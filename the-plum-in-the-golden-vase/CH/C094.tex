\chapter{大酒楼刘二撒泼~洒家店雪娥为娼}

诗曰:
骨肉伤残产业荒,一身何忍去归娼。
泪垂玉箸辞官舍,步蹴金莲入教坊。
览镜自怜倾国色,向人初学倚门妆。
春来雨露宽如海,嫁得刘郎胜阮郎。

话说陈敬济自从谢家酒楼上见了冯金宝,两个又勾搭上前情。往后没三日不和他相
会,或一日敬济有事不去,金宝就使陈三儿稍寄物事,或写情书来叫他去。一次或
五钱,或一两。以后日间供其柴米,纳其房钱。归到庙中便脸红。任道士问他何处
吃酒来,敬济只说:“在米铺和伙计畅饮三杯,解辛苦来。”他师兄金宗明一力替
他遮掩,晚夕和他一处盘弄那勾当,是不必说。朝来暮往,把任道士囊箧中细软的
本钱,也抵盗出大半花费了。

一日,也是合当有事。这洒家店的刘二,有名坐地虎,他是帅府周守备府中亲随张
胜的小舅子,专一在马头上开娼店,倚强凌弱,举放私债,与巢窝中各娼使用,加
三讨利。有一不给,捣换文书,将利作本,利上加利。嗜酒行凶,人不敢惹他。就
是打粉头的班头,欺酒客的领袖。因见陈敬济是宴公庙任道士的徒弟,白脸小厮,
谢三家大酒上把粉头郑金宝儿占住了,吃的楞楞睁睁,提着碗头大的拳头,走来谢
家楼下,问:“金宝在那里?”慌的谢三郎连忙声喏,说道:“刘二叔叔,他在楼
上第二间阁儿里便是。”这刘二大叉步上楼来。敬济正与金宝在阁儿里面饮酒,做
一处快活,把房门关闭,外边帘子挂着。被刘二一把手扯下帘子,大叫:“金宝儿
出来!”唬的陈敬济鼻口内气儿也不敢出。这刘二用脚把门跺开,金宝儿只得出来
相见,说:“刘二叔叔,有何说话?”刘二骂道:“贼淫妇,你少我三个月房钱,
却躲在这里,就不去了。”金宝笑嘻嘻说道:“二叔叔,你家去,我使妈妈就送房
钱来。”这刘二只搂心一拳,打了老婆一交,把头颅抢在阶沿下磕破,血流满地,
骂道:“贼淫妇,还等甚送来,我如今就要!”看见陈敬济在里面,走向前把桌子
只一掀,碟儿打得粉碎。那敬济便道:“阿呀,你是甚么人?走来撒野。”刘二骂
道:“我(入曰)你道士秫秫娘!”一手采过头发来,按在地下,拳捶脚踢无数。
那楼上吃酒的人,看着都立睁了。店主人谢三初时见刘二醉了,不敢惹他,次后见
打得人不像模样,上楼来解劝,说道:“刘二叔,你老人家息怒。他不晓得你老人
家大名,误言冲撞,休要和他一般见识,看小人薄面,饶他去罢。”这刘二那里依
从,尽力把敬济打了个发昏章第十一。叫将地方保甲,一条绳子,连粉头都拴在一
处墩锁,分付:“天明早解到老爷府里去。”原来守备敕书上命他保障地方,巡捕
盗贼,兼管河道。这里拿了敬济,任道士庙中尚还不知,只说晚夕米铺中上宿未回
。

却说次日,地方保甲、巡河快手押解敬济、金宝,雇头口赶清晨早到府前伺候。先
递手本与两个管事张胜、李安看,说是刘二叔地方喧闹一起,宴公庙道士一名陈宗
美,娼妇郑金宝。众军牢都问他要钱,说道:“俺们是厅上动刑的,一班十二人,
随你罢。正经两位管事的,你倒不可轻视了他。”敬济道:“身边银钱倒有,都被
夜晚刘二打我时,被人掏摸的去了。身上衣服都扯碎了,那得钱来?止有头上关顶
一根银簪儿,拔下来,与二位管事的罢。”众牢子拿着那根簪子,走来对张胜、李
安如此这般说:“他一个钱儿不拿出来,止与了这根簪儿,还是闹银的。”张胜道
:“你叫他近前,等我审问他。”众军牢不一时拥到跟前跪下,问:“你几时与任
道士做徒弟?俗名叫甚么?我从未见你。”敬济道:“小的俗名叫陈敬济,原是好
人家儿女,做道士不久。”张胜道:“你既做道士,便该习学经典,许你在外宿娼
饮酒喧嚷?你把俺帅府衙门当甚么些小衙门,不拿了钱儿来,这根簪子打水不浑,
要他做甚?”还掠与他去。分付牢子:“等住回老爷升厅,把他放在头一起。眼见
这狗男女道士,就是个吝钱的,只许你白要四方施主钱粮!休说你为官事,你就来
吃酒赴席,也带方汗巾儿揩嘴。等动刑时,着实加力拶打这厮。”又把郑金宝叫上
去。郑家有忘八跟着,上下打发了三四两银子。张胜说:“你系娼门,不过趁熟赶
些衣食为生,没甚大事。看老爷喜怒不同,看恼只是一两拶子;若喜欢,只恁放出
来也不知。”不一时,只见里面云板响,守备升厅,两边僚掾军牢森列,甚是齐整
。但见:

绯罗缴壁,紫绶桌围。当厅额挂茜罗,四下帘垂翡翠。勘官守正,戒
石上刻御制四行;人从谨廉,鹿角旁插令旗两面。军牢沉重,僚掾威
仪。执大棍授事立阶前,挟文书厅旁听发放。虽然一路帅臣,果是满
堂神道。

当时,没巧不成话,也是五百劫冤家聚会,姻缘合当凑着。春梅在府中,从去岁八
月间,已生了个哥儿小衙内。今方半岁光景,貌如冠玉,唇若涂朱。守备喜似席上
之珍,爱如无价之宝。未几,大奶奶下世,守备就把春梅册正,做了夫人。就住着
五间正房,买了两个养娘抱奶哥儿,一名玉堂,一名金匮;两个小丫鬟服侍,一名
翠花,一名兰花;又有两个身边得宠弹唱的姐儿,都十六七岁,一名海棠,一名月
桂,都在春梅房中侍奉。那孙二娘房中止使着一个丫鬟,名唤荷花儿,不在话下。
每常这小衙内,只要张胜抱他外边顽耍,遇着守备升厅,便在旁边观看。

当日,守备升厅坐下,放了告牌出去,各地方解进人来。头一起就叫上陈敬济并娼
妇郑金宝儿去。守备看了呈状,便说道:“你这厮是个道士,如何不守清规,宿娼
饮酒,骚扰地方,行止有亏。左右拿下去,打二十棍,追了度牒还俗。那娼妇郑氏
,拶一拶,敲五十敲,责令归院当差。”两边军牢向前,才待扯翻敬济,摊去衣服
,用绳索绑起,转起棍来,两边招呼要打时,可霎作怪,张胜抱着小衙内,正在月
台上站立观看,那小衙内看见打敬济,便在怀里拦不住,扑着要敬济抱。张胜恐怕
守备看见,忙走过来。那小衙内亦发大哭起来,直哭到后边春梅跟前。春梅问:“
他怎的哭?”张胜便说:“老爷厅上发放事,打那宴公庙陈道士,他就扑着要他抱
,小的走下来,他就哭了。”

这春梅听见是姓陈的,不免轻移莲步,款蹙湘裙,走到软屏后面探头观觑:“打的
那人,声音模样,倒好似陈姐夫一般,他因何出家做了道士?”又叫过张胜,问他
:“此人姓甚名谁?”张胜道:“这道士我曾问他来,他说俗名叫陈敬济。”春梅
暗道:“正是他了。”一面使张胜:“请下你老爷来。”这守备厅上打敬济才打到
十棍,一边还拶着唱的,忽听后边夫人有请,分付牢子把棍且阁住休打,一面走下
厅来。春梅说道:“你打的那道士,是我姑表兄弟,看奴面上,饶了他罢。”守备
道:“夫人何不早说,我已打了他十棍,怎生奈何?”一面出来,分付牢子:“都
与我放了。”唱的便归院去了。守备悄悄使张胜:“叫那道士回来,且休去。问了
你奶奶,请他相见。”这春梅才待使张胜请他到后堂相见,忽然沉吟想了一想,便
又分付张胜:“你且叫那人去着,待我慢慢再叫他。”度牒也不曾追。

这陈敬济打了十棍,出离了守备府,还奔来晏公庙。不想任道士听见人来说:“你
那徒弟陈宗美,在大酒楼上包着唱的郑金宝儿,惹了洒家店坐地虎刘二,打得臭死
,连老婆都拴了,解到守备府去了。行止有亏,便差军牢来拿你去审问,追度牒还
官。”这任道士听了,一者老年的着了惊怕,二来身体胖大,因打开囊箧,内又没
有许多细软东西,着了口重气,心中痰涌上来,昏倒在地。众徒弟慌忙向前扶救,
请将医者来灌下药去,通不省人事。到半夜,呜呼断气身亡。亡年六十三岁。第二
日,陈敬济来到,左右邻人说:“你还敢庙里去?你师父因为你,如此这般,得了
口重气,昨夜三更鼓死了。”这敬济听了,唬的忙忙似丧家之犬,急急如漏网之鱼
,复回清河县城中来。正是:
鹿随郑相应难辩,蝶化庄周未可知。

话分两头。却说春梅一面使张胜叫敬济且去着,一面走归房中,摘了冠儿,脱了绣
服,倒在床上,便扪心挝被,声疼叫唤起来。唬的合宅大小都慌了。下房孙二娘来
问道:“大奶奶才好好的,怎的就不好起来?”春梅说:“你每且去,休管我。”
落后守备退厅进来,见他躺在床上叫唤,也慌了。扯着他手儿问道:“你心里怎的
来?”也不言语,又问:“那个惹着你来?”也不做声。守备道:“不是我刚才打
了你兄弟,你心内恼么?”亦不应答。这守备无计奈何,走出外边麻犯起张胜、李
安来了:“你两个早知他是你奶奶兄弟,如何不早对我说?却教我打了他十下,惹
的你奶奶心中不自在。我曾教你留下他,请你奶奶相见,你如何又放他去了?你这
厮每却讨分晓!”张胜说:“小的曾禀过奶奶来,奶奶说且教他去着,小的才放他
去了。”一面走入房中,哭哭啼啼,哀告春梅:“望乞奶奶在爷前方便一言。不然
,爷要见责小的每哩。”这春梅睁圆星眼,剔起蛾眉,叫过守备近前说:“我自心
中不好,干他们甚事?那厮他不守本分,在外边做道士,且奈他些时,等我慢慢招
认他。”这守备才不麻犯张胜、李安了。

守备见他只管声唤,又使张胜请下医官来看脉,说:“老安人染了六欲七情之病,
着了重气在心。”讨将药来又不吃,都放冷了。丫头每都不敢向前说话,请将守备
来看着吃药,只呷了一口,就不吃了。守备出去了,大丫鬟月桂拿过药来,”请奶
奶吃药。”被春梅拿过来,匹脸只一泼,骂道:“贼浪奴才,你只顾拿这苦水来灌
我怎的?我肚子里有甚么?”教他跪在面前。孙二娘走来,问道:“月桂怎的?奶
奶教他跪着。”海棠道:“奶奶因他拿药与奶奶吃来,奶奶说:'我肚子里有甚么
?拿这药来灌我。'教他跪着。”孙二娘道:“奶奶,你委的今一日没曾吃甚么。
这月桂他不晓得,奶奶休打他,看我面上,饶他这遭罢。”分付海棠:“你往厨下
熬些粥儿来,与你奶奶吃口儿。”春梅于是把月桂放起来。

那海棠走到厨下,用心用意熬了一小锅粳米浓浓的粥儿,定了四碟小菜儿,用瓯儿
盛着,热烘烘拿到房中。春梅躺在床上面朝里睡,又不敢叫,直待他番身,方才请
他:“有了粥儿在此,请奶奶吃粥。”春梅把眼合着,不言语。海棠又叫道:“粥
晾冷了,请奶奶起来吃粥。”孙二娘在旁说道:“大奶奶,你这半日没吃甚么,这
回你觉好些,且起来吃些个。”那春梅一骨碌子扒起来,教奶子拿过灯来,取粥在
手,只呷了一口,往地下只一推。早是不曾把家伙打碎,被奶子接住了。就大吆喝
起来,向孙二娘说:“你平白叫我起来吃粥,你看贼奴才熬的好粥!我又不坐月子
,熬这照面汤来与我吃怎么?”分付奶子金匮:“你与我把这奴才脸上打与他四个
嘴巴!”当下真个把海棠打了四个嘴巴。孙二娘便道:“奶奶,你不吃粥,却吃些
甚么儿?却不饿着你。”春梅道:“你教我吃,我心内拦着,吃不下去。”良久,
叫过小丫鬟兰花儿来,分付道:“我心内想些鸡尖汤儿吃。你去厨房内,对那淫妇
奴才,教他洗手做碗好鸡尖汤儿与我吃。教他多放些酸笋,做的酸酸辣辣的我吃。
”孙二娘便说:“奶奶分付他,教雪娥做去。你心下想吃的就是药。”

这兰花不敢怠慢,走到厨下对雪娥说:“奶奶教你做鸡尖汤,快些做,等着要吃哩
。”原来这鸡尖汤,是雏鸡脯翅的尖儿碎切的做成汤。这雪娥一面洗手剔甲,旋宰
了两只小鸡,退刷干净,剔选翅尖,用快刀碎切成丝,加上椒料、葱花、芫荽、酸
笋、油酱之类,揭成清汤。盛了两瓯儿,用红漆盘儿,热腾腾,兰花拿到房中。春
梅灯下看了,呷了一口,怪叫大骂起来:“你对那淫妇奴才说去,做的甚么汤!精
水寡淡,有些甚味?你们只教我吃,平白叫我惹气!”慌的兰花生怕打,连忙走到
厨下对雪娥说:“奶奶嫌汤淡,好不骂哩。”这雪娥一声儿不言语,忍气吞声,从
新洗锅,又做了一碗。多加了些椒料,香喷喷,教兰花儿拿到房里来。春梅又嫌忒
咸了,拿起来照地下只一泼,早是兰花躲得快,险些儿泼了一身。骂道:“你对那
奴才说去,他不愤气做与我吃。这遭做的不好,教他讨分晓。”这雪娥听见,千不
合,万不合,悄悄说了一句:“姐姐几时这般大了,就抖搂起人来!”不想兰花回
到房里,告春梅说了。这春梅不听便罢,听了此言,登时柳眉剔竖,星眼圆睁,咬
碎银牙,通红了粉面,大叫:“与我采将那淫妇奴才来!”

须臾,使了奶娘丫鬟三四个,登时把雪娥拉到房中。春梅气狠狠的一手扯住他头发
,把头上冠子跺了,骂道:“淫妇奴才,你怎的说几时这般大?不是你西门庆家抬
举的我这般大!我买将你来伏侍我,你不愤气,教你做口子汤,不是精淡,就是苦
咸。你倒还对着丫头说我几时恁般大起来,搂搜索落我,要你何用?”一面请将守
备来,采雪娥出去,当天井跪着。前边叫将张胜、李安,旋剥褪去衣裳,打三十大
棍。两边家人点起明晃晃灯笼,张胜、李安各执大棍伺候。那雪娥只是不肯脱衣裳
。守备恐怕气了他,在跟前不敢言语。孙二娘在旁边再三劝道:“随大奶奶分付打
他多少,免褪他小衣罢。不争对着下人,脱去他衣服,他爷体面上不好看的。只望
奶奶高抬贵手,委的他的不是了。”春梅不肯,定要去他衣服打,说道:“那个拦
我,我把孩子先摔杀了,然后我也一条绳子吊死就是了。留着他便是了。”于是也
不打了,一头撞倒在地,就直挺挺的昏迷,不省人事。守备唬的连忙扶起,说道:
”随你打罢,没的气着你。”当下可怜把这孙雪娥拖番在地,褪去衣服,打了三十
大棍,打的皮开肉绽。一面使小牢子半夜叫将薛嫂儿来,即时罄身领出去办卖。

春梅把薛嫂儿叫在背地,分付:“我只要八两银子,将这淫妇奴才好歹与我卖在娼
门。随你转多少,我不管你。你若卖在别处,我打听出来,只休要见我。”那薛嫂
儿道:“我靠那里过日子,却不依你说?”当夜领了雪娥来家。那雪娥悲悲切切,
整哭到天明。薛嫂便劝道:“你休哭了,也是你的晦气,冤家撞在一处。老爷见你
到罢了,只恨你与他有些旧仇旧恨,折挫你。连老爷也做不得主儿,见他有孩子,
凡事依随他。正经下边孙二娘也让他几分。常言拐米倒做了仓官,说不的了,你休
气哭。”雪娥收泪,谢薛嫂:“只望早晚寻个好头脑我去,只有饭吃罢。”薛嫂道
:“他千万分付,只教我把你送在娼门。我养儿养女,也要天理。等我替你寻个单
夫独妻,或嫁个小本经纪人家,养活得你来也罢。”那雪娥千恩万福谢了。

薛嫂过了两日,只见邻居一个开店张妈走来叫:“薛妈,你这壁厢有甚娘子?怎的
哭的悲切?”薛嫂便道:“张妈,请进来坐。”说道:“便是这位娘子,他是大人
家出来的,因和大娘子合不着,打发出来,在我这里嫁人。情愿个单夫独妻,免得
惹气。”张妈妈道:“我那边下着一个山东卖绵花客人,姓潘,排行第五,年三十
七岁,几车花果,常在老身家安下。前日说他家有个老母有病,七十多岁,死了浑
家半年光景,没人伏侍。再三和我说,替他保头亲事,并无相巧的。我看来这位娘
子年纪到相当,嫁与他做个娘子罢。”薛嫂道:“不瞒你老人家说,这位娘子大人
家出身,不拘粗细都做的,针指女工,自不必说,又做的好汤水。今才三十五岁。
本家只要三十两银子,倒好保与他罢。”张妈妈道:“有箱笼没有?”薛嫂道:“
止是他随身衣服、簪环之类,并无箱笼。”张妈妈道:“既是如此,老身回去对那
人说,教他自家来看一看。”说毕,吃茶,坐回去了。晚夕对那人说了,次日饭罢
以后,果然领那人来相看。一见了雪娥好模样儿,年小,一口就还了二十五两,另
外与薛嫂一两媒人钱。薛嫂也没争竞,就兑了银子,写了文书。晚夕过去,次日就
上车起身。薛嫂教人改换了文书,只兑了八两银子交到府中,春梅收了,只说卖与
娼门去了。

那人娶雪娥到张妈家,止过得一夜,到第二日,五更时分,谢了张妈妈,作别上了
车,径到临清去了。此是六月天气,日子长,到马头上才日西时分。到于洒家店,
那里有百十间房子,都下着各处远方来的窠子行院唱的。这雪娥一领入一个门户,
半间房子,里面炕上坐着个五六十岁的婆子,还有个十七顶老丫头,打着盘头揸髻
,抹着铅粉红唇,穿着一弄儿软绢衣服,在炕边上弹弄琵琶。这雪娥看见,只叫得
苦,才知道那汉子潘五是个水客。买他来做粉头。起了他个名叫玉儿。这小妮子名
唤金儿,每日拿厮锣儿出去,酒楼上接客供唱,做这道路营生。这潘五进门不问长
短,把雪娥先打了一顿,睡了两日,只与他两碗饭吃,教他学乐器弹唱,学不会又
打,打得身上青红遍了。引上道儿,方与他好衣穿,妆点打扮,门前站立,倚门献
笑,眉目嘲人。正是:遗踪堪入府人眼,不买胭脂画牡丹。有诗为证:
穷途无奔更无投,南去北来休更休。
一夜彩云何处散,梦随明月到青楼。

这雪娥在洒家店,也是天假其便。一日,张胜被守备差遣往河下买几十石酒曲,宅
中造酒。这洒家店坐地虎刘二,看见他姐夫来,连忙打扫酒楼干净,在上等阁儿里
安排酒肴杯盘,请张胜坐在上面饮酒。酒博士保儿筛酒,禀问:“二叔,下边叫那
几个唱的上来递酒?”刘二分付:“叫王家老姐儿,赵家娇儿,潘家金儿,玉儿四
个上来,伏侍你张姑夫。”酒博士保儿应诺下楼。不多时,只听得胡梯畔笑声儿,
一般儿四个唱的,打扮得如花似朵,都穿着轻纱软绢衣裳,上的楼来,望上拜了四
拜,立在旁边。这张胜猛睁眼观看,内中一个粉头,可霎作怪,”到相老爷宅里打
发出来的那雪娥娘子。他如何做这道路在这里?”那雪娥亦眉眼扫见是张胜,都不
做声。这张胜便问刘二:“那个粉头是谁家的?”刘二道:“不瞒姐夫,他是潘五
屋里玉儿、金儿,这个是王老姐,一个是赵娇儿。”张胜道:“这潘家玉儿,我有
些眼熟。”因叫他近前,悄悄问他:“你莫不是雪姑娘么?怎生到于此处?”那雪
娥听见他问,便簇地两行泪下,便道:“一言难尽。”如此这般,具说一遍。”被
薛嫂撺瞒,把我卖了二十五两银子,卖在这里供筵席唱,接客迎人。”这张胜平昔
见他生的好,常是怀心。这雪娥席前殷勤劝酒,两个说得入港。雪娥和金儿不免拿
过琵琶来,唱个词儿,与张胜下酒。唱毕,彼此穿杯换盏,倚翠偎红,吃得酒浓时
,常言:“世财红粉歌楼酒,谁为三般事不迷?”这张胜就把雪娥来爱了。两个晚
夕留在阁儿里,就一处睡了。这雪娥枕边风月,耳畔山盟,和张胜尽力盘桓,如鱼
似水,百般难述。次日起来,梳洗了头面,刘二又早安排酒肴上来,与他姐夫扶头
。大盘大碗,饕食一顿,收起行装,喂饱头口,装载米曲,伴当跟随。临出门,与
了雪娥三两银子,分付刘二:“好生看顾他,休教人欺负。”自此以后,张胜但来
河下,就在洒家店与雪娥相会。往后走来走去,每月与潘五几两银子,就包住了他
,不许接人。那刘二自恁要图他姐夫欢喜,连房钱也不问他要了。各窠窝刮刷将来
,替张胜出包钱,包定雪娥柴米。有诗为证:
岂料当年纵意为,贪淫倚势把心欺。
祸不寻人人自取,色不迷人人自迷。