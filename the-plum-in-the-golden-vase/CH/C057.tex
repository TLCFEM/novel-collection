\chapter{开缘簿千金喜舍~戏雕栏一笑回嗔}

诗曰:

野寺根石壁,诸龛遍崔巍。
前佛不复辨,百身一莓苔。
惟有古殿存,世尊亦尘埃。
如闻龙象泣,足令信者哀。
公为领兵徒,咄嗟檀施开。
吾知多罗树,却倚莲花台。
诸天必欢喜,鬼物无嫌猜。

话说那山东东平府地方,向来有个永福禅寺,起建自梁武帝普通二年,开山是
那万回老祖。怎么叫做万回老祖?因那老祖做孩子的时节,才七八岁,有个哥儿从
军边上,音信不通,不知生死。他老娘思想大的孩儿,时常在家啼哭。忽一日,孩
子问母亲,说道:“娘,这等清平世界,咱家也尽挨得过,为何时时掉下泪来?娘
,你说与咱,咱也好分忧的。”老娘就说:“小孩子,你那里知道。自从你老头儿
去世,你大哥儿到边上去做了长官,四五年,信儿也没一个。不知他生死存亡,教
我老人家怎生吊的下!”说着,又哭起来。那孩子说:“早是这等,有何难哉!娘
,如今哥在那里?咱做弟郎的,早晚间走去抓寻哥儿,讨个信来,回复你老人家,
却不是好?”那婆婆一头哭,一头笑起来,说道:“怪呆子,你哥若是一百二百里
程途,便可去的,直在那辽东地面,去此一万余里,就是好汉子,也走四五个月才
到哩,你孩儿家怎么去的?”那孩子就说:“嗄,若是果在辽东,也终不在个天上
,我去寻哥儿就回也。”只见他把[革及]鞋儿系好了,把直掇儿整一整,望着婆
儿拜个揖,一溜烟去了。那婆婆叫之不应,追之不及,愈添愁闷。也有邻舍街坊、
婆儿妇女前来解劝,说道:“孩儿小,怎去的远?早晚间自回也。”因此,婆婆收
着两眶眼泪,闷闷坐的。看看红日西沉,那婆婆探头探脑向外张望,只见远远黑[
鬼戊][鬼戊]影儿里,有一个小的儿来也。那婆婆就说:“靠天靠地,靠日月三
光。若的俺小的儿子来了,也不枉了俺修斋吃素的念头。”只见那万回老祖忽地跪
到跟前说:“娘,你还未睡哩?咱已到辽东抓寻哥儿,讨的平安家信来也。”婆婆
笑道:“孩儿,你不去的正好,免教我老人家挂心。只是不要吊慌哄着老娘。那有
一万里路程朝暮往还的?”孩儿道:“娘,你不信么?”一直卸下衣包,取出平安
家信,果然是他哥儿手笔。又取出一件汗衫,带回浆洗,也是婆婆亲手缝的,毫厘
不差。因此哄动了街坊,叫做“万回”。日后舍俗出家,就叫做“万回长老”。果
然道德高妙,神通广大。曾在后赵皇帝石虎跟前,吞下两升铁针,又在梁武皇殿下
,在头顶上取出舍利三颗。因此敕建永福禅寺,做万回老祖的香火院,正不知费了
多少钱粮。正是:

神僧出世神通大,圣主尊隆圣泽深。

不想岁月如梭,时移事改。那万回老祖归天圆寂,就有些得皮得肉的上人们,
一个个多化去了。只有几个惫赖和尚,养老婆,吃烧酒,甚事儿不弄出来!不消几
日儿,把袈裟也当了,钟儿、磬儿都典了,殿上椽儿、砖儿、瓦儿换酒吃了。弄的
那雨淋风刮,佛像儿倒的,荒荒凉凉,将一片钟鼓道场,忽变作荒烟衰草。三四十
年,那一个肯扶衰起废!不想有个道长老,原是西印度国出身,因慕中国清华,打
从流沙河、星宿海走了八九个年头,才到中华区处。迤逦来到山东,就卓锡在这个
破寺里,面壁九年,不言不语,真个是:

佛法原无文字障,工夫向好定中寻。

忽一日发个念头,说道:“呀,这寺院坍塌的不成模样了,这些蠢狗才攮的秃驴,
止会吃酒[口童]饭,把这古佛道场弄得赤白白地,岂不可惜!到今日,咱不做主
,那个做主?咱不出头,那个出头?况山东有个西门大官人,居锦衣之职,他家私
巨万,富比王侯,前日饯送蔡御史,曾在咱这里摆设酒席。他见寺宇倾颓,就有个
鼎建重新的意思。若得他为主作倡,管情早晚间把咱好事成就也。咱须去走一遭。
”当时唤起法子徒孙,打起钟鼓,举集大众,上堂宣扬此意。那长老怎生打扮?但
见:

身上禅衣猩血染,双环挂耳是黄金。
手中锡杖光如镜,百八明珠耀日明。
开觉明路现金绳,提起凡夫梦亦醒。
庞眉绀发铜铃眼,道是西天老圣僧。

长老宣扬已毕,就叫行者拿过文房四宝,写了一篇疏文。好长老,真个是古佛菩萨
现身。于是辞了大众,着上禅鞋,戴上个斗笠子,一壁厢直奔到西门庆家里来。

且说西门庆辞别了应伯爵,走到吴月娘房内,把应伯爵荐水秀才的事体说了一
番,就说道:“咱前日东京去,多得众亲朋与咱把盏,如今少不的也要整酒回答他
。今日到空闲,就把这事儿完了罢。”当下就叫了玳安,吩咐买办嗄饭之类。又吩
咐小厮,分头去请各位。一面拉着月娘,走到李瓶儿房里来看官哥。李瓶儿笑嘻嘻
的接住了,就叫奶子抱出官哥儿来。只见眉目稀疏,就如粉块妆成,笑欣欣,直撺
到月娘怀里来。月娘把手接着,抱起道:“我的儿,恁的乖觉,长大来,定是聪明
伶俐的。”又向那孩子说:“儿,长大起来,恁地奉养老娘哩!”李瓶儿就说:“
娘说那里话。假饶儿子长成,讨的一官半职,也先向上头封赠起,那凤冠霞帔,稳
稳儿先到娘哩。”西门庆接口便说:“儿,你长大来还挣个文官。不要学你家老子
做个西班出身,──虽有兴头,却没十分尊重。”正说着,不想潘金莲在外边听见
,不觉怒从心上起,就骂道:“没廉耻、弄虚脾的臭娼根,偏你会养儿子!也不曾
经过三个黄梅、四个夏至,又不曾长成十五六岁,出幼过关,上学堂读书,还是个
水泡,与阎罗王合养在这里的,怎见的就做官,就封赠那老夫人?怪贼囚根子,没
廉耻的货,怎的就见的要做文官,不要象你!”正在唠唠叨叨,喃喃呐呐,一头骂
,一头着恼的时节,只见玳安走将进来,叫声“五娘”,说道:“爹在那里?”潘
金莲便骂:“怪尖嘴的贼囚根子,那个晓的你什么爹在那里!怎的到我这屋里来?
他自有五花官诰的太奶奶老封婆,八珍五鼎奉养他的在那里,那里问着我讨!”那
玳安就晓的不是路了,望六娘房里就走。走到房门前,打个咳嗽,朝着西门庆道:
“应二爹在厅上。”西门庆道:“应二爹,才送的他去,又做甚?”玳安道:“爹
出去便知。”

西门庆只得撇了月娘、李瓶儿,走到外边。见伯爵,正要问话,只见那募缘的
道长老已到西门庆门首了。高声叫:“阿弥陀佛!这是西门老爹门首么?那个掌事
的管家与吾传报一声,说道:扶桂子,保兰孙,求福有福,求寿有寿。──东京募
缘的长老求见。”原来,西门庆平日原是一个撒漫使钱的汉子,又是新得官哥,心
下十分欢喜,也要干些好事,保佑孩儿。小厮们通晓得,并不作难,一壁厢进报西
门庆。西门庆就说:“且叫他进来看。”不一时,请那长老进到花厅里面,打了个
问讯,说道:“贫僧出身西印度国,行脚到东京汴梁,卓锡在永福禅寺,面壁九年
,颇传心印。止为那宇殿倾颓,琳宫倒塌,贫僧想起来,为佛弟子,自应为佛出力
,因此上贫僧发了这个念头。前日老檀越饯行各位老爹时,悲怜本寺废坏,也有个
良心美腹,要和本寺作主。那时,诸佛菩萨已作证盟。贫僧记的佛经上说得好:如
有世间善男子、善女人以金钱喜舍庄严佛像者,主得桂于兰孙,端严美貌,日后早
登科甲,荫子封妻之报。故此特叩高门,不拘五百一千,要求老檀那开疏发心,成
就善果。”就把锦帕展开,取出那募缘疏簿,双手递上。不想那一席话儿,早已把
西门庆的心儿打动了,不觉的欢天喜地接了疏簿,就叫小厮看茶。揭开疏簿,只见
写道:

伏以白马驼经开象教,竺腾衍法启宗门。大地众僧,无不皈依佛祖;
三千世界,尽皆兰若庄严。看此瓦砾倾颓,成甚名山胜境?若不慈悲喜舍
,何称佛子仁人?今有永福禅寺,古佛道场,焚修福地。启建自梁武皇帝
,开山是万回祖师。规制恢弘,仿佛那给孤园黄金铺地;雕楼精制,依稀
似祇洹舍白玉为阶。高阁摩空,旃檀气直接九霄云表;层基亘地,
大雄殿可容千众禅僧。两翼巍峨,尽是琳宫绀宇;廊房洁净,果然精胜洞
天。那时钟鼓宣扬,尽道是寰中佛国;只这缁流济楚,却也像尘界人天。
那知岁久年深,一瞬时移事换。莽和尚纵酒撒泼,毁坏清规;呆道人懒惰
贪眠,不行打扫。渐成寂寞,断绝门徒;以致凄凉,罕稀瞻仰。兼以鸟鼠
穿蚀,那堪风雨漂摇。栋宇摧颓,一而二,二而三,支撑靡计;墙垣坍塌
,日复日,年复年,振起无人。朱红棂槅,拾来煨酒煨茶;合抱栋
梁,拿去换盐换米。风吹罗汉金消尽,雨打弥陀化作尘。吁嗟乎!金碧[
火昆]炫,一旦为灌莽荆榛。虽然有成有败,终须否极泰来。幸而有道长
老之虔诚,不忍见梵王宫之废败。发大弘愿,遍叩檀那。伏愿咸起慈悲,
尽兴恻隐。梁柱椽楹,不拘大小,喜舍到高题姓字;银钱布币,岂论丰赢
,投柜入疏簿标名。仰仗着佛祖威灵,福禄寿永永百年千载;倚靠他伽蓝
明镜,父子孙个个厚禄高官。瓜瓞绵绵,森挺三槐五桂;门庭奕奕,辉煌
金阜钱山。凡所营求,吉祥如意。疏文到日,各破悭心。谨疏。

西门庆看毕,恭恭敬敬放在桌儿上面,对长老说:“实不相瞒,在下虽不成个人家
,也有几万产业,忝居武职。不想偌大年纪,未曾生下儿子,有意做些善果。去年
第六房贱内生下孩子,咱万事已是足了。偶因饯送俺友,得到上方,因见庙字倾颓
,实有个舍财助建的念头。蒙老师下顾,那敢推辞!”拿着兔毫妙笔,正在踌躇之
际,应伯爵就说:“哥,你既有这片好心为侄儿发愿,何不一力独成,也是小可的
事体。”西门庆拿着笔笑道:“力薄,力薄。”伯爵又道:“极少也助一千。”西
门庆又笑道:“力薄,力薄。”那长老就开口说道:“老檀越在上,不是贫僧多口
,我们佛家的行径,只要随缘喜舍,终不强人所难,但凭老爹发心便是。此外亲友
,更求檀越吹嘘吹嘘。”西门庆说道:“还是老师体量。少也不成,就写上五百两
。”搁了兔毫笔,那长老打个问讯谢了。西门庆又说:“我这里内官太监、府县仓
巡,一个个都与我相好的,我明日就拿疏簿去要他们写。写的来,就不拘三百二百
、一百五十,管情与老师成就这件好事。”当日留了长老素斋,相送出门。正是:

慈悲作善豪家事,保福消灾父母心。

西门庆送了长老,转到厅上,与应伯爵坐地,道:“我正要差人请你,你来的
正好。我前日往东京,多谢众亲友们与咱把盏,今日安排小酒与众人回答,要二哥
在此相陪,不想遇着这个长老,鬼混了一会儿。”伯爵便说道:“好个长老,想是
果然有德行的。他说话中间,连咱也心动起来,做了施主。”西门庆说道:“你又
几时做施主来?疏簿又是几时写的?”应伯爵笑道:“哥,你不知道,佛经上第一
重的是心施,第二法施,第三才是财施。难道我从旁撺掇的,不当个心施?”西门
庆笑道:“二哥,只怕你有口无心哩。”两人拍手大笑,应伯爵就说:“小弟在此
等待客来,哥有正事,自与嫂子商议去。”

只见西门庆别了伯爵,转到内院里头,只见那潘金莲唠唠叨叨,没揪没采,不
觉的睡魔缠扰,打了几个喷涕,走到房中,倒在象牙床上睡去了。李瓶儿又为孩子
啼哭,自与奶子、丫鬟在房中坐地,看官哥。只有吴月娘与孙雪娥两个看着整办嗄
饭。西门庆走到面前坐的,就把道长老募缘与自己开疏的事,备细说了一番。又把
应伯爵耍笑打觑的话也说了一番。欢天喜地,大家嘻笑了一会。那吴月娘毕竟是个
正经的人,不慌不忙说下几句话儿,到是西门庆顶门上针。正是:

妻贤每至鸡鸣警,款语常闻药石言。

月娘说道:“哥,你天大的造化,生下孩儿。你又发起善念。广结良缘,岂不是俺
一家儿的福分!只是那善念头怕他不多,那恶念头怕他不尽。哥,你日后那没来回
没正经养婆娘、没搭煞贪财好色的事体少干几桩儿,却不[亻赞]下些阴功,与那
小孩子也好!”西门庆笑道:“你的醋话儿又来了。却不道天地尚有阴阳,男女自
然配合。今生偷情的、苟合的,都是前生分定,姻缘簿上注名,今生了还,难道是
生剌剌胡搊乱扯歪厮缠做的?咱闻那佛祖西天,也止不过要黄金铺地,阴司
十殿,也要些楮镪营求。咱只消尽这家私广为善事,就使强奸了[女亘]娥,和奸
了织女,拐了许飞琼,盗了西王母的女儿,也不减我泼天的富贵。”月娘笑道:“
狗吃热屎,原道是个香甜的;生血掉在牙儿内,怎生改得!”

正在笑间,只见王姑子同了薛姑子,提了一个盒儿,直闯进来,朝月娘打问讯
,又向西门庆拜了拜,说:“老爹,你倒在家里。”月娘一面让坐。看官听说,原
来这薛姑子不是从幼出家的,少年间曾嫁丈夫,在广成寺前卖蒸饼儿生理。不料生
意浅薄,与寺里的和尚、行童调嘴弄舌,眉来眼去,刮上了四五六个。常有些馒头
斋供拿来进奉他,又有那应付钱与他买花,开地狱的布,送与他做裹脚。他丈夫那
里晓得!以后,丈夫得病死了,他因佛门情熟,就做了个姑子。专一在士夫人家往
来,包揽经忏。又有那些不长进、要偷汉子的妇人,叫他牵引。闻得西门庆家里豪
富,侍妾多人,思想拐些用度,因此频频往来。有一只歌儿道得好:

尼姑生来头皮光,
拖子和尚夜夜忙。
三个光头好象师父师兄并师弟,
只是铙钹原何在里床?

薛姑子坐下,就把小盒儿揭开,说道:“咱每没有甚么孝顺,拿得施主人家几个供
佛的果子儿,权当献新。”月娘道:“要来竟自来便了,何苦要你费心!”只见潘
金莲睡觉,听得外边有人说话,又认是前番光景,便走向前来听看。见李瓶儿在房
中弄孩子,因晓得王姑于在此,也要与他商议保佑官哥。因一同走到月娘房中。大
家道个万福,各各坐地。西门庆因见李瓶儿来,又把那道长老募缘与自家开疏舍财
,替官哥求福的事情,又说一番。不想恼了潘金莲,抽身竟走,喃喃哝哝,竟自去
了。那薛姑子听了,就站将起来,合掌叫声:“佛阿!老爹你这等样好心作福,怕
不的寿年千岁,五男二女,七子团圆。只是我还有一件说与你老人家──这个因果
费不甚多,更自获福无量。咦,老檀越,你若干了这件功德,就是那老瞿昙雪山修
道,迦叶尊散发铺地,二祖师投崖饲虎,给孤老满地黄金,也比不得你功德哩!”
西门庆笑道:“姑姑且坐下,细说甚么功果,我便依你。”薛姑子就说:“我们佛
祖留下一卷《陀罗经》,专一劝人生西方净土。因为那肉眼凡夫不生尊信,故此佛
祖演说此经,劝你专心念佛,竟往西方,永永不落轮回。那佛祖说的好,如有人持
诵此经,或将此经印刷抄写,转劝一人至千万人持诵,获福无量。况且此经里面又
有《护诸童子经》儿,凡有人家生育男女,必要从此发心,方得易长易养,灾去福
来。如今这副经板现在,只没人印刷施行。老爹只消破些工料印上几千卷,装钉完
成,普施十方。那个功德真是大的紧。”西门庆道:“这也不难,只不知这一卷经
要多少纸札,多少装钉,多少印刷,有个细数才好动弹。”薛姑子又道:“老爹,
你那里去细细算他,止消先付九两银子,叫经坊里印造几千万卷,装钉完满,以后
一搅果算还他就是了。”

正说的热闹,只见陈敬济要与西门庆说话,寻到卷棚底下,刚刚凑巧遇着了潘
金莲凭栏独恼。猛抬头儿见了敬济,就是猫儿见了鱼鲜饭一般,不觉把一天愁闷都
改做春风和气。两个见没有人来,就执手相偎,剥嘴咂舌头。两个肉麻顽了一回,
又恐怕西门庆出来撞见,连算帐的事情也不提了。一双眼又象老鼠儿防猫,左顾右
盼,要做事又没个方便,只得一溜烟出去了。

且说西门庆听了薛姑子的话头,不觉又动了一片善心,就叫玳安拿拜匣,取出
一封银子,准准三十两,便交付薛姑子与王姑子:“即便同去经坊里,与我印下五
千卷经,待完了,我就算帐找他。”正话间,只见书童忙忙来报道:“请的各位客
人都到了。”少不的是吴大舅、花大舅、谢希大、常峙节这一班。西门庆忙整衣出
外迎接升堂。就叫小厮摆下桌儿,请众人一行儿分班列次,各叙长幼坐的。不一时
,大鱼大肉、时新果品,一齐儿捧将出来。只见酒逢知己,形迹都忘。猜枚的、打
鼓的、催花的,三拳两谎的,歌的歌,唱的唱,顽不尽少年场光景,说不了醉乡里
日月。正是:

秋月春花随处有,赏心乐事此时同。