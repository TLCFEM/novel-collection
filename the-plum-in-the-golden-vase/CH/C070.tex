\chapter{老太监引酌朝房~二提刑庭参太尉}

诗曰:

帝曰简才能,旌贤在股肱。
文章体一变,礼乐道逾弘。
芸阁英华人,宾门鹓鹭登。
恩筵过所望,圣泽实超恒。

话说西门庆自此与李桂姐断绝不题。却说走差人到怀庆府林千户处打听消息,
林千户将升官邸报封付与来人,又赏了五钱银子,连夜来递与提刑两位官府。当厅
夏提刑拆开,同西门庆先观本卫行来考察官员照会,其略曰:

兵部一本,尊明旨,严考核,以昭劝惩,以光圣治事:先该金吾卫提
督官校太尉太保兼太子太保朱题前事,考察禁卫官员,除堂上官自陈外,
其余两厢诏狱缉捕、内外提刑所指挥千百户、镇抚等官,各挨次格,从公
举劾,甄别贤否,具题上请,当下该部详议,黜陟升调降革等因。

奉圣旨:兵部知道,钦此钦遵。抄出到部。看得太尉朱题前事,遵奉
旧例,委的本官殚力致忠,公于考核,皆出闻见之实,而无偏执之私。足
以励人心而孚公议,无容臣等再喙。但恩威赏罚,出自朝廷,合候命下之
日,一体照例施行等因。续奉钦依拟行。

内开山东提刑所正千户夏延龄,资望既久,才练老成,昔视典牧而坊
隅安静,今理齐刑而绰有政声,宜加奖励,以冀甄升,可备卤簿之选者也
。贴刑副千户西门庆,才干有为,精察素著。家称殷实而在任不贪,国事
克勤而台工有绩。翌神运而分毫不索,司法令而齐民果仰。宜加转正,以
掌刑名者也。怀庆提刑千户所正千户林承勋,年清优学,占籍武科,继祖
职抱负不凡,提刑狱详明有法,可加奖励简任者也。副千户谢恩,年齿既
残,昔在行犹有可观,今任理刑罹软尤甚,宜罢黜革任者也。

西门庆看了他转正千户掌刑,心中大悦。夏提刑见他升指挥,管卤簿,大半日无言
,面容失色。于是又展开工部工完的本观看,上面写道:

工部一本,神运届京,天人胥庆,恳乞天恩,俯加渥典,以苏民困,
以广圣泽事。

奉圣旨:这神运奉迎大内,奠安艮岳,以承天眷,朕心嘉悦。你每既
效有勤劳,副朕事玄至意。所经过地方,委的小民困苦,着行抚按衙门,
查勘明白,着行蠲免今岁田租之半。所毁坝闸,着部里差官会同巡按御史
,即行修理。完日还差内侍孟昌龄前去致祭。蔡京、李邦彦、王炜、郑居
中、高俅,辅弼朕躬,直赞内廷,勋劳茂著,京加太师,邦彦加柱国太子
太师,王炜太傅,郑居中、高俅太保,各赏银五十两、四表礼。蔡京还荫
一子为殿中监。国师林灵素,佐国宣化,远致神运,北伐虏谋,实与天通
,加封忠孝伯,食禄一千石,赐坐龙衣一袭,肩舆人内,赐号玉真教主,
加渊澄玄妙广德真人、金门羽客、达灵玄妙先生。朱[面力]、黄经臣,
督理神运,忠勤可嘉。[面力]加太傅兼太子太傅,经臣加殿前都太尉,
提督御前人船。各荫一子为金吾卫正千户。内侍李彦、孟昌龄、贾祥、何
沂、蓝从颐着直延福五位宫近侍,各赐蟒衣玉带,仍荫弟侄一人为副千户
,俱见任管事。礼部尚书张邦昌、左侍郎兼学士蔡攸、右侍郎白时中、兵
部尚书余深、工部尚书林摅,俱加太子太保,各赏银四十两,彩缎二表礼
。巡抚两浙佥都御史张阁,升工部右侍郎。巡抚山东都御史侯[氵蒙],
升太常正卿。巡抚两浙、山东监察御史尹大谅、宋乔年,都水司郎中安忱
、伍训,各升俸一级,赏银二十两。祇迎神运千户魏承勋、徐相、
杨廷佩、司凤仪、赵友兰、扶天泽、西门庆、田九皋等,各升一级。内侍
宋推等,营将王佑等,俱各赏银十两。所官薛显忠等,各赏银五两。校尉
昌玉等,绢二匹。该衙门知道。

夏提刑与西门庆看毕,各散回家。后晌时分,有王三官差永定同文嫂拿请书,
十一日请西门庆往他府中赴席,少罄谢私之意。西门庆收下,不胜欢喜,以为其妻
指日在于掌握。不期到初十日晚夕,东京本卫经历司差人行照会:“晓谕各省提刑
官员知悉:火速赴京,赶冬节见朝谢恩,毋得违误取罪。”西门庆看了,到次日衙
门中会了夏提刑,各人到家,即收拾行装,备办贽见礼物,约早晚起程。西门庆使
玳安叫了文嫂儿,教他回王三官:“我今日不得来赴席,要上京见朝谢恩去。”文
嫂连忙去回,王三官道:“既是老伯有事,容回来洁诚具请。”西门庆一面叫将贲
四来,吩咐教他跟了去,与他五两银子,家中盘缠。留下春鸿看家,带了玳安、王
经跟随答应。又问周守备讨了四名巡捕军人,四匹小马,打点驮装轿马,排军抬扛
。夏提刑便是夏寿跟随。两家共有二十余人跟从。十二日起身离了清河县,冬天易
晚,昼夜趱行。到了怀西怀庆府会林千户,千户已上东京去了。一路天寒坐轿,天
暖乘马,朝登紫陌,暮践红尘。正是:

意急款摇青帐幕,心忙敲碎紫丝鞭。

话说一日到了东京,进得万寿门。西门庆主意要往相国寺下。夏提刑不肯,坚
执要往他亲眷崔中书家投下。西门庆不免先具拜帖拜见。正值崔中书在家,即出迎
接,至厅叙礼相见,与夏提刑道及寒温契阔之情。坐下茶毕,拱手问西门庆尊号。
西门庆道:“贱号四泉。”因问:“老先生尊号?”崔中书道:“学生性最愚朴,
名闲林下,贱名守愚,拙号逊斋。”因说道:“舍亲龙溪久称盛德,全仗扶持,同
心协恭,莫此为厚。”西门庆道:“不敢。在下常领教诲,今又为堂尊,受益恒多
,不胜感激。”夏提刑道:“长官如何这等称呼!便不见相知了。”崔中书道:“
四泉说的也是,名分使然。”言毕,彼此笑了。不一时,收拾行李。天晚了,崔中
书吩咐童仆放桌摆饭,无非是果酌肴馔之类,不必细说。当日,二人在崔中书家宿
歇不题。

到次日,各备礼物拜帖,家人跟随,早往蔡太师府中叩见。那日太师在内阁还
未出来,府前官吏人等如蜂屯蚁聚,挤匝不开。西门庆与夏提刑与了门上官吏两包
银子,拿揭帖禀进去。翟管家见了,即出来相见,让他到外边私宅。先是夏提刑先
见毕,然后西门庆叙礼,彼此道及往还酬答之意,各分宾位坐下。夏提刑先递上礼
帖:两匹云鹤金缎、两匹色缎。翟管家是十两银子。西门庆礼帖上是一匹大红绒彩
蟒、一匹玄色妆花斗牛补子员领、两匹京缎,另外梯己送翟管家一匹黑绿云绒、三
十两银子。翟谦吩咐左右:“把老爷礼都收进府中去,上簿籍。”他只受了西门庆
那匹云绒,将三十两银子连夏提刑的十两银子都不受,说道:“岂有此理。若如此
,不见至交亲情。”一面令左右放桌儿摆饭,说道:“今日圣上奉艮岳,新盖上清
宝[竹录]宫,奉安牌匾,该老爷主祭,直到午后才散。到家同李爷又往郑皇亲家
吃酒。只怕亲家和龙溪等不的,误了你每勾当。遇老爷闲,等我替二位禀就是一般
。”西门庆道:“蒙亲家费心。”翟谦因问:“亲家那里住?”西门庆就把夏龙溪
令亲家下歇说了。不一时,安放桌席端正,就是大盘大碗,汤饭点心一齐拿上来,
都是光禄烹炮,美味极品无加。每人金爵饮酒三杯,就要告辞起身。翟谦款留,令
左右又筛上一杯。西门庆因问:“亲家,俺每几时见朝?”翟谦道:“亲家,你同
不得夏大人。夏大人如今是京堂官,不在此例。你与本卫新升的副千户何大监侄儿
何永寿,他便贴刑,你便掌刑,与他作同僚了。他先谢了恩,只等着你见朝引奏毕
,一同好领札付。你凡事只会他去。”夏提刑听了,一声儿不言语。西门庆道:“
请问亲家,只怕我还要等冬至郊天回来见朝。”翟谦道:“亲家,你等不的冬至圣
上郊天回来。那日天下官员上表朝贺,还要排庆成宴,你每怎等的?不如你今日先
往鸿胪寺报了名,明日早朝谢了恩,直到那日堂上官引奏毕,领札付起身就是了。
”西门庆谢道:“蒙亲家指教,何以为报!”临起身,翟谦又拉西门庆到侧净处说
话,甚是埋怨西门庆说:“亲家,前日我的书上那等写了,大凡事要谨密,不可使
同僚每知道。亲家如何对夏大人说了?教他央了林真人帖子来,立逼着朱太尉来对
老爷说,要将他情愿不管卤簿,仍以指挥职衔在任所掌刑三年;何大监又在内廷,
转央朝廷所宠安妃刘娘娘的分上,便也传旨出来,亲对老爷和朱太尉说了,要安他
侄儿何永寿在山东理刑。两下人情阻住了,教老爷好不作难!不是我再三在老爷跟
前维持,回倒了林真人,把亲家不撑下去了?”慌的西门庆连忙打躬,说道:“多
承亲家盛情!我并不曾对一人说,此公何以知之?”翟谦道:“自古机事不密则害
成,今后亲家凡事谨慎些便了。”

西门庆千恩万谢,与夏提刑作辞出门。来到崔中书家,一面差贲四鸿胪寺报了
名。次日同夏提刑见朝,青衣冠带,正在午门前谢恩出来,刚转过西阙门来,只见
一个青衣人走向前问道:“那位是山东提刑西门老爹?”贲四问道:“你是那里的
?”那人道:“我是内府匠作监何公公来请老爹说话。”言未毕,只见一个太监,
身穿大红蟒衣,头戴三山帽,脚下粉底皂靴,从御街定声叫道:“西门大人请了!
”西门庆遂与夏提刑分别,被这太监用手一把拉在旁边一所值房内,相见作揖,慌
的西门庆倒身还礼不迭。这太监说道:“大人,你不认的我,在下是匠作监太监何
沂,见在延宁第四宫端妃马娘娘位下近侍。昨日内工完了,蒙万岁爷爷恩典,将侄
儿何永寿升受金吾卫副千户,见在贵处提刑所理刑管事,与老大人作同僚。”西门
庆道:“原来是何老太监,学生不知,恕罪,恕罪!”一面又作揖说道:“此禁地
,不敢行礼,容日到老太监外宅进拜。”于是叙礼毕,让坐,家人捧茶来吃了。茶
毕,就揭桌盒盖儿,桌上许多汤饭肴品,拿盏箸儿来安下。何太监道:“不消小杯
了,我晓的大人朝下来,天气寒冷,拿个小盏来,没甚肴馔,亵渎大人,且吃个头
脑儿罢。”西门庆道:“不当厚扰。”何太监于是满斟上一大杯,递与西门庆,西
门庆道:“承老太监所赐,学生领下。只是出去还要见官拜部,若吃得面红,不成
道理。”何太监道:“吃两盏儿烫寒何害!”因说道:“舍侄儿年幼,不知刑名,
望乞大人看我面上,同僚之间,凡事教导他教导。”西门庆道:“岂敢。老太监勿
得太谦,令侄长官虽是年幼,居气养体,自然福至心灵。”何太监道:“大人好说
。常言:学到老不会到老。天下事如牛毛,孔夫子也只识的一腿。恐有不到处,大
人好歹说与他。”西门庆道:“学生谨领。”因问:“老大监外宅在何处?学生好
来奉拜长官。”何大监道:“舍下在天汉桥东,文华坊双狮马台就是。”亦问:“
大人下处在那里?我教做官的先去叩拜。”西门庆道:“学生暂借崔中书家下。”

彼此问了住处,西门庆吃了一大杯就起身。何太监送出门,拱着手说道:“适
间所言,大人凡事看顾看顾。他还等着你一答儿引奏,好领札付。”西门庆道:“
老太监不消吩咐,学生知道。”于是出朝门,又到兵部,又遇见了夏提刑,同拜了
部官来。比及到本卫参见朱太尉,递履历手本,缴札付,又拜经历司并本所官员,
已是申刻时分。夏提刑改换指挥服色,另具手本参见了朱太尉,免行跪礼,择日南
衙到任。刚出衙门,西门庆还等着,遂不敢与他同行,让他先上马。夏延龄那里肯
?定要同行。西门庆赶着他呼“堂尊”,夏指挥道:“四泉,你我同僚在先,为何
如此称呼?”西门庆道:“名分已定,自然之理,何故大谦。”因问:“堂尊高升
美任,不还山东去了,宝眷几时搬取?”夏延龄道:“欲待搬来,那边房舍无人看
守。如今且在舍亲这边权住,直待过年,差人取家小罢了。还望长官早晚看顾一二
。房子若有人要,就央长官替我打发,自当报谢。”西门庆道:“学生谨领。请问
府上那房价值若干?”夏延龄道:“舍下此房原是一千三百两买的,后边又盖了一
层,使了二百两,如今卖原价也罢了。”

二人归到崔宅,王经向前禀说:“新升何老爹来拜,下马到厅。小的回部中还
未来家。何老爹说多拜上夏老爹、崔老爹,都投下帖。午间又差人送了两匹金缎来
。”宛红帖儿拿与西门庆看,上写着:“谨具缎帕二端,奉引贽敬。寅侍教生何永
寿顿首拜。”西门庆看了,连忙差王经封了两匹南京五彩狮补员领,写了礼帖。吃
了饭,连忙往何家回拜去。到于厅上,何千户忙出来迎接,乌纱皂履,年纪不上二
十岁,生的面如傅粉,唇若涂朱,趋下阶来揖让,退逊谦恭特甚。二人到厅上叙礼
,西门庆令玳安捧上贽见之礼,拜下去,说道:“适承光顾,兼领厚仪,又失迎迓
。今早又蒙老公公值房赐馔,感德不尽。”何千户忙还礼说:“学生叨受微职,忝
与长官同例,早晚得领教益,实为三生有幸。适间进拜不遇,又承垂顾,蓬筚光生
。”令左右收下去,一面扯椅儿分宾主坐下,左右捧茶上来。吃茶之间,彼此问号
,西门庆道:“学生贱号四泉。”何千户道:“学生贱号天泉。”又问:“长官今
日拜毕部堂了?”西门庆道:“从内里蒙公公赐酒出来,拜毕部,又到本衙门见堂
,缴了札付,拜了所司。出来就要奉谒长官,不知反先辱长官下顾。”何千户因问
:“长官今日与夏公都见朝来?”西门庆道:“夏龙溪已升了指挥直驾,今日都见
朝谢恩在一处,只到衙门见堂之时,他另具手本参见。”说毕,何千户道:“咱每
还是先与本主老爹进礼,还是先领札付?”西门庆道:“依着舍亲说,咱每先在卫
主宅中进了礼,然后大朝引奏,还在本衙门到堂同众领札付。”何千户道:“既是
如此,咱每明早备礼进了罢。”于是都会下各人礼数,何千户是两匹蟒衣、一束玉
带,西门庆是一匹大红麒麟金缎、一匹青绒蟒衣、一柄金镶玉绦环,各金华酒四坛
。明早在朱太尉宅前取齐。约会已定,茶汤两换,西门庆告辞而回,并不与夏延龄
题此事。一宿晚景题过。

到次日,早到何千户家。何千户又预备头脑小席,大盘大碗,齐齐整整,连手
下人饱餐一顿,然后同往大尉宅门前来。贲四同何家人押着礼物。那时正值朱太尉
新加太保,微宗天子又差使往南坛视牲未回,各家馈送贺礼并参见官吏人等,黑压
压在门首等候。何千户同西门庆下了马,在左近一相识人家坐的,差人打听老爷道
子响就来通报。直等到午后,忽见一人飞马而来,传报道:“老爷视牲回来,进南
薰门了。”吩咐闲杂人打开。不一时,又骑报回来,传:“老爷过天汉桥了。”少
顷,只见官吏军士各打执事旗牌,一对一对传呼,走了半日,才远远望见朱太尉八
抬八簇肩舆明轿,头戴乌纱,身穿猩红斗牛绒袍,腰横荆山白玉,悬挂太保牙牌、
黄金鱼钥,好不显赫威严!执事到了宅门首,都一字儿摆开,喝的肃静回避,无一
人声嗽。那来见的官吏人等,黑压压一群跪在街前。良久,太尉轿到跟前,左右喝
声:“起来伺候!”那众人一齐应诺,诚然声震云霄。只听东边咚咚鼓乐响动,原
来本衙门六员太尉堂官,见朱太尉新加光禄大夫、太保,又荫一子为千户,都各备
大礼,治酒庆贺,故有许多教坊伶官在此动乐。太尉才下轿,乐就止了。各项官吏
人等,预备进见。忽然一声道子响,一青衣承差手拿两个红拜帖,飞走而来,递与
门上人说:“礼部张爷与学士蔡爷来拜。”连忙禀报进去。须臾轿在门首,尚书张
邦昌与侍郎蔡攸,都是红吉服孔雀补子,一个犀带,一个金带,进去拜毕,待茶毕
,送出来。又是吏部尚书王祖道与左侍郎韩侣、右侍郎尹京也来拜,朱太尉都待茶
送了。又是皇亲喜国公、枢密使郑居中、驸马掌宗人府王晋卿,都是紫花玉带来拜
。唯郑居中坐轿,这两个都骑马。送出去,方是本衙堂上六员太尉到了:头一位是
提督管两厢捉察使孙荣,第二位管机察梁应龙,第三管内外观察典牧皇畿童大尉侄
儿童天胤,第四提督京城十三门巡察使黄经臣,第五管京营卫缉察皇城使窦监,第
六督管京城内外巡捕使陈宗善。都穿大红,头戴貂蝉,惟孙荣是太子太保玉带,余
者都是金带。下马进去。各家都有金币礼物。少顷,里面乐声响动,众太尉插金花
,与朱太尉把盏递酒,阶下一派箫韶盈耳,两行丝竹和鸣。端的食前方丈,花簇锦
筵。怎见得太尉的富贵?但见:

官居一品,位列三台。赫赫公堂,潭潭相府。虎符玉节,门庭甲仗生
寒;象板银筝,[石鬼][石田田田]排场热闹。终朝谒见,无非公子王
孙;逐岁追游,尽是侯门戚里。那里解调和燮理,一味能趋谄逢迎。端的
谈笑起干戈,真个吹嘘惊海岳。假旨令八位大臣拱手,巧辞使九重天子点
头。督择花石,江南淮北尽灾殃;进献黄杨,国库民财皆匮竭。正是:辇
下权豪第一,人间富贵无双。

须臾递毕,安席坐下。一班儿五个俳优,朝上筝[竹秦]琵琶,方响箜篌,红牙象
板,唱了一套“享富贵,受皇恩”。

当时酒进三巡,歌吟一套,六员太尉起身,朱太尉亲送出来,回到厅,乐声暂
止,管家禀事,各处官员进见。朱太尉令左右抬公案,当厅坐下,吩咐出来,先令
各勋戚中贵仕宦家人送礼的进去。须臾打发出来,才是本卫纪事、南北卫两厢、五
所、七司捉察、讥察、观察、巡察、典牧、直驾、提牢、指挥、千百户等官,各具
手本呈递。然后才传出来,叫两淮、两浙、山东、山西、关东、关西、河东、河北
、福建、广南、四川十三省提刑官挨次进见。西门庆与何千户在第五起上,抬进礼
物去,管家接了礼帖,铺在书案上,二人立在阶下,等上边叫名字。西门庆抬头见
正面五间厂厅,上面朱红牌匾,悬着徽宗皇帝御笔钦赐“执金吾堂”斗大四个金字
,甚是显赫。须臾叫名,二人应诺升阶,到滴水檐前躬身参谒,四拜一跪,听发放
。朱太尉道:“那两员千户,怎的又叫你家太监送礼来?”令左右收了,吩咐:“
在地方谨慎做官,我这里自有公道。伺候大朝引奏毕,来衙门中领札赴任。”二人
齐声应诺。左右喝:“起去!”由左角门出来。刚出大门来,寻见贲四等抬担出来
,正要走,忽见一人拿宛红帖飞马来报,说道:“王爷、高爷来了。”西门庆与何
千户闪在人家门里观看。须臾,军牢喝道,只见总督京营八十万禁军陇西公王烨,
同提督神策御林军总兵官太尉高俅,俱大红玉带,坐轿而至。那各省参见官员一涌
出来,又不得见了。西门庆与何千户走到僻处,呼跟随人扯过马来,二人方骑上马
回寓。正是:

权奸误国祸机深,开国承家戒小人。
逆贼深诛何足道,奈何二圣远蒙尘。