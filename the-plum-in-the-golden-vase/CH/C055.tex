\chapter{西门庆两番庆寿旦~苗员外一诺送歌童}

词曰:

师表方眷遇,鱼水君臣,须信从来少。宝运当千,佳辰余五,嵩岳诞
生元老。帝遣阜安宗社,人仰雍容廊庙。愿岁岁共祝眉寿,寿比山高。

却说任医官看了脉息,依旧到厅上坐下。西门庆便开言道:“不知这病症端的
何如?”任医官道:“夫人这病,原是产后不慎调理,因此得来。目下恶路不净,
面带黄色,饮食也没些要紧,走动便觉烦劳。依学生愚见,还该谨慎保重。如今夫
人两手脉息虚而不实,按之散大。这病症都只为火炎肝腑,土虚木旺,虚血妄行。
若今番不治,后边一发了不的。”说毕,西门庆道:“如今该用甚药才好?”任医
官道:“只用些清火止血的药──黄柏、知母为君,其余再加减些,吃下看住,就
好了。”西门庆听了,就叫书童封了一两银子,送任医官做药本,任医官作谢去了
。不一时,送将药来,李瓶儿屋里煎服,不在话下。

且说西门庆送了任医官去,回来与应伯爵说话。伯爵因说:“今日早晨,李三
、黄四走来,说他这宗香银子急的紧,再三央我来求哥。好歹哥看我面,接济他这
一步儿罢。”西门庆道:“既是这般急,我也只得依你了。你叫他明日来兑了去罢
。”一面让伯爵到小卷棚内,留他吃饭。伯爵因问:“李桂儿还在这里住着哩?东
京去的也该来了。”西门庆道:“正是,我紧等着还要打发他往扬州去,敢怕也只
在早晚到也。”说毕,吃了饭,伯爵别去。到次日,西门庆衙门中回来,伯爵早已
同李智、黄四坐在厅上等。见西门庆回来,都慌忙过来见了。西门庆进去换了衣服
,就问月娘取出徐家讨的二百五十两银子,又添兑了二百五十两,叫陈敬济拿了,
同到厅上,兑与李三、黄四。因说道:“我没银子,因应二哥再三来说,只得凑与
你。──我却是就要的。”李三道:“蒙老爹接济,怎敢迟延!如今关出这批银子
,一分也不敢动,就都送了来,”于是兑收明,千恩万谢去了。伯爵也就要去,被
西门庆留下。

正坐的说话,只见平安儿进来报说:“来保东京回来了。”伯爵道:“我昨日
就说也该来了。”不一时,来保进到厅上,与西门庆磕了头。西门庆便问:“你见
翟爹么?李桂姐事情怎样了?”来保道:“小的亲见翟爹。翟爹见了爹的书,随即
叫长班拿帖儿与朱太尉去说,小的也跟了去。朱太尉亲吩咐说:‘既是太师府中分
上,就该都放了。因是六黄太尉送的,难以回他,如乃未到者,俱免提;已拿到的
,且监些时。他内官性儿,有头没尾。等他性儿坦些,也都从轻处就是了。’”伯
爵道:“这等说,连齐香儿也免提了?──造化了这小淫妇儿了!”来保道:“就
是祝爹他每,也只好打几下罢了。罪,料是没了。”一面取出翟管家书递上。西门
庆看了说道:“老孙与祝麻子,做梦也不晓的是我这里人情。”伯爵道:“哥,你
也只当积阴骘罢了。”来保又说:“翟爹见小的去,好不欢喜,问爹明日可与老爷
去上寿?小的不好回说不去,只得答应:‘敢要来也。’翟爹说:‘来走走也好,
我也要与你爹会一会哩。’”西门庆道:“我到也不曾打点自去。既是这等说,只
得要去走遭了。”因吩咐来保:“你辛苦了,且到后面吃些酒饭,歇息歇息。迟一
两日,还要赶到扬州去哩。”来保应诺去了。西门庆就要进去与李桂姐说知,向伯
爵道:“你坐着,我就来。”伯爵也要去寻李三、黄四,乘机说道:“我且去着,
再来罢。”一面别去。

西门庆来到月娘房里,李桂姐已知道信了,忙走来与西门庆、月娘磕头,谢道
:“难得爹娘费心,救了我这一场大祸。拿甚么补报爹娘!”月娘道:“你既在咱
家恁一场,有些事儿,不与你处处,却为着甚么来?”桂姐道:“俺便赖爹娘可怜
救了,只造化齐香儿那小淫妇儿,他甚相干?连他都饶了。他家赚钱赚钞,带累俺
们受惊怕,俺每倒还只当替他说了个大人情,不该饶他才好!”西门庆笑道:“真
造化了这小淫妇儿了。”说了一回,挂姐便要辞了家去,道:“我家妈还不知道这
信哩,我家去说声,免得他记挂,再同妈来与爹娘磕头罢。”西门庆道:“也罢,
我不留你,你且家去说声着。”月娘道:“桂姐,你吃了饭去。”桂姐道:“娘,
我不吃饭了。”一面又拜辞西门庆与月娘众人。临去,西门庆说道:“事便完了,
你今后,这王三官儿也少招揽他了。”桂姐道:“爹说的是甚么话,还招揽他哩!
再要招揽他,就把身子烂化了。就是前日,也不是我招揽他。”月娘道:“不招揽
他就是了,又平白说誓怎的?”一面叫轿子,打发桂姐去了。西门庆因告月娘说要
上东京之事。月娘道:“既要去,须要早打点,省得临时促忙促急。”西门庆道:
“蟒袍锦绣、金花宝贝,上寿礼物,俱已完备,倒只是我的行李不曾整备。”月娘
道:“行李不打紧。”西门庆说毕,就到前边看李瓶儿去了。到次日,坐在卷棚内
,叫了陈敬济来,看着写了蔡御史的书,交与来保,又与了他盘缠,叫他明日起早
赶往扬州去,不题。

倏忽过了数日,看看与蔡太师寿诞将近,只得择了吉日,吩咐琴童、玳安、书
童、画童四个小厮跟随,各各收拾行李。月娘同玉楼、金莲众人,将各色礼物并冠
带衣服应用之物,共装了二十余扛。头一日晚夕,妻妾众人摆设酒肴和西门庆送行
。吃完酒,就进月娘房里宿歇。次日,把二十扛行李先打发出门,又发了一张通行
马牌,仰经过驿递起夫马迎送。各各停当,然后进李瓶儿房里来,看了官哥儿,与
李瓶儿说道:“你好好调理。要药,叫人去问任医官讨。我不久便来家看你。”那
李瓶儿阁着泪道:“路上小心保重。”直送出厅来,和月娘、玉楼、金莲打伙儿送
了出大门。西门庆乘了凉轿,四个小厮骑了头口,望东京进发。迤逦行来,免不得
朝登紫陌,夜宿邮亭,一路看了些山明水秀,相遇的无非都是各路文武官员进京庆
贺寿诞,生辰扛不计其数。约行了十来日,早到东京。进了万寿城门,那时天色将
晚,赶到龙德街牌楼底下,就投翟家屋里去住歇。

那翟管家闻知西门庆到了,忙出来迎接,各叙寒暄。吃了茶,西门庆叫玳安将
行李一一交盘进翟家来。翟谦交府干收了,就摆酒和西门庆洗尘。不一时,只见剔
犀官桌上,摆上珍羞美味来,只好没有龙肝凤髓罢了,其余般般俱有,便是蔡太师
自家受用,也不过如此。当值的拿上酒来,翟谦先滴了天,然后与西门庆把盏。西
门庆也回敬了。两人坐下,糖果按酒之物,流水也似递将上来。酒过两巡,西门庆
便对翟谦道:“学生此来,单为与老太师庆寿,聊备些微礼孝顺太师,想不见却。
只是学生久有一片仰高之心,欲求亲家预先禀过:但得能拜在太师门下做个干生子
,便也不枉了人生一世。不知可以启口么?”翟谦道:“这个有何难哉!我们主人
虽是朝廷大臣,却也极好奉承。今日见了这般盛礼,不惟拜做干子,定然允从,自
然还要升选官爵。”西门庆听说,不胜之喜。饮够多时,西门庆便推不吃酒了。翟
管家道:“再请一杯,怎的不吃了?”西门庆道:“明日有正经事,不敢多饮。”
再四相劝,只又吃了一杯。

翟管家赏了随从人酒食,就请西门庆到后边书房里安歇。排下暖床绡帐,银钩
锦被,香喷喷的。一班小厮扶侍西门庆脱衣上床。独宿──西门庆一生不惯,那一
晚好难捱过。巴到天明,正待起身,那翟家门户重重掩着。直挨到巳牌时分,才有
个人把钥匙一路开将出来。随后才是小厮拿手巾香汤进书房来。西门庆梳洗完毕,
只见翟管家出来和西门庆厮见,坐下。当值的就托出一个朱红盒子来,里边有三十
来样美味,一把银壶斟上酒来吃早饭。翟谦道:“请用过早饭,学生先进府去和主
翁说知,然后亲家搬礼物进来。”西门庆道:“多劳费心!”酒过数杯,就拿早饭
来吃了,收过家活。翟管家道:“且权坐一回,学生进府去便来。”

翟谦去不多时,就忙来家,向西门庆说:“老爷正在书房梳洗,外边满朝文武
官员都伺候拜寿,未得厮见哩。学生已对老爷说过了,如今先进去拜贺罢,省的住
回人杂。学生先去奉候,亲家就来罢了。”说毕去了。西门庆不胜欢喜。便教跟随
人拉同翟家几个伴当,先把那二十扛金银缎匹抬到太师府前,一行人应声去了。西
门庆即冠带,乘了轿来。只见乱哄哄,挨肩擦背,都是大小官员来上寿的。西门庆
远远望见一个官员,也乘着轿进龙德坊来。西门庆仔细一看,却认的是故人扬州苗
员外。不想那苗员外也望见西门庆,两个同下轿作揖,叙说寒温。原来这苗员外也
是个财主,他身上也现做着散官之职,向来结交在蔡太师门下,那时也来上寿,恰
遇了故人。当下,两个忙匆匆路次话了几句,问了寓处,分手而别。

西门庆来到太师府前,但见:

堂开绿野,阁起凌烟。门前宽绰堪旋马,阀阅嵬峨好竖旗。锦绣丛中
,风送到画眉声巧;金银堆里,日映出琪树花香。左右活屏风,一个个夷
光红拂;满堂死宝玩,一件件周鼎商彝。室挂明珠十二,黑夜里何用灯油
;门迎珠履三千,白日间尽皆名士。九州四海,大小官员,都来庆贺;六
部尚书,三边总督,无不低头。正是:除却万年天子贵,只有当朝宰相尊
。

西门庆恭身进了大门,翟管家接着,只见中门关着不开,官员都打从角门而入。西
门庆便问:“为何今日大事,却不开中门?”翟管家道:“中门曾经官家行幸,因
此人不敢走。”西门庆和翟谦进了几重门,门上都是武官把守,一些儿也不混乱。
见了翟谦,一个个都欠身问管家:“从何处来?”翟管家答道:“舍亲打山东来拜
寿老爷的。”说罢,又走过几座门,转几个弯,无非是画栋雕梁,金张甲第。隐隐
听见鼓乐之声,如在天上一般。西门庆又问道:“这里民居隔绝,那里来的鼓乐喧
嚷?”翟管家道:“这是老爷教的女乐,一班二十四人,都晓得天魔舞、霓裳舞、
观音舞。但凡老爷早膳、中饭、夜宴,都是奏的。如今想是早膳了。”西门庆听言
未了,又鼻子里觉得异香馥馥,乐声一发近了。翟管家道:“这里与老爷书房相近
了,脚步儿放松些。”

转个回廊,只见一座大厅,如宝殿仙宫。厅前仙鹤、孔雀种种珍禽,又有那琼
花、昙花、佛桑花,四时不谢,开的闪闪烁烁,应接不暇。西门庆还未敢闯进,交
翟管家先进去了,然后挨挨排排走到堂前。只见堂上虎皮交椅上坐一个大猩红蟒衣
的,是太师了。屏风后列有二三十个美女,一个个都是宫样妆束,执巾执扇,捧拥
着他。翟管家也站在一边。西门庆朝上拜了四拜,蔡太师也起身,就绒单上回了个
礼。──这是初相见了。落后,翟管家走近蔡太师耳边,暗暗说了几句话下来,西
门庆理会的是那话了,又朝上拜四拜,蔡太师便不答礼。──这四拜是认干爷,因
此受了。西门庆开言便以父子称呼道:“孩儿没恁孝顺爷爷,今日华诞,特备的几
件菲仪,聊表千里鹅毛之意。愿老爷寿比南山。”蔡太师道:“这怎的生受!”便
请坐下。当值的拿了把椅子上来,西门庆朝上作了个揖道:“告坐了。”就西边坐
地吃茶。翟管家慌跑出门来,叫抬礼物的都进来。须臾,二十扛礼物摆列在阶下。
揭开了凉箱盖,呈上一个礼目:大红蟒袍一套、官绿龙袍一套、汉锦二十匹、蜀锦
二十匹、火浣布二十匹、西洋布二十匹,其余花素尺头共四十匹、狮蛮玉带一围、
金镶奇南香带一围、玉杯犀杯各十对、赤金攒花爵杯八只、明珠十颗,又另外黄金
二百两,送上蔡太师做贽见礼。蔡太师看了礼目,又瞧见抬上二十来扛,心下十分
欢喜,说了声“多谢!”便叫翟管家收进库房去了。一面吩咐摆酒款待。西门庆因
见他忙冲冲,就起身辞蔡太师。太师道:“既如此,下午早早来罢。”西门庆又作
个揖,起身出来。蔡太师送了几步,便不送了。西门庆依旧和翟管家同出府来。翟
管家府内有事,也作别进去。

西门庆竟回到翟家来,脱下冠带,已整下午饭,吃了一顿。回到书房,打了个
盹,恰好蔡太师差舍人邀请赴席,西门庆谢了些扇金,着先去了。即便重整冠带,
又叫玳安封下许多赏封,做一拜匣盛了,跟随着四个小厮,复乘轿望太师府来。蔡
太师那日满朝文武官员来庆贺的,各各请酒。自次日为始,分做三停:第一日是皇
亲内相,第二日是尚书显要、衙门官员,第三日是内外大小等职。只有西门庆,一
来远客,二来送了许多礼物,蔡太师到十分欢喜,因此就是正日独独请他一个。见
西门庆到了,忙走出轩下相迎。西门庆再四谦逊,让:“爷爷先行。”自家屈着背
,轻轻跨入槛内,蔡太师道:“远劳驾从,又损隆仪。今日略坐,少表微忱。”西
门庆道:“孩儿戴天履地,全赖爷爷洪福,些小敬意,何足挂怀!”两个喁喁笑语
,真似父子一般。二十四个美女,一齐奏乐,府干当值的斟上酒来。蔡太师要与西
门庆把盏,西门庆力辞不敢,只领的一盏,立饮而尽,随即坐了桌席。西门庆叫书
童取过一只黄金桃杯,斟上一杯,满满走到蔡太师席前,双膝跪下道:“愿爷爷千
岁!”蔡太师满面欢喜道:“孩儿起来。”接过便饮个完。西门庆才起身,依旧坐
下。那时相府华筵,珍奇万状,都不必说。西门庆直饮到黄昏时候,拿赏封赏了诸
执役人,才作谢告别道:“爷爷贵冗,孩儿就此叩谢,后日不敢再来求见了。”出
了府门,仍到翟家安歇。

次日,要拜苗员外,着玳安跟寻了一日,却在皇城后李太监房中住下。玳安拿
着帖子通报了,苗员外来出迎道:“学生正想个知心朋友讲讲,恰好来得凑巧。”
就留西门庆筵燕。西门庆推却不过,只得便住了。当下山肴海错不记其数。又有两
个歌童,生的眉清目秀,顿开喉音,唱几套曲儿。西门庆指着玳安、琴童向苗员外
说道:“这班蠢材,只会吃酒饭,怎地比的那两个!”苗员外笑道:“只怕伏侍不
的老先生,若爱时,就送上也何难!”西门庆谦谢不敢夺人之好。饮到更深,别了
苗员外,依旧来翟家歇。那几日内相府管事的,各各请酒,留连了八九日。西门庆
归心如箭,便叫玳安收拾行李。翟管家苦死留住,只得又吃了一夕酒,重叙姻亲,
极其眷恋。次日早起辞别,望山东而行。一路水宿风餐,不在话下。

且说月娘家中,自从西门庆往东京庆寿,姊妹每望眼巴巴,各自在屋里做些针
指,通不出来闲耍。只有潘金莲打扮的如花似玉,乔模乔样,在丫鬓伙里,或是猜
枚,或是抹牌,说也有,笑也有,狂的通没些成色。嘻嘻哈哈,也不顾人看见,只
想着与陈敬济勾搭。每日只在花园雪洞内踅来踅去,指望一时凑巧。敬济也一心想
着妇人,不时进来寻撞,撞见无人便调戏,亲嘴咂舌做一处,只恨人多眼多,不能
尽情欢会。正是:

虽然未入巫山梦,却得时逢洛水神。

一日,吴月娘、孟玉楼、李瓶儿同一处坐地,只见玳安慌慌跑进门来,见月娘
众人磕了头,报道:“爹回来了。”月娘便问:“如今在那里?”玳安道:“小的
一路骑头口,拿着马牌先行,因此先到家。爹这时节,也差不上二十里远近了。”
月娘道:“你曾吃饭没有?”玳安道:“从早上吃来,却不曾吃中饭。”月娘便吩
咐整饭伺候,一面就和六房姊妹同伙儿到厅上迎接。正是:

诗人老去莺莺在,公子归时燕燕忙。

妻妾每在厅上等候多时,西门庆方到门前下轿了,众妻妾一齐相迎进去。西门庆先
和月娘厮见毕,然后孟玉楼、李瓶儿、潘金莲依次见了,各叙寒温。落后,书童、
琴童、画童也来磕了头,自去厨下吃饭。西门庆把路上辛苦并到翟家住下、感蔡太
师厚情请酒并与内相日日吃酒事情,备细说了一遍。因问李瓶儿:“孩子这几时好
么?你身子吃的任医官药,有些应验么?我虽则往东京,一心只吊不下家里。”李
瓶儿道:“孩子也没甚事,我身子吃药后,略觉好些。”月娘一面收好行李及蔡太
师送的下程,一面做饭与西门庆吃。到晚又设酒和西门庆接风。西门庆晚夕就在月
娘房里歇了。两个是久旱逢甘雨,他乡遇故知。欢爱之情,俱不必说。

次日,陈敬济和大姐也来见了,说了些店里的帐目。应伯爵和常峙节打听的来
家,都来探望。西门庆出来相见毕,两个一齐说:“哥一路辛苦。”西门庆便把东
京富丽的事情及太师管待情分,备细说了一遍。两人只顾称羡不已。当日,西门庆
留二人吃了一日酒。常峙节临起身向西门庆道:“小弟有一事相求,不知哥可照顾
么?”说着,只是低了脸,半含半吐。西门庆道:“但说不妨。”常峙节道:“实
为住的房子不方便,待要寻间房子安身,却没有银子。因此要求哥周济些儿,日后
少不的加些利钱送还哥。”西门庆道:“相处中说甚利钱!只我如今忙忙的,那讨
银子?且待韩伙计货船来家,自有个处。”说罢,常峙节、应伯爵作谢去了,不在
话下。

且说苗员外自与西门庆相会,在酒席上把两个歌童许下。不想西门庆归心如箭
,不曾别的他,竟自归来。苗员外还道西门庆在京,差伴当来翟家问,才晓得西门
庆家去了。苗员外自想道:“君子一言,快马一鞭。我既许了他,怎么失信!”于
是叫过两个歌童吩咐道:“我前日请山东西门大官人,曾把你两个许下他。我如今
就要送你到他家去,你们早收拾行李。”那两个歌童一齐跪告道:“小的每伏侍的
员外多年,员外不知费尽多少心力,教的俺每这些南曲,却不留下自家欢乐,怎地
到送与别人?”说罢,扑簌簌掉下泪来。那员外也觉惨然不乐,说道:“你也说的
是,咱何苦定要送人?只是:‘人而无信,不知其可也。’──那孔圣人说的话怎
么违得!如今也由不得你了,待咱修书一封,差人送你去,教他好生看觑你就是了
。”两个歌童违拗不过,只得应诺起来。苗员外就叫那门管先生写着一封书信,写
那相送歌童之意。又写个礼单儿,把些尺头书帕封了,差家人苗实赍书,护送两个
歌童往西门庆家来。两个歌童洒泪辞谢了员外,翻身上马,迤逦同望山东大道而来
。有日到了清河县,三人下马访问,一直迳到县牌坊西门庆家府里投下。

却说西门庆自从东京到家,每日忙不迭,送礼的,请酒的,日日三朋四友,以
此竟不曾到衙门里去。那日稍闲无事,才到衙门里升堂画卯,把那些解到的人犯,
同夏提刑一一审问一番。审问了半日,公事毕,方乘了一乘凉轿,几个牢子喝道,
簇拥来家。只见那苗实与两个歌童已是候的久了,就跟着西门庆的轿子,随到前厅
,跪下禀说:“小的是扬州苗员外有书拜候老爹。”随将书并礼物呈上。西门庆连
忙说道:“请起来。”一面打开副启,细细看了。见是送他歌童,心下喜之不胜,
说道:“我与你员外意外相逢,不想就蒙你员外情投意合。酒后一言,就果然相赠
,又不惮千里送来。你员外真可谓千金一诺矣。难得,难得!”两个歌童从新走过
,又磕了四个头,说道:“员外着小的们伏侍老爹,万求老爹青目!”西门庆道:
“你起来,我自然重用。”一面叫摆酒饭,管待苗实并两个歌童;一面整办厚礼─
─绫罗细软,修书答谢员外;一面就叫两个歌童,在于书房伺候。不想,韩道国老
婆王六儿,因见西门庆事忙,要时常通个信儿,没人往来,算计将他兄弟王经──
才十五六岁,也生得清秀──送来伏侍西门庆,也是这日进门。西门庆一例收下,
也叫在书房中伺候。

西门庆正在厅上分拨,忽伯爵走来。西门庆与他说知苗员外送歌童之事,就叫
玳安里面讨出酒菜儿来,留他坐,就叫两个歌童来唱南曲。那两个歌童走近席前,
并足而立,手执檀板,唱了一套《新水令》“小园昨夜放江梅”,果然是响遏行云
,调成白雪。伯爵听了,欢喜的打跌,赞说道:“哥的大福,偏有这些妙人儿送将
来。也难为这苗员外好情。”西门庆道:“我少不得寻重礼答他。”一面又与这歌
童起了两个名:一个叫春鸿,一个叫春燕。又叫他唱了几个小词儿,二人吃一回酒
,伯爵方才别去。正是:

风花弄影新莺啭,俱是筵前歌舞人。