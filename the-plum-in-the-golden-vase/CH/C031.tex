\chapter{琴童儿藏壶构衅~西门庆开宴为欢}

诗曰:

幽情怜独夜,花事复相催。
欲使春心醉,先教玉友来。
浓香犹带腻,红晕渐分腮。
莫醒沉酣恨,朝云逐梦回。

话说西门庆,次日使来保提刑所下文书。一面使人做官帽,又唤赵裁裁剪尺头
,攒造衣服,又叫许多匠人,钉了七八条带。不说西门庆家中热乱,且说吴典恩那
日走到应伯爵家,把做驿丞之事,再三央及伯爵,要问西门庆错银子,上下使用,
许伯爵十两银子相谢,说着跪在地下。慌的伯爵拉起,说道:“此是成人之美,大
官人携带你得此前程,也不是寻常小可。”因问:“你如今所用多少够了?”吴典
恩道:“不瞒老兄说,我家活人家,一文钱也没有。到明日上任参官贽见之礼,连
摆酒,并治衣类鞍马,少说也得七八十两银子。如今我写了一纸文书此,也没敢下
数儿。望老兄好歹扶持小人,事成恩有重报。”伯爵看了文书,因说:“吴二哥,
你借出这七八十两银子来也不够使。依我,取笔来写上一百两。恒是看我面,不要
你利钱,你且得手使了。到明日做了官,慢慢陆续还他也不迟。俗语说得好:借米
下得锅,讨米下不得锅。哄了一日是两晌。”吴典恩听了,谢了又谢。于是把文书
上填写了一百两之数。

两个吃了茶,一同起身,来到西门庆门首。平安儿通报了,二人进入里面,见
有许多裁缝匠人七手八脚做生活。西门庆和陈敬济在穿廊下,看着写见官手本揭帖
,见二人,作揖让坐。伯爵问道:“哥的手本札付,下了不曾?”西门庆道:“今
早使小价往提刑府下札付去了。还有东平府并本县手本,如今正要叫贲四去下。”
说毕,画童儿拿上茶来。吃毕茶,那应伯爵并不提吴主管之事,走下来且看匠人钉
带。西门庆见他拿起带来看,就卖弄说道:“你看我寻的这几条带如何?”伯爵极
口称赞夸奖道:“亏哥那里寻的,都是一条赛一条的好带,难得这般宽大。别的倒
也罢了,自这条犀角带并鹤顶红,就是满京城拿着银子也寻不出来。不是面奖,就
是东京卫主老爷,玉带金带空有,也没这条犀角带。这是水犀角,不是旱犀角。旱
犀角不值钱。水犀角号作通天犀。你不信,取一碗水,把犀角放在水内,分水为两
处,此为无价之宝。”因问:“哥,你使了多少银子寻的?”西门庆道:“你们试
估估价值。”伯爵道:“这个有甚行款,我每怎么估得出来!”西门庆道:“我对
你说了罢,此带是大街上王昭宣府里的带。昨日一个人听见我这里要,巴巴来对我
说。我着贲四拿了七十两银子,再三回了来。他家还张致不肯,定要一百两。”伯
爵道:“难得这等宽样好看。哥,你明日系出去,甚是霍绰。就是你同僚间,见了
也爱。”夸美了一回,坐下。西门庆便向吴主管问道:“你的文书下了不曾?”伯
爵道:“吴二哥正要下文书,今日巴巴的央我来激烦你。蒙你照顾他往东京押生辰
担,虽是太师与了他这个前程,就是你抬举他一般,也是他各人造化。说不的,一
品至九品都是朝廷臣子。但他告我说,如今上任,见官摆酒,并治衣服之类,共要
许多银子使,那处活变去?一客不烦二主,没奈何,哥看我面,有银子借与他几两
,率性周济了这些事儿。他到明日做上官,就衔环结草也不敢忘了哥大恩!休说他
旧在哥门下出入,就是外京外府官吏,哥也不知拔济了多少。不然,你教他那里区
处去?”因说道:“吴二哥,你拿出那符儿来,与你大官人瞧。”这吴典恩连忙向
怀中取出,递与西门庆观看。见上面借一百两银子,中人就是应伯爵,每月利行五
分。西门庆取笔把利钱抹了,说道:“既是应二哥作保,你明日只还我一百两本钱
就是了。我料你上下也得这些银子搅缠。”于是把文书收了。才待后边取银子去,
忽有夏提刑拿帖儿差了一名写字的,拿手本三班送了二十名排军来答应,就问讨上
任日期,讨问字号,衙门同僚具公礼来贺。西门庆教阴阳徐先生择定七月初二日辰
时到任,拿帖儿回夏提刑,赏了写字的五钱银子。正打发出门去了,只见陈敬济拿
着一百两银子出来,教与吴主管,说:“吴二哥,你明日只还我本钱便了。”那吴
典恩拿着银子,欢喜出门。看官听说:后来西门庆死了,家中时败势衰,吴月娘守
寡,被平安儿偷盗出解当库头面,在南瓦子里宿娼,被吴驿丞拿住,教他指攀吴月
娘与玳安有奸,要罗织月娘出官,恩将仇报。此系后事,表过不题。正是:

不结子花休要种,无义之人不可交。

那时贲四往东平府并本县下了手本来回话,西门庆留他和应伯爵,陪阴阳徐先
生摆饭。正吃着饭,只见吴大舅来拜望,徐先生就起身。良久,应伯爵也作辞出门
,来到吴主管家。吴典恩早封下十两保头钱,双手递与伯爵,磕下头去。伯爵道:
“若不是我那等取巧说着,会胜不肯与借与你。”吴典恩酬谢了伯爵,治办官带衣
类,择日见官上任不题。

那时本县正堂李知县,会了四衙同僚,差人送羊酒贺礼来,又拿帖儿送了一名
小郎来答应。年方一十八岁,本贯苏州府常熟县人,唤名小张松。原是县中门子出
身,生得清俊,面如傅粉,齿白唇红;又识字会写,善能歌唱南曲;穿着青绡直缀
,凉鞋净袜。西门庆一见小郎伶俐,满心欢喜,就拿拜帖回覆李知县,留下他在家
答应,改唤了名字叫作书童儿。与他做了一身衣服,新鞋新帽,不教他跟马,教他
专管书房,收礼帖,拿花园门钥匙。祝实念又举保了一个十四岁小厮来答应,亦改
名棋童,每日派定和琴童儿两个背书袋、夹拜帖匣跟马。

到了上任日期,在衙门中摆大酒席桌面,出票拘集三院乐工承应吹打弹唱。此
时李铭也夹在中间来了,后堂饮酒,日暮时分散归。每日骑着大白马,头戴乌纱,
身穿五彩洒线揉头狮子补子员领,四指大宽萌金茄楠香带,粉底皂靴,排军喝道,
张打着大黑扇,前呼后拥,何止十数人跟随,在街上摇摆。上任回来,先拜本府县
帅府都监,并清河左右卫同僚官,然后新朋邻舍,何等荣耀施为!家中收礼接帖子
,一日不断。正是:

白马红缨色色新,不来亲者强来亲。
时来顽铁生光彩,运去良金不发明。

西门庆自从到任以来,每日坐提刑院衙门中,升厅画卯,问理公事。光阴迅速
,不觉李瓶儿坐褥一月将满。吴大妗子、二妗子、杨姑娘、潘姥姥、吴大姨、乔大
户娘子,许多亲邻堂客女眷,都送礼来,与官哥儿做弥月。院中李桂姐、吴银儿见
西门庆做了提刑所千户,家中又生了子,亦送大礼,坐轿子来庆贺。西门庆那日在
前边大厅上摆设筵席,请堂客饮酒。春梅、迎春、玉箫、兰香都打扮起来,在席前
斟酒执壶。

原来西门庆每日从衙门中来,只到外边厅上就脱了衣服,教书童叠了,安在书
房中,止带着冠帽进后边去。到次日起来,旋使丫鬟来书房中取。新近收拾大厅西
厢房一间做书房,内安床几、桌椅、屏帏、笔砚、琴书之类。书童儿晚夕只在床脚
踏板上铺着铺睡。西门庆或在那房里歇,早晨就使出那房里丫鬟来前边取衣服。取
来取去,不想这小郎本是门子出身,生的伶俐清俊,与各房丫头打牙犯嘴惯熟,于
是暗和上房里玉箫两个嘲戏上了。那日也是合当有事,这小郎正起来,在窗户台上
搁着镜儿梳头,拿红绳扎头发。不料玉箫推开门进来,看见说道:“好贼囚,你这
咱还描眉画眼的,爹吃了粥便出来。”书童也不理,只顾扎包髻儿。玉箫道:“爹
的衣服叠了,在那里放着哩?”书童道:“在床南头安放着哩。”玉箫道:“他今
日不穿这一套。吩咐我教问你要那件玄色[囗扁]金补子、丝布员领、玉色衬衣穿
。”书童道:“那衣服在厨柜里。我昨日才收了,今日又要穿他。姐,你自开门取
了去。”那玉箫且不拿衣服,走来跟前看着他扎头,戏道:“怪贼囚,也象老婆般
拿红绳扎着头儿,梳的[髟丐]虚笼笼的!”因见他白滚纱漂白布汗褂儿上系着一
个银红纱香袋儿,一个绿纱香袋儿,就说道:“你与我这个银红的罢!”书童道:
“人家个爱物儿,你就要。”玉箫道:“你小厮家带不的这银红的,只好我带。”
书童道:“早是这个罢了,倘是个汉子儿,你也爱他罢?”被玉箫故意向他肩膀上
拧了一把,说道:“贼囚,你夹道卖门神──看出来的好画儿。”不由分说,把两
个香袋子等不的解,都揪断系儿,放在袖子内。书童道:“你子不尊贵,把人的带
子也揪断。”被玉箫发讪,一拳一把,戏打在身上。打的书童急了,说:“姐,你
休鬼混我,待我扎上这头发着!”玉箫道:“我且问你,没听见爹今日往那去?”
书童道:“爹今日与县中华主簿老爹送行,在皇庄薛公公那里摆酒,来家只怕要下
午时分,又听见会下应二叔,今日兑银子,要买对门乔大户家房子,那里吃酒罢了
。”玉箫道:“等住回,你休往那去了,我来和你说话。”书童道:“我知道。”
玉箫于是与他约会下,才拿衣服往后边去了。

少顷,西门庆出来,就叫书童,吩咐:“在家,别往那去了,先写十二个请帖
儿,都用大红纸封套,二十八日请官客吃庆官哥儿酒;教来兴儿买办东西,添厨役
茶酒,预备桌面齐整;玳安和两名排军送帖儿,叫唱的;留下琴童儿在堂客面前管
酒。”吩咐毕,西门庆上马送行去了。吴月娘众姊妹,请堂客到齐了,先在卷棚摆
茶,然后大厅上屏开孔雀,褥隐芙蓉,上坐。席间叫了四个妓女弹唱。果然西门庆
到午后时分来家,家中安排一食盒酒菜,邀了应伯爵和陈敬济,兑了七百两银子,
往对门乔大户家成房子去了。

堂客正饮酒中间,只见玉箫拿下一银执壶酒并四个梨、一个柑子,迳来厢房中
送与书童儿吃。推开门,不想书童儿不在里面,恐人看见,连壶放下,就出来了。
可霎作怪,琴童儿正在上边看酒,冷眼睃见玉箫进书房里去,半日出来,只知有书
童儿在里边,三不知叉进去瞧。不想书童儿外边去,不曾进来,一壶热酒和果子还
放在床底下。这琴童连忙把果子藏在袖里,将那一壶酒,影着身子,一直提到李瓶
儿房里。只见奶子如意儿和绣春在屋里看哥儿。琴童进门就问:“姐在那里?”绣
春道:“他在上边与娘斟酒哩。你问他怎的?”琴童儿道:“我有个好的儿,教他
替我收着。”绣春问他甚么,他又不拿出来。正说着,迎春从上边拿下一盘子烧鹅
肉、一碟玉米面玫瑰果馅蒸饼儿与奶子吃,看见便道:“贼囚,你在这里笑甚么,
不在上边看酒?”那琴童方才把壶从衣裳底下拿出来,教迎春:“姐,你与我收了
。”迎春道:“此是上边筛酒的执壶,你平白拿来做甚么?”琴童道:“姐,你休
管他。此是上房里玉箫,和书童儿小厮,七个八个,偷了这壶酒和些柑子、梨,送
到书房中与他吃。我赶眼不见,戏了他的来。你只与我好生收着,随问甚么人来抓
寻,休拿出来。我且拾了白财儿着!”因把梨和柑子掏出来与迎春瞧,迎春道:“
等住回抓寻壶反乱,你就承当?”琴童道:“我又没偷他的壶。各人当场者乱,隔
壁心宽,管我腿事!”说毕,扬长去了。迎春把壶藏放在里间桌子上,不题。

至晚,酒席上人散,查收家火,少了一把壶。玉箫往书房中寻,那里得来!问
书童,说:“我外边有事去,不知道。”那玉箫就慌了,一口推在小玉身上。小玉
骂道:“[入日]昏了你这淫妇!我后边看茶,你抱着执壶,在席间与娘斟酒。这
回不见了壶儿,你来赖我!”向各处都抓寻不着。良久,李瓶儿到房来,迎春如此
这般告诉:“琴童儿拿了一把进来,教我替他收着。”李瓶儿道:“这囚根子,他
做甚么拿进来?后边为这把壶好不反乱,玉箫推小玉,小玉推玉箫,急得那大丫头
赌身发咒,只是哭。你趁早还不快送进去哩,迟回管情就赖在你这小淫妇儿身上。
”那迎春方才取出壶,送入后边来。后边玉箫和小玉两个,正嚷到月娘面前。月娘
道:“贼臭肉,还敢嚷些甚么?你每管着那一门儿?把壶不见了!”玉箫道:“我
在上边跟着娘送酒,他守着银器家火。不见了,如今赖我。”小玉道:“大妗子要
茶,我不往后边替他取茶去?你抱着执壶儿,怎的不见了?敢屁股大──吊了心也
怎的?”月娘道:“今日席上再无闲杂人,怎的不见了东西?等住回你主子来,没
这壶,管情一家一顿。”

正乱着,只见西门庆自外来,问:“因甚嚷乱?”月娘把不见壶一节说了一遍
。西门庆道:“慢慢寻就是了,平白嚷的是些甚么?”潘金莲道:“若是吃一遭酒
,不见了一把,不嚷乱,你家是王十万!头醋不酸,到底儿薄。”看官听说:金莲
此话,讥讽李瓶儿首先生孩子,满月就不见了壶,也是不吉利。西门庆明听见,只
不做声。只见迎春送壶进来。玉箫便道:“这不是壶有了。”月娘问迎春:“这壶
端的往那里来?”迎春悉把琴童从外边拿到我娘屋里收着,不知在那里来。月娘因
问:“琴童儿那奴才,如今在那里?”玳安道:“他今日该狮子街房子里上宿去了
。”金莲在旁不觉鼻子里笑了一声。西门庆便问:“你笑怎的?”金莲道:“琴童
儿是他家人,放壶他屋里,想必要瞒昧这把壶的意思。要叫我,使小厮如今叫将那
奴才来,老实打着,问他个下落。不然,头里就赖着他那两个,正是走杀金刚坐杀
佛!”西门庆听了,心中大怒,睁眼看着金莲,说道:“依着你恁说起来,莫不李
大姐他爱这把壶?既有了,丢开手就是了,只管乱甚么!”那金莲把脸羞的飞红了
,便道:“谁说姐姐手里没钱。”说毕,走过一边使性儿去了。

西门庆就有陈敬济进来说话。金莲和孟玉楼站在一处,骂道:“恁不逢好死,
三等九做贼强盗!这两日作死也怎的?自从养了这种子,恰似生了太子一般,见了
俺每如同生刹神一般,越发通没句好话儿说了,行动就睁着两个[毛必]窟窿吆喝
人。谁不知姐姐有钱,明日惯的他每小厮丫头养汉做贼,把人说遍了,也休要管他
!”说着,只见西门庆与陈敬济说了一回话,就往前边去了。孟玉楼道:“你还不
去,他管情往你屋里去了。”金莲道:“可是他说的,有孩子屋里热闹,俺每没孩
子的屋里冷清。”正说着,只见春梅从外走来。玉楼道:“我说他往你屋里去了,
你还不信,这不是春梅叫你来了。”一面叫过春梅来问。春梅道:“我来问玉箫要
汗巾子来。”玉楼问道:“你爹在那里?”春梅道:“爹往六娘房里去了。”这金
莲听了,心上如撺上把火相似,骂道:“贼强人,到明日永世千年,就跌折脚,也
别要进我那屋里!踹踹门槛儿,教那牢拉的囚根子把踝子骨[扌歪]折了!”玉楼
道:“六姐,你今日怎的下恁毒口咒他?”金莲道:“不是这等说,贼三寸货强盗
,那鼠腹鸡肠的心儿,只好有三寸大一般。都是你老婆,无故只是多有了这点尿胞
种子罢了,难道怎么样儿的!做甚么恁抬一个灭一个,把人[足丽]到泥里!”正
是:

大风刮倒梧桐树,自有旁人说短长。

这里金莲使性儿不题。且说西门庆走到前边,薛大监差了家人,送了一坛内酒
、一牵羊、两匹金缎、一盘寿桃、一盘寿面、四样嘉肴,一者祝寿,二者来贺。西
门庆厚赏来人,打发去了。到后边,有李桂姐、吴银儿两个拜辞要家去。西门庆道
:“你每两个再住一日儿,到二十八日,我请许多官客,有院中杂耍扮戏的,教你
二位只管递酒。”桂姐道:“既留下俺每,我教人家去回妈声,放心些。”于是把
两人轿子都打发去了,不在话下。

次日,西门庆在大厅上锦屏罗列,绮席铺陈,请官客饮酒。因前日在皇庄见管
砖厂刘公公,故与薛内相都送了礼来。西门庆这里发柬请他,又邀了应伯爵、谢希
大两个相陪。从饭时,二人衣帽齐整,又早先到了。西门庆让他卷棚内待茶。伯爵
因问:“今日,哥席间请那几客?”西门庆道:“有刘、薛二内相,帅府周大人,
都监荆南江,敝同僚夏提刑,团练张总兵,卫上范千户,吴大哥,吴二哥。乔老便
今日使人来回了不来。连二位通只数客。”说毕,适有吴大舅、二舅到,作了揖,
同坐下,左右放桌儿摆饭。吃毕,应伯爵因问:“哥儿满月抱出来不曾?”西门庆
道:“也是因众堂客要看,房下说且休教孩儿出来,恐风试着他,他奶子说不妨事
。教奶子用被裹出来,他大妈屋里走了遭,应了个日子儿,就进屋去了。”伯爵道
:“那日嫂子这里请去,房下也要来走走,百忙里旧疾又举发了,起不得炕儿,心
中急的要不的。如今趁人未到,哥倒好说声,抱哥儿出来,俺每同看一看。”西门
庆一面吩咐后边:“慢慢抱哥儿出来,休要唬着他。对你娘说,大舅、二舅在这里
,和应二爹、谢爹要看一看。”月娘教奶子如意儿用红绫小被儿裹的紧紧的,送到
卷棚角门首,玳安儿接抱到卷棚内。众人观看,官哥儿穿着大红缎毛衫儿,生的面
白唇红,甚是富态,都夸奖不已。吴大舅、二舅与希大每人袖中掏出一方锦缎兜肚
,上带着一个小银坠儿;惟应伯爵是一柳五色线,上穿着十数文长命钱。教与玳安
儿好生抱回房去,休要惊唬哥儿,说道:“相貌端正,天生的就是个戴纱帽胚胞儿
。”西门庆大喜,作揖谢了。

说话中间,忽报刘公公、薛公公来了。慌的西门庆穿上衣,仪门迎接。二位内
相坐四人轿,穿过肩蟒,缨枪排队,喝道而至。西门庆先让至大厅上拜见,叙礼接
茶。落后周守备、荆都监、夏提刑等众武官都是锦绣服,藤棍大扇,军牢喝道。须
臾都到了门首,黑压压的许多伺候。里面鼓乐喧天,笙歌迭奏。西门庆迎入,与刘
、薛二内相相见。厅正面设十二张桌席。西门庆就把盏让坐。刘、薛二内再三让逊
道:“还有列位。”只见周守备道:“二位老太监齿德俱尊。常言:三岁内宦,居
冠王公之上。这个自然首坐,何消泛讲。”彼此让逊了一回。薛内相道:“刘哥,
既是列位不肯,难为东家,咱坐了罢。”于是罗圈唱了个喏,打了恭,刘内相居左
,薛内相居右,每人膝下放一条手巾,两个小厮在旁打扇,就坐下了。其次者才是
周守备、荆都监众人。须臾阶下一派箫韶,动起乐来。当日这筵席,说不尽食烹异
品,果献时新。须臾酒过五巡,汤陈三献,教坊司俳官簇拥一段笑乐院本上来。正
是:

百宝妆腰带,珍珠络臂鞲。
笑时能近眼,舞罢锦缠头。

笑院本扮完下去,就是李铭、吴惠两个小优儿上来弹唱。一个[扌栾]筝,一
个琵琶。周守备先举手让两位内相,说:“老太监吩咐,赏他二人唱那套词儿?”
刘太监道:“列位请先。”周守备道:“老太监,自然之理,不必过谦。”刘太监
道:“两个子弟唱个‘叹浮生有如一梦里’。”周守备道:“老太监,此是归隐叹
世之辞,今日西门庆大人喜事,又是华诞,唱不的。”刘太监又道:“你会唱‘虽
不是八位中紫绶臣,管领的六宫中金钗女’?”周守备道:“此是《陈琳抱妆盒》
杂记,今日庆贺,唱不的。”薛太监道:“你叫他二人上来,等我吩咐他。你记的
《普天乐》‘想人生最苦是离别’?”夏提刑大笑道:“老太监,此是离别之词,
越发使不的。”薛太监道:“俺每内官的营生,只晓的答应万岁爷,不晓得词曲中
滋味,凭他每唱罢。”夏年刑终是金吾执事人员,倚仗他刑名官,遂吩咐:“你唱
套《三十腔》。今日是你西门老爹加官进禄,又是好日子,又是弄璋之喜,宜该唱
这套。”薛内相问:“怎的是弄璋之喜?”周守备道:“二位老太监,此日又是西
门大人公子弥月之辰,俺每同僚都有薄礼庆贺。”薛内相道:“这等──”因向刘
太监道:“刘家,咱每明日都补礼来庆贺。”西门庆谢道:“学生生一豚犬,不足
为贺,到不必老太监费心。”说毕,唤玳安里边叫出吴银儿、李桂姐,席前递酒。
两个唱的打扮出来,花枝招展,望上插烛也似磕了四个头儿,起来执壶斟酒,逐一
敬奉。两个乐工,又唱一套新词,歌喉宛转,真有绕梁之声。当夜前歌后舞,锦簇
花攒,直饮至更余时分,薛内相方才起身,说道:“生等一者过蒙盛情,二者又值
喜庆,不觉留连畅饮,十分扰极,学生告辞。”西门庆道:“杯茗相邀,得蒙光降
,顿使蓬荜增辉,幸再宽坐片时,以毕余兴。”众人俱出位说道:“生等深扰,酒
力不胜。”各躬身施礼相谢。西门庆再三款留不住,只得同吴大舅、二舅等,一齐
送至大门。一派鼓乐喧天,两边灯火灿烂,前遮后拥,喝道而去。正是,得多少:

歌舞欢娱嫌日短,故烧高烛照红妆。