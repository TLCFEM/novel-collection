\chapter{来旺偷拐孙雪娥~雪娥受辱守备府}

诗曰:

菟丝附蓬麻,引蔓原不长。
失身与狂夫,不如弃道旁。
暮夜为侬好,席不暖侬床。
昏来晨一别,无乃太匆忙。
行将滨死地,老痛迫中肠。

话说吴大舅领着月娘等一簇男女,离了永福寺,顺着大树长堤前来。玳安又早在杏
花酒楼下边,人烟热闹,拣高阜去处,幕天席地设下酒肴,等候多时了。远远望月
娘众人轿子驴子到了,问道:“如何这咱才来?”月娘又把永福寺中遇见春梅告诉
一遍。不一时斟上酒来。众人坐下正饮酒,只见楼下香车绣毂往来,人烟喧杂。月
娘众人骊着高阜,把眼观看,只见人山人海围着,都看教师走马耍解。

原来是本县知县相公儿子李衙内,名唤李拱璧,年约三十余岁,见为国子上舍,一
生风流博浪,懒习诗书,专好鹰犬走马,打球蹴踘,常在三瓦两巷中走,人称他为
”李棍子”。那日穿着一弄儿轻罗软滑衣裳,头戴金顶缠棕小帽,脚踏乾黄靴,同
廊吏何不韦带领二三十好汉,拿弹弓、吹筒、球棒在于杏花村大酒楼下,看教师李
贵走马卖解,竖肩桩、隔肚带,轮枪舞棒,做各样技艺顽耍,引了许多男女围着烘
笑。那李贵诨名为山东夜叉,头带万字巾,身穿紫窄衫,销金裹肚,坐下银鬃马,
手执朱红杆明枪,背插招风令字旗,在街心扳鞍上马,往来卖弄手段。这李衙内正
看处,忽抬头看见一簇妇人在高阜处饮酒,内中一个长挑身材妇人,不觉心摇目荡
,观之不足,看之有余,口中不言,心内暗道:“不知是谁家妇女,有男子汉没有
?”一面叫过手下答应的小张闲架儿来,悄悄分付:“你去那高坡上,打听那三个
穿白的妇人是谁家的。访得的实,告我知道。”那小张闲应诺,云飞跑去。不多时
,走到跟前附耳低言回报说:“如此这般,是县门前西门庆家妻小。一个年老的姓
吴,是他妗子;一个五短身材,是他大娘子吴月娘;那个长挑身材,有白麻子的,
是第三个娘子,姓孟,名玉楼;如今都守寡在家。”这李衙内听了,独看上孟玉楼
,重赏小张闲,不在话下。

吴月娘和大舅众人观看了半日,见日色衔山,令玳安收拾了食盒,上轿骑驴一径回
家。有诗为证:
柳底花阴压路尘,一回游赏一回新。
有缘千里来相会,无缘对面不相亲。

这里月娘众人回家不题。却说那日,孙雪娥与西门大姐在家,午后时分无事,都出
大门首站立。也是天假其便,不想一个摇惊闺的过来。那时卖脂粉、花翠生活,磨
镜子,都摇惊闺。大姐说:“我镜子昏了。”使平安儿:“叫住那人,与我磨镜子
。”那人放下担儿,说道:“我不会磨镜子,我只卖些金银生活,首饰花翠。”站
立在门前,只顾眼上眼下看着雪娥。雪娥便道:“那汉子,你不会磨镜子,去罢,
只顾看我怎的!”那人说:“雪姑娘,大姑娘,不认的我了?”大姐道:“眼熟,
急忙想不起来。”那人道:“我是爹手里出去的来旺儿。”雪娥便道:“你这几年
在那里来?出落得恁胖了。”来旺儿道:“我离了爹门,到原籍徐州,家里闲着没
营生,投跟了老爹上京来做官。不想到半路里,他老爷儿死了,丁忧家去了。我便
投在城内顾银铺,学会了此银行手艺,各样生活。这两日行市迟,顾银铺教我挑副
担儿,出来街上发卖些零碎。看见娘每在门首,不敢来相认,恐怕踅门瞭户的。今
日不是你老人家叫住,还不敢相认。”雪娥道:“原来是你。教我只顾认了半日,
白想不起。既是旧儿女,怕怎的?”因问:“你担儿里卖的是甚么生活?挑进里面
,等俺每看一看。”那来旺儿一面把担儿挑入里边院子里来。打开箱子,用箧儿托
出几样首饰来:金银镶嵌不等,打造得十分奇巧。大姐与雪娥看了一回,问来旺儿
:“你还有花翠,拿出来。”这孙雪娥便留了他一对翠凤,一对柳穿金鱼儿。大姐
便称出银子来与他。雪娥两样生活,欠他一两二钱银子,约下他:“明日早来取罢
。今日你大娘不在家,和你三娘和哥儿都往坟上与你爹烧纸去了。”来旺道:“我
去年在家里,就听见人说爹死了。大娘生了哥儿,怕不的好大了。”雪娥道:“你
大娘孩儿如今才周半儿。一家儿大大小小,如宝上珠一般,全看他过日子哩。”说
话中间,来昭妻一丈青出来,倾了盏茶与他吃,那来旺儿接了茶,与他唱了个喏。
来旺也在跟前,同叙了回话。分付:“你明日来见见大娘。”那来旺儿挑担出门。

到晚上,月娘众人轿子来家。雪娥、大姐、众人丫头接着,都磕了头。玳安跟盒担
走不上,雇了匹驴儿骑来家,打发抬盒人去了。月娘告诉雪娥、大姐,说今日寺里
遇见春梅一节:“原来他把潘家的就葬在寺后首,俺每也不知。他来替他娘烧纸,
误打误撞遇见他。娘儿每又认了回亲。先是寺里长老摆斋吃了。落后他又教伴当摆
上他家的四五十攒盒,各样菜蔬下饭,筛酒上来,通吃不了。他看见哥儿,又与了
他一对簪儿,好不和气。起解行三坐五,坐着大轿子,许多跟随。又且是出落的比
旧时长大了好些,越发白胖了。”吴大妗子道:“他倒也不改常忘旧。那时在咱家
时,我见他比众丫鬟行事儿正大,说话儿沉稳,就是个才料儿。你看今日福至心灵
,恁般造化。”孟玉楼道:“姐姐没问他,我问他来。果然半年没洗换,身上怀着
喜事哩。也只是八九月里孩子,守备好不喜欢哩。薛嫂儿说的倒不差。”说了一回
,雪娥题起:“今日娘不在,我和大姐在门首,看见来旺儿。原来他又在这里学会
了银匠,挑着担儿卖金银生活花翠。俺每就不认得了,买了他几枝花翠,他问娘来
,我说往坟上烧纸去了。”月娘道:“你怎的不教他等着我来家?”雪娥道:“俺
每教他明日来。”

正坐着说话,只见奶子如意儿向前对月娘说:“哥儿来家这半日,只是昏睡不醒,
口中出冷气,身上汤烧火热的。”这月娘听见慌了,向炕上抱起孩儿来,口揾着口
儿,果然出冷汗,浑身发热,骂如意儿:“好淫妇,此是轿子冷了孩儿了。”如意
儿道:“我拿小被儿裹的紧紧的,怎得冻着?”月娘道:“再不是抱了往那死鬼坟
上,唬了他来了。那等分付教你休抱他去,你不依,浪着抱的去了。”如意儿道:
”早小玉姐姐看着,只抱了他那里看看就来了,几时唬着他来!”月娘道:“别要
说嘴,看那看儿便怎的?却把他唬了。”急忙叫来安儿:“快请刘婆子去。”不一
时,刘婆来到。看了脉息,摸了身上,说:“着了些凉寒,撞见邪祟了。”留了两
服朱砂丸,用姜汤灌下去。分付奶子抱着他,热炕上睡到半夜,出了些冷汗,身上
才凉了。于是管待刘婆子吃了茶,与了他三钱银子,叫他明日还来看看。一家子慌
的要不的,起起倒倒,整乱了半夜。

却说来旺,次日依旧挑将生活担儿,来到西门庆门首,与来昭唱喏,说:“昨日雪
姑娘留下我些生活,许下今日教我来取银子,就见见大娘。”来昭道:“你且去着
,改日来。昨日大娘来家,哥儿不好,叫医婆、太医看,下药,整乱了一夜,好不
心,今日才好些,那得工夫称银子与你。”正说着,只见月娘、玉楼、雪娥送出刘
婆子,来到大门首,看见来旺儿。那来旺儿扒在地下,与月娘、玉楼磕下两个头。
月娘道:“几时不见你,就不来这里走走。”来旺儿悉将前事说了一遍,”要来不
好来的。”月娘道:“旧儿女人家,怕怎的?你爹又没了。当初只因潘家那淫妇,
一头放火,一头放水,架的舌,把个好媳妇儿生生逼勒的吊死了,将有作没,把你
垫发了去。今日天也不容,他往那去了!”来旺儿道:“也说不的,只是娘心里明
白就是了。”说了回话,月娘问他:“卖的是甚样生活?拿出来瞧。”拣了他几件
首饰,该还他三两二钱银子,都用等子称了与他。叫他进入仪门里面,分付小玉取
一壶酒来,又是一盘点心,教他吃。那雪娥在厨上一力撺掇,又热了一大碗肉出来
与他。吃的酒饭饱了,磕头出门。月娘、玉楼众人归到后边去。雪娥独自悄悄和他
说话:“你常常来走着,怕怎的!奴有话教来昭嫂子对你说。我明日晚夕,在此仪
门里紫墙儿跟前耳房内等你。”两个递了眼色,这来旺儿就知其意,说:“这仪门
晚夕关不关?”雪娥道:“如此这般,你来先到来昭屋里,等到晚夕,踩着梯凳,
越过墙,顺着遮墙,我这边接你下来。咱二人会合一回,还有细话与你说。”这来
旺得了此话,正是欢从额起,喜向腮生,作辞雪娥,挑担儿出门。正是:不着家神
,弄不得家鬼。有诗为证:
闲来无事倚门阑,偶遇多情旧日缘。
对人不敢高声语,故把秋波送几番。

这来旺儿欢喜来家,一宿无话。到次日,也不挑担儿出来卖生活,慢慢踅来西门庆
门首,等来昭出来与他唱喏。那来昭便说:“旺哥稀罕,好些时不见你了。”来旺
儿笑道:“不是也不来,里边雪姑娘少我几钱生活银,讨讨。”来昭一面把来旺儿
让到房里坐下。来旺儿道:“嫂子怎不见?”来昭道:“你嫂子今日后边上灶哩。
”那来旺儿拿出一两银子,递与来昭,说:“这银子取壶酒来,和哥嫂吃。”来昭
道:“何消这许多。”即叫他儿子铁棍儿过来。那铁棍吊起头去--十五岁了,拿
壶出来,打了一大注酒,使他后边叫一丈青来。不一时,一丈青盖了一锡锅热饭,
一大碗杂熬下饭,两碟菜蔬,说道:“好呀,旺官儿在这里。”来昭便拿出银子与
一丈青瞧,说:“兄弟破费,要打壶酒咱两口儿吃。”一丈青笑道:“无功消受,
怎生使得?”一面放了炕桌,让来旺炕上坐。摆下酒菜,把酒来斟。来旺儿先倾头
一盏,递与来昭,次递一盏与一丈青,深深唱喏,说:“一向不见哥嫂,这盏水酒
孝顺哥嫂。”一丈青便说:“哥嫂不道酒肉吃伤了!你对真人休说假话。里边雪姑
娘昨日已央及达知我了,你两个旧情不断,托俺每两口儿如此这般周全你。你休推
睡里梦里,要知山下路,须问过来人。你若入港相会,有东西出来,休要独吃,须
把些汁水教我呷一呷,俺替你每须耽许多利害。”那来旺便跪下说:“只望哥嫂周
全,并不敢有忘。”说毕,把酒吃了一回。一丈青往后边和雪娥答了话出来,对他
说,约定晚上来,来昭屋里窝藏,待夜里关上仪门,后边人歇下,越墙而过,于中
取事。有诗为证:
报应本无私,影响皆相似。
要知祸福因,但看所为事。

这来旺得了此言,回来家,巴不到晚,踅到来昭屋里,打酒和他两口儿吃。至更深
时分,更无一人觉的,直待的大门关了,后边仪门上了拴,家中大小歇息定了,彼
此都有个暗号儿,只听墙内雪娥咳嗽之声。这来旺儿踏着梯凳,黑暗中扒过粉墙,
雪娥那边用凳子接着。两个就在西耳房堆马鞍子去处,两个相搂相抱,云雨做一处
。彼此都是旷夫寡妇,欲心如火。那来旺儿缨枪强壮,尽力弄了一回,乐极精来,
一泄如注。干毕,雪娥递与他一包金银首饰,几两碎银子,两件段子衣服,分付:
”明日晚夕你再来,我还有些细软与你。你外边寻下安身去处。往后这家中过不出
好来,不如和你悄悄出去,外边寻下房儿,成其夫妇。你又会银行手艺,愁过不得
日子?”来旺儿便说:“如今东门外细米巷,有我个姨娘,有名收生的屈老娘。你
那里曲弯小巷,倒避眼,咱两个投奔那里去。迟些时,看无动静,我带你往原籍家
里,买几亩地种去也好。”两个商量已定。这来旺就作别雪娥,依旧扒过墙来,到
来昭屋里。等至天明,开了大门,挨身出去。到黄昏时分,又来门首,踅入来昭屋
里。晚夕依旧跳过墙去,两个干事。朝来暮往,非止一日,也抵盗了许多细软东西
,金银器皿,衣服之类。来昭两口子也得抽分好些肥己,俱不必细说。

一日,后边月娘看孝哥儿出花儿,心中不快,睡得早。这雪娥房中使女中秋儿,原
是大姐使的,因李娇儿房中元宵儿被敬济要了,月娘就把中秋儿与了雪娥,把元宵
儿伏侍大姐。那一日,雪娥打发中秋儿睡下,房里打点一大包钗环头面,装在一个
匣内,用手帕盖了头,随身衣服,约定来旺儿在来昭屋里等候,两个要走。来昭便
说:“不争你走了,我看守大门,管放水鸭儿!若大娘知道,问我要人怎的?不如
你每打房上去,就骊破些瓦,还有踪迹。”来旺儿道:“哥也说得是。”雪娥又留
一个银折盂,一根金耳斡,一件青绫袄,一条黄绫裙,谢了他两口儿。直等五更鼓
,月黑之时,隔房扒过去。来昭夫妇又筛上两大钟暖酒,与来旺、雪娥吃,说:“
吃了好走,路上壮胆些。”吃到五更时分,每人拿着一根香,骊着梯子,打发两个
扒上房去,一步一步把房上瓦也跳破许多。比及扒到房檐跟前,街上人还未行走,
听巡捕的声音,这来旺儿先跳下去,后却教雪娥骊着他肩背,接搂下来。两个往前
边走,到十字路口上,被巡捕的拦住,便问:“往那里去的男女?”雪娥便唬慌了
手脚。这来旺儿不慌不忙,把手中官香弹了一弹,说道:“俺是夫妇二人,前往城
外岳庙里烧香,起的早了些,长官勿怪。”那人问:“背的包袱内是甚么?”来旺
儿道:“是香烛纸马。”那人道:“既是两口儿岳庙烧香,也是好事,你快去罢。
”这来旺儿得不的一声,拉着雪娥,往前飞走。走到城下,城门才开。打人闹里挨
出城去,转了几条街巷。

原来细米巷在个僻静去处,住着不多几家人家,都是矮房低厦。到于屈姥姥家,屈
姥姥还未开门。叫了半日,屈姥姥才起来开了门,见来旺儿领了个妇人来。原来来
旺儿本姓郑,名唤郑旺,说:“这妇人是我新寻的妻小。姨娘这里有房子,且借一
间,寄住些时,再寻房子。”递与屈姥姥三两银子,教买柴米。那屈姥姥得了银子
,只得留下。他儿子屈铛,因见郑旺夫妻二人,带着许多金银首饰东西,夜晚见财
起意,就掘开房门偷盗出来去耍钱,致被捉获,具了事件,拿去本县见官。李知县
见系贼赃之事,赃物见在,即差人押着屈铛到家,把郑旺、孙雪娥一条索子都拴了
。那雪娥唬的脸蜡黄也似黄了,换了渗淡衣裳,带着眼纱,把手上戒指都勒下来打
发了公人,押去见官。当下烘动了一街人观看,有认得的,说是西门庆家小老婆,
今被这走出的小厮来旺儿--改名郑旺通奸,拐盗财物在外居住。又被这屈铛掏摸
了,今事发见官。当下一个传十个,十个传百个,路上行人口似飞。

月娘家中自从雪娥走了,房中中秋儿见箱内细软首饰都没了,衣服丢的乱三搅四,
报与月娘。月娘吃了一惊,便问中秋儿:“你跟着他睡,走了,你岂不知?”中秋
儿便说:“他要便晚夕悄悄偷走出外边,半日方回,不知详细。”月娘又问来昭:
”你看守大门,人出去你怎不晓的?”来昭便说:“大门每日上锁,莫不他飞出去
!”落后看见房上瓦骊破许多,方知越房而去了。又不敢使人骊访,只得按纳含忍
。不想本县知县当堂理问这件事,先把屈铛夹了一顿,追出金头面四件,银首饰三
件,金环一双,银钟二个,碎银五两,衣服二件,手帕一个,匣一个。向郑旺名下
追出银三十两,金碗簪一对,金仙子一件,戒指四个。向雪娥名下追出金挑心一件
,银镯一付,金钮五付,银簪四对,碎银一包。屈姥姥名下追出银三两。就将来旺
儿问拟奴婢因奸盗取财物,屈铛系窃盗,俱系杂犯死罪,准徒五年,赃物入官。雪
娥孙氏系西门庆妾,与屈姥姥当下都当官拶了一拶。屈姥姥供明放了。雪娥责令本
县差人到西门庆家,教人递领状领孙氏。那吴月娘叫吴大舅来商议:“已是出丑,
平白又领了来家做甚么?没的玷污了家门,与死的装幌子。”打发了差人钱,回了
知县话。知县拘将官媒人来,当官辩卖。

却说守备府中,春梅打听得知,说西门庆家中孙雪娥如此这般,被来旺儿拐出,盗
了财物去在外居住,事发到官,如今当官辨卖。这春梅听见,要买他来家上灶,要
打他嘴,以报平昔之仇。对守备说:“雪娥善能上灶,会做的好茶饭汤水,买来家
中伏侍。”这守备即差张胜、李安。拿贴儿对知县说。知县自恁要做分上,只要八
两银子官价。交完银子,领到府中,先见了大奶奶并二奶奶孙氏,次后到房中来见
春梅。春梅正在房里缕金床上,锦帐之中,才起来。手下丫鬟领雪娥见面。那雪娥
见是春梅,不免低头进见。望上倒身下拜,磕了四个头。这春梅把眼瞪一瞪,唤将
当直的家人媳妇上来,”与我把这贱人撮去了(髟狄)髻,剥了上盖衣裳,打入厨
下,与我烧火做饭。”这雪娥听了,暗暗叫苦。自古世间打墙板儿翻上下,扫米却
做管仓人。既在他檐下,怎敢不低头?孙雪娥到此地步,只得摘了髻儿,换了艳服
,满脸悲恸,往厨下去了。有诗为证:
布袋和尚到明州,策杖芒鞋任处游。
饶你化身千百亿,一身还有一身愁。