\chapter{潘金莲香腮偎玉~薛姑子佛口谈经}

诗曰:

富贵如朝露,交游似聚沙。
不如竹窗里,对卷自趺跏。
静虑同聆偈,清神旋煮茶。
惟忧晓鸡唱,尘里事如麻。

话说西门庆搂抱潘金莲,一觉睡到天明。妇人见他那话还直竖一条棍相似,便
道:“达达,你饶了我罢,我来不得了。待我替你咂咂罢。”西门庆道:“怪小淫
妇儿,你若咂的过了,是你造化。”这妇人真个蹲向他腰间,按着他一只腿,用口
替他吮弄那话。吮够一个时分,精还不过,这西门庆用手按着粉项,往来只顾没棱
露脑摇撼,那话在口里吞吐不绝。抽拽的妇人口边白沫横流,残脂在茎。妇人一面
问西门庆:“二十八日应二家请俺每,去不去?”西门庆道:“怎的不去!”妇人
道:“我有桩事儿央你,依不依?”西门庆道:“怪小淫妇儿,你有甚事,说不是
。”妇人道:“你把李大姐那皮袄拿出来与我穿了罢。明日吃了酒回来,他们都穿
着皮袄,只奴没件儿穿。”西门庆道:“有王招宣府当的皮袄,你穿就是了。”妇
人道:“当的我不穿他,你与了李娇儿去。把李娇儿那皮袄却与雪娥穿。你把李大
姐那皮袄与了我,等我[扌寨]上两个大红遍地金鹤袖,衬着白绫袄儿穿,也是与
你做老婆一场,没曾与了别人。”西门庆道:“贼小淫妇儿,单管爱小便宜儿。他
那件皮袄值六十两银子哩,你穿在身上是会摇摆!”妇人道:“怪奴才,你与了张
三、李四的老婆穿了?左右是你的老婆,替你装门面,没的有这些声儿气儿的。好
不好我就不依了。”西门庆道:“你又求人又做硬儿。”妇人道:“怪硶货
,我是你房里丫头,在你跟前服软?”一面说着,把那话放在粉脸上只顾偎晃,良
久,又吞在口里挑弄蛙口,一回又用舌尖抵其琴弦,搅其龟棱,然后将朱唇裹着,
只顾动动的。西门庆灵犀灌顶,满腔春意透脑,良久精来,呼:“小淫妇儿,好生
裹紧着,我待过也!”言未绝,其精邈了妇人一口。妇人口口接着,都咽了。正是
:

自有内事迎郎意,殷勤爱把紫箫吹。

当日是安郎中摆酒,西门庆起来梳头净面出门。妇人还睡在被里,便说道:“
你趁闲寻寻儿出来罢。等住回,你又不得闲了。”这西门庆于是走到李瓶儿房中,
奶子、丫头又早起来顿下茶水供养。西门庆见如意儿薄施脂粉,长画蛾眉,笑嘻嘻
递了茶,在旁边说话儿。西门庆一面使迎春往后边讨床房里钥匙去,如意儿便问:
“爹讨来做甚么?”西门庆道:“我要寻皮袄与你五娘穿。”如意道:“是娘的那
貂鼠皮袄?”西门庆道:“就是。他要穿穿,拿与他罢。”迎春去了,就把老婆搂
在怀里,摸他奶头,说道:“我儿,你虽然生了孩子,奶头儿到还恁紧。”就两个
脸对脸儿亲嘴咂舌头做一处。如意儿道:“我见爹常在五娘身边,没见爹往别的房
里去。他老人家别的罢了,只是心多容不的人。前日爹不在,为个棒槌,好不和我
大嚷了一场。多亏韩嫂儿和三娘来劝开了。落后爹来家,也没敢和爹说。不知甚么
多嘴的人对他说,说爹要了我。他也告爹来不曾?”西门庆道:“他也告我来,你
到明日替他陪个礼儿便了。他是恁行货子,受不的人个甜枣儿就喜欢的。嘴头子虽
利害,到也没什么心。”如意儿道:“前日我和他嚷了,第二日爹到家,就和我说
好活。说爹在他身边偏多,‘就是别的娘都让我几分,你凡事只有个不瞒我,我放
着河水不洗船?’”西门庆道:“既是如此,大家取和些。”又许下老婆:“你每
晚夕等我来这房里睡。”如意道:“爹真个来?休哄俺每!”西门庆道:“谁哄你
来!”正说着,只见迎春取钥匙来。西门庆教开了床房门,又开橱柜,拿出那皮祆
来抖了抖,还用包袱包了,教迎春拿到那边房里去。如意儿就悄悄向西门庆说:“
我没件好裙袄儿,爹趁着手儿再寻件儿与了我罢。有娘小衣裳儿,再与我一件儿。
”西门庆连忙又寻出一套翠盖缎子袄儿、黄绵绸裙子,又是一件蓝潞绸绵裤儿,又
是一双妆花膝裤腿儿,与了他。老婆磕头谢了。西门庆锁上门,就使他送皮袄与金
莲房里来。

金莲才起来,在床上裹脚,只见春梅说:“如意儿送皮袄来了。”妇人便知其
意,说道:“你教他进来。”问道:“爹使你来?”如意道:“是爹教我送来与娘
穿。”金莲道:“也与了你些什么儿没有?”如意道:“爹赏了我两件绸绢衣裳年
下穿。叫我来与娘磕头。”于是向前磕了四个头。妇人道:“姐姐每这般却不好?
你主子既爱你,常言:船多不碍港,车多不碍路,那好做恶人?你只不犯着我,我
管你怎的?我这里还多着个影儿哩!”如意儿道:“俺娘已是没了,虽是后边大娘
承揽,娘在前边还是主儿,早晚望娘抬举。小媳妇敢欺心!那里是叶落归根之处?
”妇人道:“你这衣服少不得还对你大娘说声。”如意道:“小的前者也问大娘讨
来,大娘说:‘等爹开时,拿两件与你。’”妇人道:“既说知罢了。”这如意就
出来,还到那边房里,西门庆已往前厅去了。如意便问迎春:“你头里取钥匙去,
大娘怎的说?”迎春说:“大娘问:‘你爹要钥匙做什么?’我也没说拿皮袄与五
娘,只说我不知道。大娘没言语。”

却说西门庆走到厅上看设席,海盐子弟张美、徐顺、苟子孝都挑戏箱到了,李
铭等四名小优儿又早来伺候,都磕头见了。西门庆吩咐打发饭与众人吃,吩咐李铭
三个在前边唱,左顺后边答应堂客。那日韩道国娘子王六儿没来,打发申二姐买了
两盒礼物,坐轿子,他家进财儿跟着,也来与玉楼做生日。王经送到后边,打发轿
子出去了。不一时,门外韩大姨、孟大妗子都到了,又是傅伙计、甘伙计娘子、崔
本媳妇儿段大姐并贲四娘子。西门庆正在厅上,看见夹道内玳安领着一个五短身子
,穿绿缎袄儿、红裙子,不搽胭粉,两个密缝眼儿,一似郑爱香模样,便问是谁。
玳安道:“是贲四嫂。”西门庆就没言语。往后见了月娘。月娘摆茶,西门庆进来
吃粥,递与月娘钥匙。月娘道:“你开门做什么?”西门庆道:“潘六儿他说,明
日往应二哥家吃酒没皮袄,要李大姐那皮袄穿。”被月娘瞅了一眼,说道:“你自
家把不住自家嘴头了。他死了,嗔人分散他房里丫头,象你这等,就没的话儿说了
。他见放皮袄不穿,巴巴儿只要这皮袄穿。──早时他死了,他不死,你只好看一
眼儿罢了。”几句说的西门庆闭口无言。忽报刘学官来还银子,西门庆出去陪坐,
在厅上说话。只见玳安拿进帖儿说:“王招宣府送礼来了。”西门庆问:“是什么
礼?”玳安道:“是贺礼:一匹尺头、一坛南酒、四样下饭。”西门庆即叫王经拿
眷生回帖儿谢了,赏了来人五钱银子,打发去了。

只见李桂姐门首下轿,保儿挑四盒礼物。慌的玳安替他抱毡包,说道:“桂姨
,打夹道内进去罢,厅上有刘学官坐着哩。”那桂姐即向夹道内进去,来安儿把盒
子挑进月娘房里。月娘道:“爹看见不曾?”玳安道:“爹陪着客,还不见哩。”
月娘便说道:“且连盒放在明间内着。”一回客去了,西门庆进来吃饭,月娘道:
“李桂姐送礼在这里。”西门庆道:“我不知道。”月娘令小玉揭开盒儿,见一盒
果馅寿糕、一盒玫瑰糖糕、两只烧鸭、一副豕蹄。只见桂姐从房内出来,满头珠翠
,穿着大红对衿袄儿,蓝缎裙子,望着西门庆磕了四个头。西门庆道:“罢了,又
买这礼来做什么?”月娘道:“刚才桂姐对我说,怕你恼他。不干他事,说起来都
是他妈的不是:那日桂姐害头疼来,只见这王三官领着一行人,往秦玉芝儿家去,
打门首过,进来吃茶,就被人惊散了。桂姐也没出来见他。”西门庆道:“那一遭
儿没出来见他,这一遭儿又没出来见他,自家也说不过。论起来,我也难管你。这
丽春院拿烧饼砌着门不成?到处银钱儿都是一样,我也不恼。”那桂姐跪在地下只
顾不起来,说道:“爹恼的是。我若和他沾沾身子,就烂化了,一个毛孔儿里生一
个天疱疮。都是俺妈,空老了一片皮,干的营生没个主意。好的也招惹,歹的也招
惹,平白叫爹惹恼。”月娘道:“你既来说开就是了,又恼怎的?”西门庆道:“
你起来,我不恼你便了。”那桂姐故作娇态,说道:“爹笑一笑儿我才起来。你不
笑,我就跪一年也不起来。”潘金莲在旁插口道:“桂姐你起来,只顾跪着他,求
告他黄米头儿,叫他张致!如今在这里你便跪着他,明日到你家他却跪着你,──
你那时却别要理他。”把西门庆、月娘都笑了,桂姐才起来了。只见玳安慌慌张张
来报:“宋老爹、安老爹来了。”西门庆便拿衣服穿了,出去迎接。桂姐向月娘说
道:“耶[口乐][[口乐],从今后我也不要爹了,只与娘做女儿罢。”月娘道
:“你的虚头愿心,说过道过罢了。前日两遭往里头去,没在那里?”桂姐道:“
天么,天么,可是杀人!爹何曾往我家里?若是到我家里,见爹一面,沾沾身子儿
,就促死了!娘你错打听了,敢不是我那里,是往郑月儿家走了两遭,请了他家小
粉头子了。我这篇是非,就是他气不愤架的。不然,爹如何恼我?”金莲道:“各
人衣饭,他平白怎么架你是非?”桂姐道:“五娘,你不知,俺们里边人,一个气
不愤一个,好不生分!”月娘接过来道:“你每里边与外边差甚么?也是一般,一
个不愤一个。那一个有些时道儿,就要[足丽]下去。”月娘摆茶与他吃,不在话
下。

却说西门庆迎接宋御史、安郎中,到厅上叙礼。每人一匹缎子、一部书,奉贺
西门庆。见了桌席齐整,甚是称谢不尽。一面分宾主坐下,吃了茶,宋御史道:“
学生有一事奉渎四泉:今有巡抚侯石泉老先生,新升太常卿,学生同两司作东,三
十日敢借尊府置杯酒奉饯,初二日就起行上京去了。未审四泉允否?”西门庆道:
“老先生吩咐,敢不从命!但未知多少桌席?”宋御史道:“学生有分资在此。”
即唤书吏取出布、按两司连他共十二两分资来,要一张大插桌、六张散桌,叫一起
戏子。西门庆答应收了,就请去卷棚坐的。不一时,钱主事也到了。三员官会在一
处下棋。宋御史见西门庆堂庑宽广,院字幽深,书画文物极一时之盛。又见屏风前
安着一座八仙捧寿的流金鼎,约数尺高,甚是做得奇巧。炉内焚着沉檀香,烟从龟
鹤鹿口中吐出。只顾近前观看,夸奖不已。问西门庆:“这副炉鼎造得好!”因向
二官说:“我学生写书与淮安刘年兄那里,央他替我捎带一副来,送蔡老先,还不
见到。四泉不知是那里得来的?”西门庆道:“也是淮上一个人送学生的。”说毕
下棋。西门庆吩咐下边,看了两个桌盒细巧菜蔬果馅点心上来,一面叫生旦在上唱
南曲。宋御史道:“客尚未到,主人先吃得面红,说不通。”安郎中道:“天寒,
饮一杯无碍。”宋御史又差人去邀,差人禀道:“邀了,在砖厂黄老爹那里下棋,
便来也。”一面下棋饮酒,安郎中唤戏子:“你们唱个《宜春令》奉酒。”于是生
旦合声唱一套“第一来为压惊”。

唱未毕,忽吏进报:“蔡老爹和黄老爹来了。”宋御史忙令收了桌席,各整衣
冠出来迎接。蔡九知府穿素服金带,先令人投一“侍生蔡修”拜帖与西门庆。进厅
上,安郎中道:“此是主人西门大人,见在本处作千兵,也是京中老先生门下。”
那蔡知府又是作揖称道:“久仰,久仰。”西门庆道:“容当奉拜。”叙礼毕,各
宽衣服坐下。左右上了茶,各人扳话。良久,就上坐。蔡九知府居上,主位四坐。
厨役割道汤饭,戏子呈递手本,蔡九知府拣了《双忠记》,演了两折。酒过数巡,
小优儿席前唱一套《新水令》“玉鞭骄马出皇都”。蔡知府笑道:“松原直得多少
,可谓‘御史青骢马’,三公乃‘刘郎旧萦髯’。”安郎中道:“今日更不道‘江
州司马青衫湿’。”言罢,众人都笑了。西门庆又令春鸿唱了一套“金门献罢平胡
表”,把宋御史喜欢的要不的,因向西门庆道:“此子可爱。”西门庆道:“此是
小价,原是扬州人。”宋御史携着他手儿,教他递酒,赏了他三钱银子,磕头谢了
。正是:

窗外日光弹指过,席前花影坐间移。
一杯未尽笙歌送,阶下申牌又报时。

不觉日色沉西,蔡九知府见天色晚了,即令左右穿衣告辞。众位款留不住,俱
送出大门而去。随即差了两名吏典,把桌席羊酒尺头抬送到新河口去讫。宋御史亦
作辞西门庆,因说道:“今日且不谢,后日还要取扰。”各上轿而去。

西门庆送了回来,打发戏子,吩咐:“后日还是你们来,再唱一日。叫几个会
唱的来,宋老爹请巡抚侯爷哩。”戏子道:“小的知道了。”西门庆令攒上酒桌,
使玳安:“去请温师父来坐坐。”再叫来安儿:“去请应二爹去。”不一时,次第
而至,各行礼坐下。三个小优儿在旁弹唱,把酒来斟。西门庆问伯爵:“你娘们明
日都去,你叫唱的是杂耍的?”伯爵道:“哥到说得好,小人家那里抬放?将就叫
两个唱女儿唱罢了。明日早些请众位嫂子下降。”这里前厅吃酒不题。

后边,孟大姨与盂三妗子先起身去了。落后杨姑娘也要去,月娘道:“姑奶奶
你再住一日儿不是,薛师父使他徒弟取了卷来,咱晚夕叫他宣卷咱们听。”杨姑娘
道:“老身实和姐姐说,要不是我也住,明日俺第二个侄儿定亲事,使孩子来请我
,我要瞧瞧去。”于是作辞而去。众人吃到掌灯以后,三位伙计娘子也都作辞去了
,止留下段大姐没去,潘姥姥也往金莲房内去了。只有大吟子、李桂姐、申二姐和
三个姑子,郁大姐和李娇儿、孟玉楼、潘金莲,在月娘房内坐的。忽听前边散了,
小厮收下家伙来。这金莲忙抽身就往前走,到前边悄悄立在角门首。只见西门庆扶
着来安儿,打着灯,趔趄着脚儿就要往李瓶儿那边走,看见金莲在门首立着,拉了
手进入房来。那来安儿便往上房交钟箸。

月娘只说西门庆进来,把申二姐、李桂姐、郁大姐都打发往李娇儿房内去了。
问来安道:“你爹来没有?”来安道:“爹在五娘房里,不耐烦了。”月娘听了,
心内就有些恼,因向玉楼道:“你看恁没来头的行货子,我说他今日进来往你房里
去,如何三不知又摸到他屋里去了?这两日又浪风发起来,只在他前边缠。”玉楼
道:“姐姐,随他缠去!这等说,恰似咱每争他的一般。可是大师父说的笑话儿,
左右这六房里,由他串到。他爹心中所欲,你我管的他!”月娘道:“干净他有了
话!刚才听见前头散了,就慌的奔命往前走了。”因问小玉:“灶上没人,与我把
仪门拴上。后边请三位师父来,咱每且听他宣一回卷着。”又把李桂姐、申二姐、
段大姐、郁大姐都请了来。月娘向大妗子道:“我头里旋叫他使小沙弥请了《黄氏
女卷》来宣,今日可可儿杨姑娘又去了。”吩咐玉箫顿下好茶。玉楼对李娇儿说:
“咱两家轮替管茶,休要只顾累大姐姐。”于是各房里吩咐预备茶去。

不一时,放下炕桌儿,三个姑子来到,盘膝坐在炕上。众人俱各坐了,听他宣
卷。月娘洗手炷了香,这薛姑子展开《黄氏女卷》,高声演说道:

盖闻法初不灭,故归空。道本无生,每因生而不用。由法身以垂八相
,由八相以显法身。朗朗惠灯,通开世户;明明佛镜,照破昏衢。百年景
赖刹那间,四大幻身如泡影。每日尘劳碌碌,终朝业试忙忙。岂知一性圆
明,徒逞六根贪欲。功名盖世,无非大梦一场;富贵惊人,难免无常二字
。风火散时无老少,溪山磨尽几英雄!

演说了一回,又宣念偈子,又唱几个劝善的佛曲儿,方才宣黄氏女怎的出身,怎的
看经好善,又怎的死去转世为男子,又怎的男女五人一时升天。

慢慢宣完,已有二更天气。先是李娇儿房内元宵儿拿了一道茶来,众人吃了。
落后孟玉楼房中兰香,又拿了几样精制果菜、一大壶酒来,又是一大壶茶来,与大
妗子、段大姐、桂姐众人吃。月娘又教玉箫拿出四盒儿茶食饼糖之类,与三位师父
点茶。李桂姐道:“三个师父宣了这一回卷,也该我唱个曲儿孝顺。”月娘道:“
桂姐,又起动你唱?”郁大姐道:“等我先唱。”月娘道:“也罢,郁大姐先唱。
”申二姐道:“等姐姐唱了,我也唱个儿与娘们听。”桂姐不肯,道:“还是我先
唱。”因问月娘要听什么,月娘道:“你唱个‘更深静悄’罢。”当下桂姐送众人
酒,取过琵琶来,轻舒玉笋,款跨鲛绡,唱了一套。桂姐唱毕,郁大姐才要接琵琶
,早被申二姐要过去了,挂在胳膊上,先说道:“我唱个《十二月儿挂真儿》与大
妗子和娘每听罢。”于是唱道:“正月十五闹元宵,满把焚香天地烧……”那时大
妗子害夜深困的慌,也没等的申二姐唱完,吃了茶就先往月娘房内睡去了。须臾唱
完,桂姐便归李娇儿房内,段大姐便往孟玉楼房内,三位师父便往孙雪娥房里,郁
大姐、申二姐就与玉箫、小玉在那边炕屋里睡。月娘同大妗子在上房内睡,俱不在
话下。看官听说:古妇人怀孕,不侧坐,不偃卧,不听淫声,不视邪色,常玩诗书
金玉,故生子女端正聪慧,此胎教之法也。今月娘怀孕,不宜令僧尼宣卷,听其死
生轮回之说。后来感得一尊古佛出世,投胎夺舍,幻化而去,不得承受家缘。盖可
惜哉!正是:

前程黑暗路途险,十二时中自着迷。