\chapter{韩画士传真作遗爱~西门庆观戏动深悲}

诗曰:

香杳美人违,遥遥有所思。
幽明千里隔,风月两边时。
相对春那剧,相望景偏迟。
当由分别久,梦来还自疑。

话说西门庆被应伯爵劝解了一回,拭泪令小厮后边看饭去了。不一时,吴大舅
、吴二舅都到了。灵前行礼毕,与西门庆作揖,道及烦恼之意。请至厢房中,与众
人同坐。

玳安走至后边,向月娘说:“如何?我说娘每不信,怎的应二爹来了,一席话
说的爹就吃饭了。”金莲道:“你这贼,积年久惯的囚根子,镇日在外边替他做牵
头,有个拿不住他性儿的!”玳安道:“从小儿答应主子,不知心腹?”月娘问道
:“那几个陪他吃饭?”玳安道:“大舅、二舅才来,和温师父,连应二爹、谢爹
、韩伙计、姐夫,共爹八个人哩。”月娘道:“请你姐夫来后边吃罢了,也挤在上
头!”玳安道:“姐夫坐下了。”月娘吩咐:“你和小厮往厨房里拿饭去。你另拿
瓯儿粥与他吃,怕清早晨不吃饭。”玳安道:“再有谁?止我在家,都使出报丧、
买东西,王经,又使他往张亲家爹那里借云板去了。”月娘道:“书童那奴才和你
拿去是的,怕打了他纱帽展翅儿!”玳安道:“书童和画童两个在灵前,一个打磐
,一个伺候焚香烧纸哩。春鸿,爹又使他跟贲四换绢去了──嫌绢不好,要换六钱
一匹的破孝。”月娘道:“论起来,五钱的也罢,又巴巴儿换去!”又道:“你叫
下画童儿那小奴才,和他快拿去,只顾还挨甚么!”玳安于是和画童两个,大盘大
碗拿到前边,安放八仙桌席。众人正吃着饭,只见平安拿进手本来禀:“夏老爹差
写字的,送了三班军卫来这里答应。”西门庆看了,吩咐:“讨三钱银子赏他。写
期服生帖儿回你夏老爹:多谢了!”

一面吃毕饭,收了家伙。只见来保请的画师韩先生来到。西门庆与他行毕礼,
说道:“烦先生揭白传个神子儿。”那韩先生道:“小人理会得。”吴大舅道:“
动手迟了些,只怕面容改了。”韩先生道:“也不妨,就是揭白也传得。”正吃茶
毕,忽见平安来报:“门外花大舅来了。”西门庆陪花子由灵前哭涕了一回,见毕
礼数,与众人一处,因问:“甚么时侯?”西门庆道:“正丑时断气。临死还伶伶
俐俐说话儿,刚睡下,丫头起来瞧,就没了气儿。”因见韩先生旁边小童拿着屏插
,袖中取出描笔颜色来,花子由道:“姐夫如今要传个神子?”西门庆道:“我心
里疼他,少不得留个影像儿,早晚看着,题念他题念儿。”一面吩咐后边堂客躲开
,掀起帐子,领韩先生和花大舅众人到跟前。这韩先生揭起千秋幡,打一观看,见
李瓶儿勒着鸦青手帕,虽故久病,其颜色如生,姿容不改,黄恹恹的,嘴唇儿红润
可爱。那西门庆由不的掩泪而哭。来保与琴童在旁捧着屏插、颜色。韩先生一见就
知道了。众人围着他求画,应伯爵便道:“先生,此是病容,平昔好时,还生的面
容饱满,姿容秀丽。”韩先生道:“不须尊长吩咐,小人知道。敢问老爹:此位老
夫人,前者五月初一日曾在岳庙里烧香,亲见一面,可是否?”西门庆道:“正是
。那时还好哩。先生,你用心想着,传画一轴大影、一轴半身,灵前供养,我送先
生一匹缎子、十两银子。”韩先生道:“老爹吩咐,小人无不用心。”须臾,描染
出个半身来,端的玉貌幽花秀丽,肌肤嫩玉生香。拿与众人瞧,就是一幅美人图儿
。西门庆看了,吩咐玳安:“拿与你娘每瞧瞧去,看好不好。有那些儿不是,说来
好改。”

玳安拿到后边,向月娘道:“爹说叫娘每瞧瞧,六娘这影画得如何,那些儿不
象,说出去教韩先生好改。”月娘道:“成精鼓捣,人也不知死到那里去了,又描
起影来了。”潘金莲接说道:“那个是他的儿女?画下影,传下神,好替他磕头礼
拜!到明日六个老婆死了,画六个影才好。”孟玉楼和李娇儿接过来观看,说道:
“大娘,你来看,李大姐这影,倒象好时模样,打扮的鲜鲜的,只是嘴唇略扁了些
。”月娘看了道:“这左边额头略低了些,他的眉角还弯些。亏这汉子,揭白怎的
画来!”玳安道:“他在庙上曾见过六娘一面,刚才想着,就画到这等模样。”

少顷,只见王经进来说道:“娘每看了,就教拿出去。乔亲家爹来了,等乔亲
家爹瞧哩。”玳安走到前边,向韩先生道:“里边说来,嘴唇略扁了些,左额角稍
低些,眉还要略放弯些儿。”韩先生道:“这个不打紧。”随即取描笔改过了,呈
与乔大户瞧。乔大户道:“亲家母这幅尊像,真画得好,只少了口气儿。”西门庆
满心欢喜,一面递了三钟酒与韩先生,管待了酒饭,又教取出一匹尺头、十两白金
与韩先生,教他:“先攒造出半身来,就要挂,大影,不误出殡就是了。俱要用大
青大绿,冠袍齐整,绫裱牙轴。”韩先生道:“不必吩咐,小人知道。”领了银子
,教小童拿着插屏,拜辞出门。乔大户与众人又看了一回做成的棺木,便道:“亲
家母今已小殓罢了?”西门庆道:“如今仵作行人来就小殓。大殓还等到三日。”
乔大户吃毕茶,就告辞去了。

不一时,仵作行人来伺候,纸札打卷,铺下衣衾,西门庆要亲与他开光明,强
着陈敬济做孝子,与他抿了目,西门庆旋寻出一颗胡珠,安放在他口里。登时小殓
停当,照前停放端正,合家大小哭了一场。来兴又早冥衣铺里,做了四座堆金沥粉
捧盆巾盥栉毛女儿,一边两座摆下。灵前的彝炉商瓶、烛台香盒,教锡匠打造停当
,摆在桌上,耀日争辉。又兑了十两银子,教银匠打了三副银爵盏。又与应伯爵定
管丧礼簿籍:先兑了五百两银子、一百吊钱来,委付与韩伙计管帐;贲四与来兴儿
管买办,兼管外厨房;应伯爵、谢希大、温秀才、甘伙计轮番陪待吊客;崔本专管
付孝帐;来保管外库房;王经管酒房;春鸿与画童专管灵前伺候;平安与四名排军
,单管人来打云板、捧香纸;又叫一个写字带领四名排军,在大门首记门簿,值念
经日期,打伞挑幡幢。都派委已定,写了告示,贴在影壁上,各遵守去讫。只见皇
庄上薛内相差人送了六十根杉条、三十条毛竹、三百领芦席、一百条麻绳,西门庆
赏了来人五钱银子,拿期服生回帖儿打发去了。吩咐搭采匠把棚起脊搭大些,留两
个门走,把影壁夹在中间,前厨房内还搭三间罩棚,大门首扎七间榜棚,请报恩寺
十二众僧人先念倒头经,每日两个茶酒伺候茶水。

花大舅、吴二舅坐了一回,起身去了。西门庆交温秀才写孝帖儿,要刊去,令
写“荆妇奄逝”,温秀才悄悄拿与应伯爵看,伯爵道:“这个礼上说不通。见有如
今吴家嫂子在正室,如何使得?这一出去,不被人议论!就是吴大哥,心内也不自
在。等我慢慢再与他讲,你且休要写着。”陪坐至晚,各散归家去了。

西门庆晚夕也不进后边去,就在李瓶儿灵旁装一张凉床,拿围屏围着,独自宿
歇,止春鸿、书童儿近前伏侍。天明便往月娘房里梳洗,穿戴了白唐巾孝冠孝衣、
白绒袜、白履鞋,[纟至]带随身。

第二日清晨,夏提刑就来探丧吊问,慰其节哀。西门庆还礼毕,温秀才相陪,
待茶而去。到门首,吩咐写字的:“好生答应,查有不到的排军,呈来衙门内惩治
。”说毕,骑马去了。西门庆令温秀才发帖儿,差人请各亲眷,三日诵经,早来吃
斋。后晌,铺排来收拾道场,悬挂佛像,不必细说。

那日,吴银儿打听得知,坐轿子来灵前哭泣上纸。到后边,月娘相接。吴银儿
与月娘磕头,哭道:“六娘没了,我通一字不知,就没个人儿和我说声儿。可怜,
伤感人也!”孟玉楼道:“你是他干女儿,他不好了这些时,你就不来看他看儿?
”吴银儿道:“好三娘,我但知道,有个不来看的?说句假就死了!委实不知道。
”月娘道:“你不来看你娘,他倒还挂牵着你,留下件东西儿,与你做一念儿,我
替你收着哩。”因令小玉:“你取出来与银姐看。”小玉走到里面,取出包袱,打
开是一套缎子衣服、两根金头簪儿、一技金花。把吴银儿哭的泪如雨点相似,说道
:“饿早知他老人家不好,也来伏侍两日儿。”说毕,一面拜谢了月娘。月娘待茶
与他吃,留他过了三日去。

到三日,和尚打起磐子,道场诵经,挑出纸钱去。合家大小都披麻带孝。陈敬
济穿重孝[纟至]巾,佛前拜礼,街坊邻舍、亲朋长官都来吊问,上纸祭奠者,不
论其数。阴阳徐先生早来伺候大殓。祭告已毕,抬尸入棺,西门庆交吴月娘又寻出
他四套上色衣服来,装在棺内,四角又安放了四锭小银子儿。花子由说:“姐夫,
倒不消安他在里面,金银日久定要出世,倒非久远之计。”西门庆不肯,定要安放
。不一时,放下了七星板,搁上紫盖,仵作四面用长命钉一齐钉起来,一家大小放
声号哭。西门庆亦哭的呆了,口口声声只叫:“我的年小的姐姐,再不得见你了!
”良久哭毕,管待徐先生斋馔,打发去了。阖家伙计都是巾带孝服,行香之时,门
首一片皆白。温秀才举荐,北边杜中书来题铭旌。杜中书名子春,号云野,原侍真
宗宁和殿,今坐闲在家,西门庆备金帛请来。在卷棚内备果盒,西门庆亲递三杯酒
,应伯爵与温秀才相陪。铺大红官紵题旌,西门庆要写“诏封锦衣西门恭人
李氏柩”十一字,伯爵再三不肯,说:“见有正室夫人在,如何使得!”杜中书道
:“曾生过子,于礼也无碍。”讲了半日,去了“恭”字,改了“室人”。温秀才
道:“恭人系命妇,有爵;室人乃室内之人,只是个浑然通常之称。”于是用白粉
题毕,“诏封”二字贴了金,悬于灵前。又题了神主。叩谢杜中书,管待酒馔,拜
辞而去。

那日,乔大户、吴大舅、花大舅、韩姨夫、沈姨夫各家都是三牲祭桌来烧纸。
乔大户娘子并吴大妗子、二妗子、花大妗子,坐轿子来吊丧,祭祀哭泣。月娘等皆
孝髻,头须系腰,麻布孝裙,出来回礼举哀,让后边待茶摆斋。惟花大妗子与花大
舅便是重孝直身,余者都是轻孝。那日李桂姐打听得知,坐轿子也来上纸,看见吴
银儿在这里,说道:“你几时来的?怎的也不会我会儿?好人儿,原来只顾你!”
吴银儿道:“我也不知道娘没了,早知也来看看了。”月娘后边管待,俱不必细说
。

须臾过了,看看到首七,又是报恩寺十六众上僧,朗僧官为首座,引领做水陆
道场,诵《法华经》,拜三昧水忏。亲朋伙计无不毕集。那日,玉皇庙吴道官来上
纸吊孝,就揽二七经,西门庆留在卷棚内吃斋。忽见小厮来报:“韩先生送半身影
来。”众人观看,但见头戴金翠围冠,双凤珠子挑牌、大红妆花袍儿,白馥馥脸儿
,俨然如生。西门庆见了,满心欢喜。悬挂材头,众人无不夸奖:“只少口气儿!
”一面让卷棚内吃斋,嘱咐:“大影还要加工夫些。”韩先生道:“小人随笔润色
,岂敢粗心!”西门庆厚赏而去。

午间,乔大户来上祭,猪羊祭品、金银山、缎帛彩缯、冥纸炷香共约五十余抬
,地吊高撬,锣鼓细乐吹打,缨络喧阗而至。西门庆与陈敬济穿孝衣在灵前还礼。
乔大户邀了尚举人、朱堂官、吴大舅、刘学官、花千户、段亲家七八位亲朋,各在
灵前上香。三献已毕,俱跪听阴阳生读祝文曰:

维政和七年,岁次丁酉,九月庚申朔,越二十二日辛巳,眷生乔洪等
谨以刚鬣柔毛庶羞之奠,致祭于故亲家母西门孺人李氏之灵曰:呜呼!孺
人之性,宽裕温良,治家勤俭,御众慈祥,克全妇道,誉动乡邦。闺阃之
秀,兰蕙之芳,夙配君子,效聘鸾凰。蓝玉已种,浦珠已光。正期谐琴瑟
于有永,享弥寿于无疆。胡为一病,梦断黄粱?善人之殁,孰不哀伤?弱
女襁褓,沐爱姻嫱。不期中道,天不从愿,鸳伴失行。恨隔幽冥,莫睹行
藏。悠悠情谊,寓此一觞。灵其有知,来格来歆。尚飨。

官客祭毕,回礼毕,让卷棚内桌席管待。然后乔大户娘子、崔亲家母、朱堂官娘子
、尚举人娘子、段大姐众堂客女眷祭奠,地吊锣鼓,灵前吊鬼判队舞。吴月娘陪着
哭毕,请去后边待茶设席,三汤五割,俱不必细说。

西门庆正在卷棚内陪人吃酒,忽前边打的云板响。答应的慌慌张张进来禀报:
“本府胡爷上纸来了,在门首下轿子。”慌的西门庆连忙穿孝衣,灵前伺候。即使
温秀才衣巾素服出迎,左右先捧进香纸,然后胡府尹素服金带进来。许多官吏围随
,扶衣搊带,到了灵前,春鸿跪着,捧的香高高的,上了香,展拜两礼。西
门庆便道:“老先生请起,多有劳动。”连忙下来回礼。胡府尹道,“令夫人几时
没了?学生昨日才知。吊迟,吊迟!”西门庆道:“侧室一疾不救,辱承老先生枉
吊。”温秀才在旁作揖毕,请到厅上待茶一杯,胡府尹起身,温秀才送出大门,上
轿而去。上祭人吃至后晌方散。

第二日,院中郑爱月儿家来上纸。爱月儿进至灵前,烧了纸。月娘见他抬了八
盘饼馓、三牲汤饭来祭奠,连忙讨了一匹整绢孝裙与他。吴银儿与李桂姐都是三钱
奠仪,告西门庆说。西门庆道:“值甚么,每人都与他一匹整绢就是了。”月娘邀
到后边房里,摆茶管待,过夜。

晚夕,亲朋伙计来伴宿,叫了一起海盐子弟搬演戏文。李铭、吴惠、郑奉、郑
春都在这里答应。西门庆在大棚内放十五张桌席,为首的就是乔大户、吴大舅、吴
二舅、花大舅、沈姨夫、韩姨夫、倪秀才、温秀才、任医官、李智、黄四、应伯爵
、谢希大、祝实念、孙寡嘴、白赉光、常峙节、傅日新、韩道国、甘出身、贲第传
、吴舜臣、两个外甥,还有街坊六七位人,都是开桌儿。点起十数枝大烛来,堂客
便在灵前围着围屏,垂帘放桌席,往外观戏。当时众人祭奠毕,西门庆与敬济回毕
礼,安席上坐。下边戏子打动锣鼓,搬演的是韦皋、玉箫女两世姻缘《玉环记》。
不一时吊场,生扮韦皋,唱了一回下去。贴旦扮玉箫,又唱了一回下去。厨役上汤
饭割鹅。应伯爵便向西门庆说:“我闻的院里姐儿三个在这里,何不请出来,与乔
老亲家、老舅席上递杯酒儿。他倒是会看戏文,倒便益了他!”西门庆便使玳安进
入说去:“请他姐儿三个出来。”乔大户道:“这个却不当。他来吊丧,如何叫他
递起酒来?”伯爵道:“老亲家,你不知,象这样小淫妇儿,别要闲着他。──快
与我牵出来!你说应二爹说,六娘没了,只当行孝顺,也该与俺每人递杯酒儿。”
玳安进去半日,说:“听见应二爹在坐,都不出来哩。”伯爵道:“既恁说,我去
罢。”走了两步,又回坐下。西门庆笑道:“你怎的又回了?”伯爵道:“我有心
待要扯那三个小淫妇出来,等我骂两句,出了我气,我才去。”落后又使玳安请了
一遍,三个才慢条条出来。都一色穿着白绫对衿袄儿、蓝缎裙子,向席上不端不正
拜了拜儿,笑嘻嘻立在旁边。应伯爵道:“俺每在这里,你如何只顾推三阻四,不
肯出来?”那三个也不答应,向上边递了回酒,设一席坐着。下边鼓乐响动,关目
上来,生扮韦皋,净扮包知木,同到勾栏里玉箫家来。那妈儿出来迎接,包知木道
:“你去叫那姐儿出来。”妈云:“包官人,你好不着人,俺女儿等闲不便出来。
说不得一个‘请’字儿,你如何说‘叫他出来’?”那李桂姐向席上笑道:“这个
姓包的,就和应花子一般,就是个不知趣的蹇味儿!”伯爵道:“小淫妇,我不知
趣,你家妈怎喜欢我?”桂姐道:“他喜欢你?过一边儿!”西门庆道:“看戏罢
,且说甚么。再言语,罚一大杯酒!”那伯爵才不言语了。那戏子又做了一回,并
下。

厅内左边吊帘子看戏的,是吴大妗子、二妗子、杨姑娘、潘姥姥、吴大姨、孟
大姨、吴舜臣媳妇郑三姐、段大姐,并本家月娘姊妹;右边吊帘子看戏的,是春梅
、玉箫、兰香、迎春、小玉,都挤着观看。那打茶的郑纪,正拿着一盘果仁泡茶从
帘下过,被春梅叫住,问道:“拿茶与谁吃?”郑纪道:“那边六妗子娘每要吃。
”这春梅取一盏在手。不想小玉听见下边扮戏的旦儿名字也叫玉箫,便把王箫拉着
说道:“淫妇,你的孤老汉子来了。鸨子叫你接客哩,你还不出去。”使力往外一
推,直推出帘子外,春梅手里拿着茶,推泼一身。骂玉箫:“怪淫妇,不知甚么张
致,都顽的这等!把人的茶都推泼了,早是没曾打碎盏儿。”西门庆听得,使下来
安儿来问:“谁在里面喧嚷?”春梅坐在椅上道:“你去就说,玉箫浪淫妇,见了
汉子这等浪。”那西门庆问了一回,乱着席上递酒,就罢了。月娘便走过那边数落
小玉:“你出来这一日,也往屋里瞧瞧去。都在这里,屋里有谁?”小玉道:“大
姐刚才后边去的,两位师父也在屋里坐着。”月娘道:“教你们贼狗胎在这里看看
,就恁惹是招非的。”春梅见月娘过来,连忙立起身来说道:“娘,你问他。都一
个个只象有风病的,狂的通没些成色儿,嘻嘻哈哈,也不顾人看见。”那月娘数落
了一回,仍过那边去了。

那时,乔大户与倪秀才先起身去了。沈姨夫与任医官、韩姨夫也要起身,被应
伯爵拦住道:“东家,你也说声儿。俺每倒是朋友,不敢散,一个亲家都要去。沈
姨夫又不隔门,韩姨夫与任大人、花大舅都在门外。这咱晚三更天气,门也还未开
,慌的甚么?都来大坐回儿,左右关目还未了哩。”西门庆又令小厮提四坛麻姑酒
,放在面前,说:“列位只了此四坛酒,我也不留了。”因拿大赏钟放在吴大舅面
前,说道:“那位离席破坐说起身者,任大舅举罚。”于是众人又复坐下了。西门
庆令书童:“催促子弟,快吊关目上来,吩咐拣着热闹处唱罢。”须臾打动鼓板,
扮末的上来,请问面门庆:“‘寄真容’那一折可要唱?”西门庆道:“我不管你
,只要热闹。”贴旦扮玉箫唱了回。西门庆看唱到“今生难会面,因此上寄丹青”
一句,忽想起李瓶儿病时模样,不觉心中感触起来,止不住眼中泪落,袖中不住取
汗巾儿搽拭。又早被潘金莲在帘内冷眼看见,指与月娘瞧,说道:“大娘,你看他
好个没来头的行货子,如何吃着酒,看见扮戏的哭起来?”盂玉楼道:“你聪明一
场,这些儿就不知道了?乐有悲欢离合,想必看见那一段儿触着他心,他睹物思人
,见鞍思马,才掉泪来。”金莲道:“我不信。打谈的掉眼泪──替古人耽忧,这
些都是虚。他若唱的我泪出来,我才算他好戏子。”月娘道:“六姐,悄悄儿,咱
每听罢。”玉楼因向大妗子道:“俺六姐不知怎的,只好快说嘴。”

那戏子又做了一回,约有五更时分,众人齐起身。西门庆拿大杯拦门递酒,款
留不住,俱送出门。看收了家伙,留下戏厢:“明日有刘公公、薛公公来祭奠,还
做一日。”众戏子答应。管待了酒饭,归下处歇去了。李铭等四个亦归家不题。西
门庆见天色已将晓,就归后边歇息去了。正是,得多少──

红日映窗寒色浅,淡烟笼竹曙光微。