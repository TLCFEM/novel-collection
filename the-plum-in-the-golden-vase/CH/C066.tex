\chapter{翟管家寄书致赙~黄真人发牒荐亡}

词曰:

胸中千种愁,挂在斜阳树。绿叶阴阴自得春,草满莺啼处。  不见
凌波步,空想如簧语。门外重重叠叠山,遮不断愁来路。

话说西门庆陪吴大舅、应伯爵等饮酒中间,因问韩道国:“客伙中标船几时起
身?咱好收拾打包。”韩道国道:“昨日有人来会,也只在二十四日开船。”西门
庆道:“过了二十念经,打包便了。”伯爵问道:“这遭起身,那两位去?”西门
庆道:“三个人都去。明年先打发崔大哥押一船杭州货来,他与来保还往松江下五
处,置买些布货来卖。家中缎货绸绵都还有哩。”伯爵道:“哥主张极妙。常言道
:要的般般有,才是买卖。”说毕,已有起更时分,吴大舅起身说:“姐夫连日辛
苦,俺每酒已够了,告回,你可歇息歇息。”西门庆不肯,还留住,令小优儿奉酒
唱曲,每人吃三钟才放出门。西门庆赏小优四人六钱银子,再三不敢接,说:“宋
爷出票叫小的每来,官身如何敢受老爹重赏?”西门庆道:“虽然官差,此是我赏
你,怕怎的!”四人方磕头领去。西门庆便归后边歇去了。

次日早起往衙门中去,早有吴道官差了一个徒弟、两名铺排,来大厅上铺设坛
场,铺设的齐齐整整。西门庆来家看见,打发徒弟铺排斋食吃了回去。随即令温秀
才写帖儿,请乔大户、吴大舅、吴二舅、花大舅、沈姨夫、孟二舅、应伯爵、谢希
大、常峙节、吴舜臣许多亲眷并堂客,明日念经。家中厨役落作,治办斋供不题。

次日五更,道众皆来,进入经坛内,明烛焚香,打动响乐,讽诵诸经,铺排大
门首挂起长幡,悬吊榜文,两边黄纸门对一联,大书:

东极垂慈仙识乘晨而超登紫府;
南丹赦罪净魄受炼而迳上朱陵。

大厅经坛,悬挂斋题二十字,大书:“青玄救苦、颁符告简、五七转经、水火炼度
荐扬斋坛。”即日,黄真人穿大红,坐牙轿,系金带,左右围随,仪从暄喝,日高
方到。吴道官率众接至坛所,行礼毕,然后西门庆着素衣[纟至]巾,拜见递茶毕
。洞案旁边安设经筵法席,大红销金桌围,妆花椅褥,二道童侍立左右。发文书之
时,西门庆备金缎一匹;登坛之时,换了九阳雷巾,大红金云白百鹤法氅。先是表
白宣毕斋意,斋官沐手上香。然后黄真人焚香净坛,飞符召将,关发一应文书符命
,启奏三天,告盟十地。三献礼毕,打动音乐,化财行香。西门庆与陈敬济执手炉
跟随,排军喝路,前后四把销金伞、三对缨络挑搭。行香回来,安请监斋毕,又动
音乐,往李瓶儿灵前摄召引魂,朝参玉陛,旁设几筵,闻经悟道。到了午朝,高功
冠裳,步罡踏斗,拜进朱表,遣差神将,飞下罗酆。原来黄真人年约三旬,仪表非
常,妆束起来,午朝拜表,俨然就是个活神仙。但见:

星冠攒玉叶,鹤氅缕金霞。神清似长江皓月,貌古如太华乔松。踏罡
朱履进丹霄,步虚琅函浮瑞气。长髯广颊,修行到无漏之天;皓齿明眸,
佩[竹录]掌五雷之令。三更步月鸾声远,万里乘云鹤背高。就是都仙太
史临凡世,广惠真人降下方。

拜了表文,吴道官当坛颁生天宝[竹录]神虎玉札。行毕午香,卷棚内摆斋。黄真
人前,大桌面定胜;吴道官等,稍加差小;其余散众,俱平头桌席。黄真人、吴道
官皆衬缎尺头、四对披花、四匹丝绸,散众各布一匹。桌面俱令人抬送庙中,散众
各有手下徒弟收入箱中,不必细说。

吃毕午斋,都往花园内游玩散食去了。一面收下家火,从新摆上斋馔,请吴大
舅等众亲朋伙计来吃。正吃之间,忽报:“东京翟爷那里差人下书。”西门庆即出
厅上,请来人进来。只见是府前承差干办,青衣窄裤,万字头巾,乾黄靴,全副弓
箭,向前施礼。西门庆答礼相还。那人向身边取出书来递上,又是一封折赙仪银十
两。问来人上姓,那人道:“小人姓王名玉,蒙翟爷差遣,送此书来。不知老爹这
边有丧事,安老爹书到才知。”西门庆问道:“你安老爹书几时到的?”那人说:
“十月才到京。因催皇木一年已满,升都水司郎中。如今又奉敕修理河道,直到工
完回京。”西门庆问了一遍,即令来保厢房中管待斋饭,吩咐明日来讨回书。那人
问:“韩老爹在那里住?宅内捎信在此。小的见了,还要赶往东平府下书去。”西
门庆即唤出韩道国来见那人,陪吃斋饭毕,同往家中去了。

西门庆拆看书中之意,于是乘着喜欢,将书拿到卷棚内教温秀才看。说:“你
照此修一封回书答他,就捎寄十方[纟刍]纱汗巾、十方绫汗巾、十副拣金挑牙、
十个乌金酒杯作回奉之礼。他明日就来取回书。”温秀才接过书来观看,其书曰:

寓京都眷生翟谦顿首,书奉即擢大锦堂西门四泉亲家大人门下:自京
邸话别之后,未得从容相叙,心甚歉然。其领教之意,生已于家老爷前悉
陈之矣。迩者,安凤山书到,方知老亲家有鼓盆之叹,但恨不能一吊为怅
,奈何,奈何!伏望以礼节哀可也。外具赙仪,少表微忱,希管纳。又久
仰贵任荣修德政,举民有五绔之歌,境内有三留之誉,今岁考绩,必有甄
升。昨日神运都功,两次工上,生已对老爷说了,安上亲家名字。工完题
奏,必有恩典,亲家必有掌刑之喜。夏大人年终类本,必转京堂指挥列衔
矣。谨此预报,伏惟高照,不宣。
附云:此书可自省览,不可使闻之于渠。谨密,谨密!
又云:杨老爷前月二十九日卒于狱。
冬上浣具

温秀才看毕,才待袖,早被应伯爵取过来,观看了一遍,还付与温秀才收了。说道
:“老先生把回书千万加意做好些。翟公府中人才极多,休要教他笑话。”温秀才
道:“貂不足,狗尾续。学生匪才,焉能在班门中弄大斧!不过乎塞责而已。”西
门庆道:“温老先他自有个主意,你这狗才晓的甚么!”须臾,吃罢午斋,西门庆
吩咐来兴儿打发斋馔,送各亲眷街邻。又使玳安回院中李桂姐、吴银儿、郑爱月儿
、韩钏儿、洪四儿、齐香儿六家香仪人情礼去。每家回答一匹大布、一两银子。

后晌,就叫李铭、吴惠、郑奉三个小优儿来伺候。良久,道众升坛发擂,上朝
拜忏观灯,解坛送圣。天色渐晚。比及设了醮,就有起更天气。门外花大舅被西门
庆留下不去了,乔大户、沈姨夫、孟二舅告辞回家。止有吴大舅、二舅、应伯爵、
谢希大、温秀才、常峙节并众伙计在此,晚夕观看水火练度。就在大厅棚内搭高座
,扎彩桥,安设水池火沼,放摆斛食。李瓶儿灵位另有几筵帏幕,供献齐整。旁边
一首魂幡、一首红幡、一首黄幡,上书“制魔保举,受炼南宫”。先是道众音乐,
两边列座,持节捧盂剑,四个道童侍立两边。黄真人头戴黄金降魔冠,身披绛绡云
霞衣,登高座,口中念念有词。宣偈云:

太乙慈尊降驾来,夜壑幽关次第开。
童子双双前引导,死魂受炼步云阶。

宣偈毕,又熏沐焚香,念曰:“伏以玄皇阐教,广开度于冥途;正一垂科,俾炼形
而升举。恩沾幽爽,泽被饥嘘。谨运真香,志诚上请东极大慈仁者太乙救苦天尊、
十方救苦诸真人圣众,仗此真香,来临法会。切以人处尘凡,日萦俗务,不知有死
,惟欲贪生。鲜能种于善根,多随入于恶趣,昏迷弗省,恣欲贪嗔。将谓自己长存
,岂信无常易到!一朝倾逝,万事皆空。业障缠身,冥司受苦。今奉道伏为亡过室
人李氏灵魂,一弃尘缘,久沦长夜。若非荐拔于愆辜,必致难逃于苦报。恭惟天尊
秉好生之仁,救寻声之苦。洒甘露而普滋群类,放瑞光而遍烛昏衢。命三官宽考较
之条,诏十殿阁推研之笔。开囚释禁,宥过解冤。各随符使,尽出幽关。咸令登火
池之沼,悉荡涤黄华之形。凡得更生,俱归道岸。兹焚灵宝炼形真符,谨当宣奏:

太微回黄旗,无英命灵幡,
摄召长夜府,开度受生魂。”

道众先将魂幡安于水池内,焚结灵符,换红幡;次于火沼内焚郁仪符,换黄幡。高
功念:“天一生水,地二生火,水火交炼,乃成真形。”炼度毕,请神主冠帔步金
桥,朝参玉陛,皈依三宝,朝玉清,众举《五供养》。举毕,高功曰:“既受三皈
,当宣九戒。”九戒毕,道众举音乐,宣念符命并《十类孤魂》。炼度已毕,黄真
人下高座,道众音乐送至门外,化财焚烧箱库。

回来,斋功圆满,道众都换了冠服,铺排收卷道像。西门庆又早大厅上画烛齐
明,酒筵罗列。三个小优弹唱,众亲友都在堂前。西门庆先与黄真人把盏,左右捧
着一匹天青云鹤金缎、一匹色缎、十两白银,叩首下拜道:“亡室今日赖我师经功
救拔,得遂超生,均感不浅,微礼聊表寸心。”黄真人道:“小道谬忝冠裳,滥膺
玄教,有何德以达人天?皆赖大人一诚感格,而尊夫人已驾景朝元矣。此礼若受,
实为赧颜。”西门庆道:“此礼甚薄,有亵真人,伏乞笑纳!”黄真人方令小童收
了。西门庆递了真人酒,又与吴道官把盏,乃一匹金缎、五两白银,又是十两经资
。吴道官只受经资,余者不肯受,说:“小道素蒙厚爱,自恁效劳诵经,追拔夫人
往生仙界,以尽其心。受此经资尚为不可,又岂敢当此盛礼乎!”西门庆道:“师
父差矣。真人掌坛,其一应文简法事,皆乃师父费心。此礼当与师父酬劳,何为不
可?”吴道官不得已,方领下,再三致谢。西门庆与道众递酒已毕,然后吴大舅、
应伯爵等上来与西门庆散福递酒。吴大舅把盏,伯爵执壶,谢希大捧菜,一齐跪下
。伯爵道:“嫂子今日做此好事,幸请得真人在此,又是吴师父费心,嫂子自得好
处。此虽赖真人追荐之力,实是哥的虔心,嫂子的造化。”于是满斟一杯送与西门
庆。西门庆道:“多蒙列位连日劳神,言谢不尽。”说毕,一饮而尽。伯爵又斟一
盏,说:“哥,吃个双杯,不要吃单杯。”谢希大慌忙递一箸菜来吃了。西门庆回
敬众人毕,安席坐下。小优弹唱起来,厨役上割道。当夜在席前猜拳行令,品竹弹
丝,直吃到二更时分,西门庆已带半酣,众人方作辞起身而去。西门庆进来赏小优
儿三钱银子,往后边去了。正是:

人生有酒须当醉,一滴何曾到九泉。