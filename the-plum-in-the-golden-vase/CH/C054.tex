\chapter{应伯爵隔花戏金钏~任医官垂帐诊瓶儿}

词曰:

美酒斗十千,更对花前。芳樽肯放手中闲?起舞酬花花不语,似解人
怜。  不醉莫言还,请看枝间。已飘零一片减婵娟。花落明年犹自好,
可惜朱颜。

却说王姑子和李瓶儿、吴月娘,商量来日起经头停当,月娘便拿了些应用物件
送王姑子去,又教陈敬济来吩咐道:“明日你李家丈母拜经保佑官哥,你早去礼拜
礼拜。”敬济推道:“爹明日要去门外花园吃酒,留我店里照管,着别人去罢。”
原来敬济听见应伯爵请下了西门庆,便想要乘机和潘金莲弄松,因此推故。月娘见
说照顾生意,便不违拗他,放他出去了,便着书童礼拜。调拨已定,单待明日起经
。

且说西门庆和应伯爵、常峙节谈笑多时,只见琴童来回话道:“唱的叫了。吴
银儿有病去不的,韩金钏儿答应了,明日早去。”西门庆道:“吴银儿既病,再去
叫董娇儿罢。”常峙节道:“郊外饮酒,有一个尽够了,不消又去叫。”说毕,各
各别去,不在话下。

次日黎明,西门庆起身梳洗毕,月娘安排早饭吃了,便乘轿往观音庵起经。书
童、玳安跟随而行。王姑子出大门迎接,西门庆进庵来,北面皈依参拜。但见:

金仙建化,启第一之真乘;玉偈演音,集三千之妙利。宝花座上,装
成庄严世界;惠日光中,现出欢喜慈悲。香烟缭绕,直透九霄;仙鹤盘旋
,飞来[禾氐]树。访问缘由,果然稀罕;但思福果,那惜金钱!正是:
办个至诚心,何处皇天难感;愿将大佛事,保祈殇子彭[竹钱]。

王姑子宣读疏头,西门庆听了,平身更衣。王姑子捧出茶来,又拿些点心饼馓之物
摆在桌上。西门庆不吃,单呷了口清茶,便上轿回来,留书童礼拜。正是:

愿心酬毕喜匆匆,感谢灵神保佑功。
更愿皈依莲座下,却教关煞永亨通。

回来,红日才半竿,应伯爵早同常峙节来请。西门庆笑道:“那里有请吃早饭
的?我今日虽无事故,也索下午才好去。”应伯爵道:“原来哥不知,出城二十里
,有个内相花园,极是华丽,且又幽深,两三日也游玩不到哩。因此要早去,尽这
一日工夫,可不是好。”常峙节道:“今日哥既没甚事故,应哥早邀,便索去休。
”西门庆道:“既如此;常二哥和应二哥先行,我乘轿便到了。”应伯爵道:“专
待哥来。”说罢,两人出门,叫头口前去,又转到院内,立等了韩金钏儿坐轿子同
去。应伯爵先一日已着火家来园内,杀鸡宰鹅,安排筵席,又叫下两个优童随着去
了。

西门庆见三人去了多时,便乘轿出门,迤逦渐近。举头一看,但见:

千树浓阴,一湾流水。粉墙藏不谢之花,华屋掩长春之景。武陵桃放
,渔人何处识迷津?庾岭梅开,词客此中寻好句。端的是天上蓬莱,人间
阆苑。

西门庆赞叹不已道:“好景致!”下轿步人园来。应伯爵和常峙节出来迎接,园亭
内坐的。先是韩金钏儿磕了头,才是两个歌童磕头。吃了茶,伯爵就要递上酒来,
西门庆道:“且住,你每先陪我去瞧瞧景致来。”一面立起身来,搀着韩金钏手儿
同走。伯爵便引着,慢慢的步出回廊,循朱阑转过垂杨边一曲荼蘼架,踅过太湖石
、松凤亭,来到奇字亭。亭后是绕屋梅花三十树,中间探梅阁。阁上名人题咏极多
,西门庆备细看了。又过牡丹台,台上数十种奇异牡丹。又过北是竹园,园左有听
竹馆、凤来亭,匾额都是名公手迹;右是金鱼池,池上乐水亭,凭朱栏俯看金鱼,
却象锦被也似一片浮在水面。西门庆正看得有趣,伯爵催促,又登一个大楼,上写
“听月楼”。楼上也有名人题诗对联,也是刊板砂绿嵌的。下了楼,往东一座大山
,山中八仙洞,深幽广阔。洞中有石棋盘,壁上铁笛铜箫,似仙家一般。出了洞,
登山顶一望,满园都是见的。

西门庆走了半日,常峙节道:“恐怕哥劳倦了,且到园亭上坐坐,再走不迟。
”西门庆道:“十分走不过一分,却又走不得了。多亏了那些抬轿的,一日赶百来
里多路。”大家笑了,让到园亭里,西门庆坐了上位,常峙节坐东,应伯爵坐西,
韩金钏儿在西门庆侧边陪坐。大家送过酒来,西门庆道:“今日多有相扰,怎的生
受!”伯爵道:“一杯水酒,哥说那里话!”三人吃够数杯,两个歌童上来。西门
庆看那歌童生得──

粉块捏成白面,胭脂点就朱唇。绿糁糁披几寸青丝,香馥馥着满身罗
绮。秋波一转,凭他铁石心肠。檀板轻敲,遮莫金声玉振。正是但得倾城
与倾国,不论南方与北方。

两个歌童上来,拿着鼓板,合唱了一套时曲《字字锦》“群芳绽锦鲜”。唱的娇喉
婉转,端的是绕梁之声,西门庆称赞不已。常峙节道:“怪他是男子,若是妇女,
便无价了。”西门庆道:“若是妇女,咱也早叫他坐了,决不要他站着唱。”伯爵
道:“哥本是在行人,说的话也在行。”众人都笑起来。三人又吃了数杯,伯爵送
上令盆,斟一大钟酒,要西门庆行令。西门庆道:“这便不消了。”伯爵定要行令
,西门庆道:“我要一个风花雪月,第一是我,第二是常二哥,第三是主人,第四
是钏姐。但说的出来,只吃这一杯。若说不出,罚一杯,还要讲十个笑话。讲得好
便休;不好,从头再讲。如今先是我了。”拿起令钟,一饮而尽,就道:“云淡风
轻近午天。──如今该常二哥了。”常峙节接过酒来吃了,便道:“傍花随柳过前
川。──如今该主人家了。”应伯爵吃了酒,呆登登讲不出来。西门庆道:“应二
哥请受罚。”伯爵道:“且待我思量。”又迟了一回,被西门庆催逼得紧,便道:
“泄漏春光有几分。”西门庆大笑道:“好个说别字的,论起来,讲不出该一杯,
说别字又该一杯,共两杯。”伯爵笑道:“我不信,有两个‘雪’字,便受罚了两
杯?”众人都笑了,催他讲笑话。伯爵说道:“一秀才上京,泊船在扬子江。到晚
,叫艄公:‘泊别处罢,这里有贼。’艄公道:‘怎的便见得有贼?’秀才道:‘
兀那碑上写的不是江心贼?’艄公笑道:‘莫不是江心赋,怎便识差了?’秀才道
:‘赋便赋,有些贼形。’”西门庆笑道:“难道秀才也识别字?”常峙节道:“
应二哥该罚十大杯。”伯爵失惊道:“却怎的便罚十杯?”常峙节道:“你且自家
去想。”原来西门庆是山东第一个财主,却被伯爵说了“贼形”,可不骂他了!西
门庆先没理会,到被常峙节这句话提醒了。伯爵觉失言,取酒罚了两杯,便求方便
。西门庆笑道:“你若不该,一杯也不强你;若该罚时,却饶你不的。”伯爵满面
不安。又吃了数杯,瞅着常峙节道:“多嘴!”西门庆道:“再说来!”伯爵道:
“如今不敢说了。”西门庆道:“胡乱取笑,顾不的许多,且说来看。”伯爵才安
心,又说:“孔夫子西狩得麟,不能够见,在家里日夜啼哭。弟子恐怕哭坏了,寻
个牯牛,满身挂了铜钱哄他。那孔子一见便识破,道:‘这分明是有钱的牛,却怎
的做得麟!’”说罢,慌忙掩着口跪下道:“小人该死了,实是无心。”西门庆笑
着道:“怪狗才,还不起来。”金钏儿在旁笑道:“应花子成年说嘴麻犯人,今日
一般也说错了。大爹,别要理他。”说的伯爵急了,走起来把金钏儿头上打了一下
,说道:“紧自常二那天杀的韶叨,还禁的你这小淫妇儿来插嘴插舌!”不想这一
下打重了,把金钏疼的要不的,又不敢哭,[月乞][月愁]着脸,待要使性儿。
西门庆笑骂道:“你这狗才,可成个人?嘲戏了我,反又打人,该得何罪?”伯爵
一面笑着,搂了金钏说道:“我的儿,谁养的你恁娇?轻轻荡得一荡儿就待哭,亏
你挨那驴大的行货子来!”金钏儿揉着头,瞅了他一眼,骂道:“怪花子,你见来
?没的扯淡!敢是你家妈妈子倒挨驴的行货来。”伯爵笑说道:“我怎不见?只大
爹他是有名的潘驴邓小闲,不少一件,你怎的赖得过?”又道:“哥,我还有个笑
话儿,一发奉承了列位罢:一个小娘,因那话宽了,有人教道他:‘你把生矾一块
,塞在里边,敢就紧了。’那小娘真个依了他。不想那矾涩得疼了,不好过,[月
乞][月愁]着立在门前。一个走过的人看见了,说道:‘这小淫妇儿,倒象妆霸
王哩!’这小娘正没好气,听见了,便骂道:‘怪囚根子,俺樊哙妆不过,谁这里
妆霸王哩!’”说毕,一座大笑,连金钏儿也噗嗤的笑了。

少顷,伯爵饮过酒,便送酒与西门庆完令。西门庆道:“该钏姐了。”金钏儿
不肯。常峙节道:“自然还是哥。”西门庆取酒饮了,道:“月殿云梯拜洞仙。”
令完,西门庆便起身更衣散步。伯爵一面叫摆上添换来,转眼却不见了韩金钏儿。
伯爵四下看时,只见他走到山子那边蔷薇架儿底下,正打沙窝儿溺尿。伯爵看见了
,连忙折了一枝花枝儿,轻轻走去,蹲在他后面,伸手去挑弄他的花心。韩金钏儿
吃了一惊,尿也不曾溺完就立起身来,连裤腰都湿了。不防常峙节从背后又影来,
猛力把伯爵一推,扑的向前倒了一交,险些儿不曾溅了一脸子的尿。伯爵爬起来,
笑骂着赶了打,西门庆立在那边松阴下看了,笑的要不的。连韩金钏儿也笑的打跌
道:“应花子,可见天理近哩!”于是重新入席饮酒。西门庆道:“你这狗才,刚
才把俺们都嘲了,如今也要你说个自己的本色。”伯爵连说:“有有有,一财主撒
屁,帮闲道:‘不臭。’财主慌的道:‘屁不臭,不好了,快请医人!’帮闲道:
‘待我闻闻滋味看。’假意儿把鼻一嗅,口一咂,道:‘回味略有些臭,还不妨。
’”说的众人都笑了。常峙节道:“你自得罪哥哥,怎的把我的本色也说出来?”
众人又笑了一场。伯爵又要常峙节与西门庆猜枚饮酒。韩金钏儿又弹唱着奉酒。众
人欢笑,不在话下。

且说陈敬济探听西门庆出门,便百般打扮的俊俏,一心要和潘金莲弄鬼,又不
敢造次,只在雪洞里张看,还想妇人到后园来。等了半日不见来,耐心不过,就一
直迳奔到金莲房里来,喜得没有人看见。走到房门首,忽听得金莲娇声低唱了一句
道:“莫不你才得些儿便将人忘记。”已知妇人动情,便接口道:“我那敢忘记了
你!”抢进来,紧紧抱住道:“亲亲,昨日丈母叫我去观音庵礼拜,我一心放你不
下,推事故不去。今日爹去吃酒了,我绝早就在雪洞里张望。望得眼穿,并不见我
亲亲的俊影儿。因此,拚着死踅得进来。”金莲道:“硶说嘴的,你且禁声
。墙有风,壁有耳,这里说话不当稳便。”说未毕,窗缝里隐隐望见小玉手拿一幅
白绢,渐渐走近屋里来,又忽地转去了。金莲忖道:“这怪小丫头,要进房却又跑
转去,定是忘记甚东西。”知道他要再来,慌教陈敬济:“你索去休,这事不济了
。”敬济没奈何,一溜烟出去了。果然,小玉因月娘教金莲描画副裙拖送人,没曾
拿得花样,因此又跑转去。这也是金莲造化,不该出丑。待的小玉拿了花样进门,
敬济已跑去久了。金莲接着绢儿,尚兀是手颤哩。

话分两头。再表西门庆和应伯爵、常峙节,三人吃的酩酊,方才起身。伯爵再
四留不住,忙跪着告道:“莫不哥还怪我那句话么?可知道留不住哩。”西门庆笑
道:“怪狗才,谁记着你话来!”伯爵便取个大瓯儿,满满斟了一瓯递上来,西门
庆接过吃了。常峙节又把些细果供上来,西门庆也吃了,便谢伯爵起身。与了金钏
儿一两银子,叫玳安又赏了歌童三钱银子,吩咐:“我有酒,也着人叫你。”说毕
,上轿便行,两个小厮跟随。伯爵叫人家收过家活,打发了歌童,骑头口同金钏儿
轿子进城来,不题。

西门庆到家,已是黄昏时分,就进李瓶儿房里歇了。次日,李瓶儿和西门庆说
:“自从养了孩子,身上只是不净。早晨看镜子,兀那脸皮通黄了,饮食也不想,
走动却似闪肭了腿的一般。倘或有些山高水低,丢了孩子教谁看管?”西门庆见他
掉下泪来,便道:“我去请任医官来,看你脉息,吃些丸药,管就好了。”便叫书
童写个帖儿,去请任医官来。书童依命去了。

西门庆自来厅上,只见应伯爵早来谢劳。西门庆谢了相扰,两人一处坐地说话
。不多时,书童通报任医官到,西门庆慌忙出迎,和应伯爵厮见,三人依次而坐。
书童递上茶来吃了,任医官便动问:“府上是那一位贵恙?”西门庆道:“就是第
六个小妾,身子有些不好,劳老先生仔细一看。”任医官道:“莫不就是前日得哥
儿的么?”西门庆道:“正是。不知怎么生起病来。”任医官道:“且待学生进去
看看。”说毕,西门庆陪任医官进到李瓶儿屋里,就床前坐下。叫丫头把帐儿轻轻
揭开一缝,先放出李瓶儿的右手来,用帕儿包着,搁在书上。任医官道:“且待脉
息定着。”定了一回,然后把三个指头按在脉上,自家低着头,细玩脉息,多时才
放下。李瓶儿在帐缝里慢慢的缩了进去。不一时,又把帕儿包着左手,捧将出来,
搁在书上,任医官也如此看了。看完了,便向西门庆道:“老夫人两手脉都看了,
却斗胆要瞧瞧气色。”西门道:“通家朋友,但看何妨。”就教揭起帐儿。任医官
一看,只见:脸上桃花红绽色,眉尖柳叶翠含颦。那任医官略看了两眼,便对西门
庆说:“夫人尊颜,学生已是望见了。大约没有甚事,还要问个病源,才是个望、
闻、问、切。”西门庆就唤奶子。只见如意儿打扮的花花哨哨走过来,向任医官道
个万福,把李瓶儿那口燥唇干、睡炕不稳的病症,细细说了一遍。那任医官即便起
身,打个恭儿道:“老先生,若是这等,学生保的没事。大凡以下人家,他形神粗
卤,气血强旺,可以随分下药,就差了些,也不打紧的。如宅上这样大家,夫人这
样柔弱的形躯,怎容得一毫儿差池!正是药差指下,延祸四肢。以此望、闻、问、
切,一件儿少不得的。前日,王吏部的夫人也有些病症,看来却与夫人相似。学生
诊了脉,问了病源,看了气色,心下就明白得紧。到家查了古方,参以己见,把那
热者凉之,虚者补之,停停当当,不消三四剂药儿,登时好了。那吏部公也感小弟
得紧,不论尺头银两,加礼送来。那夫人又有梯己谢意,吏部公又送学生一个匾儿
,鼓乐喧天,送到家下。匾上写着‘儒医神术’四个大字。近日,也有几个朋友来
看,说道写的是甚么颜体,一个个飞得起的。况学生幼年曾读几行书,因为家事消
乏,就去学那岐黄之术。真正那‘儒医’两字,一发道的着哩!”西门庆道:“既
然不妨,极是好了。不满老先生说,家中虽有几房,只是这个房下,极与学生契合
。学生偌大年纪,近日得了小儿,全靠他扶养,怎生差池的!全仗老先生神术,与
学生用心儿调治他速好,学生恩有重报。纵是咱们武职比不的那吏部公,须索也不
敢怠慢。”任医官道:“老先生这样相处,小弟一分也不敢望谢。就是那药本,也
不敢领。”西门庆听罢,笑将起来道:“学生也不是吃白药的。近日有个笑话儿讲
得好:有一人说道:‘人家猫儿若是犯了癞的病,把乌药买来,喂他吃了就好了。
’旁边有一人问:‘若是狗儿有病,还吃甚么药?’那人应声道:‘吃白药,吃白
药。’可知道白药是狗吃的哩!”那任医官拍手大笑道:“竟不知那写白方儿的是
什么?”又大笑一回。任医官道:“老先生既然这等说,学生也止求一个匾儿罢。
谢仪断然不敢,不敢。”又笑了一回,起身,大家打恭到厅上去了。正是:

神方得自蓬莱监,脉诀传从少室君。
凡为采芝骑白鹤,时缘度世访豪门。