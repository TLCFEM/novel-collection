\chapter{韩爱姐路遇二捣鬼~普静师幻度孝哥儿}

诗曰:
旧日豪华事已空,银屏金屋梦魂中。
黄芦晚日空残垒,碧草寒烟锁故宫。
隧道鱼灯油欲尽,妆台鸾镜匣长封。
凭谁话尽兴亡事,一衲闲云两袖风。

话说韩道国与王六儿,归到谢家酒店内,无女儿,道不得个坐吃山崩,使陈三儿去
,又把那何官人勾来续上。那何官人见地方中没了刘二,除了一害,依旧又来王六
儿家行走,和韩道国商议:“你女儿爱姐,只是在府中守孝,不出来了,等我卖尽
货物,讨了赊帐,你两口跟我往湖州家去罢,省得在此做这般道路。”韩道国说:
”官人下顾,可知好哩。”一日卖尽了货物,讨上赊帐,雇了船,同王六儿跟往湖
州去了,不题。

却表爱姐在府中,与葛翠屏两个持贞守节,姊妹称呼,甚是合当。白日里与春梅做
伴儿在一处。那时金哥儿大了,年方六岁。孙二娘所生玉姐年长十岁,相伴两个孩
儿,便没甚事做。

谁知自从陈敬济死后,守备又出征去了。这春梅每日珍馐百味,绫锦衣衫,头上黄
的金,白的银,圆的珠,光照的无般不有。只是晚夕难禁独眠孤枕,欲火烧心。因
见李安一条好汉,只因打杀张胜,巡风早晚十分小心。

一日,冬月天气,李安正在班房内上宿,忽听有人敲后门,忙问道:“是谁?”只
闻叫道:“你开门则个。”李安连忙开了房门,却见一个人抢入来,闪身在灯光背
后。李安看时,却认得是养娘金匮。李安道:“养娘,你这咱晚来有甚事?”金匮
道:“不是我私来,里边奶奶差出我来的。”李安道:“奶奶叫你来怎么?”金匮
笑道:“你好不理会得。看你睡了不曾,教我把一件物事来与你。”向背上取下一
包衣服,”把与你,包内又有几件妇女衣服与你娘。前日多累你押解老爷行李车辆
,又救得奶奶一命,不然也吃张胜那厮杀了。”说毕,留下衣服,出门走了两步,
又回身道:“还有一件要紧的。”又取出一锭五十两大元宝来,撇与李安自去了。

当夜踌躇不决。次早起来,径拿衣服到家与他母亲。做娘的问道:“这东西是那里
的?”李安把夜来事说了一遍。做母亲的听言叫苦:“当初张胜干坏事,一百棍打
死,他今日把东西与你,却是甚么意思?我今六十已上年纪,自从没了你爹爹,满
眼只看着你,若是做出事来,老身靠谁?明早便不要去了。”李安道:“我不去,
他使人来叫,如何答应?”婆婆说:“我只说你感冒风寒病了。”李安道:“终不
成不去,惹老爷不见怪么?”做娘的便说:“你且投到你叔叔,山东夜叉李贵那里
住上几个月,再来看事故何如。”这李安终是个孝顺的男子,就依着娘的话,收拾
行李,往青州府投他叔叔李贵去了。春梅以后见李安不来,三、四、五次使小伴当
来叫。婆婆初时答应家中染病,次后见人来验看,才说往原籍家中,讨盘缠去了。
这春梅终是恼恨在心不题。

时光迅速,日月如梭,又早腊尽阳回,正月初旬天气。统制领兵一万三千,在东昌
府屯住已久,使家人周忠,捎书来家。教搬取春梅、孙二娘,并金哥、玉姐家小上
车。止留下周忠:“东庄上请你二爷看守宅子。”原来统制还有个族弟周宣,在庄
上住。周忠在府中,与周宣、葛翠屏、韩爱姐看守宅子。周仁与众军牢保定车辆,
往东昌府来。此一去,不为身名离故土,争知此去少回程。有词一篇,单道周统制
果然是一员好将材。当此之时,中原荡扫,志欲吞胡。但见:
四方盗起如屯峰,狼烟烈焰薰天红。
将军一怒天下安,腥膻扫尽夷从风。
公事忘私愿已久,此身许国不知有。
金戈抑日酬战征,麒麟图画功为首。
雁门关外秋风烈,铁衣披张卧寒月。
汗马卒勤二十年,赢得斑斑鬓如雪。
天子明见万里余,几番劳勣来旌书。
肘悬金印大如斗,无负堂堂七尺躯。

有日,周仁押家眷车辆到于东昌。统制见了春梅、孙二娘、金哥、玉姐,众丫鬟家
小都到了,一路平安,心中大喜。就在统制府衙后厅居住。周仁悉把”东庄上请了
二爷来宅内,同小的老子周忠看守宅舍”,说了一遍。周统制又问:“怎的李安不
见?”春梅道:“又题甚李安?那厮我因他捉获了张胜,好意赏了他两件衣服,与
他娘穿。他到晚夕巡风,进入后厅,把他二爷东庄上收的子粒银--一包五十两,
放在明间卓上,偷的去了。几番使伴当叫他,只是推病不来。落后又使叫去,他躲
的上青州原籍家去了。”统制便道:“这厮我倒看他,原来这等无恩!等我慢慢差
人拿他去。”这春梅也不题起韩爱姐之事。

过了几日,春梅见统制日逐理论军情,干朝廷国务,焦心劳思,日中尚未暇食,至
于房帏色欲之事,久不沾身。因见老家人周忠次子周义,年十九岁,生的眉清目秀
,眉来眼去,两个暗地私通,就勾搭了。朝朝暮暮,两个在房中下棋饮酒,只瞒过
统制一人不知。

一日,不想北国大金皇帝灭了辽国。又见东京钦宗皇帝登基,集大势番兵,分两路
寇乱中原。大元帅粘没喝,领十万人马,出山西太原府井陉道,来抢东京;副帅斡
离不由檀州来抢高阳关。边兵抵挡不住,慌了兵部尚书李纲、大将种师道,星夜火
牌羽书,分调山东、山西、河南、河北、关东、陕西分六路统制人马,各依要地,
防守截杀。那时陕西刘延庆领延绥之兵,关东王禀领汾绛之兵,河北王焕领魏搏之
兵,河南辛兴宗领彰德之兵,山西杨惟忠领泽潞之兵,山东周秀领青兖之兵。

却说周统制,见大势番兵来抢边界,兵部羽书火牌星火来,连忙整率人马,全装披
挂,兼道进兵。比及哨马到高阳关上,金国干离不的人马,已抢进关来,杀死人马
无数。正值五月初旬,黄沙四起,大风迷目。统制提兵进赶,不防被干离不兜马反
攻,没鞦一箭,正射中咽喉,随马而死。众番将就用钩索搭去,被这边将士向前仅
抢尸首,马戴而远,所伤军兵无数。可怜周统制一旦阵亡,亡年四十七岁。正是:
于家为国忠良将,不辩贤愚血染沙。
古人意不尽,作诗一首,以叹之曰:
胜败兵家不可期,安危端自命为之。
出师未捷身先丧,落日江流不胜悲。

巡抚张叔夜,见统制没于阵上,连忙鸣金收军,查点折伤士卒,退守东昌。星夜奏
朝廷,不在话下。部下士卒,载尸首还到东昌府。春梅合家大小,号哭动天,合棺
木盛殓,交割了兵符印信。一日,春梅与家人周仁,发丧载灵柩归清河县不题。

话分两头。单表葛翠屏与韩爱姐,自从春梅去后,两个在家清茶淡饭,守节持贞,
过其日月。正值春尽夏初天气,景物鲜明,日长针指困倦。姊妹二人闲中徐步,到
西书院花亭上。见百花盛开,莺啼燕语,触景伤情。葛翠屏心还坦然,这韩爱姐,
一心只想念陈敬济,凡事无情无绪,睹物伤悲,不觉潸然泪下。姊妹二人正在悲凄
之际,只见二爷周宣,走来劝道:“你姊妹两个少要烦恼,须索解叹。我连日做得
梦,有些不吉。梦见一张弓挂在旗竿上,旗竿折了,不知是凶是吉?”韩爱姐道:
”倒只怕老爷边上,有些说话。”正在犹疑之间,忽见家人周仁,挂着一身孝,慌
慌张张走来,报道:“祸事,老爷如此这般,五月初七日,在边关上阵亡了!大奶
奶、二奶奶家眷,载着灵车都来了。”慌了二爷周宣,收拾打扫前厅干净,停放灵
柩,摆下祭祀,合家大小,哀号起来。一面做斋累七,僧道念经。金哥、玉姐披麻
带孝,吊客往来,择日出殡,安葬于祖茔。俱不必细说。

却说二爷周宣,引着六岁金哥儿,行文书申奏朝廷,讨祭葬,袭替祖职。朝廷明降
,兵部覆题引奏:已故统制周秀,奋身报国,没于王事,忠勇可嘉。遣官谕祭一坛
,墓顶追封都督之职。伊子照例优养,出幼袭替祖职。

这春梅在内颐养之余,淫情愈盛。常留周义在香阁中,镇日不出。朝来暮往,淫欲
无度,生出骨蒸痨病症。逐日吃药,减了饮食,消了精神,体瘦如柴,而贪淫不已
。一日,过了他生辰,到六月伏暑天气,早辰晏起,不料他搂着周义在床上,一泄
之后,鼻口皆出凉气,淫津流下一洼口,就鸣呼哀哉,死在周义身上。亡年二十九
岁。这周义见没了气儿,就慌了手脚,向箱内抵盗了些金银细软,带在身边,逃走
出外。丫鬟养娘不敢隐匿,报与二爷周宣得知。把老家人周忠锁了,押着抓寻周义
。可霎作怪,正走在城外他姑娘家投住,一条索子拴将来。已知其情,恐扬出丑去
,金哥久后不可袭职,拿到前厅,不由分说,打了四十大棍,即时打死。把金哥与
孙二娘看着。一面发丧于祖茔,与统制合葬毕。房中两个养娘并海棠、月桂,都打
发各寻投向嫁人去了。止有葛翠屏与韩爱姐,再三劝他,不肯前去。

一日,不想大金人马抢了东京汴梁,太上皇帝与靖康皇帝,都被虏上北地去了。中
原无主,四下荒乱。兵戈匝地,人民逃窜。黎庶有涂炭之哭,百姓有倒悬之苦。大
势番兵已杀到山东地界,民间夫逃妻散,鬼哭神号,父子不相顾。葛翠屏已被他娘
家领去,各逃生命。止丢下韩爱姐,无处依倚,不免收拾行装,穿着随身惨淡衣衫
,出离了清河县,前往临清找寻他父母。到临清谢家店,店也关闭,主人也走了。
不想撞见陈三儿,三儿说:“你父母去年就跟了何官人,往江南湖州去了。”

这韩爱姐一路上怀抱月琴,唱小词曲,往前抓寻父母。随路饥餐渴饮,夜住晓行,
忙忙如丧家之犬,急急如漏网之鱼。弓鞋又小,千辛万苦。行了数日,来到徐州地
方,天色晚了,投在孤村里面。一个婆婆,年纪七旬之上,正在灶上杵米造饭。这
韩爱姐便向前道了万福,告道:“奴家是清河县人氏,因为荒乱,前往江南投亲,
不期天晚,权借婆婆这里投宿一宵,明早就行,房金不少。”那婆婆看这女子,不
是贫难人家婢女,生得举止典雅,容貌非俗。因说道:“既是投宿,娘子请炕上坐
,等老身造饭,有几个挑河夫子来吃。”那老婆婆炕上柴灶,登时做出一大锅稗稻
插豆子干饭,又切了两大盘生菜,撮上一包盐,只见几个汉子,都蓬头精腿,裈裤
兜裆,脚上黄泥,进来放下锹镢,便问道:“老娘有饭也未?”婆婆道:“你每自
去盛吃。”

当下各取饭菜,四散正吃。只见内一人,约四十四五年纪,紫面黄发,便问婆婆:
”这炕上坐的是甚么人?”婆婆道:“此位娘子,是清河县人氏,前往江南寻父母
去,天晚在此投宿。”那人便问:“娘子,你姓甚么?”爱姐道:“奴家姓韩,我
父亲名韩道国。”那人向前扯住问道:“姐姐,你不是我侄女韩爱姐么?”那爱姐
道:“你倒好似我叔叔韩二。”两个抱头相哭做一处。因问:“你爹娘在那里?你
在东京,如何至此?”这韩爱姐一五一十,从头说了一遍,”因我嫁在守备府里,
丈夫没了,我守寡到如今。我爹娘跟了何官人,往湖州去了。我要找寻去,荒乱中
又没人带去,胡乱单身唱词,觅些衣食前去,不想在这里撞见叔叔。”那韩二道:
”自从你爹娘上东京,我没营生过日,把房儿卖了,在这里挑河做夫子,每日觅碗
饭吃。既然如此,我和你往湖州,寻你爹娘去。”爱姐道:“若是叔叔同去,可知
好哩。”当下也盛了一碗饭,与爱姐吃。爱姐呷了一口,见粗饭,不能咽,只呷了
半碗,就不吃了。一宿晚景题过。

到次日到明,众夫子都去了,韩二交纳了婆婆房钱,领爱姐作辞出门,望前途所进
。那韩爱姐本来娇嫩,弓鞋又小,身边带着些细软钗梳,都在路上零碎盘缠。将到
淮安上船,迤逶望江南湖州来,非止一日,抓寻到湖州何官人家,寻着父母,相见
会了。不想何官人已死,家中又没妻小,止是王六儿一人,丢下六岁女儿,有几顷
水稻田地。不上一年,韩道国也死了。王六儿原与韩二旧有揸儿,就配了小叔,种
田过日。那湖州有富家子弟,见韩爱姐生的聪明标致,都来求亲。韩二再三教他嫁
人,爱姐割发毁目,出家为尼,誓不再配他人。后来至三十一岁,无疾而终。正是
:
贞骨未归三尺土,怨魂先彻九重天。
后韩二与王六儿成其夫妇,请受何官人家业田地,不在话下。

却说大金人马,抢过东昌府来,看看到清河县地界。只见官吏逃亡,城门昼诸,人
民逃窜,父子流亡。但见:
烟生四野,日蔽黄沙。封豕长蛇,互相吞噬。龙争虎斗,各自争强。皂帜红旗,布
满郊野。男啼女哭,万户惊惶。番军虏将,一似蚁聚蜂屯;短剑长枪,好似森森密
竹。一处处死尸朽骨,横三竖四;一攒攒折刀断剑,七断八截。个个携男抱女,家
家闭门关户。十室九空,不显乡村城郭;獐奔鼠窜,那契礼乐衣冠。正是:得多少
宫人红袖哭,王子白衣行。

那时,吴月娘见番兵到了,家家都关锁门户,乱窜逃去,不免也打点了些金珠宝玩
,带在身边。那时吴大舅已死,止同吴三舅、玳安、小玉,领着十五岁孝哥儿,把
家中前后都倒锁了,要往济南府投奔云理守。一来避兵,二者与孝哥完就亲事。一
路上只见人人荒乱,个个惊骇。可怜这吴月娘,穿着随身衣服,和吴二舅男女五口
,杂在人队里挨出城门,到于郊外,往前奔行。到于空野十字路口,只见一个和尚
,身披紫褐袈裟,手执九环锡杖,脚趿芒鞋,肩上背着条布袋,袋内裹着经典,大
移步迎将来,与月娘打了个问讯,高声大叫道:“吴氏娘子,你到那里去?还与我
徒弟来!”唬的月娘大惊失色,说道:“师父,你问我讨甚么徒弟?”那和尚又道
:“娘子,你休推睡里梦里,你曾记的十年前,在岱岳东峰,被殷天锡赶到我山洞
中投宿。我就是那雪洞老和尚,法号普静。你许下我徒弟,如何不与我?”吴二舅
便道:“师父出家人,如何不近道?此等荒乱年程,乱窜逃生,他有此孩儿,久后
还要接代香火,他肯舍与你出家去?”和尚道:“你真个不与我去?”吴二舅道:
”师父,你休闲说,误了人的去路。后面只怕番兵来到,朝不保暮。”和尚道:“
你既不与我徒弟,如今天色已晚,也走不出路去。番人就来,也不到此处,你且跟
我到这寺中歇一夜,明早去罢。”吴月娘问:“师父,是那寺中?”那和尚用手只
一指,道:“那路旁便是。”和尚引着来到永福寺。吴月娘认的是永福寺,曾走过
一遭。

比及来到寺中,长老僧众都走去大半,止有几个禅和尚在后边打座。佛前点着一大
盏硫璃海灯,烧看一炉香。已是日色衔山时分,当晚吴月娘与吴二舅、玳安、小玉
、孝哥儿,男女五口儿,投宿在寺中方丈内。小和尚有认的,安排了些饭食,与月
娘等吃了。那普静老师,跏趺在禅堂床上敲木鱼,口中念经。月娘与孝哥儿、小玉
在床上睡,吴二舅和玳安做一处,着了荒乱辛苦底人,都睡着了。止有小玉不曾睡
熟,起来在方丈内,打门缝内看那普静老师父念经。看看念至三更时,只见金风凄
凄,斜月朦朦,人烟寂静,万籁无声。佛前海灯,半明不暗。这普静老师见天下荒
乱,人民遭劫,阵亡横死者极多,发慈悲心,施广惠力,礼白佛言,荐拔幽魂,解
释宿冤,绝去挂碍,各去超生。于是诵念了百十遍解冤经咒。少顷,阴风凄凄,冷
气飕飕。有数十辈焦头烂额,蓬头泥面者,或断手折臂者,或有刳腹剜心者,或有
无头跛足者,或有吊颈枷锁者,都来悟领禅师经咒,列于两旁。禅师便道:“你等
众生,冤冤相报,不肯解脱,何日是了?汝当谛听吾言,随方托化去罢。偈曰:
劝尔莫结冤,冤深难解结。
一日结成冤,千日解不彻。
若将冤解冤,如汤去泼雪。
我见结冤人,尽被冤磨折。
我今此忏悔,各把性悟彻。
照见本来心,冤愆自然雪。
仗此经力深,荐拔诸恶业。
汝当各托生,再勿将冤结。

当下众魂都拜谢而去。小玉窃看,都不认得。少顷,又一大汉进来,身长七尺,形
容魁伟,全装贯甲,胸前关着一矢箭,自称”统制周秀,因与番将对敌,折于阵上
,今蒙师荐拔,今往东京,托生于沈镜为次子,名为沈守善去也。”言未已,又一
人,素体荣身,口称是清河县富户西门庆,”不幸溺血而死,今蒙师荐拔,今往东
京城内,托生富户沈通为次子沈越去也。”小玉认的是他爹,唬的不敢言语。已而
又有一人,提着头,浑身皆血,自言是陈敬济,”因被张胜所杀,蒙师经功荐拔,
今往东京城内,与王家为子去也。”已而又见一妇人,也提着头,胸前皆血。自言
:“奴是武大妻、西门庆之妾潘氏是也。不幸被仇人武松所杀。蒙师荐拔,今往东
京城内黎家为女托生去也。”已而又有一人,身躯矮小,面背青色,自言是武植,
”因被王婆唆潘氏下药吃毒而死,蒙师荐拔,今往徐州乡民范家为男,托生去也。
”已而又有一妇人,面色黄瘦,血水淋漓,自言:“妾身李氏,乃花子虚之妻,西
门庆之妾,因害血山崩而死。蒙师荐拔,今往东京城内,袁指挥家托生为女去也。
”已而又一男,自言花子虚,”不幸被妻气死,蒙师荐拔,今往东京郑千户家托生
为男。”已而又见一女人,颈缠脚带,自言西门庆家人来旺妻宋氏,”自缢身死,
蒙师荐拔,今往东京朱家为女去也。”已而又一妇人,面黄肌瘦,自言周统制妻庞
氏春梅,”因色痨而死,蒙师荐拔,今往东京与孔家为女,托生去也。”已而又一
男子,裸形披发,浑身杖痕,自言是打死的张胜,”蒙师荐拔,今往东京大兴卫贫
人高家为男去也。”已而又有一女人,项上缠着索子,自言是西门庆妾孙雪娥,不
幸自缢身死,”蒙师荐拔,今往东京城外贫民姚家为女去也。”已而又一女人,年
小,项缠脚带,自言”西门庆之女,陈敬济之妻,西门大姐是也,不幸亦缢身死,
蒙师荐拔,今往东京城外,与番役钟贵为女,托生去也。”已而又见一小男子,自
言周义,”亦被打死,蒙师荐拔,今往东京城外高家为男,名高留住儿,托生去也
。”言毕,各恍然不见。小玉唬的战栗不已。原来这和尚,只是和这些鬼说话。

正欲向床前告诉吴月娘,不料月娘睡得正熟,一灵真性,同吴二舅众男女,身带着
一百颗胡珠,一柄宝石绦环,前往济南府,投奔亲家云理守。一路到于济南府,寻
问到云参将寨门,通报进去。云参将听见月娘送亲来了,一见如故。叙毕礼数。原
来新近没了娘子,央浼邻舍王婆来陪待月娘,在后堂酒饭,甚是丰盛。吴二舅、玳
安另在一处管待。因说起避兵就亲之事,因把那百颗胡珠、宝石、绦环教与云理守
,权为茶礼。云理守收了,并不言其就亲之事。到晚,又教王婆陪月娘一处歇卧。
将言说念月娘,以挑探其意,说:“云理守虽武官,乃读书君子,从割衫襟之时,
就留心娘子。不期夫人没了,鳏居至今。今据此山城,虽是任小,上马管军,下马
管民,生杀在于掌握。娘子若不弃,愿成伉俪之欢,一双两好,令郎亦得谐秦晋之
配。等待太平之日,再回家去不迟。”月娘听言,大惊失色,半晌无言。这王婆回
报云理寺。

次日夕晚,置酒后堂,请月娘吃酒。月娘只知他与孝哥儿完亲,连忙来到席前叙坐
。云理守乃道:“嫂嫂不知,下官在此虽是山城,管着许多人马,有的是财帛衣服
,金银宝物,缺少一个主家娘子。下官一向思想娘子,如喝思浆,如热思凉。不想
今日娘子到我这里与令郎完亲,天赐姻缘,一双两好,成其夫妇,在此快活一世,
有何不可?”月娘听了,心中大怒,骂道:“云理守,谁知你人皮包着狗骨!我过
世丈夫不曾把你轻待,如何一旦出此犬马之言?”云理守笑嘻嘻向前,把月娘搂住
,求告说:“娘子,你自家中,如何走来我这里做甚?自古上门买卖好做,不知怎
的,一见你,魂灵都被你摄在身上。没奈何,好歹完成了罢。”一面拿过酒来和月
娘吃。月娘道:“你前边叫我兄弟来,等我与他说句话。”云理守笑道:“你兄弟
和玳安儿小厮,已被我杀了。”即令左右:“取那件物事,与娘子看。”不一时,
灯光下,血沥沥提了吴二舅、玳安两颗头来。唬的月娘面如土色,一面哭倒在地。
被云理守向前抱起:“娘子不须烦恼,你兄弟已死,你就与我为妻。我一个总兵官
,也不玷辱了你。”月娘自思道:“这贼汉将我兄弟家人害了命,我若不从,连我
命也丧了。”乃回嗔作喜,说道:“你须依我,奴方与你做夫妻。”云理守道:“
不拘甚事,我都依。”月娘道:“你先与我孩儿完了房,我却与你成婚。”云理守
道:“不打紧。”一面叫出云小姐来,和孝哥儿推在一处,饮合卺杯,绾同心结,
成其夫妇。然后扯月娘和他云雨。这月娘却拒阻不肯,被云理守忿然大怒,骂道:
”贱妇!你哄的我与你儿子成了婚姻,敢笑我杀不得你的孩儿?”向床头提剑,随
手而落,血溅数步之远。正是:
三尺利刀着项上,满腔鲜血湿模糊。

月娘见砍死孝哥儿,不觉大叫一声。不想撒手惊觉,却是南柯一梦。唬的浑身是汗
,遍体生津。连道:“怪哉,怪哉。”小玉在旁,便问:“奶奶怎的哭?”月娘道
:“适间做得一梦不详。”不免告诉小玉一遍。小玉道:“我倒刚才不曾睡着,悄
悄打门缝见那和尚原来和鬼说了一夜话。刚才过世俺爹、五娘、六娘和陈姐夫、周
守备、孙雪娥、来旺儿媳妇子、大姐都来说话,各四散去了。”月娘道:“这寺后
见埋着他每,夜静时分,屈死淹魂如何不来!”

娘儿们说了回话,不觉五更,鸡叫天明。吴月娘梳洗面貌,走到禅堂中,礼佛烧香
。只见普静老师在禅床上高叫:“那吴氏娘子,你如何可省悟得了么?”这月娘便
跪下参拜:“上告尊师,弟子吴氏,肉眼凡胎,不知师父是一尊古佛。适间一梦中
都已省悟了。”老师道:“既已省悟,也不消前去,你就去,也无过只是如此。倒
没的丧了五口儿性命。你这儿子,有分有缘遇着我,都是你平日一点善根所种。不
然,定然难免骨肉分离。当初,你去世夫主西门庆造恶非善,此子转身托化你家,
本要荡散其财本,倾覆其产业,临死还当身首羿处。今我度脱了他去,做了徒弟,
常言'一子出家,九祖升天',你那夫主冤愆解释,亦得超生去了。你不信,跟我
来,与你看一看。”于是叉步来到方丈内,只见孝哥儿还睡在床上。老师将手中禅
杖,向他头上只一点,教月娘众人看。忽然翻过身来,却是西门庆,项带沉枷,腰
系铁索。复用禅杖只一点,依旧是孝哥儿睡在床上。月娘见了,不觉放声大哭,原
来孝哥儿即是西门庆托生。

良久,孝哥儿醒了。月娘问他:“如何你跟了师父出家。”在佛前与他剃头,摩顶
受记。可怜月娘扯住恸哭了一场,干生受养了他一场。到十五岁,指望承家嗣业,
不想被这老师幻化去了。吴二舅、小玉、玳安亦悲不胜。当下这普静老师,领定孝
哥儿,起了他一个法名,唤做明悟。作辞月娘而去。临行,分付月娘:“你们不消
往前途去了。如今不久番兵退去,南北分为两朝,中原已有个皇帝,多不上十日,
兵戈退散,地方宁静了,你每还回家去安心度日。”月娘便道:“师父,你度托了
孩儿去了,甚年何日我母子再得见面?”不觉扯住,放声大哭起来。老师便道:“
娘子休哭!那边又有一位老师来了。”哄的众人扭颈回头,当下化阵清风不见了。
正是:  三降尘寰人不识,倏然飞过岱东峰。

不说普静老师幻化孝哥儿去了,且说吴月娘与吴二舅众人,在永福寺住了十日光景
,果然大金国立了张邦昌在东京称帝,置文武百官。徽宗、钦宗两君北,康王泥马
渡江,在建康即位,是为高宗皇帝。拜宗泽为大将,复取山东、河北。分为两朝,
天下太平,人民复业。后月娘归家,开了门户,家产器物都不曾疏失。后就把玳安
改名做西门庆,承受家业,人称呼为”西门小员外”。养活月娘到老,寿年七十岁
,善终而亡。此皆平日好善看经之报。有诗为证:

阀阅遗书思惘然,谁知天道有循环。
西门豪横难存嗣,敬济颠狂定被歼。
楼月善良终有寿,瓶梅淫佚早归泉。
可怪金莲遭恶报,遗臭千年作话传。

【全书完】