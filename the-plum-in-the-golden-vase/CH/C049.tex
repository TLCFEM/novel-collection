\chapter{请巡按屈体求荣~遇胡僧现身施药}

诗曰:

雅集无兼客,高情洽二难。
一尊倾智海,八斗擅吟坛。
话到如生旭,霜来恐不寒。
为行王舍乞,玄屑带云餐。

话说夏寿到家回复了话,夏提刑随即就来拜谢西门庆,说道:“长官活命之恩
,不是托赖长官余光这等大力量,如何了得!”西门庆笑道:“长官放心。料着你
我没曾过为,随他说去,老爷那里自有个明见。”一面在厅上放桌儿留饭,谈笑至
晚,方才作辞回家。到次日,依旧入衙门里理事,不在话下。

却表巡按曾公见本上去不行,就知道二官打点了,心中忿怒。因蔡太师所陈七
事,内多舛讹,皆损下益上之事,即赴京见朝覆命,上了一道表章。极言:“天下
之财贵于通流,取民膏以聚京师,恐非太平之治。民间结粜[亻表]籴之法不可行
,当十大钱不可用,盐钞法不可屡更。臣闻民力殚矣,谁与守邦?”蔡京大怒,奏
上徽宗天子,说他大肆倡言,阻挠国事。将曾公付吏部考察,黜为陕西庆州知州。
陕西巡按御史宋盘,就是学士蔡攸之妇兄也。太师阴令盘就劾其私事,逮其家人,
锻炼成狱,将孝序除名,窜于岭表,以报其仇。此系后事,表过不题。

再说西门庆在家,一面使韩道国与乔大户外甥崔本,拿仓钞早往高阳关户部韩
爷那里赶着挂号。留下来保家中定下果品,预备大桌面酒席,打听蔡御史船到。一
日,来保打听得他与巡按宋御史船一同京中起身,都行至东昌府地方,使人来家通
报。这里西门庆就会夏提刑起身。来保从东昌府船上就先见了蔡御史,送了下程。
然后,西门庆与夏提刑出郊五十里迎接到新河口──地名百家村。先到蔡御史船上
拜见了,备言邀请宋公之事。蔡御史道:“我知道,一定同他到府。”那时,东平
胡知府,及合属州县方面有司军卫官员、吏典生员、僧道阴阳,都具连名手本,伺
候迎接。帅府周守备、荆都监、张团练,都领人马披执跟随,清跸传道,鸡犬皆隐
迹。鼓吹迎接宋巡按进东平府察院,各处官员都见毕,呈递了文书,安歇一夜。

到次日,只见门吏来报:“巡盐蔡爷来拜。”宋御史连忙出迎。叙毕礼数,分
宾主坐下。献茶已毕,宋御史便问:“年兄几时方行?”蔡御史道:“学生还待一
二日。”因告说:“清河县有一相识西门千兵,乃本处巨族,为人清慎,富而好礼
,亦是蔡老先生门下,与学生有一面之交。蒙他远接,学生正要到他府上拜他拜。
”宋御史问道:“是那个西门千兵?”蔡御史道:“他如今见是本处提刑千户,昨
日已参见过年兄了。”宋御史令左右取手本来看,见西门庆与夏提刑名字,说道:
“此莫非与翟云峰有亲者?”蔡御史道:“就是他。如今见在外面伺候,要央学生
奉陪年兄到他家一饭。未审年兄尊意若何?”宋御史道:“学生初到此处,只怕不
好去得。”蔡御史道:“年兄怕怎的?既是云峰分上,你我走走何害?”于是吩咐
看轿,就一同起行,一面传将出来。

西门庆知了此消息,与来保、贲四骑快马先奔来家,预备酒席。门首搭照山彩
棚,两院乐人奏乐,叫海盐戏并杂耍承应。原来宋御史将各项伺候人马都令散了,
只用几个蓝旗清道官吏跟随,与蔡御史坐两顶大轿,打着双檐伞,同往西门庆家来
。当时哄动了东平府,大闹了清河县,都说:“巡按老爷也认的西门大官人,来他
家吃酒来了。”慌的周守备、荆都监、张团练,各领本哨人马把住左右街口伺候。
西门庆青衣冠带,远远迎接。两边鼓乐吹打,到大门首下了轿进去。宋御史与蔡御
史都穿着大红獬豸绣服,乌纱皂履,鹤顶红带,从人执着两把大扇。只见五间厅上
湘帘高卷,锦屏罗列。正面摆两张吃看桌席,高顶方糖,定胜簇盘,十分齐整。二
官揖让进厅,与西门庆叙礼。蔡御史令家人具贽见之礼:两端湖绸、一部文集、四
袋芽茶、一方端溪砚。宋御史只投了个宛红单拜帖,上书“侍生宋乔年拜”。向西
门庆道:“久闻芳誉。学生初临此地,尚未尽情,不当取扰。若不是蔡年兄邀来进
拜,何以幸接尊颜?”慌的西门庆倒身下拜,说道:“仆乃一介武官,属于按临之
下。今日幸蒙清顾,蓬荜生光。”于是鞠恭展拜,礼容甚谦。宋御史亦答礼相还,
叙了礼数。当下蔡御史让宋御史居左,他自在右,西门庆垂首相陪。茶汤献罢,阶
下箫韶盈耳,鼓乐喧阗,动起乐来。西门庆递酒安席已毕,下边呈献割道。说不尽
肴列珍羞,汤陈桃浪,端的歌舞声容,食前方丈。两位轿上跟从人,每位五十瓶酒
、五百点心、一百斤熟肉,都领下去。家人、吏书、门子人等,另在厢房中管待,
不必细说。当日西门庆这席酒,也费够千两金银。

那宋御史又系江西南昌人,为人浮躁,只坐了没多大回,听了一折戏文就起来
。慌的西门庆再三固留。蔡御史在旁便说:“年兄无事,再消坐一时,何遽回之太
速耶!”宋御史道:“年兄还坐坐,学生还欲到察院中处分些公事。”西门庆早令
手下,把两张桌席连金银器,已都装在食盒内,共有二十抬,叫下人夫伺候。宋御
史的一张大桌席、两坛酒、两牵羊、两封金丝花、两匹段红、一副金台盘、两把银
执壶、十个银酒杯、两个银折盂、一双牙箸。蔡御史的也是一般的。都递上揭帖。
宋御史再三辞道:“这个,我学生怎么敢领?”因看着蔡御史。蔡御史道:“年兄
贵治所临,自然之道,我学生岂敢当之!”西门庆道:“些须微仪,不过侑觞而已
,何为见外?”比及二官推让之次,而桌席已抬送出门矣。宋御史不得已,方令左
右收了揭帖,向西门庆致谢说道:“今日初来识荆,既扰盛席,又承厚贶,何以克
当?余容图报不忘也。”因向蔡御史道:“年兄还坐坐,学生告别。”于是作辞起
身。西门庆还要远送,宋御史不肯,急令请回,举手上轿而去。

西门庆回来,陪侍蔡御史,解去冠带,请去卷棚内后坐。因吩咐把乐人都打发
散去,只留下戏子。西门庆令左右重新安放桌席,摆设珍羞果品上来,二人饮酒。
蔡御史道:“今日陪我这宋年兄坐便僭了,又叨盛筵并许多酒器,何以克当?”西
门庆笑道:“微物惶恐,表意而已!”因问道:“宋公祖尊号?”蔡御史道:“号
松原。松树之松,原泉之原。”又说起:“头里他再三不来,被学生因称道四泉盛
德,与老先生那边相熟,他才来了。他也知府上与云峰有亲。”西门庆道:“想必
翟亲家有一言于彼。我观宋公为人有些蹊跷。”蔡御史道:“他虽故是江西人,倒
也没甚蹊跷处。只是今日初会,怎不做些模样!”说毕笑了。西门庆便道:“今日
晚了,老先生不回船上去罢了。”蔡御史道:“我明早就要开船长行。“西门庆道
:“请不弃在舍留宿一宵,明日学生长亭送饯。”蔡御史道:“过蒙爱厚。”因吩
咐手下人:“都回门外去罢,明早来接。”众人都应诺去了,只留下两个家人伺候
。

西门庆见手下人都去了,走下席来,叫玳安儿附耳低言,如此这般:“即去院
里坐名叫了董娇儿、韩金钏儿两个,打后门里用轿子抬了来,休交一人知道。”那
玳安一面应诺去了。西门庆复上席,陪蔡御史吃酒。海盐子弟在旁歌唱。西门庆因
问:“老先生到家多少时就来了?令堂老夫人起居康健么?”蔡御史道:“老母到
也安。学生在家,不觉荏苒半载,回来见朝,不想被曹禾论劾,将学生敝同年一十
四人之在史馆者,一时皆黜授外职。学生便选在西台,新点两淮巡盐。宋年兄便在
贵处巡按,也是蔡老先生门下。”西门庆问道:“如今安老先生在那里?”蔡御史
道:“安凤山他已升了工部主事,往荆州催攒皇木去了。也待好来也。”说毕,西
门庆教海盐子弟上来递酒。蔡御史吩咐:“你唱个《渔家傲》我听。”子弟排手在
旁正唱着,只见玳安走来请西门庆下边说话。玳安道:“叫了董娇儿、韩金钏打后
门来了,在娘房里坐着哩。”西门庆道:“你吩咐把轿子抬过一边才好。”玳安道
:“抬过一边了。”

这西门庆走至上房,两个唱的向前磕头。西门庆道:“今日请你两个来,晚夕
在山子下扶侍你蔡老爹。他如今见做巡按御史,你不可怠慢,用心扶侍他,我另酬
答你。”韩金钏儿笑道:“爹不消吩咐,俺每知道。”西门庆因戏道:“他南人的
营生,好的是南风,你每休要扭手扭脚的。”董娇儿道:“娘在这里听着,爹你老
人家羊角葱靠南墙──越发老辣了。王府门首磕了头,俺们不吃这井里水了?”

西门庆笑的往前边来。走到仪门首,只见来保和陈敬济拿着揭帖走来,与西门
庆看,说道:“刚才乔亲家爹说,趁着蔡老爹这回闲,爹倒把这件事对蔡老爹说了
罢,只怕明日起身忙了。教姐夫写了俺两个名字在此。”西门庆道:“你跟了来。
”来保跟到卷棚槅子外边站着。西门庆饮酒中间因题起:“有一事在此,不
敢干渎。”蔡御史道:“四泉,有甚事只顾吩咐,学生无不领命。”西门庆道:“
去岁因舍亲在边上纳过些粮草,坐派了些盐引,正派在贵治扬州支盐。望乞到那里
青目青目,早些支放就是爱厚。”因把揭帖递上去,蔡御史看了。上面写着:“商
人来保、崔本,旧派淮盐三万引,乞到日早掣。”蔡御史看了笑道:“这个甚么打
紧。”一面把来保叫至跟前跪下,吩咐:“与你蔡爷磕头。”蔡御史道:“我到扬
州,你等径来察院见我。我比别的商人早掣一个月。”西门庆道:“老先生下顾,
早放十日就够了。”蔡御史把原帖就袖在袖内。一面书童旁边斟上酒,子弟又唱。

唱毕,已有掌灯时分,蔡御史便说:“深扰一日,酒告止了罢。”因起身出席
,左右便欲掌灯,西门庆道:“且休掌烛,请老先生后边更衣。”于是从花园里游
玩了一回,让至翡翠轩,那里又早湘帘低簇,银烛荧煌,设下酒席。海盐戏子,西
门庆已命打发去了。书童把卷棚内家活收了,关上角门,只见两个唱的盛妆打扮,
立于阶下,向前插烛也似磕了四个头。但见:

绰约容颜金缕衣,香尘不动下阶墀。
时来水溅罗裙湿,好似巫山行雨归。

蔡御史看见,欲进不能,欲退不舍。便说道:“四泉,你如何这等爱厚?恐使不得
。”西门庆笑道:“与昔日东山之游,又何异乎?”蔡御史道:“恐我不如安石之
才,而君有王右军之高致矣。”于是月下与二妓携手,恍若刘阮之入天台。因进入
轩内,见文物依然,因索纸笔就欲留题相赠。西门庆即令书童连忙将端溪砚研的墨
浓浓的,拂下锦笺。这蔡御史终是状元之才,拈笔在手,文不加点,字走龙蛇,灯
下一挥而就,作诗一首。诗曰:

不到君家半载余,轩中文物尚依稀。
雨过书童开药圃,风回仙子步花台。
饮将醉处钟何急,诗到成时漏更催。
此去又添新怅望,不知何日是重来。

写毕,教书童粘于壁上,以为后日之遗焉。因问二妓:“你们叫甚名字?”一个道
:“小的姓董,名唤娇儿。他叫韩金钏儿。”蔡御史又道:“你二人有号没有?”
董娇儿道:“小的无名娼妓,那讨号来?”蔡御史道:“你等休要太谦。”问至再
三,韩金钏方说:“小的号玉卿。”董娇儿道:“小的贱号薇仙。”蔡御史一闻“
薇仙”二字,心中甚喜,遂留意在怀。令书童取棋桌来,摆下棋子,蔡御史与董娇
儿两个着棋。西门庆陪侍,韩金钏儿把金樽在旁边递酒,书童歌唱。蔡御史赢了一
盘棋,董娇儿吃过,又回奉蔡御史一杯。韩金钏这里也递与西门庆一杯陪饮。饮了
酒,两人又下。董娇儿赢了,连忙递酒一杯与蔡御史,西门庆在旁又陪饮一杯。饮
毕,蔡御史道:“四泉,夜深了,不胜酒力,”于是走出外边来,站立在花下。

那时正是四月半头,月色才上。西门庆道:“老先生,天色还早哩。还有韩金
钏,不曾赏他一杯酒。”蔡御史道:“正是。你唤他来,我就此花下立饮一杯。”
于是韩金钏拿大金桃杯,满斟一杯,用纤手捧递上去。董娇儿在旁捧果,蔡御史吃
过,又斟了一杯,赏与韩金钏儿。因告辞道:“四泉,今日酒大多了,令盛价收过
去罢。”于是与西门庆握手相语,说道:“贤公盛情盛德,此心悬悬。非斯文骨肉
,何以至此?向日所贷,学生耿耿在心,在京已与云峰表过。倘我后日有一步寸进
,断不敢有辜盛德。”西门庆道:“老先生何出此言?到不消介意。”

韩金钏见他一手拉着董娇儿,知局,就往后边去了。到了上房里,月娘问道:
“你怎的不陪他睡,来了?”韩金钏笑道:“他留下董娇儿了,我不来,只管在那
里做甚么?”良久,西门庆亦告了安置进来,叫了来兴儿吩咐:“明日早五更,打
发食盒酒米点心下饭,叫了厨役,跟了往门外永福寺去,与你蔡老爹送行。叫两个
小优儿答应。休要误了。”来兴儿道:“家里二娘上寿,没有人看。”西门庆道:
“留下棋童儿买东西,叫厨子后边大灶上做罢。”

不一时,书童、玳安收下家活来,又讨了一壶好茶,往花园里去与蔡老爹漱口
。翡翠轩书房床上,铺陈衾枕俱各完备。蔡御史见董娇儿手中拿着一把湘妃竹泥金
面扇儿,上面水墨画着一种湘兰平溪流水。董娇儿道:“敢烦老爹赏我一首诗在上
面。”蔡御史道:“无可为题,就指着你这薇仙号。”于是灯下拈起笔来,写了四
句在上:

小院闲庭寂不哗,一池月上浸窗纱。
邂逅相逢天未晚,紫薇郎对紫薇花。

写毕,那董娇儿连忙拜谢了。两个收拾上床就寝。书童、玳安与他家人在明间里睡
。一宿晚景不题。

次日早晨,蔡御史与了董娇儿一两银子,用红纸大包封着,到于后边,拿与西
门庆瞧。西门庆笑说道:“文职的营生,他那里有大钱与你!这个就是上上签了。
”因交月娘每人又与了他五钱银子,从后门打发去了。书童舀洗面水,打发他梳洗
穿衣。西门庆出来,在厅上陪他吃了粥。手下又早伺候轿马来接,与西门庆作辞,
谢了又谢。西门庆又道:“学生日昨所言之事,老先生到彼处,学生这里书去,千
万留神一二,足仞不浅。”蔡御史道:“休说贤公华扎下临,只盛价有片纸到,学
生无不奉行。”说毕,二人同上马,左右跟随。出城外,到于永福寺,借长老方丈
摆酒饯行。来兴儿与厨役早已安排桌席停当。李铭、吴惠两个小优弹唱。

数杯之后,坐不移时,蔡御史起身,夫马、坐轿在于三门外伺候。临行,西门
庆说起苗青之事:“乃学生相知,因诖误在旧大巡曾公案下,行牌往扬州案候捉他
。此事情已问结了。倘见宋公,望乞借重一言,彼此感激。”蔡御史道:“这个不
妨,我见宋年兄说,设使就提来,放了他去就是了。”西门庆又作揖谢了。看官听
说:后来宋御史往济南去,河道中又与蔡御史会在那船上。公人扬州提了苗青来,
蔡御史说道:“此系曾公手里案外的,你管他怎的?”遂放回去了。倒下详去东平
府,还只把两个船家,决不待时,安童便放了。正是:

公道人情两是非,人情公道最难为。
若依公道人情失,顺了人情公道亏。

当日西门庆要送至船上,蔡御史不肯,说道:“贤公不消远送,只此告别。”西门
庆道:“万惟保重,容差小价问安。”说毕,蔡御史上轿而去。

西门庆回到方丈坐下,长老走来合掌问讯,递茶,西门庆答礼相还。见他雪眉
交白,便问:“长老多大年纪?”长老道:“小僧七十有四。”西门庆道:“到还
这等康健。”因问法号,长老道:“小僧法名道坚。”又问:“有几位徒弟?”长
老道:“止有两个小徒。本寺也有三十余僧行。”西门庆道:“这寺院也宽大,只
是欠修整。”长老道:“不满老爹说,这座寺原是周秀老爹盖造,长住里没钱粮修
理,丢得坏了。”西门庆道:“原来就是你守备府周爷的香火院。我见他家庄子不
远。不打紧处,你禀了你周爷,写个缘簿,别处也再化些,我也资助你些布施。”
道坚连忙又合掌问讯谢了。西门庆吩咐玳安儿:“取一两银子谢长老。今日打搅。
”道坚道:“小僧不知老爹来,不曾预备斋供。”西门庆道:“我要往后边更更衣
去。”道坚连忙叫小沙弥开门。西门庆更了衣,因见方丈后面五间大禅堂,有许多
云游和尚在那里敲着木鱼看经。西门庆不因不由,信步走入里面观看。见一个和尚
形骨古怪,相貌搊搜,生的豹头凹眼,色若紫肝,戴了鸡蜡箍儿,穿一领肉
红直裰。颏下髭须乱[扌乍],头上有一溜光檐,就是个形容古怪真罗汉,未除火
性独眼龙。在禅床上旋定过去了,垂着头,把脖子缩到腔子里,鼻孔中流下玉箸来
。西门庆口中不言,心中暗道:“此僧必然是个有手段的高僧。不然,如何因此异
相?等我叫醒他,问他个端的。”于是高声叫:“那位僧人,你是那里人氏,何处
高僧?”叫了头一声不答应;第二声也不言语;第三声,只见这个僧人在禅床上把
身子打了个挺,伸了伸腰,睁开一只眼,跳将起来,向西门庆点了点头儿,[分鹿
]声应道:“你问我怎的?贫僧行不更名,坐不改姓,乃西域天竺国密松林齐腰峰
寒庭寺下来的胡僧,云游至此,施药济人。官人,你叫我有甚话说?”西门庆道:
“你既是施药济人,我问你求些滋补的药儿,你有也没有?”胡僧道:“我有,我
有。”又道:“我如今请你到家,你去不去?”胡僧道:“我去,我去。”西门庆
道:“你说去,即此就行。”那胡僧直竖起身来,向床头取过他的铁柱杖来拄着,
背上他的皮褡裢──褡裢内盛了两个药葫芦儿。下的禅堂,就往外走。西门庆吩咐
玳安:“叫了两个驴子,同师父先往家去等着,我就来。”胡僧道:“官人不消如
此,你骑马只顾先行。贫僧也不骑头口,管情比你先到。”西门庆道:“一定是个
有手段的高僧。不然如何开这等朗言。”恐怕他走了,吩咐玳安:“好歹跟着他同
行。”于是作辞长老上马,仆从跟随,迳直进城来家。

那日四月十七日,不想是王六儿生日,家中又是李娇儿上寿,有堂客吃酒。后
晌时分,只见王六儿家没人使,使了他兄弟王经来请西门庆。吩咐他宅门首只寻玳
安儿说话,不见玳安在门首,只顾立。立了约一个时辰,正值月娘与李娇儿送院里
李妈妈出来上轿,看见一个十五六岁扎包髻儿小厮,问是那里的。那小厮三不知走
到跟前,与月娘磕了个头,说道:“我是韩家,寻安哥说话。”月娘问:“那安哥
?”平安在旁边,恐怕他知道是王六儿那里来的,恐怕他说岔了话,向前把他拉过
一边,对月娘说:“他是韩伙计家使了来寻玳安儿,问韩伙计几时来。”以此哄过
。月娘不言语,回后边去了。

不一时玳安与胡僧先到门首,走的两腿皆酸,浑身是汗,抱怨的要不的。那胡
僧体貌从容,气也不喘。平安把王六儿那边使了王经来请爹,寻他说话一节,对玳
安儿说了一遍,道:“不想大娘看见,早是我在旁边替他摭拾过了。不然就要露出
马脚来了。等住回娘若问,你也是这般说。”那玳安走的睁睁的,只顾[扌扉]扇
子:“今日造化低也怎的?平白爹交我领了这贼秃囚来。好近路儿!从门外寺里直
走到家,路上通没歇脚儿,走的我上气儿接不着下气儿。爹交雇驴子与他骑,他又
不骑。他便走着没事,难为我这两条腿了!把鞋底子也磨透了,脚也踏破了。攘气
的营生!”平安道:“爹请他来家做甚么?”玳安道:“谁知道!他说问他讨甚么
药哩。”正说着,只闻喝道之声。西门庆到家,看见胡僧在门首,说道:“吾师真
乃人中神也。果然先到。”一面让至里面大厅上坐。西门庆叫书童接了衣裳,换了
小帽,陪他坐的。吃了茶,那胡僧睁眼观见厅堂高远,院字深沉,门上挂的是龟背
纹虾须织抹绿珠帘,地下铺狮子滚绣球绒毛线毯。正当中放一张蜻蜓腿、螳螂肚、
肥皂色起楞的桌子,桌子上安着绦环样须弥座大理石屏风。周围摆的都是泥鳅头、
楠木靶肿筋的交倚,两壁挂的画都是紫竹杆儿绫边、玛瑙轴头。正是:

鼍皮画鼓振庭堂,乌木春台盛酒器。

胡僧看毕,西门庆问道:“吾师用酒不用?”胡僧道:“贫僧酒肉齐行。”西
门庆一面吩咐小厮:“后边不消看素馔,拿酒饭来。”那时正是李娇儿生日,厨下
肴馔下饭都有。安放桌儿,只顾拿上来。先绰边儿放了四碟果子、四碟小菜,又是
四碟案酒:一碟头鱼、一碟糟鸭、一碟乌皮鸡、一碟舞鲈公。又拿上四样下饭来:
一碟羊角葱[火川]炒的核桃肉、一碟细切的[饣皆][饣禾]样子肉、一碟肥肥
的羊贯肠、一碟光溜溜的滑鳅。次又拿了一道汤饭出来:一个碗内两个肉圆子,夹
着一条花肠滚子肉,名唤一龙戏二珠汤;一大盘裂破头高装肉包子。西门庆让胡僧
吃了,教琴童拿过团靶钩头鸡脖壶来,打开腰州精制的红泥头,一股一股邈出滋阴
摔白酒来,倾在那倒垂莲蓬高脚钟内,递与胡僧。那胡僧接放口内,一吸而饮之。
随即又是两样添换上来:一碟寸扎的骑马肠儿、一碟子腌腊鹅脖子。又是两样艳物
与胡僧下酒:一碟子癞葡萄、一碟子流心红李子。落后又是一大碗鳝鱼面与菜卷儿
,一齐拿上来与胡僧打散。登时把胡僧吃的楞子眼儿,便道:“贫僧酒醉饭饱,足
以够了。”

西门庆叫左右拿过酒桌去,因问他求房术的药儿。胡僧道:“我有一枝药,乃
老君炼就,王母传方。非人不度,非人不传,专度有缘。既是官人厚待于我,我与
你几丸罢。”于是向褡裢内取出葫芦来,倾出百十丸,吩咐:“每次只一粒,不可
多了,用烧酒送下。”又将那一个葫儿捏了,取二钱一块粉红膏儿,吩咐:“每次
只许用二厘,不可多用。若是胀的慌,用手捏着,两边腿上只顾摔打,百十下方得
通。你可樽节用之,不可轻泄于人。”西门庆双手接了,说道:“我且问你,这药
有何功效?”胡僧说:

形如鸡卵,色似鹅黄。三次老君炮炼,王母亲手传方。外视轻如粪土
,内觑贵乎[王干]琅。比金金岂换,比玉玉何[亻赏]!任你腰金衣紫
,任你大厦高堂,任你轻裘肥马,任你才俊栋梁,此药用托掌内,飘然身
人洞房。洞中春不老,物外景长芳;玉山无颓败,丹田夜有光。一战精神
爽,再战气血刚。不拘娇艳宠,十二美红妆,交接从吾好,彻夜硬如枪。
服久宽脾胃,滋肾又扶阳。百日须发黑,千朝体自强。固齿能明目,阳生
[女后]始藏。恐君如不信,拌饭与猫尝:三日淫无度,四日热难当;白
猫变为黑,尿粪俱停亡;夏月当风卧,冬天水里藏。若还不解泄,毛脱尽
精光。每服一厘半,阳兴愈健强。一夜歇十女,其精永不伤。老妇颦眉蹙
,淫娼不可当。有时心倦怠,收兵罢战场。冷水吞一口,阳回精不伤。快
美终宵乐,春色满兰房。赠与知音客,永作保身方。

西门庆听了,要问他求方儿,说道:“请医须请良,传药须传方。吾师不传于
我方儿,倘或我久后用没了,那里寻师父去?随师父要多少东西,我与师父。”因
令玳安:“后边快取二十两白金来。”递与胡僧,要问他求这一枝药方。那胡僧笑
道:“贫僧乃出家之人,云游四方,要这资财何用?官人趁早收拾回去。”一面就
要起身。西门庆见他不肯传方,便道:“师父,你不受资财,我有一匹五丈长大布
,与师父做件衣服罢。”即令左右取来,双手递与胡僧。胡僧方才打问讯谢了。临
出门又吩咐:“不可多用,戒之!戒之!”言毕,背上褡裢,拴定拐杖,出门扬长
而去。正是:

柱杖挑擎双日月,芒鞋踏遍九军州。