\chapter{愿同穴一时丧礼盛~守孤灵半夜口脂香}

诗曰:

湘皋烟草碧纷纷,泪洒东风忆细君。
见说嫦娥能入月,虚疑神女解为云。
花阴昼坐闲金剪,竹里游春冷翠裙。
留得丹青残锦在,伤心不忍读回文。

话说到十月二十八日,是李瓶儿二七,玉皇庙吴道官受斋,请了十六个道众,
在家中扬幡修建斋坛。又有安郎中来下书,西门庆管待来人去了。吴道官庙中抬了
三牲祭礼来,又是一匹尺头以为奠仪。道众绕棺传咒,吴道官灵前展拜。西门庆与
敬济回礼,谢道:“师父多有破费,何以克当?”吴道官道:“小道甚是惶愧,本
该助一经追荐夫人,奈力薄,粗祭表意而已。”西门庆命收了,打发抬盒人回去。
那日三朝转经,演生神章,破九幽狱,对灵摄召,整做法事,不必细说。

第二日,先是门外韩姨夫家来上祭。那时孟玉楼兄弟孟锐做买卖来家,见西门
庆这边有丧事,跟随韩姨夫那边来上祭,讨了一分孝去,送了许多人事。西门庆叙
礼,进入玉楼房中拜见。西门庆亦设席管待,俱不在言表。

那日午间,又是本县知县李拱极、县丞钱斯成、主簿任良贵、典史夏恭基,又
有阳谷县知县狄斯朽,共五员官,都斗了分子,穿孝服来上纸帛吊问。西门庆备席
在卷棚内管待,请了吴大舅与温秀才相陪,三个小优儿弹唱。

正饮酒到热闹处,忽报:“管砖厂工部黄老爹来吊孝。”慌的西门庆连忙穿孝
衣灵前伺侯,温秀才又早迎接至大门外,让至前厅,换了衣裳进来。家人手捧香烛
纸匹金段到灵前,黄主事上了香,展拜毕,西门庆同敬济下来还礼。黄主事道:“
学生不知尊阃没了,吊迟,恕罪,恕罪!”西门庆道:“学生一向欠恭,今又承老
先生赐吊,兼辱厚仪,不胜感激。”叙毕礼,让至卷棚上面坐下。西门庆与温秀才
下边相陪,左右捧茶上来吃了。黄主事道:“昨日宋松原多致意先生,他也闻知令
夫人作过,也要来吊问,争奈有许多事情羁绊。他如今在济州住扎。先生还不知,
朝廷如今营建艮岳,敕令太尉朱[面力],往江南湖湘采取花石纲,运船陆续打河
道中来。头一运将到淮上。又钦差殿前六黄太尉来迎取卿云万态奇峰──长二丈,
阔数尺,都用黄毡盖覆,张打黄旗,费数号船只,由山东河道而来。况河中没水,
起八郡民夫牵挽。官吏倒悬,民不聊生。宋道长督率州县,事事皆亲身经历,案牍
如山,昼夜劳苦,通不得闲。况黄太尉不久自京而至,宋道长说,必须率三司官员
,要接他一接。想此间无可相熟者,委托学生来,敬烦尊府做一东,要请六黄大尉
一饭,未审尊意允否?”因唤左右:“叫你宋老爹承差上来。”有二青衣官吏跪下
,毡包内捧出一对金段、一根沉香、两根白蜡、一分绵纸。黄主事道:“此乃宋公
致赙之仪。那两封,是两司八府官员办酒分资──两司官十二员、府官八员,计二
十二分,共一百零六两。”交与西门庆:“有劳盛使一备何如?”西门庆再三辞道
:“学生有服在家,奈何,奈何?”因问:“迎接在于何时?”黄主事道:“还早
哩,也得到出月半头。黄太监京中还未起身。”西门庆道:“学生十月十二日才发
引。既是宋公祖与老先生吩咐,敢不领命!但这分资决不敢收。该多少桌席,只顾
吩咐,学生无不毕具。”黄主事道:“四泉此意差矣!松原委托学生来烦渎,此乃
山东一省各官公礼,又非松原之己出,何得见却?如其不纳,学生即回松原,再不
敢烦渎矣!”西门庆听了此言,说道:“学生权且领下。”因令玳安、王经接下去
。问备多少桌席,黄主事道:“六黄备一张吃看大桌面,宋公与两司都是平头桌席
,以下府官散席而已。承应乐人,自有差拨伺候,府上不必再叫。”说毕,茶汤两
换,作辞起身。西门庆款留,黄主事道:“学生还要到尚柳塘老先生那里拜拜,他
昔年曾在学生敝处作县令,然后转成都府推官。如今他令郎两泉,又与学生乡试同
年。”西门庆道:“学生不知老先生与尚两泉相厚,两泉亦与学生相交。”黄主事
起身,西门庆道:“烦老先生多致意宋公祖,至期寒舍拱候矣。”黄主事道:“临
期,松原还差人来通报先生,亦不可太奢。”西门庆道,“学生知道。”送出大门
,上马而去。

那县中官员,听见黄主事带领巡按上司人来,唬的都躲在山子下小卷棚内饮酒
,吩咐手下把轿马藏过一边。当时,西门庆回到卷棚与众官相见,具说宋巡按率两
司八府来,央烦出月迎请六黄太尉之事。众官悉言:“正是州县不胜忧苦。这件事
,钦差若来,凡一应祇迎、廪饩、公宴、器用、人夫,无不出于州县,州县
必取之于民,公私困极,莫此为甚。我辈还望四泉于上司处美言提拔,足见厚爱。
”言讫,都不久坐,告辞起身而去。

话休饶舌。到李瓶儿三七,有门外永福寺道坚长老,领十六众上堂僧来念经,
穿云锦袈裟,戴毗卢帽,大钹大鼓,甚是齐整。十月初八日是四七,请西门外宝庆
寺赵喇嘛,亦十六众,来念番经,结坛跳沙,洒花米行香,口诵真言。斋供都用牛
乳茶酪之类,悬挂都是九丑天魔变相,身披缨络琉璃,项挂髑髅,口咬婴儿,坐跨
妖魅,腰缠蛇螭,或四头八臂,或手执戈戟,朱发蓝面,丑恶莫比。午斋以后,就
动荤酒。西门庆那日不在家,同阴阳徐先生往坟上破土开圹去了,后晌方回。晚夕
,打发喇嘛散了。

次日,推运山头酒米、桌面肴品一应所用之物,又委付主管伙计,庄上前后搭
棚,坟内穴边又起三间罩棚。先请附近地邻来,大酒大肉管待。临散,皆肩背项负
而归,俱不必细说。

十一日白日,先是歌郎并锣鼓地吊来灵前参灵,吊《五鬼闹判》、《张天师着
鬼迷》、《钟馗戏小鬼》、《老子过函关》、《六贼闹弥陀》、《雪里梅》、《庄
周梦蝴蝶》、《天王降地水火风》、《洞宾飞剑斩黄龙》、《赵太祖千里送荆娘》
,各样百戏吊罢,堂客都在帘内观看。参罢灵去了,内外亲戚都来辞灵烧纸,大哭
一场。

到次日发引,先绝早抬出名旌、各项幡亭纸扎,僧道、鼓手、细乐、人役都来
伺候。西门庆预先问帅府周守备讨了五十名巡捕军士,都带弓马,全装结束。留十
名在家看守,四十名在材边摆马道,分两翼而行。衙门里又是二十名排军打路,照
管冥器。坟头又是二十名把门,管收祭祀。那日官员士夫、亲邻朋友来送殡者,车
马喧呼,填街塞巷。本家并亲眷轿子也有百十余顶,三院鸨子粉头小轿也有数十。
徐阴阳择定辰时起棺,西门庆留下孙雪娥并二女僧看家,平安儿同两名排军把前门
。女婿陈敬济跪在柩前摔盆,六十四人上扛,有仵作一员官立于增架上,敲响板,
指拨抬材人上肩。先是请了报恩寺僧官来起棺,转过大街口望南走。两边观看的人
山人海。那日正值晴明天气,果然好殡。但见:

和风开绮陌,细雨润芳尘,东方晓日初升,北陆残烟乍敛。冬冬咙咙
,花丧鼓不住声喧;叮叮当当,地吊锣连宵振作。铭旌招飐,大书
九尺红罗;起火轩天,冲散半天黄雾。狰狰狞狞开路鬼,斜担金斧;忽忽
洋洋险道神,端秉银戈。逍逍遥遥八洞仙,龟鹤绕定;窈窈窕窕四毛女,
虎鹿相随。热热闹闹采莲船,撒科打诨;长长大大高跷汉,贯甲顶盔。清
清秀秀小道童一十六众,都是霞衣道髻,动一派之仙音;肥肥胖胖大和尚
二十四个,个个都是云锦袈裟,转五方之法事。一十二座大绢亭,亭亭皆
绿舞红飞;二十四座小绢亭,座座尽珠围翠绕。左势下,天仓与地库相连
;右势下,金山与银山作队。掌醢厨,列八珍之罐;香烛亭,供三献之仪
。六座百花亭,现千团锦绣;一乘引魂轿,扎百结黄丝。这边把花与雪柳
争辉,那边宝盖与银幢作队。金字幡银字幡,紧护棺舆;白绢[纟散]绿
绢[纟散],同围增架。功布招[风占],孝眷声哀。打路排军,执榄杆
前后呼拥;迎丧神会,耍武艺左右盘旋。卖解犹如鹰鹞,走马好似猿猴。
竖肩桩,打斤斗,隔肚穿钱,金鸡独立,人人喝彩,个个争夸。扶肩挤背
,不辨贤愚;挨睹并观,那分贵贱!张三蠢胖,只把气吁;李四矮矬,频
将脚[足占]。白头老叟,尽将拐棒拄髭须;绿[髟丐]佳人,也带儿童
来看殡。

吴月娘与李娇儿等本家轿子十余顶,一字儿紧跟材后。西门庆总冠孝服同众亲
朋在材后,陈敬济紧扶棺舆,走出东街口。西门庆具礼,请玉皇庙吴道官来悬真。
身穿大红五彩鹤氅,头戴九阳雷巾,脚登丹舄,手执牙笏,坐在四人肩舆上,迎殡
而来。将李瓶儿大影捧于手内,陈敬济跪在前面,那殡停住了。众人听他在上高声
宣念:

恭惟
故锦衣西门恭人李氏之灵,存日阳年二十七岁,元命辛未相,正月十五日
午时受生,大限于政和七年九月十七日丑时分身故。伏以尊灵,名家秀质
,绮阁娇姝。禀花月之仪容,蕴蕙兰之佳气。郁德柔婉,赋性温和。配我
西君,克谐伉俪。处闺门而贤淑,资琴瑟以好和。曾种蓝田,寻嗟楚畹。
正宜享福百年,可惜春光三九。呜呼!明月易缺,好物难全。善类无常,
修短有数。今日棺舆载道,丹旆迎风,良夫[足辟]踊于柩前,孝眷哀矜
于巷陌。离别情深而难已,音容日远以日忘。某等谬忝冠簪,愧领玄教。
愧无新垣平之神术,恪遵玄元始之遗风。徒展崔巍镜里之容,难返庄周梦
中之蝶。漱甘露而沃琼浆,超知识登于紫府;披百宝而面七真,引净魄出
于冥途。一心无挂,四大皆空。苦,苦,苦!气化清风形归土。一灵真性
去弗回,改头换面无遍数。众听末后一句:咦!精爽不知何处去,真容留
与后人看。

吴道官念毕,端坐轿上,那轿卷坐退下去了。这里鼓乐喧天,哀声动地,殡才起身
,迤逦出南门。众亲朋陪西门庆,走至门上方乘马,陈敬济扶柩,到于山头五里原
。

原来坐营张团练,带领二百名军,同刘、薛二内相,又早在坟前高阜处搭帐房
,吹响器,打铜锣铜鼓,迎接殡到,看着装烧冥器纸扎,烟焰涨天。棺舆到山下扛
,徐先生率仵作,依罗经吊向,巳时祭告后土方隅后,才下葬掩土。西门庆易服,
备一对尺头礼,请帅府周守备点主。卫中官员并亲朋伙计,皆争拉西门庆递酒,鼓
乐喧天,烟火匝地,热闹丰盛,不必细说。

吃毕,后晌回灵,吴月娘坐魂轿,抱神主魂幡,陈敬济扶灵床,鼓手细乐十六
众小道童两边吹打。吴大舅并乔大户、吴二舅、花大舅、沈姨夫、孟二舅、应伯爵
、谢希大、温秀才、众主管伙计,都陪着西门庆进城,堂客轿子压后,到家门首燎
火而入。李瓶儿房中安灵已毕,徐先生前厅祭神洒扫,么门户皆贴辟非黄符。谢徐
先生一匹尺头、五两银子出门,各项人役打发散了。又拿出二十吊钱来,五吊赏巡
捕军人,五吊与衙门中排军,十吊赏营里人马。拿帖儿回谢周守备、张团练、夏提
刑,俱不在话下。西门庆还要留乔大户、吴大舅众人坐,众人都不肯,作辞起身。
来保进说:“搭棚在外伺候,明日来拆棚。”西门庆道:“棚且不消拆,亦发过了
你宋老爹摆酒日子来拆罢。”打发搭彩匠去了。后边花大娘子与乔大户娘子众堂客
,还等着安毕灵,哭了一场,方才去了。

西门庆不忍遽舍,晚夕还来李瓶儿房中,要伴灵宿歇。见灵床安在正面,大影
挂在旁边,灵床内安着半身,里面小锦被褥,床几、衣服、妆奁之类,无不毕具,
下边放着他的一对小小金莲,桌上香花灯烛、金碟樽俎,般般供养,西门庆大哭不
止。令迎春就在对面炕上搭铺,到夜半,对着孤灯,半窗斜月,翻复无寐,长吁短
叹,思想佳人。有诗为证:

短叹长吁对锁窗,舞鸾孤影寸心伤。
兰枯楚畹三秋雨,枫落吴江一夜霜。
夙世已违连理愿,此生难觅返魂香。
九泉果有精灵在,地下人间两断肠。

白日间供养茶饭,西门庆俱亲看着丫鬟摆下,他便对面和他同吃。举起箸儿来
:“你请些饭儿!”行如在之礼。丫鬟养娘都忍不住掩泪而哭。奶子如意儿,无人
处常在跟前递茶递水,挨挨抢抢,掐掐捏捏,插话儿应答,那消三夜两夜。这日,
西门庆因请了许多官客堂客,坟上暖墓来家,陪人吃得醉了。进来,迎春打发歇下
。到夜间要茶吃,叫迎春不应,如意儿便来递茶。因见被拖下炕来,接过茶盏,用
手扶被,西门庆一时兴动,搂过脖子就亲了个嘴,递舌头在他口内。老婆就咂起来
,一声儿不言语。西门庆令脱去衣服上炕,两个搂在被窝内,不胜欢娱,云雨一处
。老婆说:“既是爹抬举,娘也没了,小媳妇情愿不出爹家门,随爹收用便了。”
西门庆便叫:“我儿,你只用心伏侍我,愁养活不过你来!”这老婆听了,枕席之
间,无不奉承,颠鸾倒凤,随手而转,把西门庆欢喜的要不的。

次日,老婆早晨起来,与西门庆拿鞋脚,叠被褥,就不靠迎春,极尽殷勤,无
所不至。西门庆开门寻出李瓶儿四根簪儿来赏他,老婆磕头谢了。迎春知收用了他
,两个打成一路。老婆自恃得宠,脚跟已牢,无复求告于人,就不同往日,打扮乔
模乔样,在丫鬟伙内,说也有,笑也有。早被潘金莲看在眼里。

早晨,西门庆正陪应伯爵坐的,忽报宋御史差人来送贺黄太尉一桌金银酒器:
两把金壶、两副金台盏、十副小银钟、两副银折盂、四副银赏钟;两匹大红彩蟒、
两匹金缎、十坛酒、两牵羊。传报:“太尉船只已到东昌地方,烦老爹这里早备酒
席,准在十八日迎请。”西门庆收入明白,与了来人一两银子,用手本打发回去。
随即兑银与贲四、来兴儿,定桌面,粘果品,买办整理,不必细说。因向伯爵说:
“自从他不好起,到而今,我再没一日儿心闲。刚刚打发丧事出去了,又钻出这等
勾当来,教我手忙脚乱。”伯爵道:“这个哥不消抱怨,你又不曾兜揽他,他上门
儿来央烦你。虽然你这席酒替他陪几两银子,到明日,休说朝廷一位钦差殿前大太
尉来咱家坐一坐,只这山东一省官员,并巡抚巡按、人马散级,也与咱门户添许多
光辉。”西门庆道:“不是此说,我承望他到二十已外也罢,不想十八日就迎接,
忒促急促忙。这日又是他五七,我已与了吴道官写法银子去了,如何又改!不然,
双头火杖都挤在一处,怎乱得过来?”应伯爵道:“这个不打紧,我算来,嫂子是
九月十七日没了,此月二十一日正是五七。你十八日摆了酒,二十日与嫂子念经也
不迟。”西门庆道:“你说的是,我就使小厮回吴道官改日子去。”伯爵道:“哥
,我又一件:东京黄真人,朝廷差他来泰安州进金铃吊挂御香,建七昼夜罗天大醮
,如今在庙里住。趁他未起身,倒好教吴道官请他那日来做高功,领行法事。咱图
他个名声,也好看。”西门庆道:“都说这黄真人有利益,请他到好,争奈吴道官
斋日受他祭礼,出殡又起动他悬真,道童送殡,没的酬谢他,教他念这个经儿,表
意而已。今又请黄真人主行,却不难为他?”伯爵道:“斋一般还是他受,只教他
请黄真人做高功就是了。哥只多费几两银子,为嫂子,没曾为了别人。”西门庆一
面教陈敬济写帖子,又多封了五两银子,教他早请黄真人,改在二十日念经,二十
四众道士,水火炼度一昼夜。即令玳安骑头口去了。

西门庆打发伯爵去讫,进入后边。只见吴月娘说:“贲四嫂买了两个盒儿,他
女儿长姐定与人家,来磕头。”西门庆便问:“谁家?”贲四娘子领他女儿,穿着
大红缎袄儿、黄绸裙子,戴着花翠,插烛向西门庆磕了四个头。月娘在旁说:“咱
也不知道,原来这孩子与了夏大人房里抬举,昨日才相定下。这二十四日就娶过门
,只得了他三十两银子。论起来,这孩子倒也好身量,不象十五岁,到有十六七岁
的。多少时不见,就长的成成的。”西门庆道:“他前日在酒席上和我说,要抬举
两个孩子学弹唱,不知你家孩子与了他。”于是教月娘让至房内,摆茶留坐。落后
,李娇儿、孟玉楼、潘金莲、孙雪娥、大姐都来见礼陪坐。临去,月娘与了一套重
绢衣服、一两银子,李娇儿众人都有与花翠、汗巾、脂粉之类。晚上,玳安回话:
“吴道官收了银子,知道了。黄真人还在庙里住,过二十头才回东京去。十九日早
来铺设坛场。”

西门庆次日,家中厨役落作治办酒席,务要齐整,大门上扎七级彩山,厅前五
级彩山。十七日,宋御史差委两员县官来观看筵席:厅正面,屏开孔雀,地匝氍毹
,都是锦绣桌帏,妆花椅甸。黄太尉便是肘件大饭簇盘、定胜方糖,吃看大插桌;
观席两张小插桌,是巡抚、巡按陪坐;两边布按三司,有桌席列坐。其余八府官,
都在厅外棚内两边,只是五果五菜平头桌席。看毕,西门庆待茶,起身回话去了。

到次日,抚按率领多官人马,早迎到船上,张打黄旗“钦差”二字,捧着敕书
在头里走,地方统制、守御、都监、团练,各卫掌印武官,皆戎服甲胄,各领所部
人马,围随,仪杖摆数里之远。黄太尉穿大红五彩双挂绣蟒,坐八抬八簇银顶暖轿
,张打茶褐伞。后边名下执事人役跟随无数,皆骏骑咆哮,如万花之灿锦,随鼓吹
而行。黄土塾道,鸡犬不闻,樵采遁迹。人马过东平府,进清河县,县官黑压压跪
于道旁迎接,左右喝叱起去。随路传报,直到西门庆门首。教坊鼓乐,声震云霄,
两边执事人役皆青衣排伏,雁翅而列。西门庆青衣冠冕,望尘拱伺。良久,人马过
尽,太尉落轿进来,后面抚按率领大小官员,一拥而入。到于厅上,又是筝[竹秦
]、方晌、云[王敖]、龙笛、凤管,细乐响动。为首就是山东巡抚都御史侯[氵
蒙]、巡按监察御史宋乔年参见,大尉还依礼答之。其次就是山东左布政龚共、左
参政何其高、右布政陈四箴、右参政季侃廷、参议冯廷鹄、右参议汪伯彦、廉使赵
讷、采访使韩文光、提学副使陈正汇、兵备副使雷启元等两司官参见,太尉稍加优
礼。及至东昌府徐崧、东平府胡师文、兖州府凌云翼、徐州府韩邦奇、济南府张叔
夜、青州府王士奇、登州府黄甲、莱州府叶迁等八府官行厅参之礼,太尉答以长揖
而已。至于统制、制置、守御、都监、团练等官,太尉则端坐。各官听其发放,外
边伺候。然后,西门庆与夏提刑上来拜见献茶,侯巡抚、宋巡按向前把盏,下边动
鼓乐,来与太尉簪金花,捧玉斝,彼此酬饮。递酒已毕,太尉正席坐下
,抚按下边主席,其余官员并西门庆等,各依次第坐了。教坊伶官递上手本奏乐,
一应弹唱队舞,各有节次,极尽声容之盛。当筵搬演《裴晋公还带记》,一折下来
,厨役割献烧鹿、花猪、百宝攒汤、大饭烧卖。又有四员伶官,筝[竹秦]、琵琶
、箜篌,上来清弹小唱。

唱毕,汤未两陈,乐已三奏。下边跟从执事人等,宋御史差两员州官,在西门
庆卷棚内自有桌席管待。守御、都监等官,西门庆都安在前边客位,自有坐处。黄
太尉令左右拿十两银子来赏赐各项人役,随即看轿起身。众官再三款留不住,即送
出大门。鼓乐笙簧迭奏,两街仪卫喧阗,清跸传道,人马森列。多官俱上马远送,
太尉悉令免之,举手上轿而去。

宋御史、候巡抚吩咐都监以下军卫有司,直护送至皇船上来回话。桌面器皿,
答贺羊酒,具手本差东平府知府胡师文与守御周秀,亲送到船所,交付明白。回至
厅上,拜谢西门庆说:“今日负累取扰,深感,深感!分资有所不足,容当奉补。
”西门庆慌躬身施礼道:“卑职重承教爱,累辱盛仪,日昨又蒙赙礼,蜗居卑陋,
犹恐有不到处,万里公祖谅宥,幸甚!”宋御史谢毕,即令左右看轿,与候巡抚一
同起身,两司八府官员皆拜辞而去。各项人役,一哄而散。西门庆回至厅上,将伶
官乐人赏以酒食,俱令散了,止留下四名官身小优儿伺候。厅内外各官桌面,自有
本官手下人领不题。

西门庆见天色尚早,收拾家伙停当,攒下四张桌席,使人请吴大舅、应伯爵、
谢希大、温秀才、傅自新、甘出身、韩道国、贲四、崔本及女婿陈敬济,──从五
更起来,各项照管辛苦,坐饮三杯。不一时,众人来到,摆上酒来饮酒。伯爵道:
“哥,今日黄太尉坐了多大一回?欢喜不欢喜?”韩道国道:“今日六黄老公公见
咱家酒席齐整,无个不欢喜的。巡抚、巡按两位甚是知感不尽,谢了又谢。”伯爵
道:“若是第二家摆这席酒也成不的,也没咱家恁大地方,也没府上这些人手。今
日少说也有上千人进来,都要管待出去。哥就陪了几两银子,咱山东一省也响出名
去了。”温秀才道:“学生宗主提学陈老先生,也在这里预席。”西门庆问其名,
温秀才道:“名陈正汇者,乃谏垣陈了翁先生乃郎,本贯河南鄄城县人,十八岁科
举,中壬辰进士,今任本处提学副使,极有学问。”西门庆道:“他今年才二十四
岁?”正说着,汤饭上来。

众人吃毕,西门庆叫上四个小优儿,问道:“你四人叫甚名字?”答道:“小
的叫周采、梁铎、马真、韩毕。”伯爵道:“你不是韩金钏儿一家?”韩毕跪下说
道:“金钏儿、玉钏儿是小的妹子。”西门庆因想起李瓶儿来:“今日摆酒,就不
见他。”吩咐小优儿:“你们拿乐器过来,唱个‘洛阳花,梁园月’我听。”韩毕
与周采一面搊筝拨阮,唱道:

【普天乐】洛阳花,梁园月。好花须买,皓月须赊。花倚栏杆看烂熳
开,月曾把酒问团[囗栾]夜。月有盈亏,花有开谢。想人生最苦离别。
花谢了,三春近也;月缺了,中秋到也;人去了,何日来也?

唱毕,应伯爵见西门庆眼里酸酸的,便道:“哥教唱此曲,莫非想起过世嫂子来?
”西门庆看见后边上果碟儿,叫:“应二哥,你只嗔我说,有他在,就是他经手整
定。从他没了,随着丫鬟撮弄,你看象甚模样?好应口菜也没一根我吃!”温秀才
道:“这等盛设,老先生中馈也不谓无人,足可以够了。”伯爵道:“哥休说此话
。你心间疼不过,便是这等说,恐一时冷淡了别的嫂子们心。”

这里酒席上说话,不想潘金莲在软壁后听唱,听见西门庆说此话,走到后边,
一五一十告诉月娘。月娘道:“随他说去就是了,你如今却怎样的?前日他在时,
即许下把绣春教伏侍李娇儿,他到睁着眼与我叫,说:‘死了多少时,就分散他房
里丫头!’教我就一声儿再没言语。这两日凭着他那媳妇子和两个丫头,狂的有些
样儿?我但开口,就说咱们挤撮他。”金莲道:“这老婆这两日有些别改模样,只
怕贼没廉耻货,镇日在那屋里,缠了这老婆也不见的。我听见说,前日与了他两对
簪子,老婆戴在头上,拿与这个瞧,拿与那个瞧。”月娘道:“豆芽菜儿──有甚
捆儿!”众人背地里都不喜欢。正是:

遗踪堪入时人眼,多买胭脂画牡丹。