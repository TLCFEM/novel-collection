\chapter{西门庆偷娶潘金莲~武都头误打李皂隶}

诗曰:

感郎耽夙爱,着意守香奁。
岁月多忘远,情综任久淹。
于飞期燕燕,比翼誓鹣鹣。
细数从前意,时时屈指尖。

话说西门庆与潘金莲烧了武大灵,到次日,又安排一席酒,请王婆作辞,就把
迎儿交付与王婆看养。因商量道:“武二回来,却怎生不与他知道六姐是我娶了才
好?”王婆笑道:“有老身在此,任武二那厮怎地兜达,我自有话回他。大官人只
管放心!”西门庆听了,满心欢喜,又将三两银子谢他。当晚就将妇人箱笼,都打
发了家去,剩下些破桌、坏凳、旧衣裳,都与了王婆。到次日初八,一顶轿子,四
个灯笼,妇人换了一身艳色衣服,王婆送亲,玳安跟轿,把妇人抬到家中来。那条
街上,远近人家无一不知此事,都惧怕西门庆有钱有势,不敢来多管,只编了四句
口号,说得好:

堪笑西门不识羞,先奸后娶丑名留。
轿内坐着浪淫妇,后边跟着老牵头。

西门庆娶妇人到家,收拾花园内楼下三间与他做房。一个独独小角门儿进去,
院内设放花草盆景。白日间人迹罕到,极是一个幽僻去处。一边是外房,一边是卧
房。西门庆旋用十六两银子买了一张黑漆欢门描金床,大红罗圈金帐幔,宝象花拣
妆,桌椅锦杌,摆设齐整。大娘子吴月娘房里使着两个丫头,一名春梅,一
名玉箫。西门庆把春梅叫到金莲房内,令他伏侍金莲,赶着叫娘。却用五两银子另
买一个小丫头,名叫小玉,伏侍月娘。又替金莲六两银子买了一个上灶丫头,名唤
秋菊。排行金莲做第五房。先头陈家娘子陪嫁的,名唤孙雪娥,约二十年纪,生的
五短身材,有姿色。西门庆与他戴了[髟狄]髻,排行第四,以此把金莲做个第五
房。此事表过不题。

这妇人一娶过门来,西门庆就在妇人房中宿歇,如鱼似水,美爱无加。到第二
日,妇人梳妆打扮,穿一套艳色服,春梅捧茶,走来后边大娘子吴月娘房里,拜见
大小,递见面鞋脚。月娘在座上仔细观看,这妇人年纪不上二十五六,生的这样标
致。但见:

眉似初春柳叶,常含着雨恨云愁;脸如三月桃花,暗带着风情月意。
纤腰袅娜,拘束的燕懒莺慵;檀口轻盈,勾引得峰狂蝶乱。玉貌妖娆花解
语,芳容窈窕玉生香。

吴月娘从头看到脚,风流往下跑;从脚看到头,风流往上流。论风流,如水泥晶盘
内走明珠;语态度,似红杏枝头笼晓日。看了一回,口中不言,心内想道:“小厮
每来家,只说武大怎样一个老婆,不曾看见,不想果然生的标致,怪不的俺那强人
爱他。”金莲先与月娘磕了头,递了鞋脚。月娘受了他四礼。次后李娇儿、孟玉楼
、孙雪娥,都拜见了,平叙了姊妹之礼,立在旁边。月娘叫丫头拿个坐儿教他坐,
吩咐丫头、媳妇赶着他叫五娘。这妇人坐在旁边,不转睛把众人偷看。见吴月娘约
三九年纪,生的面如银盆,眼如杏子,举止温柔,持重寡言。第二个李娇儿,乃院
中唱的,生的肌肤丰肥,身体沉重,虽数名妓者之称,而风月多不及金莲也。第三
个就是新娶的孟玉楼,约三十年纪,生得貌若梨花,腰如杨柳,长挑身材,瓜子脸
儿,稀稀多几点微麻,自是天然俏丽,惟裙下双湾与金莲无大小之分。第四个孙雪
娥,乃房里出身,五短身材,轻盈体态,能造五鲜汤水,善舞翠盘之妙。这妇人一
抹儿都看在心里。过三日之后,每日清晨起来,就来房里与月娘做针指,做鞋脚,
凡事不拿强拿,不动强动。指着丫头赶着月娘,一口一声只叫大娘,快把小意儿贴
恋几次,把月娘喜欢得没入脚处,称呼他做六姐。衣服首饰拣心爱的与他,吃饭吃
茶都和他在一处。因此,李娇儿众人见月娘错敬他,都气不忿,背后常说:“俺们
是旧人,到不理论。他来了多少时,便这等惯了他。大姐姐好没分晓!”西门庆自
娶潘金莲来家,住着深宅大院,衣服头面又相趁,二人女貌郎才,正在妙年之际,
凡事如胶似漆,百依百随,淫欲之事,无日无之。且按下不题。

单表武松,八月初旬到了清河县,先去县里纳了回书。知县见了大喜,已知金
宝交得明白,赏了武松十两银子,酒食管待,不必细说。武松回到下处,换了衣服
鞋袜,戴了一顶新头巾,锁了房门,一径投紫石街来。两边众邻舍看见武松回来,
都吃一惊,捏两把汗,说道:“这番萧墙祸起了!这个太岁归来,怎肯干休!”武
松走到哥哥门前,揭起帘子,探身入来,看见小女迎儿在楼穿廊下撵线。叫声哥哥
也不应,叫声嫂嫂也不应,道:“我莫不耳聋了,如何不见哥嫂声音?”向前便问
迎儿。那迎儿见他叔叔来,吓的不敢言语。武松道:“你爹娘往那里去了?”迎儿
只是哭,不做声。正问间,隔壁王婆听得是武二归来,生怕决撒了,慌忙走过来。
武二见王婆过来,唱了喏,问道:“我哥哥往那里去了?嫂嫂也怎的不见?”婆子
道:“二哥请坐,我告诉你。你哥哥自从你去后,到四月间得个拙病死了。”武二
道:“我哥哥四月几时死的?得什么病?吃谁的药来?”王婆道:“你哥哥四月二
十头,猛可地害起心疼起来,病了八九日,求神问卜,什么药不吃到?医治不好,
死了。”武二道:“我的哥哥从来不曾有这病,如何心疼便死了?”王婆道:“都
头却怎的这般说?天有不测风云,人有旦夕祸福。今晚脱了鞋和袜,未审明朝穿不
穿。谁人保得常没事?”武二道:“我哥哥如今埋在那里?”王婆道:“你哥哥一
倒了头,家中一文钱也没有,大娘子又是没脚蟹,那里去寻坟地?亏左近一个财主
旧与大郎有一面之交,舍助一具棺木,没奈何放了三日,抬出去火葬了。”武二道
:“如今嫂嫂往那里去了?”婆子道:“他少女嫩妇的,又没的养赡过日子。胡乱
守了百日孝,他娘劝他,前月嫁了外京人去了。丢下这个业障丫头子,教我替他养
活。专等你回来交付与你,也了我一场事。”武二听言,沉吟了半晌,便撇下王婆
出门去,迳投县前下处。开了门进房里,换了一身素衣,便叫土兵街上打了一条麻
绦,买了一双绵裤,一顶孝帽戴在头上;又买了些果品点心、香烛冥纸、金银锭之
类,归到哥哥家,从新安设武大灵位。安排羹饭,点起香烛,铺设酒肴,挂起经幡
纸缯,安排得端正。约一更已后,武二拈了香,扑翻身便拜,道:“哥哥阴魂不远
,你在世时,为人软弱,今日死后,不见分明。你若负屈含冤,被人害了,托梦与
我,兄弟替你报冤雪恨!”把酒一面浇奠了,烧化冥纸,武二便放声大哭。终是一
路上来的人,哭的那两边邻舍无不凄惶。武二哭罢,将这羹饭酒肴和土兵、迎儿吃
了。讨两条席子,教土兵房外旁边睡,迎儿房中睡,他便自把条席子,就武大灵桌
子前睡。

约莫将半夜时分,武二翻来覆去那里睡得着,口里只是长吁气。那土兵齁齁的
却似死人一般,挺在那里。武二爬将起来看时,那灵桌子上琉璃灯半明半灭。武二
坐在席子上,自言自语,口里说道:“我哥哥生时懦弱,死后却无分明。”说犹未
了,只见那灵桌子下卷起一阵冷风来。但见:

无形无影,非雾非烟。盘旋似怪风侵骨冷,凛冽如杀气透肌寒。昏昏
暗暗,灵前灯火失光明;惨惨幽幽,壁上纸钱飞散乱。隐隐遮藏食毒鬼,
纷纷飘逐影魂幡。

那阵冷风,逼得武二毛发皆竖起来。定睛看时,见一个人从灵桌底下钻将出来,叫
声:“兄弟!我死得好苦也!”武二看不仔细,却待向前再问时,只见冷气散了,
不见了人。武二一交跌翻在席子上坐的,寻思道:“怪哉!似梦非梦。刚才我哥哥
正要报我知道,又被我的神气冲散了。想来他这一死,必然不明。”听那更鼓,正
打三更三点。回头看那土兵,正睡得好。于是咄咄不乐,只等天明,却再理会。

看看五更鸡叫,东方渐明。土兵起来烧汤,武二洗漱了,唤起迎儿看家,带领
土兵出了门。在街上访问街坊邻舍:“我哥哥怎的死了?嫂嫂嫁得何人去了?”那
街坊邻舍明知此事,都惧怕西门庆,谁肯来管?只说:“都头,不消访问,王婆在
紧隔壁住,只问王婆就知了。”有那多口的说:“卖梨的郓哥儿与仵作何九,二人
最知详细。”这武二竟走来街坊前去寻郓哥。只见那小猴子手里拿着个柳笼簸罗儿
,正籴米回来。武二便叫郓哥道:“兄弟!”唱喏。那小厮见是武二叫他,便道:
“武都头,你来迟了一步儿,须动不得手。只是一件,我的老爹六十岁,没人养赡
,我却难保你们打官司。”武二道:“好兄弟,跟我来。”引他到一个饭店楼上,
武二叫货卖造两分饭来。武二对郓哥道:“兄弟,你虽年幼,倒有养家孝顺之心。
我没甚么──”向身边摸出五两碎银子,递与郓哥道:“你且拿去与老爹做盘费。
待事务毕了,我再与你十来两银子做本钱。你可备细说与我:哥哥和甚人合气?被
甚人谋害了?家中嫂嫂被那一个娶去?你一一说来,休要隐匿。”这郓哥一手接过
银子,自心里想道:“这些银子,老爹也够盘费得三五个月,便陪他打官司也不妨
。”一面说道:“武二哥,你听我说,却休气苦。”于是把卖梨儿寻西门庆,后被
王婆怎地打他,不放进去,又怎地帮扶武大捉奸,西门庆怎的踢中了武大,心疼了
几日,不知怎的死了,从头至尾细说了一遍。武二听了,便道:“你这话却是实么
?”又问道:“我的嫂子实嫁与何人去了?”郓哥道:“你嫂子吃西门庆抬到家,
待捣吊底子儿,自还问他实也是虚!”武二道:“你休说谎。”郓哥道:“我便官
府面前,也只是这般说。”武二道:“兄弟,既然如此,讨饭来吃。”须臾,吃了
饭。武二还了饭钱,两个下楼来,吩咐郓哥:“你回家把盘缠交与老爹,明日早上
来县前,与我作证。”又问:“何九在那里居住?”郓哥道:“你这时候还寻何九
?他三日前听见你回,便走的不知去向了。”这武二放了郓哥家去。

到第二日,早起,先在陈先生家写了状子,走到县门前。只见郓哥也在那里伺
候,一直奔到厅上跪下,声冤起来。知县看见,认的是武松,便问:“你告什么?
因何声冤?”武二告道:“小人哥哥武大,被豪恶西门庆与嫂潘氏通奸,踢中心窝
,王婆主谋,陷害性命。何九朦胧入殓,烧毁尸伤。见今西门庆霸占嫂子在家为妾
。见有这个小厮郓哥是证见。望相公作主则个。”因递上状子。知县接着,便问:
“何九怎的不见?”武二道:“何九知情在逃,不知去向。”知县于是摘问了郓哥
口词,当下退厅与佐二官吏通同商议。原来知县、县丞、主簿、典史,上下都是与
西门庆有首尾的,因此官吏通同计较,这件事难以问理。知县随出来叫武松道:“
你也是个本县中都头,怎不省得法度?自古捉奸见双,杀人见伤。你那哥哥尸首又
没了,又不曾捉得他奸。你今只凭这小厮口内言语,便问他杀人的公事,莫非公道
忒偏向么?你不可造次,须要自己寻思。”武二道:“告禀相公,这都是实情,不
是小人捏造出来的。只望相公拿西门庆与嫂潘氏、王婆来,当堂尽法一番,其冤自
见。若有虚诬,小人情愿甘罪。”知县道:“你且起来,待我从长计较。可行时,
便与你拿人。”武二方才起来,走出外边,把郓哥留在屋里,不放回家。

早有人把这件事报与西门庆得知。西门庆听得慌了,忙叫心腹家人来保、来旺
,身边带着银两,连夜将官吏都买嘱了。到次日早晨,武二在厅上指望告禀知县,
催逼拿人。谁想这官人受了贿赂,早发下状子来,说道:“武松,你休听外人挑拨
,和西门庆做对头。这件事欠明白,难以问理。圣人云:经目之事,犹恐未真;背
后之言,岂能全信?你不可一时造次。”当该吏典在旁,便道:“都头,你在衙门
里也晓得法律,但凡人命之事,须要尸、伤、病、物、踪,五件事俱完,方可推问
。你那哥哥尸首又没了,怎生问理?”武二道:“若恁的说时,小人哥哥的冤仇,
难道终不能报便罢了?既然相公不准所告,且却有理。”遂收了状子,下厅来。来
到下处,放了郓哥归家,不觉仰天长叹一声,咬牙切齿,口中骂淫妇不绝。

武松是何等汉子,怎消洋得这口恶气!一直走到西门庆生药店前,要寻西门庆
厮打。正见他开铺子的傅伙计在柜身里面,见武二狠狠的走来,问道:“你大官人
在宅上么?”傅伙计认的是武二,便道:“不在家了。都头有甚话说?”武二道:
“且请借一步说句。”傅伙计不敢不出来,被武二引到僻静巷口。武二翻过脸来,
用手撮住他衣领,睁圆怪眼说道:“你要死,却是要活?”傅伙计道:“都头在上
,小人又不曾触犯了都头,都头何故发怒?”武二道:“你若要死,便不要说;若
要活时,对我实说。西门庆那厮如今在那里?我的嫂子被他娶了多少日子?一一说
来,我便罢休?”那傅伙计是个小胆的人,见武二发作,慌了手脚,说道:“都头
息怒,小人在他家,每月二两银子雇着,小人只开铺子,并不知他们闲帐。大官人
本不在家,刚才和一相知,往狮子街大酒楼上吃酒去了。小人并不敢说谎。”武二
听了此言,方才放了手,大叉步飞奔到狮子街来。吓的傅伙计半日移脚不动。那武
二迳奔到狮子街桥下酒楼前来。

且说西门庆正和县中一个皂隶李外传在楼上吃酒。原来那李外传专一在府县前
绰揽些公事,往来听气儿撰些钱使。若有两家告状的,他便卖串儿;或是官吏打点
,他便两下里打背。因此县中就起了他这个浑名,叫做李外传。那日见知县回出武
松状子,讨得这个消息,便来回报西门庆知道。因此西门庆让他在酒楼上饮酒,把
五两银子送他。正吃酒在热闹处,忽然把眼向楼窗下看,只见武松似凶神般从桥下
直奔酒楼前来。已知此人来意不善,不觉心惊,欲待走了,却又下楼不及,遂推更
衣,走往后楼躲避。武二奔到酒楼前,便问酒保道:“西门庆在此么?”酒保道:
“西门大官人和一相识在楼上吃酒哩。”武二拨步撩衣,飞抢上楼去。早不见了西
门庆,只见一个人坐在正面,两个唱的粉头坐在两边。认的是本县皂隶李外传,就
知是他来报信,不觉怒从心起,便走近前,指定李外传骂道:“你这厮,把西门庆
藏在那里去了?快说了,饶你一顿拳头!”李外传看见武二,先吓呆了,又见他恶
狠狠逼紧来问,那里还说得出话来!武二见他不则声,越加恼怒,便一脚把桌子踢
倒,碟儿盏儿都打得粉碎。两个粉头吓得魂都没了。李外传见势头不好,强挣起身
来,就要往楼下跑。武二一把扯回来道:“你这厮,问着不说,待要往那里去?且
吃我一拳,看你说也不说!”早飕的一拳,飞到李外传脸上。李外传叫声啊呀,忍
痛不过,只得说道:“西门庆才往后楼更衣去了,不干我事,饶我去罢!”武二听
了,就趁势儿用双手将他撮起来,隔着楼窗儿往外只一兜,说道:“你既要去,就
饶你去罢!”扑通一声,倒撞落在当街心里。武二随即赶到后楼来寻西门庆。此时
西门庆听见武松在前楼行凶,吓得心胆都碎,便不顾性命,从后楼窗一跳,顺着房
檐,跳下人家后院内去了。武二见西门庆不在后楼,只道是李外传说谎,急转身奔
下楼来,见李外传已跌得半死,直挺挺在地下,还把眼动。气不过,兜裆又是两脚
,早已哀哉断气身亡。众人道:“这是李皂隶,他怎的得罪都头来?为何打杀他?
”武二道:“我自要打西门庆,不料这厮悔气,却和他一路,也撞在我手里。”那
地方保甲见人死了,又不敢向前捉武二,只得慢慢挨上来收笼他,那里肯放松!连
酒保王鸾并两个粉头包氏、牛氏都拴了,竟投县衙里来。此时哄动了狮子街,闹了
清河县,街上议论的人,不计其数。却不知道西门庆不该死,倒都说是西门庆大官
人被武松打死了。正是:

李公吃了张公酿,郑六生儿郑九当。
世间几许不平事,都付时人话短长。