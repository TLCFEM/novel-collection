\chapter{佳人笑赏玩灯楼~狎客帮嫖丽春院}

诗曰:

楼上多娇艳,当窗并三五。
争弄游春陌,相邀开绣户。
转态结红裾,含娇入翠羽。
留宾乍拂弦,托意时移住。

话说光阴迅速,又早到正月十五日。西门庆先一日差玳安送了四盘羹菜、一坛
酒、一盘寿桃、一盘寿面、一套织金重绢衣服,写吴月娘名字,送与李瓶儿做生日
。李瓶儿才起来梳妆,叫了玳安儿到卧房里,说道:“前日打搅你大娘,今日又教
你大娘费心送礼来。”玳安道:“娘多上覆,爹也上覆二娘,不多些微礼,送二娘
赏人。”李瓶儿一面吩咐迎春罢四盘茶食管待玳安。临出门与二钱银子、一方闪色
手帕:“到家多上覆你家列位娘,我这里就使老冯拿帖儿来请。好歹明日都要光降
走走。”玳安磕头出门,两个抬盒子的与一百文钱。李瓶儿随即使老冯拿着五个柬
帖儿,十五日请月娘和李娇儿、孟玉楼、孙雪娥、潘金莲,又捎了一个帖儿,暗暗
请西门庆那日晚夕赴席。

月娘到次日,留下孙雪娥看家,同李娇儿、孟玉楼、潘金莲四顶轿子出门,都
穿着妆花锦绣衣服,来兴、来安、玳安、画童四个小厮跟随着,竟到狮子街灯市李
瓶儿新买的房子里来。这房子门面四间,到底三层:临街是楼;仪门内两边厢房,
三间客坐,一间梢间;过道穿进去,第三层三间卧房,一间厨房。后边落地紧靠着
乔皇亲花园。李瓶儿知月娘众人来看灯,临街楼上设放围屏桌席,悬挂许多花灯。
先迎接到客位内,见毕礼数,次让入后边明间内待茶,不必细说。到午间,客位内
设四张桌席,叫了两个唱的──董娇儿、韩金钏儿,弹唱饮酒。前边楼上设着细巧
添换酒席,又请月娘众人登楼看灯玩耍。楼檐前挂着湘帘,悬着灯彩。吴月娘穿着
大红妆花通袖袄儿,娇绿段裙,貂鼠皮袄。李娇儿、孟玉楼、潘金莲都是白绫袄儿
,蓝段裙。李娇儿是沉香色遍地金比甲,孟玉楼是绿遍地金比甲,潘金莲是大红遍
地金比甲,头上珠翠堆盈,凤钗半卸。俱搭伏定楼窗观看。那灯市中人烟凑集,十
分热闹。当街搭数十座灯架,四下围列诸般买卖,玩灯男女,花红柳绿,车马轰雷
。但见:

山石穿双龙戏水,云霞映独鹤朝天。金屏灯、玉楼灯见一片珠玑;荷
花灯、芙蓉灯散千围锦绣。绣球灯皎皎洁洁,雪花灯拂拂纷纷。秀才灯揖
让进止,存孔孟之遗风;媳妇灯容德温柔,效孟姜之节操。和尚灯月明与
柳翠相连,判官灯钟馗共小妹并坐。师婆灯挥羽扇假降邪神,刘海灯背金
蟾戏吞至宝。骆驼灯、青狮灯驮无价之奇珍;猿猴灯、白象灯进连城之秘
宝。七手八脚螃蟹灯倒戏清波,巨大口髯鲇鱼灯平吞绿藻。银蛾斗彩,雪
柳争辉。鱼龙沙戏,七真五老献丹书;吊挂流苏,九夷八蛮来进宝。村里
社鼓,队队喧阗;百戏货郎,桩桩斗巧。转灯儿一来一往,吊灯儿或仰或
垂。琉璃瓶映美女奇花,云母障并瀛州阆苑。王孙争看小栏下,蹴鞠齐云
;仕女相携高楼上,娇娆炫色。卦肆云集,相幙星罗:讲新春造化
如何,定一世荣枯有准。又有那站高坡打谈的,词曲杨恭;到看这扇响钹
游脚僧,演说三藏。卖元宵的高堆果馅,粘梅花的齐插枯枝。剪春娥,鬓
边斜插闹东风;祷凉钗,头上飞金光耀日。围屏画石崇之锦帐,珠帘绘梅
月之双清。虽然览不尽鳌山景,也应丰登快活年。

月娘看了一回,见楼下人乱,就和李娇儿各归席上吃酒去了。惟有潘金莲、孟
玉楼同两个唱的,只顾搭伏着楼窗子望下观看。那潘金莲一径把白绫袄袖子儿搂着
,显他那遍地金掏袖儿,露出那十指春葱来,带着六个金马镫戒指儿,探着半截身
子,口中嗑瓜子儿,把嗑的瓜子皮儿都吐落在人身上,和玉楼两个嘻笑不止。一回
指道:“大姐姐,你来看,那家房檐下挂的两盏绣球灯,一来一往,滚上滚下,倒
好看。”一回又道:“二姐姐,你来看,这对门架子上,挑着一盏大鱼灯,下面还
有许多小鱼鳖蟹儿,跟着他倒好耍子。”一回又叫:“三姐姐,你看,这首里这个
婆儿灯,那个老儿灯。”正看着,忽然一阵风来,把个婆儿灯下半截割了一个大窟
窿。妇人看见,笑个不了,引惹的那楼下看灯的人,挨肩擦背,仰望上瞧,通挤匝
不开,都压[足罗][足罗]儿。内中有几个浮浪子弟,直指着谈论。一个说道:
“一定是那公侯府里出来的宅眷。”一个又猜:“是贵戚王孙家艳妾,来此看灯。
不然如何内家妆束?”又一个说道:“莫不是院中小娘儿?是那大人家叫来这里看
灯弹唱。”又一个走过来说道:“只我认的,你们都猜不着。这两个妇人,也不是
小可人家的,他是阎罗大王的妻,五道将军的妾,是咱县门前开生药铺、放官吏债
西门大官人的妇女。你惹他怎的?想必跟他大娘来这里看灯。这个穿绿遍地金比甲
的,我不认的。那穿大红遍地金比甲儿,上戴着个翠面花儿的,倒好似卖炊饼武大
郎的娘子。大郎因为在王婆茶坊内捉奸,被大官人踢死了。把他娶在家里做妾。后
次他小叔武松告状,误打死了皂隶李外传,被大官人垫发充军去了。如今一二年不
见出来,落的这等标致了。”正说着,吴月娘见楼下围的人多了,叫了金莲、玉楼
席坐下,听着两个粉头弹唱灯词,饮酒。

坐了一回,月娘要起身,说道:“酒够了,我和二娘先行一步,留下他姊妹两
个再坐一回儿,以尽二娘之情。今日他爹不在家,家里无人,光丢着些丫头们,我
不放心。”这李瓶儿那里肯放,说道:“好大娘,奴没尽心也是的。今日大节间,
灯儿也没点,饭儿也没上,就要家去,就是西门爹不在家中,还有他姑娘们哩,怕
怎的?待月色上来,奴送四位娘去。”月娘道:“二娘,不是这等说。我又不大十
分用酒,留下他姊妹两个,就同我一般。”李瓶儿道:“大娘不用,二娘也不吃一
钟,也没这个道理。想奴前日在大娘府上,那等钟钟不辞,众位娘竟不肯饶我。今
日来到奴这湫窄之处,虽无甚物供献,也尽奴一点劳心。”于是拿大银钟递与李娇
儿,说道:“二娘好歹吃一杯儿。大娘,奴不敢奉大杯,只奉小杯儿罢。”于是满
斟递与月娘。两个唱的,月娘每人与他二钱银子。待的李娇儿吃过酒,月娘就起身
,又嘱咐玉楼、金莲道:“我两个先去,就使小厮拿灯笼来接你们,也就来罢。家
里没人。”玉楼应诺。李瓶儿送月娘、李娇儿到门首,上轿去了。归到楼上,陪玉
楼、金莲饮酒,看看天晚,楼上点起灯来,两个唱的弹唱饮酒,不在话下。

却说西门庆那日同应伯爵、谢希大两个,家中吃了饭,同往灯市里游玩。到了
狮子街东口,西门庆因为月娘众人都在李瓶儿家吃酒,恐怕他两个看见,就不往西
街去看大灯,只到卖纱灯的跟前就回了。不想转过湾来,撞遇孙寡嘴、祝实念,唱
喏说道:“连日不会哥,心中渴想。”见了应伯爵、谢希大骂道:“你两个天杀的
好人儿,你来和哥游玩,就不说叫俺一声儿!”西门庆道:“祝兄弟,你错怪了他
两个,刚才也是路上相遇。”祝实念道:“如今看了灯往那里去?”西门庆道:“
同众位兄弟到大酒楼上吃三杯儿,不是也请众兄弟家去,今日房下们都往人家吃酒
去了。”祝实念道:“比是哥请俺每到酒楼上,何不往里边望望李桂姐去?只当大
节间拜拜年,去混他混。前日俺两个在他家,他望着俺们好不哭哩!说他从腊里不
好到如今,大官人通影边儿不进去看他看。哥今日倒闲,俺们情愿相伴哥进去走走
。”西门庆因记挂晚夕李瓶儿有约,故推辞道:“今日我还有小事,明日去罢。”
怎禁这伙人死拖活拽,于是同进院中去。正是:

柳底花阴压路尘,一回游赏一回新。
不知买尽长安笑,活得苍生几户贫?

西门庆同众人到了李家,桂卿正打扮着在门首站立,一面迎接入中堂相见了。
祝实念就高叫道:“快请三妈出来!还亏俺众人,今日请的大官人来了。”少顷,
老虔婆扶拐而出,与西门庆见礼毕,说道:“老身又不曾怠慢了姐夫,如何一向不
进来看看姐儿?想必别处另叙了新表子来。”祝实念插口道:“你老人家会猜算,
俺大官人近日相了个绝色的表子,每日只在那里走,不想你家桂姐儿。刚才不是俺
二人在灯市里撞见,拉他来,他还不来哩!妈不信,问孙伯修就是了。”因指着应
伯爵、谢希大说道:“这两个天杀的,和他都是一路神祇。”老虔婆听了,
哈哈笑道:“好应二哥,俺家没恼着你,如何不在姐夫面前美言一句儿?虽故姐夫
里边头絮儿多,常言道:好子弟不嫖一个粉头,天下钱眼儿都一样。不是老身夸口
说,我家桂姐也不丑,姐夫自有眼,今也不消人说。”孙寡嘴道:“我是老实说,
哥如今新叙的这个表子,不是里面的,是外面的表子。”西门庆听了,赶着孙寡嘴
只顾打,说道:“老妈,你休听这天灾人祸的老油嘴,老杀才!”孙寡嘴和众人笑
成一块。西门庆向袖中掏出三两银子来,递与桂卿:“大节间,我请众朋友。”桂
卿不肯接,递与老妈。老妈说道:“怎么的?姐夫就笑话我家,大节下拿不出酒菜
儿管待列位老爹?又教姐夫坏钞,拿出银子。显的俺们院里人家只是爱钱了。”应
伯爵走过来说道:“老妈,你依我收了,快安排酒来俺们吃。”那虔婆说道:“这
个理上却使不得。”一壁推辞,一壁把银子接来袖了,深深道了个万福,说道:“
谢姐夫的布施。”应伯爵道:“妈,你且住。我说个笑话儿你听:一个子弟在院中
嫖小娘儿。那一日做耍,装做贫子进去。老妈见他衣服褴缕,不理他。坐了半日,
茶也不拿出来。子弟说:‘妈,我肚饥,有饭寻些来吃。’老妈道:‘米囤也晒,
那讨饭来?’子弟又道:‘既没饭,有水拿些来,我洗脸。’老妈道:‘少挑水钱
,连日没送水来。’这子弟向袖中取出十两一锭银子,放在桌上,教买米雇水去。
慌的老妈没口子道:‘姐夫吃了脸洗饭,洗了饭吃脸!’”把众人都笑了。虔婆道
:“你还是这等快取笑,可可儿的来,自古有恁说没这事。”应伯爵道:“你拿耳
朵来,我对你说:大官人新近请了花二哥表子──后巷的吴银儿了,不要你家桂姐
哩!”虔婆笑道:“我不信,俺桂姐今日不是强口,比吴银儿还比得过。我家与姐
夫是快刀儿割不断的亲戚。姐夫是何等人儿?他眼里见得多,着紧处,金子也估出
个成色来!”说毕,入去收拾酒菜去了。

少顷,李桂姐出来,家常挽着一窝丝杭州攒,金缕丝钗,翠梅花钿儿,珠子箍
儿,金笼坠子,上穿白绫对襟袄儿,下着红罗裙子,打扮的粉妆玉琢,望下道了万
福,与桂卿一边一个打横坐下。须臾,泡出茶来,桂卿、桂姐每人递了一盏,陪着
吃毕。保儿就来打抹春台,才待收拾摆放案酒,忽见帘子外探头舒脑,有几个穿褴
缕衣者──谓之架儿,进来跪下,手里拿着三四升瓜子儿:“大节间,孝顺大老爹
。”西门庆只认头一个叫于春儿,问:“你们那几个在这里?”于春道:“还有段
绵纱、青聂钺,在外边伺候。”段绵纱进来,看见应伯爵在里,说道:“应爹也在
这里。”连忙磕了头。西门庆吩咐收了他瓜子儿,打开银包儿,捏一两一块银子掠
在地下。于春儿接了,和众人扒在地下磕了个头,说道:“谢爹赏赐。”往外飞跑
。有《朝天子》单道架儿行藏:

这家子打和,那家子撮合。他的本分少,虚头大,一些儿不巧又腾挪
,绕院里都踅过。席面上帮闲,把牙儿闲嗑。攘一回才散伙,赚钱又不多
。歪厮缠怎么?他在虎口里求津唾。

西门庆打发架儿出门,安排酒上来吃。桂姐满泛金杯,双垂红袖,肴烹异品,
果献时新,倚翠偎红,花浓酒艳。酒过两巡,桂卿、桂姐一个弹筝,一个琵琶,两
个弹着唱了一套《霁景融和》。正唱在热闹处,见三个穿青衣黄板鞭者──谓之圆
社,手里捧着一只烧鹅,提着两瓶老酒,大节间来孝顺大官人,向前打了半跪。西
门庆平昔认的,一个唤白秃子,一个唤小张闲,一个是罗回子,因说道:“你们且
外边候候,待俺们吃过酒,踢三跑。”于是向桌子上拾了四盘嗄饭、一大壶酒、一
碟点心,打发众圆社吃了,整理气毬伺候。西门庆吃了一回酒,出来外面院
子里,先踢了一跑。次教桂姐上来,与两个圆社踢。一个揸头,一个对障,勾踢拐
打之间,无不假喝彩奉承。就有些不到处,都快取过去了。反来向西门庆面前讨赏
钱,说:“桂姐的行头,就数一数二的,强如二条巷董官女儿数十倍。”当下桂姐
踢了两跑下来,使的尘生眉畔,汗湿腮边,气喘吁吁,腰肢困乏。袖中取出春扇儿
摇凉,与西门庆携手,看桂卿与谢希大、张小闲踢行头。白秃子、罗回子在旁虚撮
脚儿等漏,往来拾毛。亦有《朝天子》一词,单表这踢圆的始末:

在家中也闲,到处刮涎,生理全不干,气毬儿不离在身边,每
日街头站。穷的又不趋,富贵他偏羡。从早晨只到晚,不得甚饱餐。转不
得大钱,他老婆常被人包占。

西门庆正看着众人在院内打双陆、踢气毬,饮酒,只见玳安骑马来接,悄悄
附耳低言道:“大娘、二娘家去了。花二娘叫小的请爹早些过去哩!”这西门庆听
了,暗暗叫玳安:“把马吊在后门边,等着我。”于是酒也不吃,拉桂姐到房中,
只坐了一回儿,就出来推净手,于后门上马,一溜烟走了。应伯爵使保儿去拉扯,
西门庆只说:“我家里有事。”那里肯转来!教玳安儿拿了一两五钱银子打发三个
圆社。李家恐怕他又往后巷吴银儿家去,使丫鬟直跟至院门首方回。应伯爵等众人
,还吃到二更才散。正是:

笑骂由他笑骂,欢娱我且欢娱。