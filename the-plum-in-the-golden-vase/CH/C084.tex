\chapter{吴月娘大闹碧霞宫~曾静师化缘雪涧洞}

诗曰:
一自当年折凤凰,至今情绪几惶惶。
盖棺不作横金妇,入地还从折桂郎。
彭泽晓烟归宿梦,潇湘夜雨断愁肠。
新诗写向空山寺,高挂云帆过豫章。

说话一日,吴月娘请将吴大舅来商议,要往泰安州顶上与娘娘进香,因西门庆病重
之时许的愿心。吴大舅道:“既要去,须是我同了你去。”一面备办香烛纸马祭品
之物,玳安、来安儿跟随,雇了三个头口,月娘便坐一乘暖轿,分付孟玉楼、潘金
莲、孙雪娥、西门大姐:“好生看家,同奶子如意儿、众丫头好生看孝哥儿。后边
仪门无事早早关了,休要出外边去。”又分付陈敬济:“休要那去,同傅伙计大门
首看顾。我约莫到月尽就来家了。”十五日早辰烧纸通信,晚夕辞了西门庆灵,与
众姊妹置酒作别,把房门、各库门房钥匙交付与小玉拿着。次日早五更起身,离了
家门,一行人奔大路而去。那秋深时分,天寒日短,一日行程六七十里之地。未到
黄昏,投客店村房安歇,次日再行。一路上,秋云淡淡,寒雁凄凄,树木凋落,景
物荒凉,不胜悲怆。

话休饶舌。一路无词,行了数日,到了泰安州,望见泰山,端的是天下第一名山,
根盘地脚,顶接天心,居齐鲁之邦,有岩岩之气象。吴大舅见天晚,投在客店歇宿
一宵。次日早起上山,望岱岳庙来。那岱岳库就在山前,乃累朝祀典,历代封禅,
为第一庙貌也。但见:

庙居岱岳,山镇乾坤,为山岳之尊,乃万福之领袖。山头倚槛,直望
弱水蓬莱;
绝顶攀松,都是浓云薄雾。楼台森耸,金乌展翅飞来;

殿宇棱层,玉兔腾身走到。雕梁画栋,碧瓦朱檐,凤扉亮槅映黄纱,
龟背绣帘垂锦带。遥观圣像,九猎舞舜目尧眉;
近观神颜,衮龙袍汤肩禹背。御香不断,天神飞马报丹书;

祭祀依时,老幼望风祈护福。嘉宁殿祥云香霭,正阳门瑞气盘旋。
正是:万民朝拜碧霞宫,四海皈依神圣帝。

吴大舅领月娘到了岱岳庙,正殿上进了香,瞻拜了圣像,庙祝道士在旁宣念了文书
。然后两廊都烧化了纸钱,吃了些斋食。然后领月娘上顶,登四十九盘,攀藤揽葛
上去。娘娘金殿在半空中云烟深处,约四五十里,风云雷雨都望下观看。月娘众人
从辰牌时分岱岳庙起身,登盘上顶,至申时已后方到。娘娘金殿上朱红牌扁,金书
”碧霞宫”三字。进入宫内,瞻礼娘娘金身。怎生模样?但见:

头绾九龙飞凤髻,身穿金缕绛绡衣。蓝田玉带曳长裾,白玉圭璋(敬
木)彩袖。脸如莲萼,天然眉目映云鬟;

唇似金朱,自在规模端雪体。犹如王母宴瑶池,却似嫦娥离月殿。正
大仙云描不就,威严形象画难成。

月娘瞻拜了娘娘仙容,香案边立着一个庙祝道士,约四十年纪,生的五短身材,三
溜髭须,明眸牿齿,头戴簪冠,身披绛服,足登云履,向前替月娘宣读了还愿文疏
,金炉内炷了香,焚化了纸马金银,令小童收了祭供。

原来这庙祝道士,也不是个守本分的,乃是前边岱岳庙里金住持的大徒弟,姓石,
双名伯才,极是个贪财好色之辈,趋时揽事之徒。这本地有个殷太岁,姓殷,双名
天锡,乃是本州知州高廉的妻弟。常领许多不务本的人,或张弓挟弹,牵架鹰犬,
在这上下二宫,专一睃看四方烧香妇女,人不敢惹他。这道士石伯才,专一藏奸蓄
诈,替他赚诱妇女到方丈,任意奸淫,取他喜欢。因见月娘生的姿容非俗,戴着孝
冠儿,若非官户娘子,定是豪家闺眷;又是一位苍白髭髯老子跟随,两个家童,不
免向前稽首,收谢神福:“请二位施主方丈一茶。”吴大舅便道:“不劳生受,还
要赶下山去。”伯才道:“就是下山也还早哩。”

不一时,请至方丈,里面糊的雪白,正面放一张芝麻花坐床,柳黄锦帐,香几上供
养一幅洞宾戏白牡丹图画,左右一对联,大书着:“两袖清风舞鹤,一轩明月谈经
。”伯才问吴大舅上姓,大舅道:“在下姓吴,这个就是舍妹吴氏,因为夫主来还
香愿,不当取扰上宫。”伯才道:“既是令亲,俱延上坐。”他便主位坐了,便叫
徒弟看茶。原来他手下有两个徒弟,一个叫郭守清,一个名郭守礼,皆十六岁,生
得标致,头上戴青段道髻,身穿青绢道服,脚上凉鞋净袜,浑身香气袭人。客至则
递茶递水,斟酒下菜。到晚来,背地便拿他解馋填馅。不一时,守清、守礼安放桌
儿,就摆斋上来,都是美口甜食,蒸堞饼馓,各样菜蔬,摆满春台。每人送上甜水
好茶,吃了茶,收下家火去。就摆上案酒。大盘大碗肴馔,都是鸡鹅鱼鸭上来。用
琥珀镶盏,满泛金波。吴月娘见酒来,就要起身,叫玳安近前,用红漆盘托出一匹
大布、二两白金,与石道士作致谢之礼。吴大舅便说:“不当打搅上宫,这些微礼
致谢仙长。不劳见赐酒食,天色晚来,如今还要赶下山去。”慌的石伯才致谢不已
,说:“小道不才,娘娘福荫,在本山碧霞宫做个住持,仗赖四方钱粮,不管待四
方财主,作何项下使用?今聊备粗斋薄馔,倒反劳见赐厚礼,使小道却之不恭,受
之有愧。”辞谢再三,方令徒弟收下去。一面留月娘、吴大舅坐:“好歹坐片时,
略饮三杯,尽小道一点薄情而已。”吴大舅见款留恳切,不得已和月娘坐下。不一
时,热下饭上来。石道士分付徒弟:“这个酒不中吃,另打开昨日徐知府老爷送的
那一坛透瓶香荷花酒来,与你吴老爹用。”不一时,徒弟另用热壶筛热酒上来。先
满斟一杯,双手递与月娘,月娘不肯接。吴大舅道:“舍妹他天性不用酒。”伯才
道:“老夫人一路风霜,用些何害?好歹浅用些。”一面倒去半钟,递上去与月娘
接了。又斟一杯递与吴大舅,说:“吴老爹,你老人家试用此酒,其味如何?”吴
大舅饮了一口,觉香甜绝美,其味深长,说道:“此酒甚好。”伯才道:“不瞒你
老人家说,此是青州徐知府老爹送与小道的酒。他老夫人、小姐、公子,年年来岱
岳庙烧香建醮,与小道相交极厚。他小姐;衙内又寄名在娘娘位下。见小道立心平
淡,殷勤香火,一味至诚,甚是敬爱小道。常年,这岱岳庙上下二宫钱粮,有一半
征收入库。近年多亏了我这恩主徐知府老爹题奏过,也不征收,都全放常住用度,
侍奉娘娘香火,余者接待四方香客。”这里说话,下边玳安、来安、跟从轿夫,下
边自有坐处,汤饭点心,大盘大碗酒肉,都吃饱了。

吴大舅饮了几杯,见天晚要起身。伯才道:“日色将落,晚了赶不下山去。倘不弃
,在小道方丈权宿一宵,明早下山从容些。”吴大舅道:“争奈有些小行李在店内
,诚恐一时小人罗唣。”伯才笑道:“这个何须挂意!决无丝毫差池。听得是我这
里进香的,不拘村坊店面,闻风害怕,好不好把店家拿来本州来打,就教他寻贼人
下落。”吴大舅听了,就坐住了。伯才拿大钟斟上酒来。吴大舅见酒利害,便推醉
更衣,遂往后边阁上观看随喜去了。这月娘觉身子乏困,便在床上侧侧儿。这石伯
才一面把房门拽上,外边去了。

月娘方才床上歪着,忽听里面响亮了一声,床背后纸门内跳出一个人来,淡红面貌
,三柳髭须,约三十年纪,头戴渗青巾,身穿紫锦袴衫,双手抱住月娘,说道:“
小生殷天锡,乃高太守妻弟。久闻娘子乃官豪宅眷,天然国色,思慕如渴。今既接
英标,乃三生有幸,倘蒙见怜,死生难忘也。”一面按着月娘在床上求欢。月娘唬
的慌做一团,高声大叫:“清平世界,朗朗乾坤,没事把良人妻室,强霸拦在此做
甚!”就要夺门而走。被天锡抵死拦挡不放,便跪下说:“娘子禁声,下顾小生,
恳求怜允。”那月娘越高声叫的紧了,口口大叫:“救人!”平安、玳安听见是月
娘声音,慌慌张张走去后边阁上,叫大舅说:“大舅快去,我娘在方丈和人合口哩
。”这吴大舅慌的两步做一步奔到方丈推门,那里推得开。只见月娘高声:“清平
世界,拦烧香妇女在此做甚么?”这吴大舅便叫:“姐姐休慌,我来了!”一面拿
石头把门砸开。那殷天锡见有人来,撇开手,打床背后一溜烟走了。原来这石道士
床背后都有出路。

吴大舅砸开方丈门。问月娘道:“姐姐,那厮玷污不曾?”月娘道:“不曾玷污。
那厮打床背后走了。”吴大舅寻道士,那石道士躲去一边,只教徒弟来支调。大舅
大怒,喝令手下跟随玳安、来安儿把道士门窗户壁都打碎了。一面保月娘出离碧霞
宫,上了轿子,便赶下山来。

约黄昏时分起身,走了半夜,方到山下客店内。如此这般,告店小二说。小二叫苦
连声,说:“不合惹了殷太岁,他是本州知州相公妻弟,有名殷太岁。你便去了,
俺开店之家,定遭他凌辱,怎肯干休!”吴大舅便多与他一两店钱,取了行李,保
定月娘轿子,急急奔走。后面殷天锡气不舍,率领二三十闲汉,各执腰刀短棍,赶
下山来。

吴大舅一行人,两程做一程,约四更时分,赶到一山凹里。远远树木丛中有灯光,
走到跟前,却是一座石洞,里面有一老僧秉烛念经。吴大舅问:“老师,我等顶上
烧香,被强人所赶,奔下山来,天色昏黑,迷踪失路至此。敢问老师,此处是何地
名?从那条路回得清河县去?”老僧说:“此是岱岳东峰,这洞名唤雪涧洞。贫僧
就叫雪洞禅师,法名普静,在此修行二三十年。你今遇我,实乃有缘。休往前去,
山下狼虽虎豹极多。明日早行,一直大道就是你清河县了。”吴大舅道:“只怕有
人追赶。”老师把眼一观说:“无妨,那强人赶至半山,已回去了。”因问月娘姓
氏。吴大舅道:“此乃吾妹,西门庆之妻。因为夫主,来此进香。得遇老师搭救,
恩有重报,不敢有忘。”于是在洞内歇了一夜。

次日天不亮,月娘拿出一匹大布谢老师。老师不受,说:“贫曾只化你亲生一子作
个徒弟,你意下何如?”吴大舅道:“吾妹止生一子,指望承继家业。若有多余,
就与老师作徒弟。”月娘道:“小儿还小,今才不到一周岁儿,如何来得?”老师
道:“你只许下,我如今不问你要,过十五年才问你要哩。”月娘口中不言,过十
五年再作理会,遂含糊许下老师。一面作辞老师,竟奔清河县大道而来。正是:
世上只有人心歹,万物还教天养人。
但交方寸无诸恶,狼虎丛中也立身。