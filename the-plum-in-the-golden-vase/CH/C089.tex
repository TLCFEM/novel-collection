\chapter{清明节寡妇上新坟~永福寺夫人逢故主}

词曰:

佳人命薄,叹艳代红粉,几多黄土。岂是老天浑不管,
好恶随人自取?既赋娇容,又全慧性,却遣轻归去。不
平如此,问天天更不语。可惜国色天香,随时飞谢,埋
没今如许。借问繁华何处在?多少楼台歌舞,紫陌春游,
绿窗晚秀,姊妹娇眉妩。人生失意,从来无问今古。
右调《翠楼吟》

话说月娘次日备了一张桌,并冥纸尺头之类,大姐身穿孝服,坐轿子,先叫薛嫂押
祭礼,到陈宅来。只见陈敬济正在门首站立,便问:“是那里的?”薛嫂道了万福
,说:“姐夫,你休推不知。你丈母家来与你爹烧纸,送大姐来了。”敬济便道:
”我鸡巴(入曰)的才是丈母!正月十六贴门神--来迟了半个月。人也入了土,
才来上祭。”薛嫂道:“好姐夫,你丈母说,寡妇家没脚蟹,不知亲家灵柩来家,
迟了一步,休怪。”正说着,只见大姐轿子落在门首。敬济问:“是谁?”薛嫂道
:“再有谁?你丈母心内不好,一者送大姐来家,二者敬与你爹烧纸。”敬济骂道
:“趁早把淫妇抬回去!好的死了万万千千,我要他做甚么?”薛嫂道:“常言道
:嫁夫着主。怎的说这个话?”敬济道:“我不要这淫妇了,还不与我走?”那抬
轿的只顾站立不动,被敬济向前踢了两脚,骂道:“还不与我抬了去,我把你花子
脚砸折了,把淫妇鬓毛都蒿净了!”那抬轿子的见他踢起来,只得抬轿子往家中走
不迭。比及薛嫂叫出他娘张氏来,轿子已抬去了。

薛嫂儿没奈何,教张氏收下祭礼,走来回覆吴月娘。把吴月娘气的一个发昏,说道
:“恁个没天理的短命囚根子!当初你家为了官事,搬来丈人家居住,养活了这几
年,今日反恩将仇报起来了。只恨死鬼当初揽的好货在家里,弄出事来,到今日教
我做臭老鼠,教他这等放屁辣臊。”对着大姐说:“孩儿,你是眼见的,丈人、丈
母那些儿亏了他来?你活是他家人,死是他家鬼,我家里也留以留你。你明日还去
,休要怕他,料他挟你不到井里。他好胆子,恒是杀不了人,难道世间没王法管他
也怎的!”当晚不题。

到次日,一顶轿子,教玳安儿跟随着,把大姐又送到陈敬济家来。不想陈敬济不在
家,往坟上替他父亲添土叠山子去了。张氏知礼,把大姐留下,对着玳安说:“大
官到家多多上覆亲家,多谢祭礼,休要和他一般见识。他昨日已有酒了,故此这般
。等我慢慢说他。”一面管待玳安儿,安抚来家。

至晚,陈敬济坟上回来,看见了大姐,就行踢打,骂道:“淫妇,你又来做甚么?
还说我在你家雌饭吃,你家收着俺许多箱笼,因起这大产业,不道的白养活了女婿
!好的死了万千,我要你这淫妇做甚?”大姐亦骂:“没廉耻的囚根子!没天理的
囚根子!淫妇出去吃人杀了,没的禁拿我煞气。”被敬济扯过头发,尽力打了几拳
头。他娘走来解劝,把他娘推了一交。他娘叫骂哭喊,说:“好囚根子,红了眼,
把我也不认的了!”到晚上,一顶轿子,把大姐又送将来,分付道:“不讨将寄放
妆奁箱笼来家,我把你这淫妇活杀了。”这大姐害怕,躲在家中居住,再不敢去了
。这正是:谁知好事多更变,一念翻成怨恨媒。这里不去。不题。

且说一日,三月清明佳节。吴月娘备办香烛、金钱冥纸、三牲祭物,抬了两大食盒
,要往城外坟上与西门庆上新坟祭扫。留下孙雪娥和大姐、众丫头看家。带了孟玉
楼和小玉,并奶子如意儿抱着孝哥儿,都坐轿子往坟上去。又请了吴大舅和大妗子
二人同去。出了城门,只见那郊原野旷,景物芳菲,花红柳绿,仕女游人不断。一
年四季,无过春天,最好景致。日谓之丽日,风谓之和风,吹柳眼,绽花心,拂香
尘。天色暖,谓之暄。天色寒,谓之料峭。骑的马,谓之宝马。坐的轿,谓之香车
。行的路,谓之芳径。地下飞的尘,谓之香尘。千花发蕊,万草生芽,谓之春信。
韶光淡荡,淑景融和。小桃深妆脸妖娆,嫩柳袅宫腰细腻。百转黄鹂惊回午梦,数
声紫燕说破春愁。日舒长暖澡鹅黄,水渺茫浮香鸭绿。隔水不知谁院落,秋千高挂
绿杨烟。端的春景果然是好。有诗为证:
清明何处不生烟,郊外微风挂纸钱。
人笑人歌芳草地,乍晴乍雨杏花天。
海棠枝上绵莺语,杨柳堤边醉客眠。
红粉佳人争画板,彩绳摇拽学飞仙。

吴月娘等轿子到五里原坟上,玳安押着食盒,先到厨下生起火来,厨役落作整理不
题。月娘与玉楼、小玉、奶子如意儿抱着孝哥儿,到于庄院客坐内坐下吃茶,等着
吴大妗子,不见到。玳安向西门庆坟上祭台儿,摆设桌面三牲,羹饭祭物,列下纸
钱,只等吴大妗子。原来大妗子雇不出轿子来,约已牌时分,才同吴大舅雇了两个
驴儿骑将来。月娘便说:“大妗子雇不出轿子来,这驴儿怎的骑?”一面吃了茶,
换了衣服,同来西门庆坟上祭扫。那月娘手拈着五根香,自拿一根,递一根与玉楼
,又递一根与奶子如意儿替孝哥上,那两根递与吴大舅、大妗子。月娘插在香炉内
,深深拜下去,说道:“我的哥哥,你活时为人,死后为神。今日三月清明佳节,
你的孝妻吴氏三姐、孟三姐和你周岁孩童孝哥儿,敬来与你坟前烧一陌钱纸。你保
佑他长命百岁,替你做坟前拜扫之人。我的哥哥,我和你做夫妻一场,想起你那模
样儿并说的话来,是好伤感人也。”拜毕,掩面痛哭。玉楼向前插上香,也深深拜
下,同月娘大哭了一场。玉楼上了香,奶子如意儿抱着哥儿也跪下上香,磕了头。
吴大舅、大妗子都炷了香。行毕礼数,玳安把钱纸烧了。让到庄上卷棚内,放桌席
摆饭,收拾饮酒。月娘让吴大舅、大妗子上坐。月娘与玉楼下陪。小玉和奶子如意
儿,同大妗子家使的老姐兰花,也在两边打横列坐,把酒来斟。按下这里吃酒不题
。

却表那日周守备府里也上坟。先是春梅隔夜和守备睡,假推做梦,睡梦中哭醒了。
守备慌的问:“你怎的哭?”春梅便说:“我梦见我娘向我哭泣,说养我一场,怎
地不与他清明寒食烧纸,因此哭醒了。”守备道:“这个也是养女一场,你的一点
孝心。不知你娘坟在何处?”春梅道:“在南门外永福寺后面便是。”守备说:“
不打紧,永福寺是我家香火院,明日咱家上坟,你叫伴当抬些祭物,往那里与你娘
烧分纸钱,也是好处。”至次日,守备令家人收拾食盒酒果祭品,径往城南祖坟上
。那里有大庄院、厅堂、花园、享堂、祭台。大奶奶、孙二娘并春梅,都坐四人轿
,排军喝路,上坟耍子去了。

却说吴月娘和大舅、大妗子吃了回酒,恐怕晚来,分付玳安、来安儿收拾了食盒酒
果,先往杏花村酒楼下,拣高阜去处,人烟热闹,那里设放桌席等候。又见大妗子
没轿子,都把轿子抬着,后面跟随不坐,领定一簇男女,吴大舅牵着驴儿,压后同
行,踏青游玩。三月桃花店,五里杏花村,只见那随路上坟游玩的王孙士女,花红
柳绿,闹闹喧喧,不知有多少。正走之间,也是合当有事,远远望见绿槐影里,一
座庵院,盖造得十分齐整。但见:

山门高耸,梵宇清幽。当头敕额字分明,两下金刚形势猛。五间大殿,龙鳞瓦砌碧
成行;两下僧房,龟背磨砖花嵌缝。前殿塑风调雨顺,后殿供过去未来。钟鼓楼森
立,藏经阁巍峨。旗竿高峻接青云,宝塔依稀侵碧汉。木鱼横挂,云板高悬。佛前
灯烛莹煌,炉内香烟缭绕。幢旗不断,观音殿接祖师堂;宝盖相连,鬼母位通罗汉
殿。时时护法诸天降,岁岁降魔尊者来。

吴月娘便问:“这座寺叫做甚么寺?”吴大舅便说:“此是周秀老爷香火院,名唤
永福禅林。前日姐夫在日,曾舍几拾两银子在这寺中,重修佛殿,方是这般新鲜。
”月娘向大妗子说:“咱也到这寺里看一看。”于是领着一簇男女,进入寺中来。
不一时,小沙弥看见,报与长老知道:“见有许多男女……”便出方丈来迎请,见
了吴大舅、吴月娘,向前合掌道了问讯,连忙唤小和尚开了佛殿:“请施主菩萨随
喜游玩,小僧看茶。”那小沙弥开了殿门,领月娘一簇男女,前后两廊参拜观看了
一回,然后到长老方丈。长老连忙点上茶来,吴大舅请问长老道号,那和尚答说:
”小僧法名道坚。这寺是恩主帅府周爷香火院,小僧忝在本寺长老,廊下管百十众
僧行,后边禅堂中还有许多云游僧行,常时坐禅,与四方檀越答报功德。”一面方
丈中摆斋,让月娘:“众菩萨请坐。”月娘道:“不当打搅长老宝刹。”一面拿出
五钱银子,教大舅递与长老,佛前请香烧。那和尚打问讯谢了,说道:“小僧无甚
管待,施主菩萨稍坐,略备一茶而已,何劳费心赐与布施。”不一时,小和尚放下
桌儿,拿上素菜斋食饼馓上来。那和尚在旁陪坐,才举箸儿让众人吃时,忽见两个
青衣汉子,走的气喘吁吁,暴雷也一般报与长老,说道:“长老还不快出来迎接,
府中小奶奶来祭祀来了!”慌的长老披袈裟,戴僧帽不迭,分付小沙弥连忙收了家
活,”请列位菩萨且在小房避避,打发小夫人烧了纸,祭毕去了,再款坐一会不迟
。”吴大舅告辞,和尚死活留住,又不肯放。

那和尚慌的鸣起钟鼓来,出山门迎接,远远在马道口上等候。只见一族青衣人,围
着一乘大轿,从东云飞般来,轿夫走的个个汗流满面,衣衫皆湿。那长老躬身合掌
说道:“小僧不知小奶奶前来,理合远接,接待迟了,万勿见罪。”这春梅在轿内
答道:“起动长老。”那手下伴当,又早向寺后金莲坟上,忙将祭桌纸钱来摆设下
。春梅轿子来到,也不到寺,径入寺后白杨树下金莲坟前下轿。两边青衣人伺候。
这春梅不慌不忙,来到坟前,摆了香,拜了四拜,说道:“我的娘,今日庞大姐特
来与你烧陌纸钱,你好处升天,苦处用钱。早知你死在仇人之手,奴随问怎的也娶
来府中,和奴做一处。还是奴耽误了你,悔已是迟了。”说毕,令左右把钱纸烧了
。这春梅向前放声大哭不已。

吴月娘在僧房内,只知有宅内小夫人来到,长老出山门迎接,又不见进来。问小和
尚,小和尚说:“这寺后有小奶奶的一个姐姐,新近葬下,今日清明节,特来祭扫
烧纸。”孟玉楼便道:“怕不就是春梅来了?也不见的。”月娘道:“他那得个姐
来死了葬在此处?”又问小和尚:“这府里小夫人姓甚么?”小和尚道:“姓庞,
前日与了长老四五两经钱,教替他姐姐念经,荐拔生天。”玉楼道:“我听见他爹
说春梅娘家姓庞,叫庞大姐,莫不是他?”正说话,只见长老先来,分付小沙弥:
”好看好茶。”不一时,轿子抬进方丈二门里才下。月娘和玉楼众人打僧房帘内望
外张看,怎样的小夫人。定睛仔细看时,却是春梅。但比昔时出落得长大身材,面
如满月,打扮的粉妆玉琢,头上戴着冠儿,珠翠堆满,凤钗半卸,上穿大红妆花袄
,下着翠兰缕金宽斓裙子,带着丁当禁步,比昔不同许多。但见:

宝髻巍峨,凤钗半卸。胡珠环耳边低挂,金挑凤鬓后双拖。红绣袄偏衬玉香肌,翠
纹裙下映金莲小。行动处,胸前摇响玉丁当;坐下时,一阵麝兰香喷鼻。腻粉妆成
脖颈,花钿巧帖眉尖。举止惊人,貌比幽花殊丽;姿容闲雅,性如兰蕙温柔。若非
绮阁生成,定是兰房长就。俨若紫府琼姬离碧汉,宛如蕊宫仙子下尘寰。

那长老上面独独安放一张公座椅儿,让春梅坐下。长老参见已毕,小沙弥拿上茶来
。长老递茶上去,说道:“今日小僧不知小奶奶来这里祭祀,有失迎接,万望恕罪
。”春梅道:“外日多有起动长老诵经追荐。”那和尚说:“小僧岂敢。有甚殷勤
补报恩主?多蒙小奶奶赐了许多钱衬施。小僧请了八众禅僧,整做道场,看经礼忏
一日。晚夕,又与他老人家装些厢库焚化。道场圆满,才打发两位管家进城,宅里
回小奶奶话。”春梅吃了茶,小和尚接下钟盏来。长老只顾在旁一递一句与春梅说
话,把吴月娘众人拦阻在内,又不好出来的。

月娘恐怕天晚,使小和尚请下长老来,要起身。那长老又不肯放,走来方丈禀春梅
说:“小僧有件事禀知小奶奶。”春梅道:“长老有话,但说无妨。”长老道:“
适间有几位游玩娘子,在寺中随喜,不知小奶奶来。如今他要回去,未知小奶奶尊
意如何。”春梅道:“长老何不请来相见。”那长老慌的来请。吴月娘又不肯出来
,只说:“长老不见罢。天色晚了,俺们告辞去了。”长老见收了他布施,又没管
待,又意不过,只顾再三催促。吴月娘与孟玉楼、吴大妗子推阻不过,只得出来,
春梅一见便道:“原来是二位娘与大妗子。”于是先让大妗子转上,花枝招展磕下
头去。慌的大妗子还礼不迭,说道:“姐姐,今非昔比,折杀老身。”春梅道:“
好大妗子,如何说这话,奴不是那样人。尊卑上下,自然之礼。”拜了大妗子,然
后向月娘、孟玉楼插烛也似磕头。月娘、玉楼亦欲还礼,春梅那里肯,扶起,磕下
四个头,说:“不知是娘们在这里,早知也请出来相见。”月娘道:“姐姐,你自
从出了家门在府中,一向奴多缺礼,没曾看你,你休怪。”春梅道:“好奶奶,奴
那里出身,岂敢说怪。”因见奶子如意儿抱着孝哥儿,说道:“哥哥也长的恁大了
。”月娘说:“你和小玉过来,与姐姐磕过头儿。”那如意儿和小玉二人笑嘻嘻过
来,亦与春梅都平磕了头。月娘道:“姐姐,你受他两个一礼儿。”春梅向头上拔
下一对金头银簪儿来,插在孝哥儿帽儿上。月娘说:“多谢姐姐簪儿,还不与姐姐
唱个喏儿。”如意儿抱着哥儿,真个与春梅唱个喏,把月娘喜欢的要不得。玉楼道
:“姐姐,你今日不到寺中,咱娘儿们怎得遇在一处相见。”春梅道:“便是因俺
娘他老人家新埋葬在这寺后,奴在他手里一场,他又无亲无故,奴不记挂着替他烧
张纸儿,怎生过得去。”月娘道:“我记的你娘没了好几年,不知葬在这里。”孟
玉楼道:“大娘还不知庞大姐说话,说的是潘六姐死了。多亏姐姐,如今把他埋在
这里。”月娘听了,就不言语了。吴大妗子道:“谁似姐姐这等有恩,不肯忘旧,
还葬埋了。你逢节令题念他,来替他烧钱化纸。”春梅道:“好奶奶,想着他怎生
抬举我来!今日他死的苦,这般抛露丢下,怎不埋葬他?”说毕,长老教小和尚放
桌儿,摆斋上来。两张大八仙桌子,蒸酥点心,各样素馔菜蔬,堆满春台,绝细春
芽雀舌甜水好茶。众人吃了,收下家活去。吴大舅自有僧房管待,不在话下。

孟玉楼起身,心里要往金莲坟上看看,替他烧张纸,也是姊妹一场。见月娘不动身
,拿出五分银子,教小沙弥买纸去。长老道:“娘子不消买去,我这里有金银纸,
拿几分烧去。”玉楼把银子递与长老,使小沙弥领到后边白杨树下金莲坟上,见三
尺坟堆,一堆黄土,数柳青蒿。上了根香,把纸钱点着,拜了一拜,说道:“六姐
,不知你埋在这里。今日孟三姐误到寺中,与你烧陌钱纸,你好处升天,苦处用钱
。”一面放声大哭。那奶子如意儿见玉楼往后边,也抱了孝哥儿来看一看。月娘在
方丈内和春梅说话,教奶子休抱了孩子去,只怕唬了他。如意儿道:“奶奶,不妨
事,我知道。”径抱到坟上,看玉楼烧纸哭罢回来。

春梅和月娘匀了脸,换了衣裳,分付小伴当将食盒打开,将各样细果甜食,肴品点
心攒盒,摆下两桌子,布甑内筛上酒来,银钟牙箸,请大妗子、月娘、玉楼上坐,
他便主位相陪。奶子、小玉,都在两边打横。吴大舅另放一张桌子在僧房内。正饮
酒中间,忽见两个青衣伴当走来,跪下禀道:“老爷在新庄,差小的来请小奶奶看
杂耍调百戏的。大奶奶、二奶奶都去了,请奶奶快去哩。”这春梅不慌不忙,说:
”你回去,知道了。”那二人应诺下来,又不敢去,在下边等候。大妗子、月娘便
要起身,说:“姐姐,不可打搅。天色晚了,你也有事,俺们去罢。”那春梅那里
肯放,只顾令左右将大钟来劝道:“咱娘儿们会少离多,彼此都见长着,休要断了
这门亲路。奴也没亲没故,到明日娘的好日子,奴往家里走走去。”月娘道:“我
的姐姐,说一声儿就勾了,怎敢起动你?容一日,奴去看姐姐去。”饮过一杯,月
娘说:“我酒勾了,你大妗子没轿子,十分晚了,不好行的。”春梅道:“大妗子
没轿子,我这里有跟随小马儿,拨一匹与妗子骑,关了家去。”大妗子再三不肯,
辞了,方一面收拾起身。春梅叫过长老来,令小伴当拿出一匹大布、五钱银子与长
老。长老拜谢了,送出山门。春梅与月娘拜别,看着月娘、玉楼众人上了轿子,他
也坐轿子,两下分路,一簇人明随喝道,往新庄上去了。正是:
树叶还有相逢时,岂可人无得运时。