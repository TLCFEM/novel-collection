\chapter{陈敬济感旧祭金莲~庞大姐埋尸托张胜}

诗曰:
梦中虽暂见,及觉始知非。
展不成寐,徒倚独披衣。
凄凄晓风急,腌腌月光微。
空床常达旦,所思终不归。

话说武松杀了妇人、王婆,劫去财物,逃上梁山去了,不题。且说王潮儿街上叫了
保甲来,见武松家前后门都不开,又王婆家被劫去财物,房中衣服丢的横三竖四,
就知是武松杀人劫财而去。未免打开前后门,见血沥沥两个死尸倒在地下,妇人心
肝五脏用刀插在后楼房檐下。迎儿倒扣在房中。问其故,只是哭泣。次日早衙,呈
报到本县,杀人凶刃都拿放在面前。本县新任知县也姓李,双名昌期,乃河北真定
府枣强县人氏。听见杀人公事,即委差当该吏典,拘集两邻保甲,并两家苦主王潮
、迎儿。眼同当街,如法检验。生前委被武松因忿带酒,杀潘氏、王婆二命,叠成
文案,就委地方保甲瘗埋看守。挂出榜文,四厢差人跟寻,访拿正犯武松,有人首
告者,官给赏银五十两。

守备府中张胜、李安打着一百两银子到王婆家,看见王婆、妇人俱已被武松杀死,
县中差人检尸,捉拿凶犯。二人回报到府中。春梅听见妇人死了,整哭了两三日,
茶饭都不吃。慌了守备,使人门前叫调百戏的货郎儿进去,耍与他观看,只是不喜
欢。日逐使张胜、李安打听,拿住武松正犯,告报府中知道,不在话下。

按下一头。且表陈敬济前往东京取银子,一心要赎金莲,成其夫妇。不想走到半路
,撞见家人陈定从东京来,告说家爷病重之事:“奶奶使我来请大叔往家去,嘱托
后事。”这敬济一闻其言,两程做一程,路上趱行。有日到东京他姑夫张世廉家。
张世廉已死,止有姑娘见在。他父亲陈洪已是没了三日,满家带孝。敬济参见他父
亲灵座。与他母亲张氏并姑娘磕头。张氏见他成人,母子哭做一处,通同商议:“
如今一则以喜,一则以忧。”敬济便道:“如何是喜,如何是忧?”张氏道:“喜
者,如今朝廷册立东宫,郊天大赦;忧则不想你爹爹病死在这里,你姑夫又没了,
姑娘守寡,这里住着不是常法,如今只得和你打发你爹爹灵柩回去,葬埋乡井,也
是好处。”敬济听了,心内暗道:“这一回发送,装载灵柩家小粗重上车,少说也
得许多日期耽阁,却不误了六姐?不如先诓了两车细软箱笼家去,待娶了六姐,再
来搬取灵柩不迟。”一面对张氏说道:“如今随路盗贼,十分难走。假如灵柩家小
箱笼一同起身,未免起眼,倘遇小人怎了?宁可耽迟不耽错。我先押两车细软箱笼
家去,收拾房屋。母亲随后和陈定、家眷并父亲灵柩,过年正月同起身回家,寄在
城外寺院,然后做斋念经、筑坟安葬,也是不迟。”张氏终是妇人家,不合一时听
信敬济巧言,就先打点细软箱笼,装载两大车,上插旗号,扮做香车。从腊月初一
日东京起身,不上数日,到了山东清河县家门首,对他母舅张团练说:“父亲已死
,母亲押灵车,不久就到。我押了两车行李,先来收拾打扫房屋。”他母舅听说:
”既然如此,我仍搬回家去便了。”一面就令家人搬家活,腾出房子来。敬济见母
舅搬去,满心欢喜,说:“且得冤家离眼前,落得我娶六姐来家,自在受用。我父
亲已死,我娘又疼我。先休了那个淫妇,然后一纸状子,把俺丈母告到官,追要我
寄放东西,谁敢道个不字?又挟制俺家充军人数不成!”正是:
人便如此如此,天理不然不然。

这敬济就打了一百两银子在腰里,另外又袖着十两谢王婆,来到紫石街王婆门首。
可霎作怪,只见门前街旁埋着两个尸首,上面两杆枪交叉挑着个灯笼,门前挂着一
张手榜,上书:“本县为人命事:凶犯武松,杀死潘氏、王婆二命,有人捕获首告
官司者,官给赏银五十两。”这敬济仰头看见,便立睁了。只见窝铺中站出两个人
来,喝声道:“甚么人?看此榜文做甚?见今正身凶犯捉拿不着,你是何人?”大
叉步便来捉获。敬济慌的奔走不迭,恰走到石桥下酒楼边,只见一个人,头戴万字
巾,身穿青衲袄,随后赶到桥下,说道:“哥哥,你好大胆,平白在此看他怎的?
”这敬济扭回头看时,却是一个识熟朋友--铁指甲杨二郎。二人声喏。杨二道:
”哥哥一向不见,那里去来?”敬济便把东京父死往回之事,告说一遍:“恰才这
杀死妇人,是我丈人的小,潘氏。不知他被人杀了。适才见了榜文,方知其故。”
杨二郎告道:“他是小叔武松,充配在外,遇赦回还,不知因甚杀了妇人,连王婆
子也不饶。他家还有个女孩儿,在我姑夫姚二郎家养活了三四年。昨日他叔叔杀了
人,走的不知下落。我姑夫将此女县中领出,嫁与人为妻小去了。见今这两个尸首
,日久只顾埋着,只是苦了地方保甲看守,更不知何年月日才拿住凶犯武松。”说
毕,杨二郎招了敬济,上酒楼饮酒:“与哥拂尘。”敬济见妇人已死,心中痛苦不
了,那里吃得下酒。约莫饮勾三杯,就起身下楼,作别来家。

到晚夕,买了一陌钱纸,在紫石街离王婆门首远远的石桥边,叫着妇人:“潘六姐
,我小兄弟陈敬济,今日替你烧陌钱纸。皆因我来迟了一步,误了你性命。你活时
为人,死后为神,早佑佑捉获住仇人武松,替你报仇雪恨。我在法场上看着剐他,
方趁我平生之志。”说毕哭泣,烧化了钱纸。敬济回家,闭了门户。走归房中,恰
才睡着,似睡不睡,梦见金莲身穿素服,一身带血,向敬济哭道:“我的哥哥,我
死的好苦也!实指望与你相处在一处,不期等你不来,被武松那厮害了性命。如今
阴司不收,我白日游游荡荡,夜归各处寻讨浆水,适间蒙你送了一陌钱纸与我。但
只是仇人未获,我的尸首埋在当街,你可念旧日之情,买具棺材盛了葬埋,免得日
久暴露。”敬济哭道:“我的姐姐,我可知要葬埋你。但恐我丈母那无仁义的淫妇
知道。他只恁赖我,倒趁了他机会。姐姐,你须往守备府中,对春梅说知,教他葬
埋你身尸便了。”妇人道:“刚才奴到守备府中,又被那门神户尉拦挡不放,奴须
慢慢再哀告他则个。”敬济哭着,还要拉着他说话,被他身上一阵血腥气,撇气挣
脱,却是南柯一梦。枕上听那更鼓时,正打三更三点,说道:“怪哉!我刚才分明
梦见六姐向我诉告衷肠,教我葬埋之意,又不知甚年何日拿着武松,是好伤感人也
!”正是:
梦中无限伤心事,独坐空房哭到明。

按下一头。却表县中访拿武松,约两个月有余,捕获不着,已知逃遁梁山为盗。地
方保甲邻佑呈报到官,所有两个尸首,相应责令家属领埋。王婆尸首,便有他儿子
王潮领的埋葬。止有妇人身尸,无人来领。却说府中春梅,两三日一遍,使张胜、
李安来县中打听。回去只说凶犯还未拿住,尸首照旧埋瘗,地方看守,无人敢动。
直挨过年,正月初旬时节,忽一日晚间,春梅作一梦。恍恍惚惚,梦见金莲云髻蓬
松,浑身是血,叫道:“庞大姐,我的好姐姐,奴死的好苦也!所有奴的尸首,在
街暴露日久,风吹雨洒,鸡犬作践,无人领埋。奴举眼无亲,你若念旧日母子之情
,买具棺木,把奴埋在一个去处,奴在阴司口眼皆闭。”说毕大哭不止。春梅扯住
他,还要再问他别的话,被他挣开,撇手惊觉,却是南柯一梦。从睡梦中直哭醒来
,心内犹疑不定。

次日叫进张胜、李安分付:“你二人去县中打听,那埋的妇人、婆子尸首还有也没
有。”张胜、李安应诺去了。不多时,来回报:“正犯凶身已自逃走脱了。所有杀
死身尸,地方看守,日久不便,相应责令各人家属领埋。那婆子尸首,他儿子招领
的去了。那妇人无人来领,还埋在街心。”春梅道:“既然如此,我这桩事儿,累
你二人替我干得来,我还重赏你。”二人跪下道:“小夫人说那里话,若肯在老爷
前抬举小人一二,便消受不了。虽赴汤跳水,敢说不去?”春梅走到房中,拿出十
两银子,两匹大布,委付二人道:“这死的妇人,是我一个嫡亲姐姐,嫁在西门庆
家,今日出来,被人杀死。你二人休教你老爷知道,拿这银子替我买一具棺材,把
他装殓了,抬出城外,择方便地方埋葬停当,我还重赏你。”二人道”这个不打紧
,小人就去。”李安说:“只怕县中不教你我领尸怎了?须拿老爷个贴儿,下与县
官才好。”张胜道:“只说小夫人是他妹子,嫁在府中,那县官不敢不依,何消贴
子。”于是领了银子,来到班房内。张胜便向李安说:“想必这死的妇人,与小夫
人曾在西门庆家做一处,相结的好,今日方这等为他费心。想着死了时,整哭了三
四日,不吃饭,直教老爷门前叫了调百戏货郎儿,调与他观看,还不喜欢。今日他
无亲人领去,小夫人岂肯不葬埋他?咱每若替他干得此事停当,早晚他在老爷跟前
,只方便你我,就是一点福星。见今老爷百依百随,听他说话,正经大奶奶、二奶
奶且打靠后。”说毕,二人拿银子到县前递了领状,就说他妹子在老爷府中,来领
尸首。使了六两银子,合了一具棺材,把妇人尸首掘出,把心肝填在肚内,用线缝
上,用布装殓停当,装入材内。张胜说:“就埋在老爷香火院永福寺里罢,那里有
空闲地。”就叫了两名伴当,抬到永福寺,对长老说:“这是宅内小夫人的姐姐,
要一块地儿葬埋。”长老不敢怠慢,就在寺后拣一块空心白杨树下那里葬埋。已毕
,走来宅内回春梅话,说:“除买棺材装殓,还剩四两银子。”交割明白。春梅分
付:“多有起动,你二人将这四两银子,拿二两与长老道坚,教他早晚替他念些经
忏,超度他升天。”又拿出一大坛酒,一腿猪肉,一腿羊肉:“这二两银子,你每
人将一两家中盘缠。”二人跪下,那里敢接?只说:“小夫人若肯在老爷面前抬举
小人,消受不了。这些小劳,岂敢接受银两。”春梅道:“我赏你,不收,我就恼
了。”二人只得磕头领了出来。两个班房吃酒,甚是称念小夫人好处。次日,张胜
送银子与长老念经,春梅又与五钱银子买纸,与金莲烧,俱不在话下。

却说陈定从东京载灵柩家眷到清河县城外,把灵柩寄在永福寺,等念经发送,归葬
坟内。敬济在家听见母亲张氏家小车辆到了,父亲灵柩寄停在城外永福寺,收卸行
李已毕,与张氏磕了头。张氏怪他:“就不去接我一接。”敬济只说:“心中不好
,家里无人看守。”张氏便问:“你舅舅怎的不见?”敬济道:“他见母亲到,连
忙搬回家去了。”张氏道:“且教你舅舅住着,慌搬去怎的?”一面他母舅张团练
来看姐姐。姊妹抱头而哭,置酒叙说,不必细说。

次日,张氏早使敬济拿五两银子、几陌金银钱纸,往门外与长老,替他父亲念经。
正骑头口街上走,忽撞遇他两个朋友陆大郎、杨大郎,下头口声喏。二人问道:“
哥哥那里去?”敬济悉言:“先父灵柩寄在门外寺里,明日二十日是终七,家母使
我送银子与长老,做斋念经。”二人道:“兄弟不知老伯灵柩到了,有失吊问。”
因问:“几时发引安葬?”敬济道:“也只在一二日之间,念经毕,入坟安葬。”
说罢,二人举手作别。这敬济又叫住,因问杨大郎:“县前我丈人的小,那潘氏尸
首怎不见?被甚人领的去了?”杨大郎便道:“半月前,地方因捉不着武松,禀了
本县相公,令各家领去葬埋。王婆是他儿子领去。这妇人尸首,丢了三四日,被守
备府中买了一口棺材,差人抬出城外永福寺去葬了。”敬济听了,就知是春梅在府
中收葬了他尸首。因问二郎:“城外有几个永福寺?”二郎道:“南门外只有一个
永福寺,是周秀老爷香火院,那里有几个永福寺来?”敬济听了,暗喜:“就是这
个永福寺,也是缘法凑巧,喜得六姐亦葬在此处。”一面作别二人,打头口出城,
径到永福寺中。见了长老,且不说念经之事,就先问长老道坚:“此处有守备府中
新近葬的一个妇人,在那里?”长老道:“就在寺后白杨树下。说是宅内小夫人的
姐姐。”这陈敬济且不参见他父亲灵柩,先拿钱祭物,至于金莲坟上,与他祭了,
烧化钱纸,哭道:“我的六姐,你兄弟陈敬济来与你烧一陌纸钱,你好处安身,苦
处用钱。”祭毕,然后才到方丈内他父亲灵柩跟前烧纸祭祀。递与长老经钱,教他
二十日请八众禅僧,念断七经。长老接了经衬,备办斋供。敬济到家,回了张氏话
。二十日都去寺中拈香,择吉发引,把父亲灵柩归到祖茔。安葬已毕,来家母子过
日不题。

却表吴月娘,一日二月初旬,天气融和,孟玉楼、孙雪娥、西门大姐、小玉,出来
大门首站立,观看来往车马,人烟热闹。忽见一簇男女,跟着个和尚,生的十分胖
大,头顶三尊铜佛,身上构着数枝灯树,杏黄袈裟风兜袖,赤脚行来泥没踝。当时
古人有几句,赞的这行脚僧好处:

打坐参禅,讲经说法。铺眉苦眼,习成佛祖家风;赖教求食,立起法
门规矩。白日里卖杖摇铃,黑夜间舞枪弄棒。有时门首磕光头,饿了
街前打响嘴。空色色空,谁见众生离下土?去来来去,何曾接引到西
方。

那和尚见月娘众妇人在门首,便向前道了个问讯,说道:“在家老菩萨施主,既生
在深宅大院,都是龙华一会上人。贫僧是五台山下来的,结化善缘,盖造十王功德
,三宝佛殿。仰赖十方施主菩萨,广种福田,舍资才共成胜事,种来生功果。贫僧
只是挑脚汉。”月娘听了他这般言语,便唤小玉往房中以一顶僧帽,一双僧鞋,一
吊铜钱,一斗白米。原来月娘平昔好斋僧布施,常时发心做下僧帽、僧鞋,预备来
施。这小玉取出来,月娘分付:“你叫那师父近前来,布施与他。”这小玉故做娇
态,高声叫道:“那变驴的和尚,过不过来!俺奶奶布施与你这许多东西,还不磕
头哩。”月娘便骂道:“怪堕业的小臭肉儿,一个僧家,是佛家弟子,你有要没紧
,恁谤他怎的?不当家化化的,你这小淫妇儿,到明日不知堕多少罪业!”小玉笑
道:“奶奶,这贼和尚,我叫他,他怎的把一双贼眼,眼上眼下打量我?”那和尚
双手接了鞋帽钱来,打问讯说道:“多谢施主老菩萨布施。”小玉道:“这秃厮好
无礼。这些人站着,只打两个问讯儿,就不与我打一个儿?”月娘道:“小肉儿,
还恁说白道黑道。他一个佛家之子,你也消受不的他这个问讯。”小玉道:“奶奶
,他是佛爷儿子,谁是佛爷女儿?”月娘道:“相这比丘尼姑僧,是佛的女儿。”
小玉道:“譬若说,相薛姑子、王姑子、大师父,都是佛爷女儿,谁是佛爷女婿?
”月娘忍不住笑,骂道:“这贼小淫妇儿,也学的油嘴滑舌,见见就说下道儿去了
。”小玉道:“奶奶只骂我,本等这秃和尚贼眉竖眼的只看我。”孟玉楼道:“他
看你,想必认得你,要度脱你去。”小玉道:“他若度我,我就去。”说着,众妇
女笑了一回。月娘喝道:“你这小淫妇儿,专一毁僧谤佛。”那和尚得了布施,顶
着三尊佛扬长而去了。小玉道:“奶奶还嗔我骂他,你看这贼秃,临去还看了我一
眼才去了。”有诗单道月娘修善施僧好处:
守寡看经岁月深,私邪空色久违心。
奴身好似天边月,不许浮云半点侵。

月娘众人正在门首说话,忽见薛嫂儿提着花箱儿,从街上过来。见月娘众人道了万
福。月娘问:“你往那里去来?怎的影迹儿也不来我这里走走?”薛嫂儿道:“不
知我终日穷忙的是些甚么。这两日,大街上掌刑张二老爹家,与他儿子和北边徐公
公家做亲,娶了他侄女儿,也是我和文嫂儿说的亲事。昨日三朝,摆大酒席,忙的
连守备府里咱家小大姐那里叫我,也没去,不知怎么恼我哩。”月娘问道:“你如
今往那里去?”薛嫂道:“我有桩事,敬来和你老人家说来。”月娘道:“你有话
进来说。”一面让薛嫂儿到后边上房里坐下,吃了茶。薛嫂道:“你老人家还不知
道,你陈亲家从去年在东京得病没了,亲家母叫了姐夫去,搬取老小灵柩。从正月
来家,已是念经发送,坟上安葬毕。我听说你老人家这边知道,怎不去烧张纸儿,
探望探望。”月娘道:“你不来说,俺怎得晓的,又无人打听。倒只知道潘家的吃
他小叔儿杀了,和王婆子都埋在一处,却不知如今怎样了。”薛嫂儿道:“自古生
有地儿死有处。五娘他老人家,不因那些事出去了,却不好来。平日不守本分,干
出丑事来,出去了,若在咱家里,他小叔儿怎得杀了他?还是冤有头,债有主。倒
还亏了咱家小大姐春梅,越不过娘儿们情场,差人买了口棺材,领了他尸首,葬埋
了。不然只顾暴露着,又拿不着小叔子,谁去管他?”孙雪娥在旁说:“春梅在守
备府中多少时儿,就这等大了?手里拿出银子,替他买棺材埋葬,那守备也不嗔,
当他甚么人?”薛嫂道:“耶(口乐),你还不知,守备好不喜他,每日只在他房
里歇卧,说一句依十句,一娶了他,见他生的好模样儿,乖觉伶俐,就与他西厢房
三间房住,拨了个使女伏侍他。老爷一连在他房里歇了三夜,替他裁四季衣服,上
头。三日吃酒,赏了我一两银子,一匹段子。他大奶奶五十岁,双目不明,吃长斋
,不管事。东厢孙二娘生了小姐,虽故当家,挝着个孩子。如今大小库房钥匙,倒
都是他拿着,守备好不听他说话哩。且说银子,手里拿不出来?”几句说的月娘、
雪娥都不言语。坐了一回,薛嫂起身。月娘分付:“你明日来,我这里备一张祭桌
,一匹尺头,一分冥纸,你来送大姐与他公公烧纸去。”薛嫂儿道:“你老人家不
去?”月娘道:“你只说我心中不好,改日望亲家去罢。”那薛嫂约定:“你教大
姐收拾下等着我。饭罢时候我来。”月娘道:“你如今到那里去?守备府中不去也
罢。”薛嫂道:“不去,就惹他怪死了。他使小伴当叫了我好几遍了。”月娘道:
”他叫你做甚么?”薛嫂道:“奶奶,你不知。他如今有了四五个月身孕了,老爷
好不喜欢,叫了我去,已定赏我。”提着花箱,作辞去了。雪娥便说:“老淫妇说
的没个行款也!他卖与守备多少时,就有了半肚孩子,那守备身边少说也有几房头
,莫就兴起他来,这等大道?”月娘道:“他还有正景大奶奶,房里还有一个生小
姐的娘子儿哩。”雪娥道:“可又来!到底还是媒人嘴,一尺水十丈波的。”不因
今日雪娥说话,正是:从天降下钩和线,就地引来是非来。有诗为证:
曾记当年侍主旁,谁知今日变风光。
世间万事皆前定,莫笑浮生空自忙。