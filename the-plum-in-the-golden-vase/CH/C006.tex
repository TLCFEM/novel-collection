\chapter{何九受贿瞒天~王婆帮闲遇雨}

词曰:

别后谁知珠分玉剖。忘海誓山盟天共久,偶恋着山鸡,辄弃鸾俦。从
此箫郎泪暗流,过秦楼几空回首。纵新人胜旧,也应须一别,洒泪登舟。

却说西门庆去了。到天大明,王婆拿银子买了棺材冥器,又买些香烛纸钱之类
,归来就于武大灵前点起一盏随身灯。邻舍街坊都来看望,那妇人虚掩着粉脸假哭
。众街坊问道:“大郎得何病患便死了?”那婆娘答道:“因害心疼,不想一日日
越重了,看看不能够好。不幸昨夜三更鼓死了,好是苦也!”又哽哽咽咽假哭起来
。众邻舍明知道此人死的不明,不好只顾问他。众人尽劝道:“死是死了,活的自
要安稳过。娘子省烦恼,天气暄热。”那妇人只得假意儿谢了,众人各自散去。王
婆抬了棺材来,去请仵作团头何九。但是入殓用的都买了,并家里一应物件也都买
了。就于报恩寺叫了两个禅和子,晚夕伴灵拜忏。不多时,何九先拨了几个火家整
顿。

且说何九到巳牌时分,慢慢的走来,到紫石街巷口,迎见西门庆。叫道:“老
九何往?”何九答道:“小人只去前面殓这卖炊饼的武大郎尸首。”西门庆道:“
且停一步说话。”何九跟着西门庆,来到转角头一个小酒店里,坐下在阁儿内。西
门庆道:“老九请上坐。”何九道:“小人是何等人,敢对大官人一处坐的!”西
门庆道:“老九何故见外?且请坐。”二人让了一回,坐下。西门庆吩咐酒保:“
取瓶好酒来。”酒保一面铺下菜蔬果品按酒之类,一面烫上酒来。何九心中疑忌,
想道:“西门庆自来不曾和我吃酒,今日这杯酒必有蹊跷。”两个饮够多时,只见
西门庆向袖子里摸出一锭雪花银子,放在面前说道:“老九休嫌轻微,明日另有酬
谢。”何九叉手道:“小人无半点效力之处,如何敢受大官人见赐银两!若是大官
人有使令,小人也不敢辞。”西门庆道:“老九休要见外,请收过了。”何九道:
“大官人便说不妨。”西门庆道:“别无甚事。少刻他家自有些辛苦钱。只是如今
殓武大的尸首,凡百事周全,一床锦被遮盖则个。”何九道:“我道何事!这些小
事,有甚打紧,如何敢受大官人银两?”西门庆道:“你若不受时,便是推却。”
何九自来惧西门庆是个把持官府的人,只得收了银子。又吃了几杯酒,西门庆呼酒
保来:“记了帐目,明日来我铺子内支钱。”两个下楼,一面出了店门。临行,西
门庆道:“老九是必记心,不可泄漏。改日另有补报。”吩咐罢,一直去了。

何九接了银子,自忖道:“其中缘故那却是不须提起的了。只是这银子,恐怕
武二来家有说话,留着倒是个见证。”一面又忖道:“这两日倒要些银子搅缠,且
落得用了,到其间再做理会便了。”于是一直到武大门首。只见那几个火家正在门
首伺候。王婆也等的心里火发。何九一到,便间火家:“这武大是甚病死了?”火
家道:“他家说害心疼病死了。”何九入门,揭起帘子进来。王婆接着道:“久等
多时了,阴阳也来了半日,老九如何这咱才来?”何九道:“便是有些小事绊住了
脚,来迟了一步。”只见那妇人穿着一件素淡衣裳,白布[髟狄]髻,从里面假哭
出来。何九道:“娘子省烦恼,大郎已是归天去了。”那妇人虚掩着泪眼道:“说
不得的苦!我夫心疼病症,几个日子便把命丢了。撇得奴好苦!”这何九一面上上
下下看了婆娘的模样,心里暗道:“我从来只听得人说武大娘子,不曾认得他。原
来武大郎讨得这个老婆在屋里。西门庆这十两银子使着了!”一面走向灵前,看武
大尸首。阴阳宣念经毕,揭起千秋幡,扯开白绢,定睛看时,见武大指甲青,唇口
紫,面皮黄,眼皆突出,就知是中恶。旁边那两个火家说道:“怎的脸也紫了,口
唇上有牙痕,口中出血?”何九道:“休得胡说!两日天气十分炎热,如何不走动
些!”一面七手八脚葫芦提殓了,装入棺材内,两下用长命钉钉了。王婆一力撺掇
,拿出一吊钱来与何九,打发众火家去了,就问:“几时出去?”王婆道:“大娘
子说只三日便出殡,城外烧化。”何九也便起身。那妇人当夜摆着酒请人,第二日
请四个僧念经。第三日早五更,众火家都来扛抬棺材,也有几个邻舍街坊,吊孝相
送。那妇人带上孝,坐了一乘轿子,一路上口内假哭“养家人”。来到城外化人场
上,便教举火烧化棺材。不一时烧得干干净净,把骨殖撒在池子里,原来斋堂管待
,一应都是西门庆出钱整顿。

那妇人归到家中,楼上设个灵牌,上写“亡夫武大郎之灵”。灵床子前点一盏
琉璃灯,里面贴些经幡钱纸、金银锭之类。那日却和西门庆做一处,打发王婆家去
,二人在楼上任意纵横取乐,不比先前在王婆家茶房里,只是偷鸡盗狗之欢。如今
武大已死,家中无人,两个肆意停眠整宿。初时西门庆恐邻舍瞧破,先到王婆那边
坐一回,落后带着小厮竟从妇人家后门而入。自此和妇人情沾意密,常时三五夜不
归去,把家中大小丢得七颠八倒,都不欢喜。正是:

色胆如天不自由,情深意密两绸缪。
贪欢不管生和死,溺爱谁将身体修。
只为恩深情郁郁,多因爱阔恨悠悠。
要将吴越冤仇解,地老天荒难歇休。

光阴迅速,日月如梭,西门庆刮剌那妇人将两月有余。一日,将近端阳佳节,
但见:

绿杨袅袅垂丝碧,海榴点点胭脂赤。微微风动幔,飒飒凉侵扇。处处
过端阳,家家共举觞。

却说西门庆自岳庙上回来,到王婆茶坊里坐下。那婆子连忙点一盏茶来,便问
:“大官人往那里来?怎的不过去看看大娘子?”西门庆道:“今日往庙上走走。
大节间记挂着,来看看六姐。”婆子道:“今日他娘潘妈妈在这里,怕还未去哩。
等我过去看看,回大官人。”这婆子走过妇人后门看时,妇人正陪潘妈妈在房里吃
酒,见婆子来,连忙让坐。妇人笑道:“干娘来得正好,请陪俺娘且吃个进门盏儿
,到明日养个好娃娃!”婆子笑道:“老身又没有老伴儿,那里得养出来?你年小
少壮,正好养哩!”妇人道:“常言小花不结老花儿结。”婆子便看着潘妈妈嘈道
:“你看你女儿,这等伤我,说我是老花子。到明日还用着我老花子哩!”说罢,
潘妈道:“他从小是这等快嘴,干娘休要和他一般见识。”王婆道:“你家这姐姐
,端的百伶百俐,不枉了好个妇女。到明日,不知什么有福的人受的他起。”潘妈
妈道:“干娘既是撮合山,全靠干娘作成则个!”一面安下钟箸,妇人斟酒在他面
前。婆子一连陪了几杯酒,吃得脸红红的,又怕西门庆在那边等候,连忙丢了个眼
色与妇人,告辞归家。妇人知西门庆来了,因一力撺掇他娘起身去了。将房中收拾
干净,烧些异香,从新把娘吃的残馔撇去,另安排一席齐整酒肴预备。

西门庆从后门过来,妇人接着到房中,道个万福坐下。原来妇人自从武大死后
,怎肯带孝!把武大灵牌丢在一边,用一张白纸蒙着,羹饭也不揪采。每日只是浓
妆艳抹,穿颜色衣服,打扮娇样。因见西门庆两日不来,就骂:“负心的贼,如何
撇闪了奴,又往那家另续上心甜的了?把奴冷丢,不来揪采。”西门庆道:“这两
日有些事,今日往庙上去,替你置了些首饰珠翠衣服之类。”那妇人满心欢喜。西
门庆一面唤过小厮玳安来,毡包内取出,一件件把与妇人。妇人方才拜谢收了。小
女迎儿,寻常被妇人打怕的,以此不瞒他,令他拿茶与西门庆吃。一面妇人安放桌
儿,陪西门庆吃茶。西门庆道:“你不消费心,我已与了干娘银子买东西去了。大
节间,正要和你坐一坐。”妇人道:“此是待俺娘的,奴存下这桌整菜儿。等到干
娘买来,且有一回耽搁,咱且吃着。”妇人陪西门庆脸儿相贴,腿儿相压,并肩一
处饮酒。

且说婆子提着个篮儿,走到街上打酒买肉。那时正值五月初旬天气,大雨时行
。只见红日当天,忽被黑云遮掩,俄而大雨倾盆。但见:

乌云生四野,黑雾锁长空。刷剌剌漫空障日飞来,一点点击得芭蕉声
碎。狂风相助,侵天老桧掀翻;霹雳交加,泰华嵩乔震动。洗炎驱暑,润
泽田苗,正是:江淮河济添新水,翠竹红榴洗濯清。

那婆子正打了一瓶酒,买了一篮菜蔬果品之类,在街上遇见这大雨,慌忙躲在人家
房檐下,用手帕裹着头,把衣服都淋湿了。等了一歇,那雨脚慢了些,大步云飞来
家。进入门来,把酒肉放在厨房下,走进房来,看妇人和西门庆饮酒,笑嘻嘻道:
“大官人和大娘子好饮酒!你看把婆子身上衣服都淋湿了,到明日就教大官人赔我
!”西门庆道:“你看老婆子,就是个赖精。”婆子道:“也不是赖精,大官人少
不得赔我一匹大海青。”妇人道:“干娘,你且饮盏热酒儿。”那婆子陪着饮了三
杯,说道:“老身往厨下烘衣裳去也。”一面走到厨下,把衣服烘干,那鸡鹅嗄饭
切割安排停当,用盘碟盛了果品之类,都摆在房中,烫上酒来。西门庆与妇人重斟
美酒,交杯叠股而饮。西门庆饮酒中间,看见妇人壁上挂着一面琵琶,便道:“久
闻你善弹,今日好夕弹个曲儿我下酒。”妇人笑道:“奴自幼粗学一两句,不十分
好,你却休要笑耻。”西门庆一面取下琵琶来,搂妇人在怀,看着他放在膝儿上,
轻舒玉笋,款弄冰弦,慢慢弹着,低声唱道:

冠儿不带懒梳妆,髻挽青丝云鬓光,金钗斜插在乌云上。唤梅香,开
笼箱,穿一套素缟衣裳,打扮的是西施模样。出绣房,梅香,你与我卷起
帘儿,烧一炷儿夜香。

西门庆听了,欢喜的没入脚处,一手搂过妇人粉颈来,就亲了个嘴,称夸道:
“谁知姐姐有这段儿聪明!就是小人在构栏三街两巷相交唱的,也没你这手好弹唱
!”妇人笑道:“蒙官人抬举,奴今日与你百依百顺,是必过后休忘了奴家。”西
门庆一面捧着他香腮,说道:“我怎肯忘了姐姐!”两个殢雨尤云,调笑玩
耍。少顷,西门庆又脱下他一只绣花鞋儿,擎在手内,放一小杯酒在内,吃鞋杯耍
子。妇人道:“奴家好小脚儿,你休要笑话。”不一时,二人吃得酒浓,掩闭了房
门,解衣上床玩耍。王婆把大门顶着,和迎儿在厨房中坐地。二人在房内颠鸾倒凤
,似水如鱼。那妇人枕边风月,比娼妓尤甚,百般奉承。西门庆亦施逞枪法打动。
两个女貌郎才,俱在妙龄之际。

寂静兰房簟枕凉,佳人才子意何长。
方才枕上浇红烛,忽又偷来火隔墙。
粉蝶探香花萼颤,蜻蜓戏水往来狂。
情浓乐极犹余兴,珍重檀郎莫相忘。

当日西门庆在妇人家盘桓至晚,欲回家,留了几两散碎银子与妇人做盘缠。妇
人再三挽留不住。西门庆带上眼罩,出门去了。妇人下了帘子,关上大门,又和王
婆吃了一回酒,才散。正是:

倚门相送刘郎去,烟水桃花去路迷。