\chapter{王杏庵义恤贫儿~金道士娈淫少弟}

诗曰:
阶前潜制泪,众里自嫌身。
气味如中酒,情怀似别人。
暖风张乐席,晴日看花尘。
尽是添愁处,深居乞过春。

话说陈敬济,自从西门大姐死了,被吴月娘告了一状,打了一场官司出来,唱的冯
金宝又归院中去了,刚刮剌出个命儿来。房儿也卖了,本钱儿也没了,头面也使了
,家伙也没了。又说陈定在外边打发人,克落了钱,把陈定也撵去了。家中日逐盘
费不周,坐吃山空,不时往杨大郎家中,问他这半船货的下落。一日,来到杨大郎
门首,叫声:“杨大郎在家不在?”不想杨光彦拐了他半船货物,一向在外,卖了
银两,四散躲闪。及打听得他家中吊死了老婆,他丈母县中告他,坐了半个月监,
这杨大郎就蓦地来家住着。听见敬济上门叫他,问货船下落,一径使兄弟杨二风出
来,反问敬济要人:“你把我哥哥叫的外面做买卖,这几个月通无音信,不知抛在
江中,推在河内,害了性命,你倒还来我家寻货船下落?人命要紧,你那货物要紧
?”这杨二风平昔是个刁徒泼皮,耍钱捣子,胳膊上紫肉横生,胸前上黄毛乱长,
是一条直率光棍。走出来一把扯住敬济,就问他要人。那敬济慌忙挣开手跑出回家
来。这杨二风故意拾了块三尖瓦楔,将头颅钻破,血流满面,赶将敬济来,骂道:
”我(入日)你娘娘!我见你家甚么银子来?你来我屋里放屁,吃我一顿好拳头。
”那敬济金命水命,走投无命,奔到家,把大门关闭如铁桶相似,由着杨二风牵爹
娘,骂父母,拿大砖砸门,只是鼻口内不敢出气儿。又况才打了官司出来,梦条绳
蛇也害怕,只得含忍过了。正是:
嫩草怕霜霜怕日,恶人自有恶人磨。

不消几时,把大房卖了,找了七十两银子,典了一所小房,在僻巷内居住。落后两
个丫头,卖了一个重喜儿,只留着元宵儿和他同铺歇。又过了不上半月,把小房倒
腾了,却去赁房居住。陈安也走了,家中没营运,元宵儿也死了,止是单身独自,
家伙桌椅都变卖了,只落得一贫如洗。未几,房钱不给,钻入冷铺内存身。花子见
他是个富家勤儿,生得清俊,叫他在热炕上睡,与他烧饼儿吃。有当夜的过来教他
顶火夫,打梆子摇铃。

那时正值腊月,残冬时分,天降大雪,吊起风来,十分严寒。这工敬济打了回梆子
,打发当夜的兵牌过去,不免手提铃串了几条街巷。又是风雪,地下又踏着那寒冰
,冻得耸肩缩背,战战兢兢。临五更鸡叫,只见个病花子躺在墙底下,恐怕死了,
总甲分付他看守着,寻了把草叫他烤。这敬济支更一夜,没曾睡,就歪下睡着了。
不想做了一梦,梦见那时在西门庆家,怎生受荣华富贵,和潘金莲勾搭,顽耍戏谑
,从睡梦中就哭醒来。众花子说:“你哭怎的?”这敬济便道:“你众位哥哥,我
的苦楚,你怎得知?
频年困苦痛妻亡,身上无衣口绝粮。
马死奴逃房又卖,只身独自在他乡。
朝依肆店求遗馔,暮宿庄园倚败墙。
只有一条身后路,冷铺之中去打梆。”
陈敬济晚夕在冷铺存身,白日间街头乞食。

清河县城内有一老者,姓王名宣,字廷用,年六十余岁,家道殷实,为人心慈,仗
义疏财,专一济贫拔苦,好善敬神。所生二子,皆当家成立。长子王乾,袭祖职为
牧马所掌印正千户;次子王震,充为府学庠生。老者门首搭了个主管,开着个解当
铺儿。每日丰衣足食,闲散无拘,在梵宇听经,琳宫讲道。无事在家门首施药救人
,拈素珠念佛。因后园中有两株杏树,道号为杏庵居士。

一日,杏庵头戴重檐幅巾,身穿水合道服,在门首站立。只见陈敬济打他门首过,
向前扒在地下磕了个头。忙的杏庵还礼不迭,说道:“我的哥,你是谁?老拙眼昏
,不认的你。”这敬济战战兢兢,站立在旁边说道:“不瞒你老人家,小人是卖松
槁陈洪儿子。”老者想了半日,说:“你莫不是陈大宽的令郎么?”因见他衣服褴
褛,形容憔悴,说道:“贤侄,你怎的弄得这般模样?”便问:“你父亲、母亲可
安么?”敬济道:“我爹死在东京,我母亲也死了。”杏庵道:“我闻得你在丈人
家住来?”敬济道:“家外父死了,外母把我撵出来。他女儿死了,告我到官,打
了一场官司。把房儿也卖了,有些本钱儿,都吃人坑了,一向闲着没有营生。”杏
庵道:“贤侄,你如今在那里居住?”敬济半日不言语,说:“不瞒你老人家说,
如此如此。”杏庵道:“可怜,贤侄你原来讨吃哩。想着当初,你府上那样根基人
家。我与你父亲相交,贤侄,你那咱还小哩,才扎着总角上学堂,怎就流落到此地
位?可伤,可伤。你政治家甚亲家?也不看顾你看顾儿。”敬济道:“正是。俺张
舅那里,一向也久不上门,不好去的。”

问了一回话,老者把他让到里面客位里,令小厮放桌儿,摆出点心嗄饭来,教他尽
力吃了一顿。见他身上单寒,拿出一件青布绵道袍儿,一顶毡帽,又一双毡袜、绵
鞋,又秤一两银子,五百铜钱,递与他,分付说:“贤侄,这衣服鞋袜与你身上,
那铜钱与你盘缠,赁半间房儿住;这一两银子,你拿着做上些小买卖儿,也好糊口
过日子,强如在冷铺中,学不出好人来。每月该多少房钱,来这里,老拙与你。”
这陈敬济扒在地下磕头谢了,说道:“小侄知道。”拿着银钱,出离了杏庵门首。
也不寻房子,也不做买卖,把那五百文钱,每日只在酒店面店以了其事。那一两银
子,捣了些白铜顿罐,在街上行使。吃巡逻的当土贼拿到该坊节级处,一顿拶打,
使的罄尽,还落了一屁股疮。不消两日,把身上绵衣也输了,袜儿也换嘴来吃了,
依旧原在街上讨吃。

一日,又打王杏庵门首所过,杏庵正在门首,只见敬济走来磕头,身上衣袜都没了
,止戴着那毡帽,精脚趿鞋,冻的乞乞缩缩。老者便问:“陈大官,做的买卖如何
?房钱到了,来取房钱来了?”那陈敬济半日无言可对。问之再三,方说如此这般
,都没了。老者便道:“阿呀,贤侄,你这等就不是过日子的道理。你又拈不的轻
,负不的重,但做了些小活路儿,不强如乞食,免教人耻笑,有玷你父祖之名。你
如何不依我说?”一面又让到里面,教安童拿饭来与他吃饱了。又与了他一条夹裤
,一领白布衫,一双裹脚,一吊铜钱,一斗米:“你拿去务要做上了小买卖,卖些
柴炭、豆儿、瓜子儿,也过了日子,强似这等讨吃。”这敬济口虽答应,拿钱米在
手,出离了老者门,那消几日,熟食肉面,都在冷铺内和花子打伙儿都吃了。耍钱
,又把白布衫、夹裤都输了。大正月里,又抱着肩儿在街上走,不好来见老者,走
在他门首房山墙底下,向日阳站立。

老者冷眼看见他,不叫他。他挨挨抢抢,又到根前扒在地下磕头。老者见他还依旧
如此,说道:“贤侄,这不是常策。咽喉深似海,日月快如梭,无底坑如何填得起
?你进来,我与你说,有一个去处,又清闲,又安得你身,只怕你不去。”敬济跪
下哭道:“若得老伯见怜,不拘那里,但安下身,小的情愿就去。”杏庵道:“此
去离城不远,临清马头上,有座晏公庙。那里鱼米之乡,舟船辐辏之地,钱粮极广
,清幽潇洒。庙主任道士,与老拙相交极厚,他手下也有两三个徒弟徒孙。我备分
礼物,把你送与他做个徒弟出家,学些经典吹打,与人家应福,也是好处。”敬济
道:“老伯看顾,可知好哩。”杏庵道:“既然如此,你去,明日是个好日子,你
早来,我送你去。”敬济去了。这王老连忙叫了裁缝来,就替敬济做了两件道袍,
一顶道髻,鞋袜俱全。

次日,敬济果然来到。王老教他空屋里洗了澡,梳了头,戴上道髻,里外换了新袄
新裤,上盖表绢道衣,下穿云履毡袜,备了四盘羹果,一坛酒,一匹尺头,封了五
两银子。他便乘马,雇了一匹驴儿与敬济骑着,安童、喜童跟随,两个人担了盒担
,出城门,径往临清马头晏公庙来。止七十里,一日路程。比及到晏公庙,天色已
晚,王老下马,进入庙来。只见青松郁郁,翠柏森森,两边八字红墙,正面三间朱
户,端的好座庙宇。但见:

山门高耸,殿阁棱层。高悬敕额金书,彩画出朝入相。五间大殿,塑龙王一十二尊
;两下长廊,刻水族百千万众。旗竿凌汉,帅字招风。四通八达,春秋社礼享依时
;雨顺风调,河道民间皆祭赛。万年香火威灵在,四境官民仰赖安。

山门下早有小童看见,报入方丈,任道士忙整衣出迎。王杏庵令敬济和礼物且在外
边伺候。不一时,任道士把杏庵让入方丈松鹤轩叙礼,说:“王老居上,怎生一向
不到敝庙随喜?今日何幸,得蒙下顾。”杏庵道:“只因家中俗冗所羁,久失拜望
。”叙礼毕,分宾主而坐,小童献茶。茶罢,任道士道:“老居士,今日天色已晚
,你老人家不去罢了。”分付把马牵入后槽喂息。杏庵道:“没事不登三宝殿。老
拙敬来有一事干渎,未知尊意肯容纳否?”任道士道:“老居士有何见教?只顾分
付,小道无不领命。”杏庵道:“今有故人之子,姓陈,名敬济,年方二十四岁。
生的资格清秀,倒也伶俐。只是父母去世太早,自幼失学。若说他父祖根基,也不
是无名少姓人家,有一分家当,只因不幸遭官事没了,无处栖身。老拙念他乃尊旧
日相交之情,欲送他来贵宫作一徒弟,未知尊意如何?”任道士便道:“老居士分
付,小道怎敢违阻?奈因小道命蹇,手下虽有两三个徒弟,都不省事,没一个成立
的,小道常时惹气,未知此人诚实不诚实?”杏庵道:“这个小的,不瞒尊师说,
只顾放心,一味老实本分,胆儿又小,所事儿伶范,堪可作一徒弟。”任道士问:
”几时送来?”杏庵道:“见在山门外伺候。还有些薄礼,伏乞笑纳。”慌的任道
士道:“老居干何不早说?”一面道:“有请。”于是抬盒人抬进礼物。任道士见
帖儿上写着:“谨具粗段一端,鲁酒一樽,豚蹄一副,烧鸭二只,树果二盒,白金
五两。知生王宣顿首拜。”连忙稽首谢道:“老居士何以见赐许多重礼,使小道却
之不恭,受之有愧。”

只见陈敬济头戴金梁道髻,身穿青绢道衣,脚下云履净袜,腰系丝绦,生的眉清目
秀,齿白唇红,面如傅粉,走进来向任道士倒身下拜,拜了四双八拜。任道士因问
他:“多少青春?”敬济道:“属马,交新春二十四岁了。”任道士见他果然伶俐
,取了他个法名,叫做陈宗美。原来任道士手下有两个徒弟,大徒弟姓金,名宗明
;二徒弟姓徐,名宗顺。他便叫陈宗美。王杏庵都请出来,见了礼数。一面收了礼
物,小童掌上灯来,放卓儿,先摆饭,后吃酒。肴品杯盘,堆满桌上,无非是鸡蹄
鹅鸭鱼肉之类。王老吃不多酒,徒弟轮番劝勾几巡,王老不胜酒力告辞。房中自有
床铺,安歇一宿。

到次日清晨,小童舀水净面,梳洗盥漱毕,任道士又早来递茶。不一时,摆饭,又
吃了两杯酒,喂饱头口,与了抬盒人力钱。王老临起身,叫过敬济来分付:“在此
好生用心习学经典,听师父指教。我常来看你,按季送衣服鞋袜来与你。”又向任
道士说:“他若不听教训,一任责治,老拙并不护短。”一面背地又嘱付敬济:“
我去后,你要洗心改正,习本等事业。你若再不安分,我不管你了。”那敬济应诺
道:“儿子理会了。”王老当下作辞任道士,出门上马,离晏公庙,回家去了。

敬济自此就在晏公庙做了道士。因见任道士年老赤鼻,身体魁伟,声音洪亮,一部
髭髯,能谈善饮,只专迎宾送客。凡一应大小事,都在大徒弟金宗明手里。那时,
朝廷运河初开,临清设二闸,以节水利。不拘官民,船到闸上,都来庙里,或求神
福,或来祭愿,或设卦与笤,或做好事。也有布施钱米的,也有馈送香油纸烛的,
也有留松蒿芦席的。这任道士将常署里多余钱粮,都令家下徒弟在马头上开设钱米
铺,卖将银子来,积攒私囊。

他这大徒弟金宗明,也不是个守本分的。年约三十余岁,常在娼楼包占乐妇,是个
酒色之徒。手下也有两个清洁年少徒弟,同铺歇卧,日久絮繁。因见敬济生的齿白
唇红,面如傅粉,清俊乖觉,眼里说话,就缠他同房居住。晚夕和他吃半夜酒,把
他灌醉了,在一铺歇卧。初时两头睡,便嫌敬济脚臭,叫过一个枕头上睡。睡不多
回,又说他口气喷着,令他吊转身子,屁股贴着肚子。那敬济推睡着,不理他。他
把那话弄得硬硬的,直竖一条棍,抹了些唾津在头上,往他粪门里只一顶。原来敬
济在冷铺里,被花子飞天鬼侯林儿弄过的,眼子大了,那话不觉就进去了。这敬济
口中不言,心内暗道:“这厮合败。他讨得十方便宜多了,把我不知当做甚么人儿
。与他个甜头儿,且教他在我手内纳些钱钞。”一面故意声叫起来。这金宗明恐怕
老道士听见,连忙掩住他口,说:“好兄弟,噤声!随你要的,我都依你。”敬济
道:“你既要勾搭我,我不言语,须依我三件事。”宗明道:“好兄弟,休说三件
,就是十件事,我也依你。”敬济道:“第一件,你既要我,不许你再和那两个徒
弟睡;第二件,大小房门钥匙,我要执掌;第三件,随我往那里去,你休嗔我。你
都依了我,我方依你此事。”金宗明道:“这个不打紧,我都依你。”当夜两个颠
来倒去,整狂了半夜。这陈敬济自幼风月中撞,甚么事不知道。当下被底山盟,枕
边海誓,淫声艳语,抠吮舔品,把这金宗明哄得欢喜无尽。到第二日,果然把各处
钥匙都交与他手内,就不和那两个徒弟在一处,每日只同他一铺歇卧。

一日两,两日三,这金宗明便再三称赞他老实。任道士听信,又替他使钱讨了一张
度牒。自此以后,凡事并不防范。这陈敬济因此常拿着银钱往马头上游玩,看见院
中架儿陈三儿说:“冯金宝儿他鸨子死了,他又卖在郑家,叫郑金宝儿。如今又在
大酒楼上赶趁哩,你不看他看去?”这小伙儿旧情不改,拿着银钱,跟定陈三儿,
径往马头大酒楼上来。此不来倒好,若来,正是:五百载冤家来聚会,数年前姻眷
又相逢。有诗为证:
人生莫惜金缕衣,人生莫负少年时。
有花欲折须当折,莫待无花空折枝。

原来这座酒楼乃是临清第一座酒楼,名唤谢家酒楼。里面有百十座阁儿,周围都是
绿栏杆,就紧靠着山冈,前临官河,极是人烟闹热去处,舟船往来之所。怎见得这
座酒楼齐整?但见:

雕檐映日,面栋飞云。绿栏杆低接轩窗,翠帘栊高悬户牖。吹笙品笛
,尽都是公子王孙;执盏擎杯,摆列着歌妪舞女。消磨醉眼,依青天
万叠云山;勾惹吟魂,翻瑞雪一河烟水。楼畔绿杨啼野鸟,门前翠柳
系花骢。

这陈三儿引敬济上楼,到一个阁儿里坐下。便叫店小二打抹春台,安排一分上品酒
果下饭来摆着,使他下边叫粉头去了。须臾,只见楼梯响,冯金宝上来,手中拿着
个厮锣儿,见了敬济,深深道了万福。常言情人见情人,不觉簇地两行泪下。正是
:
数声娇语如莺啭,一串珍珠落线买。

敬济一见,便拉他一处坐,问道:“姐姐,你一向在那里来?不见你。”这冯金宝
收泪道:“自从县中打断出来,我妈着了惊谎,不久得病死了,把我卖在郑五妈家
。这两日子弟稀少,不免又来在临清马头上赶趁酒客。昨日听见陈三儿说你在这里
开钱铺,要见你一见。不期今日会见一面。可不想杀我也!”说毕,又哭了。敬济
取出袖中帕儿,替他抹了眼泪,说道:“我的姐姐,你休烦恼。我如今又好了,自
从打出官司来,家业都没了,投在这晏公庙,做了道士。师父甚是托我,往后我常
来看你。”因问:“你如今在那里安下?”金宝便道:“奴就在这桥西洒家店刘二
那里。有百十房子,四外行院窠子,妓女都在那里安下,白日里便是这各酒楼赶趁
。”说着,两个挨身做一处饮酒。陈三儿烫酒上楼,拿过琵琶来。金宝弹唱了个曲
儿与敬济下酒,名《普天乐》:

泪双垂,垂双泪。三杯别酒,别酒三杯。鸾凤对拆开,折开鸾凤对。
岭外斜晖看看坠,看看坠,岭外晖。天昏地暗,徘徊不舍,不舍徘徊
。

两人吃得酒浓时,朱免解衣云雨,下个房儿。这陈敬济一向不曾近妇女,久渴的人
,今得遇金宝,尽力盘桓,尤云殢雨,未肯即休。须臾事毕,各整衣衫。敬济见天
色晚了,与金宝作别,与了金宝一两银子,与了陈三儿百文铜钱,嘱付:“姐姐,
我常来看你,咱在这搭儿里相会。你若想我,使陈三儿叫我去。”下楼来,又打发
了店主人谢三郎三钱银子酒钱。敬济回庙中去了。冯金宝送至桥边方回。正是:
盼穿秋水因钱钞,哭损花容为邓通。