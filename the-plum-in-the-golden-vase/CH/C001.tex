\chapter{西门庆热结十弟兄~武二郎冷遇亲哥嫂}

诗曰:

豪华去后行人绝,箫筝不响歌喉咽。
雄剑无威光彩沉,宝琴零落金星灭。
玉阶寂寞坠秋露,月照当时歌舞处。
当时歌舞人不回,化为今日西陵灰。

又诗曰:

二八佳人体似酥,腰间仗剑斩愚夫。
虽然不见人头落,暗里教君骨髓枯。

这一首诗,是昔年大唐国时,一个修真炼性的英雄,入圣超凡的豪杰,到后来
位居紫府,名列仙班,率领上八洞群仙,救拔四部洲沉苦一位仙长,姓吕名岩,道
号纯阳子祖师所作。单道世上人,营营逐逐,急急巴巴,跳不出七情六欲关头,打
不破酒色财气圈子。到头来同归于尽,着甚要紧!虽是如此说,只这酒色财气四件
中,惟有“财色”二者更为利害。怎见得他的利害?假如一个人到了那穷苦的田地
,受尽无限凄凉,耐尽无端懊恼,晚来摸一摸米瓮,苦无隔宿之炊,早起看一看厨
前,愧无半星烟火,妻子饥寒,一身冻馁,就是那粥饭尚且艰难,那讨馀钱沽酒!
更有一种可恨处,亲朋白眼,面目寒酸,便是凌云志气,分外消磨,怎能够与人争
气!正是:

一朝马死黄金尽,亲者如同陌路人。

到得那有钱时节,挥金买笑,一掷巨万。思饮酒真个琼浆玉液,不数那琥珀杯流;
要斗气钱可通神,果然是颐指气使。趋炎的压脊挨肩,附势的吮痈舐痔,真所谓得
势叠肩而来,失势掉臂而去。古今炎冷恶态,莫有甚于此者。这两等人,岂不是受
那财的利害处!如今再说那色的利害。请看如今世界,你说那坐怀不乱的柳下惠,
闭门不纳的鲁男子,与那秉烛达旦的关云长,古今能有几人?至如三妻四妾,买笑
追欢的,又当别论。还有那一种好色的人,见了个妇女略有几分颜色,便百计千方
偷寒送暖,一到了着手时节,只图那一瞬欢娱,也全不顾亲戚的名分,也不想朋友
的交情。起初时不知用了多少滥钱,费了几遭酒食。正是:

三杯花作合,两盏色媒人。

到后来情浓事露,甚而斗狠杀伤,性命不保,妻孥难顾,事业成灰。就如那石季伦
泼天豪富,为绿珠命丧囹圄;楚霸王气概拔山,因虞姬头悬垓下。真所谓:“生我
之门死我户,看得破时忍不过”。这样人岂不是受那色的利害处!

说便如此说,这“财色”二字,从来只没有看得破的。若有那看得破的,便见
得堆金积玉,是棺材内带不去的瓦砾泥沙;贯朽粟红,是皮囊内装不尽的臭淤粪土
。高堂广厦,玉宇琼楼,是坟山上起不得的享堂;锦衣绣袄,狐服貂裘,是骷髅上
裹不了的败絮。即如那妖姬艳女,献媚工妍,看得破的,却如交锋阵上将军叱咤献
威风;朱唇皓齿,掩袖回眸,懂得来时,便是阎罗殿前鬼判夜叉增恶态。罗袜一弯
,金莲三寸,是砌坟时破土的锹锄;枕上绸缪,被中恩爱,是五殿下油锅中生活。
只有那《金刚经》上两句说得好,他说道:“如梦幻泡影,如电复如露。”见得人
生在世,一件也少不得,到了那结束时,一件也用不着。随着你举鼎荡舟的神力,
到头来少不得骨软筋麻;由着你铜山金谷的奢华,正好时却又要冰消雪散。假饶你
闭月羞花的容貌,一到了垂眉落眼,人皆掩鼻而过之;比如你陆贾隋何的机锋,若
遇着齿冷唇寒,吾未如之何也已。到不如削去六根清净,披上一领袈裟,参透了空
色世界,打磨穿生灭机关,直超无上乘,不落是非窠,倒得个清闲自在,不向火坑
中翻筋斗也。正是:

三寸气在千般用,一日无常万事休。

说话的为何说此一段酒色财气的缘故?只为当时有一个人家,先前恁地富贵,
到后来煞甚凄凉,权谋术智,一毫也用不着,亲友兄弟,一个也靠不着,享不过几
年的荣华,倒做了许多的话靶。内中又有几个斗宠争强,迎奸卖俏的,起先好不妖
娆妩媚,到后来也免不得尸横灯影,血染空房。正是:

善有善报,恶有恶报;
天网恢恢,疏而不漏。

话说大宋徽宗皇帝政和年间,山东省东平府清河县中,有一个风流子弟,生得
状貌魁梧,性情潇洒,饶有几贯家资,年纪二十六七。这人复姓西门,单讳一个庆
字。他父亲西门达,原走川广贩药材,就在这清河县前开着一个大大的生药铺。现
住着门面五间到底七进的房子。家中呼奴使婢,骡马成群,虽算不得十分富贵,却
也是清河县中一个殷实的人家。只为这西门达员外夫妇去世的早,单生这个儿子却
又百般爱惜,听其所为,所以这人不甚读书,终日闲游浪荡。一自父母亡后,专一
在外眠花宿柳,惹草招风,学得些好拳棒,又会赌博,双陆象棋,抹牌道字,无不
通晓。结识的朋友,也都是些帮闲抹嘴,不守本分的人。第一个最相契的,姓应名
伯爵,表字光侯,原是开绸缎铺应员外的第二个儿子,落了本钱,跌落下来,专在
本司三院帮嫖贴食,因此人都起他一个浑名叫做应花子。又会一腿好气毬,
双陆棋子,件件皆通。第二个姓谢名希大,字子纯,乃清河卫千户官儿应袭子孙,
自幼父母双亡,游手好闲,把前程丢了,亦是帮闲勤儿,会一手好琵琶。自这两个
与西门庆甚合得来。其余还有几个,都是些破落户,没名器的。一个叫做祝实念,
表字贡诚。一个叫做孙天化,表字伯修,绰号孙寡嘴。一个叫做吴典恩,乃是本县
阴阳生,因事革退,专一在县前与官吏保债,以此与西门庆往来。还有一个云参将
的兄弟叫做云理守,字非去。一个叫做常峙节,表字坚初。一个叫做卜志道。一个
叫做白赉光,表字光汤。说这白赉光,众人中也有道他名字取的不好听的,他却自
己解说道:“不然我也改了,只为当初取名的时节,原是一个门馆先生,说我姓白
,当初有一个什么故事,是白鱼跃入武王舟。又说有两句书是‘周有大赉,于汤有
光’,取这个意思,所以表字就叫做光汤。我因他有这段故事,也便不改了。”说
这一干共十数人,见西门庆手里有钱,又撒漫肯使,所以都乱撮哄着他耍钱饮酒,
嫖赌齐行。正是:

把盏衔杯意气深,兄兄弟弟抑何亲。
一朝平地风波起,此际相交才见心。

说话的,这等一个人家,生出这等一个不肖的儿子,又搭了这等一班无益有损
的朋友,随你怎的豪富也要穷了,还有甚长进的日子!却有一个缘故,只为这西门
庆生来秉性刚强,作事机深诡谲,又放官吏债,就是那朝中高、杨、童、蔡四大奸
臣,他也有门路与他浸润。所以专在县里管些公事,与人把搅说事过钱,因此满县
人都惧怕他。因他排行第一,人都叫他是西门大官人。这西门大官人先头浑家陈氏
早逝,身边只生得一个女儿,叫做西门大姐,就许与东京八十万禁军杨提督的亲家
陈洪的儿子陈敬济为室,尚未过门。只为亡了浑家,无人管理家务,新近又娶了本
县清河左卫吴千户之女填房为继室。这吴氏年纪二十五六,是八月十五生的,小名
叫做月姐,后来嫁到西门庆家,都顺口叫他月娘。却说这月娘秉性贤能,夫主面上
百依百随。房中也有三四个丫鬟妇女,都是西门庆收用过的。又尝与勾栏内李娇儿
打热,也娶在家里做了第二房娘子。南街又占着窠子卓二姐,名卓丢儿,包了些时
,也娶来家做了第三房。只为卓二姐身子瘦怯,时常三病四痛,他却又去飘风戏月
,调弄人家妇女。正是:

东家歌笑醉红颜,又向西邻开玳宴。
几日碧桃花下卧,牡丹开处总堪怜。

话说西门庆一日在家闲坐,对吴月娘说道:“如今是九月廿五日了,出月初三
日,却是我兄弟们的会期。到那日也少不的要整两席齐整的酒席,叫两个唱的姐儿
,自恁在咱家与兄弟们好生玩耍一日。你与我料理料理。”吴月娘便道:“你也便
别要说起这干人,那一个是那有良心和行货!无过每日来勾使的游魂撞尸。我看你
自搭了这起人,几时曾有个家哩!现今卓二姐自恁不好,我劝你把那酒也少要吃了
。”西门庆道:“你别的话倒也中听。今日这些说话,我却有些不耐烦听他。依你
说,这些兄弟们没有好人,使着他,没有一个不依顺的,做事又十分停当,就是那
谢子纯这个人,也不失为个伶俐能事的好人。咱如今是这等计较罢,只管恁会来会
去,终不着个切实。咱不如到了会期,都结拜了兄弟罢,明日也有个靠傍些。”吴
月娘接过来道:“结拜兄弟也好。只怕后日还是别个靠你的多哩。若要你去靠人,
提傀儡儿上戏场──还少一口气儿哩。”西门庆笑道:“自恁长把人靠得着,却不
更好了。咱只等应二哥来,与他说这话罢。”

正说着话,只见一个小厮儿,生得眉清目秀,伶俐乖觉,原是西门庆贴身伏侍
的,唤名玳安儿,走到面前来说:“应二叔和谢大叔在外见爹说话哩。”西门庆道
:“我正说他,他却两个就来了。”一面走到厅上来,只见应伯爵头上戴一顶新盔
的玄罗帽儿,身上穿一件半新不旧的天青夹绉纱褶子,脚下丝鞋净袜,坐在上首。
下首坐的,便是姓谢的谢希大。见西门庆出来,一齐立起身来,边忙作揖道:“哥
在家,连日少看。”西门庆让他坐下,一面唤茶来吃,说道:“你们好人儿,这几
日我心里不耐烦,不出来走跳,你们通不来傍个影儿。”伯爵向希大道:“何如?
我说哥哥要说哩。”因对西门庆道:“哥,你怪的是。连咱自也不知道成日忙些什
么!自咱们这两只脚,还赶不上一张嘴哩。”西门庆因问道:“你这两日在那里来
?”伯爵道:“昨日在院中李家瞧了个孩子儿,就是哥这边二嫂子的侄女儿桂卿的
妹子,叫做桂姐儿。几时儿不见他,就出落的好不标致了。到明日成人的时候,还
不知怎的样好哩!昨日他妈再三向我说:‘二爹,千万寻个好子弟梳笼他。’敢怕
明日还是哥的货儿哩。”西门庆道:“有这等事!等咱空闲了去瞧瞧。”谢希大接
过来道:“哥不信,委的生得十分颜色。”西门庆道:“昨日便在他家,前几日却
在那里去来?”伯爵道:“便是前日卜志道兄弟死了,咱在他家帮着乱了几日,发
送他出门。他嫂子再三向我说,叫我拜上哥,承哥这里送了香楮奠礼去,因他没有
宽转地方儿,晚夕又没甚好酒席,不好请哥坐的,甚是过不意去。”西门庆道:“
便是我闻得他不好得没多日子,就这等死了。我前日承他送我一把真金川扇儿,我
正要拿甚答谢答谢,不想他又作了故人!”

谢希大便叹了一口气道:“咱会中兄弟十人,却又少他一个了。”因向伯爵说
:“出月初三日,又是会期,咱每少不得又要烦大官人这里破费,兄弟们顽耍一日
哩。”西门庆便道:“正是,我刚才正对房下说来,咱兄弟们似这等会来会去,无
过只是吃酒顽耍,不着一个切实,倒不如寻一个寺院里,写上一个疏头,结拜做了
兄弟,到后日彼此扶持,有个傍靠。到那日,咱少不得要破些银子,买办三牲,众
兄弟也便随多少各出些分资。不是我科派你们,这结拜的事,各人出些,也见些情
分。”伯爵连忙道:“哥说的是。婆儿烧香当不的老子念佛,各自要尽自的心。只
是俺众人们,老鼠尾巴生疮儿──有脓也不多。”西门庆笑道:“怪狗才,谁要你
多来!你说这话。”谢希大道:“结拜须得十个方好。如今卜志道兄弟没了,却教
谁补?”西门庆沉吟了一回,说道:“咱这间壁花二哥,原是花太监侄儿,手里肯
使一股滥钱,常在院中走动。他家后边院子与咱家只隔着一层壁儿,与我甚说得来
,咱不如叫小厮邀他邀去。”应伯爵拍着手道:“敢就是在院中包着吴银儿的花子
虚么?”西门庆道:“正是他!”伯爵笑道:“哥,快叫那个大官儿邀他去。与他
往来了,咱到日后,敢又有一个酒碗儿。”西门庆笑道:“傻花子,你敢害馋痨痞
哩,说着的是吃。”大家笑了一回。西门庆旋叫过玳安儿来说:“你到间壁花家去
,对你花二爹说,如此这般:‘俺爹到了出月初三日,要结拜十兄弟,敢叫我请二
爹上会哩。’看他怎的说,你就来回我话。你二爹若不在家,就对他二娘说罢。”
玳安儿应诺去了。伯爵便道:“到那日还在哥这里是,还在寺院里好?”希大道:
“咱这里无过只两个寺院,僧家便是永福寺,道家便是玉皇庙。这两个去处,随分
那里去罢。”西门庆道:“这结拜的事,不是僧家管的,那寺里和尚,我又不熟,
倒不如玉皇庙吴道官与我相熟,他那里又宽展又幽静。”伯爵接过来道:“哥说的
是,敢是永福寺和尚倒和谢家嫂子相好,故要荐与他去的。”希大笑骂道:“老花
子,一件正事,说说就放出屁来了。”

正说笑间,只见玳安儿转来了,因对西门庆说道:“他二爹不在家,俺对他二
娘说来。二娘听了,好不欢喜,说道:‘既是你西门爹携带你二爹做兄弟,那有个
不来的。等来家我与他说,至期以定撺掇他来,多拜上爹。’又与了小的两件茶食
来了。”西门庆对应、谢二人道:“自这花二哥,倒好个伶俐标致娘子儿。”说毕
,又拿一盏茶吃了,二人一齐起身道:“哥,别了罢,咱好去通知众兄弟,纠他分
资来。哥这里先去与吴道官说声。”西门庆道:“我知道了,我也不留你罢。”于
是一齐送出大门来。应伯爵走了几步,回转来道:“那日可要叫唱的?”西门庆道
:“这也罢了,弟兄们说说笑笑,到有趣些。”说毕,伯爵举手,和希大一路去了
。

话休饶舌,捻指过了四五日,却是十月初一日。西门庆早起,刚在月娘房里坐
的,只见一个才留头的小厮儿,手里拿着个描金退光拜匣,走将进来,向西门庆磕
了一个头儿,立起来站在旁边说道:“俺是花家,俺爹多拜上西门爹。那日西门爹
这边叫大官儿请俺爹去,俺爹有事出门了,不曾当面领教的。闻得爹这边是初三日
上会,俺爹特使小的先送这些分资来,说爹这边胡乱先用着,等明日爹这里用过多
少派开,该俺爹多少,再补过来便了。”西门庆拿起封袋一看,签上写着“分资一
两”,便道:“多了,不消补的。到后日叫爹莫往那去,起早就要同众爹上庙去。
”那小厮儿应道:“小的知道。”刚待转身,被吴月娘唤住,叫大丫头玉箫在食箩
里拣了两件蒸酥果馅儿与他。因说道:“这是与你当茶的。你到家拜上你家娘,你
说西门大娘说,迟几日还要请娘过去坐半日儿哩。”那小厮接了,又磕了一个头儿
,应着去了。

西门庆才打发花家小厮出门,只见应伯爵家应宝夹着个拜匣,玳安儿引他进来
见了,磕了头,说道:“俺爹纠了众爹们分资,叫小的送来,爹请收了。”西门庆
取出来看,共总八封,也不拆看,都交与月娘,道:“你收了,到明日上庙,好凑
着买东西。”说毕,打发应宝去了。立起身到那边看卓二姐。刚走到坐下,只见玉
箫走来,说道:“娘请爹说话哩。”西门庆道:“怎的起先不说来?”随即又到上
房,看见月娘摊着些纸包在面前,指着笑道:“你看这些分子,止有应二的是一钱
二分八成银子,其余也有三分的,也有五分的,都是些红的黄的,倒象金子一般。
咱家也曾没见这银子来,收他的也污个名,不如掠还他罢。”西门庆道:“你也耐
烦,丢着罢,咱多的也包补,在乎这些!”说着一直往前去了。

到了次日初二日,西门庆称出四两银子,叫家人来兴儿买了一口猪、一口羊、
五六坛金华酒和香烛纸札、鸡鸭案酒之物,又封了五钱银子,旋叫了大家人来保和
玳安儿、来兴三个:“送到玉皇庙去,对你吴师父说:‘俺爹明日结拜兄弟,要劳
师父做纸疏辞,晚夕就在师父这里散福。烦师父与俺爹预备预备,俺爹明早便来。
’”只见玳安儿去了一会,来回说:“已送去了,吴师父说知道了。”

须臾,过了初二,次日初三早,西门庆起来梳洗毕,叫玳安儿:“你去请花二
爹,到咱这里吃早饭,一同好上庙去。一发到应二叔家,叫他催催众人。”玳安应
诺去,刚请花子虚到来,只见应伯爵和一班兄弟也来了,却正是前头所说的这几个
人。为头的便是应伯爵,谢希大、孙天化、祝念实、吴典恩、云理守、常峙节、白
赉光,连西门庆、花子虚共成十个。进门来一齐箩圈作了一个揖。伯爵道:“咱时
候好去了。”西门庆道:“也等吃了早饭着。”便叫:“拿茶来。”一面叫:“看
菜儿。”须臾,吃毕早饭,西门庆换了一身衣服,打选衣帽光鲜,一齐径往玉皇庙
来。

不到数里之遥,早望见那座庙门,造得甚是雄峻。但见:

殿宇嵯峨,宫墙高耸。正面前起着一座墙门八字,一带都粉赭色红泥
;进里边列着三条甬道川纹,四方都砌水痕白石。正殿上金碧辉煌,两廊
下檐阿峻峭。三清圣祖庄严宝相列中央,太上老君背倚青牛居后殿。

进入第二重殿后,转过一重侧门,却是吴道官的道院。进的门来,两下都是些瑶草
琪花,苍松翠竹。西门庆抬头一看,只见两边门楹上贴着一副对联道:

洞府无穷岁月,
壶天别有乾坤。

上面三间敞厅,却是吴道官朝夕做作功课的所在。当日铺设甚是齐整,上面挂的是
昊天金阙玉皇上帝,两边列着的紫府星官,侧首挂着便是马、赵、温、关四大元帅
。当下吴道官却又在经堂外躬身迎接。西门庆一起人进入里边,献茶已罢,众人都
起身,四围观看。白赉光携着常峙节手儿,从左边看将过来,一到马元帅面前,见
这元帅威风凛凛,相貌堂堂,面上画着三只眼睛,便叫常峙节道:“哥,这却是怎
的说?如今世界,开只眼闭只眼儿便好,还经得多出只眼睛看人破绽哩!”应伯爵
听见,走过来道:“呆兄弟,他多只眼儿看你倒不好么?”众人笑了。常峙节便指
着下首温元帅道:“二哥,这个通身蓝的,却也古怪,敢怕是卢杞的祖宗。”伯爵
笑着猛叫道:“吴先生你过来,我与你说个笑话儿。”那吴道官真个走过来听他。
伯爵道:“一个道家死去,见了阎王,阎王问道:‘你是什么人?’道者说:‘是
道士。’阎王叫判官查他,果系道士,且无罪孽。这等放他还魂。只见道士转来,
路上遇着一个染房中的博士,原认得的,那博士问道:‘师父,怎生得转来?’道
者说:‘我是道士,所以放我转来。’那博士记了,见阎王时也说是道士。那阎王
叫查他身上,只见伸出两只手来是蓝的,问其何故。那博士打着宣科的声音道:‘
曾与温元帅搔胞。’”说的众人大笑。一面又转过右首来,见下首供着个红脸的却
是关帝。上首又是一个黑面的是赵元坛元帅,身边画着一个大老虎。白赉光指着道
:“哥,你看这老虎,难道是吃素的,随着人不妨事么?”伯爵笑道:“你不知,
这老虎是他一个亲随的伴当儿哩。”谢希大听得走过来,伸出舌头道:“这等一个
伴当随着,我一刻也成不的。我不怕他要吃我么?”伯爵笑着向西门庆道:“这等
亏他怎地过来!”西门庆道:“却怎的说?”伯爵道:“子纯一个要吃他的伴当随
不的,似我们这等七八个要吃你的随你,却不吓死了你罢了。”说着,一齐正大笑
时,吴道官走过来,说道:“官人们讲这老虎,只俺这清河县,这两日好不受这老
虎的亏!往来的人也不知吃了多少,就是猎户,也害死了十来人。”西门庆问道:
“是怎的来?”吴道官道:“官人们还不知道。不然我也不晓的,只因日前一个小
徒,到沧州横海郡柴大官人那里去化些钱粮,整整住了五七日,才得过来。俺这清
河县近着沧州路上,有一条景阳冈,冈上新近出了一个吊睛白额老虎,时常出来吃
人。客商过往,好生难走,必须要成群结伙而过。如今县里现出着五十两赏钱,要
拿他,白拿不得。可怜这些猎户,不知吃了多少限棒哩!”白赉光跳起来道:“咱
今日结拜了,明日就去拿他,也得些银子使。”西门庆道:“你性命不值钱么?”
白赉光笑道:“有了银子,要性命怎的!”众人齐笑起来。应伯爵道:“我再说个
笑话你们听:一个人被虎衔了,他儿子要救他,拿刀去杀那虎。这人在虎口里叫道
:‘儿子,你省可而的砍,怕砍坏了虎皮。’”说着众人哈哈大笑。

只见吴道官打点牲礼停当,来说道:“官人们烧纸罢。”一面取出疏纸来,说
:“疏已写了,只是那位居长?那位居次?排列了,好等小道书写尊讳。”众人一
齐道:“这自然是西门大官人居长。”西门庆道:“这还是叙齿,应二哥大如我,
是应二哥居长。”伯爵伸着舌头道:“爷,可不折杀小人罢了!如今年时,只好叙
些财势,那里好叙齿!若叙齿,这还有大如我的哩。且是我做大哥,有两件不妥:
第一不如大官人有威有德,众兄弟都服你;第二我原叫做应二哥,如今居长,却又
要叫应大哥,倘或有两个人来,一个叫‘应二哥’,一个叫‘应大哥’,我还是应
‘应二哥’,应‘应大哥’呢?”西门庆笑道:“你这搊断肠子的,单有这
些闲说的!”谢希大道:“哥,休推了。”西门庆再三谦让,被花子虚、应伯爵等
一干人逼勒不过,只得做了大哥。第二便是应伯爵,第三谢希大,第四让花子虚有
钱做了四哥。其余挨次排列。吴道官写完疏纸,于是点起香烛,众人依次排列。吴
道官伸开疏纸朗声读道:

维大宋国山东东平府清河县信士西门庆、应伯爵、谢希大、花子虚、
孙天化、祝念实、云理守、吴典恩、常峙节、白赉光等,是日沐手焚香请
旨。伏为桃园义重,众心仰慕而敢效其风;管鲍情深,各姓追维而欲同其
志。况四海皆可兄弟,岂异姓不如骨肉?是以涓今政和年月日,营备猪羊
牲礼,鸾驭金资,瑞叩斋坛,虔诚请祷,拜投昊天金阙玉皇上帝,五方值
日功曹,本县城隍社令,过往一切神祇,仗此真香,普同鉴察。伏
念庆等生虽异日,死冀同时,期盟言之永固;安乐与共,颠沛相扶,思缔
结以常新。必富贵常念贫穷,乃始终有所依倚。情共日往以月来,谊若天
高而地厚。伏愿自盟以后,相好无尤,更祈人人增有永之年,户户庆无疆
之福。凡在时中,全叨覆庇,谨疏。
政和  年  月  日文疏

吴道官读毕,众人拜神已罢,依次又在神前交拜了八拜。然后送神,焚化钱纸,收
下福礼去。不一时,吴道官又早叫人把猪羊卸开,鸡鱼果品之类整理停当,俱是大
碗大盘摆下两桌,西门庆居于首席,其余依次而坐,吴道官侧席相陪。须臾,酒过
数巡,众人猜枚行令,耍笑哄堂,不必细说。正是:

才见扶桑日出,又看曦驭衔山。
醉后倩人扶去,树梢新月弯弯。

饮酒热闹间,只见玳安儿来附西门庆耳边说道:“娘叫小的接爹来了,说三娘
今日发昏哩,请爹早些家去。”西门庆随即立起来说道:“不是我摇席破座,委的
我第三个小妾十分病重,咱先去休。”只见花子虚道:“咱与哥同路,咱两个一搭
儿去罢。”伯爵道:“你两个财主的都去了,丢下俺们怎的!花二哥你再坐回去。
”西门庆道:“他家无人,俺两个一搭里去的是,省和他嫂子疑心。”玳安儿道:
“小的来时,二娘也叫天福儿备马来了。”只见一个小厮走近前,向子虚道:“马
在这里,娘请爹家去哩。”于是二人一齐起身,向吴道官致谢打搅,与伯爵等举手
道:“你们自在耍耍,我们去也。”说着出门上马去了。单留下这几个嚼倒泰山不
谢土的,在庙流连痛饮不题。

却表西门庆到家,与花子虚别了进来,问吴月娘:“卓二姐怎的发昏来?”月
娘道:“我说一个病人在家,恐怕你搭了这起人又缠到那里去了,故此叫玳安儿恁
地说。只是一日日觉得重来,你也要在家看他的是。”西门庆听了,往那边去看,
连日在家守着不题。

却说光阴过隙,又早是十月初十外了。一日,西门庆正使小厮请太医诊视卓二
姐病症,刚走到厅上,只见应伯爵笑嘻嘻走将进来。西门庆与他作了揖,让他坐了
。伯爵道:“哥,嫂子病体如何?”西门庆道:“多分有些不起解,不知怎的好。
”因问:“你们前日多咱时分才散?”伯爵道:“承吴道官再三苦留,散时也有二
更多天气。咱醉的要不的,倒是哥早早来家的便益些。”西门庆因问道:“你吃了
饭不曾?”伯爵不好说不曾吃,因说道:“哥,你试猜。”西门庆道:“你敢是吃
了?”伯爵掩口道:“这等猜不着。”西门庆笑道:“怪狗才,不吃便说不曾吃,
有这等张致的!”一面叫小厮:“看饭来,咱与二叔吃。”伯爵笑道:“不然咱也
吃了来了,咱听得一件稀罕的事儿,来与哥说,要同哥去瞧瞧。”西门庆道:“甚
么稀罕的?”伯爵道:“就是前日吴道官所说的景阳冈上那只大虫,昨日被一个人
一顿拳头打死了。”西门庆道:“你又来胡说了,咱不信。”伯爵道:“哥,说也
不信,你听着,等我细说。”于是手舞足蹈说道:“这个人有名有姓,姓武名松,
排行第二。”先前怎的避难在柴大官人庄上,后来怎的害起病来,病好了又怎的要
去寻他哥哥,过这景阳冈来,怎的遇了这虎,怎的怎的被他一顿拳脚打死了。一五
一十说来,就象是亲见的一般,又象这只猛虎是他打的一般。说毕,西门庆摇着头
儿道:“既恁的,咱与你吃了饭同去看来。”伯爵道:“哥,不吃罢,怕误过了。
咱们倒不如大街上酒楼上去坐罢。”只见来兴儿来放桌儿,西门庆道:“对你娘说
,叫别要看饭了,拿衣服来我穿。”

须臾,换了衣服,与伯爵手拉着手儿同步出来。路上撞着谢希大,笑道:“哥
们,敢是来看打虎的么?”西门庆道:“正是。”谢希大道:“大街上好挨挤不开
哩。”于是一同到临街一个大酒楼上坐下。不一时,只听得锣鸣鼓响,众人都一齐
瞧看。只见一对对缨枪的猎户,摆将过来,后面便是那打死的老虎,好象锦布袋一
般,四个人还抬不动。末后一匹大白马上,坐着一个壮士,就是那打虎的这个人。
西门庆看了,咬着指头道:“你说这等一个人,若没有千百斤水牛般气力,怎能够
动他一动儿。”这里三个儿饮酒评品,按下不题。

单表迎来的这个壮士怎生模样?但见:

雄躯凛凛,七尺以上身材;阔面棱棱,二十四五年纪。双目直竖,远
望处犹如两点明星;两手握来,近觑时好似一双铁碓。脚尖飞起,深山虎
豹失精魂;拳手落时,穷谷熊罴皆丧魄。头戴着一顶万字头巾,上簪两朵
银花;身穿着一领血腥衲袄,披着一方红锦。

这人不是别人,就是应伯爵说所阳谷县的武二郎。只为要来寻他哥子,不意中打死
了这个猛虎,被知县迎请将来。众人看着他迎入县里。却说这时正值知县升堂,武
松下马进去,扛着大虫在厅前。知县看了武松这般模样,心中自忖道:“不恁地,
怎打得这个猛虎!”便唤武松上厅。参见毕,将打虎首尾诉说一遍。两边官吏都吓
呆了。知县在厅上赐了三杯酒,将库中众土户出纳的赏钱五十两,赐与武松。武松
禀道:“小人托赖相公福荫,偶然侥幸打死了这个大虫,非小人之能,如何敢受这
些赏赐!众猎户因这畜生,受了相公许多责罚,何不就把赏给散与众人,也显得相
公恩典。”知县道:“既是如此,任从壮士处分。”武松就把这五十两赏钱,在厅
上散与众猎户傅去了。知县见他仁德忠厚,又是一条好汉,有心要抬举他,便道:
“你虽是阳谷县人氏,与我这清河县只在咫尺。我今日就参你在我县里做个巡捕的
都头,专在河东水西擒拿贼盗,你意下如何?”武松跪谢道:“若蒙恩相抬举,小
人终身受赐。”知县随即唤押司立了文案,当日便参武松做了巡捕都头。众里长大
户都来与武松作贺庆喜,连连吃了数日酒。正要回阳谷县去抓寻哥哥,不料又在清
河县做了都头,却也欢喜。那时传得东平一府两县,皆知武松之名。正是:

壮士英雄艺略芳,挺身直上景阳冈。
醉来打死山中虎,自此声名播四方。

却说武松一日在街上闲行,只听背后一个人叫道:“兄弟,知县相公抬举你做
了巡捕都头,怎不看顾我!”武松回头见了这人,不觉的──

欣从额角眉边出,喜逐欢容笑口开。

这人不是别人,却是武松日常间要去寻他的嫡亲哥哥武大。却说武大自从兄弟分别
之后,因时遭饥馑,搬移在清河县紫石街赁房居住。人见他为人懦弱,模样猥蕤,
起了他个浑名叫做三寸丁谷树皮,俗语言其身上粗糙,头脸窄狭故也。只因他这般
软弱朴实,多欺侮也。这也不在话下。且说武大无甚生意,终日挑担子出去街上卖
炊饼度日,不幸把浑家故了,丢下个女孩儿,年方十二岁,名唤迎儿,爷儿两个过
活。那消半年光景,又消折了资本,移在大街坊张大户家临街房居住。张宅家下人
见他本分,常看顾他,照顾他依旧卖些炊饼。闲时在铺中坐地,武大无不奉承。因
此张宅家下人个个都欢喜,在大户面前一力与他说方便。因此大户连房钱也不问武
大要。

却说这张大户有万贯家财,百间房屋,年约六旬之上,身边寸男尺女皆无。妈
妈余氏,主家严厉,房中并无清秀使女。只因大户时常拍胸叹气道:“我许大年纪
,又无儿女,虽有几贯家财,终何大用。”妈妈道:“既然如此说,我叫媒人替你
买两个使女,早晚习学弹唱,服侍你便了。”大户听了大喜,谢了妈妈。过了几时
,妈妈果然叫媒人来,与大户买了两个使女,一个叫做潘金莲,一个唤做白玉莲。
玉莲年方二八,乐户人家出身,生得白净小巧。这潘金莲却是南门外潘裁的女儿,
排行六姐。因他自幼生得有些姿色,缠得一双好小脚儿,所以就叫金莲。他父亲死
了,做娘的度日不过,从九岁卖在王招宣府里,习学弹唱,闲常又教他读书写字。
他本性机变伶俐,不过十二三,就会描眉画眼,傅粉施朱,品竹弹丝,女工针指,
知书识字,梳一个缠髻儿,着一件扣身衫子,做张做致,乔模乔样。到十五岁的时
节,王招宣死了,潘妈妈争将出来,三十两银子转卖于张大户家,与玉莲同时进门
。大户教他习学弹唱,金莲原自会的,甚是省力。金莲学琵琶,玉莲学筝,这两个
同房歇卧。主家婆余氏初时甚是抬举二人,与他金银首饰装束身子。后日不料白玉
莲死了,止落下金莲一人,长成一十八岁,出落的脸衬桃花,眉弯新月。张大户每
要收他,只碍主家婆厉害,不得到手。一日主家婆邻家赴席不在,大户暗把金莲唤
至房中,遂收用了。正是:

莫讶天台相见晚,刘郎还是老刘郎。

大户自从收用金莲之后,不觉身上添了四五件病症。端的那五件?第一腰便添
疼,第二眼便添泪,第三耳便添聋,第四鼻便添涕,第五尿便添滴。自有了这几件
病后,主家婆颇知其事,与大户嚷骂了数日,将金莲百般苦打。大户知道不容,却
赌气倒赔了房奁,要寻嫁得一个相应的人家。大户家下人都说武大忠厚,见无妻小
,又住着宅内房儿,堪可与他。这大户早晚还要看觑此女,因此不要武大一文钱,
白白地嫁与他为妻。这武大自从娶了金莲,大户甚是看顾他。若武大没本钱做炊饼
,大户私与他银两。武大若挑担儿出去,大户候无人,便踅入房中与金莲厮会。武
大虽一时撞见,原是他的行货,不敢声言。朝来暮往,也有多时。忽一日大户得患
阴寒病症,呜呼死了。主家婆察知其事,怒令家僮将金莲、武大即时赶出。武大故
此遂寻了紫石街西王皇亲房子,赁内外两间居住,依旧卖炊饼。

原来这金莲自嫁武大,见他一味老实,人物猥[犭衰],甚是憎嫌,常与他合
气。报怨大户:“普天世界断生了男子,何故将我嫁与这样个货!每日牵着不走,
打着倒退的,只是一味[口床]酒,着紧处却是锥钯也不动。奴端的那世里悔气,
却嫁了他!是好苦也!”常无人处,唱个《山坡羊》为证:

想当初,姻缘错配,奴把你当男儿汉看觑。不是奴自己夸奖,他乌鸦
怎配鸾凤对!奴真金子埋在土里,他是块高号铜,怎与俺金色比!他本是
块顽石,有甚福抱着我羊脂玉体!好似粪土上长出灵芝。奈何,随他怎样
,到底奴心不美。听知:奴是块金砖,怎比泥土基!

看官听说:但凡世上妇女,若自己有几分颜色,所禀伶俐,配个好男子便罢了
,若是武大这般,虽好杀也未免有几分憎嫌。自古佳人才子相配着的少,买金偏撞
不着卖金的。

武大每日自挑担儿出去卖炊饼,到晚方归。那妇人每日打发武大出门,只在帘
子下嗑瓜子儿,一径把那一对小金莲故露出来,勾引浮浪子弟,日逐在门前弹胡博
词,撒谜语,叫唱:“一块好羊肉,如何落在狗嘴里?”油似滑的言语,无般不说
出来。因此武大在紫石街又住不牢,要往别处搬移,与老婆商议。妇人道:“贼馄
饨不晓事的,你赁人家房住,浅房浅屋,可知有小人罗唣!不如添几两银子,看相
应的,典上他两间住,却也气概些,免受人欺侮。”武大道:“我那里有钱典房?
”妇人道:“呸!浊才料,你是个男子汉,倒摆布不开,常交老娘受气。没有银子
,把我的钗梳凑办了去,有何难处!过后有了再治不迟。”武大听老婆这般说,当
下凑了十数两银子,典得县门前楼上下两层四间房屋居住。第二层是楼,两个小小
院落,甚是干净。

武大自从搬到县西街上来,照旧卖炊饼过活,不想这日撞见自己嫡亲兄弟。当
日兄弟相见,心中大喜。一面邀请到家中,让至楼上坐,房里唤出金莲来,与武松
相见。因说道:“前日景阳冈上打死大虫的,便是你的小叔。今新充了都头,是我
一母同胞兄弟。”那妇人叉手向前,便道:“叔叔万福。”武松施礼,倒身下拜。
妇人扶住武松道:“叔叔请起,折杀奴家。”武松道:“嫂嫂受礼。”两个相让了
一回,都平磕了头起来。少顷,小女迎儿拿茶,二人吃了。武松见妇人十分妖娆,
只把头来低着。不多时,武大安排酒饭,款待武松。

说话中间,武大下楼买酒菜去了,丢下妇人,独自在楼上陪武松坐地。看了武
松身材凛凛,相貌堂堂,又想他打死了那大虫,毕竟有千百斤气力。口中不说,心
下思量道:“一母所生的兄弟,怎生我家那身不满尺的丁树,三分似人七分似鬼,
奴那世里遭瘟撞着他来!如今看起武松这般人壮健,何不叫他搬来我家住?想这段
姻缘却在这里了。”于是一面堆下笑来,问道:“叔叔你如今在那里居住?每日饭
食谁人整理?”武松道:“武二新充了都头,逐日答应上司,别处住不方便,胡乱
在县前寻了个下处,每日拨两个土兵伏侍做饭。”妇人道:“叔叔何不搬来家里住
?省的在县前土兵服侍做饭腌臜。一家里住,早晚要些汤水吃时,也方便些
。就是奴家亲自安排与叔叔吃,也干净。”武松道:“深谢嫂嫂。”妇人又道:“
莫不别处有婶婶?可请来厮会。”武松道:“武二并不曾婚娶。”妇人道:“叔叔
青春多少?”武松道:“虚度二十八岁。”妇人道:“原来叔叔倒长奴三岁。叔叔
今番从那里来?”武松道:“在沧州住了一年有馀,只想哥哥在旧房居住,不道移
在这里。”妇人道:“一言难尽。自从嫁得你哥哥,吃他忒善了,被人欺负,才到
这里来。若是叔叔这般雄壮,谁敢道个不字!”武松道:“家兄从来本分,不似武
松撒泼。”妇人笑道:“怎的颠倒说!常言:人无刚强,安身不长。奴家平生性快
,看不上那三打不回头,四打和身转的”武松道:“家兄不惹祸,免得嫂嫂忧心。
”二人在楼上一递一句的说。有诗为证:

叔嫂萍踪得偶逢,娇娆偏逞秀仪容。
私心便欲成欢会,暗把邪言钓武松。

话说金莲陪着武松正在楼上说话未了,只见武大买了些肉菜果饼归家。放在厨
,走上楼来,叫道:“大嫂,你且下来则个。”那妇人应道:“你看那不晓事的,
!叔叔在此无人陪侍,却交我撇了下去。”武松道:“嫂嫂请方便。”妇人道:“
何不去间壁请王乾娘来安排?只是这般不见便。”武大便自去央了间壁王婆来。安
排端正,都拿上楼来,摆在桌子上,无非是些鱼肉果菜点心之类。随即烫酒上来。
武大叫妇人坐了主位,武松对席,武大打横。三人坐下,把酒来斟,武大筛酒在各
人面前。那妇人拿起酒来道:“叔叔休怪,没甚管待,请杯儿水酒。”武松道:“
感谢嫂嫂,休这般说。”武大只顾上下筛酒,那妇人笑容可掬,满口儿叫:“叔叔
,怎的肉果儿也不拣一箸儿?”拣好的递将过来。武松是个直性的汉子,只把做亲
嫂嫂相待。谁知这妇人是个使女出身,惯会小意儿。亦不想这妇人一片引人心。那
妇人陪武松吃了几杯酒,一双眼只看着武松的身上。武松吃他看不过,只得倒低了
头。吃了一歇,酒阑了,便起身。武大道:“二哥没事,再吃几杯儿去。”武松道
:“生受,我再来望哥哥嫂嫂罢。”都送下楼来。出的门外,妇人便道:“叔叔是
必上心搬来家里住,若是不搬来,俺两口儿也吃别人笑话。亲兄弟难比别人,与我
们争口气,也是好处。”武松道:“既是嫂嫂厚意,今晚有行李便取来。”妇人道
:“奴这里等候哩!”正是:

满前野意无人识,几点碧桃春自开。