\chapter{吴神仙冰鉴定终身~潘金莲兰汤邀午战}

词曰:

新凉睡起,兰汤试浴郎偷戏。去曾嗔怒,来便生欢喜。  奴道无心
,郎道奴如此。情如水,易开难断,若个知生死。

话说到次日,潘金莲早起,打发西门庆出门。记挂着要做那红鞋,拿着针线筐
儿,往翡翠轩台基儿上坐着,描画鞋扇。使春梅请了李瓶儿来到。李瓶儿问道:“
姐姐,你描金的是甚么?”金莲道:“要做一双大红鞋素缎子白绫平底鞋儿,鞋尖
上扣绣鹦鹉摘桃。”李瓶儿道:“我有一方大红十样锦缎子,也照依姐姐描恁一双
儿。我做高低的罢。”于是取了针线筐,两个同一处做。金莲描了一只丢下,说道
:“李大姐,你替我描这一只,等我后边把孟三姐叫了来。他昨日对我说,他也要
做鞋哩。”一直走到后边。玉楼在房中倚着护炕儿,也衲着一只鞋儿哩。看见金莲
进来,说道:“你早办!”金莲道:“我起来的早,打发他爹往门外与贺千户送行
去了。教我约下李大姐,花园里赶早凉做些生活。我才描了一只鞋,教李大姐替我
描着,迳来约你同去,咱三个一搭儿里好做。”因问:“你手里衲的是甚么鞋?”
玉楼道:“是昨日你看我开的那双玄色缎子鞋。”金莲道:“你好汉!又早衲出一
只来了。”玉楼道:“那只昨日就衲好了,这一只又衲了好些了。”金莲接过看了
一回,说:“你这个,到明日使甚么云头子?”玉楼道:“我比不得你每小后生,
花花黎黎。我老人家了,使羊皮金缉的云头子罢,周围拿纱绿线锁,好不好?”金
莲道:“也罢。你快收拾,咱去来,李瓶儿那里等着哩。”玉楼道:“你坐着吃了
茶去。”金莲道:“不吃罢,拿了茶,那里去吃来。”玉楼吩咐兰香顿下茶送去。
两个妇人手拉着手儿,袖着鞋扇,迳往外走。吴月娘在上房穿廊下坐,便问:“你
每那去?”金莲道:“李大姐使我替他叫孟三儿去,与他描鞋。”说着,一直来到
花园内。

三人一处坐下,拿起鞋扇,你瞧我的,我瞧你的,都瞧了一遍。玉楼便道:“
六姐,你平白又做平底子红鞋做甚么?不如高低好看。你若嫌木底子响脚,也似我
用毡底子,却不好?”金莲道:“不是穿的鞋,是睡鞋。他爹因我那只睡鞋,被小
奴才儿偷去弄油了,吩咐教我从新又做这双鞋。”玉楼道:“又说鞋哩,这个也不
是舌头,李大姐在这里听着。昨日因你不见了这只鞋,他爹打了小铁棍儿一顿,说
把他打的躺在地下,死了半日。惹的一丈青好不在后边海骂,骂那个淫妇王八羔子
学舌,打了他恁一顿,早是活了,若死了,淫妇、王八羔子也不得清洁!俺再不知
骂的是谁。落后小铁棍儿进来,大姐姐问他:‘你爹为甚么打你?’小厮才说:‘
因在花园里耍子,拾了一只鞋,问姑夫换圈儿来。不知是甚么人对俺爹说了,教爹
打我一顿。我如今寻姑夫,问他要圈儿去也。’说毕,一直往前跑了。原来骂的‘
王八羔子’是陈姐夫。早是只李娇儿在旁边坐着,大姐没在跟前,若听见时,又是
一场儿。”金莲道:“大姐姐没说甚么?”玉楼道:“你还说哩,大姐姐好不说你
哩!说:‘如今这一家子乱世为王,九条尾狐狸精出世了,把昏君祸乱的贬子休妻
,想着去了的来旺儿小厮,好好的从南边来了,东一帐西一帐,说他老婆养着主子
,又说他怎的拿刀弄杖,生生儿祸弄的打发他出去了,把个媳妇又逼的吊死了。如
今为一只鞋子,又这等惊天动地反乱。你的鞋好好穿在脚上,怎的教小厮拾了?想
必吃醉了,在花园里和汉子不知怎的饧成一块,才掉了鞋。如今没的摭羞,拿小厮
顶缸,又不曾为甚么大事。’”金莲听了,道:“没的扯[毛必]淡!甚么是‘大
事’?杀了人是大事了,奴才拿刀要杀主子!”向玉楼道:“孟三姐,早是瞒不了
你,咱两个听见来兴儿说了一声,唬的甚么样儿的!你是他的大老婆,倒说这个话
!你也不管,我也不管,教奴才杀了汉子才好。他老婆成日在你后边使唤,你纵容
着他不管,教他欺大灭小,和这个合气,和那个合气。各人冤有头,债有主,你揭
条我,我揭条你,吊死了,你还瞒着汉子不说。早是苦了钱,好人情说下来了,不
然怎了?你这等推干净,说面子话儿,左右是,左右我调唆汉子!也罢,若不教他
把奴才老婆、汉子一条提撵的离门离户也不算!恒数人挟不到我井里头!”玉楼见
金莲粉面通红,恼了,又劝道:“六姐,你我姐妹都是一个人,我听见的话儿,有
个不对你说?说了,只放在你心里,休要使出来。”金莲不依他。到晚等的西门庆
进入他房来,一五一十告西门庆说:“来昭媳妇子一丈青怎的在后边指骂,说你打
了他孩子,要逻揸儿和人嚷。”这西门庆不听便罢,听了记在心里。到次日,要撵
来昭三口子出门。多亏月娘再三拦劝下,不容他在家,打发他往狮子街房子里看守
,替了平安儿来家守大门。后次月娘知道,甚恼金莲,不在话下。

西门庆一日正在前厅坐,忽平安儿来报:“守备府周爷差人送了一位相面先生
,名唤吴神仙,在门首伺候见爹。”西门庆唤来人进见,递上守备帖儿,然后道:
“有请。”须臾,那吴神仙头戴青布道巾,身穿布袍草履,腰系黄丝双穗绦,手执
龟壳扇子,自外飘然进来。年约四十之上,生得神清如长江皓月,貌古似太华乔松
。原来神仙有四般古怪:身如松,声如钟,坐如弓,走如风。但见他:

能通风鉴,善究子平。观乾象,能识阴阳;察龙经,明知风水。五星
深讲,三命秘谈。审格局,决一世之荣枯;观气色,定行年之休咎。若非
华岳修真客,定是成都卖卜人。

西门庆见神仙进来,忙降阶迎接,接至厅上。神仙见西门庆,长揖稽首就坐。
须臾茶罢。西门庆动问神仙:“高名雅号,仙乡何处,因何与周大人相识?”那吴
神仙欠身道:“贫道姓吴名[百大百],道号守真。本贯浙江仙游人。自幼从师天
台山紫虚观出家。云游上国,因往岱宗访道,道经贵处。周老总兵相约,看他老夫
人目疾,特送来府上观相。”西门庆道:“老仙长会那几家阴阳?道那几家相法?
”神仙道:“贫道粗知十三家子平,善晓麻衣相法,又晓六壬神课。常施药救人,
不爱世财,随时住世。”西门庆听言,益加敬重,夸道:“真乃谓之神仙也。”一
面令左右放桌儿,摆斋管待。神仙道:“贫道未道观相,岂可先要赐斋。”西门庆
笑道:“仙长远来,一定未用早斋。待用过,看命未迟。”于是陪着神仙吃了些斋
食素馔,抬过桌席,拂抹干净,讨笔砚来。

神仙道:“请先观贵造,然后观相尊容。”西门庆便说与八字:“属虎的,二
十九岁了,七月二十八日午时生。”这神仙暗暗十指寻纹,良久说道:“官人贵造
:戊寅年,辛酉月,壬午日,丙午时。七月廿三日白戊,已交八月算命。月令提刚
辛酉,理取伤官格。子平云:伤官伤尽复生财,财旺生官福转来。立命申宫,七岁
行运辛酉,十七行壬戌,二十七癸亥,三十七甲子,四十七乙丑。官人贵造,依贫
道所讲,元命贵旺,八字清奇,非贵则荣之造。但戊土伤官,生在七八月,身忒旺
了。幸得壬午日干,丑中有癸水,水火相济,乃成大器。丙午时,丙合辛生,后来
定掌威权之职。一生盛旺,快乐安然,发福迁官,主生贵子。为人一生耿直,干事
无二,喜则合气春风,怒则迅雷烈火。一生多得妻财,不少纱帽戴。临死有二子送
老。今岁丁未流年,丁壬相合,目下丁火来克,克我者为官为鬼,必主平地登云之
喜,添官进禄之荣。大运见行癸亥,戊土得癸水滋润,定见发生。目下透出红鸾天
喜,定有熊罴之兆。又命宫驿马临申,不过七月必见矣。”西门庆问道:“我后来
运限如何?”神仙道:“官人休怪我说,但八字中不宜阴水太多,后到甲子运中,
将壬午冲破了,又有流星打搅,不出六六之年,主有呕血流浓之灾,骨瘦形衰之病
。”西门庆问道:“目下如何?”神仙道:“目今流年,日逢破败五鬼在家吵闹,
些小气恼,不足为灾,都被喜气神临门冲散了。”西门庆道:“命中还有败否?”
神仙道:“年赶着月,月赶着日,实难矣。”

西门庆听了,满心欢喜,便道:“先生,你相我面如何?”神仙道:“请尊容
转正。”西门庆把座儿掇了一掇。神仙相道:“夫相者,有心无相,相逐心生;有
相无心,相随心往。吾观官人:头圆项短,定为享福之人;体健筋强,决是英豪之
辈;天庭高耸,一生衣禄无亏;地阁方圆,晚岁荣华定取。此几椿儿好处。还有几
椿不足之处,贫道不敢说。”西门庆道:“仙长但说无妨。”神仙道:“请官人走
两步看。”西门庆真个走了几步。神仙道:“你行如摆柳,必主伤妻;若无刑克,
必损其身。妻宫克过方好。”西门庆道:“已刑过了。”神仙道:“请出手来看一
看。”西门庆舒手来与神仙看。神仙道:“智慧生于皮毛,苦乐观于手足。细软丰
润,必享福禄之人也。两目雌雄,必主富而多诈;眉生二尾,一生常自足欢娱;根
有三纹,中岁必然多耗散;奸门红紫,一生广得妻财;黄气发于高旷,旬日内必定
加官;红色起于三阳,今岁间必生贵子。又有一件不敢说,泪堂丰厚,亦主贪花;
且喜得鼻乃财星,验中年之造化;承浆地阁,管来世之荣枯。

承浆地阁要丰隆,准乃财星居正中。
生平造化皆由命,相法玄机定不容。”

神仙相毕,西门庆道:“请仙长相相房下众人。”一面令小厮:“后边请你大
娘出来。”于是李娇儿、孟玉楼、潘金莲、李瓶儿、孙雪娥等众人都跟出来,在软
屏后潜听。神仙见月娘出来,连忙道了稽首,也不敢坐,就立在旁边观相。端详了
一回,说:“娘子面如满月,家道兴隆;唇若红莲,衣食丰足,必得贵而生子;声
响神清,必益夫而发福。请出手来。”月娘从袖中露出十指春葱来。神仙道:“干
姜之手,女人必善持家,照人之鬓,坤道定须秀气。这几椿好处。还有些不足之处
,休怪贫道直说。”西门庆道:“仙长但说无妨。”“泪堂黑痣,若无宿疾,必刑
夫;眼下皴纹,亦主六亲若冰炭。

女人端正好容仪,缓步轻如出水龟。
行不动尘言有节,无肩定作贵人妻。”

相毕,月娘退后。西门庆道:“还有小妾辈,请看看。”于是李娇儿过来。神仙观
看良久:“此位娘子,额尖鼻小,非侧室,必三嫁其夫;肉重身肥,广有衣食而荣
华安享;肩耸声泣,不贱则孤;鼻梁若低,非贫即夭。请步几步我看。”李娇儿走
了几步。神仙道:

额尖露背并蛇行,早年必定落风尘。
假饶不是娼门女,也是屏风后立人。

相毕,李娇儿下去。吴月娘叫:“孟三姐,你也过来相一相。”神仙观道:“这位
娘子,三停平等,一生衣禄无亏;六府丰隆,晚岁荣华定取。平生少疾,皆因月孛
光辉;到老无灾,大抵年宫润秀。请娘子走两步。”玉楼走了两步,神仙道:

口如四字神清澈,温厚堪同掌上珠。
威命兼全财禄有,终主刑夫两有余。

玉楼相毕,叫潘金莲过来。那潘金莲只顾嘻笑,不肯过来。月娘催之再三,方才出
见。神仙抬头观看这个妇人,沉吟半日,方才说道:“此位娘子,发浓[髟丐]重
,光斜视以多淫;脸媚眉弯,身不摇而自颤。面上黑痣,必主刑夫;唇中短促,终
须寿夭。

举止轻浮惟好淫,眼如点漆坏人伦。
月下星前长不足,虽居大厦少安心。”

相毕金莲,西门庆又叫李瓶儿上来,教神仙相一相。神仙观看这个女人:“皮肤香
细,乃富室之女娘;容貌端庄,乃素门之德妇。只是多了眼光如醉,主桑中之约;
眉眉靥生,月下之期难定。观卧蚕明润而紫色,必产贵儿;体白肩圆,必受夫之宠
爱。常遭疾厄,只因根上昏沉;频遇喜祥,盖谓福星明润。此几椿好处。还有几椿
不足处,娘子可当戒之:山根青黑,三九前后定见哭声;法令细[纟亠回且],鸡
犬之年焉可过?慎之!慎之!

花月仪容惜羽翰,平生良友凤和鸾。
朱门财禄堪依倚,莫把凡禽一样看。”

相毕,李瓶儿下去。月娘令孙雪娥出来相一相。神仙看了,说道:“这位娘子,体
矮声高,额尖鼻小,虽然出谷迁乔,但一生冷笑无情,作事机深内重。只是吃了这
四反的亏,后来必主凶亡。夫四反者:唇反无棱,耳反无轮,眼反无神,鼻反不正
故也。

燕体蜂腰是贱人,眼如流水不廉真。
常时斜倚门儿立,不为婢妾必风尘。”

雪娥下去,月娘教大姐上来相一相。神仙道:“这位女娘,鼻梁低露,破祖刑家;
声若破锣,家私消散。面皮太急,虽沟洫长而寿亦夭;行如雀跃,处家室而衣食缺
乏。不过三九,当受折磨。

惟夫反目性通灵,父母衣食仅养身。
状貌有拘难显达,不遭恶死也艰辛。”

大姐相毕,教春梅也上来教神仙相相。神仙睁眼儿见了春梅,年约不上二九,头戴
银丝云髻儿,白线挑衫儿,桃红裙子,蓝纱比甲儿,缠手缠脚出来,道了万福。神
仙观看良久,相道:“此位小姐五官端正,骨格清奇。发细眉浓,禀性要强;神急
眼圆,为人急燥。山根不断,必得贵夫而生子;两额朝拱,主早年必戴珠冠。行步
若飞仙,声响神清,必益夫而得禄,三九定然封赠。但吃了这左眼大,早年克父;
右眼小,周岁克娘。左口角下这一点黑痣,主常沾啾唧之灾;右腮一点黑痣,一生
受夫敬爱。

天庭端正五官平,口若涂砂行步轻。
仓库丰盈财禄厚,一生常得贵人怜。”

神仙相毕,众妇女皆咬指以为神相。西门庆封白银五两与神仙,又赏守备府来人银
五钱,拿拜帖回谢。吴神仙再三辞却,说道:“贫道云游四方,风餐露宿,要这财
何用?决不敢受。”西门庆不得已,拿出一匹大布:“送仙长一件大衣如何?”神
仙方才受之,令小童接了,稽首拜谢。西门庆送出大门,飘然而去。正是:

柱杖两头挑日月,葫芦一个隐山川。

西门庆回到后厅,问月娘:“众人所相何如?”月娘道:“相的也都好,只是
三个人相不着。”西门庆道:“那三个相不着?”月娘道:“相李大姐有实疾,到
明日生贵子,他见今怀着身孕,这个也罢了。相咱家大姐到明日受磨折,不知怎的
磨折?相春梅后来也生贵子,或者你用好他,各人子孙也看不见。我只不信,说他
后来戴珠冠,有夫人之分。端的咱家又没官,那讨珠冠来?就有珠冠,也轮不到他
头上。”西门庆笑道:“他相我目下有平地登云之喜,加官进禄之荣,我那得官来
?他见春梅和你俱站在一处,又打扮不同,戴着银丝云髻儿,只当是你我亲生女儿
一般,或后来匹配名门,招个贵婿,故说有珠冠之分。自古算的着命,算不着好,
相逐心生,相随心灭。周大人送来,咱不好嚣了他的,教他相相除疑罢了。”说毕
,月娘房中摆下饭,打发吃了饭。

西门庆手拿芭蕉扇儿,信步闲游。来花园大卷棚聚景堂内,周围放下帘栊,四
下花木掩映。正值日午,只闻绿阴深处一派蝉声,忽然风送花香,袭人扑鼻。有诗
为证:

绿树荫浓夏日长,楼台倒影入池塘。
水晶帘动微风起,一架蔷薇满院香。

西门庆坐于椅上以扇摇凉。只见来安儿、画童儿两个小厮来井上打水。西门庆
道:“教一个来。”来安儿忙走向前,西门庆吩咐:“到后边对你春梅姐说,有梅
汤提一壶来我吃。”来安儿应诺去了。半日,只见春梅家常戴着银丝云髻儿,手提
一壶蜜煎梅汤,笑嘻嘻走来,问道:“你吃了饭了?”西门庆道:“我在后边吃了
。”春梅说:“嗔道不进房里来。说你要梅汤吃,等我放在冰里湃一湃你吃。”西
门庆点头儿。春梅湃上梅汤,走来扶着椅儿,取过西门庆手中芭蕉扇儿替他打扇,
问道:“头里大娘和你说甚么?”西门庆道:“说吴神仙相面一节。”春梅道:“
那道士平白说戴珠冠,教大娘说‘有珠冠,只怕轮不到他头上’。常言道凡人不可
貌相,海水不可斗量,从来旋的不圆,砍的圆,各人裙带上衣食,怎么料得定?莫
不长远只在你家做奴才罢!”西门庆笑道:“小油嘴儿,你若到明日有了娃儿,就
替你上了头。”于是把他搂到怀里,手扯着手儿顽耍,问道:“你娘在那里?怎的
不见?”春梅道:“娘在屋里,教秋菊热下水要洗浴。等不的,就在床上睡了。”
西门庆道:“等我吃了梅汤,鬼混他一混去。”于是春梅向冰盆内倒了一瓯儿梅汤
,与西门庆呷了一口,湃骨之凉,透心沁齿,如甘露洒心一般。

须臾吃毕,搭伏着春梅肩膀儿,转过角门来到金莲房中。看见妇人睡在正面一
张新买的螺钿床上。原是因李瓶儿房中安着一张螺钿敞厅床,妇人旋教西门庆使了
六十两银子,替他也买了这一张螺钿有栏干的床。两边槅扇都是螺钿攒造花
草翎毛,挂着紫纱帐幔,锦带银钩。妇人赤露玉体,止着红绡抹胸儿,盖着红纱衾
,枕着鸳鸯枕,在凉席之上,睡思正浓。西门庆一见,不觉淫心顿起,令春梅带上
门出去,悄悄脱了衣裤,上的床来,掀开纱被,见他玉体相互掩映,戏将两股轻开
,按麈柄徐徐插入牝中,比及星眼惊欠之际,已抽拽数十度矣。妇人睁开眼,笑道
:“怪强盗,三不知多咱进来?奴睡着了,就不知道。奴睡的甜甜的,掴混死了我
!”西门庆道:“我便罢了,若是个生汉子进来,你也推不知道罢?”妇人道:“
我不好骂的,谁人七个头八个胆,敢进我这房里来!只许你恁没大没小的罢了。”
原来妇人因前日西门庆在翡翠轩夸奖李瓶儿身上白净,就暗暗将茉莉花蕊儿搅酥油
定粉,把身上都搽遍了,搽的白腻光滑,异香可爱,欲夺其宠。西门庆见他身体雪
白,穿着新做的两只大红睡鞋。一面蹲踞在上,两手兜其股,极力而提之,垂首观
其出入之势。妇人道:“怪货,只顾端详甚么?奴的身上黑,不似李瓶儿的身上白
就是了。他怀着孩子,你便轻怜痛惜,俺每是拾的,由着这等掇弄。”西门庆问道
:“说你等着我洗澡来?”妇人问道:“你怎得知道来?”西门庆道:“是春梅说
的。”妇人道:“你洗,我叫春梅掇水来。”不一时把浴盆掇到房中,注了汤。二
人下床来,同浴兰汤,共效鱼水之欢。洗浴了一回,西门庆乘兴把妇人仰卧在浴板
之上,两手执其双足跨而提之,掀腾[扌扉]干,何止二三百回,其声如泥中螃蟹
一般响之不绝。妇人恐怕香云拖坠,一手扶着云[髟丐],一手扳着盆沿,口中燕
语莺声,百般难述。怎见这场交战?但见:

华池荡漾波纹乱,翠帏高卷秋云暗。才郎情动逞风流,美女心欢显手
段。叭叭嗒嗒弄声响,砰砰啪啪成一片。滑滑[氵刍][氵刍]怎停住,
拦拦济济难存站。一个逆水撑船,将玉股摇;一个艄公把舵,将金莲[扌
昝]。拖泥带水两情痴,殢雨尤云都不辩。任他锦帐凤鸾交,不似
兰汤鱼水战。

二人水中战斗了一回,西门庆精泄而止。拭抹身体干净,撤去浴盆。止着薄[
纟广]短襦上床,安放炕桌果酌饮酒。教秋菊:“取白酒来与你爹吃。”又拿果馅
饼与西门庆吃,恐怕他肚中饥饿。只见秋菊半日拿上一银注子酒来。妇人才斟了一
钟,摸了摸冰凉的,就照着秋菊脸上只一泼,泼了一头一脸,骂道:“好贼少死的
奴才!我吩咐教你烫了来,如何拿冷酒与爹吃?你不知安排些甚么心儿?”叫春梅
:“与我把这奴才采到院子里跪着去。”春梅道:“我替娘后边卷裹脚去来,一些
儿没在跟前,你就弄下硶儿了。”那秋菊把嘴谷都着,口里喃喃呐呐说道:
“每日爹娘还吃冰湃的酒儿,谁知今日又改了腔儿。”妇人听见骂道:“好贼奴才
,你说甚么?与我采过来!”叫春梅每边脸上打与他十个嘴巴。春梅道:“皮脸,
没的打污浊了我手。娘只教他顶着石头跪着罢。”于是不由分说,拉到院子里,教
他顶着块大石头跪着,不在话下。妇人从新叫春梅暖了酒来,陪西门庆吃了几钟,
掇去酒桌,放下纱帐子来,吩咐拽上房门,两个抱头交股,体倦而寝。正是:

若非群玉山头见,多是阳台梦里寻。