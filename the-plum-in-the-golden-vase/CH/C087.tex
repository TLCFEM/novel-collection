\chapter{王婆子贪财忘祸~武都头杀嫂祭兄}

诗曰:
悠悠嗟我里,世乱各东西。
存者问消息,死者为尘泥。
贱子家既败,壮士归来时。
行久见空巷,日暮气惨凄。
但逢狐与狸,竖毛怒裂眦。
我有镯镂剑,对此吐长霓。

话说陈敬济雇头口起身,叫了张团练一个伴当跟随,早上东京去不题。却表吴月娘
打发潘金莲出门,次日使春鸿叫薛嫂儿来,要卖秋菊。这春鸿正走到大街,撞见应
伯爵,叫住问:“春鸿,你往那里去?”春鸿道:“大娘使小的叫媒人薛嫂儿去。
”伯爵问:“叫媒人做甚么?”春鸿道:“卖五娘房里秋菊丫头。”伯爵又问:“
你五娘为甚么打发出来嫁人?”这春鸿便如此这般,”因和俺姐夫有些说话,大娘
知道了,先打发了春梅小大姐,然后打了俺姐夫一顿,赶出往家去了。昨日才打发
出俺五娘来。”伯爵听了,点了点头儿,说道:“原来你五娘和你姐夫有楂儿,看
不出人来。”又向春鸿说:“孩儿,你爹已是死了,你只顾还在他家做甚么?终是
没出产。你心里还要归你南边去?还是这里寻个人家跟罢。”春鸿道:“便是这般
说。老爹已是没了,家中大娘好不严禁,各处买卖都收了,房子也卖了,琴童儿、
画童儿都走了,也揽不过这许多人口来。小的待回南边去,又没顺便人带去。这城
内寻个人家跟,又没个门路。”伯爵道:“傻孩儿,人无远见,安身不牢。千山万
水,又往南边去做甚?你肚里会几句唱,愁这城内寻不出主儿来答应。我如今举保
个门路与你。如今大街坊张二老爹家,有万万贯家财,见顶补了你爹在提刑院做掌
刑千户。如今你二娘又在他家做了二房,我把你送到他宅中答应,他见你会唱南曲
,管情一箭就上垛,留下你做个亲随大官儿,又不比在你家里。他性儿又好,年纪
小小,又倜傥,又爱好,你就是个有造化的。”这春鸿扒倒地下就磕了个头:“有
累二爹。小的若见了张老爹,得一步之地,买礼与二爹磕头。”伯爵一把手拉着春
鸿说:“傻孩儿,你起来,我无有个不作成人的,肯要你谢?你那得钱儿来!”春
鸿道:“小的去了,只怕家中大娘抓寻小的怎了?”伯爵道:“这个不打紧。我问
你张二老爹讨个贴儿,封一两银子与他家。他家银子不敢受,不怕不把你不双手儿
送了去。”说毕,春鸿往薛嫂儿家,叫了薛嫂儿。见月娘,领秋菊出来,只卖了五
两银子,交与月娘,不在话下。

却说应伯爵领春鸿到张二官宅里见了。张二官见他生的清秀,又会唱南曲,就留下
他答应。便拿拜贴儿,封了一两银子,送往西门庆家,讨他箱子。那日吴月娘家中
正陪云离守娘子范氏吃酒。先是云离守补在清河左卫做同知,见西门庆死了,吴月
娘守寡,手里有东西,就安心有垂涎图谋之意。此日正买了八盘羹果礼物,来看月
娘。见月娘生了孝哥,范氏房内亦有一女,方两月儿,要与月娘结亲。那日吃酒,
遂两家割衫襟,做了儿女亲家,留下一双金环为定礼。听见玳安儿拿进张二官府贴
儿,并一两银子,说春鸿投在他家答应去了,使人来讨他箱子衣服。月娘见他见做
提刑官,不好不与他,银子也不曾收,只得把箱子与将出来。

初时,应伯爵对张二官说:“西门庆第五娘子潘金莲生得标致,会一手琵琶。百家
词曲,双陆象棋,无不通晓,又会写字。因为年小守不的,又和他大娘合气,今打
发出来,在王婆家嫁人。”这张二官一替两替使家人拿银子往王婆家相看,王婆只
推他大娘子分付,不倒口要一百两银子。那人来回讲了几遍,还到八十两上,王婆
还不吐口儿。落后春鸿到他宅内,张二官听见春鸿说,妇人在家养育女婿方打发出
来。这张二官就不要了,对着伯爵说:“我家现放着十五岁未出幼儿子上学攻书,
要这样妇人来家做甚?”又听见李娇儿说,金莲当初用毒药摆布死了汉子,被西门
庆占将来家,又偷小厮,把第六个娘子娘儿两个,生生吃他害杀了。以此张二官就
不要了。

话分两头。却说春梅卖到守备府中,守备见他生的标致伶俐,举止动人,心中大喜
。与了他三间房住,手下使一个小丫鬟,就一连在他房中歇了三夜。三日,替他裁
了两套衣服。薛嫂儿去,赏了薛嫂五钱银子。又买了个使女扶持他,立他做第二房
。大娘子一目失明,吃长斋念佛,不管闲事。还有生姐儿孙二娘,在东厢居住。春
梅在西厢房,各处钥匙都教他掌管,甚是宠爱他。一日,听薛嫂儿说,金莲出来在
王婆家聘嫁,这春梅晚夕啼啼哭哭对守备说:“俺娘儿两个,在一处厮守这几年,
他大气儿不着呵着我,把我当亲女儿一般看承。只知拆散开了,不想今日他也出来
了,你若肯娶将他来,俺娘儿每还在一处,过好日子。”又说他怎的好模样儿,诸
般词曲都会,又会弹琵琶。聪明俊俏,百伶百俐。属龙的,今才三十二岁儿。”他
若来,奴情愿做第三也罢。”于是把守备念转了,使手下亲随张胜、李安封了二方
手帕,二钱银子,往王婆家相看,果然生的好个出色的妇人。王婆开口指称他家大
娘子要一百两银子。张胜、李安讲了半日,还了八十两,那王婆不肯,不转口儿,
要一百两:“媒人钱不要便罢了,天也不使空人。”这张胜、李安只得又拿回银子
来禀守备。丢了两日,怎禁这春梅晚夕啼啼哭哭:“好歹再添几两银子,娶了来和
奴做伴儿,死也甘心。”守备见春梅只是哭泣,只得又差了大管家周忠,同张胜《
李安,毡包内拿着银子,打开与婆子看,又添到九十两上。婆子越发张致起来,说
:“若九十两,到不的如今,提刑张二老爹家抬的去了。”这周忠就恼了,分付李
安把银子包了,说道:“三只脚蟾便没处寻,两脚老婆愁寻不出来!这老淫妇连人
也不识。你说那张二官府怎的,俺府里老爹管不着你?不是新娶的小夫人再三在老
爷跟前说念,要娶这妇人,平白出这些银子,要他何用!”李安道:“勒掯俺两番
三次来回,贼老淫妇,越发鹦哥儿风了!”拉着周忠说:“管家,咱去来,到家回
了老爷,好不好教牢子拿去,拶与他一顿好拶子。”这婆子终是贪着陈敬济那口食
,由他骂,只是不言语。二人到府中,回禀守备说:“已添到九十两,还不肯。”
守备说:“明日兑与他一百两,拿轿子抬了来罢。”周忠说:“爷就与了一百两,
王婆还要五两媒人钱。且丢他两日,他若张致,拿到府中拶与他一顿拶子,他才怕
。”看官听说,大段金莲生有地而死有处,不争被周忠说这两句话。有分交:这妇
人从前作过事,今朝没兴一齐来。有诗为证:
人生虽未有前知,祸福因由更问谁。
善恶到头终有报,只争来早与来迟。

按下一头。单表武松自从垫发孟州牢城充军之后,多亏小管营施恩看顾。次后,施
恩与蒋门神争夺快活林酒店,被蒋门神打伤,央武松出力,反打了蒋门神一顿。不
想蒋门神妹子玉兰,嫁与张都监为妾,赚武松去,假捏贼情,将武松拷打,转又发
安平寨充军。这武松走到飞云浦,又杀了两个公人,复回身杀了张都监、蒋门神全
家老小,逃躲在施恩家。施恩写了一封书,皮箱内封了一百两银子,教武松到安平
寨与知寨刘高,教看顾他。不想路上听见太子立东宫,放郊天大赦,武松就遇赦回
家,到清河县下了文书,依旧在县当差,还做都头。来到家中,寻见上邻姚一郎,
交付迎儿。那时迎儿已长大十九岁了,收揽来家,一处居住。就有人告他说:“西
门庆已死,你嫂子又出来了,如今还在王婆家,早晚嫁人。”这汉子扣了,旧仇在
心。正是:
踏破铁鞋无觅处,得来全不费工夫。

次日,理帻穿衣,径走过间壁王婆门首。金莲正在帘下站着,见武松来,连忙闪入
里间去。武松掀开帘子便问:“王妈妈在家?”那婆子正在磨上扫面,连忙出来应
道:“是谁叫老身?”见是武松,道了万福。武松深深唱喏。婆子道:“武二哥,
且喜,几时回家来了?”武松道:“遇赦回家,昨日才到。一向多累妈妈看家,改
日相谢。”婆子笑嘻嘻道:“武二哥比旧时保养,胡子楂儿也有了,且是好身量,
在外边又学得这般知礼。”一面请他上坐,点茶吃了。武松道:“我有一桩事和妈
妈说。”婆子道:“有甚事?武二哥只顾说。”武松道:“我闻的人说,西门庆已
是死了,我嫂子出来,在你老人家这里居住。敢烦妈妈对嫂子说,他若不嫁人便罢
,若是嫁人,如是迎儿大了,娶得嫂子家去,看管迎儿,早晚招个女婿,一家一计
过日子,庶不教人笑话。”婆子初时还不吐口儿,便道:“他在便在我这里,倒不
知嫁人不嫁人。”次后听见说谢他,便道:“等我慢慢和他说。”

那妇人在帘内听见武松言语,要娶他看管迎儿,又见武松在外出落得长大身材,胖
了,比昔时又会说话儿,旧心不改,心下暗道:“我这段姻缘还落在他手里。”就
等不得王婆叫他,自己出来,向武松道了万福,说道:“既是叔叔还要奴家去看管
迎儿,招女婿成家,可知好哩。”王婆道:“我一件,只如今他家大娘子,要一百
两银子才嫁人。”武松道:“如何要这许多?”王婆道:“西门大官人,当初为他
使了许多,就打恁个银人儿也勾了。”武松道:“不打紧,我既要请嫂嫂家去,就
使一百两也罢。另外破五两银子,与你老人家。”这婆子听见,喜欢的屁滚尿流,
没口说道:“还是武二哥知礼,这几年江湖上见的事多,真是好汉。”妇人听了此
言,走到屋里,又浓浓点了一钟瓜仁泡茶,双手递与武松吃了。婆子问道:“如今
他家要发脱的紧,又有三四个官户人家争着娶,都回阻了,价钱不兑。你这银子,
作速些便好。常言先下米先吃饭,千里姻缘着线牵,休要落在别人手内。”妇人道
:“既要娶奴家,叔叔上紧些。”武松便道:“明日就来兑银子,晚夕请嫂嫂过去
。”那王婆还不信武松有这些银子,胡乱答应去了。

到次日,武松打开皮箱,拿出施恩与知寨刘高那一百两银子来,又另外包了五两碎
银子,走到王婆家,拿天平兑起来。那婆子看见白晃晃摆了一桌银子,口中不言,
心内暗道:“虽是陈敬济许下一百两,上东京去取,不知几时到来。仰着合着,我
见钟不打,去打铸钟?”又见五两谢他,连忙收了。拜了又拜,说道:“还是武二
哥知人甘苦。”武松道:“妈妈收了银子,今日就请嫂嫂过门。”婆子道:“武二
哥,且是好急性。门背后放花儿--你等不到晚了?也待我往他大娘那里交了银子
,才打发他过去。”又道:“你今日帽儿光光,晚夕做个新郎。”那武松紧着心中
不自在,那婆子不知好歹,又奚落他。打发武松出门,自己寻思:“他家大娘只叫
我发脱,又没和我断定价钱,我今胡乱与他一二十两银子就是了,绑着鬼也落他一
半多养家。”就把银凿下二十两银子,往月娘家里交割明白。月娘问:“甚么人家
娶去了?”王婆道:“兔儿沿山跑,还来归旧窝。嫁了他家小叔,还吃旧锅里粥去
了。”月娘听了,暗中跌脚,常言”仇人见仇人,分外眼睛明”,与孟玉楼说:“
往后死在他小叔子手里罢了。那汉子杀人不斩眼,岂肯干休!”

不说月娘家中叹息,却表王婆交了银子到家,下午时,教王潮先把妇人箱笼桌儿送
过去。这武松在家中又早收拾停当,打下酒肉,安排下菜蔬。晚上婆子领妇人过门
,换了孝,带着新(髟狄)髻,身穿红衣服,搭着盖头。进门来,见明间内明亮亮
点着灯烛,重立武大灵牌供养在上面,先有些疑忌,由不的发似人揪,肉如钩搭。
进入门来,到房中,武松分付迎儿把前门上了拴,后门也顶了。王婆见了,说道:
”武二哥,我去罢,家里没人。”武松道:“妈妈请进房里吃盏酒。”武松教迎儿
拿菜蔬摆在桌上,须臾烫上酒来,请妇人和王婆吃酒。那武松也不让,把酒斟上,
一连吃了四五碗酒。婆子见他吃得恶,便道:“武二哥,老身酒勾了,放我去,你
两口儿自在吃罢。”武松道:“妈妈,且休得胡说!我武二有句话问你!”只闻飕
的一声响,向衣底掣出一把二尺长刃薄背厚的朴刀来,一只手笼着刀靶,一只手按
住掩心,便睁圆怪眼,倒竖刚须,说道:“婆子休得吃惊!自古冤有头,债有主,
休推睡里梦里。我哥哥性命都在你身上!”婆子道:“武二哥,夜晚了,酒醉拿刀
弄杖,不是耍处。”武松道:“婆子休胡说,我武二就死也不怕!等我问了这淫妇
,慢慢来问你这老猪狗!若动一动步儿,先吃我五七刀子。”一面回过脸来,看着
妇人骂道:“你这淫妇听着!我的哥哥怎生谋害了?从实说来,我便饶你。”那妇
人道:“叔叔如何冷锅中豆儿炮?好没道理!你哥哥自害心疼病死了,干我甚事?
”说由未了,武松把刀子(忄乞)楂的插在桌子上,用左手揪住妇人云髻,右手匹
胸提住,把桌子一脚踢番,碟儿盏儿都打得粉碎。那妇人能有多大气脉,被这汉子
隔桌子轻轻提将起来,拖出外间灵桌子前。那婆子见势头不好,便去奔前门走,前
门又上了栓。被武松大叉步赶上,揪番在地,用腰间缠带解下来,四手四脚捆住,
如猿猴献果一般,便脱身不得,口中只叫:“都头不消动意,大娘子自做出来,不
干我事。”武松道:“老猪狗,我都知道了,你赖那个?你教西门庆那厮垫发我充
军去,今日我怎生又回家了!西门庆那厮却在那里?你不说时,先剐了这个淫妇,
后杀你这老猪狗!”提起刀来,便望那妇人脸上撇了两撇。

妇人慌忙叫道:“叔叔且饶,放我起来,等我说便了。”武松一提,提起那婆娘,
旋剥净了,跪在灵桌子前。武松喝道:“淫妇快说!”那妇人唬得魂不附体,只得
从实招说,将那时收帘子打了西门庆起,并做衣裳入马通奸,后怎的踢伤武大心窝
,王婆怎地教唆下毒,拨置烧化,又怎的娶到家去,一五一十,从头至尾,说了一
遍。王婆听见,只是暗中叫苦,说:“傻才料,你实说了,却教老身怎的支吾。”
这武松一面就灵前一手揪着妇人,一手浇奠了酒,把纸钱点着,说道:“哥哥,你
阴魂不远,今日武松与你报仇雪恨。”那妇人见势头不好,才待大叫。被武松向炉
内挝了一把香灰,塞在他口,就叫不出来了。然后劈脑揪番在地。那妇人挣扎,把
(髟狄)髻簪环都滚落了。武松恐怕他挣扎,先用油靴只顾踢他肋肢,后用两只手
去摊开他胸脯,说时迟,那时快,把刀子去妇人白馥馥心窝内只一剜,剜了个血窟
窿,那鲜血就冒出来。那妇人就星眸半闪,两只脚只顾登踏。武松口噙着刀子,双
手去斡开他胸脯,扎乞的一声,把心肝五脏生扯下来,血沥沥供养在灵前。后方一
刀割下头来,血流满地。迎儿小女在旁看见,唬的只掩了脸。武松这汉子端的好狠
也。可怜这妇人,正是三寸气在千般用,一日无常万事休。亡年三十二岁。但见:

手到处青春丧命,刀落时红粉亡身。七魄悠悠,已赴森罗殿上;三魂
渺渺,应归枉成城中。好似初春大雪压折金钱柳,腊月狂风吹折玉梅
花。这妇人娇媚不知归何处,芳魂今夜落谁家?
古人有诗一首,单悼金莲死的好苦也:
堪悼金莲诚可怜,衣裳脱去跪灵前。
谁知武二持刀杀,只道西门绑腿顽。
往事看嗟一场梦,今身不值半文钱。
世间一命还一命,报应分明在眼前。

武松杀了妇人,那婆子便叫:“杀人了!”武松听见他叫,向前一刀,也割下头来
。拖过尸首。一边将妇人心肝五脏,用刀插在后楼房檐下。

那时有初更时分,倒扣迎儿在屋里。迎儿道:“叔叔,我害怕!”武松道:“孩儿
,我顾不得你了。”武松跳过王婆家来,还要杀他儿子王潮。不想王潮合当不该死
,听见他娘这边叫,就知武松行凶,推前门不开,叫后门也不应,慌的走去街上叫
保甲。那两邻明知武松凶恶,谁敢向前。武松跳过墙来,到王婆房内,只见点着灯
,房内一人也没有。一面打开王婆箱笼,就把他衣服撇了一地。那一百两银子止交
与吴月娘二十两,还剩了八十五两,并些钗环首饰,武松都包裹了。提了朴刀,越
后墙,赶五更挨出城门,投十字坡张青夫妇那里躲住,做了头佗,上梁山为盗去了
。正是:
平生不作绉眉事,世上应无切齿人。