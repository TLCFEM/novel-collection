\chapter{吴月娘识破奸情~春梅姐不垂别泪}

诗曰:

情若连环总不解,无端招引旁人怪。好事多磨成又败,应难捱,
相冷眼谁揪采?

镇日愁眉和敛黛,阑干倚遍无聊赖。但愿五湖明月在,权宁耐,
终须还了鸳鸯债。

话说月娘取路来家,不题。单表金莲在家,和陈敬济两个就如鸡儿赶蛋相似,缠做
一处。一日,金莲眉黛低垂,腰肢宽大,终日恹恹思睡,茶饭懒咽,教敬济到房中
说:“奴有件事告你说,这两日眼皮儿懒待开,腰肢儿渐渐大,肚腹中扑扑跳,茶
饭儿怕待吃,身子好生沉困。有你爹在时,我求薛姑子符药衣胞那等安胎,白没见
个踪影。今日他没了,和你相交多少时儿,便有了孩子。我从三月内洗身上,今方
六个月,已有半肚身孕。往常时我排磕人,今日却轮到我头上。你休推睡里梦里,
趁你大娘未来家,那里讨贴坠胎的药,趁早打落了这胎气。不然,弄出个怪物来,
我就寻了无常罢了,再休想抬头见人。”敬济听了,便道:“咱家铺中诸样药都有
,倒不知那几样儿坠胎,又没方修治。你放心,不打紧处,大街坊胡太医,他大小
方脉,妇人科,都善治,常在咱家看病。等我问他那里赎取两贴,与你下胎便了。
”妇人道:“好哥哥,你上紧快去,救奴之命。”

这陈敬济包了三钱银子,径到胡太医家来。胡太医正在家,出来相见声喏,认的敬
济是西门大官人女婿,让坐说:“一向稀面,动问到舍有何见教?”敬济道:“别
无干渎。”向袖中取出白金三星:“充药资之礼,敢求下胎良剂一二贴,足见盛情
。”胡太医道:“天地之间,以好生为德。人家十个九个只要安胎的药,你如何倒
要打胎?没有,没有。”敬济见他掣肘,又添了二钱药资,说:“你休管他,各人
家自有用处。此妇女子生落不顺,情愿下胎。”这胡太医接了银子,说道:“不打
紧,我与你一服红花一扫光。吃下去,如人行五里,其胎自落矣。”于是取了两贴
,付与敬济。敬济得了药,作辞胡太医,到家递与妇人。妇人到晚夕,煎汤吃下去
,登时满肚里生疼,睡在炕上,教春梅按在肚上只情揉揣。可霎作怪,须臾坐净桶
,把孩子打下来了。只说身上来,令秋菊搅草纸倒在毛司里。次日,掏坑的汉子挑
出去,一个白胖的孩子儿。常言好事不出门,恶事传千里,不消几日,家中大小都
知金莲养女婿,偷出私孩子来了。

且说吴月娘有日来家。往回去了半个月光景,来时正值十月天气。家中大小接着,
知前拜罢,就对玉楼众姐妹,把岱岳庙中的事,从头告诉一遍,因大哭一场。合家
大小都来参见了。月娘见奶子抱孝哥儿到跟前,子母相会在一处。烧纸,置酒管待
吴大舅回家。晚夕,众姊妹与月娘接风,俱不在话下。

到第二日,月娘因路上风霜跋涉,着了辛苦,又吃了惊怕,身上疼痛沉困,整不好
了两三日。那秋菊在家,把金莲、敬济两人干的勾当,听的满耳满心,要告月娘说
。走到上房门首,又被小玉哕骂在脸上,大耳刮子打在他脸上,骂道:“贼说舌的
奴才,趁早与我走!俺奶奶远路来家,身子不快活,还未起来。气了他,倒值了多
的。”骂的秋菊忍气吞声,喏喏而退。

一日,也是合当有事,敬济进来寻衣服,妇人和他又在玩花楼上两个做得好。被秋
菊走到后边,叫了月娘来看,说道;”奴婢两番三次告大娘说不信。娘不在,两个
在家明睡到夜,夜睡到明,偷出私孩子来。与春梅两个都打成一家。今日两人又在
楼上干歹事,不是奴婢说谎,娘快些瞧去。”月娘急忙走到前边,两个正干的好,
还未下楼。春梅在房中,忽然看见,连忙上楼去说:“不好了,大娘来了。”两人
忙了手脚,没处躲避。敬济只得拿衣服下楼往外走,被月娘撞见喝骂了几句,说:
”小孩儿家没记性,有要没紧进来撞甚么?”敬济道:“铺子内人等着,没人寻衣
服。”月娘道:“我那等分付你,教小厮进来取,如何又进来寡妇房里做甚么?没
廉耻!”几句骂得敬济往外金命水命,走投无命。妇人羞的半日不敢下来。然后下
来,被月娘尽力数说了一顿,说道:“六姐,今后再休这般没廉耻!你我如今是寡
妇,比不得有汉子,香喷喷在家里。瓶儿罐儿有耳朵,有要没紧和这小厮缠甚么!
教奴才们背地排说的碜死了!常言道,男儿没性,寸铁无钢;女人无性,烂如麻糖
。其身正,不令而行;其身不正,虽令不行。你若长俊正条,肯教奴才排说?他在
我跟前说了几遍,我不信;今日亲眼看见,说不的了。我今日说过,你要自家立志
,替汉子争气。像我进香去,被强人逼勒,若是不正气的,也来不到家了。”金莲
吃月娘数说,羞的脸上红一块白一块,口里说一千个没有,只说:“我在楼上烧香
,陈姐夫自去那边寻衣裳,谁和他说甚话来!”当日月娘乱了一回,归后边去了。

晚夕,西门大姐在房内又骂敬济:“贼囚根子,敢说又没真赃实犯拿住你?你还那
等嘴巴巴的!今日两个又在楼上做甚么?说不的了!两个弄的好碜儿,只把我合在
缸底下一般。那淫妇要了我汉子,还在我面前拿话儿拴缚人,毛司里砖儿--又臭
又硬,恰似降伏着那个一般。他便羊角葱靠南墙--老辣已定。你还要在这里雌饭
吃!”敬济骂道:“淫妇,你家收着我银子,我雌你家饭吃?”使性子往前边来了
。

自此已后,敬济只在前边,无事不敢进入后边来。取东取西,只是玳安、平安两个
往楼上取去。每日饭食,晌午还不拿出来,把傅伙计饿的只拿钱街上烫面吃。正是
龙斗虎伤,苦了小獐。各处门户,日头半天就关了。由是与金莲两个恩情又间阻了
。敬济那边陈宅的房子,一向教他母舅张团练看守居住。张团练革任在家闲住,敬
济早晚往那里吃饭去,月娘也不追问。

两个隔别,约一月不得会面。妇人独在那边,挨一日似三秋,过一宵如半夏,怎禁
这空房寂静,欲火如蒸,要见他一面,难上之难。两下音信不通,这敬济无门可入
。忽一日见薛嫂儿打门首过,有心要托他寄一纸柬儿与金莲,诉其间阻之事,表此
肺腑之情。一日,推门外讨帐,骑头口径到薛嫂家,拴了驴儿,掀帘便问:“薛妈
在家?”有他儿子薛纪媳妇儿金大姐抱孩子在炕上,伴着人家卖的两个使女,听见
有人叫薛妈,出来问:“是谁?”敬济道:“是我。”问:“薛妈在家不在?”金
大姐道:“姑夫请家来坐,俺妈往人家兑了头面,讨银子去了。有甚话说,使人叫
去。”连忙点茶与敬济吃。坐不多时,只见薛嫂儿来了,与敬济道了万福,说:“
姑夫那阵风儿吹来我家!”叫金大姐:“倒茶与姑夫吃。”金大姐道:“刚才吃了
茶了。”敬济道:“无事不来。如此这般,与我五娘勾搭日久,今被秋菊丫头戳舌
,把俺两个姻缘拆散。大娘与大姐是疏淡我。我与六姐拆散不开,二人离别日久,
音信不通,欲稍寄数字进去与他。无人得到内里,须央及你,如此这般通个消息。
”向袖中取出一两银子来:“这些微礼,权与薛妈买茶吃。”那薛嫂一闻其言,拍
手打掌笑起来,说道:“谁家女婿戏丈母?世间那里有此事!姑夫,你实对我说,
端的你怎么得手来?”敬济道:“薛嫂禁声,且休取笑。我有这柬贴封好在此,好
歹明日替我送与他去。”薛嫂一手接了说:“你大娘从进香回来,我还没看他去,
两当一节,我去走走。”敬济道:“我在那里讨你信?”薛嫂道:“往铺子里寻你
回话。”说毕,敬济骑头口来家。

次日,薛嫂提着花箱儿,先进西门庆家上房看月娘。坐了一回,又到孟玉楼房中,
然后才到金莲这边。金莲正放桌儿吃粥。春梅见妇人闷闷不乐,说道:“娘,你老
人家也少要忧心。是非有无,随人说去。如今爹也没了,大娘他养不出个墓生儿来
,莫不是也来路不明?他也难管你我暗地的事。你把心放开,料天塌了还有撑天大
汉哩。人生在世,且风流了一日是一日。”于是筛上酒来,递一钟与妇人说:“娘
且吃一杯儿暖酒,解解愁闷。”因见阶下两只犬儿交恋在一处,说道:“畜生尚有
如此之乐,何况人而反不如此乎?”正饮酒,只见薛嫂儿来到,向金莲道个万福,
又与春梅拜了拜,笑道:“你娘儿们好受用。”因观二犬恋在一处,又笑道:“你
家好祥瑞,你娘儿每看着怎不解闷!”妇人道:“那阵风儿今日刮你来,怎的一向
不来走走?”一面让薛嫂坐。薛嫂儿道:“我整日干的不知甚么,只是不得闲。大
娘顶上进了香来,也不曾看的他,刚才好不怪我。西房三娘也在跟前,留了我两对
翠花,一对大翠围发,好快性,就称了八钱银子与我。只是后边雪姑娘,从八月里
要了我两对线花儿,该二钱银子,白不与我。好悭吝的人!我对你说,怎的不见你
老人家?”妇人道:“我这两日身中有些不自在,不曾出去走动。”春梅一面筛了
一钟酒,递与薛嫂儿。薛嫂忙又道万福,说:“我进门就吃酒。”妇人道:“你到
明日养个好娃娃。”薛嫂儿道:“我养不的,俺家儿子媳妇儿金大姐,倒新添了个
娃儿,才两个月来。”又道:“你老人家没了爹,终日这般冷清清了。”妇人道:
”说不得,有他在好了,如今弄的俺娘儿们一折一磨的。不瞒老薛说,如今俺家中
人多舌头多,他大娘自从有了这孩儿,把心肠儿也改变了,姊妹不似那咱亲热了。
这两日一来我心里不自在,二来因些闲话,没曾往那边去。”春梅道:“都是俺房
里秋菊这奴才,大娘不在,霹空架了俺娘一篇是非,把我也扯在里面,好不乱哩。
”薛嫂道:“就是房里使的那大姐?他怎的倒弄主子?自古穿青衣,抱黑柱。这个
使不的。”妇人使春梅:“你瞧瞧那奴才,只怕他又来听。”春梅道:“他在厨下
拣米哩!这破包篓奴才,在这屋就是走水的槽,单管屋里事儿往外学舌。”薛嫂道
:“这里没人,咱娘儿每说话。昨日陈姐夫到我那里,如此这般告诉我,干净是他
戳犯你每的事儿了。陈姐夫说,他大娘数说了他,各处门户都紧了,不许他进来取
衣裳拿药材了。把大姐搬进东厢房里住。每日晌午还不拿饭出去与他吃,饿的他只
往他母舅张老爹那里吃去。一个亲女婿不托他,倒托小厮,有这个道理?他有好一
向没得见你老人家,巴巴央及我,稍了个柬儿,多多拜上你老人家,少要心焦,左
右爹也是没了,爽利放倒身,大做一做,怕怎的?点根香怕出烟儿;放把火,倒也
罢了。”于是取出敬济封的柬贴儿递与妇人。拆开观看,别无甚话,上写《红绣鞋
》一词:

袄庙火烧皮肉,蓝桥水淹过咽喉,紧按纳风声满南州。洗净了终是染
污,成就了倒是风流,不怎么也是有。六姐妆次敬济百拜上

妇人看毕,收入袖中。薛嫂道:“他教你回个记色与他,或写几个字儿稍了去,方
信我送的有个下落。”妇人教春梅陪着薛嫂吃酒,他进入里间,半晌拿了一方白绫
帕,一个金戒指儿。帕儿上又写了一首词儿,叙其相思契阔之怀。写完,封得停当
,走出来交与薛嫂,便说:“你上覆他,教他休要使性儿,往他母舅张家那里吃饭
,惹他张舅蜃齿,说你在丈人家做买卖,却来我家吃饭。显得俺们都是没生活的一
般,教他张舅怪。或是未有饭吃,教他铺子里拿钱买些点心和伙计吃便了。你使性
儿不进来,和谁鳖气哩!却相是贼人胆儿虚一般。”薛嫂道:“等我对他说。”妇
人又与了薛嫂五钱银子。

作别出门,来到前边铺子里,寻见敬济。两个走到僻静处说话,把封的物事递与他
:“五娘说,教你休使性儿赌鳖气,教你常进来走走,休往你张舅家吃饭去,惹人
家怪。”因拿出五钱银子与他瞧:“此是里面与我的,漏眼不藏丝,久后你两个愁
不会在一答里?对出来,我脸放在那里?”敬济道:“老薛多有累你。”深深与他
唱喏。那薛嫂走了两步,又回来说:“我险些儿忘了一件事,刚才我出来,大娘又
使丫头绣春叫我进去,叫我晚上来领春梅,要打发卖他。说他与你们做牵头,和他
娘通同养汉。”敬济道:“薛妈,你且领在家。我改日到你家见他一面,有话问他
。”那薛嫂说毕,回家去了。

果然到晚夕月上的时分,走来领春梅。到月娘房中,月娘开口说:“那咱原是你手
里十六两银子买的,你如今拿十六两银子来就是了。”分付小玉:“你看着,到前
边收拾了,教他罄身儿出去,休要带出衣裳去了。”那薛嫂儿到前边,向妇人如此
这般:“他大娘教我领春梅姐来了。对我说,他与你老人家通同作弊,偷养汉子,
不管长短,只问我要原价。”妇人听见说领卖春梅,就睁了眼,半日说不出话来,
不觉满眼落泪,叫道:“薛嫂儿,你看我娘儿两个没汉子的,好苦也!今日他死了
多少时儿,就打发我身边人。他大娘这般没人心仁义,自恃他身边养了个尿胞种,
就把人(足丽)到泥里。李瓶儿孩子周半还死了哩,花麻痘疹未出,知道天怎么算
计,就心高遮了太阳!”薛嫂道:“春梅姐说,爹在日曾收用过他。”妇人道:“
收用过二字儿!死鬼把他当心肝肺肠儿一般看待!说一句,听十句,要一奉十,正
经成房立纪老婆且打靠后。他要打那个小厮十棍儿,他爹不敢打五棍儿。”薛嫂道
:“可又来,大娘差了!爹收用的恁个出色姐儿,打发他,箱笼儿也不与,又不许
带一件衣服儿,只教他罄身儿出去,邻舍也不好看的。”妇人道:“他对你说,休
教带出衣裳去?”薛嫂道:“大娘分付,小玉姐便来。教他看着,休教带衣裳出去
。”那春梅在旁,听见打发他,一点眼泪也没有。见妇人哭,说道:“娘你哭怎的
?奴去了,你耐心儿过,休要思虑坏了你。你思虑出病来,没人知你疼热。等奴出
去,不与衣裳也罢,自古好男不吃分时饭,好女不穿嫁时衣。”正说着,只见小玉
进来,说道:“五娘,你信我奶奶,倒三颠四的。小大姐扶持你老人家一场,瞒上
不瞒下,你老人拿出他箱子来,拣上色的包与他两套,教薛嫂儿替他拿了去,做个
一念儿,也是他番身一场。”妇人道:“好姐姐,你到有点仁义。”小玉道:“你
看,谁人保得常无事!虾蟆、促织儿,都是一锹土上人。兔死狐悲,物伤其类。”
一面拿出春梅箱子来,是戴的汗巾儿、翠簪儿,都教他拿去。妇人拣了两套上色罗
段衣服鞋脚,包了一大包,妇人梯己与了他几件钗梳簪坠戒指,小玉也头上拔下两
根簪子来递与春梅。余者珠子缨络、银丝云髻、遍地金妆花裙袄,一件儿没动,都
抬到后边去了。春梅当下拜辞妇人、小玉,洒泪而别。临出门,妇人还要他拜辞拜
辞月娘众人,只见小玉摇手儿。这春梅跟定薛嫂,头也不回,扬长决裂,出大门去
了。

小玉和妇人送出大门回来。小玉到上房回大娘,只说:“罄身子去了,衣服都留下
,没与他。”这金莲归到房中,往常有春梅,娘儿两个相亲相热,说知心话儿,今
日他去了,丢得屋里冷冷落落,甚是孤凄,不觉放声大哭。有诗为证:
耳畔言犹在,于今恩爱分。
房中人不见,无语自消魂。