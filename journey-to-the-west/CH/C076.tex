\chapter{心神居舍魔归性~木母同降怪体真}

话表孙大圣在老魔肚里支吾一会,那魔头倒在尘埃,无声无气,若不言语,想
是死了,却又把手放放。魔头回过气来,叫一声:“大慈大悲齐天大圣菩萨!”行者
听见道:“儿子,莫废工夫,省几个字儿,只叫孙外公罢。”那妖魔惜命,真个叫:
“外公,外公!是我的不是了!一差二误吞了你,你如今却反害我。万望大圣慈悲,
可怜蝼蚁贪生之意,饶了我命,愿送你师父过山也。”

大圣虽英雄,甚为唐僧进步。他见妖魔哀告,好奉承的人,也就回了善念,叫
道:“妖怪,我饶你,你怎么送我师父?”老魔道:“我这里也没甚么金银珠翠、玛
瑙珊瑚、琉璃琥珀、玳瑁珍奇之宝相送;我兄弟三个,抬一乘香藤轿儿,把你师父
送过此山。”行者笑道:“既是抬轿相送,强如要宝。你张开口,我出来。”那魔头
真个就张开口。那三魔走近前,悄悄的对老魔道:“大哥,等他出来时,把口往下
一咬,将猴儿嚼碎,咽下肚,却不得磨害你了。”

原来行者在里面听得,便不先出去。却把金箍棒伸出,试他一试。那怪果往下
一口,喳的一声,把个门牙都迸碎了。行者抽回棒道:“好妖怪!我倒饶你性命出
来,你反咬我,要害我命!我不出来,活活的只弄杀你。不出来,不出来!”老魔报
怨三魔道:“兄弟,你是自家人弄自家人了。且是请他出来好了,你却教我咬他。
他倒不曾咬着,却迸得我牙龈疼痛。这是怎么起的!”

三魔见老魔怪他,他又作个激将法,厉声高叫道:“孙行者,闻你名如轰雷贯
耳,说你在南天门外施威,灵霄殿下逞势;如今在西天路上降妖缚怪,原来是个小
辈的猴头!”行者道:“我何为小辈?”三怪道:“‘好汉千里客,万里去传名’。你
出来,我与你赌斗,才是好汉;怎么在人肚里做勾当!非小辈而何?”行者闻言,
心中暗想道:“是,是,是!我若如今扯断他肠,破他肝,弄杀这怪,有何难哉?
但真是坏了我的名头。也罢,也罢!你张口,我出来与你比并。但只是你这洞口窄
逼,不好使家火,须往宽处去。”三魔闻说,即点大小怪,前前后后,有三万多精,
都执着精锐器械,出洞摆开一个三才阵势,专等行者出口,一齐上阵。那二怪搀着
老魔,径至门外,叫道:“孙行者,好汉出来!此间有战场,好斗!”

大圣在他肚里,闻得外面鸦鸣鹊噪,鹤唳风声,知道是宽阔之处。却想着:“我
不出去,是失信与他;若出去,这妖精人面兽心:先时说送我师父,哄我出来咬我,
今又调兵在此。也罢,也罢!与他个两全其美:出去便出去,还与他肚里生下一个
根儿。”即转手,将尾上毫毛拔了一根,吹口仙气,叫“变”!即变一条绳儿,只有
头发粗细,倒有四十丈长短。那绳儿理出去,见风就长粗了。把一头拴着妖怪的心
肝系上,打做个活扣儿。那扣儿不扯不紧,扯紧就痛。却拿着一头,笑道:“这一
出去,他送我师父便罢;如若不送,乱动刀兵,我也没工夫与他打,只消扯此绳儿,
就如我在肚里一般!”又将身子变得小小的,往外爬;爬到咽喉之下,见妖精大张
着方口,上下钢牙,排如利刃,忽思量道:“不好,不好!若从口里出去扯这绳儿,
他怕疼,往下一嚼,却不咬断了?我打他没牙齿的所在出去。”好大圣,理着绳儿,
从他那上腭子往前爬,爬到他鼻孔里。那老魔鼻子发痒,“阿”的一声,打了个
喷嚏,却迸出行者。

行者见了风,把腰躬一躬,就长了有三丈长短,一只手扯着绳儿,一只手拿着
铁棒。那魔头不知好歹,见他出来了,就举钢刀,劈脸来砍。这大圣一只手使铁棒
相迎。又见那二怪使枪,三怪使戟,没头没脸的乱上。大圣放松了绳,收了铁棒,
急纵身驾云走了。原来怕那伙小妖围绕,不好干事。他却跳出营外,去那空阔山头
上,落下云,双手把绳尽力一扯,老魔心里才疼。他害疼,往上一挣,大圣复往下
一扯。众小妖远远看见,齐声高叫道:“大王,莫惹他!让他去罢!这猴儿不按时景:
清明还未到,他却那里放风筝也!”大圣闻言,着力气蹬了一蹬,那老魔从空中,
拍刺刺,似纺车儿一般,跌落尘埃。就把那山坡下死硬的黄土跌做个二尺浅深之坑。

慌得那二怪、三怪,一齐按下云头,上前拿住绳儿,跪在坡下,哀告道:“大
圣啊,只说你是个宽洪海量之仙,谁知是个鼠腹蜗肠之辈。实实的哄你出来,与你
见阵,不期你在我家兄心上拴了一根绳子!”行者笑道:“你这伙泼魔,十分无礼!
前番哄我出去便就咬我,这番哄我出来,却又摆阵敌我。似这几万妖兵,战我一个,
理上也不通。扯了去,扯了去见我师父!”那怪一齐叩头道:“大圣慈悲,饶我性命,
愿送老师父过山。”行者笑道:“你要性命,只消拿刀把绳子割断罢了。”老魔道:“爷
爷呀,割断外边的,这里边的拴在心上,喉咙里又的恶心,怎生是好?”行者
道:“既如此,张开口,等我再进去解出绳来。”老魔慌了道:“这一进去,又不肯
出来,却难也,却难也!”行者道:“我有本事外边就可以解得里面绳头也。解了可
实实的送我师父么?”老魔道:“但解就送,决不敢打诳语。”大圣审得是实,即便
将身一抖,收了毫毛,那怪的心就不疼了。这是孙大圣掩样的法儿,使毫毛拴着他
的心;收了毫毛,所以就不害疼也。三个妖纵身而起,谢道:“大圣请回,上复唐
僧,收拾下行李,我们就抬轿来送。”众怪偃干戈,尽皆归洞。

大圣收绳子,径转山东,远远的看见唐僧睡在地下打滚痛哭;猪八戒与沙僧解
了包袱,将行李搭分儿,在那里分哩。行者暗暗嗟叹道:“不消讲了。这定是八戒
对师父说我被妖精吃了,师父舍不得我,痛哭,那呆子却分东西散火哩。咦!不知
可是此意,且等我叫他一声看。”

落下云头,叫道:“师父!”沙僧听见,报怨八戒道:“你是个‘棺材座子,专
一害人’!师兄不曾死,你却说他死了,在这里干这个勾当!那里不叫将来了?”八
戒道:“我分明看见他被妖精一口吞了。想是日辰不好,那猴子来显魂哩。”行者到
跟前,一把挝住八戒脸,一个巴掌打了个踉跄,道:“夯货!我显甚么魂?”呆子侮
着脸道:“哥哥,你实是那怪吃了,你……你怎么又活了?”行者道:“像你这个不
济事的脓包!他吃了我,我就抓他肠,捏他肺,又把这条绳儿穿住他的心,扯他疼
痛难禁,一个个叩头哀告,我才饶了他性命。如今抬轿来送我师父过山也。”那三
藏闻言,一骨鲁爬起来,对行者躬身道:“徒弟啊,累杀你了!若信悟能之言,我已
绝矣!”行者轮拳打着八戒骂道:“这个馕糠的呆子,十分懈怠,甚不成人!师父,
你切莫恼。那怪就来送你也。”沙僧也甚生惭愧。连忙遮掩,收拾行李,扣背马匹,
都在途中等候不题。

却说三个魔头,帅群精回洞。二怪道:“哥哥,我只道是个九头八尾的孙行者,
原来是恁的个小小猴儿!你不该吞他,只与他斗时,他那里斗得过你我!洞里这几万
妖精,吐唾沫也可杀他。你却将他吞在肚里,他便弄起法来,教你受苦,怎么敢
与他比较!才自说送唐僧,都是假意,实为兄长性命要紧,所以哄他出来。决不送
他!”老魔道:“贤弟不送之故,何也?”二怪道:“你与我三千小妖,摆开阵势,
我有本事拿住这个猴头!”老魔道:“莫说三千,凭你起老营去;只是拿住他,便大
家有功。”

那二魔即点三千小妖,径到大路旁摆开,着一个蓝旗手往来传报,教:“孙行
者!赶早出来,与我二大王爷爷交战!”八戒听见,笑道:“哥啊,常言道:‘说谎不
瞒当乡人。’就来弄虚头,捣鬼!怎么说降了妖精,就抬轿来送师父,却又来叫战,
何也?”行者道:“老怪已被我降了,不敢出头,闻着个‘孙’字儿,也害头疼。
这定是二妖魔不伏气送我们,故此叫战。我道兄弟,这妖精有弟兄三个,这般义气;
我弟兄也是三个,就没些义气。我已降了大魔,二魔出来,你就与他战战,未为不
可。”八戒道:“怕他怎的!等我去打他一仗来!”行者道:“要去便去罢。”八戒笑道:
“哥啊,去便去,你把那绳儿借与我使使。”行者道:“你要怎的?你又没本事钻在
肚里,你又没本事拴在他心上,要他何用?”八戒道:“我要扣在这腰间,做个救
命索。你与沙僧扯住后手,放我出去,与他交战。估着赢了他,你便放松,我把他
拿住;若是输与他,你把我扯回来,莫教他拉了去。”真个行者暗笑道:“也是捉弄
呆子一番!”就把绳儿扣在他腰里,撮弄他出战。

那呆子举钉钯跑上山崖,叫道:“妖精,出来!与你猪祖宗打来!”那蓝旗手急
报道:“大王,有一个长嘴大耳朵的和尚来了!”二怪即出营,见了八戒,更不打话,
挺枪劈面刺来。这呆子举钯上前迎住。他两个在山坡前搭上手,斗不上七八回合,
呆子手软,架不得妖魔,急回头叫:“师兄,不好了!扯扯救命索,扯扯救命索!”
这壁厢大圣闻言,转把绳子放松了,抛将去。那呆子败了阵,往后就跑。原来那绳
子拖着走,还不觉;转回来,因松了,倒有些绊脚,自家绊倒了一跌,爬起来又一
跌。始初还跌个踵,后面就跌了个嘴抢地。被妖精赶上,开鼻子,就如蛟龙一
般,把八戒一鼻子卷住,得胜回洞。众妖凯歌齐唱,一拥而归。

这坡下三藏看见,又恼行者道:“悟空,怪不得悟能咒你死哩!原来你兄弟全无
相亲相爱之意,专怀相嫉相妒之心!他那般说,教你扯扯救命索,你怎么不扯,还
将索子丢去?如今教他被害,却如之何?”行者笑道:“师父也忒护短,忒偏心!罢
了,像老孙拿去时,你略不挂念,左右是舍命之材;这呆子才自遭擒,你就怪我。
也教他受些苦恼,方见取经之难。”三藏道:“徒弟啊,你去,我岂不挂念?想着你
会变化,断然不至伤身。那呆子生得狼,又不会腾那,这一去,少吉多凶。你还
去救他一救。”行者道:“师父不得报怨,等我去救他一救。”急纵身,赶上山,暗
中恨道:“这呆子咒我死,且莫与他个快活,且跟去看那妖精怎么摆布他,等他受
些罪,再去救他。”即捻诀念起真言,摇身一变,即变做个虫,飞将去,钉在
八戒耳朵根上,同那妖精到了洞里。

二魔帅三千小怪,大吹大打的,至洞口屯下。自将八戒拿入里边道:“哥哥,
我拿了一个来也。”老怪道:“拿来我看。”他把鼻子放松,下八戒道:“这不是?”
老怪道:“这厮没用。”八戒闻言道:“大王,没用的放出去,寻那有用的捉来吧。”
三怪道:“虽是没用,也是唐僧的徒弟猪八戒。且捆了,送在后边池塘里浸着。待
浸退了毛,破开肚子,使盐腌了晒干,等天阴下酒。”八戒大惊道:“罢了,罢了!
撞见那贩腌的妖怪也!”众怪一齐下手,把呆子四马攒蹄捆住,扛扛抬抬,送至池
塘边,往中间一推,尽皆转去。

大圣却飞起来看处,那呆子四肢朝上,掘着嘴,半浮半沉,嘴里呼呼的,着然
好笑,倒像八九月经霜落了子儿的一个大黑莲蓬。大圣见他那嘴脸,又恨他,又怜
他,说道:“怎的好么?他也是龙华会上的一个人。但只恨他动不动分行李散火,又
要撺掇师父念紧箍咒咒我。我前日曾闻得沙僧说,他攒了些私房,不知可有否。等
我且吓他一吓看。”

好大圣,飞近他耳边,假捏声音,叫道:“猪悟能!猪悟能!”八戒慌了道:“晦
气呀!我这悟能是观世音菩萨起的,自跟了唐僧,又呼做八戒,此间怎么有人知道
我叫做悟能?”呆子忍不住问道:“是那个叫我的法名?”行者道:“是我。”呆子
道:“你是那个?”行者道:“我是勾司人。”那呆子慌了道:“长官,你是那里来
的?”行者道:“我是五阎王差来勾你的。”呆子道:“长官,你且回去,上复五阎
王,他与我师兄孙悟空交得甚好,教他让我一日儿,明日来勾罢。”行者道:“胡说!
‘阎王注定三更死,谁敢留人到四更’!趁早跟我去,免得套上绳子扯拉!”呆子道:
“长官,那里不是方便,看我这般嘴脸,还想活哩。死是一定死,只等一日,这妖
精连我师父们都拿来,会一会,就都了帐也。”行者暗笑道:“也罢,我这批上有三
十个人,都在这中前后,等我拘将来就你,便有一日耽阁。你可有盘缠,把些儿我
去。”八戒道:“可怜啊!出家人那里有甚么盘缠?”行者道:“若无盘缠,索了去,
跟着我走!”

呆子慌了道:“长官不要索。我晓得你这绳儿叫做‘追命绳’,索上就要断气。
有,有,有!有便有些儿,只是不多。”行者道:“在那里?快拿出来!”八戒道:“可
怜,可怜!我自做了和尚,到如今,有些善信的人家斋僧,见我食肠大,衬钱比他
们略多些儿,我拿了攒在这里,零零碎碎有五钱银子;因不好收拾,前者到城中,
央了个银匠煎在一处,他又没天理,偷了我几分,只得四钱六分一块儿。你拿了去
罢。”行者暗笑道:“这呆子裤子也没得穿,却藏在何处?”“咄!你银子在那里?”
八戒道:“在我左耳朵眼儿里着哩。我捆了拿不得,你自家拿了去罢。”

行者闻言,即伸手在耳朵窍中摸出,真个是块马鞍儿银子,足有四钱五六分重;
拿在手里,忍不住哈哈的一声大笑。那呆子认是行者声音,在水里乱骂道:“天杀
的弼马温!到这们苦处,还来打诈财物哩!”行者又笑道:“我把你这馕糟的!老孙保
师父,不知受了多少苦难,你到攒下私房!”八戒道:“嘴脸!这是甚么私房?都是牙
齿上刮下来的,我不舍得买了嘴吃,留了买匹布儿做件衣服,你却吓了我的。还分
些儿与我。”行者道:“半分也没得与你!”八戒骂道:“买命钱让与你吧,好道也救
我出去是。”行者道:“莫发急,等我救你。”将银子藏了,即现原身,掣铁棒,把
呆子划拢,用手提着脚,扯上来,解了绳。八戒跳起来,脱下衣裳,整干了水,抖
一抖,潮漉漉的披在身上,道:“哥哥,开后门走了罢。”行者道:“后门里走,可
是个长进的?还打前门上去。”八戒道:“我的脚捆麻了,跑不动。”行者道:“快跟
我来。”

好大圣,把铁棒一路丢开解数,打将出去。那呆子忍着麻,只得跟定他。只看
见二门下靠着的是他的钉钯,走上前,推开小妖,捞过来往前乱筑;与行者打出三
四层门,不知打杀了多少小妖。

那老魔听见,对二魔道:“拿得好人!拿得好人!你看孙行者劫了猪八戒,门上
打伤小妖也!”那二魔急纵身,绰枪在手,赶出门来,应声骂道:“泼猢狲!这般无
礼,怎敢渺视我等!”大圣听得,即应声站下。那怪物不容讲,使枪便刺。行者正
是会家不忙,掣铁棒,劈面相迎。他两个在洞门外,这一场好杀:

黄牙老象变人形,义结狮王为弟兄。因为大魔来说合,同心计算吃唐僧。齐天
大圣神通广,辅正除邪要灭精。八戒无能遭毒手,悟空拯救出门行。妖王赶上施英
猛,枪棒交加各显能。那一个枪来好似穿林蟒,这一个棒起犹如出海龙。龙出海门
云霭霭,蟒穿林树雾腾腾。算来都为唐和尚,恨苦相持太没情。
那八戒见大圣与妖精交战,他在山嘴上竖着钉钯,不来帮打,只管呆呆的看着。那
妖精见行者棒重,满身解数,全无破绽,就把枪架住。开鼻子,要来卷他。行者
知道他的勾当,双手把金箍棒横起来,往上一举,被妖精一鼻子卷住腰胯,不曾卷
手。你看他两只手在妖精鼻头上丢花棒儿耍子。

八戒见了,捶胸道:“咦!那妖怪晦气呀!卷我这夯的,连手都卷住了,不能得
动;卷那们滑的,倒不卷手。他那两只手拿着棒,只消往鼻里一搠,那孔子里害疼
流涕,怎能卷得他住?”行者原无此意,倒是八戒教了他。他就把棒幌一幌,小如
鸡子,长有丈余,真个往他鼻孔里一搠。那妖精害怕,沙的一声,把鼻子放,被
行者转手过来,一把挝住,用气力往前一拉,那妖精护疼,随着手,举步跟来。八
戒方才敢近,拿钉钯望妖精胯子上乱筑。行者道:“不好,不好!那钯齿儿尖,恐筑
破皮,淌出血来,师父看见,又说我们伤生,只调柄子来打罢。”

真个呆子举钯柄,走一步,打一下,行者牵着鼻子,就似两个象奴,牵至坡下。
只见三藏凝睛盼望,见他两个嚷嚷闹闹而来,即唤:“悟净,你看悟空牵的是甚么?”
沙僧见了,笑道:“师父,大师兄把妖精揪着鼻子拉来,真爱杀人也!”三藏道:“善
哉,善哉,那般大个妖精!那般长个鼻子!你且问他:他若喜喜欢欢送我等过山呵,
饶了他,莫伤他性命。”

沙僧急纵前迎着,高声叫道:“师父说:那怪果送师父过山,教不要伤他命哩。”
那怪闻说,连忙跪下,口里呜呜的答应。原来被行者揪着鼻子,捏了,就如重伤
风一般。叫道:“唐老爷,若肯饶命,即便抬轿相送。”行者道:“我师徒俱是善胜
之人,依你言,且饶你命。快抬轿来。如再变卦,拿住决不再饶!”那怪得脱手,
磕头而去。行者同八戒见唐僧,备言前事。八戒惭愧不胜,在坡前晾晒衣服,等候
不题。

那二魔战战兢兢回洞,未到时,已有小妖报知老魔、三魔,说二魔被行者揪着
鼻子拉去。老魔悚惧,与三魔帅众方出,见二魔独回,又皆接入,问及放回之故。
二魔把三藏慈悯善胜之言,对众说了一遍。一个个面面相觑,更不敢言。

二魔道:“哥哥可送唐僧么?”老魔道:“兄弟,你说那里话!孙行者是个广施
仁义的猴头,他先在我肚里,若肯害我性命,一千个也被他弄杀了。却才揪住你鼻
子,若是扯了去不放回,只捏破你的鼻子头儿,却也惶恐。快早安排送他去罢。”
三魔笑道:“送,送,送!”老魔道:“贤弟这话,却又像尚气的了。你不送,我两
个送去罢。”

三魔又笑道:“二位兄长在上:那和尚倘不要我们送,只这等瞒过去,还是他
的造化;若要送,不知正中了我的‘调虎离山’之计哩。”老怪道:“何为‘调虎离
山’?”三怪道:“如今把满洞群妖,点将起来,万中选千,千中选百,百中选十
六个,又选三十个。”老怪道:“怎么既要十六,又要三十?”三怪道:“要三十个
会烹煮的,与他些精米、细面、竹笋、茶芽、香蕈、蘑菇、豆腐、面筋,着他二十
里,或三十里,搭下窝铺,安排茶饭,管待唐僧。”老怪道:“又要十六个何用?”
三怪道:“着八个抬,八个喝路。我弟兄相随左右,送他一程。此去向西四百余里,
就是我的城池。我那里自有接应的人马。若至城边,如此如此,着他师徒首尾不能
相顾。要捉唐僧,全在此十六个鬼成功。”老怪闻言,欢欣不已。真是如醉方醒,
似梦方觉。道:“好!好!好!”即点众妖,先选三十,与他物件;又选十六,抬一顶
香藤轿子。同出门来,又吩咐众妖:“俱不许上山闲走,孙行者是个多心的猴子,
若见汝等往来,他必生疑,识破此计。”

老怪遂帅众至大路旁高叫道:“唐老爷,今日不犯红沙,请老爷早早过山。”三
藏闻言道:“悟空,是甚人叫我?”行者指定道:“那厢是老孙降伏的妖精抬轿来送
你哩。”三藏合掌朝天道:“善哉,善哉!若不是贤徒如此之能,我怎生得去!”径直
向前,对众妖作礼道:“多承列位之爱,我弟子取经东回,向长安当传扬善果也。”
众妖叩首道:“请老爷上轿。”那三藏肉眼凡胎,不知是计;孙大圣又是太乙金仙,
忠正之性,只以为擒纵之功,降了妖怪,亦岂期他都有异谋,却也不曾详察,尽着
师父之意。即命八戒将行囊捎在马上,与沙僧紧随。他使铁棒向前开路,顾盼吉凶。
八个抬起轿子,八个一递一声喝道。三个妖扶着轿扛。师父喜喜欢欢的端坐轿上。
上了高山,依大路而行。

此一去,岂知欢喜之间愁又至。经云:“泰极否还生。”时运相逢真太岁,又值
丧门吊客星。那伙妖魔,同心合意的,侍卫左右,早晚殷勤。行经三十里献斋,五
十里又斋,未晚请歇,沿路齐齐整整。一日三餐,遂心满意,良宵一宿,好处安身。

西进有四百里余程,忽见城池相近。大圣举铁棒,离轿仅有一里之遥,见城池,
把他吓了一跌,挣挫不起。你道他只这般大胆,如何见此着唬?原来望见那城中有
许多恶气,乃是:

攒攒簇簇妖魔怪,四门都是狼精灵。斑斓老虎为都管,白面雄彪作总兵。丫叉
角鹿传文引,伶俐狐狸当道行。千尺大蟒围城走,万丈长蛇占路程。楼下苍狼呼令
使,台前花豹作人声。摇旗擂鼓皆妖怪,巡更坐铺尽山精。狡兔开门弄买卖,野猪
挑担干营生。先年原是天朝国,如今翻作虎狼城。

那大圣正当悚惧,只听得耳后风响,急回头观看,原来是三魔双手举一柄画杆
方天戟,往大圣头上打来。大圣急翻身爬起,使金箍棒劈面相迎。他两个各怀恼怒,
气,更不打话,咬着牙,各要相争。又见那老魔头,传声号令,举钢刀便砍八
戒。八戒慌得丢了马,轮着钯,向前乱筑。那二魔缠长枪,望沙僧刺来。沙僧使降
妖杖支开架子敌住。三个魔头与三个和尚,一个敌一个,在那山头舍死忘生苦战。
那十六个小妖却遵号令,各各效能。抢了白马、行囊,把三藏一拥,抬着轿子,径
至城边,高叫道:“大王爷爷定计,已拿得唐僧来了!”那城上大小妖精,一个个跑
下,将城门大开,吩咐各营卷旗息鼓,不许呐喊筛锣,说:“大王原有令在前,不
许吓了唐僧;唐僧禁不得恐吓,一吓就肉酸不中吃了。”众精都欢天喜地邀三藏,
控背躬身接主僧。把唐僧一轿子抬上金銮殿,请他坐在当中,一壁厢献茶,献饭,
左右旋绕。那长老昏昏沉沉,举眼无亲。

毕竟不知性命何如,且听下回分解。