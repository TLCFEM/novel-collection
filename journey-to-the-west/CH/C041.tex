\chapter{心猿遭火败~木母被魔擒}

善恶一时忘念,荣枯都不关心。晦明隐现任浮沉,随分饥餐渴饮。神静湛然常
寂,昏冥便有魔侵。五行蹭蹬破禅林,风动必然寒凛。

却说那孙大圣引八戒别了沙僧,跳过枯松涧,径来到那怪石崖前。果见有一座
洞府,真个也景致非凡。但见:

回銮古道幽还静,风月也听玄鹤弄。白云透出满川光,流水过桥仙意兴。猿啸
鸟啼花木奇,藤萝石磴芝兰胜。苍摇崖壑散烟霞,翠染松篁招彩凤。远列巅峰似插
屏,山朝涧绕真仙洞。昆仑地脉发来龙,有分有缘方受用。
将近行到门前,见有一座石碣,上镌八个大字,乃是“号山枯松涧火云洞”。那壁
厢一群小妖,在那里轮枪舞剑的,跳风顽耍。孙大圣厉声高叫道:“那小的们,趁
早去报与洞主知道,教他送出我唐僧师父来,免你这一洞精灵的性!牙迸半个‘不’
字,我就掀翻了你的山场,平了你的洞府!”那些小妖,闻得此言,慌忙急转身,
各归洞里,关了两扇石门,到里边来报:“大王,祸事了!”

却说那怪自把三藏拿到洞中,选剥了衣服,四马攒蹄,捆在后院里,着小妖打
干净水刷洗,要上笼蒸吃哩。急听得报声祸事,且不刷洗,便来前庭上问:“有何
祸事?”小妖道:“有个毛脸雷公嘴的和尚,带一个长嘴大耳的和尚,在门前要甚
么唐僧师父哩。但若牙迸半个‘不’字,就要掀翻山场,平洞府。”魔王微微冷
笑道:“这是孙行者与猪八戒。他却也会寻哩。我拿他师父,自半山中到此,有百
五十里,却怎么就寻上门来?”教:“小的们,把管车的,推出车去!”那一班几个
小妖,推出五辆小车儿来,开了前门。

八戒望见道:“哥哥,这妖精想是怕我们,推出车子,往那厢搬哩。”行者道:
“不是,且看他放在那里。”只见那小妖将车子按金、木、水、火、土安下,着五
个看着,五个进去通报。那魔王问:“停当了?”答应:“停当了。”教:“取过枪来。”
有那一伙管兵器的小妖,着两个抬出一杆丈八长的火尖枪,递与妖王。妖王轮枪拽
步,也无甚么盔甲,只是腰间束一条锦绣战裙,赤着脚,走出门前。行者与八戒,
抬头观看,但见那怪物:

面如傅粉三分白,唇若涂朱一表才。鬓挽青云欺靛染,眉分新月似刀裁。战裙
巧绣盘龙凤,形比哪吒更富胎。双手绰枪威凛冽,祥光护体出门来。哏声响若春雷
吼,暴眼明如掣电乖。要识此魔真姓氏,名扬千古唤红孩。

那红孩儿怪,出得门来,高叫道:“是甚么人,在我这里吆喝!”行者近前笑道:
“我贤侄,莫弄虚头。你今早在山路旁,高吊在松树梢头,是那般一个瘦怯怯的黄
病孩儿,哄了我师父。我倒好意驮着你,你就弄风儿把我师父摄将来。你如今又弄
这个样子,我岂不认得你?趁早送出我师父,不要白了面皮,失了亲情;恐你令尊
知道,怪我老孙以长欺幼,不像模样。”那怪闻言,心中大怒,咄的一声喝道:“那
泼猴头!我与你有甚亲情?你在这里满口胡柴,绰甚声经儿!那个是你贤侄?”行者
道:“哥哥,是你也不晓得。当年我与你令尊做弟兄时,你还不知在那里哩。”那怪
道:“这猴子一发胡说!你是那里人,我是那里人,怎么得与我父亲做兄弟?”行者
道:“你是不知。我乃五百年前大闹天宫的齐天大圣孙悟空是也。我当初未闹天宫
时,遍游海角天涯,四大部洲,无方不到。那时节,专慕豪杰。你令尊叫做牛魔王,
称为平天大圣,与我老孙结为七弟兄,让他做了大哥;还有个蛟魔王,称为复海大
圣,做了二哥;又有个大鹏魔王,称为混天大圣,做了三哥;又有个狮王,称为
移山大圣,做了四哥;又有个猕猴王,称为通风大圣,做了五哥;又有个狨王,
称为驱神大圣,做了六哥;惟有老孙身小,称为齐天大圣,排行第七。我老弟兄们,
那时节耍子时,还不曾生你哩!”

那怪物闻言,那里肯信,举起火尖枪就刺。行者正是那会家不忙,又使了一个
身法,闪过枪头,轮起铁棒,骂道:“你这小畜生,不识高低,看棍!”那妖精也使
身法,让过铁棒道:“泼猢狲,不达时务!看枪!”他两个也不论亲情,一齐变脸,
各使神通,跳在云端里,好杀:

行者名声大,魔王手段强。一个横举金箍棒,一个直挺火尖枪。吐雾遮三界,
喷云照四方。一天杀气凶声吼,日月星辰不见光。语言无逊让,情意两乖张。那一
个欺心失礼仪,这一个变脸没纲常。棒架威风长,枪来野性狂。一个是混元真大圣,
一个是正果善财郎。二人努力争强胜,只为唐僧拜法王。
那妖魔与孙大圣战经二十合,不分胜败。

猪八戒在旁边,看得明白:妖精虽不败阵,却只是遮拦隔架,全无攻杀之能;
行者纵不赢他,棒法精强,来往只在那妖精头上,不离了左右。八戒暗想道:“不
好啊,行者溜撒,一时间丢个破绽,哄那妖魔钻进来,一铁棒打倒,就没了我的功
劳。……”你看他抖擞精神,举着九齿钯,在空里,望妖精劈头就筑。那怪见了心
惊,急拖枪败下阵来。行者喝教八戒:“赶上!赶上!”

二人赶到他洞门前,只见妖精一只手举着火尖枪,站在那中间一辆小车儿上;
一只手捏着拳头,往自家鼻子上捶了两拳。八戒笑道:“这厮放赖不羞!你好道捶破
鼻子,淌出些血来,搽红了脸,往那里告我们去耶?”那妖魔捶了两拳,念个咒语,
口里喷出火来,鼻子里浓烟迸出,闸闸眼,火焰齐生。那五辆车子上,火光涌出。
连喷了几口,只见那红焰焰、大火烧空,把一座火云洞,被那烟火迷漫,真个是
天炽地。八戒慌了道:“哥哥,不停当!这一钻在火里,莫想得活;把老猪弄做个烧
熟的,加上香料,尽他受用哩!快走!快走!”说声走,他也不顾行者,跑过涧去了。

这行者神通广大,捏着避火诀,撞入火中,寻那妖怪。那妖怪见行者来,又吐
上几口,那火比前更胜。好火:

炎炎烈烈盈空燎,赫赫威威遍地红。却似火轮飞上下,犹如炭屑舞西东。这火
不是燧人钻木,又不是老子炮丹;非天火,非野火,乃是妖魔修炼成真三昧火。五
辆车儿合五行,五行生化火煎成。肝木能生心火旺,心火致令脾土平。脾土生金金
化水,水能生木彻通灵。生生化化皆因火,火遍长空万物荣。妖邪久悟呼三昧,永
镇西方第一名。
行者被他烟火飞腾,不能寻怪,看不见他洞门前路径,抽身跳出火中。那妖精在门
首,看得明白。他见行者走了,却才收了火具,帅群妖,转于洞内, 闭了石门, 以
为得胜, 着小的排宴奏乐,欢笑不
题。

却说行者跳过枯松涧,按下云头。只听得八戒与沙僧朗朗的在松间讲话。行者
上前喝八戒道:“你这呆子,全无人气!你就惧怕妖火,败走逃生,却把老孙丢下。
早是我有些南北哩!”八戒笑道:“哥啊,你被那妖精说着了,果然不达时务。古人
云:‘识得时务者,呼为俊杰。’那妖精不与你亲,你强要认亲;既与你赌斗,放出
那般无情的火来,又不走,还要与他恋战哩!”行者道:“那怪物的手段比我何如?”
八戒道:“不济。”“枪法比我何如?”八戒道:“也不济。老猪见他撑持不住,却来
助你一钯,不期他不识耍,就败下阵来,没天理,就放火了。”行者道:“正是你不
该来。我再与他斗几合,我取巧儿捞他一棒,却不是好?”

他两个只管论那妖精的手段,讲那妖精的火毒。沙和尚倚着松根,笑得呆了。
行者看见道:“兄弟,你笑怎么?你好道有甚手段,擒得那妖魔,破得那火阵?这桩
事,也是大家有益的事。常言道:‘众毛攒球。’你若拿得妖魔,救了师父,也是你
的一件大功绩。”沙僧道:“我也没甚手段,也不能降妖。我笑你两个都着了忙也。”
行者道:“我怎么着忙?”沙僧道:“那妖精手段不如你,枪法不如你,只是多了些
火势,故不能取胜。若依小弟说,以相生相克拿他,有甚难处?”行者闻言,呵呵
笑道:“兄弟说得有理。果然我们着忙了,忘了这事。若以相生相克之理论之,须
是以水克火;却往那里寻些水来,泼灭这妖火,可不救了师父?”沙僧道:“正是
这般。不必迟疑。”行者道:“你两个只在此间,莫与他索战,待老孙去东洋大海求
借龙兵,将些水来,泼息妖火,捉这泼怪。”八戒道:
“哥哥放心前去,我等理会得。”

好大圣,纵云离此地,顷刻到东洋。却也无心看玩海景,使个逼水法,分开波
浪。正行时,见一个巡海夜叉相撞,看见是孙大圣,急回到水晶宫里,报知那老龙
王。敖广即率龙子、龙孙、虾兵、蟹卒一齐出门迎接,请里面坐。坐定,礼毕,告
茶。行者道:“不劳茶,有一事相烦。我因师父唐僧往西天拜佛取经,经过号山枯
松涧火云洞,有个红孩儿妖精,号圣婴大王,把我师父拿了去。是老孙寻到洞边,
与他交战,他却放出火来。我们禁不得他,想着水能克火,特来问你求些水去,与
我下场大雨,泼灭了妖火,救唐僧一难。”那龙王道:“大圣差了。若要求取雨水,
不该来问我。”行者道:“你是四海龙王,主司雨泽,不来问你,却去问谁?”龙王
道:“我虽司雨,不敢擅专;须得玉帝旨意,吩咐在那地方,要几尺几寸,甚么时
辰起住,还要三官举笔,太乙移文,会令了雷公、电母、风伯、云童。俗语云:‘龙
无云而不行’哩。”行者道:“我也不用着风云雷电,只是要些雨水灭火。”龙王道:
“大圣不用风云雷电,但我一人也不能助力,着舍弟们同助大圣一功如何?”行者
道:“令弟何在?”龙王道:“南海龙王敖钦、北海龙王敖闰、西海龙王敖顺。”行
者笑道:“我若再游过三海,不如上界去求玉帝旨意了。”龙王道:“不消大圣去,
只我这里撞动铁鼓、金钟,他自顷刻而至。”行者闻其言道:“老龙王,快撞钟鼓。”

须臾间,三海龙王拥至,问:“大哥,有何事命弟等?”敖广道:“孙大圣在这
里借雨助力降妖。”三弟即引进见毕,行者备言借水之事。众神个个欢从,即点起:

鲨鱼骁勇为前部,痴口大作先锋。鲤元帅翻波跳浪,提督吐雾喷风。鲭太
尉东方打哨,都司西路催征。红眼马郎南面舞,黑甲将军北下冲。把总中军掌
号,五方兵处处英雄。纵横机巧鼋枢密,妙算玄微龟相公。有谋有智鼍丞相,多变
多能鳖总戎。横行蟹士轮长剑,直跳虾婆扯硬弓。鲇外郎查明文簿,点龙兵出离波
中。

诗曰:
四海龙王喜助功,齐天大圣请相从。
只因三藏途中难,借水前来灭火红。

那行者领着龙兵,不多时,早到号山枯松涧上。行者道:“敖氏昆玉,有烦远
。此间乃妖魔之处,汝等且停于空中,不要出头露面。让老孙与他赌斗,若赢了
他,不须列位捉拿;若输与他,也不用列位助阵;只是他但放火时,可听我呼唤,
一齐喷雨。”龙王俱如号令。

行者却按云头,入松林里,见了八戒、沙僧,叫声“兄弟。”八戒道:“哥哥来
得快哑!可曾请得龙王来?”行者道:“俱来了。你两个切须仔细,只怕雨大,莫湿
了行李,待老孙与他打去。”沙僧道:“师兄放心前去,我等俱理会得了。”

行者跳过涧,到了门首,叫声“开门!”那些小妖又去报道:“孙行者又来了。”
红孩仰面笑道:“那猴子想是火中不曾烧了他,故此又来。这一来切莫饶他,断然
烧个皮焦肉烂才罢!”急纵身,挺着长枪,教:“小的们,推出火车子来!”

他出门前,对行者道:“你又来怎的?”行者道:“还我师父来。”那怪道:“你
这猴头,忒不通变。那唐僧与你做得师父,也与我做得按酒,你还思量要他哩。莫
想,莫想!”行者闻言,十分恼怒,掣金箍棒劈头就打。那妖精,使火尖枪,急架
相迎。这一场赌斗,比前不同。好杀:

怒发泼妖魔,恼急猴王将。这一个专救取经僧,那一个要吃唐三藏。心变没亲
情,情疏无义让。这个恨不得捉住活剥皮。那个恨不得拿来生蘸酱。真个忒英雄,
果然多猛壮。棒来枪架赌输赢,枪去棒迎争下上。举手相轮二十回,两家本事一般
样。

那妖王与行者战经二十回合,见得不能取胜,虚幌一枪,急抽身,捏着拳头,
又将鼻子捶了两下,却就喷出火来。那门前车子上,烟火迸起;口眼中,赤焰飞腾。
孙大圣回头叫道:“龙王何在?”那龙王兄弟,帅众水族,望妖精火光里喷下雨来。
好雨!真个是:

潇潇洒洒,密密沉沉。潇潇洒洒,如天边坠落星辰;密密沉沉,似海口倒悬浪
滚。起初时如拳大小,次后来瓮泼盆倾。满地浇流鸭顶绿,高山洗出佛头青。沟壑
水飞千丈玉,涧泉波涨万条银。三叉路口看看满,九曲溪中渐渐平。这个是唐僧有
难神龙助,扳倒天河往下倾。
那雨淙淙大小,莫能止息那妖精的火势。原来龙王私雨,只好泼得凡火;妖精的三
昧真火,如何泼得?好一似火上浇油,越泼越灼。大圣道:“等我捻着诀,钻入火中!”
轮铁棒,寻妖要打。那妖见他来到,将一口烟,劈脸喷来。行者急回头,得眼花
雀乱,忍不住泪落如雨。原来这大圣不怕火,只怕烟。当年因大闹天宫时,被老君
放在八卦炉中,煅过一番。他幸在那巽位安身,不曾烧坏。只是风搅得烟来,把他
做火眼金睛,故至今只是怕烟。那妖又喷一口,行者当不得,纵云头走了。那妖
王却又收了火具,回归洞府。

这大圣一身烟火,炮燥难禁,径投于涧水内救火。怎知被冷水一逼,弄得火气
攻心,三魂出舍。可怜气塞胸堂喉舌冷,魂飞魄散丧残生!慌得那四海龙王在半空
里,收了雨泽,高声大叫:“天蓬元帅!卷帘将军!休在林中藏隐,且寻你师兄出来!”

八戒与沙僧听得呼他圣号,急忙解了马、挑着担奔出林来,也不顾泥泞,顺涧
边找寻。只见那上溜头,翻波滚浪,急流中淌下一个人来。沙僧见了,连衣跳下水
中,抱上岸来,却是孙大圣身躯。噫!你看他蜷局四肢伸不得,浑身上下冷如冰。
沙和尚满眼垂泪道:“师兄!可惜了你,亿万年不老长生客,如今化作个中途短命
人!”八戒笑道:“兄弟莫哭。这猴子佯推死,吓我们哩。你摸他摸,胸前还有一点
热气没有?”沙僧道:“浑身都冷了,就有一点儿热气,怎的就得回生?”八戒道:
“他有七十二般变化,就有七十二条性命。你扯着脚,等我摆布他。”真个那沙僧
扯着脚,八戒扶着头,把他拽个直,推上脚来,盘膝坐定。八戒将两手搓热,仵住
他的七窍,使一个按摩禅法。原来那行者被冷水逼了,气阻丹田,不能出声。却幸
得八戒按摸揉擦,须臾间,气透三关,转明堂,冲开孔窍,叫了一声:“师父啊!”
沙僧道:“哥啊,你生为师父,死也还在口里。且苏醒,我们在这里哩。”行者睁开
眼道:“兄弟们在这里?老孙吃了亏也!”八戒笑道:“你才子发昏的,若不是老猪救
你啊,已此了帐了,还不谢我哩!”行者却才起身,仰面道:“敖氏弟兄何在?”那
四海龙王在半空中答应道:“小龙在此伺候。”行者道:“累你远劳,不曾成得功果,
且请回去,改日再谢。”龙王帅水族,泱泱而回,不在话下。

沙僧搀着行者,一同到松林之下坐定。少时间,却定神顺气,止不住泪滴腮边。
又叫:“师父啊!
忆昔当年出大唐,岩前救我脱灾殃。
三山六水遭魔障,万苦千辛割寸肠。
托钵朝餐随厚薄,参禅暮宿或林庄。
一心指望成功果,今日安知痛受伤!”

沙僧道:“哥哥,且休烦恼。我们早安计策,去那里请兵助力,搭救师父耶。”
行者道:“那里请救么?”沙僧道:“当初菩萨吩咐,着我等保护唐僧,他曾许我们,
叫天天应,叫地地应。那里请救去?”行者道:“想老孙大闹天宫时,那些神兵,
都禁不得我。这妖精神通不小,须是比老孙手段大些的,才降得他哩。天神不济。
地煞不能,若要拿此妖魔,须是去请观音菩萨才好。奈何我皮肉酸麻,腰膝疼痛,
驾不起筋斗云,怎生请得?”八戒道:“有甚话吩咐,等我去请。”行者笑道:“也
罢,你是去得。若见了菩萨,切休仰视,只可低头礼拜。等他问时,你却将地名、
妖名说与他,再请救师父之事。他若肯来,定取擒了怪物。”八戒闻言,即便驾了
云雾,向南而去。

却说那个妖王在洞里欢喜道:“小的们,孙行者吃了亏去了。这一阵虽不得他
死,好道也发个大昏。咦,只怕他又请救兵来也。快开门,等我去看他请谁。”

众妖开了门,妖精就跳在空里观看,只见八戒往南去了。妖精想着南边再无他
处,断然是请观音菩萨,急按下云,叫:“小的们,把我那皮袋寻出来。多时不用,
只恐口绳不牢,与我换上一条,放在二门之下,等我去把八戒赚将回来,装于袋内,
蒸得稀烂,犒你们。”原来那妖精有一个如意的皮袋。众小妖拿出来,换了口绳,
安于洞门内不题。

却说那妖王久居于此,俱是熟游之地。他晓得那条路上南海去近,那条去远。
他从那近路上,一驾云头,赶过了八戒。端坐在壁岩之上,变作一个“假观世音”
模样,等候着八戒。

那呆子正纵云行处,忽然望见菩萨。他那里识得真假?这才是见像作佛。呆子
停云下拜道:“菩萨,弟子猪悟能叩头。”妖精道:“你不保唐僧去取经,却见我有
何事干?”八戒道:“弟子因与师父行至中途,遇着号山枯松涧火云洞,有个红孩
儿妖精,他把我师父摄了去。是弟子与师兄等,寻上他门,与他交战。他原来会放
火,头一阵,不曾得赢;第二阵,请龙王助雨,也不能灭火。师兄被他烧坏了,不
能行动,着弟子来请菩萨。万望垂慈,救我师父一难!”妖精道:“那火云洞洞主,
不是个伤生的;一定是你们冲撞了他也。”八戒道:“我不曾冲撞他,是师兄悟空冲
撞他的。他变作一个小孩子,吊在树上,试我师父。师父甚有善心,教我解下来,
着师兄驮他一程。是师兄掼了他一掼,他就弄风儿,把师父摄去了。”妖精道:“你
起来,跟我进那洞里见洞主,与你说个人情,你陪一个礼,把你师父讨出来罢。”
八戒道:“菩萨呀。若肯还我师父,就磕他一个头也罢。”

妖王道:“你跟来。”那呆子不知好歹,就跟着他,径回旧路,却不向南洋海,
随赴火云门。顷刻间,到了门首。妖精进去道:“你休疑忌。他是我的故人,你进
来。”呆子只得举步入门。众妖一齐呐喊,将八戒捉倒,装于袋内。束紧了口绳,
高吊在驮梁之上。妖精现了本象,坐在当中道:“猪八戒,你有甚么手段,就敢保
唐僧取经,就敢请菩萨降我?你大睁着两个眼,还不认得我是圣婴大王哩!如今拿你,
吊得三五日,蒸熟了赏赐小妖,权为案酒!”八戒听言,在里面骂道:“泼怪物!十
分无礼!若论你百计千方,骗了我吃,管教你一个个遭肿头天瘟!”呆子骂了又骂,
嚷了又嚷,不题。

却说孙大圣与沙僧正坐,只见一阵腥风,刮面而过,他就打了一个喷嚏道:“不
好,不好!这阵风,凶多吉少。想是猪八戒走错路也。”沙僧道:“他错了路,不会
问人?”行者道:“想必撞见妖精了。”沙僧道:“撞见妖精,他不会跑回?”行者
道:“不停当,你坐在这里看守,等我跑过涧去打听打听。”沙僧道:“师兄腰疼,
只恐又着他手,等小弟去罢。”行者道:“你不济事,还让我去。”

好行者,咬着牙,忍着疼,捻着铁棒,走过涧,到那火云洞前,叫声“泼怪!”
那把门的小妖,又急入里报:“孙行者又在门首叫哩!”那妖王传令叫拿,那伙小妖,
枪刀簇拥,齐声呐喊,即开门,都道:“拿住,拿住!”行者果然疲倦,不敢相迎,
将身钻在路旁,念个咒语叫“变”!即变做一个销金包袱。小妖看见,报道:“大王,
孙行者怕了;只见说一声‘拿’字,慌得把包袱丢下,走了。”妖王笑道:“那包袱
也无甚么值钱之物,左右是和尚的破偏衫,旧帽子,背进来拆洗做补衬。”一个小
妖,果将包袱背进,不知是行者变的。行者道:“好了,这个销金包袱,背着了!”
那妖精不以为事,丢在门内。

好行者,假中又假,虚里还虚,即拔一根毫毛,吹口仙气,变作个包袱一样;
他的真身,却又变作一个苍绳儿,钉在门枢上。只听得八戒在那里哼哩哼的,声音
不清,却似一个瘟猪。行者嘤的飞了去寻时,原来他吊在皮袋里也。行者钉在皮袋,
又听得他恶言恶语骂道,妖怪长,妖怪短,“你怎么假变作个观音菩萨,哄我回来,
吊我在此,还说要吃我!有一日我师兄:
大展齐天无量法,满山泼怪登时擒!
解开皮袋放我出,筑你千钯方趁心!”
行者闻言,暗笑道:“这呆子虽然在这里面受闷气,却还不倒了旗枪。老孙一定要
拿了此怪。若不如此,怎生雪恨!”

正欲设法拯救八戒出来,只听那妖王叫道:“六健将何在?”时有六个小妖,
是他知己的精灵,封为健将,都有名字:一个叫做云里雾,一个叫做雾里云;一个
叫做急如火,一个叫做快如风;一个叫做兴烘掀,一个叫做掀烘兴。六健将上前跪
下。妖王道:“你们认得老大王家么?”六健将道:“认得。”妖王道:“你与我星夜
去请老大王来,说我这里捉唐僧蒸与他吃,寿延千纪。”六怪领命,一个个厮拖厮
扯,径出门去了。行者嘤的一声,飞下袋来,跟定那六怪,躲离洞中。

毕竟不知怎的请来,且听下回分解。