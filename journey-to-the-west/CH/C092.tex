\chapter{三僧大战青龙山~四星挟捉犀牛怪}

却说孙大圣挟同二弟滚着风,驾着云,向东北艮地上,顷刻至青龙山玄英洞口,
按落云头。八戒就欲筑门,行者道:“且消停。待我进去看看师父生死如何,再好
与他争持。”沙僧道:“这门闭紧,如何得进?”行者道:“我自有法力。”

好大圣,收了棒,捻着诀,念声咒语,叫“变!”即变做个火焰虫儿。真个也
疾伶!你看他:

展翅星流光灿,古云腐草为萤。神通变化不非轻,自有徘徊之性。飞近石门悬
看,旁边瑕缝穿风。将身一纵到幽庭,打探妖魔动静。
他自飞入,只见几只牛横直倒,一个个呼吼如雷,尽皆睡熟。又至中厅里面,全
无消息。四下门户通关,不知那三个妖精睡在何处。才转过厅房,向后又照,只闻
得啼泣之声,乃是唐僧锁在后房檐柱上哭哩。行者暗暗听他哭甚,只见他哭道:
“一别长安十数年,登山涉水苦熬煎。
幸来西域逢佳节,喜到金平遇上元。
不识灯中假佛像,概因命里有灾愆。
贤徒追袭施威武,但愿英雄展大权。”
行者闻言,满心欢喜,展开翅,飞近师前。唐僧揩泪道:“呀!西方景象不同。此时
正月,蛰虫始振,为何就有萤飞?”行者忍不住,叫声:“师父,我来了!”唐僧喜
道:“悟空,我心说正月怎得萤火,原来是你。”

行者即现了本相道:“师父啊,为你不识真假,误了多少路程,费了多少心力。
我一行说不是好人,你就下拜,却被这怪侮暗灯光,盗取酥合香油,连你都摄将来
了。我当吩咐八戒、沙僧回寺看守,我即闻风追至此间。不识地名,幸遇四值功曹
传报,说此山名青龙山玄英洞。我日间与此怪斗至天晚方回,与师弟辈细道此情,
却就不曾睡,同他两个来此。我恐夜深不便交战,又不知师父下落,所以变化进来,
打听师情。”唐僧喜道:“八戒、沙僧如今在外边哩?”行者道:“在外边。才子老
孙看时,妖精都睡着。我且解了锁,搠开门,带你出去罢。”唐僧点头称谢。

行者使个解锁法,用手一抹,那锁早自开了。领着师父往前正走,忽听得妖王
在中厅内房里叫道:“小的们,紧闭门户,小心火烛。这会怎么不叫更巡逻,梆铃
都不响了?”原来那伙小妖征战一日,俱辛辛苦苦睡着;听见叫唤,却才醒了。梆
铃响处,有几个执器械的,敲着锣,从后而走,可可的撞着他师徒两个。众小妖一
齐喊道:“好和尚啊!扭开锁往那里去!”行者不容分说,掣出棒幌一幌,碗来粗细,
就打。棒起处,打死两个。其余的丢了器械,近中厅,打着门叫:“大王!不好了,
不好了,毛脸和尚在家里打杀人了!”

那三怪听见,一毂辘爬将起来,只教:“拿住!拿住!”唬得个唐僧手软脚软。
行者也不顾师父,一路棒,滚向前来。众小妖遮架不住,被他放倒三两个,推倒两
三个,打开几层门,径自出来,叫道:“兄弟们何在?”八戒、沙僧正举着钯杖等
待,道:“哥哥,如何了?”行者将变化入里解放师父,正走,被妖惊觉,顾不得
师父,打出来的事,讲说一遍不题。

那妖王把唐僧捉住,依然使铁索锁了。执着刀,轮着斧,灯火齐明,问道:“你
这厮怎样开锁,那猴子如何得进,快早供来,饶你之命!不然,就一刀两段!”慌得
那唐僧,战战兢兢的跪道:“大王爷爷!我徒弟孙悟空,他会七十二般变化。才变个
火焰虫儿,飞进来救我;不期大王知觉,被小大王等撞见,是我徒弟不知好歹,打
伤两个,众皆喊叫,举兵着火,他遂顾不得我,走出去了。”三个妖王,呵呵大笑
道:“早就惊觉,未曾走了。”叫小的们把前后门紧紧关闭。亦不喧哗。

沙僧道:“闭门不喧哗,想是暗弄我师父。我们动手耶!”行者道:“说得是。
快早打门。”那呆子卖弄神通,举钯尽力筑去,把那石门筑得粉碎,却又厉声喊骂
道:“偷油的贼怪!快送吾师出来也!”唬得那门内小妖,滚将进去,报道:“大王,
不好了,不好了!前门被和尚打破了!”三个妖王十分烦恼道:“这厮着实无礼!”即
命取披挂结束了,各持兵器,帅小妖出门迎敌。此时约有三更时候,半天中月明如
昼。走出来,更不打话,便就轮兵。这里行者抵住钺斧,八戒敌住大刀,沙僧迎住
大棍。这场好杀:

僧三众,棍杖钯。三个妖魔胆气加。钺斧钢刀藤纥,只闻风响并尘沙。初交
几合喷愁雾,次后飞腾散彩霞。钉钯解数随身滚,铁棒英豪更可夸。降妖宝杖人间
少,妖怪顽心不让他。钺斧口明尖利,藤条节一身花。大刀幌亮如门扇,和尚
神通偏赛他。这壁厢因师性命发狠打,那壁厢不放唐僧劈脸挝。斧剁棒迎争胜负,
钯轮刀砍两交搽。挞藤条降怪杖,翻翻复复逞豪华。

三僧三怪,赌斗多时,不见输赢。那辟寒大王喊一声,叫:“小的们上来!”众
精各执兵刃齐来,早把个八戒绊倒在地。被几个水牛精,揪揪扯扯,拖入洞里捆了。
沙僧见没了八戒,只见那群牛发喊哞声。即掣宝杖,望辟尘大王虚丢了架子要走,
又被群精一拥而来,拉了个踵,急挣不起,也被捉去捆了。行者觉道难为,纵筋
斗云,脱身而去。

当时把八戒、沙僧拖至唐僧前。唐僧见了,满眼垂泪道:“可怜你二人也遭了
毒手!悟空何在?”沙僧道:“师兄见捉住我们,他就走了。”唐僧道:“他既走了,
必然那里去求救。但我等不知何日方得脱网。”师徒们凄凄惨惨不题。

却说行者驾筋斗云复至慈云寺,寺僧接着,来问:“唐老爷救得否?”行者道:
“难救,难救!那妖精神通广大,我弟兄三个,与他三个斗了多时,被他呼小妖先
捉了八戒,后捉了沙僧,老孙幸走脱了。”众僧害怕道:“爷爷这般会腾云驾雾,还
捉获不得,想老师父被倾害也。”行者道:“不妨,不妨!我师父自有伽蓝、揭谛、
丁甲等神暗中护佑;却也曾吃过草还丹,料不伤命。只是那妖精有本事,汝等可好
看马匹、行李,等老孙上天去求救兵来。”众僧胆怯道:“爷爷又能上天?”行者笑
道:“天宫原是我的旧家。当年我做齐天大圣,因为乱了蟠桃会,被我佛收降,如
今没奈何,保唐僧取经,将功折罪。一路上辅正除邪,我师父该有此难,汝等却不
知也。”众僧听此言,又磕头礼拜,行者出得门,打个唿哨,即时不见。

好大圣,早至西天门外。忽见太白金星与增长天王、殷、朱、陶、许四大灵官
讲话。他见行者来,都慌忙施礼道:“大圣那里去?”行者道:“因保唐僧行至天竺
国东界金平府天县,我师被本县慈云寺僧留赏元宵。比至金灯桥,有金灯三盏,
点灯用酥合香油,价贵白金五万余两,年年有诸佛降祥受用。正看时,果有三尊佛
像降临。我师不识好歹,上桥就拜。我说不是好人,早被他侮暗灯光,连油并我师
一风摄去。我随风追袭,至天晓,到一山,幸四功曹报道:‘那山名青龙山。山有
玄英洞。洞有三怪,名辟寒大王、辟暑大王、辟尘大王。’老孙急上门寻讨,与他
赌斗一阵,未胜。是我变化入里,见师父锁住未伤,随解了欲出,又被他知觉,我
遂走了。后又同八戒、沙僧苦战,复被他将二人也捉去捆了。老孙因此特启玉帝,
查他来历,请命将降之。”

金星呵呵冷笑道:“大圣既与妖怪相持,岂看不出他的出处?”行者道:“认便
认得,是一伙牛精。只是他大有神通,急不能降也。”金星道:“那是三个犀牛之精。
他因有天文之象,累年修悟成真,亦能飞云步雾。其怪极爱干净,常嫌自己影身,
每欲下水洗浴。他的名色也多:有兕犀,有雄犀,有牯犀,有斑犀,又有胡冒犀、
堕罗犀、通天花文犀。都是一孔三毛二角,行于江海之中,能开水道。似那辟寒、
辟暑、辟尘都是角有贵气,故以此为名而称大王也。若要拿他,只是四木禽星见面
就伏。”行者连忙唱喏问道:“是那四木禽星?烦长庚老一一明示明示。”金星笑道:
“此星在斗牛宫外,罗布乾坤。你去奏闻玉帝,便见分晓。”行者拱拱手称谢,径
入天门里去。

不一时,到于通明殿下,先见葛、邱、张、许四大天师。天师问道:“何往?”
行者道:“近行至金平府地方,因我师宽放禅性,元夜观灯,遇妖魔摄去。老孙不
能收降,特来奏闻玉帝求救。”四天师即领行者至灵霄宝殿启奏。各各礼毕,备言
其事。玉帝传旨:“教点那路天兵相助?”行者奏道:“老孙才到西天门,遇长庚星
说:‘那怪是犀牛成精,惟四木禽星可以降伏。’”玉帝即差许天师同行者去斗牛宫
点四木禽星下界收降。

及至宫外,早有二十八宿星辰来接。天师道:“吾奉圣旨,教点四木禽星与孙
大圣下界降妖。”旁即闪过角木蛟、斗木獬、奎木狼、井木犴应声呼道:“孙大圣,
点我等何处降妖?”行者笑道:“原来是你。这长庚老儿却隐匿,我不解其意。早
说是二十八宿中的四木,老孙径来相请,又何必劳烦旨意?”四木道:“大圣说那
里话!我等不奉旨意,谁敢擅离?端的是那方?快早去来。”行者道:“在金平府东北
艮地青龙山玄英洞,犀牛成精。”斗木獬、奎木狼、角木蛟道:“若果是犀牛成精,
不须我们,只消井宿去罢。他能上山吃虎,下海擒犀。”行者道:“那犀不比望月之
犀,乃是修行得道,都有千年之寿者。须得四位同去才好,切勿推调。倘一时一位
拿他不住,却不又费事了?”天师道:“你们说得是甚话!旨意着你四人,岂可不去?
趁早飞行。我回旨去也。”那天师遂别行者而去。

四木道:“大圣不必迟疑,你先去索战,引他出来,我们随后动手。”行者即近
前骂道:“偷油的贼怪!还我师来!”原来那门被八戒筑破,几个小妖弄了几块板儿
搪住,在里边听得骂詈,急跑进报道:“大王,孙和尚在外面骂哩!”辟尘儿道:“他
败阵去了,这一日怎么又来?想是那里求些救兵来了。”辟寒、辟暑道:“怕他甚么
救兵!快取披挂来!小的们,都要用心围绕,休放他走了。”

那伙精不知死活,一个个各执枪刀,摇旗擂鼓,走出洞来,对行者喝道:“你
个不怕打的猢狲儿,你又来了!”行者最恼得是这“猢狲”二字,咬牙发狠,举铁
棒就打。三个妖王,调小妖,跑个圈子阵,把行者圈在垓心。那壁厢四木禽星一个
个各轮兵刃道:“孽畜,休动手!”那三个妖王看他四星,自然害怕,俱道:“不好
了,不好了!他寻将降手儿来了!小的们,各顾性命走耶!”只听得呼呼吼吼,喘喘
呵呵,众小妖都现了本身:原来是那山牛精、水牛精、黄牛精,满山乱跑。那三个
妖王,也现了本相,放下手来,还是四只蹄子,就如铁炮一般,径往东北上跑。这
大圣帅井木犴、角木蛟紧追急赶,略不放松。惟有斗木獬、奎木狼在东山凹里、山
头上、山涧中、山谷内,把些牛精打死的、活捉的,尽皆收净。却向玄英洞里解了
唐僧、八戒、沙僧。

沙僧认得是二星,随同拜谢。因问:“二位如何到此相救?”二星道:“吾等是
孙大圣奏玉帝请旨调来收怪救你也。”唐僧又滴泪道:“我悟空徒弟怎么不见进
来?”二星道:“那三个老怪是三只犀牛,他见吾等,各各顾命,向东北艮方逃遁。
孙大圣帅井木犴、角木蛟追赶去了。我二星扫荡群牛到此,特来解放圣僧。”唐僧
复又顿首拜谢,朝天又拜。八戒搀起道:“师父,礼多必诈,不须只管拜了。四星
官,一则是玉帝圣旨,二则是师兄人情。今既扫荡群妖,还不知老妖如何降伏。我
们且收拾些细软东西出来,掀翻此洞,以绝其根,回寺等候师兄罢。”奎木狼道:“天
蓬元帅说得有理。你与卷帘大将保护你师回寺安歇,待吾等还去艮方迎敌。”八戒
道:“正是,正是。你二位还协同一捉,必须剿尽,方好回旨。”二星官即时追袭。

八戒与沙僧将他洞内细软宝贝——有许多珊瑚、玛瑙、珍珠、琥珀、琚、宝
贝、美玉、良金,搜出一石,搬在外面,请师父到山崖上坐了,他又进去放起火来,
把一座洞烧成灰烬,却才领唐僧找路回金平慈云寺去。正是:
经云泰极还生否,好处逢凶实有之。
爱赏花灯禅性乱,喜游美景道心漓。
大丹自古宜长守,一失原来到底亏。
紧闭牢拴休旷荡,须臾懈怠见参差。

且不言他三众得命回寺。却表斗木獬、奎木狼二星官驾云直向东北艮方赶妖怪
来。二人在那半空中,寻看不见。直到西洋大海,远望见孙大圣在海上吆喝。他两
个按落云头道:“大圣,妖怪那里去了?”行者恨道:“你两个怎么不来追降?这会
子却冒冒失失的问甚?”斗木獬道:“我见大圣与井、角二星战败妖魔追赶,料必
擒拿。我二人却就扫荡群精,入玄英洞救出你师父、师弟。搜了山,烧了洞,把你
师父付托与你二弟领回府城慈云寺。多时不见车驾回转,故又追寻到此也。”行者
闻言,方才喜谢道:“如此却是有功,多累,多累!但那三个妖魔,被我赶到此间,
他就钻下海去。当有井、角二星,紧紧追拿,教老孙在岸边抵挡。你两个既来,且
在岸边把截,等老孙也再去来。”

好大圣,轮着棒,捻着诀,辟开水径,直入波涛深处。只见那三个妖魔在水底
下与井木犴、角木蛟舍死忘生苦斗哩。他跳近前喊道:“老孙来也!”那妖精抵住二
星官,措手不及。正在危难之处,忽听得行者叫喊,顾残生,拨转头往海心里飞跑。
原来这怪头上角,极能分水,只闻得花花花,冲开明路。这后边二星官并孙大圣并
力追之。

却说西海中有个探海的夜叉,巡海的介士,远见犀牛分开水势,又认得孙大圣
与二天星,即赴水晶宫对龙王慌慌张张报道:“大王!有三只犀牛,被齐天大圣和二
位天星赶来也!”老龙王敖顺听言,即唤太子摩昂:“快点水兵。想是犀牛精辟寒、
辟暑、辟尘儿三个惹了孙行者。今既至海,快快拔刀相助。”敖摩昂得令,即忙点
兵。

顷刻间,龟鳖鼋鼍,鳜鲤,与虾兵蟹卒等,各执枪刀,一齐呐喊,腾出水
晶宫外,挡住犀牛精。犀牛精不能前进,急退后,又有井、角二星并大圣拦阻,慌
得他失了群,各各逃生,四散奔走,早把个辟尘儿被老龙王领兵围住。孙大圣见了
心欢,叫道:“消停,消停,捉活的,不要死的!”摩昂听令,一拥上前,将辟尘儿
扳翻在地,用铁钩子穿了鼻,攒蹄捆倒。

老龙王又传号令,教分兵赶那两个,协助二星官擒拿。即时小龙王帅众前来。
只见井木犴现原身,按住辟寒儿,大口小口的啃着吃哩。摩昂高叫道:“井宿,井
宿!莫咬死他。孙大圣要活的,不要死的哩。”连喊数喊,已是被他把劲项咬断了。

摩昂吩咐虾兵蟹卒,将个死犀牛抬转水晶宫,却又与井木犴向前追赶。只见角
木蛟把那辟暑儿倒赶回来,只撞着井宿。摩昂帅龟鳖鼋鼍,撒开簸箕阵围住。那怪
只教:“饶命,饶命!”井木犴走近前,一把揪住耳朵,夺了他的刀,叫道:“不杀
你,不杀你,拿与孙大圣发落去来。”

当即倒干戈,复至水晶宫外,报道:“都捉来也。”行者见一个断了头,血淋津
的,倒在地下。一个被井木犴拖着耳朵,推跪在地。近前仔细看了道:“这头不是
兵刀伤的啊。”摩昂笑道:“不是我喊得紧,连身子都着井星官吃了。”行者道:“既
是如此,也罢,取锯子来,锯下他的这两只角,剥了皮带去。犀牛肉还留与龙王贤
父子享之。”又把辟尘儿穿了鼻,教角木蛟牵着;辟暑儿也穿了鼻,教井木犴牵着;
“带他上金平府见那刺史官,明究其由,问他个积年假佛害民,然后的决。”

众等遵言,辞龙王父子,都出西海。牵着犀牛,会着奎、斗二星,驾云雾,径
转金平府。行者足踏祥光,半空中叫道:“金平府刺史,各佐贰郎官并府城内外军
民人等听着:吾乃东土大唐差往西天取经的圣僧。你这府县,每年家供献金灯,假
充诸佛降祥者,即此犀牛之怪。我等过此,因元夜观灯,见这怪将灯油并我师父摄
去,是我请天神收伏。今已扫清山洞,剿尽妖魔,不得为害。以后你府县再不可供
献金灯,劳民伤财也。”那慈云寺里,八戒、沙僧方保唐僧进得山门,只听见行者
在半空言语,即便撇了师父,丢下担子,纵风云起到空中,问行者降妖之事。行者
道:“那一只被井星咬死,已锯角剥皮带来,两只活拿在此。”八戒道:“这两个索
性推下此城,与官员人等看看,也认得我们是圣是神。左右累四位星官收云下地,
同到府堂,将这怪的决。已此情真罪当,再有甚讲!”四星道:“天蓬帅近来知理明
律,却好呀!”八戒道:“因做了这几年和尚,也略学得些儿。”

众神果推落犀牛,一簇彩云,降至府堂之上。唬得这府县官员,城里城外人等,
都家家设香案,户户拜天神。少时间,慈云寺僧把长老用轿抬进府门,会着行者,
口中不离“谢”字道:“有劳上宿星官救出我等。因不见贤徒,悬悬在念,今幸得
胜而回!然此怪不知赶向何方才捕获也!”行者道:“自前日别了尊师,老孙上天查
访,蒙太白金星识得妖魔是犀牛,指示请四木禽星。当时奏闻玉帝,蒙旨差委,直
至洞口交战。妖王走了,又蒙斗、奎二宿救出尊师。老孙与井、角二宿并力追妖,
直赶到西洋大海,又亏龙王遣子帅兵相助。所以捕获到此审究也。”长老赞扬称谢
不已。又见那府县正官并佐贰首领,都在那里高烧宝烛,满斗焚香,朝上礼拜。

少顷间,八戒发起性来,掣出戒刀,将辟尘儿头一刀砍下,又一刀把辟暑儿头
也砍下。随即取锯子锯下四只角来。孙大圣更有主张,就教:“四位星官,将此四
只犀角,拿上界去,进贡玉帝,回缴圣旨。”把自己带来的二只:“留一只在府堂镇
库,以作向后免征灯油之证;我们带一只去,献灵山佛祖。”四星心中大喜。即时
拜别大圣,忽驾彩云回奏而去。

府县官,留住他师徒四众,大排素宴,遍请乡官陪奉。一壁厢出给告示,晓谕
军民人等,下午不许点设金灯,永蠲买油大户之役。一壁厢叫屠子宰剥犀牛之皮,
硝熟熏干,制造铠甲;把肉普给官员人等。又一壁厢动支枉罚无碍钱粮,买民间空
地,起建四星降妖之庙;又为唐僧四众建立生祠,各各树牌刻文,用传千古,以为
报谢。

师徒们索性宽怀领受。又被那二百四十家灯油大户,这家酬,那家请,略无虚
刻。八戒遂心满意受用,将洞里搜来的宝物,每样各笼些须在袖,以为各家斋筵之
赏。住经个月,犹不得起身。长老吩咐:“悟空,将余剩的宝物,尽送慈云寺僧,
以为酬礼。瞒着那些大户人家,天不明走罢;恐只管贪乐,误了取经,惹佛祖见罪,
又生灾厄,深为不便。”行者随将前件一一处分。

次日五更早起,唤八戒备马。那呆子吃了自在酒饭,睡得梦梦乍道:“这早备
马怎的?”行者喝道:“师父教走路哩!”呆子抹抹脸道:“又是这长老没正经!二百
四十家大户都请,才吃了有三十几顿饱斋,怎么又弄老猪忍饿!”长老听言骂道:“馕
糟的夯货,莫胡说,快早起来!再若强嘴,教悟空拿金箍棒打牙!”那呆子听见说打,
慌了手脚道:“师父今番变了,常时疼我,爱我,念我蠢夯护我;哥要打时,他又
劝解;今日怎么发狠转教打么?”行者道:“师父怪你为嘴,误了路程。快早收拾
行李、备马,免打!”那呆子真个怕打,跳起来穿了衣服,吆喝沙僧:“快起来,打
将来了!”沙僧也随跳起,各各收拾皆完。长老摇手道:“寂寂悄悄的,不要惊动寺
僧。”连忙上马,开了山门,找路而去。这一去,正所谓:
暗放玉笼飞彩凤,私开金锁走蛟龙。

毕竟不知天明时,酬谢之家端的如何,且听下回分解。