\chapter{情因旧恨生灾毒~心主遭魔幸破光}

话说孙大圣扶持着唐僧,与八戒、沙僧奔上大路,一直西来。不半晌,忽见一
处楼阁重重,宫殿巍巍。唐僧勒马道:“徒弟,你看那是个甚么去处?”行者举头
观看,忽然见:

山环楼阁,溪绕亭台。门前杂树密森森,宅外野花香艳艳。柳间栖白鹭,浑如
烟里玉无瑕;桃内啭黄莺,却似火中金有色。双双野鹿,忘情闲踏绿莎茵;对对山
禽,飞语高鸣红树杪。真如刘阮天台洞,不亚神仙阆苑家。
行者报道:“师父,那所在也不是王侯第宅,也不是豪富人家,却像一个庵观寺院。
到那里方知端的。”三藏闻言,加鞭促马。师徒们来至门前观看,门上嵌着一块石
板,上有“黄花观”三字。三藏下马。八戒道:“黄花观乃道士之家。我们进去会
他一会也好,他与我们衣冠虽别,修行一般。”沙僧道:“说得是。一则进去看看景
致,二来也当撒货头口。看方便处,安排些斋饭,与师父吃。”

长老依言,四众共入。但见二门上有一对春联:“黄芽白雪神仙府,瑶草琪花
羽士家。”行者笑道:“这个是烧茅炼药,弄炉火,提罐子的道士。”三藏捻他一把
道:“谨言,谨言!我们不与他相识,又不认亲,左右暂时一会,管他怎的?”说不
了,进了二门,只见那正殿谨闭,东廊下坐着一个道士,在那里丸药。你看他怎生
打扮:

戴一顶红艳艳戗金冠,穿一领黑淄淄乌皂服;踏一双绿阵
阵云头履,系一条黄拂拂吕公绦。面如瓜铁,目若朗星。准头高大类回回,唇口翻
张如达达。道心一片隐轰雷,伏虎降龙真羽士。
三藏见了,厉声高叫道:“老神仙,贫僧问讯了。”那道士猛抬头,一见心惊,丢了
手中之药,按簪儿,整衣服,降阶迎接道:“老师父,失迎了。请里面坐。”

长老欢喜上殿。推开门,见有三清圣像,供桌有炉有香,即拈香注炉,礼拜三
匝,方与道士行礼。遂至客位中,同徒弟们坐下。急唤仙童看茶。当有两个小童,
即入里边,寻茶盘,洗茶盏,擦茶匙,办茶果。忙忙的乱走,早惊动那几个冤家。

原来那盘丝洞七个女怪与这道士同堂学艺。自从穿了旧衣,唤出儿子,径来此
处。正在后面裁剪衣服,忽见那童子看茶,便问道:“童儿,有甚客来了,这般忙
冗?”仙童道:“适间有四个和尚进来,师父教来看茶。”女怪道:“可有个白胖和
尚?”道:“有。”又问:“可有个长嘴大耳朵的?”道:“有。”女怪道:“你快去递
了茶,对你师父丢个眼色,着他进来,我有要紧的话说。”

果然那仙童将五杯茶拿出去。道士敛衣,双手拿一杯递与三藏,然后与八戒、
沙僧、行者。茶罢收钟,小童丢个眼色。那道士就欠身道:“列位请坐。”教:“童
儿,放了茶盘陪侍。等我去去就来。”此时长老与徒弟们,并一个小童出殿上观玩
不题。

却说道士走进方丈中,只见七个女子齐齐跪倒,叫:“师兄,师兄,听小妹子
一言!”道士用手搀起道:“你们早间来时,要与我说甚么话,可可的今日丸药,这
枝药忌见阴人,所以不曾答你。如今又有客在外面,有话且慢慢说罢。”众怪道:“告
禀师兄。这桩事,专为客来,方敢告诉;若客去了,纵说也没用了。”道士笑道:“你
看贤妹说话,怎么专为客来才说?却不疯了?且莫说我是个清静修仙之辈,就是个俗
人家,有妻子老小家务事,也等客去了再处。怎么这等不贤,替我装幌子哩!且让
我出去。”众怪又一齐扯住道:“师兄息怒。我问你,前边那客,是那方来的?”道
士唾着脸,不答应。众怪道:“方才小童进来取茶,我闻得他说,是四个和尚。”道
士作怒道:“和尚便怎么?”众怪道:“四个和尚,内有一个白面胖的,有一个长嘴
大耳的,师兄可曾问他是那里来的?”道士道:“内中是有这两个,你怎么知道?想
是在那里见他来?”

女子道:“师兄原不知这个委曲。那和尚乃唐朝差往西天取经去的。今早到我
洞里化斋,委是妹子们闻得唐僧之名,将他拿了。”道士道:“你拿他怎的?”女子
道:“我等久闻人说,唐僧乃十世修行的真体,有人吃他一块肉,延寿长生,故此
拿了他。后被那个长嘴大耳朵的和尚把我们拦在濯垢泉里,先抢了衣服,后弄本事,
强要同我等洗浴,也止他不住。他就跳下水,变作一个鲇鱼,在我们腿裆里钻来钻
去,欲行奸骗之事。果有十分惫懒!他又跳出水去,现了本相。见我们不肯相从,
他就使一柄九齿钉钯,要伤我们性命。若不是我们有些见识,几乎遭他毒手。故此
战兢兢逃生,又着你愚外甥与他敌斗,不知存亡如何。我们特来投兄长,望兄长念
昔日同窗之雅,与我今日做个报冤之人!”

那道士闻此言,却就恼恨,遂变了声色道:“这和尚原来这等无礼,这等惫懒!
你们都放心,等我摆布他!”众女子谢道:“师兄如若动手,等我们都来相帮打他。”
道士道:“不用打,不用打,常言道:‘一打三分低。’你们都跟我来。”

众女子相随左右。他入房内,取了梯子,转过床后,爬上屋梁,拿下一个小皮
箱儿。那箱儿有八寸高下,一尺长短,四寸宽窄,上有一把小铜锁儿锁住。即于袖
中拿出一方鹅黄绫汗巾儿来。汗巾须上系着一把小钥匙儿。开了锁,取出一包儿药
来,此药乃是:

山中百鸟粪,扫积上千斤。是用铜锅煮,煎熬火候匀。千斤熬一杓,一杓炼三
分。三分还要炒,再煅再重熏。制成此毒
药,贵似宝和珍。如若尝他味,入口见阎君!
道士对七个女子道:“妹妹,我这宝贝,若与凡人吃,只消一厘,入腹就死;若与
神仙吃,也只消三厘就绝;这些和尚,只怕也有些道行,须得三厘。快取等子来。”
内一女子,急拿了一把等子道:“称出一分二厘,分作四分。”却拿了十二个红枣儿,
将枣掐破些儿,上一厘,分在四个茶钟内;又将两个黑枣儿做一个茶钟,着一个
托盘安了,对众女说:“等我去问他。不是唐朝的便罢;若是唐朝来的,就教换茶,
你却将此茶令童儿拿出。但吃了,个个身亡,就与你报了此仇,解了烦恼也。”七
女感激不尽。

那道士换了一件衣服,虚礼谦恭,走将出去,请唐僧等又至客位坐下,道:“老
师父莫怪。适间去后面吩咐小徒,教他们挑些青菜、萝卜,安排一顿素斋供养,所
以失陪。”三藏道:“贫僧素手进拜,怎么敢劳赐斋?”道士笑云:“你我都是出家
人,见山门就有三升俸粮,何言素手?敢问老师父,是何宝山?到此何干?”三藏道:
“贫僧乃东土大唐驾下差往西天大雷音寺取经者。却才路过仙宫,竭诚进拜。”道
士闻言,满面生春道:“老师乃忠诚大德之佛,小道不知,失于远候。恕罪,恕罪!”
叫:“童儿,快去换茶来。一厢作速办斋。”那小童走将进去,众女子招呼他来道:
“这里有现成好茶,拿出去。”那童子果然将五钟茶拿出。道士连忙双手拿一个红
枣儿茶钟奉与唐僧。他见八戒身躯大,就认做大徒弟;沙僧认做二徒弟;见行者身
量小,认做三徒弟;所以第四钟才奉与行者。

行者眼乖,接了茶钟,早已见盘子里那茶钟是两个黑枣儿。他道:“先生,我
与你穿换一杯。”道士笑道:“不瞒长老说。山野中贫道士,茶果一时不备。才然在
后面亲自寻果子,止有这十二个红枣,做四钟茶奉敬。小道又不可空陪,所以将两
个下色枣儿作一杯奉陪。此乃贫道恭敬之意也。”行者笑道:“说那里话?古人云:‘在
家不是贫,路上贫杀人。’你是住家儿的,何以言贫?象我们这行脚僧,才是真贫哩。
我和你换换。我和你换换。”三藏闻言道:“悟空,这仙长实乃爱客之意,你吃了罢,
换怎的?”行者无奈,将左手接了,右手盖住,看着他们。

却说那八戒,一则饥,二则渴,原来是食肠大大的,见那钟子里有三个红枣儿,
拿起来的都咽在肚里。师父也吃了,沙僧也吃了。一霎时,只见八戒脸上变色,
沙僧满眼流泪,唐僧口中吐沫。他们都坐不住,晕倒在地。

这大圣情知是毒,将茶钟,手举起来,望道士劈脸一掼。道士将袍袖隔起,当
的一声,把个钟子跌得粉碎。道士怒道:“你这和尚,十分村卤!怎么把我钟子碎
了?”行者骂道:“你这畜生!你看我那三个人是怎么说!我与你有甚相干,你却将
毒药茶药倒我的人?”道士道:“你这个村畜生,闯下祸来,你岂不知?”行者道:
“我们才进你门,方叙了坐次,道及乡贯,又不曾有个高言,那里闯下甚祸?”道
士道:“你可曾在盘丝洞化斋么?你可曾在濯垢泉洗澡么?”行者道:“濯垢泉乃七
个女怪。你既说出这话,必定与他苟合,必定也是妖精!不要走,吃我一棒!”好大
圣,去耳朵里摸出金箍棒,幌一幌,碗来粗细,望道士劈脸打来。那道士急转身躲
过,取一口宝剑来迎。

他两个厮骂厮打,早惊动那里边的女怪。他七个一拥出来,叫道:“师兄且莫
劳心,待小妹子拿他。”行者见了,越生嗔怒,双手轮铁棒,丢开解数,滚将进去
乱打。只见那七个敞开怀,腆着雪白肚子,脐孔中作出法来:骨都都丝绳乱冒,搭
起一个天篷,把行者盖在底下。

行者见事不谐,即翻身念声咒语,打个筋斗,扑的撞破天篷走了;忍着性气,
淤淤的立在空中看处,见那怪丝绳幌亮,穿穿道道,却是穿梭的经纬,顷刻间,把
黄花观的楼台殿阁都遮得无影无形。行者道:“利害,利害!早是不曾着他手。怪道
猪八戒跌了若干!似这般怎生是好!我师父与师弟却又中了毒药。这伙怪合意同心,
却不知是个甚来历,待我还去问那土地神也。”

好大圣,按落云头,捻着诀,念声“”字真言,把个土地老儿又拘来了,战
兢兢跪下路旁,叩头道:“大圣,你去救你师父的,为何又转来也?”行者道:“早
间救了师父,前去不远,遇一座黄花观。我与师父等进去看看,那观主迎接。才叙
话间,被他把毒药茶药倒我师父等。我幸不曾吃茶,使棒就打,他却说出盘丝洞化
斋,濯垢泉洗澡之事,我就知那厮是怪。才举手相敌,只见那七个女子跑出,吐放
丝绳,老孙亏有见识走了。我想你在此间为神,定知他的来历。是个甚么妖精,老
实说来,免打!”土地叩头道:“那妖精到此,住不上十年。小神自三年前检点之后,
方见他的本相,乃是七个蜘蛛精。他吐那些丝绳,乃是蛛丝。”行者闻言,十分欢
喜道:“据你说,却是小可。既这般,你回去,等我作法降他也。”那土地叩头而去。

行者却到黄花观外,将尾巴上毛捋下七十根,吹口仙气,叫“变!”即变做七
十个小行者;又将金箍棒吹口仙气,叫“变!”即变做七十个双角叉儿棒。每一个
小行者,与他一根。他自家使一根,站在外边,将叉儿搅那丝绳,一齐着力,打个
号子,把那丝绳都搅断,各搅了有十余斤。里面拖出七个蜘蛛,足有巴斗大的身躯。
一个个攒着手脚,索着头,只叫:“饶命!饶命!”此时七十个小行者,按住七个蜘
蛛,那里肯放。行者道:“且不要打他,只教还我师父、师弟来。”那怪厉声高叫道:
“师兄,还他唐僧,救我命也!”那道士从里边跑出道:“妹妹,我要吃唐僧哩,救
不得你了。”行者闻言,大怒道:“你既不还我师父,且看你妹妹的样子!”好大圣,
把叉儿棒幌一幌,复了一根铁棒,双手举起,把七个蜘蛛精,尽情打烂,却似七个
肉布袋儿,脓血淋淋。却又将尾巴摇了两摇,收了毫毛,单身轮棒,赶入里边来
打道士。

那道士见他打死了师妹,心甚不忍,即发狠举剑来迎。这一场各怀忿怒,一个
个大展神通。这一场好杀:

妖精轮宝剑,大圣举金箍。都为唐朝三藏,先教七女呜呼。如今大展经纶手,
施威弄法逞金吾。大圣神光壮,妖仙胆气粗。浑身解数如花锦,双手腾那似辘轳。
乒乓剑棒响,惨淡野云浮。言语,使机谋,一来一往如画图。杀得风响沙飞狼虎
怕,天昏地暗斗星无。
那道士与大圣战经五六十合,渐觉手软;一时间松了筋节,便解开衣带,忽辣的响
一声,脱了皂袍。行者笑道:“我儿子!打不过人,就脱剥了也是不能够的!”原来
这道士剥了衣裳,把手一齐抬起,只见那两胁下有一千只眼,眼中迸放金光,十分
利害:

森森黄雾,艳艳金光:森森黄雾,两边胁下似喷云;艳艳金光,千只眼中如放
火。左右却如金桶,东西犹似铜钟。此乃妖仙施法力,道士显神通:幌眼迷天遮日
月,罩人爆燥气朦胧;把个齐天孙大圣,困在金光黄雾中。
行者慌了手脚,只在那金光影里乱转,向前不能举步,退后不能动脚,却便似在个
桶里转的一般。无奈又爆燥不过,他急了,往上着实一跳,却撞破金光,扑的跌了
一个倒栽葱;觉道撞的头疼,急伸手摸摸,把顶梁皮都撞软了。自家心焦道:“晦
气,晦气!这颗头今日也不济了!常时刀砍斧剁,莫能伤损,却怎么被这金光撞软了
皮肉?久以后定要贡脓。纵然好了,也是个破伤风。”一会家爆燥难禁。却又自家计
较道:“前去不得,后退不得,左行不得,右行不得,往上又撞不得,却怎么好?往
下走他娘罢!”

好大圣,念个咒语,摇身一变,变做个穿山甲,又名鲮鲤鳞。真个是:

四只铁爪,钻山碎石如挝粉;满身鳞甲,破岭穿岩似切葱。两眼光明,好便似
双星幌亮;一嘴尖利,胜强如钢钻金锥。药
中有性穿山甲,俗语呼为鲮鲤鳞。
你看他硬着头,往地下一钻,就钻了有二十余里,方才出头。原来那金光只罩得十
余里。出来现了本相,力软筋麻,浑身疼痛,止不住眼中流泪。忽失声叫道:“师
父啊!
当年秉教出山中,共往西来苦用工。
大海洪波无恐惧,阳沟之内却遭风!”

美猴王正当悲切,忽听得山背后有人啼哭,即欠身揩了眼泪,回头观看。但见
一个妇人,身穿重孝,左手托一盏凉浆水饭,右手执几张烧纸黄钱,从那厢一步一
声,哭着走来。行者点头嗟叹道:“正是‘流泪眼逢流泪眼,断肠人遇断肠人!’这
一个妇人,不知所哭何事,待我问他一问。”

那妇人不一时走上路来,迎着行者。行者躬身问道:“女菩萨,你哭的是甚人?”
妇人噙泪道:“我丈夫因与黄花观观主买竹竿争讲,被他将毒药茶药死,我将这陌
纸钱烧化,以报夫妇之情。”行者听言,眼中泪下。那妇女见了作怒道:“你甚无知!
我为丈夫烦恼生悲,你怎么泪眼愁眉,欺心戏我?”

行者躬身道:“女菩萨息怒。我本是东土大唐钦差御弟唐三藏大徒弟孙悟空行
者。因往西天,行过黄花观歇马。那观中道士,不知是个甚么妖精,他与七个蜘蛛
精,结为兄妹。蜘蛛精在盘丝洞要害我师父,是我与师弟八戒、沙僧,救解得脱。
那蜘蛛精走到他这里,背了是非,说我等有欺骗之意。道士将毒药茶药倒我师父、
师弟共三人,连马四口,陷在他观里。惟我不曾吃他茶,将茶钟掼碎,他就与我相
打。正嚷时,那七个蜘蛛精跑出来吐放丝绳,将我捆住,是我使法力走脱。问及土
地,说他本相,我却又使分身法搅绝丝绳,拖出妖来,一顿棒打死。这道士即与他
报仇,举宝剑与我相斗。斗经六十回合,他败了阵,随脱了衣裳,两胁下放出千只
眼,有万道金光,把我罩定。所以进退两难,才变做一个鲮鲤鳞,从地下钻出来。
正自悲切,忽听得你哭,故此相问。因见你为丈夫,有此纸钱报答,我师父丧身,
更无一物相酬,所以自怨生悲。岂敢相戏!”

那妇女放下水饭、纸钱,对行者陪礼道:“莫怪,莫怪,我不知你是被难者。
才据你说将起来,你不认得那道士。他本是个百眼魔君,又唤做多目怪。你既然有
此变化,脱得金光,战得许久,必定有大神通,却只是还近不得那厮。我教你去请
一位圣贤,他能破得金光,降得道士。”行者闻言,连忙唱喏道:“女菩萨知此来历,
烦为指教指教。果是那位圣贤,我去请求,救我师父之难,就报你丈夫之仇。”妇
人道:“我就说出来,你去请他,降了道士,只可报仇而已,恐不能救你师父。”行
者道:“怎不能救?”妇人道:“那厮毒药最狠,药倒人,三日之间,骨髓俱烂。你
此往回恐迟了,故不能救。”行者道:“我会走路,凭他多远,千里只消半日。”女
子道:“你既会走路,听我说:此处到那里有千里之遥。那厢有一座山,名唤紫云
山。山中有个千花洞。洞里有位圣贤,唤做毗蓝婆。他能降得此怪。”行者道:“那
山坐落何方?却从何方去?”女子用手指定道:“那直南上便是。”行者回头看时,
那女子早不见了。

行者慌忙礼拜道:“是那位菩萨?我弟子钻昏了,不能相识,千乞留名,好谢!”
只见那半空中叫道:“大圣,是我。”行者急抬头看处,原是黎山老姆。赶至空中谢
道:“老姆从何来指教我也?”老姆道:“我才自龙华会上回来,见你师父有难,假
做孝妇,借夫丧之名,特来相救。你快去请他。但不可说出是我指教,那圣贤有些
多怪人。”

行者谢了。辞别,把筋斗云一纵,随到紫云山上。按定云头,就见那千花洞。
那洞外:

青松遮胜境,翠柏绕仙居。绿柳盈山道,奇花满涧渠。香
兰围石屋,芳草映岩。流水连溪碧,云封古树虚。野禽声聒聒,幽鹿步徐徐。修
竹枝枝秀,红梅叶叶舒。寒鸦栖古树,春鸟噪高樗。夏麦盈田广,秋禾遍地余。四
时无叶落,八节有花如。每生瑞霭连霄汉,常放祥云接太虚。
这大圣喜喜欢欢走将进去,一程一节,看不尽无边的景致。直入里面,更没个人儿,
见静静悄悄的,鸡犬之声也无。心中暗道:“这圣贤想是不在家了。”又进数里看时,
见一个女道姑坐在榻上。你看他怎生模样:

头戴五花纳锦帽,身穿一领织金袍。脚踏云尖凤头履,腰系攒丝双穗绦。面似
秋容霜后老,声如春燕社前娇。腹中久谙三乘法,心上常修四谛饶。悟出空空真正
果,炼成了了自逍遥。正是千花洞里佛,毗蓝菩萨姓名高。

行者止不住脚,近前叫道:“毗蓝婆菩萨,问讯了。”那菩萨即下榻,合掌回礼
道:“大圣,失迎了。你从那里来的?”行者道:“你怎么就认得我是大圣?”毗蓝
婆道:“你当年大闹天宫时,普地里传了你的形象,谁人不知,那个不识?”行者
道:“正是‘好事不出门,恶事传千里’。像我如今皈正佛门,你就不晓的了!”毗
蓝道:“几时皈正?恭喜,恭喜!”行者道:“近能脱命,保师父唐僧上西天取经,师
父遇黄花观道士,将毒药茶药倒。我与那厮赌斗,他就放金光罩住我,是我使神通
走脱了。闻菩萨能灭他的金光,特来拜请。”菩萨道:“是谁与你说的?我自赴了盂
兰会,到今三百余年,不曾出门。我隐姓埋名,更无一人知得,你却怎么得知?”
行者道:“我是个地里鬼,不管那里,自家都会访着。”毗蓝道:“也罢,也罢。我
本当不去,奈蒙大圣下临,不可灭了求经之善,我和你去来。”

行者称谢了。道:“我忒无知,擅自催促,但不知曾带甚么兵器。”菩萨道:“我
有个绣花针儿,能破那厮。”行者忍不住道:“老姆误了我,早知是绣花针,不须劳
你,就问老孙要一担也是有的。”毗蓝道:“你那绣花针,无非是钢铁金针,用不得。
我这宝贝,非钢,非铁,非金,乃我小儿日眼里炼成的。”行者道:“令郎是谁?”
毗蓝道:“小儿乃昴日星官。”行者惊骇不已。早望见金光艳艳,即回向毗蓝道:“金
光处便是黄花观也。”

毗蓝随于衣领里取出一个绣花针,似眉毛粗细,有五六分长短,拈在手,望空
抛去。少时间,响一声,破了金光。行者喜道:“菩萨,妙哉,妙哉!寻针,寻针!”
毗蓝托在手掌内道:“这不是?”行者却同按下云头,走入观里,只见那道士合了
眼,不能举步。行者骂道:“你这泼怪装瞎子哩!”耳朵里取出棒来就打。毗蓝扯住
道:“大圣莫打。且看你师父去。”

行者径至后面客位里看时,他三人都睡在地上吐痰吐沫哩。行者垂泪道:“却
怎么好!却怎么好!”毗蓝道:“大圣休悲。也是我今日出门一场,索性积个阴德,
我这里有解毒丹,送你三丸。”行者转身拜求。那菩萨袖中取出一个破纸包儿,内
将三粒红丸子递与行者,教放入口里。行者把药扳开他们牙关,每人了一丸。须
臾,药味入腹,便就一齐呕哕,遂吐出毒味,得了性命。那八戒先爬起道:“闷杀
我也!”三藏、沙僧俱醒了道:“好晕也!”行者道:“你们那茶里中了毒了。亏这毗
蓝菩萨搭救,快都来拜谢。”三藏欠身整衣谢了。

八戒道:“师兄,那道士在那里?等我问他一问,为何这般害我。”行者把蜘蛛
精上项事,说了一遍。八戒发狠道:“这厮既与蜘蛛为姊妹,定是妖精!”行者指道:
“他在那殿外立定装瞎子哩。”八戒拿钯就筑,又被毗蓝止住道:“天蓬息怒。大圣
知我洞里无人,待我收他去看守门户也。”行者道:“感蒙大德,岂不奉承!但只是
教他现本象,我们看看。”毗蓝道:“容易。”即上前用手一指,那道士扑的倒在尘
埃,现了原身,乃是一条七尺长短的大蜈蚣精。毗蓝使小指头挑起,驾祥云,径转
千花洞去。八戒打仰道:“这妈妈儿却也利害,怎么就降这般恶物?”行者笑道:“我
问他有甚兵器破他金光,他道有个绣花针儿,是他儿子在日眼里炼的。及问他令郎
是谁,他道是昴日星官。我想昴日星是只公鸡,这老妈妈子必定是个母鸡。鸡最能
降蜈蚣,所以能收伏也。”

三藏闻言,顶礼不尽。教:“徒弟们,收拾去罢。”那沙僧即在里面寻了些米粮,
安排了些斋,俱饱餐一顿。牵马挑担,请师父出门。行者从他厨中放了一把火,把
一座观霎时烧得煨烬,却拽步长行。正是:
唐僧得命感毗蓝,了性消除多目怪。

毕竟向前去还有甚么事体,且听下回分解。