\chapter{外道迷真性~元神助本心}

却说那怪将八戒拿进洞去,道:“哥哥啊,拿将一个来了。”老魔喜道:“拿来
我看。”二魔道:“这不是?”老魔道:“兄弟,错拿了,这个和尚没用。”八戒就绰
经说道:“大王,没用的和尚,放他出去罢。不当人子!”二魔道:“哥哥,不要放
他;虽然没用,也是唐僧一起的,叫做猪八戒。把他且浸在后边净水池中,浸退了
毛衣,使盐腌着,晒干了,等天阴下酒。”八戒听言道:“蹭蹬啊!撞着个贩腌腊的
妖怪了!”那小妖把八戒抬进去,抛在水里不题。

却说三藏坐在坡前,耳热眼跳,身体不安,叫声“悟空!怎么悟能这番巡山,
去之久而不来?”行者道:“师父还不晓得他的心哩。”三藏道:“他有甚心?”行
者道:“师父啊,此山若是有怪,他半步难行,一定虚张声势,跑将回来报我;想
是无怪,路途平静,他一直去了。”三藏道:“假若真个去了,却在那里相会?此间
乃是山野空阔之处,比不得那店市城井之间。”行者道:“师父莫虑,且请上马。那
呆子有些懒惰,断然走的迟慢。你把马打动些儿,我们定赶上他,一同去罢。”真
个唐僧上马,沙僧挑担,行者前面引路上山。

却说那老怪又唤二魔道:“兄弟,你既拿了八戒,断乎就有唐僧。再去巡巡山
来,切莫放过他去。”二魔道:“就行,就行。”你看他急点起五十名小妖,上山巡
逻。

正走处,只见祥云缥缈,瑞气盘旋。二魔道:“唐僧来了。”众妖道:“唐僧在
那里?”二魔道:“好人头上祥云照顶,恶人头上黑气冲天。那唐僧原是金蝉长老
临凡,十世修行的好人,所以有这祥云缥缈。”众怪都不看见,二魔用手指道:“那
不是?”那三藏就在马上打了一个寒噤;又一指,又打个寒噤。一连指了三指,他
就一连打了三个寒噤。心神不宁道:“徒弟啊,我怎么打寒噤么?”沙僧道:“打寒
噤想是伤食病发了。”行者道:“胡说,师父是走着这深山峻岭,必然小心虚惊。莫
怕,莫怕!等老孙把棒打一路与你压压惊。”好行者,理开棒,在马前丢几个解数,
上三下四,左五右六,尽按那六韬三略,使起神通。那长老在马上观之,真个是寰
中少有,世上全无。

剖开路一直前行,险些儿不唬倒那怪物。他在山顶上看见,魂飞魄丧。忽失声
道:“几年间闻说孙行者,今日才知话不虚传果是真。”众怪上前道:“大王,怎么
长他人之志气,灭自己之威风?你夸谁哩?”二魔道:“孙行者神通广大,那唐僧吃
他不成。”众怪道:“大王,你没手段,等我们着几个去报大大王,教他点起本洞大
小兵来,摆开阵势,合力齐心,怕他走了那里去!”二魔道:“你们不曾见他那条铁
棒,有万夫不当之勇。我洞中不过有四五百兵,怎禁得他那一棒?”众妖道:“这
等说,唐僧吃不成,却不把猪八戒错拿了?如今送还他罢。”二魔道:“拿便也不曾
错拿,送便也不好轻送。唐僧终是要吃,只是眼下还尚不能。”众妖道:“这般说,
还过几年么?”二魔道:“也不消几年。我看见那唐僧,只可善图,不可恶取。若
要倚势拿他,闻也不得一闻。只可以善去感他,赚得他心与我心相合,却就善中取
计,可以图之。”众妖道:“大王如定计拿他,可用我等。”二魔道:“你们都各回本
寨,但不许报与大王知道。若是惊动了他,必然走了风汛,败了我计策。我自有个
神通变化,可以拿他。”

众妖散去,他独跳下山来,在那道路之旁,摇身一变,变做个年老的道者。真
个是怎生打扮?但见他:

星冠晃亮,鹤发蓬松。羽衣围绣带,云履缀黄棕。神清目
朗如仙客,体健身轻似寿翁。说甚么清牛道士,也强如素券先生。妆成假象如真象,
捏作虚情似实情。
他在那大路旁妆做个跌折腿的道士,脚上血淋津,口里哼哼的,只叫“救人!救人!”

却说这三藏仗着孙大圣与沙僧,欢喜前来。正行处,只听得叫“师父救人!”
三藏闻得,道:“善哉!善哉!这旷野山中,四下里更无村舍,是甚么人叫?想必是虎
豹狼虫唬倒的。”这长老兜回俊马,叫道:“那有难者是甚人?可出来。”这怪从草科
里爬出,对长老马前,乒乓的只情磕头。三藏在马上见他是个道者,却又年纪高大,
甚不过意。连忙下马搀道:“请起,请起。”那怪道:“疼,疼,疼!”丢了手看处,
只见他脚上流血。三藏惊问道:“先生啊,你从那里来?因甚伤了尊足?”那怪巧语
花言,虚情假意道:“师父啊,此山西去,有一座清幽观宇。我是那观里的道士。”
三藏道:“你不在本观中侍奉香火,演习经法,为何在此闲行?”那魔道:“因前日
山南里施主家,邀道众禳星,散福来晚,我师徒二人,一路而行。行至深衢,忽遇
着一只斑斓猛虎,将我徒弟衔去。贫道战兢兢亡命走,一跤跌在乱石坡上,伤了腿
足,不知回路。今日大有天缘,得遇师父,万望师父大发慈悲,救我一命。若得到
观中,就是典身卖命,一定重谢深恩。”三藏闻言,认为真实,道:“先生啊,你我
都是一命之人,我是僧,你是道。衣冠虽别,修行之理则同。我不救你啊,就不是
出家之辈。救便救你,你却走不得路哩。”那怪道:“立也立不起来,怎生走路?”
三藏道:“也罢,也罢。我还走得路,将马让与你骑一程,到你上宫,还我马去罢。”
那怪道:“师父,感蒙厚情,只是腿胯跌伤,不能骑马。”三藏道:“正是。”叫沙和
尚:“你把行李捎在我马上,你驮他一程罢。”沙僧道:“我驮他。”那怪急回头,抹
了他一眼,道:“师父啊,我被那猛虎唬怕了,见这晦气色脸的师父,愈加惊怕,
不敢要他驮。”三藏叫道:“悟空,你驮罢。”行者连声答应道:“我驮,我驮!”那
妖就认定了行者,顺顺的要他驮,再不言语。

沙僧笑道:“这个没眼色的老道!我驮着不好,颠倒要他驮。他若看不见师父时,
三尖石上,把筋都掼断了你的哩!”行者驮了,口中笑道:“你这个泼魔,怎么敢来
惹我!你也问问老孙是几年的人儿!你这般鬼话儿,只好瞒唐僧,又好来瞒我?我认
得你是这山中的怪物!想是要吃我师父哩。我师父又非是等闲之辈,是你吃的!你要
吃他,也须是分多一半与老孙是。”那魔闻得行者口中念诵,道:“师父,我是好人
家儿孙,做了道士。今日不幸,遇着虎狼之厄,我不是妖怪。”行者道:“你既怕虎
狼,怎么不念《北斗经》?”三藏正然上马,闻得此言,骂道:“这个泼猴!‘救人
一命,胜造七级浮屠。’你驮他驮儿便罢了,且讲甚么‘北斗经’、‘南斗经’!”行
者闻言道:“这厮造化哩!我那师父是个慈悲好善之人,又有些外好里槎。我待不
驮你,他就怪我。驮便驮,须要与你讲开:若是大小便,先和我说。若在脊梁上淋
下来,臊气不堪,且污了我的衣服,没人浆洗。”那怪道:“我这般一把子年纪,岂
不知你的话说?”行者才拉将起来,背在身上。同长老、沙僧,奔大路西行。那山
上高低不平之处,行者留心慢走,让唐僧前去。

行不上三五里路,师父与沙僧下了山凹之中,行者却望不见,心中埋怨道:“师
父偌大年纪,再不晓得事体。这等远路,就是空身子也还嫌手重,恨不得了,却
又教我驮着这个妖怪!莫说他是妖怪,就是好人,这们年纪,也死得着了,掼杀他
罢,驮他怎的?”这大圣正算计要掼,原来那怪就知道了。且会遣山,就使一个“移
山倒海”的法术,就在行者背上捻诀,念动真言,把一座须弥山遣在空中,劈头来
压行者。这大圣慌的把头偏一偏,压在左肩背上。笑道:“我的儿,你使甚么重身
法来压老孙哩?这个倒也不怕,只是‘正担好挑,偏担儿难挨。’”那魔道:“一座山
压他不住!”却又念咒语,把一座峨眉山遣在空中来压。行者又把头偏一偏,压在
右肩背上。看他挑着两座大山,飞星来赶师父!那魔头看见,就吓得浑身是汗,遍
体生津道:“他却会担山!”又整性情,把真言念动,将一座泰山遣在空中,劈头压
住行者。那大圣力软筋麻,遭逢他这泰山下顶之法,只压得三尸神咋,七窍喷红。

好妖魔,使神通压倒行者,却疾驾长风,去赶唐三藏。就于云端里伸下手来,
马上挝人。慌得个沙僧丢了行李,掣出降妖棒,当头挡住。那妖魔举一口七星剑,
对面来迎。这一场好杀:

七星剑,降妖杖,万映金光如闪亮。这个圜眼凶如黑杀神,那个铁脸真是卷帘
将。那怪山前大显能,一心要捉唐三藏。这个努力保真僧,一心宁死不肯放。他两
个喷云嗳雾照天宫,播土扬尘遮斗象。杀得那一轮红日淡无光,大地乾坤昏荡荡。
来往相持八九回,不期战败沙和尚。
那魔十分凶猛,使口宝剑,流星的解数滚来,把个沙僧战得软弱难搪,回头要走,
早被他逼住宝杖,轮开大手,挝住沙僧,挟在左胁下,将右手去马上拿了三藏,脚
尖儿钩着行李,张开口,咬着马鬃,使起摄法,把他们一阵风,都拿到莲花洞里。
厉声高叫道:“哥哥!这和尚都拿来了!”

老魔闻言,大喜道:“拿来我看。”二魔道:“这不是?”老魔道:“贤弟呀,又
错拿来了也。”二魔道:“你说拿唐僧的。”老魔道:“是便就是唐僧,只是还不曾拿
住那有手段的孙行者。须是拿住他,才好吃唐僧哩。若不曾拿得他,切莫动他的人。
那猴王神通广大,变化多般。我们若吃了他师父,他肯甘心?来那门前吵闹,莫想
能得安生。”二魔笑道:“哥啊,你也忒会抬举人。若依你夸奖他,天上少有,地下
全无;自我观之,也只如此,没甚手段。”老魔道:“你拿住了?”二魔道:“他已
被我遣三座大山压在山下,寸步不能举移。所以才把唐僧、沙和尚连马、行李,都
摄将来也。”那老魔闻言,满心欢喜道:“造化,造化!拿住这厮,唐僧才是我们口
里的食哩。”叫小妖:“快安排酒来,且与你二大王奉一个得功的杯儿。”二魔道:“哥
哥,且不要吃酒,叫小的们把猪八戒捞上水来吊起。”遂把八戒吊在东廊,沙僧吊
在西边,唐僧吊在中间,白马送在槽上,行李收将进去。

老魔笑道:“贤弟好手段,两次捉了三个和尚;但孙行者虽是有山压住,也须
要作个法,怎么拿他来凑蒸,才好哩。”二魔道:“兄长请坐。若要拿孙行者,不消
我们动身,只教两个小妖,拿两件宝贝,把他装将来罢。”老魔道:“拿甚么宝贝去?”
二魔道:“拿我的‘紫金红葫芦’,你的‘羊脂玉净瓶’。”老魔将宝贝取出道:“差
那两个去?”二魔道:“差精细鬼、伶俐虫二人去。”吩咐道:“你两个拿着这宝贝,
径至高山绝顶,将底儿朝天,口儿朝地,叫一声‘孙行者!’他若应了,就已装在
里面,随即贴上‘太上老君急急如律令奉敕’的帖儿。他就一时三刻化为脓了。”
二小妖叩头,将宝贝领出去拿行者不题。

却说那大圣被魔使法压住在山根之下,遇苦思三藏,逢灾念圣僧。厉声叫道:
“师父啊!想当时你到两界山,揭了压帖,老孙脱了大难,秉教沙门;感菩萨赐与
法旨,我和你同住同修,同缘同相,同见同知,乍想到了此处,遭逢魔障,又被他
遣山压了。可怜,可怜!你死该当,只难为沙僧、八戒与那小龙化马一场!这正是树
大招风风撼树,人为名高名丧人!”叹罢,那珠泪如雨。

早惊了山神、土地与五方揭谛神众。会金头揭谛道:“这山是谁的?”土地道:
“是我们的。”——“你山下压的是谁?”土地道:“不知是谁。”揭谛道:“你等原
来不知。这压的是五百年前大闹天宫的齐天大圣孙悟空行者。如今皈依正果,跟唐
僧做了徒弟。你怎么把山借与妖魔压他?你们是死了。他若有一日脱身出来,他肯
饶你!就是从轻,土地也问个摆站,山神也问个充军,我们也领个大不应是。”那山
神、土地才怕道:“委实不知,不知。只听得那魔头念起遣山咒法,我们就把山移
将来了。谁晓得是孙大圣?”揭谛道:“你且休怕。律上有云:‘不知者不坐。’我
与你计较,放他出来,不要教他动手打你们。”土地道:“就没理了;既放出来又打?”
揭谛道:“你不知。他有一条如意金箍棒,十分利害:打着的就死,挽着的就伤;
磕一磕儿筋断,擦一擦儿皮塌哩!”

那土地、山神,心中恐惧,与五方揭谛商议了,却来到三山门外叫道:“大圣!
山神、土地、五方揭谛来见。”好行者,他虎瘦雄心还在,自然的气象昂昂,声音
朗朗道:“见我怎的?”土地道:“告大圣得知。遣开山,请大圣出来,赦小神不恭
之罪。”行者道:“遣开山,不打你。”

喝声“起去!”就如官府发放一般。那众神念动真言咒语,把山仍遣归本位,
放起行者。行者跳将起来,抖抖土,束束裙,耳后掣出棒来,叫山神、土地:“都
伸过孤拐来,每人先打两下,与老孙散散闷!”众神大惊道:“刚才大圣已吩咐,恕
我等之罪;怎么出来就变了言语要打?”行者道:“好土地!好山神!你倒不怕老孙,
却怕妖怪!”土地道:“那魔神通广大,法术高强,念动真言咒语,拘唤我等在他洞
里,一日一个轮流当值哩!”行者听见“当值”二字,却也心惊。仰面朝天,高声
大叫道:“苍天!苍天!自那混沌初分,天开地辟,花果山生了我,我也曾遍访明师,
传授长生秘诀。想我那随风变化,伏虎降龙,大闹天宫,名称大圣。更不曾把山神、
土地欺心使唤。今日这个妖魔无状,怎敢把山神、土地唤为奴仆,替他轮流当值?
天啊!既生老孙,怎么又生此辈?”

那大圣正感叹间,又见山凹里霞光焰焰而来。行者道:“山神、土地,你既在
这洞中当值,那放光的是甚物件?”土地道:“那是妖魔的宝贝放光,想是有妖精
拿宝贝来降你。”行者道:“这个却好耍子儿啊!我且问你,他这洞中有甚人与他相
往?”土地道:“他爱的是烧丹炼药,喜的是全真道人。”行者道:“怪道他变个老
道士,把我师父骗去了。既这等,你都且记打,回去罢。等老孙自家拿他。”那众
神俱腾空而散。

这大圣摇身一变,变做个老真人。你道他怎生打扮:
头挽双髻,身穿百衲衣。
手敲渔鼓简,腰系吕公绦。
斜倚大路下,专候小魔妖。
顷刻妖来到,猴王暗放刁。
不多时,那两个小妖到了。行者将金箍棒伸开,那妖不曾防备,绊着脚,扑的一跌。
爬起来,才看见行者,口里嚷道:“惫懒,惫懒!若不是我大王敬重你这行人,就和
比较起来。”行者陪笑道:“比较甚么?道人见道人,都是一家人。”那怪道:“你怎
么睡在这里,绊我一跌?”行者道:“小道童见我这老道人,要跌一跌儿做见面钱。”
那妖道:“我大王见面钱只要几两银子,你怎么跌一跌儿做见面钱?你别是一乡风,
决不是我这里道士。”行者道:“我当真不是。我是蓬莱山来的。”那妖道:“蓬莱山
是海岛神仙境界。”行者道:“我不是神仙,谁是神仙?”那妖却回嗔作喜,上前道:
“老神仙,老神仙!我等肉眼凡胎,不能识认,言语冲撞,莫怪,莫怪。”行者道:
“我不怪你。常言道:‘仙体不踏凡地’,你怎知之?我今日到你山上,要度一个成
仙了道的好人。那个肯跟我去?”精细鬼道:“师父,我跟你去。”伶俐虫道:“师
父,我跟你去。”

行者明知故问道:“你二位从那里来的?”那怪道:“自莲花洞来的。”“要往那
里去?”那怪道:“奉我大王教命,拿孙行者去的。”行者道:“拿那个?”那怪又
道:“拿孙行者。”孙行者道:“可是跟唐僧取经的那个孙行者么?”那妖道:“正是,
正是。你也认得他?”行者道:“那猴子有些无礼。我认得他。我也有些恼他。我
与你同拿他去,就当与你助功。”那怪道:“师父,不须你助功。我二大王有些法术,
遣了三座大山把他压在山下,寸步难移,教我两个拿宝贝来装他的。”行者道:“是
甚宝贝?”精细鬼道:“我的是‘红葫芦’,他的是‘玉净瓶’。”行者道:“怎么样
装他?”小妖道:“把这宝贝的底儿朝天,口儿朝地,叫他一声,他若应了,就装
在里面;贴上一张‘太上老君急急如律令奉敕’的帖子,他就一时三刻化为脓了。”
行者见说,心中暗惊道:“利害,利害!当时日值功曹报信,说有五件宝贝,这是两
件了;不知那三件又是甚么东西?”行者笑道:“二位,你把宝贝借我看看。”那小
妖那知甚么诀窍,就于袖中取出两件宝贝,双手递与行者。行者见了,心中暗喜道:
“好东西,好东西!我若把尾子一抉,飕的跳起走了,只当是送老孙。”忽又思道:
“不好,不好!抢便抢去,只是坏了老孙的名头。这叫做白日抢夺了。”复递与他去,
道:“你还不曾见我的宝贝哩。”那怪道:“师父有甚宝贝?也借与我凡人看看压灾。”

好行者,伸下手把尾上毫毛拔了一根,捻一捻,叫“变!”即变做一个一尺七
寸长的大紫金红葫芦,自腰里拿将出来道:“你看我的葫芦么?”那伶俐虫接在手,
看了道:“师父,你这葫芦长大,有样范,好看,却只是不中用。”行者道:“怎的
不中用?”那怪道:“我这两件宝贝,每一个可装千人哩。”行者道:“你这装人的,
何足稀罕?我这葫芦,连天都装在里面哩!”那怪道:“就可以装天?”行者道:“当
真的装天。”那怪道:“只怕是谎。就装与我们看看才信;不然,决不信你。”行者
道:“天若恼着我,一月之间,常装他七八遭。不恼着我,就半年也不装他一次。”
伶俐虫道:“哥啊,装天的宝贝,与他换了罢。”精细鬼道:“他装天的,怎肯与我
装人的相换?”伶俐虫道:“若不肯啊,贴他这个净瓶也罢。”行者心中暗喜道:“葫
芦换葫芦,余外贴净瓶:一件换两件,其实甚相应!”即上前扯住那伶俐虫道:“装
天可换么?”那怪道:“但装天就换;不换,我是你的儿子!”行者道:“也罢,也
罢,我装与你们看看。”

好大圣,低头捻诀,念个咒语,叫那日游神、夜游神、五方揭谛神:“即去与
我奏上玉帝,说老孙皈依正果,保唐僧去西天取经,路阻高山,师逢苦厄。妖魔那
宝,吾欲诱他换之,万千拜上,将天借与老孙装闭半个时辰,以助成功。若道半声
不肯,即上灵霄殿,动起刀兵!”

那日游神径至南天门里,灵霄殿下,启奏玉帝,备言前事。玉帝道:“这泼猴
头,出言无状。前者观音来说,放了他保护唐僧,朕这里又差五方揭谛、四值功曹,
轮流护持,如今又借天装,天可装乎?”才说装不得,那班中闪出哪吒三太子,奏
道:“万岁,天也装得。”玉帝道:“天怎样装?”哪吒道:“自混沌初分,以轻清为
天,重浊为地。天是一团清气而扶托瑶天宫阙,以理论之,其实难装;但只孙行者
保唐僧西去取经,诚所谓泰山之福缘,海深之善庆,今日当助他成功。”玉帝道:“卿
有何助?”哪吒道:“请降旨意,往北天门问真武借皂雕旗在南天门上一展,把那
日月星辰闭了。对面不见人,捉白不见黑,哄那怪道,只说装了天,以助行者成功。”
玉帝闻言:“依卿所奏。”那太子奉旨,前来北天门,见真武,备言前事。那祖师随
将旗付太子。

早有游神急降大圣耳边道:“哪吒太子来助功了。”行者仰面观之,只见祥云缭
绕,果是有神。却回头对小妖道:“装天罢。”小妖道:“要装就装,只管‘阿绵花
屎’怎的?”行者道:“我方才运神念咒来。”那小妖都睁着眼,看他怎么样装天。
这行者将一个假葫芦儿抛将上去。你想,这是一根毫毛变的,能有多重?被那山顶
上风吹去,飘飘荡荡,足有半个时辰,方才落下。只见那南天门上,哪吒太子把皂
旗拨喇喇展开,把日月星辰俱遮闭了。真是乾坤墨染就,宇宙靛装成。二小妖大惊
道:“才说话时,只好向午,却怎么就黄昏了?”行者道:“天既装了,不辨时候,
怎不黄昏!”——“如何又这等样黑?”行者道:“日月星辰都装在里面,外却无光,
怎么不黑!”小妖道:“师父,你在那厢说话哩?”行者道:“我在你面前不是?”
小妖伸手摸着道:“只见说话,更不见面目。师父,此间是甚么去处?”行者又哄
他道:“不要动脚,此间乃是渤海岸上。若塌了脚,落下去啊,七八日还不得到底
哩!”小妖大惊道:“罢!罢!罢!放了天罢。我们晓得是这样装了。若弄一会子,落
下海去,不得归家!”

好行者,见他认了真实,又念咒语,惊动太子,把旗卷起,却早见日光正午。
小妖笑道:“妙啊!妙啊!这样好宝贝,若不换啊,诚为不是养家的儿子!”那精细鬼
交了葫芦,伶俐虫拿出净瓶,一齐儿递与行者。行者却将假葫芦儿递与那怪。行者
既换了宝贝,却又干事找绝:脐下拔一根毫毛,吹口仙气,变作一个铜钱。叫道:
“小童,你拿这个钱去买张纸来。”小妖道:“何用?”行者道:“我与你写个合同
文书。你将这两件装人的宝贝换了我一件装天的宝贝,恐人心不平,向后去日久年
深,有甚反悔不便,故写此各执为照。”小妖道:“此间又无笔墨,写甚文书?我与
你赌个咒罢。”行者道:“怎么样赌?”小妖道:“我两件装人之宝,贴换你一件装
天之宝,若有反悔,一年四季遭瘟。”行者笑道:“我是决不反悔;如有反悔,也照
你四季遭瘟。”说了誓,将身一纵,把尾子了一,跳在南天门前,谢了哪吒太
子麾旗相助之功。太子回宫缴旨,将旗送还真武不题。这行者伫立霄汉之间,观看
那个小妖。

毕竟不知怎生区处,且听下回分解。