\chapter{寻洞擒妖逢老寿~当朝正主救婴儿}

却说那锦衣官把“假唐僧”扯出馆驿,与羽林军围围绕绕,直至朝门外,对黄
门官言:“我等已请唐僧到此,烦为转奏。”黄门官急进朝,依言奏上昏君,遂请进
去。

众官都在阶下跪拜,惟“假唐僧”挺立阶心,口中高叫:“比丘王,请我贫僧
何说?”君王笑道:“朕得一疾,缠绵日久不愈。幸国丈赐得一方,药饵俱已完备,
只少一味引子。特请长老,求些药引。若得病愈,与长老修建祠堂,四时奉祭,永
为传国之香火。”“假唐僧”道:“我乃出家人,只身至此,不知陛下问国丈要甚东
西作引。”昏君道:“特求长老的心肝。”“假唐僧”道:“不满陛下说。心便有几个
儿,不知要的甚么色样。”那国丈在旁指定道:“那和尚,要你的黑心。”“假唐僧”
道:“既如此,快取刀来,剖开胸腹。若有黑心,谨当奉命。”那昏君欢喜相谢,即
着当驾官取一把牛耳短刀,递与假僧。

假僧接刀在手,解开衣服,忝起胸膛,将左手抹腹,右手持刀,唿喇的响一声,
把腹皮剖开,那里头就骨都都的滚出一堆心来。唬得文官失色,武将身麻。国丈在
殿上见了道:“这是个多心的和尚!”假僧将那些心,血淋淋的,一个个捡开与众观
看,却都是些红心、白心、黄心、悭贪心、利名心、嫉妒心、计较心、好胜心、望
高心、侮慢心、杀害心、狠毒心、恐怖心、谨慎心、邪妄心、无名隐暗之心、种种
不善之心,更无一个黑心。那昏君唬得呆呆挣挣,口不能言,战兢兢的教:“收了
去!收了去!”那“假唐僧”忍耐不住,收了法,现出本相。对昏君道:“陛下全无
眼力!我和尚家都是一片好心,惟你这国丈是个黑心,好做药引。你不信,等我替
你取他的出来看看。”

那国丈听见,急睁睛仔细观看。见那和尚变了面皮,不是那般模样。咦!认得
当年孙大圣,五百年前旧有名。却抽身,腾云就起。被行者翻筋斗,跳在空中喝道:
“那里走,吃吾一棒!”那国丈即使蟠龙拐杖来迎。他两个在半空中这场好杀:

如意棒,蟠龙拐,虚空一片云。原来国丈是妖精,故将怪女称娇色。国主
贪欢病染身,妖邪要把儿童宰。相逢大圣显神通,捉怪救人将难解。铁棒当头着实
凶,拐棍迎来堪喝采。杀得那满天雾气暗城池,城里人家都失色。文武多官魂魄飞,
嫔妃绣女容颜改。唬得那比丘昏主乱身藏,战战兢兢没布摆。棒起犹如虎出山。拐
轮却似龙离海。今番大闹比丘城,致令邪正分明白。
那妖精与行者苦战二十余合,蟠龙拐抵不住金箍棒,虚幌了一拐,将身化作一道寒
光,落入皇宫内院,把进贡的妖后带出宫门,并化寒光,不知去向。

大圣按落云头,到了宫殿下。对多官道:“你们的好国丈啊!”多官一齐礼拜,
感谢神僧。行者道:“且休拜,且去看你那昏主何在。”多官道:“我主见争战时,
惊恐潜藏,不知向那座宫中去也。”行者即命:“快寻,莫被美后拐去!”多官听言,
不分内外,同行者先奔美后宫,漠然无踪,连美后也通不见了。正宫、东宫、西宫、
六院,概众后妃,都来拜谢大圣。大圣道:“且请起,不到谢处哩。且去寻你主公。”

少时,见四五个太监,搀着那昏君自谨身殿后面而来。众臣俯伏在地,齐声启
奏道:“主公!主公!感得神僧到此,辨明真假。那国丈乃是个妖邪,连美后亦不见
矣。”国王闻言,即请行者出皇宫,到宝殿,拜谢了道:“长老,你早间来的模样,
那般俊伟,这时如何就改了形容?”行者笑道:“不瞒陛下说。早间来者,是我师
父,乃唐朝御弟三藏。我是他徒弟孙悟空。还有两个师弟:猪悟能、沙悟净,见在
金亭馆驿。因知你信了妖言,要取我师父心肝做药引,是老孙变作师父模样,特来
此降妖也。”那国王闻说,即传旨着阁下太宰快去驿中请师众来朝。

那三藏听见行者现了相,在空中降妖,吓得魂飞魄散。幸有八戒、沙僧护持。
他又脸上戴着一片子臊泥,正闷闷不快,只听得人叫道:“法师,我等乃比丘国王
差来的阁下太宰,特请入朝谢恩也。”八戒笑道:“师父,莫怕,莫怕!这不是又请
你取心,想是师兄得胜,请你酬谢哩。”三藏道:“虽是得胜来请,但我这个臊脸,
怎么见人?”八戒道:“没奈何,我们且去见了师兄,自有解释。”真个那长老无计,
只得扶着八戒、沙僧挑着担,牵着马,同去驿庭之上。那太宰见了,害怕道:“爷
爷呀!这都相似妖头怪脑之类!”沙僧道:“朝士休怪丑陋。我等乃是生成的遗体。
若我师父,来见了我师兄,他就俊了。”

他三人与众来朝,不待宣召,直至殿下。行者看见,即转身下殿,迎着面,把
师父的泥脸子抓下,吹口仙气,叫:“正!”那唐僧即时复了原身,精神愈觉爽利。
国王下殿亲迎,口称法师老佛。师徒们将马拴住,都上殿来相见。

行者道:“陛下可知那怪来自何方?等老孙去与你一并擒来,剪除后患。”三宫
六院,诸嫔群妃,都在那翡翠屏后;听见行者说剪除后患,也不避内外男女之嫌,
一齐出来拜告道:“万望神僧老佛大施法力,斩草除根,把他剪除尽绝,诚为莫大
之恩,自当重报!”行者忙忙答礼,只教国王说他住居。国王含羞告道:“三年前他
到时,朕曾问他。他说离城不远,只在向南去七十里路,有一座柳林坡清华庄上。
国丈年老无儿,止后妻生一女,年方十六,不曾配人,愿进与朕。朕因那女貌娉婷,
遂纳了,宠幸在宫。不期得疾,太医屡药无功。他说我有仙方,止用小儿心煎汤为
引。是朕不才,轻信其言,遂选民间小儿,选定今日午时开刀取心。不料神僧下降,
恰恰又遇笼儿都不见了。他就说神僧十世修真,元阳未泄,得其心,比小儿心更加
万倍。一时误犯,不知神僧识透妖魔。敢望广施大法,剪其后患,朕以倾国之资酬
谢!”行者笑道:“实不相瞒。笼中小儿,是我师慈悲,着我藏了。你且休题甚么资
财相谢,待我捉了妖怪,是我的功行。”叫:“八戒,跟我去来。”八戒道:“谨依兄
命。但只是腹中空虚,不好着力。”国王即传旨教:“光禄寺快办斋供。”

不一时,斋到,八戒尽饱一餐,抖擞精神,随行者驾云而起。唬得那国王、妃
后,并文武多官,一个个朝空礼拜。都道:“是真仙真佛降临凡也!”那大圣携着八
戒,径到南方七十里之地,住下风云,找寻妖处。但只见一股清溪,两边夹岸,岸
上有千千万万的杨柳,更不知清华庄在于何处。正是那:
万顷野田观不尽,千堤烟柳隐无踪。
孙大圣寻觅不着,即捻诀,念一声“”字真言,拘出一个当方土地,战兢兢近前
跪下叫道:“大圣,柳林坡土地叩头。”行者道:“你休怕,我不打你。我问你:柳
林坡有个清华庄,在于何方?”土地道:“此间有个清华洞,不曾有个清华庄。小
神知道了,大圣想是自比丘国来的?”行者道:“正是,正是。比丘国王被一个妖
精哄了。是老孙到那厢,识得是妖怪,当时战退那怪,化一道寒光,不知去向。及
问比丘王,他说三年前进美女时,曾问其由,怪言居住城南七十里柳林坡清华庄。
适寻到此,只见林坡,不见清华庄,是以问你。”土地叩头道:“望大圣恕罪。比丘
王亦我地之主也,小神理当鉴察;奈何妖精神威法大,如我泄漏他事,就来欺凌,
故此未获。大圣今来,只去那南岸九叉头一颗杨树根下,左转三转,右转三转,用
两手齐扑树上,连叫三声‘开门’,即现清华洞府。”

大圣闻言,即令土地回去,与八戒跳过溪来,寻那颗杨树。果然有九条叉枝,
总在一颗根上。行者吩咐八戒:“你且远远的站定,待我叫开门,寻着那怪,赶将
出来,你却接应。”八戒闻命,即离树有半里远近立下。这大圣依土地之言,绕树
根,左转三转,右转三转,双手齐扑其树,叫:“开门,开门!”霎时间,一声响,
唿喇喇的门开两扇,更不见树的踪迹。那里边光明霞采,亦无人烟。行者趁神威,
撞将进去,但见那里好个去处:

烟霞幌亮,日月偷明。白云常出洞,翠藓乱漫庭。一径奇花争艳丽,遍阶瑶草
斗芳荣。温暖气,景常春,浑如阆苑,不亚蓬瀛。滑凳攀长蔓,平桥挂乱藤。蜂衔
红蕊来岩窟,蝶戏幽兰过石屏。
行者急拽步,行近前边细看。见石屏上有四个大字:“清华仙府”。他忍不住,跳过
石屏看处,只见那老怪怀中搂着个美女,喘嘘嘘的,正讲比丘国事,齐声叫道:“好
机会来,三年事,今日得完,被那猴头破了!”

行者跑近身,掣棒高叫道:“我把你这伙毛团!甚么‘好机会’,吃吾一棒!”那
老怪丢放美人,轮起蟠龙拐,急架相迎。他两个在洞前,这场好杀,比前又甚不同:

棒举迸金光,拐轮凶气发。那怪道:“你无知敢进我门来!”行者道:“我有意
降邪怪!”那怪道:“我恋国主你无干,怎的欺心来展抹?”行者道:“僧修政教本
慈悲,不忍儿童活见杀。”语去言来各恨仇,棒迎拐架当心札。促损琪花为顾生,
踢破翠苔因把滑。只杀得那洞中霞采欠光明,岩上芳菲俱掩压。乒乓惊得鸟难飞,
吆喝吓得美人散。只存老怪与猴王,呼呼卷地狂风刮。看看杀出洞门来,又撞悟能
呆性发。

原来八戒在外边,听见他们里面嚷闹,激得他心痒难挠,掣钉钯,把一棵九叉
杨树刨倒,使钯筑了几下,筑得那鲜血直冒,嘤嘤的似乎有声。他道:“这棵树成
了精也!这棵树成了精也!”按在地下,又正筑处,只见行者引怪出来。那呆子不打
话,赶上前,举钯就筑。那老怪战行者已是难敌,见八戒钯来,愈觉心慌,败了阵,
将身一幌,化道寒光,径投东走。他两个决不放松,向东赶来。

正当喊杀之际,又闻得鸾鹤声鸣,祥光缥缈。举目视之,乃南极老人星也。那
老人把寒光罩住。叫道:“大圣慢来,天蓬休赶。老道在此施礼哩。”行者即答礼道:
“寿星兄弟,那里来?”八戒笑道:“肉头老儿,罩住寒光,必定捉住妖怪了。”寿
星陪笑道:“在这里,在这里。望二公饶他命罢。”行者道:“老怪不与老弟相干,
为何来说人情?”寿星笑道:“他是我的一副脚力,不意走将来,成此妖怪。”行者
道:“既是老弟之物,只教他现出本相来看看。”寿星闻言,即把寒光放出,喝道:
“孽畜!快现本相,饶你死罪!”那怪打个转身,原来是只白鹿。寿星拿起拐杖道:
“这孽畜!连我的拐棒也偷来也!”那只鹿俯伏在地,口不能言,只管叩头滴泪。但
见他:
一身如玉简斑斑,两角参差七汊湾。
几度饥时寻药圃,有朝渴处饮云潺。
年深学得飞腾法,日久修成变化颜。
今见主人呼唤处,现身珉耳伏尘寰。

寿星谢了行者,就跨鹿而行。被行者一把扯住道:“老弟,且慢走,还有两件
事未完哩。”寿星道:“还有甚么未完之事?”行者道:“还有美人未获,不知是个
甚么怪物;还又要同到比丘城见那昏君,现相回旨也。”寿星道:“既这等说,我且
宁耐。我与天蓬下洞擒捉那美人来,同去现相可也。”行者道:“老弟略等等儿,我
们去了就来。”

那八戒抖擞精神,随行者径入清华仙府,呐声喊,叫:“拿妖精!拿妖精!”那
美人战战兢兢,正自难逃,又听得喊声大振,即转石屏之内,又没个后门出头;被
八戒喝声:“那里走!我把你这个哄汉子的臊精,看钯!”那美人手中又无兵器,不
能迎敌,将身一闪,化道寒光,往外就走;被大圣抵住寒光,乒乓一棒,那怪立不
住脚,倒在尘埃,现了本相,原来是一个白面狐狸。呆子忍不住手,举钯照头一筑,
可怜把那个倾城倾国千般笑,化作毛团狐狸形!行者叫道:“莫打烂他,且留他此身
去见昏君。”

那呆子不嫌秽污,一把揪住尾子,拖拖扯扯,跟随行者出得门来。只见那寿星
老儿手摸着鹿头骂道:“好孽畜啊!你怎么背主逃去,在此成精!若不是我来,孙大
圣定打死你了。”行者跳出来道:“老弟说甚么?”寿星道:“我嘱鹿哩!我嘱鹿哩!”
八戒将个死狐狸掼在鹿的面前道:“这可是你的女儿么?”那鹿点头幌脑,伸着嘴,
闻他几闻,呦呦发声,似有眷恋不舍之意。被寿星劈头扑了一掌道:“孽畜!你得命
足矣,又闻他怎的?”即解下勒袍腰带,把鹿扣住颈项,牵将起来,道:“大圣,
我和你比丘国相见去也。”行者道:“且住!索性把这边都扫个干净,庶免他年复生
妖孽。”

八戒闻言,举钯将柳树乱筑。行者又念声“”字真言,依然拘出当坊土地,
叫:“寻些枯柴,点起烈火,与你这方消除妖患,以免欺凌。”那土地即转身,阴风
飒飒,帅起阴兵,搬取了些迎霜草、秋青草、蓼节草、山蕊草、蒌蒿柴、龙骨柴、
芦荻柴,都是隔年干透的枯焦之物,见火如同油腻一般。行者叫:“八戒,不必筑
树。但得此物填塞洞里,放起火来,烧得个干净。”火一起,果然把一座清华妖怪
宅,烧作火池坑。

这里才喝退土地,同寿星牵着鹿,拖着狐狸,一齐回到殿前,对国王道:“这
是你的美后。与他耍子儿么?”那国王胆战心惊。又只见孙大圣引着寿星,牵着白
鹿,都到殿前,唬得那国里君臣妃后,一齐下拜。行者近前,搀住国王,笑道:“且
休拜我。这鹿儿却是国丈,你只拜他便是。”那国王羞愧无地,只道:“感谢神僧救
我一国小儿,真天恩也!”即传旨教光禄寺安排素宴,大开东阁,请南极老人与唐
僧四众,共坐谢恩。三藏拜见了寿星,沙僧亦以礼见。都问道:“白鹿既是老寿星
之物,如何得到此间为害?”寿星笑道:“前者,东华帝君过我荒山,我留坐着棋,
一局未终,这孽畜走了。及客去寻他不见,我因屈指询算,知他走在此处,特来寻
他,正遇着孙大圣施威。若果来迟,此畜休矣。”叙不了,只见报道:“宴已完备。”
好素宴:

五彩盈门,异香满座。桌挂绣纬生锦艳,地铺红毯幌霞光。宝鸭内,沉檀香袅;
御筵前,蔬品香馨。看盘高果砌楼台,龙缠斗糖摆走兽。鸳鸯锭,狮仙糖,似模似
样;鹦鹉杯,鹭鹚杓,如相如形。席前果品般般盛,案上斋肴件件精。魁圆茧栗,
鲜荔桃子。枣儿柿饼味甘甜,松子葡萄香腻酒。几般蜜食,数品蒸酥。油札糖浇,
花团锦砌。金盘高垒大馍馍,银碗满盛香稻饭。辣炒炒汤水粉条长,香喷喷相连添
换美。说不尽蘑菇木耳、嫩笋黄精,十香素菜,百味珍馐。往来绰摸不曾停,进退
诸般皆盛设。
当时叙了坐次,寿星首席,长老次席,国王前席。行者、八戒、沙僧侧席。旁又有
两三个太师相陪左右。即命教坊司动乐。国王擎着紫霞杯,一一奉酒。惟唐僧不饮。
八戒向行者道:“师兄,果子让你,汤饭等须请让我受用受用。”那呆子不分好歹,
一齐乱上,但来的吃个精空。

一席筵宴已毕,寿星告辞。那国王又近前跪拜寿星,求祛病延年之法。寿星笑
道:“我因寻鹿,未带丹药。欲传你修养之方,你又筋衰神败,不能还丹。我这衣
袖中,只有三个枣儿,是与东华帝君献茶的,我未曾吃,今送你罢。”国王吞之,
渐觉身轻病退。后得长生者,皆原于此。八戒看见,就叫道:“老寿,有火枣,送
我几个吃吃。”寿星道:“未曾带得。待改日我送你几斤。”遂出了东阁,道了谢意,
将白鹿一声喝起,飞跨背上,踏云而去。这朝中君王妃后,城中黎庶居民,各各焚
香礼拜不题。

三藏叫:“徒弟,收拾辞王。”那国王又苦留求教。行者道:“陛下,从此色欲
少贪,阴功多积,凡百事将长补短,自足以祛病延年,就是教也。”遂拿出两盘散
金碎银,奉为路费。唐僧坚辞,分文不受。国王无已,命摆銮驾,请唐僧端尘凤辇
龙车,王与嫔后,俱推轮转毂,方送出朝。六街三市,百姓群黎,亦皆盏添净水,
炉降真香,又送出城。忽听得半空中一声风响,路两边落下一千一百一十一个鹅笼,
内有小儿啼哭,暗中有原护的城隍、土地、社令、真官、五方揭谛、四值功曹、六
丁六甲、护教伽蓝等众,应声高叫道:“大圣,我等前蒙吩咐,摄去小儿鹅笼,今
知大圣功成起行,一一送来也。”那国王妃后与一应臣民,又俱下拜。行者望空道:
“有劳列位,请各归祠,我着民间祭祀谢你。”呼呼淅淅,阴风又起而退。

行者叫城里人家来认领小儿。当时传播,俱来各认出笼中之儿,欢欢喜喜,抱
出叫哥哥,叫肉儿,跳的跳,笑的笑,都叫:“扯住唐朝爷爷,到我家奉谢救儿之
恩!”无大无小,若男若女,都不怕他相貌之丑,抬着猪八戒,扛着沙和尚,顶着
孙大圣,撮着唐三藏,牵着马,挑着担,一拥回城。那国王也不能禁止。这家也开
宴,那家也设席。请不及的,或做僧帽、僧鞋、褊衫、布袜,里里外外,大小衣裳,
都来相送。如此盘桓,将有个月,才得离城。又有传下影神,立起牌位,顶礼焚香
供养。这才是:
阴功高垒恩山重,救活千千万万人。

毕竟不知向后又有甚么事体,且听下回分解。