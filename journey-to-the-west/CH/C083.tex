\chapter{心猿识得丹头~姹女还归本性}

却说三藏着妖精送出洞外,沙和尚近前问曰:“师父出来,师兄何在?”八戒
道:“他有算计,必定贴换师父出来也。”三藏用手指着妖精道:“你师兄在他肚里
哩。”八戒笑道:“腌脏杀人,在肚里做甚?出来罢!”行者在里边叫道:“张开口,
等我出来!”那怪真个把口张开。行者变得小小的,在咽喉之内,正欲出来,又
恐他无理来咬,即将铁棒取出,吹口仙气,叫“变!”变作个枣核钉儿,撑住他的
上腭子,把身一纵,跳出口外,就把铁棒顺手带出,把腰一躬,还是原身法象,举
起棒来就打。那妖精也随手取出两口宝剑,丁当架住。两个在山头上这场好杀:

双舞剑飞当面架,金箍棒起照头来。一个是天生猴属心猿体,一个是地产精灵
姹女骸。他两个,恨冲怀,喜处生仇大会垓。那个要取元阳成配偶,这个要战纯阴
结圣胎。棒举一天寒雾漫,剑迎满地黑尘筛。因长老,拜如来,恨苦相争显大才。
水火不投母道损,阴阳难合各分开。两家斗罢多时节,地动山摇树木摧。
八戒见他们赌斗,口里絮絮叨叨,返恨行者。转身对沙僧道:“兄弟,师兄胡缠!才
子在他肚里,轮起拳来,送他一个满肚红,扒开肚皮钻出来,却不了帐?怎么又从
他口里出来;却与他争战,让他这等猖狂!”沙僧道:“正是。却也亏了师兄深洞中
救出师父,返又与妖精厮战。且请师父自家坐着,我和你各持兵器,助助大哥,打
倒妖精去来。”八戒摆手道:“不,不,不!他有神通,我们不济。”沙僧道:“说那
里话!都是大家有益之事。虽说不济,却也放屁添风。”

那呆子一时兴发,掣了钉钯,叫声:“去来!”他两个不顾师父,一拥驾风赶上。
举钉钯,使宝杖,望妖精乱打。那妖精战行者一个已是不能,又见他二人,怎生抵
敌,急回头,抽身就走。行者喝道:“兄弟们赶上!”那妖精见他们赶得紧,即将右
脚上花鞋脱下来,吹口仙气,念个咒语,叫“变!”即变作本身模样,使两口剑舞
将来;将身一幌,化一阵清风,径直回去。这番也只说战他们不过,顾命而回,岂
知又有这般样事!也是三藏灾星未退:他到了洞门前牌楼下,却见唐僧在那里独坐,
他就近前一把抱住,抢了行李,咬断缰绳,连人和马,复又摄将进去不题。

且说八戒闪个空,一钯把妖精打落地,乃是一只花鞋。行者看见道:“你这两
个呆子,看着师父罢了,谁要你来帮甚么功!”八戒道:“沙和尚,如何么!我说莫
来。这猴子好的有些夹脑风。我们替他降了妖怪,返落得他生报怨!”行者道:“在
那里降了妖怪!那妖怪昨日与我战时,使了一个遗鞋计哄了。你们走了,不知师父
如何,我们快去看看!”

三人急回来,果然没了师父;连行李、白马一并无踪。慌得个八戒两头乱跑,
沙僧前后跟寻。孙大圣亦心焦性燥。正寻觅处,只见那路旁边斜着半截儿缰绳。
他一把拿起,止不住眼中流泪,放声叫道:“师父啊!我去时辞别人和马,回来只见
这些绳!”正是那“见鞍思俊马,滴泪想亲人”。八戒见他垂泪,忍不住仰天大笑。
行者骂道:“你这个夯货,又是要散火哩!”八戒又笑道:“哥啊,不是这话。师父
一定又被妖精摄进洞去了。常言道:‘事无三不成。’你进洞两遭了,再进去一遭,
管情救出师父来也。”行者揩了眼泪道:“也罢,到此地位,势不容己,我还进去。
你两个没了行李、马匹耽心,却好生把守洞口。”

好大圣,即转身跳入里面,不施变化,就将本身法相。真个是:

古怪别腮心里强,自小为怪神力壮。高低面赛马鞍鞒,眼放金光如火亮。浑身
毛硬似钢针,虎皮裙系明花响。上天撞散万云飞,下海混起千层浪。当天倚力打天
王,挡退十万八千将。官封大圣美猴精,手中惯使金箍棒。今日西方任显能,复来
洞内扶三藏。

你看他停住云光,径到了妖精宅外。见那门楼门关了,不分好歹,轮铁棒一下
打开,闯将进去。那里边静悄悄,全无人迹。东廊下不见唐僧;亭子上桌椅,与各
处家火,一件也无。原来他的洞里周围有三百余里,妖精窠穴甚多。前番摄唐僧在
此,被行者寻着,今番摄了,又怕行者来寻,当时搬了,不知去向。恼得这行者跌
脚捶胸,放声高叫道:“师父啊!你是个晦气转成的唐三藏,灾殃铸就的取经僧!噫,
这条路且是走熟了,如何不在?却教老孙那里寻找也!”

正自吆喝爆燥之间,忽闻得一阵香烟扑鼻,他回了性道:“这香烟是从后面飘
出,想是在后头哩。”拽开步,提着铁棒,走将进去看时,也不见动静。只见有三
间倒坐儿,近后壁却铺一张龙吞口雕漆供桌,桌上有一个大流金香炉,炉内有香烟
馥郁。那上面供养着一个大金字牌,牌上写着“尊父李天王位”;略次些儿,写着
“尊兄哪吒三太子位”。行者见了,满心欢喜,也不去搜妖怪,找唐僧,把铁棒捻
作个绣花针儿,在耳朵里,轮开手,把那牌子并香炉拿将起来,返云光,径出门
去。至洞口,唏唏哈哈,笑声不绝。

八戒、沙僧听见,掣放洞口,迎着行者道:“哥哥这等欢喜,想是救出师父也?”
行者笑道:“不消我们救,只问这牌子要人。”八戒道:“哥啊,这牌子不是妖精,
又不会说话,怎么问他要人?”行者放在地下道:“你们看!”沙僧近前看时,上写
着“尊父李天王之位”、“尊兄哪吒三太子位”。沙僧道:“此意何也?”行者道:“这
是那妖精家供养的。我闯入他住居之所,见人迹俱无,惟有此牌。想是李天王之女,
三太子之妹,思凡下界,假扮妖邪,将我师父摄去。不问他要人,却问谁要?你两
个且在此把守,等老孙执此牌位,径上天堂玉帝前告个御状,教天王爷儿们,还我
师父。”八戒道:“哥啊,常言道:‘告人死罪得死罪。’须是理顺,方可为之。况御
状又岂是可轻易告的?你且与我说,怎的告他。”行者笑道:“我有主张。我把这牌
位、香炉做个证见,另外再备纸状儿。”八戒道:“状儿上怎么写?你且念念我听。”
行者道:

“告状人孙悟空,年甲在牒,系东土唐朝西天取经僧唐三藏徒弟。告为假妖摄
陷人口事。今有托塔天王李靖同男哪吒太子,闺门不谨,走出亲女,在下方陷空山
无底洞变化妖邪,迷害人命无数。今将吾师摄陷曲邃之所,渺无寻处。若不状告,
切思伊父子不仁,故纵女氏成精害众。伏乞怜准,行拘至案,收邪救师,明正其罪,
深为恩便。有此上告。”
八戒、沙僧闻其言,十分欢喜道:“哥啊,告的有理,必得上风。切须早来;稍迟
恐妖精伤了师父性命。”行者道:“我快,我快!多时饭熟,少时茶滚就回。”

好大圣,执着这牌位、香炉,将身一纵,驾祥云,直至南天门外。时有把天门
的大力天王与护国天王见了行者,一个个都控背躬身,不敢拦阻,让他进去。直至
通明殿下,有张、葛、许、邱四大天师迎面作礼道:“大圣何来?”行者道:“有纸
状儿,要告两个人哩。”天师吃惊道:“这个赖皮,不知要告那个。”无奈,将他引
入灵霄殿下启奏。蒙旨宣进。

行者将牌位、香炉放下,朝上礼毕,将状子呈上。葛仙翁接了,铺在御案。玉
帝从头看了,见这等这等,即将原状批作圣旨,宣西方长庚太白金星领旨到云楼宫
宣托塔李天王见驾。行者上前奏道:“望天主好生惩治;不然,又别生事端。”玉帝
又吩咐:“原告也去。”行者道:“老孙也去?”四天师道:“万岁已出了旨意,你可
同金星去来。”

行者真个随着金星,纵云头,早至云楼宫。原来是天王住宅,号云楼宫。金星
见宫门首有个童子侍立。那童子认得金星,即入里报道:“太白金星老爷来了。”天
王遂出迎迓。又见金星捧着旨意,即命焚香。及转身,又见行者跟入,天王即又作
怒。你道他作怒为何?当年行者大闹天宫时,玉帝曾封天王为降魔大元帅,封哪吒
太子为三坛海会之神,帅领天兵,收降行者,屡战不能取胜。还是五百年前败阵的
仇气,有些恼他,故此作怒。他且忍不住道:“老长庚,你赍得是甚么旨意?”金
星道:“是孙大圣告你的状子。”那天王本是烦恼,听见说个“告”字,一发雷霆大
怒道:“他告我怎的?”金星道:“告你假妖摄陷人口事。你焚了香,请自家开读。”
那天王气呼呼的,设了香案,望空谢恩。拜毕,展开旨意看了,原来是这般这般,
如此如此,恨得他手扑着香案道:“这个猴头,他也错告我了!”金星道:“且息怒。
现有牌位、香炉在御前作证,说是你亲女哩。”天王道:“我止有三个儿子,一个女
儿。大小儿名金吒,侍奉如来,做前部护法。二小儿名木叉,在南海随观世音做徒
弟。三小儿名哪吒,在我身边,早晚随朝护驾。一女年方七岁。名贞英,人事尚未
省得,如何会做妖精!不信,抱出来你看。这猴头着实无礼!且莫说我是天上元勋,
封受先斩后奏之职,就是下界小民,也不可诬告。律云:‘诬告加三等。’”叫手下:
“将缚妖索把这猴头捆了!”那庭下摆列着巨灵神、鱼肚将、药叉雄帅,一拥上前,
把行者捆了。金星道:“李天王莫闯祸啊!我在御前同他领旨意来宣你的人。你那索
儿颇重,一时捆坏他,阁气。”天王道:“金星啊,似他这等诈伪告扰,怎该容他!
你且坐下,待我取砍妖刀砍了这个猴头,然后与你见驾回旨!”金星见他取刀,心
惊胆战。对行者道:“你干事差了。御状可是轻易告的?你也不访的实,似这般乱弄,
伤其性命,怎生是好?”行者全然不惧,笑吟吟的道:“老官儿放心,一些没事。
老孙的买卖,原是这等做,一定先输后赢。”

说不了,天王轮过刀来,望行者劈头就砍。早有那三太子赶上前,将斩腰剑架
住,叫道:“父王息怒。”天王大惊失色。噫,父见子以剑架刀,就当喝退,怎么返
大惊失色?原来天王生此子时,他左手掌上有个“哪”字,右手掌上有个“吒”字,
故名哪吒。这太子三朝儿就下海净身闯祸,踏倒水晶宫,捉住蛟龙要抽筋为绦子。
天王知道,恐生后患,欲杀之。哪吒奋怒,将刀在手,割肉还母,剔骨还父,还了
父精母血,一点灵魂,径到西方极乐世界告佛。佛正与众菩萨讲经,只闻得幢幡宝
盖有人叫道:“救命!”佛慧眼一看,知是哪吒之魂,即将碧藕为骨,荷叶为衣,念
动起死回生真言,哪吒遂得了性命。运用神力,法降九十六洞妖魔,神通广大。后
来要杀天王,报那剔骨之仇。天王无奈,告求我佛如来。如来以和为尚,赐他一座
玲珑剔透舍利子如意黄金宝塔,那塔上层层有佛,艳艳光明。唤哪吒以佛为父,解
释了冤仇。所以称为托塔李天王者,此也。今日因闲在家,未曾托着那塔,恐哪吒
有报仇之意,故吓个大惊失色。却即回手,向塔座上取了黄金宝塔,托在手间,问
哪吒道:“孩儿,你以剑架住我刀,有何话说?”哪吒弃剑叩头道:“父王,是有女
儿在下界哩。”天王道:“孩儿,我只生了你姊妹四个,那里又有个女儿哩?”哪吒
道:“父王忘了。那女儿原是个妖精。三百年前成怪,在灵山偷食了如来的香花宝
烛,如来差我父子天兵,将他拿住。拿住时,只该打死。如来吩咐道:‘积水养鱼
终不钓,深山喂鹿望长生。’当时饶了他性命。积此恩念,拜父王为父,拜孩儿为
兄,在下方供设牌位,侍奉香火。不期他又成精,陷害唐僧,却被孙行者搜寻到巢
穴之间,将牌位拿来,就做名告了御状。此是结拜之恩女,非我同胞之亲妹也。”

天王闻言,悚然惊讶道:“孩儿,我实忘了。他叫做甚么名字?”太子道:“他
有三个名字:他的本身出处,唤做金鼻白毛老鼠精;因偷香花宝烛,改名唤做半截
观音;如今饶他下界,又改了,唤做地涌夫人是也。”天王却才省悟。放下宝塔,
便亲手来解行者。行者就放起刁来道:“那个敢解我!要便连绳儿抬去见驾,老孙的
官事才赢!”慌得天王手软,太子无言,众家将委委而退。

那大圣打滚撒赖,只要天王去见驾。天王无计可施,哀求金星说个方便。金星
道:“古人云:‘万事从宽。’你干事忒紧了些儿,就把他捆住,又要杀他。这猴子
是个有名的赖皮,你如今教我怎的处!若论你令郎讲起来,虽是恩女,不是亲女,
却也晚亲义重,不拘怎生折辨,你也有个罪名。”天王道:“老星怎说个方便,就没
罪了。”金星道:“我也要和解你们,却只是无情可说。”天王笑道:“你把那奏招安
授官衔的事说说,他也罢了。”

真个金星上前,将手摸着行者道:“大圣,看我薄面,解了绳好去见驾。”行者
道:“老官儿,不用解。我会滚法,一路滚就滚到也。”金星笑道:“你这猴忒恁寡
情。我昔日也曾有些恩义儿到你,你这些些事儿,就不依我。”行者道:“你与我有
甚恩义?”金星道:“你当年在花果山为怪,伏虎降龙,强消死籍,聚群妖大肆猖
狂,上天欲要擒你,是老身力奏,降旨招安,把你宣上天堂,封你做‘弼马温’。
你吃了玉帝仙酒,后又招安,也是老身力奏,封你做‘齐天大圣’。你又不守本分,
偷桃盗酒,窃老君之丹,如此如此,才得个无灭无生。若不是我,你如何得到今日?”
行者道:“古人说得好,‘死了莫与老头儿同墓,干净会揭挑人!’我也只是做弼马
温,闹天宫罢了,再无甚大事。也罢,也罢,看你老人家面皮,还教他自己来解。”
天王才敢向前,解了缚,请行者着衣上坐,一一上前施礼。

行者朝了金星道:“老官儿,何如?我说先输后赢,买卖儿原是这等做。快催他
去见驾,莫误了我的师父。”金星道:“莫忙。弄了这一会,也吃钟茶儿去。”行者
道:“你吃他的茶,受他的私,卖放犯人,轻慢圣旨,你得何罪?”金星道:“不吃
茶,不吃茶!连我也赖将起来了!李大王,快走,快走!”天王那里敢去,怕他没的
说做有的,放起刁来,口里胡说乱道,怎生与他折辨;没奈何,又央金星,教说方
便。金星道:“我有一句话儿,你可依我?”行者道:“绳捆刀砍之事,我也通看你
面,还有甚话?你说,你说,说得好,就依你;说得不好,莫怪。”

金星道:“‘一日官事十日打。’你告了御状,说妖精是天王的女儿,天王说不
是,你两个只管在御前折辨,反复不已。我说天上一日,下界就是一年。这一年之
间,那妖精把你师父,陷在洞中,莫说成亲,若有个喜花下儿子,也生了一个小和
尚儿,却不误了大事?”行者低头想道:“是啊!我离八戒、沙僧,只说多时饭熟,
少时茶滚就回;今已弄了这半会,却不迟了?老官儿,既依你说,这旨意如何回缴?”
金星道:“教李天王点兵,同你下去降妖,我去回旨。”行者道:“你怎么样回?”
金星道:“我只说原告脱逃,被告免提。”行者笑道:“好啊!我倒看你面情罢了,你
倒说我脱逃!教他点兵在南天门外等我,我即和你回旨缴状去。”天王害怕道:“他
这一去,若有言语,是臣背君也。”行者道:“你把老孙当甚么样人?我也是个大丈
夫!‘一言既出,驷马难追。’岂又有污言顶你?”

天王即谢了行者。行者与金星回旨。天王点起本部天兵,径出南天门外。金星
与行者回见玉帝道:“陷唐僧者,乃金鼻白毛老鼠成精,假设天王父子牌位。天王
知之,已点兵收怪去了,望天尊赦罪。”玉帝已知此情,降天恩免究。行者即返云
光,到南天门外。见天王、太子,布列天兵等候。噫!那些神将,风滚滚,雾腾腾,
接住大圣,一齐坠下云头,早到了陷空山上。

八戒、沙僧眼巴巴正等,只见天兵与行者来了。呆子迎着天王施礼道:“累及,
累及!”天王道:“天蓬元帅,你却不知。只因我父子受他一炷香,致令妖精无理,
困了你师父。来迟莫怪。这个山就是陷空山了?但不知他的洞门还向那边开?”行
者道:“我这条路且是走熟了。只是这个洞叫做个无底洞,周围有三百余里。妖精
窠穴甚多。前番我师父在那两滴水的门楼里,今番静悄悄,鬼影也没个,不知又搬
在何处去也。”天王道:“‘任他设尽千般计,难脱天罗地网中。’到洞门前,再作道
理。”大家就行。

咦,约有十余里,就到了那大石边。行者指那缸口大的门儿道:“兀的便是也。”
天王道:“‘不入虎穴,安得虎子!’谁敢当先?”行者道:“我当先。”三太子道:“我
奉旨降妖,我当先。”那呆子便莽撞起来,高声叫道:“当头还要我老猪!”天王道:
“不须罗噪,但依我分摆:孙大圣和太子同领着兵将下去,我们三人在口上把守,
做个里应外合,教他上天无路,入地无门,才显些些手段。”众人都答应了一声:
“是。”

你看那行者和三太子,领了兵将,望洞里只是一溜。驾起云光,闪闪烁烁,抬
头一望,果然好个洞啊:

依旧双轮日月,照般一望山川。珠渊玉井暖韬烟,更有许多堪羡。叠叠朱楼画
阁,嶷嶷赤壁青田。三春杨柳九秋莲,兀的洞天罕见。
顷刻间,停住了云光,径到那妖精旧宅。挨门儿搜寻,吆吆喝喝,一重又一重,一
处又一处,把那三百里地,草都踏光了,那见个妖精?那见个三藏?都只说:“这孽
畜一定是早出了这洞,远远去哩。”那晓得在那东南黑角落上,望下去,另有个小
洞。洞里一重小小门,一间矮矮屋,盆栽了几种花,檐傍着数竿竹,黑气氲氲,暗
香馥馥。老怪摄了三藏,搬在这里逼住成亲,只说行者再也找不着。谁知他命合该
休:那些小怪,在里面,一个个哜哜嘈嘈,挨挨簇簇。中间有个大胆些的,伸起颈
来,望洞外略看一看,一头撞着个天兵,一声嚷道:“在这里!”那行者恼起性来,
捻着金箍棒,一下闯将进去,那里边窄小,窝着一窟妖精。三太子纵起天兵,一齐
拥上,一个个那里去躲?

行者寻着唐僧,和那龙马,和那行李。那老怪寻思无路,看着哪吒太子,只是
磕头求命。太子道:“这是玉旨来拿你,不当小可。我父子只为受了一炷香,险些
儿‘和尚拖木头,做出了寺’!”声:“天兵,取下缚妖索,把那些妖精都捆了!”
老怪也少不得吃场苦楚。返云光,一齐出洞。行者口里嘻嘻嗄嗄。天王掣开洞口,
迎着行者道:“今番却见你师父也。”行者道:“多谢了!多谢了!”就引三藏拜谢天
王,次及太子。沙僧、八戒只是要碎剐那老精,天王道:“他是奉玉旨拿的,轻易
不得。我们还要去回旨哩。”

一边天王同三太子领着天兵神将,押住妖精,去奏天曹,听候发落;一边行者
拥着唐僧,沙僧收拾行李,八戒拢马,请唐僧骑马,齐上大路。这正是:
割断丝罗干金海,打开玉锁出樊笼。

毕竟不知前去何如,且听下回分解。