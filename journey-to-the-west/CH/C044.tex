\chapter{法身元运逢车力~心正妖邪度脊关}

诗曰:
求经脱障向西游,无数名山不尽休。
兔走乌飞催昼夜,鸟啼花落自春秋。
微尘眼底三千界,锡杖头边四百州。
宿水餐风登紫陌,未期何日是回头。

话说唐三藏幸亏龙子降妖,黑水河神开路,师徒们过了黑水河,找大路一直西
来。真个是迎风冒雪,戴月披星。行够多时,又值早春天气。但见:

三阳转运,万物生辉:三阳转运,满天明媚开图画;万物生辉,遍地芳菲设绣
茵。梅残数点雪,麦涨一川云。渐开冰解山泉溜,尽放萌芽没烧痕。正是那:太昊
乘震,勾芒御辰;花香风气暖,云淡日光新。道旁杨柳舒青眼,膏雨滋生万象春。
师徒们在路上,游观景色,缓马而行,忽听得一声喝,好便似千万人呐喊之声。
唐三藏心中害怕,兜住马不能前进,急回头道:“悟空,是那里这等响振?”八戒
道:“好一似地裂山崩。”沙僧道:“也就如雷声霹雳。”三藏道:“还是人喊马嘶。”
孙行者笑道:“你们都猜不着,且住,待老孙看是何如。”

好行者,将身一纵,踏云光,起在空中,睁眼观看,远见一座城池;又近觑,
倒也祥光隐隐,不见甚么凶气纷纷。行者暗自沉吟道:“好去处!如何有响声振
耳?……那城中又无旌旗闪灼,戈戟光明,又不是炮声响振,何以若人马喧哗?”

正议间,只见那城门外,有一块沙滩空地,攒簇了许多和尚,在那里扯车儿哩。
原来是一齐着力打号,齐喊“大力王菩萨”,所以惊动唐僧。

行者渐渐按下云头来看处,呀!那车子装的都是砖瓦木植土坯之类;滩头上坡
坂最高,又有一道夹脊小路,两座大关;关下之路都是直立壁陡之崖,那车儿怎么
拽得上去?虽是天色和暖,那些人却也衣衫蓝缕。看此像十分窘迫,行者心疑道:“想
是修盖寺院。他这里五谷丰登,寻不出杂工人来,所以这和尚亲自努力。”

正自猜疑未定,只见那城门里,摇摇摆摆,走出两个少年道士来。你看他怎生
打扮?但见他:

头戴星冠,身披锦绣;头戴星冠光耀耀,身披锦绣彩霞飘。足踏云头履,腰系
熟丝绦。面如满月多聪俊,形似瑶天仙客娇。
那些和尚见道士来,一个个心惊胆战,加倍着力,恨苦的拽那车子。行者就晓得了:
“咦!想必这和尚们怕那道士;不然啊,怎么这等着力拽扯?我曾听得人言,西方路
上,有个敬道灭僧之处,断乎此间是也。我待要回报师父,奈何事不明白,返惹他
怪,敢道这等一个伶俐之人,就不能探个实信。且等下去问得明白,好回师父话。”

你道他来问谁?好大圣,按落云头,去郡城脚下,摇身一变,变做个游方的云
水全真,左臂上挂着一个水火篮儿,手敲着渔鼓,口唱着道情词,近城门,迎着两
个道士,当面躬身道:“道长,贫道起手。”那道士还礼道:“先生那里来的?”行
者道:“我弟子云游于海角,浪荡在天涯。今朝来此处,欲募善人家。动问二位道
长,这城中那条街上好道?那个巷里好贤?我贫道好去化些斋吃。”那道士笑道:“你
这先生,怎么说这等败兴的话?”行者道:“何为败兴?”道士道:“你要化些斋吃,
却不是败兴?”行者道:“出家人以乞化为由,却不化斋吃,怎生有钱买?”道士
笑道:“你是远方来的,不知我这城中之事。我这城中,且休说文武官员好道,富
民长者爱贤,大男小女见我等拜请奉斋,这般都不须挂齿,头一等就是万岁君王好
道爱贤。”行者道:“我贫道一则年幼,二则是远方乍来,实是不知。烦二位道长将
这里地名、君王好道爱贤之事,细说一遍,足见同道之情。”道士说:“此城名唤车
迟国。宝殿上君王与我们有亲。”

行者闻言,呵呵笑道:“想是道士做了皇帝?”他道:“不是。只因这二十年前,
民遭亢旱,天无点雨,地绝谷苗,不论君臣黎庶,大小人家,家家沐浴焚香,户户
拜天求雨。正都在倒悬捱命之处,忽然天降下三个仙长来,俯救生灵。”行者问道:
“是那三个仙长?”道士说:“便是我家师父。”行者道:“尊师甚号?”道士云:“我
大师父,号做虎力大仙;二师父,鹿力大仙;三师父,羊力大仙。”行者问曰:“三
位尊师,有多少法力?”道士云:“我那师父,呼风唤雨,只在翻掌之间;指水为
油,点石成金,却如转身之易;所以有这般法力,能夺天地之造化,换星斗之玄微,
君臣相敬,与我们结为亲也。”

行者道:“这皇帝十分造化。常言道:‘术动公卿。’老师父有这般手段,结了
亲,其实不亏他。噫,不知我贫道可有星星缘法,得见那老师父一面哩?”道士笑
曰:“你要见我师父,有何难处!我两个是他靠胸贴肉的徒弟,我师父却又好道爱贤,
只听见说个‘道’字,就也接出大门。若是我两个引进你,乃吹灰之力。”

行者深深的唱个大喏道:“多承举荐,就此进去罢。”道士说:“且少待片时,
你在这里坐下,等我两个把公事干了来,和你进去。”行者道:“出家人无拘无束,
自由自在,有甚公干?”道士用手指定那沙滩上僧人:“他做的是我家生活,恐他
躲懒,我们去点他一卯就来。”行者笑道:“道长差了;僧道之辈都是出家人,为何
他替我们做活,伏我们点卯?”

道士云:“你不知道。因当年求雨之时,僧人在一边拜佛,道士在一边告斗,
都请朝廷的粮偿;谁知那和尚不中用,空念空经,不能济事。后来我师父一到,唤
雨呼风,拔济了万民涂炭。却才恼了朝廷,说那和尚无用,拆了他的山门,毁了他
的佛像,追了他的度牒,不放他回乡,御赐与我们家做活,就当小厮一般。我家里
烧火的,也是他;扫地的,也是他;顶门的,也是他。因为后边还有住房,未曾完
备,着这和尚来拽砖瓦,拖木植,起盖房宇。只恐他贪顽躲懒,不肯拽车,所以着
我两个去查点查点。”

行者闻言,扯住道士滴泪道:“我说我无缘,真个无缘,不得见老师父尊面!”
道士云:“如何不得见面?”行者道:“我贫道在方上云游,一则是为性命,二则也
为寻亲。”道士问:“你有甚么亲?”行者道:“我有一个叔父,自幼出家,削发为
僧。向日年程饥馑,也来外面求乞。这几年不见回家,我念祖上之恩,特来顺便寻
访。想必是羁迟在此等地方,不能脱身,未可知也。我怎的寻着他,见一面,才可
与你进城。”道士云:“这般却是容易。我两个且坐下,即烦你去沙滩上替我一查。
只点头目有五百名数目便罢。看内中那个是你令叔。果若有呀,我们看道中情分,
放他去了,却与你进城好么?”

行者顶谢不尽,长揖一声,别了道士,敲着渔鼓,径往沙滩之上。过了双关,
转下夹脊,那和尚一齐跪下磕头道:“爷爷,我等不曾躲懒,五百名半个不少,都
在此扯车哩。”行者看见,暗笑道:“这些和尚,被道士打怕了,见我这假道士就这
般悚惧。若是个真道士,好道也活不成了。”行者又摇手道:“不要跪,休怕。我不
是监工的,我来此是寻亲的。”众僧们听说认亲,就把他圈子阵围将上来,一个个
出头露面,咳嗽打响,巴不得要认出去。道:“不知那个是他亲哩。”

行者认了一会,呵呵笑将起来。众僧道:“老爷不认亲,如何发笑?”行者道:
“你们知我笑甚么?笑你这些和尚全不长俊!父母生下你来,皆因命犯华盖,妨爷克
娘,或是不招姊妹,才把你舍断了出家;你怎的不遵三宝,不敬佛法,不去看经拜
忏,却怎么与道士佣工,作奴婢使唤?”众僧道:“老爷,你来羞我们哩!你老人家
想是个外边来的,不知我这里利害。”行者道:“果是外方来的,其实不知你这里有
甚利害。”

众僧滴泪道:“我们这一国君王,偏心无道,只喜得是老爷等辈,恼的是我们
佛子。”行者道:“为何来?”众僧道:“只因呼风唤雨,三个仙长来此处,灭了我
等;哄信君王,把我们寺拆了,度牒追了,不放归乡,亦不许补役当差,赐与那仙
长家使用,苦楚难当!但有个游方道者至此,即请拜王领赏;若是和尚来,不分远
近,就拿来与仙长家佣工。”行者道:“想必那道士还有甚么巧法术,诱了君王?—
—若只是呼风唤雨,也都是傍门小法术耳,安能动得君心?”众僧道:“他会抟砂
炼汞,打坐存神,点水为油,点石成金。如今兴盖三清观宇,对天地昼夜看经忏悔,
祈君王万年不老,所以就把君心惑动了。”

行者道:“原来这般。你们都走了便罢。”众僧道:“老爷,走不脱!那仙长奏准
君王,把我们画了影身图,四下里长川张挂。他这车迟国地界也宽,各府州县乡村
店集之方,都有一张和尚图,上面是御笔亲题。若有官职的,拿得一个和尚,高升
三级;无官职的,拿得一个和尚,就赏白银五十两,所以走不脱。且莫说是和尚,
就是剪鬃、秃子、毛稀的,都也难逃。四下里快手又多,缉事的又广,凭你怎么也
是难脱。我们没奈何,只得在此苦捱。”

行者道:“既然如此,你们死了便罢。”众僧道:“老爷,有死的。到处捉来与
本处和尚,也共有二千余众。到此熬不得苦楚,受不得煎,忍不得寒冷,服不得
水土,死了有六七百,自尽了有七八百;只有我这五百个不得死。”

行者道:“怎么不得死?”众僧道:“悬梁绳断,刀刎不疼;投河的飘起不沉,
服药的身安不损。”行者道:“你却造化,天赐汝等长寿哩!”众僧道:“老爷呀,你
少了一个字儿,是‘长受罪’哩!我等日食三餐,乃是糙米熬得稀粥。到晚就在沙
滩上冒露安身。才合眼,就有神人拥护。”行者道:“想是累苦了,见鬼么?”众僧
道:“不是鬼,乃是六丁六甲、护教伽蓝。但至夜,就来保护。但有要死的,就保
着,不教他死。”行者道:“这些神却也没理;只该教你们早死早生天,却来保护怎
的?”众僧道:“他在梦寐中劝解我们,教‘不要寻死,且苦捱着,等那东土大唐
圣僧,往西天取经的罗汉。他手下有个徒弟,乃齐天大圣,神通广大,专秉忠良之
心,与人间报不平之事,济困扶危,恤孤念寡。只等他来显神通,灭了道士,还敬
你们沙门禅教哩。’”

行者闻得此言,心中暗笑道:“莫说老孙无手段,预先神圣早传名。”他急抽身,
敲着渔鼓,别了众僧,径来城门口,见了道士。那道士迎着道:“先生,那一位是
令亲?”行者道:“五百个都与我有亲。”两个道士笑道:“你怎么就有许多亲?”
行者道:“一百个是我左邻,一百个是我右舍,一百个是我父党,一百个是我母党,
一百个是我交契。你若肯把这五百人都放了,我便与你进去;不放,我不去了。”
道士云:“你想有些风病,一时间就胡说了。那些和尚,乃国王御赐,若放一二名,
还要在师父处递了病状,然后补个死状,才了得哩。怎么说都放了!此理不通,不
通!且不要说我家没人使唤,就是朝廷也要怪。他那里长要差官查勘,或时御驾也
亲来点札,怎么敢放?”行者道:“不放么?”道士说:“不放!”行者连问三声,
就怒将起来,把耳朵里铁棒取出,迎风捻了一捻,就碗来粗细;幌了一幌,照道士
脸上一刮,可怜就打得头破血流身倒地,皮开颈折脑浆倾!

那滩上僧人,远远望见他打杀了两个道士,丢了车儿,跑将上来道:“不好了,
不好了!打杀皇亲了!”行者道:“那个是皇帝?”众僧把他簸箕阵围了。道:“他师
父,上殿不参王,下殿不辞主,朝廷常称做‘国师兄长先生’。你怎么到这里闯祸?
他徒弟出来监工,与你无干,你怎么把他来打死?那仙长不说是你来打杀,只说是
来此监工,我们害了他性命。我等怎了?且与你进城去,会了人命出来。”行者笑道:
“列位休嚷。我不是云水全真,我是来救你们的。”众僧道:“你倒打杀人,害了我
们,添了担儿,如何是救我们的?”

行者道:“我是大唐圣僧徒弟孙悟空行者,特特来此救你们性命。”众僧道:“不
是,不是,那老爷我们认得他。”行者道:“又不曾会他,如何认得?”众僧道:“我
们梦中尝见一个老者,自言太白金星,常教诲我等,说那孙行者的模样,莫教错认
了。”行者道:“他和你怎么说来?”众僧道:“他说:‘那大圣:

磕额金睛幌亮,圆头毛脸无腮。咨牙尖嘴性情乖,貌比雷公古怪。

惯使金
箍铁棒,曾将天阙攻开。如今皈正保僧来,专救人间灾害。’”
行者闻言,又嗔又喜。喜道替老孙传名!嗔道那老贼惫懒,把我的元身都说与这伙
凡人!忽失声道:“列位诚然认我不是孙行者。我是孙行者的门人,来此处学闯祸耍
子的。那里不是孙行者来了?”用手向东一指,哄得众僧回头,他却现了本相。众
僧们方才认得。一个个倒身下拜道:“爷爷!我等凡胎肉眼,不知是爷爷显化。望爷
爷与我们雪恨消灾,早进城降邪从正也!”行者道:“你们且跟我来。”众僧紧随左
右。

那大圣径至沙滩上,使个神通,将车儿拽过两关,穿过夹脊,提起来,得粉
碎。把那些砖瓦木植,尽抛下坡坂。喝教众僧:“散!莫在我手脚边,等我明日见这
皇帝,灭那道士!”众僧道:“爷爷呀,我等不敢远走;但恐在官人拿住解来,却又
吃打发赎,返又生灾。”行者道:“既如此,我与你个护身法儿。”好大圣,把毫毛
拔了一把,嚼得粉碎,每一个和尚与他一截。都教他:“捻在无名指甲里,捻着拳
头,只情走路。无人敢拿你便罢;若有人拿你,攒紧了拳头,叫一声‘齐天大圣’,
我就来护你。”众僧道:“爷爷,倘若去得远了,看不见你,叫你不应,怎么是好?”
行者道:“你只管放心,就是万里之遥,可保全无事。”

众僧有胆量大的,捻着拳头,悄悄的叫声“齐天大圣!”只见一个雷公站在面
前,手执铁棒,就是千军万马,也不能近身。此时有百十众齐叫,足有百十个大圣
护持。众僧叩头道:“爷爷,果然灵显!”行者又吩付:“叫声‘寂’字,还你收了。”
真个是叫声“寂!”依然还是毫毛在那指甲缝里。众和尚却才欢喜逃生,一齐而散。
行者道:“不可十分远遁。听我城中消息。但有招僧榜出,就进城还我毫毛也。”五
百个和尚,东的东,西的西,走的走,立的立,四散不题。

却说那唐僧在路旁,等不得行者回话,教猪八戒引马投西,遇着些僧人奔走;
将近城边,见行者还与十数个未散的和尚在那里。三藏勒马道:“悟空,你怎么来
打听个响声,许久不回?”行者引了十数个和尚,对唐僧马前施礼,将上项事说了
一遍。三藏大惊道:“这般啊,我们怎了?”那十数个和尚道:“老爷放心。孙大圣
爷爷乃天神降的,神通广大,定保老爷无虞。我等是这城里敕建智渊寺内僧人。因
这寺是先王太祖御造的,现有先王太祖神像在内,未曾拆毁。城中寺院,大小尽皆
拆了。我等请老爷赶早进城,到我荒山安下。待明日早朝,孙大圣必有处置。”行
者道:“汝等说得是;也罢,趁早进城去来。”

那长老却才下马,行到城门之下。此时已太阳西坠。过吊桥,进了三层门里,
街上人见智渊寺的和尚牵马挑包,尽皆回避。正行时,却到山门前。但见那门上高
悬着一面金字大匾,乃“敕建智渊寺”。众僧推开门,穿过金刚殿,把正殿门开了。
唐僧取袈裟披起,拜毕金身,方入。众僧叫:“看家的!”老和尚走出来,看见行者
就拜,道:“爷爷!你来了?”行者道:“你认得我是那个爷爷,就是这等呼拜?”
那和尚道:“我认得你是齐天大圣孙爷爷。我们夜夜梦中见你。太白金星常常来托
梦,说道,只等你来,我们才得性命。今日果见尊颜与梦中无异。爷爷呀,喜得早
来!再迟一两日,我等已俱做鬼矣!”行者笑道:“请起,请起。明日就有分晓。”众
僧安排了斋饭,他师徒们吃了。打扫干净方丈,安寝一宿。

二更时候,孙大圣心中有事,偏睡不着。只听那里吹打,悄悄的爬起来,穿了
衣服,跳在空中观看,原来是正南上灯烛荧煌。低下云头仔细再看,却是三清观道
士禳星哩。但见那:

灵区高殿,福地真堂:灵区高殿,巍巍壮似蓬壶景;福地真堂,隐隐清如化乐
宫。两边道士奏笙簧,正面高公擎玉简。宣理《消灾忏》,开讲《道德经》。扬尘几
度尽传符,表白一番皆俯伏。咒水发檄,烛焰飘摇冲上界;查罡布斗,香烟馥郁透
清霄。案头有供献新鲜,桌上有斋筵丰盛。
殿门前挂一联黄绫织锦的对句,绣着二十二个大字,云:“雨顺风调,愿祝天尊无
量法;河清海晏,祈求万岁有余年。”行者见三个老道士,披了法衣,想是那虎力、
鹿力、羊力大仙。下面有七八百个散众,司鼓司钟,侍香表白,尽都侍立两边。行
者暗自喜道:“我欲下去与他混一混,奈何‘单丝不线,孤掌难鸣。’且回去照顾八
戒、沙僧,一同来耍耍。”

按落祥云,径至方丈中。原来八戒与沙僧通脚睡着。行者先叫悟净。沙和尚醒
来道:“哥哥,你还不曾睡哩?”行者道:“你且起来,我和你受用些来。”沙僧道:
“半夜三更,口枯眼涩,有甚受用?”行者道:“这城里果有一座三清观。观里道
士们修醮,三清殿上有许多供养:馒头足有斗大,烧果有五六十斤一个,衬饭无数,
果品新鲜。和你受用去来!”那猪八戒睡梦里听见说吃好东西,就醒了,道:“哥哥,
就不带挈我些儿?”行者道:“兄弟,你要吃东西,不要大呼小叫,惊醒了师父。
都跟我来。”

他两个套上衣服,悄悄的走出门前,随行者踏了云头,跳将起去。那呆子看见
灯光,就要下手。行者扯住道:“且休忙。待他散了,方可下去。”八戒道:“他才
念到兴头上,却怎么肯散?”行者道:“等我弄个法儿,他就散了。”好大圣,捻着
诀,念个咒语,往巽地上吸一口气,呼的吹去,便是一阵狂风,径直卷进那三清殿
上,把他些花瓶烛台,四壁上悬挂的功德,一齐刮倒,遂而灯火无光。众道士心惊
胆战。虎力大仙道:“徒弟们且散。这阵神风所过,吹灭了灯烛香花,各人归寝,
明朝早起,多念几卷经文补数。”众道士果各退回。

这行者却引八戒、沙僧、按落云头,闯上三清殿。呆子不论生熟,拿过烧果来,
张口就啃。行者掣铁棒,着手便打。八戒缩手躲过道:“还不曾尝着甚么滋味,就
打!”行者道:“莫要小家子行。且叙礼坐下受用。”八戒道:“不羞!偷东西吃,还
要叙礼!若是请将来,却要如何!”行者道:“这上面坐的是甚么菩萨?”八戒笑道:
“三清也认不得,却认做甚么菩萨!”行者道:“那三清?”八戒道:“中间的是元
始天尊,左边的是灵宝道君,右边的是太上老君。”行者道:“都要变得这般模样,
才吃得安稳哩。”那呆子急了,闻得那香喷喷供养,要吃,爬上高台,把老君一嘴
拱下去道:“老官儿,你也坐得够了,让我老猪坐坐。”八戒变做太上老君;行者变
做元始天尊;沙僧变作灵宝道君。把原像都推下去。及坐下时,八戒就抢大馒头吃。
行者道:“莫忙哩!”八戒道:“哥哥,变得如此,还不吃等甚?”

行者道:“兄弟呀,吃东西事小,泄漏天机事大。这圣像都推在地下,倘有起
早的道士来撞钟扫地,或绊一个根头,却不走漏消息?你把他藏过一边来。”八戒道:
“此处路生,摸门不着,却那里藏他?”行者道:“我才进来时,那右手下有一重
小门儿,那里面秽气畜人,想必是个五谷轮回之所。你把他送在那里去罢。”这呆
子有些夯力量,跳下来,把三个圣像,拿在肩膊上,扛将出来;到那厢,用脚登开
门看时,原来是个大东厕。笑道:“这个弼马温着然会弄嘴弄舌!把个毛坑也与他起
个道号,叫做甚么‘五谷轮回之所’!”那呆子扛在肩上且不丢了去,口里哝哝
的祷道:

“三清,三清,我说你听:远方到此,惯灭妖精。欲享供养,无处安宁。借你
坐位,略略少停。你等坐久,也且暂下毛坑。你平日家受用无穷,做个清净道士;
今日里不免享些秽物,也做个受臭气的天尊!”
祝罢,烹的望里一,了半衣襟臭水,走上殿来。行者道:“可藏得好么?”八
戒道:“藏便藏得好;只是起些水来,污了衣服,有些腌臭气,你休恶心。”行
者笑道:“也罢,你且来受用;但不知可得个干净身子出门哩。”那呆子还变做老君。
三人坐下,尽情受用。先吃了大馒头,后吃簇盘、衬饭、点心、拖炉、饼锭、油、
蒸酥,那里管甚么冷热,任情吃起。原来孙行者不大吃烟火食,只吃几个果子,陪
他两个。那一顿如流星赶月,风卷残云,吃得罄尽。已此没得吃了,还不走路,且
在那里闲讲,消食耍子。

噫!有这般事!原来那东廊下有一个小道士,才睡下,忽然起来道:“我的手铃
儿忘记在殿上,若失落了,明日师父见责。”与那同睡者道:“你睡着,等我寻去。”
急忙中不穿底衣,止扯一领直裰,径到正殿中寻铃。摸来摸去,铃儿摸着了。正欲
回头,只听得有呼吸之声,道士害怕。急拽步往外走时,不知怎的,着一个荔枝
核子,扑的滑了一跌。当的一声,把个铃儿跌得粉碎。猪八戒忍不住呵呵大笑出来,
把个小道士唬走了三魂,惊回了七魄,一步一跌,撞到后方丈外,打着门叫:“师
公,不好了,祸事了!”三个老道士还未曾睡,即开门问:“有甚祸事?”他战战兢
兢道:“弟子忘失了手铃儿,因去殿上寻铃,只听得有人呵呵大笑,险些儿唬杀我
也!”老道士闻言,即叫:“掌灯来!看是甚么邪物?”一声传令,惊动那两廊的道
士,大大小小,都爬起来点灯着火,往正殿上观看。

不知端的何如,且听下回分解。