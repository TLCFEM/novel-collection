\chapter{花果山群妖聚义~黑松林三藏逢魔}

却说那大圣虽被唐僧逐赶,然犹思念,感叹不已,早望见东洋大海。道:“我
不走此路者,已五百年矣!”只见那海水:

烟波荡荡,巨浪悠悠:烟波荡荡接天河,巨浪悠悠通地脉。潮来汹涌,水浸湾
环:潮来汹涌,犹如霹雳吼三春;水浸湾环,却似狂风吹九夏。乘龙福老,往来必
定皱眉行;跨鹤仙童,反复果然忧虑过。近岸无村社,傍水少渔舟。浪卷千年雪,
风生六月秋。野禽凭出没,沙鸟任沉浮。眼前无钓客,耳畔只闻鸥。海底游鱼乐,
天边过雁愁。
那行者将身一纵,跳过了东洋大海,早至花果山。按落云头,睁睛观看,那山上花
草俱无,烟霞尽绝;峰岩倒塌,林树焦枯。你道怎么这等?只因他闹了天宫,拿上
界去。此山被显圣二郎神,率领那梅山七弟兄,放火烧坏了。这大圣倍加凄惨。有
一篇败山颓景的古风为证。古风云:

回顾仙山两泪垂,对山凄惨更伤悲。当时只道山无损,今日方知地有亏。可恨
二郎将我灭,堪嗔小圣把人欺。行凶掘你先灵墓,无干破尔祖坟基。满天霞雾皆消
荡,遍地风云尽散稀。东岭不闻斑虎啸,西山那见白猿啼。北溪狐兔无踪迹,南谷
獐没影遗。青石烧成千块土,碧砂化作一堆泥。洞外乔松皆倚倒,崖前翠柏尽稀
少。椿杉槐桧栗檀焦,桃杏李梅梨枣
了。柘绝桑无怎养蚕?柳稀竹少难栖鸟。峰头巧石化为尘,涧底泉干都是草。崖前
土黑没芝兰,路畔泥红藤薜攀。往日飞禽飞那处?当时走兽走何山?豹嫌蟒恶倾颓所,
鹤避蛇回败坏间。想是日前行恶念,致令目下受艰难。

那大圣正当悲切,只听得那芳草坡前,曼荆凹里,响一声,跳出七八个小猴,
一拥上前,围住叩头。高叫道:“大圣爷爷,今日来家了?”美猴王道:“你们因何
不耍不顽,一个个都潜踪隐迹?我来多时了,不见你们形影,何也?”群猴听说,
一个个垂泪告道:“自大圣擒拿上界,我们被猎人之苦,着实难捱!怎禁他硬弩强弓,
黄鹰劣犬,网扣枪钩,故此各惜性命,不敢出头顽耍;只是深潜洞府,远避窝巢。
饥去坡前偷草食,渴来涧下吸清泉。却才听得大圣爷爷声音,特来接见,伏望扶持。”
那大圣闻得此言,愈加凄惨。便问:“你们还有多少在此山上?”群猴道:“老者,
小者,只有千把。”大圣道:“我当时共有四万七千群妖,如今都往那里去了?”群
猴道:“自从爷爷去后,这山被二郎菩萨点上火,烧杀了大半。我们蹲在井里,钻
在涧内,藏于铁板桥下,得了性命。及至火灭烟消,出来时,又没花果养赡,难以
存活,别处又去了一半。我们这一半,捱苦的住在山中。这两年,又被些打猎的抢
了一半去也。”行者道:“他抢你去何干?”群猴道:“说起这猎户,可恨!他把我们
中箭着枪的,中毒打死的,拿了去剥皮剔骨,酱煮醋蒸,油煎盐炒,当做下饭食用。
或有那遭网的,遇扣的,夹活儿拿去了,教他跳圈做戏,翻筋斗,竖蜻蜓,当街上
筛锣擂鼓,无所不为的顽耍。”大圣闻此言,更十分恼怒道:“洞中有甚么人执事?”
群妖道:“还有马、流二元帅,奔、芭二将军管着哩。”大圣道:“你们去报他知道,
说我来了。”那些小妖,撞入门里报道:“大圣爷爷来家了。”那马、流、奔、芭闻
报,忙出门叩头,迎接进洞。

大圣坐在中间,群怪罗拜于前,启道:“大圣爷爷,近闻得你得了性命,保唐
僧往西天取经,如何不走西方,却回本山?”大圣道:“小的们,你不知道。那唐
三藏不识贤愚:我为他一路上捉怪擒魔,使尽了平生的手段,几番家打杀妖精;他
说我行凶作恶,不要我做徒弟,把我逐赶回来,写立贬书为照,永不听用了。”

众猴鼓掌大笑道:“造化!造化!做甚么和尚,且家来,带携我们耍子几年罢!”
叫:“快安排椰子酒来,与爷爷接风。”大圣道:“且莫饮酒。我问你:那打猎的人,
几时来我山上一度?”马、流道:“大圣,不论甚么时度,他逐日家在这里缠扰。”
大圣道:“他怎么今日不来?”马、流道:“看待来耶。”大圣吩咐:“小的们,都出
去把那山上烧酥了的碎石头与我搬将起来堆着。或二三十个一堆,或五六十个一堆,
堆着,我有用处。”那些小猴,都是一窝蜂,一个个跳天搠地,乱搬了许多堆集。
大圣看了,教:“小的们,都往洞内藏躲,让老孙作法。”

那大圣上了山巅看处,只见那南半边冬冬鼓响,当当锣鸣,闪上有千余人马,
都架着鹰犬,持着刀枪。猴王仔细看那些人,来得凶险。好男子,真个骁勇!但见:

狐皮苫肩顶,锦绮裹腰胸。袋插狼牙箭,胯挂宝雕弓。人似搜山虎,马如跳涧
龙。成群引着犬,满膀架其鹰。荆筐抬火炮,带定海东青。粘竿百十,兔叉有千
根。牛头拦路网,阎王扣子绳。一齐乱吆喝,散撒满天星。
大圣见那些人布上他的山来,心中大怒。手里捻诀,口内念念有词,往那巽地上吸
了一口气,的吹将去,便是一阵狂风。好风!但见:

扬尘播土,倒树摧林。海浪如山耸,浑波万叠侵。乾坤昏荡荡,日月暗沉沉。
一阵摇松如虎啸,忽然入竹似龙吟。万窍怒召天噫气,飞砂走石乱伤人。
大圣作起这大风,将那碎石,乘风乱飞乱舞,可怜把那些千余人马,一个个:

石打乌头粉碎,沙飞海马俱伤。人参官桂岭前忙,血染朱砂地上。附子难归故
里,槟榔怎得还乡?户骸轻粉卧山场,红娘子家中盼望。

诗曰:
人亡马死怎归家?野鬼孤魂乱似麻。
可怜抖擞英雄将,不辨贤愚血染沙。

大圣按落云头,鼓掌大笑道:“造化,造化!自从归顺唐僧,做了和尚,他每每
劝我话道:‘千日行善,善犹不足;一日行恶,恶自有余。’真有此话!我跟着他,
打杀几个妖精,他就怪我行凶;今日来家,却结果了这许多猎户。”叫:“小的们,
出来!”那群猴,狂风过去,听得大圣呼唤,一个个跳将出来。大圣道:“你们去南
山下,把那打死的猎户衣服,剥得来家,洗净血迹,穿了遮寒;把死人的尸首,都
推在那万丈深潭里;把死倒的马,拖将来,剥了皮,做靴穿,将肉腌着,慢慢的食
用;把那些弓箭枪刀,与你们操演武艺;将那杂色旗号,收来我用。”群猴一个个
领诺。

那大圣把旗拆洗,总斗做一面杂彩花旗,上写着“重修花果山,复整水帘洞,
齐天大圣”十四字。竖起杆子,将旗挂于洞外,逐日招魔聚兽,积草屯粮,不题“和
尚”二字。他的人情又大,手段又高,便去四海龙王,借些甘霖仙水,把山洗青了。
前栽榆柳,后种松楠,桃李枣梅,无所不备,逍遥自在,乐业安居不题。

却说唐僧听信狡性,纵放心猿。攀鞍上马,八戒前边开路,沙僧挑着行李西行。
过了白虎岭,忽见一带林丘,真个是藤攀葛绕,柏翠松青。三藏叫道:“徒弟呀,
山路崎岖,甚是难走,却又松林丛簇,树木森罗,切须仔细!恐有妖邪妖兽。”你看
那呆子,抖擞精神,叫沙僧带着马,他使钉钯开路,领唐僧径入松林之内。正行处,
那长老兜住马道:“八戒,我这一日其实饥了,那里寻些斋饭我吃?”八戒道:“师
父请下马,在此等老猪去寻。”长老下了马,沙僧歇了担,取出钵盂,递与八戒。
八戒道:“我去也。”长老问:“那里去?”八戒道:“莫管,我这一去,钻冰取火寻
斋至,压雪求油化饭来。”

你看他出了松林,往西行经十余里,更不曾撞着一个人家,真是有狼虎无人烟
的去处。那呆子走得辛苦,心内沉吟道:“当年行者在日,老和尚要的就有;今日
轮到我的身上,诚所谓‘当家才知柴米价,养子方晓父娘恩’。公道没去化处。”却
又走得瞌睡上来,思道:“我若就回去,对老和尚说没处化斋,他也不信我走了这
许多路。须是再多幌个时辰,才好去回话。也罢,也罢,且往这草科里睡睡。”呆
子就把头拱在草里睡下。当时也只说朦胧朦胧就起来,岂知走路辛苦的人,丢倒头,
只管睡起。

且不言八戒在此睡觉。却说长老在那林间,耳热眼跳,身心不安。急回叫沙僧
道:“悟能去化斋,怎么这早晚还不回?”沙僧道:“师父,你还不晓得哩。他见这
西方上人家斋僧的多,他肚子又大,他管你?只等他吃饱了才来哩。”三藏道:“正
是呀;倘或他在那里贪着吃斋,我们那里会他?天色晚了,此间不是个住处,须要
寻个下处方好哩。”沙僧道:“不打紧,师父,你且坐在这里,等我去寻他来。”三
藏道:“正是,正是。有斋没斋罢了,只是寻下处要紧。”沙僧绰了宝杖,径出松林
来找八戒。

长老独坐林中,十分闷倦。只得强打精神,跳将起来,把行李攒在一处,将马
拴在树上,柬下戴的斗笠,插定了锡杖,整一整缁衣,徐步幽林,权为散闷。那长
老看遍了野草山花,听不得归巢鸟噪。原来那林子内都是些草深路小的去处。只因
他情思紊乱,却走错了。他一来也是要散散闷,二来也是要寻八戒、沙僧;不期他
两个走的是直西路,长老转了一会,却走向南边去了。

出得松林,忽抬头,见那壁厢金光闪烁,彩气腾腾。仔细看处,原来是一座宝
塔,金顶放光。这是那西落的日色,映着那金顶放亮。他道:“我弟子却没缘法哩!
自离东土,发愿逢庙烧香,见佛拜佛,遇塔扫塔。那放光的不是一座黄金宝塔?怎
么就不曾走那条路?塔下必有寺院,院内必有僧家,且等我走走。这行李、白马,
料此处无人行走,却也无事。那里若有方便处,待徒弟们来,一同借歇。”

噫!长老一时晦气到了。你看他拽开步,竟至塔边。但见那:

石崖高万丈,山大接青霄。根连地厚,峰插天高。两边杂树数千科,前后藤缠
百余里。花映草梢风有影,水流云窦月无根。倒木横担深涧,枯藤结挂光峰。石桥
下,流滚滚清泉;台座上,长明明白粉。远观一似三岛天堂,近看有如蓬莱胜境。
香松紫竹绕山溪,鸦鹊猿猴穿峻岭。洞门外,有一来一往的走兽成行;树林里,有
或出或入的飞禽作队。青青香草秀,艳艳野花开。这所在分明是恶境,那长老晦气
撞将来。
那长老举步进前,才来到塔门之下,只见一个斑竹帘儿,挂在里面。他破步入门,
揭起来,往里就进,猛抬头,见那石床上,侧睡着一个妖魔。你道他怎生模样:

青靛脸,白獠牙,一张大口呀呀。两边乱蓬蓬的鬓毛,却都是些胭脂染色;三
四紫巍巍的髭髯,恍疑是那荔枝排芽。鹦嘴般的鼻儿拱拱,曙星样的眼儿巴巴。两
个拳头,和尚钵盂模样;一双蓝脚,悬崖桠槎。斜披着淡黄袍帐,赛过那织锦
袈裟。拿的一口刀,精光耀映;眠的一块石,细润无瑕。他也曾小妖排蚁阵,他也
曾老怪坐蜂衙。你看他威风凛凛,大家吆喝,叫一声爷。他也曾月作三人壶酌酒,
他也曾风生两腋盏倾茶。你看他神通浩浩,霎着下眼,游遍天涯。荒林喧鸟雀,深
莽宿龙蛇。仙子种田生白玉,道人伏火养丹砂。小小洞门,虽到不得那阿鼻地狱;
楞楞妖怪,却就是一个牛头夜叉。

那长老看见他这般模样,唬得打了一个倒退,遍体酥麻,两腿酸软,即忙的抽
身便走。刚刚转了一个身,那妖魔,他的灵性着实是强。大撑开着一双金睛鬼眼,
叫声“小的们,你看门外是甚么人!”一个小妖就伸头望门外一看,看见是个光头
的长老,连忙跑将进去,报道:“大王,外面是个和尚哩。团头大面,两耳垂肩;
嫩刮刮的一身肉,细娇娇的一张皮:且是好个和尚!”那妖闻言,呵声笑道:“这叫
做个‘蛇头上苍蝇,自来的衣食。’你众小的们!疾忙赶上也,与我拿将来!我这里
重重有赏。”那些小妖,就是一窝蜂,齐齐拥上。三藏见了,虽则是一心忙似箭,
两脚走如飞;终是心惊胆颤,腿软脚麻。况且是山路崎岖,林深日暮,步儿那里移
得动?被那些小妖,平抬将去。正是:
龙游浅水遭虾戏,虎落平原被犬欺。
纵然好事多磨障,谁像唐僧西向时?

你看那众小妖,抬得长老,放在那竹帘儿外,欢欢喜喜,报声道:“大王,拿
得和尚进来了。”那老妖,他也偷眼瞧一瞧。只见三藏头直上,貌堂堂,果然好一
个和尚。他便心中想道:“这等好和尚,必是上方人物,不当小可的;若不做个威
风,他怎肯服降哩?”陡然间,就狐假虎威,红须倒竖,血发朝天,眼睛迸裂。大
喝一声道:“带那和尚进来!”众妖们,大家响响的答应了一声“是”!就把三藏望
里面只是一推。这是“既在矮檐下,怎敢不低头!”三藏只得双手合着,与他见个
礼。

那妖道:“你是那里和尚?从那里来?到那里去?快快说明!”三藏道:“我本是唐
朝僧人,奉大唐皇帝敕命,前往西方访求经偈。经过贵山,特来塔下谒圣,不期惊
动威严,望乞恕罪。待往西方取得经回东土,永注高名也。”那妖闻言,呵呵大笑
道:“我说是上邦人物,果然是你。正要吃你哩!却来的甚好,甚好!不然,却不错
放过了?你该是我口里的食,自然要撞将来,就放也放不去,就走也走不脱!”叫小
妖:“把那和尚拿去绑了!”果然那些小妖,一拥上前,把个长老绳缠索绑,缚在那
定魂桩上。

老妖持刀又问道:“和尚,你一行有几人?终不然一人敢上西天?”三藏见他持
刀,又老实说道:“大王,我有两个徒弟,叫做猪八戒、沙和尚,都出松林化斋去
了。还有一担行李,一匹白马,都在松林里放着哩。”老妖道:“又造化了!两个徒
弟,连你三个,连马四个,彀吃一顿了!”小妖道:“我们去捉他来。”老妖道:“不
要出去,把前门关了。他两个化斋来,一定寻师父吃;寻不着,一定寻着我门上。
常言道:‘上门的买卖好做。’且等慢慢的捉他。”众小妖把前门闭了。

且不言三藏逢灾。却说那沙僧出林找八戒,直有十余里远近,不曾见个庄村。
他却站在高埠上正然观看,只听得草中有人言语,急使杖拨开深草看时,原来是呆
子在里面说梦话哩。被沙僧揪着耳朵,方叫醒了。道:“好呆子啊!师父教你化斋,
许你在此睡觉的?”那呆子冒冒失失的醒来道:“兄弟,有甚时候了?”沙僧道:“快
起来!师父说有斋没斋也罢,教你我那里寻下住处去哩。”呆子懵懵懂懂的,托着钵
盂,着钉钯,与沙僧径直回来。到林中看时,不见了师父。沙僧埋怨道:“都是
你这呆子化斋不来,必有妖精拿师父也。”八戒笑道:“兄弟,莫要胡说。那林子里
是个清雅的去处,决然没有妖精。想是老和尚坐不住,往那里观风去了。我们寻他
去来。”二人只得牵马挑担,收拾了斗篷、锡杖,出松林寻找师父。

这一回,也是唐僧不该死。他两个寻一会不见,忽见那正南下有金光闪灼。八
戒道:“兄弟啊,有福的只是有福。你看师父往他家去了。那放光的是座宝塔。谁
敢怠慢?一定要安排斋饭,留他在那里受用。我们还不走动些,也赶上去吃些斋儿。”
沙僧道:“哥啊,定不得吉凶哩。我们且去看来。”

二人雄纠纠的到了门前,——呀!闭着门哩。——只见那门上横安了一块白玉
石板,上镌着六个大字:“碗子山波月洞”。沙僧道:“哥啊,这不是甚么寺院,是
一座妖精洞府也。我师父在这里,也见不得哩。”八戒道:“兄弟莫怕。你且拴下马
匹,守着行李,待我问他的信看。”

那呆子举着钯,上前高叫:“开门,开门!”那洞内有把门的小妖,开了门。忽
见他两个的模样,急抽身,跑入里面报道:“大王,买卖来了!”老妖道:“那里买
卖?”小妖道:“洞门外有一个长嘴大耳的和尚,与一个晦气色的和尚,来叫门了!”
老妖大喜道:“是猪八戒与沙僧寻将来也!噫,他也会寻哩!怎么就寻到我这门上?既
然嘴脸凶顽,却莫要怠慢了他。”叫:“取披挂来!”小妖抬来,就结束了,绰刀在
手,径出门来。

却说那八戒、沙僧,在门前正等,只见妖魔来得凶险。你道他怎生打扮:
青脸红须赤发飘,黄金铠甲亮光饶。
裹肚衬腰石带,攀胸勒甲步云绦。
闲立山前风吼吼,闷游海外浪滔滔。
一双蓝靛焦筋手,执定追魂取命刀。
要知此物名和姓,声扬二字唤黄袍。
那黄袍老怪,出得门来,便问:“你是那方和尚,在我门首吆喝?”八戒道:“我儿
子,你不认得?我是你老爷!我是大唐差往西天去的!我师父是那御弟三藏。若在你
家里,趁早送出来,省了我钉钯筑进去!”那怪笑道:“是,是,是有一个唐僧在我
家。我也不曾怠慢他,安排些人肉包儿与他吃哩。你们也进去吃一个儿,何如?”

这呆子认真就要进去。沙僧一把扯住道:“哥啊,他哄你哩。你几时又吃人肉
哩?”呆子却才省悟。掣钉钯,望妖怪劈脸就筑。那怪物侧身躲过,使钢刀急架相
迎。两个都显神通,纵云头,跳在空中厮杀。沙僧撇了行李、白马,举宝杖,急急
帮攻。此时两个狠和尚,一个泼妖魔,在云端里,这一场好杀,正是那:

杖起刀迎,钯来刀架。一员魔将施威,两个神僧显化。九齿钯真个英雄,降妖
杖诚然凶咤。没前后左右齐来,那黄袍公然不怕。你看他蘸钢刀晃亮如银,其实的
那神通也为广大。只杀得满空中,雾绕云迷;半山里,崖崩岭咋。一个为声名,怎
肯干休?一个为师父,断然不怕。
他三个在半空中,往往来来,战经数十回合,不分胜负。各因性命要紧,其实难解
难分。

毕竟不知怎救出唐僧,且听下回分解。