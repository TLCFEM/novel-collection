\chapter{凤仙郡冒天止雨~孙大圣劝善施霖}

大道幽深,如何消息,说破鬼神惊骇。挟藏宇宙,剖判玄光,真乐世间无赛。
灵鹫峰前,宝珠拈出,明映五般光彩。照乾坤上下群生,知者寿同山海。

却说三藏师徒四众,别樵子下了隐雾山,奔上大路。行经数日,忽见一座城池
相近。三藏道:“悟空,你看那前面城池,可是天竺国么?”行者摇手道:“不是,
不是!如来处虽称极乐,却没有城池,乃是一座大山,山中有楼台殿阁,唤做灵山
大雷音寺。就到了天竺国,也不是如来住处。天竺国还不知离灵山有多少路哩。那
城想是天竺之外郡。到边前方知明白。”

不一时至城外。三藏下马,入到三层门里,见那民事荒凉,街衢冷落。又到市
口之间,见许多穿青衣者,左右摆列,有几个冠带者,立于房檐之下。他四众顺街
行走,那些人更不逊避。猪八戒村愚,把长嘴掬一掬,叫道:“让路,让路!”那些
人猛抬头,看见模样,一个个骨软筋麻,跌跌,都道:“妖精来了!妖精来了!”
唬得那檐下冠带者,战兢兢躬身问道:“那方来者?”三藏恐他们闯祸,一力当先,
对众道:“贫僧乃东土大唐驾下拜天竺国大雷音寺佛祖求经者。路过宝方,一则不
知地名,二则未落人家,才进城甚失回避,望列公恕罪。”那官人却才施礼道:“此
处乃天竺外郡,地名凤仙郡。连年干旱,郡侯差我等在此出榜,招求法师祈雨救民
也。”行者闻言道:“你的榜文何在?”众官道:“榜文在此,适间才打扫廊檐,还
未张挂。”行者道:“拿来我看看。”众官即将榜文展开,挂在檐下。行者四众上前
同看。榜上写着:

大天竺国凤仙郡郡侯上官,为榜聘明师,招求大法事。兹因郡土宽弘,军民殷
实,连年亢旱,累岁干荒。民田而军地薄,河道浅而沟浍空。井中无水,泉底无
津。富室聊以全生,穷民难以活命。斗粟百金之价,束薪五两之资。十岁女易米三
升,五岁男随人带去。城中惧法,典衣当物以存身;乡下欺公,打劫吃人而顾命。
为此出给榜文,仰望十方贤哲,祷雨救民,恩当重报。愿以千金奉谢,决不虚言。
须至榜者。
行者看罢,对众官道:“‘郡侯上官’何也?”众官道:“上官乃是姓。此我郡侯之
姓也。”行者笑道:“此姓却少。”八戒道:“哥哥不曾读书。《百家姓》后有一句‘上
官欧阳’。”三藏道:“徒弟们,且休闲讲。那个会求雨,与他求一场甘雨,以济民
瘼,此乃万善之事;如不会,就行,莫误了走路。”行者道:“祈雨有甚难事!我老
孙翻江搅海,换斗移星,踢天弄井,吐雾喷云,担山赶月,唤雨呼风:那一件儿不
是幼年耍子的勾当?何为稀罕!”

众官听说,着两个急去郡中报道:“老爷,万千之喜至也!”那郡侯正焚香默祝,
听得报声喜至,即问:“何喜?”那官道:“今日领榜,方至市口张挂,即有四个和
尚,称是东土大唐差往天竺国大雷音拜佛求经者,见榜即道能祈甘雨,特来报知。”

那郡侯即整衣步行,不用轿马多人,径至市口,以礼敦请。忽有人报道:“郡
侯老爷来了。”众人闪过。那郡侯一见唐僧,不怕他徒弟丑恶,当街心倒身下拜道:
“下官乃凤仙郡郡侯上官氏,熏沐拜请老师祈雨救民。望师大舍慈悲,运神功,拔
济拔济!”三藏答礼道:“此间不是讲话处。待贫僧到那寺观,却好行事。”郡侯道:
“老师同到小衙,自有洁净之处。”

师徒们遂牵马挑担,径至府中,一一相见。郡侯即命看茶摆斋。少顷斋至,那
八戒放量舌餐,如同饿虎。唬得那些捧盘的心惊胆战,一往一来,添汤添饭,就如
走马灯儿一般,刚刚供上,直吃得饱满方休。斋毕,唐僧谢了斋,却问:“郡侯大
人,贵处干旱几时了?”郡侯道:
“敝地大邦天竺国,凤仙外郡吾司牧。
一连三载遇干荒,草子不生绝五谷。
大小人家买卖难,十门九户俱啼哭。
三停饿死二停人,一停还似风中烛。
下官出榜遍求贤,幸遇真僧来我国。
若施寸雨济黎民,愿奉千金酬厚德!”
行者听说,满面喜生,呵呵的笑道:“莫说,莫说!若说千金为谢,半点甘雨全无。
但论积功累德,老孙送你一场大雨。”那郡侯原来十分清正贤良,爱民心重,即请
行者上坐,低头下拜道:“老师果舍慈悲,下官必不敢悖德。”行者道:“且莫讲话,
请起。但烦你好生看着我师父,等老孙行事。”沙僧道:“哥哥,怎么行事?”行者
道:“你和八戒过来,就在他这堂下随着我做个羽翼,等老孙唤龙来行雨。”八戒、
沙僧谨依使令。三个人都在堂下。郡侯焚香礼拜。三藏坐着念经。

行者念动真言,诵动咒语,即时见正东上,一朵乌云,渐渐落至堂前,乃是东
海老龙王敖广。那敖广收了云脚,化作人形,走向前,对行者躬身施礼道:“大圣
唤小龙来,那方使用?”行者道:“请起。累你远来,别无甚事;此间乃凤仙郡,
连年干旱,问你如何不来下雨?”老龙道:“启上大圣得知,我虽能行雨,乃上天
遣用之辈。上天不差,岂敢擅自来此行雨?”行者道:“我因路过此方,见久旱民
苦,特着你来此施雨救济,如何推托?”龙王道:“岂敢推托?但大圣念真言呼唤,
不敢不来。一则未奉上天御旨,二则未曾带得行雨神将,怎么动得雨部?大圣既有
拔济之心,容小龙回海点兵,烦大圣到天宫奏准,请一道降雨的圣旨,请水官放出
龙来,我却好照旨意数目下雨。”

行者见他说出理来,只得发放老龙回海。他即跳出罡斗,对唐僧备言龙王之事。
唐僧道:“既然如此,你去为之,切莫打诳语。”行者即吩咐八戒、沙僧:“保着师
父,我上天宫去也。”好大圣,说声去,寂然不见。那郡侯胆战心惊道:“孙老爷那
里去了?”八戒笑道:“驾云上天去了。”郡侯十分恭敬,传出飞报,教满城大街小
巷,不拘公卿士庶,军民人等,家家供养龙王牌位,门设清水缸,缸插杨柳枝,侍
奉香火,拜天不题。

却说行者一驾筋斗云,径到西天门外,早见护国天王引天丁、力士上前迎接道:
“大圣,取经之事完乎?”行者道:“也差不远矣。今行至天竺国界,有一外郡,
名凤仙郡。彼处三年不雨,民甚艰苦,老孙欲祈雨拯救。呼得龙王到彼,他言无旨,
不敢私自为之。特来朝见玉帝请旨。”天王道:“那壁厢敢是不该下雨哩。我向时闻
得说:那郡侯撒泼,冒犯天地,上帝见罪,立有米山、面山、黄金大锁;直等此三
事倒断,才该下雨。”行者不知此意是何,要见玉帝。天王不敢拦阻,让他进去。

径至通明殿外,又见四大天师迎道:“大圣到此何干?”行者道:“因保唐僧,
路至天竺国界,凤仙郡无雨,郡侯召师祈雨。老孙呼得龙王,意命降雨,他说未奉
玉帝旨意,不敢擅行,特来求旨,以苏民困。”四大天师道:“那方不该下雨。”行
者笑道:“该与不该,烦为引奏引奏,看老孙的人情何如。”葛仙翁道:“俗语云:‘苍
蝇包网儿——好大面皮。’”许旌阳道:“不要乱谈,且只带他进去。”邱洪济、张道
陵与葛、许四真人引至灵霄殿下,启奏道:“万岁,有孙悟空路至天竺国凤仙郡,
欲与求雨,特来请旨。”玉帝道:“那厮三年前十二月二十五日,朕出行监观万天,
浮游三界,驾至他方,见那上官正不仁,将斋天素供,推倒喂狗,口出秽言,造有
冒犯之罪,朕即立以三事,在于披香殿内。汝等引孙悟空去看。若三事倒断,即降
旨与他;如不倒断,且休管闲事。”

四天师即引行者至披香殿里看时,见有一座米山,约有十丈高下;一座面山,
约有二十丈高下。米山边有一只拳大之鸡,在那里紧一嘴,慢一嘴,那米吃。面
山边有一只金毛哈巴狗儿,在那里长一舌,短一舌,那面吃。左边悬一座铁架子,
架上挂一把金锁,约有一尺三四寸长短,锁梃有指头粗细,下面有一盏明灯,灯焰
儿燎着那锁梃。行者不知其意,回头问天师曰:“此何意也?”天师道:“那厮触犯
了上天,玉帝立此三事,直等鸡了米尽,狗得面尽,灯焰燎断锁梃,那方才该
下雨哩。”

行者闻言,大惊失色,再不敢启奏。走出殿,满面含羞。四大天师笑道:“大
圣不必烦恼,这事只宜作善可解。若有一念善慈,惊动上天,那米、面山即时就倒,
锁梃即时就断。你去劝他归善,福自来矣。”行者依言,不上灵霄辞玉帝,径来下
界复凡夫。须臾,到西天门,又见护国天王。天王道:“请旨如何?”行者将米山、
面山、金锁之事说了一遍,道:“果依你言,不肯传旨。适间天师送我,教劝那厮
归善,即福原也。”遂相别,降云下界。

那郡侯同三藏、八戒、沙僧、大小官员人等接着,都簇簇攒攒来问。行者将郡
侯喝了一声道:“只因你这厮三年前十二月二十五日冒犯了天地,致令黎民有难,
如今不肯降雨!”郡侯慌得跪伏在地道:“老师如何得知三年前事?”行者道:“你
把那斋天的素供,怎么推倒喂狗?可实实说来!”

那郡侯不敢隐瞒,道:“三年前十二月二十五日,献供斋天,在于本衙之内,
因妻不贤,恶言相斗,一时怒发无知,推倒供桌,泼了素馔,果是唤狗来吃了。这
两年忆念在心,神思恍惚,无处可以解释。不知上天见罪,遗害黎民。今遇老师降
临,万望明示,上界怎么样计较。”

行者道:“那一日正是玉皇下界之日。见你将斋供喂狗,又口出秽言,玉帝即
立三事记汝。”八戒问道:“哥,是那三事?”行者道:“披香殿立一座米山,约有
十丈高下;一座面山,约有二十丈高下。米山边有拳大的一只小鸡,在那里紧一嘴,
慢一嘴的那米吃;面山边有一个金毛哈巴狗儿,在那里长一舌,短一舌的那面
吃。左边又一座铁架子,架上挂一把黄金大锁,锁梃儿有指头粗细,下面有一盏明
灯,灯焰儿燎着那锁梃。直等那鸡米尽,狗面尽,灯燎断锁梃,他这里方才该
下雨哩。”八戒笑道:“不打紧,不打紧!哥肯带我去,变出法身来,一顿把他的米
面都吃了,锁梃弄断了,管取下雨。”行者道:“呆子莫胡说!此乃上天所设之计,
你怎么得见?”三藏道:“似这等说,怎生是好?”行者道:“不难,不难。我临行
时,四天师曾对我言,但只作善可解。”那郡侯拜伏在地,哀告道:“但凭老师指教,
下官一一皈依也。”行者道:“你若回心向善,趁早儿念佛看经,我还替你作为;汝
若仍前不改,我亦不能解释,不久天即诛之,性命不能保矣。”

那郡侯磕头礼拜,誓愿皈依。当时召请本处僧道,启建道场,各各写发文书,
申奏三天。郡侯领众拈香瞻拜,答天谢地,引罪自责。三藏也与他念经。一壁厢又
出飞报,教城里城外大家小户,不论男女人等,都要烧香念佛。自此时,一片善声
盈耳。

行者却才欢喜。对八戒、沙僧道:“你两个好生护持师父,等老孙再与他去去
来。”八戒道:“哥哥,又往那里去?”行者道:“这郡侯听信老孙之言,果然受教,
恭敬善慈,诚心念佛,我这去再奏玉帝,求些雨来。”沙僧道:“哥哥既要去,不必
迟疑,且耽搁我们行路;必求雨一坛,庶成我们之正果也。”

好大圣,又纵云头,直至天门外。还遇着护国天王。天王道:“你今又来做甚?”
行者道:“那郡侯已归善矣。”天王亦喜。正说处,早见直符使者,捧定了道家文书,
僧家关牒,到天门外传递。那符使见了行者,施礼道:“此意乃大圣劝善之功。”行
者道:“你将此文牒送去何处?”符使道:“直送至通明殿上,与天师传递到玉皇大
天尊前。”行者道:“如此,你先行,我当随后而去。”那符使入天门去了。护国天
王道:“大圣,不消见玉帝了。你只往九天应元府下,借点雷神,径自声雷掣电,
还他就有雨下也。”

真个行者依言,入天门里,不上灵霄殿求请旨意,转云步,径往九天应元府,
见那雷门使者、纠录典者、廉访典者都来迎着,施礼道:“大圣何来?”行者道:“有
事要见天尊。”三使者即为传奏。天尊随下九凤丹霞之,整衣出迎。相见礼毕,
行者道:“有一事特来奉求。”天尊道:“何事?”行者道:“我因保唐僧,至凤仙郡,
见那干旱之甚,已许他求雨,特来告借贵部官将到彼声雷。”天尊道:“我知那郡侯
冒犯上天,立有三事,不知可该下雨哩。”行者笑道:“我昨日已见玉帝请旨。玉帝
着天师引我去披香殿看那三事,乃是米山、面山、金锁。只要三事倒断,方该下雨。
我愁难得倒断,天师教我劝化郡侯等众作善,以为‘人有善念,天必从之’,庶几
可以回天心,解灾难也。今已善念顿生,善声盈耳。适间直符使者已将改行从善的
文牒奏上玉帝去了,老孙因特造尊府,告借雷部官将相助相助。”天尊道:“既如此,
差邓、辛、张、陶,帅领闪电娘子,即随大圣下降凤仙郡声雷。”

那四将同大圣,不多时,至于凤仙境界。即于半空中作起法来。只听得唿鲁鲁
的雷声,又见那淅沥沥的闪电。真个是:

电掣紫金蛇,雷轰群蛰哄。荧煌飞火光,霹雳崩山洞。列缺满天明,震惊连地
纵。红销一闪发萌芽,万里江山都撼动。
那凤仙郡,城里城外,大小官员,军民人等,整三年不曾听见雷电;今日见有雷声
霍闪,一齐跪下,头顶着香炉,有的手拈着柳枝,都念“南无阿弥陀佛!南无阿弥
陀佛!”这一声善念,果然惊动上天。正是那古诗云:
人心生一念,天地悉皆知。
善恶若无报,乾坤必有私。

且不说孙大圣指挥雷将,掣电轰雷于凤仙郡,人人归善。却说那上界直符使者,
将僧道两家的文牒,送至通明殿,四天师传奏灵霄殿。玉帝见了道:“那厮们既有
善念,看三事如何。”正说处,忽有披香殿看管的将官报道:“所立米面山俱倒了。
霎时间米面皆无。锁梃亦断。”奏未毕,又有当驾天官引凤仙郡土地、城隍、社令
等神齐来拜奏道:“本郡郡主并满城大小黎庶之家,无一家一人不皈依善果,礼佛
敬天。今启垂慈,普降甘雨,救济黎民。”玉帝闻言大喜,即传旨:“着风部、云部、
雨部,各遵号令,去下方,按凤仙郡界,即于今日今时,声雷布云,降雨三尺零四
十二点。”时有四大天师奉旨,传与各部随时下界,各逞神威,一齐振作。

行者正与邓、辛、张、陶,令闪电娘子在空中调弄,只见众神都到,合会一天。
那其间风云际会,甘雨滂沱。好雨:

漠漠浓云,蒙蒙黑雾。雷车轰轰,闪电灼灼。滚滚狂风,淙淙骤雨。所谓一念
回天,万民满望。全亏大圣施元运,万里江山处处阴。好雨倾河倒海,蔽野迷空。
檐前垂瀑布,窗外响玲珑。万户千门人念佛,六街三市水流洪。东西河道条条满,
南北溪湾处处通。槁苗得润,枯木回生。田畴麻麦盛,村堡豆粮升。客旅喜通贩卖,
农夫爱尔耘耕。从今黍稷多条畅,自然稼穑得丰登。风调雨顺民安乐,海晏河清享
太平。
一日雨下足了三尺零四十二点,众神渐渐收回。孙大圣厉声高叫道:“那四部众
神,且暂停云从,待老孙去叫郡侯拜谢列位。列位可拨开云雾,各现真身,与这凡
夫亲眼看看,他才信心供奉也。”众神听说,只得都停在空中。

这行者按落云头,径至郡里。早见三藏、八戒、沙僧,都来迎接。那郡侯一步
一拜来谢。行者道:“且慢谢我。我已留住四部神,你可传召多人同此拜谢,教
他向后好来降雨。”郡侯随传飞报,召众同酬,都一个个拈香朝拜。只见那四部神
,开明云雾,各现真身。四部者,乃雨部、雷部、云部、风部。只见那:

龙王显像,雷将舒身。云童出现,风伯垂真。龙王显像,银须苍貌世无双;雷
将舒身,钩嘴威颜诚莫比。云童出现,谁如玉面金冠;风伯垂真,曾似燥眉环眼。
齐齐显露青霄上,各各挨排现圣仪。凤仙郡界人才信,顶礼拈香恶性回。今日仰朝
天上将,洗心向善尽皈依。
众神宁待了一个时辰,人民拜之不已。孙行者又起在云端,对众作礼道:“有劳,
有劳!请列位各归本部。老孙还教郡界中人家,供养高真,遇时节醮谢。列位从此
后,五日一风,十日一雨,还来拯救拯救。”众神依言,各各转部不题。

却说大圣坠落云头,与三藏道:“事毕民安,可收拾走路矣。”那郡侯闻言,急
忙行礼道:“孙老爷说那里话!今此一场,乃无量无边之恩德。下官这里差人办备小
宴,奉答厚恩。仍买治民间田地,与老爷起建寺院,立老爷生祠,勒碑刻名,四时
享祀。虽刻骨镂心,难报万一,怎么就说走路的话!”三藏道:“大人之言虽当,但
我等乃西方挂搭行脚之僧,不敢久住。一二日间,定走无疑。”那郡侯那里肯放。
连夜差多人治办酒席,起盖祠宇。

次日,大开佳宴,请唐僧高坐;孙大圣与八戒、沙僧列坐。郡侯同本郡大小官
员部臣把杯献馔,细吹细打,款待了一日。这场果是欣然。有诗为证:
田畴久旱逢甘雨,河道经商处处通。
深感神僧来郡界,多蒙大圣上天宫。
解除三事从前恶,一念皈依善果弘。
此后愿如尧舜世,五风十雨万年丰。

一日筵,二日宴;今日酬,明日谢;扳留将有半月,只等寺院生祠完备。一日,
郡侯请四众往观。唐僧惊讶道:“功程浩大,何成之如此速耶?”郡侯道:“下官催
趱人工,昼夜不息,急急命完,特请列位老爷看看。”行者笑道:“果是贤才能干的
好贤侯也!”即时都到新寺。见那殿阁巍峨,山门壮丽,俱称赞不已。行者请师父
留一寺名。三藏道:“有,留名当唤做‘甘霖普济寺’。”郡侯称道:“甚好,甚好!”
用金贴广招僧众,侍奉香火。殿左边立起四众生祠,每年四时祭祀;又起盖雷神、
龙神等庙,以答神功。看毕,即命趱行。那一郡人民,知久留不住,各备赆仪,分
文不受。因此,合郡官员人等,盛张鼓乐,大展旌幢,送有三十里远近,犹不忍别,
遂掩泪目送,直至望不见方回。这正是:
硕德神僧留普济,齐天大圣广施恩。

毕竟不知此去还有几日方见如来,且听下回分解。