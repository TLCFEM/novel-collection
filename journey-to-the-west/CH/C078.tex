\chapter{比丘怜子遣阴神~金殿识魔谈道德}

一念才生动百魔,修持最苦奈他何。
但凭洗涤无尘垢,也用收拴有琢磨。
扫退万缘归寂灭,荡除千怪莫蹉跎。
管教跳出樊笼套,行满飞升上大罗。

话说孙大圣用尽心机,请如来收了众怪,解脱三藏师徒之难,离狮驼城西行。
又经数月,早值冬天。但见那:
岭梅将破玉,池水渐成冰。
红叶俱飘落,青松色更新。
淡云飞欲雪,枯草伏山平。
满目寒光迥,阴阴透骨冷。
师徒们冲寒冒冷,宿雨餐风。正行间,又见一座城池。三藏问道:“悟空,那厢又
是甚么所在?”行者道:“到跟前自知。若是西邸王位,须要倒换关文;若是府州
县,径过。”师徒言语未毕,早至城门之外。

三藏下马,一行四众,进了月城。见一个老军,在向阳墙下,偎风而睡。行者
近前,摇他一下,叫声:“长官。”那老军猛然惊觉,麻麻糊糊的睁开眼,看见行者,
连忙跪下磕头,叫:“爷爷!”行者道:“你休胡惊作怪。我又不是甚么恶神,你叫
‘爷爷’怎的!”老军磕头道:“你是雷公爷爷?”行者道:“胡说!吾乃东土去西天
取经的僧人。适才到此,不知地名,问你一声的。”那老军闻言,却才正了心,打
个呵欠,爬起来,伸伸腰道:“长老,长老,恕小人之罪。此处地方,原唤比丘国,
今改作小子城。”行者道:“国中有帝王否?”老军道:“有,有,有。”

行者却转身对唐僧道:“师父,此处原是比丘国,今改小子城。但不知改名之
意何故也。”唐僧疑惑道:“既云比丘,又何云小子?……”八戒道:“想是比丘王崩
了,新立王位的是个小子,故名小子城。”唐僧道:“无此理,无此理!我们且进去,
到街坊上再问。”沙僧道:“正是。那老军一则不知,二则被大哥唬得胡说。且入城
去询问。”

又入三层门里,到通衢大市观看,倒也衣冠济楚,人物清秀。但见那:
酒楼歌馆语声喧,彩铺茶房高挂帘。
万户千门生意好,六街三市广财源。
买金贩锦人如蚁,夺利争名只为钱。
礼貌庄严风景盛,河清海晏太平年。
师徒四众牵着马,挑着担,在街市上行够多时,看不尽繁华气概。但只见家家门口
一个鹅笼。三藏道:“徒弟啊,此处人家,都将鹅笼放在门首,何也?”八戒听说,
左右观之,果是鹅笼,排列五色彩缎遮幔。呆子笑道:“师父,今日想是黄道良辰,
宜结婚姻会友。都行礼哩。”行者道:“胡谈!那里就家家都行礼!其间必有缘故。等
我上前看看。”三藏扯住道:“你莫去。你嘴脸丑陋,怕人怪你。”行者道:“我变化
个儿去来。”

好大圣,捻着诀,念声咒语,摇身一变,变作一个蜜蜂儿,展开翅,飞近边前,
钻出幔里观看。原来里面坐的是个小孩儿。再去第二家笼里着,也是个小孩儿。连
看八九家,都是个小孩儿。却是男身,更无女子。有的坐在笼中顽耍,有的坐在里
边啼哭;有的吃果子,有的或睡坐。行者看罢,现原身,回报唐僧道:“那笼里是
些小孩子,大者不满七岁,小者只有五岁,不知何故。”三藏见说,疑思不定。

忽转街见一衙门,乃金亭馆驿。长老喜道:“徒弟,我们且进这驿里去。一则
问他地方,二则撒和马匹,三则天晚投宿。”沙僧道:“正是,正是,快进去耶。”
四众欣然而入。只见那在官人果报与驿丞。接入门,各各相见。叙坐定,驿丞问:
“长老自何方来?”三藏言:“贫僧东土大唐差往西天取经者。今到贵处,有关文
理当照验,权借高衙一歇。”驿丞即命看茶。茶毕,即办支应,命当直的安排管待。
三藏称谢。又问:“今日可得入朝见驾,照验关文?”驿丞道:“今晚不能,须待明
日早朝。今晚且于敝衙门宽住一宵。”少顷,安排停当,驿丞即请四众,同吃了斋
供。又教手下人打扫客房安歇。三藏感谢不尽。

既坐下,长老道:“贫僧有一件不明之事请教,烦为指示。贵处养孩儿,不知
怎生看待。”驿丞道:“‘天无二日,人无二理。’养育孩童,父精母血,怀胎十月,
待时而生;生下乳哺三年,渐成体相。岂有不知之理!”三藏道:“据尊言与敝邦无
异;但贫僧进城时,见街坊人家,各设一鹅笼,都藏小儿在内。此事不明,故敢动
问。”驿丞附耳低言道:“长老莫管他,莫问他,也莫理他、说他。请安置,明早走
路。”长老闻言,一把扯住驿丞,定要问个明白。驿丞摇头摇指,只叫:“谨言!”
三藏一发不放,执死定要问个详细。驿丞无奈,只得屏去一应在官人等。独在灯光
之下,悄悄而言道:“适所问鹅笼之事,乃是当今国主无道之事。你只管问他怎的!”
三藏道:“何为无道?必见教明白,我方得放心。”

驿丞道:“此国原是比丘国,近有民谣,改作小子城。三年前,有一老人,打
扮做道人模样,携一小女子,年方一十六岁。其女形容娇俊,貌若观音。进贡与当
今;陛下爱其色美,宠幸在宫,号为美后。近来把三宫娘娘,六院妃子,全无正眼
相觑,不分昼夜,贪欢不已。如今弄得精神瘦倦,身体羸,饮食少进,命在须臾。
太医院检尽良方,不能疗治。那进女子的道人,受我主诰封,称为国丈。国丈有海
外秘方,甚能延寿。前者去十洲、三岛,采将药来,俱已完备。但只是药引子利害;
单用着一千一百一十一个小儿的心肝,煎汤服药。服后有千年不老之功。这些鹅笼
里的小儿,俱是选就的,养在里面。人家父母,惧怕王法,俱不敢啼哭,遂传播谣
言,叫做小儿城。此非无道而何?长老明早到朝,只去倒换关文,不得言及此事。”
言毕,抽身而退。

唬得个长老骨软筋麻,止不住腮边泪堕;忽失声叫道:“昏君,昏君!为你贪欢
爱美,弄出病来,怎么屈伤这许多小儿性命!苦哉!苦哉!痛杀我也!”有诗为证,诗
曰:
邪主无知失正真,贪欢不省暗伤身。
因求永寿戕童命,为解天灾杀小民。
僧发慈悲难割舍,官言利害不堪闻。
灯前洒泪长吁叹,痛倒参禅向佛人。
八戒近前道:“师父,你是怎的起哩?‘专把别人棺材抬在自家家里哭’!不要烦恼!
常言道:‘君教臣死,臣不死不忠;父教子亡,子不亡不孝。’他伤的是他的子民,
与你何干!且来宽衣服睡觉,‘莫替古人耽忧。’”三藏滴泪道:“徒弟啊,你是一个
不慈悯的!我出家人,积功累行,第一要行方便。怎么这昏君一味胡行!从来也不见
吃人心肝,可以延寿。这都是无道之事,教我怎不伤悲!”沙僧道:“师父且莫伤悲。
等明早倒换关文,觌面与国王讲过。如若不从,看他是怎么模样的一个国丈。或恐
那国丈是个妖精,欲吃人的心肝,故设此法,未可知也。”

行者道:“悟净说得有理。师父,你且睡觉,明日等老孙同你进朝,看国丈的
好歹。如若是人,只恐他走了傍门,不知正道,徒以采药为真,待老孙将先天之要
旨,化他皈正;若是妖邪,我把他拿住,与这国王看看,教他宽欲养身,断不教他
伤了那些孩童性命。”三藏闻言,急躬身,反对行者施礼道:“徒弟啊,此论极妙,
极妙!但只是见了昏君,不可便问此事,恐那昏君不分远近,并作谣言见罪,却怎
生区处!”行者笑道:“老孙自有法力。如今先将鹅笼小儿摄离此城,教他明日无物
取心。地方官自然奏表。那昏君必有旨意,或与国丈商量,或者另行选报。那时节,
借此举奏,决不致罪坐于我也。”三藏甚喜。又道:“如今怎得小儿离城?若果能脱
得,真贤徒天大之德!可速为之,略迟缓些,恐无及也。”行者抖擞神威,即起身,
吩咐八戒、沙僧:“同师父坐着,等我施为,你看但有阴风刮动,就是小儿出城了。”
他三人一齐俱念:“南无救生药师佛!南无救生药师佛!”

这大圣出得门外,打个唿哨,起在半空,捻了诀,念动真言,叫声“净法界”,
拘得那城隍、土地、社令、真官,并五方揭谛、四值功曹、六丁六甲与护教伽蓝等
众,都到空中,对他施礼道:“大圣,夜唤吾等,有何急事?”行者道:“今因路过
比丘国,那国王无道,听信妖邪,要取小儿心肝做药引子,指望长生。我师父十分
不忍,欲要救生灭怪,故老孙特请列位,各使神通,与我把这城中各街坊人家鹅笼
里的小儿,连笼都摄出城外山凹中,或树林深处,收藏一二日,与他些果子食用,
不得饿损;再暗的护持,不得使他惊恐啼哭。待我除了邪,治了国,劝正君王,临
行时,送来还我。”

众神听令。即便各使神通,按下云头。满城中:

阴风滚滚,惨雾漫慢;阴风刮暗一天星,惨雾遮昏千里月。起初时,还荡荡悠
悠;次后来,就轰轰烈烈。悠悠荡荡,各寻门户救孩童;烈烈轰轰,都看鹅笼援骨
血。冷气侵人怎出头,寒威透体衣如铁。父母徒张皇,兄嫂皆悲切。满地卷阴风,
笼儿被神摄。此夜纵孤,天明尽欢悦。
有诗为证,诗曰:
释门慈悯古来多,正善成功说摩诃。
万圣千真皆积德,三皈五戒要从和。
比丘一国非君乱,小子千名是命讹。
行者因师同救护,这场阴骘胜波罗。
当夜有三更时分,众神把鹅笼摄去各处安藏。

行者按下祥光,径至驿庭上。只听得他三人还念“南无救生药师佛”哩。他也
心中暗喜。近前叫:“师父,我来也。阴风之起何如?”八戒道:“好阴风!”三藏
道:“救儿之事,却怎么说?”行者道:“已一一救他出去,待我们起身时送还。”
长老谢了又谢,方才就寝。

至天晓,三藏醒来,遂结束齐备道:“悟空,我趁早朝,倒换关文去也。”行者
道:“师父,你自家去,恐不济事;待老孙和你同去,看那国丈邪正如何。”三藏道:
“你去却不肯行礼,恐国王见怪。”行者道:“我不现身,暗中跟随你,就当保护。”
三藏甚喜,吩咐八戒、沙僧看守行李、马匹。却才举步,这驿丞又来相见。看这长
老打扮起来,比昨日又甚不同。但见他:

身上穿一领锦异宝佛袈裟,头戴金顶毗卢帽。九环锡杖手中拿,胸藏一点神
光妙。通关文牒紧随身,包裹袋中缠锦套。行似阿罗降世间,诚如活佛真容貌。
那驿丞相见礼毕,附耳低言,只教莫管闲事。三藏点头应声。大圣闪在门旁,念个
咒语,摇身一变,变做个虫儿,“嘤”的一声,飞在三藏帽儿上。出了馆驿,
径奔朝中。

及到朝门外,见有黄门官,即施礼道:“贫僧乃东土大唐差往西天取经者。今
到贵地,理当倒换关文。意欲见驾,伏乞转奏转奏。”那黄门官果为传奏。国王喜
道:“远来之僧,必有道行。”教请进来。黄门官复奉旨,将长老请入。

长老阶下朝见毕,复请上殿赐坐。长老又谢恩坐了。只见那国王相貌羸,精
神倦怠:举手处,揖让差池;开言时,声音断续。长老将文牒献上,那国王眼目昏
朦,看了又看,方才取宝印用了花押,递与长老。长老收讫。

那国王正要问取经原因,只听得当驾官奏道:“国丈爷爷来矣。”那国王即扶着
近侍小宦,挣下龙床,躬身迎接。慌得那长老急起身,侧立于旁。回头观看,原来
是一个老道者,自玉阶前,摇摇摆摆而进。但见他:

头上戴一顶淡鹅黄九锡云锦纱巾,身上穿一领顶梅沉香绵丝鹤氅。腰间系一
条纫蓝三股攒绒带,足下踏一对麻经葛纬云头履。手中拄一根九节枯藤盘龙拐杖,
胸前挂一个描龙刺凤团花锦囊。玉面多光润,苍髯颔下飘。金睛飞火焰,长目过眉
梢。行动云随步,逍遥香雾饶。阶下众官都拱接,齐呼国丈进王朝。
那国丈到宝殿前,更不行礼,昂昂烈烈,径到殿上。国王欠身道:“国丈仙踪,今
喜早降。”就请左手绣墩上坐。三藏起一步,躬身施礼道:“国丈大人,贫僧问讯了。”
那国丈端然高坐,亦不回礼。转面向国王道:“僧家何来?”国王道:“东土唐朝差
上西天取经者。今来倒验关文。”国丈笑道:“西方之路,黑漫漫有甚好处!”三藏
道:“自古西方乃极乐之胜境,如何不好?”那国王问道:“朕闻上古有云:‘僧是
佛家弟子。’端的不知为僧可能不死,向佛可能长生?”三藏闻言,急合掌应道:

“为僧者,万缘都罢;了性者,诸法皆空。大智闲闲,澹泊在不生之内;真机
默默,逍遥于寂灭之中。三界空而百端治,六根净而千种穷。若乃坚诚知觉,须当
识心:心净则孤明独照,心存则万境皆清。真容无欠亦无余,生前可见;幻相有形
终有坏,分外何求?行功打坐,乃为入定之原;布惠施恩,诚是
修行之本。大巧若拙,还知事事无为;善计非筹,必须头头放下。但使一心不动,
万行自全;若云采阴补阳,诚为谬语,服饵长寿,实乃虚词。只要尘尘缘总弃,物
物色皆空。素素纯纯寡爱欲,自然享寿永无穷。”
那国丈闻言,付之一笑。用手指定唐僧道:“呵,呵,呵!你这和尚满口胡柴!寂灭
门中,须云认性;你不知那性从何而灭!枯坐参禅,尽是些盲修瞎炼。俗语云:‘坐,
坐,坐!你的屁股破。火熬煎,反成祸。’更不知我这:

修仙者,骨之坚秀;达道者,神之最灵。携箪瓢而入山访友,采百药而临世济
人。摘仙花以砌笠,折香蕙以铺。歌之鼓掌,舞罢眠云。阐道法,扬太上之正教;
施符水,除人世之妖氛。夺天地之秀气,采日月之华精。运阴阳而丹结,按水火而
胎凝。二八阴消兮,若恍若惚;三九阳长兮,如杳如冥。应四时而采取药物,养九
转而修炼丹成。跨青鸾,升紫府;骑白鹤,上瑶京。参满天之华采,表妙道之殷勤。
比你那静禅释教,寂灭阴神,涅遗臭壳,又不脱凡尘。三教之中无上品,古来惟
道独称尊!”
那国王听说,十分欢喜。满朝官都喝采道:“好个‘惟道独称尊’,‘惟道独称尊’!”
长老见人都赞他,不胜羞愧。国王又叫光禄寺安排素斋,待那远来之僧出城西去。

三藏谢恩而退。才下殿,往外正走,行者飞下帽顶儿,来在耳边叫道:“师父,
这国丈是个妖邪。国王受了妖气。你先去驿中等斋,待老孙在这里听他消息。”三
藏知会了,独出朝门不题。

看那行者,一翅飞在金銮殿翡翠屏中钉下,只见那班部中闪出五城兵马官,奏
道:“我主,今夜一阵冷风,将各坊各家鹅笼里小儿,连笼都刮去了,更无踪迹。”
国王闻奏,又惊又恼,对国丈道:“此事乃天灭朕也!连月病重,御医无效。幸国丈
赐仙方,专待今日午时开刀,取此小儿心肝作引,何期被冷风刮去。非天欲灭朕而
何?”国丈笑道:“陛下且休烦恼。此儿刮去,正是天送长生与陛下也。”国王道:
“见把笼中之儿刮去,何以返说天送长生?”国丈道:“我才入朝来,见了一个绝
妙的药引,强似那一千一百一十一个小儿之心。那小儿之心,只延得陛下千年之寿;
此引子,吃了我的仙药,就可延万万年也。”国王漠然不知是何药引,请问再三,
国丈才说:“那东土差去取经的和尚,我观他器宇清净,容颜齐整,乃是个十世修
行的真体,自幼为僧,元阳未泄。比那小儿更强万倍。若得他的心肝煎汤,服我的
仙药,足保万年之寿。”

那昏君闻言,十分听信。对国丈道:“何不早说?若果如此有效,适才留住,不
放他去了。”国丈道:“此何难哉!适才吩咐光禄寺办斋待他,他必吃了斋,方才出
城。如今急传旨,将各门紧闭;点兵围了金亭馆驿,将那和尚拿来,必以礼求其心。
如果相从,即时剖而取出,遂御葬其尸,还与他立庙享祭;如若不从,就与他个武
不善作,即时捆住,剖开取之。有何难事?”那昏君如其言,即传旨,把各门闭了。
又差羽林卫大小官军,围住馆驿。

行者听得这个消息,一翅飞奔馆驿,现了本相,对唐僧道:“师父,祸事了,
祸事了!”那三藏才与八戒、沙僧领御斋,忽闻此言,唬得三尸神散,七窍烟生,
倒在尘埃,浑身是汗,眼不定睛,口不能言。慌得沙僧上前搀住,只叫:“师父苏
醒,师父苏醒!”八戒道:“有甚祸事,有甚祸事?你慢些儿说便也罢,却唬得师父
如此!”行者道:“自师父出朝,老孙回视,那国丈是个妖精。少顷,有五城兵马来
奏冷风刮去小儿之事。国王方恼,他却转教喜欢,道:‘这是天送长生与你。’要取
师父的心肝做药引,可延万年之寿。那昏君听信诬言,所以点精兵来围馆驿,差锦
衣官来请师父求心也。”八戒笑道:“行的好慈悯,救的好小儿,刮的好阴风,今番
却撞出祸来了!”

三藏战兢兢的,爬起来,扯着行者,哀告道:“贤徒啊!此事如何是好?”行者
道:“若要好,大做小。”沙僧道:“怎么叫做‘大做小’?”行者道:“若要全命,
师作徒,徒作师,方可保全。”三藏道:“你若救得我命,情愿与你做徒子、徒孙也。”
行者道:“既如此,不必迟疑。”教:“八戒,快和些泥来。”那呆子即使钉钯,筑了
些土。又不敢外面去取水,后就掳起衣服撒溺,和了一团臊泥,递与行者。行者没
奈何,将泥扑作一片,往自家脸上一安,做下个猴象的脸子,叫唐僧站起休动,再
莫言语,贴在唐僧脸上,念动真言,吹口仙气,叫:“变!”那长老即变做个行者模
样;脱了他的衣服,以行者的衣服穿上。行者却将师父的衣服穿了,捻着诀,念个
咒语,摇身变作唐僧的嘴脸。八戒、沙僧也难识认。

正当合心装扮停当,只听得锣鼓齐鸣,又见那枪刀簇拥。原来是羽林卫官,领
三千兵把馆驿围了。又见一个锦衣官走进驿庭问道:“东土唐朝长老在那里?”慌
得那驿丞战兢兢的跪下,指道:“在下面客房里。”锦衣官即至客房里道:“唐长老,
我王有请。”八戒、沙僧,左右护持“假行者”。只见“假唐僧”出门施礼道:“锦
衣大人,陛下召贫僧,有何话说?”锦衣官上前一把扯住道:“我与你进朝去。想
必有取用也。”咦!这正是:
妖诬胜慈善,慈善反招凶。

毕竟不知此去端的性命何如,且听下回分解。