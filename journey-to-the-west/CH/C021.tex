\chapter{护法设庄留大圣~须弥灵吉定风魔}

却说那五十个败残的小妖,拿着些破旗、破鼓,撞入洞里,报道:“大王,虎
先锋战不过那毛脸和尚,被他赶下东山坡去了。”老妖闻说,十分烦恼。正低头不
语,默思计策,又有把前门的小妖道:“大王,虎先锋被那毛脸和尚打杀了,拖在
门口骂战哩。”那老妖闻言,愈加烦恼道:“这厮却也无知!我倒不曾吃他师父,他
转打杀我家先锋,可恨!可恨!”叫:“取披挂来。我也只闻得讲甚么孙行者,等我
出去,看是个甚么九头八尾的和尚,拿他进来,与我虎先锋对命。”众小妖急急抬
出披挂。老妖结束齐整,绰一杆三股钢叉,帅群妖跳出本洞。那大圣停立门外,见
那怪走将出来,着实骁勇。看他怎生打扮,但见:

金盔晃日,金甲凝光。盔上缨飘山雉尾,罗袍罩甲淡鹅黄。勒甲绦盘龙耀彩,
护心镜绕眼辉煌。鹿皮靴,槐花染色;锦围裙,柳叶绒妆。手持三股钢叉利,不亚
当年显圣郎。

那老妖出得门来,厉声高叫道:“那个是孙行者?”这行者脚着虎怪的皮囊,
手执着如意的铁棒,答道:“你孙外公在此,送出我师父来!”那怪仔细观看,见行
者身躯鄙猥,面容羸瘦,不满四尺。笑道:“可怜,可怜!我只道是怎么样扳翻不倒
的好汉,原来是这般一个骷髅的病鬼!”行者笑道:“你这个儿子,忒没眼色!你外
公虽是小小的,你若肯照头打一叉柄,就长三尺。”那怪道:“你硬着头,吃吾一柄。”
大圣公然不惧。那怪果打一下来,他把腰躬一躬,足长了三尺,有一丈长短,慌得
那妖把钢叉按住,喝道:“孙行者,你怎么把这护身的变化法儿,拿来我门前使唤!
莫弄虚头,走上来,我与你见见手段!”行者笑道:“儿子啊!常言道:‘留情不举手,
举手不留情。’你外公手儿重重的,只怕你捱不起这一棒!”

那怪那容分说,拈转钢叉,望行者当胸就刺。这大圣正是会家不忙,忙家不会,
理开铁棒,使一个“乌龙掠地势”,拨开钢叉,又照头便打。他二人在那黄风洞口,
这一场好杀:

妖王发怒,大圣施威:妖王发怒,要拿行者抵先锋;大圣施威,欲捉精灵救长
老。叉来棒架,棒去叉迎。一个是镇山都总帅,一个是护法美猴王。初时还在尘埃
战,后来各起在中央。点钢叉,尖明利;如意棒,身黑箍黄。戳着的魂归冥府,
打着的定见阎王。全凭着手疾眼快,必须要力壮身强。两家舍死忘生战,不知那个
平安那个伤。

那老妖与大圣斗经三十回合,不分胜败。这行者要见功绩,使一个“身外身”
的手段:把毫毛揪下一把,用口嚼得粉碎,望上一喷,叫声“变!”变有百十个行
者,都是一样打扮,各执一根铁棒,把那怪围在空中。那怪害怕,也使一般本事:
急回头,望着巽地上,把口张了三张,的一口气,吹将出去,忽然间,一阵黄风,
从空刮起。好风!真个利害。

冷冷飕飕天地变,无影无形黄沙旋。穿林折岭倒松梅,播土扬尘崩岭坫。黄河
浪泼彻底浑,湘江水涌翻波转。碧天振动斗牛宫,争些刮倒森罗殿。五百罗汉闹喧
天,八大金刚齐嚷乱。文殊走了青毛狮,普贤白象难寻见。真武龟蛇失了群,梓
骡子飘其。行商喊叫告苍天,梢公拜许诸般愿。烟波性命浪中流,名利残生随水
办。仙山洞府黑攸攸,海岛蓬莱昏暗暗。老君难顾炼丹炉,寿星收了龙须扇。王母
正去赴蟠桃,一
风吹断裙腰钏。二郎迷失灌州城,哪吒难取匣中剑。天王不见手心塔,鲁班吊了金
头钻。雷音宝阙倒三层,赵州石桥崩两断。一轮红日荡无光,满天星斗皆昏乱。南
山鸟往北山飞,东湖水向西湖漫。雌雄拆对不相呼,子母分离难叫唤。龙王遍海找
夜叉,雷公到处寻闪电。十代阎王觅判官,地府牛头追马面。这风吹倒普陀山,卷
起观音经一卷。白莲花卸海边飞,吹倒菩萨十二院。盘古至今曾见风,不似这风来
不善。唿喇喇,乾坤险不炸崩开,万里江山都是颤!
那妖怪使出这阵狂风,就把孙大圣毫毛变的小行者刮得在那半空中,却似纺车儿一
般乱转,莫想轮得棒,如何拢得身?慌得行者将毫毛一抖,收上身来,独自个举着
铁棒,上前来打,又被那怪劈脸喷了一口黄风,把两只火眼金睛,刮得紧紧闭合,
莫能睁开;因此难使铁棒,遂败下阵来。那妖收风回洞不题。

却说猪八戒见那黄风大作,天地无光,牵着马,守着担,伏在山凹之间,也不
敢睁眼,不敢抬头,口里不住的念佛许愿;又不知行者胜负何如,师父死活何如。
正在那疑思之时,却早风定天晴。忽抬头往那洞门前看处,却也不见兵戈,不闻锣
鼓。呆子又不敢上他门,又没人看守马匹、行李,果是进退两难,怆惶不已。

忧虑间,只听得孙大圣从西边吆喝而来,他才欠身迎着道:“哥哥,好大风啊!
你从那里走来?”行者摆手道:“利害,利害,我老孙自为人,不曾见这大风。那
老妖使一柄三股钢叉,来与老孙交战;战到有三十余合,是老孙使一个身外身的本
事,把他围打,他甚着急,故弄出这阵风来,果是凶恶,刮得我站立不住,收了本
事,冒风而逃。——哏,好风!哏,好风!老孙也会呼风,也会唤雨,不曾似这个妖
精的风恶!”八戒道:“师兄,那妖精的武艺如何?”行者道:“也看得过。叉法儿
倒也齐整。与老孙也战个手平。却只是风恶了,难得赢他。”八戒道:“似这般怎生
救得师父?”行者道:“救师父且等再处,不知这里可有眼科先生,且教他把我眼
医治医治。”八戒道:“你眼怎的来?”行者道:“我被那怪一口风喷将来,吹得我
眼珠酸痛,这会子冷泪常流。”八戒道:“哥啊,这半山中,天色又晚,且莫说要甚
么眼科,连宿处也没有了!”行者道:“要宿处不难。我料着那妖精还不敢伤我师父,
我们且找上大路,寻个人家住下,过此一宵,明日天光,再来降妖罢。”八戒道:“正
是,正是。”

他却牵了马,挑了担,出山凹,行上路口。此时渐渐黄昏,只听得那路南山坡
下,有犬吠之声。二人停身观看,乃是一家庄院,影影的有灯火光明。他两个也不
管有路无路,漫草而行,直至那家门首。但见:

紫芝翳翳,白石苍苍:紫芝翳翳多青草,白石苍苍半绿苔。数点小萤光灼灼,
一林野树密排排。香兰馥郁,嫩竹新栽。清泉流曲涧,古柏倚深崖。地僻更无游客
到,门前惟有野花开。
他两个不敢擅入,只得叫一声“开门,开门”!那里有一老者,带几个年幼的农夫,
叉钯扫帚齐来,问道:“甚么人?甚么人?”行者躬身道:“我们是东土大唐圣僧的
徒弟。因往西方拜佛求经,路过此山,被黄风大王拿了我师父去了,我们还未曾救
得。天色已晚,特来府上告借一宵,万望方便方便。”那老者答礼道:“失迎,失迎。
此间乃云多人少之处,却才闻得叫门,恐怕是妖狐、老虎,及山中强盗等类,故此
小介愚顽,多有冲撞。不知是二位长老。请进,请进。”

他兄弟们牵马挑担而入,径至里边,拴马歇担,与庄老拜见叙坐。又有苍头献
茶。茶罢,捧出几碗胡麻饭。饭毕,命设铺就寝。行者道:“不睡还可,敢问善人,
贵地可有卖眼药的?”老者道:“是那位长老害眼?”行者道:“不瞒你老人家说,
我们出家人,自来无病,从不晓得害眼。”老人道:“既不害眼,如何讨药?”行者
道:“我们今日在黄风洞口救我师父,不期被那怪将一口风喷来,吹得我眼珠酸痛;
今有些眼泪汪汪,故此要寻眼药。”那老者道:“善哉!善哉!你这个长老,小小的年
纪,怎么说谎?那黄风大圣,风最利害。他那风,比不得甚么春秋风、松竹风、与
那东西南北风。……”八戒道:“想必是夹脑风、羊耳风、大麻风、偏正头风?”
长者道:“不是,不是。他叫做‘三昧神风’。”行者道:“怎见得?”老者道:“那
风,能吹天地暗,善刮鬼神愁。裂石崩崖恶,吹人命即休。你们若遇着他那风吹了
呵,还想得活哩!只除是神仙,方可得无事。”行者道:“果然,果然,我们虽不是
神仙,神仙还是我的晚辈,这条命急切难休,却只是吹得我眼珠酸痛!”那老者道:
“既如此说,也是个有来头的人。我这敝处,却无卖眼药的。老汉也有些迎风冷泪,
曾遇异人,传了一方,名唤‘三花九子膏’,能治一切风眼。”行者闻言,低头唱喏
道:“愿求些儿,点试,点试。”

那老者应承,即走进去,取出一个玛瑙石的小儿来,拔开塞口,用玉簪儿蘸
出少许与行者点上,教他不得睁开,宁心睡觉,明早就好。点毕,收了石,径领
小介们退于里面。八戒解包袱,展开铺盖,请行者安置。行者闭着眼乱摸。八戒笑
道:“先生,你的明杖儿呢?”行者道:“你这个馕糟的呆子!你照顾我做瞎子哩!”
那呆子哑哑的暗笑而睡。行者坐在铺上,转运神功,直到有三更后,方才睡下。

不觉又是五更将晓,行者抹抹脸,睁开眼道:“果然好药!比常更有百分光明!”
却转头后边望望,呀!那里得甚房舍窗门,但只见些老槐高柳,兄弟们都睡在那绿
莎茵上。那八戒醒来道:“哥哥,你嚷怎的?”行者道:“你睁开眼看看。”呆子忽
抬头,见没了人家,慌得一毂辘爬将起来道:“我的马哩?”行者道:“树上拴的不
是?”——“行李呢?”行者道:“你头边放的不是?”八戒道:“这家子惫懒也。
他搬了,怎么就不叫我们一声?通得老猪知道,也好与你送些茶果。想是躲门户的,
恐怕里长晓得,却就连夜搬了。噫!我们也忒睡得死!怎么他家拆房子,响也不听见
响响?”行者吸吸的笑道:“呆子,不要乱嚷。你看那树上是个甚么纸帖儿。”八戒
走上前,用手揭了,原来上面四句颂子云:
庄居非是俗人居,护法伽蓝点化庐。
妙药与君医眼痛,尽心降怪莫踌躇。

行者道:“这伙强神,自换了龙马,一向不曾点他,他倒又来弄虚头!”八戒道:
“哥哥莫扯架子。他怎么伏你点札!”行者道:“兄弟,你还不知哩。这护教伽蓝、
六丁六甲、五方揭谛,四值功曹,奉菩萨的法旨,暗保我师父者。自那日报了名,
只为这一向有了你,再不曾用他们,故不曾点札罢了。”八戒道:“哥哥,他既奉法
旨暗保师父,所以不能现身明显,故此点化仙庄。你莫怪他,昨日也亏他与你点眼,
又亏他管了我们一顿斋饭,亦可谓尽心矣。你莫怪他,我们且去救师父来。”行者
道:“兄弟说得是。此处到那黄风洞口不远,你且莫动身,只在林子里看马守担,
等老孙去洞里打听打听,看师父下落如何,再与他争战。”八戒道:“正是这等。讨
一个死活的实信。假若师父死了,各人好寻头干事;若是未死,我们好竭力尽心。”
行者道:“莫乱谈,我去也!”

他将身一纵,径到他门首,门尚关着睡觉。行者不叫门,且不惊动妖怪,捻着
诀,念个咒语,摇身一变,变做一个花脚蚊虫,真个小巧!有诗为证。诗曰:

扰扰微形利喙,嘤嘤声细如雷。兰房纱帐善通随,正爱炎天暖气。只怕熏烟扑
扇,偏怜灯火光辉。轻轻小小忒钻疾,飞入妖精洞里。
只见那把门的小妖,正打鼾睡,行者往他脸上叮了一口,那小妖翻身醒了。道:“我
爷哑!好大蚊子!一口就叮了一个大疙疸!”忽睁眼道:“天亮了。”又听得支的一声,
二门开了。行者嘤嘤的飞将进去,只见那老妖吩咐各门上谨慎,一壁厢收拾兵器:
“只怕昨日那阵风不曾刮死孙行者,他今日必定还来。来时定教他一命休矣。”

行者听说,又飞过那厅堂,径来后面。但见一层门,关得甚紧,行者漫门缝儿
钻将进去,原来是个大空园子,那壁厢定风桩上绳缠索绑着唐僧哩。那师父纷纷泪
落,心心只念着悟空、悟能,不知都在何处。行者停翅,叮在他光头上,叫声“师
父”。那长老认得他的声音道:“悟空啊,想杀我也!你在那里叫我哩?”行者道:“师
父,我在你头上哩。你莫要心焦,少得烦恼。我们务必拿住妖精,方才救得你的性
命。”唐僧道:“徒弟啊,几时才拿得妖精么?”行者道:“拿你的那虎怪,已被八
戒打死了。只是老妖的风势利害。料着只在今日,管取拿他。你放心莫哭,我去哑。”

说声去,嘤嘤的飞到前面。只见那老妖坐在上面,正点札各路头目;又见那洞
前有一个小妖,把个令字旗磨一磨,撞上厅来报道:“大王,小的巡山,才出门,
见一个长嘴大耳朵的和尚坐在林里;若不是我跑得快些,几乎被他捉住。却不见昨
日那个毛脸和尚。”老妖道:“孙行者不在,想必是风吹死也。再不便去那里求救兵
去了!”众妖道:“大王,若果吹杀了他,是我们的造化,只恐吹不死他,他去请些
神兵来,却怎生是好?”老妖道:“怕他怎的,怕那甚么神兵!若还定得我的风势,
只除了灵吉菩萨来是,其余何足惧也!”

行者在屋梁上,只听得他这一句言语,不胜欢喜,即抽身飞出,现本相来至林
中,叫声“兄弟!”八戒道:“哥,你往那里去来?刚才一个打令字旗的妖精,被我
赶了去也。”行者笑道:“亏你!亏你!老孙变做蚊虫儿,进他洞去探看师父,原来师
父被他绑在定风桩上哭哩。是老孙吩咐,教他莫哭,又飞在屋梁上听了一听。只见
那拿令字旗的,喘嘘嘘的,走进去报道:只是被你赶他,却不见我。老妖乱猜乱说,
说老孙是风吹杀了,又说是请神兵去了。他却自家供出一个人来,甚妙!甚妙!”八
戒道:“他供的是谁?”行者道:“他说怕甚么神兵,那个能定他的风势,只除是灵
吉菩萨来是。——但不知灵吉住在何处?……”

正商议处,只见大路旁走出一个老公公来。你看他怎生模样:

身健不扶拐杖,冰髯雪鬓蓬蓬。金花耀眼意朦胧,瘦骨衰筋强硬。屈背低头缓
步,庞眉赤脸如童。看他容貌是人称,却似寿星出洞。
八戒望见大喜道:“师兄,常言道:‘要知山下路,须问去来人。’你上前问他一声,
何如?”真个大圣藏了铁棒,放下衣襟,上前叫道:“老公公,问讯了。”那老者半
答不答的,还了个礼道:“你是那里和尚?这旷野处,有何事干?”行者道:“我们
是取经的圣僧。昨日在此失了师父,特来动问公公一声:灵吉菩萨在那里住?”老
者道:“灵吉在直南上。到那里,还有二千里路。有一山,呼名小须弥山。山中有
个道场,乃是菩萨讲经禅院。汝等是取他的经去了?”行者道:“不是取他的经,
我有一事烦他,不知从那条路去。”老者用手向南指道:“这条羊肠路就是了。”哄
得那孙大圣回头看路,那公公化作清风,寂然不见。只是路旁边下一张简帖,上有
四句颂子云:
上复齐天大圣听:老人乃是李长庚。
须弥山有飞龙杖,灵吉当年受佛兵。
行者执了帖儿,转身下路。八戒道:“哥啊,我们连日造化低了。这两日忏日里见
鬼!那个化风去的老儿是谁?”行者把帖儿递与八戒。——念了一遍道:“李长庚是
那个?”行者道:“是西方太白金星的名号。”八戒慌得望空下拜道:“恩人,恩人!
老猪若不亏金星奏准玉帝呵,性命也不知化作甚的了!”行者道:“兄弟,你却也知
感恩。但莫要出头,只藏在这树林深处,仔细看守行李、马匹,等老孙寻须弥山,
请菩萨去耶。”八戒道:“晓得,晓得!你只管快快前去!老猪学得个乌龟法,得缩头
时且缩头。”

孙大圣跳在空中,纵斗云,径往直南上去,果然速快。他点头经过三千里,
扭腰八百有余程。须臾,见一座高山,半中间有祥云出现,瑞霭纷纷,山凹里果有
一座禅院,只听得钟磬悠扬,又见那香烟缥缈。

大圣直至门前,见一道人,项挂数珠,口中念佛。行者道:“道人作揖。”那道
人躬身答礼道:“那里来的老爷?”行者道:“这可是灵吉菩萨讲经处么?”道人道:
“此间正是,有何话说?”行者道:“累烦你老人家与我传答传答:我是东土大唐
驾下御弟三藏法师的徒弟,齐天大圣孙悟空行者。今有一事,要见菩萨。”道人笑
道:“老爷字多话多,我不能全记。”行者道:“你只说是唐僧徒弟孙悟空来了。”道
人依言,上讲堂传报。那菩萨即穿袈裟,添香迎接。

这大圣才举步入门,往里观看,只见那:

满堂锦绣,一屋威严。众门人齐诵《法华经》,老班首轻敲金铸磬。佛前供养,
尽是仙果仙花;案上安排,皆是素肴素品。辉煌宝烛,条条金焰射虹霓;馥郁真香,
道道玉烟飞彩雾。正是那讲罢心闲方入定,白云片片绕松梢。静收慧剑魔头绝,般
若波罗善会高。
那菩萨整衣出迓,行者登堂,坐了客位。随命看茶。行者道:“茶不劳赐,但我师
父在黄风山有难,特请菩萨施大法力降怪救师。”菩萨道:“我受了如来法令,在此
镇押黄风怪。如来赐了我一颗‘定风丹’,一柄‘飞龙宝杖’。当时被我拿住,饶了
他的性命,放他去隐性归山,不许伤生造孽,不知他今日欲害令师,有违教令,我
之罪也。”那菩萨欲留行者,治斋相叙,行者恳辞,随取了飞龙杖,与大圣一齐驾
云。不多时,至黄风山上。菩萨道:“大圣,这妖怪有些怕我,我只在云端里住定,
你下去与他索战,诱他出来,我好施法力。”

行者依言,按落云头,不容分说,掣铁棒把他洞门打破。叫道:“妖怪!还我师
父来也!”慌得那把门小妖,急忙传报。那怪道:“这泼猴着实无礼!再不伏善,反
打破我门!这一出去,使阵神风,定要吹死!”仍前披挂,手绰钢叉,又走出门来;
见了行者,更不打话,拈叉当胸就刺。大圣侧身躲过,举棒对面相还。战不数合,
那怪吊回头,望巽地上,才待要张口呼风,只见那半空里,灵吉菩萨将飞龙宝杖丢
将下来,不知念了些甚么咒语,却是一条八爪金龙,拨喇的轮开两爪,一把抓住妖
精,提着头,两三,在山石崖边,现了本相,却是一个黄毛貂鼠。行者赶上,
举棒就打,被菩萨拦住道:“大圣,莫伤他命。我还要带他去见如来。”对行者道:
“他本是灵山脚下的得道老鼠;因为偷了琉璃盏内的清油,灯火昏暗,恐怕金刚拿
他,故此走了,却在此处成精作怪。如来照见了他,不该死罪,故着我辖押,但他
伤生造孽,拿上灵山;今又冲撞大圣,陷害唐僧,我拿他去见如来,明正其罪,才
算这场功绩哩。”行者闻言,却谢了菩萨。菩萨西归不题。

却说猪八戒在那林内,正思量行者,只听得山坂下叫声:“悟能兄弟,牵马挑
担来耶。”那呆子认得是行者声音,急收拾跑出林外,见了行者道:“哥哥,怎的干
事来?”行者道:“请灵吉菩萨,使一条飞龙杖,拿住妖精,原来是个黄毛貂鼠成
精,被他带去灵山见如来去了。我和你洞里去救师父。”那呆子才欢欢喜喜。

二人撞入里面,把那一窝狡兔、妖狐、香獐、角鹿,一顿钉钯铁棒,尽情打死,
却往后园拜救师父。师父出得门来,问道:“你两人怎生捉得妖精?如何方救得我?”
行者将那请灵吉降妖的事情,陈了一遍。师父谢之不尽。他兄弟们把洞中素物,安
排些茶饭吃了,方才出门,找大路向西而去。

毕竟不知向后如何,且听下回分解。