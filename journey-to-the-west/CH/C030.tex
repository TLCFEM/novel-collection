\chapter{邪魔侵正法~意马忆心猿}

却说那怪把沙僧捆住,也不来杀他,也不曾打他,骂也不曾骂他一句。绰起钢
刀,心中暗想道:“唐僧乃上邦人物,必知礼义;终不然我饶了他性命,又着他徒
弟拿我不成?噫!这多是我浑家有甚么书信到他那国里,走了风汛!等我去问他一
问。”那怪陡起凶性,要杀公主。

却说那公主不知,梳妆方毕,移步前来。只见那怪怒目攒眉,咬牙切齿。那公
主还陪笑脸迎道:“郎君有何事这等烦恼?”那怪咄的一声骂道:“你这狗心贱妇,
全没人伦!我当初带你到此,更无半点儿说话。你穿的锦,戴的金,缺少东西我去
寻。四时受用,每日情深。你怎么只想你父母,更无一点夫妇心?”那公主闻说,
吓得跪倒在地。道:“郎君啊,你怎么今日说起这分离的话?”那怪道:“不知是我
分离,是你分离哩!我把那唐僧拿来,算计要他受用,你怎么不先告过我,就放了
他?原来是你暗地里修了书信,教他替你传寄;不然,怎么这两个和尚又来打上我
门,教还你回去?这不是你干的事?”公主道:“郎君,你差怪我了。我何尝有甚书
去?”老怪道:“你还强嘴哩!现拿住一个对头在此,却不是证见?”公主道:“是
谁?”老妖道:“是唐僧第二个徒弟沙和尚。”原来人到了死处,谁肯认死,只得与
他放赖。公主道:“郎君且息怒,我和你去问他一声。果然有书,就打死了,我也
甘心;假若无书,却不枉杀了奴奴也?”那怪闻言,不容分说,轮开一只簸箕大小
的蓝靛手,抓住那金枝玉叶的发万根,把公主揪上前,在地下,执着钢刀,却来
审沙僧;咄的一声道:“沙和尚!你两个辄敢擅打上我们门来,可是这女子有书到他
那国,国王教你们来的?”

沙僧已捆在那里,见妖精凶恶之甚,把公主掼倒在地,持刀要杀。他心中暗想
道:“分明是他有书去,——救了我师父。此是莫大之恩。我若一口说出,他就把
公主杀了,此却不是恩将仇报?罢,罢,罢!想老沙跟我师父一场,也没寸功报效;
今日已此被缚,就将此性命与师父报了恩罢。”遂喝道:“那妖怪不要无礼!他有甚
么书来,你这等枉他,要害他性命!我们来此问你要公主,有个缘故。只因你把我
师父捉在洞中,我师父曾看见公主的模样动静。及至宝象国,倒换关文,那皇帝将
公主画影图形,前后访问。因将公主的形影,问我师父沿途可曾看见,我师父遂将
公主说起,他故知是他儿女,赐了我等御酒,教我们来拿你,要他公主还宫。此情
是实,何尝有甚书信?你要杀就杀了我老沙,不可枉害平人,大亏天理!”

那妖见沙僧说得雄壮,遂丢了刀,双手抱起公主道:“是我一时粗卤,多有冲
撞,莫怪,莫怪。”遂与他挽了青丝,扶上宝髻。软款温柔,怡颜悦色,撮哄着他
进去了。又请上坐陪礼,那公主是妇人家水性,见他错敬,遂回心转意道:“郎君
啊,你若念夫妇的恩爱,可把那沙僧的绳子略放松些儿。”老妖闻言,即命小的们
把沙僧解了绳子,锁在那里。沙僧见解缚锁住,立起来,心中暗喜道:“古人云:‘与
人方便,自己方便。’我若不方便了他,他怎肯教把我松放松放。”

那老妖又教安排酒席,与公主陪礼压惊。吃酒到半酣,老妖忽的又换了一件鲜
明的衣服,取了一口宝刀,佩在腰里。转过手,摸着公主道:“浑家,你且在家吃
酒,看着两个孩儿,不要放了沙和尚。趁那唐僧在那国里,我也赶早儿去认认亲也。”
公主道:“你认甚亲?”老妖道:“认你父王。我是他驸马,他是我丈人,怎么不去
认认?”公主道:“你去不得。”老妖道:“怎么去不得?”公主道:“我父王不是马
挣力战的江山,他本是祖宗遗留的社稷。自幼儿是太子登基,城门也不曾远出,没
有见你这等凶汉。你这嘴脸相貌,生得丑陋,若见了他,恐怕吓了他,反为不美;
却不如不去认的还好。”老妖道:“既如此说,我变个俊的儿去便罢。”公主道:“你
试变来我看看。”

好怪物,他在那酒席间,摇身一变,就变做一个俊俏之人。真个生得:

形容典雅,体段峥嵘。言语多官样,行藏正妙龄。才如子建成诗易,貌似潘安
掷果轻。头上戴一顶鹊尾冠,乌云敛伏,身上穿一件玉罗褶,广袖飘迎。足下乌靴
花折,腰间鸾带光明。丰神真是奇男子,耸壑轩昂美俊英。
公主见了,十分欢喜。那妖笑道:“浑家,可是变得好么?”公主道:“变得好,变
得好!你这一进朝啊,我父王是亲不灭,一定着文武多官留你饮宴。倘吃酒中间,
千千仔细,万万个小心,却莫要现出原嘴脸来,露出马脚,走了风汛,就不斯文了。”
老妖道:“不消吩咐,自有道理。”

你看他纵云头,早到了宝象国。按落云光,行至朝门之外。对阁门大使道:“三
驸马特来见驾,乞为转奏转奏。”那黄门奏事官来至白玉阶前,奏道:“万岁,有三
驸马来见驾,现在朝门外听宣。”那国王正与唐僧叙话。忽听得三驸马,便问多官
道:“寡人只有两个驸马,怎么又有个三驸马?”多官道:“三驸马,必定是妖怪来
了。”国王道:“可好宣他进来?”那长老心惊道:“陛下,妖精啊,不精者不灵。
他能知过去未来,他能腾云驾雾,宣他也进来,不宣他也进来,倒不如宣他进来,
还省些口面。”

国王准奏,叫宣,把怪宣至金阶。他一般的也舞蹈山呼的行礼。多官见他生得
俊丽,也不敢认他是妖精。他都是些肉眼凡胎,却当做好人。那国王见他耸壑昂霄,
以为济世之梁栋。便问他:“驸马,你家在那里居住?是何方人氏?几时得我公主配
合?怎么今日才来认亲?”那老妖叩头道:“主公,臣是城东碗子山波月庄人家。”
国王道:“你那山离此处多远?”老妖道:“不远,只有三百里。”国王道:“三百里
路,我公主如何得到那里,与你匹配?”那妖精巧语花言,虚情假意的答道:“主
公,微臣自幼儿好习弓马,采猎为生。那十三年前,带领家童数十,放鹰逐犬,忽
见一只斑斓猛虎,身驮着一个女子,往山坡下走。是微臣兜弓一箭,射倒猛虎,将
女子带上本庄,把温水温汤灌醒,救了他性命。因问他是那里人家,他更不曾题‘公
主’二字。早说是万岁的三公主,怎敢欺心,擅自配合?当得进上金殿,大小讨一
个官职荣身。只因他说是民家之女,才被微臣留在庄所。女貌郎才,两相情愿,故
配合至此多年。当时配合之后,欲将那虎宰了,邀请诸亲,却是公主娘娘教且莫杀。
其不杀之故。有几句言词,道得甚好。说道:
托天托地成夫妇,无媒无证配婚姻。
前世赤绳曾系足,今将老虎做媒人。
臣因此言,故将虎解了索子,饶了他性命。那虎带着箭伤,跑蹄剪尾而去。不知他
得了性命,在那山中,修了这几年,炼体成精,专一迷人害人。臣闻得昔年也有几
次取经的,都说是大唐来的唐僧;想是这虎害了唐僧,得了他文引,变作那取经的
模样,今在朝中哄骗主公。主公啊,那绣墩上坐的,正是那十三年前驮公主的猛虎,
不是真正取经之人!”

你看那水性的君王,愚迷肉眼,不识妖精,转把他一片虚词,当了真实。道:
“贤驸马,你怎的认得这和尚是驮公主的老虎?”那妖道:“主公,臣在山中,吃
的是老虎,穿的也是老虎,与他同眠同起,怎么不认得?”国王道:“你既认得,
可教他现出本相来看。”怪物道:“借半盏净水,臣就教他现了本相。”国王命官取
水,递与驸马。那怪接水在手,纵起身来,走上前,使个“黑眼定身法”。念了咒
语,将一口水望唐僧喷去,叫声“变!”那长老的真身,隐在殿上,真个变作一只
斑斓猛虎。此时君臣同眼观看,那只虎生得:

白额圆头,花身电目。四只蹄,挺直峥嵘;二十爪,钩弯锋利。锯牙包口,尖
耳连眉。狞狰壮若大猫形,猛烈雄如黄犊样。刚须直直插银条,刺舌喷恶气。
果然是只猛斑斓,阵阵威风吹宝殿。
国王一见,魄散魂飞。唬得那多官尽皆躲避。有几个大胆的武将,领着将军、校尉
一拥上前,使各项兵器乱砍。这一番,不是唐僧该有命不死,就是二十个僧人,也
打为肉酱。此时幸有丁甲、揭谛、功曹、护教诸神,暗在半空中护佑,所以那些人,
兵器皆不能打伤。众臣嚷到天晚,才把那虎活活的捉了。用铁绳锁了,放在铁笼里;
收于朝房之内。

那国王却传旨,教光禄寺大排筵宴,谢驸马救拔之恩。不然,险被那和尚害了。
当晚众臣朝散,那妖魔进了银安殿。又选十八个宫娥彩女,吹弹歌舞,劝妖魔饮酒
作乐。那怪物独坐上席,左右排列的,都是那艳质娇姿。你看他受用。饮酒至二更
时分,醉将上来,忍不住胡为。跳起身,大笑一声,现了本相。陡发凶心,伸开簸
箕大手,把一个弹琵琶的女子,抓将过来,咋的把头咬了一口。吓得那十七个宫
娥,没命的前后乱跑乱藏。你看那:

宫娥悚惧,彩女忙惊。宫娥悚惧,一似雨打芙蓉笼夜雨;彩女忙惊,就如风吹
芍药舞春风。碎琵琶顾命,跌伤琴瑟逃生。出门那分南北,离殿不管西东。磕损
玉面,撞破娇容。人人逃命走,各各奔残生。
那些人出去,又不敢吆喝。夜深了,又不敢惊驾。都躲在那短墙檐下,战战兢兢不
题。

却说那怪物坐在上面,自斟自酌。喝一盏,扳过人来,血淋淋的啃上两口。他
在里面受用,外面人尽传道:“唐僧是个虎精!”乱传乱嚷,嚷到金亭馆驿。此时驿
里无人,止有白马在槽上吃草吃料。他本是西海小龙王,因犯天条,锯角退鳞,变
白马,驮唐僧往西方取经。忽闻人讲唐僧是个虎精,他也心中暗想道:“我师父分
明是个好人,必然被怪把他变做虎精,害了师父。怎的好!怎的好!大师兄去得久了;
八戒、沙僧,又无音信!”他只捱到二更时分,万籁无声,却才跳将起来道:“我今
若不救唐僧,这功果休矣!休矣!”他忍不住,顿绝缰绳,抖松鞍辔,急纵身,忙显
化,依然化作龙。驾起乌云,直上九霄空里观看。有诗为证。诗曰:
三藏西来拜世尊,途中偏有恶妖氛。
今霄化虎灾难脱,白马垂缰救主人。

小龙王在半空里,只见银安殿内,灯烛辉煌。原来那八个满堂红上,点着八根
蜡烛。低下云头,仔细看处,那妖魔独自个在上面,逼法的饮酒吃人肉哩。小龙笑
道:“这厮不济!走了马脚,识破风汛,匾秤铊了。吃人可是个长进的!却不知我
师父下落何如,倒遇着这个泼怪。且等我去戏他一戏。若得手,拿住妖精再救师父
不迟。”

好龙王,他就摇身一变,也变做个宫娥。真个身体轻盈,仪容娇媚。忙移步走
入里面,对妖魔道声万福:“驸马啊,你莫伤我性命,我来替你把盏。”那妖道:“斟
酒来。”小龙接过壶来,将酒斟在他盏中,酒比锺高出三五分来,更不漫出。这是
小龙使的“逼水法”。那怪见了不识,心中喜道:“你有这般手段?”小龙道:“还
斟得有几分高哩。”那怪道:“再斟上!再斟上!”他举着壶,只情斟,那酒只情高,
就如十三层宝塔一般,尖尖满满,更不漫出些须。那怪物伸过嘴来,吃了一锺;扳
着死人,吃了一口。道:“会唱么?”小龙道:“也略晓得些儿。”依腔韵唱了一个
小曲,又奉了一锺。那怪道:“你会舞么?”小龙道:“也略晓得些儿;但只是素手,
舞得不好看。”那怪揭起衣服,解下腰间所佩宝剑,掣出鞘来,递与小龙。小龙接
了刀,就留心,在那酒席前,上三下四,左五右六,丢开了花刀法。

那怪看得眼咤,小龙丢了花字,望妖精劈一刀来。好怪物,侧身躲过,慌了手
脚,举起一根满堂红,架住宝刀。那满堂红原是熟铁打造的,连柄有八九十斤。两
个出了银安殿,小龙现了本相,却驾起云头,与那妖魔在那半空中相杀。这一场,
黑地里好杀!怎见得:

那一个是碗子山生成的怪物,这一个是西洋海罚下的真龙。一个放毫光,如喷
白电;一个生锐气,如迸红云。一个好似白牙老象走人间,一个就如金爪狸猫飞下
界。一个是擎天玉柱,一个是架海金梁。银龙飞舞,黄鬼翻腾。左右宝刀无怠慢,
往来不歇满堂红。
他两个在云端里,战够八九回合,小龙的手软筋麻,老魔的身强力壮。小龙抵敌不
住,飞起刀去,砍那妖怪,妖怪有接刀之法,一只手接了宝刀,一只手抛下满堂红
便打,小龙措手不及,被他把后腿上着了一下。急慌慌按落云头,多亏了御水河救
了性命。小龙一头钻下水去。那妖魔赶来寻他不见,执了宝刀,拿了满堂红,回上
银安殿,照旧吃酒睡觉不题。

却说那小龙潜于水底,半个时辰听不见声息,方才咬着牙,忍着腿疼跳将起去,
踏着乌云,径转馆驿。还变作依旧马匹,伏于槽下。可怜浑身是水,腿有伤痕。那
时节:
意马心猿都失散,金公木母尽雕零。
黄婆伤损通分别,道义消疏怎得成!

且不言三藏逢灾,小龙败战。却说那猪八戒,从离了沙僧,一头藏在草科里,
拱了一个猪浑塘。这一觉,直睡到半夜时候才醒。醒来时,又不知是甚么去处,摸
摸眼,定了神思,侧耳才听,噫!正是那山深无犬吠,野旷少鸡鸣。他见那星移斗
转,约莫有三更时分,心中想道:“我要回救沙僧,诚然是“单丝不线,孤掌难鸣。”……
罢,罢,罢!我且进城去见了师父,奏准当今,再选些骁勇人马,助着老猪明日来
救沙僧罢。”

那呆子急纵云头,径回城里。半霎时,到了馆驿。此时人静月明。两廊下寻不
见师父。只见白马睡在那厢,浑身水湿,后腿有盘子大小一点青痕。八戒失惊道:
“双晦气了!这亡人又不曾走路,怎么身上有汗,腿有青痕?想是歹人打劫师父,把
马打坏了。”

那白马认得是八戒,忽然口吐人言,叫声“师兄!”这呆子吓了一跌。扒起来,
往外要走,被那马探探身,一口咬住皂衣,道:“哥啊,你莫怕我。”八戒战兢兢的
道:“兄弟,你怎么今日说起话来了?你但说话,必有大不祥之事。”小龙道:“你知
师父有难么?”八戒道:“我不知。”小龙道:“你是不知!你与沙僧在皇帝面前弄了
本事,思量拿倒妖魔,请功求赏,不想妖魔本领大,你们手段不济,禁他不过。好
道着一个回来,说个信息是,却更不闻音。那妖精变做一个俊俏文人,撞入朝中,
与皇帝认了亲眷。把我师父变作一个斑斓猛虎,见被众臣捉住,锁在朝房铁笼里面。
我听得这般苦恼,心如刀割。你两日又不在不知,恐一时伤了性命。只得化龙身去
救,不期到朝里,又寻不见师父。及到银安殿外,遇见妖精,我又变做个宫娥模样,
哄那怪物。那怪叫我舞刀他看,遂尔留心,砍他一刀,早被他闪过,双手举个满堂
红,把我战败。我又飞刀砍去,他又把刀接了,下满堂红,把我后腿上着了一下;
故此钻在御水河,逃得性命。腿上青是他满堂红打的。”

八戒闻言道:“真个有这样事?”小龙道:“莫成我哄你了!”八戒道:“怎的好,
怎的好!你可挣得动么?”小龙道:“我挣得动便怎的?”八戒道:“你挣得动,便
挣下海去罢。把行李等老猪挑去高老庄上,回炉做女婿去呀。”小龙闻说,一口咬
住他直裰子,那里肯放。止不住眼中滴泪道:“师兄啊!你千万休生懒惰!”八戒道:
“不懒惰便怎么?沙兄弟已被他拿住,我是战不过他,不趁此散火,还等甚么?”

小龙沉吟半晌,又滴泪道:“师兄啊,莫说散火的话。若要救得师父,你只去
请个人来。”八戒道:“教我请谁么?”小龙道:“你趁早儿驾云回上花果山,请大
师兄孙行者来。他还有降妖的大法力,管情救了师父,也与你我报得这败阵之仇。”
八戒道:“兄弟,另请一个儿便罢了。那猴子与我有些不睦。前者在白虎岭上,打
杀了那白骨夫人,他怪我撺掇师父念紧箍儿咒。我也只当耍子,不想那老和尚当真
的念起来,就把他赶逐回去。他不知怎么样的恼我。他也决不肯来。倘或言语上略
不相对,他那哭丧棒又重,假若不知高低,捞上几下,我怎的活得成么?”小龙道:
“他决不打你。他是个有仁有义的猴王。你见了他,且莫说师父有难,只说:‘师
父想你哩。’把他哄将来,到此处,见这样个情节,他必然不忿,断乎要与那妖精
比并,管情拿得那妖精,救得我师父。”八戒道:“也罢,也罢。你倒这等尽心,我
若不去,显得我不尽心了。我这一去,果然行者肯来,我就与他一路来了;他若不
来,你却也不要望我,我也不来了。”小龙道:“你去,你去;管情他来也。”

真个呆子收拾了钉钯,整束了直裰,跳将起去,踏着云,径往东来。这一回,
也是唐僧有命。那呆子正遇顺风,撑起两个耳朵,好便似风篷一般,早过了东洋大
海,按落云头。不觉的太阳星上,他却入山寻路。

正行之际,忽闻得有人言语。八戒仔细看时,原来是行者在山凹里,聚集群妖。
他坐在一块石头崖上,面前有一千二百多猴子,分序排班,口称“万岁!大圣爷爷!”
八戒道:“且是好受用!且是好受用!怪道他不肯做和尚,只要来家哩!原来有这些好
处,许大的家业,又有这多的小猴伏侍!若是老猪有这一座山场,也不做甚么和尚
了。如今既到这里,却怎么好?必定要见他一见是。”那呆子有些怕他,又不敢明明
的见他;却往草崖边,溜阿溜的,溜在那一千二三百猴子当中挤着,也跟那些猴子
磕头。

不知孙大圣坐得高,眼又乖滑,看得他明白。便问:“那班部中乱拜的是个夷
人。是那里来的?拿上来!”说不了,那些小猴,一窝峰,把个八戒推将上来,按倒
在地。行者道:“你是那里来的夷人?”八戒低着头道:“不敢,承问了;不是夷人,
是熟人,熟人。”行者道:“我这大圣部下的群猴,都是一般模样。你这嘴脸生得各
样,相貌有些雷堆,定是别处来的妖魔。既是别处来的,若要投我部下,先来递个
脚色手本,报了名字,我好留你在这随班点扎。若不留你,你敢在这里乱拜!”八
戒低着头,拱着嘴道:“不羞!就拿出这副嘴脸来了!我和你兄弟也做了几年,又推
认不得,说是甚么夷人!”行者笑道:“抬起头来我看。”那呆子把嘴往上一伸道:“你
看么!你认不得我,好道认得嘴耶!”行者忍不住笑道:“猪八戒。”他听见一声叫,
就一毂辘跳将起来道:“正是,正是!我是猪八戒!”他又思量道:“认得就好说话
了。”

行者道:“你不跟唐僧取经去,却来这里怎的?想是你冲撞了师父,师父也贬你
回来了?有甚贬书,拿来我看。”八戒道:“不曾冲撞他。他也没甚么贬书,也不曾
赶我。”行者道:“既无贬书,又不曾赶你,你来我这里怎的?”八戒道:“师父想
你,着我来请你的。”行者道:“他也不请我,他也不想我。他那日对天发誓,亲笔
写了贬书,怎么又肯想我,又肯着你远来请我?我断然也是不好去的。”八戒就地扯
个谎,忙道:“委是想你,委是想你!”行者道:“他怎的想我来?”八戒道:“师父
在马上正行,叫声‘徒弟’,我不曾听见,沙僧又推耳聋;师父就想起你来,说我
们不济,说你还是个聪明伶俐之人,常时声叫声应,问一答十。因这般想你,专专
教我来请你的。万望你去走走,一则不孤他仰望之心,二来也不负我远来之意。”

行者闻言,跳下崖来,用手搀住八戒道:“贤弟,累你远来,且和我耍耍儿去。”
八戒道:“哥啊,这个所在路远,恐师父盼望去迟,我不耍子了。”行者道:“你也
是到此一场,看看我的山景何如。”那呆子不敢苦辞,只得随他走走。二人携手相
搀,概众小妖随后,上那花果山极巅之处。好山!自是那大圣回家,这几日,收拾
得复旧如新。但见那:

青如削翠,高似摩云。周围有虎踞龙蟠,四面多猿啼鹤唳。朝出云封山顶,暮
观日挂林间。流水潺潺鸣玉,涧泉滴滴奏瑶琴。山前有崖峰峭壁,山后有花木
华。上连玉女洗头盆,下接天河分派水。乾坤结秀赛蓬莱,清浊育成真洞府。丹青
妙笔画时难,仙子天机描不就。玲珑怪石石玲珑,玲珑结彩岭头峰。日影动千条紫
艳,瑞气摇万道红霞。洞天福地人间有,遍山新树与新花。
八戒观之不尽,满心欢喜道:“哥啊,好去处!果然是天下第一名山!”行者道:“贤
弟,可过得日子么?”八戒笑道:“你看师兄说的话。宝山乃洞天福地之处,怎么
说度日之言也?”二人谈笑多时,下了山。只见路旁有几个小猴,捧着紫巍巍的葡
萄,香喷喷的梨枣,黄森森的枇杷,红艳艳的杨梅,跪在路旁,叫道:“大圣爷爷,
请进早膳。”行者笑道:“我猪弟食肠大,却不是以果子作膳的。也罢,也罢,莫嫌
菲薄,将就吃个儿当点心罢。”八戒道:“我虽食肠大,却也随乡入乡是。拿来,拿
来,我也吃几个儿尝新。”

二人吃了果子,渐渐日高。那呆子恐怕误了救唐僧,只管催促道:“哥哥,师
父在那里盼望我和你哩。望你和我早早儿去罢。”行者道:“贤弟,请你往水帘洞里
去耍耍。”八戒坚辞道:“多感老兄盛意。奈何师父久等,不劳进洞罢。”行者道:“既
如此,不敢久留,请就此处奉别。”八戒道:“哥哥,你不去了?”行者道:“我往
那里去?我这里,天不收,地不管,自由自在,不耍子儿,做甚么和尚?我是不去,
你自去罢。但上复唐僧:既赶退了,再莫想我。”呆子闻言,不敢苦逼,只恐逼发
他性子,一时打上两棍。无奈,只得喏喏告辞,找路而去。

行者见他去了,即差两个溜撒的小猴,跟着八戒,听他说些甚么。真个那呆子
下了山,不上三四里路,回头指着行者,口里骂道:“这个猴子,不做和尚,倒做
妖怪!这个猢狲,我好意来请他,他却不去!你不去便罢!”走几步,又骂几声。那
两个小猴,急跑回来报道:“大圣爷爷,那猪八戒不大老实,他走走儿,骂几声。”
行者大怒。叫:“拿将来!”那众猴满地飞来赶上,把个八戒,扛翻倒了,抓鬃扯耳,
拉尾揪毛,捉将回去。

毕竟不知怎么处治,性命死活若何,且听下回分解。