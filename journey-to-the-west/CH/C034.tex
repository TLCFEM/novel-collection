\chapter{魔王巧算困心猿~大圣腾那骗宝贝}

却说那两个小妖,将假葫芦拿在手中,争看一会,忽抬头不见了行者。伶俐虫
道:“哥啊,神仙也会打诳语。他说换了宝贝,度我等成仙,怎么不辞就去了?”
精细鬼道:“我们相应便宜的多哩,他敢去得成?拿过葫芦来,等我装装天,也试演
试演看。”真个把葫芦往上一抛,扑的就落将下来。慌得个伶俐虫道:“怎么不装!
不装!莫是孙行者假变神仙,将假葫芦换了我们的真的去耶?”精细鬼道:“不要胡
说!孙行者是那三座山压住了,怎生得出?拿过来,等我念他那几句咒儿装了看。”
这怪也把葫芦儿望空丢起,口中念道:“若有半声不肯,就上灵霄殿上,动起刀兵!”
念不了,扑的又落将下来。两妖道:“不装,不装,一定是个假的!”

正嚷处,孙大圣在半空里听得明白,看得真实,恐怕他弄得时辰多了,紧要处
走了风汛,将身一抖,把那变葫芦的毫毛,收上身来,弄得那两妖四手皆空。精细
鬼道:“兄弟,拿葫芦来。”伶俐虫道:“你拿着的。——天呀!怎么不见了?”都去
地下乱摸,草里胡寻,吞袖子,揣腰间,那里得有?二妖吓得呆呆挣挣道:“怎的好,
怎的好!当时大王将宝贝付与我们,教拿孙行者;今行者既不曾拿得,连宝贝都不
见了。我们怎敢去回话?这一顿直直的打死了也!怎的好!怎的好!”伶俐虫道:“我
们走了罢。”精细鬼道:“往那里走么?”伶俐虫道:“不管那里走罢。若回去说没
宝贝,断然是送命了。”精细鬼道:“不要走,还回去。二大王平日看你甚好,我推
一句儿在你身上。你若肯将就,留得性命;说不过,就打死,还在此间。莫弄得两
头不着。去来,去来!”那怪商议了,转步回山。

行者在半空中见他回去,又摇身一变,变作苍蝇儿。飞下去,跟着小妖。你道
他既变了苍蝇,那宝贝却放在何处?如丢在路上,藏在草里,被人看见拿去,却不
是劳而无功?他还带在身上。带在身上啊,苍蝇不过豆粒大小,如何容得?原来他那
宝贝,与他金箍棒相同;叫做如意佛宝,随身变化,可以大,可以小,故身上亦可
容得。他嘤的一声飞下去,跟定那怪。

不一时,到了洞里。只见那两个魔头,坐在那里饮酒。小妖朝上跪下。行者就
钉在那门柜上,侧耳听着。小妖道:“大王。”二老魔即停杯道:“你们来了?”小
妖道:“来了。”又问:“拿着孙行者否?”小妖叩头,不敢声言。老魔又问,又不
敢应,只是叩头。问之再三,小妖俯伏在地:“赦小的万千死罪!赦小的万千死罪!
我等执着宝贝,走到半山之中,忽遇着蓬莱山一个神仙。他问我们那里去,我们答
道,拿孙行者去。那神仙听见说孙行者,他也恼他,要与我们帮功。是我们不曾叫
他帮功,却将拿宝贝装人的情由,与他说了。那神仙也有个葫芦,善能装天。我们
也是妄想之心,养家之意:他的装天,我的装人,与他换了罢。原说葫芦换葫芦,
伶俐虫又贴他个净瓶。谁想他仙家之物,近不得凡人之手。正试演处,就连人都不
见了。万望饶小的们死罪!”老魔听说,暴躁如雷道:“罢了,罢了!这就是孙行者
假妆神仙骗哄去了!那猴头神通广大,处处人熟,不知那个毛神,放他出来,骗去
宝贝!”

二魔道:“兄长息怒。叵耐那猴头着然无礼。既有手段,便走了也罢,怎么又
骗宝贝?我若没本事拿他,永不在西方路上为怪!”老魔道:“怎生拿他?”二魔道:
“我们有五件宝贝,去了两件,还有三件,务要拿住他。”老魔道:“还有那三件?”
二魔道:“还有‘七星剑’与‘芭蕉扇’在我身边;那一条‘幌金绳’,在压龙山压
龙洞老母亲那里收着哩。如今差两个小妖去请母亲来吃唐僧肉,就教他带幌金绳来
拿孙行者。”老魔道:“差那个去?”二魔道:“不差这样废物去!”将精细鬼、伶俐
虫一声喝起。二人道:“造化!造化!打也不曾打,骂也不曾骂,却就饶了。”

二魔道:“叫那常随的伴当巴山虎、倚海龙来。”二人跪下。二魔吩咐道:“你
却要小心。”俱应道:“小心。”“却要仔细。”俱应道:“仔细。”又问道:“你认得老
奶奶家么?”又俱应道:“认得。”“你既认得,你快早走动,到老奶奶处,多多拜
上,说请吃唐僧肉哩;就着带幌金绳来,要拿孙行者。”

二怪领命疾走,怎知那行者在旁,一一听得明白。他展开翅,飞将去,赶上巴
山虎,钉在他身上。行经二三里,就要打杀他两个;又思道:“打死他,有何难事?
但他奶奶身边有那幌金绳,又不知住在何处。等我且问他一问再打。”好行者,嘤
的一声,躲离小妖,让他先行有百十步,却又摇身一变,也变做个小妖儿,戴一顶
狐皮帽子,将虎皮裙子倒插上来勒住,赶上道:“走路的,等我一等。”那倚海龙回
头问道:“是那里来的?”行者道:“好哥啊,连自家人也认不得?”小妖道:“我
家没有你。”行者道:“怎么没我?你再认认看。”小妖道:“面生,面生,不曾相会。”
行者道:“正是。你们不曾会着我,我是外班的。”小妖道:“外班长官,是不曾会。
你往那里去?”行者道:“大王说差你二位请老奶奶来吃唐僧肉,教他就带幌金绳
来,拿孙行者。恐你二位走得缓,有些贪顽,误了正事,又差我来催你们快去。”

小妖见说着海底眼,更不疑惑,把行者果认做一家人。急急忙忙,往前飞跑。
一气又跑有八九里。行者道:“忒走快了些。我们离家有多少路了?”小怪道:“有
十五六里了。”行者道:“还有多远?”倚海龙用手一指道:“乌林子里就是。”行者
抬头见一带黑林不远,料得那老怪只在林子里外。却立定步,让那小怪前走,即取
出铁棒,走上前,着脚后一刮;可怜忒不禁打,就把两个小妖刮做一团肉饼。却拖
着脚,藏在路旁深草科里。即便拔下一根毫毛,吹口仙气,叫“变!”变做个巴山
虎,自身却变做个倚海龙。假妆做两个小妖,径往那压龙洞请老奶奶。这叫做七十
二变神通大,指物腾那手段高。

三五步,跳到林子里,正找寻处,只见有两扇石门,半开半掩,不敢擅入。只
得洋叫一声:“开门,开门!”早惊动那把门的一个女怪,将那半扇儿开了,道:“你
是那里来的?”行者道:“我是平顶山莲花洞里差来请老奶奶的。”那女怪道:“进
去。”到了二层门下,闪着头,往里观看,又见那正当中高坐着一个老妈妈儿。你
道他怎生模样?但见:

雪鬓蓬松,星光晃亮。脸皮红润皱文多,牙齿稀疏神气壮。貌似菊残霜里色,
形如松老雨余颜。头缠白练攒丝帕,耳坠黄金嵌宝环。
孙大圣见了,不敢进去,只在二门外仵着脸,脱脱的哭起来,你道他哭怎的,莫成
是怕他?就怕也便不哭。况先哄了他的宝贝,又打杀他的小妖,却为何而哭?他当时
曾下九鼎油锅,就了七八日也不曾有一点泪儿。只为想起唐僧取经的苦恼,他就
泪出痛肠,放眼便哭;心却想道:“老孙既显手段,变做小妖,来请这老怪,没有
个直直的站了说话之理,一定见他磕头才是。我为人做了一场好汉,止拜了三个人:
西天拜佛祖;南海拜观音;两界山师父救了我,我拜了他四拜。为他使碎六叶连肝
肺,用尽三毛七孔心。一卷经能值几何?今日却教我去拜此怪。若不跪拜,必定走
了风汛。苦啊!算来只为师父受困,故使我受辱于人!”到此际也没及奈何,撞将进
去,朝上跪下道:“奶奶磕头。”

那怪道:“我儿,起来。”行者暗道:“好,好,好!叫得结实!”老怪问道:“你
是那里来的?”行者道:“平顶山莲花洞,蒙二位大王有令,差来请奶奶去吃唐僧
肉;教带幌金绳,要拿孙行者哩。”老怪大喜道:“好孝顺的儿子。”就去叫抬出轿
来。行者道:“我的儿啊!妖精也抬轿!”后壁厢即有两个女怪,抬出一顶香藤轿,
放在门外,挂上青绢纬幔。老怪起身出洞,坐在轿里。后有几个小女怪,捧着减妆,
端着镜架,提着手巾,托着香盒,跟随左右。那老怪道:“你们来怎的?我往自家儿
子去处,愁那里没人伏侍,要你们去献勤塌嘴?都回去!关了门看家!”那几个小妖
果俱回去,止有两个抬轿的。老怪问道:“那差来的叫做甚么名字?”行者连忙答
应道:“他叫做巴山虎,我叫做倚海龙。”老怪道:“你两个前走,与我开路。”行者
暗想道:“可是晦气!经倒不曾取得,且来替他做皂隶。”却又不敢抵强,只得向前
引路,大四声喝起。

行了五六里远近,他就坐在石崖上。等候那抬轿的到了,行者道:“略歇歇如
何?压得肩头疼啊。”小怪那知甚么诀窍,就把轿子歇下。行者在轿后,胸脯上拔下
一根毫毛,变做一个大烧饼,抱着啃。轿夫道:“长官,你吃的是甚么?”行者道:
“不好说。这远的路,来请奶奶,没些儿赏赐,肚里饥了,原带来的干粮,等我吃
些儿再走。”轿夫道:“把些儿我们吃吃。”行者笑道:“来么,都是一家人,怎么计
较?”那小妖不知好歹,围着行者,分其干粮,被行者掣出棒,着头一磨,一个汤
着的,打得稀烂;一个擦着的,不死还哼。那老怪听得人哼,轿子里伸出头来看时,
被行者跳到轿前,劈头一棍,打了个窟窿,脑浆迸流,鲜血直冒。拖出轿来看处,
原是个九尾狐狸。行者笑道:“造孽畜,叫甚么老奶奶!你叫老奶奶,就该称老孙做
上太祖公公是!”好猴王,把他那幌金绳搜出来,笼在袖里,欢喜道:“那泼魔纵有
手段,已此三件儿宝贝姓孙了!”却又拔两根毫毛变做个巴山虎、倚海龙;又拔两
根变做两个抬轿的;他却变做老奶奶模样,坐在轿里。将轿子抬起,径回本路。

不多时,到了莲花洞口,那毫毛变的小妖,俱在前道:“开门!开门!”内有把
门的小妖,开了门道:“巴山虎、倚海龙来了?”毫毛道:“来了。”“你们请的奶奶
呢?”毫毛用手指道:“那轿内的不是?”小怪道:“你且住,等我进去先报。”报
道:“大王,奶奶来耶。”两个魔头闻说,即命排香案来接。行者听得,暗喜道:“造
化,也轮到我为人了!我先变小妖,去请老怪,磕了他一个头;这番来,我变老怪,
是他母亲,定行四拜之礼。虽不怎的,好道也赚他两个头儿!”

好大圣,下了轿子,抖抖衣服,把那四根毫毛收在身上。那把门的小妖,把空
轿抬入门里。他却随后徐行。那般娇娇啻啻,扭扭捏捏,就像那老怪的行动,径自
进去。又只见大小群妖,都来跪接。鼓乐箫韶,一派响;博山炉里,霭霭香烟。
他到正厅中,南面坐下。两个魔头,双膝跪倒,朝上叩头,叫道:“母亲,孩儿拜
揖。”行者道:“我儿起来。”

却说猪八戒吊在梁上,哈哈的笑了一声。沙僧道:“二哥,好啊!吊出笑来也!”
八戒道:“兄弟,我笑中有故。”沙僧道:“甚故?”八戒道:“我们只怕是奶奶来了,
就要蒸吃;原来不是奶奶,是旧话来了。”沙僧道:“甚么旧话?”八戒笑道:“弼
马温来了。”沙僧道:“你怎么认得是他?”八戒道:“弯倒腰,叫‘我儿起来’,那
后面就掬起猴尾巴子。我比你吊得高,所以看得明也。”沙僧道:“且不要言语,听
他说甚么话。”八戒道:“正是,正是。”

那孙大圣坐在中间,问道:“我儿,请我来有何事干?”魔头道:“母亲啊,连
日儿等少礼,不曾孝顺得。今早愚兄弟拿得东土唐僧,不敢擅吃,请母亲来献献生,
好蒸与母亲吃了延寿。”行者道:“我儿,唐僧的肉,我倒不吃;听见有个猪八戒的
耳朵甚好,可割将下来整治整治我下酒。”那八戒听见慌了道:“遭瘟的,你来为割
我耳朵的,我喊出来不好听啊!”

噫!只为呆子一句通情话,走了猴王变化的风。那里有几个巡山的小怪,把门
的众妖,都撞将进来,报道:“大王,祸事了!孙行者打杀奶奶,他妆来耶!”魔头
闻此言,那容分说,掣七星宝剑,望行者劈脸砍来。好大圣,将身一幌,只见满洞
红光,预先走了。似这般手段,着实好耍子。正是那聚则成形,散则成气。唬得个
老魔头魂飞魄散,众群精噬指摇头。

老魔道:“兄弟,把唐僧与沙僧、八戒、白马、行李都送还那孙行者,闭了是
非之门罢。”二魔道:“哥哥,你说那里话?我不知费了多少辛勤,施这计策,将那
和尚都摄将来;如今似你这等怕惧孙行者的诡谲,就俱送去还他,真所谓畏刀避剑
之人,岂大丈夫之所为也?你且请坐勿惧。我闻你说孙行者神通广大,我虽与他相
会一场,却不曾与他比试。取披挂来,等我寻他交战三合。假若他三合胜我不过,
唐僧还是我们之食;如三战我不能胜他,那时再送唐僧与他未迟。”老魔道:“贤弟
说得是。”教:“取披挂。”

众妖抬出披挂,二魔结束齐整。执宝剑,出门外,叫声“孙行者!你往那里走
了?”此时大圣已在云端里,闻得叫他名字,急回头观看。原来是那二魔。你看他
怎生打扮:
头戴凤盔欺腊雪,身披战甲幌镔铁。
腰间带是蟒龙筋,粉皮靴梅花折。
颜如灌口活真君,貌比巨灵无二别。
七星宝剑手中擎,怒气冲霄威烈烈。

二魔高叫道:“孙行者!快还我宝贝与我母亲来,我饶你唐僧取经去!”大圣忍
不住骂道:“这泼怪物,错认了你孙外公!赶早儿送还我师父、师弟、白马、行囊,
仍打发我些盘缠,往西走路。若牙缝里道半个‘不’字,就自家搓根绳儿去罢,也
免得你外公动手。”二魔闻言,急纵云,跳在空中,轮宝剑来刺。行者掣铁棒劈手
相迎。他两个在半空中,这场好杀:

棋逢对手,将遇良才:棋逢对手难藏兴,将遇良才可用功。那两员神将相交,
好便似南山虎斗,北海龙争。龙争处,鳞甲生辉;虎斗时,爪牙乱落。爪牙乱落撒
银钩,鳞甲生辉支铁叶。这一个翻翻复复,有千般解数;那一个来来往往,无半
点放闲。金箍棒,离顶门只隔三分;七星剑,向心窝惟争一。那个威风逼得斗牛
寒,这个怒气胜如雷电险。
他两个战了有三十回合,不分胜负。

行者暗喜道:“这泼怪倒也架得住老孙的铁棒!我已得了他三件宝贝,却这般苦
苦的与他厮杀,可不误了我的工夫?不若拿葫芦或净瓶装他去,多少是好。”又想道:
“不好,不好,常言道:‘物随主便。’倘若我叫他不答应,却又不误了事业?且使
幌金绳扣头罢。”好大圣,一只手使棒,架住他的宝剑;一只手把那绳抛起,刷喇
的扣了魔头。原来那魔头有个紧绳咒,有个松绳咒。若扣住别人,就念紧绳咒,莫
能得脱;若扣住自家人,就念松绳咒,不得伤身。他认得是自家的宝贝,即念松绳
咒,把绳松动,便脱出来。反望行者抛将去,却早扣住了大圣。大圣正要使“瘦身
法”,想要脱身,却被那魔念动紧绳咒,紧紧扣住,怎能得脱?褪至颈项之下,原是
一个金圈子套住。那怪将绳一扯,扯将下来,照光头上砍了七八宝剑,行者头皮儿
也不曾红了一红。那魔道:“这猴子,你这等头硬,我不砍你,且带你回去,再打
你。将我那两件宝贝趁早还我!”行者道:“我拿你甚么宝贝,你问我要?”那魔头
将身上细细搜检,却将那葫芦、净瓶都搜出来;又把绳子牵着,带至洞里道:“兄
长,拿将来了。”老魔道:“拿了谁来?”二魔道:“孙行者。你来看,你来看。”老
魔一见,认得是行者,满面欢喜道:“是他!是他!把他长长的绳儿拴在柱上耍子!”
真个把行者拴住,两个魔头,却进后面堂里饮酒。

那大圣在柱根下爬蹉,忽惊动八戒。那呆子吊在梁上,哈哈的笑道:“哥哥啊,
耳朵吃不成了!”行者道:“呆子!可吊得自在么?我如今就出去,管情救了你们。”
八戒道:“不羞!不羞!本身难脱,还想救人,罢,罢,罢!师徒们都在一处死了,好
到阴司里问路!”行者道:“不要胡说!你看我出去。”八戒道:“我看你怎么出去。”

那大圣口里与八戒说话,眼里却抹着那些妖怪。见他在里边吃酒,有几个小妖
拿盘拿盏,执壶酾酒,不住的两头乱跑,关防的略松了些儿。他见面前无人,就弄
神通:顺出棒来,吹口仙气,叫“变!”即变做一个纯钢的锉儿;扳过那颈项的圈
子,三五锉,锉做两段;扳开锉口,脱将出来,拔了一根毫毛,叫变做一个假身,
拴在那里,真身却幌一幌,变做个小妖,立在旁边。八戒又在梁上喊道:“不好了!
不好了!拴的是假货,吊的是正身!”老魔停杯便问:“那猪八戒吆喝的是甚么?”
行者已变做小妖,上前道:“猪八戒撺道孙行者教变化走了罢,他不肯走,在那里
吆喝哩。”二魔道:“还说猪八戒老实?原来这等不老实!该打二十多嘴棍!”

这行者就去拿条棍来打。八戒道:“你打轻些儿,若重了些儿,我又喊起。我
认得你!”行者道:“老孙变化,也只为你们。你怎么倒走了风息?这一洞里妖精,
都认不得,怎的偏你认得?”八戒道:“你虽变了头脸,还不曾变得屁股。那屁股
上两块红不是?我因此认得是你。”行者随往后面,演到厨中,锅底上摸了一把,将
两臀擦黑,行至前边。八戒看见,又笑道:“那个猴子去那里混了这一会,弄做个
黑屁股来了。”

行者仍站在跟前,要偷他宝贝。真个甚有见识:走上厅,对那怪扯个腿子道:
“大王,你看那孙行者拴在柱上,左右爬蹉,磨坏那根金绳,得一根粗壮些的绳子
换将下来才好。”老魔道:“说得是。”即将腰间的狮蛮带解下,递与行者。行者接
了带,把假妆的行者拴住。换下那条绳子,一窝儿窝儿笼在袖内;又拔一根毫毛,
吹口仙气,变作一根假幌金绳,双手送与那怪。那怪只因贪酒,那曾细看,就便收
下。这个是:大圣腾那弄本事,毫毛又换幌金绳。

得了这件宝贝,急转身跳出门外,现了原身。高叫:“妖怪!”那把门的小妖问
道:“你是甚人,在此呼喝?”行者道:“你快早进去报与你那泼魔,说者行孙来了。”
那小妖如言报告。老魔大惊道:“拿住孙行者,又怎么有个者行孙?”二魔道:“哥
哥,怕他怎的?宝贝都在我手里,等我拿那葫芦出去,把他装将来。”老魔道:“兄
弟仔细。”

二魔拿了葫芦,走出山门,忽看见与孙行者模样一般,只是略矮些儿。问道:
“你是那里来的?”行者道:“我是孙行者的兄弟。闻说你拿了我家兄,却来与你
寻事的。”二魔道:“是我拿了,锁在洞中。你今既来,必要索战;我也不与你交兵,
我且叫你一声,你敢应我么?”行者道:“可怕你叫上千声,我就答应你万声!”那
魔执了宝贝,跳在空中,把底儿朝天,口儿朝地,叫声“者行孙。”行者却不敢答
应,心中暗想道:“若是应了,就装进去哩。”那魔道:“你怎么不应我?”行者道:
“我有些耳闭,不曾听见。你高叫。”那怪物又叫声“者行孙。”行者在底下掐着指
头算了一算,道:“我真名字叫做孙行者,起的鬼名字叫做者行孙。真名字可以装
得,鬼名字好道装不得。”却就忍不住,应了他一声。飕的被他吸进葫芦去,贴上
帖儿。原来那宝贝,那管甚么名字真假,但绰个应的气儿,就装了去也。

大圣到他葫芦里,浑然乌黑。把头往上一顶,那里顶得动,且是塞得甚紧,却
才心中焦躁道:“当时我在山上,遇着那两个小妖,他曾告诵我说:不拘葫芦、净
瓶,把人装在里面,只消一时三刻,就化为脓了,敢莫化了我么?”一条心又想着
道:“没事,化不得我。老孙五百年前大闹天宫,被太上老君放在八卦炉中炼了四
十九日,炼成个金子心肝,银子肺腑,铜头铁背,火眼金睛,那里一时三刻就化得
我?且跟他进去,看他怎的。”

二魔拿入里面道:“哥哥,拿来了。”老魔道:“拿了谁?”二魔道:“者行孙,
是我装在葫芦里也。”老魔欢喜道:“贤弟,请坐。不要动,只等摇得响再揭帖儿。”
行者听得道:“我这般一个身子,怎么便摇得响?只除化成稀汁,才摇得响是。等我
撒泡溺罢,他若摇得响时,一定揭帖起盖,我乘空走他娘罢!”又思道,“不好,不
好!溺虽可响,只是污了这直裰。等他摇时,我但聚些唾津漱口,稀漓呼喇的,哄
他揭开,老孙再走罢。”大圣作了准备,那怪贪酒不摇。大圣作个法,意思只是哄
他来摇,忽然叫道:“天呀,孤拐都化了!”那魔也不摇。大圣又叫道:“娘啊,连
腰截骨都化了!”老魔道:“化至腰时,都化尽矣。揭起帖儿看看。”

那大圣闻言,就拔了一根毫毛,叫“变!”变作个半截的身子,在葫芦底上。
真身却变做个虫儿,钉在那葫芦口边。只见那二魔揭起帖子看时,大圣早已飞
出。打个滚,又变做个倚海龙。倚海龙却是原去请老奶奶的那个小妖。他变了,站
在旁边。那老魔扳着葫芦口,张了一张,见是个半截身子动耽,他也不认真假,慌
忙叫:“兄弟,盖上,盖上,还不曾化得了哩!”二魔依旧贴上。大圣在旁暗笑道:
“不知老孙已在此矣!”

那老魔拿了壶,满满的斟了一杯酒,近前双手递与二魔道:“贤弟,我与你递
个锺儿。”二魔道:“兄长,我们已吃了这半会酒,又递甚锺?”老魔道:“你拿住
唐僧、八戒、沙僧犹可;又索了孙行者,装了者行孙,如此功劳,该与你多递几锺。”
二魔见哥哥恭敬,怎敢不接,但一只手托着葫芦,一只手不敢去接,却把葫芦递与
倚海龙,双手去接杯,不知那倚海龙是孙行者变的。你看他端葫芦,殷勤奉侍。二
魔接酒吃了,也要回奉一杯。老魔道:“不消回酒,我这里陪你一杯罢。”

两人只管谦逊。行者顶着葫芦,眼不转睛,看他两个左右传杯,全无计较,他
就把个葫芦入衣袖。拔根毫毛,变个假葫芦,一样无二,捧在手中。那魔递了一
会酒,也不看真假,一把接过宝贝。各上席,安然坐下,依然叙饮。孙大圣撤身走
过,得了宝贝,心中暗喜道:“饶这魔头有手段,毕竟葫芦还姓孙!”

毕竟不知向后怎样施为,方得救师灭怪,且听下回分解。