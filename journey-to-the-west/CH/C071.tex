\chapter{行者假名降怪~观音现象伏妖王}

色即空兮自古,空言是色如然。人能悟彻色空禅,何用丹砂炮炼。德行全修休
懈,工夫苦用熬煎。有时行满始朝天,永驻仙颜不变。

话说那赛太岁,紧关了前后门户,搜寻行者。直嚷到黄昏时分,不见踪迹。坐
在那剥皮亭上,点聚群妖,发号施令,都教各门上提铃喝号,击鼓敲梆;一个个弓
上弦,刀出鞘,支更坐夜。原来孙大圣变做个痴苍蝇,钉在门旁。见前面防备甚紧,
他即抖开翅,飞入后宫门首看处,见金圣娘娘伏在御案上,清清滴泪,隐隐声悲。
行者飞进门去,轻轻的落在他那乌云散髻之上,听他哭的甚么。少顷间,那娘娘忽
失声道:“主公啊!我和你:
前生烧了断头香,今世遭逢泼怪王。
拆凤三年何日会?分鸳两处致悲伤。
差来长老才通信,惊散佳姻一命亡。
只为金铃难解识,相思又比旧时狂。”
行者闻言,即移身到他耳根后,悄悄的叫道:“圣宫娘娘,你休恐惧。我还是你国
差来的神僧孙长老,未曾伤命。只因自家性急,近妆台偷了金铃,你与妖王吃酒之
时,我却脱身私出了前亭,忍不住打开看看。不期扯动那塞口的绵花,那铃响一声,
迸出烟火黄沙。我就慌了手脚,把金铃丢了,现出原身,使铁棒,苦战不出。恐遭
毒手,故变作一个苍蝇儿,钉在门枢上,躲到如今。那妖王愈加严紧,不肯开门。
你可去再以夫妻之礼,哄他进来安寝,我好脱身行事,别作区处救你也。”

娘娘一闻此言,战兢兢,发似神揪;虚怯怯,心如杵筑。泪汪汪的道:“你如
今是人是鬼?”行者道:“我也不是人,我也不是鬼,如今变作个苍蝇儿在此。你
休怕,快去请那妖王也。”娘娘不信,泪滴滴悄语低声道:“你莫魇寐我。”行者道:
“我岂敢魇寐你?你若不信,展开手,等我跳下来你看。”那娘娘真个把左手张开,
行者轻轻飞下,落在他玉掌之间,好便似:
菡萏蕊头钉黑豆,牡丹花上歇游蜂;
绣球心里葡萄落,百合枝边黑点浓。
金圣宫高擎玉掌,叫声:“神僧。”行者嘤嘤的应道:“我是神僧变的。”那娘娘方才
信了,悄悄的道:“我去请那妖王来时,你却怎生行事?”行者道:“古人云:‘断
送一生惟有酒。’又云:‘破除万事无过酒。’酒之为用多端。你只以饮酒为上。你
将那贴身的侍婢,唤一个进来,指与我看,我就变作他的模样,在旁边伏侍,却好
下手。”

那娘娘真个依言,即叫:“春娇何在?”那屏风后转出一个玉面狐狸来,跪下
道:“娘娘唤春娇有何使令?”娘娘道:“你去叫他们来点纱灯,焚脑麝,扶我上前
庭,请大王安寝也。”那春娇即转前面,叫了七八个怪鹿妖狐,打着两对灯龙,一
对提炉,摆列左右。娘娘欠身叉手,那大圣早已飞去。好行者,展开翅,径飞到那
玉面狐狸头上,拔下一根毫毛,吹口仙气,叫“变!”变作一个瞌睡虫,轻轻的放
在他脸上。原来瞌睡虫到了人脸上,往鼻孔里爬;爬进孔中,即瞌睡了。那春娇果
然渐觉困倦,立不住脚,摇桩打盹,即忙寻着原睡处,丢倒头,只情呼呼的睡起。
行者跳下来,摇身一变,变做那春娇一般模样,转屏风,与众排立不题。

却说那金圣宫娘娘往前正走,有小妖看见,即报赛太岁道:“大王,娘娘来了。”
那妖王急出剥皮亭外迎迓。娘娘道:“大王啊,烟火既息,贼已无踪,深夜之际,
特请大王安置。”那妖满心欢喜道:“娘娘珍重。却才那贼乃是孙悟空。他败了我先
锋,打杀我小校,变化进来,哄了我们。我们这般搜检,他却渺无踪迹,故此心上
不安。”娘娘道:“那厮想是走脱了。大王放心勿虑,且自安寝去也。”

妖精见娘娘侍立敬请,不敢坚辞,只得吩咐群妖,各要小心火烛,谨防盗贼,
遂与娘娘径往后宫。行者假变春娇,从两班侍婢引入。娘娘叫:“安排酒来与大王
解劳。”妖王笑道:“正是,正是。快将酒来,我与娘娘压惊。”“假春娇”即同众怪
铺排了果品,整顿些腥肉,调开桌椅。那娘娘擎杯,这妖王也以一杯奉上,二人穿
换了酒杯。“假春娇”在旁,执着酒壶道:“大王与娘娘今夜才递交杯盏,请各饮干,
穿个双喜杯儿。”真个又各斟上,又饮干了。“假春娇”又道:“大王娘娘喜会,众
侍婢会唱的供唱,善舞的起舞来耶。”说未毕,只听得一派歌声,齐调音律,唱的
唱,舞的舞。他两个又饮了许多,娘娘叫住了歌舞。众侍婢分班,出屏风外摆列;
惟有“假春娇”执壶,上下奉酒。娘娘与那妖王专说得是夫妻之话。你看那娘娘一
片云情雨意,哄得那妖王骨软筋麻。只是没福,不得沾身。可怜!真是“猫咬尿胞
空欢喜”!

叙了一会,笑了一会,娘娘问道:“大王,宝贝不曾伤损么?”妖王道:“这宝
贝乃先天抟铸之物,如何得损!只是被那贼扯开塞口之绵,烧了豹皮包袱也。”娘娘
说:“怎生收拾?”妖王道:“不用收拾,我带在腰间哩。”

“假春娇”闻得此言,即拔下毫毛一把,嚼得粉碎,轻轻挨近妖王,将那毫毛
放在他身上,吹了三口仙气,暗暗的叫“变!”那些毫毛即变做三样恶物,乃虱子、
虼蚤、臭虫,攻入妖王身内,挨着皮肤乱咬。那妖王燥痒难禁,伸手入怀揣摸揉痒,
用指头捏出几个虱子来,拿近灯前观看。娘娘见了,含忖道:“大王,想是衬衣禳
了,久不曾浆洗,故生此物耳。”妖王惭愧道:“我从来不生此物,可可的今宵出丑。”
娘娘笑道:“大王何为出丑?常言道:‘皇帝身上也有三个御虱’哩。且脱下衣服来,
等我替你捉捉。”妖王真个解带脱衣。

“假春娇”在旁,着意看着那妖王身上,衣服层层皆有虼蚤跳,件件皆排大臭
虫;子母虱,密密浓浓,就如蝼蚁出窝中。不觉的揭到第三层见肉之处,那金铃上
纷纷垓垓的,也不胜其数。“假春娇”道:“大王,拿铃子来,等我也与你捉捉虱子。”
那妖王一则羞,二则慌,却也不认得真假,将三个铃儿递与“假春娇”。“假春娇”
接在手中,卖弄多时,见那妖王低着头抖这衣服,他即将金铃藏了,拔下一根毫毛,
变作三个铃儿,一般无二,拿向灯前翻检;却又把身子扭扭捏捏的,抖了一抖,将
那虱子、臭虫、虼蚤,收了归在身上,把假金铃儿递与那怪。

那怪接在手中,一发朦胧无措,那里认得甚么真假,双手托着那铃儿,递与娘
娘道:“今番你却收好了。却要仔细仔细,不要像前一番。”那娘娘接过来,轻轻的
揭开衣箱,把那假铃收了,用黄金锁锁了。却又与妖王叙饮了几杯酒,教侍婢:“净
拂牙床,展开锦被,我与大王同寝。”那妖王诺诺连声道:“没福,没福,不敢奉陪。
我还带个宫女往西宫里睡去。娘娘请自安置。”遂此各归寝处不题。

却说“假春娇”得了手,将他宝贝带在腰间,现了本象,把身子抖一抖,收去
那个瞌睡虫儿,径往前走,只听得梆铃齐响,紧打三更,好行者,捏着诀,念动真
言,使个隐身法,直至门边。又见那门上拴锁甚密,却就取出金箍棒,望门一指,
使出那解锁之法,那门就轻轻开了。急拽步出门站下,厉声高叫道:“赛太岁,还
我金圣娘娘来!”连叫两三遍,惊动大小群妖,急急看处,前门开了,即忙掌灯寻
锁,把门儿依然锁上,着几个跑入里边去报道:“大王!有人在大门外呼唤大王尊号,
要金圣娘娘哩!”那里边侍婢,即出宫门,悄悄的传言道:“莫吆喝,大王才睡着了。”
行者又在门前高叫,那小妖又不敢去惊动。如此者三四遍,俱不敢去通报。

那大圣在外嚷嚷闹闹的,直弄到天晓。忍不住,手轮着铁棒,上前打门。慌得
那大小群妖,顶门的顶门,报信的报信。那妖王一觉方醒,只闻得乱撺撺的喧哗,
起身穿了衣服,即出罗帐之外,问道:“嚷甚么?”众侍婢才跪下道:“爷爷,不知
是甚人在洞外叫骂了半夜,如今却又打门。”

妖王走出宫门,只见那几个传报的小妖,慌张张的磕头道:“外面有人叫骂,
要金圣宫娘娘哩!若说半个‘不’字,他就说出无数的歪话,甚不中听。见天晓大
王不出,逼得打门也。”那妖道:“且休开门。你去问他是那里来的,姓甚名谁。快
来回报。”

小妖急出去,隔门问道:“打门的是谁?”行者道:“我是朱紫国拜请来的外公,
来取圣宫娘娘回国哩!”那小妖听得,即以此言回报。那妖随往后宫,查问来历。
原来那娘娘才起来,还未梳洗。早见侍婢来报:“爷爷来了。”那娘娘急整衣,散挽
黑云,出宫迎迓。才坐下,还未及问,又听得小妖来报:“那来的外公已将门打破
矣。”那妖笑道:“娘娘,你朝中有多少将帅?”娘娘道:“在朝有四十八卫人马,
良将千员;各边上元帅总兵,不计其数。”妖王道:“可有个姓外的么?”娘娘道:
“我在宫,只知内里辅助君王,早晚教诲妃嫔,外事无边,我怎记得名姓!”妖王
道:“这来者称为‘外公’,我想着《百家姓》上,更无个姓外的。娘娘赋性聪明,
出身高贵,居皇宫之中,必多览书籍。记得那本书上有此姓也?”娘娘道:“止《千
字文》上有句‘外受傅训’,想必就是此矣。”

妖王喜道:“定是,定是。”即起身辞了娘娘,到剥皮亭上,结束整齐,点出妖
兵,开了门,直至外面,手持一柄宣花钺斧,厉声高叫道:“那个是朱紫国来的‘外
公’?”行者把金箍棒攥在右手,将左手指定道:“贤甥,叫我怎的?”那妖王见
了,心中大怒道:“你这厮:
相貌若猴子,嘴脸似猢狲。
七分真是鬼,大胆敢欺人!”
行者笑道:“你这个诳上欺君的泼怪,原来没眼!想我五百年前大闹天宫时,九天神
将见了我,无一个‘老’字,不敢称呼;你叫我声‘外公’,那里亏了你!”妖王喝
道:“快早说出姓甚名谁,有些甚么武艺,敢到我这里猖獗!”行者道:“你若不问
姓名犹可,若要我说出姓名,只怕你立身无地!你上来,站稳着,听我道:

生身父母是天地,日月精华结圣胎。仙石怀抱无岁数,灵根孕育甚奇哉。当年
产我三阳泰,今日归真万会谐。曾聚众妖称帅首,能降众怪拜丹崖。玉皇大帝传宣
旨,太白金星捧诏来。请我上天承职裔,官封‘弼马’不开怀。初心造反谋山洞,
大胆兴兵闹御阶。托塔天王并太子,交锋一阵尽猥衰。金星复奏玄穹帝,再降招安
敕旨来。封做齐天真大圣,那时方称栋梁材。又因搅乱蟠桃会,仗酒偷丹惹下灾。
太上老君亲奏驾,西池王母拜瑶台。情知是我欺王法,即点天兵发火牌。十万凶星
并恶曜,干戈剑戟密排排。天罗地网漫山布,齐举刀兵大会垓。恶斗一场无胜败,
观音推荐二郎来。两家对敌分高下,他有梅山兄弟侪。各逞英雄施变化,天门三圣
拨云开。老君丢了金钢套,众神擒我到金阶。不须详允书供状,罪犯凌迟杀斩灾。
斧剁锤敲难损命,刀轮剑砍怎伤怀。火烧雷打只如此,无计摧残长寿胎。押赴太清
兜率院,炉中煅炼尽安排。日期满足才开鼎,我向当中跳出来。手挺这条如意棒,
翻身打上玉龙台。各星各象皆潜躲,大闹天宫任我歪。巡视灵官忙请佛,释伽与我
逞英才。手心之内翻筋斗,游遍周天去复来。佛使先知赚哄法,被他压住在天崖。
到今五百余年矣,解脱微躯又弄乖。特保唐僧西域去,悟空行者甚明白。西方路上
降妖怪,那个妖邪不惧哉!”

那妖王听他说出悟空行者,遂道:“你原来是大闹天宫的那厮。你既脱身保唐
僧西去,你走你的路去便罢了,怎么罗织管事,替那朱紫国为奴,却到我这里寻死!”
行者喝道:“贼泼怪!说话无知!我受朱紫国拜请之礼,又蒙他称呼管待之恩,我老
孙比那王位还高千倍,他敬之如父母,事之如神明,你怎么说出‘为奴’二字!我
把你这诳上欺君之怪,不要走,吃外公一棒!”那妖慌了手脚,即闪身躲过,使宣
花斧劈面相迎。这一场好杀!你看:

金箍如意棒,风刃宣花斧。一个咬牙发狠凶,一个切齿施威武。这个是齐天大
圣降临凡,那个是作怪妖王来下土。两个喷云嗳雾照天宫,真是走石扬沙遮斗府。
往往来来解数多,翻翻复复金光吐。齐将本事施,各把神通赌。这个要取娘娘转帝
都,那个喜同皇后居山坞。这场都是没来由,舍死忘生因国主。
他两个战经五十回合,不分胜负。那妖王见行者手段高强,料不能取胜,将斧架住
他的铁棒道:“孙行者,你且住了。我今日还未早膳,待我进了膳,再来与你定雌
雄。”行者情知是要取铃铛,收了铁棒道:“‘好汉子不赶乏兔儿’,你去,你去!吃
饱些,好来领死!”

那妖急转身闯入里边,对娘娘道:“快将宝贝拿来!”娘娘道:“要宝贝何干?”
妖王道:“今早叫战者,乃是取经的和尚之徒,叫做孙悟空行者,假称‘外公’。我
与他战到此时,不分胜负。等我拿宝贝出去,放些烟火,烧这猴头。”娘娘见说,
心中怛突:欲不取出铃儿,恐他见疑;欲取出铃儿,又恐伤了孙行者性命。正自踌
躇未定,那妖王又催逼道:“快拿出来!”这娘娘无奈,只得将锁钥开了,把三个铃
儿递与妖王。妖王拿了,就走出洞。娘娘坐在宫中,泪如雨下,思量行者不知可能
逃得性命。两人却俱不知是假铃也。

那妖出了门,就占起上风,叫道:“孙行者,休走!看我摇摇铃儿!”行者笑道:
“你有铃,我就没铃?你会摇,我就不会摇?”妖王道:“你有甚么铃儿,拿出来我
看。”行者将铁棒捏做个绣花针儿,藏在耳内,却去腰间解下三个真宝贝来,对妖
王说:“这不是我的紫金铃儿?”妖王见了,心惊道:“跷蹊,跷蹊!他的铃儿怎么
与我的铃儿就一般无二!纵然是一个模子铸的,好道打磨不到,也有多个瘢儿,少
个蒂儿,却怎么这等一毫不差?”又问:“你那铃儿是那里来的?”行者道:“贤甥,
你那铃儿却是那里来的?”妖王老实,便就说道:“我这铃儿是:
太清仙君道源深,八卦炉中久炼金。
结就铃儿称至宝,老君留下到如今。”
行者笑道:“老孙的铃儿,也是那时来的。”妖王道:“怎生出处?”行者道:“我这
铃儿是:
道祖烧丹兜率宫,金铃抟炼在炉中。
二三如六循环宝,我的雌来你的雄。”
妖王道:“铃儿乃金丹之宝,又不是飞禽走兽,如何辨得雌雄?但只是摇出宝来,就
是好的!”行者道:“口说无凭,做出便见。且让你先摇。”那妖王真个将头一个铃
儿幌了三幌,不见火出;第二个幌了三幌,不见烟出;第三个幌了三幌,也不见沙
出。妖王慌了手脚道:“怪哉,怪哉!世情变了!这铃儿想是惧内,雄见了雌,所以
不出来了。”

行者道:“贤甥,住了手,等我也摇摇你看。”好猴子,一把攥了三个铃儿,一
齐摇起。你看那红火、青烟、黄沙,一齐滚出,骨都都燎树烧山!大圣口里又念个
咒语,望巽地上叫:“风来!”真个是风催火势,火挟风威,红焰焰,黑沉沉,满天
烟火,遍地黄沙!把那赛太岁唬得魄散魂飞,走头无路,在那火当中,怎逃性命!

只闻得半空中厉声高叫:“孙悟空,我来了也!”行者急回头上望,原来是观音
菩萨,左手托着净瓶,右手拿着杨柳,洒下甘露救火哩。慌得行者把铃儿藏在腰间,
即合掌倒身下拜。那菩萨将柳枝连拂几点甘露,霎时间,烟火俱无,黄沙绝迹。行
者叩头道:“不知大慈临凡,有失回避。敢问菩萨何往?”菩萨道:“我特来收寻这
个妖怪物。”

行者道:“这怪是何来历,敢劳金身下降收之?”菩萨道:“他是我跨的个金毛
。因牧童盹睡,失于防守,这孽畜咬断铁索走来,却与朱紫国王消灾也。”行者
闻言,急欠身道:“菩萨反说了。他在这里欺君骗后,败俗伤风,与那国王生灾,
却说是消灾,何也?”菩萨道:“你不知之。当时朱紫国先王在位之时,这个王还
做东宫太子,未曾登基。他年幼间,极好射猎。他率领人马,纵放鹰犬,正来到落
凤坡前,有西方佛母孔雀大明王菩萨所生二子,乃雌雄两个雀雏,停翅在山坡之下,
被此王弓开处,射伤了雌孔雀,那雌孔雀也带箭归西。佛母忏悔以后,吩咐教他拆
凤三年,身耽啾疾。那时节,我跨着这,同听此言,不期这孽畜留心,故来骗了
皇后,与王消灾。至今三年,冤愆满足,幸你来救治王患。我特来收妖邪也。”行
者道:“菩萨,虽是这般故事,奈何他玷污了皇后,败俗伤风,坏伦乱法,却是该
他死罪。今蒙菩萨亲临,饶得他死罪,却饶不得他活罪。让我打他二十棒,与你带
去罢。”菩萨道:“悟空,你既知我临凡,就当看我分上,一发都饶了罢;也算你一
番降妖之功。若是动了棍子,他也就是死了。”行者不敢违言,只得拜道:“菩萨既
收他回海,再不可令他私降人间,贻害不浅!”

那菩萨才喝了一声:“孽畜!还不还原,待何时也!”只见那怪打个滚,现了原
身,将毛衣抖抖,菩萨骑上。菩萨又望项下一看,不见那三个金铃。菩萨道:“悟
空,还我铃来。”行者道:“老孙不知。”菩萨喝道:“你这贼猴!若不是你偷了这铃,
莫说一个悟空,就是十个,也不敢近身。快拿出来!”行者笑道:“实不曾见。”菩
萨道:“既不曾见,等我念念《紧箍儿咒》。”那行者慌了,只教:“莫念,莫念!铃
儿在这里哩!”这正是:

项金铃何人解?解铃人还问系铃人。
菩萨将铃儿套在项下,飞身高坐。你看他四足莲花生焰焰,满身金缕迸森森。大
慈悲回南海不题。

却说孙大圣整束了衣裙,轮铁棒打进獬豸洞去,把群妖众怪,尽情打死,剿除
干净。直至宫中,请圣宫娘娘回国。那娘娘顶礼不尽。行者将菩萨降妖并拆凤原由
备说了一遍,寻些软草,扎了一条草龙,教:“娘娘跨上,合着眼,莫怕,我带你
回朝见主也。”

那娘娘谨遵吩咐,行者使起神通,只听得耳内风响,半个时辰,带进城,按落
云头,叫:“娘娘开眼。”那皇后睁开眼看,认得是凤阁龙楼,心中欢喜,撇了草龙,
与行者同登宝殿。那国王见了,急下龙床,就来扯娘娘玉手,欲诉离情,猛然跌倒
在地,只叫:“手疼!手疼!”八戒哈哈大笑道:“嘴脸,没福消受,一见面就蜇杀了
也!”行者道:“呆子,你敢扯他扯儿么?”八戒道:“就扯他扯儿便怎的?”行者
道:“娘娘身上生了毒刺,手上有蜇阳之毒。自到麒麟山,与那赛太岁三年,那妖
更不曾沾身。但沾身就害身疼,但沾手就害手疼。”众官听说,道:“似此怎生奈何?”
此时外面众官忧疑,内里妃嫔悚惧,旁有玉圣、银圣二宫,将君王扶起。

俱正在仓皇之际,忽听得那半空中,有人叫道:“大圣,我来也。”行者抬头观
看,只见那:

肃肃冲天鹤唳,飘飘径至朝前。缭绕祥光道道,氤氲瑞气翩翩。棕衣苫体放云
烟,足踏芒鞋罕见。手执龙须蝇帚,丝绦腰下围缠。乾坤处处结人缘,大地逍遥游
遍。此乃是大罗天上紫云仙,今日临凡解魇。
行者上前迎住道:“张紫阳何往?”紫阳真人直至殿前,躬身施礼道:“大圣,小仙
张伯端起手。”行者答礼道:“你从何来?”真人道:“小仙三年前曾赴佛会。因打
这里经过,见朱紫国王有拆凤之忧,我恐那妖将皇后玷辱,有坏人伦,后日难与国
王复合。是我将一件旧棕衣变作一领新霞裳,光生五彩,进与妖王,教皇后穿了妆
新。那皇后穿上身,即生一身毒刺。毒刺者,乃棕毛也。今知大圣成功,特来解魇。”
行者道:“既如此,累你远来,且快解脱。”真人走向前,对娘娘用手一指,即脱下
那件棕衣。那娘娘偏体如旧。真人将衣抖一抖,披在身上,对行者道:“大圣勿罪,
小仙告辞。”行者道:“且住,待君王谢谢。”真人笑道:“不劳,不劳。”遂长揖一
声,腾空而去。慌得那皇帝、皇后及大小众臣,一个个望空礼拜。

拜毕,即命大开东阁,酬谢四僧。那君王领众跪拜,夫妻才得重谐。正当欢宴
时,行者叫:“师父,拿那战书来。”长老袖中取出,递与行者。行者递与国王道:
“此书乃那怪差小校送来者。那小校已先被我打死,送来报功。后复至山中,变作
小校,进洞回复,因得见娘娘,盗出金铃,几乎被他拿住;又变化,复偷出,与他
对敌。幸遇观音菩萨将他收去,又与我说拆凤之故。……”从头至尾,细说了一遍。
那举国君臣内外,无一人不感谢称赞。唐僧道:“一则是贤王之福,二来是小徒之
功。今蒙盛宴,至矣,至矣!就此拜别,不要误贫僧向西去也。”那国王恳留不得,
遂换了关文,大排銮驾,请唐僧稳坐龙车,那君王、妃后,俱捧毂推轮,相送而别。
正是:
有缘洗尽忧疑病,绝念无思心自宁。

毕竟这去,后面再有甚么吉凶之事,且听下回分解。