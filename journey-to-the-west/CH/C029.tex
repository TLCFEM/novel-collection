\chapter{脱难江流来国土~承恩八戒转山林}

词曰:

妄想不复强灭,真如何必希求?本原自性佛前修,迷悟岂居前后?悟即刹那成正,
迷而万劫沉流。若能一念合真修,灭尽恒沙罪垢。

却说那八戒、沙僧与怪斗经个三十回合,不分胜负。你道怎么不分胜负?若论
赌手段,莫说两个和尚,就是二十个,也敌不过那妖精。只为唐僧命不该死,暗中
有那护法神祇保着他;空中又有那六丁六甲、五方揭谛、四值功曹、一十八位护教
伽蓝,助着八戒、沙僧。

且不言他三人战斗。却说那长老在洞里悲啼,思量他那徒弟。眼中流泪道:“悟
能啊,不知你在那个村中逢了善友,贪着斋供;悟净啊,你又不知在那里寻他,可
能得会?岂知我遇妖魔,在此受难!几时得会你们,脱了大难,早赴灵山!”

正当悲啼烦恼,忽见那洞里走出一个妇人来,扶着定魂桩,叫道:“那长老,
你从何来?为何被他缚在此处?”长老闻言,泪眼偷看,那妇人约有三十年纪。遂
道:“女菩萨,不消问了。我已是该死的,走进你家门来也。要吃就吃了罢,又问
怎的?”那妇人道:“我不是吃人的。我家离此西下,有三百余里。那里有座城,
叫做宝象国。我是那国王的第三个公主,乳名叫做百花羞。只因十三年前,八月十
五日夜,玩月中间,被这妖魔,一阵狂风摄将来,与他做了十三年夫妻。在此生儿
育女,杳无音信回朝。思量我那父母,不能相见。你从何来,被他拿住?”唐僧道:
“贫僧乃是差往西天取经者。不期闲步,误撞在此。如今要拿住我两个徒弟,一齐
蒸吃哩。”那公主陪笑道:“长老宽心。你既是取经的,我救得你。那宝象国是你西
方去的大路。你与我捎一封书儿去,拜上我那父母,我就教他饶了你罢。”三藏点
头道:“女菩萨,若还救得贫僧命,愿做捎书寄信人。”

那公主急转后面,即修了一纸家书,封固停当;到桩前解放了唐僧,将书付与。
唐僧得解脱,捧书在手道:“女菩萨,多谢你活命之恩。贫僧这一去,过贵处,定
送国王处。只恐日久年深,你父母不肯相认,奈何?切莫怪我贫僧打了诳语。”公主
道:“不妨,我父王无子,止生我三个姊妹,若见此书,必有相看之意。”三藏紧紧
袖了家书,谢了公主,就往外走。被公主扯住道:“前门里你出不去!那些大小妖精,
都在门外摇旗呐喊,擂鼓筛锣,助着大王,与你徒弟厮杀哩。你往后门里去罢。若
是大王拿住,还审问审问,只恐小妖儿捉了,不分好歹,挟生儿伤了你的性命。等
我去他面前,说个方便。若是大王放了你啊,待你徒弟讨个示下,寻着你一同好走。”
三藏闻言,磕了头,谨依吩咐,辞别公主,躲离后门之外,不敢自行,将身藏在荆
棘丛中。

却说公主娘娘,心生巧计,急往前来,出门外,分开了大小群妖;只听得叮叮
当当,兵刃乱响。原来是八戒、沙僧与那怪在半空里厮杀哩。这公主厉声高叫道:
“黄袍郎!”那妖王听得公主叫唤,即丢了八戒、沙僧,按落云头,揪了钢刀,搀
着公主道:“浑家,有甚话说?”公主道:“郎君啊,我才时睡在罗帏之内,梦魂中,
忽见个金甲神人。”妖魔道:“那个金甲神?上我门怎的?”公主道:“是我幼时,在
宫里,对神暗许下一桩心愿:若得招个贤郎驸马,上名山,拜仙府,斋僧布施。自
从配了你,夫妻们欢会,到今不曾题起。那金甲神人来讨誓愿,喝我醒来,却是南
柯一梦。因此,急整容来郎君处诉知,不期那桩上绑着一个僧人,万望郎君慈悯,
看我薄意,饶了那个和尚罢。只当与我斋僧还愿。不知郎君肯否?”那怪道:“浑
家,你却多心呐!甚么打紧之事。我要吃人,那里不捞几个吃吃。这个把和尚,到
得那里,放他去罢。”公主道:“郎君,放他从后门里去罢。”妖魔道:“奈烦哩。放
他去便罢,又管他甚么后门前门哩。”他遂绰了钢刀,高叫道:“那猪八戒,你过来。
我不是怕你,不与你战;看着我浑家的分上,饶了你师父也。趁早去后门首,寻着
他,往西方去罢。若再来犯我境界,断乎不饶!”

那八戒与沙僧闻得此言,就如鬼门关上放回来的一般。即忙牵马挑担,鼠窜而
行。转过那波月洞,后门之外,叫声“师父!”那长老认得声音,就在那荆棘中答
应。沙僧就剖开草径,搀着师父,慌忙的上马。这里:
狠毒险遭青面鬼,殷勤幸有百花羞。
鳌鱼脱却金钩钓,摆尾摇头逐浪游。

八戒当头领路,沙僧后随,出了那松林,上了大路。你看他两个哜哜嘈嘈,埋
埋怨怨,三藏只是解和。遇晚先投宿,鸡鸣早看天。一程一程,长亭短亭,不觉的
就走了二百九十九里。猛抬头,只见一座好城,就是宝象国。真好个处所也:

云渺渺,路迢迢;地虽千里外,景物一般饶。瑞霭祥烟笼罩,清风明月招摇。
的远山,大开图画;潺潺的流水,碎溅琼瑶。可耕的连阡带陌,足食
的密蕙新苗。渔钓的几家三涧曲,樵采的一担两峰椒。廓的廓,城的城,金汤巩固;
家的家,户的户,只斗逍遥。九重的高阁如殿宇,万丈的层台似锦标。也有那太极
殿、华盖殿、烧香殿、观文殿、宣政殿、延英殿:一殿殿的玉陛金阶,摆列着文冠
武弁;也有那大明宫、昭阳宫、长乐宫、华清宫、建章宫、未央宫:一宫宫的钟鼓
管
,撒抹了闺怨春愁。也有禁苑的,露花匀嫩脸;也有御沟的,风柳舞纤腰。通衢
上,也有个顶冠束带的,盛仪容,乘五马;幽僻中,也有个持弓挟矢的,拨云雾,
贯双雕。花柳的巷,管弦的楼,春风不让洛阳桥。取经的长老,回首大唐肝胆裂;
伴师的徒弟,息肩小驿梦魂消。
看不尽宝象国的景致。师徒三众,收拾行李、马匹,安歇馆驿中。

唐僧步行至朝门外,对阁门大使道:“有唐朝僧人,特来面驾,倒换文牒。乞
为转奏转奏。”那黄门奏事官,连忙走至白玉阶前奏道:“万岁,唐朝有个高僧,欲
求见驾,倒换文牒。”那国王闻知是唐朝大国,且又说是个方上圣僧,心中甚喜,
即时准奏。叫:“宣他进来。”把三藏宣至金阶,舞蹈山呼礼毕。两边文武多官,无
不叹道:“上邦人物,礼乐雍容如此!”那国王道:“长老,你到我国中何事?”三
藏道:“小僧是唐朝释子。承我天子敕旨,前往西方取经;原领有文牒,到陛下上
国,理合倒换。故此不识进退,惊动龙颜。”国王道:“既有唐天子文牒,取上来看。”
三藏双手捧上去,展开放在御案上。牒云:

南赡部洲大唐国奉天承运唐天子牒行:切惟朕以凉德,嗣续丕基,事神治民,
临深履薄,朝夕是惴。前者,失救泾河老龙,获谴于我皇皇后帝,三魂七魄,倏忽
阴司,已作无常之客。因有阳寿未绝,感冥君放送回生,广陈善会,修建度亡道场。
感蒙救苦观世音菩萨,金身出现,指示西方有佛有经,可度幽亡,超脱孤魂。特着
法师玄奘,远历千山,询求经偈。倘到西邦诸国,不灭善缘,照牒放行。须至牒者。
大唐贞观一十三年,秋吉日,御前文牒。(上有宝印九颗)
国王见了,取本国玉宝,用了花押,递与三藏。

三藏谢了恩,收了文牒。又奏道:“贫僧一来倒换文牒,二来与陛下寄有家书。”
国王大喜道:“有甚书?”三藏道:“陛下第三位公主娘娘,被碗子山波月洞黄袍妖
摄将去,贫僧偶尔相遇,故寄书来也。”国王闻言,满眼垂泪道:“自十三年前,不
见了公主,两班文武官,也不知贬退了多少;宫内宫外,大小婢子、太监,也不知
打死了多少。只说是走出皇宫,迷失路径,无处找寻;满城中百姓人家,也盘诘了
无数,更无下落。怎知道是妖怪摄了去!今日乍听得这句话,故此伤情流泪。”三藏
袖中取出书来献上。国王接了,见有“平安”二字,一发手软,拆不开书。传旨宣
翰林院大学士上殿读书。学士随即上殿。殿前有文武多官,殿后有后妃宫女,俱侧
耳听书。学士拆开朗诵。上写着:

不孝女百花羞顿首百拜大德父王万岁龙凤殿前暨三宫母后昭阳宫下,及举朝文
武贤卿台次:拙女幸托坤宫,感激劬劳万种。不能竭力怡颜,尽心奉孝。乃于十三
年前,八月十五日,良夜佳辰,蒙父王恩旨,着各宫排宴,赏玩月华,共乐清霄盛
会。正欢娱之间,不觉一阵香风,闪出个金睛蓝面青发魔王,将女擒住,驾祥光,
直带至半野山中无人处,难分难辨,被妖倚强。霸占为妻。是以无奈捱了一十三年。
产下两个妖儿,尽是妖魔之种。论此真是败坏人伦,有伤风化,不当传书玷辱;但
恐女死之后,不显分明。正含怨思忆父母,不期唐朝圣僧,亦被魔王擒住。是女滴
泪修书,大胆放脱,特托寄此片楮,以表寸心。伏望父王垂悯,遣上将早至碗子山
波月洞捉获黄袍怪,救女回朝,深为恩念。草草欠恭,面听不一。
逆女百花羞再顿首顿首。

那学士读罢家书,国王大哭,三宫滴泪,文武伤情,前前后后,无不哀念。

国王哭之许久,便问两班文武:“那个敢兴兵领将,与寡人捉获妖魔,救我百
花公主?”连问数声,更无一人敢答。真是木雕成的武将,泥塑就的文官。那国王
心生烦恼,泪若涌泉。只见那多官齐俯伏奏道:“陛下且休烦恼。公主已失,至今
一十三载无音,偶遇唐朝圣僧,寄书来此,未知的否。况臣等俱是凡人凡马,习学
兵书武略,止可布阵安营,保国家无侵陵之患。那妖精乃云来雾去之辈,不得与他
觌面相见,何以征救?想东土取经者,乃上邦圣僧。这和尚‘道高龙虎伏,德重鬼
神钦’,必有降妖之术。自古道:‘来说是非者,就是是非人。’可就请这长老降妖
邪,救公主,庶为万全之策。”

那国王闻言,急回头,便请三藏道:“长老若有手段,放法力,捉了妖魔,救
我孩儿回朝,也不须上西方拜佛,长发留头,朕与你结为兄弟,同坐龙床,共享富
贵如何?”三藏慌忙启上道:“贫僧粗知念佛,其实不会降妖。”国王道:“你既不
会降妖,怎么敢上西天拜佛?”那长老瞒不过,说出两个徒弟来了。奏道:“陛下,
贫僧一人,实难到此。贫僧有两个徒弟,善能逢山开路,遇水叠桥,保贫僧到此。”
国王怪道:“你这和尚大没理。既有徒弟,怎么不与他一同进来见朕?若到朝中,虽
无中意赏赐,必有随分斋供。”三藏道:“贫僧那徒弟丑陋,不敢擅自入朝,但恐惊
伤了陛下的龙体。”国王笑道:“你看你这和尚说话,终不然朕当怕他?”三藏道:
“不敢说。我那大徒弟姓猪,法名悟能八戒。他生得长嘴獠牙,刚鬃扇耳,身粗肚
大,行路生风。第二个徒弟姓沙,法名悟净和尚。他生得身长丈二,臂阔三停,脸
如蓝靛,口似血盆,眼光闪灼,牙齿排钉。他都是这等个模样,所以不敢擅领入朝。”
国王道:“你既这等样说了一遍,寡人怕他怎的?宣进来。”随即着金牌至馆驿相请。

那呆子听见来请,对沙僧道:“兄弟,你还不教下书哩。这才见了下书的好处。
想是师父下了书,国王道:捎书人不可怠慢,一定整治筵宴待他;他的食肠不济,
有你我之心,举出名来,故此着金牌来请。大家吃一顿,明日好行。”沙僧道:“哥
啊,知道是甚缘故,我们且去来。”遂将行李、马匹俱交付驿丞。各带随身兵器,
随金牌入朝。早行到白玉阶前,左右立下,朝上唱个喏,再也不动。那文武多官,
无人不怕。都说道:“这两个和尚,貌丑也罢,只是粗俗太甚!怎么见我王更不下拜,
喏毕平身,挺然而立!可怪!可怪!”八戒听见道:“列位,莫要议论。我们是这般。
乍看果有些丑,只是看下些时来,却也耐看。”

那国王见他丑陋,已是心惊;及听得那呆子说出话来,越发胆颤,就坐不稳,
跌下龙床。幸有近侍官员扶起。慌得个唐僧,跪在殿前,不住的叩头道:“陛下,
贫僧该万死!万死!我说徒弟丑陋,不敢朝见,恐伤龙体,果然惊了驾也。”那国王
战兢兢,走近前,搀起道:“长老,还亏你先说过了;若未说,猛然见他,寡人一
定唬杀了也!”

国王定性多时,便问:“猪长老、沙长老,是那一位善于降妖?”那呆子不知
好歹,答道:“老猪会降。”国王道:“怎么家降?”八戒道:“我乃是天篷元帅;只
因罪犯天条,堕落下世,幸今皈正为僧。自从东土来此,第一会降妖的是我。”国
王道:“既是天将临凡,必然善能变化。”八戒道:“不敢,不敢,也将就晓得几个
变化儿。”国王道:“你试变一个我看看。”八戒道:“请出题目,照依样子好变。”
国王道:“变一个大的罢。”

那八戒他也有三十六般变化,就在阶前,卖弄手段,却便捻诀念咒,喝一声叫
“长!”把腰一躬,就长了有八九丈长,却似个开路神一般。吓得那两班文武,战
战兢兢;一国君臣,呆呆挣挣。时有镇殿将军问道:“长老,似这等变得身高,必
定长到甚么去处,才有止极?”那呆子又说出呆话来道:“看风。东风犹可,西风
也将就;若是南风起,把青天也拱个大窟窿!”那国王大惊道:“收了神通罢。晓得
是这般变化了。”八戒把身一矬,依然现了本相,侍立阶前。

国王又问道:“长老此去,有何兵器与他交战?”八戒腰里掣出钯来道:“老猪
使的是钉钯。”国王笑道:“可败坏门面!我这里有的是鞭、简、瓜、锤、刀、枪、
钺、斧、剑、戟、矛、镰,随你选称手的拿一件去。那钯算做甚么兵器?”八戒道:
“陛下不知。我这钯,虽然粗夯,实是自幼随身之器。曾在天河水府为帅,辖押八
万水兵,全仗此钯之力。今临凡世,保护吾师,逢山筑破虎狼窝,遇水掀翻龙蜃穴,
皆是此钯。”

国王闻得此言,十分欢喜心信。即命九嫔妃子:“将朕亲用的御酒,整瓶取来,
权与长老送行。”遂满斟一爵,奉与八戒道:“长老,这杯酒,聊引奉劳之意,待捉
得妖魔,救回小女,自有大宴相酬,千金重谢。”那呆子接杯在手,人物虽是粗鲁,
行事倒有斯文。对三藏唱个大喏道:“师父,这酒本该从你饮起;但君王赐我,不
敢违背,让老猪先吃了,助助兴头,好捉妖怪。”那呆子一饮而干,才斟一爵,递
与师父。三藏道:“我不饮酒,你兄弟们吃罢。”沙僧近前接了。八戒就足下生云,
直上空里。国王见了道:“猪长老又会腾云!”

呆子去了,沙僧将酒亦一饮而干,道:“师父!那黄袍怪拿住你时,我两个与他
交战,只战个手平;今二哥独去,恐战不过他。”三藏道:“正是,徒弟啊,你可去
与他帮帮功。”沙僧闻言,也纵云跳将起去。那国王慌了,扯住唐僧道:“长老,你
且陪寡人坐坐,也莫腾云去了。”唐僧道:“可怜,可怜!我半步儿也去不得!”此时
二人在殿上叙话不题。

却说那沙僧赶上八戒道:“哥哥,我来了。”八戒道:“兄弟,你来怎的?”沙
僧道:“师父叫我来帮帮功的。”八戒大喜道:“说得是,来得好。我两个努力齐心,
去捉那怪物;虽不怎的,也在此国扬扬姓名。”你看他:
祥光辞国界,氤氲瑞气出京城。
领王旨意来山洞,努力齐心捉怪灵。

他两个不多时,到了洞口,按落云头。八戒掣钯,往那波月洞的门上,尽力气
一筑,把他那石门筑了斗来大小的个窟窿。吓得那把门的小妖开门,看见是他两个,
急跑进去报道:“大王,不好了!那长嘴大耳的和尚,与那晦气脸的和尚,又来把门
都打破了!”那怪惊道:“这个还是猪八戒、沙和尚二人。我饶了他师父,怎么又敢
复来打我的门!”小妖道:“想是忘了甚么物件,来取的。”老怪咄的一声道:“胡缠!
忘了物件,就敢打上门来?必有缘故!”急整束了披挂,绰了钢刀,走出来问道:“那
和尚,我既饶了你师父,你怎么又敢来打上我门?”八戒道:“你这泼怪干得好事
儿!”老魔道:“甚么事?”八戒道:“你把宝象国三公主骗来洞内,倚强霸占为妻,
住了一十三载,也该还他了。我奉国王旨意,特来擒你。你快快进去,自家把绳子
绑缚出来,还免得老猪动手!”那老怪闻言,十分发怒。你看他屹迸迸,咬响钢牙;
滴溜溜,睁圆环眼;雄纠纠,举起刀来;赤淋淋,拦头便砍。八戒侧身躲过,使钉
钯劈面迎来;随后又有沙僧举宝杖赶上前齐打。这一场在山头上赌斗,比前不同。
真个是:

言差语错招人恼,意毒情伤怒气生。这魔王大钢刀,着头便砍;那八戒九齿钯,
对面来迎。沙悟净丢开宝杖,那魔王抵架神兵。一猛怪,二神僧,来来往往甚消停。
这个说:“你骗国理该死罪!”那个说:“你罗闲事报不平!”这个说:“你强婚公主
伤国体!”那个说:“不干你事莫闲争!”算来只为捎书故,致使僧魔两不宁。

他们在那山坡前,战经八九个回合,八戒渐渐不济将来,钉钯难举,气力不加。
你道如何这等战他不过?当时初相战斗,有那护法诸神,为唐僧在洞,暗助八戒、
沙僧,故仅得个平手;此时诸神都在宝象国护定唐僧,所以二人难敌。

那呆子道:“沙僧,你且上前来与他斗着,让老猪出恭来。”他就顾不得沙僧,
一溜往那蒿草薜萝、荆棘葛藤里,不分好歹,一顿钻进,那管刮破头皮,搠伤嘴脸,
一毂辘睡倒,再也不敢出来。但留半边耳朵,听着梆声。

那怪见八戒走了,就奔沙僧。沙僧措手不及,被怪一把抓住,捉进洞去。小妖
将沙僧四马攒蹄捆住。

毕竟不知端的性命如何,且听下回分解。