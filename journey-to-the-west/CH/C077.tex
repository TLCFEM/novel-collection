\chapter{群魔欺本性~一体拜真如}

且不言唐长老困苦。却说那三个魔头,齐心竭力,与大圣兄弟三人,在城东半
山内,努力争持。这一场,正是那铁刷帚刷铜锅,家家挺硬。好杀:

六般体相六般兵,六样形骸六样情。六恶六根缘六欲,六门六道赌输赢。三十
六宫春自在,六六形色恨有名。这一个金箍棒,千般解数;那一个方天戟,百样峥
嵘。八戒钉钯凶更猛,二怪长枪俊又能。小沙僧宝杖非凡,有心打死;老魔头钢刀
快利,举手无情。这三个是护卫真僧无敌将,那三个是乱法欺君泼野精。起初犹可,
向后弥凶。六枚都使升空法,云端里面各翻腾。一时间吐雾喷云天地暗,哮哮吼吼
只闻声。
他六个斗罢多时,渐渐天晚。却又是风雾漫漫,霎时间,就黑暗了。

原来八戒耳大,盖着眼皮,越发昏蒙;手脚慢,又遮架不住,拖着钯,败阵就
走,被老魔举刀砍去,几乎伤命;幸躲过头脑,被口刀削断几根鬃毛,赶上张开口
咬着领头,拿入城中,丢与小怪,捆在金銮殿。老妖又驾云,起在半空助力。沙和
尚见事不谐,虚幌着宝杖,顾本身回头便走,被二怪开鼻子,响一声,连手卷住,
拿到城里,也叫小妖捆在殿下。却又腾空去叫拿行者。行者见两个兄弟遭擒,他自
家独力难撑,正是“好手不敌双拳,双拳难敌四手”。他喊一声,把棍子隔开三个
妖魔的兵器,纵筋斗驾云走了。

三怪见行者驾筋斗时,即抖抖身,现了本象,开两,赶上大圣。你道他怎
能赶上?当时如行者闹天宫,十万天兵也拿他不住者,以他会驾筋斗云,一去有十
万八千里路,所以诸神不能赶上。这妖精一翅就有九万里,两就赶过了,所以
被他一把挝住,拿在手中,左右挣挫不得。欲思要走,莫能逃脱。即使变化法遁法,
又往来难行;变大些儿,他就放松了挝住;变小些儿,他又紧了挝住。复拿了径
回城内,放了手,下尘埃。吩咐群妖,也照八戒、沙僧捆在一处。那老魔、二魔
俱下来迎接。三个魔头,同上宝殿。噫!这一番倒不是捆住行者,分明是与他送行。

此时有二更时候,众怪一齐相见毕,把唐僧推下殿来。那长老于灯光前,忽见
三个徒弟都捆在地下,老师父伏于行者身边,哭道:“徒弟啊!常时逢难,你却在外
运用神通,到那里取救降魔;今番你亦遭擒,我贫僧怎么得命!”八戒、沙僧听见
师父这般苦楚,便也一齐放声痛哭。行者微微笑道:“师父放心,兄弟莫哭;凭他
怎的,决然无伤。等那老魔安静了,我们走路。”八戒道:“哥啊,又来捣鬼了!麻
绳捆住,松些儿还着水喷,想你这瘦人儿不觉,我这胖的遭瘟哩!不信,你看两膊
上。入肉已有二寸,如何脱身?”行者笑道:“莫说是麻绳捆的,就是碗粗的棕缆,
只也当秋风过耳,何足罕哉!”

师徒们正说处,只闻得那老魔道:“三贤弟有力量,有智谋,果成妙计,拿将
唐僧来了!”叫:“小的们,着五个打水,七个刷锅,十个烧火,二十个抬出铁笼来,
把那四个和尚蒸熟,我兄弟们受用,各散一块儿与小的们吃,也教他个个长生。”
八戒听见,战兢兢的道:“哥哥,你听。那妖精计较要蒸我们吃哩!”行者道:“不
要怕,等我看他是雏儿妖精,是把势妖精。”沙和尚哭道:“哥呀!且不要说宽话,
如今已与阎王隔壁哩,且讲甚么‘雏儿’、‘把势’。”说不了,又听得二怪说:“猪
八戒不好蒸。”八戒欢喜道:“阿弥陀佛,是那个积阴骘的,说我不好蒸?”三怪道:
“不好蒸,剥了皮蒸。”八戒慌了,厉声喊道:“不要剥皮!粗自粗,汤响就烂了!”
老怪道:“不好蒸的,安在底下一格。”行者笑道:“八戒莫怕,是‘雏儿’,不是‘把
势’。”沙僧道:“怎么认得?”行者道:“大凡蒸东西,都从上边起。不好蒸的,安
在上头一格,多烧把火,圆了气,就好了;若安在底下,一住了气,就烧半年也是
不得气上的。他说八戒不好蒸,安在底下,不是雏儿是甚的!”八戒道:“哥啊,依
你说,就活活的弄杀人了!他打紧见不上气,抬开了,把我翻转过来,再烧起火,
弄得我两边俱熟,中间不夹生了?”

正讲时,又见小妖来报:“汤滚了。”老怪传令叫抬。众妖一齐上手,将八戒抬
在底下一格,沙僧抬在二格。行者估着来抬他,他就脱身道:“此灯光前好做手脚!”
拔下一根毫毛,吹口仙气,叫声“变”!即变做一个行者,捆了魔绳;将真身出神,
跳出半空里,低头看着。那群妖那知真假,见人就抬。把个“假行者”抬在上三格;
才将唐僧揪翻倒捆住,抬上第四格。干柴架起,烈火气焰腾腾。大圣在云端里嗟叹
道:“我那八戒、沙僧,还捱得两滚;我那师父,只消一滚就烂。若不用法救他,
顷刻丧矣!”

好行者,在空中捻着诀,念一声“蓝净法界,乾元亨利贞”的咒语,拘唤得
北海龙王早至。只见那云端里一朵乌云,应声高叫道:“北海小龙敖顺叩头。”行者
道:“请起,请起!无事不敢相烦,今与唐师父到此,被毒魔拿住,上铁笼蒸哩。你
去与我护持护持,莫教蒸坏了。”龙王随即将身变作一阵冷风,吹入锅下,盘旋围
护,更没火气烧锅,他三人方不损命。

将有三更尽时,只闻得老魔发放道:“手下的,我等用计劳形,拿了唐僧四众;
又因相送辛苦,四昼夜未曾得睡。今已捆在笼里,料应难脱,汝等用心看守,着十
个小妖轮流烧火,让我们退宫,略略安寝。到五更天色将明,必然烂了,可安排下
蒜泥盐醋,请我们起来,空心受用。”众妖各各遵命。三个魔头,却各转寝宫而去。

行者在云端里,明明听着这等吩咐,却低下云头,不听见笼里人声。他想着:
“火气上腾,必然也热,他们怎么不怕,又无言语?哼!莫敢是蒸死了?等我近前
再听。”

好大圣,踏着云,摇身一变,变作一个黑苍蝇儿,钉在铁笼格外听时,只闻得
八戒在里面道:“晦气,晦气!不知是闷气蒸,又不知是出气蒸哩。”沙僧道:“二哥,
怎么叫做‘闷气’、‘出气’?”八戒道:“‘闷气蒸’是盖了笼头,‘出气蒸’不盖。”
三藏在浮上一层应声道:“徒弟,不曾盖。”八戒道:“造化!今夜还不得死,这是出
气蒸了。”行者听得他三人都说话,未曾伤命,便就飞了去,把个铁笼盖,轻轻儿
盖上。三藏慌了道:“徒弟,盖上了!”八戒道:“罢了!这个是闷气蒸,今夜必是死
了!”沙僧与长老嘤嘤的啼哭。八戒道:“且不要哭,这一会烧火的换了班了。”沙
僧道:“你怎么知道?”八戒道:“早先抬上来时,正合我意:我有些儿寒湿气的病,
要他腾腾。这会子反冷气上来了。咦!烧火的长官,添上些柴便怎的?要了你的哩!”

行者听见,忍不住暗笑道:“这个夯货!冷还好捱,若热就要伤命。再说两遭,
一定走了风了,快早救他。且住!要救他须是要现本相。假如现了,这十个烧火的
看见,一齐乱喊,惊动老怪,却不又费事?等我先送他个法儿。”忽想起:“我当初
做大圣时,曾在北天门与护国天王猜枚耍子,赢得他瞌睡虫儿,还有几个,送了他
罢。”即往腰间顺带里摸摸,还有十二个。”送他十个,还留两个做种。”即将虫儿
抛了去,散在十个小妖脸上,钻入鼻孔,渐渐打盹,都睡倒了。只有一个拿火叉的,
睡不稳,揉头搓脸,把鼻子左捏右捏,不住的打喷嚏。行者道:“这厮晓得勾当了,
我再与他个‘双掭灯’。”又将一个虫儿抛在他脸上。“两个虫儿,左进右出,右出
左进,谅有一个安住。”那小妖两三个大呵欠,把腰伸一伸,丢了火叉,也扑的睡
倒,再不翻身。

行者道:“这法儿真是妙而且灵!”即现原身,走近前,叫声:“师父。”唐僧听
见道:“悟空,救我啊!”沙僧道:“哥哥,你在外面叫哩?”行者道:“我不在外面,
好和你们在里边受罪?”八戒道:“哥啊,溜撒的溜了,我们都是顶缸的,在此受
闷气哩!”行者笑道:“呆子莫嚷,我来救你。”八戒道:“哥啊,救便要脱根救,莫
又要复笼蒸。”行者却揭开笼头,解了师父,将假变的毫毛,抖了一抖,收上身来;
又一层层放了沙僧,放了八戒。那呆子才解了。巴不得就要跑。行者道:“莫忙!莫
忙!”却又念声咒语,发放了龙神,才对八戒道:“我们这去到西天,还有高山峻岭。
师父没脚力难行,等我还将马来。”

你看他轻手轻脚,走到金銮殿下,见那些大小群妖俱睡熟了。却解了缰绳,更
不惊动。那马原是龙马,若是生人,飞踢两脚,便嘶几声。行者曾养过马,授弼马
温之官,又是自家一伙,所以不跳不叫。悄悄的牵来,束紧了肚带,扣备停当,请
师父上马。长老战兢兢的骑上,也就要走。行者道:“也且莫忙。我们西去还有国
王,须要关文,方才去得;不然,将甚执照?等我还去寻行李来。”唐僧道:“我记
得进门时,众怪将行李放在金殿左手下,担儿也在那一边。”行者道:“我晓得了。”
即抽身跳在宝殿寻时,忽见光彩飘摇。行者知是行李。怎么就知?以唐僧的锦袈
裟上有夜明珠,故此放光。——急到前,见担儿原封未动,连忙拿下去,付与沙僧
挑着。

八戒牵着马,他引了路,径奔正阳门。只听得梆铃乱响,门上有锁,锁上贴了
封皮。行者道:“这等防守,如何去得?”八戒道:“后门里去吧。”行者引路,径
奔后门:“后宰门外,也有梆铃之声,门上也有封锁,却怎生是好?我这一番,若不
为唐僧是个凡体,我三人不管怎的,也驾云弄风走了。只为唐僧未超三界外,见在
五行中,一身都是父母浊骨,所以不得升驾,难逃。”八戒道:“哥哥,不消商量,
我们到那没梆铃,不防卫处,撮着师父爬过墙去罢。”行者笑道:“这个不好,此时
无奈,撮他过去,到取经回来,你这呆子口敞,延地里就对人说,我们是爬墙头的
和尚了。”八戒道:“此时也顾不得行检,且逃命去罢。”行者也没奈何,只得依他。
到那净墙边,算计爬出。

噫!有这般事!也是三藏灾星未脱。那三个魔头,在宫中正睡,忽然惊觉,说走
了唐僧,一个个披衣忙起,急登宝殿。问曰:“唐僧蒸了几滚了?”那些烧火的小
妖已是有睡魔虫,都睡着了,就是打也莫想打得一个醒来。其余没执事的,惊醒几
个,冒冒失失的答应道:“七——七——七——七滚了。”急跑近锅边,只见笼格子
乱丢在地下,烧火的还都睡着,慌得又来报道:“大王,走
——走——走——走了!”三个魔头都下殿,近锅前仔细看时,果见那笼格子乱丢
在地下,汤锅尽冷,火脚俱无。那烧火的俱呼呼鼾睡如泥。慌得众怪一齐呐喊,都
叫:“快拿唐僧,快拿唐僧!”

这一片喊声振起,把些前前后后,大大小小妖精,都惊起来。刀枪簇拥,至正
阳门下,见那封锁不动,梆铃不绝,问外边巡夜的道:“唐僧从那里走了?”俱道:
“不曾走出人来。”急赶至后宰门,封锁梆铃,一如前门;复乱抢抢的,灯笼火把,
天通红,就如白日,却明明的照见他四众爬墙哩!老魔赶近,喝声“那里走!”那
长老唬得脚软筋麻,跌下墙来,被老魔拿住。二魔捉了沙僧,三魔擒倒八戒,众妖
抢了行李、白马,只是走了行者。那八戒口里哝哝的报怨行者道:“天杀的!我
说要救便脱根救,如今却又复笼蒸了!”

众魔把唐僧擒至殿上,却不蒸了。二怪吩咐把八戒绑在殿前檐柱上,三怪吩咐
把沙僧绑在殿后檐柱上;惟老魔把唐僧抱住不放。三怪道:“大哥,你抱住他怎的?
终不然就活吃?却也没些趣味。此物比不得那愚夫俗子,拿了可以当饭;此是上邦
稀奇之物,必须待天阴闲暇之时,拿他出来,整制精洁,猜枚行令,细吹细打的吃
方可。”老魔笑道:“贤弟之言虽当,但孙行者又要来偷哩。”三魔道:“我这皇宫里
面有一座锦香亭子,亭子内有一个铁柜。依着我,把唐僧藏在柜里,关了亭子,却
传出谣言,说唐僧已被我们夹生吃了。令小妖满城讲说;那行者必然来探听消息,
若听见这话,他必死心塌地而去。待三五日不来搅扰,却拿出来,慢慢受用,如何?”
老怪、二怪俱大喜道:“是,是,是!兄弟说得有理!”可怜把个唐僧连夜拿将进去,
藏在柜中,闭了亭子。传出谣言,满城里都乱讲不题。

却说行者自夜半顾不得唐僧,驾云走脱。径至狮驼洞里,一路棍,把那万数小
妖,尽情剿绝。急回来,东方日出。到城边,不敢叫战,正是“单丝不线,孤掌难
鸣”。他落下云头,摇身一变,变作个小妖儿,演入门里,大街小巷,缉访消息。
满城里俱道:“唐僧被大王夹生儿连夜吃了。”前前后后,都是这等说。行者着实心
焦,行至金銮殿前观看,那里边有许多精灵,都戴着皮金帽子,穿着黄布直身,手
拿着红漆棍,腰挂着象牙牌,一往一来,不住的乱走。行者暗想道:“此必是穿宫
的妖怪。就变做这个模样,进去打听打听。”

好大圣,果然变得一般无二,混入金门。正走处,只见八戒绑在殿前柱上哼哩。
行者近前,叫声:“悟能。”那呆子认得声音,道:“师兄,你来了?救我一救!”行
者道:“我救你。你可知师父在那里?”八戒道:“师父没了。昨夜被妖精夹生儿吃
了。”行者闻言,忽失声泪似泉涌。八戒道:“哥哥莫哭;我也是听得小妖乱讲,未
曾眼见。你休误了,再去寻问寻问。”这行者却才收泪,又往里面找寻。忽见沙僧
绑在后檐柱上,即近前摸着他胸脯子叫道:“悟净。”沙僧也识得声音,道:“师兄,
你变化进来了?救我,救我!”行者道:“救你容易。你可知师父在那里?”沙僧滴
泪道:“哥啊!师父被妖精等不得蒸,就夹生儿吃了!”

大圣听得两个言语相同,心如刀搅,泪似水流,急纵身望空跳起,且不救八戒、
沙僧,回至城东山上,按落云头,放声大哭。叫道:“师父啊!
恨我欺天困网罗,师来救我脱沉疴。
潜心笃志同参佛,努力修身共炼魔。
岂料今朝遭蜇害,不能保你上婆娑。
西方胜境无缘到,气散魂消怎奈何!”
行者凄凄惨惨的,自思自忖,以心问心道:“这都是我佛如来坐在那极乐之境,没
得事干,弄了那三藏之经!若果有心劝善,理当送上东土,却不是个万古流传?只是
舍不得送去,却教我等来取。怎知道苦历千山,今朝到此丧命!罢,罢,罢!老孙且
驾个筋斗云,去见如来,备言前事。若肯把经与我送上东土,一则传扬善果,二则
了我等心愿;若不肯与我,教他把松箍儿咒念念,退下这个箍子,交还与他,老孙
还归本洞,称王道寡,耍子儿去吧。”

好大圣,急翻身驾起筋斗云,径投天竺。那里消一个时辰,早望见灵山不远。
须臾间,按落云头,真至鹫峰之下。忽抬头,见四大金刚挡住道:“那里走?”行
者施礼道:“有事要见如来。”当头又有昆仑山金霞岭不坏尊王永住金刚喝道:“这
泼猴甚是粗狂!前者大困牛魔,我等为汝努力,今日面见,全不为礼!有事且待先奏,
奉召方行。这里比南天门不同,教你进去出来,两边乱走!咄,还不靠开!”那大圣
正是烦恼处,又遭此抢白,气得哮吼如雷,忍不住大呼小叫,早惊动如来。

如来佛祖正端坐在九品宝莲台上,与十八尊轮世的阿罗汉讲经,即开口道:“孙
悟空来了,汝等出去接待接待。”大众阿罗,遵佛旨,两路幢幡宝盖,即出山门应
声道:“孙大圣,如来有旨相唤哩。”那山门口四大金刚却才闪开路,让行者前进。

众阿罗引至宝莲台下,见如来倒身下拜,两泪悲啼。如来道:“悟空,有何事
这等悲啼?”行者道:“弟子屡蒙教训之恩,托庇在佛爷爷之门下,自归正果,保
护唐僧,拜为师范,一路上苦不可言!今至狮陀山狮驼洞狮驼城,有三个毒魔,乃
狮王、象王、大鹏,把我师父捉将去,连弟子一概遭,都捆在蒸笼里,受汤火之
灾。幸弟子脱逃,唤龙王救免。是夜偷出师等,不料灾星难脱,复又擒回。及至天
明,入城打听,叵耐那魔十分狠毒,万样骁勇:把师父连夜夹生吃了,如今骨肉无
存。又况师弟悟能、悟净,见绑在那厢,不久性命亦皆倾矣。弟子没及奈何,特地
到此参拜如来。望大慈悲,将松箍咒儿念念,退下我这头上箍儿,交还如来,放我
弟子回花果山宽闲耍子去吧!”说未了,泪如泉涌,悲声不绝。如来笑道:“悟空少
得烦恼。那妖精神通广大,你胜不得他,所以这等心痛。”行者跪在下面,捶着胸
膛道:“不瞒如来说,弟子当年闹天宫,称大圣,自为人以来,不曾吃亏,今番却
遭这毒魔之手!”

如来闻言道:“你且休恨。那妖精我认得他。”行者猛然失声道:“如来!我听见
人讲说,那妖精与你有亲哩。”如来道:“这个刁猢狲!怎么个妖精与我有亲?”行
者笑道:“不与你有亲,如何认得?”如来道:“我慧眼观之,故此认得。那老怪与
二怪有主。”叫:“阿傩、迦叶,来!你两个分头驾云,去五台山、峨眉山宣文殊、
普贤来见。”二尊者即奉旨而去。

如来道:“这是老魔、二怪之主。但那三怪,说将起来,也是与我有些亲处。”
行者道:“亲是父党?母党?”如来道:“自那混沌分时,天开于子,地辟于丑,人
生于寅,天地再交合,万物尽皆生。万物有走兽飞禽。走兽以麒麟为之长,飞禽以
凤凰为之长。那凤凰又得交合之气,育生孔雀、大鹏。孔雀出世之时,最恶,能吃
人,四十五里路,把人一口吸之。我在雪山顶上,修成丈六金身,早被他也把我吸
下肚去。我欲从他便门而出,恐污真身,是我剖开他脊背,跨上灵山。欲伤他命,
当被诸佛劝解:伤孔雀如伤我母。故此留他在灵山会上,封他做佛母孔雀大明王菩
萨。大鹏与他是一母所生,故此有些亲处。”行者闻言笑道:“如来,若这般比论,
你还是妖精的外甥哩。”如来道:“那怪须是我去,方可收得。”行者叩头,启上如
来:“千万望挪玉一降!”

如来即下莲台,同诸佛众,径出山门。又见阿傩、迦叶,引文殊、普贤来见。
二菩萨对佛礼拜。如来道:“菩萨之兽,下山多少时了?”文殊道:“七日了。”如
来道:“山中方七日,世上几千年。不知在那厢伤了多少生灵,快随我收他去。”二
菩萨相随左右,同众飞空。只见那:
满天缥缈瑞云分,我佛慈悲降法门。
明示开天生物理,细言辟地化身文。
面前五百阿罗汉,脑后三千揭谛神。
迦叶阿傩随左右,普文菩萨殄妖氛。
大圣有此人情,请得佛祖与众前来,不多时,早望见城池。行者报道:“如来,那
放黑气的乃是狮驼国也。”如来道:“你先下去,到那城中与妖精交战,许败不许胜。
败上来,我自收他。”

大圣即按云头,径至城上,脚踏着垛儿骂道:“泼孽畜,快出来与老孙交战!”
慌得那城楼上小妖急跳下城中报道:“大王,孙行者在城上叫战哩。”老妖道:“这
猴儿两三日不来,今朝却又叫战,莫不是请了些救兵来耶?”三怪道:“怕他怎的!
我们都去看来。”三个魔头,各持兵器,赶上城来;见了行者,更不打话,举兵器
一齐乱刺。行者轮铁棒掣手相迎。斗经七八回合,行者佯输而走。

那妖王喊声大振,叫道:“那里走!”大圣筋斗一纵,跳上半空,三个精即驾云
来赶。行者将身一闪,藏在佛爷爷金光影里,全然不见。只见那过去、未来、见在
的三尊佛像与五百阿罗汉、三千揭谛神,布散左右,把那三个妖王围住,水泄不通。
老魔慌了手脚,叫道:“兄弟,不好了!那猴子真是个地里鬼!那里请得个主人公来
也!”三魔道:“大哥休得悚惧。我们一齐上前,使枪刀搠倒如来,夺他那雷音宝刹!”
这魔头不识起倒,真个举刀上前乱砍。却被文殊、普贤,念动真言,喝道:“这孽
畜还不皈正,更待怎生!”唬得老怪、二怪,不敢撑持,丢了兵器,打个滚,现了
本相。二菩萨将莲花台抛在那怪的脊背上,飞身跨坐,二怪遂泯耳皈依。

二菩萨既收了青狮、白象,只有那第三个妖魔不伏。腾开翅,丢了方天戟,扶
摇直上,轮利爪要刁捉猴王。原来大圣藏在光中,他怎敢近,如来情知此意,即闪
金光,把那鹊巢贯顶之头,迎风一幌,变做鲜红的一块血肉。妖精轮利爪刁他一下,
被佛爷把手往上一指,那妖翅膊上就了筋,飞不去,只在佛顶上,不能远遁,现了
本相,乃是一个大鹏金翅雕。即开口对佛应声叫道:“如来,你怎么使大法力困住
我也?”如来道:“你在此处多生孽障,跟我去,有进益之功。”妖精道:“你那里
持斋把素,极贫极苦;我这里吃人肉,受用无穷;你若饿坏了我,你有罪愆。”如
来道:“我管四大部洲,无数众生瞻仰,凡做好事,我教他先祭汝口。”那大鹏欲脱
难脱,要走怎走,是以没奈何,只得皈依。

行者方才转出,向如来叩头道:“佛爷,你今收了妖精,除了大害,只是没了
我师父也。”大鹏咬着牙恨道:“泼猴头,寻这等狠人困我!你那老和尚几曾吃他?如
今在那锦香亭铁柜里不是?”行者闻言,忙叩头谢了佛祖。佛祖不敢松放了大鹏,
也只教他在光焰上做个护法,引众回云,径归宝刹。

行者却按落云头,直入城里。那城里一个小妖儿也没有了。正是“蛇无头而不
行,鸟无翅而不飞”。他见佛祖收了妖王,各自逃生而去。行者才解救了八戒、沙
僧,寻着行李,马匹,与他二人说:“师父不曾吃。都跟我来。”

引他两个径入内院,找着锦香亭,打开门看,内有一个铁柜,只听得三藏有啼
哭之声。沙僧使降妖杖打开铁锁,揭开柜盖,叫声“师父”。三藏见了,放声大哭
道:“徒弟啊!怎生降得妖魔?如何得到此寻着我也?”行者把上项事,从头至尾,
细陈了一遍。三藏感谢不尽。师徒们在那宫殿里寻了些米粮,安排些茶饭,饱吃一
餐,收拾出城,找大路投西而去。正是:
真经必得真人取,意嚷心劳总是虚。

毕竟这一去,不知几时得面如来,且听下回分解。