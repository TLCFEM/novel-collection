\chapter{木母助威征怪物~金公施法灭妖邪}

话说孙大圣牵着马,挑着担,满山头寻叫师父,忽见猪八戒气的跑将来道:
“哥哥,你喊怎的?”行者道:“师父不见了,你可曾看见?”八戒道:“我原来只
跟唐僧做和尚的,你又捉弄我,教做甚么将军!我舍着命,与那妖精战了一会,得
命回来。师父是你与沙僧看着的,反来问我?”行者道:“兄弟,我不怪你。你不
知怎么眼花了,把妖精放回来拿师父。我去打那妖精,教沙和尚看着师父的,如今
连沙和尚也不见了。”八戒笑道:“想是沙和尚带师父那里出恭去了。”说不了,只
见沙僧来到。行者问道:“沙僧,师父那里去了?”沙僧道:“你两个眼都昏了,把
妖精放将来拿师父,老沙去打那妖精的,师父自家在马上坐来。”行者气得暴跳道:
“中他计了,中他计了!”沙僧道:“中他甚么计?”行者道:“这是‘分瓣梅花计’,
把我弟兄们调开,他劈心里捞了师父去了。天,天,天!却怎么好!”止不住腮边泪
滴。八戒道:“不要哭,一哭就脓包了。横竖不远,只在这座山上,我们寻去来。”

三人没计奈何,只得入山找寻。行了有二十里远近,只见那悬崖之下,有一座
洞府:

削峰掩映,怪石嵯峨。奇花瑶草馨香,红杏碧桃艳丽。崖前古树,霜皮溜雨四
十围;门外苍松,黛色参天二千尺。双双野鹤,常来洞口舞清风;对对山禽,每向
枝头啼白昼。簇簇黄藤如挂索,行行烟柳似垂金。方塘积水,深穴依山:方塘积
水,隐穷鳞未变的蛟龙;深穴依山,住多年吃人的老怪。果然不亚神仙境,真是藏
风聚气巢。
行者见了,两三步,跳到门前看处,那石门紧闭,门上横安着一块石版,石版上有
八个大字,乃“隐雾山折岳连环洞”。行者道:“八戒,动手啊!此间乃妖精住处,
师父必在他家也。”那呆子仗势行凶,举钉钯尽力筑将去,把他那石头门筑了一个
大窟窿,叫道:“妖怪!快送出我师父来,免得钉钯筑倒门,一家子都是了帐!”守
门的小妖,急急跑入报道:“大王,闯出祸来了!”老怪道:“有甚祸?”小妖道:“门
前有人把门打破,嚷道要师父哩!”老怪大惊道:“不知是那个寻将来也?”先锋道:
“莫怕,等我出去看看。”

那小妖奔至前门,从那打破的窟窿处,歪着头,往外张,见是个长嘴大耳朵,
即回头高叫:“大王莫怕他!这个是猪八戒,没甚本事,不敢无理。他若无理,开了
门,拿他进来凑蒸。怕便只怕那毛脸雷公嘴的和尚。”八戒在外边听见道:“哥啊,
他不怕我,只怕你哩。师父定在他家了。你快上前。”行者骂道:“泼孽畜!你孙外
公在这里!送我师父出来,饶你命罢!”先锋道:“大王,不好了!孙行者也寻将来了!”
老怪报怨道:“都是你定的甚么‘分瓣分瓣’,却惹得祸事临门!怎生结果?”先锋
道:“大王放心,且休埋怨。我记得孙行者是个宽洪海量的猴头,虽则他神通广大,
却好奉承。我们拿个假人头出去哄他一哄,奉承他几句,只说他师父是我们吃了。
若还哄得他去了,唐僧还是我们受用;哄不过再作理会。”老怪道:“那里得个假人
头?”先锋道:“等我做一个儿看。”

好妖怪,将一把钢刀斧,把柳树根砍做个人头模样,喷上些人血,糊糊涂涂
的,着一个小怪,使漆盘儿拿至门下,叫道:“大圣爷爷,息怒容禀。”孙行者果好
奉承,听见叫声大圣爷爷,便就止住八戒:“且莫动手,看他有甚话说。”拿盘的小
怪道:“你师父被我大王拿进洞来,洞里小妖村顽,不识好歹,这个来吞,那个来
啃,抓的抓,咬的咬,把你师父吃了,只剩了一个头在这里也。”行者道:“既吃了
便罢,只拿出人头来,我看是真是假。”

那小怪从门窟里抛出那个头来。猪八戒见了就哭道:“可怜啊!那们个师父进
去,弄做这们个师父出来也!”行者道:“呆子,你且认认是真是假。就哭!”八戒
道:“不羞!人头有个真假的?”行者道:“这是个假人头。”八戒道:“怎认得是
假?”行者道:“真人头抛出来,扑搭不响;假人头抛得像梆子声。你不信,等我
抛了你听。”拿起来往石头上一掼,当的一声响亮。沙和尚道:“哥哥,响哩!”行
者道:“响便是个假的。我教他现出本相来你看。”急掣金箍棒,扑的一下,打破了。
八戒看时,乃是个柳树根。呆子忍不住骂起来道:“我把你这伙毛团!你将我师父藏
在洞里,拿个柳树根哄你猪祖宗,莫成我师父是柳树精变的!”

慌得那拿盘的小怪,战兢兢跑去报道:“难,难,难!难,难,难!”老妖道:“怎
么有许多难?”小妖道:“猪八戒与沙和尚倒哄过了,孙行者却是个‘贩古董的—
—识货,识货。’他就认得是个假人头。如今得个真人头与他,或者他就去了。”老
怪道:“怎么得个真人头?我们那剥皮亭内有吃不了的人头选一个来。”众妖即至亭
内拣了个新鲜的头,教啃净头皮,滑塔塔的,还使盘儿拿出,叫:“大圣爷爷,先
前委是个假头。这个真正是唐老爷的头,我大王留了镇宅子的,今特献出来也。”
扑通的把个人头又从门窟里抛出,血滴滴的乱滚。

孙行者认得是个真人头,没奈何就哭。八戒、沙僧也一齐放声大哭。八戒噙着
泪道:“哥哥,且莫哭。天气不是好天气,恐一时弄臭了。等我拿将去,乘生气埋
下再哭。”行者道:“也说得是。”那呆子不嫌秽污,把个头抱在怀里,跑上山崖。
向阳处,寻了个藏风聚气的所在,取钉钯筑了一个坑,把头埋了;又筑起一个坟冢。
才叫沙僧:“你与哥哥哭着,等我去寻些甚么供养供养。”他就走向涧边,攀几根大
柳枝,拾几块鹅卵石,回至坟前,把柳枝儿插在左右,鹅卵石堆在面前。行者问道:
“这是怎么说?”八戒道:“这柳枝权为松柏,与师父遮遮坟顶;这石子权当点心,
与师父供养供养。”行者喝道:“夯货!人已死了,还将石子儿供他!”八戒道:“表
表生人意,权为孝道心。”行者道:“且休胡弄!教沙僧在此:一则庐墓,二则看守
行李、马匹。我和你去打破他的洞府,拿住妖魔,碎尸万段,与师父报仇去来。”
沙和尚滴泪道:“大哥言之极当。你两个着意,我在此处看守。”

好八戒,即脱了皂锦直裰,束一束着体小衣,举钯随着行者。二人努力向前,
不容分辨,径自把他石门打破,喊声振天,叫道:“还我活唐僧来耶!”那洞里大小
群妖,一个个魂飞魄散,都报怨先锋的不是。老妖问先锋道:“这些和尚打进门来,
却怎处治?”先锋道:“古人说得好:‘手插鱼篮——避不得腥。’一不做,二不休;
左右帅领家兵杀那和尚去来!”老怪闻言,无计可奈,真个传令,叫:“小的们,各
要齐心,将精锐器械跟我去出征。”果然一齐呐喊,杀出洞门。

这大圣与八戒,急退几步,到那山场平处,抵住群妖,喝道:“那个是出名的
头儿?那个是拿我师父的妖怪?”那群妖扎下营盘,将一面锦绣花旗闪一闪,老怪
持铁杵,应声高呼道:“那泼和尚,你认不得我?我乃南山大王,数百年放荡于此。
你唐僧已是我拿吃了,你敢如何?”行者骂道:“这个大胆的毛团!你能有多少的年
纪,敢称‘南山’二字?李老君乃开天辟地之祖,尚坐于太清之右;佛如来是治世
之尊,还坐于大鹏之下;孔圣人是儒教之尊,亦仅呼为‘夫子’。你这个孽畜,敢
称甚么南山大王,数百年之放荡!不要走!吃你外公老爷一棒!”那妖精侧身闪过,
使杵抵住铁棒,睁圆眼问道:“你这嘴脸像个猴儿模样,敢将许多言语压我!你有甚
么手段,在吾门下猖狂?”行者笑道:“我把你个无名的孽畜!是也不知老孙!你站
住,硬着胆,且听我说:

祖居东胜大神洲,天地包含几万秋。花果山头仙石卵,卵开产化我根苗。生来
不比凡胎类,圣体原从日月俦。本性自修非小可,天姿颖悟大丹头。官封大圣居云
府,倚势行凶斗斗牛。十万神兵难近我,满天星宿易为收。名扬宇宙方方晓,智贯
乾坤处处留。今幸皈依从释教,扶持长老向西游。逢山开路无人阻,遇水支桥有怪
愁。林内施威擒虎豹,崖前覆手捉貔貅。东方果正来西域,那个妖邪敢出头!孽畜
伤师真可恨,管教时下命将休!”

那怪闻言,又惊又恨。咬着牙,跳近前来,使铁杵望行者就打。行者轻轻的用
棒架住,还要与他讲话,那八戒忍不住,掣钯乱筑那怪的先锋。先锋帅众齐来。这
一场在山中平地处混战,真是好杀:

东土天邦上国僧,西方极乐取真经。南山大豹喷风雾,路阻深山独显能。施巧
计,弄乖伶,无知误捉大唐僧。相逢行者神通广,更遭八戒有声名。群妖混战山平
处,尘土纷飞天不清。那阵上小妖呼哮,枪刀乱举;这壁厢神僧叱喝,钯棒齐兴。
大圣英雄无敌手,悟能精壮喜神生。南禺老怪,部下先锋,都为唐僧一块肉,致令
舍死又亡生。这两个因师性命成仇隙,那两个为要唐僧忒恶情。往来斗经多半会,
冲冲撞撞没输赢。

孙大圣见那些小妖勇猛,连打不退。即使个分身法,把毫毛拔下一把,嚼在口
中,喷出去,叫声:“变!”都变做本身模样,一个使一条金箍棒,从前边往里打进。
那一二百个小妖,顾前不能顾后,遮左不能遮右,一个个各自逃生,败走归洞。这
行者与八戒,从阵里往外杀来。可怜那些不识俊的妖精,搪着钯,九孔血出;挽着
棒,骨肉如泥!唬得那南山大王滚风生雾,得命逃回。那先锋不能变化,早被行者
一棒打倒,现出本相,乃是个铁背苍狼怪。八戒上前扯着脚,翻过来看了道:“这
厮从小儿也不知偷了人家多少猪牙子、羊羔儿吃了!”行者将身一抖,收上毫毛道:
“呆子!不可迟慢!快赶老怪,讨师父的命去来!”八戒回头,就不见那些小行者,
道:“哥哥的法相儿都去了!”行者道:“我已收来也。”八戒道:“妙啊!妙啊!”两
个喜喜欢欢,得胜而回。

却说那老怪逃了命回洞,吩咐小妖搬石块,挑土,把前门堵了。那些得命的小
妖,一个个战兢兢的,把门都堵了,再不敢出头。这行者引八戒,赶至门首喝,
内无人答应。八戒使钯筑时,莫想得动。行者知之,道:“八戒,莫费气力,他把
门已堵了。”八戒道:“堵了门,师仇怎报?”行者道:“且回上墓前,看看沙僧去。”

二人复至本处,见沙僧还哭哩。八戒越发伤悲,丢了钯,伏在坟上,手扑着土
哭道:“苦命的师父啊!远乡的师父啊!那里再得见你耶!”行者道:“兄弟,且莫悲
切。这妖精把前门堵了,一定有个后门出入。你两个只在此间,等我再去寻看。”
八戒滴泪道:“哥啊!仔细着!莫连你也捞去了,我们不好哭得:哭一声师父,哭一
声师兄,就要哭得乱了。”行者道:“没事,我自有手段!”

好大圣,收了棒,束束裙,拽开步,转过山坡,忽听得潺潺水响。且回头看处,
原来是涧中水响,上溜头冲泄下来。又见涧那边有座门儿,门左边有一个出水的暗
沟,沟中流出红水来。他道:“不消讲!那就是后门了。若要是原嘴脸,恐有小妖开
门看见认得,等我变作个水蛇儿过去。……且住!变水蛇恐师父的阴灵儿知道,怪
我出家人变蛇缠长;变作个小螃蟹儿过去罢。……也不好,恐师父怪我出家人脚多。”
即做一个水老鼠,“飕”的一声撺过去,从那出水的沟中,钻至里面天井中。探着
头儿观看,只见那向阳处有几个小妖,拿些人肉巴子,一块块的理着晒哩。行者道:
“我的儿啊!那想是师父的肉,吃不了,晒干巴子防天阴的。我要现本相,赶上前,
一棍子打杀,显得我有勇无谋;且再变化进去,寻那老怪,看是何如。”跳出沟,
摇身又一变,变做个有翅的蚂蚁儿。真个是:
力微身小号玄驹,日久藏修有翅飞。
闲渡桥边排阵势,喜来床下斗仙机。
善知雨至常封穴,垒积尘多遂作灰。
巧巧轻轻能爽利,几番不觉过柴扉。

他展开翅,无声无影,一直飞入中堂。只见那老怪烦烦恼恼正坐,有一个小妖,
从后面跳将来报道:“大王万千之喜!”老妖道:“喜从何来?”小妖道:“我才在后
门外涧头上探看,忽听得有人大哭。即上峰头望望,原来是猪八戒、孙行者、沙
和尚在那里拜坟痛哭。想是把那个人头认做唐僧的头葬下,扛作坟墓哭哩。”行者
在暗中听说,心内欢喜道:“若出此言,我师父还藏在那里,未曾吃哩。等我再去
寻寻,看死活如何,再与他说话。”

好大圣,飞在中堂,东张西看,见旁边有个小门儿,关得甚紧;即从门缝儿里
钻去看时,原是个大园子,隐隐的听得悲声,径飞入深处,但见一丛大树,树底下
绑着两个人,一个正是唐僧。行者见了,心痒难挠,忍不住,现了本相,近前叫声:
“师父。”那长老认得,滴泪道:“悟空,你来了?快救我一救!悟空,悟空!”行者
道:“师父莫只管叫名字:面前有人,怕走了风汛。你既有命,我可救得你。那怪
只说已将你吃了,拿个假人头哄我,我们与他恨苦相持。师父放心,且再熬熬儿,
等我把那妖精弄倒,方好来解救。”

大圣念声咒语,却又摇身还变做个蚂蚁儿,复入中堂,丁在正梁之上。只见那
些未伤命的小妖,簇簇攒攒,纷纷嚷嚷。内中忽跳出一个小妖,告道:“大王,他
们见堵了门,攻打不开,死心蹋地,舍了唐僧,将假人头弄做个坟墓。今日哭一日,
明日再哭一日,后日复了三,好道回去。打听得他们散了啊,把唐僧拿出来,碎
碎剁,把些大料煎了,香喷喷的大家吃一块儿,也得个延年长寿。”又一个小妖拍
着手道:“莫说,莫说,还是蒸了吃的有味。”又一个说:“煮了吃,还省柴。”又一
个道:“他本是个稀奇之物,还着些盐儿腌腌,吃得长久。”

行者在那梁中听见,心中大怒道:“我师父与你有甚毒情,这般算计吃他!”即
将毫毛拔了一把,口中嚼啐,轻轻吹出,暗念咒语,都教变做瞌睡虫儿,往那众妖
脸上抛去。一个个钻入鼻中,小妖渐渐打盹。不一时,都睡倒了。只有那个老妖睡
不稳,他两只手揉头搓脸,不住的打涕喷,捏鼻子。行者道:“莫是他晓得了?与他
个双掭灯!”又拔一根毫毛,依母儿做了,抛在他脸上,钻于鼻孔内。两个虫儿,
一个从左进,一个从右入。那老妖起来,伸伸腰,打两个呵欠,呼呼的也睡倒了。

行者暗喜,才跳下来,现出本相。耳朵里取出棒来,幌一幌,有鸭蛋粗细,当
的一声,把旁门打破,跑至后园,高叫:“师父!”长老道:“徒弟,快来解解绳儿;
绑坏我了!”行者道:“师父不要忙,等我打杀妖精,再来解你。”急抽身跑至中堂。
正举棍要打,又滞住手道:“不好!等解了师父来打。”复至园中,又思量道:“等打
了来救。”如此者两三番,却才跳跳舞舞的到园里。长老见了,悲中作喜道:“猴儿,
想是看见我不曾伤命,所以欢喜得没是处,故这等作跳舞也?”行者才至前,将绳
解了,挽着师父就走。又听得对面树上绑的人叫道:“老爷舍大慈悲,也救我一命!”
长老立定身,叫:“悟空,那个人也解他一解。”行者道:“他是甚么人?”长老道:
“他比我先拿进一日。他是个樵子,说有母亲年老,甚是思想,倒是个尽孝的。一
发连他都救了罢。”

行者依言,也解了绳索,一同带出后门,上石崖,过了陡涧。长老谢道:“贤
徒,亏你救了他与我命!悟能、悟净都在何处?”行者道:“他两个都在那里哭你哩。
你可叫他一声。”长老果厉声高叫道:“八戒!八戒!”那呆子哭得昏头昏脑的,揩揩
鼻涕眼泪道:“沙和尚,师父回家来显魂哩!在那里叫我们不是?”行者上前,喝了
一声道:“夯货!显甚么魂?这不是师父来了?”那沙僧抬头见了,忙忙跪在面前道:
“师父,你受了多少苦啊!哥哥怎生救得你来也?”行者把上项事说了一遍。

八戒闻言,咬牙恨齿,忍不住举起钯把那坟冢,一顿筑倒,掘出那人头,一顿
筑得稀烂。唐僧道:“你筑他为何?”八戒道:“师父啊,不知他是那家的亡人,教
我朝着他哭!”长老道:“亏他救了我命哩。你兄弟们打上他门,嚷着要我,想是拿
他来搪塞;不然啊,就杀了我也。还把他埋一埋,见我们出家人之意。”那呆子听
长老此言,遂将一包稀烂骨肉埋下,也扛起个坟墓。

行者却笑道:“师父,你请略坐坐,等我剿除去来。”即又跳下石崖,过涧入洞,
把那绑唐僧与樵子的绳索拿入中堂,那老妖还睡着了,即将他四马攒蹄捆倒,使金
箍棒掬起来,握在肩上,径出后门。猪八戒远远的望见道:“哥哥好干这握头事!再
寻一个儿趁头挑着不好?”行者到跟前放下,八戒举钯就筑。行者道:“且住!洞里
还有小妖怪,未拿哩。”八戒道:“哥啊,有便带我进去打他。”行者道:“打又费工
夫了,不若寻些柴,教他断根罢。”

那樵子闻言,即引八戒去东凹里寻了些破梢竹、败叶松、空心柳、断根藤、黄
蒿、老荻、芦苇、干桑,挑了若干,送入后门里。行者点上火,八戒两耳扇起风。
那大圣将身跳上,抖一抖,收了瞌睡虫的毫毛。那些小妖及醒来,烟火齐着。可怜!
莫想有半个得命。连洞府烧得精空,却回见师父。师父听见老妖方醒声唤,便叫:
“徒弟,妖精醒了。”八戒上前一钯,把老怪筑死,现出本相,原来是个艾叶花皮
豹子精。行者道:“花皮会吃老虎,如今又会变人。这顿打死,才绝了后患也!”长
老谢之不尽,攀鞍上马。那樵子道:“老爷,向西南去不远,就是舍下。请老爷到
舍,见见家母,叩谢老爷活命之恩,送老爷上路。”

长老欣然,遂不骑马,与樵子并四众同行。向西南迤逦前来,不多路,果见那:
石径重漫苔藓,柴门篷络藤花。
四面山光连接,一林鸟雀喧哗。
密密松篁交翠,纷纷异卉奇葩。
地僻云深之处,竹篱茅舍人家。
远见一个老妪,倚着柴扉,眼泪汪汪的,儿天儿地的痛哭。这樵子看见是他母亲,
丢了长老,急忙忙先跑到柴扉前,跪下叫道:“母亲!儿来也!”老妪一把抱住道:“儿
啊!你这几日不来家,我只说是山主拿你去,害了性命,是我心疼难忍。你既不曾
被害,何以今日才来?你绳担、柯斧俱在何处?”樵子叩头道:“母亲,儿已被山主
拿去,绑在树上,实是难得性命。幸亏这几位老爷!这老爷是东土唐朝往西天取经
的罗汉。那老爷倒也被山主拿去绑在树上。他那三位徒弟老爷,神通广大,把山主
一顿打死,却是个艾叶花皮豹子精;概众小妖,俱尽烧死,却将那老老爷解下救出,
连孩儿都解救出来。此诚天高地厚之恩!不是他们,孩儿也死无疑了。如今山上太
平,孩儿彻夜行走,也无事矣。”

那老妪听言,一步一拜,拜接长老四众,都入柴扉茅舍中坐下。娘儿两个磕头
称谢不尽,慌慌忙忙的,安排些素斋酬谢。八戒道:“樵哥,我见你府上也寒薄,
只可将就一饭,切莫费心大摆布。”樵子道:“不瞒老爷说。我这山间实是寒薄,没
甚么香蕈、蘑菰、川椒、大料,只是几品野菜奉献老爷,权表寸心。”八戒笑道:“聒
噪,聒噪。放快些儿就是。我们肚中饥了。”樵子道:“就有,就有!”果然不多时,
展抹桌凳,摆将上来。果是几盘野菜。但见那:

嫩焯黄花菜,酸白鼓丁。浮蔷马齿苋,江荠雁肠英。燕子不来香且嫩,芽儿
拳小脆还青。烂煮马蓝头,白狗脚迹。猫耳朵,野落荜,灰条熟烂能中吃;剪刀
股,牛塘利,倒灌窝螺操帚荠。碎米荠,莴菜荠,几品青香又滑腻。油炒乌英花,
菱科甚可夸;蒲根菜并茭儿菜,四般近水实清华。看麦娘,娇且
佳;破破纳,不穿他,苦麻台下藩篱架。雀儿绵单,猢狲脚迹,油灼灼煎来只好吃。
斜蒿青蒿抱娘蒿,灯娥儿飞上板荞荞。羊耳秃,枸杞头,加上乌蓝不用油。几般野
菜一餐饭,樵子虔心为谢酬。

师徒们饱餐一顿,收拾起程。那樵子不敢久留,请母亲出来,再拜,再谢。樵
子只是磕头,取了一条枣木棍,结束了衣裙,出门相送。沙僧牵马,八戒挑担,行
者紧随左右,长老在马上拱手道:“樵哥,烦先引路,到大路上相别。”一齐登高下
坂,转涧寻坡。长老在马上思量道:“徒弟啊!
自从别主来西域,递递迢迢去路遥。
水水山山灾不脱,妖妖怪怪命难逃。
心心只为经三藏,念念仍求上九霄。
碌碌劳劳何日了,几时行满转唐朝!”
樵子闻言道:“老爷切莫忧思。这条大路,向西方不满千里,就是天竺国,极乐之
乡也。”长老闻言,翻身下马道:“有劳远涉。既是大路,请樵哥回府,多多拜上令
堂老安人:适间厚扰盛斋,贫僧无甚相谢,只是早晚诵经,保佑你母子平安,百年
长寿。”那樵子喏喏相辞,复回本路。师徒遂一直投西。正是:
降怪解冤离苦厄,受恩上路用心行。

毕竟不知还有几日得到西天,且听下回分解。