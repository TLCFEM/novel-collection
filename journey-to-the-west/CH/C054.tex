\chapter{法性西来逢女国~心猿定计脱烟花}

话说三藏师徒别了村舍人家,依路西进,不上三四十里,早到西梁国界。唐僧
在马上指道:“悟空,前面城池相近,市井上人语喧哗,想是西梁女国。汝等须要
仔细,谨慎规矩,切休放荡情怀,紊乱法门教旨。”三人闻言,谨遵严命。

言未尽,却至东关厢街口。那里人都是长裙短袄,粉面油头。不分老少,尽是
妇女。正在两街上做买做卖,忽见他四众来时,一齐都鼓掌呵呵,整容欢笑道:“人
种来了!人种来了!”慌得那三藏勒马难行。须臾间就塞满街道,惟闻笑话。八戒口
里乱嚷道:“我是个销猪!我是个销猪!”行者道:“呆子,莫胡谈。拿出旧嘴脸便是。”
八戒真个把头摇上两摇,竖起一双蒲扇耳,扭动莲蓬吊搭唇,发一声喊,把那些妇
女们唬得跌跌爬爬。有诗为证,诗曰:
圣僧拜佛到西梁,国内阴世少阳。
农士工商皆女辈,渔樵耕牧尽红妆。
娇娥满路呼人种,幼妇盈街接粉郎。
不是悟能施丑相,烟花围困苦难当!

遂此众皆恐惧,不敢上前。一个个都捻手矬腰,摇头咬指,战战兢兢,排塞街
傍路下,都看唐僧。孙大圣却也弄出丑相开路,沙僧也装虎维持。八戒采着马,
掬着嘴,摆着耳朵。一行前进,又见那市井上房屋齐整,铺面轩昂,一般有卖盐卖
米,酒肆茶房;鼓角楼台通货殖,旗亭候馆挂帘栊。

师徒们转湾抹角,忽见有一女官侍立街下,高声叫道:“远来的使客,不可擅
入城门。请投馆驿注名上簿,待下官执名奏驾,验引放行。”三藏闻言下马,观看
那衙门上有一匾,上书“迎阳驿”三字。长老道:“悟空,那村舍人家传言是实,
果有迎阳之驿。”沙僧笑道:“二哥,你却去‘照胎泉’边照照,看可有双影。”八
戒道:“莫弄我!我自吃了那盏儿落胎泉水,已此打下胎来了,还照他怎的?”三藏
回头吩咐道:“悟能,谨言,谨言!”遂上前与那女官作礼。

女官引路,请他们都进驿内,正厅坐下,即唤看茶。又见那手下人尽是三绺梳
头,两截穿衣之类。你看他拿茶的也笑。少顷,茶罢。女官欠身问曰:“使客何来?”
行者道:“我等乃东土大唐王驾下钦差上西天拜佛求经者。我师父便是唐王御弟,
号曰唐三藏。我乃他大徒弟孙悟空。这两个是我师弟:猪悟能、沙悟净。一行连马
五口。随身有通关文牒,乞为照验放行。”那女官执笔写罢,下来叩头道:“老爷恕
罪。下官乃迎阳驿驿丞,实不知上邦老爷,知当远接。”拜毕起身,即令管事的安
排饮馔。道:“爷爷们宽坐一时,待下官进城启奏我王,倒换关文,打发领给,送
老爷们西进。”三藏欣然而坐不题。

且说那驿丞整了衣冠,径入城中五凤楼前,对黄门官道:“我是迎阳馆驿丞,
有事见驾。”黄门即时启奏。降旨传宣至殿,问曰:“驿丞有何事来奏?”驿丞道:
“微臣在驿,接得东土大唐王御弟唐三藏。有三个徒弟,名唤孙悟空、猪悟能、沙
悟净,连马五口,欲上西天拜佛取经。特来启奏主公,可许他倒换关文放行?”女
王闻奏,满心欢喜,对众文武道:“寡人夜来梦见金屏生彩艳,玉镜展光明,乃是
今日之喜兆也。”众女官拥拜丹墀道:“主公,怎见得是今日之喜兆?”女王道:“东
土男人,乃唐朝御弟。我国中自混沌开辟之时,累代帝王,更不曾见个男人至此。
幸今唐王御弟下降,想是天赐来的。寡人以一国之富,愿招御弟为王,我愿为后,
与他阴阳配合,生子生孙,永传帝业,却不是今日之喜兆也?”众女官拜舞称扬,
无不欢悦。

驿丞又奏道:“主公之论,乃万代传家之好;但只是御弟三徒凶恶,不成相貌。”
女王道:“卿见御弟怎生模样?他徒弟怎生凶丑?”驿丞道:“御弟相貌堂堂,丰姿
英俊,诚是天朝上国之男儿,南赡中华之人物。那三徒却是形容狞恶,相貌如精。”
女王道:“既如此,把他徒弟与他领给,倒换关文,打发他往西天,只留下御弟,
有何不可?”众官拜奏道:“主公之言极当,臣等钦此钦遵。但只是匹配之事,无
媒不可。自古道:‘姻缘配合凭红叶,月老夫妻系赤绳。’”女王道:“依卿所奏,就
着当驾太师作媒,迎阳驿丞主婚,先去驿中与御弟求亲。待他许可,寡人却摆驾出
城迎接。”那太师、驿丞,领旨出朝。

却说三藏师徒们在驿厅上正享斋饭,只见外面人报:“当驾太师与我们本官老
姆来了。”三藏道:“太师来却是何意?”八戒道:“怕是女王请我们也。”行者道:
“不是相请,就是说亲。”三藏道:“悟空,假如不放,强逼成亲,却怎么是好?”
行者道:“师父只管允他,老孙自有处治。”说不了,二女官早至,对长老下拜。长
老一一还礼道:“贫僧出家人,有何德能,敢劳大人下拜?”那太师见长老相貌轩
昂,心中暗喜道:“我国中实有造化,这个男子,却也做得我王之夫。”二官拜毕起
来,侍立左右道:“御弟爷爷,万千之喜了!”三藏道:“我出家人,喜从何来?”
太师躬身道:“此处乃西梁女国,国中自来没个男子。今幸御弟爷爷降临,臣奉我
王旨意,特来求亲。”三藏道:“善哉,善哉!我贫僧只身来到贵地,又无儿女相随,
止有顽徒三个,不知大人求的是那个亲事?”驿丞道:“下官才进朝启奏,我王十
分欢喜道:夜来得一吉梦,梦见金屏生彩艳,玉镜展光明。知御弟乃中华上国男儿,
我王愿以一国之富,招赘御弟爷爷为夫,坐南面称孤,我王愿为帝后。传旨着太师
作媒,下官主婚,故此特来求这亲事也。”三藏闻言,低头不语。太师道:“大丈夫
遇时,不可错过。似此招赘之事,天下虽有;托国之富,世上实稀。请御弟速允,
庶好回奏。”长老越加痴哑。

八戒在旁掬着碓挺嘴,叫道:“太师,你去上复国王:我师父乃久修得道的罗
汉,决不爱你托国之富,也不爱你倾国之容;快些儿倒换关文,打发他往西去,留
我在此招赘,如何?”太师闻说,胆战心惊,不敢回说。驿丞道:“你虽是个男身,
但只形容丑陋,不中我王之意。”八戒笑道:“你甚不通变。常言道:‘粗柳簸箕细
柳斗,世上谁见男儿丑?’”行者道:“呆子,勿得胡谈,任师父尊意。可行则行,
可止则止。莫要担阁了媒妁工夫。”

三藏道:“悟空,凭你怎么说好。”行者道:“依老孙说,你在这里也好。自古
道‘千里姻缘似线牵’哩。那里再有这般相应处?”三藏道:“徒弟,我们在这里
贪图富贵,谁却去西天取经?那不望坏了我大唐之帝主也?”太师道:“御弟在上,
微臣不敢隐言。我王旨意,原只教求御弟为亲,教你三位徒弟赴了会亲筵宴,发付
领给,倒换关文,往西天取经去哩。”行者道:“太师说得有理。我等不必作难,情
愿留下师父,与你主为夫。快换关文。打发我们西去。待取经回来,好到此拜爷娘,
讨盘缠,回大唐也。”那太师与驿丞对行者作礼道:“多谢老师玉成之恩!”八戒道:
“太师,切莫要‘口里摆菜碟儿’。既然我们许诺,且教你主先安排一席,与我们
吃钟肯酒,如何?”太师道:“有,有,有,就教摆设筵宴来也。”那驿丞与太师欢
天喜地,回奏女主不题。

却说唐长老一把扯住行者,骂道:“你这猴头,弄杀我也!怎么说出这般话来,
教我在此招婚,你们西天拜佛,我就死也不敢如此!”行者道:“师父放心。老孙岂
不知你性情,但只是到此地,遇此人,不得不将计就计。”三藏道:“怎么叫做将计
就计?”行者道:“你若使住法儿不允他,他便不肯倒换关文,不放我们走路。倘
或意恶心毒,喝令多人,割了你肉,做甚么香袋啊,我等岂有善报?一定要使出降
魔荡怪的神通。你知我们的手脚又重,器械又凶,但动动手儿,这一国的人,尽打
杀了。他虽然阻当我等,却不是怪物妖精,还是一国人身;你又平素是个好善慈悲
的人,在路上一灵不损;若打杀无限的平人,你心何忍!诚为不善了也。”三藏听说,
道:“悟空,此论最善。但恐女主招我进去,要行夫妇之礼,我怎肯丧元阳,败坏
了佛家德行;走真精,坠落了本教人身。”行者道:“今日允了亲事,他一定以皇帝
礼,摆驾出城接你;你更不要推辞,就坐他凤辇龙车,登宝殿,面南坐下,问女王
取出御宝印信来,宣我们兄弟进朝,把通关文牒用了印,再请女王写个手字花押,
佥押了交付与我们。一壁厢教摆筵宴,就当与女王会喜,就与我们送行。待筵宴已
毕,再叫排驾,只说送我们三人出城,回来与女王配合。哄得他君臣欢悦,更无阻
挡之心,亦不起毒恶之念,却待送出城外,你下了龙车凤辇,教沙僧伺候左右,伏
侍你骑上白马,老孙却使个定身法儿,教他君臣人等皆不能动,我们顺大路只管西
行。行得一昼夜,我却念个咒,解了术法,还教他君臣们苏醒回城。一则不伤了他
的性命,二来不损了你的元神。这叫做‘假亲脱网’之计。岂非一举两全之美也?”
三藏闻言,如醉方醒,似梦初觉,乐以忘忧,称谢不尽,道:“深感贤徒高见。”四
众同心合意,正自商量不题。

却说那太师与驿丞不等宣诏,直入朝门白玉阶前,奏道:“主公佳梦最准,鱼
水之欢就矣。”女王闻奏,卷珠帘,下龙床,启樱唇,露银齿,笑吟吟娇声问曰:“贤
卿见御弟,怎么说来?”太师道:“臣等到驿,拜见御弟毕,即备言求亲之事。御
弟还有推托之辞,幸亏他大徒弟慨然见允,愿留他师父与我王为夫,面南称帝,只
教先倒换关文,打发他三人西去;取得经回,好到此拜认爷娘,讨盘费回大唐也。”
女王笑道:“御弟再有何说?”太师奏道:“御弟不言,愿配我主;只是他那二徒弟,
先要吃席肯酒。”

女王闻言,即传旨,教光禄寺排宴。一壁厢排大驾,出城迎接夫君。众女官即
钦遵王命,打扫宫殿,铺设庭台。一班儿摆宴的,火速安排;一班儿摆驾的,流星
整备。你看那西梁国虽是妇女之邦,那銮舆不亚中华之盛。但见:

六龙喷彩,双凤生祥:六龙喷彩扶车出,双凤生祥驾辇来。馥郁异香蔼,氤氲
瑞气开。金鱼玉佩多官拥,宝髻云鬟众女排。鸳鸯掌扇遮銮驾,翡翠珠帘影凤钗。
笙歌音美,弦管声谐。一片欢情冲碧汉,无边喜气出灵台。三檐罗盖摇天宇,五色
旌旗映御阶。此地自来无合卺,女王今日配男才。

不多时,大驾出城,早到迎阳馆驿。忽有人报三藏师徒道:“驾到了。”三藏闻
言,即与三徒,整衣出厅迎驾。女王卷帘下辇道:“那一位是唐朝御弟?”太师指
道:“那驿门外香案前穿衣者便是。”女王闪凤目,簇蛾眉,仔细观看,果然一表
非凡。你看他:

丰姿英伟,相貌轩昂。齿白如银砌,唇红口四方。顶平额阔天仑满,目秀眉清
地阁长。两耳有轮真杰士,一身不俗是才郎。好个妙龄聪俊风流子,堪配西梁窈窕
娘。
女王看到那心欢意美之处,不觉淫情汲汲,爱欲恣恣,展放樱桃小口,呼道:“大
唐御弟,还不来占凤乘鸾也?”三藏闻言,耳红面赤,羞答答不敢抬头。猪八戒在
旁,掬着嘴,饧眼观看那女王,却也袅娜。真个:

眉如翠羽,肌似羊脂。脸衬桃花瓣,鬟堆金凤丝。秋波湛湛妖娆态,春笋纤纤
娇媚姿。斜红绡飘彩艳,高簪珠翠显光辉。说甚么昭君美貌,果然是赛过西施。
柳腰微展鸣金,莲步轻移动玉肢。月里嫦娥难到此,九天仙子怎知斯。宫妆巧样
非凡类,诚然王母降瑶池。
那呆子看到好处,忍不住口嘴流涎,心头撞鹿,一时间骨软筋麻,好便似雪狮子向
火,不觉的都化去也。

只见那女王走近前来,一把扯住三藏,俏语娇声,叫道:“御弟哥哥,请上龙
车,和我同上金銮宝殿,匹配夫妇去来。”这长老战兢兢立站不住,似醉如痴。行
者在侧教道:“师父不必太谦,请共师娘上辇。快快倒换关文,等我们取经去罢。”
长老不敢回言,把行者抹了两抹,止不住落下泪来。行者道:“师父切莫烦恼。这
般富贵,不受用还待怎么哩?”三藏没及奈何,只得依从。揩了眼泪,强整欢容,
移步近前,与女主:

同携素手,共坐龙车。那女主喜孜孜欲配夫妻,这长老忧惶惶只思拜佛。一个
要洞房花烛交鸳侣,一个要西宇灵山见世尊。女帝真情,圣僧假意:女帝真情,指
望和谐同到老;圣僧假意,牢藏情意养元神。一个喜见男身,恨不得白昼并头谐伉
俪;一个怕逢女色,只思量即时脱网上雷音。二人和会同登辇,岂料唐僧各有心!

那些文武官见主公与长老同登凤辇,并肩而坐,一个个眉花眼笑,拨转仪从,
复入城中。孙大圣才教沙僧挑着行李,牵着白马,随大驾后边同行。猪八戒往前乱
跑,先到五凤楼前,嚷道:“好自在,好现成呀!这个弄不成,这个弄不成!吃了喜
酒进亲才是!”唬得些执仪从引导的女官,一个个回至驾边道:“主公,那一个长嘴
大耳的,在五凤楼前嚷道,要喜酒吃哩。”女主闻奏,与长老倚香肩,偎并桃腮,
开檀口,俏声叫道:“御弟哥哥,长嘴大耳的是你那个高徒?”三藏道:“是我第二
个徒弟。他生得食肠宽大,一生要图口肥;须是先安排些酒食与他吃了,方可行事。”
女主急问:“光禄寺安排筵宴,完否?”女官奏道:“已完,设了荤素两样,在东阁
上哩。”女王又问:“怎么两样?”女官奏道:“臣恐唐朝御弟与高徒等平素吃斋,
故有荤素两样。”女王却又笑吟吟,偎着长老的香腮道:“御弟哥哥,你吃荤吃素?”
三藏道:“贫僧吃素,但是未曾戒酒。须得几杯素酒,与我二徒弟吃些。”

说未了,太师启奏:“请赴东阁会宴。今宵吉日良辰,就可与御弟爷爷成亲。
明日天开黄道,请御弟爷爷登宝殿,面南,改年号即位。”女王大喜,即与长老携
手相搀,下了龙车,共入端门里。但见那:
风飘仙乐下楼台,阊阖中间翠辇来。
凤阙大开光蔼蔼,皇宫不闭锦排排。
麒麟殿内炉烟袅,孔雀屏边房影回。
亭阁峥嵘如上国,玉堂金马更奇哉。
既至东阁之下,又闻得一派笙歌声韵美,又见两行红粉貌娇娆。正中堂排设两般盛
宴:左边上首是素筵,右边上首是荤筵。下两路尽是单席。那女王敛袍袖,十指尖
尖,奉着玉杯,便来安席。行者近前道:“我师徒都是吃素。先请师父坐了左手素
席,转下三席,分左右,我兄弟们好坐。”太师喜道:“正是,正是。师徒即父子也,
不可并肩。”众女官连忙调了席面。女王一一传杯,安了他弟兄三位。行者又与唐
僧丢个眼色,教师父回礼。三藏下来,却也擎玉杯,与女王安席。那些文武官,朝
上拜谢了皇恩,各依品从,分坐两边,才住了音乐请酒。

那八戒那管好歹,放开肚子,只情吃起。也不管甚么玉屑米饭、蒸饼糖糕、蘑
菇香蕈、笋芽木耳、黄花菜石花菜、紫菜蔓菁、芋头萝菔、山药黄精,一骨辣了
个罄尽。喝了五七杯酒,口里嚷道:“看添换来!拿大觥来!再吃几觥,各人干事去。”
沙僧问道:“好筵席不吃,还要干甚事?”呆子笑道:“古人云:‘造弓的造弓,造
箭的造箭。’我们如今招的招,嫁的嫁,取经的还去取经,走路的还去走路,莫只
管贪杯误事。快早儿打发关文。正是‘将军不下马,各自奔前程。’”女王闻说,即
命取大杯来。近侍官连忙取几个鹦鹉杯、鸬鹚杓、金叵罗、银凿落、玻璃盏、水晶
盆、蓬莱碗、琥珀钟,满斟玉液,连注琼浆。果然都各饮一巡。

三藏欠身而起,对女王合掌道:“陛下,多蒙盛设,酒已够了。请登宝殿,倒
换关文,赶天早,送他三人出城罢。”女王依言,携着长老,散了筵宴,上金銮宝
殿,即让长老即位。三藏道:“不可!不可!适太师言过,明日天开黄道,贫僧才敢
即位称孤。今日即印关文,打发他去也。”

女王依言,仍坐了龙床,即取金交椅一张,放在龙床左手,请唐僧坐了,叫徒
弟们拿上通关文牒来。大圣便教沙僧解开包袱,取出关文。大圣将关文双手捧上。
那女王细看一番,上有大唐皇帝宝印九颗,下有宝象国印,乌鸡国印,车迟国印。
女王看罢,娇滴滴笑语道:“御弟哥哥又姓陈?”三藏道:“俗家姓陈,法名玄奘。
因我唐王圣恩认为御弟,赐姓我为唐也。”女王道:“关文上如何没有高徒之名?”
三藏道:“三个顽徒,不是我唐朝人物。”女王道:“既不是你唐朝人物,为何肯随
你来?”三藏道:“大的个徒弟,祖贯东胜神洲傲来国人氏;第二个乃西牛贺洲乌
斯庄人氏;第三个乃流沙河人氏。他三人都因罪犯天条,南海观世音菩萨解脱他苦,
秉善皈依,将功折罪,情愿保护我上西天取经。皆是途中收得,故此未注法名在牒。”
女王道:“我与你添注法名,好么?”三藏道:“但凭陛下尊意。”女王即令取笔砚
来,浓磨香翰,饱润香毫,牒文之后,写上孙悟空、猪悟能、沙悟净三人名讳,却
才取出御印,端端正正印了;又画个手字花押,传将下去。孙大圣接了,教沙僧包
裹停当。

那女王又赐出碎金碎银一盘,下龙床递与行者道:“你三人将此权为路费,早
上西天;待汝等取经回来,寡人还有重谢。”行者道:“我们出家人,不受金银,途
中自有乞化之处。”女王见他不受,又取出绫锦十匹,对行者道:“汝等行色匆匆,
裁制不及,将此路上做件衣服遮寒。”行者道:“出家人穿不得绫锦,自有护体布衣。”
女王见他不受,教:“取御米三升,在路权为一饭。”八戒听说个“饭”字,便就接
了,捎在包袱之间。行者道:“兄弟,行李见今沉重,且倒有气力挑米?”八戒笑
道:“你那里知道,米好的是个日消货。只消一顿饭,就了帐也。”遂此合掌谢恩。

三藏道:“敢烦陛下相同贫僧送他三人出城,待我嘱付他们几句,教他好生西
去,我却回来,与陛下永受荣华。无挂无牵,方可会鸾交凤友也。”女王不知是计,
便传旨摆驾,与三藏并倚香肩,同登凤辇,出西城而去。满城中都盏添净水,炉降
真香。一则看女王銮驾,二来看御弟男身。没老没小,尽是粉容娇面,绿鬓云鬟之
辈。不多时,大驾出城,到西关之外。

行者、八戒、沙僧,同心合意,结束整齐,径迎着銮舆,厉声高叫道:“那女
王不必远送,我等就此拜别。”长老慢下龙车,对女王拱手道:“陛下请回,让贫僧
取经去也。”女王闻言,大惊失色,扯住唐僧道:“御弟哥哥,我愿将一国之富,招
你为夫,明日高登宝位,即位称君,我愿为君之后,喜筵通皆吃了,如何却又变卦?”
八戒听说,发起个风来,把嘴乱扭,耳朵乱摇,闯至驾前,嚷道:“我们和尚家和
你这粉骷髅做甚夫妻!放我师父走路!”那女王见他那等撒泼弄丑,唬得魂飞魄散,
跌入辇驾之中。

沙僧却把三藏抢出人丛,伏侍上马。只见那路旁闪出一个女子,喝道:“唐御
弟,那里走!我和你耍风月儿去来!”沙僧骂道:“贼辈无知!”掣宝杖劈头就打。那
女子弄阵旋风,呜的一声,把唐僧摄将去了,无影无踪,不知下落何处。咦!正是:
脱得烟花网,又遇风月魔。

毕竟不知那女子是人是怪,老师父的性命得死得生,且听下回分解。