\chapter{鬼王夜谒唐三藏~悟空神化引婴儿}

却说三藏坐于宝林寺禅堂中,灯下念一会《梁皇水忏》,看一会《孔雀真经》,
只坐到三更时候,却才把经本包在囊里。正欲起身去睡,只听得门外扑剌剌一声响
,淅零零刮阵狂风。那长老恐吹灭了灯,慌忙将偏衫袖子遮住。又见那灯或明或
暗,便觉有些心惊胆战。此时又困倦上来,伏在经案上盹睡。虽是合眼朦胧,却还
心中明白,耳内嘤嘤听着那窗外阴风飒飒。好风,真个那:

淅淅潇潇,飘飘荡荡:淅淅潇潇飞落叶,飘飘荡荡卷浮云。满天星斗皆昏昧,
遍地尘沙尽洒纷。一阵家猛,一阵家纯。纯时松竹敲清韵,猛处江湖波浪浑。刮得
那山鸟难栖声哽哽,海鱼不定跳喷喷。东西馆阁门窗脱,前后房廊神鬼圭。佛殿花
瓶吹堕地,琉璃摇落慧灯昏。香炉倒香灰迸,烛架歪斜烛焰横。幢幡宝盖都摇拆,
钟鼓楼台撼动根。
那长老昏梦中听着风声一时过处,又闻得禅堂外,隐隐的叫一声“师父!”忽抬头
梦中观看,门外站着一条汉子:浑身上下,水淋淋的,眼中垂泪,口里不住叫:“师
父,师父!”三藏欠身道:“你莫是魍魉妖魅,神怪邪魔,至夜深时,来此戏我?我
却不是那贪欲贪嗔之类。我本是个光明正大之僧,奉东土大唐旨意,上西天拜佛求
经者。我手下有三个徒弟,都是降龙伏虎之英豪,扫怪除魔之壮士。他若见了你,
碎尸粉骨,化作微尘。此是我大慈悲之意、方便之心。你趁早儿潜身远遁,莫上我
的禅门来。”那人倚定禅堂道:“师父,我不是妖魔鬼怪,亦不是魍魉邪神。”三藏
道:“你既不是此类,却深夜来此何为?”那人道:“师父,你舍眼看我一看。”长
老果仔细定睛看处,——呀!只见他:

头戴一顶冲天冠,腰束一条碧玉带,身穿一领飞龙舞凤赭黄袍,足踏一双云头
绣口无忧履,手执一柄列斗罗星白玉圭。面如东岳长生帝,形似文昌开化君。
三藏见了,大惊失色。急躬身厉声高叫道:“是那一朝陛下?请坐。”用手忙搀,扑
了个空虚,回身坐定。再看处,还是那个人。

长老便问:“陛下,你是那里皇王?何邦帝主?想必是国土不宁,谗臣欺虐,半
夜逃生至此。有何话说,说与我听。”这人才泪滴腮边谈旧事,愁攒眉上诉前因,
道:“师父啊,我家住在正西,离此只有四十里远近。那厢有座城池,便是兴基之
处。”三藏道:“叫做甚么地名?”那人道:“不瞒师父说,便是朕当时创立家邦,
改号乌鸡国。”三藏道:“陛下这等惊慌,却因甚事至此?”那人道:“师父啊,我
这里五年前,天年干旱,草子不生,民皆饥死,甚是伤情。”三藏闻言,点头叹道:
“陛下啊,古人云:‘国正天心顺。’想必是你不慈恤万民。既遭荒歉,怎么就躲离
城郭?且去开了仓库,赈济黎民;悔过前非,重兴今善,放赦了那枉法冤人;自然
天心和合,雨顺风调。”那人道:“我国中仓廪空虚,钱粮尽绝。文武两班停俸禄,
寡人膳食亦无荤。仿效禹王治水,与万民同受甘苦,沐浴斋戒,昼夜焚香祈祷。如
此三年,只干得河枯井涸。正都在危急之处,忽然锺南山来了一个全真,能呼风唤
雨,点石成金。先见我文武多官,后来见朕,当即请他登坛祈祷,果然有应,只见
令牌响处,顷刻间大雨滂沱。寡人只望三尺雨足矣,他说久旱不能润泽,又多下了
二寸。朕见他如此尚义,就与他八拜为交,以‘兄弟’称之。”三藏道:“此陛下万
千之喜也。”那人道:“喜自何来?”三藏道:“那全真既有这等本事,若要雨时,
就教他下雨;若要金时,就教他点金。还有那些不足,却离了城阙来此?”那人道:
“朕与他同寝食者,只得二年。又遇着阳春天气,红杏夭桃,开花绽蕊,家家士女,
处处王孙,俱去游春赏玩。那时节,文武归衙,嫔妃转院。朕与那全真携手缓步,
至御花园里,忽行到八角琉璃井边,不知他抛下些甚么物件,井中有万道金光。哄
朕到井边看甚么宝贝,他陡起凶心,扑通的把寡人推下井内;将石板盖住井口,拥
上泥土,移一株芭蕉栽在上面。可怜我啊,已死去三年,是一个落井伤生的冤屈之
鬼也!”

唐僧见说是鬼,唬得筋力酥软,毛骨耸然。没奈何,只得将言又问他道:“陛
下,你说的这话,全不在理。既死三年,那文武多官,三宫皇后,遇三朝见驾殿上,
怎么就不寻你?”那人道:“师父啊,说起他的本事,果然世间罕有!自从害了朕,
他当时在花园内摇身一变,就变做朕的模样,更无差别。现今占了我的江山,暗侵
了我的国土。他把我两班文武,四百朝官,三宫皇后,六院嫔妃,尽属了他矣。”

三藏道:“陛下,你忒也懦。”那人道:“何懦?”三藏道:“陛下,那怪倒有些
神通,变作你的模样,侵占你的乾坤,文武不能识,后妃不能晓,只有你死的明白;
你何不在阴司阎王处具告,把你的屈情伸诉伸诉。”那人道:“他的神通广大,官吏
情熟,——都城隍常与他会酒,海龙王尽与他有亲;东岳天齐是他的好朋友,十代
阎罗是他的异兄弟。因此这般,我也无门投告。”

三藏道:“陛下,你阴司里既没本事告他,却来我阳世间作甚?”那人道:“师
父啊,我这一点冤魂,怎敢上你的门来?山门前有那护法诸天、六丁六甲、五方揭
谛、四值功曹、一十八位护教伽蓝,紧随鞍马。却才被夜游神一阵神风,把我送将
进来。他说我三年水灾该满,着我来拜谒师父。他说你手下有一个大徒弟,是齐天
大圣,极能斩怪降魔。今来志心拜恳,千乞到我国中,拿住妖魔,辨明邪正。朕当
结草衔环,报酬师恩也!”

三藏道:“陛下,你此来是请我徒弟与你去除却那妖怪么?”那人道:“正是,
正是!”三藏道:“我徒弟干别的事不济,但说降妖捉怪,正合他宜。陛下啊,虽是
着他拿怪,但恐理上难行。”那人道:“怎么难行?”三藏道:“那怪既神通广大,
变得与你相同;满朝文武,一个个言和心顺;三宫妃嫔,一个个意合情投;我徒弟
纵有手段,决不敢轻动干戈。倘被多官拿住,说我们欺邦灭国,问一款大逆之罪,
困陷城中,却不是画虎刻鹄也?”

那人道:“我朝中还有人哩。”三藏道:“却好,却好!想必是一代亲王侍长,发
付何处镇守去了?”那人道:“不是;我本宫有个太子,是我亲生的储君。”三藏道:
“那太子想必被妖魔贬了?”那人道:“不曾。他只在金銮殿上,五凤楼中,或与
学士讲书,或共全真登位。自此三年,禁太子不入皇宫,不能彀与娘娘相见。”三
藏道:“此是何故?”那人道:“此是妖怪使下的计策。只恐他母子相见,闲中论出
长短,怕走了消息;故此两不会面,他得永住常存也。”

三藏道:“你的灾屯,想应天付,却与我相类。当时我父曾被水贼伤生。我母
被水贼欺占,经三个月,分娩了我。我在水中逃了性命,幸金山寺恩师,救养成人。
记得我幼年无父母,此间那太子失双亲,惭惶不已!”

又问道:“你纵有太子在朝,我怎的与他相见?”那人道:“如何不得见?”三
藏道:“他被妖魔拘辖,连一个生身之母尚不得见,我一个和尚,欲见何由?”那
人道:“他明早出朝来也。”三藏问:“出朝作甚?”那人道:“明日早朝,领三千人
马,架鹰犬,出城采猎,师父断得与他相见。见时肯将我的言语说与他,他便信了。”
三藏道:“他本是肉眼凡胎,被妖魔哄在殿上,那一日不叫他几声父王?他怎肯信我
的言语?”那人道:“既恐他不信,我留下一件表记与你罢。”三藏问:“是何物件?”
那人把手中执的金厢白玉圭放下道:“此物可以为记。”三藏道:“此物何如?”那
人道:“全真自从变作我的模样,只是少变了这件宝贝。他到宫中,说那求雨的全
真拐了此圭去了。自此三年,还没此物。我太子若看见,他睹物思人,此仇必报。”
三藏道:“也罢,等我留下,着徒弟与你处置。却在那里等么?”那人道:“我也不
敢等。我这去,还央求夜游神,再使一阵神风,把我送进皇宫内院,托一梦与我那
正宫皇后,教他母子们合意,你师徒们同心。”三藏点头应承道:“你去罢。”

那冤魂叩头拜别,举步相送,不知怎么蹋了脚,跌了一个筋斗,把三藏惊醒,
却原来是南柯一梦。慌得对着那盏昏灯,连忙叫:“徒弟,徒弟!”八戒醒来道:“甚
么‘土地土地’?当时我做好汉,专一吃人度日,受用腥膻,其实快活;偏你出家,
教我们保护你跑路!原说只做和尚,如今拿做奴才,日间挑包袱牵马,夜间提尿瓶
务脚!这早晚不睡,又叫徒弟作甚?”

三藏道:“徒弟,我刚才伏在案上打盹,做了一个怪梦。”行者跳将起来道:“师
父,梦从想中来。你未曾上山,先怕妖怪;又愁雷音路远,不能得到;思念长安,
不知何日回程:所以心多梦多。似老孙一点真心,专要西方见佛,更无一个梦儿到
我。”三藏道:“徒弟,我这桩梦,不是思乡之梦。才然合眼,见一阵狂风过处,禅
房门外有一朝皇帝,自言是乌鸡国王。浑身水湿,满眼泪垂。”这等这等,如此如
此,将那梦中话一一的说与行者。行者笑道:“不消说了,他来托梦与你,分明是
照顾老孙一场生意。必然是个妖怪在那里篡位谋国。等我与他辨个真假。想那妖魔,
棍到处,立业成功。”三藏道:“徒弟,他说那怪神通广大哩。”行者道:“怕他甚么
广大!早知老孙到,教他即走无方!”三藏道:“我又记得留下一件宝贝做表记。”八
戒答道:“师父莫要胡缠;做个梦便罢了,怎么只管当真?”沙僧道:“‘不信直中
直,须防仁不仁’。我们打起火,开了门,看看如何便是。”

行者果然开门。一齐看处,只见星月光中,阶檐上,真个放着一柄金厢白玉圭。
八戒近前拿起道:“哥哥,这是甚么东西?”行者道:“这是国王手中执的宝贝,名
唤玉圭。师父啊,既有此物,想此事是真。明日拿妖,全都在老孙身上。只是要你
三桩儿造化低哩。”八戒道:“好,好,好!做个梦罢了,又告诵他。他那些儿不会
作弄人哩?就教你三桩儿造化低。”三藏回入里面道:“是那三桩?”行者道:“明日
要你顶缸、受气、遭瘟。”八戒笑道:“一桩儿也是难的,三桩儿却怎么耽得?”唐
僧是个聪明的长老,便问:“徒弟啊,此三事如何讲?”行者道:“也不消讲,等我
先与你二件物。”

好大圣,拔了一根毫毛,吹口仙气,叫声“变”!变做一个红金漆匣儿,把白
玉圭放在内盛着,道:“师父,你将此物捧在手中,到天晓时,穿上锦袈裟,去
正殿坐着念经,等我去看看他那城池。端的是个妖怪,就打杀他,也在此间立个功
绩;假若不是,且休撞祸。”三藏道:“正是,正是。”行者道:“那太子不出城便罢;
若真个应梦出城来,我定引他来见你。”三藏道:“见了我如何迎答?”行者道:“来
到时,我先报知,你把那匣盖儿扯开些,等我变作二寸长的一个小和尚,钻在匣儿
里,你连我捧在手中。那太子进了寺来,必然拜佛;你尽他怎的下拜,只是不睬他。
他见你不动身,一定教拿你;你凭他拿下去,打也由他,绑也由他,杀也由他。”
三藏道:“呀!他的军令大,真个杀了我,怎么好?”行者道:“没事,有我哩。若
到那紧关处,我自然护你。他若问时,你说是东土钦差上西天拜佛取经进宝的和尚。
他道:‘有甚宝贝?’你却把锦袈裟对他说一遍,说道:‘此是三等宝贝。还有头
一等、第二等的好物哩。’但问处,就说这匣内有一件宝贝,上知五百年,下知五
百年,中知五百年,共一千五百年过去未来之事,俱尽晓得。却把老孙放出来。我
将那梦中话告诵那太子,他若肯信,就去拿了那妖魔,一则与他父王报仇,二来我
们立个名节;他若不信,再将白玉圭拿与他看。只恐他年幼,还不认得哩。”三藏
闻言,大喜道:“徒弟啊,此计绝妙!但说这宝贝,一个叫做锦袈裟,一个叫做白
玉圭;你变的宝贝却叫做甚名?”行者道:“就叫做‘立帝货’罢。”三藏依言,记
在心上。师徒们一夜那曾得睡,盼到天明,恨不得点头唤出扶桑日,喷气吹散满天
星。

不多时,东方发白。行者又吩咐了八戒、沙僧,教他两个:“不可搅扰僧人,
出来乱走。待我成功之后,共汝等同行。”才别了唐僧,打了唿哨,一筋斗跳在空
中。睁火眼平西看处,果见有一座城池。你道怎么就看见了?当时说那城池离寺只
有四十里,故此凭高就望见了。

行者近前仔细看处,又见那怪雾愁云漠漠,妖风怨气纷纷。行者在空中赞叹道:
“若是真王登宝座,自有祥光五色云;
只因妖怪侵龙位,腾腾黑气锁金门。”
行者正然感叹。忽听得炮声响,又只见东门开处,闪出一路人马,真个是采猎之
军,果然势勇。但见:

晓出禁城东,分围浅草中。彩旗开映日,白马骤迎风。鼍鼓冬冬擂,标枪对对
冲。架鹰军猛烈,牵犬将骁雄。火炮连天振,粘竿映日红。人人支弩箭,个个挎雕
弓。张网山坡下,铺绳小径中。一声惊霹雳,千骑拥貔熊。狡兔身难保,乖獐智亦
穷。狐狸该命尽,麋鹿丧当终。山雉难飞脱,野鸡怎避凶?他都要捡占山场擒猛兽,
摧残林木射飞虫。

那些人出得城来,散步东郊,不多时,有二十里向高田地,又只见中军营里,
有小小的一个将军:顶着盔,贯着甲,果肚花,十八札,手执青锋宝剑,坐下黄骠
马,腰带满弦弓。真个是:
隐隐君王像,昂昂帝主容。
规模非小辈,行动显真龙。
行者在空暗喜道:“不须说,那个就是皇帝的太子了。等我戏他一戏。”

好大圣,按落云头,撞入军中太子马前。摇身一变,变作一个白兔儿,只在太
子马前乱跑。太子看见,正合欢心,拈起箭,拽满弓,一箭正中了那兔儿。原来是
那大圣故意教他中了,却眼乖手疾,一把接住那箭头,把箭翎花落在前边,丢开脚
步跑了。那太子见箭中了玉兔,兜开马,独自争先来赶。不知马行的快,行者如风;
马行的迟,行者慢走;只在他面前不远。看他一程一程,将太子哄到宝林寺山门之
下,行者现了本身,——不见兔儿,只见一枝箭插在门槛上。——径撞进去,见唐
僧道:“师父,来了!来了!”却又一变,变做二寸长短的小和尚儿,钻在红匣之内。

却说那太子赶到山门前,不见了白兔,只见门槛上插住一枝雕翎箭。太子大惊
失色道:“怪哉!怪哉!分明我箭中了玉兔,玉兔怎么不见,只见箭在此间!想是年多
日久,成了精魅也。”拔了箭,抬头看处,山门上有五个大字,写着“敕建宝林寺”。
太子道:“我知之矣。向年间曾记得我父王在金銮殿上差官些金帛与这和尚修理
佛殿佛像,不期今日到此。正是‘因过道院逢僧话,又得浮生半日闲’。我且进去
走走。”

那太子跳下马来,正要进去。只见那保驾的官将与三千人马赶上,簇簇拥拥,
都入山门里面。慌得那本寺众僧,都来叩头拜接。接入正殿中间,参拜佛像。却才
举目观瞻,又欲游廊玩景,忽见正当中坐着一个和尚,太子大怒道:“这个和尚无
礼!我今半朝銮驾进山,虽无旨意知会,不当远接,此时军马临门,也该起身;怎
么还坐着不动?”教:“拿下来!”说声“拿”字,两边校尉,一齐下手,把唐僧抓
将下来,急理绳索便捆。行者在匣里默默的念咒,教道:“护法诸天、六丁六甲,
我今设法降妖,这太子不能知识,将绳要捆我师父,汝等即早护持;若真捆了,汝
等都该有罪!”那大圣暗中吩咐,谁敢不遵,却将三藏护持定了;有些人摸也摸不
着他光头,好似一壁墙挡住,难拢其身。

那太子道:“你是那方来的,使这般隐身法欺我!”三藏上前施礼道:“贫僧无
隐身法,乃是东土唐僧,上雷音寺拜佛求经进宝的和尚。”太子道:“你那东土虽是
中原,其穷无比,有甚宝贝,你说来我听。”三藏道:“我身上穿的这袈裟,是第三
样宝贝。还有第一等,第二等更好的物哩!”太子道:“你那衣服,半边苫身,半边
露臂,能值多少物,敢称宝贝!”三藏道:“这袈裟虽不全体,有诗几句,诗曰:
佛衣偏袒不须论,内隐真如脱世尘。
万线千针成正果,九珠八宝合元神。
仙娥圣女恭修制,遗赐禅僧静垢身。
见驾不迎犹自可,你的父冤未报枉为人!”
太子闻言,心中大怒道:“这泼和尚胡说!你那半片衣,凭着你口能舌便,夸好夸强。
我的父冤从何未报,你说来我听。”

三藏进前一步,合掌问道:“殿下,为人生在天地之间,能有几恩?”太子道:
“有四恩。”三藏道:“那四恩?”太子道:“感天地盖载之恩,日月照临之恩,国
王水土之恩,父母养育之恩。”三藏笑曰:“殿下言之有失。人只有天地盖载,日月
照临,国王水土,那得个父母养育来?”太子怒道:“和尚是那游手游食削发逆君
之徒!人不得父母养育,身从何来?”三藏道:“殿下,贫僧不知;但只这红匣内有
一件宝贝,叫做‘立帝货’,他上知五百年,中知五百年,下知五百年,共知一千
五百年过去未来之事,便知无父母养育之恩,令贫僧在此久等多时矣。”

太子闻说,教:“拿来我看。”三藏扯开匣盖儿,那行者跳将出来,呀的,
两边乱走。太子道:“这星星小人儿,能知甚事?”行者闻言嫌小,却就使个神通,
把腰伸一伸,就长了有三尺四五寸。众军士吃惊道:“若是这般快长,不消几日,
就撑破天也。”行者长到原身,就不长了。太子才问道:“立帝货,这老和尚说你能
知未来过去吉凶,你却有龟作卜,有蓍作筮?凭书句断人祸福?”行者道:“我一毫
不用,只是全凭三寸舌,万事尽皆知。”太子道:“这厮又是胡说。自古以来,《周
易》之书,极其玄妙,断尽天下吉凶,使人知所趋避;故龟所以卜,蓍所以筮。听
汝之言,凭据何理?妄言祸福,扇惑人心!”

行者道:“殿下且莫忙,等我说与你听。你本是乌鸡国王的太子。你那里五年
前,年程荒旱,万民遭苦,你家皇帝共臣子,秉心祈祷。正无点雨之时,锺南山来
了一个道士,他善呼风唤雨,点石为金。君王忒也爱小,就与他拜为兄弟。这桩事
有么?”太子道:
“有!有!有!你再说说。”行者道:“后三年不见全真,称孤的却是谁?”太子道:“果
是有个全真,父王与他拜为兄弟,食则同食,寝则同寝。三年前在御花园里玩景,
被他一阵神风,把父王手中金厢白玉圭,摄回锺南山去了。至今父王还思慕他。因
不见他,遂无心赏玩,把花园紧闭了,已三年矣。做皇帝的,非我父王而何?”

行者闻言,哂笑不绝。太子再问不答,只是哂笑。

太子怒道:“这厮当言不言,如何这等哂笑?”行者又道:“还有许多话哩;奈
何左右人众,不是说处。”太子见他言语有因,将袍袖一展,教军士且退。那驾上
官将,急传令,将三千人马,都出门外住扎。此时殿上无人,太子坐在上面,长老
立在前边,左手旁立着行者。本寺诸僧皆退。行者才正色上前道:“殿下,化风去
的是你生身之父母,见坐位的,是那祈雨之全真。”太子道:“胡说,胡说!我父自
全真去后,风调雨顺,国泰民安,照依你说,就不是我父王了。还是我年孺,容得
你;若我父王听见你这番话,拿了去,碎尸万段!”把行者咄的喝下来。

行者对唐僧道:“何如?我说他不信。果然,果然!如今却拿那宝贝进与他,倒
换关文,往西方去罢。”三藏即将红匣子递与行者。行者接过来,将身一抖,那匣
儿卒不见了,原是他毫毛变的,被他收上身去。却将白玉圭双手捧上,献与太子。

太子见了道:“好和尚,好和尚!你五年前本是个全真,来骗了我家的宝贝,如
今又妆做和尚来进献!”叫:“拿下!”一声传令,把长老唬得慌忙指着行者道:“你
这弼马温!专撞空头祸,带累我哩!”行者近前一齐拦住道:“休嚷,莫走了风!我不
叫做立帝货,还有真名哩。”太子怒道:“你上来!我问你个真名字,好送法司定罪!”

行者道:“我是那长老的大徒弟,名唤悟空孙行者。因与我师父上西天取经,
昨宵到此觅宿。我师父夜读经卷,至三更时分,得一梦。梦见你父王道,他被那全
真欺害,推在御花园八角琉璃井内,全真变作他的模样。满朝官不能知,你年幼亦
无分晓,禁你入宫,关了花园,大端怕漏了消息。你父王今夜特来请我降魔,我恐
不是妖邪;自空中看了,果然是个妖精。正要动手拿他,不期你出城打猎。你箭中
的玉兔,就是老孙。老孙把你引到寺里,见师父,诉此衷肠,句句是实。你既然认
得白玉圭,怎么不念鞠养恩情,替亲报仇?”

那太子闻言,心中惨戚,暗自伤愁道:“若不信此言语,他却有三分儿真实;
若信了,怎奈殿上见是我父王。”这才是进退两难心问口,三思忍耐口问心。行者
见他疑惑不定,又上前道:“殿下不必心疑,请殿下驾回本国,问你国母娘娘一声,
看他夫妻恩爱之情,比三年前如何。只此一问,便知真假矣。”那太子回心道:“正
是!且待我问我母亲去来。”

他跳起身,笼了玉圭就走。行者扯住道:“你这些人马都回,却不走漏消息,
我难成功?但要你单人独马进城,不可扬名卖弄。莫入正阳门,须从后宰门进去。
到宫中见你母亲,切休高声大气,须是悄语低言:恐那怪神通广大,一时走了消息,
你娘儿们性命俱难保也。”太子谨遵教命,出山门吩咐将官:“稳在此扎营,不得移
动。我有一事,待我去了就来一同进城。”看他:
指挥号令屯军士,上马如飞即转城。

这一去,不知见了娘娘,有何话说,且听下回分解。