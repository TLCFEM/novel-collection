\chapter{观音赴会问原因~小圣施威降大圣}

且不言天神围绕,大圣安歇。话表南海普陀落伽山大慈大悲救苦救难灵感观世
音菩萨,自王母娘娘请赴蟠桃大会,与大徒弟惠岸行者,同登宝阁瑶池,见那里荒
荒凉凉,席面残乱;虽有几位天仙,俱不就座,都在那里乱纷纷讲论。菩萨与众仙
相见毕,众仙备言前事。菩萨道:“既无盛会,又不传杯,汝等可跟贫僧去见玉帝。”
众仙怡然随往。至通明殿前,早有四大天师、赤脚大仙等众,俱在此迎着菩萨,即
道玉帝烦恼,调遣天兵,擒怪未回等因。菩萨道:“我要见见玉帝,烦为转奏。”天
师邱弘济,即入灵霄宝殿,启知宣入。时有太上老君在上,王母娘娘在后。

菩萨引众同入里面,与玉帝礼毕,又与老君、王母相见,各坐下,便问:“蟠
桃盛会如何?”玉帝道:“每年请会,喜喜欢欢,今年被妖猴作乱,甚是虚邀也。”
菩萨道:“妖猴是何出处?”玉帝道:“妖猴乃东胜神洲傲来国花果山石卵化生的。
当时生出,即目运金光,射冲斗府。始不介意,继而成精,降龙伏虎,自削死籍。
当有龙王、阎王启奏。朕欲擒拿,是长庚星启奏道:‘三界之间,凡有九窍者,可
以成仙。’朕即施教育贤,宣他上界,封为御马监弼马温官。那厮嫌恶官小,反了
天宫。即差李天王与哪吒太子收降,又降诏抚安,宣至上界,就封他做个‘齐天大
圣’,只是有官无禄。他因没事干管理,东游西荡。朕又恐别生事端,着他代管蟠
桃园。他又不遵法律,将老树大桃尽行偷吃。及至设会,他乃无禄人员,不曾请他;
他就设计赚哄赤脚大仙,却自变他相貌入会,将仙肴仙酒尽偷吃了,又偷老君仙丹,
又偷御酒若干,去与本山众猴享乐。朕心为此烦恼,故调十万天兵,天罗地网收伏。
这一日不见回报,不知胜负如何。”菩萨闻言,即命惠岸行者道:“你可快下天宫,
到花果山,打探军情如何。如遇相敌,可就相助一功,务必的实回话。”

惠岸行者整整衣裙,执一条铁棍,驾云离阙,径至山前。见那天罗地网,密密
层层,各营门提铃喝号,将那山围绕的水泄不通。惠岸立住,叫:“把营门的天丁,
烦你传报:我乃李天王二太子木叉,南海观音大徒弟惠岸,特来打探军情。”那营
里五岳神兵,即传入辕门之内。早有虚日鼠、昴日鸡、星日马、房日兔,将言传到
中军帐下。李天王发下令旗,教开天罗地网,放他进来。此时东方才亮。惠岸随旗
进入,见四大天王与李天王下拜。拜讫,李天王道:“孩儿,你自那厢来者?”惠
岸道:“愚男随菩萨赴蟠桃会,菩萨见胜会荒凉,瑶池寂寞,引众仙并愚男去见玉
帝。玉帝备言父王等下界收伏妖猴,一日不见回报,胜负未知,菩萨因命愚男到此
打听虚实。”李天王道:“昨日到此安营下寨,着九曜星挑战,被这厮大弄神通,九
曜星俱败走而回。后我等亲自提兵,那厮也排开阵势。我等十万天兵,与他混战至
晚,他使个分身法战退。及收兵查勘时,止捉得些狼虫虎豹之类,不曾捉得他半个
妖猴。今日还未出战。”

说不了,只见辕门外有人来报道:“那大圣引一群猴精,在外面叫战。”四大天
王与李天王并太子正议出兵。木叉道:“父王,愚男蒙菩萨吩咐,下来打探消息,
就说若遇战时,可助一功。今不才愿往,看他怎么个大圣!”天王道:“孩儿,你随
菩萨修行这几年,想必也有些神通,切须在意。”

好太子,双手轮着铁棍,束一束绣衣,跳出辕门,高叫:“那个是齐天大圣?”
大圣挺如意棒,应声道:“老孙便是。你是甚人,辄敢问我?”木叉道:“吾乃李天
王第二太子木叉,今在观音菩萨宝座前为徒弟护教,法名惠岸是也。”大圣道:“你
不在南海修行,却来此见我做甚?”木叉道:“我蒙师父差来打探军情,见你这般
猖獗,特来擒你!”大圣道:“你敢说那等大话!且休走!吃老孙这一棒!”木叉全然
不惧,使铁棒劈手相迎。他两个立那半山中,辕门外,这场好斗:

棍虽对棍铁各异,兵纵交兵人不同。一个是太乙散仙呼大圣,一个是观音徒弟
正元龙。浑铁棍乃千锤打,六丁六甲运神功;如意棒是天河定,镇海神珍法力洪。
两个相逢真对手,往来解数实无穷。这个的阴手棍,万千凶,绕腰贯索疾如风;那
个的夹枪棒,不放空,左遮右挡怎相容。那阵上旌旗闪闪,这阵上鼍鼓冬冬。万员
天将团团绕,一洞妖猴簇簇丛。怪雾愁云漫地府,狼烟煞气射天宫。昨朝混战还犹
可,今日争持更又凶。堪羡猴王真本事,木叉复败又逃生。
这大圣与惠岸战经五六十合,惠岸臂膊酸麻,不能迎敌,虚幌一幌,败阵而走。大
圣也收了猴兵,安扎在洞门之外。只见天王营门外,大小天兵,接住了太子,让开
大路,径入辕门,对四天王、李托塔、哪吒,气哈哈的,喘息未定:“好大圣!好大
圣!着实神通广大!孩儿战不过,又败阵而来也!”李天王见了心惊,即命写表求助,
便差大力鬼王与木叉太子上天启奏。

二人当时不敢停留,闯出天罗地网,驾起瑞霭祥云。须臾,径至通明殿下,见
了四大天师,引至灵霄宝殿,呈上表章。惠岸又见菩萨施礼。菩萨道:“你打探的
如何?”惠岸道:“始领命到花果山,叫开天罗地网门,见了父亲,道师父差命之
意。父王道:‘昨日与那猴王战了一场,止捉得他虎豹狮象之类,更未捉他一个猴
精。’正讲间,他又索战,是弟子使铁棍与他战经五六十合,不能取胜,败走回营。
父亲因此差大力鬼王同弟子上界求助。”菩萨低头思忖。

却说玉帝拆开表章,见有求助之言,笑道:“叵耐这个猴精,能有多大手段,
就敢敌过十万天兵!李天王又来求助,却将那路神兵助之?”言未毕,观音合掌启
奏:“陛下宽心,贫僧举一神,可擒这猴。”玉帝道:“所举者何神?”菩萨道:“乃
陛下令甥显圣二郎真君,见居灌洲灌江口,享受下方香火。他昔日曾力诛六怪,又
有梅山兄弟与帐前一千二百草头神,神通广大。奈他只是听调不听宣,陛下可降一
道调兵旨意,着他助力,便可擒也。”玉帝闻言,即传调兵的旨意,就差大力鬼王
赍调。

那鬼王领了旨,即驾起云,径至灌江口。不消半个时辰,直至真君之庙。早有
把门的鬼判,传报至里道:“外有天使,捧旨而至。”二郎即与众弟兄,出门迎接旨
意,焚香开读。旨意上云:

花果山妖猴齐天大圣作乱。因在宫偷桃、偷酒、偷丹,搅乱蟠桃大会,见着十
万天兵,一十八架天罗地网,围山收伏,未曾得胜。今特调贤甥同义兄弟即赴花果
山助力剿除。成功之后,高升重赏。
真君大喜道:“天使请回,吾当就去拔刀相助也。”鬼王回奏不题。

这真君即唤梅山六兄弟——乃康、张、姚、李四太尉,郭申、直健二将军,聚
集殿前道:“适才玉帝调遣我等往花果山收降妖猴,同去去来。”众兄弟俱忻然愿往。
即点本部神兵,驾鹰牵犬,搭弩张弓,纵狂风,霎时过了东洋大海,径至花果山。
见那天罗地网,密密层层,不能前进,因叫道:“把天罗地网的神将听着:吾乃二
郎显圣真君,蒙玉帝调来,擒拿妖猴者,快开营门放行。”一时,各神一层层传入。
四大天王与李天王俱出辕门迎接。相见毕,问及胜败之事,天王将上项事备陈一遍。
真君笑道:“小圣来此,必须与他斗个变化。列公将天罗地网,不要幔了顶上,只
四围紧密,让我赌斗。若我输与他,不必列公相助,我自有兄弟扶持;若赢了他,
也不必列公绑缚,我自有兄弟动手。只请托塔天王与我使个照妖镜,住立空中。恐
他一时败阵,逃窜他方,切须与我照耀明白,勿走了他。”天王各居四维,众天兵
各挨排列阵去讫。

这真君领着四太尉、二将军,连本身七兄弟,出营挑战;分付众将,紧守营盘,
收全了鹰犬。众草头神得令。真君只到那水帘洞外,见那一群猴,齐齐整整,排作
个蟠龙阵势;中军里,立一竿旗,上书“齐天大圣”四字。真君道:“那泼妖,怎
么称得起齐天之职?”梅山六弟道:“且休赞叹,叫战去来。”那营口小猴见了真君,
急走去报知。那猴王即掣金箍棒,整黄金甲,登步云履,按一按紫金冠,腾出营门,
急睁睛观看,那真君的相貌,果是清奇,打扮得又秀气。真个是:

仪容清俊貌堂堂,两耳垂肩目有光。头戴三山飞凤帽,身穿一领淡鹅黄。缕金
靴衬盘龙袜,玉带团花八宝妆。腰挎弹弓新月样,手执三尖两刃枪。斧劈桃山曾救
母,弹打棕罗双凤凰。力诛八怪声名远,义结梅山七圣行。心高不认天家眷,性傲
归神住灌江。赤城昭惠英灵圣,显化无边号二郎。

大圣见了,笑嘻嘻的,将金箍棒掣起,高叫道:“你是何方小将,辄敢大胆到
此挑战?”真君喝道:“你这厮有眼无珠,认不得我么!吾乃玉帝外甥,敕封昭惠灵
显王二郎是也。今蒙上命,到此擒你这反天宫的弼马温猢狲,你还不知死活!”大
圣道:“我记得当年玉帝妹子思凡下界,配合杨君,生一男子,曾使斧劈桃山的,
是你么?我待要骂你几声,曾奈无甚冤仇;待要打你一棒,可惜了你的性命。你这
郎君小辈,可急急回去,唤你四大天王出来。”真君闻言,心中大怒道:“泼猴!休
得无礼!吃吾一刃!”大圣侧身躲过,疾举金箍棒,劈手相还。他两个这场好杀:

昭惠二郎神,齐天孙大圣,这个心高欺敌美猴王,那个面生压伏真梁栋。两个
乍相逢,各人皆赌兴。从来未识浅和深,今日方知轻与重。铁棒赛飞龙,神锋如舞
凤。左挡右攻,前迎后映。这阵上梅山六弟助威风,那阵上马流四将传军令。摇
旗擂鼓各齐心,呐喊筛锣都助兴。两个钢刀有见机,一来一往无丝缝。金箍棒是海
中珍,变化飞腾能取胜;若还身慢命该休,但要差池为蹭蹬。
真君与大圣斗经三百余合,不知胜负。那真君抖搜神威,摇身一变,变得身高万丈,
两只手,举着三尖两刃神锋,好便似华山顶上之峰,青脸獠牙,朱红头发,恶狠狠,
望大圣着头就砍。这大圣也使神通,变得与二郎身躯一样,嘴脸一般,举一条如意
金箍棒,却就如昆仑顶上的擎天之柱,抵住二郎神。唬得那马、流元帅,战兢兢,
摇不得旌旗;崩、芭二将,虚怯怯,使不得刀剑。这阵上,康、张、姚、李、郭申、
直健传号令,撒放草头神,向他那水帘洞外,纵着鹰犬,搭弩张弓,一齐掩杀。可
怜冲散妖猴四健将,捉拿灵怪二三千!那些猴,抛戈弃甲,撇剑丢枪;跑的跑,喊
的喊;上山的上山,归洞的归洞:好似夜猫惊宿鸟,飞洒满天星。众兄弟得胜不题。

却说真君与大圣变做法天象地的规模,正斗时,大圣忽见本营中妖猴惊散,自
觉心慌,收了法象,掣棒抽身就走。真君见他败走,大步赶上道:“那里走?趁早归
降,饶你性命!”大圣不恋战,只情跑起。将近洞口,正撞着康、张、姚、李四太
尉,郭申、直健二将军,一齐帅众挡住道:“泼猴!那里走!”大圣慌了手脚,就把
金箍棒捏做绣花针,藏在耳内,摇身一变,变作个麻雀儿,飞在树梢头钉住。那六
兄弟,慌慌张张,前后寻觅不见,一齐吆喝道:“走了这猴精也!走了这猴精也!”

正嚷处,真君到了,问:“兄弟们,赶到那厢不见了?”众神道:“才在这里围
住,就不见了。”二郎圆睁凤目观看,见大圣变了麻雀儿,钉在树上,就收了法象,
撇了神锋,卸下弹弓,摇身一变,变作个饿鹰儿,抖开翅,飞将去扑打。大圣见了,
搜的一翅飞起去,变作一只大鹚老,冲天而去。二郎见了,急抖翎毛,摇身一变,
变作一只大海鹤,钻上云霄来。大圣又将身按下,入涧中,变作一个鱼儿,淬入
水内。二郎赶至涧边,不见踪迹。心中暗想道:“这猢狲必然下水去也,定变作鱼
虾之类。等我再变变拿他。”果一变变作个鱼鹰儿,飘荡在下溜头波面上,等待片
时。那大圣变鱼儿,顺水正游,忽见一只飞禽,似青鹞,毛片不青;似鹭鸶,顶上
无缨;似老鹳,腿又不红:“想是二郎变化了等我哩!……”急转头,打个花就走。
二郎看见道:“打花的鱼儿,似鲤鱼,尾巴不红;似鳜鱼,花鳞不见;似黑鱼,头
上无星;似鲂鱼,鳃上无针。他怎么见了我就回去了?必然是那猴变的。”赶上来,
刷的啄一嘴。那大圣就撺出水中,一变,变作一条水蛇,游近岸,钻入草中。二郎
因他不着,他见水响中,见一条蛇撺出去,认得是大圣,急转身,又变了一只朱
绣顶的灰鹤,伸着一个长嘴,与一把尖头铁钳子相似,径来吃这水蛇。水蛇跳一跳,
又变做一只花鸨,木木樗樗的,立在蓼汀之上。二郎见他变得低贱,——花鸨乃鸟
中至贱至淫之物,不拘鸾、凤、鹰、鸦都与交群——故此不去拢傍,即现原身,走
将去,取过弹弓拽满,一弹子把他打个踵。

那大圣趁着机会,滚下山崖,伏在那里又变,变一座土地庙儿:大张着口,似
个庙门;牙齿变做门扇,舌头变做菩萨,眼睛变做窗棂。只有尾巴不好收拾,竖在
后面,变做一根旗竿。真君赶到崖下,不见打倒的鸨鸟,只有一间小庙;急睁凤眼,
仔细看之,见旗竿立在后面,笑道:“是这猢狲了!他今又在那里哄我。我也曾见庙
宇,更不曾见一个旗竿竖在后面的。断是这畜生弄喧,他若哄我进去,他便一口咬
住。我怎肯进去?等我掣拳先捣窗棂,后踢门扇!”大圣听得,心惊道:“好狠,好
狠!门扇是我牙齿,窗棂是我眼睛;若打了牙,捣了眼,却怎么是好?”扑的一个
虎跳,又冒在空中不见。

真君前前后后乱赶,只见四太尉、二将军,一齐拥至道:“兄长,拿住大圣了
么?”真君笑道:“那猴儿才自变座庙宇哄我。我正要捣他窗棂,踢他门扇,他就
纵一纵,又渺无踪迹。可怪!可怪!”众皆愕然,四望更无形影。真君道:“兄弟们
在此看守巡逻,等我上去寻他。”急纵身驾云,起在半空。见那李天王高擎照妖镜,
与哪吒住立云端,真君道:“天王,曾见那猴王么?”天王道:“不曾上来。我这里
照着他哩。”真君把那赌变化,弄神通,拿群猴一事说毕,却道:“他变庙宇,正打
处,就走了。”李天王闻言,又把照妖镜四方一照,呵呵的笑道:“真君,快去!快
去!那猴使了个隐身法,走出营围,往你那灌江口去也。”二郎听说,即取神锋,回
灌江口来赶。

却说那大圣已到灌江口,摇身一变,变作二郎爷爷的模样,按下云头,径入庙
里。鬼判不能相认,一个个磕头迎接。他坐中间,点查香火:见李虎拜还的三牲,
张龙许下的保福,赵甲求子的文书,钱丙告病的良愿。正看处,有人报:“又一个
爷爷来了。”众鬼判急急观看,无不惊心。真君却道:“有个甚么齐天大圣,才来这
里否?”众鬼判道:“不曾见甚么大圣,只有一个爷爷在里面查点哩。”真君撞进门,
大圣见了,现出本相道:“郎君不消嚷,庙宇已姓孙了。”这真君即举三尖两刃神锋,
劈脸就砍。那猴王使个身法,让过神锋,掣出那绣花针儿,幌一幌,碗来粗细,赶
到前,对面相还。两个嚷嚷闹闹,打出庙门,半雾半云,且行且战,复打到花果山,
慌得那四大天王等众,提防愈紧。这康、张太尉等迎着真君,合心努力,把那美猴
王围绕不题。

话表大力鬼王既调了真君与六兄弟提兵擒魔去后,却上界回奏。玉帝与观音菩
萨、王母并众仙卿,正在灵霄殿讲话,道:“既是二郎已去赴战,这一日还不见回
报。”观音合掌道:“贫僧请陛下同道祖出南天门外,亲去看看虚实如何?”玉帝道:
“言之有理。”即摆驾,同道祖、观音、王母与众仙卿至南天门。早有些天丁、力
士接着,开门遥观,只见众天丁布罗网,围住四面;李天王与哪吒擎照妖镜,立在
空中;真君把大圣围绕中间,纷纷赌斗哩。菩萨开口对老君说:“贫僧所举二郎神
如何?果有神通,已把那大圣围困,只是未得擒拿。我如今助他一功,决拿住他也。”
老君道:“菩萨将甚兵器?怎么助他?”菩萨道:“我将那净瓶杨柳抛下去,打那猴
头;即不能打死,也打个一跌,教二郎小圣好去拿他。”老君道:“你这瓶是个磁器,
准打着他便好,如打不着他的头,或撞着他的铁棒,却不打碎了?你且莫动手,等
我老君助他一功。”菩萨道:“你有甚么兵器?”老君道:“有,有,有。”捋起衣袖,
左膊上取下一个圈子,说道:“这件兵器,尺锟钢抟炼的,被我将还丹点成,养就
一身灵气,善能变化,水火不侵,又能套诸物;一名‘金钢琢’,又名‘金钢套’。
当年过函关,化胡为佛,甚是亏他。早晚最可防身。等我丢下去打他一下。”话毕,
自天门上往下一掼,滴流流,径落花果山营盘里,可可的着猴王头上一下。猴王只
顾苦战七圣,却不知天上坠下这兵器,打中了天灵,立不稳脚,跌了一跤,爬将起
来就跑;被二郎爷爷的细犬赶上,照腿肚子上一口,又扯了一跌。他睡倒在地,骂
道:“这个亡人!你不去妨家长,却来咬老孙!”急翻身爬不起来,被七圣一拥按住,
即将绳索捆绑,使勾刀穿了琵琶骨,再不能变化。

那老君收了金钢琢,请玉帝同观音、王母、众仙等,俱回灵霄殿。这下面四大
天王与李天王诸神,俱收兵拔寨,近前向小圣贺喜,都道:“此小圣之功也!”小圣
道:“此乃天尊洪福,众神威权,我何功之有?”康、张、姚、李道:“兄长不必多
叙,且押这厮去上界见玉帝,请旨发落去也。”真君道:“贤弟,汝等未受天,不
得面见玉帝。教天甲神兵押着,我同天王等上界回旨。你们帅众在此搜山,搜净之
后,仍回灌口。待我请了赏,讨了功,回来同乐。”四太尉、二将军,依言领诺。
这真君与众即驾云头,唱凯歌,得胜朝天。不多时,到通明殿外。天师启奏道:“四
大天王等众已捉了妖猴齐天大圣了。来此听宣。”玉帝传旨,即命大力鬼王与天丁
等众,押至斩妖台,将这厮碎剁其尸。咦!正是:

欺诳今遭刑宪苦,英雄气概等时休。

毕竟不知那猴王性命何如,且听下回分解。