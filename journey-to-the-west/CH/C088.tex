\chapter{禅到玉华施法会~心猿木母授门人}

话说唐僧喜喜欢欢别了郡侯,在马上向行者道:“贤徒,这一场善果,真胜似
比丘国搭救儿童,皆尔之功也。”沙僧道:“比丘国只救得一千一百一十一个小儿,
怎似这场大雨,滂沱浸润,活够者万万千千性命!弟子也暗自称赞大师兄的法力通
天,慈恩盖地也。”八戒笑道:“哥的恩也有,善也有,却只是外施仁义,内包祸心。
但与老猪走,就要作践人。”行者道:“我在那里作践你?”八戒道:“也够了,也
够了!常照顾我捆,照顾我吊,照顾我煮,照顾我蒸!今在凤仙郡施了恩惠与万万之
人,就该住上半年,带挈我吃几顿自在饱饭,却只管催趱行路!”长老闻言,喝道:
“这个呆子,怎么只思量掳嘴!快走路,再莫斗口!”八戒不敢言,掬掬嘴,挑着行
囊,打着哈哈,师徒们奔上大路。此时光景如梭,又值深秋之候。但见:

水痕收,山骨瘦。红叶纷飞,黄花时候。霜晴觉夜长,月白穿窗透。家家烟火
夕阳多,处处湖光寒水溜。白苹香,红蓼茂。桔绿橙黄,柳衰谷秀。荒村雁落碎芦
花,野店鸡声收菽豆。
四众行够多时,又见城垣影影。长老举鞭遥指叫:“悟空,你看那里又有一座城池,
却不知是甚去处。”行者道:“你我俱未曾到,何以知之?且行至边前问人。”

说不了,忽见树丛里走出一个老者,手持竹杖,身着轻衣,足踏一对棕鞋,腰
束一条扁带,慌得唐僧滚鞍下马,上前道个问讯。那老者扶杖还礼道:“长老那方
来的?”唐僧合掌道:“贫僧东土唐朝差往雷音拜佛求经者。今至宝方,遥望城垣,
不知是甚去处,特问老施主指教。”那老者闻言,口称:“有道禅师,我这敝处,乃
天竺国下郡,地名玉华县。县中城主,就是天竺皇帝之宗室,封为玉华王。此王甚
贤,专敬僧道,重爱黎民。老禅师若去相见,必有重敬。”三藏谢了。那老者径穿
树林而去。

三藏才转身对徒弟备言前事。他三人欣喜,扶师父上马。三藏道:“没多路,
不须乘马。”四众遂步至城边街道观看。原来那关厢人家,做买做卖的,人烟凑集,
生意亦甚茂盛。观其声音相貌,与中华无异。三藏吩咐:“徒弟们谨慎。切不可放
肆。”那八戒低了头,沙僧掩着脸,惟孙行者搀着师父。两边人都来争看,齐声叫
道:“我这里只有降龙伏虎的高僧,不曾见降猪伏猴的和尚。”八戒忍不住,把嘴一
掬道:“你们可曾看见降猪王的和尚?”唬得满街上人,跌跌,都往两边闪过。
行者笑道:“呆子,快藏了嘴,莫装扮。仔细脚下过桥。”那呆子低着头,只是笑。
过了吊桥,入城门内,又见那大街上酒楼歌馆,热闹繁华。果然是神州都邑。有诗
为证,诗曰:
锦城铁瓮万年坚,临水依山色色鲜。
百货通湖船入市,千家沽酒店垂帘。
楼台处处人烟广,巷陌朝朝客贾喧。
不亚长安风景好,鸡鸣犬吠亦般般。
三藏心中暗喜道:“人言西域诸番,更不曾到此。细观此景,与我大唐何异!所为极
乐世界,诚此之谓也。”又听得人说,白米四钱一石,麻油八厘一斤,真是五谷丰
登之处。

行够多时,方到玉华王府。府门左右,有长史府、审理厅、典膳所、待客馆。
三藏道:“徒弟,此间是府,等我进去,朝王验牒而行。”八戒道:“师父进去,我
们可好在衙门前站立?”三藏道:“你不看这门上是‘待客馆’三字!你们都去那里
坐下,看有草料,买些喂马。我见了王,倘或赐斋,便来唤你等同享。”行者道:“师
父放心前去。老孙自当理会。”那沙僧把行李挑至馆中。馆中有看馆的人役,见他
们面貌丑陋,也不敢问他,也不敢教他出去,只得让他坐下不题。

却说老师父换了衣帽,拿了关文,径至王府前。早见引礼官迎着问道:“长老
何来?”三藏道:“东土大唐差来大雷音拜佛祖求经之僧,今到贵地,欲倒换关文,
特来朝参千岁。”引礼官即为传奏。那王子果然贤达,即传旨召进。

三藏至殿下施礼。王子即请上殿赐坐。三藏将关文献上。王子看了,又见有各
国印信手押,也就欣然将宝印了,押了花字,收折在案;问道:“国师长老,自你
那大唐至此,历遍诸邦,共有几多路程?”三藏道:“贫僧也未记程途。但先年蒙
观音菩萨在我王御前显身,曾留了颂子,言西方十万八千里。贫僧在路,已经过一
十四遍寒暑矣。”王子笑道:“十四遍寒暑,即十四年了。想是途中有甚耽搁。”三
藏道:“一言难尽!万蛰千魔,也不知受了多少苦楚,才到得宝方!”那王子十分欢
喜。即着典膳官备素斋管待。三藏:“启上殿下,贫僧有三个小徒,在外等候,不
敢领斋,但恐迟误行程。”王子教:“当殿官,快去请长老三位徒弟,进府同斋。”

当殿官随出外相请。都道:“未曾见,未曾见。”有跟随的人道:“待客馆中坐
着三个丑貌和尚,想必是也。”当殿官同众至馆中,即问看馆的道:“那个是大唐取
经僧的高徒?我主有旨,请吃斋也。”八戒正坐打盹,听见一个“斋”字,忍不住,
跳起身来答道:“我们是,我们是。”当殿官一见了,魂飞魄丧,都战战的道:“是
个猪魈,猪魈!”行者听见,一把扯住八戒道:“兄弟,放斯文些,莫撒村野。”那
众官见了行者,又道:“是个猴精,猴精!”沙僧拱手道:“列位休得惊恐。我三人
都是唐僧的徒弟。”众官见了,又道:“灶君,灶君!”孙行者即教八戒牵马,沙僧
挑担,同众入玉华王府。当殿官先入启知。

那王子举目见那等丑恶,却也心中害怕。三藏合掌道:“千岁放心。顽徒虽是
貌丑,却都心良。”八戒朝上唱个喏道:“贫僧问讯了。”王子愈觉心惊。三藏道:“顽
徒都是山野中收来的,不会行礼,万望赦罪。”王子奈着惊恐,教典膳官请众僧官
去暴纱亭吃斋。

三藏谢了恩,辞王下殿,同至亭内,埋怨八戒道:“你这夯货,全不知一毫礼
体!索性不开口,便也罢了;怎么那般粗鲁!一句话,足足冲倒泰山!”行者笑道:“还
是我不唱喏的好,也省些力气。”沙僧道:“他唱喏又不等齐,预先就抒着个嘴吆喝。”
八戒道:“活淘气!活淘气!师父前日教我,见人打个问讯儿是礼;今日打问讯,又
说不好,教我怎的干么!”三藏道:“我教你见了人打个问讯,不曾教你见王子就此
歪缠!常言道:‘物有几等物,人有几等人。’如何不分个贵贱?”正说处,见那典
膳官带领人役,调开桌椅,摆上斋来。师徒们却不言语,各各吃斋。

却说那王子退殿进宫,宫中有三个小王子,见他面容改色,即问道:“父王今
日为何有此惊恐?”王子道:“适才有东土大唐差来拜佛取经的一个和尚,倒换关
文,却一表非凡。我留他吃斋,他说有徒弟在府前,我即命请。少时进来,见我不
行大礼,打个问讯,我已不快。及抬头看时,一个个丑似妖魔,心中不觉惊骇,故
此面容改色。”原来那三个小王子比众不同,一个个好武好强,便就伸拳掳袖道:“莫
敢是那山里走来的妖精,假装人像;待我们拿兵器出去看来!”

好王子,大的个拿一条齐眉棍,第二个轮一把九齿钯,第三个使一根乌油黑棒
子,雄纠纠,气昂昂的,走出王府。吆喝道:“甚么取经的和尚!在那里?”时有典
膳官员人等跪下道:“小王,他们在这暴纱亭吃斋哩。”小王子不分好歹,闯将进去,
喝道:“汝等是人是怪,快早说来,饶你性命!”唬得三藏面容失色,丢下饭碗,躬
着身道:“贫僧乃唐朝来取经者。人也,非怪也。”小王子道:“你便还像个人,那
三个丑的,断然是怪!”八戒只管吃饭不睬。沙僧与行者欠身道:“我等俱是人。面
虽丑而心良,身虽夯而性善。汝三个却是何来,却这样海口轻狂?”旁有典膳等官
道:“三位是我王之子小殿下。”八戒丢了碗道:“小殿下,各拿兵器怎么?莫是要与
我们打哩?”

二王子掣开步,双手舞钯,便要打八戒。八戒嘻嘻笑道:“你那钯只好与我这
钯做孙子罢了!”即揭衣,腰间取出钯来,幌一幌,金光万道;丢了解数,有瑞气
千条;把个王子唬得手软筋麻,不敢舞弄。行者见大的个使一条齐眉棍,跳阿跳的,
即耳朵里取出金箍棒来,幌一幌,碗来粗细,有丈二三长短;着地下一捣,捣了有
三尺深浅,竖在那里,笑道:“我把这棍子送你罢!”那王子听言,即丢了自己棍,
去取那棒,双手尽气力一拔,莫想得动分毫;再又端一端,摇一摇,就如生根一般。
第三个撒起莽性,使乌油杆棒来打。被沙僧一手劈开,取出降妖宝杖,拈一拈,艳
艳光生,纷纷霞亮,唬得那典膳等官,一个个呆呆挣挣,口不能言。三个小王子一
齐下拜道:“神师,神师!我等凡人不识,万望施展一番,我等好拜授也。”行者走
近前,轻轻的把棒拿将起来道:“这里窄狭,不好展手,等我跳在空中,耍一路儿,
你们看看。”

好大圣,唿哨一声,将筋斗一纵,两只脚踏着五色祥云,起在半空,离地约有
三百步高下,把金箍棒丢开个撒花盖顶,黄龙转身,一上一下,左旋右转。起初时
人与棒似锦上添花,次后来不见人,只见一天棒滚。八戒在底下喝声采,也忍不住
手脚,厉声喊道:“等老猪也去耍耍来!”好呆子,驾起风头,也到半空,丢开钯,
上三下四,左五右六,前七后八,满身解数,只听得呼呼风响。正使到热闹处,沙
僧对长老道:“师父,也等老沙去操演操演。”好和尚,双着脚一跳,轮着杖,也起
在空中,只见那锐气氤氲,金光缥缈;双手使降妖杖丢一个丹凤朝阳,饿虎扑食,
紧迎慢挡,捷转忙撺。弟兄三个即展神通,都在那半空中,一齐扬威耀武。这才是:
真禅景象不凡同,大道缘由满太空。
金木施威盈法界,刀圭展转合圆通。
神兵精锐随时显,丹器花生到处崇。
天竺虽高还戒性,玉华王子总归中。
唬得那三个小王子,跪在尘埃。暴纱亭大小人员,并王府里老王子,满城中军民男
女,僧尼道俗,一应人等,家家念佛磕头,户户拈香礼拜。果然是:
见象归真度众僧,人间作福享清平。
从今果正菩提路,尽是参禅拜佛人。
他三个各逞雄才,使了一路,按下祥云,把兵器收了。到唐僧面前问讯,谢了师恩,
各各坐下不题。

那三个小王子,急回宫里,告奏老王道:“父王万千之喜!今有莫大之功也!适
才可曾看见半空中舞弄么?”老王道:“我才见半空霞彩,就于宫院内同你母亲等
众焚香启拜,更不知是那里神仙降聚也。”小王子道:“不是那里神仙,就是那取经
僧三个丑徒弟。一个使金箍铁棒,一个使九齿钉钯,一个使降妖宝杖,把我三个的
兵器,比的通没有分毫。我们教他使一路,他嫌‘地上窄狭,不好支吾,等我起在
空中,使一路你看’。他就各驾云头,满空中祥云缥缈,瑞气氤氲。才然落下,都
坐在暴纱亭里。做儿的十分欢喜,欲要拜他为师,学他手段,保护我邦。此诚莫大
之功!不知父王以为何如?”老王闻言,信心从愿。

当时父子四人,不摆驾,不张盖,步行到暴纱亭。他四众收拾行李,欲进府谢
斋,辞王起行;偶见玉华王父子上亭来倒身下拜,慌得长老舒身,扑地还礼;行者
等闪过旁边,微微冷笑。众拜毕,请四众进府堂上坐。四众欣然而入。老王起身道:
“唐老师父,孤有一事奉求,不知三位高徒,可能容否?”三藏道:“但凭千岁吩
咐,小徒不敢不从。”老王道:“孤先见列位时,只以为唐朝远来行脚僧,其实肉眼
凡胎,多致轻亵。适见孙师、猪师、沙师起舞在空,方知是仙是佛。孤三个犬子,
一生好弄武艺,今谨发虔心,欲拜为门徒,学些武艺。万望老师开天地之心,普运
慈舟,传度小儿,必以倾城之资奉谢。”行者闻言,忍不住呵呵笑道:“你这殿下,
好不会事!我等出家人,巴不得要传几个徒弟。你令郎既有从善之心,切不可说起
分毫之利;但只以情相处,足为爱也。”王子听言,十分欢喜。随命大排筵宴,就
于本府正堂摆列。噫!一声旨意,即刻俱完。但见那:

结彩飘摇,香烟馥郁。戗金桌子挂绞绡,幌人眼目;彩漆椅儿铺锦绣,添座风
光。树果新鲜,茶汤香喷。三五道闲食清甜,一两餐馒头丰洁。蒸酥蜜煎更奇哉,
油札糖浇真美矣。有几瓶香糯素酒,斟出来,赛过琼浆;献几番阳羡仙茶,捧到手,
香欺丹桂。般般品品皆齐备,色色行行尽出奇。
一壁厢叫承应的歌舞吹弹,撮弄演戏。他师徒们并王父子,尽乐一日。不觉天晚,
散了酒席。又叫即于暴纱亭铺没床帏,请师安宿;待明早竭诚焚香,再拜求传武艺。
众皆听从,即备香汤,请师沐浴,众却归寝。此时那:
众鸟高栖万籁沉,诗人下榻罢哦吟。
银河光显天弥亮,野径荒凉草更深。
砧杵叮咚敲别院,关山杳动乡心。
寒蛩声朗知人意,呖呖床头破梦魂。

一宵晚景题过。明早,那老王父子,又来相见这长老。昨日相见,还是王礼,
今日就行师礼。那三个小王子,对行者、八戒、沙僧当面叩头,拜问道:“尊师之
兵器,还借出与弟子们看看。”八戒闻言,欣然取出钉钯,抛在地下。沙僧将宝杖
抛出,倚在墙边。二王子与三王子跳起去便拿,就如蜻蜓撼石柱,一个个挣得红头
赤脸,莫想拿动半分毫。大王子见了,叫道:“兄弟,莫费力了。师父的兵器,俱
是神兵,不知有多少重哩!”八戒笑道:“我的钯也没多重,只有一藏之数,连柄五
千零四十八斤。”三王子问沙僧道:“师父宝杖多重?”沙僧笑道:“也是五千零四
十八斤。”大王子求行者的金箍棒看。行者去耳朵里取出一个针儿来,迎风幌一幌,
就有碗来粗细,直直的竖立面前。那王父子都皆悚惧,众官员个个心惊。三个小王
子礼拜道:“猪师、沙师之兵,俱随身带在衣下,即可取之。孙师为何自耳中取出?
见风即长,何也?”行者笑道:“你不知我这棒不是凡间等闲可有者。这棒是:

鸿蒙初判陶熔铁,大禹神人亲所设。湖海江河浅共深,曾将此棒知之切。开山
治水太平时,流落东洋镇海阙。日久年深放彩霞,能消能长能光洁。老孙有分取将
来,变化无方随口诀。要大弥于宇宙间,要小却似针儿节。棒名如意号金箍,天上
人间称一绝。重该一万三千五百斤,或粗或细能生灭。也曾助我闹天宫,也曾随我
攻地阙。伏虎降龙处处通,炼魔荡怪方方彻。举头一指太阳昏,天地鬼神皆胆怯。
混沌仙传到至今,原来不是凡间铁。”
那王子听言,个个顶礼不尽。三人向前重重拜礼,虔心求授。行者道:“你三人不
知学那般武艺。”王子道:“愿使棍的就学棍,惯使钯的就学钯,爱用杖的就学杖。”
行者笑道:“教便也容易,只是你等无力量,使不得我们的兵器,恐学之不精,如
‘画虎不成反类狗’也。古人云:‘训教不严师之惰,学问无成子之罪。’汝等既有
诚心,可去焚香来拜了天地,我先传你些神力,然后可授武艺。”

三个小王子闻言,满心欢喜。即便亲抬香案,沐手焚香,朝天礼拜。拜毕,请
师传法。行者转下身来,对唐僧行礼道:“告尊师,恕弟子之罪。自当年在两界山
蒙师父大德救脱弟子,秉教沙门,一向西来,虽不曾重报师恩,却也曾渡水登山,
竭尽心力。今来佛国之乡,幸遇贤王三子,投拜我等,欲学武艺。彼既为我等之徒
弟,即为我师之徒孙也。谨禀过我师,庶好传授。”三藏十分大喜。八戒、沙僧见
行者行礼,也那转身朝三藏磕头道:“师父,我等愚鲁,拙口钝腮,不会说话,望
师父高坐法位,也让我两个各招个徒弟耍耍;也是西方路上之忆念。”三藏俱欣然
允之。

行者才教三个王子就于暴纱亭后,静室之间,画了罡斗;教三人都俯伏在内,
一个个瞑目宁神。这里却暗暗念动真言,诵动咒语,将仙气吹入他三人心腹之中,
把元神收归本舍,传与口诀,各授得万千之膂力,运添了火候,却像个脱抬换骨之
法。运遍了子午周天,那三个小王子,方才苏醒,一齐爬将起来,抹抹脸,精神抖
擞,一个个骨壮筋强:大王子就拿得金箍棒,二王子就轮得九齿钯,三王子就举得
降妖杖。

老王见了,欢喜不胜。又排素宴,启谢他师徒四众。就在筵前各传各授:学棍
的演棍,学钯的演钯,学杖的演杖。虽然打几个转身,丢几般解数,终是有些着力:
走一路,便喘气嘘嘘,不能耐久;盖他那兵器都有变化,其进退攻扬,随消随长,
皆有变化自然之妙,此等终是凡夫,岂能以遽及也。当日散了筵宴。

次日,三个王子又来称谢道:“感蒙神师授赐了膂力,纵然轮得师的神器,只
是转换艰难;意欲命工匠依师神器式样,减削斤两,打造一般,未知师父肯容否?”
八戒道:“好!好!好!说得像话。我们的器械,一则你们使不得,二则我们要护法降
魔,正该另造另造。”王子又随宣召铁匠,买办钢铁万斤,就于王府内前院搭厂,
支炉铸造。先一日将钢铁炼熟,次日请行者三人将金箍棒、九齿钯、降妖杖,都取
出放在篷厂之间,看样造作,遂此昼夜不收。

噫!这兵器原是他们随身之宝,一刻不可离者,各藏在身,自有许多光彩护体;
今放在厂院中几日,那霞光有万道冲天,瑞气有千般罩地。其夜有一妖精,离城只
有七十里远近,山唤豹头山,洞唤虎口洞。夜坐之间,忽见霞光瑞气,即驾云头而
看。原是州城之光彩,他按下云来,近前观看,乃是这三般兵器放光。妖精又喜又
爱道:“好宝贝!好宝贝!这是甚人用的,今放在此?……也是我的缘法,拿了去呀!
拿了去呀!”他爱心一动,弄起威风,将三般兵器,一股收之,径转本洞。正是那:
道不须臾离,可离非道也。
神兵尽落空,枉费参修者。

毕竟不知怎生寻得这兵器,且听下回分解。