\chapter{心猿钻透阴阳窍~魔王还归大道真}

却说孙大圣进于洞口,两边观看。只见:

骷髅若岭,骸骨如林。人头发成毡片,人皮肉烂作泥尘。人筋缠在树上,干
焦晃亮如银。真个是尸山血海,果然腥臭难闻。东边小妖,将活人拿了剐肉;西下
泼魔,把人肉鲜煮鲜烹。若非美猴王如此英雄胆,第二个凡夫也进不得他门。
不多时,行入二层门里看时,呀!这里却比外面不同:清奇幽雅,秀丽宽平;左右
有瑶草仙花,前后有乔松翠竹。又行七八里远近,才到三层门。闪着身,偷着眼看
处,那上面高坐三个老妖,十分狞恶。中间的那个生得:

凿牙锯齿,圆头方面。声吼若雷,眼光如电。仰鼻朝天,赤眉飘焰。但行处,
百兽心慌;若坐下,群魔胆战。这一个是兽中王,青毛狮子怪。
左手下那个生得:

凤目金睛,黄牙粗腿。长鼻银毛,看头似尾。圆额皱眉,身躯磊磊。细声如窈
窕佳人,玉面似牛头恶鬼。这一个是藏齿修身多年的黄牙老象。
右手下那一个生得:

金翅鲲头,星睛豹眼。振北图南,刚强勇敢。变生翱翔,笑龙惨。抟风翮百
鸟藏头,舒利爪诸禽丧胆。这个是云程九万的大鹏雕。
那两下列着有百十大小头目,一个个全装披挂,介胄整齐,威风凛凛,杀气腾腾。

行者见了,心中欢喜。一些儿不怕,大踏步,径直进门,把梆铃卸下。朝上叫
声“大王”。三个老魔,笑呵呵问道:“小钻风,你来了?”行者应声道:“来了”。
“你去巡山,打听孙行者的下落何如?”行者道:“大王在上,我也不敢说起。”老
魔道:“怎么不敢说?”行者道:“我奉大王命,敲着梆铃,正然走处,猛抬头,只
看见一个人,蹲在那里磨扛子,还像个开路神,若站将起来,足有十数丈长短。他
就着那涧崖石上,抄一把水,磨一磨,口里又念一声,说他那扛子到此还不曾显个
神通,他要磨明,就来打大王。我因此知他是孙行者,特来报知。”

那老魔闻此言,浑身是汗,唬得战呵呵的道:“兄弟,我说莫惹唐僧。他徒弟
神通广大,预先作了准备,磨棍打我们,却怎生是好?”教:“小的们,把洞外大
小俱叫进来,关了门,让他过去罢。”那头目中有知道的报:“大王,门外小妖,已
都散了。”老魔道:“怎么都散了?想是闻得风声不好也。快早关门!快早关门!”众
妖乒乓把前后门尽皆牢拴紧闭。

行者自心惊道:“这一关了门,他再问我家长里短的事,我对不来,却不弄走
了风,被他拿住?且再唬他一唬,教他开着门,好跑。”又上前道:“大王,他还说
得不好。”老魔道:“他又说甚么?”行者道:“他说拿大大王剥皮,二大王剐骨,
三大王抽筋。你们若关了门不出去啊,他会变化,一时变了个苍蝇儿,自门缝里飞
进,把我们都拿出去,却怎生是好?”老魔道:“兄弟们仔细。我这洞里,递年家
没个苍蝇,但是有苍蝇进来,就是孙行者。”行者暗笑道:“就变个苍蝇唬他一唬,
好开门。”大圣闪在旁边,伸手去脑后拔了一根毫毛,吹一口仙气,叫“变!”即变
做一个金苍蝇,飞去望老魔劈脸撞了一头。那老怪慌了道:“兄弟,不停当,那话
儿进门来了!”惊得那大小群妖,一个个丫钯扫帚,都上前乱扑苍蝇。

这大圣忍不住,的笑出声来。干净他不宜笑,这一笑笑出原嘴脸来了,却
被那第三个老妖魔,跳上前,一把扯住道:“哥哥,险些儿被他瞒了!”老魔道:“贤
弟,谁瞒谁?”三怪道:“刚才这个回话的小妖,不是小钻风,他就是孙行者。必
定撞见小钻风,不知是他怎么打杀了,却变化来哄我们哩。”行者慌了道:“他认得
我了!”即把手摸摸,对老怪道:“我怎么是孙行者?我是小钻风。大王错认了。”老
魔笑道:“兄弟,他是小钻风。他一日三次在面前点卯,我认得他。”又问:“你有
牌儿么?”行者道:“有。”掳着衣服,就拿出牌子。老怪一发认实道:“兄弟,莫
屈了他。”三怪道:“哥哥,你不曾看见他?他才子闪着身,笑了一声,我见他就露
出个雷公嘴来。见我扯住时,他又变作个这等模样。”叫:“小的们,拿绳来!”众
头目即取绳索。三怪把行者扳翻倒,四马攒蹄捆住;揭起衣裳看时,足足是个弼马
温。原来行者有七十二般变化,若是变飞禽、走兽、花木、器皿、昆虫之类,却就
连身子滚去了;但变人物,却只是头脸变了,身子变不过来。果然一身黄毛,两块
红股,一条尾巴。老妖看着道:“是孙行者的身子,小钻风的脸皮。是他了!”教:
“小的们,先安排酒来,与你三大王递个得功之杯。既拿倒了孙行者,唐僧坐定是
我们口里食也。”三怪道:“且不要吃酒。孙行者溜撒,他会逃遁之法,只怕走了。
教小的们抬出瓶来,把孙行者装在瓶里,我们才好吃酒。”

老魔大笑道:“正是!正是!”即点三十六个小妖,入里面开了库房门,抬出瓶
来。你说那瓶有多大?只得二尺四寸高。怎么用得三十六个人抬?那瓶乃阴阳二气之
宝,内有七宝八卦、二十四气,要三十六人,按天罡之数,才抬得动。不一时,将
宝瓶抬出,放在三层门外,展得干净,揭开盖,把行者解了绳索,剥了衣服,就着
那瓶中仙气,飕的一声,吸入里面,将盖子盖上,贴了封皮。却去吃酒道:“猴儿
今番入我宝瓶之中,再莫想那西方之路!若还能够拜佛求经,除是转背摇车,再去
投胎夺舍是。”你看那大小群妖,一个个笑呵呵都去贺功不题。

却说大圣到了瓶中,被那宝贝将身束得小了,索性变化,蹲在当中;半晌,倒
还荫凉,忽失声笑道:“这妖精外有虚名,内无实事。怎么告诵人说这瓶装了人,
一时三刻,化为脓血?若似这般凉快,就住上七八年也无事!”咦!大圣原来不知那
宝贝根由:假若装了人,一年不语,一年荫凉;但闻得人言,就有火来烧了。大圣
未曾说完,只见满瓶都是火焰。幸得他有本事,坐在中间,捻着避火诀,全然不惧。
耐到半个时辰,四周围钻出四十条蛇来咬。行者轮开手,抓将过来,尽力气一攥,
攥做八十段。少时间,又有三条火龙出来,把行者上下盘绕,着实难禁,自觉慌张
无措道:“别事好处,这三条火龙难为。再过一会不出,弄得火气攻心,怎了?”
他想道:“我把身子长一长,券破罢。”好大圣,捻着诀,念声咒,叫“长”!即长
了丈数高下,那瓶紧靠着身,也就长起去;他把身子往下一小,那瓶儿也就小下来
了。行者心惊道:“难、难、难!怎么我长他也长,我小他也小?如之奈何!”说不了,
孤拐上有些疼痛,急伸手摸摸,却被火烧软了,自己心焦道:“怎么好?孤拐烧软了!
弄做个残疾之人了!”忍不住吊下泪来,这正是:
遭魔遇苦怀三藏,着难临危虑圣僧。
道:“师父啊!当年皈正,蒙观音菩萨劝善,脱离天灾,我与你苦历诸山,收殄多怪,
降八戒,得沙僧,千辛万苦,指望同证西方,共成正果。何期今日遭此毒魔,老孙
误入于此,倾了性命,撇你在半山之中,不能前进!想是我昔日名高,故有今朝之
难!”正此凄怆,忽想起:“菩萨当年在蛇盘山曾赐我三根救命毫毛,不知有无,且
等我寻一寻看。”即伸手浑身摸了一把,只见脑后有三根毫毛,十分挺硬。忽喜道:
“身上毛都如彼软熟,只此三根如此硬枪,必然是救我命的。”即便咬着牙,忍着
疼,拔下毛,吹口仙气,叫“变”!一根即变作金钢钻,一根变作竹片,一根变作
绵绳。扳张篾片弓儿,牵着那钻,照瓶底下飕飕的一顿钻,钻成一个眼孔,透进光
亮。喜道:“造化,造化,却好出去也!”才变化出身,那瓶复荫凉了。怎么就凉?
原来被他钻了,把阴阳之气泄了,故此遂凉。

好大圣,收了毫毛,将身一小,就变做个虫儿,十分轻巧,细如须发,长
似眉毛,自孔中钻出;且还不走,径飞在老魔头上钉着。那老魔正饮酒,猛然放下
杯儿道:“三弟,孙行者这回化了么?”三魔笑道:“还到此时哩?”老魔教传令抬
上瓶来。那下面三十六个小妖即便抬瓶,瓶就轻了许多,慌得众小妖报道:“大王,
瓶轻了!”老魔喝道:“胡说!宝贝乃阴阳二气之全功,如何轻了!”内中有一个勉强
的小妖,把瓶提上来道:“你看这不轻了?”老魔揭盖看时,只见里面透亮,忍不
住失声叫道:“这瓶里空者,控也!”大圣在他头上,也忍不住道一声“我的儿啊!
搜者,走也!”众怪听见道:“走了,走了!”即传令:“关门,关门!”

那行者将身一抖,收了剥去的衣服,现本相,跳出洞外。回头骂道:“妖精不
要无礼!瓶子钻破,装不得人了,只好拿了出恭。”喜喜欢欢,嚷嚷闹闹,踏着云头,
径转唐僧处。那长老正在那里撮土为香,望空祷祝。行者且停云头,听他祷祝甚的。
那长老合掌朝天道:
“祈请云霞众位仙,六丁六甲与诸天。
愿保贤徒孙行者,神通广大法无边。”

大圣听得这般言语,更加努力,收敛云光,近前叫道:“师父,我来了!”长老
搀住道:“悟空,劳碌!你远探高山,许久不回,我甚忧虑。端的这山中有何吉凶?”
行者笑道:“师父,才这一去,一则是东土众僧有缘有分,二来是师父功德无量无
边,三也亏弟子法力!”将前项妆钻风、陷瓶里及脱身之事,细陈了一遍。“今得见
尊师之面,实为两世之人也!”长老感谢不尽道:“你这番不曾与妖精赌斗么?”行
者道:“不曾。”长老道:“这等保不得我过山了?”行者是个好胜的人,叫喊道:“我
怎么保你过山不得?”长老道:“不曾与他见个胜负,只这般含糊,我怎敢前进!”
大圣笑道:“师父,你也忒不通变。常言道:‘单丝不线,孤掌难鸣。’那魔三个,
小妖千万,教老孙一人,怎生与他赌斗?”长老道:“寡不敌众,是你一人也难处。
八戒、沙僧他也都有本事,教他们都去,与你协力同心,扫净山路,保我过去罢。”
行者沉吟道:“师言最当。着沙僧保护你,着八戒跟我去罢。”那呆子慌了道:“哥
哥没眼色!我又粗夯,无甚本事,走路扛风,跟你何益?”行者道:“兄弟,你虽无
甚本事,好道也是个人。俗云:‘放屁添风。’你也可壮我些胆气。”八戒道:“也罢,
也罢,望你带挈带挈。但只急溜处,莫捉弄我。”长老道:“八戒在意,我与沙僧在
此。”

那呆子抖擞神威,与行者纵着狂风,驾着云雾,跳上高山,即至洞口。早见那
洞门紧闭,四顾无人。行者上前,执铁棒,厉声高叫道:“妖怪开门,快出来与老
孙打耶!”

那洞里小妖报入,老魔心惊胆战道:“几年都说猴儿狠,话不虚传果是真!”二
老怪在旁问道:“哥哥怎么说?”老魔道:“那行者早间变小钻风混进来,我等不能
相识。幸三贤弟认得,把他装在瓶里。他弄本事,钻破瓶儿,却又摄去衣服走了。
如今在外叫战,谁敢与他打个头仗?”更无一人答应。又问,又无人答,都是那装
聋推哑。老魔发怒道:“我等在西方大路上,忝着个丑名,今日孙行者这般藐视,
若不出去与他见阵,也低了名头。等我舍了这老性命去与他战上三合!三合战得过,
唐僧还是我们口里食;战不过,那时关了门,让他过去吧。”遂取披挂结束了,开
门前走。

行者与八戒在门旁观看,真是好一个怪物:

铁额铜头戴宝盔,盔缨飘舞甚光辉。辉辉掣电双睛亮,亮亮铺霞两鬓飞。勾爪
如银尖且利,锯牙似凿密还齐。身披金甲无丝缝,腰束龙绦有见机。手执钢刀明晃
晃,英雄威武世间稀。一声吆喝如雷震,问道敲门者是谁?
大圣转身道:“是你孙老爷齐天大圣也。”老魔笑道:“你是孙行者?大胆泼猴!我不
惹你,你却为何在此叫战?”行者道:“‘有风方起浪,无潮水自平’。你不惹我,
我好寻你?只因你狐群狗党,结为一伙,算计吃我师父,所以来此施为。”老魔道:
“你这等雄纠纠的,嚷上我门,莫不是要打么?”行者道:“正是。”老魔道:“你
休猖獗!我若调出妖兵,摆开阵势,摇旗擂鼓,与你交战,显得我是坐家虎,欺负
你了。我只与你一个对一个,不许帮丁!”行者闻言,叫:“猪八戒走过,看他把老
孙怎的!”那呆子真个闪在一边。老魔道:“你过来,先与我做个桩儿,让我尽力气
着光头砍上三刀,就让你唐僧过去;假若禁不得,快送你唐僧来,与我做一顿下饭!”
行者闻言笑道:“妖怪,你洞里若有纸笔,取出来,与你立个合同。自今日起,就
砍到明年,我也不与你当真!”

那老魔抖擞威风,丁字步站定,双手举刀,望大圣劈顶就砍。这大圣把头往上
一迎,只闻一声响,头皮儿红也不红。那老魔大惊道:“这猴子好个硬头儿!”
大圣笑道:“你不知。老孙是:

生就铜头铁脑盖,天地乾坤世上无。斧砍锤敲不得碎,幼年曾入老君炉。四斗
星官监临造,二十八宿用工夫。水浸几番不得坏,周围搭板筋铺。唐僧还恐不坚
固,预先又上紫金箍。”
老魔道:“猴儿不要说嘴!看我这二刀来!决不容你性命!”行者道:“不见怎的,左
右也只这般砍罢了。”老魔道:“猴儿,你不知这刀:

金火炉中造,神功百炼熬。锋刃依三略,刚强按六韬。却似苍蝇尾,犹如白蟒
腰。入山云荡荡,下海浪滔滔。琢磨无遍数,煎熬几百遭。深山古洞放,上阵有功
劳。搀着你这和尚天灵盖,一削就是两个瓢!”
大圣笑道:“这妖精没眼色!把老孙认做个瓢头哩!也罢,误砍误让,教你再砍一刀
看怎么。”

那老魔举刀又砍,大圣把头迎一迎,乒乓的劈做两半个;大圣就地打个滚,变
做两个身子。那妖一见慌了,手按下钢刀。猪八戒远远望见,笑道:“老魔好砍两
刀的,却不是四个人了?”老魔指定行者道:“闻你能使分身法,怎么把这法儿拿
出在我面前使?”大圣道:“何为分身法?”老魔道:“为甚么先砍你一刀不动,如
今砍你一刀,就是两个人?”大圣笑道:“妖怪,你切莫害怕。砍上一万刀,还你
二万个人!”老魔道:“你这猴儿,你只会分身,不会收身。你若有本事收做一个,
打我一棍去吧。”大圣道:“不许说谎。你要砍三刀,只砍了我两刀;教我打一棍,
若打了棍半,就不姓孙!”老魔道:“正是,正是。”

好大圣,就把身搂上来,打个滚,依然一个身子,掣棒劈头就打。那老魔举刀
架住道:“泼猴无礼!甚么样个哭丧棒,敢上门打人?”大圣喝道:“你若问我这条
棍,天上地下,都有名声。”老魔道:“怎见名声?”他道:

“棒是九转镔铁炼,老君亲手炉中煅。禹王求得号‘神珍’,四海八河为定验。
中间星斗暗铺陈,两头箝裹黄金片。花纹密布鬼神惊,上造龙纹与凤篆。名号‘灵
阳棒’一条,深藏海藏人难见。成形变化要飞腾,飘摇五色霞光现。老孙得道取归
山,无穷变化多经验。时间要大瓮来粗,或小些微如铁线。粗如南岳细如针,长短
随吾心意变。轻轻举动彩云生,亮
亮飞腾如闪电。攸攸冷气逼人寒,条条杀雾空中现。降龙伏虎谨随身,天涯海角都
游遍。曾将此棍闹天宫,威风打散蟠桃宴。天王赌斗未曾赢,哪吒对敌难交战。棍
打诸神没躲藏,天兵十万都逃窜。雷霆众将护灵霄,飞身打上通明殿。掌朝天使尽
皆惊,护驾仙卿俱搅乱。举棒掀翻北斗宫,回首振开南极院。金阙天皇见棍凶,特
请如来与我见。兵家胜负自如然,困苦灾危无可辨。整整挨排五百年,亏了南海菩
萨劝。大唐有个出家僧,对天发下洪誓愿。枉死城中度鬼魂,灵山会上求经卷。西
方一路有妖魔,行动甚是不方便。已知铁棒世无双,央我途中为侣伴。邪魔汤着赴
幽冥,肉化红尘骨化面。处处妖精棒下亡,论万成千无打算。上方击坏斗牛宫,下
方压损森罗殿。天将曾将九曜追,地府打伤催命判。半空丢下振山川,胜如太岁新
华剑。全凭此棍保唐僧,天下妖魔都打遍!”

那魔闻言,战兢兢舍着性命,举刀就砍。猴王笑吟吟,使铁棒前迎。他两个先
时在洞前撑持,然后跳起去,都在半空里厮杀。这一场好杀:

天河定底神珍棒,棒名如意世间高。夸称手段魔头恼,大杆刀擎法力豪。门外
争持还可近,空中赌斗怎相饶!一个随心更面目,一个立地长身腰。杀得满天云气
重,遍野雾飘摇。那一个几番立意吃三藏,这一个广施法力保唐朝。都因佛祖传经
典,邪正分明恨苦交。

那老魔与大圣斗经二十余合,不分输赢。原来八戒在底下见他两个战到好处,
忍不住掣钯架风,跳将起去,望妖魔劈脸就筑。那魔慌了,不知八戒是个头性子,
冒冒失失的唬人,他只道嘴长耳大,手硬钯凶,败了阵,丢了刀,回头就走。大圣
喝道:“赶上,赶上!”这呆子仗着威风,举着钉钯,即忙赶下怪去。

老魔见他赶的相近,在坡前立定,迎着风头,幌一幌现了原身,张开大口,就
要来吞八戒。八戒害怕,急抽身往草里一钻,也管不得荆针棘刺,也顾不得刮破头
疼,战兢兢的,在草里听着梆声。随后行者赶到,那怪也张口来吞,却中了他的机
关,收了铁棒,迎将上去,被老魔一口吞之。唬得个呆子在草里囊囊咄咄的埋怨道:
“这个弼马温,不识进退!那怪来吃你,你如何不走,反去迎他!这一口吞在肚中,
今日还是个和尚,明日就是个大恭也!”那魔得胜而去。这呆子才钻出草来,溜回
旧路。

却说三藏在那山坡下,正与沙僧盼望,只见八戒喘呵呵的跑来。三藏大惊道:
“八戒,你怎么这等狼狈?悟空如何不见?”呆子哭哭啼啼道:“师兄被妖精一口吞
下肚去了!”三藏听言,唬倒在地。半晌间跌脚拳胸道:“徒弟呀!只说你善会降妖,
领我西天见佛,怎知今日死于此怪之手!苦哉,苦哉!我弟子同众的功劳,如今都化
作尘土矣!”

那师父十分苦痛。你看那呆子,他也不来劝解师父,却叫:“沙和尚,你拿将
行李来,我两个分了罢。”沙僧道:“二哥,分怎的?”八戒道:“分开了,各人散
火:你往流沙河,还去吃人;我往高老庄,看看我浑家。将白马卖了,与师父买个
寿器送终。”长老气的,闻得此言,叫皇天放声大哭。且不题。

却说那老魔吞了行者,以为得计,径回本洞。众妖迎问出战之功。老魔道:“拿
了一个来了。”二魔喜道:“哥哥拿的是谁?”老魔道:“是孙行者。”二魔道:“拿
在何处?”老魔道:“被我一口吞在腹中哩。”第三个魔头大惊道:“大哥啊,我就
不曾吩咐你。孙行者不中吃!”那大圣肚里道:“忒中吃、又禁饥,再不得饿。”慌
得那小妖道:“大王,不好了!孙行者在你肚里说话哩!”老魔道:“怕他说话!有本
事吃了他,没本事摆布他不成?你们快去烧些盐白汤,等我灌下肚去,把他哕出来,
慢慢的煎了吃酒。”小妖真个冲了半盆盐汤。老怪一饮而干,洼着口,着实一呕,
那大圣在肚里生了根,动也不动;却又拦着喉咙,往外又吐,吐得头晕眼花,黄胆
都破了,行者越发不动。老魔喘息了,叫声:“孙行者,你不出来?”行者道:“早
哩,正好不出来哩!”老魔道:“你怎么不出?”行者道:“你这妖精,甚不通变。
我自做和尚,十分淡薄:如今秋凉,我还穿个单直裰。这肚里倒暖,又不透风,等
我住过冬才好出来。”

众妖听说,都道:“大王,孙行者要在你肚里过冬哩!”老魔道:“你要过冬,
我就打起禅来,使个搬运法,一冬不吃饭,就饿杀那弼马温!”大圣道:“我儿子,
你不知事!老孙保唐僧取经,从广里过,带了个折迭锅儿,进来煮杂碎吃。将你这
里边的肝、肠、肚、肺,细细儿受用,还够盘缠到清明哩!”那二魔大惊道:“哥啊,
这猴子他干得出来!”三魔道:“哥啊,吃了杂碎也罢,不知在那里支锅。”行者道:
“三叉骨上好支锅。”三魔道:“不好了!假若支起锅,烧动火烟,炒到鼻孔里,打
嚏喷么?”行者笑道:“没事!等老孙把金箍棒往顶门里一搠,搠个窟窿:一则当天
窗,二来当烟洞。”

老魔听说,虽说不怕,却也心惊。只得硬着胆叫:“兄弟们,莫怕!把我那药酒
拿来,等我吃几钟下去,把猴儿药杀了罢!”行者暗笑道:“老孙五百年前大闹天宫
时,吃老君丹,玉皇酒,王母桃,及凤髓龙肝,那样东西我不曾吃过?是甚么药酒,
敢来药我?”那小妖真个将药酒筛了两壶,满满斟了一钟,递与老魔。老魔接在手
中,大圣在肚里就闻得酒香,道:“不要与他吃!”好大圣,把头一扭,变做个喇叭
口子,张在他喉咙之下。那怪的咽下,被行者的接吃了。第二钟咽下,被行者
的又接吃了。一连咽了七八钟,都是他接吃了。老魔放下钟道:“不吃了。这酒
常时吃两钟,腹中如火;却才吃了七八钟,脸上红也不红!”原来这大圣吃不多酒,
接了他七八钟吃了,在肚里撒起酒风来,不住的支架子,跌四平,踢飞脚;抓住肝
花打秋千,竖蜻蜓,翻根头乱舞。那怪物疼痛难禁,倒在地下。

毕竟不知死活如何,且听下回分解。