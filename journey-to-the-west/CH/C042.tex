\chapter{大圣殷勤拜南海~观音慈善缚红孩}

话说那六健将出洞门,径往西南上,依路而走。行者心中暗想道:“他要请老
大王吃我师父,老大王断是牛魔王。我老孙当年与他相会,真个意合情投,交游甚
厚。至如今我归正道,他还是邪魔。虽则久别,还记得他模样,且等老孙变作牛魔
王,哄他一哄,看是何如。”好行者,躲离了六个小妖,展开翅,飞向前边,离小
妖有十数里远近,摇身一变,变作个牛魔王;拔下几根毫毛,叫“变!”即变作几
个小妖。在那山凹里,驾鹰牵犬,搭弩张弓,充作打围的样子,等候那六健将。

那一伙厮拖厮扯,正行时,忽然看见牛魔王坐在中间,慌得兴烘掀、掀烘兴扑
的跪下道:“老大王爷爷在这里也。”那云里雾、雾里云、急如火、快如风都是肉眼
凡胎,那里认得真假,也就一同跪倒,磕头道:“爷爷!小的们是火云洞圣婴大王处
差来,请老大王爷爷去吃唐僧肉,寿延千纪哩。”行者借口答道:“孩儿们起来,同
我回家去,换了衣服来也。”小妖叩头道:“望爷爷方便,不消回府罢。路程遥远,
恐我大王见责。小的们就此请行。”行者笑道:“好乖儿女。也罢,也罢,向前开路,
我和你去来。”六怪抖擞精神,向前喝路。大圣随后而来。

不多时,早到了本处。快如风、急如火撞进洞里,报:“大王,老大王爷爷来
了。”妖王欢喜道:“你们却中用,这等来的快。”即便叫:“各路头目,摆队伍,开
旗鼓,迎接老大王爷爷。”满洞群妖,遵依旨令,齐齐整整,摆将出去。这行者昂
昂烈烈,挺着胸脯,把身子抖了一抖,却将那架鹰犬的毫毛,都收回身上。拽开大
步,径走入门里,坐在南面当中。

红孩儿当面跪下,朝上叩头道:“父王,孩儿拜揖。”行者道:“孩儿免礼。”那
妖王四大拜拜毕,立于下手。行者道:“我儿,请我来有何事?”妖王躬身道:“孩
儿不才,昨日获得一人,乃东土大唐和尚。常听得人讲,他是一个十世修行之人,
有人吃他一块肉,寿似蓬瀛不老仙。愚男不敢自食,特请父王同享唐僧之肉,寿延
千纪。”行者闻言,打了个失惊道:“我儿,是那个唐僧?”妖王道:“是往西天取
经的人也。”行者道:“我儿,可是孙行者师父么?”妖王道:“正是。”行者摆手摇
头道:“莫惹他,莫惹他!别的还好惹,孙行者是那样人哩,我贤郎,你不曾会他?
那猴子神通广大,变化多端。他曾大闹天宫。玉皇上帝差十万天兵,布下天罗地网,
也不曾捉得他。你怎么敢吃他师父!快早送出去还他,不要惹那猴子。他若打听着
你吃了他师父,他也不来和你打,他只把那金箍棒往山腰里搠个窟窿,连山都掬了
去。我儿,弄得你何处安身,教我倚靠何人养老!”

妖王道:“父王说那里话,长他人志气,灭孩儿的威风。那孙行者共有兄弟三
人,领唐僧在我半山之中,被我使个变化,将他师父摄来。他与那猪八戒当时寻到
我的门前,讲甚么攀亲托熟之言,被我怒发冲天,与他交战几合,也只如此,不见
甚么高作。那猪八戒刺邪里就来助战,是孩儿吐出三昧真火,把他烧败了一阵。慌
得他去请四海龙王助雨,又不能灭得我三昧真火;被我烧了一个小发昏,连忙着猪
八戒去请南海观音菩萨。是我假变观音,把猪八戒赚来,见吊在如意袋中,也要蒸
他与众小的们吃哩。那行者今早又来我的门首吆喝,我传令教拿他,慌得他把包袱
都丢下走了。却才去请父王来看看唐僧活象,方可蒸与你吃,延寿长生不老也。”

行者笑道:“我贤郎啊,你只知有三昧火赢得他,不知他有七十二般变化哩!”
妖王道:“凭他怎么变化,我也认得。谅他决不敢进我门来。”行者道:“我儿,你
虽然认得他,他却不变大的,如狼大象,恐进不得你门;他若变作小的,你却难
认。”妖王道:“凭他变甚小的。我这里每一层门上,有四五个小妖把守,他怎生得
入!”行者道:“你是不知。他会变苍蝇、蚊子、虼,或是蜜蜂、蝴蝶并虫等
项,又会变我模样,你却那里认得?”妖王道:“勿虑;他就是铁胆铜心,也不敢
近我门来也。”

行者道:“既如此说,贤郎甚有手段,实是敌得他过,方来请我吃唐僧的肉;
奈何我今日还不吃哩。”妖王道:“如何不吃?”行者道:“我近来年老,你母亲常劝
我作些善事。我想无甚作善,且持些斋戒。”妖王道:“不知父王是长斋,是月斋?”
行者道:“也不是长斋,也不是月斋,唤做‘雷斋’。每月只该四日。”妖王问:“是
那四日?”行者道:“三辛逢初六。今朝是辛酉日,一则当斋,二来酉不会客。且
等明日,我去亲自刷洗蒸他,与儿等同享罢。”

那妖王闻言,心中暗想道:“我父王平日吃人为生,今活够有一千余岁,怎么
如今又吃起斋来了?想当初作恶多端,这三四日斋戒,那里就积得过来。此言有假,
可疑,可疑!”即抽身走出二门之下,叫六健将来问:“你们老大王是那里请来的?”
小妖道:“是半路请来的。”妖王道:“我说你们来的快。不曾到家么?”小妖道:
“是,不曾到家。”妖王道:“不好了,着了他假也!这不是老大王!”小妖一齐跪下
道:“大王,自家父亲,也认不得?”妖王道:“观其形容动静都像,只是言语不像。
只怕着了他假,吃了人亏。你们都要仔细:会使刀的,刀要出鞘;会使枪的,枪要
磨明;会使棍的,使棍;会使绳的,使绳。待我再去问他,看他言语如何。若果是
老大王,莫说今日不吃,明日不吃,便迟个月何妨!假若言语不对,只听我哏的一
声,就一齐下手。”群魔各各领命讫。

这妖王复转身到于里面,对行者当面又拜。行者道:“孩儿,家无常礼,不须
拜;但有甚话,只管说来。”妖王伏于地下道:“愚男一则请来奉献唐僧之肉,二来
有句话儿上请。我前日闲行,驾祥光,直至九霄空内,忽逢着祖廷道陵张先生。”
行者道:“可是做天师的张道陵么?”妖王道:“正是。”行者问曰:“有甚话说?”
妖王道:“他见孩儿生得五官周正,三停平等,他问我是几年、那月、那日、那时
出世。儿因年幼,记得不真。先生子平精熟,要与我推看五星。今请父王,正欲问
此。倘或下次再得会他,好烦他推算。”行者闻言,坐在上面暗笑道:“好妖怪呀!
老孙自归佛果,保唐师父,一路上也捉了几个妖精,不似这厮克剥。他问我甚么家
长礼短,少米无柴的话说,我也好信口捏脓答他。他如今问我生年月日,我却怎么
知道!”好猴王,也十分乖巧:巍巍端坐中间,也无一些儿惧色,面上反喜盈盈的
笑道:“贤郎请起。我因年老,连日有事不遂心怀,把你生时果偶然忘了。且等到
明日回家,问你母亲便知。”

妖王道:“父王把我八个字时常不离口论说,说我有同天不老之寿,怎么今日
一旦忘了!岂有此理!必是假的!”哏的一声,群妖枪刀簇拥,望行者没头没脸的札
来。这大圣使金箍棒架住了,现出本象,对妖精道:“贤郎,你却没理。那里儿子
好打爷的?”那妖王满面羞惭,不敢回视。行者化金光,走出他的洞府。小妖道:
“大王,孙行者走了。”妖王道:“罢,罢,罢!让他走了罢,我吃他这一场亏也!且
关了门,莫与他打话,只来刷洗唐僧,蒸吃便罢。”

却说那行者搴着铁棒,呵呵大笑,自涧那边而来。沙僧听见,急出林迎着道:
“哥啊,这半日方回,如何这等哂笑,想救出师父来也?”行者道:“兄弟,虽不
曾救得师父,老孙却得个上风来了。”沙僧道:“甚么上风?”行者道:“原来猪八
戒被那怪假变观音哄将回来,吊于皮袋之内。我欲设法救援,不期他着甚么六健将
去请老大王来吃师父肉。是老孙想着他老大王必是牛魔王,就变了他的模样,充将
进去,坐在中间。他叫父王,我就应他;他便叩头,我就直受。着实快活,果然得
了上风!”沙僧道:“哥啊,你便图这般小便宜,恐师父性命难保。”行者道:“不须
虑,等我去请菩萨来。”沙僧道:“你还腰疼哩。”行者道:“我不疼了。古人云:‘人
逢喜事精神爽。’你看着行李、马匹,等我去。”沙僧道:“你置下仇了,恐他害我
师父。你须快去快来。”行者道:“我来得快,只消顿饭时,就回来矣。”

好大圣,说话间躲离了沙僧,纵筋斗云,径投南海。在那半空里,那消半个时
辰,望见普陀山景。须臾,按下云头,直至落伽崖上。端肃正行,只见二十四路诸
天迎着道:“大圣,那里去?”行者作礼毕,道:“要见菩萨。”诸天道:“少停,容
通报。”时有鬼子母诸天来潮音洞外报道:“菩萨得知,孙悟空特来参见。”菩萨闻
报,即命进去。

大圣敛衣皈命,捉定步,径入里边,见菩萨倒身下拜。菩萨道:“悟空,你不
领金蝉子西方求经去,却来此何干?”行者道:“上告菩萨。弟子保护唐僧前行,
至一方,乃号山枯松涧火云洞。有一个红孩儿妖精,唤作圣婴大王,把我师父摄去。
是弟子与猪悟能等寻至门前,与他交战。他放出三昧火来,我等不能取胜,救不出
师父。急上东洋大海,请到四海龙王,施雨水,又不能胜火,把弟子都熏坏了,几
乎丧了残生。”菩萨道:“既他是三昧火,神通广大,怎么去请龙王,不来请我?”
行者道:“本欲来的,只是弟子被烟熏了,不能驾云,却教猪八戒来请菩萨。”菩萨
道:“悟能不曾来呀。”行者道:“正是。未曾到得宝山,被那妖精假变做菩萨模样,
把猪八戒又赚入洞中,现吊在一个皮袋里,也要蒸吃哩。”

菩萨听说,心中大怒道:“那泼妖敢变我的模样!”恨了一声,将手中宝珠净瓶
往海心里扑的一掼,唬得那行者毛骨竦然,即起身侍立下面,道:“这菩萨火性不
退,好是怪老孙说的话不好,坏了他的德行,就把净瓶掼了。可惜,可惜!早知送
了我老孙,却不是一件大人事?”

说不了,只见那海当中,翻波跳浪,钻出个瓶来。原来是一个怪物驮着出来。
行者仔细看那驮瓶的怪物,怎生模样:

根源出处号帮泥,水底增光独显威。世隐能知天地性,安藏偏晓鬼神机。藏身
一缩无头尾,展足能行快似飞。文王画卦曾元卜,常纳庭台伴伏羲。云龙透出千般
俏,号水推波把浪吹。条条金线穿成甲,点点装成彩玳瑁。九宫八卦袍披定,散碎
铺遮绿灿衣。生前好勇龙王幸,死后还驮佛祖碑。要知此物名和姓,兴风作浪恶乌
龟。
那龟驮着净瓶,爬上崖边,对菩萨点头二十四点,权为二十四拜。行者见了,暗笑
道:“原来是看瓶的。想是不见瓶,就问他要。”菩萨道:“悟空,你在下面说甚么?”
行者道:“没说甚么。”菩萨教:“拿上瓶来。”这行者即去拿瓶,唉!莫想拿得他动。
好便似蜻蜓撼石柱,怎生摇得半分毫?行者上前跪下道:“菩萨,弟子拿不动。”菩
萨道:“你这猴头,只会说嘴。瓶儿你也拿不动,怎么去降妖缚怪?”行者道:“不
瞒菩萨说。平日拿得动,今日拿不动。想是吃了妖精亏,筋力弱了。”菩萨道:“常
时是个空瓶;如今是净瓶抛下海去,这一时间,转过了三江五湖,八海四渎,溪源
潭洞之间,共借了一海水在里面。你那里有架海的斤量,此所以拿不动也。”行者
合掌道:“是弟子不知。”

那菩萨走上前,将右手轻轻的提起净瓶,托在左手掌上。只见那龟点点头,钻
下水去了。行者道:“原来是个养家看瓶的夯货!”菩萨坐定道:“悟空,我这瓶中
甘露水浆,比那龙王的私雨不同,能灭那妖精的三昧火。待要与你拿了去,你却拿
不动;待要着善财龙女与你同去,你却又不是好心,专一只会骗人。你见我这龙女
貌美,净瓶又是个宝物,你假若骗了去,却那有工夫又来寻你?你须是留些甚么东
西作当。”行者道:“可怜!菩萨这等多心。我弟子自秉沙门,一向不干那样事了。
你教我留些当头,却将何物?我身上这件绵布直裰,还是你老人家赐的。这条虎皮
裙子,能值几个铜钱?这根铁棒,早晚却要护身。但只是头上这个箍儿,是个金的,
却又被你弄了个方法儿长在我头上,取不下来。你今要当头,情愿将此为当。你念
个松箍儿咒,将此除去罢;不然,将何物为当?”菩萨道:“你好自在啊!我也不要
你的衣服、铁棒、金箍;只将你那脑后救命的毫毛拔一根与我作当罢。”行者道:“这
毫毛,也是你老人家与我的。但恐拔下一根,就拆破群了,又不能救我性命。”菩
萨骂道:“你这猴子!你便一毛也不拔,教我这善财也难舍。”行者笑道:“菩萨,你
却也多疑。正是‘不看僧面看佛面’。千万救我师父一难罢!”那菩萨:
逍遥欣喜下莲台,云步香飘上石崖。
只为圣僧遭障害,要降妖怪救回来。

孙大圣十分欢喜,请观音出了潮音仙洞。诸天大神都列在普陀岩上。菩萨道:
“悟空,过海。”行者躬身道:“请菩萨先行。”菩萨道:“你先过去。”行者磕头道:
“弟子不敢在菩萨面前施展。若驾筋斗云啊,掀露身体,恐菩萨怪我不敬。”菩萨
闻言,即着善财龙女去莲花池里,劈一瓣莲花,放在石岩下边水上,教行者:“你
上那莲花瓣儿,我渡你过海。”行者见了道:“菩萨,这花瓣儿,又轻又薄,如何载
得我起!这一翻跌下水去,却不湿了虎皮裙?走了硝,天冷怎穿!”菩萨喝道:“你
且上去看!”行者不敢推辞,舍命往上跳。果然先见轻小,到上面比海船还大三分。
行者欢喜道:“菩萨,载得我了。”菩萨道:“既载得,如何不过去?”行者道:“又
没个篙、桨、篷、桅,怎生得过?”菩萨道:“不用。”只把他一口气吹开吸拢,又
着实一口气,吹过南洋苦海,得登彼岸。行者却脚实地,笑道:“这菩萨卖弄神
通,把老孙这等呼来喝去,全不费力也!”

那菩萨吩咐概众诸天各守仙境,着善财龙女闭了洞门,他却纵祥云,躲离普陀
岩,到那边叫:“惠岸何在?”惠岸乃托塔李天王第二个太子,俗名木叉是也。乃
菩萨亲传授的徒弟,不离左右,称为护法惠岸行者,即对菩萨合掌伺候。菩萨道:
“你快上界去,见你父王,问他借天罡刀来一用。”惠岸道:“师父用着几何?”菩
萨道:“全副都要。”

惠岸领命,即驾云头,径入南天门里,到云楼宫殿,见父王下拜。天王见了,
问:“儿从何来?”木叉道:“师父是孙悟空请来降妖,着儿拜上父王,将天罡刀借
了一用。”天王即唤哪吒将刀取三十六把,递与木叉。木叉对哪吒说:“兄弟,你回
去多拜上母亲:我事紧急,等送刀来再磕头罢。”忙忙相别,按落祥光,径至南海,
将刀捧与菩萨。

菩萨接在手中,抛将去,念个咒语,只见那刀化作一座千叶莲台。菩萨纵身上
去,端坐在中间。行者在旁暗笑道:“这菩萨省使俭用。那莲花池里有五色宝莲台,
舍不得坐将来,却又问别人去借。”菩萨道:“悟空,休言语,跟我来也。”却才都
驾着云头,离了海上。白鹦哥展翅前飞,孙大圣与惠岸随后。

顷刻间,早见一座山头。行者道:“这山就是号山了。从此处到那妖精门首,
约摸有四百余里。”菩萨闻言,即命住下祥云;在那山头上念一声“”字咒语,
只见那山左山右,走出许多神鬼,却乃是本山土地众神,都到菩萨宝莲座下磕头。
菩萨道:“汝等俱莫惊张。我今来擒此魔王。你与我把这团围打扫干净,要三百里
远近地方,不许一个生灵在地。将那窝中小兽,窟内雏虫,都送在巅峰之上安生。”
众神遵依而退。须臾间,又来回复。菩萨道:“既然干净,俱各回祠。”遂把净瓶扳
倒,唿喇喇倾出水来,就如雷响。真个是:

漫过山头,冲开石壁;漫过山头如海势,冲开石壁似汪洋。黑雾涨天全水气,
沧波影日幌寒光。遍崖冲玉浪,满海长金莲。菩萨大展降魔法,袖中取出定身禅。
化做落伽仙景界,真如南海一般般。秀蒲挺出昙花嫩,香草舒开贝叶鲜。紫竹几竿
鹦鹉歇,青松数簇鹧鸪喧。万叠波涛连四野,只闻风吼水漫
天。
孙大圣见了,暗中赞叹道:“果然是一个大慈大悲的菩萨!若老孙有此法力,将瓶儿
望山一倒,管甚么禽兽蛇虫哩!”菩萨叫:“悟空,伸手过来。”行者即忙敛袖,将
左手伸出。菩萨拔杨柳枝,蘸甘露,把他手心里写一个“迷”字。教他:“捏着拳
头,快去与那妖精索战,许败不许胜。败将来我这跟前,我自有法力收他。”

行者领命。返云光,径来至洞口。一只手使拳,一只手使棒,高叫道:“妖怪
开门!”那些小妖,又进去报道:“孙行者又来了!”妖王道:“紧关了门,莫睬他!”
行者叫道:“好儿子!把老子赶在门外,还不开门!”小妖又报道:“孙行者骂出那话
儿来了!”妖王只教:“莫睬他!”行者叫两次,见不开门,心中大怒,举铁棒,将
门一下打了一个窟窿。慌得那小妖跌将进去道:“孙行者打破门了!”

妖王见报几次,又听说打破前门,急纵身跳将出去,挺长枪,对行者骂道:“这
猴子,老大不识起倒!我让你得些便宜,你还不知尽足,又来欺我!打破我门,你该
个甚么罪名?”行者道:“我儿,你赶老子出门,你该个甚么罪名?”

那妖王羞怒,绰长枪劈胸便刺;这行者举铁棒,架隔相还。一番搭上手,斗经
四五个回合,行者捏着拳头,拖着棒,败将下来。那妖王立在山前道:“我要刷洗
唐僧去哩!”行者道:“好儿子,天看着你哩,你来!”那妖精闻言,愈加嗔怒,喝
一声,赶到面前,挺枪又刺。这行者轮棒又战几合,败阵又走。那妖王骂道:“猴
子,你在前有二三十合的本事,你怎么如今正斗时就要走了,何也?”行者笑道:
“贤郎,老子怕你放火。”妖精道:“我不放火了,你上来。”行者道:“既不放火,
走开些。好汉子莫在家门前打人。”那妖精不知是诈,真个举枪又赶。行者拖了棒,
放了拳头。那妖王着了迷乱,只情追赶。前走的如流星过度,后走的如弩箭离弦。

不一时,望见那菩萨了。行者道:“妖精,我怕你了。你饶我罢。你如今赶至
南海观音菩萨处,怎么还不回去?”那妖王不信,咬着牙,只管赶来。行者将身一
幌,藏在那菩萨的神光影里。

这妖精见没了行者。走近前,睁圆眼,对菩萨道:“你是孙行者请来的救兵么?”
菩萨不答应。妖王拈转长枪,喝道:“咄!你是孙行者请来的救兵么?”菩萨也不答
应。妖精望菩萨劈心刺一枪来。那菩萨化道金光,径走上九霄空内。行者跟定道:
“菩萨,你好欺伏我罢了!那妖精再三问你,你怎么推聋装痖,不敢做声,被他一
枪搠走了,却把那个莲台都丢下耶!”菩萨只教:“莫言语,看他再要怎的。”此时
行者与木叉俱在空中,并肩同看。只见那妖呵呵冷笑道:“泼猴头,错认了我也!他
不知把我圣婴当作个甚人。几番家战我不过,又去请个甚么脓包菩萨来,却被我一
枪,搠得无形无影去了,又把个宝莲台儿丢了。且等我上去坐坐。”好妖精,他也
学菩萨,盘手盘脚的,坐在当中。行者看见道:“好,好,好!莲花台儿好送人了!”
菩萨道:“悟空,你又说甚么?”行者道:“说甚,说甚,莲台送了人了!那妖精坐
放臀下,终不得你还要哩?”菩萨道:“正要他坐哩。”行者道:“他的身躯小巧,
比你还坐得稳当。”菩萨叫:“莫言语,且看法力。”

他将杨柳枝往下指定,叫一声“退!”只见那莲台花彩俱无,祥光尽散,原来
那妖王坐在刀尖之上。即命木叉:“使降妖杵,把刀柄儿打打去来。”那木叉按下云
头,将降魔杵,如筑墙一般,筑了有千百余下。那妖精,穿通两腿刀尖出,血流成
汪皮肉开。好怪物,你看他咬着牙,忍着痛,且丢了长枪,用手将刀乱拔。行者却
道:“菩萨啊,那怪物不怕痛,还拔刀哩。”菩萨见了,唤上木叉,“且莫伤他生命。”
却又把杨柳枝垂下,念声“”字咒语,那天罡刀都变做倒须钩儿,狼牙一般,莫
能褪得。那妖精却才慌了,扳着刀尖,痛声苦告道:“菩萨,我弟子有眼无珠,不
识你广大法力。千乞垂慈,饶我性命!再不敢恃恶,愿入法门戒行也。”

菩萨闻言,却与二行者、白鹦哥低下金光,到了妖精面前。问道:“你可受吾
戒行么?”妖王点头滴泪道:“若饶性命,愿受戒行。”菩萨道:“你可入我门么?”
妖王道:“果饶性命,愿入法门。”菩萨道:“既如此,我与你摩顶受戒。”就袖中取
出一把金剃头刀儿,近前去,把那怪分顶剃了几刀,剃作一个太山压顶,与他留下
三个顶搭,挽起三个窝角揪儿。行者在旁笑道:“这妖精大晦气!弄得不男不女,不
知像个甚么东西!”菩萨道:“你今既受我戒,我却也不慢你,称你做善财童子,如
何?”那妖点头受持,只望饶命。菩萨却用手一指,叫声“退!”撞的一声,天罡
刀都脱落尘埃,那童子身躯不损。

菩萨叫:“惠岸,你将刀送上天宫,还你父王,莫来接我,先到普陀岩会众诸
天等候。”那木叉领命,送刀上界,回海不题。

却说那童子野性不定,见那腿疼处不疼,臀破处不破,头挽了三个揪儿,他走
去绰起长枪,望菩萨道:“那里有甚真法力降我!原来是个掩样术法儿,不受甚戒,
看枪!”望菩萨劈脸刺来。恨得个行者轮铁棒要打。菩萨只叫:“莫打,我自有惩治。”
却又袖中取出一个金箍儿来道:“这宝贝原是我佛如来赐我往东土寻取经人的‘金、
紧、禁’三个箍儿。紧箍儿,先与你戴了;禁箍儿,收了守山大神;这个金箍儿,
未曾舍得与人,今观此怪无礼,与他罢。”好菩萨,将箍儿迎风一幌,叫声“变!”
即变作五个箍儿,望童子身上抛了去,喝声“着!”一个套在他头顶上,两个套在
他左右手上,两个套在他左右脚上。菩萨道:“悟空,走开些,等我念念金箍儿咒。”
行者慌了道:“菩萨呀,请你来此降妖,如何却要咒我?”菩萨道:“这篇咒,不是
紧箍儿咒,咒你的;是金箍儿咒,咒那童子的。”行者却才放心,紧随左右,听得
他念咒。菩萨捻着诀,默默的念了几遍,那妖精搓耳揉腮,攒蹄打滚。正是:
一句能通遍沙界,广大无边法力深。

毕竟不知那童子怎的皈依,且听下回分解。