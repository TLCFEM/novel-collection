\chapter{黑河妖孽擒僧去~西洋龙子捉鼍回}

却说那菩萨念了几遍,却才住口,那妖精就不疼了。又正性起身看处,颈项里
与手足上都是金箍,勒得疼痛
,便就除那箍儿时,莫想褪得动分毫。这宝贝已此是见肉生根,越抹越痛。行者笑
道:“我那乖乖,菩萨恐你养不大,与你戴个颈圈镯头哩。”那童子闻此言,又生烦
恼,就此绰起枪来,望行者乱刺。行者急闪身,立在菩萨后面,叫:“念咒!念咒!”

那菩萨将杨柳枝儿,蘸了一点甘露,洒将去,叫声“合!”只见他丢了枪,一
双手合掌当胸,再也不能开放。至今留了一个“观音扭”,即此意也。那童子开不
得手,拿不得枪,方知是法力深微。没奈何,才纳头下拜。

菩萨念动真言,把净瓶倒,将那一海水,依然收去,更无半点存留。对行者
道:“悟空,这妖精已是降了,却只是野心不定,等我教他一步一拜,只拜到落伽
山,方才收法。你如今快早去洞中,救你师父去来!”行者转身叩头道:“有劳菩萨
远涉,弟子当送一程。”菩萨道:“你不消送,恐怕误了你师父性命。”行者闻言,
欢喜叩别。那妖精早归了正果,五十三参,参拜观音。

且不题善菩萨收了童子,却说那沙僧久坐林间,盼望行者不到;将行李捎在马
上,一只手执着降妖宝杖,一只手牵着缰绳,出松林向南观看,只见行者欣喜而来。

沙僧迎着道:“哥哥,你怎么去请菩萨,此时才来!焦杀我也!”行者道:“你还
做梦哩。老孙已请了菩萨,降了妖怪。”行者却将菩萨的法力,备陈了一遍。沙僧
十分欢喜道:“救师父去也!”

他两个才跳过涧去,撞到门前,拴下马匹。举兵器齐打入洞里,剿净了群妖,
解下皮袋,放出八戒来。那呆子谢了行者道:“哥哥,那妖精在那里?等我去筑他几
钯,出出气来!”行者道:“且寻师父去。”

三人径至后边,只见师父赤条条,捆在院中哭哩。沙僧连忙解绳,行者即取衣
服穿上。三人跪在面前道:“师父吃苦了。”三藏谢道:“贤徒啊,多累你等。怎生
降得妖魔也?”行者又将请菩萨,收童子之言,备陈一遍。三藏听得,即忙跪下,
朝南礼拜。行者道:“不消谢他,转是我们与他作福,收了一个童子。”——如今说
童子拜观音,五十三参,参参见佛,即此是也。——教沙僧,将洞内宝物收了。且
寻米粮,安排斋饭,管待了师父。那长老得性命全亏孙大圣,取真经只靠美猴精。
师徒们出洞来,攀鞍上马,找大路,笃志投西。

行经一个多月,忽听得水声振耳。三藏大惊道:“徒弟呀,又是那里水声?”
行者笑道:“你这老师父,忒也多疑,做不得和尚。我们一同四众,偏你听见甚么
水声。你把那《多心经》又忘了也?”唐僧道:“《多心经》乃浮屠山乌巢禅师口授,
共五十四句,二百七十个字。我当时耳传,至今常念,你知我忘了那句儿?”行者
道:“老师父,你忘了‘无眼耳鼻舌身意’。我等出家人,眼不视色,耳不听声,鼻
不嗅香,舌不尝味,身不知寒暑,意不存妄想——如此谓之祛褪六贼。你如今为求
经,念念在意;怕妖魔,不肯舍身;要斋吃,动舌;喜香甜,嗅鼻;闻声音,惊耳;
睹事物,凝眸:招来这六贼纷纷,怎生得西天见佛?”三藏闻言,默默沉虑道:“徒
弟啊,我
一自当年别圣君,奔波昼夜甚殷勤。
芒鞋踏破山头雾,竹笠冲开岭上云。
夜静猿啼殊可叹,月明鸟噪不堪闻。
何时满足三三行,得取如来妙法文!”
行者听毕,忍不住鼓掌大笑道:“这师父原来只是思乡难息!若要那三三行满,有何
难哉!常言道:‘功到自然成’哩。”八戒回头道:“哥啊,若照依这般魔障凶高,就
走上一千年也不得成功!”沙僧道:“二哥,你和我一般,拙口钝腮,不要惹大哥热
擦。且只捱肩磨担,终须有日成功也。”

师徒们正话间,脚走不停,马蹄正疾,见前面有一道黑水滔天,马不能进。四
众停立岸边,仔细观看。但见那:

层层浓浪,叠叠浑波:层层浓浪翻乌潦,叠叠浑波卷黑油。近观不照人身影,
远望难寻树木形。滚滚一地墨,滔滔千里灰。水沫浮来如积炭,浪花飘起似翻煤。
牛羊不饮,鸦鹊难飞。牛羊不饮嫌深黑,鸦鹊难飞怕渺弥。只是岸上芦知节令,
滩头花草斗青奇。湖泊江河天下有,溪源泽洞世间多。人生皆有相逢处,谁见西方
黑水河!
唐僧下马道:“徒弟,这水怎么如此浑黑?”八戒道:“是那家泼了靛缸了。”沙僧
道:“不然,是谁家洗笔砚哩。”行者道:“你们且休胡猜乱道,且设法保师父过去。”
八戒道:“这河若是老猪过去不难;或是驾了云头,或是下河负水,不消顿饭时,
我就过去了。”沙僧道:“若教我老沙,也只消纵云水,顷刻而过。”行者道:“我
等容易,只是师父难哩。”三藏道:“徒弟啊,这河有多少宽么?”八戒道:“约摸
有十来里宽。”三藏道:“你三个计较,着那个驮我过去罢。”行者道:“八戒驮得。”
八戒道:“不好驮。若是驮着腾云,三尺也不能离地。常言道:‘背凡人重若丘山。’
若是驮着负水,转连我坠下水去了。”

师徒们在河边,正都商议,只见那上溜头,有一人棹下一只小船儿来。唐僧喜
道:“徒弟,有船来了。叫他渡我们过去。”沙僧厉声高叫道:“棹船的,来渡人!来
渡人!”船上人道:“我不是渡船,如何渡人?”沙僧道:“天上人间,方便第一。
你虽不是渡船,我们也不是常来打搅你的。我等是东土钦差取经的佛子,你可方便
方便,渡我们过去,谢你。”那人闻言,却把船儿棹近岸边,扶着桨道:“师父啊,
我这船小,你们人多,怎能全渡?”三藏近前看了,那船儿原来是一段木头刻的,
中间只有一个舱口,只好坐下两个人。三藏道:“怎生是好?”沙僧道:“这般啊,
两遭儿渡罢。”八戒就使心术,要躲懒讨乖,道:“悟净,你与大哥在这边看着行李、
马匹,等我保师父先过去,却再来渡马。教大哥跳过去罢。”行者点头道:“你说的
是。”

那呆子扶着唐僧,那梢公撑开船,举棹冲流,一直而去。方才行到中间,只听
得一声响,卷浪翻波,遮天迷目。那阵狂风十分利害!好风:

当空一片炮云起,中溜千层黑浪高。两岸飞沙迷日色,四边树倒振天号。翻江
搅海龙神怕,播土扬尘花木雕。呼呼响若春雷吼,阵阵凶如饿虎哮。蟹鳖鱼虾朝上
拜,飞禽走兽失窝巢。五湖船户皆遭难,四海人家命不牢。溪内渔翁难把钩,河间
梢子怎撑篙?揭瓦翻砖房屋倒,惊天动地泰山摇。

这阵风,原来就是那棹船人弄的。他本是黑水河中怪物。眼看着那唐僧与猪八戒,
连船儿淬在水里,无影无形,不知摄了那方去也。

这岸上,沙僧与行者心慌道:“怎么好?老师父步步逢灾,才脱了魔障,幸得这
一路平安,又遇着黑水!”沙僧道:“莫是翻了船,我们往下溜头找寻去。”行
者道:“不是翻船,若翻船,八戒会水,他必然保师父负水而出。我才见那个棹船
的有些不正气,想必就是这厮弄风,把师父拖下水去了。”沙僧闻言道:“哥哥何不
早说!你看着马与行李,等我下水找寻去来。”行者道:“这水色不正,恐你不能去。”
沙僧道:“这水比我那流沙河如何?去得,去得!”

好和尚,脱了褊衫,扎抹了手脚,轮着降妖宝杖,扑的一声,分开水路,钻入
波中。大踏步行将进去。正走处,只听得有人言语。沙僧闪在旁边,偷睛观看,那
壁厢有一座亭台,台门外横封了八个大字,乃是“衡阳峪黑水河神府”。又听得那
怪物坐在上面道:“一向辛苦,今日方能得物。这和尚乃十世修行的好人,但得吃
他一块肉,便做长生不老人。我为他也等够多时,今朝却不负我志。”教:“小的们!
快把铁笼抬出来,将这两个和尚囫囵蒸熟,具柬去请二舅爷来,与他暖寿。”沙僧
闻言,按不住心头火起,掣宝杖,将门乱打。口中骂道:“那泼物,快送我唐僧师
父与八戒师兄出来!”唬得那门内妖邪,急跑去报:“祸事了!”老怪问:“甚么祸
事?”小妖道:“外面有一个晦气色脸的和尚,打着前门骂,要人哩。”
那怪闻言,即唤取披挂。小妖抬出披挂,老妖结束整齐。手提一根竹节钢鞭,走出
门来,真个是凶顽毒象。但见:

方面圜睛霞彩亮,卷唇巨口血盆红。几根铁线稀髯摆,再鬓朱砂乱发蓬。形似
显灵真太岁,貌如发怒狠雷公。身披铁甲团花灿,头戴金盔嵌宝浓。竹节钢鞭提手
内,行时滚滚拽狂风。生来本是波中物,脱去原流变化凶。要问妖邪真姓字,前身
唤做小鼍龙。
那怪喝道:“是甚人在此打我门哩?”沙僧道:“我把你个无知的泼怪!你怎么弄玄
虚,变作梢公,架船将我师父摄来?快早送还,饶你性命!”那怪呵呵笑道:“这和
尚不知死活!你师父是我拿了,如今要蒸熟了请人哩!你上来,与我见个雌雄!三合
敌得我啊,还你师父;如三合敌不得,连你一发都蒸吃了,休想西天去也!”沙僧
闻言大怒,轮宝杖,劈头就打。那怪举钢鞭,急架相迎。两个在水底下,这场好杀:

降妖杖,竹节鞭,二人怒发各争先。一个是黑水河中千载怪,一个是灵霄殿外
旧时仙。那个因贪三藏肉中吃,这个为保唐僧命可怜。都来水底相争斗,各要功成
两不然。杀得虾鱼对对摇头躲,蟹鳖双双缩首潜。只听水府群妖齐擂鼓,门前众怪
乱争喧。好个沙门真悟净,单身独力展威权!跃浪翻波无胜败,鞭迎杖架两牵连。
算来只为唐和尚,欲取真经拜佛天。
他二人战经三十回合,不见高低。沙僧暗想道:“这怪物是我的对手,枉自不能取
胜,且引他出去,教师兄打他。”这沙僧虚丢了个架子,拖着宝杖就走。那妖精更
不赶来,道:“你去罢,我不与你斗了。我且具柬帖儿去请客哩。”

沙僧气呼呼跳出水来,见了行者道:“哥哥,这怪物无礼。”行者问:“你下去
许多时才出来,端的是甚妖邪?可曾寻见师父?”沙僧道:“他这里边,有一座亭台;
台门外横书八个大字,唤做‘衡阳峪黑水河神府’。我闪在旁边,听着他在里面说
话,教小的们刷洗铁笼,待要把师父与八戒蒸熟了,去请他舅爷来暖寿。是我发起
怒来,就去打门。那怪物提一条竹节钢鞭走出来,与我斗了这半日,约有三十合,
不分胜负。我却使个佯输法,要引他出来,着你助阵。那怪物乖得紧,他不来赶我,
只要回去具柬请客,我才上来了。”行者道:“不知是个甚么妖邪?”沙僧道:“那
模样像一个大鳖;不然,便是个鼍龙也。”行者道:“不知那个是他舅爷?”

说不了,只见那下湾里走出一个老人,远远的跪下,叫:“大圣,黑水河河神
叩头。”行者道:“你莫是那棹船的妖邪,又来骗我么?”那老人磕头滴泪道:“大
圣,我不是妖邪,我是这河内真神。那妖精旧年五月间,从西洋海趁大潮来于此处,
就与小神交斗。奈我年迈身衰,敌他不过,把我坐的那衡阳峪黑水河神府,就占夺
去住了,又伤了我许多水族。我却没奈何,径往海内告他。原来西海龙王是他的母
舅,不准我的状子,教我让与他住。我欲启奏上天,奈何神微职小,不能得见玉帝。
今闻得大圣到此,特来参拜投生。万望大圣与我出力报冤!”行者闻言道:“这等说,
四海龙王都该有罪。他如今摄了我师父与师弟,扬言要蒸熟了,去请他舅爷暖寿,
我正要拿他,幸得你来报信。这等啊,你陪着沙僧在此看守,等我去海中,先把那
龙王捉来,教他擒此怪物。”河神道:“深感大圣大恩!”

行者即驾云,径至西洋大海。按筋斗,捻了避水诀,分开波浪;正然走处,撞
见一个黑鱼精捧着一个浑金的请书匣儿,从下流头似箭如梭钻将上来,被行者扑个
满面,掣铁棒分顶一下,可怜就打得脑浆迸出,腮骨查开,都的一声,飘出水面。
他却揭开匣儿看处,里边有一张简帖,上写着:

愚甥鼍洁顿首百拜,启上二舅爷敖老大人台下:向承佳惠,感感。今因获得二
物,乃东土僧人,实为世间之罕物。甥不敢自用。因念舅爷圣诞在迩,特设菲筵,
预祝千寿。万望车驾速临是荷!

行者笑道:“这厮却把供状先递与老孙也!”正才袖了帖子,往前再行。早有一
个探海的夜叉,望见行者,急抽身撞上水晶宫报大王:“齐天大圣孙爷爷来了!”那
龙王敖顺即领众水族,出宫迎接道:“大圣,请入小宫少座,献茶。”行者道:“我
还不曾吃你的茶,你倒先吃了我的酒也!”龙王笑道:“大圣一向皈依佛门,不动荤
酒,却几时请我吃酒来?”行者道:“你便不曾去吃酒,只是惹下一个吃酒的罪名
了。”敖顺大惊道:“小龙为何有罪?”行者袖中取出简帖儿,递与龙王。

龙王见了,魂飞魄散,慌忙跪下,叩头道:“大圣恕罪!那厮是舍妹第九个儿子。
因妹夫错行了风雨,刻减了雨数,被天曹降旨,着人曹官魏徵丞相,梦里斩了。舍
妹无处安身,是小龙带他到此,恩养成人。前年不幸,舍妹疾故,惟他无方居住,
我着他在黑水河养性修真。不期他作此恶孽,小龙即差人去擒他来也。”行者道:“你
令妹共有几个贤郎?都在那里作怪?”龙王道:“舍妹有九个儿子。那八个都是好的。
第一个小黄龙,见居淮渎;第二个小骊龙,见住济渎;第三个青背龙,占了江渎;
第四个赤髯龙,镇守河渎;第五个徒劳龙,与佛祖司钟;第六个稳兽龙,与神宫镇
脊;第七个敬仲龙,与玉帝守擎天华表;第八个蜃龙,在大家兄处,砥据太岳。此
乃第九个鼍龙,因年幼无甚执事,自旧年才着他居黑水河养性,待成名,别迁调用;
谁知他不遵吾旨,冲撞大圣也。”

行者闻言,笑道:“你妹妹有几个妹丈?”敖顺道:“只嫁得一个妹丈,乃泾河
龙王。向年已此被斩,舍妹孀居于此,前年疾故了。”行者道:“一夫一妻,如何生
这几个杂种?”敖顺道:“此正谓‘龙生九种,九种各别。’”行者道:“我才心中烦
恼,欲将简帖为证,上奏天庭,问你个通同作怪,抢夺人口之罪;据你所言,是那
厮不遵教诲,我且饶你这次:一则是看你昆玉分上;二来只该怪那厮年幼无知,你
也不甚知情。你快差人擒来,救我师父,再作区处。”敖顺即唤太子摩昂:“快点五
百虾鱼壮兵,将小鼍捉来问罪。一壁厢安排酒席,与大圣陪礼。”行者道:“龙王再
勿多心。既讲开饶了你便罢,又何须办酒?我今须与你令郎同回:一则老师父遭愆,
二则我师弟盼望。”

那老龙苦留不住,又见龙女捧茶来献。行者立饮他一盏香茶,别了老龙,随与
摩昂领兵,离了西海。早到黑水河中。行者道:“贤太子,好生捉怪,我上岸去也。”
摩昂道:“大圣宽心,小龙子将他拿上来先见了大圣,惩治了他罪名,把师父送上
来,才敢带回海内,见我家父。”行者欣然相别。捏了避水诀,跳出波津,径到了
东边崖上。沙僧与那河神迎着道:“师兄,你去时从空而去,怎么回来却自河内而
回?”行者把那打死鱼精,得简帖,见龙王,与太子同领兵来之事,备陈了一遍。
沙僧十分欢喜,都立在岸边,候接师父不题。

却说那摩昂太子着介士先到他水府门前,报与妖怪道:“西海老龙王太子摩昂
来也。”那怪正坐,忽闻摩昂来,心中疑惑道:“我差黑鱼精投简帖拜请二舅爷,这
早晚不见回话,怎么舅爷不来,却是表兄来耶?”正说间,只见那巡河的小怪,又
来报:“大王,河内有一枝兵,屯于水府之西,旗号上书着‘西海储君摩昂小帅’。”
妖怪道:“这表兄却也狂妄:想是舅爷不得来,命他来赴宴;既是赴宴,如何又领
兵劳士?咳!但恐其间有故。”教:“小的们,将我的披挂钢鞭伺候,恐一时变暴。待
我且出去迎他,看是何如。”众妖领命,一个个擦掌摩拳准备。

这鼍龙出得门来,真个见一枝海兵扎营在右。只见:

征旗飘绣带,画戟列明霞。宝剑凝光彩,长枪缨绕花。弓弯如月小,箭插似狼
牙。大刀光灿灿,短棍硬沙沙。鲸鳌并蛤蚌,蟹鳖共鱼虾。大小齐齐摆,干戈似密
麻。不是元戎令,谁敢乱爬碴!

鼍怪见了,径至那营门前,厉声高叫:“大表兄,小弟在此拱候,有请。”有一
个巡营的螺螺,急至中军帐,“报千岁殿下,外有鼍龙叫请哩。”太子按一按顶上金
盔,束一束腰间宝带,手提一根三棱简,拽开步,跑出营去,道:“你来请我怎么?”
鼍龙进礼道:“小弟今早有简帖拜请舅爷,想是舅爷见弃,着表兄来的,兄长既来
赴席,如何又劳师动众?不入水府,扎营在此,又贯甲提兵,何也?”太子道:“你
请舅爷做甚?”妖怪道:“小弟一向蒙恩赐居于此,久别尊颜,未得孝顺。昨日捉
得一个东土僧人,我闻他是十世修行的元体,人吃了他,可以延寿,欲请舅爷看过,
上铁笼蒸熟,与舅爷暖寿哩。”太子喝道:“你这厮十分懵懂!你道僧人是谁?”妖
怪道:“他是唐朝来的僧人,往西天取经的和尚。”太子道:“你只知他是唐僧,不
知他手下徒弟利害哩。”妖怪道:“他有一个长嘴的和尚,唤做个猪八戒,我也把他
捉住了,要与唐和尚一同蒸吃。还有一个徒弟,唤做沙和尚,乃是一条黑汉子,晦
气色脸,使一根宝杖。昨日在这门外与我讨师父,被我帅出河兵,一顿钢鞭,战得
他败阵逃生,也不见怎的利害。”

太子道:“原来是你不知!他还有一个大徒弟,是五百年前大闹天宫上方太乙金
仙齐天大圣;如今保护唐僧往西天拜佛求经,是普陀岩大慈大悲观音菩萨劝善,与
他改名,唤做孙悟空行者。你怎么没得做,撞出这件祸来?他又在我海内遇着你的
差人,夺了请帖,径入水晶宫,拿捏我父子们,有‘结连妖邪,抢夺人口’之罪。
你快把唐僧、八戒送上河边,交还了孙大圣,凭着我与他陪礼,你还好得性命;若
有半个‘不’字,休想得全生居于此也!”那怪鼍闻此言,心中大怒道:“我与你
嫡亲的姑表,你倒反护他人!听你所言,就教把唐僧送出;天地间那里有这等容易
事也!你便怕他,莫成我也怕他?他若有手段,敢来我水府门前,与我交战三合,我
才与他师父;若敌不过我,就连他也拿来,一齐蒸熟,也没甚么亲人,也不去请客,
自家关了门,教小的们唱唱舞舞,我坐在上面,自自在在,吃他娘不是!”

太子见说,开口骂道:“这泼邪!果然无状!且不要教孙大圣与你对敌,你敢与
我相持么?”那怪道:“要做好汉,怕甚么相持!”教:“取披挂!”呼唤一声,众小
妖跟随左右,献上披挂,捧上钢鞭。他两个变了脸,各逞英雄;传号令,一齐擂鼓。
这一场比与沙僧争斗,甚是不同。但见那:

旌旗照耀,戈戟摇光。这壁厢营盘解散,那壁厢门户开张。摩昂太子提金简,
鼍怪轮鞭急架偿。一声炮响河兵烈,三棒锣鸣海士狂。虾与虾争,蟹与蟹斗。鲸鳌
吞赤鲤,起黄。鲨鲻吃鲭鱼走,牡蛎擒蛏蛤蚌慌。少扬刺硬如铁棍,司
针利似锋芒。追白蟮,鲈脍捉乌鲳。一河水怪争高下,两处龙兵定弱强。混战
多时波浪滚,摩昂太子赛金刚。喝声
金简当头重,拿住妖鼍作怪王。
这太子将三棱简闪了一个破绽,那妖精不知是诈,钻将进来;被他使个解数,把妖
精右臂,只一简,打了个踵;赶上前,又一拍脚,跌倒在地。众海兵一拥上前,
揪翻住,将绳子背绑了双手,将铁索穿了琵琶骨,拿上岸来。押至孙行者面前道:
“大圣,小龙子捉住妖鼍,请大圣定夺。”

行者与沙僧见了道:“你这厮不遵旨令。你舅爷原着你在此居住,教你养性存
身,待你名成之日,别有迁用;你怎么强占水神之宅,倚势行凶,欺心诳上,弄玄
虚,骗我师父、师弟?我待要打你这一棒,奈何老孙这棒子甚重,略打打儿就了了
性命。你将我师父安在何处哩?”那怪叩头不住道:“大圣,小鼍不知大圣大名。
却才逆了表兄,骋强背理,被表兄把我拿住。今见大圣,幸蒙大圣不杀之恩,感谢
不尽。你师父还捆在那水府之间,望大圣解了我的铁索,放了我手,等我到河中送
他出来。”摩昂在旁道:“大圣,这厮是个逆怪,他极奸诈;若放了他,恐生恶念。”
沙和尚道:“我认得他那里,等我寻师父去。”

他两个跳入水中,径至水府门前。那里门扇大开,更无一个小卒。直入亭台里
面,见唐僧、八戒,赤条条都捆在那里。沙僧即忙解了师父,河神亦随解了八戒,
一家背着一个,出水面,径至岸边。

猪八戒见那妖精锁绑在侧,急掣钯上前就筑,口里骂道:“泼邪畜!你如今不吃
我了?”行者扯住道:“兄弟,且饶他死罪罢。看敖顺贤父子之情。”摩昂进礼道:
“大圣,小龙子不敢久停。既然救得你师父,我带这厮去见家父;虽大圣饶了他死
罪,家父决不饶他活罪,定有发落处置,仍回复大圣谢罪。”行者道:“既如此,你
领他去罢。多多拜上令尊,尚容面谢。”那太子押着那妖鼍,投水中,帅领海兵,
径转西洋大海不题。

却说那黑水河神谢了行者,道:“多蒙大圣复得水府之恩!”唐僧道:“徒弟啊,
如今还在东岸,如何渡此河也?”河神道:“老爷勿虑,且请上马,小神开路,引
老爷过河。”那师父才骑了白马,八戒采着缰绳,沙和尚挑了行李,孙行者扶持左
右,只见河神作起阻水的法术,将上流挡住。须臾,下流撤干,开出一条大路。师
徒们行过西边,谢了河神,登崖上路。这正是:
禅僧有救来西域,彻地无波过黑河。
毕竟不知怎生得拜佛求经,且听下回分解。