\chapter{盘丝洞七情迷本~濯垢泉八戒忘形}

话表三藏别了朱紫国王,整顿鞍马西进。行够多少山原,历尽无穷水道,不觉
的秋去冬残,又值春光明媚。师徒们正在路踏青玩景,忽见一座庵林。三藏滚鞍下
马,站立大道之旁。行者问道:“师父,这条路平坦无邪,因何不走?”八戒道:“师
兄好不通情!师父在马上坐得困了,也让他下来关关风是。”三藏道:“不是关风;
我看那里是个人家,意欲自去化些斋吃。”行者笑道:“你看师父说的是那里话。你
要吃斋,我自去化。俗语云:‘一日为师,终身为父。’岂有为弟子者高坐,教师父
去化斋之理?”三藏道:“不是这等说。平日间一望无边无际,你们没远没近的去
化斋,今日人家逼近,可以叫应,也让我去化一个来。”八戒道:“师父没主张。常
言道‘三人出外,小的儿苦。’你况是个父辈,我等俱是弟子。古书云:‘有事弟子
服其劳。’等我老猪去。”三藏道:“徒弟啊,今日天气晴明,与那风雨之时不同。
那时节,汝等必定远去;此个人家,等我去。有斋无斋,可以就回走路。”沙僧在
旁笑道:“师兄,不必多讲。师父的心性如此,不必违拗。苦恼了他,就化将斋来,
他也不吃。”

八戒依言,即取出钵盂,与他换了衣帽。拽开步,直至那庄前观看,却也好座
住场。但见:

石桥高耸,古树森齐:石桥高耸,潺潺流水接长溪;古树森齐,聒聒幽禽鸣远
岱。桥那边有数椽茅屋,清清雅雅若仙
庵;又有那一座蓬窗,白白明明欺道院。窗前忽见四佳人,都在那里刺凤描鸾做针
线。
长老见那人家没个男儿,只有四个女子,不敢进去。将身立定,闪在乔林之下。只
见那女子,一个个:
闺心坚似石,兰性喜如春。
娇脸红霞衬,朱唇绛脂匀。
蛾眉横月小,蝉鬓迭云新。
若到花间立,游蜂错认真。
少停有半个时辰,一发静悄悄,鸡犬无声。自家思虑道:“我若没本事化顿斋饭,
也惹那徒弟笑我:敢道为师的化不出斋来,为徒的怎能去拜佛。”

长老没计奈何,也带了几分不是,趋步上桥。又走了几步,只见那茅屋里面有
一座木香亭子,亭子下又有三个女子在那里踢气球哩。你看那三个女子,比那四个
又生得不同。但见那:

飘扬翠袖,摇拽缃裙:飘扬翠袖,低笼着玉笋纤纤;摇拽缃裙,半露出金莲窄
窄。形容体势十分全,动静脚跟千样。拿头过论有高低,张泛送来真又楷。转身
踢个出墙花,退步翻成大过海。轻接一团泥,单枪急对拐。明珠上佛头,实捏来尖
。窄砖偏会拿,卧鱼将脚。平腰折膝蹲,扭顶翘跟。扳凳能喧泛,披肩甚脱
洒。绞裆任往来,锁项随摇摆。踢的是黄河水倒流,金鱼滩上买。那个错认是头儿,
这个转身就打拐。端然捧上臁,周正尖来。提跟草鞋,倒插回头采。退步泛肩
妆,钩儿只一歹。版篓下来长,便把夺门揣。踢到美心时,佳人齐喝采。一个个汗
流粉腻透罗裳,兴懒情疏方叫海。
言不尽,又有诗为证,诗曰:
蹴当场三月天,仙风吹下素婵娟。
汗沾粉面花含露,尘染蛾眉柳带烟。
翠袖低垂笼玉笋,缃裙斜拽露金莲。
几回踢罢娇无力,云鬓蓬松宝髻偏。
三藏看得时辰久了,只得走上桥头,应声高叫道:“女菩萨,贫僧这里随缘布施些
儿斋吃。”那些女子听见,一个个喜喜欢欢抛了针线,撇了气球,都笑笑吟吟的接
出门来道:“长老,失迎了。今到荒庄,决不敢拦路斋僧,请里面坐。”三藏闻言,
心中暗道:“善哉,善哉!西方正是佛地!女流尚且注意斋僧,男子岂不虔心向佛?”

长老向前问讯了,相随众女入茅屋。过木香亭看处,呀!原来那里边没甚房廊,
只见那:

峦头高耸,地脉遥长:峦头高耸接云烟,地脉遥长通海岳。门近石桥,九曲九
湾流水顾;园栽桃李,千株千颗斗华。藤薜挂悬三五树,芝兰香散万千花。远观
洞府欺蓬岛,近睹山林压太华。正是妖仙寻隐处,更无邻舍独成家。
有一女子上前,把石头门推开两扇,请唐僧里面坐。那长老只得进去。忽抬头看时,
铺设的都是石桌、石凳,冷气阴阴。长老心惊,暗自思忖道:“这去处少吉多凶,
断然不善。”众女子喜笑吟吟,都道:“长老请坐。”长老没奈何,只得坐了。少时
间,打个冷禁。众女子问道:“长老是何宝山?化甚么缘?还是修桥补路,建寺礼塔,
还是造佛印经?请缘簿出来看看。”长老道:“我不是化缘的和尚。”女子道:“既不
化缘,到此何干?”长老道:“我是东土大唐差去西天大雷音求经者。适过宝方,
腹间饥馁,特造檀府,募化一斋,贫僧就行也。”众女子道:“好!好!好!常言道:‘远
来的和尚好看经。’妹妹们!不可怠慢,快办斋来。”

此时有三个女子陪着,言来语去,论说些因缘。那四个到厨中撩衣敛袖,炊火
刷锅。你道他安排的是些甚么东西?原来是人油炒炼,人肉煎熬;熬得黑糊充作面
筋样子,剜的人脑煎作豆腐块片。两盘儿捧到石桌上放下,对长老道:“请了。仓
卒间,不曾备得好斋,且将就吃些充腹。后面还有添换来也。”

那长老闻了一闻,见那腥膻,不敢开口,欠身合掌道:“女菩萨,贫僧是胎里
素。”众女子笑道:“长老,此是素的。”长老道:“阿弥陀佛!若像这等素的啊,我
和尚吃了,莫想见得世尊,取得经卷。”众女子道:“长老,你出家人,切莫拣人布
施。”长老道:“怎敢,怎敢!我和尚奉大唐旨意,一路西来,微生不损,见苦就救;
遇谷粒手拈入口,逢丝缕联缀遮身,怎敢拣主布施!”众女子笑道:“长老虽不拣人
布施,却只有些上门怪人。莫嫌粗淡,吃些儿罢。”长老道:“实是不敢吃,恐破了
戒。望菩萨养生不若放生,放我和尚出去罢。”

那长老挣着要走,那女子拦住门,怎么肯放,俱道:“上门的买卖,倒不好做!
‘放了屁儿,却使手掩。’你往那里去?”他一个个都会些武艺,手脚又活,把长
老扯住,顺手牵羊,扑的掼倒在地。众人按住,将绳子捆了,悬梁高吊。这吊有个
名色,叫做“仙人指路”。原来是一只手向前,牵丝吊起;一只手拦腰捆住,将绳
吊起;两只脚向后一条绳吊起:三条绳把长老吊在梁上,却是脊背朝上,肚皮朝下。
那长老忍着疼,噙着泪,心中暗恨道:“我和尚这等命苦!只说是好人家化顿斋吃,
岂知道落了火坑!徒弟啊!速来救我,还得见面;但迟两个时辰,我命休矣!”

那长老虽然苦恼,却还留心看着那些女子。那些女子把他吊得停当,便去脱剥
衣服。长老心惊,暗自忖道:“这一脱了衣服,是要打我的情了。或者夹生儿吃我
的情也有哩。”原来那女子们只解了上身罗衫,露出肚腹,各显神通:一个个腰眼
中冒出丝绳,有鸭蛋粗细,骨都都的,迸玉飞银,时下把庄门瞒了不题。

却说那行者、八戒、沙僧,都在大道之旁。他二人都放马看担,惟行者是个顽
皮,他且跳树攀枝,摘叶寻果。忽回头,只见一片光亮,慌得跳下树来,吆喝道:
“不好,不好!师父造化低了!”行者用手指道:“你看那庄院如何?”八戒、沙僧
共目视之,那一片,如雪又亮如雪,似银又光似银。八戒道:“罢了,罢了!师父遇
着妖精了!我们快去救他也!”行者道:“贤弟莫嚷。你都不见怎的,等老孙去来。”
沙僧道:“哥哥仔细。”行者道:“我自有处。”

好大圣,束一束虎皮裙,掣出金箍棒,拽开脚,两三步跑到前边,看见那丝绳
缠了有千百层厚,穿穿道道,却似经纬之势;用手按了一按,有些粘软沾人。行者
更不知是甚么东西,他即举棒道:“这一棒,莫说是几千层,就有几万层,也打断
了!”正欲打,又停住手道:“若是硬的便可打断,这个软的,只好打匾罢了。假如
惊了他,缠住老孙,反为不美。等我且问他一问再打。”

你道他问谁?即捻一个诀,念一个咒,拘得个土地老儿在庙里似推磨的一般乱
转。土地婆儿道:“老儿,你转怎的?好道是羊儿风发了。”土地道:“你不知,你不
知!有一个齐天大圣来了,我不曾接他,他那里拘我哩。”婆儿道:“你去见他便了,
却如何在这里打转?”土地道:“若去见他,他那棍子好不重,他管你好歹就打哩!”
婆儿道:“他见你这等老了,那里就打你?”土地道:“他一生好吃没钱酒,偏打老
年人。”

两口儿讲一会,没奈何只得走出去,战兢兢的,跪在路旁,叫道:“大圣,当
境土地叩头。”行者道:“你且起来,不要假忙。我且不打你,寄下在那里。我问你,
此间是甚地方?”土地道:“大圣从那厢来?”行者道:“我自东土往西来的。”土
地道:“大圣东来,可曾在那山岭上?”行者道:“正在那山岭上。我们行李、马匹
还都歇在那岭上不是!”土地道:“那岭叫做盘丝岭。岭下有洞,叫做盘丝洞。洞里
有七个妖精。”行者道:“是男怪女怪?”土地道:“是女怪。”行者道:“他有多大
神通?”土地道:“小神力薄威短,不知他有多大手段;只知那正南上,离此有三
里之遥,有一座濯垢泉,乃天生的热水,原是上方七仙姑的浴池。自妖精到此居住,
占了他的濯垢泉,仙姑更不曾与他争竞,平白地就让与他了。我见天仙不惹妖魔怪,
必定精灵有大能。”行者道:“占了此泉何干?”土地道:“这怪占了浴池,一日三
遭,出来洗澡。如今巳时已过,午时将来哑。”行者听言道:“土地,你且回去,等
我自家拿他罢。”那土地老儿磕了一个头,战兢兢的,回本庙去了。

这大圣独显神通,摇身一变,变作个麻苍蝇儿,钉在路旁草梢上等待。须臾间,
只听得呼呼吸吸之声,犹如蚕食叶,却似海生潮。只好有半盏茶时,丝绳皆尽,依
然现出庄村,还像当初模样。又听得呀的一声,柴扉响处,里边笑语喧哗,走出七
个女子。行者在暗中细看,见他一个个携手相搀,挨肩执袂,有说有笑的,走过桥
来,果是标致。但见:

比玉香尤胜,如花语更真。柳眉横远岫,檀口破樱唇。钗头翘翡翠,金莲闪绛
裙。却似嫦娥临下界,仙子落凡尘。
行者笑道:“怪不得我师父要来化斋,原来是这一般好处。这七个美人儿,假若留
住我师父,要吃也不够一顿吃,要用也不够两日用,要动手轮流一摆布就是死了。
且等我去听他一听,看他怎的算计。”

好大圣,“嘤”的一声,飞在那前面走的女子云髻上钉住。才过桥来,后边的
走向前来呼道:“姐姐,我们洗了澡,来蒸那胖和尚吃去。”行者暗笑道:“这怪物
好没算计!煮还省些柴,怎么转要蒸了吃!”那些女子采花斗草向南来。不多时,到
了浴池。但见一座门墙,十分壮丽。遍地野花香艳艳,满旁兰蕙密森森。后面一个
女子,走上前,唿哨的一声,把两扇门儿推开,那中间果有一塘热水。这水自开辟
以来,太阳星原贞有十,后被羿善开弓,射落九乌坠地,止存金乌一星,乃太阳之
真火也。天地有九处汤泉,俱是众乌所化。那九阳泉,乃香冷泉、伴山泉、温泉、
东合泉、潢山泉、孝安泉、广汾泉、汤泉,此泉乃濯垢泉。有诗为证,诗曰:

一气无冬夏,三秋永注春。炎波如鼎沸,热浪似汤新。分溜滋禾稼,停流荡俗
尘。涓涓珠泪泛,滚滚玉团津。润滑原非酿,清平还自温。瑞祥本地秀,造化乃天
真。佳人洗处冰肌滑,涤荡尘烦玉体新。
那浴池约有五丈余阔,十丈多长,内有四尺深浅,但见水清彻底。底下水一似滚珠
泛玉,骨都都冒将上来。四面有六七个孔窍通流。流去二三里之遥,淌到田里,还
是温水。池上又有三间亭子。亭子中近后壁放着一张八只脚的板凳。两山头放着两
个描金彩漆的衣架。行者暗中喜嘤嘤的,一翅飞在那衣架头上钉住。

那些女子见水又清又热,便要洗浴,即一齐脱了衣服,搭在衣架上。一齐下去,
被行者看见:

褪放纽扣儿,解开罗带结。酥胸白似银,玉体浑如雪。肘膊赛冰铺,香肩欺粉
贴。肚皮软又绵,脊背光还洁。膝腕半围团,金莲三寸窄。中间一段情,露出风流
穴。
那女子都跳下水去,一个个跃浪翻波,负水顽耍。行者道:“我若打他啊,只消把
这棍子往池中一搅,就叫做‘滚汤泼老鼠,一窝儿都是死。’可怜!可怜!打便打死
他,只是低了老孙的名头。常言道:‘男不与女斗。’我这般一个汉子,打杀这几个
丫头,着实不济。不要打他,只送他一个绝后计,教他动不得身,出不得水,多少
是好。”好大圣,捏着诀,念个咒,摇身一变,变作一个饿老鹰,但见:

毛犹霜雪,眼若明星。妖狐见处魂皆丧,狡兔逢时胆尽惊。钢爪锋芒快,雄姿
猛气横。会使老拳供口腹,不辞亲手逐飞腾。万里寒空随上下,穿云检物任他行。
呼的一翅,飞向前,轮开利爪,把他那衣架上搭的七套衣服,尽情雕去,径转岭头,
现出本相来见八戒、沙僧道:“你看。”那呆子迎着对沙僧笑道:“师父原来是典当
铺里拿了去的。”沙僧道:“怎见得?”八戒道:“你不见师兄把他些衣服都抢将来
也?”行者放下道:“此是妖精穿的衣服。”八戒道:“怎么就有这许多?”行者道:
“七套。”八戒道:“如何这般剥得容易,又剥得干净?”行者道:“那曾用剥。原
来此处唤做盘丝岭。那庄村唤做盘丝洞。洞中有七个女怪,把我师父拿住,吊在洞
里,都向濯垢泉去洗浴。那泉却是天地产成的一塘子热水。他都算计着洗了澡要把
师父蒸吃。是我跟到那里,见他脱了衣服下水,我要打他,恐怕污了棍子,又怕低
了名头,是以不曾动棍,只变做一个饿老鹰,雕了他的衣服。他都忍辱含羞,不敢
出头,蹲在水中哩。我等快去解下师父走路罢。”八戒笑道:“师兄,你凡干事,只
要留根。既见妖精,如何不打杀他,却就去解师父!他如今纵然藏羞不出,到晚间
必定出来。他家里还有旧衣服,穿上一套,来赶我们。纵然不赶,他久住在此,我
们取了经,还从那条路回去。常言道:‘宁少路边钱,莫少路边拳。’那时节,他拦
住了吵闹,却不是个仇人也?”行者道:“凭你如何主张?”八戒道:“依我,先打
杀了妖精,再去解放师父:此乃‘斩草除根’之计。”行者道:“我是不打他。你要
打,你去打他。”

八戒抖擞精神,欢天喜地,举着钉钯,拽开步,径直跑到那里。忽的推开门看
时,只见那七个女子,蹲在水里,口中乱骂那鹰哩,道:“这个匾毛畜生!猫嚼头的
亡人!把我们衣服都雕去了,教我们怎的动手!”八戒忍不住笑道:“女菩萨,在这
里洗澡哩。也携带我和尚洗洗,何如?”那怪见了,作怒道:“你这和尚,十分无
礼!我们是在家的女流,你是个出家的男子。古书云:‘七年男女不同席。’你好和
我们同塘洗澡?”八戒道:“天气炎热,没奈何,将就容我洗洗儿罢。那里调甚么
书担儿,同席不同席!”呆子不容说,丢了钉钯,脱了皂锦直裰,扑的跳下水来。
那怪心中烦恼,一齐上前要打。不知八戒水势极熟,到水里摇身一变,变做一个鲇
鱼精。那怪就都摸鱼,赶上拿他不住:东边摸,忽的又渍了西去;西边摸,忽的又
渍了东去;滑的,只在那腿裆里乱钻。原来那水有搀胸之深,水上盘了一会,
又盘在水底,都盘倒了,喘嘘嘘的,精神倦怠。

八戒却才跳将上来,现了本相,穿了直裰,执着钉钯,喝道:“我是那个?你把
我当鲇鱼精哩!”那怪见了,心惊胆战,对八戒道:“你先来是个和尚,到水里变作
鲇鱼,及拿你不住,却又这般打扮;你端的是从何到此?是必留名。”八戒道:“这
伙泼怪当真的不认得我!我是东土大唐取经的唐长老之徒弟,乃天蓬元帅悟能八戒
是也。你把我师父吊在洞里,算计要蒸他受用!我的师父,又好蒸吃?快早伸过头来,
各筑一钯,教你断根!”那些妖闻此言,魂飞魄散,就在水中跪拜道:“望老爷方便
方便!我等有眼无珠,误捉了你师父,虽然吊在那里,不曾敢加刑受苦。望慈悲饶
了我的性命,情愿贴些盘费,送你师父往西天去也。”八戒摇手道:“莫说这话!俗
语说得好:‘曾着卖糖君子哄,到今不信口甜人。’是便筑一钯,各人走路!”

呆子一味粗夯,显手段,那有怜香惜玉之心,举着钯,不分好歹,赶上前乱筑。
那怪慌了手脚,那里顾甚么羞耻,只是性命要紧,随用手侮着羞处,跳出水来,都
跑在亭子里站立,作出法来:脐孔中骨都都冒出丝绳,瞒天搭了个大丝篷,把八戒
罩在当中。

那呆子忽抬头,不见天日,即抽身往外便走。那里举得脚步!原来放了绊脚索,
满地都是丝绳,动动脚,跌个踵:左边去,一个面磕地;右边去,一个倒栽葱;
急转身,又跌了个嘴地;忙爬起,又跌了个竖蜻蜓。也不知跌了多少跟头,把个
呆子跌得身麻脚软,头晕眼花,爬也爬不动,只睡在地下呻吟。那怪物却将他困住,
也不打他,也不伤他,一个个跳出门来,将丝篷遮住天光,各回本洞。

到了石桥上站下,念动真言,霎时间,把丝篷收了,赤条条的,跑入洞里,侮
着那话,从唐僧面前笑嘻嘻的跑过去。走入石房,取几件旧衣穿了,径至后门口立
定,叫:“孩儿们何在?”原来那妖精一个有一个儿子,却不是他养的,都是他结
拜的干儿子。有名唤做蜜、蚂、、班、蜢、蜡、蜻:蜜是蜜蜂,蚂是蚂蜂,是
蜂,班是班毛,蜢是牛蜢,蜡是抹蜡,蜻是蜻蜓。原来那妖精幔天结网,掳住这
七般虫蛭,却要吃他。古云:“禽有禽言,兽有兽语。”当时这些虫哀告饶命,愿拜
为母,遂此春采百花供怪物,夏寻诸卉孝妖精。忽闻一声呼唤,都到面前,问:“母
亲有何使令?”众怪道:“儿啊,早间我们错惹了唐朝来的和尚,才然被他徒弟拦
在池里,出了多少丑,几乎丧了性命!汝等努力,快出门前去退他一退。如得胜后,
可到你舅舅家来会我。”那些怪既得逃生,往他师兄处,孽嘴生灾不题。你看这些
虫蛭,一个个摩拳擦掌,出来迎敌。

却说八戒跌得昏头昏脑,猛抬头,见丝篷丝索俱无,他才一步一探,爬将起来,
忍着疼,找回原路。见了行者,用手扯住道:“哥哥,我的头可肿,脸可青么?”
行者道:“你怎的来?”八戒道:“我被那厮将丝绳罩住,放了绊脚索,不知跌了多
少跟头,跌得我腰拖背折,寸步难移。却才丝篷索子俱空,方得了性命回来也。”
沙僧见了道:“罢了,罢了!你闯下祸来也!那怪一定往洞里去伤害师父,我等快去
救他!”

行者闻言,急拽步便走。八戒牵着马,急急来到庄前。但见那石桥上有七个小
妖儿挡住道:“慢来,慢来!吾等在此!”行者看了道:“好笑!干净都是些小人儿!”
长的也只有二尺五六寸,不满三尺;重的也只有八九斤,不满十斤。喝道:“你是
谁?”那怪道:“我乃七仙姑的儿子。你把我母亲欺辱了,还敢无知,打上我门!不
要走,仔细!”好怪物,一个个手之舞之,足之蹈之,乱打将来。八戒见了生嗔,
本是跌恼了的性子,又见那伙虫蛭小巧,就发狠举钯来筑。

那些怪见呆子凶猛,一个个现了本象,飞将起去,叫声“变!”须臾间,一个
变十个,十个变百个,百个变千个,千个变万个,个个都变成无穷之数。只见:
满天飞抹蜡,遍地舞蜻蜓。
蜜蚂追头额,蜂扎眼睛。
班毛前后咬,牛蜢上下叮。
扑面漫漫黑,神鬼惊。
八戒慌了道:“哥啊,只说经好取,西方路上,虫儿也欺负人哩!”行者道:“兄弟,
不要怕,快上前打!”八戒道:“扑头扑脸,浑身上下,都叮有十数层厚,却怎么打?”
行者道:“没事,没事!我自有手段!”沙僧道:“哥啊,有甚手段,快使出来罢。一
会子光头上都叮肿了!”

好大圣,拔了一把毫毛,嚼得粉碎,喷将出去,即变做些黄、麻、、白、雕、
鱼、鹞。八戒道:“师兄,又打甚么市语?黄啊、麻啊哩?”行者道:“你不知。黄
是黄鹰,麻是麻鹰,是鹰,白是白鹰,雕是雕鹰、鱼是鱼鹰,鹞是鹞鹰。那妖
精的儿子是七样虫,我的毫毛是七样鹰。”鹰最能虫,一嘴一个,爪打翅敲,须
臾,打得罄尽,满空无迹,地积尺余。

三兄弟方才闯过桥去,径入洞里。只见老师父吊在那里哼哼的哭哩。八戒近前
道:“师父,你是要来这里吊了耍子,不知作成我跌了多少跟头哩!”沙僧道:“且
解下师父再说。”行者即将绳索挑断,放下唐僧,都问道:“妖精那里去了?”唐僧
道:“那七个怪都赤条条的往后边叫儿子去了。”行者道:“兄弟们,跟我来寻去。”

三人各持兵器,往后园里寻处,不见踪迹。都到那桃李树上寻遍不见。八戒道:
“去了!去了!”沙僧道:“不必寻他,等我扶师父去也。”弟兄们复来前面,请唐僧
上马道:“师父,下次化斋,还让我们去。”唐僧道:“徒弟呵,以后就是饿死,也
再不自专了。”八戒道:“你们扶师父走着,等老猪一顿钯筑倒他这房子,教他来时
没处安身。”行者笑道:“筑还费力,不若寻些柴来,与他个断根罢。”好呆子,寻
了些朽松、破竹、干柳、枯藤,点上一把火,烘烘的都烧得干净。师徒却才放心前
来。

咦!毕竟这去,不知那怪的吉凶如何,且听下回分解。