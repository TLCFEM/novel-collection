\chapter{猿熟马驯方脱壳~功成行满见真如}

话表寇员外既得回生,复整理了幢鼓乐,僧道亲友,依旧送行不题。却说唐
僧四众,上了大路。果然西方佛地,与他处不同。见了些琪花瑶草,古柏苍松。所
过地方,家家向善,户户斋僧。每逢山下人修行,又见林间客诵经。师徒们夜宿晓
行,又经有六七日,忽见一带高楼,几层杰阁。真个是:

冲天百尺,耸汉凌空。低头观落日,引手摘飞星。豁达窗轩吞宇宙,嵯峨栋宇
接云屏。黄鹤信来秋树老,彩鸾书到晚风清。此乃是灵宫宝阙,琳馆珠庭;真堂谈
道,宇宙传经。花向春来美,松临雨过青。紫芝仙果年年秀,丹凤仪翔万感灵。
三藏举鞭遥指道:“悟空,好去处耶!”行者道:“师父,你在那假境界,假佛像处,
倒强要下拜;今日到了这真境界,真佛像处,倒还不下马,是怎的说?”三藏闻言,
慌得翻身跳下来,已到了那楼阁门首。只见一个道童,斜立山门之前,叫道:“那
来的莫非东土取经人么?”长老急整衣,抬头观看。见他:

身披锦衣,手摇玉麈:身披锦衣,宝阁瑶池常赴宴;手摇玉麈,丹台紫府每挥
尘。肘悬仙,足踏履鞋。飘然真羽士,秀丽实奇哉。炼就长生居胜境,修成永寿
脱尘埃。圣僧不识灵山客,当年金顶大仙来。
孙大圣认得他,即叫:“师父,此乃是灵山脚下玉真观金顶大仙,他来接我们哩。”
三藏方才醒悟,进前施礼。大仙笑道:“圣僧今年才到。我被观音菩萨哄了。他十
年前领佛金旨,向东土寻取经人,原说二三年就到我处。我年年等候,渺无消息,
不意今年才相逢也。”三藏合掌道:“有劳大仙盛意,感激,感激!”遂此四众牵马
挑担,同入观里。却又与大仙一一相见。即命看茶摆斋,又叫小童儿烧香汤与圣僧
沐浴了,好登佛地。正是那:
功满行完宜沐浴,炼驯本性合天真。
千辛万苦今方息,九戒三皈始自新。
魔尽果然登佛地,灾消故得见沙门。
洗尘涤垢全无染,反本还原不坏身。
师徒们沐浴了,不觉天色将晚。就于玉真观安歇。

次早,唐僧换了衣服,披上锦袈裟,戴了毗卢帽,手持锡杖,登堂拜辞大仙。
大仙笑道:“昨日缕,今日鲜明,观此相,真佛子也。”三藏拜别就行。大仙道:
“且住,等我送你。”行者道:“不必你送,老孙认得路。”大仙道:“你认得的是云
路。圣僧还未登云路,当从本路而行。”行者道:“这个讲得是。老孙虽走了几遭,
只是云来云去,实不曾踏着此地。既有本路,还烦你送送。我师父拜佛心重,幸勿
迟疑。”那大仙笑吟吟,携着唐僧手,接引旃坛上法门。

原来这条路不出山门,就自观宇中堂穿出后门便是。大仙指着灵山道:“圣僧,
你看那半天中有祥光五色,瑞蔼千重的,就是灵鹫高峰,佛祖之圣境也。”唐僧见
了就拜。行者笑道:“师父,还不到拜处哩。常言道:‘望山走倒马。’离此镇还有
许远,如何就拜!若拜到顶上,得多少头磕是?”大仙道:“圣僧,你与大圣、天蓬、
卷帘四位,已此到于福地,望见灵山,我回去也。”三藏遂拜辞而去。

大圣引着唐僧等,徐徐缓步,登了灵山。不上五六里,见了一道活水,滚浪飞
流,约有八九里宽阔,四无人迹。三藏心惊道:“悟空,这路来得差了。敢莫大仙
错指了?此水这般宽阔,这般汹涌,又不见舟楫,如何可渡?”行者笑道:“不差,
你看那壁厢不是一座大桥?要从那桥上行过去,方成正果哩。”长老等又近前看时,
桥边有一扁,扁上有“凌云渡”三字。原来是一根独木桥。正是:
远看横空如玉栋,近观断水一枯槎。
维河架海还容易,独木单梁人怎!
万丈虹霓平卧影,千寻白练接天涯。
十分细滑浑难渡,除是神仙步彩霞。
三藏心惊胆战道:“悟空,这桥不是人走的。我们别寻路径去来。”行者笑道:“正
是路,正是路!”八戒慌了道:“这是路,那个敢走?水面又宽,波浪又涌,独独一
根木头,又细又滑,怎生动脚?”行者道:“你都站下,等老孙走个儿你看。”

好大圣,拽开步,跳上独木桥,摇摇摆摆。须臾,跑将过去,在那边招呼道:
“过来,过来!”唐僧摇手。八戒、沙僧咬指道:“难,难,难!”行者又从那边跑
过来,拉着八戒道:“呆子,跟我走,跟我走!”那八戒卧倒在地道:“滑,滑,滑!
走不得,你饶我罢,让我驾风雾过去。”行者按住道:“这是甚么去处,许你驾风雾?
必须从此桥上走过,方可成佛。”八戒道:“哥啊,佛做不成也罢,实是走不得!”

他两个在那桥边,滚滚爬爬,扯扯拉拉的耍斗,沙僧走去劝解,才撒脱了手。
三藏回头,忽见那下溜中有一人撑一只船来,叫道:“上渡,上渡!”长老大喜道:
“徒弟,休得乱顽。那里有只渡船儿来了。”他三个跳起来站定,同眼观看,那船
儿来得至近,原来是一只无底的船儿。行者火眼金睛,早已认得是接引佛祖,又称
为南无宝幢光王佛。行者却不题破,只管叫:“这里来,撑拢来!”霎时撑近岸边,
又叫:“上渡,上渡!”三藏见了,又心惊道:“你这无底的破船儿,如何渡人?”佛
祖道:“我这船:
鸿蒙初判有声名,幸我撑来不变更。
有浪有风还自稳,无终无始乐升平。
六尘不染能归一,万劫安然自在行。
无底船儿难过海,今来古往渡群生。”
孙大圣合掌称谢道:“承盛意,接引吾师。师父,上船去。他这船儿虽是无底,却
稳;纵有风浪,也不得翻。”长老还自惊疑,行者叉着膊子,往上一推。那师父踏
不住脚,毂辘的跌在水里,早被撑船人一把扯起,站在船上。师父还抖衣服,垛鞋
脚,报怨行者。行者却引沙僧、八戒,牵马挑担,也上了船,都立在之上。那
佛祖轻轻用力撑开,只见上溜头泱下一个死尸。长老见了大惊。行者笑道:“师父
莫怕。那个原来是你。”八戒也道:“是你,是你!”沙僧拍着手,也道:“是你,是
你!”那撑船的打着号子,也说:“那是你!可贺,可贺!”

他们三人,也一齐声相和。撑着船,不一时,稳稳当当的过了凌云仙渡。三藏
才转身,轻轻的跳上彼岸。有诗为证。诗曰:
脱却胎胞骨肉身,相亲相爱是元神。
今朝行满方成佛,洗净当年六六尘。
此诚所谓广大智慧,登彼岸无极之法。四众上岸回头,连无底船儿却不知去向。行
者方说是接引佛祖。三藏方才省悟,急转身,反谢了三个徒弟。行者道:“两不相
谢。彼此皆扶持也。我等亏师父解脱,借门路修功,幸成了正果;师父也赖我等保
护,秉教伽持,喜脱了凡胎。师父,你看这面前花草松篁,鸾凤鹤鹿之胜境,比那
妖邪显化之处,孰美孰恶?何善何凶?”三藏称谢不已。一个个身轻体快,步上灵山。
早见那雷音古刹:

顶摩霄汉中,根接须弥脉。巧峰排列,怪石参差。悬崖下
瑶草琪花,曲径旁紫芝香蕙。仙猿摘果入桃林,却似火烧金;白鹤栖松立枝头,浑
如烟捧玉。彩凤双双,青鸾对对。彩凤双双,向日一鸣天下瑞;青鸾对对,迎风耀
舞世间稀。又见那黄森森金瓦叠鸳鸯,明幌幌花砖铺玛瑙。东一行,西一行,尽都
是蕊宫珠阙;南一带,北一带,看不了宝阁珍楼。天王殿上放霞光,护法堂前喷紫
焰。浮屠塔显,优钵花香。正是地胜疑天别,云闲觉昼长。红尘不到诸缘尽,万劫
无亏大法堂。
师徒们逍逍遥遥,走上灵山之巅。又见青松林下列优婆,翠柏丛中排善士。长老就
便施礼,慌得那优婆塞、优婆夷、比丘僧、比丘尼合掌道:“圣僧且休行礼。待见
了牟尼,却来相叙。”行者笑道:“早哩,早哩!且去拜上位者。”

那长老手舞足蹈,随着行者,直至雷音寺山门之外。那厢有四大金刚迎住道:
“圣僧来耶?”三藏躬身道:“是弟子玄奘到了。”答毕,就欲进门。金刚道:“圣僧
少待,容禀过再进。”那金刚着一个转山门报与二门上四大金刚,说唐僧到了;二
门上又传入三门上,说唐僧到了;三山门内原是打供的神僧,闻得唐僧到时,急至
大雄殿下,报与如来至尊释迦牟尼文佛说:“唐朝圣僧,到于宝山,取经来了。”佛
爷爷大喜。即召聚八菩萨、四金刚、五百阿罗、三千揭谛、十一大曜、十八伽蓝,
两行排列,却传金旨,召唐僧进。那里边,一层一节,钦依佛旨,叫:“圣僧进来。”
这唐僧循规蹈矩,同悟空、悟能、悟净,牵马挑担,径入山门。正是:
当年奋志奉钦差,领牒辞王出玉阶。
清晓登山迎雾露,黄昏枕石卧云霾。
挑禅远步三千水,飞锡长行万里崖。
念念在心求正果,今朝始得见如来。

四众到大雄宝殿殿前,对如来倒身下拜。拜罢,又向左右再拜。各各三匝已遍,
复向佛祖长跪,将通关文牒奉上。如来一一看了,还递与三藏。三藏作礼,启
上道:“弟子玄奘,奉东土大唐皇帝旨意,遥诣宝山,拜求真经,以济众生。望我
佛祖垂恩,早赐回国。”如来方开怜悯之口,大发慈悲之心,对三藏言曰:“你那东
土乃南赡部洲,只因天高地厚,物广人稠,多贪多杀,多淫多诳,多欺多诈;不遵
佛教,不向善缘,不敬三光,不重五谷;不忠不孝,不义不仁,瞒心昧己,大斗小
秤,害命杀牲:造下无边之孽,罪盈恶满,致有地狱之灾;所以永堕幽冥,受那许
多碓捣磨舂之苦,变化畜类。有那许多披毛顶角之形,将身还债,将肉饲人。其永
堕阿鼻,不得超升者,皆此之故也。虽有孔氏在彼立下仁义礼智之教,帝王相继,
治有徒流绞斩之刑,其如愚昧不明,放纵无忌之辈何耶!我今有经三藏,可以超脱
苦恼,解释灾愆。三藏:有法一藏,谈天;有论一藏,说地;有经一藏,度鬼。共
计三十五部,该一万五千一百四十四卷。真是修真之径,正善之门。凡天下四大部
洲之天文、地理、人物、鸟兽、花木、器用、人事,无般不载。汝等远来,待要全
付与汝取去,但那方之人,愚蠢村强,毁谤真言,不识我沙门之奥旨。”叫:“阿傩、
伽叶,你两个引他四众,到珍楼之下,先将斋食待他。斋罢,开了宝阁,将我那三
藏经中,三十五部之内,各检几卷与他,教他传流东土,永注洪恩。”

二尊者即奉佛旨,将他四众,领至楼下。看不尽那奇珍异宝,摆列无穷。只见
那设供的诸神,铺排斋宴,并皆是仙品、仙肴、仙茶、仙果,珍馐百味,与凡世不
同。师徒们顶礼了佛恩,随心享用。其实是:
宝焰金光映目明,异香奇品更微精。
千层金阁无穷丽,一派仙音入耳清。
素味仙花人罕见,香茶异食得长生。
向来受尽千般苦,今日荣华喜道成。

这番造化了八戒,便宜了沙僧:佛祖处正寿长生,脱胎换骨之馔,尽着他受用。
二尊者陪奉四众餐毕,却入宝阁,开门登看。那厢有霞光瑞气,笼罩千重;彩雾祥
云,遮漫万道。经柜上,宝箧外,都贴了红签,楷书着经卷名目。乃是:
《涅经》一部七百四十八卷
《菩萨经》一部一千二十一卷
《虚空藏经》一部四百卷
《首楞严经》一部一百一十卷
《恩意经大集》一部五十卷
《决定经》一部一百四十卷
《宝藏经》一部四十五卷
《华严经》一部五百卷
《礼真如经》一部九十卷
《大般若经》一部九百一十六卷
《大光明经》一部三百卷
《未曾有经》一部一千一百一十卷
《维摩经》一部一百七十卷
《三论别经》一部二百七十卷
《金刚经》一部一百卷
《正法论经》一部一百二十卷
《佛本行经》一部八百卷
《五龙经》一部三十二卷
《菩萨戒经》一部一百一十六卷
《大集经》一部一百三十卷
《摩竭经》一部三百五十卷
《法华经》一部一百卷
《瑜伽经》一部一百卷
《宝常经》一部二百二十卷
《西天论经》一部一百三十卷
《僧祇经》一部一百五十七卷
《佛国杂经》一部一千九百五十卷
《起信论经》一部一千卷
《大智度经》一部一千八十卷
《宝威经》一部一千二百八十卷
《本阁经》一部八百五十卷
《正律文经》一部二百卷
《大孔雀经》一部二百二十卷
《维识论经》一部一百卷
《具舍论经》一部二百卷

阿傩、伽叶引唐僧看遍经名,对唐僧道:“圣僧东土到此,有些甚么人事送我
们?快拿出来,好传经与你去。”三藏闻言道:“弟子玄奘,来路迢遥,不曾备得。”
二尊者笑道:“好,好,好!白手传经继世,后人当饿死矣!”行者见他讲口扭捏,
不肯传经,他忍不住叫噪道:“师父,我们去告如来,教他自家来把经与老孙也。”
阿傩道:“莫嚷!此是甚么去处,你还撒野放刁?到这边来接着经!”八戒、沙僧耐住
了性子,劝住了行者,转身来接。一卷卷收在包里,驮在马上,又捆了两担,八戒
与沙僧挑着,却来宝座前叩头,谢了如来,一直出门。逢一位佛祖,拜两拜;见一
尊菩萨,拜两拜。又到大门,拜了比丘僧、尼,优婆夷、塞,一一相辞,下山奔路
不题。

却说那宝阁上有一尊燃灯古佛,他在阁上,暗暗的听着那传经之事,心中甚明,
原是阿傩、伽叶将无字之经传去,却自笑云:“东土众僧愚迷,不识无字之经,却
不枉费了圣僧这场跋涉?”问:“座边有谁在此?”只见白雄尊者闪出。古佛吩咐道:
“你可作起神威,飞星赶上唐僧,把那无字之经夺了,教他再来求取有字真经。”
白雄尊者即驾狂风,滚离了雷音寺山门之外,大作神威。那阵好风,真个是:

佛前勇士,不比巽二风神。仙窍怒号,远赛吹嘘少女。这一阵,鱼龙皆失穴,
江海逆波涛。玄猿捧果难来献,黄鹤回云找旧巢。丹凤清音鸣不美,锦鸡喔运叫声
嘈。青松枝折,优钵花飘。翠竹竿竿倒,金莲朵朵摇。钟声远送三千里,经韵轻飞
万壑高。崖下奇花残美色,路旁瑶草偃鲜苗。彩鸾难舞翅,白鹿躲山崖。荡荡异香
漫宇宙,清清风气彻云霄。
那唐长老正行间,忽闻香风滚滚,只道是佛祖之祯祥,未曾堤防。又闻得响一声,
半空中伸下一只手来,将马驮的经,轻轻抢去,唬得个三藏捶胸叫唤,八戒滚地来
追,沙和尚护守着经担,孙行者急赶去如飞。那白雄尊者,见行者赶得将近,恐他
棍头上没眼,一时间不分好歹,打伤身体,即将经包碎,抛落尘埃。行者见经包
破落,又被香风吹得飘零,却就按下云头顾经,不去追赶。那白雄尊者收风敛雾,
回报古佛不题。

八戒去追赶,见经本落下,遂与行者收拾背着,来见唐僧。唐僧满眼垂泪道:
“徒弟呀!这个极乐世界,也还有凶魔欺害哩!”沙僧接了抱着的散经,打开看时,
原来雪白,并无半点字迹。慌忙递与三藏道:“师父,这一卷没字。”行者又打开一
卷,看时,也无字。八戒打开一卷,也无字。三藏叫:“通打开来看看。”卷卷俱是
白纸。长老短叹长吁的道:“我东土人果是没福!似这般无字的空本,取去何用?怎
么敢见唐王!诳君之罪,诚不容诛也!”行者早已知之,对唐僧道:“师父,不消说
了。这就是阿傩、伽叶那厮,问我要人事,没有,故将此白纸本子与我们来了。快
回去告在如来之前,问他财作弊之罪。”八戒嚷道:“正是,正是,告他去来!”
四众急急回山,无好步,忙忙又转上雷音。

不多时,到于山门之外,众皆拱手相迎,笑道:“圣僧是换经来的?”三藏点头
称谢。众金刚也不阻挡,让他进去,直至大雄殿前。行者嚷道:“如来!我师徒们受
了万蜇千魔,千辛万苦,自东土拜到此处,蒙如来吩咐传经,被阿傩、伽叶财不
遂,通同作弊,故意将无字的白纸本儿教我们拿去,我们拿他去何用?望如来敕治!”
佛祖笑道:“你且休嚷。他两个问你要人事之情,我已知矣。但只是经不可轻传,
亦不可以空取。向时众比丘圣僧下山,曾将此经在舍卫国赵长者家与他诵了一遍,
保他家生者安全,亡者超脱,只讨得他三斗三升米粒黄金回来。我还说他们忒卖贱
了,教后代儿孙没钱使用。你如今空手来取,是以传了白本。白本者,乃无字的真
经,倒也是好的。因你那东土众生,愚迷不悟,只可以此传之耳。”即叫:“阿傩、
伽叶,快将有字的真经,每部中各检几卷与他,来此报数。”

二尊者复领四众,到珍楼宝阁之下,仍问唐僧要些人事。三藏无物奉承,即命
沙僧取出紫金钵孟,双手奉上道:“弟子委是穷寒路遥,不曾备得人事。这钵盂乃
唐王亲手所赐,教弟子持此,沿路化斋。今特奉上,聊表寸心。万望尊者不鄙轻亵
将此收下,待回朝奏上唐王,定有厚谢。只是以有字真经赐下,庶不孤钦差之意,
远涉之劳也。”那阿傩接了,但微微而笑。被那些管珍楼的力士,管香积的庖丁,
看阁的尊者,你抹他脸,我扑他背,弹指的,扭唇的,一个个笑道:“不羞,不羞!
需索取经的人事!”须臾,把脸皮都羞皱了,只是拿着钵盂不放。伽叶却才进阁检
经,一一查与三藏。三藏却叫:“徒弟们,你们都好生看看,莫似前番。”他三人接
一卷,看一卷,却都是有字的。传了五千零四十八卷,乃一藏之数。收拾齐整,驮
在马上;剩下的,还装了一担,八戒挑着。自己行囊,沙僧挑着。行者牵了马,唐
僧拿了锡杖,按一按毗卢帽,抖一抖锦袈裟,才喜喜欢欢,到我佛如来之前。正是
那:
大藏真经滋味甜,如来造就甚精严。
须知玄奘登山苦,可笑阿傩却爱钱。
先次未详亏古佛,后来真实始安然。
至今得意传东土,大众均将雨露沾。

阿傩、伽叶引唐僧来见如来。如来高升莲座,指令降龙、伏虎二大罗汉敲响云
磬,遍请三千诸佛、三千揭谛、八金刚、四菩萨、五百尊罗汉、八百比丘僧、大众
优婆塞、比丘尼、优婆夷,各天各洞福地灵山大小尊者圣僧,该坐的请登宝座,该
立的侍立两旁。一时间,天乐遥闻,仙音嘹亮,满空中祥光叠叠,瑞气重重,诸佛
毕集,参见了如来。如来问:“阿傩、伽叶,传了多少经卷与他?可一一报数。”二
尊者即开报:“现付去唐朝:
《涅经》四百卷
《菩萨经》三百六十卷
《虚空藏经》二十卷
《首楞严经》三十卷
《恩意经大集》四十卷
《决定经》四十卷
《宝藏经》二十卷
《华严经》八十一卷
《礼真如经》三十卷
《大般若经》六百卷
《金光明品经》五十卷
《未曾有经》五百五十卷
《维摩经》三十卷
《三论别经》四十二卷
《金刚经》一卷
《正法论经》二十卷
《佛本行经》一百一十六卷
《五龙经》二十卷
《菩萨戒经》六十卷
《大集经》三十卷
《摩竭经》一百四十卷
《法华经》十卷
《瑜伽经》三十卷
《宝常经》一百七十卷
《西天论经》三十卷
《僧祇经》一百一十卷
《佛国杂经》一千六百三十八卷
《起信论经》五十卷
《大智度经》九十卷
《宝威经》一百四十卷
《本阁经》五十六卷
《正律文经》十卷
《大孔雀经》十四卷
《维识论经》十卷
《具舍论经》十卷
在藏总经,共三十五部,各部中检出五千零四十八卷,与东土圣僧传留在唐。现俱
收拾整顿于人马驮担之上,专等谢恩。”

三藏四众拴了马,歇了担,一个个合掌躬身,朝上礼拜。如来对唐僧言曰:“此
经功德,不可称量。虽为我门之龟鉴,实乃三教之源流。若到你那南赡部洲,示与
一切众生,不可轻慢。非沐浴斋戒,不可开卷。宝之!重之!盖此内有成仙了道之奥
妙,有发明万化之奇方也。”三藏叩头谢恩,信受奉行,依然对佛祖遍礼三匝,承
谨归诚,领经而去;去到三山门,一一又谢了众圣不题。

如来因打发唐僧去后,才散了传经之会。旁又闪上观世音菩萨合掌启佛祖道:
“弟子当年领金旨向东土寻取经之人,今已成功,共计得一十四年,乃五千零四十
日,还少八日,不合藏数。望我世尊,早赐圣僧回东转西,须在八日之内,庶完藏
数,准弟子缴还金旨。”如来大喜道:“所言甚当。准缴金旨。”即叫八大金刚吩咐
道:“汝等快使神威,驾送圣僧回东,把真经传留,即引圣僧西回。须在八日之内,
以完一藏之数。勿得迟违。”金刚随即赶上唐僧,叫道:“取经的,跟我来!”唐僧
等俱身轻体健,荡荡飘飘,随着金刚,驾云而起。这才是:
见性明心参佛祖,功完行满即飞升。

毕竟不知回东土怎生传授,且听下回分解。