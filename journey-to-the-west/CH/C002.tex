\chapter{悟彻菩提真妙理~断魔归本合元神}

话表美猴王得了姓名,怡然踊跃,对菩提前作礼启谢。那祖师即命大众引孙悟
空出二门外,教他洒扫应对,进退周旋之节。众仙奉行而出。悟空到门外,又拜了
大众师兄,就于廊庑之间,安排寝处。次早,与众师兄学言语礼貌,讲经论道,习
字焚香,每日如此。闲时即扫地锄园,养花修树,寻柴燃火,挑水运浆。凡所用之
物,无一不备。在洞中不觉六七年。一日,祖师登坛高坐,唤集诸仙,开讲大道。
真个是:

天花乱坠,地涌金莲。妙演三乘教,精微万法全。慢摇麈尾喷珠玉,响振雷霆
动九天。说一会道,讲一会禅,三家配合本如然。开明一字皈诚理,指引无生了性
玄。
孙悟空在旁闻讲,喜得他抓耳挠腮,眉花眼笑。忍不住手之舞之,足之蹈之。忽被
祖师看见,叫孙悟空道:“你在班中,怎么颠狂跃舞,不听我讲?”悟空道:“弟子
诚心听讲,听到老师父妙音处,喜不自胜,故不觉作此踊跃之状,望师父恕罪!”
祖师道:“你既识妙音,我且问你,你到洞中多少时了?”悟空道:“弟子本来懵懂,
不知多少时节。只记得灶下无火,常去山后打柴,见一山好桃树,我在那里吃了七
次饱桃矣。”祖师道:“那山唤名烂桃山。你既吃七次,想是七年了。你今要从我学
些甚么道?”悟空道:“但凭尊师教诲,只是有些道气儿,弟子便就学了。”

祖师道:“‘道’字门中有三百六十傍门,傍门皆有正果。不知你学那一门哩?”
悟空道:“凭尊师意思。弟子倾心听从。”祖师道:“我教你个‘术’字门中之道,
如何?”悟空道:“术门之道怎么说?”祖师道:“术字门中,乃是些请仙扶鸾,问
卜揲蓍,能知趋吉避凶之理。”悟空道:“似这般可得长生么?”祖师道:“不能,
不能!”悟空道:“不学,不学!”

祖师又道:“教你‘流’字门中之道,如何?”悟空又问:“流字门中,是甚又
理?”祖师道:“流字门中,乃是儒家、释家、道家、阴阳家、墨家、医家,或看
经,或念佛,并朝真降圣之类。”悟空道:“似这般可得长生么?”祖师道:“若要
长生,也似‘壁里安柱’。”悟空道:“师父,我是个老实人,不晓得打市语。怎么
谓之‘壁里安柱’?”祖师道:“人家盖房,欲图坚固,将墙壁之间,立一顶柱,
有日大厦将颓,他必朽矣。”悟空道:“据此说,也不长久。不学,不学!”

祖师道:“教你‘静’字门中之道,如何?”悟空道:“静字门中,是甚正果?”
祖师道:“此是休粮守谷,清静无为,参禅打坐,戒语持斋,或睡功,或立功,并
入定坐关之类。”悟空道:“这般也能长生么?”祖师道:“也似‘窑头土坯’。”悟
空笑道:“师父果有些滴。一行说我不会打市语。怎么谓之‘窑头土坯’?”祖
师道:“就如那窑头上,造成砖瓦之坯,虽已成形,尚未经水火煅炼,一朝大雨滂
沱,他必滥矣。”悟空道:“也不长远。不学,不学!”

祖师道:“教你‘动’字门中之道,如何?”悟空道:“动门之道,却又怎么?”
祖师道:“此是有为有作,采阴补阳,攀弓踏弩,摩脐过气,用方炮制,烧茅打鼎,
进红铅,炼秋石,并服妇乳之类。”悟空道:“似这等也得长生么?”祖师道:“此
欲长生,亦如‘水中捞月’。”悟空道:“师父又来了!怎么叫做‘水中捞月’?”祖
师道:“月在长空,水中有影,虽然看见,只是无捞摸处,到底只成空耳。”悟空道:
“也不学,不学!”

祖师闻言,咄的一声,跳下高台,手持戒尺,指定悟空道:“你这猢狲,这般
不学,那般不学,却待怎么?”走上前,将悟空头上打了三下,倒背着手,走入里
面,将中门关了,撇下大众而去。唬得那一班听讲的,人人惊惧,皆怨悟空道:“你
这泼猴,十分无状,师父传你道法,如何不学,却与师父顶嘴?这番冲撞了他,不
知几时才出来呵!”此时俱甚报怨他,又鄙贱嫌恶他。悟空一些儿也不恼,只是满
脸陪笑。原来那猴王,已打破盘中之谜,暗暗在心,所以不与众人争竞,只是忍耐
无言。祖师打他三下者,教他三更时分存心;倒背着手,走入里面,将中门关上者,
教他从后门进步,秘处传他道也。

当日悟空与众等,喜喜欢欢,在三星仙洞之前,盼望天色,急不能到晚。及黄
昏时,却与众就寝,假合眼,定息存神。山中又没支更传箭,不知时分,只自家将
鼻孔中出入之气调定。约到子时前后,轻轻的起来,穿了衣服,偷开前门,躲离大
众,走出外,抬头观看。正是那:
月明清露冷,八极迥无尘。
深树幽禽宿,源头水溜汾。
飞萤光散影,过雁字排云。
正直三更候,应该访道真。

你看他从旧路径至后门外,只见那门儿半开半掩。悟空喜道:“老师父果然注
意与我传道,故此开着门也。”即曳步近前,侧身进得门里,只走到祖师寝榻之下。
见祖师蜷身躯,朝里睡着了。悟空不敢惊动,即跪在榻前。那祖师不多时觉来,
舒开两足,口中自吟道:

“难,难,难!道最玄,莫把金丹作等闲。不遇至人传妙诀,空言口困舌头干!”
悟空应声叫道:“师父,弟子在此跪候多时。”祖师闻得声音是悟空,即起披衣盘坐,
喝道:“这猢狲!你不在前边去睡,却来我这后边作甚?”悟空道:“师父昨日坛前
对众相允,教弟子三更时候,从后门里传我道理,故此大胆径拜老爷榻下。”祖师
听说,十分欢喜,暗自寻思道:“这厮果然是个天地生成的!不然,何就打破我盘中
之暗谜也?”悟空道:“此间更无六耳,止只弟子一人,望师父大舍慈悲,传与我
长生之道罢,永不忘恩!”祖师道:“你今有缘,我亦喜说。既识得盘中暗谜,你近
前来,仔细听之,当传与你长生之妙道也。”悟空叩头谢了,洗耳用心,跪于榻下。
祖师云:

“显密圆通真妙诀,惜修性命无他说。都来总是精气神,谨固牢藏休漏泄。休
漏泄,体中藏,汝受吾传道自昌。口诀记来多有益,屏除邪欲得清凉。得清凉,光
皎洁,好向丹台赏明月。月藏玉兔日藏乌,自有龟蛇相盘结。相盘结,性命坚,却
能火里种金莲。攒簇五行颠倒用,功完随作佛和仙。”

此时说破根源,悟空心灵福至,切切记了口诀,对祖师拜谢深恩,即出后门观
看。但见东方天色微舒白,西路金光大显明。依旧路转到前门,轻轻的推开进去,
坐在原寝之处,故将床铺摇响道:“天光了!天光了!起耶!”那大众还正睡哩,不知
悟空已得了好事。当日起来打混,暗暗维持,子前午后,自己调息。

却早过了三年,祖师复登宝座,与众说法。谈的是公案比语,论的是外像包皮。
忽问:“悟空何在?”悟空近前跪下:“弟子有。”祖师道:“你这一向修些甚么道
来?”悟空道:“弟子近来法性颇通,根源亦渐坚固矣。”祖师道:“你既通法性,
会得根源,已注神体,却只是防备着‘三灾利害’。”悟空听说,沉吟良久道:“师
父之言谬矣。我尝闻道高德隆,与天同寿;水火既济,百病不生。却怎么有个‘三
灾利害’?”祖师道:“此乃非常之道:夺天地之造化,浸日月之玄机;丹成之后,
鬼神难容。虽驻颜益寿,但到了五百年后,天降雷灾打你,须要见性明心,预先躲
避。躲得过,寿与天齐;躲不过,就此绝命。再五百年后,天降火灾烧你。这火不
是天火,亦不是凡火,唤做‘阴火’。自本身涌泉穴下烧起,直透泥垣宫,五脏成
灰,四肢皆朽,把千年苦行,俱为虚幻。再五百年,又降风灾吹你。这风不是东南
西北风,不是和熏金朔风,亦不是花柳松竹风,唤做‘风’。自囟门中吹入六腑,
过丹田,穿九窍,骨肉消疏,其身自解。所以都要躲过。”

悟空闻说,毛骨悚然,叩头礼拜道:“万望老爷垂悯,传与躲避三灾之法,到
底不敢忘恩。”祖师道:“此亦无难,只是你比他人不同,故传不得。”悟空道:“我
也头圆顶天,足方履地,一般有九窍四肢,五脏六腑,何以比人不同?”祖师道:
“你虽然像人,却比人少腮。”原来那猴子孤拐面,凹脸尖嘴。悟空伸手一摸,笑
道:“师父没成算!我虽少腮,却比人多这个素袋,亦可准折过也。”祖师说:“也罢,
你要学那一般?有一般天罡数,该三十六般变化;有一般地煞数,该七十二般变化。”
悟空道:“弟子愿多里捞摸,学一个地煞变化罢。”祖师道:“既如此,上前来,传
与你口诀。”遂附耳低言,不知说了些甚么妙法。这猴王也是他一窍通时百窍通,
当时习了口诀,自修自炼,将七十二般变化,都学成了。

忽一日,祖师与众门人在三星洞前戏玩晚景。祖师道:“悟空,事成了未曾?”
悟空道:“多蒙师父海恩,弟子功果完备,已能霞举飞升也。”祖师道:“你试飞举
我看。”

悟空弄本事,将身一耸,打了个连扯跟头,跳离地有五六丈,踏云霞去勾有顿
饭之时,返复不上三里远近,落在面前,手道:“师父,这就是飞举腾云了。”祖
师笑道:“这个算不得腾云,只算得爬云而已。自古道:‘神仙朝游北海暮苍梧。’
似你这半日,去不上三里,即爬云也还算不得哩!”悟空道:“怎么为‘朝游北海暮
苍梧’?”祖师道:“凡腾云之辈,早辰起自北海,游过东海、西海、南海,复转
苍梧。苍梧者,却是北海零陵之语话也。将四海之外,一日都游遍,方算得腾云。”
悟空道:“这个却难,却难!”祖师道:“‘世上无难事,只怕有心人。’”悟空闻得此
言,叩头礼拜,启道:“师父,‘为人须为彻’,索性舍个大慈悲,将此腾云之法,
一发传与我罢,决不敢忘恩。”祖师道:“凡诸仙腾云,皆跌足而起,你却不是这般。
我才见你去,连扯方才跳上。我今只就你这个势,传你个‘筋斗云’罢。”悟空又
礼拜恳求,祖师却又传个口诀道:“这朵云,捻着诀,念动真言,攒紧了拳,将身
一抖,跳将起来,一筋斗就有十万八千里路哩!”大众听说,一个个嘻嘻笑道:“悟
空造化!若会这个法儿,与人家当铺兵,送文书,递报单,不管那里都寻了饭吃。”
师徒们天昏各归洞府。这一夜,悟空即运神炼法,会了筋斗云。逐日家无拘无束,
自在逍遥,此亦长生之美。

一日,春归夏至,大众都在松树下会讲多时。大众道:“悟空,你是那世修来
的缘法?前日老师父附耳低言,传与你的躲三灾变化之法,可都会么?”悟空笑道:
“不瞒诸兄长说,一则是师父传授,二来也是我昼夜殷勤,那几般儿都会了。”大
众道:“趁此良时,你试演演,让我等看看。”悟空闻说,抖搜精神,卖弄手段道:
“众师兄请出个题目。要我变化甚么?”大众道:“就变棵松树罢。”悟空捻着诀,
念动咒语,摇身一变,就变做一棵松树。真个是:
郁郁含烟贯四时,凌云直上秀贞姿。
全无一点妖猴像,尽是经霜耐雪枝。
大众见了,鼓掌呵呵大笑。都道:“好猴儿,好猴儿!”不觉的嚷闹,惊动了祖师。

祖师急拽杖出门来问道:“是何人在此喧哗?”大众闻呼,慌忙检束,整衣向
前。悟空也现了本相,杂在丛中道:“启上尊师,我等在此会讲,更无外姓喧哗。”
祖师怒喝道:“你等大呼小叫,全不像个修行的体段!修行的人,口开神气散,舌动
是非生。如何在此嚷笑?”大众道:“不敢瞒师父,适才孙悟空演变化耍子。教他
变棵松树,果然是棵松树,弟子们俱称扬喝采,故高声惊冒尊师,望乞恕罪。”祖
师道:“你等起去。”叫:“悟空,过来!我问你弄甚么精神,变甚么松树?这个工夫,
可好在人前卖弄?假如你见别人有,不要求他?别人见你有,必然求你。你若畏祸,
却要传他;若不传他,必然加害:你之性命又不可保。”悟空叩头道:“只望师父恕
罪!”祖师道:“我也不罪你,但只是你去罢。”悟空闻此言,满眼堕泪道:“师父,
教我往那里去?”祖师道:“你从那里来,便从那里去就是了。”悟空顿然醒悟道:
“我自东胜神洲傲来国花果山水帘洞来的。”祖师道:“你快回去,全你性命;若在
此间,断然不可!”悟空领罪:“上告尊师,我也离家有二十年矣,虽是回顾旧日儿
孙,但念师父厚恩未报,不敢去。”祖师道:“那里甚么恩义?你只不惹祸不牵带我
就罢了!”悟空见没奈何,只得拜辞,与众相别。祖师道:“你这去,定生不良。凭
你怎么惹祸行凶,却不许说是我的徒弟。你说出半个字来,我就知之,把你这猢狲
剥皮锉骨,将神魂贬在九幽之处,教你万劫不得翻身!”悟空道:“决不敢提起师父
一字,只说是我自家会的便罢。”

悟空谢了。即抽身,捻着诀,丢个连扯,纵起筋斗云,径回东胜。那里消一个
时辰,早看见花果山水帘洞。美猴王自知快乐,暗暗的自称道:
“去时凡骨凡胎重,得道身轻体亦轻。
举世无人肯立志,立志修玄玄自明。
当时过海波难进,今日回来甚易行。
别语叮咛还在耳,何期顷刻见东溟。”
悟空按下云头,直至花果山,找路而走。忽听得鹤唳猿啼,鹤唳声冲霄汉外,猿啼
悲切甚伤情,即开口叫道:“孩儿们,我来了也!”那崖下石坎边,花草中,树木里,
若大若小之猴,跳出千千万万,把个美猴王围在当中,叩头叫道:“大王,你好宽
心!怎么一去许久?把我们俱闪在这里,望你诚如饥渴!近来被一妖魔在此欺虐,强
要占我们水帘洞府,是我等舍死忘生,与他争斗。这些时,被那厮抢了我们家火,
捉了许多子侄,教我们昼夜无眠,看守家业。幸得大王来了!大王若再年载不来,
我等连山洞尽属他人矣!”

悟空闻说,心中大怒道:“是甚么妖魔,辄敢无状!你且细细说来,待我寻他报
仇!”众猴叩头:“告上大王,那厮自称混世魔王,住居在直北下。”悟空道:“此间
到他那里,有多少路程?”众猴道:“他来时云,去时雾,或风或雨,或电或雷,
我等不知有多少路。”悟空道:“既如此,你们休怕,且自顽耍,等我寻他去来!”

好猴王,将身一纵,跳起去,一路筋斗,直至北下观看,见一座高山,真是十
分险峻。好山:

笔峰挺立,曲涧深沉。笔峰挺立透空霄,曲涧深沉通地户。两崖花木争奇,几
处松篁斗翠。左边龙,熟熟驯驯;右边虎,平平伏伏。每见铁牛耕,常有金钱种。
幽禽声,丹凤朝阳立。石磷磷,波净净,古怪跷蹊真恶狞。世上名山无数多,
花开花谢蘩还众。争如此景永长存,八节四时浑不动。诚为三界坎源山,滋养五行
水脏洞!

美猴王正默观看景致,只听得有人言语。径自下山寻觅,原来那陡崖之前,乃
是那水脏洞。洞门外有几个小妖跳舞,见了悟空就走。悟空道:“休走!借你口中言,
传我心内事。我乃正南方花果山水帘洞洞主。你家甚么混世鸟魔,屡次欺我儿孙,
我特寻来,要与他见个上下!”

那小妖听说,疾忙跑入洞里,报道:“大王!祸事了!”魔王道:“有甚祸事?”
小妖道:“洞外有猴头称为花果山水帘洞洞主。他说你屡次欺他儿孙,特来寻你,
见个上下哩。”魔王笑道:“我常闻得那些猴精说他有个大王,出家修行去,想是今
番来了。你们见他怎生打扮,有甚器械?”小妖道:“他也没甚么器械,光着个头,
穿一领红色衣,勒一条黄丝绦,足下踏一对乌靴,不僧不俗,又不像道士神仙,赤
手空拳,在门外叫哩。”魔王闻说:“取我披挂兵器来!”那小妖即时取出。

那魔王穿了甲胄,绰刀在手,与众妖出得门来,即高声叫道:“那个是水帘洞
洞主?”悟空急睁睛观看,只见那魔王:

头戴乌金盔,映日光明;身挂皂罗袍,迎风飘荡。下穿着黑铁甲,紧勒皮条;
足踏着花褶靴,雄如上将。腰广十围,身高三丈。手执一口刀,锋刃多明亮。称为
混世魔,磊落凶模样。

猴王喝道:“这泼魔这般眼大,看不见老孙!”魔王见了,笑道:“你身不满四
尺,年不过三旬,手内又无兵器,怎么大胆猖狂,要寻我见甚么上下?”悟空骂道:
“你这泼魔,原来没眼!你量我小,要大却也不难。你量我无兵器,我两只手彀着
天边月哩!你不要怕,只吃老孙一拳!”纵一纵,跳上去,劈脸就打。那魔王伸手架
住道:“你这般矬矮,我这般高长,你要使拳,我要使刀,使刀就杀了你,也吃人
笑,待我放下刀,与你使路拳看。”悟空道:“说得是。好汉子!走来!”那魔王丢开
架子便打,这悟空钻进去相撞相迎。他两个拳捶脚踢,一冲一撞。原来长拳空大,
短簇坚牢。那魔王被悟空掏短胁,撞丫裆,几下筋节,把他打重了。他闪过,拿起
那板大的钢刀,望悟空劈头就砍。悟空急撤身,他砍了一个空。悟空见他凶猛,即
使身外身法,拔一把毫毛,丢在口中嚼碎,望空喷去,叫一声“变”!即变做三二
百个小猴,周围攒簇。

原来人得仙体,出神变化无方。不知这猴王自从了道之后,身上有八万四千毛
羽,根根能变,应物随心。那些小猴,眼乖会跳,刀来砍不着,枪去不能伤。你看
他前踊后跃,钻上去,把个魔王围绕,抱的抱,扯的扯,钻裆的钻裆,扳脚的扳脚,
踢打毛,抠眼睛,捻鼻子,抬鼓弄,直打做一个攒盘。这悟空才去夺得他的刀来,
分开小猴,照顶门一下,砍为两段。领众杀进洞中,将那大小妖精,尽皆剿灭。却
把毫毛一抖,收上身来。又见那收不上身者,却是那魔王在水帘洞擒去的小猴,悟
空道:“汝等何为到此?”约有三五十个,都含泪道:“我等因大王修仙去后,这两
年被他争吵,把我们都摄将来,那不是我们洞中的家火?石盆、石碗都被这厮拿来
也。”悟空道:“既是我们的家火,你们都搬出外去。”随即洞里放起火来,把那水
脏洞烧得枯干,尽归了一体。对众道:“汝等跟我回去。”众猴道:“大王,我们来
时,只听得耳边风响,虚飘飘到于此地,更不识路径,今怎得回乡?”悟空道:“这
是他弄的个术法儿,有何难也!我如今一窍通,百窍通,我也会弄。你们都合了眼,
休怕!”

好猴王,念声咒语,驾阵狂风,云头落下。叫:“孩儿们,睁眼。”众猴脚实
地,认得是家乡,个个欢喜,都奔洞门旧路。那在洞众猴,都一齐簇拥同入,分班
序齿,礼拜猴王。安排酒果,接风贺喜,启问降魔救子之事。悟空备细言了一遍,
众猴称扬不尽道:“大王去到那方,不意学得这般手段!”悟空又道:“我当年别汝
等,随波逐流,飘过东洋大海,径至南赡部洲,学成人像,着此衣,穿此履,摆摆
摇摇,云游了八九年余,更不曾有道;又渡西洋大海,到西牛贺洲地界,访问多时,
幸遇一老祖,传了我与天同寿的真功果,不死长生的大法门。”众猴称贺。都道:“万
劫难逢也!”悟空又笑道:“小的们,又喜我这一门皆有姓氏。”众猴道:“大王姓
甚?”悟空道:“我今姓孙,法名悟空。”众猴闻说,鼓掌忻然道:“大王是老孙,
我们都是二孙、三孙、细孙、小孙——一家孙、一国孙、一窝孙矣!”都来奉承老
孙,大盆小碗的,椰子酒、葡萄酒、仙花、仙果,真个是合家欢乐!咦!
贯通一姓身归本,只待荣迁仙名。

毕竟不知怎生结果,居此界终始如何,且听下回分解。