\chapter{四僧宴乐御花园~一怪空怀情欲喜}

话表孙行者三人,随着宣召官至午门外,黄门官即时传奏宣进。他三个齐齐站
定,更不下拜。国王问道:“那三位是圣僧驸马之高徒?姓甚名谁?何方居住?因甚事
出家?取何经卷?”行者即近前,意欲上殿。旁有护驾的喝道:“不要走!有甚话,
立下奏来。”行者笑道:“我们出家人,得一步就进一步。”随后八戒、沙僧亦俱近
前。长老恐他村鲁惊驾,便起身叫道:“徒弟啊,陛下问你来因,你即奏上。”行者
见他那师父在旁侍立,忍不住大叫一声道:“陛下轻人重己!既招我师为驸马,如何
教他侍立?世间称女夫谓之‘贵人’,岂有贵人不坐之理!”国王听说,大惊失色。
欲退殿,恐失了观瞻。只得硬着胆,教近侍的取绣墩来,请唐僧坐了。行者才奏道:

“老孙祖居东胜神洲傲来国花果山水帘洞。父天母地,石裂吾生。曾拜至人,
学成大道。复转仙乡,啸聚在洞天福地。下海降龙,登山擒兽。消死名,上生籍,
官拜齐天大圣。玩赏琼楼,喜游宝阁。会天仙,日日歌欢;居圣境,朝朝快乐。只
因乱却蟠桃宴,大反天宫,被佛擒伏,困压在五行山下,饥餐铁弹,渴饮铜汁,五
百年未尝茶饭。幸我师出东土,拜西方,观音教令脱天灾,离大难,皈正在瑜伽门
下。旧讳悟空,称名行者。”

国王闻得这般名重,慌得下了龙床,走将来,以御手挽定长老道:“驸马,也
是朕之天缘,得遇你这仙姻仙眷。”三藏满口谢恩,请国王登位。复问:“那位是第
二高徒?”八戒掬嘴扬威道:

“老猪先世为人,贪欢爱懒。一生混沌,乱性迷心。未识天高地厚,难明海阔
山遥。正在幽闲之际,忽然遇一真人。半句话,解开业网;两三言,劈破灾门。当
时省悟,立地投师,谨修二八之工夫,敬炼三三之前后。行满飞升,得超天府。荷
蒙玉帝厚恩,官赐天蓬元帅,管押河兵,逍遥汉阙。只因蟠桃酒醉,戏弄嫦娥,谪
官衔,遭贬临凡;错投胎,托生猪像。住福陵山,造恶无边。遇观音,指明善道。
皈依佛教,保护唐僧。径往西天,拜求妙典。法讳悟能,称为八戒。”
国王听言,胆战心惊,不敢观觑。这呆子越弄精神,摇着头,掬着嘴,撑起耳朵“呵
呵”大笑。三藏又怕惊驾,即叱道:“八戒收敛!”方才叉手拱立,假扭斯文。又问:
“第三位高徒,因甚皈依?”沙和尚合掌道:

“老沙原系凡夫,因怕轮回访道。云游海角,浪荡天涯。常得衣钵随身,每炼
心神在舍。因此虔诚,得逢仙侣。养就孩儿,配缘姹女。工满三千,合和四相。超
天界,拜玄穹,官授卷帘大将,侍御凤辇龙车。也为蟠桃会上,失手打破玻璃盏,
贬在流沙河,改头换面,造孽伤生。幸喜菩萨远游东土,劝我皈依,等候唐朝佛子,
往西天求经果正。从立自新,复修大觉。指河为姓,法讳悟净,称名沙僧。”

国王见说,多惊多喜。喜的是女儿招了活佛,惊的是三个实乃妖神。正在惊喜
之间,忽有正台阴阳官奏道:“婚期已定本年本月十二日。壬子辰良,周堂通利,
宜配婚姻。”国王道:“今日是何日辰?”阴阳官奏:“今日初八,乃戊申之日,猿
猴献果,正宜进贤纳事。”国王大喜,即着当驾官打扫御花园馆阁楼亭,且请驸马
同三位高徒安歇,待后安排合卺佳筵,着公主匹配。众等钦遵,国王退朝,多官皆
散不题。

却说三藏师徒们都到御花园,天色渐晚,摆了素膳。八戒喜道:“这一日也该
吃饭了。”管办人即将素米饭、面饭等物,整担挑来。那八戒吃了又添,添了又吃,
直吃得撑肠拄腹,方才住手。少顷,又点上灯,设铺盖,各自归寝。长老见左右无
人,却恨责行者,怒声叫道:“悟空!你这猢狲,番番害我!我说只去倒换关文,莫
向彩楼前去,你怎么直要引我去看看?如今看得好么!却惹出这般事来,怎生是好?”
行者陪笑道:“师父说,‘先母也是抛打绣球,遇旧缘,成其夫妇’。似有慕古之意,
老孙才引你去。又想着那个给孤布金寺长老之言,就此检视真假。适见那国王之面,
略有些晦暗之色,但只未见公主何如耳。”

长老道:“你见公主便怎的?”行者道:“老孙的火眼金睛,但见面,就认得真
假善恶,富贵贫穷,却好施为,辨明邪正。”沙僧与八戒笑道:“哥哥近日又学得会
相面了。”行者道:“相面之士,当我孙子罢了。”三藏喝道:“且休调嘴!只是他如
今定要招我,果何以处之?”行者道:“且到十二日会喜之时,必定那公主出来参
拜父母,等老孙在旁观看。若还是个真女人,你就做了驸马,享用国内之荣华也罢。”
三藏闻言,越生嗔怒,骂道:“好猢狲!你还害我哩!却是悟能说的,我们十节儿已
上了九节七八分了,你还把热舌头铎我!快早夹着,你休开那臭口!再若无礼,我就
念起咒来,教你了当不得!”行者听说念咒,慌得跪在面前道:“莫念,莫念!若是
真女人,待拜堂时,我们一齐大闹皇宫,领你去也。”师徒说话,不觉早已入更。
正是:

沉沉宫漏,荫荫花香。绣户垂珠箔,闲庭绝火光。秋千索冷空留影,羌笛声残
静四方。绕屋有花笼月灿,隔空无树显星芒。杜鹃啼歇,蝴蝶梦长。银汉横天宇,
白云归故乡。正是离
人情切处,风摇嫩柳更凄凉。
八戒道:“师父,夜深了,有事明早再议。且睡,且睡!”师徒们果然安歇。

一宵夜景已题,早又金鸡唱晓。五更三点,国王即登殿设朝。但见:
宫殿开轩紫气高,风吹御乐透青霄。
云移豹尾旌旗动,日射螭头玉佩摇。
香雾细添宫柳绿,露珠微润苑花娇。
山呼舞蹈千官列,海晏河清一统朝。
众文武百官朝罢,又宣:“光禄寺安排十二日会喜佳筵。今日且整春,请驸马在
御花园中款玩。”吩咐仪制司领三位贤亲去会同馆少坐,着光禄寺安排三席素宴去
彼奉陪。两处俱着教坊司奏乐,伏侍赏春景消迟日也。八戒闻得,应声道:“陛下,
我师徒自相会,更无一刻相离。今日既在御花园饮宴,带我们去耍两日,好教师父
替你家做驸马;不然,这个买卖生意弄不成。”那国王见他丑陋,说话粗俗,又见
他扭头捏颈,掬嘴巴,摇耳朵,即像有些风气,犹恐搅破亲事,只得依从;便教:
“在永镇华夷阁里安排二席,我与驸马同坐。留春亭上,安排三席,请三位别坐。
恐他师徒们坐次不便。”那呆子才朝上唱个喏,叫声多谢。各各而退。又传旨教内
宫官排宴,着三宫六院后妃与公主上头,就为添妆子,以待十二日佳配。

将有巳时前后,那国王排驾,请唐僧都到御花园内观看。好去处:

径铺彩石,槛凿雕栏:径铺彩石,径边石畔长奇葩;槛凿雕栏,槛外栏中生异
卉。夭桃迷翡翠,嫩柳闪黄鹂。步觉幽香来袖满,行沾清味上衣多。凤台龙沼,竹
阁松轩。凤台之上,
吹箫引凤来仪;龙沼之间,养鱼化龙而去。竹阁有诗,费尽推敲裁白雪;松轩文集,
考成珠玉注青编。假山拳石翠,曲水碧波深。牡丹亭,蔷薇架,迭锦铺绒;茉藜槛,
海棠畦,堆霞砌玉。芍药异香,蜀葵奇艳。白梨红杏斗芳菲,紫蕙金萱争烂熳。丽
春花、木笔花、杜鹃花,夭夭灼灼;含笑花、凤仙花、玉簪花,战战巍巍。一处处
红透胭脂润,一丛丛芳浓锦绣围。更喜东风回暖日,满园娇媚逞光辉。

一行君王几位,观之良久。早有仪制司官邀请行者三人入留春亭。国王携唐僧
上华夷阁,各自饮宴。那歌舞吹弹,铺张陈设,真是:
峥嵘阊阖曙光生,凤阁龙楼瑞霭横。
春色细铺花草绣,天光遥射锦袍明。
笙歌缭绕如仙宴,杯飞传玉液清。
君悦臣欢同玩赏,华夷永镇世康宁。

此时长老见那国王敬重,无计可奈,只得勉强随喜,诚是外喜而内忧也。坐间
见壁上挂着四面金屏,屏上画着春夏秋冬四景,皆有题咏,皆是翰林名士之诗:

春景诗曰:
周天一气转洪钧,大地熙熙万象新。
桃李争妍花烂熳,燕来画栋迭香尘。

夏景诗曰:
熏风拂拂思迟迟,宫院榴葵映日辉。
玉笛音调惊午梦,芰荷香散到庭帏。

秋景诗曰:
金井梧桐一叶黄,珠帘不卷夜来霜。
燕知社日辞巢去,雁折芦花过别乡。

冬景诗曰:
天雨飞云暗淡寒,朔风吹雪积千山。
深宫自有红炉暖,报道梅开玉满栏。

那国王见唐僧恣意看诗,便道:“驸马喜玩诗中之味,必定善于吟哦。如不吝
珠玉,请依韵各和一首如何?”长老是个对景忘情,明心见性之意;见国王钦重,
命和前韵,他不觉忽谈一句道:“日暖冰消大地钧。”国王大喜,即召侍卫官:“取
文房四宝,请驸马和完录下,俟朕缓缓味之。”长老欣然不辞,举笔而和:

和春景诗曰:
日暖冰消大地钧,御园花卉又更新。
和风膏雨民沾泽,海晏河清绝俗尘。

和夏景诗曰:
斗指南方白昼迟,槐云榴火斗光辉。
黄鹂紫燕啼宫柳,巧转双声入绛帏。

和秋景诗曰:
香飘橘绿与橙黄,松柏青青喜降霜。
篱菊半开攒锦绣,笙歌韵彻水云乡。

和冬景诗曰:
瑞雪初晴气味寒,奇峰巧石玉团山。
炉烧兽炭煨酥酪,袖手高歌倚翠栏。
国王见和大喜。称唱道:“好个‘袖手高歌倚翠栏’!”遂命教坊司以新诗奏乐,尽
日而散。

行者三人在留春亭亦尽受用,各饮了几杯,也都有些酣意。正欲去寻长老,只
见长老已同国王在一阁。八戒呆性发作,应声叫道:“好快活,好自在!今日也受用
这一下了!却该趁饱儿睡觉去也!”沙僧笑道:“二哥忒没修养。这气饱饫,如何睡
觉?”八戒道:“你那里知,俗语云‘吃了饭儿不挺尸,肚里没板脂’哩!”

唐僧与国王相别,只谨言,只谨言。既至亭内,嗔责他三人道:“这夯货,越
发村了!这是甚么去处,只管大呼小叫!倘或恼着国王,却不被他伤害性命?”八戒
道:“没事,没事!我们与他亲家礼道的,他便不好生怪。常言道:‘打不断的亲,
骂不断的邻。’大家耍子,怕他怎的?”长老叱道,教:“拿过呆子来,打他二十禅
杖!”行者果一把揪翻,长老举杖就打。呆子喊叫道:“驸马爷爷!饶罪!饶罪!”旁
有陪宴官劝住。呆子爬将起来,突突嚷嚷的道:“好贵人,好驸马,亲还未成,就
行起王法来了!”行者侮着他嘴道:“莫胡说,莫胡说,快早睡去!”他们又在留春
亭住了一宿。到明早,依旧宴乐。

不觉乐了三四日,正值十二日佳辰。有光禄寺三部各官回奏道:“臣等自八日
奉旨,驸马府已修完,专等妆奁铺设。合卺宴亦已完备,荤素共五百余席。”国王
心喜,正欲请驸马赴席,忽有内宫官对御前启奏道:“万岁,正宫娘娘有请。”国王
遂退入内宫,只见那三宫皇后,六院嫔妃,引领着公主,都在昭阳宫谈笑。真个是
花团锦簇!那一片富丽妖娆,真胜似天堂月殿,不亚于仙府瑶宫。有喜会佳姻新词
四首为证。

喜词云:

喜,喜,喜!欣然乐矣!结婚姻,恩爱美。巧样宫妆,嫦娥怎比。龙钗与凤,
艳艳飞金缕。樱唇皓齿朱颜,袅娜如花轻
体。锦重重,五彩丛中;香拂拂,千金队里。

会词云:

会,会,会!妖娆娇媚。赛毛嫱,欺楚妹。倾国倾城,比花比玉。妆饰更鲜妍,
钗环多艳丽。兰心蕙性清高,粉脸冰肌荣贵。黛眉一线远山微,窈窕嫣攒锦队。

佳词云:

佳,佳,佳!玉女仙娃。深可爱,实堪夸。异香馥郁,脂粉交加。天台福地远,
怎似国王家。笑语纷然娇态,笙歌缭绕喧哗。花堆锦砌千般美,看遍人间怎若他。

姻词云:

姻,姻,姻!兰麝香喷。仙子阵,美人群。嫔妃换彩,公主妆新。云鬓堆鸦髻,
霓裳压凤裙。一派仙音嘹,两行朱紫缤纷。当年曾结乘鸾信,今朝幸喜会佳姻。

却说国王驾到,那后妃引着公主,并彩女、宫娥都来迎接。国王喜孜孜,进了
昭阳宫坐下。后妃等朝拜毕,国王道:“公主贤女,自初八日结彩抛球,幸遇圣僧,
想是心愿已足。各衙门官,又能体朕心,各项事俱已完备。今日正是佳期,可早赴
合卺之宴,不要错过时辰。”

那公主走近前,倒身下拜,奏道:“父王,乞赦小女万千之罪。有一言启奏:
这几日闻得宫官传说,唐圣僧有三个徒弟,他生得十分丑恶,小女不敢见他,恐见
时必生恐惧。万望父王将他发放出城方好,不然惊伤弱体,反为祸害也。”国王道:
“孩儿不说,朕几乎忘了。果然生得有些丑恶。连日教他在御花园里留春亭管待。
趁今日就上殿,打发他关文,教他出城,却好会宴。”公主叩头谢了恩。国王即出
驾上殿,传旨:“请驸马共他三位。”

原来那唐僧捏指头儿算日子,熬至十二日,天未明,就与他三人计较道:“今
日却是十二了,这事如何区处?”行者道:“那国王我已识得他有些晦气,还未沾
身,不为大害;但只不得公主见面,若得出来,老孙一觑,就知真假,方才动作。
你只管放心。他如今一定来请,打发我等出城。你自应承莫怕。我闪闪身儿就来,
紧紧随护你也。”

师徒们正讲,果见当驾官同仪制司来请。行者笑道:“去来!去来!必定是与我
们送行,好留师父会合。”八戒道:“送行必定有千百两黄金白银,我们也好买些人
事回去。到我那丈人家,也再会亲耍子儿去耶。”沙僧道:“二哥箝着口,休乱说,
只凭大哥主张。”

遂此将行李、马匹,俱随那些官到于丹墀下。国王见了,教请行者三位近前道:
“汝等将关文拿上来,朕当用宝花押交付汝等,外多备盘缠,送你三位早去灵山见
佛。若取经回来,还有重谢。留驸马在此,勿得悬念。”行者称谢。遂教沙僧取出
关文递上。国王看了,即用了印,押了花字,又取黄金十锭,白金二十锭,聊达亲
礼。八戒原来财色心重,即去接了。行者朝上唱个喏道:“聒噪,聒噪!”便转身要
走,慌得个三藏一毂辘爬起,扯住行者,咬响牙根道:“你们都不顾我就去了!”行
者把手捏着三藏手掌,丢个眼色道:“你在这里宽怀欢会,我等取了经,回来看你。”
那长老似信不信的,不肯放手。多官都看见,以为实是相别而去。早见国王又请驸
马上殿,着多官送三位出城。长老只得放了手上殿。

行者三人,同众出了朝门,各自相别。八戒道:“我们当真的走哩?”行者不
言语,只管走至驿中。驿丞接入,看茶,摆饭。行者对八戒、沙僧道:“你两个只
在此,切莫出头。但驿丞问甚么事情,且含糊答应,莫与我说话。我保师父去也。”

好大圣,拔一根毫毛,吹口仙气,叫“变!”即变作本身模样,与八戒、沙僧
同在驿内。真身却幌的跳在半空,变作一个蜜蜂儿,其实小巧。但见:

翅黄口甜尾利,随风飘舞颠狂。最能摘蕊与偷香,度柳穿花摇荡。辛苦几番淘
染,飞来飞去空忙。酿成浓美自何尝,只好留存名状。
你看他轻轻的飞入朝中。远见那唐僧在国王左边绣墩上坐着,愁眉不展,心存焦燥。
径飞至他毗卢帽上,悄悄的爬及耳边,叫道:“师父,我来了,切莫忧虑。”这句话,
只有唐僧听见,那伙凡人,莫想知觉。唐僧听见,始觉心宽。不一时,宫官来请道:
“万岁,合卺嘉筵已排设在鹊宫中。娘娘与公主,俱在宫伺候。专请万岁同贵人
会亲也。”国王喜之不尽,即同驸马进宫而去。正是那:
邪主爱花花作祸,禅心动念念生愁。

毕竟不知唐僧在内宫怎生解脱,且听下回分解。