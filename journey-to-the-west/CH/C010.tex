\chapter{二将军宫门镇鬼~唐太宗地府还魂}

却说太宗与魏征在便殿对弈,一递一着,摆开阵势。正合《烂柯经》云:

博弈之道,贵乎严谨。高者在腹,下者在边,中者在角,此棋家之常法。法曰:
“宁输一子,不失一先。击左则视右,攻后则瞻前。有先而后,有后而先。两生勿
断,皆活勿连。阔不可太疏,密不可太促。与其恋子以求生,不若弃之而取胜;与
其无事而独行,不若固之而自补。彼众我寡,先谋其生;我众彼寡,务张其势。善
胜者不争,善阵者不战;善战者不败,善败者不乱。夫棋始以正合,终以奇胜。凡
敌无事而自补者,有侵绝之意;弃小而不救者,有图大之心;随手而下者,无谋之
人;不思而应者,取败之道。《诗》云:‘惴惴小心,如临于谷。’此之谓也。”

诗曰:
棋盘为地子为天,色按阴阳造化全。
下到玄微通变处,笑夸当日烂柯仙。

君臣两个对弈此棋,正下到午时三刻,一盘残局未终,魏征忽然踏伏在案边,
鼾鼾盹睡。太宗笑曰:“贤卿真是匡扶社稷之心劳,创立江山之力倦,所以不觉盹
睡。”太宗任他睡着,更不呼唤。

不多时,魏征醒来,俯伏在地道:“臣该万死!臣该万死!却才晕困,不知所为,
望陛下赦臣慢君之罪!”太宗道:“卿有何慢罪?且起来,拂退残棋,与卿从新更着。”
魏征谢了恩,却才拈子在手,只听得朝门外大呼小叫。原来是秦叔宝、徐茂功等,
将着一个血淋的龙头,掷在帝前,启奏道:“陛下,海浅河枯曾有见,这般异事却
无闻。”太宗与魏征起身道:“此物何来?”叔宝、茂功道:“千步廊南,十字街
头,云端里落下这颗龙头,微臣不敢不奏。”唐王惊问魏征:“此是何说?”魏征
转身叩头道:“是臣才一梦斩的。”唐王闻言,大惊道:“贤卿盹睡之时,又不曾
见动身动手,又无刀剑,如何却斩此龙?”魏征奏道:“主公,臣的

身在君前,梦离陛下:身在君前对残局,合眼朦胧;梦离陛下乘瑞云,出神抖
搜。那条龙,在剐龙台上,被天兵将绑缚其中。是臣道:‘你犯天条,合当死罪。
我奉天命,斩汝残生。’龙闻哀苦,臣抖精神。龙闻哀苦,伏爪收鳞甘受死;臣抖
精神,撩衣进步举霜锋。一声刀过处,龙头因此落虚空。”
太宗闻言,心中悲喜不一。喜者:夸奖魏征好臣,朝中有此豪杰,愁甚江山不稳?
悲者:谓梦中曾许救龙,不期竟致遭诛。只得强打精神,传旨着叔宝将龙头悬挂市
曹,晓谕长安黎庶。一壁厢赏了魏征,众官散讫。

当晚回宫,心中只是忧闷:想那梦中之龙,哭啼啼哀告求生,岂知无常,难免
此患。思念多时,渐觉神魂倦怠,身体不安。当夜二更时分,只听得宫门外有号泣
之声,太宗愈加惊恐。正朦胧睡间,又见那泾河龙王,手提着一颗血淋淋的首级,
高叫:“唐太宗!还我命来,还我命来,你昨夜满口许诺救我,怎么天明时反宣人曹
官来斩我?你出来,你出来,我与你到阎君处折辨折辨!”他扯住太宗,再三嚷闹不
放。太宗箝口难言,只挣得汗流遍体。正在那难分难解之时,只见正南上香云缭绕,
彩雾飘,有一个女真人上前,将杨柳枝用手一摆,那没头的龙,悲悲啼啼,径往
西北而去。原来这是观音菩萨,领佛旨,上东土,寻取经人,此住长安城都土地庙
里,夜闻鬼泣神号,特来喝退业龙,救脱皇帝。那龙径到阴司地狱具告不题。

却说太宗苏醒回来,只叫“有鬼!有鬼!”慌得那三宫皇后,六院嫔妃,与近侍
太监,战兢兢,一夜无眠。不觉五更三点,那满朝文武多官,都在朝门外候朝。等
到天明,犹不见临朝,唬得一个个惊惧踌躇。及日上三竿,方有旨意出来道:“朕
心不快,众官免朝。”不觉倏五七日,众官忧惶,都正要撞门见驾问安,只见太后
有旨,召医官入宫用药。众人在朝门等候讨信。少时,医官出来,众问何疾。医官
道:“皇上脉气不正,虚而又数,狂言见鬼;又诊得十动一代,五脏无气,恐不讳
只在七日之内矣。”众官闻言,大惊失色。

正怆惶间,又听得太后有旨宣徐茂功、护国公、尉迟公见驾。三公奉旨,急入
到分宫楼下。拜毕,太宗正色强言道:“贤卿,寡人十九岁领兵,南征北伐,东挡
西除,苦历数载,更不曾见半点邪祟,今日却反见鬼!”尉迟公道:“创立江山,
杀人无数,何怕鬼乎?”太宗道:“卿是不信。朕这寝宫门外,入夜就抛砖弄瓦,
鬼魅呼号,着然难处。白日犹可,昏夜难禁。”叔宝道:“陛下宽心,今晚臣与敬
德把守宫门,看有甚么鬼祟。”太宗准奏。茂功谢恩而出。当日天晚,各取披挂,
他两个介胄整齐,执金瓜钺斧,在宫门外把守。好将军!他看他怎生打扮:

头戴金盔光烁烁,身披铠甲龙鳞。护心宝镜幌祥云,狮蛮收紧扣,绣带彩霞新。
这一个凤眼朝天星斗怕,那一个环睛映电月光浮。他本是英雄豪杰旧勋臣,只落得
千年称户尉,万古作门神。
二将军侍立门旁,一夜天晚,更不曾见一点邪祟。是夜,太宗在宫,安寝无事,晓
来宣二将军。重重赏道:“朕自得疾,数日不能得睡,今夜仗二将军威势甚安。
卿且请出安息安息,待晚间再一护卫。”二将谢恩而出。遂此二三夜把守俱安。只
是御膳减损,病转觉重。太宗又不忍二将辛苦,又宣叔宝、敬德与杜、房诸公入宫。
吩咐道:“这两日朕虽得安,却只难为秦、胡二将军彻夜辛苦。朕欲召巧手丹青,
传二将军真容,贴于门上,免得劳他,如何?”众臣即依旨,选两个会写真的,着
胡、秦二公,依前披挂,照样画了,贴在门上。夜间也即无事。

如此二三日,又听得后宰门,乒乓乒乓,砖瓦乱响,晓来急宣众臣曰:“连日
前门幸喜无事,今夜后门又响,却不又惊杀寡人也!”茂功进前奏道:“前门不安,
是敬德、叔宝护卫;后门不安,该着魏征护卫。”太宗准奏。又宣魏征今夜把守后
门。征领旨,当夜结束整齐,提着那诛龙的宝剑,侍立在后宰门前,真个的好英雄
也!他怎生打扮:

熟绢青巾抹额,锦袍玉带垂腰。兜风氅袖采霜飘,压赛垒荼神貌。脚踏乌靴坐
折,手持利刃凶骁。圆睁两眼四边瞧,那个邪神敢到?

一夜通明,也无鬼魅。虽是前后门无事,只是身体渐重。一日,太后又传旨,
召众臣商议殡殓后事。太宗又宣徐茂功,吩咐国家大事,叮嘱仿刘蜀主托孤之意。
言毕,沐浴更衣,待时而已。旁闪魏征,手扯龙衣,奏道:“陛下宽心,臣有一事,
管保陛下长生。”太宗道:“病势已入膏肓,命将危矣,如何保得?”征云:“臣
有书一封,进与陛下,捎去到冥司,付酆都判官崔圭。”太宗道:“崔圭是谁?”
征云:“崔圭乃是太上先皇帝驾前之臣,先受磁州令,后升礼部侍郎。在日与臣八
拜为交,相知甚厚。他如今已死,现在阴司做掌生死文簿的酆都判官,梦中常与臣
相会。此去若将此书付与他,他念微臣薄分,必然放陛下回来。管教魂魄还阳世,
定取龙颜转帝都。”太宗闻言,接在手中,笼入袖里,遂瞑目而亡。那三宫六院、
皇后嫔妃、侍长储君及两班文武,俱举哀戴孝;又在白虎殿上,停着梓宫不题。

却说太宗渺渺茫茫,魂灵径出五凤楼前,只见那御林军马,请大驾出朝采猎。
太宗欣然从之,缥渺而去。行多时,人马俱无。独自个散步荒郊草野之间。正惊惶
难寻道路,只见那一边,有一人高声大叫道:“大唐皇帝,往这里来!往这里来!”
太宗闻言,抬头观看,只见那人:

头顶乌纱,腰围犀角。头顶乌纱飘软带,腰围犀角显金厢。手擎牙笏凝祥霭,
身着罗袍隐瑞光。脚踏一双粉底靴,登云促雾;怀揣一本生死簿,注定存亡。鬓发
蓬松飘耳上,胡须飞舞绕腮旁。昔日曾为唐国相,如今掌案侍阎王。
太宗行到那边,只见他跪拜路旁,口称“陛下,赦臣失远迎之罪!”太宗问曰:
“你是何人?因甚事前来接拜?”那人道:“微臣半月前,在森罗殿上,见泾河鬼龙
告陛下许救反诛之故,第一殿秦广大王即差鬼使催请陛下,要三曹对案。臣已知之,
故来此间候接。不期今日来迟,望乞恕罪,恕罪。”太宗道:“你姓甚名谁?是何官
职?”那人道:“微臣存日,在阳曹侍先君驾前,为磁州令,后拜礼部侍郎,姓崔
名圭。今在阴司,得受酆都掌案判官。”太宗大喜,近前来御手忙搀道:“先生远
劳。朕驾前魏征,有书一封,正寄与先生,却好相遇。”判官谢恩,问书在何处。
太宗即向袖中取出递与崔圭。圭拜接了,拆封而看。其书曰:

辱爱弟魏征,顿首书拜大都案契兄崔老先生台下:忆昔交游,音容如在,倏尔
数载,不闻清教。常只是遇节令设蔬品奉祭,未卜享否?又承不弃,梦中临示,始知
我兄长大人高迁。奈何阴阳两隔,天各一方,不能面觌。今因我太宗文皇帝倏然而
故,料是对案三曹,必然得与兄长相会。万祈俯念生日交情,方便一二,放我陛下
回阳,殊为爱也。容再修谢。不尽。
那判官看了书,满心欢喜道:“魏人曹前日梦斩老龙一事,臣已早知,甚是夸奖不
尽。又蒙他早晚看顾臣的子孙,今日既有书来,陛下宽心,微臣管送陛下还阳,重
登玉阙。”太宗称谢了。

二人正说间,只见那边有一对青衣童子,执幢幡宝盖,高叫道:“阎王有请,
有请。”太宗遂与崔判官并二童子举步前进。忽见一座城,城门上挂着一面大牌,
上写着“幽冥地府鬼门关”七个大金字。那青衣将幢幡摇动,引太宗径入城中,顺
街而走。只见那街旁边有先主李渊,先兄建成,故弟元吉,上前道:“世民来了!
世民来了!”那建成、元吉就来揪打索命。太宗躲闪不及,被他扯住。幸有崔判官
唤一青面獠牙鬼使,喝退了建成、元吉,太宗方得脱身而去。行不数里,见一座碧
瓦楼台,真个壮丽。但见:

飘飘万叠彩霞堆,隐隐千条红雾观。耿耿檐飞怪兽头,辉辉瓦叠鸳鸯片。门钻
几路赤金钉,槛设一横白玉段。窗牖近光放晓烟,帘栊幌亮穿红电。楼台高耸接青
霄,廊庑平排连宝院。兽鼎香云袭御衣,绛纱灯火明宫扇。左边猛烈摆牛头,右下
峥嵘罗马面。接亡送鬼转金牌,引魄招魂垂素练。唤作阴司总会门,下方阎老森罗
殿。
太宗正在外面观看,只见那壁厢环叮当,仙香奇异,外有两对提烛,后面却是十
代阎王降阶而至。是那十代阎君:秦广王、初江王、宋帝王、仵官王、阎罗王、平
等王、泰山王、都市王、卞城王、转轮王。十王出在森罗宝殿,控背躬身,迎迓太
宗。太宗谦下,不敢前行。十王道:“陛下是阳间人王,我等是阴间鬼王,分所当
然,何须过让?”太宗道:“朕得罪麾下,岂敢论阴阳人鬼之道?”逊之不已。太
宗前行,径入森罗殿上,与十王礼毕,分宾主坐定。

约有片时,秦广王拱手而进言曰:“泾河鬼龙告陛下许救而反杀之,何也?”
太宗道:“朕曾夜梦老龙求救,实是允他无事;不期他犯罪当刑,该我那人曹官魏
征处斩。朕宣魏征在殿着棋,不知他一梦而斩。这是那人曹官出没神机,又是那龙
王犯罪当死,岂是朕之过也?”十王闻言,伏礼道:“自那龙未生之前,南斗星死
簿上已注定该遭杀于人曹之手,我等早已知之。但只是他在此折辨,定要陛下来此,
三曹对案,是我等将他送入轮藏,转生去了。今又有劳陛下降临,望乞恕我催促之
罪。”

言毕,命掌生死簿判官:“急取簿子来,看陛下阳寿天禄该有几何?”崔判官
急转司房,将天下万国国王天禄总簿,先逐一检阅。只见南赡部洲大唐太宗皇帝注
定贞观一十三年。崔判官吃了一惊,急取浓墨大笔,将“一”字上添了两画,却将
簿子呈上。十王从头看时,见太宗名下注定三十三年,阎王惊问:“陛下登基多少
年了?”太宗道:“朕即位,今一十三年了。”阎王道:“陛下宽心勿虑,还有二
十年阳寿。此一来已是对案明白,请返本还阳。”太宗闻言,躬身称谢。十阎王差
崔判官、朱太尉二人,送太宗还魂。太宗出森罗殿,又起手问十王道:“朕宫中老
少安否如何?”十王道:“俱安,但恐御妹寿似不永。”太宗又再拜启谢:“朕回
阳世,无物可酬谢,惟答瓜果而已。”十王喜曰:“我处颇有东瓜,西瓜、只少南
瓜。”太宗道:“朕回去即送来,即送来。”从此遂相揖而别。

那太尉执一首引魂,在前引路。崔判官随后保着太宗,径出幽司。太宗举目
而看,不是旧路,问判官曰:“此路差矣?”判官道:“不差。阴司里是这般,有
去路,无来路。如今送陛下自‘转轮藏’出身:一则请陛下游观地府,一则教陛下
转托超生。”太宗只得随他两个,引路前来。

径行数里,忽见一座高山,阴云垂地,黑雾迷空。太宗道:“崔先生,那厢是
甚么山?”判官道:“乃幽冥背阴山。”太宗悚惧道:“朕如何去得?”判官道:
“陛下宽心,有臣等引领。”太宗战战兢兢,相随二人,上得山岩,抬头观看。只
见:

形多凸凹,势更崎岖。峻如蜀岭,高似庐岩。非阳世之名
山,实阴司之险地。荆棘丛丛藏鬼怪,石崖磷磷隐邪魔。耳畔不闻兽鸟噪,眼前惟
见鬼妖行。阴风飒飒,黑雾漫漫。阴风飒飒,是神兵口内哨来烟;黑雾漫漫,是鬼
祟暗中喷出气。一望高低无景色,相看左右尽猖亡。那里山也有,峰也有,岭也有,
洞也有,涧也有;只是山不生草,峰不插天,岭不行客,洞不纳云,涧不流水。岸
前皆魍魉,岭下尽神魔。洞中收野鬼,涧底隐邪魂。山前山后,牛头马面乱喧呼;
半掩半藏,饿鬼穷魂时对泣。催命的判官,急急忙忙传信票;追魂的太尉,吆吆喝
喝趱公文。急脚子,旋风滚滚;勾司人,黑雾纷纷。
太宗全靠着那判官保护,过了阴山。

前进又历了许多衙门,一处处俱是悲声振耳,恶怪惊心。太宗又道:“此是何
处?”判官道:“此是阴山背后‘一十八层地狱’。”太宗道:“是那十八层?”
判官道:“你听我说:

吊筋狱、幽枉狱、火坑狱,寂寂寥寥,烦烦恼恼,尽皆是生前作下千般业,死
后通来受罪名。酆都狱、拔舌狱、剥皮狱,哭哭啼啼,凄凄惨惨,只因不忠不孝伤
天理,佛口蛇心堕此门。磨捱狱、碓捣狱、车崩狱,皮开肉绽,抹嘴咨牙,乃是瞒
心昧己不公道,巧语花言暗损人。寒冰狱、脱壳狱、抽肠狱,垢面蓬头,愁眉皱眼,
都是大斗小秤欺痴蠢,致使灾屯累自身。油锅狱、黑暗狱、刀山狱,战战兢兢,悲
悲切切,皆因强暴欺良善,藏头缩颈苦伶仃。血池狱、阿鼻狱、秤杆狱,脱皮露骨,
折臂断筋,也只为谋财害命,宰畜屠生,堕落千年难解释,沉沦永世不翻身。一个
个紧缚牢拴,绳缠索绑。差些赤发鬼、黑脸鬼,长枪短剑;牛头鬼、马面鬼,铁简
铜锤。只打得皱眉苦面血淋淋,叫地叫天无效应。——正是人生却莫把心欺,神鬼
昭彰放过谁?善恶到头终有报,只争来早与来迟。”
太宗听说,心中惊惨。

进前又走不多时,见一伙鬼卒,各执幢幡,路旁跪下道:“桥梁使者来接。”
判官喝令起去,上前引着太宗,从金桥而过。太宗又见那一边有一座银桥,桥上行
几个忠孝贤良之辈,公平正大之人,亦有幢幡接引;那壁厢又有一桥,寒风滚滚,
血浪滔滔,号泣之声不绝。太宗问道:“那座桥是何名色?”判官道:“陛下,那
叫做奈河桥。若到阳间,切须传记。那桥下都是些:

奔流浩浩之水,险峻窄窄之路。俨如匹练搭长江,却似火坑浮上界。阴气逼人
寒透骨,腥风扑鼻味钻心。波翻浪滚,往来并没渡人船;赤脚蓬头,出入尽皆作业
鬼。桥长数里,阔只三。高有百尺,深却千重。上无扶手栏杆,下有抢人恶怪。
枷缠身,打上奈河险路。你看那桥边神将甚凶顽,河内孽魂真苦恼。桠杈树上,
挂的是青红黄紫色丝衣;壁斗崖前,蹲的是毁骂公婆淫泼妇。铜蛇铁狗任争餐,永
堕奈河无出路。”

诗曰:
时闻鬼哭与神号,血水浑波万丈高。
无数牛头并马面,狰狞把守奈河桥。

正说间,那几个桥梁使者,早已回去了。太宗心又惊惶,点头暗叹,默默悲伤,
相随着判官、太尉,早过了奈河恶水,血盆苦果。前又到枉死城,只听哄哄人嚷,
分明说“李世民来了!李世民来了!”太宗听叫,心惊胆战。见一伙拖腰折臂、有足
无头的鬼魅,上前拦住,都叫道:“还我命来!还我命来!”慌得那太宗藏藏躲躲,
只叫“崔先生救我!崔先生救我!”判官道:“陛下,那些人都是那六十四处烟尘,
七十二处草寇,众王子、众头目的鬼魂;尽是枉死的冤业,无收无管,不得超生,
又无钱钞盘缠,都是孤寒饿鬼。陛下得些钱钞与他,我才救得哩。”太宗道:“寡
人空身到此,却那里得有钱钞?”判官道:“陛下,阳间有一人,金银若干,在我
这阴司里寄放。陛下可出名立一约,小判可作保,且借他一库,给散这些饿鬼,方
得过去。”太宗问曰:“此人是谁?”判官道:“他是河南开封府人氏,姓相名良。
他有十三库金银在此。陛下若借用过他的,到阳间还他便了。”太宗甚喜,情愿出
名借用。遂立了文书与判官,借他金银一库,着太尉尽行给散。判官复吩咐道:“这
些金银,汝等可均分用度,放你大唐爷爷过去。他的阳寿还早哩。我领了十王钧语,
送他还魂,教他到阳间做一个水陆大会,度汝等超生,再休生事。”众鬼闻言,得
了金银,俱唯唯而退。判官令太尉摇动引魂,领太宗出离了枉死城中,奔上平阳
大路,飘飘荡荡而去。

毕竟不知从那条路出身,且听下回分解。