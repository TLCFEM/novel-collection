\chapter{婴儿戏化禅心乱~猿马刀圭木母空}

却说那孙大圣,兄弟三人,按下云头,径至朝内。只见那君臣储后,几班儿拜
接谢恩。行者将菩萨降魔收怪的那一节,陈诉与他君臣听了,一个个顶礼不尽。正
都在贺喜之间,又听得黄门官来奏:“主公,外面又有四个和尚来也。”八戒慌了道:
“哥哥,莫是妖精弄法,假捏文殊菩萨,哄了我等,却又变作和尚,来与我们斗智
哩?”行者道:“岂有此理!”即命宣进来看。

众文武传令,着他进来。行者看时,原来是那宝林寺僧人,捧着那冲天冠、碧
玉带、赭黄袍、无忧履进得来也。行者大喜道:“来得好,来得好!”且教道人过来,
摘下包巾,戴上冲天冠;脱了布衣,穿上赭黄袍;解了绦子,系上碧玉带;褪了僧
鞋,登上无忧履;教太子拿出白玉圭来,与他执在手里,早请上殿称孤。正是自古
道:“朝廷不可一日无君。”那皇帝那里肯坐,哭啼啼,跪在阶心道:“我已死三年,
今蒙师父救我回生,怎么又敢妄自称尊;请那一位师父为君,我情愿领妻子城外为
民足矣。”那三藏那里肯受,一心只是要拜佛求经。又请行者,行者笑道:“不瞒列
位说。老孙若肯要做皇帝,天下万国九州皇帝,都做遍了。只是我们做惯了和尚,
是这般懒散。若做了皇帝,就要留头长发,黄昏不睡,五鼓不眠;听有边报,心神
不安;见有灾荒,忧愁无奈。我们怎么弄得惯?你还做你的皇帝,我还做我的和尚,
修功行去也。”那国王苦让不过,只得上了宝殿,南面称孤,大赦天下,封赠了宝
林寺僧人回去。却才开东阁,筵宴唐僧。一壁厢传旨宣召丹青,写下唐僧师徒四位
喜容,供养在金銮殿上。

那师徒们安了邦国,不肯久停,欲辞王驾投西。那皇帝与三宫妃后、太子、诸
臣,将镇国的宝贝,金银缎帛,献与师父酬恩。那三藏分毫不受,只是倒换关文,
催悟空等背马早行。那国王甚不过意,摆整朝銮驾请唐僧上坐,着两班文武引导,
他与三宫妃后并太子一家儿,捧毂推轮,送出城廓,却才下龙辇,与众相别。国王
道:“师父啊,到西天经回之日,是必还到寡人界内一顾。”三藏道:“弟子领命。”
那皇帝阁泪汪汪,遂与众臣回去了。

那唐僧一行四僧,上了羊肠大路,一心里专拜灵山。正值秋尽冬初时节,但见:
霜凋红叶林林瘦,雨熟黄粱处处盈。
日暖岭梅开晓色,风摇山竹动寒声。

师徒们离了乌鸡国,夜住晓行,将半月有余。忽又见一座高山,真个是摩天碍
日。三藏马上心惊,急兜缰忙呼行者。行者道:“师父有何吩咐?”三藏道:“你看
前面又有大山峻岭,须要仔细提防,恐一时又有邪物来侵我也。”行者笑道:“只管
走路,莫再多心。老孙自有防护。”那长老只得宽怀,加鞭策马,奔至山岩,果然
也十分险峻。但见得:

高不高,顶上接青霄;深不深,涧中如地府。山前常见骨都都白云,腾腾黑
雾。红梅翠竹,绿柏青松。山后有千万丈挟魂灵台,台后有古古怪怪藏魔洞。洞中
有叮叮当当滴水泉,泉下更有弯弯曲曲流水涧。又见那跳天搠地献果猿,丫丫叉叉
带角鹿,呢呢痴痴看人獐。至晚巴山寻穴虎,待晓翻波出水龙。登得洞门唿喇的响,
惊得飞禽扑鲁的起,看那林中走兽鞠律律的行。见此一伙禽和兽,吓得人心磴磴
惊。堂倒洞堂堂倒洞,洞当当倒洞当仙。青石染成千块玉,碧纱笼罩万堆
烟。

师徒们正当悚惧,又只见那山凹里有一朵红云,直冒到九霄空内,结聚了一团
火气。行者大惊,走近前,把唐僧着脚,推下马来,叫:“兄弟们,不要走了,
妖怪来矣。”慌得个八戒急掣钉钯,沙僧忙轮宝杖,把唐僧围护在当中。

话分两头。却说红光里,真是个妖精。他数年前,闻得人讲:“东土唐僧往西
天取经,乃是金蝉长老转生,十世修行的好人。有人吃他一块肉,延生长寿,与天
地同休。”他朝朝在山间等候,不期今日到了。他在那半空里,正然观看,只见三
个徒弟,把唐僧围护在马上,各各准备。这精灵夸赞不尽道:“好和尚!我才看着一
个白面胖和尚骑了马,真是那唐朝圣僧,却怎么被三个丑和尚护持住了!一个个伸
拳敛袖,各执兵器,似乎要与人打的一般。噫!不知是那个有眼力的,想应认得我
了。似此模样,莫想得那唐僧的肉吃。”沉吟半晌,以心问心的自家商量道:“若要
倚势而擒,莫能得近;或者以善迷他,却到得手。但哄得他心迷惑,待我在善内生
机,断然拿了。且下去戏他一戏。”

好妖怪,即散红光,按云头落下。去那山坡里,摇身一变,变作七岁顽童,赤
条条的,身上无衣,将麻绳捆了手足,高吊在那松树梢头,口口声声,只叫“救人!
救人!”

却说那孙大圣忽抬头再看处,只见那红云散尽,火气全无。便叫:“师父,请
上马走路。”唐僧道:“你说妖怪来了,怎么又敢走路?”行者道:“我才然间,见
一朵红云从地而起,到空中结做一团火气,断然是妖精。这一会红云散了,想是个
过路的妖精,不敢伤人。我们去耶!”八戒笑道:“师兄说话最巧,妖精又有个甚么
过路的。”行者道:“你那里知道。若是那山那洞的魔王设宴,邀请那诸山各洞之精
赴会,却就有东南西北四路的精灵都来赴会;故此他只有心赴会,无意伤人。此乃
过路之妖精也。”

三藏闻言,也似信不信的,只得攀鞍在马,顺路奔山前进。正行时,只听得叫
声“救人!”长老大惊道:“徒弟呀,这半山中,是那里甚么人叫?”行者上前道:
“师父只管走路,莫缠甚么‘人轿’、‘骡轿’、‘明轿’、‘睡轿’。这所在,就有轿,
也没个人抬你。”唐僧道:“不是扛抬之轿,乃是叫唤之叫。”行者笑道:“我晓得,
莫管闲事,且走路。”

三藏依言,策马又进。行不上一里之遥,又听得叫声“救人!”长老道:“徒弟,
这个叫声,不是鬼魅妖邪;若是鬼魅妖邪,但有出声,无有回声。你听他叫一声,
又叫一声,想必是个有难之人。我们可去救他一救。”行者道:“师父,今日且把这
慈悲心略收起收起,待过了此山,再发慈悲罢。这去处凶多吉少。你知道那倚草附
木之说,是物可以成精。诸般还可,只有一般蟒蛇,但修得年远日深,成了精魅,
善能知人小名儿。他若在草科里,或山凹中,叫人一声,人不答应还可;若答应一
声,他就把人元神绰去,当夜跟来,断然伤人性命。且走,且走!古人云:‘脱得去,
谢神明。’切不可听他。”长老只得依他,又加鞭催马而去。

行者心中暗想:“这泼怪不知在那里,只管叫阿叫的;等我老孙送他一个‘卯
酉星法’,教他两不见面。”好大圣,叫沙和尚前来:“拢着马,慢慢走着,让老孙
解解手。”你看他让唐僧先行几步,却念个咒语,使个移山缩地之法,把金箍棒往
后一指,他师徒过此峰头,往前走了,却把那怪物撇下。他再拽开步,赶上唐僧,
一路奔山。只见那三藏又听得那山背后叫声“救人!”长老道:“徒弟呀,那有难的
人,大没缘法,不曾得遇着我们。我们走过他了;你听他在山后叫哩。”八戒道:“在
便还在山前,只是如今风转了也。”行者道:“管他甚么转风不转风,且走路。”因
此,遂都无言语,恨不得一步过此山,不题话下。

却说那妖精在山坡里,连叫了三四声,更无人到。他心中思量道:“我等唐僧
在此,望见他离不上三里,却怎么这半晌还不到?……想是抄下路去了。”他抖一抖
身躯,脱了绳索,又纵红光,上空再看。不觉孙大圣仰面回观,识得是妖怪,又把
唐僧撮着脚推下马来道:“兄弟们,仔细,仔细!那妖精又来也!”慌得那八戒、沙
僧各持兵刀,将唐僧又围护在中间。

那精灵见了,在半空中称羡不已道:“好和尚!我才见那白面和尚坐在马上,却
怎么又被他三人藏了?这一去见面方知。先把那有眼力的弄倒了,方才捉得唐僧。
不然啊,徒费心机难获物,枉劳情兴总成空。”却又按下云头,恰似前番变化,高
吊在松树山头等候。这番却不上半里之地。

却说那孙大圣抬头再看,只见那红云又散,复请师父上马前行。三藏道:“你
说妖精又来,如何又请走路?”行者道:“这还是个过路的妖精,不敢惹我们。”长
老又怀怒道:“这个泼猴,十分弄我!正当有妖魔处,却说无事;似这般清平之所,
却又恐吓我,不时的嚷道有甚妖精。虚多实少,不管轻重,将我着脚,下马来,
如今却解说甚么过路的妖精。假若跌伤了我,却也过意不去!这等,这等!……”行
者道:“师父莫怪。若是跌伤了你的手足,却还好医治;若是被妖精捞了去,却何
处跟寻?”三藏大怒,哏哏的,要念紧箍儿咒,却是沙僧苦劝,只得上马又行。

还未曾坐得稳,只听又叫“师父救人啊!”长老抬头看时,原来是个小孩童,
赤条条的,吊在那树上,兜住缰,便骂行者道:“这泼猴多大惫懒!全无有一些儿善
良之意,心心只是要撒泼行凶哩!我那般说叫唤的是个人声,他就千言万语只嚷是
妖怪!你看那树上吊的不是个人么?”大圣见师父怪下来了,却又觌面看见模样,
一则做不得手脚,二来又怕念紧箍儿咒,低着头,再也不敢回言。让唐僧到了树下。
那长老将鞭梢指着问道:“你是那家孩儿?因有甚事,吊在此间?说与我,好救你。”
噫!分明他是个精灵,变化得这等,那师父却是个肉眼凡胎,不能相识。

那妖魔见他下问,越弄虚头,眼中噙泪,叫道:“师父呀,山西去有一条枯松
涧。涧那边有一庄村。我是那里人家。我祖公公姓红,只因广积金银,家私巨万,
混名唤做红百万。年老归世已久,家产遗与我父。近来人事奢侈,家私渐废,改名
唤做红十万,专一结交四路豪杰,将金银借放,希图利息。怎知那无籍之人,设骗
了去啊,本利无归。我父发了洪誓,分文不借。那借金银人,身贫无计,结成凶党,
明火执杖,白日杀上我门,将我财帛尽情劫掳,把我父亲杀了;见我母亲有些颜色,
拐将去做甚么压寨夫人。那时节,我母亲舍不得我,把我抱在怀里,哭哀哀,战兢
兢,跟随贼寇;不期到此山中,又要杀我,多亏我母亲哀告,免教我刀下身亡,却
将绳子吊我在树上,只教冻饿而死。那些贼将我母亲不知掠往那里去了。我在此已
吊三日三夜,更没一个人来行走。不知那世里修积,今生得遇老师父。若肯舍大慈
悲,救我一命回家,就典身卖命,也酬谢师恩。致使黄沙盖面,更不敢忘也。”

三藏闻言,认了真实,就教八戒解放绳索,救他下来。那呆子也不识人,便要
上前动手。行者在旁,忍不住喝了一声道:“那泼物!有认得你的在这里哩!莫要只
管架空捣鬼,说谎哄人!你既家私被劫,父被贼伤,母被人掳,救你去交与谁人?你
将何物与我作谢?这谎脱节了耶!”

那怪闻言,心中害怕,就知大圣是个能人,暗将他放在心上;却又战战兢兢,
滴泪而言曰:“师父,虽然我父母空亡,家财尽绝,还有些田产未动,亲戚皆存。”
行者道:“你有甚么亲戚?”妖怪道:
“我外公家在山南,姑娘住居岭北。涧头李四,是我姨夫;林内红三,是我族伯。
还有堂叔、堂兄都住在本庄左右。老师父若肯救我,到了庄上,见了诸亲,将老师
父拯救之恩,一一对众言说,典卖些田产,重重酬谢也。”

八戒听说,扛住行者道:“哥哥,这等一个小孩子家,你只管盘诘他怎的!他说
得是,强盗只打劫他些浮财,莫成连房屋田产也劫得去?若与他亲戚们说了,我们
纵有广大食肠,也吃不了他十亩田价。救他下来罢。”呆子只是想着吃食,那里管
甚么好歹,使戒刀挑断绳索,放下怪来。

那怪对唐僧马下,泪汪汪只情磕头。长老心慈,便叫:“孩儿,你上马来,我
带你去。”那怪道:“师父啊,我手脚都吊麻了,腰胯疼痛,一则是乡下人家,不惯
骑马。”唐僧叫八戒驮着,那妖怪抹了一眼道:“师父,我的皮肤都冻熟了,不敢要
这位师父驮。他的嘴长耳大,脑后鬃硬,搠得我慌。”唐僧道:“教沙和尚驮着。”
那怪也抹了一眼道:“师父,那些贼来打劫我家时,一个个都搽了花脸,带假胡子,
拿刀弄杖的。我被他唬怕了,见这位晦气脸的师父,一发没了魂了,也不敢要他驮。”
唐僧教孙行者驮着。行者呵呵笑道:“我驮,我驮!”

那怪物暗自欢喜。顺顺当当的要行者驮他。行者把他扯在路旁边,试了一试,
只好有三斤十来两重。行者笑道:“你这个泼怪物,今日该死了;怎么在老孙面前
捣鬼!我认得你是个‘那话儿’呵。”妖怪道:“师父,我是好人家儿女,不幸遭此
大难,我怎么是个甚么‘那话儿’?”行者道:“你既是好人家儿女,怎么这等骨
头轻?”妖怪道:“我骨格儿小。”行者道:“你今年几岁了?”那怪道:“我七岁了。”
行者笑道:“一岁长一斤,也该七斤。你怎么不满四斤重么?”那怪道:“我小时失
乳。”行者说:“也罢,我驮着你;若要尿尿把把,须和我说。”三藏才与八戒、沙
僧前走,行者背着孩儿随后,一行径投西去。有诗为证,诗曰:
道德高隆魔障高,禅机本静静生妖。
心君正直行中道,木母痴顽外裈。
意马不言怀爱欲,黄婆无语自忧焦。
客邪得志空欢喜,毕竟还从正处消。

孙大圣驮着妖魔,心中埋怨唐僧,不知艰苦,“行此险峻山场,空身也难走,
却教老孙驮人。这厮莫说他是妖怪,就是好人,他没了父母,不知将他驮与何人,
倒不如掼杀他罢。”那怪物却早知觉了。便就使个神通,往四下里吸了四口气,吹
在行者背上,便觉重有千斤。行者笑道:“我儿啊,你弄重身法压我老爷哩!”那怪
闻言,恐怕大圣伤他,却就解尸,出了元神,跳将起去,伫立在九霄空里。这行者
背上越重了。猴王发怒,抓过他来,往那路旁边赖石头上滑辣的一掼,将尸骸掼得
像个肉饼一般。还恐他又无礼,索性将四肢扯下,丢在路两边,俱粉碎了。

那物在空中,明明看着,忍不住心头火起道:“这猴和尚,十分惫懒!就作我是
个妖魔,要害你师父,却还不曾见怎么下手哩,你怎么就把我这等伤损!早是我有
算计,出神走了。不然,是无故伤生也。若不趁此时拿了唐僧,再让一番,越教他
停留长智。”好怪物,就在半空里弄了一阵旋风,呼的一声响亮,走石扬沙,诚然
凶狠。好风:
淘淘怒卷水云腥,黑气腾腾闭日明。
岭树连根通拔尽,野梅带干悉皆平。
黄沙迷目人难走,怪石伤残路怎平。
滚滚团团平地暗,遍山禽兽发哮声。
刮得那三藏马上难存,八戒不敢仰视,沙僧低头掩面。孙大圣情知是怪物弄风,急
纵步来赶时,那怪已骋风头,将唐僧摄去了,无踪无影,不知摄向何方,无处跟寻。

一时间,风声暂息,日色光明。行者上前观看,只见白龙马,战兢兢发喊声嘶;
行李担,丢在路下;八戒伏于崖下呻吟,沙僧蹲在坡前叫唤。行者喊:“八戒!”那
呆子听见是行者的声音,却抬头看时,狂风已静。爬起来,扯住行者道:“哥哥,
好大风啊!”沙僧却也上前道:“哥哥,这是一阵旋风。”又问:“师父在那里?”八
戒道:“风来得紧,我们都藏头遮眼,各自躲风,师父也伏在马上的。”行者道:“如
今却往那里去了?”沙僧道:“是个灯草做的,想被一风卷去也。”

行者道:“兄弟们,我等自此就该散了!”八戒道:“正是,趁早散了,各寻头
路,多少是好。那西天路无穷无尽,几时能到得!”沙僧闻言,打了一个失惊,浑
身麻木道:“师兄,你都说的是那里话。我等因为前生有罪,感蒙观世音菩萨劝化,
与我们摩顶受戒,改换法名,皈依佛果,情愿保护唐僧上西方拜佛求经,将功折罪。
今日到此,一旦俱休,说出这等各寻头路的话来,可不违了菩萨的善果,坏了自己
的德行,惹人耻笑,说我们有始无终也!”行者道:“兄弟,你说的也是。奈何师父
不听人说。我老孙火眼金睛,认得好歹。才然这风,是那树上吊的孩儿弄的。我认
得他是个妖精,你们不识,那师父也不识,认作是好人家儿女,教我驮着他走。是
老孙算计要摆布他,他就弄个重身法压我。是我把他掼得粉碎,他想是又使解尸之
法,弄阵旋风,把我师父摄去也。因此上怪他每每不听我说,故我意懒心灰,说各
人散了。既是贤弟有此诚意,教老孙进退两难。八戒,你端的要怎的处?”八戒道:
“我才自失口乱说了几句,其实也不该散。哥哥,没及奈何,还信沙弟之言,去寻
那妖怪救师父去。”行者却回嗔作喜道:“兄弟们,还要来结同心,收拾了行李、马
匹,上山找寻怪物,搭救师父去。”

三个人附葛扳藤,寻坡转涧,行经有五七十里,却也没个音信。那山上飞禽走
兽全无,老柏乔松常见。孙大圣着实心焦,将身一纵,跳上那巅●峰头,喝一声叫
“变!”变作三头六臂,似那大闹天宫的本象。将金箍棒,幌一幌,变作三根金箍
棒,劈哩扑辣的,往东打一路,往西打一路,两边不住的乱打。八戒见了道:“沙
和尚,不好了。师兄是寻不着师父,恼出气心风来了。”

那行者打了一会,打出一伙穷神来。都披一片,挂一片,●无裆,裤无口的,
跪在山前,叫:“大圣,山神、土地来见。”行者道:“怎么就有许多山神、土地?”
众神叩头道:“上告大圣。此山唤做‘六百里钻头号山’。我等是十里一山神,十里
一土地,共该三十名山神,三十名土地。昨日已此闻大圣来了,只因一时会不齐,
故此接迟,致令大圣发怒。万望恕罪。”行者道:“我且饶你罪名。我问你:“这山
上有多少妖精?”众神道:“爷爷呀,只有得一个妖精,把我们头也摩光了;弄得
我们少香没纸,血食全无,一个个衣不充身,食不充口,还吃得有多少妖精哩!”
行者道:“这妖精在山前住,是山后住?”众神道:“他也不在山前山后。这山中有
一条涧,叫做枯松涧。涧边有一座洞,叫做火云洞。那洞里有一个魔王,神通广大,
常常的把我们山神、土地拿了去,烧火顶门,黑夜与他提铃喝号。小妖儿又讨甚么
常例钱。”行者道:“汝等乃是阴鬼之仙,有何钱钞?”众神道:“正是没钱与他,
只得捉几个山獐、野鹿,早晚间打点群精;若是没物相送,就要来拆宙宇,剥衣裳,
搅得我等不得安生!万望大圣与我等剿除此怪,拯救山上生灵。”行者道:“你等既
受他节制,常在他洞下,可知他是那里妖精,叫做甚么名字?”众神道:“说起他
来,或者大圣也知道。他是牛魔王的儿子,罗刹女养的。他曾在火焰山修行了三百
年,炼成‘三昧真火’,却也神通广大。牛魔王使他来镇守号山,乳名叫做红孩儿,
号叫做圣婴大王。”

行者闻言,满心欢喜。喝退了土地、山神,却现了本象,跳下峰头,对八戒、
沙僧道:“兄弟们放心,再不须思念。师父决不伤生。妖精与老孙有亲。”八戒笑道:
“哥哥,莫要说谎。你在东胜神洲,他这里是西牛贺洲,路程遥远,隔着万水千山,
海洋也有两道,怎的与你有亲?”行者道:“刚才这伙人都是本境土地、山神。我
问他妖怪的原因,他道是牛魔王的儿子,罗刹女养的,名字唤做红孩儿,号圣婴大
王。想我老孙五百年前大闹天宫时,遍游天下名山,寻访大地豪杰,那牛魔王曾与
老孙结七弟兄。一般五六个魔王,止有老孙生得小巧,故此把牛魔王称为大哥。这
妖精是牛魔王的儿子,我与他父亲相识,若论将起来,还是他老叔哩。他怎敢害我
师父?我们趁早去来。”沙和尚笑道:“哥啊,常言道:‘三年不上门,当亲也不亲’
哩。你与他相别五六百年,又不曾往还杯酒,又没有个节礼相邀,他那里与你认甚
么亲耶?”行者道:“你怎么这等量人!常言道:‘一叶浮萍归大海,为人何处不相
逢!’纵然他不认亲,好道也不伤我师父。不望他相留酒席,必定也还我个囫囵唐
僧。”三兄弟各办虔心,牵着白马,马上驮着行李,找大路一直前进。

无分昼夜,行了百十里远近,忽见一松林,林中有一条曲涧,涧下有碧澄澄的
活水飞流,那涧梢头有一座石板桥,通着那厢洞府。行者道:“兄弟,你看那壁厢
有石崖磷磷,想必是妖精住处了。我等从众商议:那个管看守行李、马匹,那个肯
跟我过去降妖。”八戒道:“哥哥,老猪没甚坐性,我随你去罢。”行者道:“好,好!”
教沙僧:“将马匹、行李俱潜在树林深处,小心守护,待我两个上门去寻师父耶。”
那沙僧依命,八戒相随,与行者各持兵器前来。正是:
未炼婴儿邪火胜,心猿木母共扶持。

毕竟不知这一去吉凶何如,且听下回分解。