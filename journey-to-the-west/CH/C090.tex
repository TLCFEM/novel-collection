\chapter{师狮授受同归一~盗道缠禅静九灵}

却说孙大圣同八戒、沙僧出城头,觌面相迎,见那伙妖精都是些杂毛狮子:黄
狮精在前引领,狻猊狮、抟象狮在左,白泽狮、伏狸狮在右,猱狮、雪狮在后,中
间却是一个九头狮子。那青脸儿怪执一面锦绣团花宝幢,紧挨着九头狮子;刁钻古
怪儿、古怪刁钻儿打两面红旗,齐齐的都布在坎宫之地。

八戒莽撞,走近前骂道:“偷宝贝的贼怪!你去那里,伙这几个毛团来此怎的?”
黄狮精切齿骂道:“泼狠秃厮!昨日三个敌我一个,我败回去,让你为人罢了;你怎
么这般狠恶,烧了我的洞府,损了我的山场,伤了我的眷族!我和你冤仇深如大海!
不要走!吃你老爷一铲!”好八戒,举钯就迎。两个才交手,还未见高低,那猱狮精
轮一根铁蒺藜,雪狮精使一条三楞简,径来奔打。八戒发一声喊道:“来得好!”你
看他横冲直抵,斗在一处。这壁厢,沙和尚急掣降妖杖,近前相助。又见那狻猊精、
白泽精与抟象、伏狸二精,一拥齐上。这里孙大圣使金箍棒架住群精。狻猊使闷棍,
白泽使铜锤,抟象使钢枪,伏狸使钺斧。那七个狮子精,这三个狠和尚,好杀:

棍锤枪斧三楞简,蒺藜骨朵四明铲。七狮七器甚锋芒,围战三僧齐呐喊。大圣
金箍铁棒凶,沙僧宝杖人间罕。八戒颠风骋势雄,钉钯幌亮光华惨。前遮后挡各施
功,左架右迎都勇敢。城头王子助威风,擂鼓筛锣齐壮胆。投来抢去弄神通,杀
得昏蒙天地反!

那一伙妖精,齐与大圣三人,战经半日,不觉天晚。八戒口吐粘涎,看看脚软,
虚幌一钯,败下阵去,被那雪狮、猱狮二精喝道:“那里走,看打!”呆子躲闪不及,
被他照脊梁上打了一简,睡在地下,只叫:“罢了,罢了!”两个精把八戒采鬃拖尾,
扛将去见那九头狮子,报道:“祖爷,我等拿了一个来也。”

说不了,沙僧、行者也都战败。众妖精一齐赶来,被行者拔一把毫毛,嚼碎喷
将去,叫声:“变!”即变做百十个小行者,围围绕绕,将那白泽、狻猊、抟象、伏
狸并金毛狮怪围裹在中。沙僧、行者却又上前攒打。到晚,拿住狻猊、白泽。走了
伏狸、抟象。金毛报知老妖,老怪见失了二狮,吩咐:“把猪八戒捆了,不可伤他
性命。待他还我二狮,却将八戒与他。他若无知,坏了我二狮,即将八戒杀了对命!”
当晚群妖安歇城外不题。

却说孙大圣把两个狮子精抬近城边,老王见了,即传令开门,差二三十个校尉,
拿绳扛出门,绑了狮精,扛入城里。孙大圣收了法毛,同沙僧径至城楼上,见了唐
僧。唐僧道:“这场事甚是利害呀!悟能性命,不知有无?”行者道:“没事!我们把
这两个妖精拿了,他那里断不敢伤。且将二精牢拴紧缚,待明早抵换八戒也。”

三个小王子对行者叩头道:“师父先前赌斗,只见一身;及后佯输而回,却怎
么就有百十位师身?及至拿住妖精,近城来还是一身,此是甚么法力?”行者笑道:
“我身上有八万四千毫毛,以一化十,以十化百,百千万亿之变化,皆身外身之法
也。”那王子一个个顶礼,即时摆上斋来,就在城楼上吃了。各垛口上都要灯笼旗
帜,梆铃锣鼓,支更传箭,放炮呐喊。

早又天明。老怪即唤黄狮精定计道:“汝等今日用心拿那行者、沙僧,等我暗
自飞空上城,拿他那师父并那老王父子,先转九曲盘桓洞,待你得胜回报。”

黄狮领计,便引猱狮、雪狮、抟象、伏狸各执兵器到城边,滚风酿雾的索战。
这里行者与沙僧跳出城头,厉声骂道:“贼泼怪!快将我师弟八戒送还我,饶你性命!
不然,都教你粉骨碎尸!”那妖精那容分说,一拥齐来。这大圣弟兄两个,各运机
谋,挡住五个狮子。这杀比昨日又甚不同:

呼呼刮地狂风恶,暗暗遮天黑雾浓。走石飞沙神鬼怕,推林倒树虎狼惊。钢枪
狠狠钺斧明,棍铲铜锤太毒情。恨不得囫囵吞行者,活活泼泼擒住小沙僧。这大圣
一条如意棒,卷舒收放甚精灵。沙僧那柄降妖杖,灵霄殿外有名声。今番干运神通
广,西域施功扫荡精。

这五个杂毛狮子精与行者、沙僧正自杀到好处,那老怪驾着黑云,径直腾至城
楼上,摇一摇头,唬得那城上文武大小官员并守城人夫等,都滚下城去;被他奔入
楼中,张开口,把三藏与老王父子一顿噙出,复至坎宫地下,将八戒也着口噙之。
原来他九个头就有九张口。一口噙着唐僧,一口噙着八戒,一口噙着老王,一口噙
着大王子,一口噙着二王子,一口噙着三王子:六口噙着六人,还空了三张口,发
声喊叫道:“我先去也!”这五个小狮精见他祖得胜,一个个愈展雄才。

行者闻得城上人喊嚷,情知中了他计,急唤沙僧仔细;他却把臂膊上毫毛,尽
皆拔下,入口嚼烂喷出,变作千百个小行者,一拥攻上。当时拖倒猱狮,活捉了雪
狮,拿住了抟象狮,扛翻了伏狸狮,将黄狮打死,烘烘的嚷到州城之下,倒转走脱
了青脸儿与刁钻古怪、古怪刁钻儿二怪。

那城上官看见,却又开门,将绳把五个狮精又捆了,抬进城去。还未发落,只
见那王妃哭哭啼啼,对行者礼拜道:“神师啊,我殿下父子并你师父,性命休矣!这
孤城怎生是好?”大圣收了法毛,对王妃作礼道:“贤后莫愁。只因我拿他七个狮
精,那老妖弄摄法,定将我师父与殿下父子摄去,料必无伤。待明日绝早,我兄弟
二人去那山中,管情捉住老妖,还你四个王子。”那王妃一簇女眷闻得此言,都对
行者下拜道:“愿求殿下父子全生,皇图坚固!”拜毕,一个个含泪还宫。行者吩咐
各官:“将打死那黄狮精,剥了皮;六个活狮精,牢牢拴锁。取些斋饭来,我们吃
了睡觉。你们都放心,保你无事。”

至次日,大圣领沙僧驾起祥云,不多时,到于竹节山头。按云头观看,好座高
山!但见:

峰排突兀,岭峻崎岖。深涧下潺水漱,陡崖前锦绣花香。回峦重迭,古道湾
环。真是鹤来松有伴,果然云去石无依。玄猿觅果向晴辉,麋鹿寻花欢日暖。青鸾
声淅呖,黄鸟语绵蛮。春来桃李争妍,夏至柳槐竞茂。秋到黄花布锦,冬交白雪飞
绵。四时八节好风光,不亚瀛洲仙景象。
他两个正在山头上看景,忽见那青脸儿,手拿一条短棍,径跑出崖谷之间。行者喝
道:“那里走!老孙来也!”唬得那小妖一翻一滚的跑下崖谷。他两个一直追来,又
不见踪迹。向前又转几步,却是一座洞府。两扇花斑石门,紧紧关闭。门上横嵌
着一块石版,楷镌了十个大字,乃是“万灵竹节山,九曲盘桓洞。”

那小妖原来跑进洞去,即把洞门闭了。到中间对老妖道:“爷爷,外面又有两
个和尚来了。”老妖道:“你大王并猱狮、雪狮、抟象、伏狸,可曾来?”小妖道:
“不见,不见!只是两个和尚,在山峰高处眺望。我看见回头就跑,他赶将来,我
却闭门来也。”老妖听说,低头不语。半晌,忽的吊下泪来,叫声“苦啊!我黄狮孙
死了!猱狮孙等又尽被和尚捉进城去矣!此恨怎生报得!”八戒捆在旁边,与王父子、
唐僧,俱攒在一处,惶惶受苦;听见老妖说声“众孙被和尚捉进城去”,暗暗
喜道:“师父莫怕,殿下休愁。我师兄已得胜,捉了众妖,寻到此间救拔吾等也。”
说罢,又听得老妖叫:“小的们,好生在此看守,等我出去拿那两个和尚进来,一
发惩治。”

你看他身无披挂,手不拈兵,大踏步,走到前边,只闻得孙行者吆喝哩。他就
大开了洞门,不答话,径奔行者。行者使铁棒,当头支住。沙僧轮宝杖就打。那老
妖把头摇一摇,左右八个头,一齐张开口,把行者、沙僧轻轻的又衔于洞内。教:
“取绳索来!”那刁钻古怪、古怪刁钻与青脸儿是昨夜逃生而回者,即拿两条绳,
把他二人着实捆了。

老妖问道:“你这泼猴,把我那七个儿孙捉了,我今拿住你和尚四个,王子四
个,也足以抵得我儿孙之命!小的们,选荆条柳棍来,且打这猴头一顿,与我黄狮
孙报报冤仇!”那三个小妖,各执柳棍,专打行者。行者本是熬炼过的身体,那些
些柳棍儿,只好与他拂痒,他那里做声;凭他怎么捶打,略不介意。八戒、唐僧与
王子见了,一个个毛骨悚然。少时,打折了柳棍。直打到天晚,也不计其数。沙僧
见打得多了,甚不过意道:“我替他打百十下罢。”老妖道:“你且莫忙,明日就打
到你了。一个个挨次儿打将来。”八戒着忙道:“后日就打到我老猪也!”打一会,
渐渐的天昏了。老妖叫:“小的们,且住,点起灯火来,你们吃些饮食,让我到锦
云窝略睡睡去。汝三人都是遭过害的,却用心看守,待明早再打。”三个小妖移过
灯来,拿柳棍又打行者脑盖,就像敲梆子一般,剔剔托,托托剔,紧几下,慢几下。
夜将深了,却都盹睡。

行者就使个遁法,将身一小,脱出绳来,抖一抖毫毛,整束了衣服,耳朵内取
出棒来,幌一幌,有吊桶粗细,二丈长短,朝着三个小妖道:“你这孽畜,把你老
爷就打了许多棍子!老爷还只照旧,老爷也把这棍子略你,看道如何!”把三个
小妖轻轻一,就做三个肉饼;却又剔亮了灯,解放沙僧。八戒捆急了,忍不住
大声叫道:“哥哥!我的手脚都捆肿了,倒不来先解放我!”这呆子喊了一声,却早
惊动老妖。老妖一毂辘爬起来道:“是谁人解放?”那行者听见,一口吹息灯,也
顾不得沙僧等众,使铁棒,打破几重门走了。那老妖到中堂里叫:“小的们,怎么
没了灯光?只莫走了人也?”叫一声,没人答应;又叫一声,又没人答应;及取灯
火来看时,只见地下血淋淋的三块肉饼,老王父子及唐僧、八戒俱在,只不见了行
者、沙僧。点着火,前后赶看,忽见沙僧还背贴在廊下站哩;被他一把拿住倒,
照旧捆了。又找寻行者,但见几层门尽皆破损,情知是行者打破走了;也不去追赶,
将破门补的补,遮的遮,固守家业不题。

却说孙大圣出了那九曲盘桓洞,跨祥云,径转玉华州。但见那城头上各厢的土
地、神与城隍之神迎空拜接。行者道:“汝等怎么今夜才见?”城隍道:“小神等
知大圣下降玉华州,因有贤王款留,故不敢见,今知王等遇怪,大圣降魔,特来叩
接。”行者正在嗔怪处,又见金头揭谛、六甲六丁神将,押着一尊土地,跪在面前
道:“大圣,吾等捉得这个地里鬼来也。”行者喝道:“汝等不在竹节山护我师父,
却怎么嚷到这里?”丁甲神道:“大圣,那妖精自你逃时,复捉住卷帘大将,依然
捆了。我等见他法力甚大,却将竹节山土地押解至此。他知那妖精的根由,乞大圣
问他一问,便好处治,以救圣僧、贤王之苦。”行者听言,甚喜。那土地战兢兢叩
头道:“那老妖前年下降竹节山。那九曲盘桓洞原是六狮之窝。那六个狮子,自得
老妖至此,就都拜为祖翁。祖翁乃是个九头狮子,号为九灵元圣。若得他灭,须去
到东极妙岩宫,请他主人公来,方可收伏。他人莫想擒也。”行者闻言,思忆半晌
道:“东极妙岩宫,是太乙救苦天尊啊。他坐下正是个九头狮子。这等说,……”
便教:“揭谛、金甲,还同土地回去,暗中护师父、师弟并州王父子。本处城隍
守护城池,走出去来。”众神各各遵守去讫。

这大圣纵筋斗云,连夜前行。约有寅时分,到了东天门外,正撞着广目天王与
天丁、力士一行仪从。众皆停住,拱手迎道:“大圣何往?”行者对众礼毕,道:“前
去妙岩宫走走。”天王道:“西天路不走,却又东天来做甚?”行者道:“因到玉华
州,蒙州王相款,遣三子拜我等弟兄为师,习学武艺,不期遇着一伙狮怪。今访得
妙岩宫太乙救苦天尊乃怪之主人公也,欲请他为我降怪救师。”天王道:“那厢因你
欲为人师,所以惹出这一窝狮子来也。”行者笑道:“正为此,正为此!”众天丁、
力士一个个拱手,让道而行。大圣进了东天门,不多时,到妙岩宫前。但见:

彩云重迭,紫气茏葱。瓦漾金波焰,门排玉兽崇。花盈双阙红霞绕,日映骞林
翠雾笼。果然是万真环拱,千圣兴隆。殿阁层层锦,窗轩处处通。苍龙盘护神光蔼,
黄道光辉瑞气浓。这的是青华长乐界,东极妙岩宫。
那宫门里立着一个穿霓帔的仙童,忽见孙大圣,即入宫报道:“爷爷,外面是闹天
宫的齐天大圣来了。”太乙救苦天尊听得,即唤侍卫众仙迎接。迎至宫中。只见天
尊高坐九色莲花座上,百亿瑞光之中。见了行者,下座来相见。

行者朝上施礼。天尊答礼道:“大圣,这几年不见,前闻得你弃道归佛,保唐
僧西天取经,想是功行完了。”行者道:“功行未完,却也将近;但如今因保唐僧到
玉华州,蒙王子遣三子拜老孙等为师,习学武艺,把我们三件神兵照样打造,不期
夜间被贼偷去。及天明寻找,原是城北豹头山虎口洞一个金毛狮子成精盗去。老孙
用计取出,那精就伙了若干狮精与老孙大闹。内有一个九头狮子,神通广大,将我
师父与八戒并王父子四人都衔去,到一竹节山九曲盘桓洞。次日,老孙与沙僧跟寻,
亦被衔去。老孙被他捆打无数,幸而弄法走了。他们正在彼处受罪。问及当坊土地,
始知天尊是他主人,特来奉请收降解救。”

天尊闻言,即令仙将到狮子房唤出狮奴来问。那狮奴熟睡,被众将推摇方醒,
揪至中厅来见。天尊问道:“狮兽何在?”那奴儿垂泪叩头,只教:“饶命!饶命!”
天尊道:“孙大圣在此,且不打你。你快说为何不谨,走了九头狮子。”狮奴道:“爷
爷,我前日在大千甘露殿中见一瓶酒,不知偷去吃了,不觉沉醉睡着,失于拴锁,
是以走了。”天尊道:“那酒是太上老君送的,唤做‘轮回琼液’。你吃了该醉三日
不醒。那狮兽今走几日了?”大圣道:“据土地说,他前年下降,到今二三年矣。”
天尊笑道:“是了,是了,天宫里一日,在凡世就是一年。”叫狮奴道:“你且起来,
饶你死罪,跟我与大圣下方去收他来。汝众仙都回去,不用跟随。”

天尊遂与大圣、狮奴,踏云径至竹节山。只见那五方揭谛、六丁六甲、本山土
地都来跪接。行者道:“汝等护,可曾伤着我师?”众神道:“妖精着了恼睡了,
更不曾动甚刑罚。”天尊道:“我那元圣儿也是一个久修得道的真灵,他喊一声,上
通三圣,下彻九泉,等闲也便不伤生。孙大圣,你去他门首索战,引他出来,我好
收之。”

行者听言,果掣棒跳近洞口,高骂道:“泼妖精,还我人来也!泼妖精,还我人
来也!”连叫了数声。那老妖睡着了,无人答应。行者性急起来,轮铁棒,往里打
进,口中不住的喊骂。那老妖方才惊醒,心中大怒。爬起来,喝一声:“赶战!”摇
摇头,便张口来衔。

行者回头跳出。妖精赶到外边,骂道:“贼猴!那里走!”行者立在高崖上笑道:
“你还敢这等大胆无礼!你死活也不知哩!这不是你老爷主公在此?”那妖精赶到崖
前,早被天尊念声咒语,喝道:“元圣儿,我来了!”那妖认得是主人,不敢展挣,
四只脚伏之于地,只是磕头。旁边跑过狮奴儿,一把挝住项毛,用拳着项上打够百
十,口里骂道:“你这畜生,如何偷走,教我受罪!”那狮兽合口无言,不敢摇动。
狮奴儿打得手困,方才住了。即将锦安在他身上,天尊骑了,喝声教走。他就纵
身驾起彩云,径转妙岩宫去。

大圣望空称谢了。却入洞中,先解玉华王,次解唐三藏,次又解了八戒、沙僧
并三王子。共搜他洞里物件,逍逍停停,将众领出门外。八戒就取了若干枯柴,前
后堆上,放起火来,把一个九曲盘桓洞,烧做个乌焦破瓦窑!大圣又发放了众神,
还教土地在此镇守。却令八戒、沙僧,各各使法,把王父子背驮回州。他搀着唐僧。
不多时,到了州城,天色渐晚,当有妃后官员,都来接见了。摆上斋筵,共坐享之。
长老师徒还在暴纱亭安歇。王子们入宫各寝。一宵无话。

次日,王又传旨,大开素宴。合府大小官员,一一谢恩。行者又叫屠子来,把
那六个活狮子杀了,共那黄狮子都剥了皮,将肉安排将来受用。殿下十分欢喜,即
命杀了。把一个留在本府内外人用,一个与王府长史等官分用;把五个都剁做一二
两重的块子,差校尉散给州城内外军民人等,各吃些须:一则尝尝滋味,二则押押
惊恐。那些家家户户,无不瞻仰。

又见那铁匠人等造成了三般兵器,对行者磕头道:“爷爷,小的们工都完了。”
问道:“各重多少斤两?”铁匠道:“金箍棒有千斤,九齿钯与降妖杖各有八百斤。”
行者道:“也罢了。”叫请三位王子出来,各人执兵器。三子对老王道:“父王,今
日兵器完矣。”老王道:“为此兵器,几乎伤了我父子之命。”小王子道:“幸蒙神师
施法,救出我等,却又扫荡妖邪,除了后患。诚所谓海晏河清,太平之世界也!”
当时老王父子赏劳了匠作,又至暴纱亭拜谢了师恩。

三藏又教大圣等快传武艺,莫误行程。他三人就各轮兵器,在王府院中,一一
传授。不数日,那三个王子尽皆操演精熟,其余攻退之方,紧慢之法,各有七十二
到解数,无不知之。一则那诸王子心坚,二则亏孙大圣先授了神力,此所以那千斤
之棒,八百斤之钯杖,俱能举能运。较之初时自家弄的武艺,真天渊也!有诗为证,
诗曰:
缘因善庆遇神师,习武何期动怪狮。
扫荡群邪安社稷,皈依一体定边夷。
九灵数合元阳理,四面精通道果之。
授受心明遗万古,玉华永乐太平时。

那王子又大开筵宴,谢了师教。又取出一大盘金银,用答微情。行者笑道:“快
拿进去,快拿进去,我们出家人,要他何用?”八戒在旁道:“金银实不敢受,奈
何我这件衣服被那些狮子精扯拉破了,但与我们换件衣服,足为爱也。”那王子随
命针工,照依色样,取青锦、红锦、茶褐锦各数匹,与三位各做了一件。三人欣然
领受,各穿了锦布直裰,收拾了行装起程。只见那城里城外,若大若小,无一人不
称是罗汉临凡,活佛下界。鼓乐之声,旌旗之色,盈街塞道。正是家家户外焚香火,
处处门前献彩灯。送至许远方回。他四众方得离城西去。这一去顿脱群思,潜心正
果。才是:
无虑无忧来佛界,诚心诚意上雷音。

毕竟不知到灵山还有几多路程,何时行满,且听下回分解。