\chapter{朱紫国唐僧论前世~孙行者施为三折肱}

善正万缘收,名誉传扬四部洲。智慧光明登彼岸,飕飕,云生天际头。诸
佛共相酬,永住瑶台万万秋。打破人间蝴蝶梦,休休,涤净尘氛不惹愁。

话表三藏师徒,洗污秽之胡同,上遥逍之道路,光阴迅速,又值炎天。正是:
海榴舒锦弹,荷叶绽青盘。
两路绿杨藏乳燕,行人避暑扇摇纨。

进前行处,忽见有一城池相近。三藏勒马叫:“徒弟们,你看那是甚么去处?”
行者道:“师父原来不识字,亏你怎么领唐王旨意离朝也!”三藏道:“我自幼为僧,
千经万典皆通,怎么说我不识字?”行者道:“既识字,怎么那城头上杏黄旗,明
书三个大字,就不认得,却问是甚去处何也?”三藏喝道:“这泼猴胡说!那旗被风
吹得乱摆,纵有字也看不明白!”行者道:“老孙偏怎看见?”八戒、沙僧道:“师
父,莫听师兄捣鬼。这般遥望,城池尚不明白,如何就见是甚字号?”行者道:“却
不是‘朱紫国’三字?”三藏道:“朱紫国必是西邦王位,却要倒换关文。”行者道:
“不消讲了。”

不多时,至城门下马,过桥,入进三层门里,真个好个皇州!但见:

门楼高耸,垛迭齐排。周围活水通流,南北高山相对。六街三市货资多,万户
千家生意盛。果然是个帝王都会处,天府大京城。绝域梯航至,遐方玉帛盈。形胜
连山远,宫垣接汉清。三关严锁钥,万古乐升平。
师徒们在那大街市上行时,但见人物轩昂,衣冠齐整,言语清朗,真不亚大唐世界。
那两边做买做卖的,忽见猪八戒相貌丑陋,沙和尚面黑身长,孙行者脸毛额廓,丢
了买卖,都来争看。三藏只叫:“不要撞祸,低着头走!”八戒遵依,把个把子嘴揣
在怀里;沙僧不敢仰视;惟行者东张西望,紧随唐僧左右。那些人有知事的,看看
儿就回去了。有那游手好闲的,并那顽童们,烘烘笑笑,都上前抛瓦丢砖,与八戒
作戏。唐僧捏着一把汗,只教莫要生事,那呆子不敢抬头。

不多时,转过隅头,忽见一座门墙,上有“会同馆”三字。唐僧道:“徒弟,
我们进这衙门去也。”行者道:“进去怎的?”唐僧道:“会同馆乃天下通会通同之
所,我们也打搅得。且到里面歇下。待我见驾,倒换了关文,再赶出城走路。”八
戒闻言,掣出嘴来,把那些随看的人,唬倒了数十个。他上前道:“师父说的是。
我们且到里边藏下,免得这伙鸟人吵嚷。”遂进馆去。那些人方渐渐而退。

却说那馆中有两个大使,乃是一正一副,都在厅上查点人夫,要往那里接官。
忽见唐僧来到,个个心惊,齐道:“是甚么人,是甚么人?往那里走?”三藏合掌道:
“贫僧乃东土大唐驾下,差往西天取经者。今到宝方,不敢私过,有关文欲倒验放
行,权借高衙暂歇。”那两个馆使听言,屏退左右,一个个整冠束带,下厅迎上相
见。即命打扫客房安歇,教办清素支应。三藏谢了。二官带领人夫,出厅而去。手
下人请老爷客房安歇,三藏便走。行者恨道:“这厮惫懒!怎么不让老孙在正厅?”
三藏道:“他这里不服我大唐管属,又不与我国相连,况不时又有上司过客往来,
所以不好留此相待。”行者道:“这等说,我偏要他相待!”

正说处,有管事的送支应来,乃是一盘白米、一盘白面、两把青菜、四块豆腐、
两个面筋、一盘干笋、一盘木耳。三藏教徒弟收了,谢了管事的。管事的道:“西
房里有干净锅灶,柴火方便,请自去做饭。”三藏道:“我问你一声,国王可在殿上
么?”管事的道:“我万岁爷爷久不上朝,今日乃黄道良辰,正与文武多官议出黄
榜。你若要倒换关文,趁此急去,还赶上;到明日,就不能够了,不知还有多少时
伺候哩。”三藏道:“悟空,你们在此安排斋饭,等我急急去验了关文回来,吃了走
路。”八戒急取出袈裟关文。三藏整束了进朝,只是吩咐徒弟们,切不可出外去生
事。

不一时,已到五凤楼前。说不尽那殿阁峥嵘,楼台壮丽。直至端门外,烦奏事
官转达天廷,欲倒验关文。那黄门官果至玉阶前,启奏道:“朝门外有东土大唐钦
差一员僧,前往西天雷音寺拜佛求经,欲倒换通关文牒,听宣。”国王闻言,喜道:
“寡人久病,不曾登基;今上殿出榜招医,就有高僧来国!”即传旨宣至阶下。三
藏即礼拜俯伏。国王又宣上金殿赐坐,命光禄寺办斋。三藏谢了恩,将关文献上。

国王看毕,十分欢喜道:“法师,你那大唐,几朝君正?几辈臣贤?至于唐王,
因甚作疾回生,着你远涉山川求经?”这长老因问,即欠身合掌道:“贫僧那里:

三皇治世,五帝分伦。尧舜正位,禹汤安民。成周子众,各立乾坤。倚强欺弱,
分国称君。邦君十八,分野边尘。后成十二,宇宙安淳。因无车马,却又相吞。七
雄争胜,六国归秦。天生鲁沛,各怀不仁。江山属汉,约法钦遵。汉归司马,晋又
纷纭。南北十二,宋齐梁陈。列祖相继,大隋绍真。赏花无道,涂炭多民。我王李
氏,国号唐君。高祖晏驾,当今世民。河清海晏,大德宽仁。兹因长安城北,有个
怪水龙神,刻减甘
雨,应该损身。夜间托梦,告王救。王言准赦,早召贤臣。款留殿内,慢把棋轮。
时当日午,那贤臣梦斩龙身。”
国王闻言,忽作呻吟之声,问道:“法师,那贤臣是那邦来者?”三藏道:“就是我
王驾前丞相,姓魏名徵。他识天文,知地理,辨阴阳,乃安邦立国之大宰辅也。因
他梦斩了泾河龙王,那龙王告到阴司,说我王许救又杀之,故我王遂得促病,渐觉
身危。魏徵又写书一封,与我王带至冥司,寄与酆都城判官崔圭。少时,唐王身死,
至三日复得回生。亏了魏徵,感崔判官改了文书,加王二十年寿。今要做水陆大会,
故遣贫僧远涉道途,询求诸国,拜佛祖,取大乘经三藏,超度孽苦升天也。”那国
王又呻吟叹道:“诚乃是天朝大国,君正臣贤!似我寡人久病多时,并无一臣拯救。”
长老听说,偷睛观看,见那皇帝面黄肌瘦,形脱神衰。长老正欲启问,有光禄寺官,
奏请唐僧奉斋。王传旨,教:“在披香殿,连朕之膳摆下,与法师同享。”三藏谢了
恩,与王同进膳进斋不题。

却说行者在会同馆中,着沙僧安排茶饭,并整治素菜。沙僧道:“茶饭易煮,
蔬菜不好安排。”行者问道:“如何?”沙僧道:“油、盐、酱、醋俱无也。”行者道:
“我这里有几文衬钱,教八戒上街买去。”那呆子躲懒道:“我不敢去。嘴脸欠俊,
恐惹下祸来,师父怪我。”行者道:“公平交易,又不讹他,又不抢他,何祸之有!”
八戒道:“你才不曾看见獐智?在这门前扯出嘴来,把人唬倒了十来个;若到闹市丛
中,也不知唬杀多少人是!”行者道:“你只知闹市丛中,你可曾看见那市上卖的是
甚么东西?”八戒道:“师父只教我低着头,莫撞祸,实是不曾看见。”行者道:“酒
店、米铺、磨坊,并绫罗杂货不消说;着然又好茶房、面店,大烧饼、大馍馍,饭
店又有好汤饭、好椒料、好蔬菜,与那异品的糖糕、蒸酥、点心、卷子、油食、蜜
食,……无数好东西,我去买些儿请你如何?”那呆子闻说,口内流涎,喉咙里
的咽唾,跳起来道:“哥哥!这遭我扰你,待下次趱钱,我也请你回席。”行者暗
笑道:“沙僧,好生煮饭,等我们去买调和来。”沙僧也知是耍呆子,只得顺口应承
道:“你们去,须是多买些,吃饱了来。”那呆子捞个碗盏拿了,就跟行者出门。有
两个在官人问道:“长老那里去!”行者道:“买调和。”那人道:“这条街往西去,
转过拐角鼓楼,那郑家杂货店,凭你买多少,油、盐、酱、醋、姜、椒、茶叶俱全。”

他二人携手相搀,径上街西而去。行者过了几处茶房,几家饭店,当买的不买,
当吃的不吃。八戒叫道:“师兄,这里将就买些用罢。”那行者原是耍他,那里肯买,
道:“贤弟,你好不经纪!再走走,拣大的买吃。”两个人说说话儿,又领了许多人
跟随争看。不时,到了鼓楼边,只见那楼下无数人喧嚷,挤挤挨挨,填街塞路。八
戒见了道:“哥哥,我不去了。那里人嚷得紧,只怕是拿和尚的。又况是面生可疑
之人,拿了去,怎的了?”行者道:“胡谈!和尚又不犯法,拿我怎的?我们走过去,
到郑家店买些调和来。”八戒道:“罢,罢,罢!我不撞祸。这一挤到人丛里,把耳
朵了两拄,唬得他跌跌爬爬,跌死几个,我倒偿命是!”行者道:“既然如此,你
在这壁根下站定,等我过去买了回来,与你买素面烧饼吃罢。”那呆子将碗盏递与
行者,把嘴拄着墙根,背着脸,死也不动。

这行者走至楼边,果然挤塞。直挨入人丛里听时,原来是那皇榜张挂楼下,故
多人争看。行者挤到近处,闪开火眼金睛,仔细看时,那榜上却云:

朕西牛贺洲朱紫国王,自立业以来,四方平服,百姓清安。近因国事不祥,沉
疴伏枕,淹延日久难痊。本国太医院,屡选良方,未能调治。今出此榜文,普招天
下贤士。不拘北往东来,中华外国,若有精医药者,请登宝殿,疗理朕躬。稍得病
愈,愿将社稷平分,决不虚示。为此出给张挂。须至榜者。
览毕,满心欢喜道:“古人云:‘行动有三分财气。’早是不在馆中呆坐。即此不必
买甚调和,且把取经事宁耐一日,等老孙做个医生耍耍。”

好大圣,弯倒腰,丢了碗盏,拈一撮土,往上洒去,念声咒语,使个隐身法,
轻轻的上前揭了榜。又朝着巽地上吸口仙气吹来,那阵旋风起处,他却回身,径到
八戒站处,只见那呆子嘴拄着墙根,却是睡着了一般。行者更不惊他,将榜文折了,
轻轻揣在他怀里,拽转步,先往会同馆去了不题。

却说那楼下众人,见风起时,各各蒙头闭眼。不觉风过时,没了皇榜,众皆悚
惧。那榜原有十二个太监,十二个校尉,早朝领出。才挂不上三个时辰,被风吹去,
战兢兢左右追寻。忽见猪八戒怀中露出个纸边儿来,众人近前道:“你揭了榜来
耶?”那呆子猛抬头,把嘴一,唬得那几个校尉,踉踉,跌倒在地。他却转
身要走,又被面前几个胆大的扯住道:“你揭了招医的皇榜,还不进朝医治我万岁
去,却待何往?”那呆子慌慌张张道:“你儿子便揭了皇榜!你孙子便会医治!”校
尉道:“你怀中揣的是甚?”呆子却才低头看时,真个有一张字纸。展开一看,咬
着牙骂道:“那猢狲害杀我也!”恨一声,便要扯破,早被众人架住道:“你是死了!
此乃当今国王出的榜文,谁敢扯坏?你既揭在怀中,必有医国之手,快同我去!”八
戒喝道:“汝等不知。这榜不是我揭的,是我师兄孙悟空揭的。他暗暗揣在我怀中,
他却丢下我去了。若得此事明白,我与你寻他去。”众人道:“说甚么乱话!‘现钟
不打打铸钟’?你现揭了榜文,教我们寻谁!不管你,扯了去见主上!”那伙人不分
清白,将呆子推推扯扯。这呆子立定脚,就如生了根一般,十来个人也弄他不动。
八戒道:“汝等不知高低!再扯一会,扯得我呆性子发了,你却休怪!”

不多时,闹动了街人,将他围绕。内有两个年老的太监道:“你这相貌稀奇,
声音不对,是那里来的,这般村强?”八戒道:“我们是东土差往西天取经的。我
师父乃唐王御弟法师,却才入朝,倒换关文去了。我与师兄来此买办调和,我见楼
下人多,未曾敢去,是我师兄教我在此等候。他原来见有榜文,弄阵旋风揭了,暗
揣我怀内,先去了。”那太监道:“我头前见个白面胖和尚,径奔朝门而去,想就是
你师父?”八戒道:“正是,正是。”太监道:“你师兄往那里去了?”八戒道:“我
们一行四众。师父去倒换关文,我三众并行囊、马匹俱歇在会同馆。师兄弄了我,
他先回馆中去了。”太监道:“校尉,不要扯他。我等同到馆中,便知端的。”八戒
道:“你这两个奶奶知事。”众校尉道:“这和尚委不识货!怎么赶着公公叫起奶奶来
耶?”八戒笑道:“不羞!你这反了阴阳的!他二位老妈妈儿,不叫他做婆婆、奶奶,
倒叫他做公公!”众人道:“莫弄嘴!快寻你师兄去。”

那街上人吵吵闹闹,何止三五百,共扛到馆门首。八戒道:“列位住了。我师
兄却不比我任你们作戏。他却是个猛烈认真之士。汝等见了,须要行个大礼,叫他
声‘孙老爷’,他就招架了。不然啊,他就变了嘴脸,这事却弄不成也。”众太监、
校尉俱道:“你师兄果有手段,医好国王,他也该有一半江山,我等合该下拜。”那
些闲杂人都在门外喧哗。八戒领着一行太监、校尉,径入馆中。只听得行者与沙僧
在客房里正说那揭榜之事耍笑哩。八戒上前扯住,乱嚷道:“你可成个人!哄我去买
素面、烧饼、馍馍我吃,原来都是空头。又弄旋风,揭了甚么皇榜,暗暗的揣在我
怀里,拿我装胖!这可成个弟兄!”行者笑道:“你这呆子,想是错了路,走向别处
去。我过鼓楼,买了调和,急回来寻你不见,我先来了。在那里揭甚皇榜?”八戒
道:“现有看榜的官员在此。”

说不了,只见那几个太监、校尉朝上礼拜道:“孙老爷,今日我王有缘,天遣
老爷下降,是必大展经纶手,微施三折肱,治得我王病愈,江山有分,社稷平分也。”
行者闻言,正了声色,接了八戒的榜文,对众道:“你们想是看榜的官么?”太监
叩头道:“奴婢乃司礼监内臣。这几个是锦衣校尉。”行者道:“这招医榜,委是我
揭的,故遣我师弟引见。既然你主有病,常言道:‘药不跟卖,病不讨医。’你去教
那国王亲来请我。我有手到病除之功。”太监闻言,无不惊骇。校尉道:“口出大言,
必有度量。我等着一半在此哑请,着一半入朝启奏。”

当分了四个太监,六个校尉,更不待宣召,径入朝,当阶奏道:“主公万千之
喜!”那国王正与三藏膳毕清谈,忽闻此奏,问道:“喜自何来?”太监奏道:“奴
婢等早领出招医皇榜,鼓楼下张挂,有东土大唐远来取经的一个圣僧孙长老揭了,
现在会同馆内,要王亲自去请他,他有手到病除之功。故此特来启奏。”国王闻言,
满心欢喜,就问唐僧道:“法师有几位高徒?”三藏合掌答曰:“贫僧有三个顽徒。”
国王问:“那一位高徒善医?”三藏道:“实不瞒陛下说。我那顽徒,俱是山野庸才,
只会挑包背马,转涧寻波,带领贫僧登山岭,或者到峻险之处,可以伏魔擒怪,
捉虎降龙而已;更无一个能知药性者。”国王道:“法师何必太谦?朕当今日登殿,
幸遇法师来朝,诚天缘也。高徒既不知医,他怎肯揭我榜文,教寡人亲迎?断然有
医国之能也。”叫:“文武众卿,寡人身虚力怯,不敢乘辇;汝等可替寡人,俱到朝
外,敦请孙长老,看朕之病。汝等见他,切不可轻慢,称他做‘神僧孙长老’,皆
以君臣之礼相见。”

那众臣领旨,与看榜的太监、校尉径至会同馆,排班参拜。唬得那八戒躲在厢
房,沙僧闪于壁下。那大圣,看他坐在当中,端然不动。八戒暗地里怨恶道:“这
猢狲活活的折杀也!怎么这许多官员礼拜,更不还礼,也不站将起来!”不多时,礼
拜毕,分班启奏道:“上告神僧孙长老。我等俱朱紫国王之臣,今奉王旨,敬以洁
礼参请神僧,入朝看病。”行者方才立起身来,对众道:“你王如何不来?”众臣道:
“我王身虚力怯,不敢乘辇,特令臣等行代君之礼,拜请神僧也。”行者道:“既如
此说,列位请前行,我当随至。”众臣各依品从,作队而走。行者整衣而起。八戒
道:“哥哥,切莫攀出我们来。”行者道:“我不攀你,只要你两个与我收药。”沙僧
道:“收甚么药?”行者道:“凡有人送药来与我,照数收下,待我回来取用。”二
人领诺不题。

这行者即同多官,顷间便到。众臣先走,奏知那国王,高卷珠帘,闪龙睛凤目,
开金口御言,便问:“那一位是神僧孙长老?”行者进前一步,厉声道:“老孙便是。”
那国王听得声音凶狠,又见相貌刁钻,唬得战兢兢,跌在龙床之上。慌得那女官内
宦,急扶入宫中。道:“唬杀寡人也!”众官都嗔怨行者道:“这和尚怎么这等粗鲁
村疏!怎敢就擅揭榜!”

行者闻言,笑道:“列位错怪了我也。若像这等慢人,你国王之病,就是一千
年也不得好。”众臣道:“人生能有几多阳寿?就一千年也还不好?”行者道:“他如
今是个病君,死了是个病鬼,再转世也还是个病人,却不是一千年也还不好?”众
臣怒曰:“你这和尚,甚不知礼!怎么敢这等满口胡柴!”行者笑道:“不是胡柴。你
都听我道来:

医门理法至微玄,大要心中有转旋。望闻问切四般事,缺一之时不备全:第一
望他神气色,润枯肥瘦起和眠;第二闻声清与浊,听他真语及狂言;三问病原经几
日,如何饮食怎生便;四才切脉明经络,浮沉表里是何般。我不望闻并问切,今生
莫想得安然。”
那两班文武丛中,有太医院官,一闻此言,对众称扬道:“这和尚也说得有理。就
是神仙看病,也须望、闻、问、切,谨合着神圣功巧也。”众官依此言,着近侍传
奏道:“长老要用望、闻、问、切之理,方可认病用药。”那国王睡在龙床上,声声
唤道:“叫他去罢,寡人见不得生人面了!”近侍的出宫来道:“那和尚,我王旨意,
教你去罢,见不得生人面哩。”行者道:“若见不得生人面啊,我会‘悬丝诊脉’。”
众官暗喜道:“悬丝诊脉,我等耳闻,不曾眼见。再奏去来。”那近侍的又入宫奏道:
“主公,那孙长老不见主公之面,他会悬丝诊脉。”国王心中暗想道:“寡人病了三
年,未曾试此,宣他进来。”近侍的即忙传出道:“主公已许他悬丝诊脉,快宣孙长
老进宫诊视。”

行者却就上了宝殿。唐僧迎着骂道:“你这波猴,害了我也!”行者笑道:“好
师父,我倒与你壮观,你返说我害你?”三藏喝道:“你跟我这几年,那曾见你医
好谁来!你连药性也不知,医书也未读,怎么大胆撞这个大祸!”行者笑道:“师父,
你原来不晓得。我有几个草头方儿,能治大病,管情医得他好便是。就是医杀了,
也只问得个庸医杀人罪名,也不该死,你怕怎的!不打紧,不打紧,你且坐下看我
的脉理如何。”长老又道:“你那曾见《素问》、《难经》、《本草》、《脉诀》,是甚般
章句,怎生注解,就这等胡说散道,会甚么悬丝诊脉!”行者笑道:“我有金线在身,
你不曾见哩。”即伸手下去,尾上拔了三根毫毛,捻一把,叫声“变!”即变作三条
丝线,每条各长二丈四尺,按二十四气,托于手内,对唐僧道:“这不是我的金线?”
近侍宦官在旁道:“长老且休讲口,请入宫中诊视去来。”行者别了唐僧,随着近侍
入宫看病。正是那:
心有秘方能治国,内藏妙诀注长生。

毕竟这去不知看出甚么病来,用甚么药品。欲知端的,且听下回分解。