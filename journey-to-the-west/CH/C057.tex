\chapter{真行者落伽山诉苦~假猴王水帘洞誊文}

却说孙大圣恼恼闷闷,起在空中,欲待回花果山水帘洞,恐本洞小妖见笑,笑
我出乎尔反乎尔,不是个大丈夫之器;欲待要投奔天宫,又恐天宫内不容久住;欲
待要投海岛,却又羞见那三岛诸仙;欲待要奔龙宫,又不伏气求告龙王;真个是无
依无倚,苦自忖量道:“罢,罢,罢!我还去见我师父,还是正果。”

遂按下云头,径至三藏马前侍立道:“师父,恕弟子这遭!向后再不敢行凶,受
师父教诲。千万还得我保你西天去也。”唐僧见了,更不答应,兜住马,即念紧箍
儿咒。颠来倒去,又念有二十余遍,把大圣咒倒在地,箍儿陷在肉里有一寸来深浅,
方才住口道:“你不回去,又来缠我怎的?”行者只教:“莫念,莫念!我是有处过
日子的,只怕你无我去不得西天。”三藏发怒道:“你这猢狲杀生害命,连累了我多
少,如今实不要你了!我去得去不得,不干你事。快走,快走!迟了些儿,我又念真
言。这番决不住口,把你脑浆都勒出来哩!”大圣疼痛难忍,见师父更不回心,没
奈何,只得又驾筋斗云,起在空中。忽然省悟道:“这和尚负了我心,我且向普陀
崖告诉观音菩萨去来。”

好大圣,拨回筋斗,那消一个时辰,早至南洋大海。住下祥光,直至落伽山上,
撞入紫竹林中,忽见木叉行者迎面作礼道:“大圣何往?”行者道:“要见菩萨。”
木叉即引行者至潮音洞口,又见善财童子作礼道:“大圣何来?”行者道:“有事要
告菩萨。”善财听见一个“告”字,笑道:“好刁嘴猴儿!还像当时我拿住唐僧被你
欺哩!我菩萨是个大慈大悲,大愿大乘,救苦救难,无边无量的圣善菩萨,有甚不
是处,你要告他?”行者满怀闷气,一闻此言,心中怒发,咄的一声,把善财童子
喝了个倒退,道:“这个背义忘恩的小畜生,着实愚鲁!你那时节作怪成精,我请菩
萨收了你,皈正迦持,如今得这等极乐长生,自在逍遥,与天同寿,还不拜谢老孙,
转倒这般侮慢!我是有事来告求菩萨,却怎么说我刁嘴要告菩萨?”善财陪笑道:“还
是个急猴子。我与你作笑耍子,你怎么就变脸了?”

正讲处,只见白鹦哥飞来飞去,知是菩萨呼唤,木叉与善财,遂向前引导,至
宝莲台下。行者望见菩萨,倒身下拜,止不住泪如泉涌,放声大哭。菩萨教木叉与
善财扶起道:“悟空,有甚伤感之事,明明说来。莫哭,莫哭,我与你救苦消灾也。”
行者垂泪再拜道:“当年弟子为人,曾受那个气来?自蒙菩萨解脱天灾,秉教沙门,
保护唐僧往西天拜佛求经,我弟子舍身拚命,救解他的魔障,就如老虎口里夺脆骨,
蛟龙背上揭生鳞。只指望归真正果,洗业除邪,怎知那长老背义忘恩,直迷了一片
善缘,更不察皂白之苦!”菩萨道:“且说那皂白原因来我听。”行者即将那打杀草
寇前后始终,细陈了一遍。却说唐僧因他打死多人,心生怨恨,不分皂白,遂念紧
箍儿咒,赶他几次。上天无路,入地无门,特来告诉菩萨。菩萨道:“唐三藏奉旨
投西,一心要秉善为僧,决不轻伤性命。似你有无量神通,何苦打死许多草寇!草
寇虽是不良,到底是个人身,不该打死。比那妖禽怪兽、鬼魅精魔不同。那个打死,
是你的功绩;这人身打死,还是你的不仁。但祛退散,自然救了你师父。据我公论,
还是你的不善。”

行者噙泪叩头道:“纵是弟子不善,也当将功折罪,不该这般逐我。万望菩萨,
舍大慈悲,将松箍儿咒念念,褪下金箍,交还与你,放我仍往水帘洞逃生去罢!”
菩萨笑道:“紧箍儿咒,本是如来传我的。当年差我上东土寻取经人,赐我三件宝
贝,乃是锦袈裟、九环锡杖、金紧禁三个箍儿。秘授与咒语三篇,却无甚么松箍
儿咒。”行者道:“既如此,我告辞菩萨去也。”菩萨道:“你辞我往那里去?”行者
道:“我上西天,拜告如来,求念松箍儿咒去也。”菩萨道:“你且住,我与你看看
祥晦如何。”行者道:“不消看,只这样不祥也够了。”菩萨道:“我不看你,看唐僧
的祥晦。”

好菩萨,端坐莲台,运心三界,慧眼遥观,遍周宇宙,霍时间开口道:“悟空,
你那师父顷刻之际,就有伤身之难,不久便来寻你。你只在此处,待我与唐僧说,
教他还同你去取经,了成正果。”孙大圣只得皈依,不敢造次,侍立于宝莲台下不
题。

却说唐长老自赶回行者,教八戒引马,沙僧挑担,连马四口,奔西走不上五十
里远近,三藏勒马道:“徒弟,自五更时出了村舍,又被那弼马温着了气恼,这半
日饥又饥,渴又渴,那个去化些斋来我吃?”八戒道:“师父且请下马,等我看可
有邻近的庄村,化斋去也。”三藏闻言,滚下马来。呆子纵起云头,半空中仔细观
看,一望尽是山岭,莫想有个人家。八戒按下云来,对三藏道:“却是没处化斋。
一望之间,全无庄舍。”三藏道:“既无化斋之处,且得些水来解渴也可。”八戒道:
“等我去南山涧下取些水来。”沙僧即取钵盂,递与八戒。八戒托着钵盂,驾起云
雾而去。那长老坐在路旁,等够多时,不见回来,可怜口干舌苦难熬。有诗为证。
诗曰:
保神养气谓之精,情性原来一禀形。
心乱神昏诸病作,形衰精败道元倾。
三花不就空劳碌,四大萧条枉费争。
土木无功金水绝,法身疏懒几时成!
沙僧在旁,见三藏饥渴难忍,八戒又取水不来,只得稳了行囊,拴牢了白马道:“师
父,你自在着,等我去催水来。”长老含泪无言,但点头相答。沙僧急驾云光,也
向南山而去。

那师父独炼自熬,困苦太甚。正在怆徨之际,忽听得一声响亮,唬得长老欠身
看处,原来是孙行者跪在路旁,双手捧着一个磁杯道:“师父,没有老孙,你连水
也不能够哩。这一杯好凉水,你且吃口水解渴,待我再去化斋。”长老道:“我不吃
你的水!立地渴死,我当任命!不要你了,你去罢!”行者道:“无我你去不得西天也。”
三藏道:“去得去不得,不干你事!泼猢狲!只管来缠我做甚!”那行者变了脸,发怒
生嗔,喝骂长老道:“你这个狠心的泼秃,十分贱我!”轮铁棒,丢了磁杯,望长老
脊背上砑了一下。那长老昏晕在地,不能言语,被他把两个青毡包袱,提在手中,
驾筋斗云,不知去向。

却说八戒托着钵盂,只奔山南坡下,忽见山凹之间,有一座草舍人家。原来在
先看时,被山高遮住,未曾见得;今来到边前,方知是个人家。呆子暗想道:“我
若是这等丑嘴脸,决然怕我,枉劳神思,断然化不得斋饭。须是变好,须是变好!”
好呆子,捻着诀,念个咒,把身摇了七八摇,变作一个食痨病黄胖和尚,口里哼哼
的,埃近门前,叫道:“施主,厨中有剩饭,路上有饥人。贫僧是东土来,往
西天取经的。我师父在路饥渴了,家中有锅巴冷饭,千万化些儿救口。”原来那家
子男人不在,都去插秧种谷去了;只有两个女人在家,正才煮了午饭,盛起两盆,
却收拾送下田,锅里还有些饭与锅巴,未曾盛了。那女人见他这等病容,却又说东
土往四天去的话,只恐他是病昏了胡说;又怕跌倒,死在门首。只得哄哄翕翕,将
些剩饭锅巴,满满的与了一钵。呆子拿转来,现了本象,径回旧路。

正走间,听得有人叫“八戒”。八戒抬头看时,却是沙僧站在山崖上喊道:“这
里来!这里来!”及下崖,迎至面前道:“这涧里好清水不舀,你往那里去的?”八
戒笑道:“我到这里,见山凹子有个人家,我去化了这一钵干饭来了。”沙僧道:“饭
也用着,只是师父渴得紧了,怎得水去?”八戒道:“要水也容易;你将衣襟来兜
着这饭,等我使钵盂去舀水。”

二人欢欢喜喜,回至路上,只见三藏面磕地,倒在尘埃;白马撒缰,在路旁长
嘶跑跳;行李担不见踪影。慌得八戒跌脚捶胸,大呼小叫道:“不消讲,不消讲,
这还是孙行者赶走的余党,来此打杀师父,抢了行李去了!”沙僧道:“且去把马拴
住!”只叫:“怎么好,怎么好!这诚所谓半途而废,中道而止也!”叫一声:“师父!”
满眼抛珠,伤心痛哭。八戒道:“兄弟,且休哭。如今事已到此,取经之事,且莫
说了。你看着师父的尸灵,等我把马骑到那个府州县乡村店集卖几两银子,买口棺
木,把师父埋了,我两个各寻道路散伙。”

沙僧实不忍舍,将唐僧扳转身体,以脸温脸,哭一声:“苦命的师父!”只见那
长老口鼻中吐出热气,胸前温暖。连叫:“八戒,你来,师父未伤命哩!”那呆子才
近前扶起。长老苏醒,呻吟一会,骂道:“好泼猢狲,打杀我也!”沙僧、八戒问道:
“是那个猢狲?”长老不言,只是叹息。却讨水吃了几口,才说:“徒弟,你们刚
去,那悟空更来缠我。是我坚执不收,他遂将我打了一棒,青毡包袱都抢去了。”
八戒听说,咬响口中牙,发起心头火道:“叵耐这泼猴子,怎敢这般无礼!”教沙僧
道:“你伏侍师父,等我到他家讨包袱去!”沙僧道:“你且休发怒。我们扶师父到
那山凹人家化些热茶汤,将先化的饭热热,调理师父,再去寻他。”

八戒依言,把师父扶上马,拿着钵盂,兜着冷饭,直至那家门首。只见那家止
有个老婆子在家,忽见他们,慌忙躲过。沙僧合掌道:“老母亲,我等是东土唐朝
差往西天去者。师父有些不快,特拜府上,化口热茶汤,与他吃饭。”那妈妈道:“适
才有个食痨病和尚,说是东土差来的,已化斋去了,又有个甚么东土的。我没人在
家,请别转转。”长老闻言,扶着八戒,下马躬身道:“老婆婆,我弟子有三个徒弟,
合意同心,保护我上天竺国大雷音拜佛求经。只因我大徒弟,唤孙悟空,一生凶恶,
不遵善道,是我逐回,不期他暗暗走来,着我背上打了一棒,将我行囊衣钵抢去。
如今要着一个徒弟寻他取讨,因在那空路上不是坐处,特来老婆婆府上权安息一时。
待讨将行李来就行,决不敢久住。”那妈妈道:“刚才一个食痨病黄胖和尚,他化斋
去了,也说是东土往西天去的,怎么又有一起?”八戒忍不住笑道:“就是我。因
我生得嘴长耳大,恐你家害怕,不肯与斋,故变作那等模样。你不信,我兄弟衣兜
里不是你家锅巴饭?”那妈妈认得果是他与的饭,遂不拒他,留他们坐了。却烧了
一罐热茶,递与沙僧泡饭。

沙僧即将冷饭泡了,递与师父。师父吃了几口,定性多时道:“那个去讨行李?”
八戒道:“我前年因师父赶他回去,我曾寻他一次,认得他花果山水帘洞。等我去,
等我去!”长老道:“你去不得。那猢狲原与你不和,你又说话粗鲁,或一言两句之
间,有些差池,他就要打你。着悟净去罢。”沙僧应承道:“我去,我去。”长老又
吩咐沙僧道:“你到那里,须看个头势。他若肯与你包袱,你就假谢谢拿来;若不
肯,切莫与他争竞,径至南海菩萨处,将此情告诉,请菩萨去问他要。”沙僧一一
听从。向八戒道:“我今寻他去,你千万莫,好生供养师父。这人家亦不可撒
泼,恐他不肯供饭。我去就回。”八戒点头道:“我理会得。但你去讨得讨不得,次
早回来,不要弄做‘尖担担柴两头脱’也。”沙僧遂捻了诀,驾起云光,直奔东胜
神洲而去。真个是:
身在神飞不守舍,有炉无火怎烧丹。
黄婆别主求金老,木母延师奈病颜。
此去不知何日返,这回难量几时还。
五行生克情无顺,只待心猿复进关。

那沙僧在半空里,行经三昼夜,方到了东洋大海。忽闻波浪之声,低头观看,
真个是:

黑雾涨天阴气盛,沧溟衔日晓光寒。
他也无心观玩,望仙山渡过瀛洲,向东方直抵花果山界。乘海风,踏水势,又多时,
却望见高峰排戟,峻壁悬屏。即至峰头,按云找路下山,寻水帘洞。步近前,只听
得一派喧声,见那山中无数猴精,滔滔乱嚷。沙僧又近前仔细再看,原来是孙行者
高坐石台之上,双手扯着一张纸,朗朗的念道:

“东土大唐王皇帝李,驾前敕命御弟圣僧陈玄奘法师,上西方天竺国娑婆灵山
大雷音寺专拜如来佛祖求经。朕因促病侵身,魂游地府,幸有阳数臻长,感冥君放
送回生,广陈善会,修建度亡道场。盛蒙救苦救难观世音菩萨金身出现,指示西方
有佛有经,可度幽亡超脱,特着法师玄奘,远历千山,询求经偈。倘过西邦诸国,
不灭善缘,照牒施行。

大唐贞观一十三年秋吉日御前文牒。

自别大国以来,经度诸邦,中途收得大徒弟孙悟空行者,二徒弟猪悟能八戒,
三徒弟沙悟净和尚。”
念了从头又念。沙僧听得是通关文牒,止不住近前厉声高叫:“师兄,师父的关文
你念他怎的?”那行者闻言,急抬头,不认得是沙僧,叫:“拿来!拿来!”众猴一
齐围绕,把沙僧拖拖扯扯,拿近前来,喝道:“你是何人,擅敢近吾仙洞?”沙僧
见他变了脸,不肯相认,只得朝上行礼道:“上告师兄。前者实是师父性暴,错怪
了师兄,把师兄咒了几遍,逐赶回家。一则弟等未曾劝解,二来又为师父饥渴去寻
水化斋。不意师兄好意复来,又怪师父执法不留,遂把师父打倒,昏晕在地,将行
李抢去。后救转师父,特来拜兄。若不恨师父,还念昔日解脱之恩,同小弟将行李
回见师父,共上西天,了此正果。倘怨恨之深,不肯同去,千万把包袱赐弟,兄在
深山,乐桑榆晚景,亦诚两全其美也。”

行者闻言,呵呵冷笑道:“贤弟,此论甚不合我意。我打唐僧,抢行李,不因
我不上西方,亦不因我爱居此地;我今熟读了牒文,我自己上西方拜佛求经,送上
东土,我独成功,教那南赡部洲人立我为祖,万代传名也。”沙僧笑道:“师兄言之
欠当。自来没个‘孙行者取经’之说。我佛如来造下三藏真经,原着观音菩萨向东
土寻取经人求经,要我们苦历千山,询求诸国,保护那取经人。菩萨曾言:取经人
乃如来门生,号曰金蝉长老。只因他不听佛祖谈经,贬下灵山,转生东土,教他果
正西方,复修大道。遇路上该有这般魔障,解脱我等三人,与他做护法。兄若不得
唐僧去,那个佛祖肯传经与你!却不是空劳一场神思也?”那行者道:“贤弟,你原
来蒙懂,但知其一,不知其二。谅你说你有唐僧,同我保护,我就没有唐僧?我这
里另选个有道的真僧在此,老孙独力扶持,有何不可!已选明日大走起身去矣。你
不信,待我请来你看。”叫“小的们,快请老师父出来!”果跑进去,牵出一匹白马,
请出一个唐三藏,跟着一个八戒,挑着行李;一个沙僧,拿着锡杖。

这沙僧见了大怒道:“我老沙行不更名,坐不改姓,那里又有一个沙和尚!不要
无礼,吃我一杖!”好沙僧,双手举降妖杖,把一个“假沙僧”劈头一下打死,原
来这是一个猴精。那行者恼了,轮金箍棒,帅众猴,把沙僧围了。沙僧东冲西撞,
打出路口,纵云雾逃生道:“这泼猴如此惫懒,我告菩萨去来!”那行者见沙僧打死
一个猴精,把沙和尚逼得走了,他也不来追赶。回洞教小的们把打死的妖尸拖在一
边,剥了皮,取肉煎炒,将椰子酒、葡萄酒,同众猴都吃了。另选一个会变化的妖
猴,还变一个沙和尚,从新教道,要上西方不题。

沙僧一驾云离了东海,行经一昼夜,到了南海。正行时,早见落伽山不远,急
至前,低停云雾观看。好去处!果然是:

包乾之奥,括坤之区。会百川而浴日滔星,归众流而生风漾月。潮发腾凌大鲲
化,波翻浩荡巨鳌游。水通西北海,浪合正东洋。四海相连同地脉,仙方洲岛各仙
宫。休言满地蓬莱,且看普陀云洞。好景致!山头霞彩壮元精,岩下祥风漾月晶。
紫竹林中飞孔雀,绿杨枝上语灵鹦。琪花瑶草年年秀,宝树金莲岁岁生。白鹤几番
朝顶上,素鸾数次到山亭。游鱼也解修
真性,跃浪穿波听讲经。
沙僧徐步落伽山,玩看仙境。只见木叉行者当面相迎道:“沙悟净,你不保唐僧取
经,却来此何干?”沙僧作礼毕,道:“有一事特来朝见菩萨,烦为引见引见。”木
叉情知是寻行者,更不题起,即先进去对菩萨道:“外有唐僧的小徒弟沙悟净朝拜。”
孙行者在台下听见,笑道:“这定是唐僧有难,沙僧来请菩萨的。”菩萨即命木叉门
外叫进。这沙僧倒身下拜。拜罢,抬头正欲告诉前事,忽见孙行者站在旁边,等不
得说话,就掣降妖杖望行者劈脸便打。这行者更不回手,彻身躲过。沙僧口里乱骂
道:“我把你个犯十恶造反的泼猴!你又来影瞒菩萨哩!”菩萨喝道:“悟净不要动
手。有甚事先与我说。”

沙僧收了宝杖,再拜台下,气冲冲的对菩萨道:“这猴一路行凶,不可数计。
前日在山坡下打杀两个剪路的强人,师父怪他;不期晚间就宿在贼窝主家里,又把
一伙贼人尽情打死,又血淋淋提一个人头来与师父看。师父唬得跌下马来,骂了他
几句,赶他回来。分别之后,师父饥渴太甚,教八戒去寻水。久等不来,又教我去
寻他。不期孙行者见我二人不在,复回来把师父打一铁棍,将两个青毡包袱抢去。
我等回来,将师父救醒,特来他水帘洞寻他讨包袱,不想他变了脸,不肯认我,将
师父关文念了又念。我问他念了做甚,他说不保唐僧,他要自上西天取经,送上东
土,算他的功果,立他为祖,万古传扬。我又说:‘没唐僧,那肯传经与你?’他
说他选了一个有道的真僧。及请出,果是一匹白马,一个唐僧,后跟着八戒、沙僧。
我道:‘我便是沙和尚,那里又有个沙和尚?’是我赶上前,打了他一宝杖,原来
是个猴精。他就帅众拿我,是我特来告请菩萨。不知他会使筋斗云,预先到此处;
又不知他将甚巧语花言,影瞒菩萨也。”菩萨道:“悟净,不要赖人。悟空到此,今
已四日。我更不曾放他回去,他那里有另请唐僧,自去取经之意?”沙僧道:“见
如今水帘洞有一个孙行者,怎敢欺诳?”菩萨道:“既如此,你休发急,教悟空与
你同去花果山看看。是真难灭,是假易除。到那里自见分晓。”这大圣闻言,即与
沙僧辞了菩萨。这一去,到那:
花果山前分皂白,水帘洞口辨真邪。

毕竟不知如何分辨,且听下回分解。