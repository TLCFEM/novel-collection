\chapter{神狂诛草寇~道昧放心猿}

诗曰:
灵台无物谓之清,寂寂全无一念生。
猿马牢收休放荡,精神谨慎莫峥嵘。
除六贼,悟三乘,万缘都罢自分明。
色邪永灭超真界,坐享西方极乐城。

话说唐三藏咬钉嚼铁,以死命留得一个不坏之身;感蒙行者等打死蝎子精,救
出琵琶洞。一路无词,又早是朱明时节。但见那:
熏风时送野兰香,濯雨才晴新竹凉。
艾叶满山无客采,蒲花盈涧自争芳。
海榴娇艳游蜂喜,溪柳阴浓黄雀狂。
长路那能包角黍,龙舟应吊汨罗江。
他师徒们行赏端阳之景,虚度中天之节,忽又见一座高山阻路。长老勒马回头叫道:
“悟空,前面有山,恐又生妖怪,是必谨防。”行者等道:“师父放心。我等皈命投
诚,怕甚妖怪!”长老闻言甚喜。加鞭催骏马,放辔趱蛟龙。

须臾,上了山崖,举头观看,真个是:

顶巅松柏接云青,石壁荆榛挂野藤。万丈崔巍,千层悬削:万丈崔巍峰岭峻,
千层悬削壑崖深。苍苔碧藓铺阴石,古
桧高槐结大林。林深处,听幽禽,巧声实堪吟。涧内水流如泻玉,路旁花落似
堆金。山势恶,不堪行,十步全无半步平。狐狸糜鹿成双遇,白鹿玄猿作对迎。忽
闻虎啸惊人胆,鹤鸣振耳透天庭。黄梅红杏堪供食,野草闲花不识名。
四众进山,缓行良久,过了山头。下西坡,乃是一段平阳之地。猪八戒卖弄精神,
教沙和尚挑着担子,他双手举钯,上前赶马。那马更不惧他,凭那呆子嗒笞笞的赶,
只是缓行不紧。行者道:“兄弟,你赶他怎的?让他慢慢走罢了。”八戒道:“天色将
晚,自上山行了这一日,肚里饿了,大家走动些,寻个人家化些斋吃。”行者闻言
道:“既如此,等我教他快走。”把金箍棒幌一幌,喝了一声,那马溜了缰,如飞似
箭,顺平路往前去了。你说马不怕八戒,只怕行者何也?行者五百年前曾受玉帝封
在大罗天御马监养马,官名“弼马温”,故此传留至今,是马皆惧猴子。那长老挽
不住缰口,只扳紧着鞍鞒,让他放了一路辔头,有二十里向开田地,方才缓步而行。

正走处,忽听得一棒锣声,路两边闪出三十多人,一个个枪刀棍棒,拦住路口
道:“和尚!那里走!”唬得个唐僧战兢兢,坐不稳,跌下马来,蹲在路旁草料里,
只叫:“大王饶命!大王饶命!”那为头的两个大汉道:“不打你,只是有盘缠留下。”
长老方才省悟,知他是伙强人,却欠身抬头观看。但见他:

一个青脸獠牙欺太岁,一个暴睛圜眼赛丧门。鬓边红发如飘火,颔下黄须似插
针。他两个头戴虎皮花磕脑,腰系貂裘彩战裙。一个手中执着狼牙棒,一个肩上横
担挞藤。果然不亚巴山虎,真个犹如出水龙。
三藏见他这般凶恶,只得走起来,合掌当胸道:“大王,贫僧是东土唐王差往西天
取经者。自别了长安,年深日久,就有些盘缠也使尽了。出家人专以乞化为由,那
得个财帛!万望大王方便方便,让贫僧过去罢!”那两个贼帅众向前道:“我们在这
里起一片虎心,截住要路,专要些财帛,甚么方便方便!你果无财帛,快早脱下衣
服,留下白马,放你过去!”三藏道:“阿弥陀佛!贫僧这件衣服,是东家化布,西
家化针,零零碎碎化来的。你若剥去,可不害杀我也?只是这世里做得好汉,那世
里变畜生哩!”

那贼闻言大怒,掣大棍,上前就打。这长老口内不言,心中暗想道:“可怜!你
只说你的棍子,还不知我徒弟的棍子哩!”那贼那容分说,举着棒,没头没脸的打
来。长老一生不会说慌,遇着这急难处,没奈何,只得打个诳语道:“二位大王,
且莫动手。我有个小徒弟,在后面就到。他身上有几两银子,把与你罢。”那贼道:
“这和尚是也吃不得亏,且捆起来。”众娄罗一齐下手,把一条绳捆了,高高吊在
树上。

却说三个撞祸精,随后赶来。八戒呵呵大笑道:“师父去得好快,不知在那里
等我们哩。”忽见长老在树上,他又说:“你看师父。等便罢了,却又有这般心肠,
爬上树去,扯着藤儿打秋千耍子哩!”行者见了道:“呆子,莫乱谈。师父吊在那里
不是?你两个慢来,等我去看看。”好大圣,急登高坡细看,认得是伙强人。心中暗
喜道:“造化,造化,买卖上门了!”即转步,摇身一变,变做个干干净净的小和尚,
穿一领缁衣,年纪只有二八,肩上背着一个蓝布包袱。拽开步,来到前边,叫道:
“师父,这是怎么说话?这都是些甚么歹人?”三藏道:“徒弟呀,还不救我一救,
还问甚的?”行者道:“是干甚勾当的?”三藏道:“这一伙拦路的,把我拦住,要
买路钱。因身边无物,遂把我吊在这里,只等你来计较计较。不然,把这匹马送与
他罢。”行者闻言笑道:“师父不济。天下也有和尚,似你这样皮松的却少。唐太宗
差你往西天见佛,谁教你把这龙马送人?”三藏道:“徒弟呀,似这等吊起来,打
着要,怎生是好?”行者道:“你怎么与他说来?”三藏道:“他打的我急了,没奈
何,把你供出来也。”行者道:“师父,你好没搭撒。你供我怎的?”三藏道:“我
说你身边有些盘缠,且教道莫打我,是一时救难的话儿。”行者道:“好,好,好!
承你抬举。正是这样供。若肯一个月供得七八十遭,老孙越有买卖。”

那伙贼见行者与他师父讲话,撒开势,围将上来道:“小和尚,你师父说你腰
里有盘缠,趁早拿出来,饶你们性命!若道半个‘不’字,就都送了你的残生!”行
者放下包袱道:“列位长官,不要嚷。盘缠有些在此包袱,不多,只有马蹄金二十
来锭,粉面银二三十锭,散碎的未曾见数。要时就连包儿拿去,切莫打我师父。古
书云:‘德者,本也;财者,末也。’此是末事。我等出家人,自有化处;若遇着个
斋僧的长者,衬钱也有,衣服也有,能用几何?只望放下我师父来,我就一并奉承。”
那伙贼闻言,都甚欢喜道:“这老和尚悭吝,这小和尚倒还慷慨。”教:“放下来。”
那长老得了性命,跳上马,顾不得行者,操着鞭,一直跑回旧路。

行者忙叫道:“走错路了。”提着包袱,就要追去。那伙贼拦住道:“那里走?将
盘缠留下,免得动刑!”行者笑道:“说开,盘缠须三分分之。”那贼头道:“这小和
尚忒乖,就要瞒着他师父留起些儿。也罢,拿出来看。若多时,也分些与你背地里
买果子吃。”行者道:“哥呀,不是这等说。我那里有甚盘缠?说你两个打劫别人的
金银,是必分些与我。”那贼闻言大怒,骂道:“这和尚不知死活!你倒不肯与我,
返问我要!不要走,看打!”轮起一条挞藤棍,照行者光头上打了七八下。行者只
当不知,且满面陪笑道:“哥呀,若是这等打,就打到来年打罢春,也是不当真的。”
那贼大惊道:“这和尚好硬头!”行者笑道:“不敢,不敢,承过奖了。也将就看得
过。”那贼那容分说,两三个一齐乱打。行者道:“列位息怒,等我拿出来。”

好大圣,耳中摸一摸,拔出一个绣花针儿道:“列位,我出家人,果然不曾带
得盘缠,只这个针儿送你罢。”那贼道:“晦气呀,把一个富贵和尚放了,却拿住这
个穷秃驴!你好道会做裁缝?我要针做甚的?”行者听说不要,就拈在手中,幌了一
幌,变作碗来粗细的一条棍子。那贼害怕道:“这和尚生得小,倒会弄术法儿。”行
者将棍子插在地下道:“列位拿得动,就送你罢。”两个贼上前抢夺,可怜就如蜻蜓
撼石柱,莫想弄动半分毫。这条棍本是如意金箍棒,天秤称的,一万三千五百斤重,
那伙贼怎么知得。大圣走上前,轻轻的拿起,丢一个蟒翻身拗步势,指着强人道:
“你都造化低,遇着我老孙了!”那贼上前来,又打了五六十下。行者笑道:“你也
打得手困了,且让老孙打一棒儿,却休当真。”你看他展开棍子,幌一幌,有井栏
粗细,七八丈长短;荡的一棍,把一个打倒在地,嘴唇土,再不做声。那一个开
言骂道:“这秃厮老大无礼!盘缠没有,转伤我一个人!”行者笑道:“且消停,且消
停!待我一个个打来,一发教你断了根罢!”“荡”的又一棍,把第二个又打死了,
唬得那众娄罗撇枪弃棍,四路逃生而走。

却说唐僧骑着马,往东正跑,八戒、沙僧拦住道:“师父往那里去?错走路了。”
长老兜马道:“徒弟啊,趁早去与你师兄说,教他棍下留情,莫要打杀那些强盗。”
八戒道:“师父住下,等我去来。”呆子一路跑到前边,厉声高叫道:“哥哥,师父
教你莫打人哩。”行者道:“兄弟,那曾打人?”八戒道:“那强盗往那里去了?”
行者道:“别个都散了,只是两个头儿在这里睡觉哩。”八戒笑道:“你两个遭瘟的,
好道是熬了夜,这般辛苦,不往别处睡,却睡在此处!”呆子行到身边,看看道:“倒
与我是一起的,干净张着口睡,淌出些粘涎来了。”行者道:“是老孙一棍子打出豆
腐来了。”八戒道:“人头上又有豆腐?”行者道:“打出脑子来了!”

八戒听说打出脑子来,慌忙跑转去,对唐僧道:“散了伙也!”三藏道:“善哉,
善哉!往那条路上去了?”八戒道:“打也打得直了脚,又会往那里去走哩!”三藏
道:“你怎么说散伙?”八戒道:“打杀了,不是散伙是甚的?”三藏问:“打的怎
么模样?”八戒道:“头上打了两个大窟窿。”三藏教:“解开包,取几文衬钱,快
去那里讨两个膏药与他两个贴贴。”八戒笑道:“师父好没正经。膏药只好贴得活人
的疮肿,那里好贴得死人的窟窿?”三藏道:“真打死了?”就恼起来,口里不住
的絮絮叨叨,猢狲长,猴子短,兜转马,与沙僧、八戒至死人前,见那血淋淋的,
倒卧山坡之下。

这长老甚不忍见,即着八戒:“快使钉钯,筑个坑子埋了,我与他念卷《倒头
经》。”八戒道:“师父左使了人也。行者打杀人,还该教他去烧埋,怎么教老猪做
土工?”行者被师父骂恼了,喝着八戒道:“泼懒夯货!趁早儿去埋!迟了些儿,就
是一棍!”呆子慌了,往山坡下筑了有三尺深,下面都是石脚石根,扛住钯齿;呆
子丢了钯,便把嘴拱;拱到软处,一嘴有二尺五,两嘴有五尺深,把两个贼尸埋了,
盘作一个坟堆。三藏叫:“悟空,取香烛来,待我祷祝,好念经。”行者努着嘴道:
“好不知趣!这半山之中,前不巴村,后不着店,那讨香烛?就有钱也无处去买。”
三藏恨恨的道:“猴头过去!等我撮土焚香祷告。”这是三藏离鞍悲野冢,圣僧善念
祝荒坟。祝云:

拜惟好汉,听祷原因:念我弟子,东土唐人。奉太宗皇帝旨意,上西方求取经
文。适来此地,逢尔多人,不知是何府何州何县,都在此山内结党成群。我以好话,
哀告殷勤。尔等不听,返善生嗔。却遭行者,棍下伤身。切念尸骸暴露,吾随掩土
盘坟。折青竹,为香烛,无光彩,有心勤;取顽石,作施食,无滋味,有诚真。你
到森罗殿下兴词,倒树寻根,他姓孙,我姓陈,各居异姓。冤有头,债有主,切莫
告我取经僧人。
八戒笑道:“师父推了干净。他打时却也没有我们两个。”三藏真个又撮土祷告道:
“好汉告状,只告行者,也不干八戒、沙僧之事。”大圣闻言,忍不住笑道:“师父,
你老人家忒没情义。为你取经,我费了多少殷勤劳苦,如今打死这两个毛贼,你倒
教他去告老孙。虽是我动手打,却也只是为你。你不往西天取经,我不与你做徒弟,
怎么会来这里,会打杀人!索性等我祝他一祝。”攥着铁棒,望那坟上捣了三下,道:
“遭瘟的强盗,你听着!我被你前七八棍,后七八棍,打得我不疼不痒的,触恼了
性子,一差二误,将你打死了,尽你到那里去告,我老孙实是不怕:玉帝认得我,
天王随得我;二十八宿惧我,九曜星官怕我;府县城隍跪我,东岳天齐怖我;十代
阎君曾与我为仆从,五路猖神曾与我当后生:不论三界五司,十方诸宰,都与我情
深面熟,随你那里去告!”三藏见说出这般恶话,却又心惊道:“徒弟呀,我这祷祝
是教你体好生之德,为良善之人;你怎么就认真起来?”行者道:“师父,这不是
好耍子的勾当。且和你赶早寻宿去。”那长老只得怀嗔上马。

孙大圣有不睦之心,八戒、沙僧亦有嫉妒之意,师徒都面是背非。依大路向西
正走,忽见路北下有一座庄院。三藏用鞭指定道:“我们到那里借宿去。”八戒道:
“正是。”遂行至庄舍边下马。看时,却也好个住场。但见:

野花盈径,杂树遮扉。远岸流山水,平畦种麦葵。蒹葭露润轻鸥宿,杨柳风微
倦鸟栖。青柏间松争翠碧,红蓬映蓼斗芳菲。村犬吠,晚鸡啼,牛羊食饱牧童归。
爨烟结雾黄粱熟,正是山家入暮时。

长老向前,忽见那村舍门里走出一个老者,即与相见,道了问讯。那老者问道:
“僧家从那里来?”三藏道:“贫僧乃东土大唐钦差往西天求经者。适路过宝方,
天色将晚,特来檀府告宿一宵。”老者笑道:“你贵处到我这里,程途迢递,怎么涉
水登山,独自得此?”三藏道:“贫僧还有三个徒弟同来。”老者问:“高徒何在?”
三藏用手指道:“那大路旁立的便是。”老者猛抬头,看见他们面貌丑陋,急回身往
里就走;被三藏扯住道:“老施主,千万慈悲,告借一宿!”老者战兢兢钳口难言,
摇着头,摆着手道:“不,不,不,不像人模样!是,是,是几个妖精!”三藏陪笑
道:“施主切休恐惧。我徒弟生得是这等相貌,不是妖精。”老者道:“爷爷呀,一
个夜叉,一个马面,一个雷公!”行者闻言,厉声高叫道:“雷公是我孙子,夜叉是
我重孙,马面是我玄孙哩!”那老者听见,魄散魂飞,面容失色,只要进去。三藏
搀住他,同到草堂,陪笑道:“老施主,不要怕他。他都是这等粗鲁,不会说话。”

正劝解处,只见后面走出一个婆婆,携着五六岁的一个小孩儿,道:“爷爷,
为何这般惊恐?”老者才叫:“妈妈,看茶来。”那婆婆真个丢了孩儿,入里面捧出
二钟茶来。茶罢,三藏却转下来,对婆婆作礼道:“贫僧是东土大唐差往西天取经
的。才到贵处,拜求尊府借宿,因是我三个徒弟貌丑,老家长见了虚惊也。”婆婆
道:“见貌丑的就这等虚惊,若见了老虎豺狼,却怎么好?”老者道:“妈妈呀,人
面丑陋还可,只是言语一发吓人。我说他像夜叉、马面、雷公,他吆喝道,雷公是
他孙子,夜叉是他重孙,马面是他玄孙。我听此言,故然悚惧。”唐僧道:“不是,
不是。像雷公的,是我大徒孙悟空。像马面的,是我二徒猪悟能。像夜叉的,是我
三徒沙悟净。他们虽是丑陋,却也秉教沙门,皈依善果,不是甚么恶魔毒怪,怕他
怎么!”

公婆两个,闻说他名号,皈正沙门之言,却才定性回惊,教:“请来,请来。”
长老出门叫来。又吩咐道:“适才这老者甚恶你等。今进去相见,切勿抗礼,各要
尊重些。”八戒道:“我俊秀,我斯文,不比师兄撒泼。”行者笑道:“不是嘴长耳大
脸丑,便也是一个好男子。”沙僧道:“莫争讲,这里不是那抓乖弄俏之处。且进去,
且进去!”

遂此把行囊、马匹,都到草堂上,齐同唱了个喏,坐定。那妈妈儿贤慧,即便
携转小儿,吩咐煮饭,安排一顿素斋,他师徒吃了。渐渐晚了,又掌起灯来,都在
草堂上闲叙。长老才问:“施主高姓?”老者道:“姓杨。”又问年纪,老者道:“七
十四岁。”又问:“几位令郎?”老者道:“止得一个。适才妈妈携的是小孙。”长老:
“请令郎相见拜揖。”老者道:“那厮不中拜。老拙命苦,养不着他,如今不在家了。”
三藏道:“何方生理?”老者点头而叹:“可怜,可怜!若肯何方生理,是吾之幸也!
那厮专生恶念,不务本等,专好打家截道,杀人放火!相交的都是些狐群狗党!自五
日之前出去,至今未回。”三藏闻说,不敢言喘,心中暗想道:“或者悟空打杀的就
是也。”长老神思不安,欠身道:“善哉,善哉!如此贤父母,何生恶逆儿!”行者近
前道:“老官儿,似这等不良之肖,奸盗邪淫之子,连累父母,要他何用!等我替你
寻他来打杀了罢。”老者道:“我待也要送了他,奈何再无以次人丁,纵是不才,一
定还留他与老汉掩土。”沙僧与八戒笑道:“师兄,莫管闲事,你我不是官府。他家
不肖,与我何干!且告施主,见赐一束草儿,在那厢打铺睡觉,天明走路。”老者即
起身,着沙僧到后园里拿两个稻草,教他们在园中草团瓢内安歇。行者牵了马,八
戒挑了行李,同长老俱到团瓢内安歇不题。

却说那伙贼内果有老杨的儿子。自天早在山前被行者打死两个贼首,他们都四
散逃生。约摸到四更时候,又结坐一伙,在门前打门。老者听得门响,即披衣道:
“妈妈,那厮们来也。”妈妈道:“既来,你去开门,放他来家。”老者方才开门,
只见那一伙贼都嚷道:“饿了,饿了!”这老杨的儿子忙入里面,叫起他妻来,打米
煮饭;却厨下无柴,往后园里拿柴到厨房里,问妻道:“后园里白马是那里的?”
其妻道:“是东土取经的和尚,昨晚至此借宿,公公婆婆管待他一顿晚斋,教他在
草团瓢内睡哩。”

那厮闻言,走出草堂,拍手打掌笑道:“兄弟们,造化!造化!冤家在我家里也!”
众贼道:“那个冤家?”那厮道:“却是打死我们头儿的和尚,来我家借宿,现睡在
草团瓢里。”众贼道:“却好,却好!拿住这些秃驴,一个个剁成肉酱,一则得那行
囊、白马,二来与我们头儿报仇!”那厮道:“且莫忙,你们且去磨刀。等我煮饭熟
了,大家吃饱些,一齐下手。”真个那些贼磨刀的磨刀,磨枪的磨枪。

那老儿听得此言,悄悄的走到后园,叫起唐僧四位道:“那厮领众来了。知得
汝等在此,意欲图害。我老拙念你远来,不忍伤害。快早收拾行李,我送你往后门
出去罢!”三藏听说,战兢兢的叩头谢了老者,即唤八戒牵马,沙僧挑担,行者拿
了九环锡杖。老者开后门,放他去了,依旧悄悄的来前睡下。

却说那厮们磨快了刀枪,吃饱了饭食,时已五更天气,一齐来到园中看处,却
不见了。即忙点灯着火。寻彀多时,四无踪迹,但见后门开着。都道:“从后门走
了!走了!”发一声喊,“赶将上拿来。”一个个如飞似箭,直赶到东方日出,却才望
见唐僧。

那长老忽听得喊声,回头观看,后面有二三十人,枪刀簇簇而来。便叫:“徒
弟啊,贼兵追至,怎生奈何!”行者道:“放心,放心,老孙了他去来!”三藏勒马
道:“悟空,切莫伤人,只吓退他便罢。”行者那肯听信,急掣棒回首相迎道:“列
位那里去?”众贼骂道:“秃厮无礼!还我大王的命来!”那厮们圈子阵把行者围在
中间,举枪刀乱砍乱搠。这大圣把金箍棒幌一幌,碗来粗细,把那伙贼打得星落云
散,汤着的就死,挽着的就亡;着的骨折,擦着的皮伤;乖些的跑脱几个,痴些
的都见阎王!

三藏在马上,见打倒许多人,慌的放马奔西。猪八戒与沙和尚,紧随鞭镫而去。
行者问那不死带伤的贼人道:“那个是那杨老儿的儿子?”那贼哼哼的告道:“爷
爷,那穿黄的是!”行者上前,夺过刀来,把个穿黄的割下头来,血淋淋提在手中,
收了铁棒,拽开云步,赶到唐僧马前,提着头道:“师父,这是杨老儿的逆子,被
老孙取将首级来也。”三藏见了,大惊失色,慌得跌下马来,骂道:“这泼猢狲唬杀
我也!快拿过,快拿过!”八戒上前,将人头一脚踢下路旁,使钉钯筑些土盖了。

沙僧放下担子,搀着唐僧道:“师父请起。”那长老在地下正了性,口中念起紧
箍儿咒来,把个行者勒得耳红面赤,眼胀头昏,在地下打滚,只教:“莫念!莫念!”
那长老念够有十余遍,还不住口。行者翻筋斗,竖蜻蜓,疼痛难禁,只叫:“师父
饶我罪罢!有话便说。莫念!莫念!”三藏却才住口道:“没话说,我不要你跟了,你
回去罢!”行者忍疼磕头道:“师父,怎的就赶我去耶?”三藏道:“你这泼猴,凶
恶太甚,不是个取经之人。昨日在山坡下,打死那两个贼头,我已怪你不仁。及晚
了到老者之家,蒙他赐斋借宿;又蒙他开后门放我等逃了性命;虽然他的儿子不肖,
与我无干,也不该就枭他首;况又杀死多人,坏了多少生命,伤了天地多少和气。
屡次劝你,更无一毫善念,要你何为!快走,快走!免得又念真言!”行者害怕,只
教:“莫念,莫念!我去也!”说声去,一路筋斗云,无影无踪,遂不见了。咦!这正
是:
心有凶狂丹不熟,神无定位道难成。

毕竟不知那大圣投向何方,且听下回分解。