\chapter{观音院僧谋宝贝~黑风山怪窃袈裟}

却说他师徒两个,策马前来,直至山门首观看,果然是一座寺院。但见那:

层层殿阁,迭迭廊房。三山门外,巍巍万道彩云遮;五福堂前,艳艳千条红雾
绕。两路松篁,一林桧柏:两路松篁,无年无纪自清幽;一林桧柏,有色有颜随傲
丽。又见那钟鼓楼高,浮屠塔峻。安禅僧定性,啼树鸟音闲。寂寞无尘真寂寞,清
虚有道果清虚。

诗曰:
上刹园隐翠窝,招提胜景赛娑婆。
果然净土人间少,天下名山僧占多。
长老下了马,行者歇了担,正欲进门,只见那门里走出一众僧来。你看他怎生模样:
头戴左笄帽,身穿无垢衣。
铜环双坠耳,绢带束腰围。
草履行来稳,木鱼手内提。
口中常作念,般若总皈依。
三藏见了,侍立门旁,道个问讯,那和尚连忙答礼。笑道:“失瞻。”问:“是那里
来的?请入方丈献茶。”三藏道:“我弟子乃东土钦差,上雷音寺拜佛求经。至此处
天色将晚,欲借上刹一宵。”那和尚道:“请进里坐,请进里坐。”三藏方唤行者牵
马进来。那和尚忽见行者相貌,有些害怕,便问:“那牵马的是个甚么东西?”三
藏道:“悄言,悄言!他的性急,若听见你说是甚么东西,他就恼了。他是我的徒弟。”
那和尚打了个寒噤,咬着指头道:“这般一个丑头怪脑的,好招他做徒弟!”三藏道:
“你看不出来哩,丑自丑,甚是有用。”

那和尚只得同三藏与行者进了山门。山门里,又见那正殿上书四个大字,是“观
音禅院”。三藏又大喜道:“弟子屡感菩萨圣恩,未及叩谢;今遇禅院,就如见菩萨
一般,甚好拜谢。”那和尚闻言,即命道人开了殿门,请三藏朝拜。那行者拴了马,
丢了行李,同三藏上殿。三藏展背舒身,铺胸纳地,望金像叩头。那和尚便去打鼓,
行者就去撞钟。三藏俯伏台前,倾心祷祝。祝拜已毕,那和尚住了鼓,行者还只管
撞钟不歇,或紧或慢,撞了许久。那道人道:“拜已毕了,还撞钟怎么?”行者方
丢了钟杵,笑道:“你那里晓得!我这是‘做一日和尚撞一日钟’的。”

此时却惊动那寺里大小僧人、上下房长老,听得钟声乱响,一齐拥出道:“那
个野人在这里乱敲钟鼓!”行者跳将出来,“咄”的一声道:“是你孙外公撞了耍子
的!”那些和尚一见了,唬得跌跌滚滚,都爬在地下道:“雷公爷爷!”行者道:“雷
公是我的重孙儿哩!起来,起来,不要怕,我们是东土大唐来的老爷。”众僧方才礼
拜;见了三藏,都才放心不怕。内有本寺院主请道:“老爷们到后方丈中奉茶。”遂
而解缰牵马,抬了行李,转过正殿,径入后房,序了坐次。

那院主献了茶,又安排斋供。天光尚早。三藏称谢未毕,只见那后面有两个小
童,搀着一个老僧出来。看他怎生打扮:

头上戴一顶毗卢方帽,猫睛石的宝顶光辉;身上穿一领锦绒褊衫,翡翠毛的金
边晃亮。一对僧鞋攒八宝,一根拄杖嵌云星。满面皱痕,好似骊山老母;一双昏眼,
却如东海龙君。口
不关风因齿落,腰驼背屈为筋挛。
众僧道:“师祖来了。”三藏躬身施礼迎接道:“老院主,弟子拜揖。”那老僧还了礼,
又各叙坐。老僧道:“适间小的们说,东土唐朝来的老爷,我才出来奉见。”三藏道:
“轻造宝山,不知好歹,恕罪!恕罪!”老僧道:“不敢!不敢!”因问:“老爷,东土
到此,有多少路程?”三藏道:“出长安边界,有五千余里;过两界山,收了一众
小徒,一路来,行过西番哈国,经两个月,又有五六千里,才到了贵处。”老僧
道:“也有万里之遥了。我弟子虚度一生,山门也不曾出去,诚所谓‘坐井观天’,
樗朽之辈。”三藏又问:“老院主高寿几何?”老僧道:“痴长二百七十岁了。”行者
听见道:“这还是我万代孙儿哩!”三藏瞅了他一眼道:“谨言!莫要不识高低,冲撞
人。”那和尚便问:“老爷,你有多少年纪了?”行者道:“不敢说。”那老僧也只当
一句疯话,便不介意,也不再问,只叫献茶。有一个小幸童,拿出一个羊脂玉的盘
儿,有三个法蓝镶金的茶钟;又一童,提一把白铜壶儿,斟了三杯香茶。真个是色
欺榴蕊艳,味胜桂花香。三藏见了,夸爱不尽道:“好物件!好物件!真是美食美器!”
那老僧道:“污眼!污眼!老爷乃天朝上国,广览奇珍,似这般器具,何足过奖?老爷
自上邦来,可有甚么宝贝,借与弟子一观?”三藏道:“可怜!我那东土,无甚宝贝;
就有时,路程遥远,也不能带得。”

行者在旁道:“师父,我前日在包袱里,曾见那领袈裟,不是件宝贝?拿与他看
看何如?”众僧听说袈裟,一个个冷笑。行者道:“你笑怎的?”院主道:“老爷才
说袈裟是件宝贝,言实可笑。若说袈裟,似我等辈者,不止二三十件;若论我师祖,
在此处做了二百五六十年和尚,足有七八百件!”叫:“拿出来看看。”那老和尚也
是他一时卖弄,便叫道人开库房,头陀抬柜子,就抬出十二柜,放在天井中,开了
锁,两边设下衣架,四围牵了绳子,将袈裟一件件抖开挂起,请三藏观看。果然是
满堂绮绣,四壁绫罗!

行者一一观之,都是些穿花纳锦,刺绣销金之物。笑道:“好,好,好!收起,
收起!把我们的也取出来看看。”三藏把行者扯住,悄悄的道:“徒弟,莫要与人斗
富。你我是单身在外,只恐有错。”行者道:“看看袈裟,有何差错?”三藏道:“你
不曾理会得。古人有云:‘珍奇玩好之物,不可使见贪婪奸伪之人。’倘若一经入目,
必动其心;既动其心,必生其计。汝是个畏祸的,索之而必应其求,可也;不然,
则殒身灭命,皆起于此,事不小矣。”行者道:“放心,放心,都在老孙身上!”你
看他不由分说,急急的走了去,把个包袱解开,早有霞光迸迸;尚有两层油纸裹定,
去了纸,取出袈裟,抖开时,红光满室,彩气盈庭。众僧见了,无一个不心欢口赞。
真个好袈裟!上头有:
千般巧妙明珠坠,万样稀奇佛宝攒。
上下龙须铺彩绮,兜罗四面锦沿边。
体挂魍魉从此灭,身披魑魅入黄泉。
托化天仙亲手制,不是真僧不敢穿。

那老和尚见了这般宝贝,果然动了奸心,走上前,对三藏跪下,眼中垂泪道:
“我弟子真是没缘!”三藏搀起道:“老院师有何话说?”他道:“老爷这件宝贝,
方才展开,天色晚了,奈何眼目昏花,不能看得明白,岂不是无缘!”三藏教:“掌
上灯来,让你再看。”那老僧道:“爷爷的宝贝,已是光亮;再点了灯,一发晃眼,
莫想看得仔细。”行者道:“你要怎的看才好?”老僧道:“老爷若是宽恩放心,教
弟子拿到后房,细细的看一夜,明早送还老爷西去,不知尊意何如?”三藏听说,
吃了一惊,埋怨行者道:“都是你,都是你!”行者笑道:“怕他怎的?等我包起来,
教他拿了去看。但有疏虞,尽是老孙管整。”那三藏阻当不住,他把袈裟递与老僧
道:“凭你看去;只是明早照旧还我,不得损污些须。”老僧喜喜欢欢,着幸童将袈
裟拿进去,却吩咐众僧,将前面禅堂扫净,取两张藤床,安设铺盖,请二位老爷安
歇;一壁厢又教安排明早斋送行,遂而各散。师徒们关了禅堂,睡下不题。

却说那和尚把袈裟骗到手,拿在后房灯下,对袈裟号啕痛哭,慌得那本寺僧,
不敢先睡。小幸童也不知为何,却去报与众僧道:“公公哭到二更时候,还不歇声。”
有两个徒孙,是他心爱之人,上前问道:“师公,你哭怎的?”老僧道:“我哭无缘,
看不得唐僧宝贝!”小和尚道:“公公年纪高大,发过了。他的袈裟,放在你面前,
你只消解开看便罢了,何须痛哭?”老僧道:“看的不长久。我今年二百七十岁,
空挣了几百件袈裟。怎么得有他这一件?怎么得做个唐僧?”小和尚道:“师公差了。
唐僧乃是离乡背井的一个行脚僧。你这等年高,享用也够了,倒要像他做行脚僧,
何也?”老僧道:“我虽是坐家自在,乐乎晚景,却不得他这袈裟穿穿。若教我穿
得一日儿,就死也闭眼,也是我来阳世间为僧一场!”

众僧道:“好没正经!你要穿他的,有何难处?我们明日留他住一日,你就穿他
一日;留他住十日,你就穿他十日,便罢了。何苦这般痛哭?”老僧道:“纵然留
他住了半载,也只穿得半载,到底也不得气长。他要去时,只得与他去,怎生留得
长远?”

正说话处,有一个小和尚,名唤广智,出头道:“公公,要得长远,也容易。”
老僧闻言,就欢喜起来道:“我儿,你有甚么高见?”广智道:“那唐僧两个是走路
的人,辛苦之甚,如今已睡着了。我们想几个有力量的,拿了枪刀,打开禅堂,将
他杀了,把尸首埋在后园,只我一家知道,却又谋了他的白马、行囊,却把那袈裟
留下,以为传家之宝,岂非子孙长久之计耶?”老和尚见说,满心欢喜,却才揩了
眼泪道:“好,好,好!此计绝妙!”即便收拾枪刀。

内中又有一个小和尚,名唤广谋,就是那广智的师弟,上前来道:“此计不妙。
若要杀他,须要看看动静。那个白脸的似易,那个毛脸的似难;万一杀他不得,却
不反招己祸?我有一个不动刀枪之法,不知你尊意如何?”老僧道:“我儿,你有何
法?”广谋道:“依小孙之见,如今唤聚东山大小房头,每人要干柴一束,舍了那
三间禅堂,放起火来,教他欲走无门,连马一火焚之。就是山前山后人家看见,只
说是他自不小心,走了火,将我禅堂都烧了。那两个和尚,却不都烧死?又好掩人
耳目。袈裟岂不是我们传家之宝?”那些和尚闻言,无不欢喜。都道:“强,强,
强!此计更妙,更妙!”遂教各房头搬柴来。唉!这一计,正是弄得个高寿老僧该尽
命,观音禅院化为尘!原来他那寺里,有七八十个房头,大小有二百余众。当夜一
拥搬柴,把个禅堂,前前后后,四面围绕不通,安排放火不题。

却说三藏师徒,安歇已定。那行者却是个灵猴,虽然睡下,只是存神炼气,朦
胧着醒眼。忽听得外面不住的人走,揸揸的柴响风生。他心疑惑道:“此时夜静,
如何有人行得脚步之声?莫敢是贼盗,谋害我们的?……”他就一骨鲁跳起。欲要开
门出看,又恐惊醒师父。你看他弄个精神,摇身一变,变做一个蜜峰儿。真个是:

口甜尾毒,腰细身轻。穿花度柳飞如箭,粘絮寻香似落星。小小微躯能负重,
嚣嚣薄翅会乘风。却自椽棱下,钻出看分明。
只见那众僧们,搬柴运草,已围住禅堂放火哩。行者暗笑道:“果依我师父之言!他
要害我们性命,谋我的袈裟,故起这等毒心。我待要拿棍打他啊,可怜又不禁打,
一顿棍都打死了,师父又怪我行凶。罢,罢,罢!与他个‘顺手牵羊,将计就计’,
教他住不成罢!”

好行者,一筋斗跳上南天门里,唬得个庞、刘、苟、毕躬身,马、赵、温、关
控背,俱道:“不好了,不好了!那闹天宫的主子又来了!”行者摇着手道:“列位免
礼,休惊。我来寻广目天王的。”说不了,却遇天王早到,迎着行者道:“久阔,久
阔。前闻得观音菩萨来见玉帝,借了四值功曹、六丁六甲并揭谛等,保护唐僧往西
天取经去,说你与他做了徒弟,今日怎么得闲到此?”行者道:“且休叙阔。唐僧
路遇歹人,放火烧他,事在万分紧急,特来寻你借‘辟火罩儿’,救他一救。快些
拿来使使,即刻返上。”天王道:“你差了;既是歹人放火,只该借水救他,如何要
辟火罩?”行者道:“你那里晓得就里。借水救之,却烧不起来,倒相应了他;只
是借此罩,护住了唐僧无伤,其余管他,尽他烧去。快些,快些!此时恐已无及。
莫误了我下边干事!”那天王笑道:“这猴子还是这等起不善之心,只顾了自家,就
不管别人。”行者道:“快着,快着!莫要调嘴,害了大事!”那天王不敢不借,遂将
罩儿递与行者。

行者拿了,按着云头,径到禅堂房脊上,罩住了唐僧与白马、行李。他却去那
后面老和尚住的方丈房上头坐,着意保护那袈裟。看那些人放起火来,他转捻诀念
咒,望巽地上吸一口气吹将去,一阵风起,把那火转刮得烘烘乱着,好火,好火!
但见:

黑烟漠漠,红焰腾腾:黑烟漠漠,长空不见一天星;红焰腾腾,大地有光千里
赤。起初时,灼灼金蛇;次后来,威威血马。南方三逞英雄,回禄大神施法力。
燥干柴烧烈火性,说甚么燧人钻木;熟油门前飘彩焰,赛过了老祖开炉。正是那无
情火发,怎禁这有意行凶;不去弭灾,反行助虐。风随火势,焰飞有千丈余高;火
趁风威,灰迸上九霄云外。乒乒乓乓,好便似残年爆竹;泼泼喇喇,却就如军中炮
声。烧得那当场佛像莫能逃,东院伽蓝无处躲。胜如赤壁夜鏖兵,赛过阿房宫内火!
这正是星星之火,能烧万顷之田。须臾间,风狂火盛,把一座观音院,处处通红。
你看那众和尚,搬箱抬笼,抢桌端锅,满院里叫苦连天。孙行者护住了后边方丈,
辟火罩罩住了前面禅堂,其余前后火光大发,真个是照天红焰辉煌,透壁金光照耀!

不期火起之时,惊动了一山兽怪。这观音院正南二十里远近,有座黑风山,山
中有一个黑风洞,洞中有一个妖精,正在睡醒翻身。只见那窗门透亮,只道是天明。
起来看时,却是正北下的火光晃亮,妖精大惊道:“呀!这必是观音院里失了火,这
些和尚好不小心!我看时,与他救一救来。”好妖精,纵起云头,即至烟火之下,果
然冲天之火,前面殿宇皆空,两廊烟火方灼。他大拽步,撞将进去,正呼唤叫取水
来,只见那后房无火,房脊上有一人放风。他却情知如此,急入里面看时,见那方
丈中间有些霞光彩气,台案上有一个青毡包袱。他解开一看,见是一领锦袈裟,
乃佛门之异宝。正是财动人心,他也不救火,他也不叫水,拿着那袈裟,趁哄打劫,
拽回云步,径转东山而去。

那场火只烧到五更天明,方才灭息。你看那众僧们,赤赤精精,啼啼哭哭,都
去那灰内寻铜铁,拨腐炭,扑金银。有的在墙筐里,苫搭窝棚;有的赤壁根头,支
锅造饭;叫冤叫屈,乱嚷乱闹不题。

却说行者取了辟火罩,一筋斗送上南天门,交与广目天王道:“谢借!谢借!”
天王收了道:“大圣至诚了。我正愁你不还我的宝贝,无处寻讨,且喜就送来也。”
行者道:“老孙可是那当面骗物之人?这叫做‘好借好还,再借不难。’”天王道:“许
久不面,请到宫少坐一时,何如?”行者道:“老孙比在前不同‘烂板凳,高谈阔
论’了;如今保唐僧,不得身闲。容叙!容叙!”急辞别坠云,又见那太阳星上。径
来到禅堂前,摇身一变,变做个蜜蜂儿,飞将进去,现了本象看时,那师父还沉睡
哩。

行者叫道:“师父,天亮了,起来罢。”三藏才醒觉,翻身道:“正是。”穿了衣
服,开门出来,忽抬头,只见些倒壁红墙,不见了楼台殿宇。大惊道:“呀!怎么这
殿宇俱无?都是红墙,何也?”行者道:“你还做梦哩!今夜走了火的。”三藏道:“我
怎不知?”行者道:“是老孙护了禅堂,见师父浓睡,不曾惊动。”三藏道:“你有
本事护了禅堂,如何就不救别房之火?”行者笑道:“好教师父得知。果然依你昨
日之言,他爱上我们的袈裟,算计要烧杀我们。若不是老孙知觉,到如今皆成灰骨
矣!”三藏闻言,害怕道:“是他们放的火么?”行者道:“不是他是谁?”三藏道:
“莫不是怠慢了你,你干的这个勾当?”行者道:“老孙是这等惫懒之人,干这等
不良之事?实实是他家放的。老孙见他心毒,果是不曾与他救火,只是与他略略助
些风的。”三藏道:“天那,天那!火起时,只该助水,怎转助风?”行者道:“你可
知古人云:‘人没伤虎心,虎没伤人意。’他不弄火,我怎肯弄风?”三藏道:“袈
裟何在?敢莫是烧坏了也?”行者道:“没事,没事,烧不坏,那放袈裟的方丈无火。”
三藏恨道:“我不管你,但是有些儿伤损,我只把那话儿念动念动,你就是死了!”
行者慌了道:“师父,莫念!莫念!管寻还你袈裟就是了。等我去拿来走路。”三藏才
牵着马,行者挑了担,出了禅堂,径往后方丈去。

却说那些和尚,正悲切间,忽的看见他师徒牵马挑担而来,唬得一个个魂飞魄
散道:“冤魂索命来了!”行者喝道:“甚么冤魂索命?快还我袈裟来!”众僧一齐跪
倒,叩头道:“爷爷呀!冤有冤家,债有债主。要索命不干我们事,都是广谋与老和
尚定计害你的,莫问我们讨命。”行者咄的一声道:“我把你这些该死的畜生!那个
问你讨甚么命!只拿袈裟来还我走路!”其间有两个胆量大的和尚道:“老爷,你们
在禅堂里已烧死了,如今又来讨袈裟,端的还是人,是鬼?”行者笑道:“这伙孽
畜!那里有甚么火来?你去前面看看禅堂,再来说话!”众僧们爬起来往前观看,那
禅堂外面的门窗扇,更不曾燎灼了半分。众人悚惧,才认得三藏是种神僧,行者
是尊护法。一齐上前叩头道:“我等有眼无珠,不识真人下界!你的袈裟在后面方丈
中老师祖处哩!”三藏行过了三五层败壁破墙,嗟叹不已。只见方丈果然无火,众
僧抢入里面,叫道:“公公!唐僧乃是神人,未曾烧死,如今反害了自己家当!趁早
拿出袈裟,还他去也。”

原来这老和尚寻不见袈裟,又烧了本寺的房屋,正在万分烦恼焦燥之处,一闻
此言,怎敢答应?因寻思无计,进退无方,拽开步,躬着腰,往那墙上着实撞了一
头,可怜只撞得脑破血流魂魄散,咽喉气断染红沙!有诗为证。诗曰:
堪叹老衲性愚蒙,枉作人间一寿翁。
欲得袈裟传远世,岂知佛宝不凡同!
但将容易为长久,定是萧条取败功。
广智广谋成甚用?损人利己一场空!
慌得个众僧哭道:“师公已撞杀了,又不见袈裟,怎生是好?”行者道:“想是汝等
盗藏起也!都出来,开具花名手本,等老孙逐一查点!”那上下房的院主,将本寺和
尚、头陀、幸童、道人尽行开具手本二张,大小人等,共计二百三十名。行者请师
父高坐,他却一一从头唱名搜检,都要解放衣襟,分明点过,更无袈裟。又将那各
房头搬抢出去的箱笼物件,从头细细寻遍,那里得有踪迹。

三藏心中烦恼,懊恨行者不尽,却坐在上面念动那咒。行者扑的跌倒在地,抱
着头,十分难禁,只教“莫念,莫念!管寻还了袈裟!”那众僧见了,一个个战兢兢
的,上前跪下劝解,三藏才合口不念。行者一骨鲁跳起来,耳朵里掣出铁棒要打那
些和尚,被三藏喝住道:“这猴头!你头痛还不怕,还要无礼?休动手,且莫伤人,
再与我审问一问!”众僧们磕头礼拜,哀告三藏道:“老爷饶命!我等委实的不曾看
见。这都是那老死鬼的不是。他昨晚看着你的袈裟,只哭到更深时候,看也不曾敢
看,思量要图长久,做个传家之宝,设计定策,要烧杀老爷;自火起之候,狂风大
作,各人只顾救火,搬抢物件,更不知袈裟去向。”行者大怒,走进方丈屋里,把
那触死鬼尸首抬出,选剥了细看,浑身更无那件宝贝;就把个方丈掘地三尺,也无
踪影。

行者忖量半晌,问道:“你这里可有甚么妖怪成精么?”院主道:“老爷不问,
莫想得知。我这里正东南有座黑风山。黑风洞内有一个黑大王。我这老死鬼常与他
讲道。他便是个妖精。别无甚物。”行者道:“那山离此有多远近?”院主道:“只
有二十里,那望见山头的就是。”行者笑道:“师父放心,不须讲了,一定是那黑怪
偷去无疑。”三藏道:“他那厢离此有二十里,如何就断得是他?”行者道:“你不
曾见夜间那火,光腾万里,亮透三天,且休说二十里,就是二百里也照见了!坐定
是他见火光耀,趁着机会暗暗的来到这里,看见我们袈裟是件宝贝,必然趁哄掳
去也。等老孙去寻他一寻。”三藏道:“你去了时,我却何倚?”行者道:“这个放
心,暗中自有神灵保护,明中等我叫那些和尚伏侍。”即唤众和尚过来,道:“汝等
着几个去埋那老鬼,着几个伏侍我师父,看守我白马!”众僧领诺。行者又道:“汝
等莫顺口儿答应,等我去了,你就不来奉承。看师父的,要怡颜悦色;养白马的,
要水草调匀;假有一毫儿差了,照依这个样棍,与你们看看!”他掣出棍子,照那
火烧的砖墙扑的一下,把那墙打得粉碎,又震倒了有七八层墙。众僧见了个个骨软
身麻,跪着磕头滴泪道:“爷爷宽心前去,我等竭力虔心,供奉老爷,决不敢一毫
怠慢!”好行者,急纵筋斗云,径上黑风山,寻找这袈裟。正是那:

金禅求正出京畿,仗锡投西涉翠微。
虎豹狼虫行处有,工商士客见时稀。
路逢异国愚僧妒,全仗齐天大圣威。
火发风生禅院废,黑熊夜盗锦衣。

毕竟此去不知袈裟有无,吉凶如何,且听下回分解。