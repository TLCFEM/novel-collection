\chapter{圣僧夜阻通天水~金木垂慈救小童}

却说那国王倚着龙床,泪如泉涌,只哭到天晚不住。行者上前高呼道:“你怎
么这等昏乱!见放着那道士的尸骸,一个是虎,一个是鹿,那羊力是一个羚羊。不
信时,捞上骨头来看。那里人有那样骷髅?他本是成精的山兽,同心到此害你。因
见气数还旺,不敢下手。若再过二年,你气数衰败,他就害了你性命,把你江山一
股儿尽属他了。幸我等早来,除妖邪救了你命。你还哭甚!哭甚!急打发关文,送我
出去。”国王闻此,方才省悟。那文武多官俱奏道:“死者果然是白鹿、黄虎;油锅
里果是羊骨。圣僧之言,不可不听!”国王道:“既是这等,感谢圣僧。今日天晚,
教太师且请圣僧至智渊寺;明日早朝,大开东阁,教光禄寺安排素净筵宴酬谢。”
果送至寺里安歇。

次日五更时候,国王设朝,聚集多官,传旨:“快出招僧榜文,四门各路张挂。”
一壁厢大排筵宴,摆驾出朝,至智渊寺门外,请了三藏等,共入东阁赴宴,不在话
下。

却说那脱命的和尚闻有招僧榜,个个欣然,都入城来寻孙大圣,交纳毫毛谢恩。
这长老散了宴,那国王换了关文,同皇后嫔妃,两班文武,送出朝门。只见那些和
尚跪拜道旁,口称:“齐天大圣爷爷!我等是沙滩上脱命僧人。闻知爷爷扫除妖孽,
救拔我等,又蒙我王出榜招僧,特来交纳毫毛,叩谢天恩。”行者笑道:“汝等来了
几何?”僧人道:“五百名,半个不少。”行者将身一抖,收了毫毛。对君臣僧俗人
说道:“这些和尚,实是老孙放了。车辆是老孙运转双关,穿夹脊,碎了。那两
个妖道也是老孙打死了。今日灭了妖邪,方知是禅门有道。向后来,再不可胡为乱
信。望你把三教归一:也敬僧,也敬道,也养育人才。我保你江山永固。”国王依
言,感谢不尽,遂送唐僧出城去讫。

这一去,只为殷勤经三藏,努力修持光一元。晓行夜住,渴饮饥餐,不觉的春
尽夏残,又是秋光天气。

一日,天色已晚。唐僧勒马道:“徒弟,今宵何处安身也?”行者道:“师父,
出家人莫说那在家人的话。”三藏道:“在家人怎么?出家人怎么?”行者道:“在家
人,这时候温床暖被,怀中抱子,脚后蹬妻,自自在在睡觉;我等出家人,那里能
够!便是要带月披星,餐风宿水,有路且行,无路方住。”八戒道:“哥哥,你只知
其一,不知其二。如今路多峻,我挑着重担,着实难走,须要寻个去处,好眠一
觉,养养精神,明日方好捱担;不然,却不累倒我也?”行者道:
“趁月光再走一程,到有人家之所再住。”师徒们没奈何,只得相随行者往前。

又行不多时,只听得滔滔浪响。八戒道:“罢了!来到尽头路了!”沙僧道:“是
一股水挡住也。”唐僧道:“却怎生得渡?”八戒道:“等我试之,看深浅何如。”三
藏道:“悟能,你休乱谈。水之浅深,如何试得?”八戒道:“寻一个鹅卵石,抛在
当中。若是溅起水泡来,是浅;若是骨都都下有声,是深。”行者道:“你去试试
看。”那呆子在路旁摸了一块顽石,望水中抛去,只听得骨都都泛起鱼津,沉下水
底。他道:“深,深,深!去不得!”唐僧道:“你虽试得深浅,却不知有多少宽阔。”
八戒道:“这个却不知,不知。”行者道:“等我看看。”好大圣,纵筋斗云,跳在空
中,定睛观看,但见那:

洋洋光浸月,浩浩影浮天。灵派吞华岳,长流贯百川。千层汹浪滚,万叠峻波
颠。岸口无渔火,沙头有鹭眠。茫然浑似海,一望更无边。
急收云头,按落河边道:“师父,宽哩!宽哩!去不得!老孙火眼金睛,白日里常看千
里,凶吉晓得是。夜里也还看三五百里。如今通看不见边岸,怎定得宽阔之数?”

三藏大惊,口不能言,声音哽咽道:“徒弟啊,似这等怎了?”沙僧道:“师父
莫哭。你看那水边立的,可不是个人么?”行者道:“想是扳罾的渔人,等我问他
去来。”拿了铁棒,两三步,跑到面前看处,呀!不是人,是一面石碑。碑上有三个
篆文大字,下边两行,有十个小字。三个大字,乃“通天河”。十个小字,乃“径
过八百里,亘古少人行。”行者叫:“师父,你来看看。”三藏看见,滴泪道:“徒弟
呀,我当年别了长安,只说西天易走;那知道妖魔阻隔,山水迢遥!”

八戒道:“师父,你且听,是那里鼓钹声音?想是做斋的人家。我们且去赶些斋
饭吃,问个渡口寻船,明日过去罢。”三藏马上听得,果然有鼓钹之声。“却不是道
家乐器,足是我僧家举事。我等去来。”

行者在前引马,一行闻响而来。那里有甚正路,没高没低,漫过沙滩,望见一
簇人家住处,约摸有四五百家,却也都住得好。但见:

倚山通路,傍岸临溪。处处柴扉掩,家家竹院关。沙头宿鹭梦魂清,柳外啼鹃
喉舌冷。短笛无声,寒砧不韵。红蓼枝摇月,黄芦叶斗风。陌头村犬吠疏篱,渡口
老渔眠钓艇。灯火稀,人烟静,半空皎月如悬镜。忽闻一阵白香,却是西风隔岸
送。

三藏下马,只见那路头上有一家儿,门外竖一首幢幡,内里有灯烛荧煌,香烟
馥郁。三藏道:“悟空,此处比那山凹河边,却是不同。在人间屋檐下,可以遮得
冷露,放心稳睡。你都莫来,让我先到那斋公门首告求。若肯留我,我就招呼汝等;
假若不留,你却休要撒泼。汝等脸嘴丑陋,只恐唬了人,闯出祸来,却倒无住处矣。”
行者道:“说得有理。请师父先去,我们在此守待。”

那长老才摘了斗笠,光着头,抖抖褊衫,拖着锡杖,径来到人家门外。见那门
半开半掩,三藏不敢擅入。聊站片时,只见里面走出一个老者,项下挂着数珠,口
念阿弥陀佛,径自来关门,慌得这长老合掌高叫:“老施主,贫僧问讯了。”那老者
还礼道:“你这和尚,却来迟了。”三藏道:“怎么说?”老者道:“来迟无物了。早
来啊,我舍下斋僧,尽饱吃饭,熟米三升,白布一段,铜钱十文。你怎么这时才来?”
三藏躬身道:“老施主,贫僧不是赶斋的。”老者道:“既不赶斋,来此何干?”三
藏道:“我是东土大唐钦差往西天取经者。今到贵处,天色已晚。听得府上鼓钹之
声,特来告借一宿,天明就行也。”那老者摇手道:“和尚,出家人休打诳语。东土
大唐,到我这里,有五万四千里路。你这等单身,如何来得?”三藏道:“老施主
见得最是。但我还有三个小徒,逢山开路,遇水叠桥,保护贫僧,方得到此。”老
者道:“既有徒弟,何不同来?”教:“请,请,我舍下有处安歇。”三藏回头,叫
声“徒弟,这里来。”

那行者本来性急,八戒生来粗鲁,沙僧却也莽撞,三个人听得师父招呼,牵着
马,挑着担,不问好歹,一阵风,闯将进去。那老者看见,唬得跌倒在地,口里只
说是“妖怪来了!妖怪来了!”三藏搀起道:“施主莫怕。不是妖怪,是我徒弟。”老
者战兢兢道:“这般好俊师父,怎么寻这样丑徒弟!”三藏道:“虽然相貌不终,却
倒会降龙伏虎,捉怪擒妖。”老者似信不信的,扶着唐僧慢走。

却说那三个凶顽,闯入厅房上,拴了马,丢下行李。那厅中原有几个和尚念经。
八戒掬着长嘴,喝道:“那和尚,念的是甚么经?”那些和尚,听见问了一声,忽
然抬头:

观看外来人,嘴长耳朵大,身粗背膊宽,声响如雷咋。行者与沙僧,容貌更丑
陋。厅堂几众僧,无人不害怕。黎还念经,班首教行罢。难顾磬和铃,佛像且丢
下。一齐吹息灯,惊
散光乍乍。跌跌与爬爬,门槛何曾跨!你头撞我头,似倒葫芦架。清清好道场,翻
成大笑话。
这兄弟三人,见那些人跌跌爬爬,鼓着掌哈哈大笑。那些僧越加悚惧,磕头撞脑,
各顾性命,通跑净了。

三藏搀那老者,走上厅堂,灯火全无,三人嘻嘻哈哈的还笑。唐僧骂道:“这
泼物,十分不善!我朝朝教诲,日日叮咛,古人云:‘不教而善,非圣而何!教而后
善,非贤而何!教亦不善,非愚而何!’汝等这般撒泼,诚为至下至愚之类!走进门
不知高低,唬倒了老施主,惊散了念经僧,把人家好事都搅坏了,却不是堕罪与我?”
说得他们不敢回言。那老者方信是他徒弟,急回头作礼道:“老爷,没大事,没大
事,才然关了灯,散了花,佛事将收也。”八戒道:“既是了帐,摆出满散的斋来,
我们吃了睡觉。”老者叫:“掌灯来!掌灯来!”家里人听得,大惊小怪道:“厅上念
经,有许多香烛,如何又教掌灯?”几个僮仆出来看时,这个黑洞洞的,即便点火
把灯笼,一拥而至。忽抬头见八戒、沙僧,慌得丢了火把,急抽身关了中门。往里
嚷道:“妖怪来了!妖怪来了!”行者拿起火把,点上灯烛,扯过一张交椅,请唐僧
坐在上面。他兄弟们坐在两旁。那老者坐在前面。

正叙坐间,只听得里面门开处,又走出一个老者,拄着拐杖,道:“是甚么邪
魔,黑夜里来我善门之家?”前面坐的老者,急起身迎到屏门后道:“哥哥莫嚷,
不是邪魔,乃东土大唐取经的罗汉。徒弟们相貌虽凶,果然是山恶人善。”那老者
方才放下拄杖,与他四位行礼。礼毕,也坐了面前,叫:“看茶来。排斋。”连叫数
声,几个僮仆,战战兢兢,不敢拢帐。

八戒忍不住问道:“老者,你这盛价,两边走怎的?”老者道:“教他们捧斋来
侍奉老爷。”八戒道:“几个人伏侍?”老者道:“八个人。”八戒道:“这八个人伏
侍那个?”老者道:“伏侍你四位。”八戒道:“那白面师父,只消一个人;毛脸雷
公嘴的,只消两个人;那晦气脸的,要八个人;我得二十个人伏侍方够。”老者道:
“这等说,想是你的食肠大些。”八戒道:“也将就看得过。”老者道:“有人,有人。”
七大八小,就叫出有三四十人出来。

那和尚与老者,一问一答的讲话,众人方才不怕。却将上面排了一张桌,请唐
僧上坐;两边摆了三张桌,请他三位坐;前面一张桌,坐了二位老者。先排上素果
品菜蔬,然后是面饭、米饭、闲食、粉汤,排得齐齐整整。唐长老举起箸来,先念
一卷《启斋经》。那呆子一则有些急吞,二来有些饿了,那里等唐僧经完,拿过红
漆木碗来,把一碗白米饭,扑的丢下口去,就了了。旁边小的道:“这位老爷忒没
算计,不笼馒头,怎的把饭笼了,却不污了衣服?”八戒笑道:“不曾笼,吃了。”
小的道:“你不曾举口,怎么就吃了?”八戒道:“儿子们便说谎!分明吃了;不信,
再吃与你看。”那小的们,又端了碗,盛一碗递与八戒。呆子幌一幌,又丢下口去
就了了。众僮仆见了道:“爷爷呀!你是‘磨砖砌的喉咙,着实又光又溜!’”那唐僧
一卷经还未完,他已五六碗过手了。然后却才同举箸,一齐吃斋。呆子不论米饭面
饭,果品闲食,只情一捞乱,口里还嚷:“添饭,添饭!”渐渐不见来了。

行者叫道:“贤弟,少吃些罢。也强似在山凹里忍饿,将就够得半饱也好了。”
八戒道:“嘴脸!常言道:‘斋僧不饱,不如活埋’哩。”行者教:“收了家火,莫睬
他!”二老者躬身道:“不瞒老爷说。白日里倒也不怕,似这大肚子长老,也斋得起
百十众;只是晚了,收了残斋,只蒸得一石面饭、五斗米饭与几桌素食,要请几个
亲邻与众僧们散福;不期你列位来,唬得众僧跑了,连亲邻也不曾敢请,尽数都供
奉了列位。如不饱,再教蒸去。”八戒道:“再蒸去,再蒸去!”

话毕,收了家火桌席。三藏拱身,谢了斋供。才问:“老施主,高姓?”老者
道:“姓陈。”三藏合掌道:“这是我贫僧华宗了。”老者道:“老爷也姓陈?”三藏
道:“是,俗家也姓陈。请问适才做的甚么斋事?”八戒笑道:“师父问他怎的!岂
不知道?必然是‘青斋’、‘平安斋’、‘了场斋’罢了。”老者道:“不是,不是。”
三藏又问:“端的为何?”老者道:“是一场‘预修亡斋’。”八戒笑得打跌道:“公
公忒没眼力!我们是扯谎架桥,哄人的大王,你怎么把这谎话哄我!和尚家岂不知斋
事?只有个‘预修寄库斋’、‘预修填还斋’,那里有个‘预修亡斋’的?你家人又不
曾有死的,做甚亡斋?”

行者闻言,暗喜道:“这呆子乖了些也。老公公,你是错说了。怎么叫做‘预
修亡斋’?”那二位欠身道:“你等取经,怎么不走正路,却到我这里来?”行
者道:“走的是正路,只见一股水挡住,不能得渡;因闻鼓钹之声,特来造府借宿。”
老者道:“你们到水边,可曾见些甚么?”行者道:“止见一面石碑,上书‘通天河’
三字,下书‘径过八百里,亘古少人行’十字,再无别物。”老者道:“再往上岸走
走,好的离那碑记只有里许,有一座灵感大王庙,你不曾见?”行者道:“未见。
请公公说说,何为灵感?”那两个老者一齐垂泪道:“老爷啊!那大王:
感应一方兴庙宇,威灵千里佑黎民。
年年庄上施甘露,岁岁村中落庆云。”
行者道:“施甘雨,落庆云,也是好意思,你却这等伤情烦恼,何也?”那老者跌
脚捶胸,哏了一声道:“老爷啊!
虽则恩多还有怨,纵然慈惠却伤人。
只因要吃童男女,不是昭彰正直神。”

行者道:“要吃童男女么?”老者道:“正是。”行者道:“想必轮到你家了?”
老者道:“今年正到舍下。我们这里,有百家人家居住。此处属车迟国元会县所管,
唤做陈家庄。这大王一年一次祭赛,要一个童男,一个童女,猪羊牲醴供献他。他
一顿吃了,保我们风调雨顺;若不祭赛,就来降祸生灾。”行者道:“你府上几位令
郎?”老者捶胸道:“可怜!可怜!说甚么令郎,羞杀我等!这个是我舍弟,名唤陈清。
老拙叫做陈澄。我今年六十三岁,他今年五十八岁,儿女上都艰难。我五十岁上还
没儿子,亲友们劝我纳了一妾,没奈何,寻下一房,生得一女。今年才交八岁,取
名唤做一秤金。”八戒道:“好贵名!怎么叫做一秤金?”老者道:“我因儿女艰难,
修桥补路,建寺立塔,布施斋僧,有一本帐目,那里使三两,那里使五两;到生女
之年,却好用过有三十斤黄金。三十斤为一秤,所以唤做一秤金。”

行者道:“那个的儿子么?”老者道:“舍弟有个儿子,也是偏出,今年七岁了,
取名唤做陈关保。”行者问:“何取此名?”老者道:“家下供养关圣爷爷,因在关
爷之位下求得这个儿子,故名关保。我兄弟二人,年岁百二,止得这两个人种,不
期轮次到我家祭赛,所以不敢不献。故此父子之情,难割难舍,先与孩儿做个超生
道场。故曰‘预修亡斋’者,此也。”

三藏闻言,止不住腮边泪下道:“这正是古人云:‘黄梅不落青梅落,老天偏害
没儿人。’”行者笑道:“等我再问他。老公公,你府上有多大家当?”二老道:“颇
有些儿,水田有四五十顷,旱田有六七十顷,草场有八九十处;水黄牛有二三百头,
驴马有三二十匹,猪羊鸡鹅无数。舍下也有吃不着的陈粮,穿不了的衣服。家财产
业,也尽得数。”行者道:“你这等家业,也亏你省将起来的。”老者道:“怎见我省?”
行者道:“既有这家私,怎么舍得亲生儿女祭赛?拚了五十两银子,可买一个童男;
拚了一百两银子,可买一个童女。连绞缠不过二百两之数,可就留下自己儿女后代,
却不是好?”二老滴泪道:“老爷!你不知道。那大王甚是灵感,常来我们人家行走。”
行者道:“他来行走,你们看见他是甚么嘴脸?有几多长短?”二老道:“不见其形,
只闻得一阵香风,就知是大王爷爷来了,即忙满斗焚香,老少望风下拜。他把我们
这人家,匙大碗小之事,他都知道。老幼生时年月,他都记得。只要亲生儿女,他
方受用。不要说二三百两没处买,就是几千万两,也没处买这般一模一样同年同月
的儿女。”

行者道:“原来这等。也罢,也罢,你且抱你令郎出来,我看看。”那陈清急入
里面,将关保儿抱出厅上,放在灯前。小孩儿那知死活,笼着两袖果子,跳跳舞舞
的,吃着耍子。行者见了,默默念声咒语,摇身一变,变作那关保儿一般模样。两
个孩儿,搀着手,在灯前跳舞,唬得那老者谎忙跪着唐僧道:“老爷,不当人子!不
当人子!这位老爷才然说话,怎么就变作我儿一般模样,叫他一声,齐应齐走!却折
了我们年寿!请现本相,请现本相!”行者把脸抹了一把,现了本相。那老者跪在面
前道:“老爷原来有这样本事。”行者笑道:“可像你儿子么?”老者道:“像,像,
像!果然一般嘴脸、一般声音、一般衣服、一般长短。”行者道:“你还没细看哩。
取秤来称称,可与他一般轻重。”老者道:“是,是,是;是一般重。”行者道:“似
这等可祭赛得过么?”老者道:“忒好,忒好!祭得过了!”

行者道:“我今替这个孩儿性命,留下你家香烟后代,我去祭赛那大王去也。”
那陈清跪地磕头道:“老爷果若慈悲替得,我送白银一千两,与唐老爷做盘缠往西
天去。”行者道:“就不谢谢老孙?”老者道:“你已替祭,没了你也。”行者道:“怎
的得没了?”老者道:“那大王吃了。”行者道:“他敢吃我?”老者道:“不吃你,
好道嫌腥。”行者笑道:“任从天命。吃了我,是我的命短;不吃,是我的造化。我
与你祭赛去。”

那陈清只管磕头相谢,又允送银五百两;惟陈澄也不磕头,也不说谢,只是倚
着那屏门痛哭。行者知之,上前扯住道:“老大,你这不允我,不谢我,想是舍不
得你女儿么?”陈澄才跪下道:“是,舍不得。敢蒙老爷盛情,救替了我侄子也够
了。但只是老拙无儿,止此一女,就是我死之后,他也哭得痛切,怎么舍得!”

行者道:“你快去蒸上五斗米的饭,整治些好素菜,与我那长嘴师父吃,教他
变作你的女儿,我兄弟同去祭赛。索性行个阴骘,救你两个儿女性命,如何?”那
八戒听得此言,心中大惊,道:“哥哥,你要弄精神,不管我死活,就要攀扯我。”
行者道:“贤弟,常言道:‘鸡儿不吃无工之食。’你我进门,感承盛斋,你还嚷吃
不饱哩,怎么就不与人家救些患难?”八戒道:“哥啊,你便会变化,我却不会哩。”
行者道:“你也有三十六般变化,怎么不会?”唐僧叫:“悟能,你师兄说得最是,
处得甚当。常言‘救人一命,胜造七级浮屠。’一则感谢厚情,二来当积阴德。况
凉夜无事,你兄弟耍耍去来。”八戒道:“你看师父说的话!我只会变山,变树,变
石头,变癞象,变水牛,变大胖汉还可;若变小女儿,有几分难哩。”行者道:“老
大莫信他,抱出你令爱来看。”

那陈澄急入里边,抱将一秤金孩儿,到了厅上。一家子,妻妾大小,不分老幼
内外,都出来磕头礼拜,只请救孩儿性命。那女儿头上戴一个八宝垂珠的花翠箍;
身上穿一件红闪黄的丝袄,上套着一件官绿缎子棋盘领的披风;腰间系一条大红
花绢裙;脚下踏一双虾蟆头浅红丝鞋;腿上系两只绡金膝裤儿;也袖着果子吃哩。
行者道:“八戒,这就是女孩儿。你快变的像他,我们祭赛去。”八戒道:“哥呀,
似这般小巧俊秀,怎变?”行者叫:“快些!莫讨打!”八戒慌了道:“哥哥不要打,
等我变了看。”

这呆子念动咒语,把头摇了几摇,叫“变!”真个变过头来,就也像女孩儿面
目,只是肚子胖大,郎伉不像。行者笑道:“再变变!”八戒道:“凭你打了罢!变不
过来,奈何?”行者道:“莫成是丫头的头,和尚的身子?弄的这等不男不女,却怎
生是好?你可布起罡来。”他就吹他一口仙气,果然即时把身子变过,与那孩儿一般。
便教:“二位老者,带你宝眷与令郎令爱进去,不要错了。一会家,我兄弟躲懒讨
乖,走进去,转难识认。你将好果子与他吃,不可教他哭叫;恐大王一时知觉,走
了风讯。等我两人耍子去也!”

好大圣,吩咐沙僧保护唐僧,他变作陈关保,八戒变作一秤金。二人俱停当了,
却问:“怎么供献?还是捆了去,是绑了去?蒸熟了去,是剁碎了去?”八戒道:“哥
哥,莫要弄我。我没这个手段。”老者道:“不敢,不敢!只是用两个红漆丹盘,请
二位坐在盘内,放在桌上,着两个后生抬一张桌子,把你们抬上庙去。”行者道:“好,
好,好!拿盘子出来,我们试试。”那老者即取出两个丹盘。行者与八戒坐上,四个
后生,抬起两张桌子,往天井里走走儿,又抬回放在堂上。行者欢喜道:“八戒,
像这般子走走耍耍,我们也是上台盘的和尚了。”八戒道:“若是抬了去,还抬回来,
两头抬到天明,我也不怕;只是抬到庙里,就要吃哩,这个却不是耍子!”行者道:
“你只看着我。着吃我时,你就走了罢。”八戒道:“知他怎么吃哩?如先吃童男,
我便好跑;如先吃童女,我却如何?”老者道:“常年祭赛时,我这里有胆大的,
钻在庙后,或在供桌底下,看见他先吃童男,后吃童女。”八戒道:“造化,造化!”
兄弟正然谈论,只听得外面锣鼓喧天,灯火照耀,同庄众人打开前门,叫:“抬出
童男童女来!”这老者哭哭啼啼,那四个后生将他二人抬将出去。

端的不知性命何如,且听下回分解。