\chapter{难灭伽持圆大觉~法王成正体天然}

话说唐三藏固住元阳,出离了烟花苦套,随行者投西前进。不觉夏时,正值那
熏风初动,梅雨丝丝。好光景:
冉冉绿阴密,风轻燕引雏。
新荷翻沼面,修竹渐扶苏。
芳草连天碧,山花遍地铺。
溪边蒲插剑,榴火壮行图。
师徒四众,耽炎受热,正行处,忽见那路旁有两行高柳,柳阴中走出一个老母,右
手下搀着一个小孩儿,对唐僧高叫道:“和尚,不要走了,快早儿拨马东回,进西
去都是死路。”唬得个三藏跳下马来,打个问讯道:“老菩萨,古人云:‘海阔从鱼
跃,天空任鸟飞。’怎么西进便没路了?”那老母用手朝西指道:“那里去,有五六
里远近,乃是灭法国。那国王前生那世里结下冤仇,今世里无端造罪。二年前许下
一个罗天大愿,要杀一万个和尚。这两年陆陆续续,杀够了九千九百九十六个无名
和尚,只要等四个有名的和尚,凑成一万,好做圆满哩。你们去,若到城中,都是
送命王菩萨!”三藏闻言,心中害怕,战兢兢的道:“老菩萨,深感盛情,感谢不尽!
但请问可有不进城的方便路儿,我贫僧转过去罢。”那老母笑道:“转不过去,转不
过去。只除是会飞的,就过去了也。”八戒在旁边卖嘴道:“妈妈儿莫说黑话。我们
都会飞哩。”

行者火眼金睛,其实认得好歹——那老母搀着孩儿,原是观音菩萨与善财童子
——慌得倒身下拜,叫道:“菩萨,弟子失迎,失迎!”那菩萨一朵祥云,轻轻驾起,
吓得个唐长老立身无地,只情跪着磕头。八戒、沙僧也慌跪下,朝天礼拜。一时间,
祥云缥渺,径回南海而去。

行者起来,扶着师父道:“请起来,菩萨已回宝山也。”三藏起来道:“悟空,
你既认得是菩萨,何不早说?”行者笑道:“你还问话不了,我即下拜,怎么还是
不早哩?”八戒、沙僧对行者道:“感蒙菩萨指示,前边必是灭法国,要杀和尚,
我等怎生奈何?”行者道:“呆子休怕!我们曾遭着那毒魔狠怪,虎穴龙潭,更不曾
伤损;此间乃是一国凡人,有何惧哉?只奈这里不是住处。天色将晚,且有乡村人
家,上城买卖回来的,看见我们是和尚,嚷出名去,不当稳便。且引师父找下大路,
寻个僻静之处,却好商议。”真个三藏依言,一行都闪下路来,到一个坑坎之下,
坐定。行者道:“兄弟,你两个好生保守师父,待老孙变化了,去那城中看看,寻
一条僻路,连夜去也。”三藏叮嘱道:“徒弟啊,莫当小可。王法不容。你须仔细!”
行者笑道:“放心,放心!老孙自有道理。”

好大圣,话毕,将身一纵,唿哨的跳在空中。怪哉:
上面无绳扯,下头没棍撑。
一般同父母,他便骨头轻。
伫立在云端里,往下观看。只见那城中喜气冲融,祥光荡漾。行者道:“好个去处!
为何灭法?”看一会,渐渐天昏,又见那:

十字街灯光灿烂,九重殿香蔼钟鸣。七点皎星照碧汉,八方客旅卸行踪。六军
营,隐隐的画角才吹;五鼓楼,点点的铜壶初滴。四边宿雾昏昏,三市寒烟蔼蔼。
两两夫妻归绣幕,一轮明月上东方。

他想着:“我要下去,到街坊打看路径,这般个嘴脸,撞见人,必定说是和尚;
等我变一变了。”捻着诀,念动真言,摇身一变,变做个扑灯蛾儿:

形细翼硗轻巧,灭灯扑烛投明。本来面目化生成,腐草中间灵应。每爱炎光触
焰,忙忙飞绕无停。紫衣香翅赶流萤,最喜夜深风静。
但见他翩翩翻翻,飞向六街三市。傍房檐,近屋角。正行时,忽见那隅头拐角上一
湾子人家,人家门首挂着个灯笼儿。他道:“这人家过元宵哩?怎么挨排儿都点灯
笼?”他硬硬翅,飞近前来,仔细观看。正当中一家子方灯笼上,写着“安歇往来
商贾”六字,下面又写着“王小二店”四字。行者才知是开饭店的。又伸头打一看,
看见有八九个人,都吃了晚饭,宽了衣服,卸了头巾,洗了脚手,各各上床睡了。
行者暗喜道:“师父过得去了。”你道他怎么就知过得去?他要起个不良之心,等那
些人睡着,要偷他的衣服、头巾,装做俗人进城。

噫,有这般不遂意的事!正思忖处,只见那小二走向前,吩咐:“列位官人,仔
细些。我这里君子小人不同,各人的衣物、行李都要小心着。”你想那在外做买卖
的人,那样不仔细?又听得店家吩咐,越发谨慎。他都爬起来道:“主人家说得有理。
我们走路的人辛苦,只怕睡着,急忙不醒,一时失所,奈何?你将这衣服、头巾、
搭联都收进去,待天将明,交付与我们起身。”那王小二真个把些衣物之类,尽情
都搬进他屋里去了。行者性急,展开翅,就飞入里面,丁在一个头巾架上。又见王
小二去门首摘了灯笼,放下吊搭,关了门窗,却才进房,脱衣睡下。那王小二有个
婆子,带了两个孩子,哇哇聒噪,急忙不睡。那婆子又拿了一件破衣,补补纳纳,
也不见睡。行者暗想道:“若等这婆子睡了下手,却不误了师父?”又恐更深,城
门闭了,他就忍不住,飞下去,望灯上一扑。真是“舍身投火焰,焦额探残生”。
那盏灯早已息了。他又摇身一变,变作个老鼠,哇哇的叫了两声,跳下来,拿
着衣服、头巾,往外就走。那婆子慌慌张张的道:“老头子,不好了!夜耗子成精也!”

行者闻言,又弄手段,拦着门,厉声高叫道:“王小二,莫听你婆子胡说。我
不是夜耗子成精。明人不做暗事。吾乃齐天大圣临凡,保唐僧往西天取经。你这国
王无道,特来借此衣冠,装扮我师父。一时过了城去,就便送还。”那王小二听言,
一毂辘起来,黑天摸地,又是着忙的人,捞着裤子当衫子,左穿也穿不上,右套也
套不上。

那大圣使个摄法,早已驾云出去。复翻身,径至路下坑坎边前。三藏见星光月
皎,探身凝望,见是行者,来至近前,即开口叫道:“徒弟,可过得灭法国么?”
行者上前放下衣物道:“师父,要过灭法国,和尚做不成。”八戒道:“哥,你勒
那个哩?不做和尚也容易,只消半年不剃头,就长出毛来也。”行者道:“那里等得
半年!眼下就都要做俗人哩!”那呆子慌了道:“但你说话,通不察理。我们如今都
是和尚,眼下要做俗人,却怎么戴得头巾?就是边儿勒住,也没收顶绳处。”三藏喝
道:“不要打花,且干正事!端的何如?”

行者道:“师父,他这城池,我已看了。虽是国王无道杀僧,却倒是个真天子,
城头上有祥光喜气。城中的街道,我也认得。这里的乡谈,我也省得,会说。却才
在饭店内借了这几件衣服、头巾,我们且扮作俗人,进城去借了宿,至四更天就起
来,教店家安排了斋吃;捱到五更时候,挨城门而去,奔大路西行,就有人撞见扯
住,也好折辨:只说是上邦钦差的,灭法王不敢阻滞,放我们来的。”沙僧道:“师
兄处的最当。且依他行。”真个长老无奈,脱了褊衫,去了僧帽,穿了俗人的衣服,
戴了头巾。沙僧也换了。八戒的头大,戴不得巾儿,被行者取了些针线,把头巾扯
开,两顶缝做一顶,与他搭在头上;拣件宽大的衣服,与他穿了。然后自家也换上
一套道:“列位,这一去,把‘师父徒弟’四个字儿且收起。”八戒道:“除了此四
字,怎的称呼?”行者道:“都要做弟兄称呼:师父叫做唐大官儿,你叫做朱三官
儿,沙僧叫做沙四官儿,我叫做孙二官儿。但到店中,你们切休言语,只让我一个
开口答话。等他问甚么买卖,只说是贩马的客人。把这白马做个样子,说我们是十
弟兄,我四个先来赁店房卖马。那店家必然款待我们,我们受用了,临行时,等我
拾块瓦查儿,变块银子谢他,却就走路。”长老无奈,只得曲从。

四众忙忙的牵马挑担,跑过那边。此处是个太平境界,入更时分,尚未关门。
径直进去,行到王小二店门首,只听得里边叫哩。有的说:“我不见了头巾!”有的
说:“我不见了衣服!”行者只推不知,引着他们,往斜对门一家安歇。那家子还未
收灯笼,即近门叫道:“店家,可有闲房儿,我们安歇?”那里边有个妇人答应道:
“有,有,有。请官人们上楼。”说不了,就有一个汉子来牵马。行者把马儿递与
牵进去。他引着师父,从灯影儿后面,径上楼门。那楼上有方便的桌椅,推开窗格,
映月光齐齐坐下。只见有人点上灯来。行者拦门,一口吹息道:“这般月亮不用灯。”

那人才下去,又一个丫环拿四碗清茶。行者接住,楼下又走上一个妇人来,约
有五十七八岁的模样,一直上楼,站着旁边。问道:“列位客官,那里来的?有甚宝
货?”行者道:“我们是北方来的,有几匹粗马贩卖。”那妇人道:“贩马的客人尚
还小。”行者道:“这一位是唐大官,这一位是朱三官,这一位是沙四官,我学生是
孙二官。”妇人笑道:“异姓。”行者道:“正是异姓同居。我们共有十个弟兄,我四
个先来赁店房打火;还有六个在城外借歇;领着一群马,因天晚不好进城。待我们
赁了房子,明早都进来。只等卖了马才回。”那妇人道:“一群有多少马?”行者道:
“大小有百十匹儿,都像我这个马的身子,却只是毛片不一。”妇人笑道:“孙二官
人诚然是个客纲客纪。早是来到舍下,第二个人家也不敢留你。我舍下院落宽阔,
槽札齐备,草料又有,凭你几百匹马都养得下。却一件:我舍下在此开店多年,也
有个贱名。先夫姓赵,不幸去世久矣。我唤做赵寡妇店。我店里三样儿待客。如今
先小人,后君子,先把房钱讲定后,好算帐。”行者道:“说得是。你府上是那三样
待客?常言道:‘货有高低三等价,客无远近一般看。’你怎么说三样待客?你可试说
说我听。”

赵寡妇道:“我这里是上、中、下三样。上样者:五果五菜的筵席。狮仙斗糖
桌面,二位一张,请小娘儿来陪唱陪歇。每位该银五钱,连房钱在内。”行者笑道:
“相应啊!我那里五钱银子还不够请小娘儿哩。”寡妇又道:“中样者:合盘桌儿,
只是水果、热酒,筛来凭自家猜枚行令,不用小娘儿,每位只该二钱银子。”行者
道:“一发相应!下样儿怎么?”妇人道:“不敢在尊客面前说。”行者道:“也说说
无妨。我们好拣相应的干。”妇人道:“下样者:没人伏侍,锅里有方便的饭,凭他
怎么吃;吃饱了,拿个草儿,打个地铺,方便处睡觉;天光时,凭赐几文饭钱,决
不争竞。”八戒听说道:“造化,造化!老朱的买卖到了!等我看着锅吃饱了饭,灶门
前睡他娘!”行者道:“兄弟,说那里话!你我在江湖上,那里不赚几两银子!把上样
的安排将来。”

那妇人满心欢喜,即叫:“看好茶来。厨下快整治东西。”遂下楼去,忙叫:“宰
鸡宰鹅,煮腌下饭。”又叫:“杀猪杀羊,今日用不了,明日也可用。看好酒。拿白
米做饭,白面捍饼。”三藏在楼上听见道:“孙二官,怎好?他去宰鸡鹅,杀猪羊,
倘送将来,我们都是长斋,那个敢吃?”行者道:“我有主张。”去那楼门边跌跌脚
道:“赵妈妈,你上来。”那妈妈上来道:“二官人有甚吩咐?”行者道:“今日且莫
杀生,我们今日斋戒。”寡妇惊讶道:“官人们是长斋,是月斋?”行者道:“俱不
是,我们唤做‘庚申斋’。今朝乃是庚申日,当斋;只过三更后,就是辛酉,便开
斋了。你明日杀生罢。如今且去安排些素的来,定照上样价钱奉上。”

那妇人越发欢喜。跑下去教:“莫宰!莫宰!取些木耳、闽笋、豆腐、面筋,园
里拔些青菜,做粉汤,发面蒸卷子,再煮白米饭,烧香茶。”咦!那些当厨的庖丁,
都是每日家做惯的手段,霎时间就安排停当,摆在楼上。又有现成的狮仙糖果,四
众任情受用。又问:“可吃素酒?”行者道:“止唐大官不用,我们也吃几杯。”寡
妇又取了一壶暖酒。他三个方才斟上,忽听得乒乓板响。行者道:“妈妈,底下倒
了甚么家火了?”寡妇道:“不是,是我小庄上几个客子送租米来晚了,教他在底
下睡;因客官到,没人使用,教他们抬轿子去院中请小娘儿陪你们。想是轿杠撞得
楼板响。”行者道:“早是说哩。快不要去请。一则斋戒日期,二则兄弟们未到。索
性明日进来,一家请个表子,在府上耍耍时,待卖了马起身。”寡妇道:“好人!好
人!又不失了和气,又养了精神。”教:“抬进轿子来,不要请去。”四众吃了酒饭。
收了家火,都散讫。

三藏在行者耳根边悄悄的道:“那里睡?”行者道:“就在楼上睡。”三藏道:“不
稳便。我们都辛辛苦苦的,倘或睡着,这家子一时再有人来收拾,见我们或滚了帽
子,露出光头,认得是和尚,嚷将起来,却怎么好?”行者道:“是啊!”又去楼前
跌跌脚。寡妇又上来道:“孙官人又有甚吩咐?”行者道:“我们在那里睡?”妇人
道:“楼上好睡。又没蚊子,又是南风。大开着窗子,忒好睡觉。”行者道:“睡不
得。我这朱三官儿有些寒湿气,沙四官儿有些漏肩风。唐大哥只要在黑处睡,我也
有些儿羞明。此间不是睡处。”

那妈妈走下去,倚着柜栏叹气。他有个女儿,抱着个孩子近前道:“母亲,常
言道:‘十日滩头坐,一日行九滩。’如今炎天,虽没甚买卖,到交秋时,还做不了
的生意哩。你嗟叹怎么?”妇人道:“儿啊,不是愁没买卖。今日晚间,已是将收
铺子,入更时分,有这四个马贩子来赁店房,他要上样管待。实指望赚他几钱银子,
他却吃斋,又赚不得他钱,故此嗟叹。”那女儿道:“他既吃了饭,不好往别人家去。
明日还好安排荤酒,如何赚不得他钱?”妇人又道:“他都有病,怕风,羞亮,都
要在黑处睡。你想家中都是些单浪瓦儿的房子,那里去寻黑暗处?不若舍一顿饭与
他吃了,教他往别家去罢。”女儿道:“母亲,我家有个黑处,又无风色,甚好,甚
好。”妇人道:“是那里?”女儿道:“父亲在日曾做了一张大柜。那柜有四尺宽,
七尺长,三尺高下,里面可睡六七个人。教他们往柜里睡去罢。”妇人道:“不知可
好,等我问他一声。孙官人,舍下蜗居,更无黑处,止有一张大柜,不透风,又不
透亮,往柜里睡去如何?”行者道:“好!好!好!”即着几个客子把柜抬出,打开盖
儿,请他们下楼。行者引着师父,沙僧拿担,顺灯影后径到柜边。八戒不管好歹,
就先进柜去。沙僧把行李递入,搀着唐僧进去,沙僧也到里边。行者道:“我的
马在那里?”旁有伏侍的道:“马在后屋拴着吃草料哩。”行者道:“牵来。把槽抬
来,紧挨着柜儿拴住。”方才进去,叫:“赵妈妈,盖上盖儿,插上锁钉,锁上锁子,
还替我们看看,那里透亮,使些纸儿糊糊,明日早些儿来开。”寡妇道:“忒小心了!”
遂此各各关门去睡不题。

却说他四个到了柜里。可怜啊!一则乍戴个头巾,二来天气炎热,又闷住了气,
略不透风,他都摘了头巾,脱了衣服,又没把扇子,只将僧帽扑扑扇扇。你挨着我,
我挤着你。直到有二更时分,却都睡着。惟行者有心闯祸,偏他睡不着,伸过手,
将八戒腿上一捻。那呆子缩了脚,口里哼哼的道:“睡了罢!辛辛苦苦的,有甚么心
肠还捻手捻脚的耍子?”行者捣鬼道:“我们原来的本身是五千两,前者马卖了三
千两,如今两搭联里现有四千两,这一群马还卖他三千两,也有一本一利。够了,
够了!”八戒要睡的人,那里答对。

岂知他这店里走堂的,挑水的,烧火的,素与强盗一伙。听见行者说有许多银
子,他就着几个溜出去,伙了二十多个贼,明火执杖的来打劫马贩子。冲开门进来,
唬得那赵寡妇娘女们战战兢兢的关了房门,尽他外边收拾。原来那贼不要店中家火,
只寻客人。到楼上不见形迹,收着火把,四下照看,只见天井中一张大柜,柜脚上
拴着一匹白马,柜盖紧锁,掀翻不动。众贼道:“走江湖的人,都有手眼。看这柜
势重,必是行囊财帛锁在里面。我们偷了马,抬柜出城,打开分用,却不是好?”
那些贼果找起绳扛,把柜抬着就走,幌阿幌的。八戒醒了道:“哥哥,睡罢。摇甚
么?”行者道:“莫言语!没人摇。”三藏与沙僧忽地也醒了,道:“是甚人抬着我们
哩?”行者道:“莫嚷,莫嚷!等他抬!抬到西天,也省得走路。”

那贼得了手,不往西去,倒抬向城东,杀了守门的军,打开城门出去。当时就
惊动六街三市,各铺上火甲人夫,都报与巡城总兵、东城兵马司。那总兵、兵马,
事当干己,即点人马弓兵,出城赶贼。那贼见官军势大,不敢抵敌,放下大柜,丢
了白马,各自落草逃走。众官军不曾拿得半个强盗,只是夺下柜,捉住马,得胜而
回。总兵在灯光下,见那马,好马:

鬃分银线,尾玉条。说甚么八骏龙驹,赛过了款段。千金市骨,万里追
风。登山每与青云合,啸月浑如白雪匀。真是蛟龙离海岛,人间喜有玉麒麟。
总兵官把自家马儿不骑,就骑上这个白马,帅军兵进城,把柜子抬在总府,同兵马
写个封皮封了,令人巡守,待天明启奏,请旨定夺。官军散讫不题。

却说唐长老在柜里埋怨行者道:“你这个猴头,害杀我也!若在外边,被人拿住,
送与灭法国王,还好折辨;如今锁在柜里,被贼劫去,又被官军夺来,明日见了国
王,现现成成的开刀请杀,却不凑了他一万之数?”行者道:“外面有人!打开柜,
拿出来不是捆着,便是吊着。且忍耐些儿,免了捆吊。明日见那昏君,老孙自有对
答,管你一毫儿也不伤。且放心睡睡。”

挨到三更时分,行者弄个手段,顺出棒来,吹口仙气,叫“变!”即变做三尖
头的钻儿,挨柜脚两三钻,钻了一个眼子。收了钻,摇身一变,变做个蝼蚁儿,
将出去。现原身,踏起云头,径入皇宫门外。那国王正在睡浓之际。他使个“大分
身普会神法”,将左臂上毫毛都拔下来,吹口仙气,叫:“变!”都变做小行者。右
臂上毛,也都拔下来,吹口仙气,叫“变!”都变做瞌睡虫;念一声“”字真言,
教当坊土地,领众布散皇宫内院,五府六部,各衙门大小官员宅内,但有品职者,
都与他一个瞌睡虫,人人稳睡,不许翻身。又将金箍棒取在手中,掂一掂,幌一幌,
叫声“宝贝,变!”即变做千百口剃头刀儿;他拿一把,吩咐小行者各拿一把,都
去皇宫内院、五府六部、各衙门里剃头。咦!这才是:
法王灭法法无穷,法贯乾坤大道通。
万法原因归一体,三乘妙相本来同。
钻开玉柜明消息,布散金毫破蔽蒙。
管取法王成正果,不生不灭去来空。
这半夜剃削成功。念动咒语,喝退土地神。将身一抖,两臂上毫毛归伏。将剃头
刀总捻成真,依然认了本性,还是一条金箍棒,收来些小之形,藏于耳内。复翻身
还做蝼蚁,钻入柜内。现了本相,与唐僧守困不题。

却说那皇宫内院,宫娥彩女,天不亮起来梳洗,一个个都没了头发。穿宫的大
小太监,也都没了头发。一拥齐来,到于寝宫外,奏乐惊寝,个个噙泪,不敢传言。
少时,那三宫皇后醒来,也没了头发。忙移灯到龙床下看处,锦被窝中,睡着一个
和尚,皇后忍不住言语出来,惊醒国王。那国王急睁睛,见皇后的头光,他连忙爬
起来道:“梓童,你如何这等?”皇后道:“主公亦如此也。”那皇帝摸摸头,唬得
三尸呻咋,七魄飞空,道:“朕当怎的来耶!”正慌忙处,只见那六院嫔妃,宫娥彩
女,大小太监,皆光着头跪下道:“主公,我们做了和尚耶!”国王见了,眼中流泪
道:“想是寡人杀害和尚……”即传旨吩咐:“汝等不得说出落发之事,恐文武群臣,
褒贬国家不正。且都上殿设朝。”

却说那五府六部,合衙门大小官员,天不明都要去朝王拜阙。原来这半夜一个
个也没了头发。各人都写表启奏此事。只听那:
静鞭三响朝皇帝,表奏当今剃发因。

毕竟不知那总兵官夺下柜里贼赃如何,与唐僧四众的性命如何,且听下回分解。