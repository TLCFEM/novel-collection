\chapter{心主夜间修药物~君王筵上论妖邪}

话表孙大圣同近侍宦官,到于皇宫内院,直至寝宫门外立定。将三条金线与宦
官拿入里面,吩咐:“教内宫妃后,或近侍太监,先系在圣躬左手腕下,按寸、关、
尺三部上,却将线头从窗儿穿出与我。”真个那宦官依此言,请国王坐在龙床,
按寸、关、尺,以金线一头系了,一头理出窗外。

行者接了线头,以自己右手大指先托着食指,看了寸脉;次将中指按大指,看
了关脉;又将大指托定无名指,看了尺脉;调停自家呼吸,分定四气、五郁、七表、
八里、九候、浮中沉,沉中浮,辨明了虚实之端;又教解下左手,依前系在右手腕
下部位。行者即以左手指,一一从头诊视毕,却将身抖了一抖,把金线收上身来。
厉声高呼道:“陛下左手寸脉强而紧,关脉涩而缓,尺脉芤且沉;右手寸脉浮而滑,
关脉迟而结,尺脉数而牢。夫左寸强而紧者,中虚心痛也;关涩而缓者,汗出肌麻
也;尺芤而沉者,小便赤而大便带血也。右手寸脉浮而滑者,内结经闭也;关迟而
结者,宿食留饮也;尺数而牢者,烦满虚寒相持也。诊此贵恙:是一个惊恐忧思,
号为‘双鸟失群’之证。”那国王在内闻言,满心欢喜。打起精神,高声应道:“指
下明白!指下明白!果是此疾!请出外面用药来也。”

大圣却才缓步出宫。早有在旁听见的太监,已先对众报知。须臾,行者出来,
唐僧即问如何。行者道:“诊了脉,如今对证制药哩。”众官上前道:“神僧长老,
适才说‘双鸟失群’之证,何也?”行者笑道:“有雌雄二鸟,原在一处同飞,忽
被暴风骤雨惊散,雌不能见雄,雄不能见雌,雌乃想雄,雄亦想雌:这不是‘双鸟
失群’也?”众官闻说,齐声喝采道:“真是神僧!真是神医!”称赞不已。当有太
医官问道:“病势已看出矣,但不知用何药治之?”行者道:“不必执方,见药就要。”
医官道:“经云:‘药有八百八味,人有四百四病。’病不在一人之身,药岂有全用
之理!如何见药就要?”行者道:“古人云:‘药不执方,合宜而用。’故此全征药品,
而随便加减也。”那医官不复再言。即出朝门之外,差本衙当值之人,遍晓满城生
熟药铺,即将药品,每味各办三斤,送与行者。行者道:“此间不是制药处,可将
诸药之数并制药一应器皿,都送入会同馆,交与我师弟二人收下。”医官听命,即
将八百八味每味三斤及药碾、药磨、药罗、药乳并乳钵、乳槌之类都送至馆中,一
一交付收讫。

行者往殿上请师父同至馆中制药。那长老正自起身,忽见内宫传旨,教阁下留
住法师,同宿文华殿。待明朝服药之后,病痊酬谢,倒换关文送行。三藏大惊道:
“徒弟啊,此意是留我做当头哩。若医得好,欢喜起送;若医不好,我命休矣。你
须仔细上心,精虔制度也!”行者笑道:“师父放心,在此受用。老孙自有医国之手。”

好大圣,别了三藏,辞了众臣,径至馆中。八戒迎着笑道:“师兄,我知道你
了。”行者道:“你知甚么?”八戒道:“知你取经之事不果,欲作生涯无本,今日
见此处富庶,设法要开药铺哩。”行者喝道:“莫胡说!医好国王,得意处辞朝走路,
开甚么药铺!”八戒道:“终不然,这八百八味药,每味三斤,共计二千四百二十四
斤,只医一人,能用多少?不知多少年代方吃得了哩!”行者道:“那里用得许多?他
那太医院官都是些愚盲之辈,所以取这许多药品,教他没处捉摸,不知我用的是那
几味,难识我神妙之方也。”

正说处,只见两个馆使,当面跪下道:“请神僧老爷进晚斋。”行者道:“早间
那般待我,如今却跪而请之,何也?”馆使叩头道:“老爷来时,下官有眼无珠,
不识尊颜。今闻老爷大展三折之肱,治我一国之主,若主上病愈,老爷江山有分,
我辈皆臣子也,礼当拜请。”行者见说,欣然登堂上坐。八戒、沙僧分坐左右。摆
上斋来。沙僧便问道:“师兄,师父在那里哩?”行者笑道:“师父被国王留住作当
头哩。只待医好了病,方才酬谢送行。”沙僧又问:“可有些受用么?”行者道:“国
王岂无受用!我来时,他已有三个阁老陪侍左右,请入文华殿去也。”八戒道:“这
等说,还是师父大哩。他倒有阁老陪侍,我们只得两个馆使奉承。且莫管他,让老
猪吃顿饱饭也。”兄弟们遂自在受用一番。

天色已晚。行者叫馆使:“收了家火,多办些油蜡,我等到夜静时,方好制药。”
馆使果送若干油蜡,各命散讫。

至半夜,天街人静,万籁无声。八戒道:“哥哥,制何药?赶早干事。我瞌睡了。”
行者道:“你将大黄取一两来,碾为细末。”沙僧乃道:“大黄味苦,性寒,无毒,
其性沉而不浮,其用走而不守,夺诸郁而无壅滞,定祸乱而致太平,名之曰‘将军’。
此行药耳。但恐久病虚弱,不可用此。”行者笑道:“贤弟不知。此药利痰顺气,荡
肚中凝滞之寒热。你莫管我。你去取一两巴豆,去壳去膜,捶去油毒,碾为细末来。”
八戒道:“巴豆味辛,性热,有毒;削坚积,荡肺腑之沉寒;通闭塞,利水谷之道
路;乃斩关夺门之将,不可轻用。”行者道:“贤弟,你也不知。此药破结宣肠,能
理心膨水胀。快制来。我还有佐使之味辅之也。”他二人即时将二药碾细道:“师兄,
还用那几十味?”行者道:“不用了。”八戒道:“八百八味,每味三斤,只用此二
两,诚为起夺人了。”行者将一个花磁盏子,道:“贤弟莫讲,你拿这个盏儿,将锅
脐灰刮半盏过来。”八戒道:“要怎的?”行者道:“药内要用。”沙僧道:“小弟不
曾见药内用锅灰。”行者道:“锅灰名为‘百草霜’,能调百病,你不知道。”那呆子
真个刮了半盏,又碾细了。行者又将盏子,递与他道:“你再去把我们的马尿等半
盏来。”八戒道:“要他怎的?”行者道:“要丸药。”沙僧又笑道:“哥哥,这事不
是耍子。马尿腥臊,如何入得药品?我只见醋糊为丸,陈米糊为丸,炼蜜为丸,或
只是清水为丸,那曾见马尿为丸?那东西腥腥臊臊,脾虚的人,一闻就吐;再服巴
豆、大黄,弄得人上吐下泻,可是耍子?”行者道:“你不知就里。我那马,不是
凡马。他本是西海龙身。若得他肯去便溺,凭你何疾,服之即愈。但急不可得耳。”
八戒闻言,真个去到马边。那马斜伏地下睡哩。呆子一顿脚踢起,衬在肚下,等了
半会,全不见撒尿。他跑将来,对行者说:“哥啊,且莫去医皇帝,且快去医医马
来。那亡人干结了,莫想尿得出一点儿!”行者笑道:“我和你去。”沙僧道:“我也
去看看。”

三人都到马边,那马跳将起来,口吐人言,厉声高叫道:“师兄,你岂不知?我
本是西海飞龙,因为犯了天条,观音菩萨救了我,将我锯了角,退了鳞,变作马,
驮师父往西天取经,将功折罪。我若过水撒尿,水中游鱼,食了成龙;过山撒尿,
山中草头得味,变作灵芝,仙僮采去长寿;我怎肯在此尘俗之处轻抛却也?”行者
道:“兄弟谨言。此间乃西方国王,非尘俗也,亦非轻抛弃也。常言道:‘众毛攒裘。’
要与本国之王治病哩。医得好时,大家光辉。不然,恐俱不得善离此地也。”那马
才叫声:“等着。”你看他往前扑了一扑,往后蹲了一蹲,咬得那满口牙支支的响
,仅努出几点儿,将身立起。八戒道:“这个亡人!就是金汁子,再撒些儿也罢!”
那行者见有少半盏,道:“够了,够了!拿去罢。”沙僧方才欢喜。

三人回至厅上,把前项药饵搅和一处,搓了三个大丸子。行者道:“兄弟,忒
大了。”八戒道:“只有核桃大。若论我吃,还不够一口哩!”遂此收在一个小盒儿
里。兄弟们连衣睡下,一夜无词。

早是天晓。却说那国王耽病设朝,请唐僧见了,即命众官快往会同馆参拜神僧
孙长老取药去。

多官随至馆中,对行者拜伏于地道:“我王特命臣等拜领妙剂。”行者叫八戒取
盒儿,揭开盖子,递与多官。多官启问:“此药何名?好见王回话。”行者道:“此名
‘乌金丹’。”八戒二人,暗中作笑道:“锅灰拌的,怎么不是乌金!”多官又问道:
“用何引子?”行者道:“药引儿两般都下得。有一般易取者,乃六物煎汤送下。”
多官问:“是何六物?”行者道:

“半空飞的老鸦屁,紧水负的鲤鱼尿,王母娘娘搽脸粉,老君炉里炼丹灰,玉
皇戴破的头巾要三块,还要五根困龙须:六物煎汤送此药,你王忧病等时除。”
多官闻言道:“此物乃世间所无者。请问那一般引子是何?”行者道:“用无根水送
下。”众官笑道:“这个易取。”行者道:“怎见得易取?”多官道:“我这里人家俗
论:若用无根水,将一个碗盏,到井边,或河下,舀了水,急转步,更不落地,亦
不回头,到家与病人吃药,便是。”行者道:“井中河内之水,俱是有根的。我这无
根水,非此之论,乃是天上落下者,不沾地就吃,才叫做‘无根水’。”多官又道:
“这也容易。等到天阴下雨时,再吃药便罢了。”遂拜谢了行者,将药持回献上。

国王大喜,即命近侍接上来。看了道:“此是甚么丸子?”多官道:“神僧说是
‘乌金丹’,用无根水送下。”国王便教宫人取无根水。众官道:“神僧说,无根水
不是井河中者,乃是天上落下不沾地的才是。”国王即唤当驾官传旨,教请法官求
雨。众官遵依出榜不题。

却说行者在会同馆厅上,叫猪八戒道:“适间允他天落之水,才可用药,此时
急忙,怎么得个雨水?我看这王,倒也是个大贤大德之君,我与你助他些儿雨下药,
如何?”八戒道:“怎么样助?”行者道:“你在我左边立下,做个辅星。”又叫沙
僧,“你在我右边立下,做个弼宿。等老孙助他些无根水儿。”好大圣,步了罡袂,
念声咒语。早见那正东上,一朵乌云,渐近于头顶上。叫道:“大圣,东海龙王敖
广来见。”行者道:“无事不敢捻烦,请你来助些无根水与国王下药。”龙王道:“大
圣呼唤时,不曾说用水,小龙只身来了,不曾带得雨器,亦未有风云雷电,怎生降
雨?”行者道:“如今用不着风云雷电,亦不须多雨,只要些须引药之水便了。”龙
王道:“既如此,待我打两个喷涕,吐些涎津溢,与他吃药罢。”行者大喜道:“最
好,最好!不必迟疑,趁早行事。”

那老龙在空中,渐渐低下乌云,直至皇宫之上,隐身潜象,一口津唾,遂化
作甘霖。那满朝官齐声喝采道:“我主万千之喜!天公降下甘雨来也!”国王即传旨,
教:“取器皿盛着。不拘宫内外及官大小,都要等贮仙水,拯救寡人。”

你看那文武多官并三宫六院妃嫔与三千彩女,八百娇娥,一个个擎杯托盏,举
碗持盘,等接甘雨。那老龙在半空,运化津涎,不离了王宫前后。将有一个时辰,
龙王辞了大圣回海。众臣将杯盂碗盏收来,也有等着一点两点者,也有等着三点五
点者,也有一点不曾等着者,共合一处,约有三盏之多,总献至御案。真个是异香
满袭金銮殿,佳味熏飘天子庭!

那国王辞了法师,将着“乌金丹”并甘雨至宫中,先吞了一丸,吃了一盏甘雨;
再吞了一丸,又饮了一盏甘雨;三次,三丸俱吞了,三盏甘雨俱送下。不多时,腹
中作响,如辘轳之声不绝;即取净桶,连行了三五次;服了些米饮,倒在龙床之
上。有两个妃子,将净桶捡看,说不尽那秽污痰涎,内有糯米饭块一团。妃子近龙
床前来报:“病根都行下来也!”国王闻此言,甚喜,又进一次米饭。少顷,渐觉心
胸宽泰,气血调和,就精神抖擞,脚力强健。下了龙床,穿上朝服,即登宝殿,见
了唐僧,辄倒身下拜。那长老忙忙还礼。拜毕,以御手搀着,便教阁下:“快具简
帖,帖上写朕‘再拜顿首’字样,差官奉请法师高徒三位。一壁厢大开东阁,光禄
寺排宴酬谢。”多官领旨,具简的具简,排宴的排宴,正是国家有倒山之力,霎时
俱完。

却说八戒见官投简,喜不自胜道:“哥啊,果是好妙药!今来酬谢,乃兄长之功。”
沙僧道:“二哥说那里话!常言道:‘一人有福,带挈一屋。’我们在此合药,俱是有
功之人。只管受用去,再休多话。”咦!你看他弟兄们俱欢欢喜喜,径入朝来。

众官接引,上了东阁,早见唐僧、国王、阁老,已都在那里安排筵宴哩。这行
者与八戒、沙僧,对师父唱了个喏,随后众官都至。只见那上面有四张素桌面,都
是吃一看十的筵席;前面有一张荤桌面,也是吃一看十的珍馐。左右有四五百张单
桌面,真个排得齐整:

古云:珍馐百味,美禄千钟。琼膏酥酪,锦缕肥红。宝妆花彩艳,果品味香浓。
斗糖龙缠列狮仙,饼锭拖炉摆凤侣。荤有猪羊鸡鹅鱼鸭般般肉,素有蔬肴笋芽木耳
并蘑菇。几样香汤饼,数次透酥糖。滑软黄粱饭,清新菇米糊。色色粉汤香又辣,
般般添换美还甜。君臣举盏方安席,名分品级慢传壶。
那国王御手擎杯,先与唐僧安坐。三藏道:“贫僧不会饮酒。”国王道:“素酒。法
师饮此一杯,何如?”三藏道:“酒乃僧家第戒。”国王甚不过意道:“法师戒饮,
却以何物为敬?”三藏道:“顽徒三众代饮罢。”国王却才欢喜,转金卮,递与行者。
行者接了酒,对众礼毕,吃了一杯。国王见他吃得爽利,又奉一杯。行者不辞,又
吃了。国王笑道:“吃个三宝钟儿。”行者不辞,又吃了。国王又叫斟上,“吃个四
季杯儿。”

八戒在旁,见酒不到他,忍得他咽唾;又见那国王苦劝行者,他就叫将起
来道:“陛下,吃的药也亏了我,那药里有马……”这行者听说,恐怕呆子走了消
息,却将手中酒递与八戒。八戒接着就吃,却不言语。国王问道:“神僧说药里有
马,是甚么马?”行者接过口来道:“我这兄弟,是这般口敞。但有个经验的好方
儿,他就要说与人。陛下早间吃药,内有马兜铃。”国王问众官道:“马兜铃是何品
味?能医何证?”时有太医院官在旁道:“主公:
兜铃味苦寒无毒,定喘消痰大有功。
通气最能除血蛊,补虚宁嗽又宽中。”
国王笑道:“用得当,用得当!猪长老再饮一杯。”呆子亦不言语,却也吃了个三宝
钟。国王又递了沙僧酒,也吃了三杯,却俱叙坐。

饮宴多时,国王又擎大爵,奉与行者。行者道:“陛下请坐。老孙依巡痛饮,
决不敢推辞。”国王道:“神僧恩重如山,寡人酬谢不尽。好歹进此一巨觥,朕有话
说。”行者道:“有甚话说了,老孙好饮。”国王道:“寡人有数载忧疑病,被神僧一
贴灵丹打通,所以就好了。”行者笑道:“昨日老孙看了陛下,已知是忧疑之疾,但
不知忧惊何事?”国王道:“古人云:‘家丑不可外谈。’奈神僧是朕恩主,惟不笑,
方可告之。”行者道:“怎敢笑话,请说无妨。”

国王道:“神僧东来,不知经过几个邦国?”行者道:“经有五六处。”又问:“他
国之后,不知是何称呼。”行者道:“国王之后,都称为正宫、东宫、西宫。”国王
道:“寡人不是这等称呼:将正宫称为金圣宫,东宫称为玉圣宫,西宫称为银圣宫。
现今只有银、玉二后在宫。”行者道:“金圣宫因何不在宫中?”国王滴泪道:“不
在已三年矣。”行者道:“向那厢去了?”国王道:“三年前,正值端阳之节,朕与
嫔后都在御花园海榴亭下解粽插艾,饮菖蒲雄黄酒,看斗龙舟。忽然一阵风至,半
空中现出一个妖精,自称赛太岁,说他在麒麟山獬豸洞居住,洞中少个夫人,访得
我金圣宫生得貌美姿娇,要做个夫人,教朕快早送出。如若三声不献出来,就要先
吃寡人,后吃众臣,将满城黎民,尽皆吃绝。那时节,朕却忧国忧民,无奈,将金
圣宫推出海榴亭外,被那妖响一声摄将去了。寡人为此着了惊恐,把那粽子凝滞在
内;况又昼夜忧思不息,所以成此苦疾三年。今得神僧灵丹服后,行了数次,尽是
那三年前积滞之物,所以这会体健身轻,精神如旧。今日之命,皆是神僧所赐,岂
但如泰山之重而已乎!”

行者闻得此言,满心喜悦,将那巨觥之酒,两口吞之,笑问国王曰:“陛下原
来是这等惊忧!今遇老孙,幸而获愈。但不知可要金圣宫回国?”那国王滴泪道:“朕
切切思思,无昼无夜,但只是没一个能获得妖精的。岂有不要他回国之理!”行者
道:“我老孙与你去伏妖邪,那时何如?”国王跪下道:“若救得朕后,朕愿领三宫
九嫔,出城为民,将一国江山,尽付神僧,让你为帝。”八戒在旁,见出此言,行
此礼,忍不住呵呵大笑道:“这皇帝失了体统!怎么为老婆就不要江山,跪着和尚?”
行者急上前,将国王搀起道:“陛下,那妖精自得金圣宫去后,这一向可曾再来?”
国王道:“他前年五月节摄了金圣宫,至十月间来,要取两个宫娥,是说伏侍娘娘,
朕即献出两个。至旧年三月间,又来要两个宫娥;七月间,又要去两个;今年二月
里,又要去两个;不知到几时又要来也。”行者道:“似他这等频来,你们可怕他么?”
国王道:“寡人见他来得多遭,一则惧怕,二来又恐有伤害之意,旧年四月内,是
朕命工起了一座避妖楼,但闻风响,知是他来,即与二后、九嫔,入楼躲避。”行
者道:“陛下不弃,可携老孙去看那避妖楼一番,何如?”那国王即将左手携着行
者出席。众官亦皆起身。猪八戒道:“哥哥,你不达理!这般御酒不吃,摇席破坐的,
且去看甚么哩?”国王闻说,情知八戒是为嘴,即命当驾官抬两张素桌面,看酒在
避妖楼外伺候。呆子却才不嚷,同师父、沙僧笑道:“翻席去也。”

一行文武官引导,那国王并行者相搀,穿过皇宫到了御花园内,更不见楼台殿
阁。行者道:“避妖楼何在?”说不了,只见两个太监,拿两根红漆扛子,往那空
地上掬起一块四方石板。国王道:“此间便是。这底下有三丈多深,成的九间朝
殿。内有四个大缸,缸内满注清油,点着灯火,昼夜不息。寡人听得风响,就入里
边躲避,外面着人盖上石板。”行者笑道:“那妖精还是不害你;若要害你,这里如
何躲得?”正说间,只见那正南上,呼呼的,吹得风响,播土扬尘。唬得那多官齐
声报怨道:“这和尚盐酱口,讲起甚么妖精,妖精就来了!”慌得那国王丢了行者,
即钻入地穴。唐僧也就跟入。众官亦躲个干净。

八戒、沙僧也都要躲,被行者左右手扯住他两个道:“兄弟们,不要怕得。我
和你认他一认,看是个甚么妖精。”八戒道:“可是扯淡!认他怎的?众官躲了,师父
藏了,国王避了,我们不去了罢,炫的是那家世!”那呆子左挣右挣,挣不得脱手,
被行者拿定多时,只见那半空里闪出一个妖精。你看他怎生模样:

九尺长身多恶狞,一双环眼闪金灯。两轮查耳如撑扇,四个钢牙似插钉。鬓绕
红毛眉竖焰,鼻垂糟准孔开明。髭髯几缕朱砂线,颧骨满面青。两臂红筋蓝靛
手,十条尖爪把枪擎。豹皮裙子腰间系,赤脚蓬头若鬼形。
行者见了道:“沙僧,你可认得他?”沙僧道:“我又不曾与他相识,那里认得!”
又问:“八戒,你可认得他?”八戒道:“我又不曾与他会茶会酒,又不是宾朋邻里,
我怎么认得他!”行者道:“他却像东岳天齐手下把门的那个醮面金睛鬼。”八戒道:
“不是,不是!”行者道:“你怎知他不是?”八戒道:我岂不知,鬼乃阴灵也,一
日至晚,交申酉戌亥时方出。今日还在巳时,那里有鬼敢出来?就是鬼,也不会驾
云。纵会弄风,也只是一阵旋风耳,有这等狂风?或者他就是赛太岁也。”行者笑道:
“好呆子,倒也有些论头!既如此说,你两个护持在此,等老孙去问他个名号,好
与国王救取金圣宫来朝。”八戒道:“你去自去,切莫供出我们来。”行者昂然不答,
急纵祥光,跳将上去。咦!正是:
安邦先却君王病,守道须除爱恶心。

毕竟不知此去,到于空中,胜败如何,怎么擒得妖怪,救得金圣宫,且听下回
分解。