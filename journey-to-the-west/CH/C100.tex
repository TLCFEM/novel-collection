\chapter{东土~五圣成真}

且不言他四众脱身,随金刚驾风而起。却说陈家庄救生寺内多人,天晓起来,
仍治果肴来献,至楼下,不见了唐僧。这个也来问,那个也来寻,俱慌慌张张,莫
知所措,叫苦连天的道:“清清把个活佛放去了!”一会家无计,将办来的品物,俱
抬在楼上祭祀烧纸。以后每年四大祭,二十四小祭。还有那告病的,保安的,求亲
许愿,求财求子的,无时无日,不来烧香祭赛。真个是金炉不断千年火,玉盏常明
万载灯。不题。

却说八大金刚使第二阵香风,把他四众,不一日,送至东土,渐渐望见长安。
原来那太宗自贞观十三年九月望前三日送唐僧出城,至十六年,即差工部官在西安
关外起建了望经楼接经。太宗年年亲至其地。

恰好那一日出驾复到楼上,忽见正西方满天瑞霭,阵阵香风,金刚停在空中叫
道:“圣僧,此间乃长安城了。我们不好下去,这里人伶俐,恐泄漏吾像。孙大圣
三位也不消去,汝自去传了经与汝主,即便回来。我在霄汉中等你,与你一同缴旨。”
大圣道:“尊者之言虽当,但吾师如何挑得经担,如何牵得这马,须得我等同去一
送。烦你在空少等,谅不敢误。”金刚道:“前日观音菩萨启过如来,往来只在八
日,方完藏数。今已经四日有余,只怕八戒贪图富贵,误了期限。”八戒笑道:“师
父成佛,我也望成佛,岂有贪图之理!泼大粗人,都在此等我,待交了经,就来与你
回向也!”呆子挑着担,沙僧牵着马,行者领着圣僧,都按下云头,落于望经楼边。

太宗同多官一齐见了,即下楼相迎道:“御弟来也?”唐僧即倒身下拜。太宗
搀起,又问:“此三者何人?”唐僧道:“是途中收的徒弟。”太宗大喜,即命侍
官:“将朕御车马扣背,请御弟上马,同朕回朝。”唐僧谢了恩,骑上马。大圣轮
金箍棒紧随。八戒、沙僧俱扶马挑担,随驾后共入长安。真个是:
当年清宴乐升平,文武安然显俊英。
水陆场中僧演法,金銮殿上主差卿。
关文敕赐唐三藏,经卷原因配五行。
苦炼凶魔种种灭,功成今喜上朝京。

唐僧四众,随驾入朝。满城中无一不知是取经人来了。却说那长安唐僧旧住的
洪福寺大小僧人,看见几株松树一颗颗头俱向东,惊讶道:“怪哉,怪哉!今夜未曾
刮风,如何这树头都扭过来了?”内有三藏的旧徒道:“快拿衣服来!取经的老师父
来了!”众僧问道:“你何以知之?”旧徒曰:“当年师父去时,曾有言道:‘我
去之后,或三五年,或六七年,但看松树枝头若是东向,我即回矣。’我师父佛口
圣言,故此知之。”急披衣而出。至西街时,早已有人传播说:“取经的人适才方
到,万岁爷爷接入城来了。”众僧听说,又急急跑来,却就遇着。一见大驾,不敢
近前,随后跟至朝门之外。

唐僧下马,同众进朝。唐僧将龙马与经担,同行者、八戒、沙僧,站在玉阶之
下。太宗传宣:“御弟上殿。”赐坐。唐僧又谢恩坐了,教把经卷抬来。行者等取
出,近侍官传上。太宗又问:“多少经数?怎生取来?”三藏道:“臣僧到了灵山,
参见佛祖,蒙差阿傩、伽叶二尊者先引至珍楼内赐斋,次到宝阁内传经。那尊者需
索人事,因未曾备得,不曾送他,他遂以经与了。当谢佛祖之恩,东行,忽被妖风
抢了经去。幸小徒有些神通赶夺,却俱抛掷散漫。因展看,皆是无字空本。臣等着
惊,复去拜告恳求。佛祖道:‘此经成就之时,有比丘圣僧将下山与舍卫国赵长者
家看诵了一遍,保他家生者安全,亡者超脱,止讨了他三斗三升米粒黄金,意思
还嫌卖贱了,后来子孙没钱使用。’我等知二尊者需索人事,佛祖明知,只得将钦
赐紫金钵盂送他,方传了有字真经。此经有三十五部。各部中检了几卷传来,共计
五千零四十八卷。此数盖合一藏也。”

太宗更喜,教:“光禄寺设宴开东阁酬谢。”忽见他三徒立在阶下,容貌异常,
便问:“高徒果外国人耶?”长老俯伏道:“大徒弟姓孙,法名悟空,臣又呼他为
孙行者。他出身原是东胜神洲傲来国花果山水帘洞人氏。因五百年前大闹天宫,被
佛祖困压在西番两界山石匣之内,蒙观音菩萨劝善,情愿皈依,是臣到彼救出,甚
亏此徒保护。二徒弟姓猪,法名悟能,臣又呼他为猪八戒。他出身原是福陵山云栈
洞人氏。因在乌斯藏高老庄上作怪,即蒙菩萨劝善,亏行者收之。一路上挑担有力,
涉水有功。三徒弟姓沙,法名悟净,臣又呼他为沙和尚。他出身原是流沙河作怪者,
也蒙菩萨劝善,秉教沙门。那匹马不是主公所赐者。”太宗道:“毛片相同,如何
不是?”三藏道:“臣到蛇盘山鹰愁涧涉水,原马被此马吞之,亏行者请菩萨问此
马来历,原是西海龙王之子,因有罪,也蒙菩萨救解,教他与臣作脚力。当时变作
原马,毛片相同。幸亏他登山越岭,跋涉崎岖。去时骑坐,来时驮经,亦甚赖其力
也。”太宗闻言,称赞不已。又问:“远涉西方,端的路程多少?”三藏道:“总
记菩萨之言,有十万八千里之远。途中未曾记数。只知经过了一十四遍寒暑。日日
山,日日岭。遇林不小,遇水宽洪。还经几座国王,俱有照验印信。”叫:“徒弟,
将通关文牒取上来,对主公缴纳。”当时递上。太宗看了,乃贞观一十三年九月望
前三日给。太宗笑道:“久劳远涉。今已贞观二十七年矣。”牒文上有宝象国印,
乌鸡国印,车迟国印,西梁女国印,祭赛国印,朱紫国印,狮驼国印,比丘国印,
灭法国印;又有凤仙郡印,玉华州印,金平府印。太宗览毕,收了。

早有当驾官请宴,即下殿携手而行。又问:“高徒能礼貌乎?”三藏道:“小
徒俱是山村旷野之妖身,未谙中华圣朝之礼数。万望主公赦罪。”太宗笑道:“不
罪他,不罪他。都同请东阁赴宴去也。”三藏又谢了恩,招呼他三众,都到阁内观
看。果是中华大国,比寻常不同。你看那:

门悬彩绣,地衬红毡。异香馥郁,奇品新鲜。琥珀杯,琉璃盏,镶金点翠;黄
金盘,白玉碗,嵌锦花缠。烂煮蔓菁,糖浇香芋。蘑菇甜美,海菜清奇。几次添来
姜辣笋,数番办上蜜调葵。面筋椿树叶,木耳豆腐皮。石花仙菜,蕨粉干薇。花椒
煮莱菔,芥末拌瓜丝。几盘素品还犹可,数种奇稀果夺魁。核桃柿饼,龙眼荔枝。
宣州茧栗山东枣,江南银杏兔头梨。榛松莲肉葡萄大,榧子瓜仁菱米齐。橄榄林檎,
苹婆沙果。慈菇嫩藕,脆李杨梅。无般不备,无件不齐。还有些蒸酥蜜食兼嘉馔,
更有那美酒香茶与异奇。说不尽百味珍馐真上品,果然是中华大国异西夷。
师徒四众与文武多官,俱侍列左右。太宗皇帝仍正坐当中。歌舞吹弹,整齐严肃,
遂尽乐一日。正是:
君王嘉会赛唐虞,取得真经福有余。
千古流传千古盛,佛光普照帝王居。

当日天晚,谢恩宴散。太宗回宫,多官回宅。唐僧等归于洪福寺,只见寺僧磕
头迎接。方进山门,众僧报道:“师父,这树头儿今早俱忽然向东。我们记得师父
之言,遂出城来接。果然到了!”长老喜之不胜,遂入方丈。此时八戒也不嚷茶饭,
也不弄喧头。行者、沙僧,个个稳重。只因道果完成,自然安静。当晚睡了。

次早,太宗升朝,对群臣言曰:“朕思御弟之功,至深至大,无以为酬。一夜
无寐,口占几句俚谈,权表谢意。但未曾写出。”叫:“中书官来,朕念与你,你
一一写之。”其文云:

盖闻二仪有象,显覆载以含生;四时无形,潜寒暑以化物。是以窥天鉴地,庸
愚皆识其端;明阴洞阳,贤哲罕穷其数。然天地包乎阴阳,而易识者,以其有象也;
阴阳处乎天地,而难穷者,以其无形也。故知象显可征,虽愚不惑;形潜莫睹,在
智犹迷。况乎佛道崇虚,乘幽控寂。弘济万品,典御十方。举威灵而无上,抑神力
而无下;大之则弥于宇宙,细之则摄于毫厘。无灭无生,历千劫而亘古;若隐若显,
运百福而长今。妙道凝玄,遵之莫知其际;法流湛寂,挹之莫测其源。故知蠢蠢凡
愚,区区庸鄙,投其旨趣,能无疑惑者哉!然则大教之兴,基乎西土。腾汉庭而皎梦,
照东域而流慈。古者,分形分迹之时,言未驰而成化;当常见常隐之世,民仰德而
知遵。及乎晦影归真,迁移越世,金容掩色,不镜三千之光;丽象开图,空端四八
之相。于是微言广被,拯禽类于三途;遗训遐宣,导群生于十地。佛有经,能分大
小之乘;更有法,传讹邪正之术。我僧玄奘法师者,法门之领袖也。幼怀慎敏,早
悟三空之功;长契神清,先包四忍之行。松风水月,未足比其清华;仙露明珠,讵
能方其朗润!故以智通无累,神测未形。超六尘而迥出,使千古而传芳。凝心内境,
悲正法之陵迟;栖虑玄门,慨深文之讹谬。思欲分条振理,广彼前闻;截伪续真,
开兹后学。是以翘心净土,法游西域。乘危远迈,策杖孤征。积雪晨飞,途间失地;
惊沙夕起,空外迷天。万里山川,拨烟霞而进步;百重寒暑,蹑霜雨而前踪。诚重
劳轻,求深欲达。周游西宇,十有四年。穷历异邦,询求正教。双林八水,味道餐
风;鹿苑鹫峰,瞻奇仰异。承至言于先圣,受真教于上贤。探妙门,精穷奥业。
三乘六律之道,驰骤于心田;一藏百箧之文,波涛于海口。爰自所历之国无涯,求
取之经有数。总得大乘要文,凡三十五部,计五千四十八卷,译布中华,宣扬胜业。
引慈云于西极,注法雨于
东陲。圣教缺而复全,苍生罪而还福。湿火宅之干焰,共拔迷途;朗金水之昏波,
同臻彼岸。是知恶因业坠,善以缘升。升坠之端,惟人自作。譬之桂生高岭,云露
方得泫其花;莲出绿波,飞尘不能染其叶。非莲性自洁而桂质本贞,良由所附者高,
则微物不能累;所凭者净,则浊类不能沾。夫以卉木无知,犹资善而成善,矧乎人
伦有识,不缘庆而成庆?方冀真经传布,并日月而无穷;景福遐敷,与乾坤而永大也
欤!
写毕,即召圣僧。此时长老已在朝门外候谢。闻宣急入,行俯伏之礼。太宗传请上
殿,将文字递与长老。览遍,复下谢恩,奏道:“主公文辞高古,理趣渊微。但不
知是何名目。”太宗道:“朕夜口占,答谢御弟之意,名曰‘圣教序’。不知好否。”
长老叩头,称谢不已。太宗又曰:

“朕才愧璋,言惭金石。至于内典,尤所未闻。口占叙文,诚为鄙拙。秽翰
墨于金简,标瓦砾于珠林。循躬省虑,腼面恧心。甚不足称,虚劳致谢。”

当时多官齐贺,顶礼圣教御文,遍传内外。太宗道:“御弟将真经演诵一番,
何如?”长老道:“主公,若演真经,须寻佛地。宝殿非可诵之处。”太宗甚喜。
即问当驾官:“长安城寺,有那座寺院洁净?”班中闪上大学士萧奏道:“城中
有一雁塔寺,洁净。”太宗即令多官:“把真经各虔捧几卷,同朕到雁塔寺,请御
弟谈经去来。”多官遂各各捧着,随太宗驾幸寺中,搭起高台,铺设齐整。长老仍
命:“八戒、沙僧,牵龙马,理行囊;行者在我左右。”又向太宗道:“主公欲将
真经传流天下,须当誊录副本,方可布散。原本还当珍藏,不可轻亵。”太宗又笑
道:“御弟之言,甚当,甚当!”随召翰林院及中书科各官誊写真经。又建一寺,
在城之东,名曰誊黄寺。

长老捧几卷登台,方欲讽诵,忽闻得香风缭绕,半空中有八大金刚现身高叫道:
“诵经的,放下经卷,跟我回西去也。”这底下行者三人,连白马,平地而起。长
老亦将经卷丢下,也从台上起于九霄,相随腾空而去。慌得那太宗与多官望空下拜。
这正是:
圣僧努力取经编,西宇周流十四年。
苦历程途遭患难,多经山水受。
功完八九还加九,行满三千及大千。
大觉妙文回上国,至今东土永留传。
太宗与多官拜毕,即选高僧,就于雁塔寺里,修建水陆大会,看诵《大藏真经》,
超脱幽冥孽鬼,普施善庆。将誊录过经文,传布天下不题。

却说八大金刚,驾香风,引着长老四众,连马五口,复转灵山。连去连来,适
在八日之内。此时灵山诸神,都在佛前听讲。八金刚引他师徒进去,对如来道:“弟
子前奉金旨,驾送圣僧等,已到唐国,将经交纳,今特缴旨。”遂叫唐僧等近前受
职。

如来道:“圣僧,汝前世原是我之二徒,名唤金蝉子。因为汝不听说法,轻慢
我之大教,故贬汝之真灵,转生东土。今喜皈依,秉我迦持,又乘吾教,取去真经,
甚有功果,加升大职正果,汝为旃檀功德佛。孙悟空,汝因大闹天宫,吾以甚深法
力,压在五行山下,幸天灾满足,归于释教;且喜汝隐恶扬善,在途中炼魔降怪有
功,全终全始,加升大职正果,汝为斗战胜佛。猪悟能,汝本天河水神,天蓬元帅。
为汝蟠桃会上酗酒戏了仙娥,贬汝下界投胎,身如畜类。幸汝记爱人身,在福陵山
云栈洞造孽,喜归大教,入吾沙门,保圣僧在路,却又有顽心,色情未泯。因汝挑
担有功,加升汝职正果,做净坛使者。”八戒口中嚷道:“他们都成佛,如何把我
做个净坛使者?”如来道:“因汝口壮身慵,食肠宽大。盖天下四大部洲,瞻仰吾
教者甚多,凡诸佛事,教汝净坛,乃是个有受用的品级。如何不好——沙悟净,汝
本是卷帘大将,先因蟠桃会上打碎玻璃盏,贬汝下界,汝落于流沙河,伤生吃人造
孽,幸皈吾教,诚敬迦持,保护圣僧,登山牵马有功,加升大职正果,为金身罗汉。”
又叫那白马:“汝本是西洋大海广晋龙王之子。因汝违逆父命,犯了不孝之罪,幸
得皈身皈法,皈我沙门,每日家亏你驮负圣僧来西,又亏你驮负圣经去东,亦有功
者,加升汝职正果,为八部天龙。”

长老四众,俱各叩头谢恩。马亦谢恩讫。仍命揭谛引了马下灵山后崖,化龙池
边,将马推入池中。须臾间,那马打个展身,即退了毛皮,换了头角,浑身上长起
金鳞,腮颔下生出银须,一身瑞气,四爪祥云,飞出化龙池,盘绕在山门里擎天华
表柱上。诸佛赞扬如来的大法。孙行者却又对唐僧道:“师父,此时我已成佛,与
你一般,莫成还戴金箍儿,你还念甚么紧箍咒儿勒我?趁早儿念个松箍儿咒,脱下
来,打得粉碎,切莫叫那甚么菩萨再去捉弄他人。”唐僧道:“当时只为你难管,
故以此法制之。今已成佛,自然去矣。岂有还在你头上之理!你试摸摸看。”行者举
手去摸一摸,果然无之。此时旃檀佛、斗战佛、净坛使者、金身罗汉,俱正果了本
位。天龙马亦自归真。有诗为证,诗曰:
一体真如转落尘,合和四相复修身。
五行论色空还寂,百怪虚名总莫论。
正果旃檀皈大觉,完成品职脱沉沦。
经传天下恩光阔,五圣高居不二门。

五圣果位之时,诸众佛祖、菩萨、圣僧、罗汉、揭谛、比丘、优婆夷塞、各山
各洞的神仙、大神、丁甲、功曹、伽蓝、土地,一切得道的师仙,始初俱来听讲,
至此各归方位。你看那:

灵鹫峰头聚霞彩,极乐世界集祥云。金龙稳卧,玉虎安然。乌兔任随来往,龟
蛇凭汝盘旋。丹凤青鸾情爽爽,玄猿白鹿意怡怡。八节奇花,四时仙果。乔松古桧,
翠柏修篁。五色
梅时开时结,万年桃时熟时新。千果千花争秀,一天瑞霭纷纭。
大众合掌皈依。都念:

“南无燃灯上古佛。南无药师琉璃光王佛。南无释迦牟尼佛。南无过去未来现
在佛。南无清净喜佛。南无毗卢尸佛。南无宝幢王佛。南无弥勒尊佛。南无阿弥陀
佛。南无无量寿佛。南无接引归真佛。南无金刚不坏佛。南无宝光佛。南无龙尊王
佛。南无精进喜佛。南无宝月光佛。南无现无愚佛。南无婆留那佛。南无那罗延佛。
南无功德华佛。南无才功德佛。南无善游步佛。南无旃檀光佛。南无摩尼幢佛。南
无慧炬照佛。南无海德光明佛。南无大慈光佛。南无慈力王佛。南无贤善首佛。南
无广庄严佛。南无金华光佛。南无才光明佛。南无智慧胜佛。南无世静光佛。南无
日月光佛。南无日月珠光佛。南无慧幢胜王佛。南无妙音声佛。南无常光幢佛。南
无观世灯佛。南无法胜王佛。南无须弥光佛。南无大慧力王佛。南无金海光佛。南
无大通光佛。南无才光佛。南无旃檀功德佛。南无斗战胜佛。南无观世音菩萨。南
无大势至菩萨。南无文殊菩萨。南无普贤菩萨。南无清净大海众菩萨。南无莲池海
会佛菩萨。南无西天极乐诸菩萨。南无三千揭谛大菩萨。南无五百阿罗大菩萨。南
无比丘夷塞尼菩萨。南无无边无量法菩萨。南无金刚大士圣菩萨。南无净坛使者菩
萨。南无八宝金身罗汉菩萨。南无八部天龙广力菩萨。

如是等一切世界诸佛,
愿以此功德,庄严佛净土。上报四重恩,下济三途苦。
若有见闻者,悉发菩提心。同生极乐国,尽报此一身。
十方三世一切佛,诸尊菩萨摩诃萨,摩诃般若波罗密。”

《西游记》至此终。