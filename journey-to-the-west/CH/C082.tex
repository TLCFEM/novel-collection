\chapter{姹女求阳~元神护道}

却说八戒跳下山,寻着一条小路。依路前行,有五六里远近,忽见两个女怪,
在那井上打水。他怎么认得是两个女怪?见他头上戴一顶一尺二三寸高的篾丝鬏髻,
甚不时兴。呆子走近前,叫声:“妖怪。”那怪闻言大怒,两人互相说道:“这和尚
惫懒!我们又不与他相识,平时又没有调得嘴惯,他怎么叫我们做妖怪!”那怪恼了,
轮起抬水的杠子,劈头就打。

这呆子手无兵器,遮架不得,被他捞了几下,侮着头跑上山来道:“哥啊,回
去罢!妖怪凶!”行者道:“怎么凶?”八戒道:“山凹里两个女妖精在井上打水,我
只叫了他一声,就被他打了我三四杠子!”行者道:“你叫他做甚么的?”八戒道:
“我叫他做妖怪。”行者笑道:“打得还少。”八戒道:“谢你照顾!头都打肿了,还
说少哩!”行者道:“‘温柔天下去得,刚强寸步难移。’他们是此地之怪,我们是远
来之僧,你一身都是手,也要略温存。你就去叫他做妖怪,他不打你,打我?‘人
将礼乐为先。’”八戒道:“一发不晓得!”行者道:“你自幼在山中吃人,你晓得有
两样木么?”八戒道:“不知。是甚么木?”行者道:“一样是杨木,一样是檀木。
杨木性格甚软,巧匠取来,或雕圣像,或刻如来,装金立粉,嵌玉装花,万人烧香
礼拜,受了多少无量之福。那檀木性格刚硬,油房里取了去,做柞撒,使铁箍箍了
头,又使铁锤往下打,只因刚强,所以受此苦楚。”八戒道:“哥啊,你这好话儿,
早与我说说也好,却不受他打了。”行者道:“你还去问他个端的。”八戒道:“这去
他认得我了。”行者道:“你变化了去。”八戒道:“哥啊,且如我变了,却怎么问么?”
行者道:“你变了去,到他跟前,行个礼儿,看他多大年纪:若与我们差不多,叫
他声‘姑娘’;若比我们老些儿,叫他声‘奶奶’。”八戒笑道:“可是蹭蹬!这般许
远的田地,认得是甚么亲!”行者道:“不是认亲,要套他的话哩。若是他拿了师父,
就好下手;若不是他,却不误了我们别处干事?”八戒道:“说得有理,等我再去。”

好呆子,把钉钯撒在腰里,下山凹,摇身一变,变做个黑胖和尚。摇摇摆摆,
走近怪前,深深唱个大喏道:“奶奶,贫僧稽首了。”那两个喜道:“这个和尚却好,
会唱个喏儿,又会称道一声儿。”问道:“长老,那里来的?”八戒道:“那里来的。”
又问:“那里去的?”又道:“那里去的。”又问:“你叫做甚么名字?”又答道:“我
叫做甚么名字。”那怪笑道:“这和尚好便好,只是没来历,会说顺口话儿。”八戒
道:“奶奶,你们打水怎的?”那怪道:“和尚,你不知道:我家老夫人今夜里摄了
一个唐僧在洞内,要管待他;我洞中水不干净,差我两个来此打这阴阳交媾的好水,
安排素果素菜的筵席,与唐僧吃了,晚间要成亲哩。”

那呆子闻得此言,急抽身跑上山叫:“沙和尚,快拿将行李来,我们分了罢!”
沙僧道:“二哥,又分怎的?”八戒道:“分了便你还去流沙河吃人,我去高老庄探
亲,哥哥去花果山称圣,白龙马归大海成龙。师父已在这妖精洞内成亲哩!我们都
各安生理去也!”行者道:“这呆子又胡说了!”八戒道:“你的儿子胡说!才那两个
抬水的妖精说,安排素筵席与唐僧吃了成亲哩!”行者道:“那妖精把师父困在洞里,
师父眼巴巴的望我们去救,你却在此说这样话!”八戒道:“怎么救?”行者道:“你
两个牵着马,挑着担,我们跟着那两个女怪,做个引子,引到那门前,一齐下手。”

真个呆子只得随行。行者远远的标着那两怪,渐入深山,有一二十里远近,忽
然不见。八戒惊道:“师父是日里鬼拿去了!”行者道:“你好眼力!怎么就看出他本
相来?”八戒道:“那两个怪,正抬着水走,忽然不见,却不是个日里鬼?”行者
道:“想是钻进洞去了。等我去看。”

好大圣,急睁火眼金睛,漫山看处,果然不见动静。只见那陡崖前,有一座玲
珑剔透细妆花、堆五采、三檐四簇的牌楼。他与八戒、沙僧近前观看,上有六个大
字,乃“陷空山无底洞”。行者道:“兄弟呀,这妖精把个架子支在这里,还不知门
向那里开哩。”沙僧说:“不远,不远!好生寻!”都转身看时,牌楼下,山脚下有一
块大石,约有十余里方圆;正中间有缸口大的一个洞儿,爬得光溜溜的。八戒道:
“哥啊,这就是妖精出入洞也。”行者看了道:“怪哉!我老孙自保唐僧,瞒不得你
两个,妖精也拿了些,却不见这样洞府。八戒,你先下去试试,看有多少浅深,我
好进去救师父。”八戒摇头道:“这个难,这个难!我老猪身子夯夯的,若塌了脚吊
下去,不知二三年可得到底哩!”行者道:“就有多深么?”八戒道:“你看!”大圣
伏在洞边上,仔细往下看处,咦!深啊!周围足有三百余里,回头道:“兄弟,果然
深得紧!”八戒道:“你便回去罢。师父救不得耶!”行者道:“你说那里话,‘莫生
懒惰意,休起怠荒心’。且将行李歇下,把马拴在牌楼柱上,你使钉钯,沙僧使杖,
拦住洞门,让我进去打听打听。若师父果在里面,我将铁棒把妖精从里打出,跑至
门口,你两个却在外面挡住:这是里应外合。打死精灵,才救得师父。”二人遵命。

行者却将身一纵,跳入洞中,足下彩云生万道,身边瑞气护千层。不多时,到
于深远之间,那里边明明朗朗,一般的有日色,有风声,又有花草果木。行者喜道:
“好去处啊!想老孙出世,天赐与水帘洞,这里也是个洞天福地!”

正看时,又见有一座二滴水的门楼,团团都是松竹,内有许多房舍。又想道:
“此必是妖精的住处了。我且到那里边去打听打听。——且住!若是这般去啊,他
认得我了,且变化了去。”摇身捻诀,就变做个苍蝇儿,轻轻的飞在门楼上听听。
只见那怪高坐在草亭内。他那模样,比在松林里救他,寺里拿他,便是不同,越发
打扮得俊了:

发盘云髻似堆鸦,身着绿绒花比甲。一对金莲刚半折,十指如同春笋发。团团
粉面若银盘,朱唇一似樱桃滑。端端正正美人姿,月里嫦娥还喜恰。今朝拿住取经
僧,便要欢娱同枕榻。
行者且不言语,听他说甚话。少时,绽破樱桃,喜孜孜的叫道:“小的们,快排素
筵席来,我与唐僧哥哥吃了成亲。”行者暗笑道:“真个有这话!我只道八戒作耍子
乱说哩!等我且飞进去寻寻,看师父在那里。不知他的心性如何。假若被他摩弄动
了啊,留他在这里也罢。”即展翅,飞到里边看处,那东廊下上明下暗的红纸格子
里面,坐着唐僧哩。

行者一头撞破格子眼,飞在唐僧光头上丁着,叫声:“师父。”三藏认得声音,
叫道:“徒弟,救我命啊!”行者道:“师父不济呀,那怪精安排筵宴,与你吃了成
亲哩。或生下一男半女,也是你和尚之后代,你愁怎的?”长老闻言,咬牙切齿道:
“徒弟,我自出了长安,到两界山中收你,一向西来,那个时辰动荤?那一日子有
甚歪意?今被这妖精拿住,要求配偶,我若把真阳丧了,我就身堕轮回,打在那阴
山背后,永世不得翻身!”行者笑道:“莫发誓。既有真心往西天取经,老孙带你去
罢。”三藏道:“进来的路儿,我通忘了。”行者道:“莫说你忘了。他这洞,不比走
进来走出去的,是打上头往下钻。如今救了你,要打底下往上钻。若是造化高,钻
着洞口儿,就出去了;若是造化低,钻不着,还有个闷杀的日子了。”三藏满眼垂
泪道:“似此艰难,怎生是好?”行者道:“没事,没事。那妖精整治酒与你吃,没
奈何,也吃他一钟;只要斟得急些儿,斟起一个喜花儿来,等我变作个虫儿,
飞在酒泡之下,他把我一口吞下肚去,我就捻破他的心肝,扯断他的肺腑,弄死那
妖精,你才得脱身出去。”三藏道:“徒弟,这等说,只是不当人子。”行者道:“只
管行起善来,你命休矣。妖精乃害人之物,你惜他怎的!”三藏道:“也罢,也罢,
你只是要跟着我。”正是那孙大圣护定唐三藏,取经僧全靠着美猴王。

他师徒两个,商量未定,早是那妖精安排停当,走近东廊外,开了门锁,叫声:
“长老。”唐僧不敢答应。又叫一声,又不敢答应。他不敢答应者何意?想着:“口
开神气散,舌动是非生。”却又一条心儿想着,若死住法儿不开口,怕他心狠,顷
刻间就害了性命。正是那进退两难心问口,三思忍耐口问心。正自狐疑,那怪又叫
一声:“长老。”唐僧没奈何,应他一声道:“娘子,有。”那长老应出这一句言来,
真是肉落千斤。人都说唐僧是个真心的和尚,往西天拜佛求经,怎么与这女妖精答
话?不知此时正是危急存亡之秋,万分出于无奈,虽是外有所答,其实内无所欲。

妖精见长老应了一声,他推开门,把唐僧搀起来,和他携手挨背,交头接耳,
你看他做出那千般娇态,万种风情,岂知三藏一腔子烦恼。行者暗中笑道:“我师
父被他这般哄诱,只怕一时动心。”正是:
真僧魔苦遇娇娃,妖怪娉婷实可夸。
淡淡翠眉分柳叶,盈盈丹脸衬桃花。
绣鞋微露双钩凤,云髻高盘两鬓鸦。
含笑与师携手处,香飘兰麝满袈裟。

妖精挽着三藏,行近草亭道:“长老,我办了一杯酒,和你酌酌。”唐僧道:“娘
子,贫僧自不用荤。”妖精道:“我知你不吃荤,因洞中水不洁净,特命山头上取阴
阳交媾的净水,做些素果素菜筵席,和你耍子。”唐僧跟他进去观看,果然见那:

盈门下,绣缠彩结;满庭中,香喷金猊。摆列着黑油垒钿
桌,朱漆篾丝盘。垒细桌上,有异样珍羞;篾丝盘中,盛稀奇素物。林檎橄榄、莲
肉葡萄、榧柰榛松、荔枝龙眼、山栗风菱、枣儿柿子、胡桃银杏、金桔香橙。果子
随山有,蔬菜更时新。豆腐面筋、木耳鲜笋、蘑菇香蕈、山药黄精。石花菜、黄花
菜,青油煎炒;扁豆角、江豆角,熟酱调成。王瓜瓠子,白果蔓菁。镟皮茄子鹌鹑
做,剔种冬瓜方旦名。烂煨芋头糖拌着,白煮萝卜醋浇烹。椒姜辛辣般般美,咸淡
调和色色平。
那妖精露尖尖之玉指,捧晃晃之金杯,满斟美酒,递与唐僧,口里叫道:“长老哥
哥,妙人,请一杯交欢酒儿。”三藏羞答答的,接了酒,望空浇奠,心中暗祝道:“护
法诸天、五方揭谛,四值功曹:弟子陈玄奘,自离东土,蒙观世音菩萨差遣列位众
神暗中保护,拜雷音,见佛求经,今在途中,被妖精拿住,强逼成亲,将这一杯酒
递与我吃。此酒果是素酒,弟子勉强吃了,还得见佛成功;若是荤酒,破了弟子之
戒,永堕轮回之苦!”

孙大圣,他却变得轻巧,在耳根后,若像一个耳报;但他说话,惟三藏听见,
别人不闻。他知师父平日好吃葡萄做的素酒,教吃他一钟。那师父没奈何吃了,急
将酒满斟一钟,回与妖怪。果然斟起有一个喜花儿。行者变作个虫儿,轻轻的
飞入喜花之下。那妖精接在手,且不吃,把杯儿放住,与唐僧拜了两拜,口里娇娇
怯怯,叙了几句情话。却才举杯,那花儿已散,就露出虫来。妖精也认不得是行者
变的,只以为虫儿,用小指挑起,往下一弹。

行者见事不谐,料难入他腹,即变做个饿老鹰。真个是:

玉瓜金睛铁翮,雄姿猛气抟云。妖狐狡兔见他昏,千里山河时遁。饥处迎风逐
雀,饱来高贴天门。老拳钢硬最伤人,得志凌霄嫌近。
飞起来,轮开玉爪,响一声掀翻桌席,把些素果素菜,盘碟家火,尽皆碎,撇却
唐僧,飞将出去。唬得妖精心胆皆裂,唐僧的骨肉通酥。妖情战战兢兢,搂住唐僧
道:“长老哥哥,此物是那里来的?”三藏道:“贫僧不知。”妖精道:“我费了许多
心,安排这个素宴与你耍耍,却不知这个扁毛畜生,从那里飞来,把我的家火打碎!”
众小妖道:“夫人,打碎家火犹可,将些素品都泼散在地,秽了怎用?”三藏分明
晓得是行者弄法,他那里敢说。那妖精道:“小的们,我知道了。想必是我把唐僧
困住,天地不容,故降此物。你们将碎家火拾出去,另安排些酒肴,不拘荤素,我
指天为媒,指地作订,然后再与唐僧成亲。”依然把长老送在东廊里坐下不题。

却说行者飞出去,现了本相,到于洞口,叫声:“开门!”八戒笑道:“沙僧,
哥哥来了。”他二人撒开兵器。行者跳出,八戒上前扯住道:“可有妖精?可有师父?”
行者道:“有,有,有。”八戒道:“师父在里边受罪哩?绑着是捆着?要蒸是要煮?”
行者道:“这个事倒没有,只是安排素宴,要与他干那个事哩。”八戒道:“你造化,
你造化,你吃了陪亲酒来了!”行者道:“呆子啊!师父的性命也难保,吃甚么陪亲
酒!”八戒道:“你怎的就来了?”行者把见唐僧施变化的上项事说了一遍,道:“兄
弟们,再休胡思乱想。师父已在此间,老孙这一去,一定救他出来。”

复翻身入里面,还变做个苍蝇儿,丁在门楼上所之。只闻得这妖怪气的,
在亭子上吩咐:“小的们,不论荤素,拿来烧纸。借烦天地为媒订,务要与他成亲。”

行者听见,暗笑道:“这妖精全没一些儿廉耻!青天白日的,把个和尚关在家里
摆布。且不要忙,等老孙再进去看看。”“嘤”的一声,飞在东廊之下,见那师父坐
在里边,清滴滴腮边泪淌。行者钻将进去,丁在他头上,又叫声:“师父。”长老认
得声音,跳起来,咬牙恨道:“猢狲啊!别人胆大,还是身包胆;你的胆大,就是胆
包身!你弄变化神通,打破家火,能值几何!斗得那妖精淫兴发了,那里不分荤素安
排,定要与我交媾,此事怎了!”行者暗中陪笑道:“师父莫怪,有救你处。”唐僧
道:“那里救得我?”

行者道:“我才一翅飞起去时,见他后边有个花园。你哄他往园里去耍子,我
救了你罢。”唐僧道:“园里怎么样救?”行者道:“你与他到园里,走到桃树边,
就莫走了。等我飞上桃枝,变作个红桃子。你要吃果子,先拣红的儿摘下来。红的
是我。他必然也要摘一个,他把红的定要让他。他若一口吃了,我却在他肚里,等
我捣破他的皮袋,扯断他的肝肠,弄死他,你就脱身了。”三藏道:“你若有手段,
就与他赌斗便了;只要钻在他肚里怎么?”行者道:“师父,你不知趣。他这个洞,
若好出入,便可与他赌斗,只为出入不便,曲道难行,若就动手,他这一窝子,老
老小小,连我都扯住,却怎么了?须是这般手干。大家才得干净。”三藏点头听信,
只叫:“你跟定我。”行者道:“晓得,晓得,我在你头上。”

师徒们商量定了,三藏才欠起身来,双手扶着那格子,叫道:“娘子,娘子。”
那妖精听见,笑唏唏的跑近跟前道:“妙人哥哥,有甚话说?”三藏道:“娘子,我
出了长安,一路西来,无日不山,无日不水。昨天镇海寺投宿,偶得伤风重疾,今
日出了汗,略才好些;又蒙娘子盛情,携入仙府,只得坐了这一日,又觉心神不爽。
你带我往那里略散散心,耍耍儿去么?”那妖精十分欢喜道:“妙人哥哥倒有些兴
趣。我和你去花园里耍耍。”叫:“小的们,拿钥匙来开了园门,打扫路径。”众妖
都跑去开门收拾。

这妖精开了格子,搀出唐僧。你看那许多小妖,都是油头粉面,袅娜娉婷,簇
簇拥拥,与唐僧径上花园而去。好和尚!他在这绮罗队里无他故,锦绣丛中作哑聋。
若不是这铁打的心肠朝佛去,第二个酒色凡夫也取不得经。一行都到了花园之外。
那妖精俏语低声叫道:“妙人哥哥,这里耍耍,真可散心释闷。”唐僧与他携手相搀,
同入园内,抬头观看,其实好个去处。但见那:

萦回曲径,纷纷尽点苍苔;窈窕绮窗,处处暗笼绣箔。微
风初动,轻飘飘展开蜀锦吴绫;细雨才收,娇滴滴露出冰肌玉质。日灼鲜杏,红如
仙子晒霓裳;月映芭蕉,青似太真摇羽扇。粉墙四面,万株杨柳啭黄鹂;闲馆周围,
满院海棠飞粉蝶。更看那凝香阁、青蛾阁、解酲阁、相思阁,层层卷映,朱帘上,
钩控虾须;又见那养酸亭、披素亭、画眉亭、四雨亭,个个峥嵘,华扁上,字书鸟
篆。看那浴鹤池、洗觞池、怡月池、濯缨池,青萍绿藻耀金鳞;又有墨花轩、异箱
轩、适趣轩、慕云轩,玉斗琼卮浮绿蚁。池亭上下,有太湖石、紫英石、鹦落石、
锦川石,青青栽着虎须蒲,轩阁东西,有木假山、翠屏山、啸风山、玉芝山,处处
丛生凤尾竹。荼架、蔷薇架,近着秋千架,浑如锦帐罗帏;松柏亭、辛夷亭,对
着木香亭,却似碧城绣。芍药栏,牡丹丛,朱朱紫紫斗华;夜合台,茉藜槛,
岁岁年年生妩媚。涓涓滴露紫含笑,堪面堪描,艳艳烧空红拂桑,宜题宜赋。论景
致,休夸阆苑蓬莱;较芳菲,不数姚黄魏紫。若到三春闲斗草,园中只少玉琼花。

长老携着那怪,步赏花园,看不尽的奇葩异卉。行过了许多亭阁,真个是渐入
佳境。忽抬头,到了桃树林边,行者把师父头上一掐,那长老就知。

行者飞在桃树枝儿上,摇身一变,变作个红桃儿,其实红得可爱。长老对妖精
道:“娘子,你这苑内花香,枝头果熟。苑内花香蜂竞采,枝头果熟鸟争衔。怎么
这桃树上果子青红不一,何也?”妖精笑道:“天无阴阳,日月不明;地无阴阳,
草木不生;人无阴阳,不分男女。这桃树上果子,向阳处,有日色相烘者先熟,故
红;背阴处无日者还生,故青:此阴阳之道理也。”三藏道:“谢娘子指教。其实贫
僧不知。”即向前伸手摘了个红桃。妖精也去摘了一个青桃。三藏躬身将红桃奉与
妖怪道:“娘子,你爱色,请吃这个红桃,拿青的来我吃。”妖精真个换了。且暗喜
道:“好和尚啊,果是个真人!一日夫妻未做,却就有这般恩爱也。”那妖精喜喜欢
欢的,把唐僧亲敬。这唐僧把青桃拿过来就吃。那妖精喜相陪,把红桃儿张口便咬。
启朱唇,露银牙,未曾下口,原来孙行者十分性急,毂辘一个跟头,翻入他咽候之
下,径到肚腹之中。妖精害怕,对三藏道:“长老啊,这个果子利害。怎么不容咬
破,就滚下去了?”三藏道:“娘子,新开园的果子爱吃,所以去得快了。”妖精道:
“未曾吐出核子,他就撺下去了。”三藏道:“娘子意美情佳,喜吃之甚,所以不及
吐核,就下去了。”

行者在他肚里,复了本相。叫声:“师父,不要与他答嘴,老孙已得了手也!”
三藏道:“徒弟方便着些。”妖精听见道:“你和那个说话哩?”三藏道:“和我徒弟
孙悟空说话哩。”妖精道:“孙悟空在那里?”三藏道:“在你肚里哩。却才吃的那
个红桃子不是?”妖精慌了道:“罢了,罢了!这猴头钻在我肚里,我是死也!孙行
者!你千方百计的钻在我肚里怎的?”行者在里边恨道:“也不怎的!只是吃了你的
六叶连肝肺,三毛七孔心,五脏都淘净,弄做个梆子精!”妖精听说,唬得魂飞魄
散,战战兢兢的,把唐僧抱住道:“长老啊!我只道:
夙世前缘系赤绳,鱼水相和两意浓。
不料鸳鸯今拆散,何期鸾凤又西东!
蓝桥水涨难成事,佛庙烟沉嘉会空。
着意一场今又别,何年与你再相逢!”
行者在他肚里听见说时,只怕长老慈心,又被他哄了。便就轮拳跳脚,支架子,理
四平,几乎把个皮袋儿捣破了。那妖精忍不得疼痛,倒在尘埃,半晌家不敢言语。
行者见不言语,想是死了,却把手略松一松。他又回过气来,叫:“小的们,在那
里?”原来那些小妖,自进园门来,各人知趣,都不在一处,各自去采花斗草,任
意随心耍子,让那妖精与唐僧两个自在叙情儿。忽听得叫,却才都跑将来。又见妖
精倒在地上,面容改色,口里哼哼的爬不动,连忙搀起,围在一处道:“夫人,怎
的不好?想是急心疼了?”妖精道:“不是,不是!你莫要问,我肚里已有了人也!快
把这和尚送出去,留我性命!”那些小妖,真个都来扛抬。行者在肚里叫道:“那个
敢抬!要便是你自家献我师父出去,出到外边,我饶你命!”那怪精没计奈何,只是
惜命之心,急挣起来,把唐僧背在身上,拽开步,往外就走。小妖跟随着:“老夫
人,往那里去?”妖精道:“‘留得五湖明月在,何愁没处下金钩!’把这厮送出去,
等我别寻一个头儿吧!”

好妖精,一纵云光,直到洞口。又闻得叮叮当当,兵刃乱响。三藏道:“徒弟,
外面兵器响哩。”行者道:“是八戒揉钯哩。你叫他一声。”三藏便叫:“八戒!”八
戒听见道:“沙和尚!师父出来也!”二人掣开钯杖,妖精把唐僧驮出。咦!正是:
心猿里应降邪怪,土木司门接圣僧。

毕竟不知那妖精性命如何,且听下回分解。