\chapter{平顶山功曹传信~莲花洞木母逢灾}

话说唐僧复得了孙行者,师徒们一心同体,共诣西方。自宝象国救了公主,承
君臣送出城西。说不尽沿路饥餐渴饮,夜住晓行。却又值三春景候。那时节:

轻风吹柳绿如丝,佳景最堪题。时催鸟语,暖烘花发,遍地芳菲。海棠庭院来
双燕,正是赏春时。红尘紫陌,绮罗弦管,斗草传卮。
师徒们正行赏间,又见一山挡路。唐僧道:“徒弟们仔细。前遇山高,恐有虎狼阻
挡。”行者道:“师父,出家人莫说在家话。你记得那乌巢和尚的《心经》云‘心无
挂碍:无挂碍,方无恐怖,远离颠倒梦想’之言?但只是‘扫除心上垢,洗净耳边
尘。不受苦中苦,难为人上人。’你莫生忧虑,但有老孙,就是塌下天来,可保无
事。怕甚么虎狼!”长老勒回马道:“我
当年奉旨出长安,只忆西来拜佛颜。
舍利国中金象彩,浮屠塔里玉毫斑。
寻穷天下无名水,历遍人间不到山。
逐逐烟波重叠叠,几时能彀此身闲?”
行者闻说,笑呵呵道:“师要身闲,有何难事?若功成之后,万缘都罢,诸法皆空。
那时节,自然而然,却不是身闲也?”长老闻言,只得乐以忘忧。放辔催银,兜
缰趱玉龙。

师徒们上得山来,十分险峻,真个嵯峨。好山:

巍巍峻岭,削削尖峰。湾环深涧下,孤峻陡崖边。湾环深涧下,只听得唿喇喇
戏水蟒翻身;孤峻陡崖边,但见那出林虎剪尾。往上看,峦头突兀透青霄;
回眼观,壑下深沉邻碧落。上高来,似梯似凳;下低行,如堑如坑。真个是古怪巅
峰岭,果然是连尖削壁崖。巅峰岭上,采药人寻思怕走;削壁崖前,打柴夫寸步难
行。胡羊野马乱撺梭,狡兔山牛如布阵。山高蔽日遮星斗,时逢妖兽与苍狼。草径
迷漫难进马,怎得雷音见佛王?

长老勒马观山,正在难行之处。只见那绿莎坡上,伫立着一个樵夫。你道他怎
生打扮:

头戴一顶老蓝毡笠,身穿一领毛皂衲衣:老蓝毡笠,遮烟盖日果稀奇;毛皂衲
衣,乐以忘忧真罕见。手持钢斧快磨明,刀伐干柴收束紧。担头春色,幽然四序融
融,身外闲情,常是三星淡淡。到老只于随分过,有何荣辱暂关山?
那樵子:
正在坡前伐朽柴,忽逢长老自东来。
停柯住斧出林外,趋步将身上石崖。
对长老厉声高叫道:“那西进的长老!暂停片时,我有一言奉告:此山有一伙毒魔狠
怪,专吃你东来西去的人哩。”长老闻言,魂飞魄散,战兢兢坐不稳雕鞍。急回头,
忙呼徒弟道:“你听那樵夫报道:‘此山有毒魔狠怪。’谁敢去细问他一问?”行者
道:“师父放心,等老孙去问他一个端的。”

好行者,拽开步,径上山来,对樵子叫声“大哥”,道个问讯。樵夫答礼道:“长
老啊,你们有何缘故来此?”行者道:“不瞒大哥说,我们是东土差来西天取经的。
那马上是我的师父。他有些胆小。适蒙见教,说有甚么毒魔狠怪,故此我来奉问一
声:那魔是几年之魔,怪是几年之怪?还是个把势?还是个雏儿?烦大哥老实说说,
我好着山神、土地递解他起身。”

樵子闻言,仰天大笑道:“你原来是个风和尚。”行者道:“我不风啊,这是老
实话。”樵子道:“你说是老实,便怎敢说把他递解起身?”行者道:“你这等长他
那威风,胡言乱语的拦路报信,莫不是与他有亲?不亲必邻,不邻必友。”樵子笑道:
“你这个风泼和尚,忒没道理。我倒是好意,特来报与你们。教你们走路时,早晚
间防备,你倒转赖在我身上。且莫说我不晓得妖魔出处;就晓得啊,你敢把他怎么
的递解?解往何处?”行者道:“若是天魔,解与玉帝;若是土魔,解与土府。西方
的归佛,东方的归圣。北方的解与真武,南方的解与火德。是蛟精解与海主,是鬼
祟解与阎王。各有地头方向。我老孙到处里人熟,发一张批文,把他连夜解着飞跑。”

那樵子止不住呵呵冷笑道:“你这个风泼和尚,想是在方上云游,学了些书符
咒水的法术,只可驱邪缚鬼,还不曾撞见这等狠毒的怪哩。”行者道:“怎见他狠
毒?”樵子道:“此山径过有六百里远近,名唤平顶山。山中有一洞,名唤莲花洞。
洞里有两个魔头,他画影图形,要捉和尚;抄名访姓,要吃唐僧。你若别处来的还
好,但犯了一个‘唐’字儿,莫想去得,去得!”行者道:“我们正是唐朝来的。”
樵子道:“他正要吃你们哩。”行者道:“造化,造化!但不知他怎的样吃哩?”樵子
道:“你要他怎的吃?”行者道:“若是先吃头,还好耍子;若是先吃脚,就难为了。”
樵子道:“先吃头怎么说?先吃脚怎么说?”行者道:“你还不曾经着哩。若是先吃
头,一口将他咬下,我已死了,凭他怎么煎炒熬煮,我也不知疼痛;若是先吃脚,
他啃了孤拐,嚼了腿亭,吃到腰截骨,我还急忙不死,却不是零零碎碎受苦?此所
以难为也。”樵子道:“和尚,他那里有这许多工夫,只是把你拿住,捆在笼里,囫
囵蒸吃了!”行者笑道:“这个更好,更好,疼倒不忍疼,只是受些闷气罢了。”樵
子道:“和尚不要调嘴。那妖怪随身有五件宝贝,神通极大极广。就是擎天的玉柱,
架海的金梁,若保得唐朝和尚去,也须要发发昏是。”行者道:“发几个昏么?”樵
子道:“要发三四个昏是。”行者道:“不打紧,不打紧。我们一年,常发七八百个
昏儿,这三四个昏儿易得发,发发儿就过去了。”

好大圣,全然无惧,一心只是要保唐僧,脱樵夫,拽步而转。径至山坡马头
前道:“师父,没甚大事。有便有个把妖精儿,只是这里人胆小,放他在心上。有
我哩,怕他怎的?走路,走路!”长老见说,只得放怀随行。

正行处,早不见了那樵夫。长老道:“那报信的樵子如何就不见了?”八戒道:
“我们造化低,撞见日里鬼了。”行者道:“想是他钻进林子里寻柴去了。等我看看
来。”好大圣,睁开火眼金睛,漫山越岭的望处,却无踪迹。忽抬头往云端里一看,
看见是日值功曹,他就纵云赶上,骂了几声“毛鬼!”道:“你怎么有话不来直说,
却那般变化了,演样老孙?”慌得那功曹施礼道:“大圣,报信来迟,勿罪,勿罪。
那怪果然神通广大,变化多端。只看你腾那乖巧,运动神机,仔细保你师父;假若
怠慢了些儿,西天路莫想去得。”

行者闻言,把功曹叱退,切切在心。按云头,径来山上。只见长老与八戒、沙
僧,簇拥前进。他却暗想:“我若把功曹的言语实实告诵师父,师父他不济事,必
就哭了;假若不与他实说,梦着头,带着他走,常言道:‘乍入芦圩,不知深浅。’
倘或被妖魔捞去,却不又要老孙费心?……且等我照顾八戒一照顾,先着他出头与
那怪打一仗看。若是打得过他,就算他一功;若是没手段,被怪拿去,等老孙再去
救他不迟:却好显我本事出名。”正自家计较,以心问心道:“只恐八戒躲懒便不肯
出头。师父又有些护短。等老孙羁勒他羁勒。”

好大圣,你看他弄个虚头,把眼揉了一揉,揉出些泪来。迎着师父,往前径走。
八戒看见,连忙叫:“沙和尚,歇下担子,拿出行李来,我两个分了罢!”沙僧道:
“二哥,分怎的?”八戒道:“分了罢!你往流沙河还做妖怪,老猪往高老庄上盼盼
浑家。把白马卖了,买口棺木,与师父送老,大家散火。还往西天去哩?”长老在
马上听见。道:“这个夯货!正走路,怎么又胡说了?”八戒道:“你儿子便胡说!你
不看见孙行者那里哭将来了?他是个钻天入地,斧砍火烧,下油锅都不怕的好汉;
如今戴了个愁帽,泪汪汪的哭来,必是那山险峻,妖怪凶狠。似我们这样软弱的人
儿,怎么去得?”长老道:“你且休胡谈。待我问他一声,看是怎么说话。”

问道:“悟空,有甚话当面计较。你怎么自家烦恼?这般样个哭包脸,是虎唬我
也?”行者道:“师父啊,刚才那个报信的,是日值功曹。他说妖精凶狠,此处难
行,果然的山高路峻,不能前进。改日再去罢。”长老闻言,恐惶悚惧,扯住他虎
皮裙子道:“徒弟呀,我们三停路已走了停半,因何说退悔之言?”行者道:“我没
个不尽心的。但只恐魔多力弱,行势孤单。‘纵然是块铁,下炉能打得几根钉?’”
长老道:“徒弟啊,你也说得是。果然一个人也难。兵书云:‘寡不可敌众。’我这
里还有八戒、沙僧,都是徒弟,凭你调度使用,或为护将帮手,协力同心,扫清山
径,领我过山,却不都还了正果?”

那行者这一场扭捏,只逗出长老这几句话来。他了泪道:“师父啊,若要过
得此山,须是猪八戒依得我两件事儿,才有三分去得;假若不依我言,替不得我手,
半分儿也莫想过去。”八戒道:“师兄,不去就散火罢。不要攀我。”长老道:“徒弟,
且问你师兄,看他教你做甚么。”呆子真个对行者说道:“哥哥,你教我做甚事?”
行者道:“第一件是看师父,第二件是去巡山。”八戒道:“看师父是坐,巡山去是
走;终不然教我坐一会又走,走一会又坐。两处怎么顾盼得来?”行者道:“不是
教你两件齐干,只是领了一件便罢。”八戒又笑道:“这等也好计较。但不知看师父
是怎样,巡山是怎样。你先与我讲讲,等我依个相应些儿的去干罢。”行者道:“看
师父啊:师父去出恭,你伺候;师父要走路,你扶持;师父要吃斋,你化斋。若他
饿了些儿,你该打;黄了些儿脸皮,你该打;瘦了些儿形骸,你该打。”八戒慌了
道:“这个难,难,难!伺候扶持,通不打紧,就是不离身驮着,也还容易;假若教
我去乡下化斋,他这西方路上,不识我是取经的和尚,只道是那山里走出来的一个
半壮不壮的健猪,伙上许多人,叉钯扫帚,把老猪围倒,拿家去宰了,腌着过年,
这个却不就遭瘟了?”行者道:“巡山去罢。”八戒道:“巡山便怎么样儿?”行者
道:“就入此山,打听有多少妖怪,是甚么山,是甚么洞,我们好过去。”八戒道:
“这个小可,老猪去巡山罢。”那呆子就撒起衣裙,挺着钉钯,雄赳赳,径入深山!
气昂昂,奔上大路。

行者在旁,忍不住嘻嘻冷笑。长老骂道:“你这个泼猴!兄弟们全无爱怜之意,
常怀嫉妒之心。你做出这样獐智,巧言令色,撮弄他去甚么巡山,却又在这里笑他!”
行者道:“不是笑他,我这笑中有味。你看猪八戒这一去,决不巡山,也不敢见妖
怪,不知往那里去躲闪半会,捏一个谎来,哄我们也。”长老道:“你怎么就晓得他?”
行者道:“我估出他是这等。不信,等我跟他去看看,听他一听:一则帮副他手段
降妖,二来看他可有个诚心拜佛。”长老道:“好,好,好!你却莫去捉弄他。”行者
应诺了。径直赶上山坡,摇身一变,变作个虫儿。其实变得轻巧。但见他:

翅薄舞风不用力,腰尖细小如针。穿蒲抹草过花阴,疾似流星还甚。眼睛明映
映,声气渺喑喑。昆虫之类惟他小,亭亭款款机深。几番闲日歇幽林,一身浑不见,
千眼莫能寻。
嘤的一翅飞将去,赶上八戒,钉在他耳朵后面鬃根底下。那呆子只管走路,怎知道
身上有人,行有七八里路,把钉钯撇下,吊转头来,望着唐僧,指手画脚的骂道:
“你罢软的老和尚,捉掐的弼马温,面弱的沙和尚!他都在那里自在,捉弄我老猪
来跄路!大家取经,都要望成正果,偏是教我来巡甚么山!哈,哈,哈!晓得有妖怪,
躲着些儿走。还不够一半,却教我去寻他,这等晦气哩!我往那里睡觉去,睡一觉
回去,含含糊糊的答应他,只说是巡了山,就了其帐也。”那呆子一时间侥幸,搴
着钯,又走。只见山凹里一弯红草坡,他一头钻得进去,使钉钯扑个地铺,毂辘的
睡下。把腰伸了一伸,道声“快活!就是那弼马温,也不得像我这般自在!”原来行
者在他耳根后,句句儿听着哩;忍不住,飞将起来,又捉弄他一捉弄。又摇身一变,
变作个啄木虫儿。但见:

铁嘴尖尖红溜,翠翎艳艳光明。一双钢爪利如钉,腹馁何妨林静。最爱枯槎朽
烂,偏嫌老树伶仃。圜睛决尾性丢灵,辟剥之声堪听。

这虫不大不小的,上秤称,只有二三两重。红铜嘴,黑铁脚,刷剌的一翅飞
下来。那八戒丢倒头,正睡着了,被他照嘴唇上揸的一下。那呆子慌得爬将起来,
口里乱嚷道:“有妖怪,有妖怪!把我戳了一枪去了!嘴上好不疼呀!”伸手摸摸,泱
出血来了。他道:“蹭蹬啊!我又没甚喜事,怎么嘴上挂了红耶?”他看着这血手,
口里絮絮叨叨的两边乱看,却不见动静,道:“无甚妖怪,怎么戳我一枪么?”忽
抬头往上看时,原来是个啄木虫,在半空中飞哩。呆子咬牙骂道:“这个亡人!弼马
温欺负我罢了,你也来欺负我!我晓得了。他一定不认我是个人,只把我嘴当一段
黑朽枯烂的树,内中生了虫,寻虫儿吃的,将我啄了这一下也。等我把嘴揣在怀里
睡罢。”那呆子毂辘的依然睡倒。行者又飞来,着耳根后又啄了一下。呆子慌得爬
起来道:“这个亡人,却打搅得我狠!想必这里是他的窠巢,生蛋布雏,怕我占了,
故此这般打搅。罢,罢,罢!不睡他了!”搴着钯,径出红草坡,找路又走。可不喜
坏了孙行者,笑倒个美猴王。行者道:“这夯货大睁着两个眼,连自家人也认不得!”

好大圣,摇身又一变,还变做个虫,钉在他耳朵后面,不离他身上。那呆
子入深山,又行有四五里,只见山凹中有桌面大的四四方方三块青石头。呆子放下
钯,对石头唱个大喏。行者暗笑道:“这呆子!石头又不是人,又不会说话,又不会
还礼,唱他喏怎的,可不是个瞎帐?”原来那呆子把石头当着唐僧、沙僧、行者三
人,朝着他演习哩。他道:“我这回去,见了师父,若问有妖怪,就说有妖怪。他
问甚么山,我若说是泥捏的,土做的,锡打的,铜铸的,面蒸的,纸糊的,笔画的,
他们见说我呆哩,若讲这话,一发说呆了,我只说是石头山。他问甚么洞,也只说
是石头洞。他问甚么门,却说是钉钉的铁叶门。他问里边有多远,只说入内有三层。
十分再搜寻,问门上钉子多少,只说老猪心忙记不真。此间编造停当,哄那弼马温
去!”

那呆子捏合了,拖着钯,径回本路。怎知行者在耳朵后,一一听得明显。行者
见他回来,即腾两翅预先回去。现原身,见了师父。师父道:“悟空,你来了,悟
能怎不见回?”行者笑道:“他在那里编谎哩。就待来也。”长老道:“他两个耳朵
盖着眼,愚拙之人也,他会编甚么谎?又是你捏合甚么鬼话赖他哩。”行者道:“师
父,你只是这等护短。这是有对问的话。”把他那钻在草里睡觉,被啄木虫叮醒,
朝石头唱喏,编造甚么石头山、石头洞、铁叶门、有妖精的话,预先说了。

说毕不多时,那呆子走将来。又怕忘了那谎,低着头,口里温习。被行者喝了
一声道:“呆子!念甚么哩?”八戒掀起耳朵来看看道:“我到了地头了!”那呆子上
前跪倒。长老搀起道:“徒弟,辛苦啊。”八戒道:“正是。走路的人,爬山的人,
第一辛苦了。”长老道:“可有妖怪么?”八戒道:“有妖怪,有妖怪!一堆妖怪哩!”
长老道:“怎么打发你来?”八戒说:“他叫我做猪祖宗,猪外公,安排些粉汤素食,
教我吃了一顿,说道,摆旗鼓送我们过山哩。”行者道:“想是在草里睡着了,说得
是梦话?”呆子闻言,就吓得矮了二寸道:“爷爷呀!我睡他怎么晓得?……”行者
上前,一把揪住道:“你过来,等我问你。”呆子又慌了,战战兢兢的道:“问便罢
了,揪扯怎的?”行者道:“是甚么山?”八戒道:“是石头山。”“甚么洞?”道:
“是石头洞。”“甚么门?”道:“是钉钉铁叶门。”“里边有多远?”道:“入内是三
层。”行者道:“你不消说了,后半截我记得真。恐师父不信,我替你说了罢。”八
戒道:“嘴脸!你又不曾去,你晓得那些儿,要替我说?”行者笑道:“‘门上钉子有
多少,只说老猪心忙记不真。’可是么?”那呆子即慌忙跪倒。行者道:“朝着石头
唱喏,当做我三人,对他一问一答。可是么?又说:‘等我编得谎儿停当,哄那弼马
温去!’可是么?”那呆子连忙只是磕头道:“师兄,我去巡山,你莫成跟我去听的?”
行者骂道:“我把你个馕糠的夯货!这般要紧的所在,教你去巡山,你却去睡觉!不
是啄木虫叮你醒来,你还在那里睡哩。及叮醒,又编这样大谎,可不误了大事?你
快伸过孤拐来,打五棍记心!”

八戒慌了道:“那个哭丧棒重,擦一擦儿皮塌,挽一挽儿筋伤,若打五下,就
是死了!”行者道:“你怕打,却怎么扯谎?”八戒道:“哥哥呀,只是这一遭儿,
以后再不敢了。”行者道:“一遭便打三棍罢。”八戒道:“爷爷呀,半棍儿也禁不得!”
呆子没计奈何,扯住师父道:“你替我说个方便儿。”长老道:“悟空说你编谎,我
还不信。今果如此,其实该打。但如今过山少人使唤,悟空,你且饶他,待过了山,
再打罢。”行者道:“古人云:‘顺父母言情,呼为大孝。’师父说不打,我就且饶你。
你再去与他巡山。若再说谎误事,我定一下也不饶你!”

那呆子只得爬起来又去。你看他奔上大路,疑心生暗鬼,步步只疑是行者变化
了跟住他。故见一物,即疑是行者。走有七八里,见一只老虎,从山坡上跑过,他
也不怕,举着钉钯道:“师兄来听说谎的?这遭不编了。”又走处,那山风来得甚猛,
呼的一声,把颗枯木刮倒,滚至面前,他又跌脚捶胸的道:“哥啊!这是怎的起!一
行说不敢编谎罢了,又变甚么树来打人!”又走向前,只见一个白颈老鸦,当头喳
喳的连叫几声,他又道:“哥哥,不羞,不羞!我说不编就不编了,只管又变着老鸦
怎的?你来听么?”原来这一番行者却不曾跟他去,他那里却自惊自怪,乱疑乱猜,
故无往而不疑是行者随他身也。呆子惊疑且不题。

却说那山叫做平顶山,那洞叫做莲花洞。洞里两妖:一唤金角大王,一唤银角
大王。金角正坐,对银角说:“兄弟,我们多少时不巡山了?”银角道:“有半个月
了。”金角道:“兄弟,你今日与我去巡巡。”银角道:“今日巡山怎的?”金角道:
“你不知。近闻得东土唐朝差个御弟唐僧往西方拜佛,一行四众,叫做孙行者、猪
八戒、沙和尚,连马五口。你看他在那处,与我把他拿来。”银角道:“我们要吃人,
那里不捞几个。这和尚到得那里,让他去罢。”金角道:“你不晓得。我当年出天界,
尝闻得人言:唐僧乃金蝉长老临凡,十世修行的好人,一点元阳未泄。有人吃他肉,
延寿长生哩。”银角道:“若是吃了他肉就可以延寿长生,我们打甚么坐,立甚么功,
炼甚么龙与虎,配甚么雌与雄?只该吃他去了。等我去拿他来。”金角道:“兄弟,
你有些性急,且莫忙着。你若走出门,不管好歹,但是和尚就拿将来,假如不是唐
僧,却也不当人子。我记得他的模样,曾将他师徒画了一个影,图了一个形,你可
拿去。但遇着和尚,以此照验照验。”又将某人是某名字,一一说了。银角得了图
像,知道姓名,即出洞,点起三十名小怪,便来山上巡逻。

却说八戒运拙。正行处,可可的撞见群魔,当面挡住道:“那来的甚么人?”
呆子才抬起头来,掀着耳朵,看见是些妖魔,他就慌了,心中暗道:“我若说是取
经的和尚,他就捞了去;只是说走路的。”小妖回报道:“大王,是走路的。”那三
十名小怪,中间有认得的,有不认得的,旁边有听着指点说话的,道:“大王,这
个和尚,像这图中猪八戒模样。”叫挂起影神图来。八戒看见,大惊道:“怪道这些
时没精神哩!原来是他把我的影神传将来也!”小妖用枪挑着,银角用手指道:“这
骑白马的是唐僧。这毛脸的是孙行者。”八戒听见道:“城隍,没我便也罢了,猪头
三牲,清醮二十四分。……”口里唠叨,只管许愿。那怪又道:“这黑长的是沙和
尚,这长嘴大耳的是猪八戒。”呆子听见说他,慌得把个嘴揣在怀里藏了。那怪叫:
“和尚,伸出嘴来!”八戒道:“胎里病,伸不出来。”那怪令小妖使钩子钩出来。
八戒慌得把个嘴伸出道:“小家形。罢了,这不是?你要看便就看,钩怎的?”

那怪认得是八戒,掣出宝刀,上前就砍。这呆子举钉钯按住道:“我的儿,休
无礼!看钯!”那怪笑道:“这和尚是半路出家的。”八戒道:“好儿子,有些灵性!你
怎么就晓得老爷是半路出家的?”那怪道:“你会使这钯,一定是在人家园圃中筑
地,把他这钯偷将来也。”八戒道:“我的儿,你那里认得老爷这钯。我不比那筑地
之钯。这是:

巨齿铸来如龙爪,渗金妆就似虎形。若逢对敌寒风洒,但遇相持火焰生。能替
唐僧消障碍,西天路上捉妖精。轮动烟霞遮日月,使起昏云暗斗星。筑倒泰山老虎
怕,掀翻大海老龙惊。饶你这妖有手段,一钯九个血窟窿!”

那怪闻言,那里肯让。使七星剑,丢开解数,与八戒一往一来,在山中赌斗,
有二十回合,不分胜负。八戒发起狠来,舍死的相迎。那怪见他耳朵,喷粘涎,
舞钉钯,口里吆吆喝喝的,也尽有些悚惧,即回头招呼小怪,一齐动手。若是一个
打一个,其实还好。他见那些小妖齐上,慌了手脚,遮架不住,败了阵,回头就跑。
原来是道路不平,未曾细看,忽被萝藤绊了个踉跄。挣起来正走,又被一个小妖
睡倒在地,扳着他脚跟,扑的又跌了个狗吃屎;被一群赶上按住,抓鬃毛,揪耳朵,
扯着脚,拉着尾,扛扛抬抬,擒进洞去。咦!正是:
一身魔发难消灭,万种灾生不易除。

毕竟不知猪八戒性命如何,且听下回分解。