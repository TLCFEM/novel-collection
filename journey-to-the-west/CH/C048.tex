\chapter{魔弄寒风飘大雪~僧思拜佛履层冰}

话说陈家庄众信人等,将猪羊牲醴与行者、八戒,喧喧嚷嚷,直抬至灵感庙里
排下。将童男女设在上首。行者回头,看见那供桌上香花蜡烛,正面一个金字牌位,
上写“灵感大王之神”,更无别的神像。众信摆列停当,一齐朝上叩头道:“大王爷
爷,今年、今月、今日、今时,陈家庄祭主陈澄等众信,年甲不齐,谨遵年例,供
献童男一名陈关保,童女一名陈一秤金,猪羊牲醴如数,奉上大王享用。保佑风调
雨顺,五谷丰登。”祝罢,烧了纸马,各回本宅不题。

那八戒见人散了,对行者道:“我们家去罢。”行者道:“你家在那里?”八戒
道:“往老陈家睡觉去。”行者道:“呆子又乱谈了。既允了他,须与他了这愿心才
是哩。”八戒道:“你倒不是呆子,反说我是呆子!只哄他耍耍便罢,怎么就与他祭
赛,当起真来!”行者道:“莫胡说。为人为彻。一定等那大王来吃了,才是个全始
全终;不然,又教他降灾贻害,反为不美。”

正说间,只听得呼呼风响。八戒道:“不好了!风响是那话儿来了!”行者只叫:
“莫言语,等我答应。”顷刻间,庙门外来了一个妖邪。你看他怎生模样:

金甲金盔灿烂新,腰缠宝带绕红云。眼如晚出明星皎,牙似重排锯齿分。足下
烟霞飘荡荡,身边雾霭暖熏熏。行时阵阵阴风冷,立处层层煞气温。却似卷帘扶驾
将,犹如镇寺大门神。
那怪物拦住庙门问道:“今年祭祀的是那家?”行者笑吟吟的答道:“承下问,庄头
是陈澄、陈清家。”那怪闻答,心中疑似道:“这童男胆大,言谈伶俐。常来供养受
用的,问一声不言语,再问声,唬了魂;用手去捉,已是死人。怎么今日这童男善
能应对?”

怪物不敢来拿,又问:“童男女叫甚名字?”行者笑道:“童男陈关保,童女一
秤金。”怪物道:“这祭赛乃上年旧规,如今供献我,当吃你。”行者道:“不敢抗拒,
请自在受用。”怪物听说,又不敢动手,拦住门喝道:“你莫顶嘴!我常年先吃童男,
今年倒要先吃童女!”八戒慌了道:“大王还照旧罢,不要吃坏例子。”

那怪不容分说,放开手,就捉八戒。呆子扑的跳下来,现了本相,掣钉钯,劈
手一筑,那怪物缩了手,往前就走,只听得当的一声响。八戒道:“筑破甲了!”行
者也现本相看处,原来是冰盘大小两个鱼鳞。喝声“赶上!”二人跳到空中。那怪
物因来赴会,不曾带得兵器,空手在云端里问道:“你是那方和尚,到此欺人,破
了我的香火,坏了我的名声!”行者道:“这泼物原来不知。我等乃东土大唐圣僧三
藏奉钦差西天取经之徒弟。昨因夜寓陈家,闻有邪魔,假号灵感,年年要童男女祭
赛,是我等慈悲,拯救生灵,捉你这泼物!趁早实实供来,一年吃两个童男女,你
在这里称了几年大王,吃了多少男女?一个个算还我,饶你死罪!”那怪闻言就走,
被八戒又一钉钯,未曾打着。他化一阵狂风,钻入通天河内。

行者道:“不消赶他了。这怪想是河中之物。且待明日设法拿他,送我师父过
河。”八戒依言,径回庙里,把那猪羊祭醴,连桌面一齐搬到陈家。此时唐长老、
沙和尚,共陈家兄弟,正在厅中候信,忽见他二人将猪羊等物都丢在天井里。三藏
迎来问道:“悟空,祭赛之事何如?”行者将那称名赶怪钻入河中之事,说了一遍。
二老十分欢喜,即命打扫厢房,安排床铺,请他师徒就寝不题。

却说那怪得命,回归水内,坐在宫中,默默无言。水中大小眷族问道:“大王
每年享祭,回来欢喜,怎么今日烦恼?”那怪道:“常年享毕,还带些余物与汝等
受用,今日连我也不曾吃得。造化低,撞着一个对头,几乎伤了性命。”众水族问:
“大王,是那个?”那怪道:“是一个东土大唐圣僧的徒弟,往西天拜佛求经者,
假变男女,坐在庙里。我被他现出本相,险些儿伤了性命。一向闻得人讲:唐三藏
乃十世修行好人,但得吃他一块肉延寿长生。不期他手下有这般徒弟。我被他坏了
名声,破了香火,有心要捉唐僧,只怕不得能够。”

那水族中,闪上一个斑衣鳜婆,对怪物跬跬拜拜,笑道:“大王,要捉唐僧,
有何难处!但不知捉住他,可赏我些酒肉?”那怪道:“你若有谋,合同用力,捉了
唐僧,与你拜为兄妹,共席享之。”鳜婆拜谢了道:“久知大王有呼风唤雨之神通,
搅海翻江之势力,不知可会降雪?”那怪道:“会降。”又道:“既会降雪,不知可
会作冷结冰?”那怪道:“更会!”鳜婆鼓掌笑道:“如此极易,极易!”那怪道:“你
且将极易之功,讲来我听。”鳜婆道:“今夜有三更天气,大王不必迟疑,趁早作法,
起一阵寒风,下一阵大雪,把通天河尽皆冻结。着我等善变化者,变作几个人形,
在于路口,背包持伞,担担推车,不住的在冰上行走。那唐僧取经之心甚急,看见
如此人行,断然踏冰而渡。大王稳坐河心,待他脚踪响处,迸裂寒冰,连他那徒弟
们一齐坠落水中,一鼓可得也!”那怪闻言,满心欢喜道:“甚妙,甚妙!”即出水
府,踏长空兴风作雪,结冷凝冻成冰不题。

却说唐长老师徒四人,歇在陈家。将近天晓,师徒们衾寒枕冷。八戒咳歌打战
睡不得,叫道:“师兄,冷啊!”行者道:“你这呆子,忒不长俊!出家人寒暑不侵,
怎么怕冷?”三藏道:“徒弟,果然冷。你看,就是那:

重衾无暖气,袖手似揣冰。此时败叶垂霜蕊,苍松挂冻铃。地裂因寒甚,池平
为水凝。渔舟不见叟,山寺怎逢僧。樵子愁柴少,王孙喜炭增。征人须似铁,诗客
笔如菱。皮袄犹嫌
薄,貂裘尚恨轻。蒲团僵老衲,纸帐旅魂惊。绣被重褥,浑身战抖铃。”
师徒们都睡不得,爬起来穿了衣服。开门看处,呀!外面白茫茫的,原来下雪哩!行
者道:“怪道你们害冷哩。却是这般大雪!”四人眼同观看,好雪!但见那:

彤云密布,惨雾重浸:彤云密布,朔风凛凛号空;惨雾重浸,大雪纷纷盖地。
真个是:六出花,片片飞琼!千林树,株株带玉。须臾积粉,顷刻成盐。白鹦歌失
素,皓鹤羽毛同。平添吴楚千江水,压倒东南几树梅。却便似战退玉龙三百万,果
然如败鳞残甲满天飞。那里得东郭履,袁安卧,孙康映读;更不见子猷舟,王恭氅,
苏武餐毡。但只是几家村舍如银砌,万里江山似玉团。好雪!柳絮漫桥,梨花盖舍。
柳絮漫桥,桥边渔叟挂蓑衣;梨花盖舍,舍下野翁煨骨。客子难沽酒,苍头苦觅
梅。洒洒潇潇裁蝶翅,飘飘荡荡剪鹅衣。团团滚滚随风势,叠叠层层道路迷。阵阵
寒威穿小幕,飕飕冷气透幽帏。丰年祥瑞从天降,堪贺人间好事宜。

那场雪,纷纷洒洒,果如剪玉飞绵。师徒们叹玩多时,只见陈家老者,着两个
僮仆,扫开道路,又两个送出热汤洗面。须臾,又送滚茶乳饼,又抬出炭火;俱到
厢房,师徒们叙坐。

长老问道:“老施主,贵处时令,不知可分春夏秋冬?”陈老笑道:“此间虽是
僻地,但只风俗人物,与上国不同,至于诸凡谷牲畜,都是同天共日,岂有不分
四时之理?”三藏道:“既分四时,怎么如今就有这般大雪,这般寒冷?”陈老道:
“此时虽是七月,昨日已交白露,就是八月节了。我这里常年八月间就有霜雪。”
三藏道:“甚比我东土不同。我那里交冬节方有之。”

正话间,又见僮仆来安桌子,请吃粥。粥罢之后,雪比早间又大,须臾,平地
有二尺来深。三藏心焦垂泪。陈老道:“老爷放心,莫见雪深忧虑。我舍下颇有几
石粮食,供养得老爷们半生。”三藏道:“老施主不知贫僧之苦。我当年蒙圣恩赐了
旨意,摆大驾亲送出关,唐王御手擎杯奉饯,问道:‘几时可回?’贫僧不知有山
川之险,顺口回奏:‘只消三年,可取经回国。’自别后,今已七八个年头,还未见
佛面,恐违了钦限;又怕的是妖魔凶狠,所以焦虑。今日有缘得寓潭府,昨夜愚徒
们略施小惠报答,实指望求一船只渡河;不期天降大雪,道路迷漫,不知几时才得
功成回故土也!”陈老道:“老爷放心,正是多的日子过了,那里在这几日。且待天
晴,化了冰,老拙倾家费产,必处置送老爷过河。”

只见一僮又请进早斋。到厅上吃毕。叙不多时,又午斋相继而进。三藏见品物
丰盛,再四不安道:“既蒙见留,只可以家常相待。”陈老道:“老爷感蒙替祭救命
之恩,虽逐日设筵奉款,也难酬难谢。”

此后大雪方住,就有人行走。陈老见三藏不快,又打扫花园,大盆架火,请去
雪洞里闲耍散闷。八戒笑道:“那老儿忒没算计!春二三月好赏花园;这等大雪,又
冷,赏玩何物!”行者道:“呆子不知事!雪景自然幽静。一则游赏,二来与师父宽
怀。”陈老道:“正是,正是。”遂此邀请到园。但见:

景值三秋,风光如腊。苍松结玉蕊,衰柳挂银花。阶下玉苔堆粉屑,窗前翠竹
吐琼芽。巧石山头,养鱼池内:巧石山头,削削尖峰排玉笋;养鱼池内,清清活水
作冰盘。临岸芙蓉娇色浅,傍崖木槿嫩枝垂。秋海棠,全然压倒;腊梅树,聊发新
枝。牡丹亭、海榴亭、丹桂亭,亭亭尽鹅毛堆积;放怀处、款客处、遣兴处,处处
皆蝶翅铺漫。两篱黄菊玉绡金,几树丹枫红间白。无数闲庭冷难到,且观雪洞冷如
冰。那里边、放一个兽面象足铜火盆,热烘烘炭火才生;那上下、有几张虎皮搭苫
漆
交椅,软温温纸窗铺设。四壁上,挂几轴名公古画,却是那:七贤过关,寒江独钓,
叠嶂层峦团雪景;苏武餐毡,折梅逢使,琼林玉树写寒文。说不尽那:家近水亭鱼
易买,雪迷山径酒难沽。真个可堪容膝处,算来何用访蓬壶?
众人观玩良久,就于雪洞里坐下,对邻叟道取经之事。又捧香茶饮毕。陈老问:“列
位老爷,可饮酒么?”三藏道:“贫僧不饮,小徒略饮几杯素酒。”陈老大喜,即命:
“取素果品,炖暖酒,与列位汤寒。”那僮仆即抬桌围炉,与两个邻叟,各饮了几
杯,收了家火。

不觉天色将晚,又仍请到厅上晚斋。只听得街上行人都说:“好冷天啊!把通天
河冻住了!”三藏闻言道:“悟空,冻住河,我们怎生是好?”陈老道:“乍寒乍冷,
想是近河边浅水处冻结。”那行人道:“把八百里都冻的似镜面一般,路口上有人走
哩!”三藏听说有人走,就要去看。陈老道:“老爷莫忙。今日晚了,明日去看。”
遂此别却邻叟。又晚斋毕,依然歇在厢房。

及次日天晓,八戒起来道:“师兄,今夜更冷,想必河冻住也。”三藏迎着门,
朝天礼拜道:“众位护教大神,弟子一向西来,虔心拜佛,苦历山川,更无一声报
怨。今至于此,感得皇天佑助,结冻河水,弟子空心权谢,待得经回,奏上唐皇,
竭诚酬答。”礼拜毕,遂教悟净背马,趁冰过河。陈老又道:“莫忙,待几日雪融冰
解,老拙这里办船相送。”沙僧道:“就行也不是话,再住也不是话。口说无凭,耳
闻不如眼见。我背了马,且请师父亲去看看。”陈老道:“言之有理。”教:“小的们,
快去背我们六匹马来!且莫背唐僧老爷马。”

就有六个小价跟随。一行人径往河边来看,真个是:

雪积如山耸,云收破晓晴。寒凝楚塞千峰瘦,冰结江湖一片平。朔风凛凛,滑
冻棱棱。池鱼偎密藻,野鸟恋枯槎。塞外
征夫俱坠指,江头梢子乱敲牙。裂蛇腹,断鸟足,果然冰山千百尺。万壑冷浮银,
一川寒浸玉。东方自信出僵蚕,北地果然有鼠窟。王祥卧,光武渡,一夜溪桥连底
固。曲沼结棱层,深渊重叠。通天阔水更无波,皎洁冰漫如陆路。
三藏与一行人到了河边,勒马观看。真个那路口上有人行走。三藏问道:“施主,
那些人上冰往那里去?”陈老道:“河那边乃西梁女国。这起人都是做买卖的。我
这边百钱之物,到那边可值万钱;那边百钱之物,到这边亦可值万钱。利重本轻,
所以人不顾生死而去。常年家有五七人一船,或十数人一船,飘洋而过。见如今河
道冻住,故舍命而步行也。”三藏道:“世间事惟名利最重。似他为利的,舍死忘生;
我弟子奉旨全忠,也只是为名,与他能差几何!”教:“悟空,快回施主家,收拾行
囊,叩背马匹,趁此层冰,早奔西方去也。”行者笑吟吟答应。

沙僧道:“师父啊,常言道:‘千日吃了千升米。’今已托赖陈府上,且再住几
日,待天晴化冻,办船而过。忙中恐有错也。”三藏道:“悟净,怎么这等愚见!若
是正二月,一日暖似一日,可以待得冻解。此时乃八月,一日冷似一日,如何可便
望解冻!却不又误了半载行程?”

八戒跳下马来:“你们且休讲闲口,等老猪试看有多少厚薄。”行者道:“呆子,
前夜试水,能去抛石;如今冰冻重漫,怎生试得?”八戒道:“师兄不知。等我举
钉钯筑他一下。假若筑破,就是冰薄,且不敢行;若筑不动,便是冰厚,如何不行?”
三藏道:“正是,说得有理。”那呆子撩衣拽步,走上河边,双手举钯,尽力一筑,
只听扑的一声,筑了九个白迹,手也振得生疼。呆子笑道:“去得!去得!连底都锢
住了。”

三藏闻言,十分欢喜,与众同回陈家。只教收拾走路。那两个老者苦留不住,
只得安排些干粮烘炒,做些烧饼馍馍相送。一家子磕头礼拜,又捧出一盘子散碎金
银,跪在面前道:“多蒙老爷活子之恩,聊表途中一饭之敬。”三藏摆手摇头,只是
不受道:“贫僧出家人,财帛何用?就途中也不敢取出。只是以化斋度日为正事。收
了干粮足矣。”二老又再三央求,行者用指尖儿捻了一小块,约有四五钱重,递与
唐僧道:“师父,也只当些衬钱,莫教空负二老之意。”

遂此相向而别。径至河边冰上,那马蹄滑了一滑,险些儿把三藏跌下马来。沙
僧道:“师父,难行!”八戒道:“且住!问陈老官讨个稻草来我用。”行者道:“要稻
草何用?”八戒道:“你那里得知?要稻草包着马蹄方才不滑,免教跌下师父来也。”
陈老在岸上听言,急命人家中取一束稻草,却请唐僧上岸下马。八戒将草包裹马足,
然后踏冰而行。

别陈老,离河边,行有三四里远近,八戒把九环锡杖递与唐僧道:“师父,你
横此在马上。”行者道:“这呆子奸诈!锡杖原是你挑的,如何又叫师父拿着?”八
戒道:“你不曾走过冰凌,不晓得;凡是冰冻之上,必有凌眼;倘或着凌眼,脱
将下去,若没横担之物,骨都的落水,就如一个大锅盖盖住,如何钻得上来!须是
如此架住方可。”行者暗笑道:“这呆子倒是个积年走冰的!”果然都依了他。长老
横担着锡杖,行者横担着铁棒,沙僧横担着降妖宝杖,八戒肩挑着行李,腰横着钉
钯,师徒们放心前进。这一直行到天晚,吃了些干粮,却又不敢久停,对着星月光
华,映的冰冻上亮灼灼、白茫茫,只情奔走,果然是马不停蹄。师徒们莫能合眼,
走了一夜。天明又吃些干粮,望西又进。

正行时,只听得冰底下扑喇喇一声响,险些儿唬倒了白马。三藏大惊道:“徒
弟呀!怎么这般响?”八戒道:“这河忒也冻得结实,地凌响了。或者这半中间连
底通锢住了也。”三藏闻言,又惊又喜,策马前进,趱行不题。

却说那妖邪自从回归水府,引众精在于冰下。等候多时,只听得马蹄响处,他
在底下弄个神通,滑喇的迸开冰冻,慌得孙大圣跳上空中。早把那白马落于水内,
三人尽皆脱下。

那妖邪将三藏捉住,引群精径回水府。厉声高叫:“鳜妹何在?”老鳜婆迎门
施礼道:“大王,不敢!不敢!”妖邪道:“贤妹何出此言!‘一言既出,驷马难追。’
原说听从汝计,捉了唐僧,与你拜为兄妹。今日果成妙计,捉了唐僧,就好昧了前
言?”教:“小的们,抬过案桌,磨快刀来,把这和尚剖腹剜心,剥皮剐肉;一壁
厢响动乐器,与贤妹共而食之,延寿长生也。”鳜婆道:“大王,且休吃他,恐他徒
弟们寻来吵闹,且宁耐两日,让那厮不来寻,然后剖开,请大王上坐,众眷族环列,
吹弹歌舞,奉上大王,从容自在享用,却不好也?”那怪依言,把唐僧藏于宫后,
使一个六尺长的石匣,盖在中间不题。

却说八戒、沙僧,在水里捞着行囊,放在白马身上驮了。分开水路,涌浪翻波,
负水而出。只见行者在半空中看见,问道:“师父何在?”八戒道:“师父姓‘陈’,
名‘到底’了。如今没处找寻,且上岸再作区处。”原来八戒本是天蓬元帅临凡,
他当年掌管天河八万水兵大众;沙和尚是流沙河内出身;白马本是西海龙孙:故此
能知水性。大圣在空中指引。须臾,回转东崖,晒刷了马匹,掠了衣裳,大圣云
头按落,一同到于陈家庄上。

早有人报与二老道:“四个取经的老爷,如今只剩了三个来也。”兄弟即忙接出
门外,果见衣裳还湿,道:“老爷们,我等那般苦留,却不肯住,只要这样方休。
怎么不见三藏老爷?”八戒道:“不叫做三藏了,改名叫做‘陈到底’也。”二老垂
泪道:“可怜!可怜!我说等雪融备船相送,坚执不从,致令丧了性命!”行者道:“老
儿,莫替古人耽忧。我师父管他不死长命。老孙知道,决然是那灵感大王弄法算计
去了。你且放心,与我们浆浆衣服,晒晒关文,取草料喂着白马,等我弟兄寻着那
厮,救出师父,索性剪草除根,替你一庄人除了后患,庶几永永得安生也。”陈老
闻言,满心欢喜,即命安排斋供。

兄弟三人,饱餐一顿。将马匹、行囊,交与陈家看守。各整兵器,径赴道边寻
师擒怪。正是:
误踏层冰伤本性,大丹脱漏怎周全?

毕竟不知怎么救得唐僧,且听下回分解。