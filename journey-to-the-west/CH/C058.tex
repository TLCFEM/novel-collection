\chapter{二心搅乱大乾坤~一体难修真寂灭}

这行者与沙僧拜辞了菩萨,纵起两道祥光,离了南海。原来行者筋斗云快,沙
和尚仙云觉迟,行者就要先行。沙僧扯住道:“大哥不必这等藏头露尾,先去安根。
待小弟与你一同走。”大圣本是良心,沙僧却有疑意。真个二人同驾云而去。不多
时,果见花果山。按下云头,二人洞外细看,果见一个行者,高坐石台之上,与群
猴饮酒作乐。模样与大圣无异:也是黄发金箍,金睛火眼;身穿也是绵布直裰,腰
系虎皮裙;手中也拿一条儿金箍铁棒,足下也踏一双麂皮靴;也是这等毛脸雷公嘴,
朔腮别土星,查耳额颅阔,獠牙向外生。

这大圣怒发,一撒手,撇了沙和尚,掣铁棒上前骂道:“你是何等妖邪,敢变
我的相貌,敢占我的儿孙,擅居吾仙洞,擅作这威福!”那行者见了,公然不答,
也使铁棒来迎。二行者在一处,果是不分真假。好打呀:

两条棒,二猴精,这场相敌实非轻。都要护持唐御弟,各施功绩立英名。真猴
实受沙门教,假怪虚称佛子情。盖为神通多变化,无真无假两相平。一个是混元一
气齐天圣,一个是久炼千灵缩地精。这个是如意金箍棒,那个是随心铁杆兵。隔架
遮拦无胜败,撑持抵敌没输赢。先前交手在洞外,少顷争持起半空。

他两个各踏云光,跳斗上九霄云内。沙僧在旁,不敢下手,见他们战此一场,
诚然难认真假;欲待拔刀相助,又恐伤了真的。忍耐良久,且纵身跳下山崖,使降
妖宝杖,打近水帘洞外,惊散群妖,掀翻石凳,把饮酒食肉的器皿,尽情打碎;寻
他的青毡包袱,四下里全然不见。原来他水帘洞本是一股瀑布飞泉,遮挂洞门,远
看似一条白布帘儿,近看乃是一股水脉,故曰水帘洞。沙僧不知进步来历,故此难
寻。即便纵云,赶到九霄云里,轮着宝杖,又不好下手。

大圣道:“沙僧,你既助不得力,且回复师父,说我等这般这般,等老孙与此
妖打上南海落伽山菩萨前辨个真假。”道罢,那行者也如此说。沙僧见两个相貌声
音,更无一毫差别,皂白难分,只得依言,拨转云头,回复唐僧不题。

你看那两个行者,且行且斗,直嚷到南海,径至落伽山,打打骂骂,喊声不绝。
早惊动护法诸天,即报入潮音洞里道:“菩萨,果然两个孙悟空打将来也。”那菩萨
与木叉行者、善财童子、龙女降莲台出门喝道:“那孽畜那里走!”这两个递相揪住
道:“菩萨,这厮果然像弟子模样。才自水帘洞打起,战斗多时,不分胜负。沙悟
净肉眼愚蒙,不能分识,有力难助,是弟子教他回西路去回复师父,我与这厮打到
宝山,借菩萨慧眼,与弟子认个真假,辨明邪正。”道罢,那行者也如此说一遍。
众诸天与菩萨都看良久,莫想能认。菩萨道:“且放了手,两边站下,等我再看。”
果然撒手,两边站定。这边说:“我是真的!”那边说:“他是假的!”

菩萨唤木叉与善财上前,悄悄吩咐:“你一个帮住一个,等我暗念紧箍儿咒,
看那个害疼的便是真,不疼的便是假。”他二人果各帮一个。菩萨暗念真言,两个
一齐喊疼,都抱着头,地下打滚,只叫:“莫念,莫念!”菩萨不念,他两个又一齐
揪住,照旧嚷斗。菩萨无计奈何,即令诸天、木叉,上前助力。众神恐伤真的,亦
不敢下手。菩萨叫声:“孙悟空”,两个一齐答应。菩萨道:“你当年官拜‘弼马温’,
大闹天宫时,神将皆认得你;你且上界去分辨回话。”这大圣谢恩,那行者也谢恩。

二人扯扯拉拉,口里不住的嚷斗,径至南天门外,慌得那广目天王帅马、赵、
温、关四大天将,及把门大小众神,各使兵器挡住道:“那里走,此间可是争斗之
处!”大圣道:“我因保护唐僧往西天取经,在路上打杀贼徒,那三藏赶我回去,我
径到普陀崖见观音菩萨诉告,不想这妖精,几时就变作我的模样,打倒唐僧,抢去
包袱。有沙僧至花果山寻讨,只见这妖精占了我的巢穴。后到普陀崖告请菩萨,又
见我侍立台下,沙僧诳说是我驾筋斗云,又先在菩萨处遮饰。菩萨却是个正明,不
听沙僧之言,命我同他到花果山看验。原来这妖精果像老孙模样。才自水帘洞打到
普陀山见菩萨,菩萨也难识认,故打至此间,烦诸天眼力,与我认个真假。”道罢,
那行者也似这般这般说了一遍。众天神看够多时,也不能辨。他两个喝道:“你
们既不能认,让开路,等我们去见玉帝!”

众神搪抵不住,放开天门,直至灵霄宝殿。马元帅同张、葛、许、邱四天师奏
道:“下界有一般两个孙悟空,打进天门,口称见王。”说不了,两个直嚷将进来,
唬得那玉帝即降立宝殿,问曰:“你两个因甚事擅闹天宫,嚷至朕前寻死!”大圣口
称:“万岁,万岁!臣今皈命,秉教沙门,再不敢欺心诳上;只因这个妖精变作臣的
模样,……”如此如彼,把前情备陈了一遍。“……指望与臣辨个真假!”那行者也
如此陈了一遍。玉帝即传旨宣托塔李天王,教:“把‘照妖镜’来照这厮谁真谁假,
教他假灭真存。”天王即取镜照住,请玉帝同众神观看。镜中乃是两个孙悟空的影
子;金箍、衣服,毫发不差。玉帝亦辨不出,赶出殿外。这大圣呵呵冷笑,那行者
也哈哈欢喜,揪头抹颈,复打出天门,坠落西方路上道:“我和你见师父去!我和你
见师父去!”

却说那沙僧自花果山辞他两个,又行了三昼夜,回至本庄,把前事对唐僧说了
一遍。唐僧自家悔恨道:“当时只说是孙悟空打我一棍,抢去包袱,岂知却是妖精
假变的行者!”沙僧又告道:“这妖又假变一个长老,一匹白马;又有一个八戒挑着
我们包袱,又有一个变作是我。我忍不住恼怒,一杖打死,原是一个猴精。因此惊
散,又到菩萨处诉告。菩萨着我与师兄又同去识认,那妖果与师兄一般模样。我难
助力,故先来回复师父。”三藏闻言,大惊失色。八戒哈哈大笑道:“好,好,好!
应了这施主家婆婆之言了!他说有几起取经的,这却不又是一起?”

那家子老老小小的,都来问沙僧:“你这几日往何处讨盘缠去的?”沙僧笑道:
“我往东胜神洲花果山寻大师兄取讨行李,又到南海普陀山拜见观音菩萨,却又到
花果山,方才转回至此。”那老者又问:“往返有多少路程?”沙僧道:“约有二十
余万里。”老者道:“爷爷呀,似这几日,就走了这许多路,只除是驾云,方能够得
到!”八戒道:“不是驾云,如何过海?”沙僧道:“我们那算得走路,若是我大师
兄,只消一二日,可往回也。”那家子听言,都说是神仙。八戒道:“我们虽不是神
仙,神仙还是我们的晚辈哩!”

正说间,只听半空中喧哗人嚷。慌得都出来看,却是两个行者打将来。八戒见
了,忍不住手痒道:“等我去认认看。”好呆子,急纵身跳起,望空高叫道:“师兄
莫嚷,我老猪来也!”那两个一齐应道:“兄弟,来打妖精,来打妖精!”那家子又
惊又喜道:“是几位腾云驾雾的罗汉歇在我家!就是发愿斋僧的,也斋不着这等好
人!”更不计较茶饭,愈加供养。又说:“这两个行者只怕斗出不好来,地覆天翻,
作祸在那里!”三藏见那老者当面是喜,背后是忧,即开言道:“老施主放心,莫生
忧叹。贫僧收伏了徒弟,去恶归善,自然谢你。”那老者满口回答道:“不敢,不敢!”
沙僧道:“施主休讲,师父可坐在这里,等我和二哥去,一家扯一个来到你面前,
你就念念那话儿,看那个害疼的就是真的,不疼的就是假的。”三藏道:“言之极当。”

沙僧果起在半空道:“二位住了手,我同你到师父面前辨个真假去。”这大圣放
了手,那行者也放了手。沙僧搀住一个,叫道:“二哥,你也搀住一个。”果然搀住,
落下云头,径至草舍门外。三藏见了,就念紧箍儿咒。二人一齐叫苦道:“我们这
等苦斗,你还咒我怎的?莫念,莫念!”那长老本心慈善,遂住了口不念,却也不认
得真假。他两个挣脱手,依然又打。这大圣道:“兄弟们,保着师父,等我与他打
到阎王前折辨去也!”那行者也如此说。二人抓抓,须臾,又不见了。

八戒道:“沙僧,你既到水帘洞,看见‘假八戒’挑着行李,怎么不抢将来?”
沙僧道:“那妖精见我使宝杖打他‘假沙僧’,他就乱围上来要拿,是我顾性命走了。
及告菩萨,与行者复至洞口,他两个打在空中,是我去掀翻他的石凳,打散他的小
妖,只见一股瀑布泉水流,竟不知洞门开在何处,寻不着行李,所以空手回复师命
也。”八戒道:“你原来不晓得。我前年请他去时,先在洞门外相见;后被我说泛了
他,他就跳下,去洞里换衣来时,我看见他将身往水里一钻。那一股瀑布水流,就
是洞门。想必那怪将我们包袱收在那里面也。”三藏道:“你既知此门,你可趁他都
不在家,可先到他洞里取出包袱,我们往西天去罢。他就来,我也不用他了。”八
戒道:“我去。”沙僧说:“二哥,他那洞前有千数小猴,你一人恐弄他不过,反为
不美。”八戒笑道:“不怕!不怕!”急出门,纵着云雾,径上花果山寻取行李不题。

却说那两个行者又打嚷到阴山背后,唬得那满山鬼战战兢兢,藏藏躲躲。有先
跑的,撞入阴司门里,报上森罗宝殿道:“大王,背阴山上,有两个齐天大圣打得
来也!”慌得那第一殿秦广王传报与二殿楚江王、三殿宋帝王、四殿卞城王、五殿
阎罗王、六殿平等王、七殿泰山王、八殿都市王、九殿忤官王、十殿转轮王。一殿
转一殿,霎时间,十王会齐,又着人飞报与地藏王。尽在森罗殿上,点聚阴兵,等
擒真假。只听得那强风滚滚,惨雾漫漫,二行者一翻一滚的,打至森罗殿下。

阴君近前挡住道:“大圣有何事,闹我幽冥?”这大圣道:“我因保唐僧西天取
经,路过西梁国,至一山,有强贼截劫我师,是老孙打死几个,师父怪我,把我逐
回。我随到南海菩萨处诉告,不知那妖精怎么就绰着口气,假变作我的模样,在半
路上打倒师父,抢夺了行李。师弟沙僧,向我本山取讨包袱,这妖假立师名,要往
西天取经。沙僧逃遁至南海见菩萨,我正在侧。他备说原因,菩萨又命我同他至花
果山观看,果被这厮占了我巢穴。我与他争辨到菩萨处,其实相貌、言语等俱一般,
菩萨也难辨真假。又与这厮打上天堂,众神亦果难辨,因见我师。我师念紧箍咒试
验,与我一般疼痛。故此闹至幽冥,望阴君与我查看生死簿,看‘假行者’是何出
身,快早追他魂魄,免教二心沌乱。”那怪亦如此说一遍。阴君闻言,即唤管簿判
官一一从头查勘,更无个“假行者”之名。再看毛虫文簿,那猴子一百三十条已是
孙大圣幼年得道之时,大闹阴司,消死名一笔勾之,自后来凡是猴属,尽无名号。
查勘毕,当殿回报。阴君各执笏,对行者道:“大圣,幽冥处既无名号可查,你还
到阳间去折辨。”

正说处,只听得地藏王菩萨道:“且住!且住!等我着谛听与你听个真假。”原来
那谛听是地藏菩萨经案下伏的一个兽名。他若伏在地下,一霎时,将四大部洲山川
社稷,洞天福地之间,蠃虫、鳞虫、毛虫、羽虫、昆虫、天仙、地仙、神仙、人仙、
鬼仙可以照鉴善恶,察听贤愚。那兽奉地藏钧旨,就于森罗庭院之中,俯伏在地。
须臾,抬起头来,对地藏道:“怪名虽有,但不可当面说破,又不能助力擒他。”地
藏道:“当面说出便怎么?”谛听道:“当面说出,恐妖精恶发,搔扰宝殿,致令阴
府不安。”又问:“何为不能助力擒拿?”谛听道:“妖精神通,与孙大圣无二。幽
冥之神,能有多少法力,故此不能擒拿。”地藏道:“似这般怎生祛除?”谛听言:
“佛法无边。”地藏早已省悟。即对行者道:“你两个形容如一,神通无二,若要辨
明,须到雷音寺释迦如来那里,方得明白。”两个一齐嚷道:“说的是!说的是!我和
你西天佛祖之前折辨去!”那十殿阴君送出,谢了地藏,回上翠云宫,着鬼使闭了
幽冥关隘不题。

看那两个行者,飞云奔雾,打上西天。有诗为证,诗曰:
人有二心生祸灾,天涯海角致疑猜。
欲思宝马三公位,又忆金銮一品台。
南征北讨无休歇,东挡西除未定哉。
禅门须学无心诀,静养婴儿结圣胎。
他两个在那半空里,扯扯拉拉,抓抓,且行且斗。直嚷至大西天灵鹫仙山雷音
宝刹之外。早见那四大菩萨、八大金刚、五百阿罗、三千揭谛、比丘尼、比丘僧、
优婆塞、优婆夷诸大圣众,都到七宝莲台之下,各听如来说法。那如来正讲到这:

不有中有,不无中无。不色中色,不空中空。非有为有,非无为无。非色为色,
非空为空。空即是空,色即是色。色无定色,色即是空。空无定空,空即是色。知
空不空,知色不色。名为照了,始达妙音。
概众稽首皈依。流通诵读之际,如来降天花普散缤纷,即离宝座,对大众道:“汝
等俱是一心,且看二心竞斗而来也。”

大众举目看之,果是两个行者,天喝地,打至雷音胜境。慌得那八大金刚,
上前挡住道:“汝等欲往那里去?”这大圣道:“妖精变作我的模样,欲至宝莲台下,
烦如来为我辨个虚实也。”

众金刚抵挡不住。直嚷至台下,跪于佛祖之前,拜告道:“弟子保护唐僧,来
造宝山,求取真经,一路上炼魔缚怪,不知费了多少精神。前至中途,偶遇强徒劫
掳,委是弟子二次打伤几人。师父怪我赶回,不容同拜如来金身。弟子无奈,只得
投奔南海,见观音诉苦。不期这个妖精,假变弟子声音、相貌,将师父打倒,把行
李抢去。师弟悟净寻至我山,被这妖假捏巧言,说有真僧取经之故。悟净脱身至南
海,备说详细。观音知之,遂令弟子同悟净再至我山。因此,两人比并真假,打至
南海,又打到天宫,又曾打见唐僧,打见冥府,俱莫能辨认。故此大胆轻造,千乞
大开方便之门,广垂慈悯之念,与弟子辨明邪正,庶好保护唐僧亲拜金身,取经回
东土,永扬大教。”

大众听他两张口一样声俱说一遍,众亦莫辨;惟如来则通知之。正欲道破,忽
见南下彩云之间,来了观音,参拜我佛。

我佛合掌道:“观音尊者,你看那两个行者,谁是真假?”菩萨道:“前日在弟
子荒境,委不能辨。他又至天宫、地府,亦俱难认。特来拜告如来,千万与他辨明
辨明。”如来笑道:“汝等法力广大,只能普阅周天之事,不能遍识周天之物,亦不
能广会周天之种类也。”菩萨又请示周天种类。如来才道:“周天之内有五仙:乃天、
地、神、人、鬼。有五虫:乃蠃、鳞、毛、羽、昆。这厮非天、非地、非神、非人、
非鬼;亦非蠃、非鳞、非毛、非羽、非昆。又有四猴混世,不入十类之种。”

菩萨道:“敢问是那四猴?”如来道:“第一是灵明石猴,通变化,识天时,知
地利,移星换斗;第二是赤尻马猴,晓阴阳,会人事,善出入,避死延生;第三是
通臂猿猴,拿日月,缩千山,辨休咎,乾坤摩弄;第四是六耳猕猴,善聆音,能察
理,知前后,万物皆明。此四猴者,不入十类之种,不达两间之名。我观‘假悟空’
乃六耳猕猴也。此猴若立一处,能知千里外之事;凡人说话,亦能知之;故此善聆
音,能察理,知前后,万物皆明。与真悟空同象同音者,六耳猕猴也。”

那猕猴闻得如来说出他的本象,胆战心惊,急纵身,跳起来就走。如来见他走
时,即令大众下手。早有四菩萨、八金刚、五百阿罗、三千揭谛、比丘僧、比丘尼、
优婆塞、优婆夷、观音、木叉,一齐围绕。孙大圣也要上前。如来道:“悟空休动
手,待我与你擒他。”那猕猴毛骨悚然,料着难脱,即忙摇身一变,变作个蜜蜂儿,
往上便飞。如来将金钵盂撇起去,正盖着那蜂儿,落下来。大众不知,以为走了。
如来笑云:“大众休言。妖精未走,见在我这钵盂之下。”大众一发上前,把钵盂揭
起,果然见了本象,是一个六耳猕猴。孙大圣忍不住,轮起铁棒,劈头一下打死,
至今绝此一种。

如来不忍,道声:“善哉!善哉!”大圣道:“如来不该慈悯他。他打伤我师父,
抢夺我包袱,依律问他个得财伤人,白昼抢夺,也该个斩罪哩!”如来道:“你自快
去保护唐僧来此求经罢。”大圣叩头谢道:“上告如来得知。那师父定是不要我;我
此去,若不收留,却不又劳一番神思!望如来方便,把松箍儿咒念一念,褪下这个
金箍,交还如来,放我还俗去罢。”如来道:“你休乱想,切莫放刁。我教观音送你
去,不怕他不收。好生保护他去,那时功成归极乐,汝亦坐莲台。”

那观音在旁听说,即合掌谢了圣恩。领悟空,辄驾云而去。随后木叉行者、白
鹦哥,一同赶上。不多时,到了中途草舍人家。沙和尚看见,急请师父拜门迎接。
菩萨道:“唐僧,前日打你的,乃‘假行者’六耳猕猴也。幸如来知识,已被悟空
打死。你今须是收留悟空。一路上魔障未消,必得他保护你,才得到灵山,见佛取
经。再休嗔怪。”三藏叩头道:“谨遵教旨。”

正拜谢时,只听得正东上狂风滚滚,众目视之,乃猪八戒背着两个包袱,驾风
而至。呆子见了菩萨,倒身下拜道:“弟子前日别了师父至花果山水帘洞寻得包袱,
果见一个‘假唐僧’、‘假八戒’,都被弟子打死,原是两个猴身。却入里,方寻着
包袱。当时查点,一物不少。却驾风转此。更不知两行者下落如何。”菩萨把如来
识怪之事,说了一遍。那呆子十分欢喜,称谢不尽。师徒们拜谢了,菩萨回海,却
都照旧合意同心,洗冤解怒。又谢了那村舍人家,整束行囊、马匹,找大路而西。
正是:
中道分离乱五行,降妖聚会合元明。
神归心舍禅方定,六识祛降丹自成。

毕竟这去,不知三藏几时得面佛求经,且听下回分解。