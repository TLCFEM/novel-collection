\chapter{陷虎穴金星解厄~双叉岭伯钦留僧}

诗曰:
大有唐王降敕封,钦差玄奘问禅宗。
坚心磨琢寻龙穴,着意修持上鹫峰。
边界远游多少国,云山前度万千重。
自今别驾投西去,秉教迦持悟大空。

却说三藏自贞观十三年九月望前三日,蒙唐王与多官送出长安关外。一二日马
不停蹄,早至法门寺。本寺住持上房长老,滞头众僧有五百余人,两边罗列,接至
里面,相见献茶。茶罢进斋。斋后不觉天晚。正是那:
影动星河近,月明无点尘。
雁声鸣远汉,砧韵响西邻。
归鸟栖枯树,禅僧讲梵音。
蒲团一榻上,坐到夜将分。
众僧们灯下议论佛门定旨,上西天取经的原由。有的说水远山高,有的说路多虎豹;
有的说峻岭陡崖难度,有的说毒魔恶怪难降。三藏箝口不言,但以手指自心,点头
几度。众僧们莫解其意,合掌请问道:“法师指心点头者,何也?”三藏答曰:“心
生,种种魔生;心灭,种种魔灭。我弟子曾在化生寺对佛设下洪誓大愿,不由我不
尽此心。这一去,定要到西天,见佛求经,使我们法轮回转,愿圣主皇图永固。”
众僧闻得此言,人人称羡,个个宣扬,都叫一声“忠心赤胆大阐法师!”夸赞不尽,
请师入榻安寐。

早又是竹敲残月落,鸡唱晓云生。那众僧起来,收拾茶水早斋。玄奘遂穿了袈
裟,上正殿,佛前礼拜,道:“弟子陈玄奘,前往西天取经,但肉眼愚迷,不识活
佛真形。今愿立誓:路中逢庙烧香,遇佛拜佛,遇塔扫塔。但愿我佛慈悲,早现丈
六金身,赐真经,留传东土。”祝罢,回方丈进斋。斋毕,那二从者整顿了鞍马,
促趱行程。三藏出了山门,辞别众僧。众僧不忍分别,直送有十里之遥,噙泪而返。
三藏遂直西前进。正是那季秋天气。但见:

数村木落芦花碎,几树枫杨红叶坠。路途烟雨故人稀,黄菊丽,山骨细,水寒
荷破人憔悴。
白苹红蓼霜天雪,落霞孤鹜长空坠。依稀黯淡野云飞,玄鸟去,宾
鸿至,嘹嘹呖呖声宵碎。

师徒们行了数日,到了巩州城。早有巩州合属官吏人等,迎接入城中。安歇一
夜,次早出城前去。一路饥餐渴饮,夜住晓行。两三日,又至河州卫。此乃是大唐
的山河边界。早有镇边的总兵与本处僧道,闻得是钦差御弟法师,上西方见佛,无
不恭敬;接至里面供给了,着僧纲请往福原寺安歇。本寺僧人,一一参见,安排晚
斋。斋毕,吩咐二从者饱喂马匹,天不明就行。及鸡方鸣,随唤从者,却又惊动寺
僧,整治茶汤斋供。斋罢,出离边界。

这长老心忙,太起早了。原来此时秋深时节,鸡鸣得早,只好有四更天气。一
行三人,连马四口,迎着清霜,看着明月,行有数十里远近,见一山岭,只得拨草
寻路,说不尽崎岖难走,又恐怕错了路径。正疑思之间,忽然失足,三人连马都跌
落坑坎之中。三藏心慌,从者胆战。却才悚惧,又闻得里面哮吼高呼,叫:“拿将
来!拿将来!”只见狂风滚滚,拥出五六十个妖邪,将三藏、从者揪了上去。这法师
战战兢兢的,偷眼观看,上面坐的那魔王,十分凶恶。真个是:
雄威身凛凛,猛气貌堂堂。
电目飞光艳,雷声振四方。
锯牙舒口外,凿齿露腮旁。
锦绣围身体,文斑裹脊梁。
钢须稀见肉,钩爪利如霜。
东海黄公惧,南山白额王。
唬得个三藏魂飞魄散,二从者骨软筋麻。魔王喝令绑了,众妖一齐将三人用绳索绑
缚。正要安排吞食,只听得外面喧哗,有人来报:“熊山君与特处士二位来也。”
三藏闻言,抬头观看,前走的是一条黑汉。你道他是怎生模样:
雄豪多胆量,轻健夯身躯。
涉水惟凶力,跑林逞怒威。
向来符吉梦,今独露英姿。
绿树能攀折,知寒善谕时。
准灵惟显处,故此号山君。
又见那后边来的是一条胖汉。你道怎生模样:
嵯峨双角冠,端肃耸肩背。
性服青衣稳,蹄步多迟滞。
宗名父作牯,原号母称。
能为田者功,因名特处士。

这两个摇摇摆摆,走入里面,慌得那魔王奔出迎接。熊山君道:“寅将军,一
向得意,可贺,可贺!”特处士道:“寅将军丰姿胜常,真可喜,真可喜!”魔王
道:“二公连日如何?”山君道:“惟守素耳。”处士道:“惟随时耳。”三个叙
罢,各坐谈笑。

只见那从者绑得痛切悲啼。那黑汉道:“此三者何来?”魔王道:“自送上门
来者。”处士笑云:“可能待客否?”魔王道:“奉承,奉承!”山君道:“不可
尽用,食其二,留其一可也。”魔王领诺,即呼左右,将二从者剖腹剜心,剁碎其
尸。将首级与心肝奉献二客,将四肢自食,其余骨肉,分给各妖。只听得之声,
真似虎啖羊羔。霎时食尽。把一个长老几乎唬死。这才是初出长安第一场苦难。

正怆慌之间,渐渐的东方发白,那二怪至天晓方散。俱道:“今日厚扰,容日
后竭诚奉酬。”方一拥而退。不一时,红日高升。三藏昏昏沉沉,也辨不得东西南
北。正在那不得命处,忽然见一老叟,手持拄杖而来。走上前,用手一拂,绳索皆
断。对面吹了一口气,三藏方苏。跪拜于地道:“多谢老公公!搭救贫僧性命!”老
叟答礼道:“你起来。你可曾疏失了甚么东西?”三藏道:“贫僧的从人,已是被
怪食了;只不知行李、马匹在于何处?”老叟用杖指定道:“那厢不是一匹马,两
个包袱?”三藏回头看时,果是他的物件,并不曾失落,心才略放下些。问老叟曰:
“老公公,此处是甚所在?公公何由在此?”老叟道:“此是双叉岭,乃虎狼巢穴处。
你为何堕此?”三藏道:“贫僧鸡鸣时,出河州卫界,不料起得早了,冒霜拨露,
忽失落此地。见一魔王,凶顽太甚。将贫僧与二从者绑了。又见一条黑汉,称是熊
山君;一条胖汉,称是特处士;走进来,称那魔王是寅将军。他三个把我二从者吃
了,天光才散。不想我是那里有这大缘大分,感得老公公来此救我?”老叟道:“处
士者是个野牛精。山君者是个熊罴精。寅将军者是个老虎精。左右妖邪,尽都是山
精树鬼,怪兽苍狼。只因你的本性元明,所以吃不得你。你跟我来,引你上路。”
三藏不胜感激,将包袱捎在马上,牵著缰绳,相随老叟径出了坑坎之中,走上大路。
却将马拴在道旁草头上,转身拜谢那公公,那公公遂化作一阵清风,跨一只朱顶白
鹤,腾空而去。只见风飘飘遗下一张简帖,书上四句颂子。颂子云:
吾乃西天太白星,特来搭救汝生灵。
前行自有神徒助,莫为艰难报怨经。
三藏看了,对天礼拜道:“多谢金星,度脱此难。”拜毕,牵了马匹,独自个孤孤
凄凄,往前苦进。这岭上,真个是:

寒飒飒雨林风,响潺潺涧下水。香馥馥野花开,密丛丛乱石磊。闹嚷嚷鹿与猿,
一队队獐和麂。喧杂杂鸟声多,静悄悄人事靡。那长老,战兢兢心不宁;这马儿,
力怯怯蹄难举。
三藏舍身拚命。上了那峻岭之间。行经半日,更不见个人烟村舍。一则腹中饥了,
二则路又不平。正在危急之际,只见前面有两只猛虎咆哮,后边有几条长蛇盘绕。
左有毒虫,右有怪兽。三藏孤身无策,只得放下身心,听天所命。又无奈那马腰软
蹄弯,便屎俱下,伏倒在地,打又打不起,牵又牵不动。苦得个法师衬身无地,真
个有万分凄楚,已自分必死,莫可奈何。却说他虽有灾,却有救应。正在那不得
命处,忽然见毒虫奔走,妖兽飞逃;猛虎潜踪,长蛇隐迹。三藏抬头看时,只见一
人,手执钢叉,腰悬弓箭,自那山坡前转出,果然是一条好汉。你看他:

头上戴一顶,艾叶花斑豹皮帽;身上穿一领,羊绒织锦叵罗衣;腰间束一条狮
蛮带,脚下一对麂皮靴。环眼圆睛如吊客,圈须乱扰似河奎。悬一囊毒药弓矢,
拿一杆点钢大叉。雷声震破山虫胆,勇猛惊残野雉魂。
三藏见他来得渐近,跪在路旁,合掌高叫道:“大王救命,大王救命!”那条汉到
边前,放下钢叉,用手搀起道:“长老休怕。我不是歹人,我是这山中的猎户,姓
刘名伯钦,绰号镇山太保。我才自来,要寻两只山虫食用,不期遇著你,多有冲撞。”
三藏道:“贫僧是大唐驾下钦差往西天拜佛求经的和尚。适间来到此处,遇著些狼
虎蛇虫,四边围绕,不能前进。忽见太保来,众兽皆走,救了贫僧性命,多谢,多
谢!”伯钦道:“我在这里住人,专倚打些狼虎为生,捉些蛇虫过活,故此众兽怕
我走了。你既是唐朝来的,与我都是乡里。此间还是大唐的地界,我也是唐朝的百
姓,我和你同食皇王的水土,诚然是一国之人,你休怕,跟我来。到我舍下歇马,
明朝我送你上路。”三藏闻言,满心欢喜,谢了伯钦,牵马随行。

过了山坡,又听得呼呼风响。伯钦道:“长老休走,坐在此间。风响处,是个
山猫来了。等我拿他家去管待你。”三藏见说,又胆战心惊,不敢举步。那太保执
了钢叉,拽开步,迎将上去。只见一只斑斓虎,对面撞见。他看见伯钦,急回头就
走。这太保霹雳一声,咄道:“那业畜!那里走!”那虎见赶得急,转身轮爪扑来。
这太保三股叉举手迎敌,唬得个三藏软瘫在草地。这和尚自出娘肚皮,那曾见这样
凶险的勾当?太保与那虎在那山坡下,人虎相持,果是一场好斗。但见:

怒气纷纷,狂风滚滚:怒气纷纷,太保冲冠多膂力;狂风滚滚,斑彪逞势喷红
尘。那一个张牙舞爪,这一个转步回身。三股叉擎天幌日,千花尾扰雾飞云。这一
个当胸乱刺,那一个劈面来吞。闪过的再生人道,撞着的定见阎君。只听得那斑彪
哮吼,太保声哏。斑彪哮吼,振裂山川惊鸟兽;太保声哏,喝开天府现星辰。那一
个金睛怒出,这一个壮胆生嗔。可爱镇山刘太保,堪夸据地兽之君。人虎贪生争胜
负,些儿有慢丧三魂。
他两个斗了有一个时辰,只见那虎爪慢腰松,被太保举叉平胸刺倒,可怜呵,钢叉
尖穿透心肝,霎时间血流满地。揪著耳朵,拖上路来,好男子!气不连喘,面不改色,
对三藏道:“造化,造化!这只山猫,够长老食用几日。”三藏夸赞不尽,道:“太
保真山神也!”伯钦道:“有何本事,敢劳过奖?这个是长老的洪福。去来!赶早儿
剥了皮,煮些肉,管待你也。”他一只手执著叉,一只手拖着虎,在前引路。三藏
牵着马,随后而行。迤逦行过山坡,忽见一座山庄。那门前真个是:

参天古树,漫路荒藤。万壑风尘冷,千崖气象奇。一径野花香袭体,数竿幽竹
绿依依。草门楼,篱笆院,堪描堪画;石板桥,白土壁,真乐真稀。秋容萧索,爽
气孤高。道傍黄叶落,岭上白云飘。疏林内山禽聒聒,庄门外细犬嘹嘹。
伯钦到了门首,将死虎掷下,叫:“小的们何在?”只见走出三四个家僮,都是怪
形恶相之类,上前拖拖拉拉,把只虎扛将进去。伯钦吩咐教:“赶早剥了皮,安排
将来待客。”复回头迎接三藏进内。彼此相见。三藏又拜谢伯钦厚恩怜悯救命。伯
钦道:“同乡之人,何劳致谢。”

坐定茶罢,有一老妪,领着一个媳妇,对三藏进礼。伯钦道:“此是家母、山
妻。”三藏道:“请令堂上坐,贫僧奉拜。”老妪道:“长老远客,各请自珍,不
劳拜罢。”伯钦道:“母亲呵,他是唐王驾下,差往西天见佛求经者。适间在岭头
上遇着孩儿,孩儿念一国之人,请他来家歇马,明日送他上路。”老妪闻言,十分
欢喜道:“好,好,好!就是请他,不得这般恰好。明日你父亲周忌,就浼长老做些
好事,念卷经文,到后日送他去罢。”这刘伯钦,虽是一个杀虎手,镇山的太保,
他却有些孝顺之心。闻得母言,就要安排香纸,留住三藏。

话说间,不觉的天色将晚。小的们排开桌凳,拿几盘烂熟虎肉,热腾腾的放在
上面。伯钦请三藏权用,再另办饭。三藏合掌当胸道:“善哉!贫僧不瞒太保说,自
出娘胎,就做和尚,更不晓得吃荤。”伯钦闻得此说,沉吟了半晌道:“长老,寒
家历代以来,不晓得吃素;就是有些竹笋,采些木耳,寻些干菜,做些豆腐,也都
是獐鹿虎豹的油煎,却无甚素处。有两眼锅灶,也都是油腻透了,这等奈何?反是我
请长老的不是。”三藏道:“太保不必多心,请自受用。我贫僧就是三五日不吃饭,
也可忍饿,只是不敢破了斋戒。”伯钦道:“倘或饿死,却如之何?”三藏道:“感
得太保天恩,搭救出虎狼丛里,就是饿死,也强如喂虎。”伯钦的母亲闻说,叫道:
“孩儿不要与长老闲讲,我自有素物,可以管待。”伯钦道:“素物何来?”母亲
道:“你莫管我,我自有素的。”叫媳妇将小锅取下,着火烧了油腻,刷了又刷,
洗了又洗,却仍安在灶上,先烧半锅滚水,别用;却又将些山地榆叶子,着水煎作
茶汤;然后将些黄粱粟米,煮起饭来;又把些干菜煮熟;盛了两碗,拿出来铺在桌
上。老母对着三藏道:“长老请斋。这是老身与儿妇,亲自动手整理的些极洁极净
的茶饭。”三藏下来谢了,方才上坐。那伯钦另设一处,铺排些没盐没酱的老虎肉、
香獐肉、蟒蛇肉,狐狸肉、兔肉,点剁鹿肉干巴,满盘满碗的,陪着三藏吃斋。方
坐下,心欲举箸,只见三藏合掌诵经,唬得个伯钦不敢动箸,急起身立在旁边。三
藏念不数句,却教“请斋”。伯钦道:“你是个念短头经的和尚?”三藏道:“此
非是经,乃是一卷揭斋之咒。”伯钦道:“你们出家人,偏有许多计较,吃饭便也
念诵念诵。”

吃了斋饭,收了盘碗,渐渐天晚,伯钦引着三藏出中宅,到后边走走。穿过夹
道,有一座草亭。推开门,入到里面,只见那四壁上挂几张强弓硬弩,插几壶箭;
过梁上搭两块血腥的虎皮:墙根头插着许多枪刀叉棒;正中间设两张坐器。伯钦请
三藏坐坐。三藏见这般凶险腌脏,不敢久坐,遂出了草亭。又往后再行,是一座大
园子,却看不尽那丛丛菊蕊堆黄,树树枫杨挂赤。又见呼的一声,跑出十来只肥鹿,
一大阵黄獐,见了人,呢呢痴痴,更不恐惧。三藏道:“这獐鹿想是太保养家了的?”
伯钦道:“似你那长安城中人家,有钱的集财宝,有庄的集聚稻粮;似我们这打猎
的,只得聚养些野兽,备天阴耳。”他两个说话闲行,不觉黄昏,复转前宅安歇。

次早,那合家老小都起来,就整素斋,管待长老,请开启念经。这长老净了手,
同太保家堂前拈了香,拜了家堂,三藏方敲响木鱼,先念了净口业的真言,又念了
净身心的神咒,然后开《度亡经》一卷。诵毕,伯钦又请写荐亡疏一道,再开念《金
刚经》、《观音经》。一一朗音高诵。诵毕,吃了午斋。又念《法华经》、《弥陀
经》。各诵几卷,又念一卷《孔雀经》,及谈洗业的故事。早又天晚。献过了
种种香火,化了众神纸马,烧了荐亡文疏,佛事已毕,又各安寝。

却说那伯钦的父亲之灵,超荐得脱沉沦,鬼魂儿早来到东家宅内,托一梦与合
宅长幼道:“我在阴司里苦难难脱,日久不得超生。今幸得圣僧,念了经卷,消了
我的罪业,阎王差人送我上中华富地,长者人家托生去了。你们可好生谢送长老,
不要怠慢,不要怠慢。我去也。”这才是:万法庄严端有意,荐亡离苦出沉沦。

那合家儿梦醒,又早太阳东上。伯钦的娘子道:“太保,我今夜梦见公公来家,
说他在阴司苦难难脱,日久不得超生。今幸得圣僧念了经卷,消了他的罪业,阎王
差人送他上中华富地长者人家托生去,教我们好生谢那长老,不得怠慢。他说罢,
径出门,徉徜去了。我们叫他不应,留他不住,醒来却是一梦。”伯钦道:“我也
是那等一梦,与你一般。我们起去对母亲说去。”他两口子正欲去说,只见老母叫
道:“伯钦孩儿,你来,我与你说话。”二人至前,老母坐在床上道:“儿呵,我
今夜得了个喜梦,梦见你父亲来家,说多亏了长老超度,已消了罪业,上中华富地
长者家去托生。”夫妻们俱呵呵大笑道:“我与媳妇皆有此梦,正来告禀,不期母
亲呼唤,也是此梦。”

遂叫一家大小起来,安排谢意,替他收拾马匹,都至前拜谢道:“多谢长老超
荐我亡父脱难超生,报答不尽!”三藏道:“贫僧有何能处,敢劳致谢?”伯钦把
三口儿的梦话,对三藏陈诉一遍,三藏也喜。早供给了素斋,又具白银一两为谢。
三藏分文不受。一家儿又恳恳拜央,三藏毕竟分文未受。但道:“是你肯发慈悲送
我一程,足感至爱。”伯钦与母妻无奈,急做了些粗面烧饼干粮,叫伯钦远送。三
藏欢喜收纳。太保领了母命,又唤两三个家僮,各带捕猎的器械,同上大路。看不
尽那山中野景,岭上风光。

行经半日,只见对面处,有一座大山,真个是高接青霄,崔巍险峻。三藏不一
时,到了边前。那太保登此山如行平地。正走到半山之中,伯钦回身,立于路下道:
“长老,你自前进,我却告回。”三藏闻言,滚鞍下马道:“千万敢劳太保再送一
程!”伯钦道:“长老不知。此山唤做两界山。东半边属我大唐所管,西半边乃是
鞑靼的地界。那厢狼虎,不伏我降,我却也不能过界,你自去罢。”三藏心惊,轮
开手,牵衣执袂,滴泪难分。正在那叮咛拜别之际,只听得山脚下叫喊如雷道:“我
师父来也!我师父来也!”唬得个三藏痴呆,伯钦打挣。

毕竟不知是甚人叫喊,且听下回分解。