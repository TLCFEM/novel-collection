\chapter{三藏有灾沉水宅~观音救难现鱼篮}

却说孙大圣与八戒、沙僧辞陈老来至河边,道:“兄弟,你两个议定,那一个
先下水。”八戒道:“哥啊,我两个手段不见怎的,还得你先下水。”行者道:“不瞒
贤弟说,若是山里妖精,全不用你们费力;水中之事,我去不得。就是下海行江,
我须要捻着避水诀,或者变化甚么鱼蟹之形,才去得;若是那般捻诀,却轮不得铁
棒,使不得神通,打不得妖怪。我久知你两个乃惯水之人,所以要你两个下去。”
沙僧道:“哥啊,小弟虽是去得,但不知水底如何。我等大家都去。哥哥变作甚么
模样;或是我驮着你,分开水道,寻着妖怪的巢穴,你先进去打听打听。若是师父
不曾伤损,还在那里,我们好努力征讨;假若不是这怪弄法,或者杀师父,或者
被妖吃了,我等不须苦求,早早的别寻道路何如?”行者道:“贤弟说得有理。你
们那个驮我?”八戒暗喜道:“这猴子不知捉弄了我多少,今番原来不会水,等老
猪驮他,也捉弄他捉弄!”呆子笑嘻嘻的叫道:“哥哥,我驮你。”行者就知有意,
却便将计就计道:“是,也好,你比悟净还有些膂力。”八戒就背着他。

沙僧剖开水路,弟兄们同入通天河内。向水底下行有百十里远近,那呆子要捉
弄行者,行者随即拔下一根毫毛,变做假身,伏在八戒背上,真身变作一个猪虱子,
紧紧的贴在他耳朵里。八戒正行,忽然打个踵,得故子把行者往前一掼,扑的跌
了一跤。原来那个假身本是毫毛变的,却就飘起去,无影无形。沙僧道:“二哥,
你是怎么说?不好生走路,就跌在泥里,便也罢了,却把大哥不知跌了那里去了!”
八戒道:“那猴子不禁跌,一跌就跌化了。兄弟,莫管他死活,我和你且去寻师父
去。”沙僧道:“不好,还得他来。他虽不知水性,他比我们乖巧。若无他来,我不
与你去。”行者在八戒耳朵里,忍不住高叫道:“悟净!老孙在这里也。”沙僧听得,
笑道:“罢了!这呆子是死了!你怎么就敢捉弄他!如今弄得闻声不见面,却怎是
好?”八戒慌得跪在泥里磕头道:“哥哥,是我不是了。待救了师父,上岸陪礼。
你在那里做声?就影杀我也!你请现原身出来。我驮着你,再不敢冲撞你了。”行者
道:“是你还驮着我哩。我不弄你,你快走,快走!”那呆子絮絮叨叨,只管念诵着
陪礼,爬起来与沙僧又进。

行了又有百十里远近,忽抬头望见一座楼台,上有“水鼋之第”四个大字。沙
僧道:“这厢想是妖精住处,我两个不知虚实,怎么上门索战。”行者道:“悟净,
那门里外可有水么?”沙僧道:“无水。”行者道:“既无水,你再藏隐在左右,待
老孙去打听打听。”

好大圣,爬离了八戒耳朵里,却又摇身一变,变作个长脚虾婆,两三跳跳到门
里。睁眼看时,只见那怪坐在上面,众水族摆列两边,有个斑衣鳜婆坐于侧手,都
商议要吃唐僧。行者留心,两边寻找不见,忽看见一个大肚虾婆走将来,径往西廊
下立定。行者跳到面前,称呼道:“姆姆,大王与众商议要吃唐僧,唐僧却在那里?”
虾婆道:“唐僧被大王降雪结冰,昨日拿在宫后石匣中间,只等明日,他徒弟们不
来吵闹,就奏乐享用也。”

行者闻言,演了一会,径直寻到宫后,看果有一个石匣,却像人家槽房里的猪
槽,又似人间一口石棺材之样,量量足有六尺长短;却伏在上面,听了一会,只听
得三藏在里面嘤嘤的哭哩。行者不言语,侧耳再听,那师父挫得牙响,哏了一声道:
“自恨江流命有愆,生时多少水灾缠。
出娘胎腹淘波浪,拜佛西天堕渺渊。
前遇黑河身有难,今逢冰解命归泉。
不知徒弟能来否,可得真经返故园?”
行者忍不住叫道:“师父莫恨水灾。《经》云:‘土乃五行之母,水乃五行之源,无
土不生,无水不长。’老孙来了!”三藏闻得道:“徒弟啊,救我耶!”行者道:“你
且放心,待我们擒住妖精,管教你脱难。”三藏道:“快些儿下手!再停一日,足足
闷杀我也!”行者道:“没事,没事,我去也!”急回头,跳将出去,到门外现了原
身,叫:“八戒!”那呆子与沙僧近道:“哥哥,如何?”行者道:“正是此怪骗了师
父。师父未曾伤损,被怪物盖在石匣之下。你两个快早挑战,让老孙先出水面。你
若擒得他就擒;擒不得,做个佯输,引他出水,等我打他。”沙僧道:“哥哥放心先
去,待小弟们鉴貌辨色。”这行者捻着避水诀,钻出波中,停立岸边等候不题。

你看那猪八戒行凶,闯至门前,厉声高叫:“泼怪物,送我师父出来!”慌得那
门里小妖,急报:“大王,门外有人要师父哩!”妖邪道:“这定是那泼和尚来了。”
教:“快取披挂兵器来!”众小妖连忙取出。妖邪结束了,执兵器在手,即命开门,
走将出来。八戒与沙僧对列左右,见妖邪怎生披挂。好怪物!你看他:

头戴金盔晃且辉,身披金甲掣虹霓。腰围宝带团珠翠,足踏烟黄靴样奇。鼻准
高隆如峤耸,天庭广阔若龙仪。眼光闪灼圆还暴,牙齿钢锋尖又齐。短发蓬松飘火
焰,长须潇洒挺金锥。口咬一枝青嫩藻,手拿九瓣赤铜锤。一声咿哑门开处,响似
三春惊蛰雷。这等形容人世少,敢称灵显大王威。

妖邪出得门来,随后有百十个小妖,一个个轮枪舞剑,摆开两哨,对八戒道:
“你是那寺里和尚?为甚到此喧嚷?”八戒喝道:“我把你这打不死的泼物!你前夜
与我顶嘴,今日如何推不知来问我?我本是东土大唐圣僧之徒弟,往西天拜佛求经
者。你弄玄虚,假做甚么灵感大王,专在陈家庄要吃童男童女,我本是陈清家一秤
金,你不认得我么?”那妖邪道:“你这和尚,甚没道理!你变做一秤金,该一个冒
名顶替之罪。我倒不曾吃你,反被你伤了我手背。已此让了你,你怎么又寻上我的
门来?”八戒道:“你既让我,却怎么又弄冷风,下大雪,冻结坚冰,害我师父?快
早送我师父出来,万事皆休!牙迸半个‘不’字,你只看看手中钯!决不饶你!”妖
邪闻言,微微冷笑道:“这和尚卖此长舌,胡夸大口。果然是我作冷下雪冻河,摄
你师父。你今嚷上门来,思量取讨,只怕这一番不比那一番了。那时节,我因赴会,
不曾带得兵器,误中你伤。你如今且休要走,我与你交敌三合。三合敌得我过,还
你师父;敌不过,连你一发吃了。”

八戒道“好乖儿子,正是这等说,仔细看钯!”妖邪道:“你原来是半路上出家
的和尚。”八戒道:“我的儿,你真个有些灵感,怎么就晓得我是半路出家的?”妖
邪道:“你会使钯,想是雇在那里种园,把他钉钯拐将来也。”八戒道:“儿子,我
这钯,不是那筑地之钯。你看:

巨齿铸就如龙爪,逊金妆来似蟒形。若逢对敌寒风洒,但遇相持火焰生。能与
圣僧除怪物,西方路上捉妖精。轮动烟云遮日月,使开霞彩照分明。筑倒太山千虎
怕,掀翻大海万龙惊。饶你威灵有手段,一筑须教九窟窿!”

那个妖邪,那里肯信,举铜锤劈头就打。八戒使钉钯架住道:“你这泼物,原
来也是半路上成精的邪魔!”那怪道:“你怎么认得我是半路上成精的?”八戒道:
“你会使铜锤,想是雇在那个银匠家扯炉,被你得了手,偷将出来的。”妖邪道:“这
不是打银之锤。你看:

九瓣攒成花骨朵,一竿虚孔万年青。原来不比凡间物,出
处还从仙苑名。绿房紫瑶池老,素质清香碧沼生。因我用功抟炼过,坚如钢锐彻
通灵。枪刀剑戟浑难赛,钺斧戈矛莫敢经。纵让你钯能利刃,汤着吾锤迸折钉!”

沙和尚见他两个攀话,忍不住近前高叫道:“那怪物!休得浪言!古人云:‘口说
无凭,做出便见。’不要走!且吃我一杖!”妖邪使锤杆架住道:“你也是半路里出家
的和尚。”沙僧道:“你怎么认得?”妖邪道:“你这个模样,像一个磨博士出身。”
沙僧道:“如何认得我像个磨博士?”妖邪道:“你不是磨博士,怎么会使赶面杖?”
沙僧骂道:“你这孽障,是也不曾见!

这般兵器人间少,故此难知宝杖名。出自月宫无影处,梭罗仙木琢磨成。外边
嵌宝霞光耀,内里钻金瑞气凝。先日也曾陪御宴,今朝秉正保唐僧。西方路上无知
识,上界宫中有大名。唤做降妖真宝杖,管教一下碎天灵!”

那妖邪不容分说,三家变脸,这一场,在水底下好杀:

铜锤宝杖与钉钯,悟能悟净战妖邪。一个是天蓬临世界,一个是上将降天涯。
他两个、夹攻水怪施威武,这一个独抵神僧势可夸。有分有缘成大道,相生相克秉
恒沙。土克水,水干见底;水生木,木旺开花。禅法参修归一体,还丹炮炼伏三家。
土是母,发金芽,金生神水产婴娃;水为本,润木华,木有辉煌烈火霞。攒簇五行
皆别异,故然变脸各争差。看他那、铜锤九瓣光明好,宝杖千丝彩绣佳。钯按阴阳
分九曜,不明解数乱如麻。捐躯弃命因僧难,舍死忘生为释迦。致使铜锤忙不坠,
左遮宝杖右遮钯。
三人在水底下斗经两个时辰,不分胜败。猪八戒料道不得赢他,对沙僧丢了个眼色,
二人诈败佯输,各拖兵器,回头就走。那怪物教:“小的们,扎住在此,等我赶上
这厮,捉将来与汝等凑吃哑!”你看他如风吹败叶,似雨打残花,将他两个赶出水
面。

那孙大圣在东岸上,眼不转睛,只望着河边水势。忽然见波浪翻腾,喊声号吼,
八戒先跳上岸道:“来了!来了!”沙僧也到岸边道:“来了!来了!”那妖邪随后叫:
“那里走!”才出头,被行者喝道:“看棍!”那妖邪闪身躲过,使铜锤急架相还。
一个在河边涌浪,一个在岸上施威。搭上手未经三合,那妖遮架不住,打个花,又
淬于水里,遂此风平浪息。

行者回转高崖道:“兄弟们!辛苦啊。”沙僧道:“哥啊,这妖精,他在岸上觉到
不济,在水底也尽利害哩!我与二哥左右齐攻,只战得两平,却怎么处置,救师父
也?”行者道:“不必疑迟,恐被他伤了师父。”八戒道:“哥哥,我这一去哄他出
来,你莫做声,但只在半空中等候。估着他钻出头来,却使个捣蒜打,照他顶门上
着着实实一下!纵然打不死他,好道也护疼发晕,却等老猪赶上一钯,管教他了帐!”
行者道:“正是,正是!这叫做‘里迎外合’,方可济事。”他两个复入水中不题。

却说那妖邪败阵逃生,回归本宅。众妖接到宫中,鳜婆上前问道:“大王赶那
两个和尚到那方来?”妖邪道:“那和尚原来还有一个帮手。他两个跳上岸去,那
帮手轮一条铁棒打我,我闪过与他相持。也不知他那棍子有多少斤重,我的铜锤莫
想架得他住。战未三合,我却败回来也。”鳜婆道:“大王,可记得那帮手是甚相貌?”
妖邪道:“是一个毛脸雷公嘴,查耳朵,折鼻梁,火眼金睛和尚。”鳜婆闻说,打了
一个寒噤道:“大王啊!亏了你识俊,逃了性命!若再三合,决然不得全生!那和尚我
认得他。”妖邪道:“你认得他是谁?”鳜婆道:“我当年在东洋海内,曾闻得老龙
王说他的名誉。乃是五百年前大闹天宫,混元一气上方太乙金仙美猴王齐天大圣。
如今归依佛教,保唐僧往西天取经,改名唤做孙悟空行者。他的神通广大,变化多
端。大王,你怎么惹他!今后再莫与他战了。”

说不了,只见门里小妖来报:“大王,那两个和尚又来门前索战哩!”妖精道:
“贤妹所见甚长,再不出去,看他怎么。”急传令,教:“小的们,把门关紧了。正
是‘任君门外叫,只是不开门。’让他缠两日,性摊了回去时,我们却不自在受用
唐僧也?”那小妖一齐都搬石头,塞泥块,把门闭杀。八戒与沙僧连叫不出,呆子
心焦,就使钉钯筑门。那门已此紧闭牢关,莫想能够;被他七八钯,筑破门扇,里
面却都是泥土石块,高叠千层。沙僧见了道:“二哥,这怪物惧怕之甚,闭门不出,
我和你且回上河崖,再与大哥计较去来。”八戒依言,径转东岸。

那行者半云半雾,提着铁棒等哩。看见他两个上来,不见妖怪,即按云头,迎
至边岸,问道:“兄弟,那话儿怎么不上来?”沙僧道:“那怪物紧闭宅门,再不出
来见面;被二哥打破门扇看时,那里面都使些泥土石块实实的叠住了。故此不能得
战,却来与哥哥计议,再怎么设法去救师父。”行者道:“似这般却也无法可治。你
两个只在河岸上巡视着,不可放他往别处走了,待我去来。”八戒道:“哥哥,你往
那里去?”行者道:“我上普陀岩拜问菩萨,看这妖怪是那里出身,姓甚名谁。寻
着他的祖居,拿了他的家属,捉了他的四邻,却来此擒怪救师。”八戒笑道:“哥啊,
这等干,只是忒费事,担搁了时辰了。”行者道:“管你不费事,不担搁,我去就来!”

好大圣,急纵祥光,躲离河口,径赴南海。那里消半个时辰,早望见落伽山不
远。低下云头,径至普陀崖上。只见那二十四路诸天与守山大神、木叉行者、善财
童子、捧珠龙女,一齐上前,迎着施礼道:“大圣何来?”行者道:“有事要见菩萨。”
众神道:“菩萨今早出洞,不许人随,自入竹林里观玩。知大圣今日必来,吩咐我
等在此候接大圣,不可就见。请在翠岩前聊坐片时,待菩萨出来,自有道理。”

行者依言,还未坐下,又见那善财童子上前施礼道:“孙大圣,前蒙盛意,幸
菩萨不弃收留,早晚不离左右,专侍莲台之下,甚得善慈。”行者知是红孩儿,笑
道:“你那时节魔业迷心,今朝得成正果,才知老孙是好人也。”

行者久等不见,心焦道:“列位与我传报传报,但迟了,恐伤吾师之命。”诸天
道:“不敢报。菩萨吩咐,只等他自出来哩。”行者性急,那里等得,急纵身往里便
走。噫!

这个美猴王,性急能鹊薄。诸天留不住,要往里边。拽步入深林,睁眼偷觑
着。远观救苦尊,盘坐衬残箬。懒散怕梳妆,容颜多绰约。散挽一窝丝,未曾戴缨
络。不挂素蓝袍,贴身小袄缚。漫腰束锦裙,赤了一双脚。披肩绣带无,精光两臂
膊。玉手执钢刀,正把竹皮削。
行者见了,忍不住厉声高叫道:“菩萨,弟子孙悟空志心朝礼。”菩萨教:“外面俟
候。”行者叩头道:“菩萨,我师父有难,特来拜问通天河妖怪根源。”菩萨道:“你
且出去,待我出来。”行者不敢强,只得走出竹林,对众诸天道:“菩萨今日又重置
家事哩。怎么不坐莲台,不妆饰,不喜欢,在林里削篾做甚?”诸天道:“我等却
不知。今早出洞,未曾妆束,就入林中去了。又教我等在此接候大圣,必然为大圣
有事。”行者没奈何,只得等候。

不多时,只见菩萨手提一个紫竹篮儿出林,道:“悟空,我与你救唐僧去来。”
行者慌忙跪下道:“弟子不敢催促,且请菩萨着衣登座。”菩萨道:“不消着衣,就
此去也。”那菩萨撇下诸天,纵祥云腾空而去。孙大圣只得相随。

顷刻间,到了通天河界。八戒与沙僧看见道:“师兄性急,不知在南海怎么乱
嚷乱叫,把一个未梳妆的菩萨逼将来也。”说不了,到于河岸。二人下拜道:“菩萨,
我等擅干,有罪,有罪!”菩萨即解下一根束袄的丝绦,将篮儿拴定,提着丝绦,
半踏云彩,抛在河中,往上溜头扯着,口念颂子道:“死的去,活的住!死的去,活
的住!”念了七遍,提起篮儿,但见那篮里亮灼灼一尾金鱼,还斩眼动鳞。菩萨叫:
“悟空,快下水救你师父耶。”行者道:“未曾拿住妖邪,如何救得师父?”菩萨道:
“这篮儿里不是?”八戒与沙僧拜问道:“这鱼儿怎生有那等手段?”菩萨道:“他
本是我莲花池里养大的金鱼。每日浮头听经,修成手段。那一柄九瓣铜锤,乃是一
枝未开的菡萏,被他运炼成兵。不知是那一日,海潮泛涨,走到此间。我今早扶栏
看花,却不见这厮出拜。掐指巡纹,算着他在此成精,害你师父,故此未及梳妆,
运神功,织个竹篮儿擒他。”

行者道:“菩萨,既然如此,且待片时,我等叫陈家庄众信人等,看看菩萨的
金面:一则留恩,二来说此收怪之事,好教凡人信心供养。“菩萨道:“也罢,你快
去叫来。”那八戒与沙僧,一齐飞跑至庄前,高呼道:“都来看活观音菩萨!都来看
活观音菩萨!”一庄老幼男女,都向河边,也不顾泥水,都跪在里面,磕头礼拜。
内中有善图画者,传下影神,这才是鱼篮观音现身。当时菩萨就归南海。

八戒与沙僧,分开水道,径往那水鼋之第,找寻师父。原来那里边水怪鱼精,
尽皆死烂。却入后宫,揭开石匣,驮着唐僧,出离波津,与众相见。那陈清兄弟,
叩头称谢道:“老爷不依小人劝留,致令如此受苦。”行者道:“不消说了。你们这
里人家,下年再不用祭赛。那大王已此除根,永无伤害。陈老儿,如今才好累你,
快寻一只船儿,送我们过河去也。”那陈清道:“有,有,有!”就教解板打船。众
庄客闻得此言,无不喜舍。那个道,我买桅篷;这个道,我办篙桨。有的说,我出
绳索;有的说,我雇水手。

正都在河边上吵闹,忽听得河中间高叫:“孙大圣不要打船,花费人家财物。
我送你师徒们过去。”众人听说,个个心惊,胆小的走了回家,胆大的战兢兢贪看。
须臾,那水里钻出一个怪来,你道怎生模样:
方头神物非凡品,九助灵机号水仙。
曳尾能延千纪寿,潜身静隐百川渊。
翻波跳浪冲江岸,向日朝风卧海边。
养气含灵真有道,多年粉盖癞头鼋。
那老鼋又叫:“大圣,不要打船,我送你师徒过去。”行者轮着铁棒道:“我把你这
个孽畜!若到边前,这一棒就打死你!”老鼋道:“我感大圣之恩,情愿办好心送你
师徒,你怎么反要打我?”行者道:“与你有甚恩惠?”老鼋道:“大圣,你不知这
底下水鼋之第,乃是我的住宅。自历代以来,祖上传留到我。我因省悟本根,养成
灵气,在此处修行,被我将祖居翻盖了一遍,立做一个水鼋之第。那妖邪乃九年前
海啸波翻,他赶潮头,来于此处,仗逞凶顽,与我争斗;被他伤了我许多儿女,夺
了我许多眷族。我斗他不过,将巢穴白白的被他占了。今蒙大圣至此搭救唐师父,
请了观音菩萨扫净妖氛,收去怪物,将第宅还归于我,我如今团老小,再不须挨
土帮泥,得居旧舍。此恩重若丘山,深如大海。且不但我等蒙惠,只这一庄上人,
免得年年祭赛,全了多少人家儿女,此诚所谓‘一举而两得’之恩也!敢不报答?”

行者闻言,心中暗喜,收了铁棒道:“你端的是真实之情么?”老鼋道:“因大
圣恩德洪深,怎敢虚谬?”行者道:“既是真情,你朝天赌咒。”那老鼋张着红口,
朝天发誓道:“我若真情不送唐僧过此通天河,将身化为血水!”行者笑道:“你上
来,你上来。”老鼋却才负近岸边,将身一纵,爬上河崖。众人近前观看,有四丈
围圆的一个大白盖。行者道:“师父,我们上他身,渡过去也。”三藏道:“徒弟呀,
那层冰厚冻,尚且,况此鼋背;恐不稳便。”老鼋道:“师父放心。我比那层冰
厚冻,稳得紧哩。但歪一歪,不成功果!”行者道:“师父啊,凡诸众生,会说人话,
决不打诳语。”教:“兄弟们,快牵马来。”

到了河边,陈家庄老幼男女,一齐来拜送。行者教把马牵在白鼋盖上,请唐僧
站在马的颈项左边,沙僧站在右边,八戒站在马后,行者站在马前;又恐那鼋无礼,
解下虎筋绦子,穿在老鼋的鼻之内,扯起来,像一条缰绳;却使一只脚踏在盖上,
一只脚登在头上;一只手执着铁棒,一只手扯着缰绳,叫道:“老鼋,慢慢走啊。
歪一歪儿,就照头一下!”老鼋道:“不敢,不敢!”他却蹬开四足,踏水面如行平
地。众人都在岸上,焚香叩头,都念“南无阿弥陀佛。”这正是真罗汉临凡,活菩
萨出现。众人只拜的望不见形影方回,不题。

却说那师父驾着白鼋,那消一日,行过了八百里通天河界,干手干脚的登岸。
三藏上崖,合手称谢道:“老鼋累你,无物可赠,待我取经回谢你罢。”老鼋道:“不
劳师父赐谢。我闻得西天佛祖无灭无生,能知过去未来之事。我在此间,整修行了
一千三百余年;虽然延寿身轻,会说人语,只是难脱本壳。万望老师父到西天与我
问佛祖一声,看我几时得脱本壳,可得一个人身。”三藏响允道:“我问,我问。”
那老鼋才淬水中去了。行者遂伏侍唐僧上马。八戒挑着行囊,沙僧跟随左右。师徒
们找大路,一直奔西。这的是:
圣僧奉旨拜弥陀,水远山遥灾难多。
意志心诚不惧死,白鼋驮渡过天河。

毕竟不知此后还有多少路程,还有甚么凶吉,且听下回分解。