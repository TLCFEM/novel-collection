\chapter{镇元仙赶捉取经僧~孙行者大闹五庄观}

却说他兄弟三众,到了殿上,对师父道:“饭将熟了,叫我们怎的?”三藏道:
“徒弟,不是问饭。他这观里,有甚么人参果,似孩子一般的东西,你们是那一个
偷他的吃了?”八戒道:“我老实。不晓得,不曾见。”清风道:“笑的就是他,
笑的就是他!”行者喝道:“我老孙生的是这个笑容儿,莫成为你不见了甚么果子,
就不容我笑?”三藏道:“徒弟息怒。我们是出家人,休打诳语,莫吃昧心食。果
然吃了他的,陪他个礼罢。何苦这般抵赖?”

行者见师父说得有理,他就实说道:“师父,不干我事。是八戒隔壁听见那两
个道童吃甚么人参果,他想一个儿尝新,着老孙去打了三个,我兄弟各人吃了一个。
如今吃也吃了,待要怎么?”明月道:“偷了我四个,这和尚还说不是贼哩!”八戒
道:“阿弥陀佛!既是偷了四个,怎么只拿出三个来分,预先就打起一个偏手?”那
呆子倒转胡嚷。

二仙童问得是实,越加毁骂。就恨得个大圣钢牙咬响,火眼睁圆,把条金箍棒
了又,忍了又忍道:“这童子这样可恶,只说当面打人,也罢,受他些气儿,
等我送他一个绝后计,教他大家都吃不成!”好行者,把脑后的毫毛拔了一根,吹
口仙气,叫“变!”变做个假行者,跟定唐僧,陪着悟能、悟净,忍受着道童嚷骂;
他的真身,出一个神,纵云头,跳将起去,径到人参园里,掣金箍棒往树上乒乓一
下,又使个推山移岭的神力,把树一推推倒。可怜叶落开根出土,道人断绝草还
丹!那大圣推倒树,却在枝儿上寻果子,那里得有半个。原来这宝贝遇金而落,他
的棒刃头却是金裹之物,况铁又是五金之类,所以敲着就振下来;既下来,又遇土
而入,因此上边再没一个果子。他道:“好,好,好,大家散火!”他收了铁棒,径
往前来,把毫毛一抖,收上身来。那些人肉眼凡胎,看不明白。

却说那仙童骂彀多时,清风道:“明月,这些和尚也受得气哩,我们就像骂鸡
一般,骂了这半会,通没个招声。想必他不曾偷吃。倘或树高叶密,数得不明,不
要诳骂了他。我和你再去查查。”明月道:“也说得是。”他两个果又到园中,只见
那树倒开,果无叶落。唬得清风脚软跌根头,明月腰酥打骸垢。那两个魂飞魄散。
有诗为证,诗曰:
三藏西临万寿山,悟空断送草还丹。
开叶落仙根露,明月清风心胆寒。
他两个倒在尘埃,语言颠倒,只叫“怎的好,怎的好!害了我五庄观里的丹头,断
绝我仙家的苗裔!师父来家,我两个怎的回话?”明月道:“师兄莫嚷。我们且整了
衣冠,莫要惊张了这几个和尚。这个没有别人,定是那个毛脸雷公嘴的那厮,他来
出神弄法,坏了我们的宝贝。若是与他分说,那厮毕竟抵赖,定要与他相争,争起
来,就要交手相打,你想我们两个,怎么敌得过他四个?且不如去哄他一哄,只说
果子不少,我们错数了,转与他陪个不是。他们的饭已熟了,等他吃饭时,再贴他
些儿小菜。他一家拿着一个碗,你却站在门左,我却站在门右,扑的把门关倒,把
锁锁住,将这几层门都锁了,不要放他。待师父来家,凭他怎的处置。他又是师父
的故人,饶了他,也是师父的人情;不饶他,我们也拿住个贼在,庶几可以免我等
之罪。”清风闻言道:“有理,有理。”

他两个强打精神,勉生欢喜,从后园中径来殿上,对唐僧控背躬身道:“师父,
适间言语粗俗,多有冲撞,莫怪,莫怪。”三藏问道:“怎么说?”清风道:“果子
不少,只因树高叶密,不曾看得明白;才然又去查查,还是原数。”那八戒就趁脚
儿跷道:“你这个童儿,年幼不知事体,就来乱骂,白口咀咒,枉赖了我们也!不当
人子!”行者心上明白,口里不言,心中暗想道:“是谎,是谎,果子已了了帐,怎
的说这般话?想必有起死回生之法?……”三藏道:“既如此,盛将饭来,我们吃了
去罢。”

那八戒便去盛饭,沙僧安放桌椅。二童忙取小菜,却是些酱瓜、酱茄、糟萝卜、
醋豆角、腌窝蕖、绰芥菜,共排了七八碟儿,与师徒们吃饭;又提一壶好茶,两个
茶锺,伺候左右。那师徒四众,却才拿起碗来,这童儿一边一个,扑的把门关上,
插上一把两铜锁。八戒笑道:“这童子差了。你这里风俗不好,却怎的关了门里
吃饭?”明月道:“正是,正是,好歹吃了饭儿开门。”清风骂道:“我把你这个害
馋劳、偷嘴的秃贼!你偷吃了我的仙果,已该一个擅食田园瓜果之罪,却又把我的
仙树推倒,坏了我五庄观里仙根,你还要说嘴哩!若能彀到得西方参佛面,只除是
转背摇车再托生!”三藏闻言,丢下饭碗,把个石头放在心上。那童子将那前山门、
二山门,通都上了锁。却又来正殿门首,恶语恶言,贼前贼后,只骂到天色将晚,
才去吃饭。饭毕,归房去了。

唐僧埋怨行者道:“你这个猴头,番番撞祸!你偷吃了他的果子,就受他些气儿,
让他骂几句便也罢了;怎么又推倒他的树!若论这般情由,告起状来,就是你老子
做官,也说不通。”行者道:“师父莫闹。那童儿都睡去了,只等他睡着了,我们连
夜起身。”沙僧道:“哥啊,几层门都上了锁,闭得甚紧,如何走么!”行者笑道:“莫
管,莫管,老孙自有法儿。”八戒道:“愁你没有法儿哩!你一变,变甚么虫蛭儿,
瞒格子眼里就飞将出去,只苦了我们不会变的,便在此顶缸受罪哩!”唐僧道:“他
若干出这个勾当,不同你我出去啊,我就念起旧话经儿,他却怎生消受!”八戒闻
言,又愁又笑道:“师父,你说的那里话?我只听得佛教中有卷《楞严经》、《法华经》、
《孔雀经》、《观音经》、《金刚经》,不曾听见个甚那‘旧话儿经’啊。”行者道:“兄
弟,你不知道。我顶上戴的这个箍儿,是观音菩萨赐与我师父的;师父哄我戴了,
就如生根的一般,莫想拿得下来;——叫做紧箍儿咒,又叫做紧箍儿经。他‘旧话
儿经’,即此是也。但若念动,我就头疼,故有这个法儿难我。师父,你莫念,我
决不负你,管情大家一齐出去。”

说话后,都已天昏,不觉东方月上。行者道:“此时万籁无声,冰轮明显,正
好走了去罢。”八戒道:“哥啊,不要捣鬼。门俱锁闭,往那里走?”行者道:“你
看手段!”好行者,把金箍棒捻在手中,使一个“解锁法”,往门上一指,只听得突
的一声响,几层门双俱落,唿喇的开了门扇。八戒笑道:“好本事!就是叫小炉
儿匠使掭子,便也不像这等爽利。”行者道:“这个门儿,有甚稀罕!就是南天门,
指一指也开了。”却请师父出了门,上了马,八戒挑着担,沙僧拢着马,径投西路
而去。行者道:“你们且慢行。等老孙去照顾那两个童儿睡一个月。”三藏道:“徒
弟,不可伤他性命;不然,又一个得财伤人的罪了。”行者道:“我晓得。”行者复
进去,来到那童儿睡的房门外。他腰里有带的瞌睡虫儿,原来在东天门与增长天王
猜枚耍子赢的。他摸出两个来,瞒窗眼儿弹将进去,径奔到那童子脸上,鼾鼾沉睡,
再莫想得醒。他才拽开云步,赶上唐僧,顺大路一直西奔。

这一夜马不停蹄,只行到天晓。三藏道:“这个猴头弄杀我也!你因为嘴,带累
我一夜无眠!”行者道:“不要只管埋怨。天色明了,你且在这路旁边树林中将就歇
歇,养养精神再走。”那长老只得下马,倚松根权作禅床坐下。沙僧歇了担子打盹。
八戒枕着石睡觉。孙大圣偏有心肠,你看他跳树扳枝顽耍。四众歇息不题。

却说那大仙自元始宫散会,领众小仙出离兜率,径下瑶天,坠祥云,早来到万
寿山五庄观门首。看时,只见观门大开,地上干净。大仙道:“清风、明月,却也
中用。常时节日高三丈,腰也不伸;今日我们不在,他倒肯起早,开门扫地。”众
小仙俱悦。行至殿上,香火全无,人踪俱寂,那里有明月、清风!众仙道:“他两个
想是因我们不在,拐了东西走了。”大仙道:“岂有此理!修仙的人,敢有这般坏心
的事!想是昨晚忘却关门,就去睡了,今早还未醒哩。”众仙到他房门首看处,真个
关着房门,鼾鼾沉睡;这外边打门乱叫,那里叫得醒来。众仙撬开门板,着手扯下
床来,也只是不醒。大仙笑道:“好仙童啊!成仙的人,神满再不思睡,却怎么这般
困倦?莫不是有人做弄了他也?快取水来。”一童急取水半盏递与大仙。大仙念动咒
语,一口水,喷在脸上,随即解了睡魔。

二人方醒,忽睁睛,抹抹脸,抬头观看,认得是仙师与世同君和仙兄等众,慌
得那清风顿首,明月叩头道:“师父啊!你的故人,原是东来的和尚——一伙强盗,
十分凶狠!”大仙笑道:“莫惊恐,慢慢的说来。”

清风道:“师父啊,当日别后不久,果有个东土唐僧,一行有四个和尚,连马
五口。弟子不敢违了师命,问及来因,将人参果取了两个奉上。那长老俗眼愚心,
不识我们仙家的宝贝。他说是三朝未满的孩童,再三不吃,是弟子各吃了一个。不
期他那手下有三个徒弟,有一个姓孙的,名悟空行者,先偷四个果子吃了。是弟子
们向伊理说,实实的言语了几句,他却不容,暗自里弄了个出神的手段。苦啊!……”
二童子说到此处,止不住腮边泪落。众仙道:“那和尚打你来?”明月道:“不曾打,
只是把我们人参树打倒了。”大仙闻言,更不恼怒。道:“莫哭,莫哭,你不知那姓
孙的,也是个太乙散仙,也曾大闹天宫,神通广大。既然打倒了宝树,你可认得那
些和尚?”清风道:“都认得。”大仙道:“既认得,都跟我来。众徒弟们,都收拾
下刑具,等我回来打他。”众仙领命。

大仙与明月、清风纵起祥光,来赶三藏。顷刻间就有千里之遥。大仙在云端里
平西观看,不见唐僧;及转头向东看时,倒多赶了九百余里。原来那长老一夜马不
停蹄,只行了一百二十里路;大仙的云头一纵,赶过了九百余里。仙童道:“师父,
那路旁树下坐的是唐僧。”大仙道:“我已见了。你两个回去安排下绳索,等我自家
拿他。”清风、明月先回不题。

那大仙按落云头,摇身一变,变作个行脚全真。你道他怎生模样:

穿一领百衲袍,系一条吕公绦。手摇麈尾,渔鼓轻敲。三耳草鞋登脚下,九阳
巾子把头包。飘飘风满袖,口唱月儿高。
径直来到树下,对唐僧高叫道:“长老,贫道起手了。”那长老忙忙答礼道:“失瞻,
失瞻!”大仙问:“长老是那方来的?为何在途中打坐?”三藏道:“贫僧乃东土大唐
差往西天取经者。路过此间,权为一歇。”大仙佯讶道:“长老东来,可曾在荒山经
过?”长老道:“不知仙官是何宝山?”大仙道:“万寿山五庄观,便是贫道栖止处。”
行者闻言,他心中有物的人,忙答道:“不曾,不曾!我们是打上路来的。”那大仙
指定笑道:“我把你这个泼猴!你瞒谁哩?你倒在我观里,把我人参果树打倒,你连
夜走在此间,还不招认,遮饰甚么!不要走,趁早去还我树来!”那行者闻言,心中
恼怒,掣铁棒不容分说,望大仙劈头就打。大仙侧身躲过,踏祥光,径到空中。行
者也腾云,急赶上去。

大仙在半空现了本相,你看他怎生打扮:

头戴紫金冠,无忧鹤氅穿。履鞋登足下,丝带束腰间。体如童子貌,面似美人
颜。三须飘颔下,鸦翎叠鬓边。相迎行者无兵器,止将玉麈手中拈。
那行者没高没低的,棍子乱打。大仙把玉麈左遮右挡,奈了他两三回合,使一个“袖
里乾坤”的手段,在云端里,把袍袖迎风轻轻的一展,刷地前来,把四僧连马一袖
子笼住。八戒道:“不好了!我们都装在里了!”行者道:“呆子,不是,我
们被他笼在衣袖中哩。”八戒道:“这个不打紧;等我一顿钉钯,筑他个窟窿,脱将
下去,只说他不小心,笼不牢,吊的了罢!”那呆子使钯乱筑,那里筑得动:手捻
着虽然是个软的,筑起来就比铁还硬。

那大仙转祥云,径落五庄观坐下,叫徒弟拿绳来。众小仙一一伺候。你看他从
袖子里,却像撮傀儡一般,把唐僧拿出,缚在正殿檐柱上;又拿出他三个,每一根
柱上,绑了一个;将马也拿出拴在庭下,与他些草料;行李抛在廊下;又道:“徒
弟,这和尚是出家人,不可用刀枪,不可加钺,且与我取出皮鞭来,打他一顿,
与我人参果出气!”众仙即忙取出一条鞭——不是甚么牛皮、羊皮、麂皮、犊皮的,
原来是龙皮做的七星鞭,着水浸在那里。令一个有力量的小仙,把鞭执定道:“师
父,先打那个?”大仙道:“唐三藏做大不尊,先打他。”

行者闻言,心中暗道:“我那老和尚不禁打;假若一顿鞭打坏了啊,却不是我
造的业?”他忍不住,开言道:“先生差了。偷果子是我,吃果子是我,推倒树也
是我,怎么不先打我,打他做甚?”大仙笑道:“这泼猴倒言语膂烈。这等便先打
他。”小仙问:“打多少?”大仙道:“照依果数,打三十鞭。”那小仙轮鞭就打。行
者恐仙家法大,睁圆眼瞅定,看他打那里。原来打腿。行者就把腰扭一扭,叫声“变!”
变作两条熟铁腿,看他怎么打。那小仙一下一下的,打了三十,天早向午了。

大仙又吩咐道:“还该打三藏训教不严,纵放顽徒撒泼。”那仙又轮鞭来打。行
者道:“先生又差了。偷果子时,我师父不知,他在殿上与你二童讲话,是我兄弟
们做的勾当。纵是有教训不严之罪,我为弟子的,也当替打。再打我罢。”大仙笑
道:“这泼猴,虽是狡猾奸顽,却倒也有些孝意。既这等,还打他罢。”小仙又打了
三十。行者低头看看,两只腿似明镜一般,通打亮了,更不知些疼痒。此时天色将
晚。大仙道:“且把鞭子浸在水里,待明朝再拷打他。”小仙且收鞭去浸,各各归房。
晚斋已毕,尽皆安寝不题。

那长老泪眼双垂,怨他三个徒弟道:“你等闯出祸来,却带累我在此受罪,这
是怎的起?”行者道:“且休报怨,打便先打我。你又不曾吃打,倒转嗟呀怎的?”
唐僧道:“虽然不曾打,却也绑得身上疼哩。”沙僧道:“师父,还有陪绑的在这里
哩。”行者道:“都莫要嚷,再停会儿走路。”八戒道:“哥哥又弄虚头了。这里麻绳
喷水,紧紧的绑着,还比关在殿上,被你使解锁法搠开门走哩!”行者道:“不是夸
口说,那怕他三股的麻绳喷上了水,就是碗粗的棕缆,也只好当秋风!”

正话处,早已万籁无声,正是天街人静。好行者,把身子小一小,脱下索来道:
“师父去哑!”沙僧慌了道:“哥哥,也救我们一救!”行者道:“悄言,悄言!”他
却解了三藏,放下八戒、沙僧,整束了偏衫,扣背了马匹,廊下拿了行李,一齐出
了观门。又教八戒:“你去把那崖边柳树伐四颗来。”八戒道:“要他怎的?”行者
道:“有用处。快快取来!”

那呆子有些夯力,走了去,一嘴一颗,就拱了四颗,一抱抱来。行者将枝梢折
了,教兄弟二人复进去,将原绳照旧绑在柱上。那大圣念动咒语,咬破舌尖,将血
喷在树上,叫“变!”一根变作长老,一根变作自身,那两根变作沙僧、八戒;都
变得容貌一般,相貌皆同,问他也就说话,叫名也就答应。他两个却才放开步,赶
上师父。这一夜依旧马不停蹄,躲离了五庄观。

只走到天明,那长老在马上摇桩打盹。行者见了,叫道:“师父不济!出家人怎
的这般辛苦?我老孙千夜不眠,也不晓得困倦。且下马来,莫教走路的人,看见笑
你。权在山坡下藏风聚气处,歇歇再走。”

不说他师徒在路暂住。且说那大仙,天明起来,吃了早斋,出在殿上。教拿鞭
来:“今日却该打唐三藏了。”那小仙轮着鞭,望唐僧道:“打你哩。”那柳树也应道:
“打么。”乒乓打了三十。轮过鞭来,对八戒道:“打你哩。”那柳树也应道:“打么。”
及打沙僧,也应道:“打么。”及打到行者,那行者在路,偶然打个寒噤道:“不好
了!”三藏问道:“怎么说?”行者道:“我将四颗柳树变作我师徒四众,我只说他
昨日打了我两顿,今日想不打了;却又打我的化身,所以我真身打噤。收了法罢。”
那行者慌忙念咒收法。

你看那些道童害怕,丢了皮鞭,报道:“师父啊,为头打的是大唐和尚,这一
会打的都是柳树之根!”大仙闻言,呵呵冷笑,夸不尽道:“孙行者,真是一个好猴
王!曾闻他大闹天宫,布地网天罗,拿他不住,果有此理。你走了便也罢,却怎么
绑些柳树在此,冒名顶替?决莫饶他,赶去来!”

那大仙说声赶,纵起云头,往西一望,只见那和尚挑包策马,正然走路。大仙
低下云头,叫声“孙行者,往那里走!还我人参树来!”八戒听见道:“罢了,对头
又来了!”行者道:“师父,且把善字儿包起,让我们使些凶恶,一发结果了他,脱
身去罢。”唐僧闻言,战战兢兢,未曾答应,沙僧掣宝杖,八戒举钉钯,大圣使铁
棒,一齐上前,把大仙围住在空中,乱打乱筑。这场恶斗,有诗为证,诗曰:
悟空不识镇元仙,与世同君妙更玄。
三件神兵施猛烈,一根麈尾自飘然。
左遮右挡随来往,后架前迎任转旋。
夜去朝来难脱体,淹留何日到西天!
他兄弟三众,各举神兵,一齐攻打,那大仙只把蝇帚儿演架。那里有半个时辰,他
将袍袖一展,依然将四僧一马并行李,一袖笼去。返云头,又到观里。众仙接着,
仙师坐于殿上。却又在袖儿里一个个搬出,将唐僧绑在阶下矮槐树上;八戒、沙僧
各绑在两边树上;将行者捆倒,行者道:“想是调问哩。”不一时,捆绑停当。教把
长头布取十匹来。行者笑道:“八戒!这先生好意思,拿出布来与我们做中袖哩!减
省些儿,做个一口中罢了。”那小仙将家机布搬将出来。大仙道:“把唐三藏、猪八
戒、沙和尚都使布裹了!”众仙一齐上前裹了。行者笑道:“好,好,好!夹活儿就
大殓了!”须臾,缠裹已毕。又教拿出漆来。众仙即忙取了些自收自晒的生熟漆,
把他三个布裹漆漆了,浑身俱裹漆,上留着头脸在外。八戒道:“先生,上头倒不
打紧,只是下面还留孔儿,我们好出恭。”那大仙又教把大锅抬出来。行者笑道:“八
戒,造化!抬出锅来,想是煮饭我们吃哩。”八戒道:“也罢了,让我们吃些饭儿,
做个饱死的鬼也好看。”众仙果抬出一口大锅支在阶下。大仙叫架起干柴,发起烈
火,教:“把清油拗上一锅,烧得滚了,将孙行者下油锅扎他一扎,与我人参树报
仇!”

行者闻言,暗喜道:“正可老孙之意。这一向不曾洗澡,有些儿皮肤燥痒,好
歹荡荡,足感盛情。”顷刻间,那油锅将滚。大圣却又留心:恐他仙法难参,油锅
里难做手脚,急回头四顾,只见那台下东边是一座日规台,西边是一个石狮子。行
者将身一纵,滚到西边,咬破舌尖,把石狮子喷了一口,叫声“变!”变作他本身
模样,也这般捆作一团;他却出了元神,起在云端里,低头看着道士。

只见那小仙报道:“师父,油锅滚透了。”大仙教“把孙行者抬下去!”四个仙
童抬不动;八个来,也抬不动;又加四个,也抬不动。众仙道:“这猴子恋土难移,
小自小,倒也结实。”却教二十个小仙,扛将起来,往锅里一掼,烹的响了一声,
溅起些滚油点子,把那小道士们脸上烫了几个燎浆大泡!只听得烧火的小童喊道:
“锅漏了!锅漏了!”说不了,油漏得罄尽,锅底打破。原来是一个石狮子放在里面。

大仙大怒道:“这个泼猴,着然无礼!教他当面做了手脚!你走了便罢,怎么又
捣了我的灶?这泼猴枉自也拿他不住;就拿住他,也似抟砂弄汞,捉影捕风。罢,
罢,罢!饶他去罢。且将唐三藏解下,另换新锅,把他扎一扎,与人参树报报仇罢。”
那小仙真个动手,拆解布漆。

行者在半空里听得明白。他想着:“师父不济:他若到了油锅里,一滚就死,
二滚就焦,到三五滚,他就弄做个稀烂的和尚了!我还去救他一救。”

好大圣,按落云头,上前叉手道:“莫要拆坏了布漆,我来下油锅了。”那大仙
惊骂道:“你这猢猴!怎么弄手段捣了我的灶?”行者笑道:“你遇着我就该倒灶,
干我甚事?我才自也要领你些油汤油水之爱,但只是大小便急了,若在锅里开风,
恐怕污了你的熟油,不好调菜吃;如今大小便通干净了,才好下锅。不要扎我师父,
还来扎我。”那大仙闻言,呵呵冷笑,走出殿来,一把扯住。

毕竟不知有何话说,端的怎么脱身,且听下回分解。