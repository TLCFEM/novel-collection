\chapter{四海千山皆拱伏~九幽十类尽除名}

却说美猴王荣归故里,自剿了混世魔王,夺了一口大刀。逐日操演武艺,教小
猴砍竹为标,削木为刀,治旗幡,打哨子,一进一退,安营下寨,顽耍多时。忽然
静坐处,思想道:“我等在此,恐作耍成真,或惊动人王,或有禽王、兽王认此犯
头,说我们操兵造反,兴师来相杀,汝等都是竹竿木刀,如何对敌?须得锋利剑戟
方可。如今奈何?”众猴闻说,个个惊恐道:“大王所见甚长,只是无处可取。”

正说间,转上四个老猴,两个是赤尻马猴,两个是通背猿猴,走在面前道:“大
王,若要治锋利器械,甚是容易。”悟空道:“怎见容易?”四猴道:“我们这山,
向东去,有二百里水面,那厢乃傲来国界。那国界中有一王位,满城中军民无数,
必有金银铜铁等匠作。大王若去那里,或买或造些兵器,教演我等,守护山场,诚
所谓保泰长久之机也。”悟空闻说,满心欢喜道:“汝等在此顽耍,待我去来。”

好猴王,即纵筋斗云,霎时间过了二百里水面。果然那厢有座城池,六街三市,
万户千门,来来往往,人都在光天化日之下。悟空心中想道:“这里定有现成的兵
器,我待下去买他几件,还不如使个神通觅他几件倒好。”他就捻起诀来,念动咒
语,向巽地上吸一口气,呼的吹将去,便是一阵狂风,飞沙走石,好惊人也!

炮云起处荡乾坤,黑雾阴霾大地昏。江海波翻鱼蟹怕,山林树折虎狼奔。诸般
买卖无商旅,各样生涯不见人。殿上君
王归内院,阶前文武转衙门。千秋宝座都吹倒,五凤高楼幌动根。
风起处,惊散了那傲来国君王,三市六街,都慌得关门闭户,无人敢走。

悟空才按下云头,径闯入朝门里。直寻到兵器馆、武库中,打开门扇看时,那
里面无数器械:刀、枪、剑、戟、斧、钺、戈、镰、鞭、钯、挝、简、弓、弩、叉、
矛,件件俱备。一见甚喜道:“我一人能拿几何?还使个分身法搬将去罢。”好猴王,
即拔一把毫毛,入口嚼烂,喷将出去,念动咒语,叫声:“变!”变做千百个小猴,
都乱搬乱抢:有力的拿五七件,力小的拿三二件,尽数搬个罄净。径踏云头,弄个
摄法,唤转狂风,带领小猴,俱回本处。

却说那花果山大小儿猴,正在那洞门外顽耍,忽听得风声响处,见半空中,丫
丫叉叉,无边无岸的猴精,唬得都乱跑乱躲。少时,美猴王按落云头,收了云雾,
将身一抖,收了毫毛,将兵器都乱堆在山前,叫道:“小的们!都来领兵器!”众猴
看时,只见悟空独立在平阳之地,俱跑来叩头问故。悟空将前使狂风、搬兵器一应
事说了一遍。众猴称谢毕,都去抢刀夺剑,挝斧争枪,扯弓扳弩,吆吆喝喝,耍了
一日。

次日,依旧排营。悟空会聚群猴,计有四万七千余口。早惊动满山怪兽,都是
些狼、虫、虎、豹、、麂、獐、、狐、狸、獾、、狮、象、狻猊、猩猩、熊、
鹿、野豕、山牛、羚羊、青兕、狡儿、神獒……各样妖王,共有七十二洞,都来参
拜猴王为尊。每年献贡,四时点卯。也有随班操演的,也有随节征粮的,齐齐整整,
把一座花果山造得似铁桶金城。各路妖王,又有进金鼓,进彩旗,进盔甲的,纷纷
攘攘,日逐家习舞兴师。

美猴王正喜间,忽对众说道:“汝等弓弩熟谙,兵器精通,奈我这口刀着实榔
,不遂我意,奈何?”四老猴上前启奏道:“大王乃是仙圣,凡兵是不堪用;但
不知大王水里可能去得?”悟空道:“我自闻道之后,有七十二般地煞变化之功:
筋斗云有莫大的神通;善能隐身遁身,起法摄法;上天有路,入地有门;步日月无
影,入金石无碍;水不能溺,火不能焚。那些儿去不得?”四猴道:“大王既有此神
通,我们这铁板桥下,水通东海龙宫。大王若肯下去,寻着老龙王。问他要件甚么
兵器,却不趁心?”悟空闻言甚喜道:“等我去来。”

好猴王,跳至桥头,使一个闭水法,捻着诀,扑的钻入波中,分开水路,径入
东洋海底。正行间,忽见一个巡海的夜叉,挡住问道:“那推水来的,是何神圣?说
个明白,好通报迎接。”悟空道:“吾乃花果山天生圣人孙悟空,是你老龙王的紧邻,
为何不识?”

那夜叉听说,急转水晶宫传报道:“大王,外面有个花果山天生圣人孙悟空,
口称是大王紧邻,将到宫也。”东海龙王敖广即忙起身,与龙子、龙孙、虾兵、蟹
将出宫迎道:“上仙请进,请进。”直至宫里相见,上坐献茶毕,问道:“上仙几时
得道,授何仙术?”悟空道:“我自生身之后,出家修行,得一个无生无灭之体。
近因教演儿孙,守护山洞,奈何没件兵器。久闻贤邻享乐瑶宫贝阙,必有多余神器,
特来告求一件。”

龙王见说,不好推辞,即着鳜都司取出一把大捍刀奉上。悟空道:“老孙不会
使刀,乞另赐一件。”龙王又着大尉,领鳝力士,抬出一杆九股叉来。悟空跳下
来,接在手中,使了一路,放下道:“轻,轻,轻!又不趁手!再乞另赐一件。”龙王
笑道:“上仙,你不曾看这叉,有三千六百斤重哩!”悟空道:“不趁手!不趁手!”
龙王心中恐惧,又着提督、鲤总兵抬出一柄画杆方天戟。那戟有七千二百斤重。
悟空见了,跑近前接在手中,丢几个架子,撒两个解数,插在中间道:“也还轻,
轻,轻!”老龙王一发害怕道:“上仙,我宫中只有这根戟重,再没甚么兵器了。”
悟空笑道:“古人云:‘愁海龙王没宝哩!’你再去寻寻看。若有可意的,一一奉价。”
龙王道:“委的再无。”

正说处,后面闪过龙婆、龙女道:“大王,观看此圣,决非小可。我们这海藏
中,那一块天河定底的神珍铁,这几日霞光艳艳,瑞气腾腾,敢莫是该出现,遇此
圣也?”龙王道:“那是大禹治水之时,定江海浅深的一个定子,是一块神铁,能
中何用?”龙婆道:“莫管他用不用,且送与他,凭他怎么改造,送出宫门便了。”

老龙王依言,尽向悟空说了。悟空道:“拿出来我看。”龙王摇手道:“扛不动,
抬不动,须上仙亲去看看。”悟空道:“在何处?你引我去。”龙王果引导至海藏中间,
忽见金光万道。龙王指定道:“那放光的便是。”悟空撩衣上前,摸了一把,乃是一
根铁柱子,约有斗来粗,二丈有余长。他尽力两手挝过道:“忒粗忒长些,再短细
些方可用。”说毕,那宝贝就短了几尺,细了一围。悟空又颠一颠道:“再细些更好!”
那宝贝真个又细了几分。悟空十分欢喜,拿出海藏看时,原来两头是两个金箍,中
间乃一段乌铁;紧挨箍有镌成的一行字,唤做“如意金箍棒”,重一万三千五百斤。
心中暗喜道:“想必这宝贝如人意!”一边走,一边心思口念,手颠着道:“再短细
些更妙!”拿出外面,只有二丈长短,碗口粗细。你看他弄神通,丢开解数,打转
水晶宫里,唬得老龙王胆战心惊,小龙子魂飞魄散;龟鳖鼋鼍皆缩颈,鱼虾鳌蟹尽
藏头。

悟空将宝贝执在手中,坐在水晶宫殿上。对龙王笑道:“多谢贤邻厚意。”龙王
道:“不敢,不敢。”悟空道:“这块铁虽然好用,还有一说。”龙王道:“上仙还有
甚说?”悟空道:“当时若无此铁,倒也罢了;如今手中既拿着他,身上更无衣服
相趁,奈何?你这里若有披挂,索性送我一副,一总奉谢。”龙王道:“这个却是没
有。”悟空道:“‘一客不犯二主。’若没有,我也定不出此门。”龙王道:“烦上仙再
转一海,或者有之。”悟空又道:“‘走三家不如坐一家。’千万告求一副。”龙王道:
“委的没有;如有即当奉承。”悟空道:“真个没有,就和你试试此铁!”龙王慌了
道:“上仙,切莫动手,切莫动手!待我看舍弟处可有,当送一副。”悟空道:“令弟
何在?”龙王道:“舍弟乃南海龙王敖钦、北海龙王敖顺、西海龙王敖闰是也。”悟
空道:“我老孙不去,不去!俗语谓‘赊三不敌见二’,只望你随高就低的送一副便
了。”老龙道:“不须上仙去。我这里有一面铁鼓,一口金钟;凡有紧急事,擂得鼓
响,撞得钟鸣,舍弟们就顷刻而至。”悟空道:“既是如此,快些去擂鼓撞钟!”真
个那鼍将便去撞钟,鳖帅即来擂鼓。

少时,钟鼓响处,果然惊动那三海龙王,须臾来到,一齐在外面会着。

敖钦道:“大哥,有甚紧事,擂鼓撞钟?”老龙道:“贤弟!不好说!有一个花果
山甚么天生圣人,早间来认我做邻居,后要求一件兵器,献钢叉嫌小,奉画戟嫌轻。
将一块天河定底神珍铁,自己拿出手,丢了些解数。如今坐在宫中,又要索甚么披
挂。我处无有,故响钟鸣鼓,请贤弟来。你们可有甚么披挂,送他一副,打发出门
去罢了。”敖钦闻言,大怒道:“我兄弟们,点起兵,拿他不是!”老龙道:“莫说拿,
莫说拿!那块铁,挽着些儿就死,磕着些儿就亡;挨挨儿皮破,擦擦儿筋伤!”西海
龙王敖闰说:“二哥不可与他动手;且只凑副披挂与他,打发他出了门,启表奏上
上天,天自诛也。”北海龙王敖顺道:“说的是。我这里有一双藕丝步云履哩。”西
海龙王敖闰道:“我带了一副锁子黄金甲哩。”南海龙王敖钦道:“我有一顶凤翅紫
金冠哩。”老龙大喜,引入水晶宫相见了,以此奉上。悟空将金冠、金甲、云履都
穿戴停当,使动如意棒,一路打出去,对众龙道:“聒噪,聒噪!”四海龙王甚是不
平,一边商议进表上奏不题。

你看这猴王,分开水道,径回铁板桥头,撺将上来,只见四个老猴,领着众猴,
都在桥边等候。忽然见悟空跳出波外,身上更无一点水湿,金灿灿的,走上桥来。
唬得众猴一齐跪下道:“大王,好华彩耶,好华彩耶!”悟空满面春风,高登宝座,
将铁棒竖在当中。那些猴不知好歹,都来拿那宝贝,却便似蜻蜓撼铁树,分毫也不
能禁动。一个个咬指伸舌道:“爷爷呀!这般重,亏你怎的拿来也!”悟空近前,舒
开手,一把挝起,对众笑道:“物各有主。这宝贝镇于海藏中,也不知几千百年,
可可的今岁放光。龙王只认做是块黑铁,又唤做天河镇底神珍。那厮每都扛抬不动,
请我亲去拿之。那时此宝有二丈多长,斗来粗细;被我挝他一把,意思嫌大,他就
小了许多;再教小些,他又小了许多;再教小些,他又小了许多;急对天光看处,
上有一行字,乃‘如意金箍棒,一万三千五百斤。’你都站开,等我再叫他变一变
着。”他将那宝贝颠在手中,叫:“小!小!小!”即时就小做一个绣花针儿相似,可
以在耳朵里面藏下。众猴骇然,叫道:“大王!还拿出来耍耍!”猴王真个去耳朵
里拿出,托放掌上叫:“大!大!大!”即又大做斗来粗细,二丈长短。他弄到欢喜处,
跳上桥,走出洞外,将宝贝在手中,使一个法天象地的神通,把腰一躬,叫声“长!”
他就长的高万丈,头如泰山,腰如峻岭,眼如闪电,口似血盆,牙如剑戟;手中那
棒,上抵三十三天,下至十八层地狱,把些虎豹狼虫,满山群怪,七十二洞妖王,
都唬得磕头礼拜,战兢兢魄散魂飞。霎时收了法象,将宝贝还变做个绣花针儿,藏
在耳内,复归洞府。慌得那各洞妖王,都来参贺。

此时遂大开旗鼓,响振铜锣。广设珍馐百味,满斟椰液萄浆,与众饮宴多时。
却又依前教演。猴王将那四个老猴封为健将;将两个赤尻马猴唤做马、流二元帅;
两个通背猿猴唤做崩、芭二将军。将那安营下寨、赏罚诸事,都付与四健将维持。
他放下心,日逐腾云驾雾,遨游四海,行乐千山。施武艺,遍访英豪;弄神通,广
交贤友。此时又会了个七弟兄,乃牛魔王、蛟魔王、鹏魔王、狮驼王、猕猴王、
狨王,连自家美猴王七个。日逐讲文论武,走传觞,弦歌吹舞,朝去暮回,无般
儿不乐。把那万里之遥,只当庭闱之路,所谓点头径过三千里,扭腰八百有余程。

一日,在本洞分付四健将安排筵宴,请六王赴饮,杀牛宰马,祭天享地,着众
怪跳舞欢歌,俱吃得酩酊大醉。送六王出去,却又赏大小头目,在铁板桥边松
阴之下,霎时间睡着。四健将领众围护,不敢高声。

只见那美猴王睡里见两人拿一张批文,上有“孙悟空”三字,走近身,不容分
说,套上绳,就把美猴王的魂灵儿索了去,踉踉跄跄,直带到一座城边。猴王渐觉
酒醒,忽抬头观看,那城上有一铁牌,牌上有三个大字,乃“幽冥界”。美猴王顿
然醒悟道:“幽冥界乃阎王所居,何为到此?”那两人道:“你今阳寿该终,我两人
领批,勾你来也。”猴王听说,道:“我老孙超出三界外,不在五行中,已不伏他管
辖,怎么朦胧,又敢来勾我?”那两个勾死人只管扯扯拉拉,定要拖他进去。那猴
王恼起性来,耳躲中掣出宝贝,幌一幌,碗来粗细;略举手,把两个勾死人打为肉
酱。自解其索,丢开手,轮着棒,打入城中。唬得那牛头鬼东躲西藏,马面鬼南奔
北跑,众鬼卒奔上森罗殿,报着:“大王!祸事,祸事!外面一个毛脸雷公,打将来
了!”慌得那十代冥王急整衣来看,见他相貌凶恶,即排下班次,应声高叫道:“上
仙留名,上仙留名!”猴王道:“你既认不得我,怎么差人来勾我?”十王道:“不
敢,不敢!想是差人差了。”猴王道:“我本是花果山水帘洞天生圣人孙悟空。你等
是甚么官位?”十王躬身道:“我等是阴间天子十代冥王。”悟空道:“快报名来,
免打!”十王道:“我等是秦广王、初江王、宋帝王、仵官王、阎罗王、平等王、泰
山王、都市王、卞城王、转轮王。”悟空道:“汝等既登王位,乃灵显感应之类,为
何不知好歹?我老孙修仙了道,与天齐寿,超升三界之外,跳出五行之中,为何着
人拘我?”十王道:“上仙息怒。普天下同名同姓者多,敢是那勾死人错走了也?”
悟空道:“胡说,胡说!常言道:‘官差吏差,来人不差。’你快取生死簿子来我看!”
十王闻言,即请上殿查看。

悟空执着如意棒,径登森罗殿上,正中间南面坐下。十王即拿掌案的判官取出
文簿来查。那判官不敢怠慢,便到司房里,捧出五六簿文书并十类簿子,逐一查看。
裸虫、毛虫、羽虫、昆虫、鳞介之属,俱无他名。又看到猴属之类,原来这猴似人
相,不入人名;似裸虫,不居国界;似走兽,不伏麒麟管;似飞禽,不受凤凰辖。
另有个簿子,悟空亲自检阅,直到那魂字一千三百五十号上,方注着孙悟空名字,
乃天产石猴,该寿三百四十二岁,善终。悟空道:“我也不记寿数几何,且只消了
名字便罢!取笔过来!”那判官慌忙捧笔,饱掭浓墨。悟空拿过簿子,把猴属之类,
但有名者,一概勾之。下簿子道:“了帐,了帐!今番不伏你管了!”一路棒,打
出幽冥界。那十王不敢相近,都去翠云宫,同拜地藏王菩萨,商量启表,奏闻上天,
不在话下。

这猴王打出城中,忽然绊着一个草纥,跌了个踵,猛的醒来,乃是南柯一
梦。才觉伸腰,只闻得四健将与众猴高叫道:“大王,吃了多少酒,睡这一夜,还
不醒来?”悟空道:“睡还小可,我梦见两个人,来此勾我,把我带到幽冥界城门
之外,却才醒悟。是我显神通,直嚷到森罗殿,与那十王争吵,将我们的生死簿子
看了,但有我等名号,俱是我勾了,都不伏那厮所辖也。”众猴磕头礼谢。自此,
山猴多有不老者,以阴司无名故也。美猴王言毕前事,四健将报知各洞妖王,都来
贺喜。不几日,六个义兄弟,又来拜贺;一闻销名之故,又个个欢喜,每日聚乐不
题。

却表启那个高天上圣大慈仁者玉皇大天尊玄穹高上帝,一日,驾坐金阙云宫灵
霄宝殿,聚集文武仙卿早朝之际,忽有丘弘济真人启奏道:“万岁,通明殿外,有
东海龙王敖广进表,听天尊宣诏。”玉皇传旨:着宣来。

敖广宣至灵霄殿下,礼拜毕。旁有引奏仙童,接上表文。玉皇从头看过。表曰:

水元下界东胜神洲东海小龙臣敖广启奏大天圣主玄穹高上帝君:近因花果山
生、水帘洞住妖仙孙悟空者,欺虐小龙, 强坐水宅, 索兵器, 施法施威; 要披
挂, 骋凶骋势。惊伤水
族,唬走龟鼍。南海龙战战兢兢,西海龙凄凄惨惨,北海龙缩首归降,臣敖广舒身
下拜,献神珍之铁棒,凤翅之金冠,与那锁子甲、步云履,以礼送出。他仍弄武艺,
显神通,但云‘聒噪,聒噪’,果然无敌,甚为难制。臣今启奏,伏望圣裁。恳乞
天兵,收此妖孽,庶使海岳清宁,下元安泰。奉奏。
圣帝览毕,传旨:“着龙神回海,朕即遣将擒拿。”老龙王顿首谢去。下面又有葛仙
翁天师启奏道:“万岁,有冥司秦广王赍奉幽冥教主地藏王菩萨表文进上。”旁有传
言玉女,接上表文,玉皇亦从头看过。表曰:

幽冥境界,乃地之阴司。天有神而地有鬼,阴阳轮转;禽有生而兽有死,反复
雌雄。生生化化,孕女成男,此自然之数,不能易也。今有花果山水帘洞天产妖猴
孙悟空,逞恶行凶,不服拘唤。弄神通,打绝九幽鬼使;恃势力,惊伤十代慈王。
大闹森罗,强销名号。致使猴属之类无拘,猕猴之畜多寿;寂灭轮回,各无生死。
贫僧具表,冒渎天威。伏乞调遣神兵,收降此妖,整理阴阳,永安地府。谨奏。
玉皇览毕,传旨:“着冥君回归地府,朕即遣将擒拿。”秦广王亦顿首谢去。

大天尊宣众文武仙卿,问曰:“这妖猴是几年产育,何代出身,却就这般有道?”
一言未已,班中闪出千里眼、顺风耳道:“这猴乃三百年前天产石猴。当时不以为
然,不知这几年在何方修炼成仙,降龙伏虎,强销死籍也。”玉帝道:“那路神将下
界收伏?”言未已,班中闪出太白长庚星,俯伏启奏道:“上圣三界中,凡有九窍
者,皆可修仙。奈此猴乃天地育成之体,日月孕就之身,他也顶天履地,服露餐霞;
今既修成仙道,有降龙伏虎之能,与人何以异哉?臣启陛下,可念生化之慈恩,降
一道招安圣旨,把他宣来上界,授他一个大小官职,与他籍名在,拘束此间;若
受天命,后再升赏;若违天命,就此擒拿。一则不动众劳师,二则收仙有道也。”
玉帝闻言甚喜,道:“依卿所奏。”即着文曲星官修诏,着太白金星招安。

金星领了旨,出南天门外,按下祥云,直至花果山水帘洞。对众小猴道:“我
乃天差天使,有圣旨在此,请你大王上界。快快报知!”

洞外小猴,一层层传至洞天深处,道:“大王,外面有一老人,背着一角文书,
言是上天差来的天使,有圣旨请你也。”美猴王听得大喜:道:“我这两日,正思量
要上天走走,却就有天使来请。”叫:“快请进来!”

猴王急整衣冠,门外迎接。金星径入当中,面南立定道:“我是西方太白金星,
奉玉帝招安圣旨下界,请你上天,拜受仙。”悟空笑道:“多感老星降临。”教:“小
的们,安排筵宴款待!”金星道:“圣旨在身,不敢久留;就请大王同往,待荣迁之
后,再从容叙也。”悟空道:“承光顾,空退,空退!”即唤四健将,分付:“谨慎教
演儿孙,待我上天去看看路,却好带你们上去同居住也。”四健将领诺。这猴王与
金星纵起云头,升在空霄之上。正是那:
高迁上品天仙位,名列云班宝中。

毕竟不知授个甚么官爵,且听下回分解。