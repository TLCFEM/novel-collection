\chapter{姹女育阳求配偶~心猿护主识妖邪}

却说比丘国君臣黎庶,送唐僧四众出城,有二十里之远,还不肯舍。三藏勉强
下辇,乘马辞别而行。目送者直至望不见踪影方回。

四众行够多时,又过了冬残春尽,看不了野花山树,景物芳菲。前面又见一座
高山峻岭。三藏心惊,问道:“徒弟,前面高山,有路无路?是必小心!”行者笑道:
“师父这话,也不像个走长路的,却似个公子王孙,坐井观天之类。自古道:‘山
不碍路,路自通山。’何以言有路无路?”三藏道:“虽然是山不碍路,但恐峻之
间生怪物,密查深处出妖精。”八戒道:“放心,放心!这里来相近极乐不远,管取
太平无事!”

师徒正说,不觉的到了山脚下。行者取出金箍棒,走上石崖,叫道:“师父,
此间乃转山的路儿,忒好步。快来,快来!”长老只得放怀策马。沙僧教:“二哥,
你把担子挑一肩儿。”真个八戒接了担子挑上。沙僧拢着缰绳,老师父稳坐雕鞍,
随行者都奔山崖上大路。但见那山:

云雾笼峰顶,潺涌涧中。百花香满路,万树密丛丛。梅青李白,柳绿桃红。
杜鹃啼处春将暮,紫燕呢喃社已终。嵯峨石,翠盖松。崎岖岭道,突兀玲珑。削壁
悬崖峻,薜萝草木。千岩竞秀如排戟,万壑争流远浪洪。
老师父缓观山景,忽闻啼鸟之声,又起思乡之念。兜马叫道:“徒弟!
我自天牌传旨意,锦屏风下领关文。
观灯十五离东土,才与唐王天地分。
甫能龙虎风云会,却又师徒拗马军。
行尽巫山峰十二,何时对子见当今?”
行者道:“师父,你常以思乡为念,全不似个出家人。放心且走,莫要多忧。古人
云:‘欲求生富贵,须下死工夫。’”三藏道:“徒弟,虽然说得有理,但不知西天路
还在那里哩!”八戒道:“师父,我佛如来舍不得那三藏经,知我们要取去,想是搬
了;不然,如何只管不到?”沙僧道:“莫胡谈!只管跟着大哥走。只把工夫捱他,
终须有个到之之日。”

师徒正自闲叙,又见一派黑松大林。唐僧害怕,又叫道:“悟空,我们才过了
那崎岖山路,怎么又遇这个深黑松林?是必在意。”行者道:“怕他怎的!”三藏道:
“说那里话!‘不信直中直,须防仁不仁。’我也与你走过好几处松林,不似这林深
远。”你看:

东西密摆,南北成行。东西密摆彻云霄,南北成行侵碧汉。密查荆棘周围结,
蓼却缠枝上下盘。藤来缠葛,葛去缠藤:藤来缠葛,东西客旅难行;葛去缠藤,南
北经商怎进。这林中,住半年,那分日月;行数里,不见斗星。你看那背阴之处千
般景,向阳之所万丛花。又有那千年槐,万载桧,耐寒松,山桃果,野芍药,旱芙
蓉,一攒攒密砌重堆,乱纷纷神仙难画。又听得百鸟声:鹦鹉哨,杜鹃啼;喜鹊穿
枝,乌鸦反哺;黄鹂飞舞,百舌调音;鹧鸪鸣,紫燕语;八哥儿学人说话,画眉郎
也会看经。又见那大虫摆尾,老虎磕牙;多年狐妆娘子,日久苍狼吼振林。就是
托塔天王来到此,纵会降妖也失魂!
孙大圣公然不惧。使铁棒上前劈开大路,引唐僧径入深林。逍逍遥遥,行经半日,
未见出林之路。唐僧叫道:“徒弟,一向西来,无数的山林崎,幸得此间清雅,
一路太平。这林中奇花异卉,其实可人情意!我要在此坐坐:一则歇马;二则腹中
饥了,你去那里化些斋来我吃。”行者道:“师父请下马,老孙化斋去来。”那长老
果然下了马。八戒将马拴在树上,沙僧歇下行李,取了钵盂,递与行者。行者道:
“师父稳坐,莫要惊怕。我去了就来。”三藏端坐松阴之下,八戒、沙僧却去寻花
觅果闲耍。

却说大圣纵筋斗,到了半空,伫定云光,回头观看,只见松林中祥云缥缈,瑞
霭氤氲。他忽失声叫道:“好啊,好啊!”你道他叫好做甚?原来夸奖唐僧,说他是
金蝉长老转世,十世修行的好人,所以有此祥瑞罩头。若我老孙,方五百年前大闹
天宫之时,云游海角,放荡天涯,聚群精自称齐天大圣,降龙伏虎,消了死籍;头
戴着三额金冠,身穿着黄金铠甲,手执着金箍棒,足踏着步云履,手下有四万七千
群怪,都称我做大圣爷爷,着实为人。如今脱却天灾,做小伏低,与你做了徒弟,
想师父头顶上有祥云瑞霭罩定,径回东土,必定有些好处,老孙也必定得个正果。

正自家这等夸念中间,忽然见林南下有一股子黑气,骨都都的冒将上来。行者
大惊道:“那黑气里必定有邪了;我那八戒、沙僧却不会放甚黑气。”那大圣在半空
中,详察不定。

却说三藏坐在林中,明心见性,讽念那《摩诃般若波罗密多心经》,忽听得嘤
嘤的叫声“救人”。三藏大惊道:“善哉,善哉!这等深林里,有甚么人叫?想是狼虫
虎豹唬倒的,待我看看。”

那长老起身挪步,穿过千年柏,隔起万年松,附葛攀藤,近前视之,只见那大
树上绑着一个女子,上半截使葛藤绑在树上,下半截埋在土里。长老立定脚,问他
一句道:“女菩萨,你有甚事,绑在此间?”咦!分明这厮是个妖怪,长老肉眼凡胎,
却不能认得。那怪见他来问,泪如泉涌。你看他桃腮垂泪,有沉鱼落雁之容;星眼
含悲,有闭月羞花之貌。长老实不敢近前,又开口问道:“女菩萨,你端的有何罪
过?说与贫僧,却好救你。”那妖精巧语花言,虚情假意,忙忙的答应道:“师父,
我家住在贫婆国,离此有二百余里。父母在堂,十分好善,一生的和亲爱友。时遇
清明,邀请诸亲及本家老小拜扫先茔,一行轿马,都到了荒郊野外。至茔前,摆开
祭礼,刚烧化纸马,只闻得锣鸣鼓响,跑出一伙强人,持刀弄杖,喊杀前来,慌得
我们魂飞魄散。父母诸亲,得马得轿的,各自逃了性命;奴奴年幼,跑不动,唬倒
在地,被众强人拐来山内,大大王要做夫人,二大王要做妻室,第三第四个都爱我
美色,七八十家一齐争吵,大家都不忿气,所以把奴奴绑在林间,众强人散盘而去。
今已五日五夜,看看命尽,不久身亡!不知是那世里祖宗积德,今日遇着老师父到
此。千万发大慈悲,救我一命,九泉之下,决不忘恩!”说罢,泪下如雨。

三藏真个慈心,也就忍不住吊下泪来,声音哽咽。叫道:“徒弟。”那八戒、沙
僧,正在林中寻花觅果,猛听得师父叫得凄怆,呆子道:“沙和尚,师父在此认了
亲耶。”沙僧笑道:“二哥胡缠!我们走了这些时,好人也不曾撞见一个,亲从何来?”
八戒道:“不是亲,师父那里与人哭么?我和你去看来。”沙僧真个回转旧处,牵了
马,挑了担,至跟前叫:“师父,怎么说?”唐僧用手指定那树上,叫:“八戒,解
下那女菩萨来,救他一命。”呆子不分好歹,就去动手。

却说那大圣在半空中,又见那黑气浓厚,把祥光尽情盖了,道声:“不好,不
好!黑气罩暗祥光,怕不是妖邪害俺师父!化斋还是小事,且去看我师父去。”即返
云头,按落林里。只见八戒乱解绳儿。行者上前,一把揪住耳朵,扑的了一跌。
呆子抬头看见,爬起来说道:“师父教我救人,你怎么恃你有力,将我掼这一跌!”
行者笑道:“兄弟,莫解他。他是个妖怪,弄喧儿,骗我们哩。”三藏喝道:“你这
泼猴,又来胡说了!怎么这等一个女子,就认得他是个妖怪!”行者道:“师父原来
不知。这都是老孙干过的买卖,想人肉吃的法儿。你那里认得!”八戒着嘴道:“师
父,莫信这弼马温哄你!这女子乃是此间人家。我们东土远来,不与相较,又不是
亲眷,如何说他是妖精!他打发我们丢了前去,他却翻筋斗,弄神法转来和他干巧
事儿,倒踏门也!”行者喝道:“夯货,莫乱谈!我老孙一向西来,那里有甚惫懒处?
似你这个重色轻生,见利忘义的馕糟,不识好歹,替人家哄了招女婿,绑在树上哩!”
三藏道:“也罢,也罢。八戒啊,你师兄常时也看得不差。既这等说,不要管他,
我们去罢。”行者大喜道:“好了,师父是有命的了!请上马。出松林外,有人家化
斋你吃。”四人果一路前进,把那怪撇了。

却说那怪绑在树上,咬牙恨齿道:“几年家闻人说孙悟空神通广大,今日见他,
果然话不虚传。那唐僧乃童身修行,一点元阳未泄,正欲拿他去配合,成太乙金仙,
不知被此猴识破吾法,将他救去了。若是解了绳,放我下来,随手捉将去,却不是
我的人儿也?今被他一篇散言碎语带去,却又不是劳而无功?等我再叫他两声,看是
如何。”好妖精,不动绳索,把几声善言善语,用一阵顺风,嘤嘤的吹在唐僧耳内。
你道叫的甚么?他叫道:“师父啊,你放着活人的性命还不救,昧心拜佛取何经?”

唐僧在马上听得又这般叫唤,即勒马叫:“悟空,去救那女子下来罢。”行者道:
“师父走路,怎么又想起他来了?”唐僧道:“他又在那里叫哩。”行者问:“八戒,
你听见么?”八戒道:“耳大遮住了,不曾听见。”又问:“沙僧,你听见么?”沙
僧道:“我挑担前走,不曾在心,也不曾听见。”行者道:“老孙也不曾听见。师父,
他叫甚么?偏你听见。”唐僧道:“他叫得有理。说道:‘活人性命还不救,昧心拜佛
取何经?’‘救人一命,胜造七级浮屠。’快去救他下去,强似取经拜佛。”行者笑
道:“师父要善将起来,就没药医。你想你离了东土,一路西来,却也过了几重山
场,遇着许多妖怪,常把你拿将进洞,老孙来救你,使铁棒,常打死千千万万;今
日一个妖精的性命,舍不得,要去救他?”唐僧道:“徒弟呀,古人云:‘勿以善小
而不为,勿以恶小而为之。’还去救他救罢。”行者道:“师父既然如此,只是这个
担儿,老孙却担不起。你要救他,我也不敢苦劝你;劝一会,你又恼了。任你去救。”
唐僧道:“猴头莫多话!你坐着,等我和八戒救他去。”

唐僧回至林里,教八戒解了上半截绳子,用钯筑出下半截身子。那怪跌跌鞋,
束束裙,喜孜孜跟着唐僧出松林,见了行者。行者只是冷笑不止。唐僧骂道:“泼
猴头!你笑怎的?”行者道:“我笑你‘时来逢好友,运去遇佳人。’”三藏又骂道:
“泼猢狲,胡说!我自出娘肚皮,就做和尚。如今奉旨西来,虔心礼佛求经,又不
是利禄之辈,有甚运退时!”行者笑道:“师父,你虽是自幼为僧,却只会看经念佛,
不曾见王法条律。这女子生得年少标致,我和你乃出家人,同他一路行走,倘或遇
着歹人,把我们拿送官司,不论甚么取经拜佛,且都打做奸情;纵无此事,也要问
个拐带人口:师父追了度牒,打个小死;八戒该问充军;沙僧也问摆站;我老孙也
不得干净,饶我口能,怎么折辩,也要问个不应。”三藏喝道:“莫胡说!终不然,
我救他性命,有甚贻累不成!带了他去。凡有事,都在我身上。”

行者道:“师父虽说有事在你,却不知你不是救他,反是害他。”三藏道:“我
救他出林,得其活命,怎么反是害他?”行者道:“他当时绑在林间,或三五日,
十日,半月,没饭吃,饿死了,还得个完全身体归阴;如今带他出来,你坐得是个
快马,行路如风,我们只得随你,那女子脚小,挪步艰难,怎么跟得上走?一时把
他丢下,若遇着狼虫虎豹,一口吞之,却不是反害其生也?”三藏道:“正是呀。
这件事却亏你格。如何处置?”行者笑道:“抱他上来,和你同骑着马走罢。”三藏
沉吟道:“我那里好与他同马!”“他怎生得去?”三藏道:“教八戒驮他走罢。”行
者笑道:“呆子造化到了!”八戒道:“‘远路没轻担。’教我驮人,有甚造化?”行
者道:“你那嘴长,驮着他,转过嘴来,计较私情话儿,却不便益?”八戒闻此言,
捶胸爆跳道:“不好!不好!师父要打我几下,宁可忍疼。背着他决不得干净:师兄
一生会赃埋人。我驮不成!”三藏道:“也罢,也罢。我也还走得几步,等我下来,
慢慢的同走,着八戒牵着空马罢。”行者大笑道:“呆子倒有买卖。师父照顾你牵马
哩。”三藏道:“这猴头又胡说了!古人云:‘马行千里,无人不能自往。’假如我在
路上慢走,你好丢了我去?我若慢,你们也慢。大家一处同这女菩萨走下山去,或
到庵观寺院,有人家之处,留他在那里,也是我们救他一场。”行者道:“师父说得
有理。快请前进。”

三藏撩前走,沙僧挑担,八戒牵着空马,行者拿着棒,引着女子,一行前进。
不上二三十里,天色将晚。又见一座楼台殿阁。三藏道:“徒弟,那里必定是座庵
观寺院,就此借宿了,明日早行。”行者道:“师父说得是。各各走动些。”霎时到
了门首。吩咐道:“你们略站远些,等我先去借宿。若有方便处,着人来叫你。”众
人俱立在柳阴之下,惟行者拿铁棒,辖着那女子。

长老拽步近前,只见那门东倒西歪,零零落落。推开看时,忍不住心中凄惨:
长廊寂静,古刹萧疏;苔藓盈庭,蒿蓁满径;惟萤火之飞灯,只蛙声而代漏。长老
忽然吊下泪来。真个是:

殿宇雕零倒塌,廊房寂寞倾颓。断砖破瓦十余堆,尽是些歪梁折柱。前后尽生
青草,尘埋朽烂香厨。钟楼崩坏鼓无皮,琉璃香灯破损。佛祖金身没色,罗汉倒卧
东西。观音淋坏尽成泥,杨柳净瓶坠地。日内并无僧入,夜间尽宿狐狸。只听风响
吼如雷,都是虎豹藏身之处。四下墙垣皆倒,亦无门扇关居。
有诗为证,诗曰:
多年古刹没人修,狼狈凋零倒更休。
猛风吹裂伽蓝面,大雨浇残佛像头。
金刚跌损随淋洒,土地无房夜不收。
更有两般堪叹处,铜钟着地没悬楼。
三藏硬着胆,走进二层门。见那钟鼓楼俱倒了,止有一口铜钟,札在地下。上半截
如雪之白,下半截如靛之青。原来是日久年深,上边被雨淋白,下边是土气上的铜
青。三藏用手摸着钟,高叫道:“钟啊!你

也曾悬挂高楼吼,也曾鸣远彩梁声。也曾鸡啼就报晓,也曾天晚送黄昏。不知
化铜的道人归何处,铸铜匠作那边存。想他二命归阴府,他无踪迹你无声。”
长老高声赞叹,不觉的惊动寺里之人。那里边有一个侍奉香火的道人,他听见人语,
扒起来,拾一块断砖,照钟上打将去。那钟当的响了一声,把个长老唬了一跌;挣
起身要走,又绊着树根,扑的又是一跌。长老倒在地下,抬头又叫道:“钟啊!

贫僧正然感叹你,忽的叮当响一声。想是西天路上无人到,日久多年变作精。”

那道人赶上前,一把搀住道:“老爷请起。不干钟成精之事。却才是我打得钟
响。”三藏抬头见他的模样丑黑,道:“你莫是魍魉妖邪?我不是寻常之人,我是大
唐来的,我手下有降龙伏虎的徒弟。你若撞着他,性命难存也!”道人跪下道:“老
爷休怕。我不是妖邪,我是这寺里侍奉香火的道人。却才听见老爷善言相赞,就欲
出来迎接;恐怕是个邪鬼敲门,故此拾一块断砖,把钟打一下压惊,方敢出来。老
爷请起。”那唐僧方然正性道:“住持,险些儿唬杀我也。你带我进去。”

那道人引定唐僧,直至三层门里看处,比外边甚是不同。但见那:

青砖砌就彩云墙,绿瓦盖成琉璃殿。黄金装圣象,白玉造阶台。大雄殿上舞青
光,毗罗阁下生锐气。文殊殿,结采飞云;轮藏堂,描花堆翠。三檐顶上宝瓶尖,
五福楼中平绣盖。千株翠竹摇禅榻,万种青松映佛门。碧云宫里放金光,紫雾丛中
飘瑞霭。朝闻四野香风远,暮听山高画鼓鸣。应有朝阳补破衲,岂无对月了残经?
又只见半壁灯光明后院,一行香雾照中庭。
三藏见了,不敢进去。叫:“道人,你这前边十分狼狈,后边这等齐整,何也?”
道人笑道:“老爷,这山中多有妖邪强寇,天色清明,沿山打劫,天阴就来寺里藏
身,被他把佛像推倒垫坐,木植搬来烧火。本寺僧人软弱,不敢与他讲论,因此把
这前边破房都舍与那些强人安歇,从新另化了些施主,盖得那一所寺院。清混各一,
这是西方的事情。”三藏道:“原来是如此。”

正行间,又见山门上有五个大字,乃“镇海禅林寺”。才举步。入门里,忽
见一个和尚走来。你看他怎生模样:

头戴左笄绒锦帽,一对铜圈坠耳根。身着颇罗毛线服,一双白眼亮如银。手中
摇着播郎鼓,口念番经听不真。三藏原来不认得,这是西方路上喇嘛僧。
那喇嘛和尚,走出门来,看见三藏眉清目秀,额阔顶平,耳垂肩,手过膝,好似罗
汉临凡,十分俊雅。他走上前扯住,满面笑唏唏的与他捻手捻脚,摸他鼻子,揪他
耳朵,以示亲近之意。携至方丈中,行礼毕,却问:“老师父何来?”三藏道:“弟
子乃东土大唐驾下钦差往西方天竺国大雷音寺拜佛取经者。适行至宝方天晚,特奔
上刹借宿一宵,明日早行。望垂方便一二。”那和尚笑道:“不当人子!不当人子!我
们不是好意要出家的,皆因父母生身,命犯华盖,家里养不住,才舍断了出家;既
做了佛门弟子,切莫说脱空之话。”三藏道:“我是老实话。”和尚道:“那东土到西
天,有多少路程!路上有山,山中有洞,洞内有精。像你这个单身,又生得娇嫩,
那里像个取经的!”三藏道:“院主也见得是。贫僧一人,岂能到此。我有三个徒弟,
逢山开路,遇水叠桥,保我弟子,所以到得上刹。”那和尚道:“三位高徒何在?”
三藏道:“现在山门外伺候。”那和尚慌了道:“师父,你不知我这里有虎狼、妖贼、
鬼怪伤人。白日里不敢远出,未经天晚,就关了门户,这早晚把人放在外边!”叫:
“徒弟,快去请将进来。”

有两个小喇嘛儿,跑出外去,看见行者,唬了一跌;见了八戒,又是一跌;扒
起来往后飞跑,道:“爷爷,造化低了!你的徒弟不见,只有三四个妖怪站在那门首
也。”三藏问道:“怎么模样?”小和尚道:“一个雷公嘴,一个碓挺嘴,一个青脸
獠牙。旁有一个女子,倒是个油头粉面。”三藏笑道:“你不认得。那三个丑的,是
我徒弟。那一个女子,是我打松林里救命来的。”那喇嘛道:“爷爷呀,这们好俊师
父,怎么寻这般丑徒弟?”三藏道:“他丑自丑,却俱有用。你快请他进来。若再
迟了些儿,那雷公嘴的有些闯祸,不是个人生父母养的,他就打进来也。”

那小和尚即忙跑出,战兢兢的跪下道:“列位老爷,唐老爷请哩。”八戒笑道:
“哥啊,他请便罢了,却这般战兢兢的,何也?”行者道:“看见我们丑陋害怕。”
八戒道:“可是扯淡!我们乃生成的,那个是好要丑哩!”行者道:“把那丑且略收拾
收拾。”呆子真个把嘴揣在怀里,低着头,牵着马,沙僧挑着担,行者在后面,拿
着棒,辖着那女子,一行进去。穿过了倒塌房廊,入三层门里。拴了马,歇了担,
进方丈中,与喇嘛僧相见,分了坐次。那和尚入里边,引出七八十个小喇嘛来;见
礼毕,收拾办斋管待。正是:

积功须在慈悲念,佛法兴时僧赞僧。

毕竟不知怎生离寺,且听下回分解。