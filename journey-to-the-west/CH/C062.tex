\chapter{涤垢洗心惟扫塔~缚魔归正乃修身}

十二时中忘不得,行功百刻全收。五年十万八千周,休教神水涸,莫纵火光愁。

水火调停无损处,五行联络如钩。阴阳和合上云楼,乘鸾登紫府,跨鹤赴瀛洲。

这一篇词牌,名《临江仙》。单道唐三藏师徒四众,水火既济,本性清凉。借
得纯阴宝扇,息燥火遥山。不一日行过了八百之程。师徒们散诞逍遥,向西而去。
正值秋末冬初时序,见了些:

野菊残英落,新梅嫩蕊生。村村纳禾稼,处处食香羹。平林木落远山现,曲涧
霜浓幽壑清。应钟气,闭蛰营。纯阴阳,月帝玄溟;盛水德,舜日怜晴。地气下降,
天气上升。虹藏不见影,池沼渐生冰。悬崖挂索藤花败,松竹凝寒色更青。
四众行够多时,前又遇城池相近。唐僧勒住马叫徒弟:“悟空,你看那厢楼阁峥嵘,
是个甚么去处?”行者抬头观看,乃是一座城池。真个是:

龙蟠形势,虎踞金城。四垂华盖近,百转紫墟平。玉石桥栏排巧兽,黄金台座
列贤明。真个是神洲都会,天府瑶京。万里邦畿固,千年帝业隆。蛮夷拱服君恩远,
海岳朝元圣会盈。御阶洁净,辇路清宁。酒肆歌声闹,花楼喜气生。未央宫外长春
树,应许朝阳彩凤鸣。
行者道:“师父,那座城池,是一国帝王之所。”八戒笑道:“天下府有府城,县有
县城,怎么就见是帝王之所?”行者道:“你不知帝王之居,与府县自是不同。你
看他四面有十数座门,周围有百十余里,楼台高耸,云雾缤纷。非帝京邦国,何以
有此壮丽?”沙僧道:“哥哥眼明,虽识得是帝王之处,却唤做甚么名色?”行者
道:“又无牌匾旌号,何以知之?须到城中询问,方可知也。”

长老策马,须臾到门。下马过桥,进门观看。只见六街三市,货殖通财;又见
衣冠隆盛,人物豪华。正行时,忽见有十数个和尚,一个个披枷戴锁,沿门乞化,
着实的蓝缕不堪。三藏叹曰:“兔死狐悲,物伤其类。”叫:“悟空,你上前去问他
一声,为何这等遭罪?”

行者依言,即叫:“那和尚,你是那寺里的?为甚事披枷戴锁?”众僧跪倒道:
“爷爷,我等是金光寺负屈的和尚。”行者道:“金光寺坐落何方?”众僧道:“转
过隅头就是。”行者将他带在唐僧前,问道:“怎生负屈,你说我听。”众僧道:“爷
爷,不知你们是那方来的,我等似有些面善。此问不敢在此奉告,请到荒山,具说
苦楚。”长老道:“也是。我们且到他那寺中去,仔细询问缘由。”

同至山门,门上横写七个金字,“敕建护国金光寺”。师徒们进得门来观看,但
见那:

古殿香灯冷,虚廊叶扫风。凌云千尺塔,养性几株松。满地落花无客过,檐前
蛛网任攀笼。空架鼓,枉悬钟,绘壁尘多彩象朦。讲座幽然僧不见,禅堂静矣鸟常
逢。凄凉堪叹息,寂寞苦无穷。佛前虽有香炉设,灰冷花残事事空。
三藏心酸,止不住眼中出泪。众僧们顶着枷锁,将正殿推开,请长老上殿拜佛。长
老进殿,奉上心香,叩齿三咂,却转于后面,见那方丈檐柱上又锁着六七个小和尚,
三藏甚不忍见。及到方丈,众僧俱来叩头,问道:“列位老爷像貌不一,可是东土
大唐来的么?”行者笑道:“这和尚有甚未卜先知之法?我们正是。你怎么认得?”
众僧道:“爷爷,我等有甚未卜先知之法,只是痛负了屈苦,无处分明,日逐家只
是叫天叫地。想是惊动天神,昨日夜间,各人都得一梦:说有个东土大唐来的圣僧,
救得我等性命,庶此冤苦可伸。今日果见老爷这般异像,故认得也。”

三藏闻言大喜道:“你这里是何地方?有何冤屈?”众僧跪告:“爷爷,此城名
唤祭赛国,乃西邦大去处。当年有四夷朝贡:南,月陀国;北,高昌国;东,西梁
国;西,本钵国。年年进贡美玉明珠,娇妃骏马。我这里不动干戈,不去征讨,他
那里自然拜为上邦。”三藏道:“既拜为上邦,想是你这国王有道,文武贤良。”众
僧道:“爷爷,文也不贤,武也不良,国君也不是有道。我这金光寺,自来宝塔上
祥云笼罩,瑞霭高升;夜放霞光,万里有人曾见;昼喷彩气,四国无不同瞻。故此
以为天府神京,四夷朝贡。只是三年之前,孟秋朔日,夜半子时,下了一场血雨。
天明时,家家害怕,户户生悲。众公卿奏上国王,不知天公甚事见责。当时延请道
士打醮,和尚看经,答天谢地。谁晓得我这寺里黄金宝塔污了,这两年外国不来朝
贡。我王欲要征伐,众臣谏道:我寺里僧人偷了塔上宝贝,所以无祥云瑞霭,外国
不朝。昏君更不察理。那些赃官,将我僧众拿了去,千般拷打,万样追求。当时我
这里有三辈和尚:前两辈已被拷打不过,死了;如今又捉我辈,问罪枷锁。老爷在
上,我等怎敢欺心,盗取塔中之宝!万望爷爷怜念,方以类聚,物以群分,舍大慈
大悲,广施法力,拯救我等性命!”

三藏闻言,点头叹道:“这桩事暗昧难明。一则是朝廷失政,二来是汝等有灾。
既然天降血雨,污了宝塔,那时节何不启本奏君,致令受苦?”众僧道:“爷爷,
我等凡人,怎知天意,况前辈俱未辨得,我等如何处之!”三藏道:“悟空,今日甚
时分了?”行者道:“有申时前后。”三藏道:“我欲面君倒换关文,奈何这众僧之
事,不得明白,难以对君奏言。我当时离了长安,在法门寺里立愿:上西方逢庙烧
香,遇寺拜佛,见塔扫塔。今日至此,遇有受屈僧人,乃因宝塔之累。你与我办一
把新笤帚,待我沐浴了,上去扫扫,即看这污秽之事何如,不放光之故何如,访着
端的,方好面君奏言,解救他们这苦难也。”

这些枷锁的和尚听说,连忙去厨房取把厨刀,递与八戒道:“爷爷,你将此刀
打开那柱子上锁的小和尚铁锁,放他去安排斋饭香汤,伏侍老爷进斋沐浴。我等且
上街化把新笤帚来与老爷扫塔。”八戒笑道:“开锁有何难哉?不用刀斧,教我那一
位毛脸老爷,他是开锁的积年。”行者真个近前,使个解锁法,用手一抹,几把锁
俱退落下。那小和尚俱跑到厨中,净刷锅灶,安排茶饭。三藏师徒们吃了斋,渐渐
天昏。只见那枷锁的和尚,拿了两把笤帚进来,三藏甚喜。

正说处,一个小和尚点了灯,来请洗澡。此时满天星月光辉,谯楼上更鼓齐发。
正是那:
四壁寒风起,万家灯火明。
六街关户牖,三市闭门庭。
钓艇归深树,耕犁罢短绳。
樵夫柯斧歇,学子诵书声。
三藏沐浴毕,穿了小袖褊衫,束了环绦,足下换一双软公鞋,手里拿一把新笤帚,
对众僧道:“你等安寝,待我扫塔去来。”行者道:“塔上既被血雨所污,又况日久
无光,恐生恶物;一则夜静风寒,又没个伴侣:自去恐有差池。老孙与你同上如何?”
三藏道:“甚好,甚好!”

两人各持一把,先到大殿上,点起琉璃灯,烧了香,佛前拜道:“弟子陈玄奘
奉东土大唐差往灵山参见我佛如来取经,今至祭赛国金光寺,遇本僧言宝塔被污,
国王疑僧盗宝,衔冤取罪,上下难明。弟子竭诚扫塔,望我佛威灵,早示污塔之原
因,莫致凡夫之冤屈。”祝罢,与行者开了塔门,自下层望上而扫。只见这塔,真
是:

峥嵘倚汉,突兀凌空。正唤做五色琉璃塔,千金舍利峰。梯转如穿窟,门开似
出笼。宝瓶影射天边月,金铎声传海上风。但见那虚檐拱斗,绝顶留云:虚檐拱斗,
作成巧石穿花凤;绝顶留云,造就浮屠绕雾龙。远眺可观千里外,高登似在九霄中。
层层门上琉璃灯,有尘无火;步步檐前白玉栏,积垢飞虫。塔心里,佛座上,香烟
尽绝;窗棂外,神面前,蛛网牵蒙。炉中多鼠粪,盏内少油熔。只因暗失中间宝,
苦杀僧人命落空。三藏发心将塔扫,管教重见旧时容。
唐僧用帚子扫了一层,又上一层。如此扫至第七层上,却早二更时分。那长老渐觉
困倦,行者道:“困了,你且坐下,等老孙替你扫罢。”三藏道:“这塔是多少层数?”
行者道:“怕不有十三层哩。”长老耽着劳倦道:“是必扫了,方趁本愿。”又扫了三
层,腰酸腿痛,就于十层上坐倒道:“悟空,你替我把那三层扫净下来罢。”行者抖
擞精神,登上第十一层,霎时又上到第十二层。正扫处,只听得塔顶上有人言语。
行者道:“怪哉,怪哉!这早晚有三更时分,怎么得有人在这顶上言语?断乎是邪物
也!且看看去。”

好猴王,轻轻的挟着笤帚,撒起衣服,钻出前门,踏着云头观看。只见第十三
层塔心里坐着两个妖精,面前放一盘下饭,一只碗,一把壶,在那里猜拳吃酒哩。
行者使个神通,丢了笤帚,掣出金箍棒,拦住塔门喝道:“好怪物,偷塔上宝贝的
原来是你!”两个怪物慌了,急起身,拿壶拿碗乱掼,被行者横铁棒拦住道:“我若
打死你,没人供状。”只把棒逼将去。那怪贴在壁上,莫想挣扎得动。口里只叫:“饶
命,饶命!不干我事!自有偷宝贝的在那里也。”行者使个拿法,一只手抓将过来,
径拿下第十层塔中。报道:“师父,拿住偷宝贝之贼了!”三藏正自盹睡,忽闻此言,
又惊又喜道:“是那里拿来的?”行者把怪物揪到面前跪下道:“他在塔顶上猜拳吃
酒耍子,是老孙听得喧哗,一纵云,跳到顶上拦住,未曾着力。但恐一棒打死,没
人供伏,故此轻轻捉来。师父可取他个口词,看他是那里妖精,偷的宝贝在于何处。”

那怪物战战兢兢,口叫:“饶命!”遂从实供道:“我两个是乱石山碧波潭万圣
龙王差来巡塔的。他叫做奔波儿灞,我叫做灞波儿奔。他是鲇鱼怪,我是黑鱼精。
因我万圣老龙生了一个女儿,就唤做万圣公主。那公主花容月貌,有二十分人才。
招得一个驸马,唤做九头驸马,神通广大。前年与龙王来此,显大法力,下了一阵
血雨,污了宝塔,偷了塔中的舍利子佛宝。公主又去大罗天上,灵霄殿前,偷了王
母娘娘的九叶灵芝草,养在那潭底下,金光霞彩,昼夜光明。近日闻得有个孙悟空
往西天取经,说他神通广大,沿路上专一寻人的不是,所以这些时常差我等来此巡
拦。若还有那孙悟空到时,好准备也。”行者闻言,嘻嘻冷笑道:“那孽畜等这等无
礼!怪道前日请牛魔王在那里赴会!原来他结交这伙泼魔,专干不良之事!”

说未了,只见八戒与两三个小和尚,自塔下提着两个灯笼走上来道:“师父,
扫了塔不去睡觉,在这里讲甚么哩?”行者道:“师弟,你来正好。塔上的宝贝,
乃是万圣老龙偷了去。今着这两个小妖巡塔,探听我等来的消息,却才被我拿住也。”
八戒道:“叫做甚么名字,甚么妖精?”行者道:“才然供了口词,一个叫做奔波儿
灞,一个叫做灞波儿奔;一个是鲇鱼怪,一个是黑鱼精。”八戒掣钯就打,道:“既
是妖精,取了口词,不打死何待?”行者道:“你不知。且留着活的,好去见皇帝
讲话,又好做凿眼去寻贼追宝。”好呆子,真个收了钯,一家一个,都抓下塔来。
那怪只叫:“饶命!”八戒道:“正要你鲇鱼、黑鱼做些鲜汤,与那负冤屈的和尚吃
哩!”

两三个小和尚,喜喜欢欢,提着灯笼,引长老下了塔。一个先跑报众僧道:“好
了,好了!我们得见青天了!偷宝贝的妖怪,已是爷爷们捉将来矣!”行者教:“拿铁
索来,穿了琵琶骨,锁在这里。汝等看守,我们睡觉去,明日再做理会。”那些和
尚都紧紧的守着,让三藏们安寝。

不觉的天晓。长老道:“我与悟空入朝,倒换关文去来。”长老即穿了锦袈裟,
戴了卢帽,整束威仪,拽步前进。行者也束一束虎皮裙,整一整绵布直裰,取了
关文同去。八戒道:“怎么不带这两个妖贼?”行者道:“待我们奏过了,自有驾帖
着人来提他。”

遂行至朝门外。看不尽那朱雀黄龙,清都绛阙。三藏到东华门,对阁门大使作
礼道:“烦大人转奏,贫僧是东土大唐差去西天取经者,意欲面君,倒换关文。”那
黄门官果与通报,至阶前奏道:“外面有两个异容异服僧人,称言南赡部洲东土唐
朝差往西方拜佛求经,欲朝我王,倒换关文。”

国王闻言,传旨教宣。长老即引行者入朝。文武百官,见了行者,无不惊怕。
有的说是猴和尚,有的说是雷公嘴和尚。个个悚然,不敢久视。长老在阶前舞蹈山
呼的行拜,大圣叉着手,斜立在旁,公然不动。长老启奏道:“臣僧乃南赡部洲东
土大唐国差来拜西方天竺国大雷音寺佛求取真经者。路经宝方,不敢擅过。有随身
关文,乞倒验方行。”那国王闻言大喜。传旨教宣唐朝圣僧上金銮殿,安绣墩赐坐。
长老独自上殿,先将关文捧上,然后谢恩敢坐。

那国王将关文看了一遍,心中喜悦道:“似你大唐王有疾,能选高僧,不避路
途遥远,拜我佛取经;寡人这里和尚,专心只是做贼,败国倾君!”三藏闻言,合
掌道:“怎见得败国倾君?”国王道:“寡人这国,乃是西域上邦,常有四夷朝贡,
皆因国内有个金光寺,寺内有座黄金宝塔,塔上有光彩冲天。近被本寺贼僧,暗窃
了其中之宝,三年无有光彩,外国这二年也不来朝,寡人心痛恨之。”三藏合掌笑
道:“万岁,‘差之毫厘,失之千里’矣。贫僧昨晚到于天府,一进城门,就见十数
个枷纽之僧。问及何罪,他道是金光寺负冤屈者。因到寺细审,更不干本寺僧人之
事:贫僧入夜扫塔,已获那偷宝之妖贼矣。”国王大喜道:“妖贼安在?”三藏道:
“现被小徒锁在金光寺里。”

那国王急降金牌:“着锦衣卫快到金光寺取妖贼来,寡人亲审。”三藏又奏道:
“万岁,虽有锦衣卫,还得小徒去方可。”国王道:“高徒在那里?”三藏用手指道:
“那玉阶旁立者便是。”国王见了,大惊道:“圣僧如此丰姿,高徒怎么这等像貌?”
孙大圣听见了,厉声高叫道:“陛下,‘人不可貌相,海水不可斗量。’若爱丰姿者,
如何捉得妖贼也?”国王闻言,回惊作喜道:“圣僧说的是。朕这里不选人才,只
要获贼得宝归塔为上。”再着当驾官看车盖,教锦衣卫好生伏侍圣僧去取妖贼来。
那当驾官即备大轿一乘,黄伞一柄,锦衣卫点起校尉,将行者八抬八绰,大四声喝
路,径至金光寺。自此惊动满城百姓,无处无一人不来看圣僧及那妖贼。

八戒、沙僧听得喝道,只说是国王差官,急出迎接,原来是行者坐在轿上。呆
子当面笑道:“哥哥,你得了本身也!”行者下了轿,搀着八戒道:“我怎么得了本
身?”八戒道:“你打着黄伞,抬着八人轿,却不是猴王之职分?故说你得了本身。”
行者道:“且莫取笑。”遂解下两个妖物,押见国王。沙僧道:“哥哥,也带挈小弟
带挈。”行者道:“你只在此看守行李、马匹。”那枷锁之僧道:“爷爷们都去承受皇
恩,等我们在此看守。”行者道:“既如此,等我去奏过国王,却来放你。”八戒揪
着一个妖贼,沙僧揪着一个妖贼,孙大圣依旧坐了轿,摆开头搭,将两个妖怪押赴
当朝。

须臾,至白玉阶。对国王道:“那妖贼已取来了。”国王遂降龙床,与唐僧及文
武多官,同目视之。那怪一个是暴腮乌甲,尖嘴利牙;一个是滑皮大肚,巨口长须。
虽然是有足能行,大抵是变成的人像。国王问曰:“你是何方贼怪,那处妖精,几
年侵吾国土,何年盗我宝贝,一盘共有多少贼徒,都唤做甚么名字,从实一一供来!”
二怪朝上跪下,颈内血淋淋的,更不知疼痛。供道:

“三载之外,七月初一,有个万圣龙王,帅领许多亲戚,住居在本国东南,离
此处路有百十。潭号碧波,山名乱石。生女
多娇,妖娆美色。招赘一个九头驸马,神通无敌。他知你塔上珍奇,与龙王合盘做
贼,先下血雨一场,后把舍利偷讫。见如今照耀龙宫,纵黑夜明如白日。公主施能,
寂寂密密,又偷了王母灵芝,在潭中温养宝物。我两个不是贼头,乃龙王差来小卒。
今夜被擒,所供是实。”
国王道:“既取了供,如何不供自家名字?”那怪道:“我唤做奔波儿灞,他唤做灞
波儿奔。奔波儿灞是个鲇鱼怪,灞波儿奔是个黑鱼精。”国王教锦衣卫好生收监。
传旨:“赦了金光寺众僧的枷锁,快教光禄寺排宴,就于麒麟殿上谢圣僧获贼之功,
议请圣僧捕擒贼首。”

光禄寺即时备了荤素两样筵席。国王请唐僧四众上麒麟殿叙坐。问道:“圣僧
尊号?”唐僧合掌道:“贫僧俗家姓陈,法名玄奘。蒙君赐姓唐,贱号三藏。”国王
又问:“圣僧高徒何号?”三藏道:“小徒俱无号。第一个名孙悟空,第二个名猪悟
能,第三个名沙悟净:此乃南海观世音菩萨起的名字。因拜贫僧为师,贫僧又将悟
空叫做行者;悟能叫做八戒;悟净叫做和尚。”国王听毕,请三藏坐了上席;孙行
者坐了侧首左席;猪八戒、沙和尚坐了侧首右席。俱是素果、素菜、素茶、素饭。
前面一席荤的,坐了国王;下首有百十席荤的,坐了文武多官。众臣谢了君恩,徒
告了师罪,坐定。国王把盏,三藏不敢饮酒,他三个各受了安席酒。下边只听得管
弦齐奏,乃是教坊司动乐。你看八戒放开食嗓,真个是虎咽狼吞,将一席果菜之类,
吃得罄尽。少顷间,添换汤饭又来,又吃得一毫不剩。巡酒的来,又杯杯不辞。这
场筵席,直乐到午后方散。

三藏谢了盛宴。国王又留住道:“这一席聊表圣僧获怪之功。”教光禄寺:“快
翻席到建章宫里,再请圣僧定捕贼首,取宝归塔之计。”三藏道:“既要捕贼取宝,
不劳再宴。贫僧等就此辞王,就擒捉妖怪去也。”国王不肯,一定请到建章宫,又
吃了一席。国王举酒道:“那位圣僧帅众出师,降妖捕贼?”三藏道:“教大徒弟孙
悟空去。”大圣拱手应承。国王道:“孙长老既去,用多少人马?几时出城?”八戒
忍不住高声叫道:“那里用甚么人马!又那里管甚么时辰!趁如今酒醉饭饱,我共师
兄去,手到擒来!”三藏甚喜道:“八戒这一向勤紧啊!”行者道:“既如此,着沙僧
弟保护师父,我两个去来。”那国王道:“二位长老既不用人马,可用兵器?”八戒
笑道:“你家的兵器,我们用不得。我弟兄自有随身器械。”国王闻说,即取大觥来,
与二位长老送行。孙大圣道:“酒不吃了,只教锦衣卫把两个小妖拿来,我们带了
他去做凿眼。”国王传旨,即时提出。二人挟着两个小妖,驾风头,使个摄法,径
上东南去了。噫!他那:
君臣一见腾风雾,才识师徒是圣僧。

毕竟不知此去如何擒获,且听下回分解。