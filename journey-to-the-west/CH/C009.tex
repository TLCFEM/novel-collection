\chapter{袁守诚妙算无私曲~老龙王拙计犯天条}

诗曰:
都城大国实堪观,八水周流绕四山。
多少帝王兴此处,古来天下说长安。

此单表陕西大国长安城,乃历代帝王建都之地。自周、秦、汉以来,三州花似
锦,八水绕城流。三十六条花柳巷,七十二座管弦楼。华夷图上看,天下最为头。
真是奇胜之方。今却是大唐太宗文皇帝登基,改元龙集贞观。此时已登极十三年,
岁在己巳。且不说他驾前有安邦定国的英豪,与那创业争疆的杰士。

却说长安城外泾河岸边,有两个贤人:一个是渔翁,名唤张稍;一个是樵子,
名唤李定。他两个是不登科的进士,能识字的山人。一日,在长安城里,卖了肩上
柴,货了篮中鲤,同入酒馆之中,吃了半酣,各携一瓶,顺泾河岸边,徐步而回。
张稍道:“李兄,我想那争名的,因名丧体;夺利的,为利亡身;受爵的,抱虎而
眠;承恩的,袖蛇而走。算起来,还不如我们水秀山青,逍遥自在;甘淡薄,随缘
而过。”李定道:“张兄说得有理。但只是你那水秀,不如我的山青。”张稍道:“你
山青不如我的水秀。有一《蝶恋花》词为证。词曰:

烟波万里扁舟小,静依孤篷,西施声音绕。涤虑洗心名利少,闲攀蓼穗蒹葭草。

数点沙鸥堪乐道,柳岸芦湾,妻子同
欢笑。一觉安眠风浪俏,无荣无辱无烦恼。”
李定道:“你的水秀,不如我的山青。也有个《蝶恋花》词为证。词曰:

云林一段松花满,默听莺啼,巧舌如调管。红瘦绿肥春正暖,倏然夏至光阴转。

又值秋来容易换,黄花香,堪供玩。迅速严冬如指拈,逍遥四季无人管。”
渔翁道:“你山青不如我水秀,受用些好物。有一《鹧鸪天》为证:

仙乡云水足生涯,摆橹横舟便是家。活剖鲜鳞烹绿鳖,旋蒸紫蟹煮红虾。
青
芦笋,水荇芽,菱角鸡头更可夸。娇藕老莲芹叶嫩,慈菇茭白鸟英花。”
樵夫道:“你水秀不如我山青,受用些好物。亦有一《鹧鸪天》为证:

崔巍峻岭接天涯,草舍茅庵是我家。腌腊鸡鹅强蟹鳖,獐兔鹿胜鱼虾。
香
椿叶,黄楝芽,竹笋山茶更可夸。紫李红桃梅杏熟,甜梨酸枣木樨花。”
渔翁道:“你山青真个不如我的水秀。又有《天仙子》一首:

一叶小舟随所寓,万迭烟波无恐惧。垂钩撒网捉鲜鳞,没酱腻,偏有味,老妻
稚子团圆会。
鱼多又货长安市,换得香醪吃个醉。蓑衣当被卧秋江,鼾鼾睡,无
忧虑,不恋人间荣与贵。”
樵子道:“你水秀还不如我的山青。也有《天仙子》一首:

舍数椽山下盖,松竹梅兰真可爱。穿林越岭觅干柴,没
人怪,从我卖,或少或多凭世界。

将钱沽酒随心快,瓦钵磁瓯殊自在。醉
了卧松阴,无挂碍,无利害,不管人间兴与败。”
渔翁道:“李兄,你山中不如我水上生意快活。有一《西江月》为证:

红蓼花繁映月,黄芦叶乱摇风。碧天清远楚江空,牵搅一潭星动。

入网大
鱼作队,吞钩小鳜成丛。得来烹煮味偏浓,笑傲江湖打哄。”
樵夫道:“张兄,你水上还不如我山中的生意快活。亦有《西江月》为证:

败叶枯藤满路,破梢老竹盈山。女萝干葛乱牵攀,折取收绳杀担。

虫蛀空
心榆柳,风吹断头松楠。采来堆积备冬寒,换酒换钱从俺。”
渔翁道:“你山中虽可比过,还不如我水秀的幽雅。有一《临江仙》为证:

潮落旋移孤艇去,夜深罢棹歌来。蓑衣残月甚幽哉,宿鸥惊不起,天际彩云开。

困卧芦洲无个事,三竿日上还捱。随心尽意自安排,朝臣寒待漏,争似我宽怀?”
樵夫道:“你水秀的幽雅,还不如我山青更幽雅。亦有《临江仙》可证:

苍径秋高拽斧去,晚凉抬担回来。野花插鬓更奇哉,拨云寻路出,待月叫门开。

稚子山妻欣笑接,草床木枕捱。蒸梨炊黍旋铺排,瓮中新酿熟,真个壮幽怀!”
渔翁道:“这都是我两个生意,赡身的勾当,你却没有我闲时节的好处。有诗为证,
诗曰:
闲看天边白鹤飞,停舟溪畔掩苍扉。
倚篷教子搓钓线,罢棹同妻晒网围。
性定果然知浪静,身安自是觉风微。
绿蓑青笠随时着,胜挂朝中紫绶衣。”
樵夫道:“你那闲时又不如我的闲时好也。亦有诗为证。诗曰:
闲观缥缈白云飞,独坐茅庵掩竹扉。
无事训儿开卷读,有时对客把棋围。
喜来策杖歌芳径,兴到携琴上翠微。
草履麻绦粗布被,心宽强似着罗衣。”
张稍道:“李定,我两个‘真是微吟可相狎,不须檀板共金樽。’但散道词章,不为
稀罕;且各联几句,看我们渔樵攀话何如?”李定道:“张兄言之最妙。请兄先吟。”

“舟停绿水烟波内,家住深山旷野中。偏爱溪桥春水涨,最怜岩岫晓云蒙。龙
门鲜鲤时烹煮,虫蛀干柴日燎烘。钓网多般堪赡老,担绳二事可容终。小舟仰卧观
飞雁,草径斜听唳鸿。口舌场中无我分,是非海内少吾踪。溪边挂晒缯如锦,石
上重磨斧似锋。秋月晖晖常独钓,春山寂寂没人逢。鱼多换酒同妻饮,柴剩沽壶共
子丛。自唱自斟随放荡,长歌长叹任颠风。呼兄唤弟邀船伙,挈友携朋聚野翁。行
令猜拳频递盏,拆牌道字漫传钟。烹虾煮蟹朝朝乐,炒鸭鸡日日丰。愚妇煎茶情
散诞,山妻造饭意从容。晓来举杖淘轻浪,日出担柴过大。雨后披蓑擒活鲤,风
前弄斧伐枯松。潜踪避世妆痴蠢,隐姓埋名作哑聋。”
张稍道:“李兄,我才僭先起句,今到我兄,也先起一联,小弟亦当续之:

风月佯狂山野汉,江湖寄傲老余丁。清闲有分随潇洒,口舌无闻喜太平。月夜
身眠茅屋稳,天昏体盖箬蓑轻。忘情结识松梅友,乐意相交鸥鹭盟。名利心头无算
计,干戈耳畔不闻声。随时一酌香醪酒,度日三餐野菜羹。两束柴薪为活计,一竿
钓线是营生。闲呼稚子磨钢斧,静唤憨儿补旧缯。春到爱观杨柳绿,时融喜看荻芦
青。夏天避暑修新竹,六月乘凉摘嫩菱。霜降鸡肥常日宰,重阳蟹壮及时烹。冬来
日上还沉睡,数九天高自不蒸。八节山中随放性,四时湖里任陶情。采薪自有仙家
兴,垂钓全无世俗形。门外野花香艳艳,船头绿水浪平平。身安不说三公位,性定
强如十里城。十里城高防阃令,三公位显听宣声。乐山乐水真是罕,谢天谢地谢神
明。”
他二人既各道词章,又相联诗句,行到那分路去处,躬身作别。张稍道:“李兄呵,
途中保重!上山仔细看虎。假若有些凶险,正是‘明日街头少故人’!”李定闻言,
大怒道:“你这厮惫懒!好朋友也替得生死,你怎么咒我?我若遇虎遭害,你必遇浪
翻江!”张稍道:“我永世也不得翻江。”李定道:“‘天有不测风云,人有暂时祸福。’
你怎么就保得无事?”张稍道:“李兄,你虽这等说,你还没捉摸;不若我的生意
有捉摸,定不遭此等事。”李定道:“你那水面上营生,极凶极险,隐隐暗暗,有甚
么捉摸?”张稍道:“你是不晓得。这长安城里,西门街上,有一个卖卦的先生。
我每日送他一尾金色鲤,他就与我袖传一课。依方位,百下百着。今日我又去买卦,
他教我在泾河湾头东边下网,西岸抛钓,定获满载鱼虾而归。明日上城来,卖钱沽
酒,再与老兄相叙。”二人从此叙别。

这正是“路上说话,草里有人。”原来这泾河水府有一个巡水的夜叉,听见了
百下百着之言,急转水晶宫,慌忙报与龙王道:“祸事了!祸事了!”龙王问:“有甚
祸事?”夜叉道:“臣巡水去到河边,只听得两个渔樵攀话。相别时,言语甚是利
害。那渔翁说:长安城里,西门街上,有个卖卦先生,算得最准;他每日送他鲤鱼
一尾,他就袖传一课,教他百下百着。若依此等算准,却不将水族尽情打了?何以
壮观水府,何以跃浪翻波,辅助大王威力?”龙王甚怒,急提了剑,就要上长安城,
诛灭这卖卦的。旁边闪过龙子、龙孙、虾臣、蟹士、鲥军师、鳜少卿、鲤太宰,一
齐启奏道:“大王且息怒。常言道:‘过耳之言,不可听信。’大王此去,必有云从,
必有雨助,恐惊了长安黎庶,上天见责。大王隐显莫测,变化无方,但只变一秀士,
到长安城内,访问一番。果有此辈,容加诛灭不迟;若无此辈,可不是妄害他人也?”
龙王依奏,遂弃宝剑,也不兴云雨,出岸上,摇身一变,变作一个白衣秀士。真个:

丰姿英伟,耸壑昂霄。步履端祥,循规蹈矩。语言遵孔孟,礼貌体周文。身穿
玉色罗服,头戴逍遥一字巾。
上路来拽开云步,径到长安城西门大街上。只见一簇人,挤挤杂杂,闹闹哄哄,内
有高谈阔论的道:“属龙的本命,属虎的相冲。寅辰巳亥,虽称合局,但只怕的是
日犯岁君。”龙王闻言,情知是那卖卜之处。走上前,分开众人,望里观看。只见:

四壁珠玑,满堂绮绣。宝鸭香无断,磁瓶水恁清。两边罗列王维画,座上高悬
鬼谷形。端溪砚,金烟墨,相衬着霜毫大笔;火珠林,郭璞数,谨对了台政新经。
六爻熟谙,八卦精通。能知天地理,善晓鬼神情。一子午安排定,满腹星辰布列
清。真个那未来事,过去事,观如月镜;几家兴,几家败,鉴若神明。知凶定吉,
断死言生。开谈风雨迅,下笔鬼神惊。招牌有字书名姓,神课先生袁守诚。
此人是谁?原来是当朝钦天监台正先生袁天罡的叔父,袁守诚是也。那先生果然相
貌稀奇,仪容秀丽;名扬大国,术冠长安。龙王入门来,与先生相见。礼毕,请龙
上坐,童子献茶。先生问曰:“公来问何事?”龙王曰:“请卜天上阴晴事如何。”
先生即袖传一课,断曰:“云迷山顶,雾罩林梢。若占雨泽,准在明朝。”龙曰:“明
日甚时下雨?雨有多少尺寸?”先生道:“明日辰时布云,巳时发雷,午时下雨,未
时雨足,共得水三尺三寸零四十八点。”龙王笑曰:“此言不可作戏。如是明日有雨,
依你断的时辰、数目,我送课金五十两奉谢。若无雨,或不按时辰数目,我与你实
说:定要打坏你的门面,扯碎你的招牌,即时赶出长安,不许在此惑众!”先生欣
然而答:“这个一定任你。请了,请了。明朝雨后来会。”

龙王辞别,出长安,回水府。大小水神接着,问曰:“大王访那卖卦的如何?”
龙王道:“有,有,有!但是一个掉嘴口,讨春的先生。我问他几时下雨,他就说明
日下雨;问他甚么时辰,甚么雨数,他就说辰时布云,巳时发雷,午时下雨,未时
雨足,得水三尺三寸零四十八点。我与他打了个赌赛:若果如他言,送他谢金五十
两;如略差些,就打破他门面,赶他起身,不许在长安惑众。”众水族笑曰:“大王
是八河都总管,司雨大龙神,有雨无雨,惟大王知之;他怎敢这等胡言?那卖卦的
定是输了!定是输了!”

此时龙子、龙孙与那鱼卿、蟹士正欢笑谈此事未毕,只听得半空中叫:“泾河
龙王接旨。”众抬头上看,是一个金衣力士,手擎玉帝敕旨,径投水府而来。慌得
龙王整衣端肃,焚香接了旨。金衣力士回空而去。龙王谢恩,拆封看时,上写着:
敕命八河总,驱雷掣电行;
明朝施雨泽,普济长安城。
旨意上时辰、数目,与那先生判断者毫发不差。唬得那龙王魂飞魄散。少顷苏醒,
对众水族曰:“尘世上有此灵人!真个是能通天彻地,却不输与他呵!”鲥军师奏云:
“大王放心。要赢他有何难处?臣有小计,管教灭那厮的口嘴。”龙王问计,军师道:
“行雨差了时辰,少些点数,就是那厮断卦不准,怕不赢他?那时碎招牌,赶他
跑路,果何难也?”龙王依他所奏,果不担忧。

至次日,点札风伯、雷公、云童、电母,直至长安城九霄空上。他挨到那巳时
方布云,午时发雷,未时落雨,申时雨止,却只得三尺零四十点:改了他一个时辰,
克了他三寸八点。雨后发放众将班师。他又按落云头,还变作白衣秀士,到那西门
里大街上,撞入袁守诚卦铺,不容分说,就把他招牌、笔、砚等一齐碎。那先生
坐在椅上,公然不动。这龙王又轮起门板便打,骂道:“这妄言祸福的妖人,擅惑
众心的泼汉!你卦又不灵,言又狂谬!说今日下雨的时辰、点数俱不相对,你还危然
高坐,趁早去,饶你死罪!”守诚犹公然不惧分毫,仰面朝天冷笑道:“我不怕!我
不怕!我无死罪,只怕你倒有个死罪哩!别人好瞒,只是难瞒我也。我认得你,你不
是秀士,乃是泾河龙王。你违了玉帝敕旨,改了时辰,克了点数,犯了天条。你在
那‘剐龙台’上,恐难免一刀,你还在此骂我?”

龙王见说,心惊胆战,毛骨悚然。急丢了门板,整衣伏礼,向先生跪下道:“先
生休怪。前言戏之耳,岂知弄假成真,果然违犯天条,奈何?望先生救我一救!不然,
我死也不放你。”守诚曰:“我救你不得,只是指条生路与你投生便了。”龙曰:“愿
求指教。”先生曰:“你明日午时三刻,该赴人曹官魏征处听斩。你果要性命,须当
急急去告当今唐太宗皇帝方好。那魏征是唐王驾下的丞相,若是讨他个人情,方保
无事。”龙王闻言,拜辞含泪而去。不觉红日西沉,太阴星上。但见:

烟凝山紫归鸦倦,远路行人投旅店。渡头新雁宿眭沙,银河现。催更筹,孤村
灯火光无焰。

风袅炉烟清道院,蝴蝶梦中人不见。月移花影上栏杆,星光乱。
漏声换,不觉深沉夜已半。
这泾河龙王也不回水府;只在空中,等到子时前后,收了云头,敛了雾角,径来皇
宫门首。此时唐王正梦出宫门之外,步月花阴。忽然龙王变作人相,上前跪拜。口
叫“陛下,救我!救我!”太宗云:“你是何人?朕当救你。”龙王云:“陛下是真龙,
臣是业龙。臣因犯了天条,该陛下贤臣人曹官魏征处斩,故来拜求,望陛下救我一
救!”太宗曰:“既是魏征处斩,朕可以救你。你放心前去。”龙王欢喜,叩谢而去。

却说那太宗梦醒后,念念在心。早已至五鼓三点,太宗设朝,聚集两班文武官
员。但见那:

烟笼凤阙,香蔼龙楼,光摇丹动,云拂翠华流。君臣相契同尧舜,礼乐威严
近汉周。侍臣灯,宫女扇,双双映彩;孔雀屏,麒麟殿,处处光浮。山呼万岁,华
祝千秋。静鞭三下响,衣冠拜冕旒。宫花灿烂天香袭,堤柳轻柔御乐讴。珍珠帘,
翡翠帘,金钩高控;龙凤扇,山河扇,宝辇停留。文官英秀,武将抖搜。御道分高
下,丹墀列品流。金章紫绶乘三象,地久天长万万秋。
众官朝贺已毕,各各分班。唐王闪凤目龙睛,一一从头观看,只见那文官内是房玄
龄、杜如晦、徐世𪟝、许敬宗、王圭等,武官内是马三宝、段志贤、殷开山、程咬
金、刘洪纪、胡敬德、秦叔宝等,一个个威仪端肃,却不见魏征丞相。唐王召徐世
𪟝上殿道:“朕夜间得一怪梦:梦见一人,迎面拜谒,口称是泾河龙王,犯了天条,
该人曹官魏征处斩,拜告寡人救他,朕已许诺。今日班前独不见魏征,何也?”世
𪟝对曰:“此梦告准,须臾魏征来朝,陛下不要放他出门。过此一日,可救梦中之
龙。”唐王大喜,即传旨,着当驾官宣魏征入朝。

却说魏征丞相在府,夜观乾象,正宝香,只闻得九霄鹤唳,却是天差仙使,
捧玉帝金旨一道,着他午时三刻,梦斩泾河老龙。这丞相谢了天恩,斋戒沐浴,在
府中试慧剑,运元神,故此不曾入朝。一见当驾官赍旨来宣,惶惧无任;又不敢违
迟君命,只得急急整衣束带,同旨入朝,在御前叩头请罪。唐王出旨道:“赦卿无
罪。”那时诸臣尚未退朝,至此,却命卷帘散朝。独留魏征,宣上金銮,召入便殿,
先议论安邦之策,定国之谋。将近巳末午初时候,却命宫人,取过大棋来,“朕与
贤卿对弈一局。”众嫔妃随取棋枰,铺设御案。魏征谢了恩,即与唐王对弈。

毕竟不知胜负如何,且听下回分解。