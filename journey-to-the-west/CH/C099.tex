\chapter{九九数完魔灭尽~三三行满道归根}

话表八金刚既送唐僧回国不题。那三层门下,有五方揭谛、四值功曹、六丁六
甲、护教伽蓝,走向观音菩萨前启道:“弟子等向蒙菩萨法旨,暗中保护圣僧,今
日圣僧行满,菩萨缴了佛祖金旨,我等望菩萨准缴法旨。”菩萨亦甚喜道:“准缴,
准缴。”又问道:“那唐僧四众,一路上心行何如?”诸神道:“委实心虔志诚,料
不能逃菩萨洞察;但只是唐僧受过之苦,真不可言。他一路上历过的灾愆患难,弟
子已谨记在此。这就是他灾难的簿子。”菩萨从头看了一遍。上写着:
蒙差揭谛皈依旨

谨记唐僧难数清
金蝉遭贬第一难
出胎几杀第二难
满月抛江第三难
寻亲报冤第四难
出城逢虎第五难
折从落坑第六难
双叉岭上第七难
两界山头第八难
陡涧换马第九难
夜被火烧第十难
失却袈裟十一难
收降八戒十二难
黄风怪阻十三难
请求灵吉十四难
流沙难渡十五难
收得沙僧十六难
四圣显化十七难
五庄观中十八难
难活人参十九难
贬退心猿二十难
黑松林失散二十一难
宝象国捎书二十二难

金銮殿变虎二十三难
平顶山逢魔二十四难
莲花洞高悬二十五难
乌鸡国救主二十六难
被魔化身二十七难
号山逢怪二十八难
风摄圣僧二十九难
心猿遭害三十难
请圣降妖三十一难
黑河沉没三十二难
搬运车迟三十三难
大赌输赢三十四难
祛道兴僧三十五难
路逢大水三十六难
身落天河三十七难
鱼篮现身三十八难
金山遇怪三十九难
普天神难伏四十难
问佛根源四十一难
吃水遭毒四十二难
西梁国留婚四十三难
琵琶洞受苦四十四难
再贬心猿四十五难
难辨猕猴四十六难
路阻火焰山四十七难
求取芭蕉扇四十八难
收缚魔王四十九难
赛城扫塔五十难
取宝救僧五十一难
棘林吟咏五十二难
小雷音遇难五十三难
诸天神遭困五十四难
稀柿秽阻五十五难
朱紫国行医五十六难
拯救疲癃五十七难
降妖取后五十八难
七情迷没五十九难
多目遭伤六十难
路阻狮驼六十一难
怪分三色六十二难
城里遇灾六十三难
请佛收魔六十四难
比丘救子六十五难
辨认真邪六十六难
松林救怪六十七难
僧房卧病六十八难
无底洞遭困六十九难
灭法国难行七十难
隐雾山遇魔七十一难
凤仙郡求雨七十二难
失落兵器七十三难
会庆钉钯七十四难
竹节山遭难七十五难
玄英洞受苦七十六难
赶捉犀牛七十七难
天竺招婚七十八难
铜台府监禁七十九难
凌云渡脱胎八十难
路经十万八千里
圣僧历难簿分明
菩萨将难簿目过了一遍,急传声道:“佛门中‘九九’归真。圣僧受过八十难,还
少一难,不得完成此数。”即令揭谛:“赶上金刚,还生一难者。”这揭谛得令,飞
云一驾向东来。一昼夜赶上八大金刚,附耳低言道:“如此如此,谨遵菩萨法旨,
不得违误。”八金刚闻得此言,刷的把风按下,将他四众,连马与经,坠落下地。
噫!正是那:
九九归真道行难,坚持笃志立玄关。
必须苦练邪魔退,定要修持正法还。
莫把经章当容易,圣僧难过许多般。
古来妙合参同契,毫发差殊不结丹。

三藏脚踏了凡地,自觉心惊。八戒呵呵大笑道:“好,好,好!这正是要快得迟。”
沙僧道:“好,好,好!因是我们走快了些儿,教我们在此歇歇哩。”大圣道:“俗语
云:‘十日滩头坐,一日行九滩。’”三藏道:“你三个且休斗嘴。认认方向,看这是
甚么地方。”沙僧转头四望道:“是这里,是这里!师父,你听听水响。”行者道:“水
响想是你的祖家了。”八戒道:“他祖家乃流沙河。”沙僧道:“不是,不是。此通天
河也。”三藏道:“徒弟啊,仔细看在那岸。”行者纵身跳起,用手搭凉篷,仔细看
了,下来道:“师父,此是通天河西岸。”三藏道:“我记起来了。东岸边原有个陈
家庄。那年到此,亏你救了他儿女,深感我们,要造船相送,幸白鼋伏渡。我记得
西岸上,四无人烟。这番如何是好?”八戒道:“只说凡人会作弊,原来这佛面前
的金刚也会作弊。他奉佛旨,教送我们东回,怎么到此半路上就丢下我们?如今岂
不进退两难!怎生过去!”沙僧道:“二哥休报怨。我的师父已得了道。前在凌云渡
已脱了凡胎,今番断不落水。教师兄同你我都作起摄法,把师父驾过去也。”行者
频频的暗笑道:“驾不去,驾不去!”你看他怎么就说个驾不去?若肯使出神通,说
破飞升之奥妙,师徒们就一千个河也过去了;只因心里明白,知道唐僧九九之数未
完,还该有一难,故羁留于此。

师徒们口里纷纷的讲,足下徐徐的行,直至水边,忽听得有人叫道:“唐圣僧,
唐圣僧!这里来,这里来!”四众皆惊。举头观看,四无人迹,又没舟船,却是一个
大白赖头鼋在岸边探着头叫道:“老师父,我等了你这几年,却才回也?”行者笑
道:“老鼋,向年累你,今岁又得相逢。”三藏与八戒、沙僧都欢喜不尽。行者道:
“老鼋,你果有接待之心,可上岸来。”那鼋即纵身爬上河来。行者叫把马牵上他
身。八戒还蹲在马尾之后。唐僧站在马颈左边。沙僧站在右边。行者一脚踏着老鼋
的项,一脚踏着老鼋的头叫道:“老鼋,好生走稳着。”那老鼋蹬开四足,踏水面如
行平地,将他师徒四众,连马五口,驮在身上,径回东岸而来。诚所谓:
不二门中法奥玄,诸魔战退识人天。
本来面目今方见,一体原因始得全。
秉证三乘随出入,丹成九转任周旋。
挑包飞杖通休讲,幸喜还元遇老鼋。
老鼋驮着他们,波踏浪,行经多半日,将次天晚,好近东岸,忽然问曰:“老师
父,我向年曾央到西方见我佛如来,与我问声归着之事,还有多少年寿,果曾问否?”
原来那长老自到西天玉真观沐浴,凌云渡脱胎,步上灵山,专心拜佛及参诸佛菩萨
圣僧等众,意念只在取经,他事一毫不理,所以不曾问得老鼋年寿,无言可答;却
又不敢欺,打诳语,沉吟半晌,不曾答应。老鼋即知不曾替问,他就将身一幌,唿
喇的淬下水去,把他四众连马并经,通皆落水。咦!还喜得唐僧脱了胎,成了道。
若似前番,已经沉底。又幸白马是龙,八戒、沙僧会水,行者笑巍巍显大神通,把
唐僧扶驾出水,登彼东岸。只是经包、衣服、鞍辔俱湿了。

师徒方登岸整理,忽又一阵狂风,天色昏暗,雷俱作,走石飞沙。但见那:

一阵风,乾坤播荡;一声雷,振动山川;一个,钻云飞火;一天雾,大地遮
漫。风气呼号,雷声激烈;掣红绡,雾迷星月。风鼓的尘沙扑面,雷惊的虎豹藏
形;幌的飞禽叫噪,雾漫的树木无踪。那风搅得个通天河波浪翻腾,那雷振得个
通天河鱼龙丧胆;那照得个通天河彻底光明,那雾盖得个通天河岸崖昏惨。好风!
颓山烈石松篁倒;好雷!惊蛰伤人威势豪。好!流天照野金蛇走;好雾!混混漫空
蔽九霄。
唬得那三藏按住了经包,沙僧压住了经担,八戒牵住了白马,行者却双手轮起铁棒,
左右护持。原来那风、雾、雷、乃是些阴魔作号,欲夺所取之经。劳攘了一夜,
直到天明,却才止息。长老一身水衣,战兢兢的道:“悟空,这是怎的起?”行者
气呼呼的道:“师父,你不知就里。我等保护你取获此经,乃是夺天地造化之功,
可以与乾坤并久,日月同明,寿享长春,法身不朽:此所以为天地不容,鬼神所忌,
欲来暗夺之耳。一则这经是水湿透了;二则是你的正法身压住,雷不能轰,电不能
照,雾不能迷;又是老孙抡着铁棒,使纯阳之性,护持住了;及至天明,阳气又盛:
所以不能夺去。”

三藏、八戒、沙僧方才省悟,各谢不尽。少顷,太阳高照,却移经于高崖上,
开包晒晾。至今彼处晒经之石尚存。他们又将衣鞋都晒在崖旁,立的立,坐的坐,
跳的跳。真个是:
一体纯阳喜向阳,阴魔不敢逞强梁。
须知水胜真经伏,不怕风雷雾光。
自此清平归正觉,从今安泰到仙乡。
晒经石上留踪迹,千古无魔到此方。

他四众检看经本,一一晒晾,早见几个打鱼人,来过河边,抬头看见。内有认
得的道:“老师父可是前年过此河往西天取经的?”八戒道:“正是,正是。你是那
里人?怎么认得我们?”渔人道:“我们是陈家庄上人。”八戒道:“陈家庄离此有多
远?”渔人道:“过此冲南有二十里,就是也。”八戒道:“师父,我们把经搬到陈
家庄上晒去。他那里有住坐,又有得吃,就教他家与我们浆浆衣服,却不是好?”
三藏道:“不去罢。在此晒干了,就收拾找路回也。”那几个渔人,行过南冲,恰遇
着陈澄。叫道:“二老官,前年在你家替祭儿子的师父回来了。”陈澄道:“你在那
里看见?”渔人回指道:“都在那石上晒经哩。”

陈澄随带了几个佃户,走过冲来望见,跑近前跪下道:“老爷取经回来,功成
行满,怎么不到舍下,却在这里盘弄?快请,快请到舍。”行者道:“等晒干了经,
和你去。”陈澄又问道:“老爷的经典、衣物,如何湿了?”三藏道:“昔年亏白鼋
驮渡河西,今年又蒙他驮渡河东。已将近岸,被他问昔年托问佛祖寿年之事,我本
未曾问得,他遂淬在水内,故此湿了。”又将前后事细说了一遍。那陈澄拜请甚恳,
三藏无已,遂收拾经卷。不期石上把《佛本行经》沾住了几卷,遂将经尾沾破了。
所以至今《本行经》不全,晒经石上犹有字迹。三藏懊悔道:“是我们怠慢了,不
曾看顾得!”行者笑道:“不在此,不在此!盖天地不全。这经原是全全的,今沾破
了,乃是应不全之奥妙也。岂人力所能与耶!”师徒们果收拾毕,同陈澄赴庄。

那庄上人家,一个传十,十个传百,百个传千,若老若幼,都来接看。陈清闻
说,就摆香案,在门前迎迓;又命鼓乐吹打。少顷到了,迎入。陈清领合家人眷,
俱出来拜见,拜谢昔日救女儿之恩。随命看茶摆斋。三藏自受了佛祖的仙品、仙肴,
又脱了凡胎成佛,全不思凡间之食。二老苦劝,没奈何,略见他意。孙大圣自来不
吃烟火食,也道:“够了。”沙僧也不甚吃。八戒也不似前番,就放下碗。行者道:
“呆子也不吃了?”八戒道:“不知怎么,脾胃一时就弱了。”遂此收了斋筵,却又
问取经之事。三藏又将先至玉真观沐浴,凌云渡脱胎,及至雷音寺参如来,蒙珍楼
赐宴,宝阁传经,始被二尊者索人事未遂,故传无字之经,后复拜告如来,始得授
一藏之数,并白鼋淬水,阴魔暗夺之事,细细陈了一遍,就欲拜别。

那二老举家,如何肯放,且道:“向蒙救拔儿女深恩莫报,已创建一座院宇,
名曰救生寺,专侍奉香火不绝。”又唤出原替祭之儿女陈关保、一秤金叩谢,复请
至寺观看。三藏却又将经包儿收在他家堂前,与他念了一卷《宝常经》。后至寺中,
只见陈家又设馔在此。还不曾坐下,又一起来请。还不曾举箸,又一起来请。络绎
不绝,争不上手。三藏俱不敢辞,略略见意。只见那座寺果盖得齐整:

山门红粉腻,多赖施主功。一座楼台从此立,两廊房宇自今兴。朱红隔扇,七
宝玲珑。香气飘云汉,清光满太空。几株嫩柏还浇水,数干乔松未结丛。活水迎前,
通天叠叠翻波浪;高崖倚后,山脉重重接地龙。
三藏看毕,才上高楼。楼上果装塑着他四众之像。八戒看见,扯着行者道:“兄长
的相儿甚像。”沙僧道:“二哥,你的又像得紧。只是师父的又忒俊了些儿。”三藏
道:“却好,却好!”遂下楼来。下面前殿后廊,还有摆斋的候请。行者却问:“向
日大王庙儿如何了?”众老道:“那庙当年拆了。老爷,这寺自建立之后,年年成
熟,岁岁丰登,却是老爷之福庇。”行者笑道:“此天赐耳,与我们何与!但只我们
自今去后,保你这一庄上人家,子孙繁衍,六畜安生,年年风调雨顺,岁岁雨顺风
调。”众等却叩头拜谢。

只见那前前后后,更有献果献斋的,无限人家。八戒笑道:“我的蹭蹬!那时节
吃得,却没人家连请十请;今日吃不得,却一家不了,又是一家。”饶他气满,略
动手,又吃过八九盘素食;纵然胃伤,又吃了二三十个馒头。已皆尽饱,又有人来
相邀。三藏道:“弟子何能,感蒙至爱!望今夕暂停,明早再领。”

时已深夜。三藏守定真经,不敢暂离,就于楼下打坐看守。将及三更,三藏悄
悄的叫道:“悟空,这里人家,识得我们道成事完了。自古道:‘真人不露相,露相
不真人。’恐为久淹,失了大事。”行者道:“师父说得有理。我们趁此深夜,人皆
熟睡,寂寂的去了罢。”八戒却也知觉,沙僧尽自分明,白马也能会意。遂此起了
身,轻轻的抬上驮垛,挑着担,从庑廊驮出。到于山门,只见门上有锁。行者又使
个解锁法,开了二门、大门,找路望东而去。只听得半空中有八大金刚叫道:“逃
走的,跟我来!”那长老闻得香风荡荡,起在空中。这正是:
丹成识得本来面,体健如如拜主人。

毕竟不知怎生见那唐王,且听下回分解。