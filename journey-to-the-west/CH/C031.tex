\chapter{猪八戒义激猴王~孙行者智降妖怪}

义结孔怀,法归本性。金顺木驯成正果,心猿木母合丹元。共登极乐世界,同
来不二法门。经乃修行之总径,佛配自己之元神。兄和弟会成三契,妖与魔色应五
行。剪除六门趣,即赴大雷音。

却说那呆子被一窝猴子捉住了,扛抬扯拉,把一件直裰子揪破。口里劳劳叨叨
的,自家念诵道:“罢了,罢了,这一去有个打杀的情了!”不一时,到洞口。那大
圣坐在石崖之上,骂道:“你这馕糠的夯货!你去便罢了,怎么骂我?”八戒跪在地
下道:“哥啊,我不曾骂你;若骂你,就嚼了舌头根。我只说哥哥不去,我自去报
师父便了。怎敢骂你?”行者道:“你怎么瞒得过我?我这左耳往上一扯,晓得三十
三天人说话;我这右耳往下一扯,晓得十代阎王与判官算帐。你今走路把我骂,我
岂不听见?”八戒道:“哥啊,我晓得。你贼头鼠脑的,一定又变作个甚么东西儿,
跟着我听的。”行者叫:“小的们,选大棍来!先打二十个见面孤拐,再打二十个背
花,然后等我使铁棒与他送行!”八戒慌得磕头道:“哥哥,千万看师父面上,饶了
我罢!”行者道:“我想那师父好仁义儿哩!”八戒又道:“哥哥,不看师父啊,请看
海上菩萨之面,饶了我罢!”

行者见说起菩萨,却有三分儿转意道:“兄弟,既这等说,我且不打你。你却
老实说,不要瞒我。那唐僧在那里有难,你却来此哄我?”八戒道:“哥哥,没甚
难处,实是想你。”行者骂道:“这个好打的夯货!你怎么还要者嚣?我老孙身回水帘
洞,心逐取经僧。那师父步步有难,处处该灾。你趁早儿告诵我,免打!”八戒闻
得此言,叩头上告道:“哥啊,分明要瞒着你,请你去的;不期你这等样灵。饶我
打,放我起来说罢。”行者道:“也罢,起来说。”众猴撒开手,那呆子跳得起来,
两边乱张。行者道:“你张甚么?”八戒道:“看看那条路儿空阔,好跑。”行者道:
“你跑到那里?我就让你先走三日,老孙自有本事赶转你来!快早说来!这一恼发我
的性子,断不饶你!”

八戒道:“实不瞒哥哥说。自你回后,我与沙僧,保师父前行。只见一座黑松
林,师父下马,教我化斋。我因许远,无一个人家,辛苦了,略在草里睡睡。不想
沙僧别了师父,又来寻我。你晓得师父没有坐性;他独步林间玩景,出得林,见一
座黄金宝塔放光,他只当寺院,不期塔下有个妖精,名唤黄袍,被他拿住。后边我
与沙僧回寻,止见白马、行囊,不见师父,随寻至洞口,与那怪厮杀。师父在洞,
幸亏了一个救星。原是宝象国王第三个公主,被那怪摄来者。他修了一封家书,托
师父寄去,遂说方便,解放了师父。到了国中,递了书子,那国王就请师父降妖,
取回公主。哥啊,你晓得,那老和尚可会降妖?我二人复去与战。不知那怪神通广
大,将沙僧又捉了。我败阵而走,伏在草中。那怪变做个俊俏文人入朝,与国王认
亲,把师父变作老虎。又亏了白龙马夜现龙身,去寻师父。师父倒不曾寻见,却遇
着那怪在银安殿饮酒。他变一宫娥,与他巡酒、舞刀,欲乘机而砍,反被他用满堂
红打伤马腿。就是他教我来请师兄的,说道:‘师兄是个有仁有义的君子。君子不
念旧恶,一定肯来救师父一难。’万望哥哥念‘一日为师,终身为父’之情,千万
救他一救!”

行者道:“你这个呆子!我临别之时,曾叮咛又叮咛,说道:‘若有妖魔捉住师
父,你就说老孙是他大徒弟。’怎么却不说我?”八戒又思量道:“请将不如激将,
等我激他一激。”道:“哥啊,不说你还好哩;只为说你,他一发无状!”行者道:“怎
么说?”八戒道:“我说:‘妖精,你不要无礼,莫害我师父!我还有个大师兄,叫
做孙行者。他神通广大,善能降妖。他来时教你死无葬身之地!’那怪闻言,越加
忿怒,骂道:‘是个甚么孙行者,我可怕他!他若来,我剥了他皮,抽了他筋,啃了
他骨,吃了他心!饶他猴子瘦,我也把他剁着油烹!’”行者闻言,就气得抓耳挠
腮,暴躁乱跳道:“是那个敢这等骂我!”八戒道:“哥哥息怒,是那黄袍怪这等骂
来,我故学与你听也。”行者道:“贤弟,你起来。不是我去不成,既是妖精敢骂我,
我就不能不降他。我和你去。老孙五百年前大闹天宫,普天的神将看见我,一个个
控背躬身,口口称呼大圣。这妖怪无礼,他敢背前面后骂我!我这去,把他拿住,
碎尸万段,以报骂我之仇!报毕,我即回来。”八戒道:“哥哥,正是。你只去拿了
妖精,报了你仇,那时来与不来,任从尊命。”

那猴才跳下崖,撞入洞里,脱了妖衣,整一整锦直裰,束一束虎皮裙,执了铁
棒,径出门来。慌得那群猴拦住道:“大圣爷爷,你往那里去?带挈我们耍子几年也
好。”行者道:“小的们,你说那里话!我保唐僧的这桩事,天上地下,都晓得孙悟
空是唐僧的徒弟。他倒不是赶我回来,倒是教我来家看看,送我来家自在耍子。如
今只因这件事,你们却都要仔细看守家业,依时插柳栽松,毋得废坠。待我还去保
唐僧,取经回东土。功成之后,仍回来与你们共乐天真。”众猴各各领命。

那大圣才和八戒携手驾云,离了洞,过了东洋大海,至西岸,住云光,叫道:
“兄弟,你且在此慢行,等我下海去净净身子。”八戒道:“忙忙的走路,且净甚么
身子?”行者道:你那里知道。我自从回来,这几日弄得身上有些妖精气了。师父
是个爱干净的,恐怕嫌我。”八戒于此始识得行者是片真心,更无他意。

须臾洗毕,复驾云西进。只见那金塔放光。八戒指道:“那不是黄袍怪家?沙僧
还在他家里。”行者道:“你在空中,等我下去看看那门前如何,好与妖精见阵。”
八戒道:“不要去,妖精不在家。”行者道:“我晓得。”

好猴王,按落祥光,径至洞门外观看。只见有两个小孩子,在那里使弯头棍,
打毛球,抢窝耍子哩。一个有十来岁,一个有八九岁了。正戏处,被行者赶上前,
也不管他是张家李家的,一把抓着顶搭子,提将过来。那孩子吃了唬,口里夹骂带
哭的乱嚷,惊动那波月洞的小妖,急报与公主道:“奶奶,不知甚人把二位公子抢
去也!”原来那两个孩子是公主与那怪生的。

公主闻言,忙忙走出洞门来。只见行者提着两个孩子,站在那高崖之上,意欲
往下惯。慌得那公主厉声高叫道:“那汉子,我与你没甚相干,怎么把我儿子拿去?
他老子利害,有些差错,决不与你干休!”行者道:“你不认得我?我是那唐僧的大
徒弟孙悟空行者。我有个师弟沙和尚在你洞里。你去放他出来,我把这两个孩儿还
你。似这般两个换一个,还是你便宜。”

那公主闻言,急往里面,喝退那几个把门的小妖,亲动手,把沙僧解了。沙僧
道:“公主,你莫解我:恐你那怪来家,问你要人,带累你受气。”公主道:“长老
啊,你是我的恩人,你替我折辩了家书,救了我一命,我也留心放你;不期洞门之
外,你有个大师兄孙悟空来了,叫我放你哩。”

噫!那沙僧一闻孙悟空的三个字,好便似醍醐灌顶,甘露滋心。一面天生喜,
满腔都是春。也不似闻得个人来,就如拾着一方金玉一般。你看他手拂衣,走出
门来,对行者施礼道:“哥哥,你真是从天而降也!万乞救我一救!”行者笑道:“你
这个沙尼!师父念紧箍儿咒,可肯替我方便一声?都弄嘴施展!要保师父,如何不走
西方路,却在这里‘蹲’甚么?”沙僧道:“哥哥,不必说了。君子人既往不咎。
我等是个败军之将,不可语勇。救我救儿罢!”行者道:“你上来。”沙僧才纵身跳
上石崖。

却说那八戒停立空中,看见沙僧出洞,即按下云头,叫声“沙兄弟,心忍,心
忍!”沙僧见身道:“二哥,你从那里来?”八戒道:“我昨日败阵,夜间进城,会
了白马,知师父有难,被黄袍使法,变做个老虎。那白马与我商议,请师兄来的。”
行者道:“呆子,且休叙阔,把这两个孩子,你抱着一个,先进那宝象城去激那怪
来,等我在这里打他。”沙僧道:“哥啊,怎么样激他?”行者道:“你两个驾起云,
站在那金銮殿上,莫分好歹,把那孩子往那白玉阶前一掼。有人问你是甚人,你便
说是黄袍妖精的儿子。被我两个拿将来也。那怪听见,管情回来,我却不须进城与
他斗了。若在城上厮杀,必要喷云嗳雾,播土扬尘,惊扰那朝廷与多官黎庶,俱不
安也。”八戒笑道:“哥哥,你但干事,就左我们。”行者道:“如何为左你?”八戒
道:“这两个孩子,被你抓来,已此唬破胆了;这一会声都哭哑,再一会必死无疑;
我们拿他往下一掼,掼做个肉子,那怪赶上肯放?定要我两个偿命。你却还不是
个干净人?连见证也没你,你却不是左我们?”行者道:“他若扯你,你两个就与他
打将这里来。这里有战场宽阔,我在此等候打他。”沙僧道:“正是,正是。大哥说
得有理。我们去来。”他两个才倚仗威风,将孩子拿去。

行者即跳下石崖,到他塔门之下。那公主道:“你这和尚,全无信义:你说放
了你师弟,就与我孩儿,怎么你师弟放去,把我孩儿又留,反来我门首做甚?”行
者陪笑道:“公主休怪。你来的日子已久,带你令郎去认他外公去哩。”公主道:“和
尚莫无礼。我那黄袍郎比众不同。你若唬了我的孩儿,与他柳柳惊是。”

行者笑道:“公主啊,为人生在天地之间,怎么便是得罪?”公主道:“我晓得。”
行者道:“你女流家,晓得甚么?”公主道:“我自幼在宫,曾受父母教训。记得古
书云:‘五刑之属三千,而罪莫大于不孝。’”行者道:“你正是个不孝之人。盖‘父
兮生我,母兮鞠我。哀哀父母,生我劬劳!’故孝者,百行之原,万善之本,却怎
么将身陪伴妖精,更不思念父母?非得不孝之罪,如何?”公主闻此正言,半晌家
耳红面赤,惭愧无地。忽失口道:“长老之言最善。我岂不思念父母?只因这妖精将
我摄骗在此,他的法令又谨,我的步履又难,路远山遥,无人可传音信。欲要自尽,
又恐父母疑我逃走,事终不明。故没奈何,苟延残喘,诚为天地间一大罪人也!”
说罢,泪如泉涌。行者道:“公主不必伤悲。猪八戒曾告诉我,说你有一封书,曾
救了我师父一命,你书上也有思念父母之意。老孙来,管与你拿了妖精,带你回朝
见驾,别寻个佳偶,侍奉双亲到老。你意如何?”公主道:“和尚啊,你莫要寻死。
昨者你两个师弟,那样好汉,也不曾打得过我黄袍郎。你这般一个筋多骨少的瘦鬼,
一似个螃蟹模样,骨头都长在外面,有甚本事,你敢说拿妖魔之话?”行者笑道:
“你原来没眼色,认不得人。俗语云:‘尿泡虽大无斤两,秤铊虽小压千斤。’他们
相貌:空大无用,走路抗风;穿衣费布,种火心空;顶门腰软,吃食无功。咱老孙
小自小,筋节。”那公主道:“你真个有手段么?”行者道:“我的手段,你是也不
曾看见。绝会降妖,极能伏怪。”公主道:“你却莫误了我耶。”行者道:“决然误你
不得。”公主道:“你既会降妖伏怪,如今却怎样拿他?”行者说:“你且回避回避,
莫在我这眼前:倘他来时,不好动手脚,只恐你与他情浓了,舍不得他。”公主道:
“我怎的舍不得他?其稽留于此者,不得已耳!”行者道:“你与他做了十三年夫妻,
岂无情意?我若见了他,不与他儿戏,一棍便是一棍,一拳便是一拳,须要打倒他,
才得你回朝见驾。”

那公主果然依行者之言,往僻静处躲避。也是他姻缘该尽,故遇着大圣来临。
那猴王把公主藏了,他却摇身一变,就变做公主一般模样,回转洞中,专候那怪。

却说八戒、沙僧,把两个孩子,拿到宝象国中,往那白玉阶前下,可怜都掼
做个肉饼相似,鲜血迸流,骨骸粉碎。慌得那满朝多官报道:“不好了!不好了!天
上掼下两个人来了!”八戒厉声高叫道:“那孩子是黄袍妖精的儿子,被老猪与沙弟
拿将来也!”

那怪还在银安殿,宿酒未醒。正睡梦间,听得有人叫他名字,他就翻身,抬头
观看,只见那云端里是猪八戒、沙和尚二人吆喝。妖怪心中暗想道:“猪八戒便也
罢了;沙和尚是我绑在家里,他怎么得出来?我的浑家,怎么肯放他?我的孩儿,怎
么得到他手?这怕是猪八戒不得我出去与他交战,故将此计来我。我若认了这个
泛头,就与他打啊,噫!我却还害酒哩!假若被他筑上一钯,却不灭了这个威风,识
破了那个关窍。且等我回家看看,是我的儿子不是我的儿子,再与他说话不迟。”

好妖怪,他也不辞王驾,转山林,径去洞中查信息。此时朝中已知他是个妖怪
了。原来他夜里吃了一个宫娥,还有十七个脱命去的,五更时,奏了国王,说他如
此如此。又因他不辞而去,越发知他是怪。那国王即着多官看守着假老虎不题。

却说那怪径回洞口。行者见他来时,设法哄他,把眼挤了一挤,扑簌簌泪如雨
落,儿天儿地的,跌脚捶胸,于此洞里嚎啕痛哭。那怪一时间,那里认得。上前搂
住道:“浑家,你有何事,这般烦恼?”那大圣编成的鬼话,捏出的虚词,泪汪汪
的告道:“郎君啊!常言道:‘男子无妻财没主,妇女无夫身落空!’你昨日进朝认亲,
怎不回来?今早被猪八戒劫了沙和尚,又把我两个孩儿抢去,是我苦告,更不肯饶。
他说拿去朝中认认外公。这半日不见孩儿,又不知存亡如何,你又不见来家,教我
怎生割舍?故此止不住伤心痛哭。”那怪闻言,心中大怒道:“真个是我的儿子?”
行者道:“正是,被猪八戒抢去了。”

那妖魔气得乱跳道:“罢了,罢了!我儿被他掼杀了,已是不可活也!只好拿那
和尚来与我儿子偿命报仇罢!浑家,你且莫哭。你如今心里觉道怎么?且医治一医
治。”行者道:“我不怎的,只是舍不得孩儿,哭得我有些心疼。”妖魔道:“不打紧,
你请起来,我这里有件宝贝,只在你那疼上摸一摸儿,就不疼了。却要仔细,休使
大指儿弹着;若使大指儿弹着啊,就看出我本相来了。”行者闻言,心中暗笑道:“这
泼怪,倒也老实;不动刑法,就自家供了。等他拿出宝贝来,我试弹他一弹,看他
是个甚么妖怪。”

那怪携着行者,一直行到洞里深远密闭之处。却从口中吐出一件宝贝,有鸡子
大小,是一颗舍利子玲珑内丹。行者心中暗喜道“好东西耶!这件物不知打了多少
坐工,炼了几年磨难,配了几转雌雄,炼成这颗内丹舍利。今日大有缘法,遇着老
孙。”那猴子拿将过来,那里有甚么疼处,特故意摸了一摸,一指头弹将去。那妖
慌了,劈手来抢。你思量,那猴子好不溜撒,把那宝贝一口吸在肚里。那妖魔着
拳头就打,被行者一手隔住,把脸抹了一抹,现出本相,道声“妖怪,不要无礼!
你且认认看,我是谁?”

那妖怪见了,大惊道:“呀!浑家,你怎么拿出这一副嘴脸来耶?”行者骂道:
“我把你这个泼怪!谁是你浑家?连你祖宗也还不认得哩!”那怪忽然省悟道:“我像
有些认得你哩。”行者道:“我且不打你,你再认认看。”那怪道:“我虽见你眼熟,
一时间却想不起姓名。你果是谁?从那里来的?你把我浑家估倒在何处,却来我家诈
诱我的宝贝?着实无礼,可恶!”行者道:“你是也不认得我。我是唐僧的大徒弟,
叫做孙悟空行者。我是你五百年前的旧祖宗哩!”那怪道:“没有这话,没有这话!
我拿住唐僧时,止知他有两个徒弟,叫做猪八戒、沙和尚,何曾见有人说个姓孙的。
你不知是那里来的个怪物,到此骗我!”行者道:“我不曾同他二人来,是我师父因
老孙惯打妖怪,杀伤甚多,他是个慈悲好善之人,将我逐回,故不曾同他一路行走。
你是不知你祖宗名姓。”那怪道:“你好不丈夫啊!既受了师父赶逐,却有甚么嘴脸,
又来见人!”行者道:“你这个泼怪,岂知‘一日为师,终身为父’,‘父子无隔宿之
仇’!你伤害我师父,我怎么不来救他?你害他便也罢;却又背前面后骂我,是怎的
说?”妖怪道:“我何尝骂你?”行者道:“是猪八戒说的。”那怪道:“你不要信他。
那个猪八戒,尖着嘴,有些会说老婆舌头,你怎听他?”行者道:“且不必讲此闲
话。只说老孙今日到你家里,你好怠慢了远客。虽无酒馔款待,头却是有的。快快
将头伸过来,等老孙打一棍儿,当茶!”那怪闻得说打,呵呵大笑道:“孙行者,你
差了计较了!你既说要打,不该跟我进来。我这里大小群妖,还有百十。饶你满身
是手,也打不出我的门去。”行者道:“不要胡说!莫说百十个,就有几千几万,只
要一个个查明白了好打,棍棍无空,教你断根绝迹!”

那怪闻言,急传号令,把那山前山后群妖,洞里洞外诸怪,一齐点起,各执器
械,把那三四层门,密密拦阻不放。行者见了,满心欢喜,双手理棍,喝声叫“变!”
变的三头六臂;把金箍捧幌一幌,变做三根金箍捧。你看他六只手,使着三根棒,
一路打将去,好便似虎入羊群,鹰来鸡栅;可怜那小怪,汤着的,头如粉碎;刮着
的,血似水流。往来纵横,如入无人之境。止剩一个老妖,赶出门来骂道:“你这
泼猴,其实惫懒!怎么上门子欺负入家!”行者急回头,用手招呼道:“你来,你来,
打倒你,才是功绩!”

那怪物举宝刀,分头便砍;好行者,掣铁棒,觌面相迎。这一场,在那山顶上,
半云半雾的杀哩:

大圣神通大,妖魔本事高。这个横理生金棒,那个斜举蘸钢刀。悠悠刀起明霞
亮,轻轻棒架彩云飘。往来护顶翻多次,反复浑身转数遭。一个随风更面目,一个
立地把身摇。那个大睁火眼伸猿膊,这个明幌金睛折虎腰。你来我去交锋战,刀迎
捧架不相饶。猴王铁棍依三略,怪物钢刀按六韬。一个惯行手段为魔主,一个广施
法力保唐僧。猛烈的猴王添猛烈,英豪的怪物长英豪。死生不顾空中打,都为唐僧
拜佛遥。

他两个战有五六十合,不分胜负。行者心中暗喜道:“这个泼怪,他那口刀,
倒也得住老孙的这根棒。等老孙丢个破绽与他,看他可认得。”好猴王,双手举棍,
使一个“高探马”的势子。那怪不识是计,见有空儿,舞着宝刀,径奔下三路砍;
被行者急转个“大中平”,挑开他那口刀,又使个“叶底偷桃势”,望妖精头顶一棍,
就打得他无影无踪。急收棍子看处,不见了妖精。行者大惊道:“我儿啊,不禁打,
就打得不见了。果是打死,好道也有些脓血,如何没一毫踪影?想是走了。”急纵身
跳在云端里看处,四边更无动静。“老孙这双眼睛,不管那里,一抹都见,却怎么
走得这等溜撒?我晓得了:那怪说有些儿认得我,想必不是凡间的怪,多是天上来
的精。”

那大圣一时忍不住怒发,着铁棒,打个筋斗,只跳到南天门上。慌得那庞、
刘、苟、毕、张、陶、邓、辛等众,两边躬身控背,不敢拦阻,让他打入天门,直
至通明殿下。早有张、葛、许、邱四大天师问道:“大圣何来?”行者道:“因保唐
僧至宝象国,有一妖魔,欺骗国女,伤害吾师,老孙与他赌斗。正斗间,不见了这
怪。想那怪不是凡间之怪,多是天上之精,特来查勘,那一路走了甚么妖神。”天
师闻言,即进灵霄殿上启奏,蒙差查勘九曜星官、十二元辰、东西南北中央五斗,
河汉群辰、五岳四渎、普天神圣都在天上,更无一个敢离方位。又查那斗牛宫外,
二十八宿,颠倒只有二十七位,内独少了奎星。

天师回奏道:“奎木狼下界了。”玉帝道:“多少时不在天了?”天师道:“四卯
不到。三日点卯一次,今已十三日了。”玉帝道:“天上十三日,下界已是十三年。”
即命本部收他上界。

那二十七宿星员,领了旨意,出了天门,各念咒语,惊动奎星。你道他在那里
躲避?他原来是孙大圣大闹天宫时打怕了的神将,闪在那山涧里潜灾,被水气隐住
妖云,所以不曾看见他。他听得本部星员念咒,方敢出头,随众上界。被大圣拦住
天门要打,幸亏众星劝住,押见玉帝。那怪腰间取出金牌,在殿下叩头纳罪。玉帝
道:“奎木狼,上界有无边的胜景,你不受用,却私走一方,何也?”奎宿叩头奏
道:“万岁,赦臣死罪。那宝象国王公主,非凡人也。他本是披香殿侍香的玉女,
因欲与臣私通,臣恐点污了天宫胜境,他思凡先下界去,托生于皇宫内院,是臣不
负前期,变作妖魔,占了名山,摄他到洞府,与他配了一十三年夫妻。‘一饮一啄,
莫非前定。’今被孙大圣到此成功。”玉帝闻言,收了金牌,贬他去兜率宫与太上老
君烧火,带俸差操,有功复职,无功重加其罪。

行者见玉帝如此发放,心中欢喜。朝上唱个大喏,又向众神道:“列位,起动
了。”天师笑道:“那个猴子还是这等村俗。替他收了怪神,也倒不谢天恩,却就喏
喏而退。”玉帝道:“只得他无事,落得天上清平是幸。”

那大圣按落祥光,径转碗子山波月洞,寻出公主。将那思凡下界收妖的言语正
然陈诉,只听得半空中八戒、沙僧厉声高叫道:“师兄,有妖精,留几个儿我们打
耶。”行者道:“妖精已尽绝矣。”沙僧道:“既把妖精打绝,无甚挂碍,将公主引入
朝中去罢。不要睁眼。兄弟们,使个缩地法来。”

那公主只闻得耳内风响,霎时间径回城里。他三人将公主带上金銮殿上。那公
主参拜了父王、母后,会了姊妹,各官俱来拜见。那公主才启奏道:“多亏孙长老
法力无边,降了黄袍怪,救奴回国。”那国王问曰:“黄袍是个甚怪?”行者道:“陛
下的驸马,是上界的奎星;令爱乃侍香的玉女,因思凡降落人间,不非小可,都因
前世前缘,该有这些姻眷。那怪被老孙上天宫启奏玉帝,玉帝查得他四卯不到,下
界十三日,就是十三年了,盖天上一日,下界一年。随差本部星宿,收他上界,贬
在兜率宫立功去讫;老孙却救得令爱来也。”那国王谢了行者的恩德,便教:“看你
师父去来。”

他三人径下宝殿,与众官到朝房里,抬出铁笼,将假虎解了铁索。别人看他是
虎,独行者看他是人。原来那师父被妖术魇住,不能行走,心上明白,只是口眼难
开。行者笑道:“师父啊,你是个好和尚,怎么弄出这般个恶模样来也?你怪我行凶
作恶,赶我回去,你要一心向善,怎么一旦弄出个这等嘴脸?”八戒道:“哥啊,
救他救儿罢。不要只管揭挑他了。”行者道:“你凡事撺唆,是他个得意的好徒弟,
你不救他,又寻老孙怎的?原与你说来,待降了妖精,报了骂我之仇,就回去的。”
沙僧近前跪下道:“哥啊,古人云:‘不看僧面看佛面。’兄长既是到此,万望救他
一救。若是我们能救,也不敢许远的来奉请你也。”行者用手挽起道:“我岂有安心
不救之理?快取水来。”那八戒飞星去驿中,取了行李、马匹,将紫金钵盂取出,盛
水半盂,递与行者。行者接水在手,念动真言,望那虎劈头一口喷上,退了妖术,
解了虎气。

长老现了原身,定性睁睛,才认得是行者。一把搀住道:“悟空!你从那里来
也?”沙僧侍立左右,把那请行者,降妖精,救公主,解虎气,并回朝上项事,备
陈了一遍。三藏谢之不尽,道:“贤徒,亏了你也!亏了你也!这一去,早诣西方,
径回东土,奏唐王,你的功劳第一。”行者笑道:“莫说!莫说!但不念那话儿,足感
爱厚之情也。”国王闻此言,又劝谢了他四众。整治素筵,大开东阁。他师徒受了
皇恩,辞王西去。国王又率多官远送。这正是:

君回宝殿定江山,僧去雷音参佛祖。

毕竟不知此后又有甚事,几时得到西天,且听下回分解。