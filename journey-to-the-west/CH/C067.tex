\chapter{拯救驼罗禅性稳~脱离秽污道心清}

话说三藏四众,躲离了小西天,欣然上路。行经个月程途,正是春深花放之时,
见了几处园林皆绿暗,一番风雨又黄昏。三藏勒马道:“徒弟啊,天色晚矣,往那
条路上求宿去?”行者笑道:“师父放心。若是没有借宿处,我三人都有些本事,
叫八戒砍草,沙和尚扳松,老孙会做木匠,就在这路上搭个蓬庵,好道也住得年把。
你忙怎的!”八戒道:“哥呀,这个所在,岂是住场!满山多虎豹狼虫,遍地有魑魅
魍魉。白日里尚且难行,黑夜里怎生敢宿?”行者道:“呆子!越发不长进了!不是
老孙海口,只这条棒子,在手里,就是塌下天来,也撑得住!”

师徒们正然讲论,忽见一座山庄不远。行者道:“好了,有宿处了!”长老问:
“在何处?”行者指道:“那树丛里不是个人家?我们去借宿一宵,明早走路。”

长老欣然促马,至庄门外下马。只见那柴扉紧闭。长老敲门道:“开门,开门。”
里面有一老者,手拖藜杖,足踏蒲鞋,头顶乌巾,身穿素服,开了门,便问:“是
甚人在此大呼小叫?”三藏合掌当胸,躬身施礼道:“老施主,贫僧乃东土差往西
天取经者。适到贵地天晚,特造尊府假宿一宵。万望方便方便。”老者道:“和尚,
你要西行,却是去不得啊。此处乃小西天。若到大西天,路途甚远。且休道前去艰
难,只这个地方,已此难过。”三藏问:“怎么难过?”老者用手指道:“我这庄村
西去三十余里,有一条稀柿同,山名七绝。”三藏道:“何为‘七绝’?”老者道:
“这山径过有八百里,满山尽是柿果。古云:‘柿树有七绝:一,益寿;二,多阴;
三,无鸟巢;四,无虫;五,霜叶可玩;六,嘉实;七,枝叶肥大。’故名七绝山。
我这敝处地阔人稀,那深山亘古无人走到。每年家熟烂柿子落在路上,将一条夹石
胡同,尽皆填满;又被雨露雪霜,经霉过夏,作成一路污秽。这方人家,俗呼为稀
屎同。但刮西风,有一股秽气,就是淘东圊也不似这般恶臭。如今正值春深,东南
风大作,所以还不闻见也。”三藏心中烦闷不言。

行者忍不住,高叫道:“你这老儿甚不通便!我等远来投宿,你就说出这许多话
来唬人!十分你家窄逼没处睡,我等在此树下蹲一蹲,也就过了此宵,何故这般絮
聒?”那老者见了他相貌丑陋,便也拧住口,惊嘬嘬的,硬着胆,喝了一声,用藜
杖指定道:“你这厮,骨挝脸,磕额头,塌鼻子,凹颉腮,毛眼毛睛,痨病鬼,不
知高低,尖着个嘴,敢来冲撞我老人家!”行者陪笑道:“老官儿,你原来有眼无珠,
不识我这痨病鬼哩!相法云:‘形容古怪,石中有美玉之藏。’你若以言貌取人,干
净差了。我虽丑便丑,却倒有些手段。”老者道:“你是那方人氏?姓甚名谁?有何手
段?”行者笑道:“我

祖居东胜大神洲,花果山前自幼修。身拜灵台方寸祖,学成武艺甚全周:也能
搅海降龙母,善会担山赶日头;缚怪擒魔称第一,移星换斗鬼神愁。偷天转地英名
大,我是变化无穷美石猴!”
老者闻言,回嗔作喜。躬着身,便教:“请!请入寒舍安置。”

遂此,四众牵马挑担,一齐进去。只见那荆针棘刺,铺设两边;二层门是砖石
垒的墙壁,又是荆棘苫盖;入里才是三间瓦房。老者便扯椅安坐待茶,又叫办饭。
少顷,移过桌子,摆着许多面筋豆腐,芋苗萝白,辣芥蔓菁,香稻米饭,醋烧葵汤,
师徒们尽饱一餐。吃毕,八戒扯过行者,背云:“师兄,这老儿始初不肯留宿,今
返设此盛斋,何也?”行者道:“这个能值多少钱!到明日,还要他十果十菜的送我
们哩!”八戒道:“不羞!凭你那几句大话,哄他一顿饭吃了,明日却要跑路,他又
管待送你怎的?”行者道:“不要忙,我自有个处治。”

不多时,渐渐黄昏,老者又叫掌灯。行者躬身问道:“公公高姓?”老者道:“姓
李。”行者道:“贵地想就是李家庄?”老者道:“不是,这里唤做驼罗庄,共有五
百多人家居住。别姓俱多,惟我姓李。”行者道:“李施主,府上有何善意,赐我等
盛斋?”那老者起身道:“才闻得你说会拿妖怪,我这里却有个妖怪,累你替我们
拿拿,自有重谢。”行者就朝上唱个喏道:“承照顾了!”八戒道:“你看他惹祸!听
见说拿妖怪,就是他外公也不这般亲热,预先就唱个喏!”行者道:“贤弟,你不知。
我唱个喏就是下了个定钱,他再不去请别人了。”

三藏闻言道:“这猴儿凡事便要自专。倘或那妖精神通广大,你拿他不住,可
不是我出家人打诳语么?”行者笑道:“师父莫怪,等我再问了看。”那老者道:“还
问甚?”行者道:“你这贵处,地势清平,又许多人家居住,更不是偏僻之方,有
甚么妖精,敢上你这高门大户?”老者道:“实不瞒你说。我这里久矣康宁。只这
三年六月间,忽然一阵风起,那时人家甚忙,打麦的在场上,插秧的在田里,俱着
了慌,只说是天变了。谁知风过处,有个妖精,将人家牧放的牛马吃了,猪羊吃了,
见鸡鹅囫囵咽,遇男女夹活吞。自从那次,这二年常来伤害。长老啊,你若有手段,
拿了他,扫净此土,我等决然重谢,不敢轻慢。”行者道:“这个却是难拿。”八戒
道:“真是难拿,难拿!我们乃行脚僧,借宿一宵,明日走路,拿甚么妖精!”老者
道:“你原来是骗饭吃的和尚。初见时夸口弄舌,说会换斗移星,降妖缚怪,及说
起此事,就推却难拿!”

行者道:“老儿,妖精好拿;只是你这方人家不齐心,所以难拿。”老者道:“怎
见得人心不齐?”行者道:“妖精搅扰了三年,也不知伤害了多少生灵。我想着每
家只出银一两,五百家可凑五百两银子,不拘到那里,也寻一个法官把妖拿了,却
怎么就甘受他三年磨折?”老者道:“若论说使钱,好道也羞杀人!我们那家不花费
三五两银子!前年曾访着山南里有个和尚,请他到此拿妖,未曾得胜。”行者道:“那
和尚怎的拿来?”老者道:

“那个僧伽,披领袈裟。先谈《孔雀》,后念《法华》。香焚炉内,手把铃拿。
正然念处,惊动妖邪。风生云起,径至庄家。僧和怪斗,其实堪夸:一递一拳捣,
一递一把抓。和尚还相应,相应没头发。须臾妖怪胜,径直返烟霞。原来晒干疤。
我等近前看,光头打的似个烂西瓜!”
行者笑道:“这等说,吃了亏也。”老者道:“他只拚得一命,还是我们吃亏:与他
买棺木殡葬,又把些银子与他徒弟。那徒弟心还不歇,至今还要告状,不得干净!”

行者道:“再可曾请甚么人拿他?”老者道:“旧年又请了一个道士。”行者道:
“那道士怎么拿他?”老者道:“那道士:

头戴金冠,身穿法衣。令牌敲响,符水施为。驱神使将,拘到妖魑。狂风滚滚,
黑雾迷迷。即与道士,两个相持。斗到天晚,怪返云霓。乾坤清朗朗,我等众人齐。
出来寻道士,死在山溪。捞得上来大家看,却如一个落汤鸡!”
行者笑道:“这等说,也吃亏了。”老者道:“他也只舍得一命,我们又使够闷数钱
粮。”行者道:“不打紧,不打紧,等我替你拿他来。”老者道:“你若果有手段拿得
他,我请几个本庄长者与你写个文书:若得胜,凭你要多少银子相谢,半分不少;
如若有亏,切莫和我等放赖,各听天命。”行者笑道:“这老儿被人赖怕了。我等不
是那样人。快请长者去。”

那老者满心欢喜,即命家僮,请几个左邻、右舍、表弟、姨兄、亲家、朋友,
共有八九位老者,都来相见。会了唐僧,言及拿妖一事,无不欣然。众老问:“是
那一位高徒去拿?”行者叉手道:“是我小和尚。”众老悚然道:“不济,不济!那妖
精神通广大,身体狼。你这个长老,瘦瘦小小,还不够他填牙齿缝哩!”行者笑
道:“老官儿,你估不出人来。我小自小,结实,都是‘吃了磨刀水的,秀气在内’
哩!”众老见说,只得依从道:“长老,拿住妖精,你要多少谢礼?”行者道:“何
必说要甚么谢礼!俗语云:‘说金子幌眼,说银子傻白,说铜钱腥气!’我等乃积德
的和尚,决不要钱。”众老道:“既如此说,都是受戒的高僧。既不要钱,岂有空劳
之理!我等各家俱以鱼田为活。若果降了妖孽,净了地方,我等每家送你两亩良田,
共凑一千亩,坐落一处,你师徒们在上起盖寺院,打坐参禅,强似方上云游。”行
者又笑道:“越不停当!但说要了田,就要养马当差,纳粮办草,黄昏不得睡,五鼓
不得眠。好倒弄杀人也!”众老道:“诸般不要,却将何谢?”行者道:“我出家人,
但只是一茶一饭,便是谢了。”众老喜道:“这个容易。但不知你怎么拿他。”行者
道:“他但来,我就拿住他。”众老道:“那怪大着哩!上拄天,下拄地;来时风,去
时雾。你却怎生近得他?”行者笑道:“若论呼风驾雾的妖精,我把他当孙子罢了;
若说身体长大,有那手段打他!”

正讲处,只听得呼呼风响,慌得那八九个老者,战战兢兢道:“这和尚盐酱口!
说妖精,妖精就来了!”那老李开了腰门,把几个亲戚,连唐僧,都叫:“进来,进
来!妖怪来了!”唬得那八戒也要进去,沙僧也要进去。行者两只手扯住两个道:“你
们忒不循理!出家人,怎么不分内外!站住,不要走!跟我去天井里,看看是个甚么
妖精!”八戒道:“哥啊,他们都是经过帐的,风响便是妖来。他都去躲,我们又不
与他有亲,又不相识,又不是交契故人,看他做甚?”原来行者力量大,不容说,
一把拉在天井里站下。那阵风越发大了。好风:
倒树摧林狼虎忧,播江搅海鬼神愁。
掀翻华岳三峰石,提起乾坤四部洲。
村舍人家皆闭户,满庄儿女尽藏头。
黑云漠漠遮星汉,灯火无光遍地幽。
慌得那八戒战战兢兢,伏之于地,把嘴拱开土,埋在地下,却如钉了钉一般。沙僧
蒙着头脸,眼也难睁。

行者闻风认怪,一霎时,风头过处,只见那半空中隐隐的两盏灯来,即低头叫
道:“兄弟们!风过了,起来看!”那呆子扯出嘴来,抖抖灰土,仰着脸,朝天一望,
见有两盏灯光,忽失声笑道:“好耍子,好耍子!原来是个有行止的妖精!该和他做
朋友!”沙僧道:“这般黑夜,又不曾觌面相逢,怎么就知好歹?”八戒道:“古人
云:‘夜行以烛,无烛则止。’你看他打一对灯笼引路,必定是个好的。”沙僧道:“你
错看了。那不是一对灯笼,是妖精的两只眼亮。”这呆子就唬矮了三寸,道:“爷爷
呀!眼有这般大啊,不知口有多少大哩!”行者道:“贤弟莫怕。你两个护持着师父,
待老孙上去讨他个口气,看他是甚妖精。”八戒道:“哥哥,不要供出我们来。”

好行者,纵身打个唿哨,跳到空中。执铁棒,厉声高叫道:“慢来,慢来!有吾
在此!”那怪见了,挺住身躯,将一根长枪乱舞。行者执了棍势,问道:“你是那方
妖怪?何处精灵?”那怪更不答应,只是舞枪。行者又问,又不答,只是舞枪。行
者暗笑道:“好是耳聋口哑!不要走,看棍!”那怪更不怕,乱舞枪遮拦。在那半空
中,一来一往,一上一下,斗到三更时分,未见胜败。八戒、沙僧,在李家天井里,
看得明白。原来那怪只是舞枪遮架,更无半分儿攻杀。行者一条棒不离那怪的头上。
八戒笑道:“沙僧,你在这里护持,让老猪去帮打帮打,莫教那猴子独干这功,领
头一钟酒。”

好呆子,就跳起云头,赶上就筑。那怪物又使一条枪抵住。两条枪,就如飞蛇
掣电。八戒夸奖道:“这妖精好枪法!不是‘山后枪’,乃是‘缠丝枪’;也不是‘马
家枪’,却叫做个‘软柄枪’!”行者道:“呆子莫胡谈!那里有个甚么‘软柄枪’!”
八戒道:“你看他使出枪尖来架住我们,不见枪柄,不知收在何处。”行者道:“或
者是个‘软柄枪’;但这怪物还不会说话,想是还未归人道,阴气还重。只怕天明
时阳气胜,他必要走。但走时,一定赶上,不可放他。”八戒道:“正是,正是!”

又斗多时,不觉东方发白。那怪不敢恋战,回头就走。行者与八戒,一齐赶来,
忽闻得污秽之气旭人,乃是七绝山稀柿同也。八戒道:“是那家淘毛厕哩!哏!臭气
难闻!”行者侮着鼻子,只叫:“快快赶妖精!快快赶妖精!”那怪物撺过山去,现了
本象,乃是一条红鳞大蟒。你看他:

眼射晓星,鼻喷朝雾。密密牙排钢剑,弯弯爪曲金钩。头戴一条肉角,好便似
千千块玛瑙攒成;身披一派红鳞,却就如万万片胭脂砌就。盘地只疑为锦被,飞空
错认作虹霓。歇卧处有腥气冲天,行动时有赤云罩体。大不大,两边人不见东西;
长不长,一座山跨占南北。
八戒道:“原来是这般一个长蛇!若要吃人啊,一顿也得五百个,还不饱足!”行者
道:“那软柄枪乃是两条信。我们赶他软了,从后打出去!”

这八戒纵身赶上,将钯便筑。那怪物一头钻进窟里,还有七八尺长尾巴丢在外
边。八戒放下钯,一把挝住道:“着手!着手!”尽力气往外乱扯,莫想扯得动一毫。
行者笑道:“呆子!放他进去,自有处置,不要这等倒扯蛇。”八戒真个撒了手,那
怪缩进去了。八戒怨道:“才不放手时,半截子已是我们的了!是这般缩了,却怎么
得他出来?这不是叫做没蛇弄了?”行者道:“这厮身体狼,窟穴窄小,断然转身
不得,一定是个照直撺的,定有个后门出头。你快去后门外拦住,等我在前门外打。”

那呆子真个一溜烟,跑过山去。果见有个孔窟,他就扎定脚。还不曾站稳,不
期行者在前门外使棍子往里一捣,那怪物护疼,径往后门撺出。八戒未曾防备,被
他一尾巴打了一跌,莫能挣挫得起,睡在地下忍疼。行者见窟中无物,搴着棍,穿
进去叫赶妖怪。那八戒听得吆喝,自己害羞,忍着疼,爬起来,使钯乱扑。行者见
了,笑道:“妖怪走了,你还扑甚的了?”八戒道:“老猪在此‘打草惊蛇’哩!”
行者道:“活呆子!快赶上!”

二人赶过涧去,见那怪盘做一团,竖起头来,张开巨口,要吞八戒。八戒慌得
往后便退。这行者反迎上前,被他一口吞之。八戒捶胸跌脚,大叫道:“哥耶!倾了
你也!”行者在妖精肚里,支着铁棒道:“八戒莫愁,我叫他搭个桥儿你看!”那怪
物躬起腰来,就似一道路东虹。八戒道:“虽是像桥,只是没人敢走。”行者道:“我
再叫他变做个船儿你看!”在肚里将铁棒撑着肚皮。那怪物肚皮贴地,翘起头来,
就似一只赣保船。八戒道:“虽是像船,只是没有桅篷,不好使风。”行者道:“你
让开路,等我叫他使个风你看。”又在里面尽着力把铁棒从脊背上一搠将出去,约
有五七丈长,就似一根桅杆。那厮忍疼挣命,往前一撺,比使风更快,撺回旧路,
下了山,有二十余里,却才倒在尘埃,动荡不得,呜呼丧矣。八戒随后赶上来,又
举钯乱筑。行者把那物穿了一个大洞,钻将出来道:“呆子!他死也死了,你还筑他
怎的?”八戒道:“哥啊,你不知我老猪一生好打死蛇?”遂此收了兵器,抓着尾
巴,倒拉将来。

却说那驼罗庄上李老儿与众等,对唐僧道:“你那两个徒弟,一夜不回,断然
倾了命也。”三藏道:“决不妨事。我们出去看看。”须臾间,只见行者与八戒拖着
一条大蟒,吆吆喝喝前来,众人却才欢喜。满庄上老幼男女,都来跪拜道:“爷爷!
正是这个妖精,在此伤人!今幸老爷施法,斩怪除邪,我辈庶各得安生也!”众家都
是感激,东请西邀,各各酬谢。师徒们被留住五七日,苦辞无奈,方肯放行。又各
家见他不要钱物,都办些干粮果品,骑骡压马,花红彩旗,尽来饯行。此处五百人
家,到有七八百人相送。

一路上喜喜欢欢,不时到了七绝山稀柿同口。三藏闻得那般恶秽,又见路道填
塞,道:“悟空,似此怎生度得?”行者侮着鼻子道:“这个却难也。”三藏见行者
说难,便就眼中垂泪。李老儿与众上前道:“老爷勿得心焦。我等送到此处,都已
约定意思了。令高徒与我们降了妖精,除了一庄祸害,我们各办虔心,另开一条好
路,送老爷过去。”行者笑道:“你这老儿,俱言之欠当。你初然说这山径过有八百
里,你等又不是大禹的神兵,那里会开山凿路!若要我师父过去,还得我们着力,
你们都成不得。”三藏下马,道:“悟空,怎生着力么!”行者笑道:“眼下就要过山,
却也是难;若说再开条路,却又难也。须是还从旧胡同过去。只恐无人管饭。”李
老儿道:“长老说那里话!凭你四位担搁多少时,我等俱养得起,怎么说无人管饭!”
行者道:“既如此,你们去办得两石米的干饭,再做些蒸饼馍馍来。等我那长嘴和
尚吃饱了,变了大猪,拱开旧路,我师父骑在马上,我等扶持着,管情过去了。”

八戒闻言,道:“哥哥,你们都要图个干净,怎么独教老猪出臭?”三藏道:“悟
能,你果有本事拱开胡同,领我过山,注你这场头功。”八戒笑道:“师父在上,列
位施主们都在此,休笑话。我老猪本来有三十六般变化。若说变轻巧华丽飞腾之物,
委实不能;若说变山,变树,变石块,变土墩,变赖象、科猪、水牛、骆驼,真个
全会。只是身体变得大,肚肠越发大。须是吃得饱了,才好干事。”众人道:“有东
西,有东西!我们都带得有干粮、果品、烧饼、在此。原要开山相送的。且都
拿出来,凭你受用。待变化了,行动之时,我们再着人回去做饭送来。”八戒满心
欢喜,脱了皂直裰,丢了九齿钯,对众道:“休笑话,看老猪干这场臭功。”

好呆子,捻着诀,摇身一变,果然变做一个大猪。真个是:

嘴长毛短半脂膘,自幼山中食药苗。黑面环睛如日月,圆头大耳似芭蕉。修成
坚骨同天寿,炼就粗皮比铁牢。鼻
音呱诂叫,喳喳喉响喷喁哮。白蹄四只高千尺,剑鬣长身百丈饶。从见人间肥豕彘,
未观今日老猪魈。唐僧等众齐称赞,羡美天蓬法力高。
孙行者见八戒变得如此,即命那些相送人等,快将干粮等物推攒一处,叫八戒受用。
那呆子不分生熟,一涝食之,却上前拱路。

行者叫沙僧脱了脚,好生挑担,请师父稳坐雕鞍。他也脱了鞋,吩咐众人回
去:“若有情,快早送些饭来与我师弟接力。”那些人有七八百相送随行,多一半有
骡马的,飞星回庄做饭;还有三百人步行的,立于山下遥望他行。原来此庄至山,
有三十余里;待回取饭来,又三十余里;往回担搁,约有百里之遥,他师徒们已此
去得远了。众人不舍,催趱骡马,进胡同,连夜赶至,次日方才赶上。叫道:“取
经的老爷,慢行,慢行!我等送饭来也!”长老闻言,谢之不尽,道:“真是善信之
人!”叫八戒住了,再吃些饭食壮神。那呆子拱了两日,正在饥饿之际。那许多人
何止有七八石饭食。他也不论米饭、面饭、收积来一涝用之。饱餐一顿,却又上前
拱路。三藏与行者、沙僧谢了众人,分手两别。正是:
驼罗庄客回家去,八戒开山过同来。
三藏心诚神力拥,悟空法显怪魔衰。
千年稀柿今朝净,七绝胡同此日开。
六欲尘情皆剪绝,平安无阻拜莲台。

这一去不知还有多少路程,还遇甚么妖怪,且听下回分解。