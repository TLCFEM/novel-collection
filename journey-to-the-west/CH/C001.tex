\chapter{灵根育孕源流出~心性修持大道生}

诗曰:
混沌未分天地乱,茫茫渺渺无人见。
自从盘古破鸿蒙,开辟从兹清浊辨。
覆载群生仰至仁,发明万物皆成善。
欲知造化会元功,须看《西游释厄传》。

盖闻天地之数,有十二万九千六百岁为一元。将一元分为十二会,乃子、丑、
寅、卯、辰、巳、午、未、申、酉、戌、亥之十二支也。每会该一万八百岁。且就
一日而论:子时得阳气而丑则鸡鸣,寅不通光而卯则日出,辰时食后而巳则挨排,
日午天中而未则西蹉,申时晡而日落酉,戌黄昏而人定亥。譬于大数,若到戌会之
终,则天地昏而万物否矣。再去五千四百岁,交亥会之初,则当黑暗,而两间人
物俱无矣,故曰混沌。又五千四百岁,亥会将终,贞下起元,近子之会,而复逐渐
开明。邵康节曰“冬至子之半,天心无改移。一阳初动处,万物未生时”,到此,天
始有根;再五千四百岁,正当子会,轻清上腾,有日,有月,有星,有辰。日、月、
星、辰,谓之四象,故曰“天开于子”。又经五千四百岁,子会将终,近丑之会,而
逐渐坚实。《易》曰:大哉乾元,至哉坤元!万物资生,乃顺承天。至此,地始凝结。
再五千四百岁,正当丑会,重浊下凝,有水,有火,有山,有石,有土。水、火、
山、石、土,谓之五形,故曰“地辟于丑”。又经五千四百岁,丑会终而寅会之初,
发生万物,历曰“天气下降,地气上升;天地交合,群物皆生”。至此,天清地爽,
阴阳交合。再五千四百岁,正当寅会,生人,生兽,生禽,正谓天地人,三才定位。
故曰“人生于寅”。

感盘古开辟,三皇治世,五帝定伦,世界之间,遂分为四大部洲:曰东胜神洲、
曰西牛贺洲、曰南赡部洲、曰北俱芦洲。这部书单表东胜神洲。海外有一国土,名
曰傲来国。国近大海,海中有一座名山,唤为花果山。此山乃十洲之祖脉,三岛之
来龙,自开清浊而立,鸿蒙判后而成。真个好山!有词赋为证。赋曰:

势镇汪洋,威宁瑶海:势镇汪洋,潮涌银山鱼入穴;威宁瑶海,波翻雪浪蜃离
渊。水火方隅高积土,东海之处耸崇巅。丹崖怪石,削壁奇峰。丹崖上,彩凤双鸣;
削壁前,麒麟独卧。峰头时听锦鸡鸣,石窟每观龙出入。林中有寿鹿仙狐,树上有
灵禽玄鹤。瑶草奇花不谢,青松翠柏长春。仙桃常结果,修竹每留云。一条涧壑藤
萝密,四面原堤草色新。正是百川会处擎天柱,万劫无移大地根。
那座山正当顶上,有一块仙石。其石有三丈六尺五寸高,有二丈四尺围圆。三丈六
尺五寸高,按周天三百六十五度;二丈四尺围圆,按政历二十四气。上有九窍八孔,
按九宫八卦。四面更无树木遮阴,左右倒有芝兰相衬。盖自开辟以来,每受天真地
秀,日精月华,感之既久,遂有灵通之意。内育仙胞,一日迸裂,产一石卵,似圆
球样大。因见风化作一个石猴。五官俱备,四肢皆全。便就学爬学走,拜了四方。
目运两道金光,射冲斗府。惊动高天上圣大慈仁者玉皇大天尊玄穹高上帝,驾座金
阙云宫灵霄宝殿,聚集仙卿,见有金光焰焰,即命千里眼、顺风耳开南天门观看。
二将果奉旨出门外,看的真,听的明。须臾回报道:“臣奉旨观听金光之处,乃东胜
神洲海东傲来小国之界,有一座花果山,山上有一仙石,石产一卵,见风化一石猴,
在那里拜四方,眼运金光,射冲斗府。如今服饵水食,金光将潜息矣。”玉帝垂赐恩
慈曰:“下方之物,乃天地精华所生,不足为异。”

那猴在山中,却会行走跳跃,食草木,饮涧泉,采山花,觅树果;与狼虫为伴,
虎豹为群,獐鹿为友,猕猿为亲;夜宿石崖之下,朝游峰洞之中。真是“山中无甲
子,寒尽不知年。”

一朝天气炎热,与群猴避暑,都在松阴之下顽耍。你看他一个个:

跳树攀枝,采花觅果;抛弹子,么儿;跑沙窝,砌宝塔;赶蜻蜓,扑蜡;
参老天,拜菩萨;扯葛藤,编草;捉虱子,咬又掐;理毛衣,剔指甲;挨的挨,
擦的擦;推的推,压的压;扯的扯,拉的拉。青松林下任他顽,绿水涧边随洗濯。
一群猴子耍了一会,却去那山涧中洗澡。见那股涧水奔流,真个似滚瓜涌溅。古云:
“禽有禽言,兽有兽语。”众猴都道:“这股水不知是那里的水。我们今日赶闲无事,
顺涧边往上溜头寻看源流,耍子去耶!”喊一声,都拖男挈女,唤弟呼兄,一齐跑来,
顺涧爬山,直至源流之处,乃是一股瀑布飞泉。但见那:
一派白虹起,千寻雪浪飞。
海风吹不断,江月照还依。
冷气分青嶂,余流润翠微。
潺名瀑布,真似挂帘帷。
众猴拍手称扬道:“好水,好水!原来此处远通山脚之下,直接大海之波。”又道:“那
一个有本事的,钻进去寻个源头出来,不伤身体者,我等即拜他为王。”连呼了三声,
忽见丛杂中跳出一个石猴,应声高叫道:“我进去,我进去!”好猴!也是他:
今日芳名显,时来大运通。
有缘居此地,天遣入仙宫。

你看他瞑目蹲身,将身一纵,径跳入瀑布泉中,忽睁睛抬头观看,那里边却无
水无波,明明朗朗的一架桥梁。他住了身,定了神,仔细再看,原来是座铁板桥。
桥下之水,冲贯于石窍之间,倒挂流出去,遮闭了桥门。却又欠身上桥头,再走再
看,却似有人家住处一般,真个好所在。但见那:

翠藓堆蓝,白云浮玉,光摇片片烟霞。虚窗静室,滑凳板生花。乳窟龙珠倚挂,
萦回满地奇葩。锅灶傍崖存火迹,樽靠案见肴渣。石座石床真可爱,石盆石碗更
堪夸。又见那一竿两竿修竹,三点五点梅花。几树青松常带雨,浑然像个人家。
看罢多时,跳过桥中间,左右观看,只见正当中有一石碣。碣上有一行楷书大字,
镌着“花果山福地,水帘洞洞天”。石猿喜不自胜,急抽身往外便走,复瞑目蹲身,
跳出水外,打了两个呵呵道:“大造化,大造化!”众猴把他围住,问道:“里面怎么
样?水有多深?”石猴道:“没水,没水!原来是一座铁板桥。桥那边是一座天造地
设的家当。”众猴道:“怎见得是个家当?”石猴笑道:“这股水乃是桥下冲贯石窍,
倒挂下来遮闭门户的。桥边有花有树,乃是一座石房。房内有石锅、石灶、石碗、
石盆、石床、石凳。中间一块石碣上,镌着‘花果山福地,水帘洞洞天。’真个是我
们安身之处。里面且是宽阔,容得千百口老小。我们都进去住,也省得受老天之气。
这里边:
刮风有处躲,下雨好存身。
霜雪全无惧,雷声永不闻。
烟霞常照耀,祥瑞每蒸熏。
松竹年年秀,奇花日日新。”

众猴听得,个个欢喜。都道:“你还先走,带我们进去,进去!”石猴却又瞑目
蹲身,往里一跳,叫道:“都随我进来!进来!”那些猴有胆大的,都跳进去了;胆小
的,一个个伸头缩颈,抓耳挠腮,大声叫喊,缠一会,也都进去了。跳过桥头,一
个个抢盆夺碗,占灶争床,搬过来,移过去,正是猴性顽劣,再无一个宁时,只搬
得力倦神疲方止。

石猿端坐上面道:“列位呵,‘人而无信,不知其可。’你们才说有本事进得来,
出得去,不伤身体者,就拜他为王。我如今进来又出去,出去又进来,寻了这一个
洞天与列位安眠稳睡,各享成家之福,何不拜我为王?”众猴听说,即拱伏无违。
一个个序齿排班,朝上礼拜,都称“千岁大王”。自此,石猿高登王位,将“石”字
儿隐了,遂称美猴王。有诗为证,诗曰:
三阳交泰产群生,仙石胞含日月精。
借卵化猴完大道,假他名姓配丹成。
内观不识因无相,外合明知作有形。
历代人人皆属此,称王称圣任纵横。

美猴王领一群猿猴、猕猴、马猴等,分派了君臣佐使,朝游花果山,暮宿水帘
洞,合契同情,不入飞鸟之丛,不从走兽之类,独自为王,不胜欢乐。是以:
春采百花为饮食,夏寻诸果作生涯。
秋收芋栗延时节,冬觅黄精度岁华。

美猴王享乐天真,何期有三五百载。一日,与群猴喜宴之间,忽然忧恼,堕下
泪来。众猴慌忙罗拜道:“大王何为烦恼?”猴王道:“我虽在欢喜之时,却有一点
儿远虑,故此烦恼。”众猴又笑道:“大王好不知足!我等日日欢会,在仙山福地,古
洞神洲,不伏麒麟辖,不伏凤凰管,又不伏人间王位所拘束,自由自在,乃无量之
福,为何远虑而忧也?”猴王道:“今日虽不归人王法律,不惧禽兽威严,将来年老
血衰,暗中有阎王老子管着,一旦身亡,可不枉生世界之中,不得久注天人之内?”
众猴闻此言,一个个掩面悲啼,俱以无常为虑。只见那班部中,忽跳出一个通背猿
猴,厉声高叫道:“大王若是这般远虑,真所谓道心开发也!如今五虫之内,惟有三
等名色,不伏阎王老子所管。”猴王道:“你知那三等人?”猿猴道:“乃是佛与仙与
神圣三者,躲过轮回,不生不灭,与天地山川齐寿。”猴王道:“此三者居于何所?”
猿猴道:“他只在阎浮世界之中,古洞仙山之内。”猴王闻之,满心欢喜,道:“我明
日就辞汝等下山,云游海角,远涉天涯,务必访此三者,学一个不老长生,常躲过
阎君之难。”噫!这句话,顿教跳出轮回网,致使齐天大圣成。众猴鼓掌称扬,都道:
“善哉!善哉!我等明日越岭登山,广寻些果品,大设筵宴送大王也。”

次日,众猴果去采仙桃,摘异果,刨山药,精,芝兰香蕙,瑶草奇花,般般
件件,整整齐齐,摆开石凳石桌,排列仙酒仙肴。但见那:

金丸珠弹,红绽黄肥:金丸珠弹腊樱桃,色真甘美;红绽黄肥熟梅子,味果香
酸。鲜龙眼,肉甜皮薄;火荔枝,核小囊红。林檎碧实连枝献,枇杷缃苞带叶擎。
兔头梨子鸡心枣,消渴除烦更解酲。香桃烂杏,美甘甘似玉液琼浆;脆李杨梅,酸
荫荫如脂酥膏酪。红囊黑子熟西瓜,四瓣黄皮大柿子。石榴裂破,丹砂粒现火晶珠;
芋栗剖开,坚硬肉团金玛瑙。胡桃银杏可传茶,椰子葡萄能做酒。榛松榧柰满盘盛,
桔蔗柑橙盈案摆。熟煨山药,烂煮黄精。捣碎茯苓并薏苡,石锅微火漫炊羹。人间
纵有珍羞味,怎比山猴乐更宁?
群猴尊美猴王上坐,各依齿肩排于下边,一个个轮流上前奉酒奉花,奉果,痛饮了
一日。

次日,美猴王早起,教:“小的们,替我折些枯松,编作筏子,取个竹竿作篙,
收拾些果品之类,我将去也。”果独自登筏,尽力撑开,飘飘荡荡,径向大海波中,
趁天风,来渡南赡部洲地界。这一去,正是那:
天产仙猴道行隆,离山驾筏趁天风。
飘洋过海寻仙道,立志潜心建大功。
有分有缘休俗愿,无忧无虑会元龙。
料应必遇知音者,说破源流万法通。
也是他运至时来,自登木筏之后,连日东南风紧,将他送到西北岸前,乃是南赡部
洲地界。持篙试水,偶得浅水,弃了筏子,跳上岸来,只见海边有人捕鱼、打雁、
挖蛤、淘盐。他走近前,弄个把戏,妆个虎,吓得那些人丢筐弃网,四散奔跑。
将那跑不动的拿住一个,剥了他的衣裳,也学人穿在身上,摇摇摆摆,穿州过府,
在市廛中,学人礼,学人话。朝餐夜宿,一心里访问佛仙神圣之道,觅个长生不老
之方。见世人都是为名为利之徒,更无一个为身命者。正是那:
争名夺利几时休?早起迟眠不自由!
骑着驴骡思骏马,官居宰相望王侯。
只愁衣食耽劳碌,何怕阎君就取勾?
继子荫孙图富贵,更无一个肯回头!

猴王参访仙道,无缘得遇。在于南赡部洲,串长城,游小县,不觉八九年余。
忽行至西洋大海,他想着海外必有神仙。独自个依前作筏,又飘过西海,直至西牛
贺洲地界。登岸遍访多时,忽见一座高山秀丽,林麓幽深。他也不怕狼虫,不惧虎
豹,登山顶上观看。果是好山:

千峰排戟,万仞开屏。日映岚光轻锁翠,雨收黛色冷含青。瘦藤缠老树,古渡
界幽程。奇花瑞草,修竹乔松:修竹乔松,万载常青欺福地;奇花瑞草,四时不谢
赛蓬瀛。幽鸟啼声近,源泉响溜清。重重谷壑芝兰绕,处处崖苔藓生。起伏峦头
龙脉好,必有高人隐姓名。
正观看间,忽闻得林深之处,有人言语,急忙趋步,穿入林中,侧耳而听,原来是
歌唱之声。歌曰:

“观棋柯烂,伐木丁丁,云边谷口徐行。卖薪沽酒,狂笑自陶情。苍径秋高,
对月枕松根,一觉天明。认旧林,登崖过岭,持斧断枯藤。

收来成一担,行歌市上,易米三升。更无些子争竞,时价平平。不会机谋巧算,
没荣辱,恬淡延生。相逢处,非仙即道,静坐讲《黄庭》。”
美猴王听得此言,满心欢喜道:“神仙原来藏在这里!”即忙跳入里面,仔细再看,
乃是一个樵子,在那里举斧砍柴。但看他打扮非常:

头上戴箬笠,乃是新笋初脱之箨;身上穿布衣,乃是木绵拈就之纱;腰间系环
绦,乃是老蚕口吐之丝;足下踏草履,乃是枯莎槎就之爽。手执钢斧,担挽火麻
绳;扳松劈枯树,争似此樵能!

猴王近前叫道:“老神仙!弟子起手。”那樵汉慌忙丢了斧,转身答礼道:“不当
人,不当人!我拙汉衣食不全,怎敢当‘神仙’二字?”猴王道:“你不是神仙,如
何说出神仙的话来?”樵夫道:“我说甚么神仙话?”猴王道:“我才来至林边,只
听的你说:‘相逢处非仙即道,静坐讲《黄庭》。’《黄庭》乃道德真言,非神仙而何?”
樵夫笑道:“实不瞒你说,这个词名做《满庭芳》,乃一神仙教我的。那神仙与我舍
下相邻,他见我家事劳苦,日常烦恼,教我遇烦恼时,即把这词儿念念,一则散心,
二则解困。我才有些不足处思虑,故此念念。不期被你听了。”猴王道:“你家既与
神仙相邻,何不从他修行,学得个不老之方,却不是好?”樵夫道:“我一生命苦:
自幼蒙父母养育至八九岁,才知人事,不幸父丧,母亲居孀。再无兄弟姊妹,只我
一人,没奈何,早晚侍奉。如今母老,一发不敢抛离。却又田园荒芜,衣食不足,
只得斫两束柴薪,挑向市廛之间,货几文钱,籴几升米,自炊自造,安排些茶饭,
供养老母,所以不能修行。”猴王道:“据你说起来,乃是一个行孝的君子,向后必
有好处。但望你指与我那神仙住处,却好拜访去也。”樵夫道:“不远,不远。此山
叫做灵台方寸山。山中有座斜月三星洞。那洞中有一个神仙,称名须菩提祖师。那
祖师出去的徒弟,也不计其数,见今还有三四十人从他修行。你顺那条小路儿,向
南行七八里远近,即是他家了。”猴王用手扯住樵夫道:“老兄,你便同我去去。若
还得了好处,决不忘你指引之恩。”樵夫道:“你这汉子,甚不通变。我方才这般与
你说了,你还不省?假若我与你去了,却不误了我的生意,老母何人奉养?我要斫柴,
你自去,自去!”

猴王听说,只得相辞。出深林,找上路径,过一山坡,约有七八里远,果然望
见一座洞府。挺身观看,真好去处,但见:

烟霞散彩,日月摇光。千株老柏,万节修篁:千株老柏带雨,半空青冉冉;万
节修篁含烟,一壑色苍苍。门外奇花布锦,桥边瑶草喷香。石崖突兀青苔润,悬壁
高张翠藓长。时闻仙
鹤唳,每见凤凰翔。仙鹤唳时,声振九霄汉远;凤凰翔起,翎毛五色彩云光。玄
猿白鹿随隐见,金狮玉象任行藏。细观灵福地,真个赛天堂!
又见那洞门紧闭,静悄悄杳无人迹。忽回头,见崖头立一石碑,约有三丈余高,八
尺余阔,上有一行十个大字,乃是“灵台方寸山,斜月三星洞”。美猴王十分欢喜道:
“此间人果是朴实。果有此山此洞。”看勾多时,不敢敲门。且去跳上松枝梢头,摘
松子吃了顽耍。少顷间,只听得呀的一声,洞门开处,里面走出一个仙童,真个丰
姿英伟,像貌清奇,比寻常俗子不同。但见他:
髻双丝绾,宽袍两袖风。
貌和身自别,心与相俱空。
物外长年客,山中永寿童。
一尘全不染,甲子任翻腾。
那童子出得门来,高叫道:“甚么人在此搔扰?”猴王扑的跳下树来,上前躬身道:
“仙童,我是个访道学仙之弟子,更不敢在此搔扰。”仙童笑道:“你是个访道的么?”
猴王道:“是。”童子道:“我家师父,正才下榻,登坛讲道,还未说出原由,就教我
出来开门。说:‘外面有个修行的来了,可去接待接待。’想必就是你了?”猴王笑
道:“是我,是我。”童子道:“你跟我进来。”

这猴王整衣端肃,随童子径入洞天深处观看:一层层深阁琼楼,一进进珠宫贝
阙,说不尽那静室幽居,直至瑶台之下。见那菩提祖师端坐在台上,两边有三十个
小仙侍立台下。果然是:
大觉金仙没垢姿,西方妙相祖菩提。
不生不灭三三行,全气全神万万慈。
空寂自然随变化,真如本性任为之。
与天同寿庄严体,历劫明心大法师。

美猴王一见,倒身下拜,磕头不计其数,口中只道:“师父,师父!我弟子志心
朝礼,志心朝礼!”祖师道:“你是那方人氏?且说个乡贯姓名明白,再拜。”猴王道:
“弟子乃东胜神洲傲来国花果山水帘洞人氏。”祖师喝令:“赶出去!他本是个撒诈捣
虚之徒,那里修甚么道果!”猴王慌忙磕头不住道:“弟子是老实之言,决无虚诈。”
祖师道:“你既老实,怎么说东胜神洲?那去处到我这里,隔两重大海,一座南赡部
洲,如何就得到此?”猴王叩头道:“弟子飘洋过海,登界游方,有十数个年头,方
才访到此处。”祖师道:“既是逐渐行来的也罢。你姓甚么?”猴王又道:“我无性。
人若骂我,我也不恼;若打我,我也不嗔,只是陪个礼儿就罢了。一生无性。”祖师
道:“不是这个性。你父母原来姓甚么?”猴王道:“我也无父母。”祖师道:“既无
父母,想是树上生的?”猴王道:“我虽不是树上生,却是石里长的。我只记得花果
山上有一块仙石,其年石破,我便生也。”祖师闻言暗喜,道:“这等说,却是个天
地生成的。你起来走走我看。”猴王纵身跳起,拐呀拐的走了两遍。祖师笑道:“你
身躯虽是鄙陋,却像个食松果的猢狲。我与你就身上取个姓氏,意思教你姓‘猢’。
猢字去了个兽傍,乃是个古月。古者,老也;月者,阴也。老阴不能化育,教你姓
‘狲’倒好。狲字去了兽傍,乃是个子系。子者,儿男也;系者,婴细也。正合婴
儿之本论。教你姓‘孙’罢。”猴王听说,满心欢喜,朝上叩头道:“好,好,好!今
日方知姓也。万望师父慈悲!既然有姓,再乞赐个名字,却好呼唤。”祖师道:“我门
中有十二个字,分派起名,到你乃第十辈之小徒矣。”猴王道:“那十二个字?”祖
师道:“乃‘广大智慧,真如性海,颖悟圆觉’十二字。排到你,正当‘悟’字。与
你起个法名叫做‘孙悟空’,好么?”猴王笑道:“好,好,好!自今就叫做孙悟空也!”
正是:
鸿蒙初辟原无姓,打破顽空须悟空。

毕竟不知向后修些甚么道果,且听下回分解。