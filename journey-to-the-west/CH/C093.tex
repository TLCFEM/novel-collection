\chapter{给孤园问古谈因~天竺国朝王遇偶}

起念断然有爱,留情必定生灾。灵明何事辨三台?行满自归元海。不论成仙成
佛,须从个里安排。清清净净绝尘埃,果正飞升上界。

却说寺僧,天明不见了三藏师徒,都道:“不曾留得,不曾别得,不曾求告得,
清清的把个活菩萨放得走了!”正说处,只见南关厢有几个大户来请。众僧扑掌道:
“昨晚不曾防御,今夜都驾云去了。”众人齐望空拜谢。此言一讲,满城中官员人
等,尽皆知之。叫此大户人家,俱治办五牲花果,往生祠祭献酬恩不题。

却说唐僧四众,餐风宿水,一路平宁,行有半个多月。忽一日,见座高山,唐
僧又悚惧道:“徒弟,那前面山岭峻峭,是必小心!”行者笑道:“这边路上将近佛
地,断乎无甚妖邪。师父放怀勿虑。”唐僧道:“徒弟,虽然佛地不远。但前日那寺
僧说,到天竺国都下有二千里,还不知是有多少路哩。”行者道:“师父,你好是又
把乌巢禅师《心经》忘记了也?”三藏道:“《般若心经》是我随身衣钵。自那乌巢
禅师教后,那一日不念,那一时得忘?颠倒也念得来,怎会忘得!”行者道:“师父
只是念得,不曾求那师父解得。”三藏说:“猴头,怎又说我不曾解得!你解得么?”
行者道:“我解得,我解得。”自此,三藏、行者再不作声。

旁边笑倒一个八戒,喜坏一个沙僧,说道:“嘴巴!替我一般的做妖精出身,又
不是那里禅和子,听过讲经,那里应佛僧,也曾见过说法?弄虚头,找架子,说甚
么‘晓得,解得’!怎么就不作声?听讲,请解!”沙僧说:“二哥,你也信他。大哥
扯长话,哄师父走路。他晓得弄棒罢了,他那里晓得讲经!”三藏道:“悟能、悟净,
休要乱说。悟空解得是无言语文字,乃是真解。”

他师徒们正说话间,却倒也走过许多路程,离了几个山冈,路旁早见一座大寺。
三藏道:“悟空,前面是座寺啊。你看那寺:

倒也不小不大,却也是琉璃碧瓦;半新半旧,却也是八字红墙。隐隐见苍松偃
盖,也不知是几千百年间故物到于今;潺潺听流水鸣弦,也不道是那朝代时分开山
留得在。山门上,大书着‘布金禅寺’;悬扁上,留题着‘上古遗迹’。”
行者看得是“布金禅寺”,八戒也道是“布金禅寺”。三藏在马上沉思道:“‘布金’……
‘布金’……这莫不是舍卫国界了么?”八戒道:“师父,奇啊!我跟师父几年,再
不曾见识得路,今日也识得路了。”三藏说道:“不是。我常看经诵典,说是佛在舍
卫城祗树给孤园。这园说是给孤独长者问太子买了,请佛讲经。太子说:‘我这园
不卖。他若要买我的时,除非黄金满布园地。’给孤独长者听说,随以黄金为砖,
布满园地,才买得太子祇园,才请得世尊说法。我想这布金寺莫非就是这个故事。”
八戒笑道:“造化!若是就是这个故事,我们也去摸他块把砖儿送人。”大家又笑了
一会,三藏才下得马来。

进得山门,只见山门下,挑担的,背包的,推车的,整车坐下;也有睡的去睡,
讲的去讲。忽见他们师徒四众,俊的又俊,丑的又丑,大家有些害怕,却也就让开
些路儿。三藏生怕惹事,口中不住只叫:“斯文!斯文!”这时节,却也大家收敛。
转过金刚殿后,早有一位禅僧走出,却也威仪不俗。真是:
面如满月光,身似菩提树。
拥锡袖飘风,芒鞋石头路。
三藏见了问讯。那僧即忙还礼道:“师从何来?”三藏道:“弟子陈玄奘,奉东土大
唐皇帝之旨,差往西天拜佛求经。路过宝方,造次奉谒,便求借一宿,明日就行。”
那僧道:“荒山十方常住,都可随喜;况长老东土神僧,但得供养,幸甚。”三藏谢
了,随即唤他三人同行。过了回廊香积,径入方丈。相见礼毕,分宾主坐定。行者
三人,亦垂手坐了。

话说这时寺中听说到了东土大唐取经僧人,寺中若大若小,不问长住、挂榻、
长老、行童,一一都来参见。茶罢,摆上斋供。这时长老还正开斋念偈,八戒早是
要紧,馒头、素食、粉汤一搅直下。这时方丈却也人多,有知识的,赞说三藏威仪;
好耍子的,都看八戒吃饭。却说沙僧眼溜,看见头底,暗把八戒捏了一把,说道:
“斯文!”八戒着忙,急的叫将起来,说道:“斯文,斯文!肚里空空!”沙僧笑道:
“二哥,你不晓的。天下多少‘斯文’,若论起肚子里来,正替你我一般哩。”八戒
方才肯住。三藏念了结斋,左右彻了席面,三藏称谢。

寺僧问起东土来因,三藏说到古迹,才问布金寺名之由。那僧答曰:“这寺原
是舍卫国给孤独园寺,又名祇园。因是给孤独长者请佛讲经,金砖布地,又易今名。
我这寺一望之前,乃是舍卫国。那时给孤独长者正在舍卫国居住。我荒山原是长者
之祇园,因此遂名给孤布金寺。寺后边还有祇园基址。近年间,若遇时雨滂沱,还
淋出金银珠儿。有造化的,每每拾着。”三藏道:“话不虚传果是真!”

又问道:“才进宝山,见门下两廊有许多骡马车担的行商,为何在此歇宿?”
众僧道:“我这山唤做百脚山。先年且是太平,近因天气循环,不知怎的,生几个
蜈蚣精,常在路下伤人。虽不至于伤命,其实人不敢走。山下有一座关,唤做鸡鸣
关。但到鸡鸣之时,才敢过去。那些客人,因到晚了,惟恐不便,权借荒山一宿,
等鸡鸣后便行。”三藏道:“我们也等鸡鸣后去罢。”师徒们正说处,又见拿上斋来,
却与唐僧等吃毕。

此时上弦月皎。三藏与行者步月闲行,又见个道人来报道:“我们老师爷要见
见中华人物。”三藏急转身,见一个老和尚,手持竹杖,向前作礼道:“此位就是中
华来的师父?”三藏答礼道:“不敢。”老僧称赞不已。因问:“老师高寿?”三藏
道:“虚度四十五年矣。敢问老院主尊寿?”老僧笑道:“比老师痴长一花甲也。”
行者道:“今年是一百零五岁了。你看我有多少年纪?”老僧道:“师家貌古神清,
况月夜眼花,急看不出来”。叙了一会,又向后廊看看。三藏道:“才说给孤园基址,
果在何处?”老僧道:“后门外就是。”快教开门,但见是一块空地,还有些碎石迭
的墙脚。三藏合掌叹曰:
“忆昔檀那须达多,曾将金宝济贫疴。
祇园千古留名在,长者何方伴觉罗?”

他都玩着月,缓缓而行。行近后门外,至台上,又坐了一坐,忽闻得有啼哭之
声。三藏静心诚听,哭的是爷娘不知苦痛之言。他就感触心酸,不觉泪堕,回问众
僧道:“是甚人在何处悲切?”老僧见问,即命众僧先回去煎茶,见无人,方才对
唐僧、行者下拜。三藏搀起道:“老院主,为何行此礼?”老僧道:“弟子年岁百余,
略通人事。每于禅静之间,也曾见过几番景象。若老爷师徒,弟子聊知一二,与他
人不同。若言悲切之事,非这位师家,明辨不得。”行者道:“你且说,是甚事?”

老僧道:“旧年今日,弟子正明性月之时,忽闻一阵风响,就有悲怨之声。弟
子下榻,到祇园基上看处,乃是一个美貌端正之女。我问他:‘你是谁家女子?为甚
到于此地?’那女子道:‘我是天竺国国王的公主。因为月下观花,被风刮来的。’
我将他锁在一间敝空房里,将那房砌作个监房模样,门上止留一小孔,仅递得碗过。
当日与众僧传道:‘是个妖邪,被我捆了。’但我僧家乃慈悲之人,不肯伤他性命。
每日与他两顿粗茶粗饭,吃着度命。那女子也聪明,即解吾意,恐为众僧点污,就
装风作怪,尿里眠,屎里卧。白日家说胡话,呆呆邓邓的;到夜静处,却思量父母
啼哭。我几番家进城乞化打探公主之事,全然无损。故此坚收紧锁,更不放出。今
幸老师来国,万望到了国中,广施法力,辨明辨明。一则救拔良善,二则昭显神通
也。”三藏与行者听罢,切切在心。正说处,只见两个小和尚请吃茶安置,遂而回
去。

八戒与沙僧在方丈中,突突哝哝的道:“明日要鸡鸣走路,此时还不来睡!”行
者道:“呆子又说甚么?”八戒道:“睡了罢,这等夜深,还看甚么景致。”因此,
老僧散去,唐僧就寝。正是那:
人静月沉花梦悄,暖风微透壁窗纱。
铜壶点点看三汲,银汉明明照九华。

当夜睡还未久,即听鸡鸣。那前边行商烘烘皆起,引灯造饭。这长老也唤醒八
戒、沙僧,扣马收拾。行者叫点灯来。那寺僧已先起来,安排茶汤点心,在后候敬。
八戒欢喜,吃了一盘馍馍,把行李、马匹牵出。三藏、行者对众辞谢。老僧又向行
者道:“悲切之事,在心,在心!”行者笑道:“谨领,谨领!我到城中,自能聆音而
察理,见貌而辨色也。”那伙行商,哄哄嚷嚷的,也一同上了大路。将有寅时,过
了鸡鸣关。至巳时,方见城垣。真是铁瓮金城,神洲天府。那城:
虎锯龙蟠形势高,凤楼麟阁彩光摇。
御沟流水如环带,福地依山插锦标。
晓日旌旗明辇路,春风箫鼓遍溪桥。
国王有道衣冠胜,五谷丰登显俊豪。

当日入于东市街,众商各投旅店。他师徒们进城,正走处,有一个会同馆驿,
三藏等径入驿内。那驿内管事的,即报驿丞道:“外面有四个异样的和尚,牵一匹
白马进来了。”驿丞听说有马,就知是官差的,出厅迎迓。三藏施礼道:“贫僧是东
土唐朝钦差灵山大雷音见佛求经的。随身有关文,入朝照验。借大人高衙一歇,事
毕就行。”驿丞答礼道:“此衙门原设待使客之处,理当款迓。请进,请进。”三藏
喜悦,教徒弟们都来相见。那驿丞看见嘴脸丑陋,暗自心惊,不知是人是鬼,战兢
兢的,只得看茶,摆斋。三藏见他惊怕,道:“大人勿惊,我等三个徒弟,相貌虽
丑,心地俱良。俗谓‘山恶人善’,何以惧为!”

驿丞闻言,方才定了心性,问道:“国师,唐朝在于何方?”三藏道:“在南赡
部洲中华之地。”又问:“几时离家?”三藏道:“贞观十三年,今已历过十四载,
苦经了些万水千山,方到此处。”驿丞道:“神僧,神僧!”三藏问道:“上国天年几
何?”驿丞道:“我敝处乃大天竺国,自太祖太宗传到今,已五百余年。现在位的
爷爷,爱山水花卉,号做怡宗皇帝,改元靖宴,今已二十八年了。”三藏道:“今日
贫僧要去见驾倒换关文,不知可得遇朝?”驿丞道:“好,好,正好!近因国王的公
主娘娘,年登二十青春,正在十字街头,高结彩楼,抛打绣球,撞天婚招驸马。今
日正当热闹之际,想我国王爷爷还未退朝。若欲倒换关文,趁此时好去。”三藏欣
然要走,只见摆上斋来,遂与驿丞、行者等吃了。

时已过午。三藏道:“我好去了。”行者道:“我保师父去。”八戒道:“我去。”
沙僧道:“二哥罢么。你的嘴脸不见怎的,莫到朝门外装胖,还教大哥去。”三藏道:
“悟净说得好。呆子粗夯,悟空还有些细腻。”那呆子掬着嘴道:“除了师父,我三
个的嘴脸也差不多儿。”

三藏却穿了袈裟,行者拿了引袋同去。只见街坊上,士农工商,文人墨客,愚
夫俗子,齐咳咳都道:“看抛绣球去也!”三藏立于道旁,对行者道:“他这里人物
衣冠,宫室器用,言语谈吐,也与我大唐一般。我想着我俗家先母也是抛打绣球遇
旧姻缘,结了夫妇。此处亦有此等风俗。”行者道:“我们也去看看,如何?”三藏
道:“不可,不可!你我服色不便,恐有嫌疑。”行者道:“师父,你忘了那给孤布金
寺老僧之言:一则去看彩楼,二则去辨真假。似这般忙忙的,那皇帝必听公主之喜
报,那里视朝理事?且去去来!”三藏听说,真与行者相随。见各项人等俱在那里看
打绣球。呀!那知此去,却是:
渔翁抛下钩和线,从今钓出是非来。

话表那个天竺国王,因爱山水花卉,前年带后妃公主在御花园,月夜赏玩,惹
动一个妖邪,把真公主摄去,他却变做一个假公主。知得唐僧今年今月今日今时到
此,他假借国家之富,搭起彩楼。欲招唐僧为偶,采取元阳真气,以成太乙上仙。
正当午时三刻,三藏与行者杂入人丛,行近楼下,那公主才拈香焚起,祝告天地。
左右有五七十胭娇绣女,近侍的捧着绣球。那楼八窗玲珑。公主转睛观看,见唐僧
来得至近,将绣球取过来,亲手抛在唐僧头上。唐僧着了一惊,把个毗卢帽子打歪,
双手忙扶着那球。那球毂辘的滚在他衣袖之内。那楼上齐声发喊道:“打着个和尚
了!打着个和尚了!”

噫!十字街头,那些客商人等,济济哄哄,都来奔抢绣球,被行者喝一声,把
牙一,把腰躬一躬,长了有三丈高,使个神威,弄出丑脸,唬得些人跌跌爬爬,
不敢相近。

霎时人散,行者还现了本像。那楼上绣女宫娥并大小太监,都来对唐僧下拜道:
“贵人,贵人,请入朝堂贺喜。”三藏急还礼,扶起众人,回头埋怨行者道:“你这
猴头,又是撮弄我也!”行者笑道:“绣球儿打在你头上,滚在你袖里,干我何事,
埋怨怎么?”三藏道:“似此怎生区处?”行者道:“师父,你且放心。便入朝见驾,
我回驿报与八戒、沙僧等候。若是公主不招你便罢,倒换了关文就行;如必欲招你,
你对国王说,‘召我徒弟来,我要吩咐他一声。’那时召我三个入朝,我其间自能辨
别真假。此是‘倚婚降怪’之计。”唐僧无已从言,行者转身回驿。

那长老被众宫娥等撮拥至楼前。公主下楼,玉手相搀,同登宝辇,摆开仪从,
回转朝门。早有黄门官先奏道:“万岁,公主娘娘搀着一个和尚,想是绣球打着,
现在午门外候旨。”那国王见说,心甚不喜,意欲赶退,又不知公主之意何如,只
得含情宣入。公主与唐僧遂至金銮殿下,正是:
一对夫妻呼万岁,两门邪正拜千秋。
礼毕,又宣至殿上,开言问道:“僧人何来,遇朕女抛球得中?”唐僧俯伏奏道:“贫
僧乃南赡部洲大唐皇帝差往西天大雷音寺拜佛求经的。因有长路关文,特来朝王倒
换。路过十字街彩楼之下,不期公主娘娘抛绣球,打在贫僧头上。贫僧是出家异教
之人,怎敢与玉叶金枝为偶。万望赦贫僧死罪,倒换关文,打发早赴灵山,见佛求
经,回我国土,永注陛下之天恩也!”

国王道:“你乃东土圣僧,正是‘千里姻缘使线牵’。寡人公主,今登二十岁未
婚,因择今日年月日时俱利,所以结彩楼抛绣球,以求佳偶。可可的你来抛着,朕
虽不喜,却不知公主之意如何。”那公主叩头道:“父王,常言:‘嫁鸡逐鸡,嫁犬
逐犬。’女有誓愿在先,结了这球,告奏天地神明,撞天婚抛打;今日打着圣僧,
即是前世之缘,遂得今生之遇,岂敢更移!愿招他为驸马。”国王方喜。即宣钦天监
正台官选择日期。一壁厢收拾妆奁,又出旨晓谕天下。

三藏闻言,更不谢恩,只教:“放赦!放赦!”国王道:“这和尚甚不通理。朕以
一国之富,招你做驸马,为何不在此享用,念念只要取经!再若推辞,教锦衣官校
推出斩了!”长老唬得魂不附体,只得战兢兢叩头启奏道:“感蒙陛下天恩。但贫僧
一行四众,还有三个徒弟在外,今当领纳,只是不曾吩咐得一言,万望召他到此,
倒换关文,教他早去,不误了西来之意。”国王遂准奏道:“你徒弟在何处?”三藏
道:“都在会同馆驿。”随即差官召圣僧徒弟领关文西去,留圣僧在此为驸马。长老
只得起身侍立。有诗为证:
大丹不漏要三全,苦行难成恨恶缘。
道在圣传修在己,善由人积福由天。
休逞六根多贪欲,顿开一性本来原。
无爱无思自清净,管教解脱得超然。
当时差官至会同馆驿,宣召唐僧徒弟不题。

却说行者自彩楼下别了唐僧,走两步,笑两声,喜喜欢欢的回驿。八戒、沙僧
迎着道:“哥哥,你怎么那般喜笑?师父如何不见?”行者道:“师父喜了。”八戒道:
“还未到地头,又不曾见佛取得经回,是何来之喜?”行者笑道:“我与师父只走
至十字街彩楼之下,可可的被当朝公主抛绣球打中了师父,师父被些宫娥、彩女、
太监推拥至楼前,同公主坐辇入朝,招为驸马,此非喜而何?”八戒听说,跌脚捶
胸道:“早知我去好来!都是那沙僧惫懒,你不阻我啊,我径奔彩楼之下,一绣球打
着我老猪,那公主招了我,却不美哉妙哉!俊刮标致,停当,大家造化耍子儿,何
等有趣!”沙僧上前,把他脸上一抹道:“不羞!不羞!好个嘴巴骨子!‘三钱银子买
个老驴——自夸骑得!’要是一绣球打着你,就连夜烧‘退送纸’也还道迟了,敢
惹你这晦气进门!”八戒道:“你这黑子不知趣!丑自丑,还有些风味。自古道:‘皮
肉粗糙,骨格坚强,各有一得可取。”行者道:“呆子莫胡谈,且收拾行李,但恐师
父着了急,来叫我们,却好进朝保护他。”八戒道:“哥哥又说差了。师父做了驸马,
到宫中与皇帝的女儿交欢,又不是爬山路,遇怪逢魔,要你保护他怎的!他那样
一把子年纪,岂不知被窝里之事,要你去扶?”行者一把揪住耳朵,轮拳骂道:
“你这个淫心不断的夯货!说那甚胡话!”

正吵闹间,只见驿丞来报道:“圣上有旨,差官来请三位神僧。”八戒道:“端
的请我们为何?”驿丞道:“老神僧幸遇公主娘娘,打中绣球,招为驸马,故此差
官来请。”行者道:“差官在那里?教他进来。”那官看行者施礼。礼毕,不敢仰视,
只管暗念诵道:“是鬼,是怪?……是雷公,夜叉?……”行者道:“那官儿,有话不
说,为何沉吟?”那官儿慌得战战兢兢的,双手举着圣旨,口里乱道:“我公主有
请会亲……我主公会亲有请!”八戒道:“我这里没刑具,不打你,你慢慢说,不要
怕。”行者道:“莫成道怕你打?怕你那脸哩!快收拾挑担牵马进朝,见师父议事去
也!”这正是:
路逢狭道难回避,定教恩爱反为仇。

毕竟不知见了国王有何话说,且听下回分解。