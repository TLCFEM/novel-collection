\chapter{万寿山大仙留故友~五庄观行者窃人参}

却说那三人穿林入里,只见那呆子绷在树上,声声叫喊,痛苦难禁。行者上前
笑道:“好女婿呀!这早晚还不起来谢亲,又不到师父处报喜,还在这里卖解儿耍子
哩。咄!你娘呢?你老婆呢?好个绷巴吊拷的女婿呀!”那呆子见他来抢白着羞,咬着
牙,忍着疼,不敢叫喊。沙僧见了,老大不忍,放下行李,上前解了绳索救下。呆
子对他们只是磕头礼拜,其实羞耻难当。有《西江月》为证:

色乃伤身之剑,贪之必定遭殃。佳人二八好容妆,更比夜叉凶壮。

只有一
个原本,再无微利添囊。好将资本谨收藏,坚守休教放荡。
那八戒撮土焚香,望空礼拜。行者道:“你可认得那些菩萨么?”八戒道:“我已此
晕倒昏迷,眼花撩乱,那认得是谁?”行者把那简帖儿递与八戒。八戒见了是颂子,
更加惭愧。沙僧笑道:“二哥有这般好处哩,感得四位菩萨来与你做亲!”八戒道:
“兄弟再莫题起。不当人子了!从今后,再也不敢妄为。就是累折骨头,也只是摩
肩压担,随师父西域去也。”三藏道:“既如此说才是。”

行者遂领师父上了大路。在路餐风宿水,行罢多时,忽见有高山挡路。三藏勒
马停鞭道:“徒弟,前面一山,必须仔细,恐有妖魔作耗,侵害吾党。”行者道:“马
前但有我等三人,怕甚妖魔?”因此,长老安心前进。只见那座山,真是好山:

高山峻极,大势峥嵘。根接昆仑脉,顶摩霄汉中。白鹤每来栖桧柏,玄猿时复
挂藤萝。日映晴林,叠叠千条红雾绕;风生阴壑,飘飘万道彩云飞。幽鸟乱啼青竹
里,锦鸡齐斗野花间。只见那千年峰、五福峰、芙蓉峰,巍巍凛凛放毫光;万岁石、
虎牙石、三尖石,突突磷磷生瑞气。崖前草秀,岭上梅香。荆棘密森森,芝兰清淡
淡。深林鹰凤聚千禽,古洞麒麟辖万兽。涧水有情,曲曲弯弯多绕顾;峰峦不断,
重重叠叠自周回。又见那绿的槐,斑的竹,青的松,依依千载斗华;白的李,红
的桃,翠的柳,灼灼三春争艳丽。龙吟虎啸,鹤舞猿啼。麋鹿从花出,青鸾对日鸣。
乃是仙山真福地,蓬莱阆苑只如然。又见些花开花谢山头景,云去云来岭上峰。
三藏在马上欢喜道:“徒弟,我一向西来,经历许多山水,都是那嵯峨险峻之处,
更不似此山好景,果然的幽趣非常。若是相近雷音不远路,我们好整肃端严见世尊。”
行者笑道:“早哩,早哩,正好不得到哩!”沙僧道:“师兄,我们到雷音有多少远?”
行者道:“十万八千里。十停中还不曾走了一停哩。”八戒道:“哥啊,要走几年才
得到?”行者道:“这些路,若论二位贤弟,便十来日也可到;若论我走,一日也
好走五十遭,还见日色;若论师父走,莫想,莫想!”唐僧道:“悟空,你说得几时
方可到?”行者道:“你自小时走到老,老了再小,老小千番也还难;只要你见性
志诚,念念回首处,即是灵山。”沙僧道:“师兄,此间虽不是雷音,观此景致,必
有个好人居止。”行者道:“此言却当。这里决无邪祟,一定是个圣僧、仙辈之乡。
我们游玩慢行。”不题。

却说这座山名唤万寿山;山中有一座观,名唤五庄观;观里有一尊仙,道号镇
元子,混名与世同君。那观里出一般异宝,乃是混沌初分,鸿蒙始判,天地未开之
际,产成这颗灵根。盖天下四大部洲,惟西牛贺洲五庄观出此,唤名“草还丹”,
又名“人参果”。三千年一开花,三千年一结果,再三千年才得熟,短头一万年方
得吃。似这万年,只结得三十个果子。果子的模样,就如三朝未满的小孩相似,四
肢俱全,五官咸备。人若有缘,得那果子闻了一闻,就活三百六十岁;吃一个,就
活四万七千年。

当日镇元大仙得元始天尊的简帖,邀他到上清天上弥罗宫中听讲“混元道果”。
大仙门下出的散仙,也不计其数,见如今还有四十八个徒弟,都是得道的全真。当
日带领四十六个上界去听讲,留下两个绝小的看家:一个唤做清风,一个唤做明月。
清风只有一千三百二十岁,明月才交一千二百岁。镇元子吩咐二童道:“不可违了
大天尊的简帖,要往弥罗宫听讲,你两个在家仔细。不日有一个故人从此经过,却
莫怠慢了他。可将我人参果打两个与他吃,权表旧日之情。”二童道:“师父的故人
是谁?望说与弟子,好接待。”大仙道:“他是东土大唐驾下的圣僧,道号三藏,今
往西天拜佛求经的和尚。”二童笑道:“孔子云:‘道不同,不相为谋。’我等是太乙
玄门,怎么与那和尚做甚相识!”大仙道:“你那里得知。那和尚乃金蝉子转生,西
方圣老如来佛第二个徒弟。五百年前,我与他在‘兰盆会’上相识。他曾亲手传茶,
佛子敬我,故此是为故人也。”

二仙童闻言,谨遵师命。那大仙临行,又叮咛嘱咐道:“我那果子有数,只许
与他两个,不得多费。”清风道:“开园时,大众共吃了两个,还有二十八个在树,
不敢多费。”大仙道:“唐三藏虽是故人,须要防备他手下人罗唣,不可惊动他知。”
二童领命讫,那大仙承众徒弟飞升,径朝天界。

却说唐僧四众,在山游玩,忽抬头,见那松篁一簇,楼阁数层。唐僧道:“悟
空,你看那里是甚么去处?”行者看了道:“那所在,不是观宇,定是寺院。我们
走动些,到那厢方知端的。”不一时,来于门首观看,见那:

松坡冷淡,竹径清幽。往来白鹤送浮云,上下猿猴时献
果。那门前池宽树影长,石裂苔花破。宫殿森罗紫极高,楼台缥缈丹霞堕。真个是
福地灵区,蓬莱云洞。清虚人事少,寂静道心生。青鸟每传王母信,紫鸾常寄老君
经。看不尽那巍巍道德之风,果然漠漠神仙之宅。

三藏离鞍下马。又见那山门左边有一通碑,碑上有十个大字,乃是“万寿山福
地,五庄观洞天”。长老道:“徒弟,真个是一座观宇。”沙僧道:“师父,观此景鲜
明,观里必有好人居住。我们进去看看,若行满东回,此间也是一景。”行者道:“说
得好。”遂都一齐进去。又见那二门上有一对春联:
长生不老神仙府,与天同寿道人家。
行者笑道:“这道士说大话唬人。我老孙五百年前大闹天宫时,在那太上老君门首,
也不曾见有此话说。”八戒道:“且莫管他,进去,进去,或者这道士有些德行,未
可知也。”

及至二层门里,只见那里面急急忙忙,走出两个小童儿来。看他怎生打扮:
骨清神爽容颜丽,顶结丫髻短发。
道服自然襟绕雾,羽衣偏是袖飘风。
环绦紧束龙头结,芒履轻缠蚕口绒。
丰采异常非俗辈,正是那清风明月二仙童。
那童子控背躬身,出来迎接道:“老师父,失迎,请坐。”长老欢喜,遂与二童子上
了正殿观看。原来是向南的五间大殿,都是上明下暗的雕花格子。那仙童推开格子,
请唐僧入殿,只见那壁中间挂着五彩装成的“天地”二大字,设一张朱红雕漆的香
几,几上有一副黄金炉瓶,炉边有方便整香。

唐僧上前,以左手拈香注炉,三匝礼拜。拜毕,回头道:“仙童,你五庄观真
是西方仙界,何不供养三清、四帝、罗天诸宰,只将‘天地’二字侍奉香火?”童
子笑道:“不瞒老师说。这两个字,上头的,礼上还当;下边的,还受不得我们的
香火。是家师父谄佞出来的。”三藏道:“何为谄佞?”童子道:“三清是家师的朋
友,四帝是家师的故人;九曜是家师的晚辈,元辰是家师的下宾。”

那行者闻言,就笑得打跌。八戒道:“哥啊,你笑怎的?”行者道:“只讲老孙
会捣鬼,原来这道童会捆风!”三藏道:“令师何在?”童子道:“家师元始天尊降
简请到上清天弥罗宫听讲‘混元道果’去了,不在家。”

行者闻言,忍不住喝了一声道:“这个臊道童!人也不认得,你在那个面前捣鬼,
扯甚么空心架子!那弥罗宫有谁是太乙天仙?请你这泼牛蹄子去讲甚么!”三藏见他
发怒,恐怕那童子回言,斗起祸来。便道:“悟空,且休争竞。我们既进来就出去,
显得没了方情。常言道:‘鹭鸶不吃鹭鸶肉。’他师既是不在,搅扰他做甚?你去山
门前放马,沙僧看守行李,教八戒解包袱。取些米粮,借他锅灶,做顿饭吃,待临
行,送他几文柴钱,便罢了。各依执事,让我在此歇息歇息,饭毕就行。”他三人
果各依执事而去。

那明月、清风,暗自夸称不尽道:“好和尚!真个是西方爱圣临凡,真元不昧。
师父命我们接待唐僧,将人参果与他吃,以表故旧之情;又教防着他手下人罗唣。
果然那三个嘴脸凶顽,性情粗糙。幸得就把他们调开了;若在边前,却不与他人参
果见面。”清风道:“兄弟,还不知那和尚可是师父的故人。问他一问看,莫要错了。”
二童子又上前道:“启问老师可是大唐往西天取经的唐三藏?”长老回礼道:“贫僧
就是。仙童为何知我贱名?”童子道:“我师临行,曾吩咐教弟子远接。不期车驾
来促,有失迎迓。老师请坐,待弟子办茶来奉。”三藏道:“不敢。”那明月急转本
房,取一杯香茶,献与长老。茶毕,清风道:“兄弟,不可违了师命,我和你去取
果子来。”

二童别了三藏,同到房中,一个拿了金击子,一个拿了丹盘,又多将丝帕垫着
盘底,径到人参园内。那清风爬上树去,使金击子敲果;明月在树下,以丹盘等接。

须臾,敲下两个果来,接在盘中,径至前殿奉献道:“唐师父,我五庄观土僻
山荒,无物可奉,土仪素果二枚,权为解渴。”那长老见了,战战兢兢,远离三尺
道:“善哉,善哉!今岁倒也年丰时稔,怎么这观里作荒吃人?这个是三朝未满的孩
童,如何与我解渴?”清风暗道:“这和尚在那口舌场中,是非海里,弄得眼肉胎
凡,不识我仙家异宝。”明月上前道:“老师,此物叫做‘人参果’,吃一个儿不妨。”
三藏道:“胡说,胡说!他那父母怀胎,不知受了多少苦楚,方生下。未及三日,怎
么就把他拿来当果子?”清风道:“实是树上结的。”长老道:“乱谈,乱谈!树上又
会结出人来?拿过去,不当人子!”

那两个童儿,见千推万阻不吃,只得拿着盘子,转回本房。那果子却也跷蹊,
久放不得;若放多时,即僵了,不中吃。二人到于房中,一家一个,坐在床边上,
只情吃起。

噫,原来有这般事哩!他那道房,与那厨房紧紧的间壁。这边悄悄的言语,那
边即便听见。八戒正在厨房里做饭,先前听见说,取金击子,拿丹盘,他已在心;
又听见他说,唐僧不认得是人参果,即拿在房里自吃,口里忍不住流涎道:“怎得
一个儿尝新!”自家身子又狼,不能够得动,只等行者来,与他计较。他在那锅
门前,更无心烧火,不时的伸头探脑,出来观看。

不多时,见行者牵将马来,拴在槐树上,径往后走。那呆子用手乱招道:“这
里来,这里来!”行者转身,到于厨房门首,道:“呆子,你嚷甚的?想是饭不够吃。
且让老和尚吃饱,我们前边大人家,再化吃去罢。”八戒道:“你进来,不是饭少。
这观里有一件宝贝,你可晓得?”行者道:“甚么宝贝?”八戒笑道:“说与你,你
不曾见;拿与你,你不认得。”行者道:“这呆子笑话我老孙。老孙五百年前,因访
仙道时,也曾云游在海角天涯。那般儿不曾见?”八戒道:“哥啊,人参果你曾见
么?”行者惊道:“这个真不曾见。但只常闻得人说,人参果乃是草还丹,人吃了
极能延寿。如今那里有得?”八戒道:“他这里有。那童子拿两个与师父吃,那老
和尚不认得,道是三朝未满的孩儿,不曾敢吃。那童子老大惫懒,师父既不吃,便
该让我们,他就瞒着我们,才自在这隔壁房里,一家一个,的吃了出去,
就急得我口里水泱。怎么得一个儿尝新?我想你有些溜撒,去他那园子里偷几个来
尝尝,如何?”行者道:“这个容易。老孙去,手到擒来。”急抽身,往前就走。八
戒一把扯住道:“哥啊,我听得他在这房里说,要拿甚么金击子去打哩。须是干得
停当,不可走露风声。”行者道:“我晓得,我晓得。”

那大圣使一个隐身法,闪进道房看时,原来那两个道童,吃了果子,上殿与唐
僧说话,不在房里。行者四下里观看,看有甚么金击子,但只见窗棂上挂着一条赤
金:有二尺长短,有指头粗细;底下是一个蒜疙疸的头子;上边有眼,系着一根绿
绒绳儿。他道:“想必就是此物叫做金击子。”他却取下来,出了道房,径入后边去,
推开两扇门,抬头观看,呀!却是一座花园!但见:

朱栏宝槛,曲砌峰山。奇花与丽日争妍,翠竹共青天斗碧。流杯亭外,一弯绿
柳似拖烟;赏月台前,数簇乔松如泼靛。红拂拂,锦巢榴;绿依依,绣墩草。青茸
茸,碧砂兰;攸荡荡,临溪水。丹桂映金井梧桐,锦槐傍朱栏玉砌。有或红或白千
叶桃,有或香或黄九秋菊。荼架,映着牡丹亭;木槿台,相连芍药圃。看不尽傲
霜君子竹,欺雪大夫松。更有那鹤庄鹿宅,方沼圆池;泉流碎玉,地萼堆金;朔风
触绽梅花白,春来点破海棠红。诚所谓人间第一仙景,西方魁首花丛。
那行者观看不尽,又见一层门,推开看处,却是一座菜园:

布种四时蔬菜,菠芹姜苔。笋瓜瓠茭白,葱蒜芫荽韭薤。窝蕖童蒿苦,
葫芦茄子须栽。蔓菁萝卜羊头埋,红苋青菘紫芥。
行者笑道:“他也是个自种自吃的道士。”走过菜园,又见一层门。推开看处,呀!
只见那正中间有根大树,真个是青枝馥郁,绿叶阴森,那叶儿却似芭蕉模样,直上
去有千尺余高,根下有七八丈围圆。那行者倚在树下,往上一看,只见向南的枝上,
露出一个人参果,真个像孩儿一般。原来尾间上是个蒂,看他丁在枝头,手脚乱
动,点头幌脑,风过处似乎有声。行者欢喜不尽,暗自夸称道:“好东西呀!果然罕
见,果然罕见!”他倚着树,飕的一声,撺将上去。

那猴子原来第一会爬树偷果子。他把金击子敲了一下,那果子扑的落将下来。
他也随跳下来跟寻,寂然不见;四下里草中找寻,更无踪影。行者道:“跷蹊,跷
蹊!想是有脚的会走;就走也跳不出墙去。我知道了,想是花园中土地不许老孙偷
他果子,他收了去也。”他就捻着诀,念一口“”字咒,拘得那花园土地前来,
对行者施礼道:“大圣,呼唤小神,有何吩咐?”行者道:“你不知老孙是盖天下有
名的贼头。我当年偷蟠桃、盗御酒、窃灵丹,也不曾有人敢与我分用;怎么今日偷
他一个果子,你就抽了我的头分去了!这果子是树上结的,空中过鸟也该有分,老
孙就吃他一个,有何大害?怎么刚打下来,你就捞了去?”土地道:“大圣,错怪了
小神也。这宝贝乃是地仙之物,小神是个鬼仙,怎么敢拿去?就是闻也无福闻闻。”
行者道:“你既不曾拿去,如何打下来就不见了?”土地道:“大圣只知这宝贝延寿,
更不知他的出处哩。”

行者道:“有甚出处?”土地道:“这宝贝三千年一开花,三千年一结果,再三
千年方得成熟。短头一万年,只结得三十个。有缘的,闻一闻,就活三百六十岁;
吃一个,就活四万七千年。却是只与五行相畏。”行者道:“怎么与五行相畏?”土
地道:“这果子遇金而落,遇木而枯,遇水而化,遇火而焦,遇土而入。敲时必用
金器,方得下来。打下来,却将盘儿用丝帕衬垫方可;若受些木器,就枯了,就吃
也不得延寿。吃他须用磁器,清水化开食用,遇火即焦而无用。遇土而入者,大圣
方才打落地上,他即钻下土去了。这个土有四万七千年,就是钢钻钻他也钻不动些
须,比生铁也还硬三四分。人若吃了,所以长生。大圣不信时,可把这地下打打儿
看。”行者即掣金箍棒,筑了一下,响一声,迸起棒来,土上更无痕迹。行者道:“果
然!果然!我这棍,打石头如粉碎,撞生铁也有痕。怎么这一下打不伤些儿?这等说,
我却错怪了你了,你回去罢。”那土地即回本庙去讫。

大圣却有算计:爬上树,一只手使击子,一只手将锦布直裰的襟儿扯起来做个
兜子等住,他却串枝分叶,敲了三个果,兜在襟中。跳下树,一直前来,径至厨房
里去。那八戒笑道:“哥哥,可有么?”行者道:“这不是?老孙的手到擒来。这个
果子,也莫背了沙僧,可叫他一声。”八戒即招手叫道:“悟净,你来。”那沙僧撇
下行李,跑进厨房道:“哥哥,叫我怎的?”行者放开衣兜道:“兄弟,你看这个是
甚的东西?”沙僧见了道:“是人参果。”行者道:“好啊!你倒认得。你曾在那里吃
过的?”沙僧道:“小弟虽不曾吃,但旧时做卷帘大将,扶侍鸾舆赴蟠桃宴,尝见
海外诸仙将此果与王母上寿。见便曾见,却未曾吃。哥哥,可与我些儿尝尝?”行
者道:“不消讲,兄弟们一家一个。”

他三人将三个果各各受用。那八戒食肠大,口又大,一则是听见童子吃时,便
觉馋虫拱动,却才见了果子,拿过来,张开口,毂辘的囫囵舌咽下肚,却白着眼胡
赖,向行者、沙僧道:“你两个吃的是甚么?”沙僧道:“人参果。”八戒道:“甚么
味道?”行者道:“悟净,不要睬他,你倒先吃了,又来问谁?”八戒道:“哥哥,
吃的忙了些,不像你们细嚼细咽,尝出些滋味。我也不知有核无核,就吞下去了。
哥啊,为人为彻;已经调动我这馋虫,再去弄个儿来,老猪细细的吃吃。”行者道:
“兄弟,你好不知止足!这个东西,比不得那米食面食,撞着尽饱。像这一万年只
结得三十个,我们吃他这一个,也是大有缘法,不等小可。罢罢罢!彀了!”他欠起
身来,把一个金击子,瞒窗眼儿,丢进他道房里,竟不睬他。

那呆子只管絮絮叨叨的唧哝,不期那两个道童复进房来取茶去献,只听得八戒
还嚷甚么“人参果吃得不快活,再得一个儿吃吃才好。”清风听见,心疑道:“明月,
你听那长嘴和尚讲‘人参果还要个吃吃’。师父别时叮咛,教防他手下人罗唣,莫
敢是他偷了我们宝贝么?”明月回头道:“哥耶,不好了!不好了!金击子如何落在
地下!我们去园里看看来!”

他两个急急忙忙的走去,只见花园开了。清风道:“这门是我关的,如何开了?”
又急转过花园,只见菜园门也开了。忙入人参园里,倚在树下,望上查数;颠倒来
往,只得二十二个。明月道:“你可会算帐?”清风道:“我会,你说将来。”明月
道:“果子原是三十个。师父开园,分吃了两个,还有二十八个;适才打两个与唐
僧吃,还有二十六个;如今止剩得二十二个,却不少了四个?不消讲,不消讲,定
是那伙恶人偷了,我们只骂唐僧去来。”

两个出了园门,径来殿上,指着唐僧,秃前秃后,秽语污言,不绝口的乱骂;
贼头鼠脑,臭短臊长,没好气的胡嚷。唐僧听不过道:“仙童啊,你闹的是甚么?消
停些儿;有话慢说不妨,不要胡说散道的。”清风说:“你的耳聋?我是蛮话,你不
省得?你偷吃了人参果,怎么不容我说?”唐僧道:“人参果怎么模样?”明月道:
“才拿来与你吃,你说像孩童的不是?”唐僧道:“阿弥陀佛!那东西一见,我就心
惊胆战,还敢偷他吃哩!就是害了馋痞,也不敢干这贼事。不要错怪了人。”清风道:
“你虽不曾吃,还有手下人要偷吃的哩。”三藏道:“这等也说得是,你且莫嚷,等
我问他们看。果若是偷了,教他赔你。”明月道:“赔呀!就有钱那里去买!”三藏道:
“纵有钱没处买呵,常言道:‘仁义值千金。’教他陪你个礼,便罢了。也还不知是
他不是他哩。”明月道:“怎的不是他?他那里分不均,还在那里嚷哩。”三藏叫声“徒
弟,且都来。”沙僧听见道:“不好了,决撒了!老师父叫我们,小道童胡厮骂,不
是旧话儿走了风,却是甚的!”行者道:“活羞杀人!这个不过是饮食之类,若说出
来,就是我们偷嘴了,只是莫认。”八戒道:“正是,正是,昧了罢。”他三人只得
出了厨房,走上殿去。

咦!毕竟不知怎么与他抵赖,且听下回分解。