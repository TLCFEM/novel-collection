\chapter{镇海寺心猿知怪~黑松林三众寻师}

话表三藏师徒到镇海禅林寺,众僧相见,安排斋供。四众食毕,那女子也得些
食力。渐渐天昏,方丈里点起灯来。众僧一则是问唐僧取经来历,二则是贪看那女
子,都攒攒簇簇,排列灯下。

三藏对那初见的喇嘛僧道:“院主,明日离了宝山,西去的路途如何?”那僧
双膝跪下,慌得长老一把扯住道:“院主请起。我问你个路程,你为何行礼?”那
僧道:“老师父明日西行,路途平正,不须费心。只是眼下有件事儿不尴尬,一进
门就要说,恐怕冒犯洪威,却才斋罢,方敢大胆奉告:老师东来,路遥辛苦,都在
小和尚房中安歇甚好;只是这位女菩萨,不方便,不知请他那里睡好。”三藏道:“院
主,你不要生疑,说我师徒们有甚邪意。早间打黑松林过,撞见这个女子绑在树上。
小徒孙悟空不肯救他,是我发菩提心,将他救了,到此随院主送他那里睡去。”那
僧谢道:“既老师宽厚,请他到天王殿里,就在天王爷爷身后,安排个草铺,教他
睡罢。”三藏道:“甚好,甚好。”遂此时,众小和尚引那女子往殿后睡去。长老就
在方丈中,请众院主自在,遂各散去。三藏吩咐悟空:“辛苦了,早睡早起。”遂一
处都睡了,不敢离侧,护着师父。渐入夜深,正是那:
玉兔高升万簌宁,天街寂静断人行。
银河耿耿星光灿,鼓发谯楼趱换更。

一宵晚话不题。及天明了,行者起来,教八戒、沙僧收拾行囊、马匹,却请师
父走路。此时长老还贪睡未醒。行者近前叫声:“师父。”那师父把头抬了一抬,又
不曾答应得出。行者问:“师父怎么说?”长老呻吟道:“我怎么这般头悬眼胀,浑
身皮骨皆疼?”八戒听说,伸手去摸摸,身上有些发热。呆子笑道:“我晓得了。
这是昨晚见没钱的饭,多吃了几碗,倒沁着头睡,伤食了。”行者喝道:“胡说!等
我问师父,端的何如。”三藏道:“我半夜之间,起来解手,不曾戴得帽子,想是风
吹了。”行者道:“这还说得是。如今可走得路么?”三藏道:“我如今起坐不得,
怎么上马?但只误了路啊!”行者道:“师父说那里话!常言道:‘一日为师,终身为
父。’我等与你做徒弟,就是儿子一般。又说道:‘养儿不用阿金溺银,只是见景生
情便好。’你既身子不快,说甚么误了行程,便宁耐几日,何妨!”兄弟们都伏侍着
师父,不觉的早尽午来昏又至,良宵才过又侵晨。

光阴迅速,早过了三日。那一日,师父欠身起来叫道:“悟空,这两日病体沉
疴,不曾问得你,那个脱命的女菩萨,可曾有人送些饭与他吃?”行者笑道:“你
管他怎的,且顾了自家的病着。”三藏道:“正是,正是。你且扶我起来,取出我的
纸、笔、墨,寺里借个砚台来使使。”行者道:“要怎的?”长老道:“我要修一封
书,并关文封在一处,你替我送上长安驾下,见太宗皇帝一面。”行者道:“这个容
易。我老孙别事无能,若说送书,人间第一。你把书收拾停当与我,我一筋斗送到
长安,递与唐王,再一筋斗转将回来,你的笔砚还不干哩。但只是你寄书怎的?且
把书意念念我听。念了再写不迟。”长老滴泪道:“我写着:

臣僧稽首三顿首,万岁山呼拜圣君;文武两班同入目,公卿四百共知闻。当年
奉旨离东土,指望灵山见世尊。不料途中遭厄难,何期半路有灾。僧病沉疴难进
步,佛门深远接天门。有经无命空劳碌,启奏当今别遣人。”
行者听得此言,忍不住呵呵大笑道:“师父,你忒不济,略有些些病儿,就起这个
意念。你若是病重,要死要活,只消问我。我老孙自有个本事。问道‘那个阎王敢
起心?那个判官敢出票?那个鬼使来勾取?’若恼了我,我拿出那大闹天宫之性子,
又一路棍,打入幽冥,捉住十代阎王,一个个抽了他的筋,还不饶他哩!”三藏道:
“徒弟呀,我病重了,切莫说这大话。”

八戒上前道:“师兄,师父说不好,你只管说好!十分不尴尬,我们趁早商量,
先卖了马,典了行囊,买棺木送终散火。”行者道:“呆子又胡说了!你不知道。师
父是我佛如来第二个徒弟,原叫做金蝉长老;只因他轻慢佛法,该有这场大难。”
八戒道:“哥啊,师父既是轻慢佛法,贬回东土,在是非海内,口舌场中,托化做
人身,发愿往西天拜佛求经,遇妖精就捆,逢魔头就吊,受诸苦恼,也够了;怎么
又叫他害病?”行者道:“你那里晓得,老师父不曾听佛讲法,打了一个盹,往下
一失,左脚下了一粒米,下界来,该有这三日病。”八戒惊道:“像老猪吃东西泼
泼撒撒的,也不知害多少年代病是!”行者道:“兄弟,佛不与你众生为念。你又不
知。人云:‘锄禾日当午,汗滴禾下土。谁知盘中餐,粒粒皆辛苦!’师父只今日一
日,明日就好了。”三藏道:“我今日比昨不同:咽喉里十分作渴。你去那里,有凉
水寻些来我吃。”行者道:“好了!师父要水吃,便是好了。等我取水去。”

即时取了钵盂,往寺后面香积厨取水。忽见那些和尚一个个眼儿通红,悲啼哽
咽,只是不敢放声大哭。行者道:“你们这些和尚,忒小家子样!我们住几日,临行
谢你,柴火钱照日算还。怎么这等脓包!”众僧慌跪下道:“不敢,不敢!”行者道:
“怎么不敢?想是我那长嘴和尚,食肠大,吃伤了你的本儿也?”众僧道:“老爷,
我这荒山,大大小小,也有百十众和尚,每一人养老爷一日,也养得起百十日。怎
么敢欺心,计较甚么食用!”

行者道:“既不计较,你却为甚么啼哭?”众僧道:“老爷,不知是那山里来的
妖邪在这寺里。我们晚夜间着两个小和尚去撞钟打鼓,只听得钟鼓响罢,再不见人
回。至次日找寻,只见僧帽、僧鞋,丢在后边园里,骸骨尚存,将人吃了。你们住
了三日,我寺里不见了六个和尚。故此,我兄弟们不由的不怕,不由的不伤。因见
你老师父贵恙,不敢传说,忍不住泪珠偷垂也。”行者闻言,又惊又喜道:“不消说
了,必定是妖魔在此伤人也。等我与你剿除他。”众僧道:“老爷,妖精不精者不灵。
一定会腾云驾雾,一定会出幽入冥。古人道得好:‘莫信直中直,须防仁不仁。’老
爷,你莫怪我们说:你若拿得他住哩,便与我荒山除了这条祸根,正是三生有幸了;
若还拿他不住啊,却有好些儿不便处。”行者道:“怎叫做好些不便处?”那众僧道:
“直不相瞒老爷说。我这荒山,虽有百十众和尚,却都只是自小儿出家的:

发长寻刀削,衣单破衲缝。早晨起来洗着脸,叉手躬身,皈依大道;夜来收拾
烧着香,虔心叩齿,念的弥陀。举头看见佛,莲九品,三乘,慈航共法云,愿见
园释世尊;低头看见心,受五戒,度大千,生生万法中,愿悟顽空与色空。诸檀
越来啊,老的、小的、长的、矮的、胖的、瘦的,一个个敲木鱼,击金磬,挨挨拶
拶,两卷《法华经》,一策《梁王忏》;诸檀越不来啊,新的、旧的、生的、熟的、
村的、俏的,一个个合着掌,瞑着目,悄悄冥冥,入定蒲团上,牢关月下门。一任
他莺啼鸟语闲争斗,不上我方便慈悲大法乘。因此上,也不会伏虎,也不会降龙;
也不识的怪,也不识的精。你老爷若还惹起那妖魔啊,我百十个和尚只够他斋一饱:
一则堕落我众生轮回;二则灭抹了这禅林古迹;三则如来会上,全没半点儿光辉。
这却是好些儿不便处。”
行者闻得众和尚说出这一端的话语,他便怒从心上起,恶向胆边生,高叫一声:“你
这众和尚好呆哩!只晓得那妖精,就不晓得我老孙的行止么?”众僧轻轻的答道:“实
不晓得。”行者道:“我今日略节说说,你们听着:

我也曾花果山伏虎降龙,我也曾上天堂大闹天宫。饥时把老君的丹,略略咬了
两三颗;渴时把玉帝的酒,轻轻了六七钟。睁着一双不白不黑的金睛眼,天惨淡,
月朦胧;拿着一条不短不长的金箍棒,来无影,去无踪。说甚么大惊小怪,那怕他
惫懒脓!一赶赶上去,跑的跑,颤的颤,躲的躲,慌的慌;一捉捉将来,锉的锉,
烧的烧,磨的磨,舂的舂。正是八仙同过海,独自显神通!众和尚,我拿这妖精与
你看看,你才认得我老孙!”
众僧听着,暗点头道:“这贼秃开大口,说大话,想是有些来历。”都一个个诺诺连
声。只有那喇嘛僧道:“且住!你老师父贵恙,你拿这妖精不至紧。俗语道:‘公子
登筵,不醉便饱;壮士临阵,不死即伤。’你两下里角斗之时,倘贻累你师父,不
当稳便。”行者道:“有理,有理!我且送凉水与师父吃了再来。”掇起钵盂,着上凉
水,转出香积厨,就到方丈,叫声:“师父,吃凉水哩。”

三藏正当烦渴之时,便抬起头来,捧着水,只是一吸。真个“渴时一滴如甘露,
药到真方病即除。”行者见长老精神渐爽,眉目舒开,就问道:“师父,可吃些汤饭
么?”三藏道:“这凉水就是灵丹一般,这病儿了一半,有汤饭也吃得些。”行者
连声高高叫道:“我师父好了,要汤饭吃哩。”教那些和尚忙忙的安排。淘米,煮饭,
捍面,烙饼,蒸馍馍,做粉汤,抬了四五桌。唐僧只吃得半碗儿米汤;行者、沙僧
止用了一席;其余的都是八戒一肚餐之。家火收去,点起灯来,众僧各散。

三藏道:“我们今住几日子?”行者道:“三整日矣。明朝向晚,便就是四个日
头。”三藏道:“三日误了许多路程。”行者道:“师父,也算不得路程,明日去罢。”
三藏道:“正是。就带几分病儿,也没奈何。”行者道:“既是明日要去,且让我今
晚捉了妖精者。”三藏惊道:“又捉甚么妖精?”行者道:“有个妖精在这寺里,等
老孙替他捉捉。”唐僧道:“徒弟呀,我的病身未可,你怎么又兴此念!倘那怪有神
通,你拿他不住啊,却又不是害我?”行者道:“你好灭人威风!老孙到处降妖,你
见我弱与谁的?只是不动手,动手就要赢。”三藏扯住道:“徒弟,常言说得好,‘遇
方便时行方便,得饶人处且饶人。操心怎似存心好,争气何如忍气高!’”孙大圣见
师父苦苦劝他,不许降妖,他说出老实话来道:“师父,实不瞒你说。那妖在此吃
了人了。”唐僧大惊道:“吃了甚么人?”行者说道:“我们住了三日,已是吃了这
寺里六个小和尚了。”长老道:“‘兔死狐悲,物伤其类。’他既吃了寺内之僧,我亦
僧也,我放你去;只但用心仔细些。”行者道:“不消说。老孙的手到就消除了。”

你看他灯光前吩咐八戒、沙僧看守师父;他喜孜孜跳出方丈,径来佛殿看时,
天上有星,月还未上,那殿里黑暗暗的。他就吹出真火,点起琉璃,东边打鼓,西
边撞钟。响罢,摇身一变,变做个小和尚儿,年纪只有十二三岁,披着黄绢褊衫,
白布直裰,手敲着木鱼,口里念经。等到一更时分,不见动静。二更时分,残月才
升,只听见呼呼的一阵风响。好风:

黑雾遮天暗,愁云照地昏。四方如泼墨,一派靛妆浑。先刮时扬尘播土,次后
来倒树摧林。扬尘播土星光现,倒树摧林月色昏。只刮得嫦娥紧抱梭罗树,玉兔团
团找药盆。九曜星官皆闭户,四海龙王尽掩门。庙里城隍觅小鬼,空中仙子怎腾云?
地府阎罗寻马面,判官乱跑赶头巾。刮动昆仑顶上石,卷得江湖波浪混。
那风才然过处,猛闻得兰麝香熏,环声响,即欠身抬头观看,呀!却是一个美貌
佳人,径上佛殿。

行者口里呜哩呜喇,只情念经。那女子走近前,一把搂住道:“小长老,念的
甚么经?”行者道:“许下的。”女子道:“别人都自在睡觉,你还念经怎么?”行
者道:“许下的,如何不念?”女子搂住,与他亲个嘴道:“我与你到后面耍耍去。”
行者故意的扭过头去道:“你有些不晓事!”女子道:“你会相面?”行者道:“也晓
得些儿。”女子道:“你相我怎的样子?”行者道:“我相你有些儿偷生熟,被公
婆赶出来的。”女子道:“相不着,相不着!我
不是公婆赶逐,不因熟偷生。
奈我前生命薄,投配男子年轻。
不会洞房花烛,避夫逃走之情。
趁如今星光月皎,也是有缘千里来相会,我和你到后园中交欢配鸾俦去也。”行者
闻言,暗点头道:“那几个愚僧,都被色欲引诱,所以伤了性命。他如今也来哄我。”
就随口答应道:“娘子,我出家人年纪尚幼,却不知甚么交欢之事。”女子道:“你
跟我去,我教你。”行者暗笑道:“也罢,我跟他去,看他怎生摆布。”

他两个搂着肩,携着手,出了佛殿,径至后边园里。那怪把行者使个绊子腿,
跌倒在地。口里“心肝哥哥”的乱叫,将手就去掐他的臊根。行者道:“我的儿,
真个要吃老孙哩!”却被行者接住他手,使个小坐跌法,把那怪一辘轳揪翻在地上。
那怪口里还叫道:“心肝哥哥,你倒会跌你的娘哩!”行者暗算道:“不趁此时下手
他,还到几时!正是‘先下手为强,后下手遭殃。’”就把手一叉,腰一躬,一跳跳
起来,现出原身法象,轮起金箍铁棒,劈头就打。那怪倒也吃了一惊。他心想道:
“这个小和尚,这等利害!”打开眼一看,原来是那唐长老的徒弟姓孙的。他也不
惧他。你说这精怪是甚么精怪:

金作鼻,雪铺毛。地道为门屋,安身处处牢。养成三百年前气,曾向灵山走几
遭。一饱香花和蜡烛,如来吩咐下天曹。托塔天王恩爱女,哪吒太子认同胞。也不
是个填海鸟,也不是
个戴山鳌。也不怕的雷焕剑,也不怕的吕虔刀。往往来来,一任他水流江汉阔;上
上下下,那论他山耸泰恒高?你看他月貌花容娇滴滴,谁识得是个鼠老成精逞黠豪!
他自恃的神通广大,便随手架起双股剑,玎玎的响,左遮右格,随东倒西。行
者虽强些,却也捞他不倒。阴风四起,残月无光。你看他两人,后园中一场好杀:

阴风从地起,残月荡微光。阒静梵王宇,阑珊小鬼廊。后园里一片战争场:孙
大士,天上圣;毛姹女,女中王;赌赛神通未肯降。一个儿扭转芳心嗔黑秃,一个
儿圆睁慧眼恨新妆。两手剑飞,那认得女菩萨;一根棍打,狠似个活金刚。响处金
箍如电掣,霎时铁白耀星芒。玉楼抓翡翠,金殿碎鸳鸯。猿啼巴月小,雁叫楚天长。
十八尊罗汉,暗暗喝采;三十二诸天,个个慌张。

那孙大圣精神抖擞,棍儿没半点差池。妖精自料敌他不住,猛可的眉头一蹙,
计上心来,抽身便走。行者喝道:“泼货,那走!快快来降!”那妖精只是不理,真
往后退。等行者赶到紧急之时,即将左脚上花鞋脱下来,吹口仙气,念个咒语,叫
一声“变!”就变做本身模样,使两口剑舞将来;真身一幌,化阵清风而去。这却
不是三藏的灾星?他便径撞到方丈里,把唐三藏摄将去云头上,杳杳冥冥,霎霎眼,
就到了陷空山,进了无底洞,叫小的们安排素筵席成亲不题。

却说行者斗得心焦性燥,闪一个空,一棍把那妖精打落下来,乃是一只花鞋。
行者晓得中了他计,连忙转身来看师父。那有个师父?只见那呆子和沙僧口里呜哩
呜哪说甚么。行者怒气填胸,也不管好歹,捞起棍来一片打,连声叫道:“打死你
们!打死你们!”那呆子慌得走也没路;沙僧却是个灵山大将,见得事多,就软款温
柔,近前跪下道:“兄长,我知道了。想你要打杀我两个,也不去救师父,径自回
家去哩。”行者道:“我打杀你两个,我自去救他!”沙僧笑道:“兄长说那里话!无
我两个,真是‘单丝不线,孤掌难鸣。’兄啊,这行囊、马匹,谁与看顾?宁学管鲍
分金,休仿孙庞斗智。自古道:‘打虎还得亲兄弟,上阵须教父子兵。’望兄长且饶
打,待天明和你同心戮力,寻师去也。”

行者虽是神通广大,却也明理识时。见沙僧苦苦哀告,便就回心道:“八戒,
沙僧,你都起来。明日找寻师父,却要用力。”那呆子听见饶了,恨不得天也许下
半边,道:“哥啊,这个都在老猪身上。”兄弟们思思想想,那曾得睡,恨不得点头
唤出扶桑日,一口吹散满天星。

三众只坐到天晓,收拾要行,早有寺僧拦门来问:“老爷那里去?”行者笑道:
“不好说。昨日对众夸口,说与他们拿妖精,妖精未曾拿得,倒把我个师父不见了。
我们寻师父去哩。”众僧害怕道:“老爷,小可的事,倒带累老师;却往那里去寻?”
行者道:“有处寻他。”众僧又道:“既去莫忙,且吃些早斋。”连忙的端了两三盆汤
饭。八戒尽力吃个干净,道:“好和尚!我们寻着师父,再到你这里来耍子。”行者
道:“还到这里吃他饭哩!你去天王殿里看看那女子在否。”众僧道:“老爷,不在了,
不在了。自是当晚宿了一夜,第二日就不见了。”

行者喜喜欢欢的辞了众僧,着八戒、沙僧牵马挑担,径回东走。八戒道:“哥
哥差了。怎么又往东行?”行者道:“你岂知道!前日在那黑松林绑的那个女子,老
孙火眼金睛,把他认透了,你们都认做好人。今日吃和尚的也是他,摄师父的也是
他!你们救得好女菩萨!今既摄了师父,还从旧路上找寻去也。”二人叹服道:“好,
好,好!真是粗中有细!去来,去来!”

三人急急到于林内,只见那:

云霭霭,雾漫漫;石层层,路盘盘。狐踪兔迹交加走,虎豹豺狼往复钻。林内
更无妖怪影,不知三藏在何端。
行者心焦,掣出棒来,摇身一变,变作大闹天宫的本相,三头六臂,六只手,理着
三根棒,在林里辟哩拨喇的乱打。八戒见了道:“沙僧,师兄着了恼,寻不着师父,
弄做个气心风了。”原来行者打了一路,打出两个老头儿来!一个是山神,一个是土
地。上前跪下道:“大圣,山神、土地来见。”八戒道:“好灵根啊!打了一路,打出
两个山神、土地;若再打一路,连太岁都打出来也。”行者问道:“山神、土地,汝
等这般无礼!在此处专一结伙强盗,强盗得了手,买些猪羊祭赛你,又与妖精结掳,
打伙儿把我师父摄来,如今藏在何处?快快的从实供来,免打!”二神慌了道:“大
圣错怪了我耶。妖精不在小神山上,不伏小神管辖。但只夜间风响处,小神略知一
二。”行者道:“既知,一一说来!”土地道:“那妖精摄你师父去,在那正南下,离
此有千里之遥。那厢有座山,唤做陷空山。山中有个洞,叫做无底洞。是那山里妖
精,到此变化摄去也。”行者听言,暗自惊心,喝退了山神、土地,收了法身,现
出本相与八戒、沙僧道:“师父去得远了。”八戒道:“远便腾云赶去!”

好呆子,一纵狂风先起,随后是沙僧驾云。那白马原是龙子出身,驮了行李,
也踏了风雾。大圣即起筋斗,一直南来。不多时,早见一座大山,阻住云脚。三人
采住马,都按定云头。见那山:

顶摩碧汉,峰接青霄。周围杂树万万千,来往飞禽喳喳噪。虎豹成阵走,獐鹿
打丛行。向阳处,琪花瑶草馨香;背阴方,腊雪顽冰不化。崎岖峻岭,削壁悬崖。
直立高峰,湾环深涧。松郁郁,石磷磷,行人见了悚其心。打柴樵子全无影,采药
仙童不见踪。眼前虎豹能兴雾,遍地狐狸乱弄风。
八戒道:“哥啊,这山如此峻,必有妖邪。”行者道:“不消说了。‘山高原有怪,
岭峻岂无精!’”叫:“沙僧,我和你且在此,着八戒先下山凹里打听打听看那条路
好走,端的可有洞府,再看是那里开门,俱细细打探,我们好一齐去寻师父救他。”
八戒道:“老猪晦气!先拿我顶缸!”行者道:“你夜来说都在你身上,如何打仰?”
八戒道:“不要嚷,等我去。”呆子放下钯,抖抖衣裳,空着手,跳下高山,找寻路
径。

这一去,毕竟不知好歹如何,且听下回分解。