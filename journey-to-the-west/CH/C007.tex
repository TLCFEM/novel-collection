\chapter{八卦炉中逃大圣~五行山下定心猿}

富贵功名,前缘分定,为人切莫欺心。正大光明,忠良善果弥深。些些狂妄天
加谴,眼前不遇待时临。问东君因甚,如今祸害相侵。只为心高图罔极,不分上下
乱规箴。

话表齐天大圣被众天兵押去斩妖台下,绑在降妖柱上,刀砍斧剁,枪刺剑刳,
莫想伤及其身。南斗星奋令火部众神,放火煨烧,亦不能烧着。又着雷部众神,以
雷屑钉打,越发不能伤损一毫。那大力鬼王与众启奏道:“万岁,这大圣不知是何
处学得这护身之法,臣等用刀砍斧剁,雷打火烧,一毫不能伤损,却如之何?”玉
帝闻言道:“这厮这等,这等……如何处治?”太上老君即奏道:“那猴吃了蟠桃,
饮了御酒,又盗了仙丹,我那五壶丹,有生有熟,被他都吃在肚里,运用三昧火,
煅成一块,所以浑做金钢之躯,急不能伤。不若与老道领去,放在八卦炉中,以文
武火煅炼。炼出我的丹来,他身自为灰烬矣。”玉帝闻言,即教六丁六甲,将他解
下,付与老君。老君领旨去讫。一壁厢宣二郎显圣,赏赐金花百朵,御酒百瓶,还
丹百粒,异宝明珠,锦绣等件,教与义兄弟分享。真君谢恩,回灌江口不题。

那老君到兜率宫,将大圣解去绳索,放了穿琵琶骨之器,推入八卦炉中,命看
炉的道人,架火的童子,将火扇起煅炼。原来那炉是乾、坎、艮、震、巽、离、坤、
兑八卦。他即将身钻在“巽宫”位下。巽乃风也,有风则无火。只是风搅得烟来,
把一双眼红了,弄做个老害病眼,故唤作“火眼金睛”。

真个光阴迅速,不觉七七四十九日,老君的火候俱全。忽一日,开炉取丹。那
大圣双手侮着眼,正自揉搓流涕,只听得炉头声响,猛睁睛看见光明,他就忍不住,
将身一纵,跳出丹炉,唿喇一声,蹬倒八卦炉,往外就走。慌得那架火、看炉与丁
甲一班人来扯,被他一个个都放倒,好似癫痫的白额虎,风狂的独角龙。老君赶上
抓一把,被他一,了个倒栽葱,脱身走了。即去耳中掣出如意棒,迎风幌一幌,
碗来粗细,依然拿在手中,不分好歹,却又大乱天宫,打得那九曜星闭门闭户,四
天王无影无形。好猴精!有诗为证。诗曰:
混元体正合先天,万劫千番只自然。
渺渺无为浑太乙,如如不动号初玄。
炉中久炼非铅汞,物外长生是本仙。
变化无穷还变化,三皈五戒总休言。
又诗:
一点灵光彻太虚,那条拄杖亦如之:
或长或短随人用,横竖横排任卷舒。
又诗:
猿猴道体配人心,心即猿猴意思深。
大圣齐天非假论,官封弼马是知音。
马猿合作心和意,紧缚牢拴莫外寻。
万相归真从一理,如来同契住双林。
这一番,那猴王不分上下,使铁棒东打西敌,更无一神可挡。只打到通明殿里,灵
霄殿外。幸有佑圣真君的佐使王灵官执殿。他看大圣纵横,掣金鞭近前挡住道:“泼
猴何往!有吾在此,切莫猖狂!”这大圣不由分说,举棒就打。那灵官鞭起相迎。两
个在灵霄殿前厮浑一处。好杀:

赤胆忠良名誉大,欺天诳上声名坏。一低一好幸相持,豪杰英雄同赌赛。铁棒
凶,金鞭快,正直无私怎忍耐?这个是太乙雷声应化尊,那个是齐天大圣猿猴怪。
金鞭铁棒两家能,都是神宫仙器械。今日在灵霄宝殿弄威风,各展雄才真可爱。一
个欺心要夺斗牛宫,一个竭力匡扶玄圣界。苦争不让显神通,鞭棒往来无胜败。
他两个斗在一处,胜败未分,早有佑圣真君又差将佐发文到雷府,调三十六员雷将
齐来,把大圣围在垓心,各骋凶恶鏖战。那大圣全无一毫惧色,使一条如意棒,左
遮右挡,后架前迎。一时,见那众雷将的刀枪剑戟、鞭简挝锤、钺斧金瓜、旄镰月
铲,来的甚紧,他即摇身一变,变做三头六臂;把如意棒幌一幌,变作三条;六只
手使开三条棒,好便似纺车儿一般,滴流流,在那垓心里飞舞。众雷神莫能相近。
真个是:

圆陀陀,光灼灼,亘古常存人怎学?入火不能焚,入水何曾溺?光明一颗摩尼珠,
剑戟刀枪伤不着。也能善,也能恶,眼前善恶凭他作。善时成佛与成仙,恶处披毛
并带角。无穷变化闹天宫,雷将神兵不可捉。
当时众神把大圣攒在一处,却不能近身,乱嚷乱斗,早惊动玉帝。遂传旨着游奕灵
官同翊圣真君上西方请佛老降伏。

那二圣得了旨,径到灵山胜境,雷音宝刹之前,对四金刚、八菩萨礼毕,即烦
转达。众神随至宝莲台下启知,如来召请。二圣礼佛三匝,侍立台下。如来问:“玉
帝何事,烦二圣下临?”二圣即启道:“向时花果山产一猴,在那里弄神通,聚众
猴搅乱世界。玉帝降招安旨,封为弼马温,他嫌官小反去。当遣李天王、哪吒太子
擒拿未获,复招安他,封做齐天大圣,先有官无禄。着他代管蟠桃园,他即偷桃;
又走至瑶池,偷肴偷酒,搅乱大会,仗酒又暗入兜率宫,偷老君仙丹,反出天宫。
玉帝复遣十万天兵,亦不能收伏。后观世音举二郎真君同他义兄弟追杀,他变化多
端,亏老君抛金钢琢打重,二郎方得拿住。解赴御前,即命斩之。刀砍斧剁,火烧
雷打,俱不能伤,老君奏准领去,以火煅炼。四十九日开鼎,他却又跳出八卦炉,
打退天丁,径入通明殿里,灵霄殿外;被佑圣真君的佐使王灵官挡住苦战,又调三
十六员雷将,把他困在垓心,终不能相近。事在紧急,因此,玉帝特请如来救驾。”
如来闻诏,即对众菩萨道:“汝等在此稳坐法堂,休得乱了禅位,待我炼魔救驾去
来。”

如来即唤阿傩、迦叶二尊者相随,离了雷音,径至灵霄门外。忽听得喊声振耳,
乃三十六员雷将围困着大圣哩。佛祖传法旨:“教雷将停息干戈,放开营所,叫那
大圣出来,等我问他有何法力。”众将果退。大圣也收了法象,现出原身近前,怒
气昂昂,厉声高叫道:“你是那方善士,敢来止住刀兵问我?”如来笑道:“我是西
方极乐世界释迦牟尼尊者,南无阿弥陀佛。今闻你猖狂村野,屡反天宫,不知是何
方生长,何年得道,为何这等暴横?”大圣道:“我本:

天地生成灵混仙,花果山中一老猿。水帘洞里为家业,拜友寻师悟太玄。炼就
长生多少法,学来变化广无边。因在凡间嫌地窄,立心端要住瑶天。灵霄宝殿非他
久,历代人王有分传。强者为尊该让我,英雄只此敢争先。”
佛祖听言,呵呵冷笑道:“你那厮乃是个猴子成精,焉敢欺心,要夺玉皇上帝龙位?
他自幼修持,苦历过一千七百五十劫。每劫该十二万九千六百年。你算,他该多少
年数,方能享受此无极大道?你那个初世为人的畜生,如何出此大言!不当人子!不
当人子!折了你的寿算!趁早皈依,切莫胡说!但恐遭了毒手,性命顷刻而休,可惜
了你的本来面目!”大圣道:“他虽年劫修长,也不应久占在此。常言道:‘皇帝轮
流做,明年到我家。’只教他搬出去,将天宫让与我便罢了;若还不让,定要搅攘,
永不清平!”佛祖道:“你除了长生变化之法,再有何能,敢占天宫胜境?”大圣道:
“我的手段多哩!我有七十二般变化,万劫不老长生。会驾筋斗云,一纵十万八千
里。如何坐不得天位?”佛祖道:“我与你打个赌赛:你若有本事,一筋斗打出我
这右手掌中,算你赢,再不用动刀兵苦争战,就请玉帝到西方居住,把天宫让你;
若不能打出手掌,你还下界为妖,再修几劫,却来争吵。”那大圣闻言,暗笑道:“这
如来十分好呆!我老孙一筋斗去十万八千里。他那手掌,方圆不满一尺,如何跳不
出去?”急发声道:“既如此说,你可做得主张?”佛祖道:“做得!做得!”伸开右
手,却似个荷叶大小。

那大圣收了如意棒,抖擞神威,将身一纵,站在佛祖手心里,却道声:“我出
去也!”你看他一路云光,无影无形去了。佛祖慧眼观看,见那猴王风车子一般相
似不住,只管前进。大圣行时,忽见有五根肉红柱子,撑着一股青气。他道:“此
间乃尽头路了。这番回去,如来作证,灵霄宫定是我坐也。”又思量说:“且住!等
我留下些记号,方好与如来说话。”拔下一根毫毛,吹口仙气,叫“变!”变作一管
浓墨双毫笔,在那中间柱子上写一行大字云:“齐天大圣,到此一游。”写毕,收了
毫毛。又不庄尊,却在第一根柱子根下撒了一泡猴尿。翻转筋斗云,径回本处,站
在如来掌内道:“我已去,今来了。你教玉帝让天宫与我。”

如来骂道:“我把你这个尿精猴子!你正好不曾离了我掌哩!”大圣道:“你是不
知。我去到天尽头,见五根肉红柱,撑着一股青气,我留个记在那里,你敢和我同
去看么?”如来道:“不消去,你只自低头看看。”那大圣睁圆火眼金睛,低头看时,
原来佛祖右手中指写着“齐天大圣,到此一游。”大指丫里,还有些猴尿臊气,大
圣吃了一惊道:“有这等事!有这等事!我将此字写在撑天柱子上,如何却在他手指
上?莫非有个未卜先知的法术。我决不信,不信!等我再去来!”

好大圣,急纵身又要跳出,被佛祖翻掌一扑,把这猴王推出西天门外,将五指
化作金、木、水、火、土五座联山,唤名“五行山”,轻轻的把他压住。众雷神与
阿傩、迦叶,一个个合掌称扬道:“善哉!善哉!
当年卵化学为人,立志修行果道真。
万劫无移居胜境,一朝有变散精神。
欺天罔上思高位,凌圣偷丹乱大伦。
恶贯满盈今有报,不知何日得翻身。”

如来佛祖殄灭了妖猴,即唤阿傩、迦叶同转西方极乐世界。时有天蓬、天佑急
出灵霄宝殿道:“请如来少待,我主大驾来也。”佛祖闻言,回首瞻仰。须臾,果见
八景鸾舆,九光宝盖;声奏玄歌妙乐,咏哦无量神章;散宝花,喷真香,直至佛前
谢曰:“多蒙大法收殄妖邪,望如来少停一日,请诸仙做一会筵奉谢。”如来不敢违
悖,即合掌谢道:“老僧承大天尊宣命来此,有何法力?还是天尊与众神洪福。敢劳
致谢?”玉帝传旨,即着雷部众神,分头请三清、四御、五老、六司、七元、八极、
九曜、十都、千真万圣,来此赴会,同谢佛恩。又命四大天师、九天仙女,大开玉
京金阙、太玄宝宫、洞阳玉馆,请如来高座七宝灵台,调设各班坐位,安排龙肝凤
髓,玉液蟠桃。

不一时,那玉清元始天尊、上清灵宝天尊、太清道德天尊、五真君、五斗星
君、三官四圣、九曜真君、左辅、右弼、天王、哪吒,玄虚一应灵通,对对旌旗,
双双幡盖,都捧着明珠异宝,寿果奇花,向佛前拜献曰:“感如来无量法力,收伏
妖猴。蒙大天尊设宴呼唤,我等皆来陈谢。请如来将此会立一名,如何?”如来领
众神之托曰:“今欲立名,可作个‘安天大会’。”各仙老异口同声,俱道:“好个‘安
天大会’!好个‘安天大会’!”言讫,各坐座位,走传觞,簪花鼓瑟,果好会也。
有诗为证。诗曰:
宴设蟠桃猴搅乱,安天大会胜蟠桃。
龙旗鸾辂祥光蔼,宝节幢幡瑞气飘。
仙乐玄歌音韵美,凤箫玉管响声高。
琼香缭绕群仙集,宇宙清平贺圣朝。

众皆畅然喜会,只见王母娘娘引一班仙子、仙娥、美姬、毛女,飘飘荡荡舞向
佛前,施礼曰:“前被妖猴搅乱蟠桃嘉会,请众仙众佛,俱未成功。今蒙如来大法
链锁顽猴,喜庆‘安天大会’,无物可谢,今是我净手亲摘大株蟠桃数颗奉献。”真
个是:
半红半绿喷甘香,艳丽仙根万载长。
堪笑武陵源上种,争如天府更奇强!
紫纹娇嫩寰中少,缃核清甜世莫双。
延寿延年能易体,有缘食者自非常。
佛祖合掌向王母谢讫。王母又着仙姬、仙子唱的唱,舞的舞。满会群仙,又皆赏赞。
正是:
缥缈天香满座,缤纷仙蕊仙花。
玉京金阙大荣华,异品奇珍无价。
对对与天齐寿,双双万劫增加。
桑田沧海任更差,他自无惊无讶。
王母正着仙姬仙子歌舞,觥筹交错,不多时,忽又闻得:

一阵异香来鼻噢,惊动满堂星与宿。天仙佛祖把杯停,各各抬头迎目候。霄汉
中间现老人,手捧灵芝飞蔼绣。葫芦藏蓄万年丹,宝名书千纪寿。洞里乾坤任自
由,壶中日月随成就。遨游四海乐清闲,散淡十洲容辐辏。曾赴蟠桃醉几遭,醒时
明月还依旧。长头大耳短身躯,南极之方称老寿。
寿星又到。见玉帝礼毕,又见如来,申谢曰:“始闻那妖猴被老君引至兜率宫煅炼,
以为必致平安,不期他又反出。幸如来善伏此怪,设宴奉谢,故此闻风而来。更无
他物可献,特具紫芝瑶草,碧藕金丹奉上。”诗曰:
碧藕金丹奉释迦,如来万寿若恒沙。
清平永乐三乘锦,康泰长生九品花。
无相门中真法主,色空天上是仙家。
乾坤大地皆称祖,丈六金身福寿赊。
如来欣然领谢。寿星得座,依然走传觞。只见赤脚大仙又至。向玉帝前囟礼毕,
又对佛祖谢道:“深感法力,降伏妖猴。无物可以表敬,特具交梨二颗,火枣数枚
奉献。”诗曰:
大仙赤脚枣梨春,敬献弥陀寿算长。
七宝莲台山样稳,千金花座锦般妆。
寿同天地言非谬,福比洪波话岂狂。
福寿如期真个是,清闲极乐那西方。

如来又称谢了。叫阿傩、迦叶,将各所献之物,一一收起,方向玉帝前谢宴。
众各酩酊。只见个巡视灵官来报道:“那大圣伸出头来了。”佛祖道:“不妨,不妨。”
袖中只取出一张帖子,上有六个金字:“嘛呢叭”。递与阿傩,叫贴在那山顶
上。这尊者即领帖子,拿出天门,到那五行山顶上,紧紧的贴在一块四方石上。那
座山即生根合缝,可运用呼吸之气,手儿爬出,可以摇挣摇挣。阿傩回报道:“已
将帖子贴了。”

如来即辞了玉帝众神,与二尊者出天门之外,又发一个慈悲心,念动真言咒语,
将五行山,召一尊土地神祇,会同五方揭谛,居住此山监押。但他饥时,与他铁丸
子吃;喝时,与他溶化的铜汁饮。待他灾愆满日,自有人救他。正是:
妖猴大胆反天宫,却被如来伏手降。
渴饮溶铜捱岁月,饥餐铁弹度时光。
天灾苦困遭磨折,人事凄凉喜命长。
若得英雄重展挣,他年奉佛上西方。
又诗曰:
伏逞豪强大势兴,降龙伏虎弄乖能。
偷桃偷酒游天府,受承恩在玉京。
恶贯满盈身受困,善根不绝气还升。
果然脱得如来手,且待唐朝出圣僧。

毕竟不知向后何年何月,方满灾殃,且听下回分解。