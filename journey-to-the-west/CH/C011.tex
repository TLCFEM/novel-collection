\chapter{还受生唐王遵善果~度孤魂萧正空门}

诗曰:
百岁光阴似水流,一生事业等浮沤。
昨朝面上桃花色,今日头边雪片浮。
白蚁阵残方是幻,子规声切想回头。
古来阴能延寿,善不求怜天自周。

却说唐太宗随着崔判官、朱太尉,自脱了冤家债主,前进多时,却来到“六道
轮回”之所,又见那腾云的,身披霞帔;受的,腰挂金鱼;僧尼道俗,走兽飞禽,
魑魅魍魉,滔滔都奔走那轮回之下,各进其道。唐王问曰:“此意何如?”判官道:
“陛下明心见性,是必记了,传与阳间人知。这唤做‘六道轮回’:行善的,升化
仙道;尽忠的,超生贵道;行孝的,再生福道;公平的,还生人道;积德的,转生
富道;恶毒的,沉沦鬼道。”唐王听说,点头叹曰:
善哉真善哉!作善果无灾!
善心常切切,善道大开开。
莫教兴恶念,是必少刁乖。
休言不报应,神鬼有安排。”
判官送唐王直至那“超生贵道门”,拜呼唐王道:“陛下呵,此间乃出头之处,小
判告回,着朱太尉再送一程。”唐王谢道:“有劳先生远。”判官道:“陛下到
阳间,千万做个水陆大会,超度那无主的冤魂,切勿忘了。若是阴司里无报怨之声,
阳世间方得享太平之庆。凡百不善之处,俱可一一改过。普谕世人为善,管教你后
代绵长,江山永固。”唐王一一准奏,辞了崔判官,随着朱太尉,同入门来。那太
尉见门里有一匹海骝马,鞍齐备,急请唐王上马,太尉左右扶持。马行如箭,早
到了渭水河边,只见那水面上有一对金色鲤鱼在河里翻波跳斗。唐王见了心喜,兜
马贪看不舍。太尉道:“陛下,趱动些,趁早赶时辰进城去也。”那唐王只管贪看,
不肯前行,被太尉撮着脚,高呼道:“还不走,等甚!”扑的一声,望那渭河推下
马去,却就脱了阴司,径回阳世。

却说那唐朝驾下有徐茂功、秦叔宝、胡敬德、段志贤、马三宝、程咬金、高士
廉、李世、房玄龄、杜如晦、萧、傅奕、张道源、张士衡、王等两班文武,
俱保着那东宫太子与皇后、嫔妃、宫娥、侍长,都在那白虎殿上举哀。一壁厢议传
哀诏,要晓谕天子,欲扶太子登基。时有魏征在旁道:“列位且住。不可!不可!假
若惊动州县,恐生不测。且再按候一日,我主必还魂也。”下边闪上许敬宗道:“魏
丞相言之甚谬。自古云:‘泼水难收,人逝不返。’你怎么还说这等虚言,惑乱人
心,是何道理!”魏征道:“不瞒许先生说,下官自幼得授仙术,推算最明,管取
陛下不死。”

正讲处,只听得棺中连声大叫道:“杀我耶!杀我耶!”唬得个文官武将心
慌,皇后嫔妃胆战。一个个:

面如秋后黄桑叶,腰似春前嫩柳条。储君脚软,难扶丧杖尽哀仪;侍长魂飞,
怎戴梁冠遵孝礼?嫔妃打跌,彩女欹斜:嫔妃打跌,却如狂风吹倒败芙蓉;彩女欹斜,
好似骤雨冲歪娇菡萏。众臣悚惧,骨软筋麻。战战兢兢,痴痴痖痖。把一座白虎殿
却像断梁桥;闹丧台就如倒塌寺。
此时众宫人走得精光,那个敢近灵扶柩。多亏了正直的徐茂功,凛烈的魏丞相,有
胆量的秦琼,忒猛撞的敬德,上前来扶着棺材,叫道:“陛下有甚么放不下心处,
说与我等,不要弄鬼,惊骇了眷族。”魏征道:“不是弄鬼,此乃陛下还魂也。快
取器械来!”打开棺盖,果见太宗坐在里面。还叫“死我了!是谁救捞?”茂功等
上前扶起道:“陛下苏醒莫怕,臣等都在此护驾哩。”唐王方才开眼道:“朕适才
好苦:躲过阴司恶鬼难,又遭水面丧身灾。”众臣道:“陛下宽心勿惧,有甚水灾
来?”唐王道:“朕骑着马,正行至渭水河边,见双头鱼戏;被朱太尉欺心,将朕
推下马来,跌落河中,几乎死。”魏征道:“陛下鬼气尚未解。”急着太医院进
安神定魄汤药,又安排粥膳。连服一二次,方才反本还原,知得人事。——计唐王
死去,已三昼夜,复回阳间为君。诗曰:
万古江山几变更,历来数代败和成。
周秦汉晋多奇事,谁似唐王死复生?
当日天色已晚,众臣请王归寝,各各散讫。次早,脱却孝衣,换了彩服,一个个红
袍乌帽,一个个紫绶金章,在那朝门外等候宣召。

却说太宗自服了安神定魄之剂,连进了数次粥汤,被众臣扶入寝室,一夜稳睡,
保养精神,直至天明方起,抖擞威仪,你看他怎生打扮:

戴一顶冲天冠,穿一领赭黄袍。系一条蓝田碧玉带,踏一对创业无忧履。貌堂
堂,赛过当朝;威烈烈,重兴今日。好一个清平有道的大唐王,起死回生的李陛下。
唐王上金銮宝殿,聚集两班文武,山呼已毕,依品分班。只听得传旨道:“有事出
班来奏,无事退朝。”那东厢闪过徐茂功、魏征、王、杜如晦、房玄龄、袁天罡、
李淳风、许敬宗等;西厢闪过殷开山、刘洪基、马三宝、段志贤、程咬金、秦叔宝、
胡敬德、薛仁贵等:一齐上前,在白玉阶前,俯伏启奏道:“陛下前朝一梦,如何
许久方觉?”太宗道:“日前接得魏征书,朕觉神魂出殿,只见羽林军请朕出猎。
正行时,人马无踪,又见那先君父王与先兄弟争嚷。正难解处,见一人乌帽皂袍,
乃是判官崔,喝退先兄弟。朕将魏征书传递与他。正看时,又见青衣者,执幢幡,
引朕入内,到森罗殿上,与十代阎王叙坐。他说那泾河龙诬告我许救转杀之事,是
朕将前言陈具一遍。他说已三曹对过案了,急命取生死文簿,检看我的阳寿。时有
崔判官传上簿子。阎王看了道,寡人有三十三年天禄,才过得一十三年,还该我二
十年阳寿,即着朱太尉、崔判官,送朕回来。朕与十王作别,允了送他瓜果谢恩。
自出了森罗殿,见那阴司里,不忠不孝,非礼非义,作践五谷,明欺暗骗,大斗小
秤,奸盗诈伪,淫邪欺罔之徒,受那些磨烧舂锉之苦,煎熬吊剥之刑,有千千万万,
看之不足。又过着枉死城中,有无数的冤魂,尽都是六十四处烟尘的草寇,七十二
处叛贼的魂灵,挡住了朕之来路。幸亏崔判官作保,借得河南相老儿的金银一库,
买转鬼魂,方得前行。崔判官教朕回阳世,千万作一场水陆大会,超度那无主的孤
魂,将此言叮咛分别。出了那六道轮回之下,有朱太尉请朕上马。飞也相似行到渭
水河边,我看见那水面上有双头鱼戏。正欢喜处,他将我撮着脚,推下水中,朕方
得还魂也。”众臣闻此言,无不称贺,遂此编行传报,天下各府县官员,上表称庆
不题。

却说太宗又传旨赦天下罪人,又查狱中重犯。时有审官将刑部绞斩罪人,查有
四百余名呈上。太宗放赦回家,拜辞父母兄弟,托产与亲戚子侄,明年今日赴曹,
仍领应得之罪。众犯谢恩而退。又出恤孤榜文,又查宫中老幼彩女共有三千人,出
旨配军。自此,内外俱善。有诗为证,诗曰:
大国唐王恩德洪,道过尧舜万民丰。
死囚四百皆离狱,怨女三千放出宫。
天下多官称上寿,朝中众宰贺元龙。
善心一念天应佑,福荫应传十七宗。
太宗既放宫女,出死囚已毕;又出御制榜文,遍传天下。榜曰:

乾坤浩大,日月照鉴分明;宇宙宽洪,天地不容奸党。使心用术,果报只在今
生;善布浅求,获福休言后世。千般巧计,不如本分为人;万种强徒,怎似随缘节
俭。心行慈善,何须努力看经?意欲损人,空读如来一藏!

自此时,盖天下无一人不行善者。一壁厢又出招贤榜,招人进瓜果到阴司里去;
一壁厢将宝藏库金银一库,差鄂国公胡敬德上河南开封府,访相良还债。榜张数日,
有一赴命进瓜果的贤者,本是均州人,姓刘名全,家有万贯之资。只因妻李翠莲在
门首拔金钗斋僧,刘全骂了他几句,说他不遵妇道,擅出闺门。李氏忍气不过,自
缢而死。撇下一双儿女年幼,昼夜悲啼。刘全又不忍见,无奈,遂舍了性命,弃了
家缘,撇了儿女,情愿以死进瓜,将皇榜揭了,来见唐王。王传旨意,教他去金亭
馆里,头顶一对南瓜,袖带黄钱,口噙药物。

那刘全果服毒而死——一点魂灵,顶着瓜果,早到鬼门关上。把门的鬼使喝道:
“你是甚人,敢来此处?”刘全道:“我奉大唐太宗皇帝钦差,特进瓜果与十代阎
王受用的。”那鬼使欣然接引。刘全径至森罗宝殿,见了阎王,将瓜果进上道:“奉
唐王旨意,远进瓜果,以谢十王宽宥之恩。”阎王大喜道:“好一个有信有德的太
宗皇帝!”遂此收了瓜果。便问那进瓜的人姓名,那方人氏。刘全道:“小人是均
州城民籍,姓刘名全。因妻李氏缢死,撇下儿女,无人看管,小人情愿舍家弃子,
捐躯报国,特与我王进贡瓜果,谢众大王厚恩。”十王闻言,即命查勘刘全妻李氏。
那鬼使速取来在森罗殿下,与刘全夫妻相会。诉罢前言,回谢十王恩宥。那阎王却
检生死簿子看时,他夫妻们都有登仙之寿,急差鬼使送回。鬼使启上道:“李翠莲
归阴日久,尸首无存,魂将何附?”阎王道:“唐御妹李玉英,今该促死;你可借
他尸首,教他还魂去也。”那鬼使领命,即将刘全夫妻二人还魂。带定出了阴司,
那阴风绕绕,径到了长安大国,将刘全的魂灵,推入金亭馆里;将翠莲的灵魂,带
进皇宫内院。只见那玉英宫主,正在花阴下,徐步绿苔而行,被鬼使扑个满怀,推
倒在地,活捉了他魂;却将翠莲的魂灵,推入玉英身内。鬼使回转阴司不题。

却说宫院中的大小侍婢,见玉英跌死,急走金銮殿,报与三宫皇后道:“宫主
娘娘跌死也!”皇后大惊,随报太宗。太宗闻言,点头叹曰:“此事信有之也。朕
曾问十代阎君:‘老幼安乎?’他道:‘俱安;但恐御妹寿促。’果中其言。”合
宫人都来悲切,尽到花阴下看时,只见那宫主微微有气。唐王道:“莫哭!莫哭!休
惊了他。”遂上前将御手扶起头来,叫道:“御妹苏醒苏醒。”那宫主忽的翻身,
叫:“丈夫慢行,等我一等!”太宗道:“御妹,是我等在此。”宫主抬头睁眼观
看道:“你是谁人,敢来扯我?”太宗道:“是你皇兄、皇嫂。”宫主道:“我那
里得个甚么皇兄、皇嫂!我娘家姓李,我的乳名唤做李翠莲;我丈夫姓刘名全。两口
儿都是均州人氏。因为我三个月前,拔金钗在门首斋僧,我丈夫怪我擅出内门,不
遵妇道,骂了我几句,是我气塞胸堂,将白绫带悬梁缢死,撇下一双儿女,昼夜悲
啼。今因我丈夫被唐王钦差,赴阴司进瓜果,阎王怜悯,放我夫妻回来。他在前走。
因我来迟,赶不上他,我绊了一跌。你等无礼!不知姓名,怎敢扯我!”太宗闻言,
与众宫人道:“想是御妹跌昏了,胡说哩。”传旨教太医院进汤药,将玉英扶入宫
中。

唐王当殿,忽有当驾官奏道:“万岁,今有进瓜果人刘全还魂,在朝门外等旨。”
唐王大惊,急传旨,将刘全召进,俯伏丹墀。太宗问道:“进瓜果之事何如?”刘
全道:“臣顶瓜果,径至鬼门关,引上森罗殿,见了那十代阎君,将瓜果奉上,备
言我王殷勤致谢之意。阎君甚喜,多多拜上我王道:‘真是个有信有德的太宗皇帝!’”
唐王道:“你在阴司见些甚么来?”刘全道:“臣不曾远行,没见甚的,只闻得阎
王问臣乡贯、姓名。臣将弃家舍子,因妻缢死,愿来进瓜之事,说了一遍。他急差
鬼使,引过我妻,就在森罗殿下相会。一壁厢又检看死生文簿,说我夫妻都有登仙
之寿,便差鬼使送回。臣在前走,我妻后行,幸得还魂。但不知妻投何所。”唐王
惊问道:“那阎王可曾说你妻甚么?”刘全道:“阎王不曾说甚么,只听得鬼使说:
‘李翠莲归阴日久,尸首无存。’阎王道:‘唐御妹李玉英今该促死,教翠莲即借
玉英尸还魂去罢。’臣不知‘唐御妹’是甚地方,家居何处,我还未曾得去找寻哩。”

唐王闻奏,满心欢喜,当对多官道:“朕别阎君,曾问宫中之事;他言老幼俱
安,但恐御妹寿促。却才御妹玉英,花阴下跌死,朕急扶看,须臾苏醒,口叫‘丈
夫慢行,等我一等!’朕只道是他跌昏了胡言。又问他详细,他说的话,与刘全一
般。”魏征奏道:“御妹偶尔寿促,少苏醒即说此言,此是刘全妻借尸还魂之事。
此事也有。可请宫主出来,看他有甚话说。”唐王道:“朕才命太医院去进药,不
知何如。”便教妃嫔入宫去请。那宫主在里面乱嚷道:“我吃甚么药!这里那是我家!
我家是清凉瓦屋,不像这个害黄病的房子,花狸狐哨的门扇!放我出去!放我出去!”

正嚷处,只见四五个女官,两三个太监,扶着他直至殿上。唐王道:“你可认
得你丈夫么?”玉英道:“说那里话,我两个从小儿的结发夫妻,与他生男长女,
怎的不认得?”唐王叫内官搀他下去。那宫主下了宝殿,直至白玉阶前,见了刘全,
一把扯住道:“丈夫,你往那里去,就不等我一等!我跌了一跤,被那些没道理的人
围住我嚷,这是怎的说!”那刘全听他说的话是妻之言,观其人非妻之面,不敢相
认。唐王道:“这正是山崩地裂有人见,捉生替死却难逢!”好一个有道的君王:
即将御妹的妆奁、衣物、首饰,尽赏赐了刘全,就如陪嫁一般。又赐与他永免差徭
的御旨,着他带领御妹回去。他夫妻两个,便在阶前谢了恩,欢欢喜喜还乡。有诗
为证:
人生人死是前缘,短短长长各有年。
刘全进瓜回阳世,借尸还魂李翠莲。
他两个辞了君王,径来均州城里,见旧家业儿女俱好,两口儿宣扬善果不题。

却说那尉迟公将金银一库,上河南开封府访看相良,原来卖水为活,同妻张氏
在门首贩卖乌盆瓦器营生,但赚得些钱儿,只以盘缠为足,其多少斋僧布施,买金
银纸锭,记库焚烧,故有此善果臻身。阳世间是一条好善的穷汉,那世里却是个积
玉堆金的长者。尉迟公将金银送上他门,唬得那相公、相婆魂飞魄散;又兼有本府
官员,茅舍外车马骈集,那老两口子如痴如哑,跪在地下,只是磕头礼拜。尉迟公
道:“老人家请起。我虽是个钦差官,却赍着我王的金银送来还你。”他战兢兢的
答道:“小的没有甚么金银放债,如何敢受这不明之财?”尉迟公道:“我也妨得
你是个穷汉;只是你斋僧布施,尽其所用,就买办金银纸锭,烧记阴司,阴司里有
你积下的钱钞。是我太宗皇帝死去三日,还魂复生,曾在那阴司里借了你一库金银,
今此照数送还与你。你可一一收下,等我好去回旨。”那相良两口儿只是朝天礼拜,
那里敢受。道:“小的若受了这些金银,就死得快了。虽然是烧纸记库,此乃冥冥
之事;况万岁爷爷那世里借了金银,有何凭据?我决不敢受。”尉迟公道:“陛下说,
借你的东西,有崔判官作保可证。你收下罢。”相良道:“就死也是不敢受的。”
尉迟公见他苦苦推辞,只得具本差人启奏。太宗见了本,知相良不受金银。道:“此
诚为善良长者!”即传旨教胡敬德将金银与他修理寺院,起盖生祠,请僧作善,就
当还他一般。旨意到日,敬德望阙谢恩,宣旨众皆知之。遂将金银买到城里军民无
碍的地基一段,周围有五十亩宽阔,在上兴工,起盖寺院,名“敕建相国寺”。左
有相公相婆的生祠,镌碑刻石,上写着“尉迟公监造”。即今大相国寺是也。

工完回奏,太宗甚喜。却又聚集多官,出榜招僧,修建“水陆大会”,超度冥
府孤魂。榜行天下,着各处官员推选有道的高僧,上长安做会。那消个月之期,天
下多僧俱到。唐王传旨,着太史丞傅奕选举高僧,修建佛事。傅奕闻旨,即上疏止
浮图,以言无佛。表曰:

西域之法,无君臣父子,以三涂六道,蒙诱愚蠢,追既往之罪,窥将来之福;
口诵梵言,以图偷免。且生死寿夭,本诸自然;刑德威福,系之人主。今闻俗徒矫
托,皆云由佛。自五帝、三王,未有佛法;君明臣忠,年祚长久。至汉明帝始立胡
神,然惟西域桑门,自传其教。实乃夷犯中国,不足为信。
太宗闻言,遂将此表掷付群臣议之。

时有宰相萧,出班俯囟奏曰:“佛法兴自屡朝,弘善遏恶,冥助国家,理无
废弃。佛,圣人也。非圣者无法,请置严刑。”傅奕与萧论辨,言礼本于事亲事
君,而佛背亲出家,以匹夫抗天子,以继体悖所亲;萧不生于空桑,乃遵无父之
教,正所谓非孝者无亲。萧但合掌曰:“地狱之设,正为是人。”太宗召太仆卿
张道源,中书令张士衡,问佛事营福,其应何如。二臣对曰:“佛在清净仁恕,果
正佛空。周武帝以三教分次:大慧禅师有赞幽远,历众供养而无不显;五祖投胎,
达摩现象。自古以来,皆云三教至尊而不可毁,不可废。伏乞陛下圣鉴明裁。”太
宗甚喜道:“卿之言合理。再有所陈者,罪之。”遂着魏征与萧、张道源,邀请
诸佛,选举一名有大德行者作坛主,设建道场。众皆顿首谢恩而退。自此时出了法
律:但有毁僧谤佛者,断其臂。

次日,三位朝臣,聚众僧,在那山川坛里,逐一从头查选。内中选得一名有德
行的高僧。你道他是谁人?

灵通本讳号金蝉:只为无心听佛讲,转托尘凡苦受磨,降生世俗遭罗网。投胎
落地就逢凶,未出之前临恶党。父是海州陈状元,外公总管当朝长。出身命犯落江
星,顺水随波逐浪
泱。海岛金山有大缘,迁安和尚将他养。年方十八认亲娘,特赴京都求外长。总管
开山调大军,洪州剿寇诛凶党。状元光蕊脱天罗,子父相逢堪贺奖。复谒当今受主
恩,凌烟阁上贤名响。恩官不受愿为僧,洪福沙门将道访。小字江流古佛儿,法名
唤做陈玄奘。
当日对众举出玄奘法师。这个人自幼为僧,出娘胎,就持斋受戒。他外公见是当朝
一路总管殷开山。他父亲陈光蕊,中状元,官拜文渊殿大学士。一心不爱荣华,只
喜修持寂灭。查得他根源又好,德行又高;千经万典,无所不通;佛号仙音,无般
不会。当时三位引至御前,扬尘舞蹈。拜罢奏曰:“臣等,蒙圣旨,选得高僧一
名陈玄奘。”太宗闻其名,沉思良久道:“可是学士陈光蕊之儿玄奘否?”江流儿
叩头曰:“臣正是。”太宗喜道:“果然举之不错。诚为有德行有禅心的和尚。朕
赐你左僧纲,右僧纲,天下大阐都僧纲之职。”玄奘顿首谢恩,受了大阐官爵。又
赐五彩织金袈裟一件,毗卢帽一顶。教他用心再拜明僧,排次黎班首;书办旨意,
前赴化生寺,择定吉日良时,开演经法。

玄奘再拜领旨而出,遂到化生寺里,聚集多僧,打造禅榻,装修功德,整理音
乐。选得大小明僧共计一千二百名,分派上中下三堂。诸所佛前,物件皆齐,头头
有次。选到本年九月初三日,黄道良辰,开启做七七四十九日“水陆大会”。即具
表申奏,太宗及文武国戚皇亲,俱至期赴会,拈香听讲。

毕竟不知圣意如何,且听下回分解。