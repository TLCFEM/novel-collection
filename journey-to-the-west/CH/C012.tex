\chapter{玄奘秉诚建大会~观音显象化金蝉}

诗曰:
龙集贞观正十三,王宣大众把经谈。
道场开演无量法,云雾光乘大愿龛。
御敕垂恩修上刹,金蝉脱壳化西涵。
普施善果超沉没,秉教宣扬前后三。

贞观十三年,岁次己巳,九月甲戌,初三日,癸卯良辰。陈玄奘大阐法师,聚
集一千二百名高僧,都在长安城化生寺开演诸品妙经。那皇帝早朝已毕,帅文武多
官,乘凤辇龙车,出离金銮宝殿,径上寺来拈香。怎见那銮驾?真个是:

一天瑞气,万道祥光。仁风轻淡荡,化日丽非常。千官环佩分前后,五卫旌旗
列两旁。执金瓜,擎斧钺,双双对对;绛纱烛,御炉香,霭霭堂堂。龙飞凤舞,鹗
荐鹰扬。圣明天子正,忠义大臣良。介福千年过舜禹,升平万代赛尧汤。又见那曲
柄伞,滚龙袍,辉光相射;玉连环,彩凤扇,瑞霭飘扬。珠冠玉带,紫绶金章。护
驾军千队,扶舆将两行。这皇帝沐浴虔诚尊敬佛,皈依善果喜拈香。
唐王大驾,早到寺前。吩咐住了音乐响器。下了车辇,引着多官,拜佛拈香。三匝
已毕,抬头观看,果然好座道场。但见:

幢幡飘舞,宝盖飞辉:幢幡飘舞,凝空道道彩霞摇;宝盖飞辉,映日翩翩红电
彻。世尊金像貌臻臻,罗汉玉容威烈烈。瓶插仙花,炉焚檀降:瓶插仙花,锦树辉
辉漫宝刹;炉焚檀降,香云霭霭透清霄。时新果品砌朱盘,奇样糖酥堆彩案。高僧
罗列诵真经,愿拔孤魂离苦难。
太宗文武俱各拈香,拜了佛祖金身,参了罗汉。又见那大阐都纲陈玄奘法师引众僧
罗拜唐王。礼毕,分班各安禅位。法师献上济孤榜文与太宗看。榜曰:

至德渺茫,禅宗寂灭。清净灵通,周流三界。千变万化,统摄阴阳。体用真常,
无穷极矣。观彼孤魂,深宜哀愍。此奉太宗圣命,选集诸僧,参禅讲法。大开方便
门庭,广运慈悲舟楫,普济苦海群生,脱免沉疴六趣。引归真路,普玩鸿蒙;动止
无为,混成纯素。仗此良因,邀赏清都绛阙;乘吾胜会,脱离地狱凡笼,早登极乐
任逍遥,来往西方随自在。

诗曰:
一炉永寿香,几卷超生。
无边妙法宣,无际天恩沐。
冤孽尽消除,孤魂皆出狱。
愿保我邦家,清平万咸福。
太宗看了,满心欢喜。对众僧道:“汝等秉立丹衷,切休怠慢佛事。待后功成完备,
各各福有所归,朕当重赏,决不空劳。”那一千二百僧,一齐顿首称谢。当日三斋
已毕,唐王驾回。待七日正会,复请拈香。时天色将晚,各官俱退。怎见得好晚?
你看那:
万里长空淡落辉,归鸦数点下栖迟。
满城灯火人烟静,正是禅僧入定时。
一宿晚景题过。次早,法师又升坐,聚众诵经不题。

却说南海普陀山观世音菩萨,自领了如来佛旨,在长安城访察取经的善人,日
久未逢真实有德行者。忽闻得太宗宣扬善果,选举高僧,开建大会,又见得法师坛
主,乃是江流儿和尚,正是极乐中降来的佛子,又是他原引送投胎的长老,菩萨十
分欢喜,就将佛赐的宝贝,捧上长街,与木叉货卖。你道他是何宝贝?有一件锦
异宝袈裟、九环锡杖。还有那金、紧、禁三个箍儿,密密藏收,以俟后用。只将袈
裟、锡杖出卖。长安城里,有那选不中的愚僧,倒有几贯村钞。见菩萨变化个疥癞
形容,身穿破衲,赤脚光头,将袈裟捧定,艳艳生光,他上前问道:“那癞和尚,
你的袈裟要卖多少价钱?”菩萨道:“袈裟价值五千两,锡杖价值二千两。”那愚僧
笑道:“这两个癞和尚是疯子!是傻子!这两件粗物,就卖得七千两银子,只是除非
穿上身长生不老,就得成佛作祖,也值不得这许多!拿了去!卖不成!”那菩萨更不
争吵,与木叉往前又走。行勾多时,来到东华门前,正撞着宰相萧散朝而回,众
头踏喝开街道。那菩萨公然不避,当街上拿着袈裟,径迎着宰相。宰相勒马观看,
见袈裟艳艳生光,着手下人问那卖袈裟的要价几何。菩萨道:“袈裟要五千两,锡
杖要二千两。”萧道:“有何好处,值这般高价?”菩萨道:“袈裟有好处,有不好
处;有要钱处,有不要钱处。”萧道:“何为好?何为不好?”菩萨道:“着了我袈
裟,不入沉沦,不堕地狱,不遭恶毒之难,不遇虎狼之灾,便是好处;若贪淫乐祸
的愚僧,不斋不戒的和尚,毁经谤佛的凡夫,难见我袈裟之面,这便是不好处。”
又问道:“何为要钱,不要钱?”菩萨道:“不遵佛法,不敬三宝,强买袈裟、锡杖,
定要卖他七千两,这便是要钱;若敬重三宝,见善随喜,皈依我佛,承受得起,我
将袈裟、锡杖,情愿送他,与我结个善缘,这便是不要钱。”萧闻言,倍添春色,
知他是个好人。即便下马,与菩萨以礼相见。口称“大法长老,恕我萧之罪。我
大唐皇帝十分好善,满朝的文武,无不奉行。即今起建‘水陆大会’,这袈裟正好
与大都阐陈玄奘法师穿用。我和你入朝见驾去来。”

菩萨欣然从之,拽转步,径进东华门里。黄门官转奏,蒙旨宣至宝殿。见萧
引着两个疥癞僧人,立于阶下,唐王问曰:“萧来奏何事?”萧俯伏阶前道:“臣
出了东华门前,偶遇二僧,乃卖袈裟与锡杖者。臣思法师玄奘可着此服,故领僧人
启见。”太宗大喜,便问那袈裟价值几何。菩萨与木叉侍立阶下,更不行礼,因问
袈裟之价,答道:“袈裟五千两,锡杖二千两。”太宗道:“那袈裟有何好处,就值
许多?”菩萨道:

“这袈裟,龙披一缕,免大鹏吞噬之灾;鹤挂一丝,得超凡入圣之妙。但坐处,
有万神朝礼;凡举动,有七佛随身。

这袈裟是冰蚕造练抽丝,巧匠翻腾为线。仙娥织就,神女机成,方方簇幅绣花
缝,片片相帮堆锦。玲珑散碎斗妆花,色亮飘光喷宝艳。穿上满身红雾绕,脱来
一段彩云飞。三天门外透玄光,五岳山前生宝气。重重嵌就西番莲,灼灼悬珠星斗
象。四角上有夜明珠,攒顶间一颗祖母绿。虽无全照原本体,也有生光八宝攒。

这袈裟,闲时折迭,遇圣才穿。闲时折迭,千层包裹透虹霓;遇圣才穿,惊动
诸天神鬼怕。上边有如意珠、摩尼珠、辟尘珠、定风珠;又有那红玛瑙、紫珊瑚、
夜明珠、舍利子。偷月沁白,与日争红。条条仙气盈空,朵朵祥光捧圣:条条仙气
盈空,照彻了天关;朵朵祥光捧圣,影遍了世界。照山川,惊虎豹;影海岛,动鱼
龙。沿边两道销金锁,叩领连环白玉琮。

诗曰:
三宝巍巍道可尊,四生六道尽评论。
明心解养人天法,见性能传智慧灯。
护体庄严金世界,身心清净玉壶冰。
自从佛制袈裟后,万劫谁能敢断僧?”

唐王在那宝殿上闻言,十分欢喜。又问:“那和尚,九环杖有甚好处?”菩萨
道:“我这锡杖,是那:
铜镶铁造九连环,九节仙藤永驻颜。
入手厌看青骨瘦,下山轻带白云还。
摩呵五祖游天阙,罗卜寻娘破地关。
不染红尘些子秽,喜伴神僧上玉山。”
唐王闻言,即命展开袈裟,从头细看,果然是件好物。道:“大法长老,实不瞒你。
朕今大开善教,广种福田,见在那化生寺聚集多僧,敷演经法。内中有一个大有德
行者,法名玄奘。朕买你这两件宝物,赐他受用。你端的要价几何?”菩萨闻言,
与木叉合掌皈依,道声佛号,躬身上启道:“既有德行,贫僧情愿送他,决不要钱。”
说罢,抽身便走。唐王急着萧扯住,欠身立于殿上,问曰:“你原说袈裟五千两,
锡杖二千两,你见朕要买,就不要钱,敢是说朕心倚恃君位,强要你的物件?——
更无此理。朕照你原价奉偿,却不可推避。”菩萨起手道:“贫僧有愿在前,原说果
有敬重三宝,见善随喜,皈依我佛,不要钱,愿送与他。今见陛下明德止善,敬我
佛门,况又高僧有德有行,宣扬大法,理当奉上,决不要钱。贫僧愿留下此物告回。”
唐王见他这等勤恳,甚喜。随命光禄寺,大排素宴酬谢。菩萨又坚辞不受,畅然而
去。依旧望都土地庙中,隐避不题。

却说太宗设午朝,着魏征赍旨,宣玄奘入朝。那法师正聚众登坛,讽经诵偈,
一闻有旨,随下坛整衣,与魏征同往见驾。太宗道:“求证善事,有劳法师,无物
酬谢。早间萧迎着二僧,愿送锦异宝袈裟一件,九环锡杖一条。今特召法师领
去受用。”玄奘叩头谢恩。太宗道:“法师如不弃,可穿上与朕看看。”长老遂将袈
裟抖开,披在身上,手持锡杖,侍立阶前。君臣个个欣然。诚为如来佛子。你看他:

凛凛威颜多雅秀,佛衣可体如裁就。辉光艳艳满乾坤,结彩纷纷凝宇宙。朗朗
明珠上下排,层层金线穿前后。兜罗四面锦沿边,万样稀奇铺绮绣。八宝妆花缚钮
丝,金环束领攀绒扣。佛天大小列高低,星象尊卑分左右。玄奘法师大有缘,现前
此物堪承受。浑如极乐活阿罗,赛过西方真觉秀。锡杖叮当斗九环,毗卢帽映多丰
厚。诚为佛子不虚传,胜似菩提无诈谬。

当时文武阶前喝采,太宗喜之不胜。即着法师穿了袈裟,持了宝杖;又赐两队
仪从,着多官送出朝门,教他上大街行道,往寺里去,就如中状元夸官的一般。这
去玄奘再拜谢恩,在那大街上,烈烈轰轰,摇摇摆摆。你看那长安城里,行商坐贾、
公子王孙、墨客文人、大男小女,无不争看夸奖,俱道:“好个法师!真是个活罗汉
下降,活菩萨临凡。”玄奘直至寺里,僧人下榻来迎。一见他披此袈裟,执此锡杖,
都道是地藏王来了,各各归依,侍于左右。玄奘上殿,炷香礼佛,又对众感述圣恩
已毕,各归禅座。又不觉红轮西坠。正是那:

日落烟迷草树,帝都钟鼓初鸣。叮叮三响断人行,前后街前寂静。

上刹辉煌灯火,孤村冷落无声。禅僧入定理残经,正好炼魔养性。

光阴拈指,却当七日正会。玄奘又具表,请唐王拈香。此时善声遍满天下。太
宗即排驾,率文武多官,后妃国戚,早赴寺里。那一城人,无论大小尊卑,俱诣寺
听讲。当有菩萨与木叉道:“今日是水陆正会,以一七继七七,可矣了。我和你杂
在众人丛中,一则看他那会何如,二则看金蝉子可有福穿我的宝贝,三则也听他讲
的是那一门经法。”两人随投寺里。正是有缘得遇旧相识,般若还归本道场。入到
寺里观看,真个是天朝大国,果胜裟婆;赛过祇园舍卫,也不亚上刹招提。那一派
仙音响亮,佛号喧哗,这菩萨直至多宝台边,果然是明智金蝉之相。诗曰:
万象澄明绝点埃,大典玄奘坐高台。
超生孤魂暗中到,听法高流市上来。
施物应机心路远,出生随意藏门开。
对看讲出无量法,老幼人人放喜怀。
又诗曰:
因游法界讲堂中,逢见相知不俗同。
尽说目前千万事,又谈尘劫许多功。
法云容曳舒群岳,教网张罗满太空。
检点人生归善念,纷纷天雨落花红。
那法师在台上,念一会《受生度亡经》,谈一会《安邦天宝篆》,又宣一会《劝修功
卷》。这菩萨近前来,拍着宝台,厉声高叫道:“那和尚,你只会谈‘小乘教法’,
可会谈‘大乘’么?”玄奘闻言,心中大喜,翻身跳下台来,对菩萨起手道:“老
师父,弟子失瞻,多罪。见前的盖众僧人,都讲的是‘小乘教法’,却不知‘大乘
教法’如何。”菩萨道:“你这小乘教法,度不得亡者超升,只可浑俗和光而已;我
有大乘佛法三藏,能超亡者升天,能度难人脱苦,能修无量寿身,能作无来无去。”

正讲处,有那司香巡堂官急奏唐王道:“法师正讲谈妙法,被两个疥癞游僧,
扯下来乱说胡话。”王令擒来,只见许多人将二僧推拥进后法堂。见了太宗,那僧
人手也不起,拜也不拜,仰面道:“陛下问我何事?”唐王却认得他,道:“你是前
日送袈裟的和尚?”菩萨道:“正是。”太宗道:“你既来此处听讲,只该吃些斋便
了,为何与我法师乱讲,扰乱经堂,误我佛事?”菩萨道:“你那法师讲的是小乘
教法,度不得亡者升天。我有大乘佛法三藏,可以度亡脱苦,寿身无坏。”太宗正
色喜问道:“你那大乘佛法,在于何处?”菩萨道:“在大西天天竺国大雷音寺我佛
如来处,能解百冤之结,能消无妄之灾。”太宗道:“你可记得么?”菩萨道:“我
记得。”太宗大喜道:“教法师引去,请上台开讲。”

那菩萨带了木叉,飞上高台,遂踏祥云,直至九霄,现出救苦原身,托了净瓶
杨柳。左边是木叉惠岸,执着棍,抖擞精神。喜的个唐王朝天礼拜,众文武跪地焚
香。满寺中僧尼道俗,士人工贾,无一人不拜祷道:“好菩萨!好菩萨!”有调为证。
但见那:

瑞霭散缤纷,祥光护法身。九霄华汉里,现出女真人。那菩萨,头上戴一顶—
—金叶纽,翠花铺,放金光,生锐气的垂珠缨络;身上穿一领——淡淡色,浅浅妆,
盘金龙,飞彩凤的结素蓝袍;胸前挂一面——对月明,舞清风,杂宝珠,攒翠玉的
砌香环;腰间系一条——冰蚕丝,织金边,登彩云,促瑶海的锦绣绒裙;面前又
领一个——飞东洋,游普世,感恩行孝,黄毛红嘴白鹦哥;手内托着一个——施恩
济世的宝瓶,瓶内插着一枝——洒青霄,撒大恶,扫开残雾垂杨柳。玉环穿绣扣,
金莲足下深。三天许出入,这才是救苦救难观世音。

喜的个唐太宗,忘了江山;爱的那文武官,失却朝礼;盖众多人,都念“南无
观世音菩萨”。太宗即传旨,教巧手丹青,描下菩萨真象。旨意一声,选出个图神
写圣远见高明的吴道子。此人即后图功臣于凌烟阁者,当时展开妙笔,图写真形。
那菩萨祥云渐远,霎时间不见了金光。只见那半空中,滴溜溜落下一张简帖,上有
几句颂子,写得明白。颂曰:

礼上大唐君,西方有妙文。程途十万八千里,大乘进殷勤。此经回上国,能超
鬼出群。若有肯去者,求正果金身。

太宗见了颂子,即命众僧:“且收胜会,待我差人取得大乘经来,再秉丹诚,
重修善果。”众官无不遵依。

当时在寺中问曰:“谁肯领朕旨意,上西天拜佛求经?”问不了,旁边闪过法
师,帝前施礼道:“贫僧不才,愿效犬马之劳,与陛下求取真经,祈保我王江山永
固。”唐王大喜,上前将御手扶起道:“法师果能尽此忠贤,不怕程途遥远,跋涉山
川,朕情愿与你拜为兄弟。”玄奘顿首谢恩。唐王果是十分贤德,就去那寺里佛前,
与玄奘拜了四拜,口称“御弟圣僧”。玄奘感谢不尽道:“陛下,贫僧有何德何能,
敢蒙天恩眷顾如此?我这一去,定要捐躯努力,直至西天;如不到西天,不得真经,
即死也不敢回国,永堕沉沦地狱。”随在佛前拈香,以此为誓。唐王甚喜,即命回
銮,待选良利日辰,发牒出行,遂此驾回各散。

玄奘亦回洪福寺里。那本寺多僧与几个徒弟,早闻取经之事,都来相见。因问:
“发誓愿上西天,实否?”玄奘道:“是实。”他徒弟道:“师父呵,尝闻人言,西
天路远,更多虎豹妖魔;只怕有去无回,难保身命。”玄奘道:“我已发了弘誓大愿,
不取真经,永堕沉沦地狱。大抵是受王恩宠,不得不尽忠以报国耳。我此去真是渺
渺茫茫,吉凶难定。”又道:“徒弟们,我去之后,或三二年,或五七年,但看那山
门里松枝头向东,我即回来;不然,断不回矣。”众徒将此言切切而记。

次早,太宗设朝,聚集文武,写了取经文牒,用了通行宝印。有钦天监奏曰:
“今日是人专吉星,堪宜出行远路。”唐王大喜。又见黄门官奏道:“御弟法师朝门
外候旨。”随即宣上宝殿道:“御弟,今日是出行吉日。这是通关文牒。朕又有一个
紫金钵盂,送你途中化斋而用。再选两个长行的从者,又银的马一匹,送为远行
脚力。你可就此行程。”玄奘大喜,即便谢了恩,领了物事,更无留滞之意。唐王
排驾,与多官同送至关外,只见那洪福寺僧与诸徒将玄奘的冬夏衣服,俱送在关外
相等。唐王见了,先教收拾行囊、马匹,然后着官人执壶酌酒。太宗举爵,又问曰:
“御弟雅号甚称?”玄奘道:“贫僧出家人,未敢称号。”太宗道:“当时菩萨说,
西天有经三藏。御弟可指经取号,号作‘三藏’何如?”玄奘又谢恩,接了御酒道:
“陛下,酒乃僧家头一戒,贫僧自为人,不会饮酒。”太宗道:“今日之行,比他事
不同。此乃素酒,只饮此一杯,以尽朕奉饯之意。”三藏不敢不受。接了酒,方待
要饮,只见太宗低头,将御指拾一撮尘土,弹入酒中。三藏不解其意。太宗笑道:
“御弟呵,这一去,到西天,几时可回?”三藏道:“只在三年,径回上国。”太宗
道:“日久年深,山遥路远,御弟可进此酒:宁恋本乡一捻土,莫爱他乡万两金。”
三藏方悟捻土之意,复谢恩饮尽,辞谢出关而去。唐王驾回。

毕竟不知此去何如,且听下回分解。