\chapter{心猿正处诸缘伏~劈破傍门见月明}

却说孙行者按落云头,对师父备言菩萨借童子,老君收去宝贝之事。三藏称谢
不已,死心塌地,办虔诚,舍命投西。攀鞍上马,猪八戒挑着行李,沙和尚拢着马
头,孙行者执了铁棒,剖开路,径下高山前进。说不尽那水宿风餐,披霜冒露。师
徒们行罢多时,前又一山阻路。

三藏在那马上高叫:“徒弟啊,你看那里山势崔巍,须是要仔细提防,恐又有
魔障侵身也。”行者道:“师父休要胡思乱想,只要定性存神,自然无事。”三藏道:
“徒弟呀,西天怎么这等难行?我记得离了长安城。在路上春尽夏来,秋残冬至,
有四五个年头,怎么还不能得到?”行者闻言,呵呵笑道:“早哩,早哩,还不曾
出大门哩!”八戒道:“哥哥不要扯谎。人间就有这般大门?”行者道:“兄弟,我
们还在堂屋里转哩!”沙僧笑道:“师兄,少说大话吓我。那里就有这般大堂屋,却
也没处买这般大过梁啊。”行者道:“兄弟,若依老孙看时,把这青天为屋瓦,日月
作窗棂;四山五岳为梁柱,天地犹如一敞厅!”八戒听说道:“罢了,罢了,我们只
当转些时回去罢!”行者道:“不必乱谈,只管跟着老孙走路。”

好大圣,横担了铁棒,领定了唐僧,剖开山路,一直前进。那师父在马上遥观,
好一座山景。真个是:

山顶嵯峨摩斗柄,树梢仿佛接云霄。青烟堆里,时闻得谷口猿啼;乱翠阴中,
每听得松间鹤唳。啸风山魅立溪间,戏弄
樵夫;成器狐狸坐崖畔,惊张猎户。好山!看那八面崔巍,四围峻。古怪乔松盘
翠盖,枯摧老树挂藤萝。泉水飞流,寒气透人毛发冷;巅峰屹,清风射眼梦魂惊。
时听大虫哮吼,每闻山鸟时鸣。麂鹿成群穿荆棘,往来跳跃;獐结党寻野食,前
后奔跑。伫立草坡,一望并无客旅;行来深凹,四边俱有豺狼。应非佛祖修行处,
尽是飞禽走兽场。
那师父战战兢兢,进此深山,心中凄惨,兜住马,叫声“悟空啊!我,
自从益智登山盟,王不留行送出城。
路上相逢三棱子,途中催趱马兜铃。
寻坡转涧求荆芥,迈岭登山拜茯苓。
防己一身如竹沥,茴香何日拜朝廷?”

孙大圣闻言,呵呵冷笑道:“师父不必挂念,少要心焦。且自放心前进,还你
个‘功到自然成’也。”师徒们玩着山景,信步行时,早不觉红轮西坠。正是:
十里长亭无客走,九重天上现星辰。
八河船只皆收港,七千州县尽关门。
六宫五府回官宰,四海三江罢钓纶。
两座楼头钟鼓响,一轮明月满乾坤。

那长老在马上遥观,只见那山凹里有楼台叠叠,殿阁重重。三藏道:“徒弟,
此时天色已晚,幸得那壁厢有楼阁不远,想必是庵观寺院,我们都到那里借宿一宵,
明日再行罢。”行者道:“师父说得是。不要忙,等我且看好歹如何。”那大圣跳在
空中,仔细观看,果然是座山门。但见:

八字砖墙泥红粉,两边门上钉金钉。叠叠楼台藏岭畔,层
层宫阙隐山中。万佛阁对如来殿,朝阳楼应大雄门。七层塔屯云宿雾,三尊佛神现
光荣。文殊台对伽蓝舍,弥勒殿靠大慈厅。看山楼外青光舞,步虚阁上紫云生。松
关竹院依依绿,方丈禅堂处处清。雅雅幽幽供乐事,川川道道喜回迎。参禅处有禅
僧讲,演乐房多乐器鸣。妙高台上昙花坠,说法坛前贝叶生。正是那林遮三宝地,
山拥梵王宫。半壁灯烟光闪灼,一行香霭雾朦胧。
孙大圣按下云头,报与三藏道:“师父,果然是一座寺院,却好借宿,我们去来。”

这长老放开马,一直前来,径到了山门之外。行者道:“师父,这一座是甚么
寺?”三藏道:“我的马蹄才然停住,脚尖还未出镫,就问我是甚么寺,好没分晓!”
行者道:“你老人家自幼为僧,须曾讲过儒书,方才去演经法;文理皆通,然后受
唐王的恩宥,门上有那般大字,如何不认得?”长老骂道:“泼猢狲,说话无知!我
才面西催马,被那太阳影射,奈何门虽有字,又被尘垢朦胧,所以未曾看见。”行
者闻言,把腰儿躬一躬,长了二丈余高,用手展去灰尘道:“师父,请看。”上有五
个大字,乃是“敕建宝林寺”。行者收了法身。道:“师父,这寺里谁进去借宿?”
三藏道:“我进去。你们的嘴脸丑陋,言语粗疏,性刚气傲,倘或冲撞了本处僧人,
不容借宿,反为不美。”行者道:“既如此,请师父进去,不必多言。”

那长老却丢了锡杖,解下斗篷,整衣合掌,径入山门。只见两边红漆栏杆里面,
高坐着一对金刚,装塑的威仪恶丑:

一个铁面钢须似活容,一个燥眉圜眼若玲珑。左边的拳头骨突如生铁,右边的
手掌赛赤铜。金甲连环光灿烂,明盔绣带映飘风。西方真个多供佛,石鼎中间
香火红。
三藏见了,点头长叹道:“我那东土,若有人也将泥胎塑这等大菩萨,烧香供养啊,
我弟子也不往西天去矣。”

正叹息处,又到了二层山门之内。见有四大天王之相,乃是持国、多闻、增长、
广目,按东北西南风调雨顺之意。进了二层门里,又见有乔松四树,一树树翠盖蓬
蓬,却如伞状。忽抬头,乃是大雄宝殿。那长老合掌皈依,舒身下拜。拜罢起来,
转过佛台,到于后门之下。又见有倒座观音普度南海之相。那壁上都是良工巧匠装
塑的那些虾、鱼、蟹、鳖,出头露尾,跳海水波潮耍子。长老又点头三五度,感叹
万千声道:“可怜啊!鳞甲众生都拜佛,为人何不肯修行!”

正赞叹间,又见三门里走出一个道人。那道人忽见三藏相貌稀奇,丰姿非俗,
急趋步上前施礼道:“师父那里来的?”三藏道:“弟子是东土大唐驾下差来,上西
天拜佛求经的。今到宝方,天色将晚,告借一宿。”那道人道:“师父莫怪,我做不
得主。我是这里扫地撞钟打勤劳的道人。里面还有个管家的老师父哩,待我进去禀
他一声。他若留你,我就出来奉请;若不留你,我却不敢羁迟。”三藏道:“累及你
了。”

那道人急到方丈报道:“老爷,外面有个人来了。”那僧官即起身,换了衣服,
按一按毗卢帽,披上袈裟,急开门迎接。问道人:“那里人来?”道人用手指定道:
“那正殿后边不是一个人?”那三藏光着一个头,穿一领二十五条达摩衣,足下登
一双拖泥带水的达公鞋,斜倚在那后门首。僧官见了,大怒道:“道人少打!你岂不
知我是僧官,但只有城上来的士夫降香,我方出来迎接。这等个和尚,你怎么多虚
少实,报我接他!看他那嘴脸,不是个诚实的,多是云游方上僧,今日天晚,想是
要来借宿。我们方丈中,岂容他打搅!教他往前廊下蹲罢了,报我怎么!”抽身转去。

长老闻言,满眼垂泪道:“可怜!可怜!这才是‘人离乡贱’!我弟子从小儿出家,
做了和尚,又不曾拜忏吃荤生歹意,看经怀怒坏禅心;又不曾丢瓦抛砖伤佛殿,阿
罗脸上剥真金。噫,可怜啊!不知是那世里触伤天地,教我今生常遇不良人!和尚,
你不留我们宿便罢了,怎么又说这等惫懒话,教我们在前道廊下去‘蹲’?此话不
与行者说还好,若说了,那猴子进来,一顿铁棒,把孤拐都打断你的!”长老道:“也
罢,也罢。常言道:‘人将礼乐为先。’我且进去问他一声,看意下如何。”

那师父踏脚迹,跟他进方丈门里。只见那僧官脱了衣服,气呼呼的坐在那里,
不知是念经,又不知是与人家写法事,见那桌案上有些纸札堆积。唐僧不敢深入,
就立于天井里,躬身高叫道:“老院主,弟子问讯了!”那和尚就有些不耐烦他进里
边来的意思,半答不答的还了个礼,道:“你是那里来的?”三藏道:“弟子乃东土
大唐驾下差来,上西天拜活佛求经的。经过宝方,天晚,求借一宿,明日不犯天光
就行了。万望老院主方便,方便。”那僧官才欠起身来道:“你是那唐三藏么?”三
藏道:“不敢,弟子便是。”僧官道:“你既往西天取经,怎么路也不会走?”三藏
道:“弟子更不曾走贵处的路。”他道:“正西去,只有四五里远近,有一座三十里
店,店上有卖饭的人家,方便好宿。我这里不便,不好留你们远来的僧。”三藏合
掌道:“院主,古人有云:‘庵观寺院,都是我方上人的馆驿,见山门就有三升米分。’
你怎么不留我,却是何情?”僧官怒声叫道:“你这游方的和尚,便是有些油嘴油
舌的说话!”三藏道:“何为油嘴油舌?”僧官道:“古人云:‘老虎进了城,家家都
闭门。虽然不咬人,日前坏了名。’”三藏道:“怎么‘日前坏了名’?”他道:“向
年有几众行脚僧,来于山门口坐下,是我见他寒薄,一个个衣破鞋无,光头赤脚,
我叹他那般褴褛,即忙请入方丈,延之上坐;款待了斋饭,又将故衣各借一件与他,
就留他住了几日。怎知他贪图自在衣食,更不思量起身,就住了七八个年头。住便
也罢,又干出许多不公的事来。”三藏道:“有甚么不公的事?”僧官道:“你听我
说:

闲时沿墙抛瓦,闷来壁上扳钉。冷天向火折窗棂,夏日拖
门拦径。布扯为脚带,牙香偷换蔓菁。常将琉璃把油倾,夺碗夺锅赌胜。”

三藏听言,心中暗道:“可怜啊!我弟子可是那等样没脊骨的和尚?”欲待要哭,
又恐那寺里的老和尚笑他;但暗暗扯衣揩泪,忍气吞声,急走出去,见了三个徒弟。
那行者见师父面上含怒,向前问:“师父,寺里和尚打你来?”唐僧道:“不曾打。”
八戒说:“一定打来。不是,怎么还有些哭包声?”那行者道:“骂你来?”唐僧道:
“也不曾骂。”行者道:“既不曾打,又不曾骂,你这般苦恼怎么?好道是思乡哩?”
唐僧道:“徒弟,他这里不方便。”行者笑道:“这里想是道士?”唐僧怒道:“观里
才有道士,寺里只是和尚。”行者道:“你不济事;但是和尚,即与我们一般。常言
道:‘既在佛会下,都是有缘人。’你且坐,等我进去看看。”

好行者,按一按顶上金箍,束一束腰间裙子,执着铁棒,径到大雄宝殿上,指
着那三尊佛像道:“你本是泥塑金装假象,内里岂无感应?我老孙保领大唐圣僧往西
天拜佛求取真经,今晚特来此处投宿,趁早与我报名!假若不留我等,就一顿棍打
碎金身,教你还现本相泥土!”

这大圣正在前边发狠,捣叉子乱说。只见一个烧晚香的道人,点了几枝香,来
佛前炉里插;被行者咄的一声,唬了一跌;爬起来看见脸,又是一跌;吓得滚滚
,跑入方丈里,报道:“老爷!外面有个和尚来了!”那僧官道:“你这伙道人都少
打!一行说教他往前廊下去‘蹲’,又报甚么!再说打二十!”道人说:“老爷,这个
和尚,比那个和尚不同:生得恶躁,没脊骨。”僧官道:“怎的模样?”道人道:“是
个圆眼睛,查耳朵,满面毛,雷公嘴。手执一根棍子,咬牙恨恨的,要寻人打哩。”
僧官道:“等我出去看。”

他即开门,只见行者撞进来了。真个生得丑陋:七高八低孤拐脸,两只黄眼睛,
一个磕额头;獠牙往外生,就像属螃蟹的,肉在里面,骨在外面。那老和尚慌得把
方丈门关了。行者赶上,扑的打破门扇,道:“赶早将干净房子打扫一千间,老孙
睡觉!”僧官躲在房里,对道人说:“怪他生得丑么?原来是说大话,折作的这般嘴
脸。我这里连方丈、佛殿、钟鼓楼、两廊,共总也不上三百间,他却要一千间睡觉。
却打那里来?”道人说:“师父,我也是吓破胆的人了,凭你怎么答应他罢。”那僧
官战索索的高叫道:“那借宿的长老,我这小荒山不方便,不敢奉留,往别处去宿
罢。”

行者将棍子变得盆来粗细,直壁壁的竖在天井里,道:“和尚,不方便,你就
搬出去!”僧官道:“我们从小儿住的寺,师公传与师父,师父传与我辈,我辈要远
继儿孙。他不知是那里勾当,冒冒实实的,教我们搬哩。”道人说:“老爷,十分不
,搬出去也罢。扛子打进门来了。”僧官道:“你莫胡说!我们老少众大四五百
名和尚,往那里搬?搬出去,却也没处住。”行者听见道:“和尚,没处搬,便着一
个出来打样棍!”老和尚叫:“道人你出去与我打个样棍来。”那道人慌了道:“爷爷
呀!那等个大扛子,教我去打样棍!”老和尚道:“‘养军千日,用军一朝’。你怎么
不出去?”道人说:“那扛子莫说打来,若倒下来,压也压个肉泥!”老和尚道:“也
莫要说压,只道竖在天井里,夜晚间走路,不记得啊,一头也撞个大窟窿!”道人
说:“师父,你晓得这般重,却教我出去打甚么样棍?”他自家里面转闹起来。

行者听见道:“是也禁不得。假若就一棍打杀一个,我师父又怪我行凶了。且
等我另寻一个甚么打与你看看。”忽抬头,只见方丈门外有一个石狮子,却就举起
棍来,乒乓一下,打得粉乱麻碎。那和尚在窗眼儿里看见,就吓得骨软筋麻,慌忙
往床下拱;道人就往锅门里钻;口中不住叫:“爷爷!棍重,棍重!禁不得!方便,方
便!”

行者道:“和尚,我不打你。我问你:这寺里有多少和尚?”僧官战索索的道:
“前后是二百八十五房头,共有五百个有度牒的和尚。”行者道:“你快去把那五百
个和尚都点得齐齐整整,穿了长衣服出去,把我那唐朝的师父接进来,就不打你了。”
僧官道:“爷爷,若是不打,便抬也抬进来。”行者道:“趁早去!”僧官叫:“道人,
你莫说吓破了胆,就是吓破了心,便也去与我叫这些人来接唐僧老爷爷来。”

那道人没奈何,舍了性命,不敢撞门,从后边狗洞里钻将出去,径到正殿上,
东边打鼓,西边撞钟。钟鼓一齐响处,惊动了两廊大小僧众,上殿问道:“这早还
不晚哩,撞钟打鼓做甚?”道人说:“快换衣服,随老师父排班,出山门外迎接唐
朝来的老爷。”那众和尚,真个齐齐整整,摆班出门迎接。有的披了袈裟,有的着
了偏衫,无的穿着个一口钟直裰。十分穷的,没有长衣服,就把腰裙接起两条披在
身上。行者看见道:“和尚,你穿的是甚么衣服?”和尚见他丑恶,道:“爷爷,不
要打,等我说。这是我们城中化的布。此间没有裁缝,是自家做的个‘一裹穷’。”

行者闻言暗笑,押着众僧,出山门下跪下。那僧官磕头高叫道:“唐老爷,请
方丈里坐。”八戒看见道:“师父老大不济事。你进去时,泪汪汪,嘴上挂得油瓶。
师兄怎么就有此獐智,教他们磕头来接?”三藏道:“你这个呆子,好不晓礼!常言
道:‘鬼也怕恶人哩。’”

唐僧见他们磕头礼拜,甚是不过意。上前叫:“列位请起。”众僧叩头道:“老
爷,若和你徒弟说声方便,不动扛子,就跪一个月也罢。”唐僧叫:“悟空,莫要打
他。”行者道:“不曾打,若打,这会已打断了根矣。”那些和尚却才起身,牵马的
牵马,挑担的挑担,抬着唐僧,驮着八戒,挽着沙僧,一齐都进山门里去。却到后
面方丈中,依叙坐下。

众僧却又礼拜。三藏道:“院主请起,再不必行礼,作践贫僧。我和你都是佛
门弟子。”僧官道:“老爷是上国钦差,小和尚有失迎接。今到荒山,奈何俗眼不识
尊仪,与老爷邂逅相逢。动问老爷:一路上是吃素?是吃荤?我们好去办饭。”三藏
道:“吃素。”僧官道:“徒弟,这个爷爷好的吃荤。”行者道:“我们也吃素。都是
胎里素。”那和尚道:“爷爷呀,这等凶汉也吃素!”有一个胆量大的和尚,近前又
问:“老爷既然吃素,煮多少米的饭方彀吃?”八戒道:“小家子和尚!问甚么!一家
煮上一石米。”那和尚都慌了,便去刷洗锅灶,各房中安排茶饭。高掌明灯,调开
桌椅,管待唐僧。

师徒们都吃罢了晚斋,众僧收拾了家火,三藏称谢道:“老院主,打搅宝山了。”
僧官道:“不敢,不敢。怠慢,怠慢。”三藏道:
“我师徒却在那里安歇?”僧官道:“老爷不要忙,小和尚自有区处。”叫:“道人,
那壁厢有几个人听使令的?”道人说:“师父,有。”僧官吩咐道:“你们着两个去
安排草料,与唐老爷喂马;着几个去前面把那三间禅堂,打扫干净,铺设床帐,快
请老爷安歇。”

那些道人听命,各各整顿齐备。却来请唐老爷安寝。他师徒们牵马挑担,出方
丈,径至禅堂门首看处,只见那里面灯火光明,两梢间铺着四张藤屉床。行者见了,
唤那办草料的道人,将草料抬来,放在禅堂里面,拴下白马,教道人都出去。三藏
坐在中间。灯下,两班儿,立五百个和尚,都伺候着,不敢侧离。三藏欠身道:“列
位请回,贫僧好自在安寝也。”众僧决不敢退。僧官上前,吩咐大众:“伏侍老爷安
置了再回。”三藏道:“即此就是安置了,都就请回。”众人却才敢散,去讫。

唐僧举步出门小解,只见明月当天,叫“徒弟”。行者、八戒、沙僧都出来侍
立。因感这月清光皎洁,玉宇深沉,真是一轮高照,大地分明。对月怀归,口占一
首古风长篇。诗云:

皓魄当空宝镜悬,山河摇影十分全。琼楼玉宇清光满,冰鉴银盘爽气旋。万里
此时同皎洁,一年今夜最明鲜。浑如霜饼离沧海,却似冰轮挂碧天。别馆寒窗孤客
闷,山村野店老翁眠。乍临汉苑惊秋鬓,才到秦楼促晚奁。庾亮有诗传晋史,袁
宏不寐泛江船。光浮杯面寒无力,清映庭中健有仙。处处窗轩吟白雪,家家院宇弄
冰弦。今宵静玩来山寺,何日相同返故园?
行者闻言,近前答曰:“师父啊,你只知月色光华,心怀故里,更不知月中之意,
乃先天法象之规绳也。月至三十日,阳魂之金散尽,阴魄之水盈轮,故纯黑而无光,
乃曰‘晦’。此时与日相交,在晦朔两日之间,感阳光而有孕。至初三日一阳现,
初八日二阳生,魄中魂半,其平如绳,故曰‘上弦’。至今十五日,三阳备足,是
以团圆,故曰‘望’。至十六日一阴生,二十二日二阴生,此时魂中魄半,其平如
绳,故曰‘下弦’。至三十日三阴备足,亦当晦。此乃先天采炼之意。我等若能温
养二八,九九成功,那时节,见佛容易,返故田亦易也。诗曰:
前弦之后后弦前,药味平平气象全。
采得归来炉里炼,志心功果即西天。”
那长老听说,一时解悟,明彻真言。满心欢喜,称谢了悟空。沙僧在旁笑道:“师
兄此言虽当,只说的是弦前属阳,弦后属阴,阴中阳半,得水之金;更不道:
水火相搀各有缘,全凭土母配如然。
三家同会无争竞,水在长江月在天。”
那长老闻得,亦开茅塞。正是理明一窍通千窍,说破无生即是仙。八戒上前扯住长
老道:“师父,莫听乱讲,误了睡觉。这月啊:
缺之不久又团圆,似我生来不十全。
吃饭嫌我肚子大,拿碗又说有粘涎。
他都伶俐修来福,我自痴愚积下缘。
我说你取经还满三涂业,摆尾摇头直上天!”

三藏道:“也罢,徒弟们走路辛苦,先去睡下。等我把这卷经来念一念。”行者
道:“师父差了。你自幼出家,做了和尚,小时的经文,那本不熟?却又领了唐王旨
意,上西天见佛,求取‘大乘真典’。如今功未完成,佛未得见,经未曾取,你念
的是那卷经儿?”三藏道:“我自出长安,朝朝跋涉,日日奔波,小时的经文恐怕
生了;幸今夜得闲,等我温习温习。”行者道:“既这等说,我们先去睡也。”他三
人各往一张藤床上睡下。长老掩上禅堂门,高剔银缸,铺开经本,默默看念。正是
那:
楼头初鼓人烟静,野浦渔舟火灭时。

毕竟不知那长老怎么样离寺,且听下回分解。