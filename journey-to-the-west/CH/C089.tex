\chapter{黄狮精虚设钉钯宴~金木土计闹豹头山}

却说那院中几个铁匠,因连日辛苦,夜间俱自睡了。及天明起来打造,篷下不
见了三般兵器,一个个呆挣神惊,四下寻找。只见那三个王子出宫来看,那铁匠一
齐磕头道:“小主啊,神师的三般兵器,都不知那里去了!”

小王子听言,心惊胆战道:“想是师父今夜收拾去了。”急奔暴纱亭看时,见白
马尚在廊下,忍不住叫道:“师父还睡哩!”沙僧道:“起来了。”即将房门开了,让
王子进里看时,不见兵器,慌慌张张问道:“师父的兵器都收来了?”行者跳起道:
“不曾收啊!”王子道:“三般兵器,今夜都不见了。”八戒连忙爬起道:“我的钯在
么?”小王道:“适才我等出来,只见众人前后找寻不见,弟子恐是师父收了,却
才来问。老师的宝贝,俱是能长能消,想必藏在身边哄弟子哩。”行者道:“委的未
收。都寻去来。”

随至院中篷下,果然不见踪影。八戒道:“定是这伙铁匠偷了!快拿出来!略迟
了些儿,就都打死!打死!”那铁匠慌得磕头滴泪道:“爷爷!我们连日辛苦,夜间睡
着,及至天明起来,遂不见了。我等乃一概凡人,怎么拿得动,望爷爷饶命!饶命!”
行者无语,暗恨道:“还是我们的不是。既然看了式样,就该收在身边,怎么却丢
放在此!那宝贝霞彩光生,想是惊动甚么歹人,今夜窃去也。”八戒不信道:“哥哥
说那里话!这般个太平境界,又不是旷野深山,怎得个歹人来!定是铁匠欺心,他见
我们的兵器光彩,认得是三件宝贝,连夜走出王府,伙些人来,抬的抬,拉的拉,
偷出去了!拿过来打呀!打呀!”众匠只是磕头发誓。

正嚷处,只见老王子出来,问及前事,却也面无人色,沉吟半晌,道:“神师
兵器,本不同凡,就有百十余人也禁挫不动;况孤在此城,今已五代,不是大胆海
口,孤也颇有个贤名在外;这城中军民匠作人等,也颇惧孤之法度,断是不敢欺心。
望神师再思可矣。”

行者笑道:“不用再思,也不须苦赖铁匠。我问殿下:你这州城四面,可有甚
么山林妖怪?”王子道:“神师此问,甚是有理。孤这州城之北,有一座豹头山。
山中有一座虎口洞。往往人言洞内有仙,又言有虎狼,又言有妖怪。孤未曾访得端
的,不知果是何物。”行者笑道:“不消讲了,定是那方歹人,知道俱是宝贝,一夜
偷将去了。”叫:“八戒、沙僧,你都在此保着师父,护着城池,等老孙寻访去来。”
又叫铁匠们不可住了炉火,一一炼造。

好猴王,辞了三藏,唿哨一声,形影不见。早跨到豹头山上。原来那城相去只
有三十里,一瞬即到。径上山峰观看,果然有些妖气。真是:

龙脉悠长,地形远大。尖峰挺挺插天高,陡涧沉沉流水急。山前有瑶草铺茵,
山后有奇花布锦。乔松老柏,古树修篁。山鸦山鹊乱飞鸣,野鹤野猿皆啸唳。悬崖
下,麋鹿双双;峭壁前,獾狐对对。一起一伏远来龙,九曲九湾潜地脉。埂头相接
玉华州,万古千秋兴胜处。
行者正然看时,忽听得山背后有人言语,急回头视之,乃两个狼头妖怪,朗朗的说
着话,向西北上走。行者揣道:“这定是巡山的怪物,等老孙跟他去听听,看他说
些甚的。”

捻着诀,念个咒,摇身一变,变做个蝴蝶儿,展开翅,翩翩翻翻,径自赶上。
果然变得有样范:

一双粉翅,两道银须。乘风飞去急,映日舞来徐。渡水过
墙能疾俏,偷香弄絮甚欢娱。体轻偏爱鲜花味,雅态芳情任卷舒。
他飞在那个妖精头直上,飘飘荡荡,听他说话。那妖猛的叫道:“二哥,我大王连
日侥幸:前月里得了一个美人儿,在洞内盘桓,十分快乐。昨夜里又得了三般兵器,
果然是无价之宝。明朝开宴庆‘钉钯会’哩。我们都有受用。”这个道:“我们也有
些侥幸:拿这二十两银子买猪羊去。如今到了乾方集上,先吃几壶酒儿。把东西开
个花帐儿,落他二三两银子,买件绵衣过寒,却不是好?”两个怪说说笑笑的,上
大路急走如飞。

行者听得要庆钉钯会,心中暗喜;欲要打杀他,争奈不管他事;况手中又无兵
器。他即飞向前边,现了本相,在路口上立定。那怪看看走到身边,被他一口法唾
喷将去,念一声“咤”,即使个定身法,把两个狼头精定住。眼睁睁,口也
难开;直挺挺,双脚站住。又将他扳翻倒,揭衣搜捡,果是有二十两银子,着一条
搭包儿打在腰间裙带上,又各挂着一个粉漆牌儿,一个上写着“刁钻古怪”,一个
上写着“古怪刁钻”。

好大圣,取了他银子,解了他牌儿,返跨步回至州城。到王府中,见了王子、
唐僧并大小官员、匠作人等,具言前事。八戒笑道:“想是老猪的宝贝,霞彩光明,
所以买猪羊,治筵席庆贺哩。但如今怎得他来?”行者道:“我兄弟三人俱去。这
银子是买办猪羊的,且将这银子赏了匠人,教殿下寻几个猪羊。八戒,你变做刁钻
古怪,我变做古怪刁钻,沙僧装做个贩猪羊的客人,走进那虎口洞里,得便处,各
人拿了兵器,打绝那妖邪,回来却收拾走路。”沙僧笑道:“妙,妙,妙!不宜迟,
快走!”老王果依此计,即教管事的买办了七八口猪,四五腔羊。

他三人辞了师父,在城外大显神通。八戒道:“哥哥,我未曾看见那刁钻古怪,
怎生变得他模样?”行者道:“那怪被老孙使了定身法定住在那里,直到明日此时
方醒。我记得他的模样,你站下,等我教你变。如此如彼,就是他的模样了。”那
呆子真个口里念着咒,行者吹口仙气,霎时就变得与那刁钻古怪一般无二,将一个
粉牌儿带在腰间。行者即变做古怪刁钻,腰间也带了一个牌儿。沙僧打扮得像个贩
猪羊的客人。一起儿赶着猪羊,上大路,径奔山来。不多时,进了山凹里,又遇见
一个小妖。他生得嘴脸也恁地凶恶!看那:

圆滴溜两只眼,如灯幌亮;红刺一头毛,似火飘光。糟鼻子,口,獠牙
尖利;查耳朵,砍额头,青脸泡浮。身穿一件浅黄衣,足踏一双莎蒲履。雄雄纠纠
若凶神,急急忙忙如恶鬼。
那怪左胁下挟着一个彩漆的请书匣儿,迎着行者三人叫道:“古怪刁钻,你两个来
了?买了几口猪羊?”行者道:“这赶的不是?”那怪朝沙僧道:“此位是谁?”行
者道:“就是贩猪羊的客人。还少他几两银子,带他来家取的。你往那里去?”那
怪道:“我往竹节山去请老大王明早赴会。”行者绰他的口气儿,就问:“共请多少
人?”那怪道:“请老大王坐首席,连本山大王共头目等众,约有四十多位。”正说
处,八戒道:“去罢,去罢。猪羊都四散走了。”行者道:“你去邀着,等我讨他帖
儿看看。”那怪见自家人,即揭开取出,递与行者。行者展开看时,上写着:

明辰敬治肴酌,庆‘钉钯嘉会’,屈尊过山一叙。幸勿外,至感。右启祖翁九
灵元圣老大人尊前。门下孙黄狮顿首百拜。
行者看毕,仍递与那怪。那怪放在匣内,径往东南上去了。

沙僧问道:“哥哥,帖儿上是甚么话头?”行者道:“乃庆钉钯会的请帖。名字
写着‘门下孙黄狮顿首百拜’。请的是祖翁九灵元圣老大人。”沙僧笑道:“黄狮想
必是个金毛狮子成精。但不知九灵元圣是个何物?”八戒听言,笑道:“是老猪的
货了!”行者道:“怎见得是你的货?”八戒道:“古人云:‘癞母猪专赶金毛狮子。’
故知是老猪之货物也。”他三人说说笑笑,赶着猪羊。却就望见虎口洞门。但见那
门儿外:
周围山绕翠,一脉气连城。
峭壁扳青蔓,高崖挂紫荆。
鸟声深树匝,花影洞门迎。
不亚桃源洞,堪宜避世情。

渐渐近于门口,又见一丛大大小小的杂项妖精,在那花树之下顽耍。忽听得八
戒“呵呵”赶猪羊到时,都来迎接,便就捉猪的猪,捉羊的捉羊,一齐捆倒。

早惊动里面妖王,领十数个小妖,出来问道:“你两个来了?买了多少猪羊?”
行者道:“买了八口猪,七腔羊,共十五个牲口。猪银该一十六两,羊银该九两。
前者领银二十两,仍欠五两。这个就是客人,跟来找银子的。”妖王听说,即唤:“小
的们,取五两银子,打发他去。”行者道:“这客人,一则来找银子,二来要看看嘉
会。”那妖大怒,骂道:“你这个刁钻儿惫懒!你买东西罢了,又与人说甚么会不会!”
八戒上前道:“主人公得了宝贝,诚是天下之奇珍,就教他看看怕怎的?”那怪咄
的一声道:“你这古怪也可恶!我这宝贝,乃是玉华州城中得来的,倘这客人看了,
去那州中传说,说得人知,那王子一时来访求,却如之何?”行者道:“主公,这
个客人,乃乾方集后边的人,去州许远,又不是他城中人也,那里去传说?二则他
肚里也饥了,我两个也未曾吃饭。家中有现成酒饭,赏他些吃了,打发他去罢。”
说不了,有一小妖,取了五两银子,递与行者。行者将银子递与沙僧道:“客人,
收了银子,我与你进后面去吃些饭来。”

沙僧仗着胆,同八戒、行者进于洞内。到二层厂厅之上,只见正中间桌上,高
高的供养着一柄九齿钉钯,真个是光彩映目;东山头靠着一条金箍棒,西山头靠着
一条降妖杖。那怪王随后跟着道:“客人,那中间放光亮的就是钉钯。你看便看,
只是出去千万莫与人说。”沙僧点头称谢了。

噫!这正是“物见主,必定取”。那八戒一生是个鲁夯的人,他见了钉钯,那里
与他叙甚么情节,跑上去,拿下来,轮在手中,现了本相。丢了解数,望妖精劈脸
就筑。这行者、沙僧也奔至两山头各拿器械,现了原身。三弟兄一齐乱打,慌得那
怪王急抽身闪过,转入后边,取一柄四明铲,杆长利,赶到天井中,支住他三般
兵器,厉声喝道:“你是甚么人,敢弄虚头,骗我宝贝!”行者骂道:“我把你这个
贼毛团!你是认我不得!我们乃东土圣僧唐三藏的徒弟。因至玉华州倒换关文,蒙贤
王教他三个王子拜我们为师,学习武艺,将我们宝贝作样,打造如式兵器。因放在
院中,被你这贼毛团夤夜入城偷来,倒说我弄虚头骗你宝贝。不要走,就把我们这
三件兵器,各奉承你几下尝尝!”那妖精就举铲来敌。这一场,从天井中斗出前门。
看他三僧攒一怪!好杀:

呼呼棒若风,滚滚钯如雨。降妖杖举满天霞,四明铲伸云生绮。好似三仙炼大
丹,火光彩幌惊神鬼。行者施威甚有能,妖精盗宝多无礼!天蓬八戒显神通,大将
沙僧英更美。弟兄合意运机谋,虎口洞中兴斗起。那怪豪强弄巧乖,四个英雄堪厮
比。当时杀至日头西,妖邪力软难相抵。
他们在豹头山战斗多时,那妖精抵敌不住,向沙僧前喊一声:“看铲!”沙僧让个身
法躲过,妖精得空而走,向东南巽宫上乘风飞去。八戒拽步要赶,行者道:“且让
他去。自古道:‘穷寇勿追。’且只来断他归路。”八戒依言。

三人径至洞口,把那百十个若大若小的妖精,尽皆打死。原来都是些虎狼彪豹,
马鹿山羊。被大圣使个手法,将他那洞里细软物件并打死的杂项兽身与赶来的猪羊,
通皆带出。沙僧就取出干柴放起火来。八戒使两个耳朵风,把一个巢穴霎时烧得
干净,却将带出的诸物,即转州城。

此时城门尚开,人家未睡。老王父子与唐僧俱在暴纱亭盼望。只见他们扑哩扑
剌的丢下一院子死兽、猪羊及细软物件。一齐叫道:“师父,我们已得胜回来也!”
那殿下喏喏相谢。唐长老满心欢喜。三个小王子跪拜于地,沙僧搀起道:“且莫谢,
都近前看看那物件。”王子道:“此物俱是何来?”行者笑道:“那虎狼彪豹,马鹿
山羊,都是成精的妖怪。被我们取了兵器,打出门来。那老妖是个金毛狮子。他使
一柄四明铲,与我等战到天晚,败阵逃生,往东南上走了。我等不曾赶他,却扫除
他归路,打杀这些群妖,搜寻他这些物件,带将来的。”

老王听说,又喜又忧。喜的是得胜而回,忧的是那妖日后报仇。行者道:“殿
下放心。我已虑之熟,处之当矣。一定与你扫除尽绝,方才起行,决不至贻害于后。
我午间去时,撞见一个青脸红毛的小妖送请书。我看他帖子上写着:‘明辰敬治肴
酌庆钉钯嘉会,屈尊车从过山一叙。幸勿外,万感!右启祖翁九灵元圣老大人尊前。’
名字是‘门下孙黄狮顿首百拜’。才子那妖精败阵,必然向他祖翁处去会话。明辰
断然寻我们报仇,当情与你扫荡干净。”老王称谢了,摆上晚斋。师徒们斋毕,各
归寝处不题。

却说那妖精果然向东南方奔到竹节山。那山中有一座洞天之处,唤名九曲盘桓
洞。洞中的九灵元圣是他的祖翁。当夜足不停风,行至五更时分,到于洞口,敲门
而进。小妖见了道:“大王,昨晚有青脸儿下请书,老爷留他住到今早,欲同他去
赴你钉钯会,你怎么又绝早亲来邀请?”妖精道:“不好说,不好说!会成不得了!”
正说处,见青脸儿从里边走出道:“大王,你来怎的?老大王爷爷起来就同我去赴会
哩。”妖精慌张张的,只是摇手不言。

少顷,老妖起来了,唤入。这妖精丢了兵器,倒身下拜,止不住腮边泪落。老
妖道:“贤孙,你昨日下柬,今早正欲来赴会,你又亲来,为何发悲烦恼?”

妖精叩头道:“小孙前夜对月闲行,只见玉华州城中有光彩冲空。急去看时,
乃是王府院中三般兵器放光:一件是九齿渗金钉钯,一件是宝杖,一件是金箍棒。
小孙即使神法摄来,立名‘钉钯嘉会’,着小的们买猪羊果品等物,设宴庆会,请
祖爷爷赏之,以为一乐。昨差青脸来送柬之后,只见原差买猪羊的刁钻儿等赶着几
个猪羊,又带了一个贩卖的客人来找银子。他定要看看会去,是小孙恐他外面传说,
不容他看。他又说肚中饥饿,讨些饭吃,因教他后边吃饭。他走到里边,看见兵器,
说是他的。三人就各抢去一件,现出原身:一个是毛脸雷公嘴的和尚,一个是长嘴
大耳朵的和尚,一个是晦气色脸的和尚。他都不分好歹,喊一声乱打。是小孙急取
四明铲赶出与他相持,问是甚么人敢弄虚头。他道是东土大唐差往西天去的唐僧之
徒弟,因过州城,倒换关文,被王子留住,习学武艺,将他这三件兵器作样子打造,
放在院内,被我偷来:遂此不忿相持。不知那三个和尚叫做甚各,却真有本事。小
孙一人敌他三个不过,所以败走祖爷处。望拔刀相助,拿那和尚报仇,庶见我祖爱
孙之意也!”

老妖闻言,默想片时,笑道:“原来是他。我贤孙,你错惹了他也!”妖精道:
“祖爷知他是谁?”老妖道:“那长嘴大耳者,乃猪八戒;晦气色脸者,乃沙和尚;
这两个犹可。那毛脸雷公嘴者,叫做孙行者。这个人其实神通广大:五百年前曾大
闹天宫,十万天兵也不曾拿得住。他专意寻人的。他便就是个搜山揭海,破洞攻城,
闯祸的个都头!你怎么惹他?也罢,等我和你去,把那厮连玉华王子都擒来替你出
气!”那妖精听说,即叩头而谢。

当时老妖点猱狮、雪狮、狻猊、白泽、伏狸、抟象诸孙,各执锋利器械,黄狮
引领,各纵狂风,径至豹头山界。只闻得烟火之气扑鼻,又闻得有哭泣之声。仔细
看时,原来是刁钻、古怪二人在那里叫主公哭主公哩。妖精近前喝道:“你是真刁
钻儿,假刁钻儿?”二怪跪倒,噙泪叩头道:“我们怎是假的?昨日这早晚领了银子
去买猪羊,走至山西边大冲之内,见一个毛脸雷公嘴的和尚,他啐了我们一口,我
们就脚软口强,不能言语,不能移步;被他扳倒,把银子搜了去,牌儿解了去,我
两个昏昏沉沉,直到此时才醒。及到家,见烟火未息,房舍尽皆烧了。又不见主公
并大小头目。故在此伤心痛哭。不知这火是怎生起的?”

那妖精闻言,止不住泪如泉涌,双脚齐跌,喊声振天,恨道:“那秃厮!十分作
恶!怎么干出这般毒事,把我洞府烧尽,美人烧死,家当老小一空。气杀我也,气
杀我也!”老妖叫猱狮扯他过来道:“贤孙,事已至此,徒恼无益。且养全锐气,到
州城里拿那和尚去。”那妖精犹不肯住哭,道:“老爷!我那们个山场,非一日治的;
今被这秃厮尽毁,我却要此命做甚的!”挣起来,往石崖上撞头磕脑;被雪狮、猱
狮等苦劝方止。当时丢了此处,都奔州城。

只听得那风滚滚,雾腾腾,来得甚近。唬得那城外各关厢人等,拖男挟女,顾
不得家私,都往州城中走。走入城门,将门闭了。有人报入王府中道:“祸事!祸事!”
那王子唐僧等,正在暴纱亭吃早斋,听得人报祸事,却出门来问。众人道:“一群
妖精,飞沙走石,喷雾掀风的,来近城了!”老王大惊道:“怎么好?”行者笑道:
“都放心,都放心。这是虎口洞妖精,昨日败阵,往东南方去伙了那甚么九灵元圣
儿来也。等我同兄弟们出去。吩咐教关了四门,汝等点人夫看守城池。”那王子果
传令把四门闭了,点起人夫上城。他父子并唐僧在城楼上点札,旌旗蔽日,炮火连
天。行者三人,却半云半雾,出城迎敌。这正是:
失却慧兵缘不谨,顿教魔起众邪凶。

毕竟不知这场胜败如何,且听下回分解。