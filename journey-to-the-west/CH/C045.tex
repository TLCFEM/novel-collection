\chapter{三清观大圣留名~车迟国猴王显法}

却说孙大圣左手把沙和尚捻一把,右手把猪八戒捻一把,他二人却就省悟。坐
在高处,倥着脸,不言不语。凭那些道士点灯着火,前后照看。他三个就如泥塑金
装一般模样。虎力大仙道:“没有歹人,如何把供献都吃了。”鹿力大仙道:“却像
人吃的勾当,有皮的都剥了皮,有核的都吐出核,却怎么不见人形?”羊力大仙道:
“师兄勿疑。想是我们虔心志意,在此昼夜诵经,前后申文,又是朝廷名号,断然
惊动天尊。想是三清爷爷圣驾降临,受用了这些供养。趁今仙从未返,鹤驾在斯,
我等可拜告天尊,恳求些圣水金丹,进与陛下,却不是长生永寿,见我们的功果也?”
虎力大仙道:“说的是。”教:“徒弟们动乐诵经!一壁厢取法衣来,等我步罡拜祷。”
那些小道士俱遵命,两班儿摆列齐整。当的一声磬响,齐念一卷《黄庭道德真经》。
虎力大仙披了法衣,擎着玉简,对面前舞蹈扬尘,拜伏于地,朝上启奏道:

“诚惶诚恐,稽首归依。臣等兴教,仰望清虚。灭僧鄙俚,敬道光辉。敕修宝
殿,御制庭闱。广陈供养,高挂龙旗。通宵秉烛,镇日香菲。一诚达上,寸敬虔归。
今蒙降驾,未返仙车,望赐些金丹圣水,进与朝廷,寿比南山。”
八戒闻言,心中忐忑,默对行者道:“这是我们的不是:吃了东西,且不走路,只
等这般祷祝,却怎么答应?”行者又捻一把,忽地开口,叫声:“晚辈小仙,且休
拜祝。我等自蟠桃会上来的,不曾带得金丹圣水,待改日再来垂赐。”那些大小道
士听见说出话来,一个个抖衣而战道:“爷爷呀!活天尊临凡,是必莫放,好歹求个
长生的法儿!”鹿力大仙上前,又拜云:

“扬尘顿首,谨办丹诚。微臣归命,俯仰三清。自来此界,兴道除僧。国王心
喜,敬重玄龄。罗天大醮,彻夜看经。幸天尊之不弃,降圣驾而临庭。俯求垂念,
仰望恩荣。是必留些圣水,与弟子们延寿长生。”
沙僧捻着行者,默默的道:“哥呀,要得紧,又来祷告了。”行者道:“与他些罢。”
八戒寂寂道:“那里有得?”行者道:“你只看着我;我有时,你们也都有了。”

那道士吹打已毕,行者开言道:“那晚辈小仙,不须伏拜。我欲不留些圣水与
你们,恐灭了苗裔;若要与你,又忒容易了。”众道闻言,一齐俯伏叩头道:“万望
天尊念弟子恭敬之意,千乞喜赐些须。我弟子广宣道德,奏国王普敬玄门。”行者
道:“既如此,取器皿来。”那道士一齐顿首谢恩。虎力大仙爱强,就抬一口大缸,
放在殿上;鹿力大仙端一砂盆安在供桌之上;羊力大仙把花瓶摘了花,移在中间。
行者道:“你们都出殿前,掩上格子,不可泄了天机,好留与你些圣水。”众道一齐
跪伏丹墀之下,掩了殿门。

那行者立将起来,掀着虎皮裙,撒了一花瓶臊溺。猪八戒见了,欢喜道:“哥
啊,我把你做这几年兄弟,只这些儿不曾弄我。我才吃了些东西,道要干这个事儿
哩。”那呆子揭衣服,忽喇喇,就似吕梁洪倒下坂来,沙沙的溺了一砂盆。沙和尚
却也撒了半缸。依旧整衣端坐在上道:“小仙领圣水。”

那些道士,推开格子,磕头礼拜谢恩,抬出缸去,将那瓶盆总归一处,教:“徒
弟,取个锺子来尝尝。”小道士即便拿了一个茶锺,递与老道士。道士舀出一锺来,
喝下口去,只情抹唇咂嘴。鹿力大仙道:“师兄好吃么?”老道士努着嘴道:“不甚
好吃,有些酣之味。”羊力大仙道:“等我尝尝。”也喝了一口,道:“有些猪溺臊
气。”行者坐在上面,听见说出这话儿来,已此识破了,道:“我弄个手段,索性留
个名罢。”大叫云:

“道号!道号!你好胡思!那个三清,肯降凡基?吾将真姓,说与你知。大唐僧众,
奉旨来西。良宵无事,下降宫闱。吃了供养,闲坐嬉嬉。蒙你叩拜,何以答之?那
里是甚么圣水,你们吃的都是我一溺之尿!”
那道士闻得此言,拦住门,一齐动叉钯、扫帚、瓦块、石头,没头没脸,往里面乱
打。好行者,左手挟了沙僧,右手挟了八戒,闯出门,驾着祥光,径转智渊寺方丈。
不敢惊动师父,三人又复睡下。

早是五鼓三点。那国王设朝,聚集两班文武,四百朝官,但见绛纱灯火光明,
宝鼎香云。此时唐三藏醒来,叫:“徒弟,徒弟,伏侍我倒换关文去来。”行者
与沙僧、八戒急起身,穿了衣服,侍立左右道:“上告师父。这昏君信着那些道士,
兴道灭僧,恐言语差错,不肯倒换关文;我等护持师父,都进朝去也。”

唐僧大喜,披了锦袈裟。行者带了通关文牒,教悟净捧着钵盂,悟能拿了锡
杖;将行囊、马匹,交与智渊寺僧看守。径到五凤楼前,对黄门官作礼,报了姓名。
言是东土大唐取经的和尚来此倒换关文,烦为转奏。那阁门大使,进朝俯伏金阶,
奏曰:“外面有四个和尚,说是东土大唐取经的,欲来倒换关文,现在五凤楼前候
旨。”国王闻奏道:“这和尚没处寻死,却来这里寻死!那巡捕官员,怎么不拿他解
来?”旁边闪过当驾的太师,启奏道:“东土大唐,乃南赡部洲,号曰中华大国。
到此有万里之遥,路多妖怪。这和尚一定有些法力,方敢西来。望陛下看中华之远
僧,且召来验牒放行,庶不失善缘之意。”国王准奏,把唐僧等宣至金銮殿下。师
徒们排列阶前,捧关文递与国王。

国王展开方看,又见黄门官来奏:“三位国师来也。”慌得国王收了关文,急下
龙座,着近侍的设了绣墩,躬身迎接。三藏等回头观看,见那大仙,摇摇摆摆,后
带着一双丫髻蓬头的小童儿,往里直进。两班官控背躬身,不敢仰视。

他上了金銮殿,对国王径不行礼。那国王道:“国师,朕未曾奉请,今日如何
肯降?”老道士云:“有一事奉告,故来也。那四个和尚是那国来的?”国王道:“是
东土大唐差去西天取经的,来此倒换关文。”那三道士鼓掌大笑道:“我说他走了,
原来还在这里!”国王惊道:“国师有何话说?他才来报了姓名,正欲拿送国师使用,
怎奈当驾太师所奏有理,朕因看远来之意,不灭中华善缘,方才召入验牒;不期国
师有此问。想是他冒犯尊颜,有得罪处也?”道士笑云:“陛下不知。他昨日来的,
在东门外打杀了我两个徒弟,放了五百个囚僧,碎车辆,夜间闯进观来,把三清
圣像毁坏,偷吃了御赐供养。我等被他蒙蔽了,只道是天尊下降;求些圣水金丹,
进与陛下,指望延寿长生;不期他遗些小便,哄瞒我等。我等各喝了一口,尝出滋
味,正欲下手擒拿,他却走了。今日还在此间,正所谓‘冤家路儿窄’也!”那国
王闻言发怒,欲诛四众。

孙大圣合掌开言,厉声高叫道:“陛下暂息雷霆之怒,容僧等启奏。”国王道:
“你冲撞了国师,国师之言,岂有差谬!”行者道:“他说我昨日到城外打杀他两个
徒弟,是谁知证?我等且屈认了,着两个和尚偿命,还放两个去取经。他又说我
碎车辆,放了囚僧,此事亦无见证,料不该死,再着一个和尚领罪罢了。他说我毁
了三清,闹了观宇,这又是栽害我也。”

国王道:“怎见栽害?”行者道:“我僧乃东土之人,乍来此处,街道尚且不通,
如何夜里就知他观中之事?既遗下小便,就该当时捉住,却这早晚坐名害人。天下
假名托姓的无限,怎么就说是我?望陛下回嗔详察。”那国王本来昏乱,被行者说了
一遍,他就决断不定。

正疑惑之间,又见黄门官来奏:“陛下,门外有许多乡老听宣。”国王道:“有
何事干?”即命宣来。宣至殿前,有三四十名乡老,朝上磕头道:“万岁,今年一
春无雨,但恐夏月干荒,特来启奏,请那位国师爷爷祈一场甘雨,普济黎民。”国
王道:“乡老且退,就有雨来也。”乡老谢恩而出。

国王道:“唐朝僧众,朕敬道灭僧为何?只为当年求雨,我朝僧人,更未尝求得
一点;幸天降国师,拯援涂炭。你今远来,冒犯国师,本当即时问罪;姑且恕你,
敢与我国师赌胜求雨么?若祈得一场甘雨,济度万民,朕即饶你罪名,倒换关文,
放你西去。若赌不过,无雨,就将汝等推赴杀场典刑示众。”行者笑道:“小和尚也
晓得些儿求祷。”

国王见说,即命打扫坛场;一壁厢教:“摆驾,寡人亲上五凤楼观看。”当时多
官摆驾。须臾,上楼坐了。唐三藏随着行者、沙僧、八戒,侍立楼下。那三道士陪
国王坐在楼上。少时间,一员官飞马来报:“坛场诸色皆备,请国师爷爷登坛。”

那虎力大仙,欠身拱手,辞了国王,径下楼来。行者向前拦住道:“先生那里
去?”大仙道:“登坛祈雨。”行者道:“你也忒自重了,更不让我远乡之僧。也罢,
这正是‘强龙不压地头蛇’。先生先去,必须对君前讲开。”大仙道:“讲甚么?”
行者道:“我与你都上坛祈雨,知雨是你的,是我的?不见是谁的功绩了。”国王在
上听见心中暗喜道:“那小和尚说话,倒有些筋节。”沙僧听见暗笑道:“不知一肚
子筋节,还不曾拿出来哩!”大仙道:“不消讲,陛下自然知之。”行者道:“虽然知
之,奈我远来之僧,未曾与你相会。那时彼此混赖,不成勾当。须讲开方好行事。”
大仙道:“这一上坛,只看我的令牌为号:一声令牌响,风来;二声响,云起;三
声响,雷闪齐鸣;四声响,雨至;五声响,云散雨收。”行者笑道:“妙啊!我僧是
不曾见,请了,请了!”

大仙拽开步前进,三藏等随后,径到了坛门外。抬头观看,那里有一座高台,
约有三丈多高。台左右插着二十八宿旗号,顶上放一张桌子,桌上有一个香炉,炉
中香烟霭霭。两边有两只烛台,台上风烛煌煌。炉边靠着一个金牌,牌上镌的是雷
神名号。底下有五个大缸,都注着满缸清水,水上浮着杨柳枝。杨柳枝上托着一面
铁牌,牌上书的是雷霆都司的符字。左右有五个大桩,桩上写着五方蛮雷使者的名
录。每一桩边立两个道士,各执铁锤,伺候着打桩。台后面有许多道士在那里写作
文书。正中间设一架纸炉,又有几个象生的人物,都是那执符使者,土地赞教之神。

那大仙走进去,更不谦逊,直上高台立定。旁边有个小道士,捧了几张黄纸书
就的符字,一口宝剑,递与大仙。大仙执着宝剑,念声咒语,将一道符在烛上烧了。
那底下两三个道士,拿过一个执符的象生、一道文书,亦点火焚之。那上面乒的一
声令牌响,只见那半空里,悠悠的风色飘来。猪八戒口里作念道:“不好了,不好
了!这道士果然有本事,令牌响了一下,果然就刮风!”行者道:“兄弟悄悄的,你
们再莫与我说话。只管护持师父,等我干事去来。”

好大圣,拔下一根毫毛,吹口仙气,叫“变!”就变作一个“假行者”,立在唐
僧手下。他的真身,出了元神,赶到半空中。高叫:“那司风的是那个?”慌得那
风婆婆捻住布袋,巽二郎札住口绳,上前施礼。行者道:“我保护唐朝圣僧西天取
经,路过车迟国,与那妖道赌胜祈雨,你怎么不助老孙,反助那道士?我且饶你,
把风收了。若有一些风儿,把那道士的胡子吹得动动,各打二十铁棒!”风婆婆道:
“不敢,不敢!”遂而没些风气。八戒忍不住,乱嚷道:“那先儿请退!令牌已响,
怎么不见一些风儿?你下来,让我们上去!”

那道士又执令牌,烧了符檄,扑的又打了一下,只见那空中云雾遮满。孙大圣
又当头叫道:“布云的是那个?”慌得那推云童子、布雾郎君当面施礼。行者又将
前事说了一遍。那云童、雾子也收了云雾,放出太阳星耀耀,一天万里更无云。八
戒笑道:“这先儿只好哄这皇帝,搪塞黎民,全没些真实本事!令牌响了两下,如何
又不见云生?”

那道士心中焦躁,仗宝剑,解散了头发,念着咒,烧了符,再一令牌打将下去,
只见那南天门里,邓天君领着雷公、电母到当空,迎着行者施礼。行者又将前项事
说了一遍。道:“你们怎么来的志诚!是何法旨!”天君道:“那道士五雷法是个真的。
他发了文书,烧了文檄,惊动玉帝,玉帝掷下旨意,径至‘九天应元雷声普化天尊’
府下。我等奉旨前来,助雷电下雨。”行者道:“既如此,且都住了,伺候老孙行事。”
果然雷也不鸣,电也不灼。

那道士愈加着忙,又添香、烧符、念咒打下令牌。半空中,又有四海龙王,一
齐拥至。行者当头喝道:“敖广,那里去?”那敖广、敖顺、敖钦、敖闰上前施礼。
行者又将前项事说了一遍。道:“向日有劳,未曾成功;今日之事,望为助力。”龙
王道:“遵命!遵命!”行者又谢了敖顺道:“前日亏令郎缚怪,搭救师父。”龙王道:
“那厮还锁在海中,未敢擅便,正欲请大圣发落。”行者道:“凭你怎么处治了罢。
如今且助我一功。那道士四声令牌已毕,却轮到老孙下去干事了。但我不会发符、
烧檄、打甚令牌,你列位却要助我行行。”

邓天君道:“大圣吩咐,谁敢不从!但只是得一个号令,方敢依令而行;不然,
雷雨乱了,显得大圣无款也。”行者道:“我将棍子为号罢。”那雷公大惊道:“爷爷
呀!我们怎吃得这棍子?”行者道:“不是打你们,但看我这棍子往上一指,就要刮
风。”那风婆婆、巽二郎没口的答应道:“就放风!”“棍子第二指,就要布云。”那
推云童子、布雾郎君道:“就布云,就布云!”“棍子第三指,就要雷电皆鸣。”那雷
公、电母道:“奉承!奉承!”“棍子第四指,就要下雨。”那龙王道:“遵命,遵命!”
“棍子第五指,就要大日晴天,却莫违误。”

吩咐已毕,遂按下云头,把毫毛一抖,收上身来。那些人肉眼凡胎,那里晓得?
行者遂在旁边高叫道:“先生请了。四声令牌俱已响毕,更没有风云雷雨,该让我
了。”

那道士无奈,不敢久占,只得下了台让他。努着嘴,径往楼上见驾。行者道:
“等我跟他去,看他说些甚的。”只听得那国王问道:“寡人这里洗耳诚听,你那里
四声令响,不见风雨,何也?”道士云:“今日龙神都不在家。”行者厉声道:“陛
下,龙神俱在家;只是这国师法不灵,请他不来。等和尚请来你看。”国王道:“即
去登坛,寡人还在此候雨。”

行者得旨,急抽身到坛所,扯着唐僧道:“师父请上台。”唐僧道:“徒弟,我
却不会祈雨。”八戒笑道:“他害你了。若还没雨,拿上柴蓬,一把火了帐!”行者
道:“你不会求雨,好的会念经。等我助你。”那长老才举步登坛,到上面,端然坐
下,定性归神,默念那《密多心经》。正坐处,忽见一员官,飞马来问:“那和尚,
怎么不打令牌,不烧符檄?”行者高声答道:“不用,不用,我们是静功祈祷。”那
官去回奏不题。

行者听得老师父经文念尽,却去耳朵内取出铁棒,迎风幌了一幌,就有丈二长
短,碗来粗细。将棍望空一指,那风婆婆见了,急忙扯开皮袋,巽二郎解放口绳;
只听得呼呼风响,满城中揭瓦翻砖,扬砂走石。看起来,真个好风,却比那寻常之
风不同也。但见:

折柳伤花,摧林倒树。九重殿损壁崩墙,五凤楼摇梁撼柱。天边红日无光,地
下黄砂有翅。演武厅前武将惊,会文阁内文官惧。三宫粉黛乱青丝,六院嫔妃蓬宝
髻。侯伯金冠落绣缨,宰相乌纱飘展翅。当驾有言不敢谈,黄门执本无由递。金鱼
玉带不依班,象简罗衫无品叙。彩阁翠屏尽损伤,绿窗朱户皆狼狈。金銮殿瓦走砖
飞,锦云堂门歪碎。这阵狂风果是凶,刮得那君王父子难相会;六街三市没人踪,
万户千门皆紧闭!
正是那狂风大作,孙行者又显神通,把金箍棒钻一钻,望空又一指。只见那:

推云童子,布雾郎君:推云童子显神威,骨都都触石遮天;布雾郎君施法力,
浓漠漠飞烟盖地。茫茫三市暗,冉冉六街昏。因风离海上,随雨出昆仑。顷刻漫天
地,须臾蔽世尘。宛然如混沌,不见凤楼门。
此时昏雾朦胧,浓云。孙行者又把金箍棒钻一钻,望空又一指。慌得那:

雷公奋怒,电母生嗔:雷公奋怒,倒骑火兽下天关;电母生嗔,乱掣金蛇离斗
府。唿喇喇施霹雳,振碎了铁叉山;淅沥沥闪红绡,飞出了东洋海。呼呼隐隐滚车
声,烨烨煌煌飘稻米。万萌万物精神改,多少昆虫蛰已开。君臣楼上心惊骇,商贾
闻声胆怯忙。
那沉雷护闪,乒乒乓乓,一似那地裂山崩之势,唬得那满城人,户户焚香,家家化
纸。孙行者高呼:“老邓!仔细替我看那贪赃坏法之官,忤逆不孝之子,多打死几个
示众!”那雷越发振响起来。行者却又把铁棒望上一指。只见那:

龙施号令,雨漫乾坤。势如银汉倾天堑,疾似云流过海门。楼头声滴滴,窗外
响潇潇。天上银河泻,街前白浪滔。淙淙如瓮捡,滚滚似盆浇。孤庄将漫屋,野岸
欲平桥。真个桑田变沧海,霎时陆岸滚波涛。神龙借此来相助,抬起长江望下浇。

这场雨,自辰时下起,只下到午时前后。下得那车迟城,里里外外,水漫了街
衢。那国王传旨道:“雨够了,雨够了!十分再多,又坏了禾苗,反为不美。”五
凤楼下听事官策马冒雨来报:“圣僧,雨够了。”行者闻言,将金箍棒往上又一指。
只见霎时间,雷收风息,雨散云收。国王满心欢喜,文武尽皆称赞道:“好和尚,
这正是‘强中更有强中手’!就是我国师求雨虽灵,若要晴,细雨儿还下半日,便
不清爽;怎么这和尚要晴就晴,顷刻间杲杲日出,万里就无云也?”

国王教回銮,倒换关文,打发唐僧过去。正用御宝时,又被那三个道士上前阻
住道:“陛下,这场雨全非和尚之功,还是我道门之力。”国王道:“你才说龙王不
在家,不曾有雨;他走上去,以静功祈祷,就雨下来,怎么又与他争功,何也?”
虎力大仙道:“我上坛发了文书,烧了符檄,击了令牌,那龙王谁敢不来?想是别方
召请,风、云、雾、雷、电五司俱不在,一闻我令,随赶而来;适遇着我下他上,
一时撞着这个机会,所以就雨。从根算来,还是我请的龙,下的雨,怎么算作他的
功果?”那国王昏乱,听此言,却又疑惑未定。

行者近前一步,合掌奏道:“陛下,这些傍门法术,也不成个功果,算不得我
的他的;如今有四海龙王,现在空中,我僧未曾发放,他还不敢遽退。那国师若能
叫得龙王现身,就算他的功劳。”国王大喜道:“寡人做了二十三年皇帝,更不曾看
见活龙是怎么模样。你两家各显法力,不论僧道,但叫得来的,就是有功;叫不出
的,有罪。”那道士怎么有那样本事?就叫,那龙王见大圣在此,也不敢出头。道士
云:“我辈不能,你是叫来。”

那大圣仰面朝空,厉声高叫:“敖广何在?弟兄们都现原身来看!”那龙王听唤,
即忙现了本身。四条龙,在半空中度雾穿云,飞舞向金銮殿上。但见:

飞腾变化,绕雾盘云。玉爪垂钩白,银鳞舞镜明。髯飘素练根根爽,角耸轩昂
挺挺清。磕额崔巍,圆睛幌亮。隐显莫能测,飞扬不可评。祷雨随时布雨,求晴即
便天晴。这才是有灵有圣真龙象,祥瑞缤纷绕殿庭。
那国王在殿上焚香,众公卿在阶前礼拜。国王道:“有劳贵体降临,请回。寡人改
日醮谢。”行者道:“列位众神各自归去,这国王改日醮谢哩。”那龙王径自归海,
众神各各回天。这正是:
广大无边真妙法,至真了性劈傍门。

毕竟不知怎么除邪,且听下回分解。