\chapter{孙行者大闹黑风山~观世音收伏熊罴怪}

话说孙行者一筋斗跳将起去,唬得那观音院大小和尚并头陀、幸童、道人等,
一个个朝天礼拜道:“爷爷呀!原来是腾云驾雾的神圣下界!怪道火不能伤!恨我那个
不识人的老剥皮,使心用心,今日反害了自己!”三藏道:“列位请起,不须恨了。
这去寻着袈裟,万事皆休;但恐找寻不着,我那徒弟性子有些不好,汝等性命不知
如何,恐一人不能脱也。”众僧闻得此言,一个个提心吊胆,告天许愿,只要寻得袈
裟,各全性命不题。

却说孙大圣到空中,把腰儿扭了一扭,早来到黑风山上。住了云头,仔细看,
果然是座好山。况正值春光时节,但见:

万壑争流,千崖竞秀。鸟啼人不见,花落树犹香。雨过天连青壁润,风来松卷
翠屏张。山草发,野花开,悬崖峭嶂;薜萝生,佳木丽,峻岭平岗。不遇幽人,那
寻樵子?涧边双鹤饮,石上野猿狂。矗矗堆螺排黛色,巍巍拥翠弄岚光。

那行者正观山景,忽听得芳草坡前,有人言语。他却轻步潜踪,闪在那石崖之
下,偷睛观看。原来是三个妖魔,席地而坐:上首的是一条黑汉,左首下是一个道
人,右首下是一个白衣秀士。都在那里高谈阔论。讲的是立鼎安炉,抟砂炼汞,白
雪黄芽,傍门外道。正说中间,那黑汉笑道:“后日是我母难之日,二公可光顾光
顾?”白衣秀士道:“年年与大王上寿,今年岂有不来之理?”黑汉道:“我夜来得
了一件宝贝,名唤锦佛衣,诚然是件玩好之物。我明日就以他为寿,大开筵宴,
邀请各山道官,庆贺佛衣,就称为‘佛衣会’如何?”道人笑道:“妙,妙,妙!我
明日先来拜寿,后日再来赴宴。”

行者闻得佛衣之言,定以为是他宝贝。他就忍不住怒气,跳出石崖,双手举起
金箍棒,高叫道:“我把你这伙贼怪!你偷了我的袈裟,要做甚么‘佛衣会’,趁早
儿将来还我!”喝一声“休走!”轮起棒,照头一下,慌得那黑汉化风而逃,道人驾
云而走;只把个白衣秀士,一棒打死。拖将过来看处,却是一条白花蛇怪。索性提
起来,做五七断,径入深山,找寻那个黑汉。

转过尖峰,抹过峻岭,又见那壁陡崖前,耸出一座洞府,但见那:

烟霞渺渺,松柏森森:烟霞渺渺采盈门,松柏森森青绕户。桥踏枯槎木,峰巅
绕薜萝。鸟衔红蕊来云壑,鹿践芳丛上石台。那门前、时催花发,风送花香。临堤
绿柳转黄鹂,傍岸夭桃翻粉蝶。虽然旷野不堪夸,却赛蓬莱山下景。

行者到于门首,又见那两扇石门,关得甚紧。门上有一横石板,明书六个大字,
乃“黑风山黑风洞”。即便轮棒叫声“开门!”那里面有把门的小妖,开了门出来,
问道:“你是何人,敢来击吾仙洞?”行者骂道:“你个作死的孽畜!甚么个去处,
敢称仙洞!‘仙’字是你称的?快进去报与你那黑汉,教他快送老爷的袈裟出来,饶
你一窝性命!”

小妖急急跑到里面,报道:“大王!‘佛衣会’做不成了!门外有一个毛脸雷公
嘴的和尚,来讨袈裟哩!”那黑汉被行者在芳草坡前赶将来,却才关了门,坐还未
稳。又听得那话,心中暗想到:“这厮不知是那里来的,这般无礼,他敢嚷上我的
门来!”教:“取披挂。”随结束了,绰一杆黑缨枪,走出门来。这行者闪在门外,
执着铁棒,睁睛观看,只见那怪果生得凶险:
碗子铁盔火漆光,乌金铠甲亮辉煌。
皂罗袍罩风兜袖,黑绿丝绦穗长。
手执黑缨枪一杆,足踏乌皮靴一双。
眼幌金睛如掣电,正是山中黑风王。
行者暗笑道:“这厮真个如烧窑的一般,筑煤的无二!想必是在此处刷炭为生,怎么
这等一身乌黑?”那怪厉声高叫道:“你是个甚么和尚,敢在我这里大胆?”行者
执铁棒,撞至面前,大咤一声道:“不要闲讲!快还你老外公的袈裟来!”那怪道:“你
是那寺里和尚?你的袈裟在那里失落了,敢来我这里索取?”行者道:“我的袈裟,
在直北观音院后方丈里放着;只因那院里失了火,你这厮,趁哄掳掠,盗了来,要
做‘佛衣会’庆寿,怎敢抵赖?快快还我,饶你性命!若牙迸半个‘不’字,我推倒
了黑风山,平了黑风洞,把你这一洞妖邪,都碾为粉!”

那怪闻言,呵呵冷笑道:“你这个泼物!原来昨夜那火就是你放的!你在那方丈
屋上,行凶招风,是我把一件袈裟拿来了,你待怎么!你是那里来的?姓甚名谁?有
多大手段,敢那等海口浪言!”行者道:“是你也认不得你老外公哩!你老外公乃大
唐上国驾前御弟三藏法师之徒弟,姓孙,名悟空行者。若问老孙的手段,说出来,
教你魂飞魄散,死在眼前!”那怪道:“我不曾会你,有甚么手段,说来我听。”行
者笑道:“我儿子,你站稳着,仔细听之!我:

自小神通手段高,随风变化逞英豪。养性修真熬日月,跳出轮回把命逃。一点
诚心曾访道,灵台山上采药苗。那山有个老仙长,寿年十万八千高。老孙拜他为师
父,指我长生路一条。他说身内有丹药,外边采取枉徒劳。得传大品天仙诀,若无
根本实难熬。回光内照宁心坐,身中日月坎离交。万事不思全寡欲,六根清净体坚
牢。返老还童容易得,超凡入圣路非
遥。三年无漏成仙体,不同俗辈受煎熬。十洲三岛还游戏,海角天涯转一遭。活该
三百多余岁,不得飞升上九霄。下海降龙真宝贝,才有金箍棒一条。花果山前为帅
首,水帘洞里聚群妖。玉皇大帝传宣诏,封我齐天极品高。几番大闹灵霄殿,数次
曾偷王母桃。天兵十万来降我,层层密密布枪刀。战退天王归上界,哪吒负痛领兵
逃。显圣真君能变化,老孙硬赌跌平交。道祖观音同玉帝,南天门上看降妖。却被
老君助一阵,二郎擒我到天曹。将身绑在降妖柱,即命神兵把首枭。刀砍锤敲不得
坏,又教雷打火来烧。老孙其实有手段,全然不怕半分毫。送在老君炉里炼,六丁
神火慢煎熬。日满开炉我跳出,手持铁棒绕天跑。纵横到处无遮挡,三十三天闹一
遭。我佛如来施法力,五行山压老孙腰。整整压该五百载,幸逢三藏出唐朝。吾今
皈正西方去,转上雷音见玉毫。你去乾坤四海问一问,我是历代驰名第一妖!”

那怪闻言笑道:“你原来是那闹天宫的弼马温么?”行者最恼的是人叫他弼马
温;听见这一声,心中大怒。骂道:“你这贼怪!偷了袈裟不还,倒伤老爷!不要走!
看棒!”那黑汉侧身躲过,绰长枪,劈手来迎。两家这场好杀:

如意棒,黑缨枪,二人洞口逞刚强。分心劈脸刺,着臂照头伤。这个横丢阴棍
手,那个直拈急三枪。白虎爬山来探爪,黄龙卧道转身忙。喷彩雾,吐毫光,两个
妖仙不可量:一个是修正齐天圣,一个是成精黑大王。这场山里相争处,只为袈裟
各不良。
那怪与行者斗了十数回合,不分胜负。渐渐红日当午,那黑汉举枪架住铁棒道:“孙
行者,我两个且收兵,等我进了膳来,再与你赌斗。”行者道:“你这个孽畜,教做
汉子?好汉子,半日儿就要吃饭?似老孙在山根下,整压了五百余年,也未曾尝些汤
水,那里便饿哩?莫推故!休走!还我袈裟来,方让你去吃饭!”那怪虚幌一枪,撤身
入洞,关了石门,收回小怪,且安排筵宴,书写请帖,邀请各山魔王庆会不题。

却说行者攻门不开,也只得回观音院。那本寺僧人已葬埋了那老和尚,都在方
丈里伏侍唐僧。早斋已毕,又摆上午斋。正那里添汤换水,只见行者从空降下,众
僧礼拜,接入方丈,见了三藏。

三藏道:“悟空,你来了?袈裟如何?”行者道:“已有了根由。早是不曾冤了
这些和尚。原来是那黑风山妖怪偷了。老孙去暗暗的寻他,只见他与一个白衣秀士,
一个老道人,坐在那芳草坡前讲话。也是个不打自招的怪物,他忽然说出道:后日
是他母难之日,邀请诸邪来做生日;夜来得了一件锦佛衣,要以此为寿,作一大
宴,唤做‘庆赏佛衣会’。是老孙抢到面前,打了一棍,那黑汉化风而走,道人也
不见了,只把个白衣秀士打死,乃是一条白花蛇成精。我又急急赶到他洞口,叫他
出来与他赌斗。他已承认了,是他拿回。战够这半日,不分胜负。那怪回洞,却要
吃饭,关了石门,惧战不出。老孙却来回看师父,先报此信。已是有了袈裟的下落,
不怕他不还我。”

众僧闻言,合掌的合掌,磕头的磕头,都念声“南无阿弥陀佛!今日寻着下落,
我等方有了性命矣!”行者道:“你且休喜欢畅快,我还未曾到手,师父还未曾出门
哩。只等有了袈裟,打发得我师父好好的出门,才是你们的安乐处;若稍有些须不
虞,老孙可是好惹的主子!可曾有好茶饭与我师父吃?可曾有好草料喂马?”众僧俱
满口答应道:“有,有,有!更不曾一毫待怠慢了老爷。”三藏道:“自你去了这半日,
我已吃过了三次茶汤,两餐斋供了。他俱不曾敢慢我。但只是你还尽心竭力去寻取
袈裟回来。”行者道:“莫忙!既有下落,管情拿住这厮,还你原物。放心,放心!”
正说处,那上房院主,又整治素供,请孙老爷吃斋。行者却吃了些须,复驾祥云,
又去找寻。

正行间,只见一个小怪,左胁下夹着一个花梨木匣儿,从大路而来。行者度他
匣内必有甚么柬札,举起棒,劈头一下,可怜不禁打,就打得似个肉饼一般;却拖
在路旁,揭开匣儿观看,果然是一封请帖。帖上写着:

侍生熊罴顿首拜,启上大阐金池老上人丹房:屡承佳惠,感激渊深。夜观回禄
之难,有失救护,谅仙机必无他害。生偶得佛衣一件,欲作雅会,谨具花酌,奉扳
清赏。至期,千乞仙驾过临一叙是荷。先二日具。
行者见了,呵呵大笑道:“那个老剥皮,死得他一毫儿也不亏!他原来与妖精结党!
怪道他也活了二百七十岁。想是那个妖精,传他些甚么服气的小法儿,故有此寿。
老孙还记得他的模样,等我就变做那和尚,往他洞里走走,看我那袈裟放在何处。
假若得手,即便拿回,却也省力。”

好大圣,念动咒语,迎着风一变,果然就像那老和尚一般,藏了铁棒,拽开步,
径来洞口,叫声“开门”。那小妖开了门,见是这般模样,急转身报道:“大王,金
池长老来了。”那怪大惊道:“刚才差了小的去下简帖请他,这时候还未到那里哩,
如何他就来得这等迅速?想是小的不曾撞着他,断是孙行者呼他来讨袈裟的。管事
的,可把佛衣藏了,莫教他看见。”

行者进了前门,但见那天井中,松篁交翠,桃李争妍,丛丛花发,簇簇兰香,
却也是个洞天之处。又见那二门上有一联对子,写着:“静隐深山无俗虑,幽居仙
洞乐天真。”行者暗道:“这厮也是个脱垢离尘知命的怪物。”入门里,往前又进,
到于三层门里,都是些画栋雕梁,明窗彩户。只见那黑汉子,穿的是黑绿丝袢祆,
罩一领鸦青花绫披风,戴一顶乌角软巾,穿一双麂皮皂靴;见行者进来,整顿衣巾,
降阶迎接道:“金池老友,连日欠亲。请坐,请坐。”行者以礼相见。见毕而坐,坐
定而茶。茶罢,妖精欠身道:“适有小简奉启,后日一叙,何老友今日就下顾也?”
行者道:“正来进拜,不期路遇华翰,见有‘佛衣雅会’,故此急急奔来,愿求见见。”
那怪笑道:“老友差矣。这袈裟本是唐僧的,他在你处住札,你岂不曾看见,反来
就我看看?”行者道:“贫僧借来,因夜晚还不曾展看,不期被大王取来。又被火
烧了荒山,失落了家私。那唐僧的徒弟,又有些骁勇,乱忙中,四下里都寻觅不见。
原来是大王的洪福收来,故特来一见。”

正讲处,只见有一个巡山的小妖,来报道:“大王,祸事了!下请书的小校,被
孙行者打死在大路旁边,他绰着经儿,变化做金池长老,来骗佛衣也!”那怪闻言,
暗道:“我说那长老怎么今日就来,又来得迅速,果然是他!”急纵身,拿过枪来,
就刺行者。行者耳躲里急掣出棍子,现了本相,架住枪尖,就在他那中厅里跳出,
自天井中,斗到前门外,唬得那洞里群魔都丧胆,家间老幼尽无魂。这场在山头好
赌斗,比前番更是不同。好杀:

那猴王胆大充和尚,这黑汉心灵隐佛衣。语去言来机会巧,随机应变不差池。
袈裟欲见无由见,宝贝玄微真妙微。小怪寻山言祸事,老妖发怒显神威。翻身打出
黑风洞,枪棒争持辨是非。棒架长枪声响亮,枪迎铁棒放光辉。悟空变化人间少,
妖怪神通世上稀。这个要把佛衣来庆寿,那个不得袈裟肯善归?这番苦战难分手,
就是活佛临凡也解不得围。
他两个从洞口打上山头,自山头杀在云外,吐雾喷风,飞砂走石,只斗到红日沉西,
不分胜败。那怪道:“姓孙的,你且住了手。今日天晚,不好相持。你去,你去!待
明早来,与你定个死活。”行者叫道:“儿子莫走!要战便像个战的,不可以天晚相
推。”看他没头没脸的,只情使棍子打来,这黑汉又化阵清风,转回本洞,紧闭石
门不出。

行者却无计策奈何,只得也回观音院里。按落云头,道声“师父”。那三藏眼
儿巴巴的,正望他哩。忽见到了面前,甚喜;又见他手里没有袈裟,又惧;问道:
“怎么这番还不曾有袈裟来?”行者袖中取出个简帖儿来,递与三藏道:“师父,
那怪物与这死的老剥皮,原是朋友。他着一个小妖送此帖来,还请他去赴‘佛衣会’。
是老孙就把那小妖打死,变做那老和尚,进他洞去,骗了一钟茶吃。欲问他讨袈裟
看看,他不肯拿出。正坐间,忽被一个甚么巡风的,走了风信,他就与我打将起来。
只斗到这早晚,不分上下。他见天晚,闪回洞去,紧闭石门。老孙无奈,也暂回来。”
三藏道:“你手段比他何如?”行者道:“我也硬不多儿,只战个手平。”

三藏才看了简帖,又递与那院主道:“你师父敢莫也是妖精么?”那院主慌忙
跪下道:“老爷,我师父是人;只因那黑大王修成人道,常来寺里与我师父讲经,
他传了我师父些养神服气之术,故以朋友相称。”行者道:“这伙和尚没甚妖气,他
一个个头圆顶天,足方履地,但比老孙肥胖长大些儿,非妖精也。你看那帖儿上写
着‘侍生熊罴’,此物必定是个黑熊成精。”三藏道:“我闻得古人云:‘熊与猩猩相
类。’都是兽类,他却怎么成精?”行者笑道:“老孙是兽类,见做了齐天大圣,与
他何异?大抵世间之物,凡有九窍者,皆可以修行成仙。”三藏又道:“你才说他本
事与你手平,你却怎生得胜,取我袈裟回来?”行者道:“莫管,莫管,我有处治。”

正商议间,众僧摆上晚斋,请他师徒们吃了。三藏教掌灯,仍去前面禅堂安歇。
众僧都挨墙倚壁,苫搭窝棚,各各睡下,只把个后方丈让与那上下院主安身。此时
夜静,但见:

银河现影,玉宇无尘。满天星灿烂,一水浪收痕。万籁声宁,千山鸟绝。溪边
渔火息,塔上佛灯昏。昨夜黎钟鼓响,今宵一遍哭声闻。
是夜在禅堂歇宿。那三藏想着袈裟,那里得稳睡?忽翻身见窗外透白,急起叫道:“悟
空,天明了,快寻袈裟去。”行者一骨鲁跳将起来。早见众僧侍立,供奉汤水,行
者道:“你等用心伏侍我师父,老孙去也。”三藏下床,扯住道:“你往那里去?”
行者道:“我想这桩事都是观音菩萨没理,他有这个禅院在此,受了这里人家香火,
又容那妖精邻住。我去南海寻他,与他讲三讲,教他亲来问妖精讨袈裟还我。”三
藏道:“你这去,几时回来?”行者道:“时少只在饭罢,时多只在晌午,就成功了。
那些和尚,可好伏侍,老孙去也。”

说声去,早已无踪。须臾间,到了南海。停云观看,但见那:

汪洋海远,水势连天。祥光笼宇宙,瑞气照山川。千层雪浪吼青霄,万迭烟波
滔白昼。水飞四野,浪滚周遭:水飞四野振轰雷,浪滚周遭鸣霹雳。休言水势,且
看中间。五色朦胧宝叠山。红黄紫皂绿和蓝。才见观音真胜境,试看南海落伽山。
好去处!山峰高耸,顶透虚空。中间有千样奇花,百般瑞草。风摇宝树,日映金莲。
观音殿瓦盖琉璃,潮音洞门铺玳瑁。绿杨影里语鹦哥,紫竹林中啼孔雀。罗纹石上,
护法威严;玛瑙滩前,木叉雄壮。

这行者观不尽那异景非常,径直按云头,到竹林之下。早有诸天迎接道:“菩
萨前者对众言大圣归善,甚是宣扬。今保唐僧,如何得暇到此?”行者道:“因保
唐僧,路逢一事,特见菩萨,烦为通报。”诸天遂来洞口报知。菩萨唤入。

行者遵法而行,至宝莲台下拜了。菩萨问曰:“你来何干?”行者道:“我师父
路遇你的禅院,你受了人间香火,容一个黑熊精在那里邻住,着他偷了我师父袈裟,
屡次取讨不与,今特来问你要的。”菩萨道:“这猴子说话,这等无状!既是熊精偷
了你的袈裟,你怎来问我取讨?都是你这个孽猴大胆,将宝贝卖弄,拿与小人看见,
你却又行凶,唤风发火,烧了我的留云下院,反来我处放刁!”行者见菩萨说出这
话,知他晓得过去未来之事,慌忙礼拜道:“菩萨,乞恕弟子之罪,果是这般这等。
但恨那怪物不肯与我袈裟,师父又要念那话儿咒语,老孙忍不得头疼,故此来拜烦
菩萨。望菩萨慈悲之心,助我去拿那妖精,取衣西进也。”菩萨道:“那怪物有许多
神通,却也不亚于你。也罢,我看唐僧面上,和你去走一遭。”行者闻言,谢恩再
拜。即请菩萨出门,遂同驾祥云,早到黑风山。坠落云头,依路找洞。

正行处,只见那山坡前,走出一个道人,手拿着一个玻璃盘儿,盘内安着两粒
仙丹,往前正走;被行者撞个满怀,掣出棒,就照头一下,打得脑里浆流出,腔中
血迸撺。菩萨大惊道:“你这个猴子,还是这等放泼!他又不曾偷你袈裟,又不与你
相识,又无甚冤仇,你怎么就将他打死?”行者道:“菩萨,你认他不得。他是那
黑熊精的朋友。他昨日和一个白衣秀士,都在芳草坡前坐讲。后日是黑精的生日,
请他们来庆‘佛衣会’。今日他先来拜寿,明日来庆‘佛衣会’,所以我认得。定是
今日替那妖去上寿。”菩萨说:“既是这等说来,也罢。”

行者才去把那道人提起来看,却是一只苍狼。旁边那个盘儿底下却有字,刻道:
“凌虚子制”。行者见了,笑道:“造化!造化!老孙也是便益,菩萨也是省力。这怪
叫做不打自招,那怪教他今日了劣。”菩萨说道:“悟空,这教怎么说?”行者道:
“菩萨,我悟空有一句话儿,叫做将计就计,不知菩萨可肯依我?”菩萨道:“你
说。”行者说道:“菩萨,你看这盘儿中是两粒仙丹,便是我们与那妖魔的贽见;这
盘儿后面刻的四个字,说‘凌虚子制’,便是我们与那妖魔的勾头。菩萨若要依得
我时,我好替你作个计较,也就不须动得干戈,也不须劳得征战,妖魔眼下遭瘟,
佛衣眼下出现;菩萨要不依我时,菩萨往西,我悟空往东,佛衣只当相送,唐三藏
只当落空。”菩萨笑道:“这猴熟嘴!”行者道:“不敢,倒是一个计较。”菩萨说:“你
这计较怎说?”行者道:“这盘上刻那‘凌虚子制’,想这道人就叫做凌虚子。菩萨,
你要依我时,可就变做这个道人,我把这丹吃了一粒,变上一粒,略大些儿。菩萨
你就捧了这个盘儿,两粒仙丹,去与那妖上寿,把这丸大些的让与那妖。待那妖一
口吞之,老孙便于中取事,他若不肯献出佛衣,老孙将他肚肠,就也织将一件出来。”

菩萨没法,只得也点点头儿。行者笑道:“如何?”尔时菩萨乃以广大慈悲,
无边法力,亿万化身,以心会意,以意会身,恍惚之间,变作凌虚仙子:
鹤氅仙风飒,飘欲步虚。
苍颜松柏老,秀色古今无。
去去还无住,如如自有殊。
总来归一法,只是隔邪躯。
行者看道:“妙啊!妙啊!还是妖精菩萨,还是菩萨妖精?”菩萨笑道:“悟空,菩萨、
妖精,总是一念;若论本来,皆属无有。”行者心下顿悟,转身却就变做一粒仙丹:
走盘无不定,圆明未有方。
三三勾漏合,六六少翁商。
瓦铄黄金焰,牟尼白昼光。
外边铅与汞,未许易论量。
行者变了那颗丹,终是略大些儿。菩萨认定,拿了那个玻璃盘儿,径到妖洞门口,
看时,果然是:

崖深岫险,云生岭上;柏苍松翠,风飒林间。崖深岫险,果是妖邪出没人烟少;
柏苍松翠,也可仙真修隐道情多。山有涧,涧有泉,潺潺流水咽鸣琴,便堪洗耳;
崖有鹿,林有鹤,幽幽仙籁动间岑,亦可赏心。这是妖仙有分降菩提,弘誓无边垂
恻隐。
菩萨看了,心中暗喜道:“这孽畜占了这座山洞,却是也有些道分。”因此心中已此
有个慈悲。

走到洞口,只见守洞小妖,都有些认得道:“凌虚仙长来了。”一边传报,一边
接引。那妖早已迎出二门道:“凌虚,有劳仙驾珍顾,蓬荜有辉。”菩萨道:“小道
敬献一粒仙丹,敢称千寿。”他二人拜毕,方才坐定,又叙起他昨日之事。菩萨不
答,连忙拿丹盘道:“大王,且见小道鄙意。”觑定一粒大的,推与那妖道:“愿大
王千寿!”那妖亦推一粒,递与菩萨道:“愿与凌虚子同之。”让毕,那妖才待要咽,
那药顺口儿一直滚下。现了本相,理起四平,那妖滚倒在地。菩萨现相,问妖取了
佛衣。行者早已从鼻孔中出去。菩萨又怕那妖无礼,却把一个箍儿,丢在那妖头上。
那妖起来,提枪要刺,行者、菩萨早已起在空中,菩萨将真言念起。那怪依旧头疼,
丢了枪,满地乱滚。半空里笑倒个美猴王,平地下滚坏个黑熊怪。

菩萨道:“孽畜!你如今可皈依么?”那怪满口道:“心愿皈依,只望饶命!”行
者道:“恐耽搁了工夫,”意欲就打。菩萨急止住道:“休伤他命。我有用他处哩。”
行者道:“这样怪物,不打死他,反留他在何处用哩?”菩萨道:“我那落伽山后,
无人看管,我要带他去做个守山大神。”行者笑道:“诚然是个救苦慈尊,一灵不损。
若是老孙有这样咒语,就念上他娘千遍!这回儿就有许多黑熊,都教他了帐!”

却说那怪苏醒多时,公道难禁疼痛,只得跪在地下哀告道:“但饶性命,愿皈
正果!”菩萨方坠落祥光,又与他摩顶受戒,教他执了长枪,跟随左右。那黑熊才
一片野心今日定,无穷顽性此时收。菩萨吩咐道:“悟空,你回去罢。好生伏侍唐
僧是,休懈惰生事。”行者道:“深感菩萨远来,弟子还当回送回送。”菩萨道:“免
送。”行者才捧着袈裟,叩头而别。菩萨亦带了熊罴,径回大海。有诗为证。诗曰:
祥光霭霭凝金象,万道缤纷实可夸。
普济世人垂悯恤,遍观法界现金莲。
今来多为传经意,此去原无落点瑕。
降怪成真归大海,空门复得锦袈裟。

毕竟不知向后事情如何,且听下回分解。