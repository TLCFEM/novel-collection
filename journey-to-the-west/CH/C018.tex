\chapter{观音院唐僧脱难~高老庄大圣除魔}

行者辞了菩萨,按落云头,将袈裟挂在香楠树上,掣出棒来,打入黑风洞里。
那洞里那得一个小妖?原来是他见菩萨出现,降得那老怪就地打滚,急急都散走了。
行者一发行凶,将他那几层门上,都积了干柴,前前后后,一齐发火,把个黑风洞
烧做个“红风洞”,却拿了袈裟,驾祥光,转回直北。

话说那三藏望行者急忙不来,心甚疑惑;不知是请菩萨不至,不知是行者托故
而逃。正在那胡猜乱想之中,只见半空中彩雾灿灿,行者忽坠阶前,叫道:“师父,
袈裟来了。”三藏大喜。众僧亦无不欢悦道:“好了!好了!我等性命,今日方才得全
了。”三藏接了袈裟道:“悟空,你早间去时,原约到饭罢晌午,如何此时日西方回?”
行者将那请菩萨施变化降妖的事情,备陈了一遍。三藏闻言,遂设香案,朝南礼拜
罢。道:“徒弟啊,既然有了佛衣,可快收拾包裹去也。”行者道:“莫忙,莫忙。
今日将晚,不是走路的时候,且待明日早行。”众僧们一齐跪下道:“孙老爷说得是:
一则天晚,二来我等有些愿心儿,今幸平安,有了宝贝,待我还了愿,请老爷散了
福,明早再送西行。”行者道:“正是,正是。”你看那些和尚,都倾囊倒底,把那
火里抢出的余资,各出所有,整顿了些斋供,烧了些平安无事的纸,念了几卷消灾
解厄的经。当晚事毕。

次早方刷扮了马匹,包裹了行囊出门。众僧远送方回。行者引路而去,正是那
春融时节。但见那:

草衬玉骢蹄迹软,柳摇金线露华新。桃杏满林争艳丽,薜萝绕径放精神。沙堤
日暖鸳鸯睡,山涧花香蛱蝶驯。这般秋去冬残春过半,不知何年行满得真文。
师徒们行了五七日荒路,忽一日天色将晚,远远的望见一村人家。三藏道:“悟空,
你看那壁厢有座山庄相近,我们去告宿一宵,明日再行何如?”行者道:“且等老
孙去看看吉凶,再作区处。”那师父挽住丝缰,这行者定睛观看,真个是:

竹篱密密,茅屋重重。参天野树迎门,曲水溪桥映户。道旁杨柳绿依依,园内
花开香馥馥。此时那夕照沉西,处处山林喧鸟雀;晚烟出爨,条条道径转牛羊。又
见那食饱鸡豚眠屋角,醉酣邻叟唱歌来。
行者看罢道:“师父请行。定是一村好人家,正可借宿。”

那长老催动白马,早到街衢之口。又见一个少年,头裹绵布,身穿蓝袄,持伞
背包,敛裈扎裤,脚踏着一双三耳草鞋,雄纠纠的,出街忙走。行者顺手一把扯住
道:“那里去?我问你一个信儿:此间是甚么地方?”那个人只管苦挣,口里嚷道:
“我庄上没人?只是我好问信!”行者陪着笑道:“施主莫恼。‘与人方便,自己方
便’。你就与我说说地名何害?我也可解得你的烦恼。”那人挣不脱手,气得乱跳道:
“蹭蹬,蹭蹬!家长的屈气受不了,又撞着这个光头,受他的清气!”行者道:“你
有本事,劈开我的手,你便就去了也罢。”那人左扭右扭,那里扭得动,却似一把
铁钤住一般,气得他丢了包袱,撇了伞,两只手,雨点似来抓行者。行者把一只
手扶着行李,一只手抵住那人,凭他怎么支吾,只是不能抓着。行者愈加不放,急
得爆燥如雷。三藏道:“悟空,那里不有人来了?你再问那人就是,只管扯住他怎的?
放他去罢。”行者笑道:“师父不知。若是问了别人没趣,须是问他,才有买卖。”
那人被行者扯住不过,只得说出道:“此处乃是乌斯藏国界之地,唤做高老庄。一
庄人家有大半姓高,故此唤做高老庄。你放了我去罢。”行者又道:“你这样行装,
不是个走近路的。你实与我说,你要往那里去,端的所干何事,我才放你。”

这人无奈,只得以实情告诉道:“我是高太公的家人,名叫高才。我那太公有
个老女儿,年方二十岁,更不曾配人,三年前被一个妖精占了。那妖整做了这三年
女婿。我太公不悦,说道:‘女儿招了妖精,不是长法:一则败坏家门,二则没个
亲家来往。’一向要退这妖精。那妖精那里肯退,转把女儿关在他后宅,将有半年,
再不放出与家内人相见。我太公与了我几两银子,教我寻访法师,拿那妖怪。我这
些时不曾住脚,前前后后,请了有三四个人,都是不济的和尚,脓包的道士,降不
得那妖精。刚才骂了我一场,说我不会干事,又与了我五钱银子做盘缠,教我再去
请好法师降他。不期撞着你这个纥刺星扯住,误了我走路,故此里外受气,我无奈,
才与你叫喊。不想你又有些拿法,我挣不过你,所以说此实情。你放我去罢。”行
者道:“你的造化,我有营生。这才是凑四合六的勾当。你也不须远行,莫要化费
了银子。我们不是那不济的和尚,脓包的道士,其实有些手段,惯会拿妖。这正是
‘一来照顾郎中,二来又医得眼好’。烦你回去上复你那家主,说我们是东土驾下
差来的御弟圣僧,往西天拜佛求经者,善能降妖缚怪。”高才道:“你莫误了我。我
是一肚子气的人,你若哄了我,没甚手段,拿不住那妖精,却不又带累我来受气?”
行者道:“管教不误了你。你引我到你家门首去来。”

那人也无计奈何,真个提着包袱,拿了伞,转步回身,领他师徒到于门首道:
“二位长老,你且在马台上略坐坐,等我进去报主人知道。”行者才放了手,落担
牵马,师徒们坐立门旁等候。

那高才入了大门,径往中堂上走,可可的撞见高太公。太公骂道:“你那个蛮
皮畜生,怎么不去寻人,又回来做甚?”高才放下包伞道:“上告主人公得知,小
人才行出街口,忽撞见两个和尚:一个骑马,一个挑担。他扯住我不放,问我那里
去。我再三不曾与他说及,他缠得没奈何,不得脱手,遂将主人公的事情,一一说
与他知。他却十分欢喜,要与我们拿那妖怪哩。”高老道:“是那里来的?”高才道:
“他说是东土驾下差来的御弟圣僧,前往西天拜佛求经的。”太公道:“既是远来的
和尚,怕不真有些手段。他如今在那里?”高才道:“现在门外等候。”

那太公即忙换了衣服,与高才出来迎接,叫声“长老”。三藏听见,急转身,
早已到了面前。那老者戴一顶乌绫巾,穿一领葱白蜀锦衣,踏一双糙米皮的犊子靴,
系一条黑绿绦子,出来笑语相迎,便叫:“二位长老,作揖了。”三藏还了礼,行者
站着不动。那老者见他相貌凶丑,便就不敢与他作揖。行者道:“怎么不唱老孙喏?”
那老儿有几分害怕,叫高才道:“你这小厮却不弄杀我也?家里现有一个丑头怪脑的
女婿打发不开,怎么又引这个雷公来害我?”行者道:“老高,你空长了许大年纪,
还不省事!若专以相貌取人,干净错了。我老孙丑自丑,却有些本事。替你家擒得
妖精,捉得鬼魅,拿住你那女婿,还了你女儿,便是好事,何必谆谆以相貌为言!”
太公见说,战兢兢的,只得强打精神,叫声“请进”。这行者见请,才牵了白马,
教高才挑着行李,与三藏进去。他也不管好歹,就把马拴在敞厅柱上,扯过一张退
光漆交椅,叫三藏坐下。他又扯过一张椅子,坐在旁边。那高老道:“这个小长老,
倒也家怀。”行者道:“你若肯留我住得半年,还家怀哩。”

坐定,高老问道:“适间小价说,二位长老是东土来的?”三藏道:“便是。贫
僧奉朝命往西天拜佛求经,因过宝庄,特借一宿,明日早行。”高老道:“二位原是
借宿的,怎么说会拿怪?”行者道:“因是借宿,顺便拿几个妖怪儿耍耍的。动问
府上有多少妖怪?”高老道:“天那!还吃得有多少哩!只这一个怪女婿,也被他磨
慌了!”行者道:“你把那妖怪的始末,有多大手段,从头儿说说我听,我好替你拿
他。”

高老道:“我们这庄上,自古至今,也不晓得有甚么鬼祟魍魉,邪魔作耗。只
是老拙不幸,不曾有子,止生三个女儿:大的唤名香兰,第二的名玉兰,第三的名
翠兰。那两个从小儿配与本庄人家,止有小的个,要招个女婿,指望他与我同家过
活,做个养老女婿,撑门抵户,做活当差。不期三年前,有一个汉子,模样儿倒也
精致,他说是福陵山上人家,姓猪,上无父母,下无兄弟,愿与人家做个女婿。我
老拙见是这般一个无根无绊的人,就招了他。一进门时,倒也勤谨:耕田耙地,不
用牛具;收割田禾,不用刀杖。昏去明来,其实也好;只是一件,有些会变嘴脸。”
行者道:“怎么变么?”高老道:“初来时,是一条黑胖汉,后来就变做一个长嘴大
耳朵的呆子,脑后又有一溜鬃毛,身体粗糙怕人,头脸就像个猪的模样。食肠却又
甚大:一顿要吃三五斗米饭;早间点心,也得百十个烧饼才够。喜得还吃斋素,若
再吃荤酒,便是老拙这些家业田产之类,不上半年,就吃个罄净!”三藏道:“只因
他做得,所以吃得。”高老道:“吃还是件小事,他如今又会弄风,云来雾去,走石
飞砂,唬得我一家并左邻右舍,俱不得安生。又把那翠兰小女关在后宅子里,一发
半年也不曾见面,更不知死活如何。因此知他是个妖怪,要请个法师与他去退去退。”
行者道:“这个何难?老儿你管放心,今夜管情与你拿住,教他写个退亲文书,还你
女儿如何?”高老大喜道:“我为招了他不打紧,坏了我多少清名,疏了我多少亲
眷;但得拿住他,要甚么文书?就烦与我除了根罢。”行者道:“容易!容易!入夜之
时,就见好歹。”

老儿十分欢喜,才教展抹桌椅,摆列斋供。斋罢,将晚,老儿问道:“要甚兵
器?要多少人随?趁早好备。”行者道:“兵器我自有。”老儿道:“二位只是那根锡杖,
锡杖怎么打得妖精?”行者随于耳内取出一个绣花针来,捻在手中,迎风幌了一幌,
就是碗来粗细的一根金箍铁棒,对着高老道:“你看这条棍子,比你家兵器如何?可
打得这怪否?”高老又道:“既有兵器,可要人跟?”行者道:“我不用人,只是要
几个年高有德的老儿,陪我师父清坐闲叙,我好撇他而去。等我把那妖精拿来,对
众取供,替你除了根罢。”那老儿即唤家僮,请了几个亲故朋友。一时都到。相见
已毕,行者道:“师父,你放心稳坐,老孙去也。”

你看他着铁棒,扯着高老道:“你引我去后宅子里,妖精的住处看看。”高老
遂引他到后宅门首。行者道:“你去取钥匙来。”高老道:“你且看看。若是用得钥
匙,却不请你了。”行者笑道:“你那老儿,年纪虽大,却不识耍。我把这话儿哄你
一哄,你就当真。”走上前,摸了一摸,原来是铜汁灌的锁子。狠得他将金箍棒一
捣,捣开门扇,里面却黑洞洞的。行者道:“老高,你去叫你女儿一声,看他可在
里面。”那老儿硬着胆叫道:“三姐姐。”那女儿认得是他父亲的声音,才少气无力
的应了一声道:“爹爹,我在这里哩。”行者闪金睛,向黑影里仔细看时,你道他怎
生模样?但见那:

云鬓乱堆无掠,玉容未洗尘淄。一片兰心依旧,十分娇态倾颓。樱唇全无气血,
腰肢屈屈偎偎。愁蹙蹙,蛾眉淡;瘦怯怯,语声低。
他走来看见高老,一把扯住,抱头大哭。行者道:“且莫哭!且莫哭!我问你,妖怪
往那里去了?”女子道:“不知往那里去。这些时,天明就去,入夜方来。云云雾
雾,往回不知何所。因是晓得父亲要祛退他,他也常常防备,故此昏来朝去。”行
者道:“不消说了。老儿,你带令爱往前边宅里,慢慢的叙阔,让老孙在此等他。
他若不来,你却莫怪;他若来了,定与你剪草除根。”那老高欢欢喜喜的,把女儿
带将前去。

行者却弄神通,摇身一变,变得就如那女子一般,独自个坐在房里等那妖精。
不多时,一阵风来,真个是走石飞砂。好风:

起初时微微荡荡,向后来渺渺茫茫。微微荡荡乾坤大,渺渺茫茫无阻碍。雕花
折柳胜摁麻,倒树摧林如拔菜。翻江搅海鬼神愁,裂石崩山天地怪。衔花糜鹿失来
踪,摘果猿猴迷在外。七层铁塔侵佛头,八面幢幡伤宝盖。金梁玉柱起根摇,房上
瓦飞如燕块。举棹梢公许愿心,开船忙把猪羊赛。当坊土地弃祠堂,四海龙王朝上
拜。海边撞损夜叉船,长城刮倒半边塞。
那阵狂风过处,只见半空里来了一个妖精,果然生得丑陋:黑脸短毛,长喙大耳;
穿一领青不青、蓝不蓝的梭布直裰,系一条花布手巾。行者暗笑道:“原来是这个
买卖!”好行者,却不迎他,也不问他,且睡在床上推病,口里哼哼的不绝。

那怪不识真假,走进房,一把搂住,就要亲嘴。行者暗笑道:“真个要来弄老
孙哩!”即使个拿法,托着那怪的长嘴,叫做个小跌。漫头一料,扑的掼下床来。
那怪爬起来,扶着床边道:“姐姐,你怎么今日有些怪我?想是我来得迟了?”行者
道:“不怪!不怪!”那妖道:“既不怪我,怎么就丢我这一跌?”行者道:“你怎么
就这等样小家子,就搂我亲嘴?我因今日有些不自在,若每常好时,便起来开门等
你了。你可脱了衣服睡是。”那怪不解其意,真个就去脱衣。行者跳起来,坐在净
桶上。那怪依旧复来床上摸一把,摸不着人,叫道:“姐姐,你往那里去了?请脱衣
服睡罢。”行者道:“你先睡,等我出个恭来。”那怪果先解衣上床。行者忽然叹口
气,道声“造化低了!”那怪道:“你恼怎的?造化怎么得低的?我得到了你家,虽是
吃了些茶饭,却也不曾白吃你的:我也曾替你家扫地通沟,搬砖运瓦,筑土打墙,
耕田耙地,种麦插秧,创家立业。如今你身上穿的锦,戴的金,四时有花果享用,
八节有蔬菜烹煎,你还有那些儿不趁心处,这般短叹长吁,说甚么造化低了!”行
者道:“不是这等说。今日我的父母,隔着墙,丢砖料瓦的,甚是打我骂我哩。”那
怪道:“他打骂你怎的?”行者道:“他说我和你做了夫妻,你是他门下一个女婿,
全没些儿礼体。这样个丑嘴脸的人,又会不得姨夫,又见不得亲戚,又不知你云来
雾去,端的是那里人家,姓甚名谁,败坏他清德,玷辱他门风,故此这般打骂,所
以烦恼。”那怪道:“我虽是有些儿丑陋,若要俊,却也不难。我一来时,曾与他讲
过,他愿意方才招我。今日怎么又说起这话!我家住在福陵山云栈洞。我以相貌为
姓,故姓猪,官名叫做猪刚鬣。他若再来问你,你就以此话与他说便了。”

行者暗喜道:“那怪却也老实,不用动刑,就供得这等明白。既有了地方、姓
名,不管怎的也拿住他。”行者道:“他要请法师来拿你哩。”那怪笑道:“睡着!睡
着!莫睬他!我有天罡数的变化,九齿的钉钯,怕甚么法师、和尚、道士?就是你老
子有虔心,请下九天荡魔祖师下界,我也曾与他做过相识,他也不敢怎的我。”行
者道:“他说请一个五百年前大闹天宫姓孙的齐天大圣,要来拿你哩。”那怪闻得这
个名头,就有三分害怕道:“既是这等说,我去了罢。两口子做不成了。”行者道:
“你怎的就去?”那怪道:“你不知道。那闹天宫的弼马温,有些本事,只恐我弄
他不过,低了名头,不像模样。”

他套上衣服,开了门,往外就走;被行者一把扯住,将自己脸上抹了一抹,现
出原身。喝道:“好妖怪,那里走!你抬头看看我是那个?”那怪转过眼来,看见行
者咨牙嘴,火眼金睛,磕头毛脸,就是个活雷公相似,慌得他手麻脚软,划剌的
一声,挣破了衣服,化狂风脱身而去。行者急上前,掣铁棒,望风打了一下。那怪
化万道火光,径转本山而去。行者驾云,随后赶来,叫声“那里走!你若上天,我
就赶到斗牛宫!你若入地,我就追至枉死狱!”

咦!毕竟不知这一去赶至何方,有何胜败,且听下回分解。