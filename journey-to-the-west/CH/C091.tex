\chapter{金平府元夜观灯~玄英洞唐僧供状}

修禅何处用工夫?马劣猿颠速剪除。
牢捉牢拴生五彩,暂停暂住堕三途。
若教自在神丹漏,才放从容玉性枯。
喜怒忧思须扫净,得玄得妙恰如无。

话表唐僧师徒四众离了玉华城,一路平稳,诚所谓极乐之乡。去有五六日程途,
又见一座城池。唐僧问行者道:“此又是甚么处所?”行者道:“是座城池。但城上
有杆无旗,不知地方,俟近前再问。”及至东关厢,见那两边茶坊酒肆喧哗,米市
油房热闹。街衢中有几个无事闲游的浪子,见猪八戒嘴长,沙和尚脸黑,孙行者眼
红,都拥拥簇簇的争看,只是不敢近前而问。唐僧捏着一把汗,惟恐他们惹祸。又
走过几条巷口,还不到城。忽见有一座山门,门上有“慈云寺”三字,唐僧道:“此
处略进去歇歇马,打一个斋如何?”行者道:“好,好。”四众遂一齐而入。但见那
里边:

珍楼壮丽,宝座峥嵘。佛阁高云外,僧房静月中。丹霞缥缈浮屠挺,碧树阴森
轮藏清。真净土,假龙宫,大雄殿上紫云笼。两廊不绝闲人戏,一塔常开有客登。
炉中香火时时,台上灯花夜夜荧。忽闻方丈金钟韵,应佛僧人朗诵经。

四众正看时,又见廊下走出一个和尚,对唐僧作礼道:“老师何来?”唐僧道:
“弟子中华唐朝来者。”那和尚倒身下拜,慌得唐僧搀起道:“院主何为行此大礼?”
那和尚合掌道:“我这里向善的人,看经念佛,都指望修到你中华地托生;才见老
师丰采衣冠,果然是前生修到的,方得此受用,故当下拜。”唐僧笑道:“惶恐!惶
恐!我弟子乃行脚僧,有何受用!若院主在此闲养自在,才是享福哩。”那和尚领唐
僧入正殿,拜了佛像。唐僧方才招呼:“徒弟来耶。”原来行者三人,自见那和尚与
师父讲话,他都背着脸,牵着马,守着担,立在一处,和尚不曾在心。忽的闻唐僧
叫“徒弟”,他三人方才转面。那和尚见了,慌得叫:“爷爷呀!你高徒如何恁般丑
样?”唐僧道:“丑则虽丑,倒颇有些法力。我一路甚亏他们保护。”

正说处,里面又走出几个和尚作礼。先见的那和尚对后的说道:“这老师是中
华大唐来的人物。那三位是他高徒。”从僧且喜且惧道:“老师中华大国,到此何
为?”唐僧言:“我奉唐王圣旨,向灵山拜佛求经。适过宝方,特奔上刹,一则求
问地方,二则打顿斋食就行。”那僧人个个欢喜,又邀入方丈。方丈里又有几个与
人家做斋的和尚。这先进去的又叫道:“你们都来看看中华人物。原来中华有俊的,
有丑的。俊的真个难描难画,丑的却十分古怪。”那许多僧同斋主都来相见,见毕,
各坐下。

茶罢,唐僧问道:“贵处是何地名?”众僧道:“我这里乃天竺国外郡,金平府
是也。”唐僧道:“贵府至灵山还有许多远近?”众僧道:“此间到都下有二千里。
这是我等走过的。西去到灵山,我们未走,不知还有多少路,不敢妄对。”唐僧谢
了。

少时,摆上斋来。斋罢,唐僧要行,却被众僧并斋主款留道:“老师宽住一二
日,过了元宵,耍耍去不妨。”唐僧惊问道:“弟子在路,只知有山,有水,怕的是
逢怪,逢魔,把光阴都错过了,不知几时是元宵佳节。”众僧笑道:“老师拜佛与悟
禅心重,故不以此为念。今日乃正月十三,到晚就试灯。后日十五上元。直到十八
九,方才谢灯。我这里人家好事,本府太守老爷爱民,各地方俱高张灯火,彻夜笙
箫。还有个‘金灯桥’,乃上古传留,至今丰盛。老爷们宽住数日,我荒山颇管待
得起。”唐僧无奈,遂俱住下。当晚只听得佛殿上钟鼓喧天,乃是街坊众信人等,
送灯来献佛。唐僧等都出方丈来看了灯,各自归寝。

次日,寺僧又献斋。吃罢,同步后园闲耍。果然好个去处。正是:

时维正月,岁届新春。园林幽雅,景物妍森。四时花木争奇,一派峰峦叠翠。
芳草阶前萌动,老梅枝上生馨。红入桃花嫩,青归柳色新。金谷园富丽休夸,辋川
图流风慢说。水流一道,野凫出没无常;竹种千竿,墨客推敲未定。芍药花、牡丹
花、紫薇花、含笑花,天机方醒;山茶花、红梅花、迎春花、瑞香花,艳质先开。
阴崖积雪犹含冻,远树浮烟已带春。又见那鹿向池边照影,鹤来松下听琴。东几厦,
西几亭,客来留宿;南几堂,北几塔,僧静安禅。花卉中,有一两座养性楼,重檐
高拱;山水内,有三四处炼魔室,静几明窗。真个是天然堪隐逸,又何须他处觅蓬
瀛。

师徒们玩赏一日,殿上看了灯,又都去看灯游戏。但见那:

玛瑙花城,琉璃仙洞,水晶云母诸宫:似重重锦绣,迭迭玲珑。星桥影幌乾坤
动,看数株火树摇红。六街箫鼓,千门璧月,万户香风。几处鳌峰高耸,有鱼龙出
海,鸾凤腾空。羡灯光月色,和气融融。绮罗队里,人人喜听笙歌,车马轰轰:看
不尽花容玉貌,风流豪侠,佳景无穷。
众等既在本寺里看了灯,又到东门厢各街上游戏。到二更时,方才回转安置。

次日,唐僧对众僧道:“弟子原有扫塔之愿,趁今日上元佳节,请院主开了塔
门,让弟子了此愿心。”众僧随开了门。沙僧取了袈裟,随从唐僧。到了一层,就
披了袈裟,拜佛祷祝毕,即将笤帚扫了一层,卸了袈裟,付与沙僧。又扫二层,一
层层直扫上绝顶。那塔上层层有佛,处处开窗,扫一层,赏玩赞美一层。扫毕下来,
已此天晚,又都点上灯火。

此夜正是十五元宵。众僧道:“老师父,我们前晚只在荒山与关厢看灯,今晚
正节,进城里看看金灯如何?”唐僧欣然从之,同行者三人及本寺多僧进城看灯。
正是:

三五良宵节,上元春色和。花灯悬闹市,齐唱太平歌。又见那六街三市灯亮,
半空一鉴初升。那月如冯夷推上烂银盘,这灯似仙女织成铺地锦。灯映月,增一倍
光辉;月照灯,添十分灿烂。观不尽铁锁星桥,看不了灯花火树。雪花灯、梅花灯,
春冰剪碎;绣屏灯、画屏灯,五彩攒成。核桃灯、荷花灯,灯楼高挂;青狮灯、白
象灯,灯架高檠。儿灯、鳖儿灯,棚前高弄;羊儿灯、兔儿灯,檐下精神。鹰儿
灯、凤儿灯,相连相并;虎儿灯、马儿灯,同走同行。仙鹤灯、白鹿灯,寿星骑坐;
金鱼灯、长鲸灯,李白高乘。鳌山灯,神仙聚会;走马灯,武将交锋。万千家灯火
楼台,十数里云烟世界。那壁厢,索琅琅玉飞来;这壁厢,毂辘辘香车辇过。看
那红妆楼上,倚着栏,隔着帘,并着肩,携着手,双双美女贪欢;绿水桥边,闹吵
吵,锦簇簇,醉醺醺,笑呵呵,对对游人戏彩。满城中箫鼓喧哗,彻夜里笙歌不断。
有诗为证,诗曰:
锦绣场中唱彩莲,太平境内簇人烟。
灯明月皎元宵夜,雨顺风调大有年。

此时正是金吾不禁。乱烘烘的,无数人烟。有那跳舞的,跷的,装鬼的,骑
象的,东一攒,西一簇,看之不尽。却才到金灯桥上,唐僧与众僧近前看处,原来
是三盏金灯。那灯有缸来大,上照着玲珑剔透的两层楼阁,都是细金丝儿编成;内
托着琉璃薄片,其光幌月,其油喷香。

唐僧回问众僧道:“此灯是甚油?怎么这等异香扑鼻?”众僧道:“老师不知。
我这府后有一县,名唤天县。县有二百四十里。每年审造差徭,共有二百四十家
灯油大户。府县的各项差徭犹可,惟有此大户甚是吃累:每家当一年,要使二百多
两银子。此油不是寻常之油,乃是酥合香油。这油每一两值价银二两,每一斤值三
十二两银子。三盏灯,每缸有五百斤,三缸共一千五百斤,共该银四万八千两。还
有杂项缴缠使用,将有五万余两,只点得三夜。”行者道:“这许多油,三夜何以就
点得尽?”众僧道:“这缸内每缸有四十九个大灯马,都是灯草扎的把,裹了丝绵,
有鸡子粗细;只点过今夜,见佛爷现了身,明夜油也没了,灯就昏了。”八戒在旁
笑道:“想是佛爷连油都收去了。”众僧道:“正是此说。满城里人家,自古及今,
皆是这等传说。但油干了,人俱说是佛祖收了灯,自然五谷丰登;若有一年不干,
却就年成荒旱,风雨不调。所以人家都要这供献。”

正说处,只听得半空中呼呼风响,唬得些看灯的人尽皆四散。那些和尚也立不
住脚道:“老师父,回去罢。风来了。是佛爷降祥,到此看灯也。”唐僧道:“怎见
得是佛来看灯?”众僧道:“年年如此,不上三更,就有风来。知道是诸佛降祥,
所以人皆回避。”唐僧道:“我弟子原是思佛念佛拜佛的人,今逢佳景,果有诸佛降
临,就此拜拜,多少是好。”众僧连请不回。

少时,风中果现出三位佛身,近灯来了。慌得那唐僧跑上桥顶,倒身下拜。行
者急忙扯起道:“师父,不是好人,必定是妖邪也。”说不了,见灯光昏暗,“呼”
的一声,把唐僧抱起,驾风而去。噫!不知是那山那洞真妖怪,积年假佛看金灯。
唬得那八戒两边寻找,沙僧左右招呼。行者叫道:“兄弟!不须在此叫唤。师父乐极
生悲,已被妖精摄去了!”那几个和尚害怕道:“爷爷,怎见得是妖精摄去?”行者
笑道:“原来你这伙凡人,累年不识,故被妖邪惑了,只说是真佛降祥,受此灯供。
刚才风到处,现佛身者,就是三个妖精。我师父亦不能识,上桥顶就拜,即被他侮
暗灯光,将器皿盛了油,连我师父都摄去。我略走迟了些儿,所以他三个化风而遁。”
沙僧道:“师兄,这般却如之何?”行者道:“不必迟疑。你两个同众回寺,看守马
匹、行李,等老孙趁此风追赶去也。”

好大圣,急纵筋斗云,起在半空,闻着那腥风之气,往东北上径赶。赶至天晓,
尔风息。见有一座大山,十分险峻,着实嵯峨。好山:

重重丘壑,曲曲源泉。藤萝悬削壁,松柏挺虚岩。鹤鸣晨雾里,雁唳晓云间。
峨峨矗矗峰排戟,突突磷磷石砌磐。顶巅高万仞,峻岭迭千湾。野花佳木知春发,
杜宇黄莺应景妍。能巍奕,实岩,古怪崎岖险又艰。停玩多时人不语,只听虎豹
有声鼾。香獐白鹿随来往,玉兔青狼去复还。深涧水流千万里,回湍激石响潺潺。
大圣在山崖上,正自找寻路径,只见四个人,赶着三只羊,从西坡下,齐吆喝“开
泰”。大圣闪火眼金睛,仔细观看,认得是年、月、日、时四值功曹使者,隐象化
形而来。

大圣即掣出铁棒,幌一幌,碗来粗细,有丈二长短,跳下崖来,喝道:“你都
藏头缩颈的那里走!”四值功曹见他说出风息,慌得喝散三羊,现了本相,闪下路
旁施礼道:“大圣,恕罪,恕罪!”行者道:“这一向也不曾用着你们,你们见老孙
宽慢,都一个个弄懈怠了,见也不来见我一见,是怎么说!你们不在暗中保吾师,
都往那里去?”功曹道:“你师父宽了禅性,在于金平府慈云寺贪欢,所以泰极生
否,乐盛成悲,今被妖邪捕获。他身边有护法伽蓝保着哩。吾等知大圣连夜追寻,
恐大圣不识山林,特来传报。”行者道:“你既传报,怎么隐姓埋名,赶着三个羊儿,
吆吆喝喝作甚?”功曹道:“设此三羊,以应开泰之言,唤做‘三阳开泰’,破解你
师之否塞也。”

行者恨恨的要打,见有此意,却就免之。收了棒,回嗔作喜道:“这座山,可
是妖精之处?”功曹道:“正是,正是。此山名青龙山。内有洞,名玄英洞。洞中
有三个妖精:大的个名辟寒大王,第二个号辟暑大王,第三个号辟尘大王,这妖精
在此有千年了。他自幼儿爱食酥合香油。当年成精,到此假装佛象,哄了金平府官
员人等,设立金灯,灯油用酥合香油。他年年到正月半,变佛像收油;今年见你师
父,他认得是圣僧之身,连你师父都摄在洞内,不日要割剐你师之肉,使酥合香油
煎吃哩。你快用工夫,救援去也。”

行者闻言,喝退四功曹,转过山崖,找寻洞府。行未数里,只见那涧边有一石
崖。崖下是座石屋。屋有两扇石门,半开半掩。门旁立有石碣,上有六字,却是“青
龙山玄英洞”。行者不敢擅入,立定步,叫声:“妖怪,快送我师父出来!”那里“唿
喇”一声,大开了门,跑出一阵牛头精,邓邓呆呆的问道:“你是谁,敢在这里呼
唤!”行者道:“我本是东土大唐取经的圣僧唐三藏之大徒弟。路过金平府观灯,我
师被你家魔头摄来,快早送还,免汝等性命!如或不然,掀翻你窝巢,教你群精都
化为脓血!”

那些小妖听言,急入里边报道:“大王!祸事了,祸事了!”三个老妖正把唐僧
拿在那洞中深远处,那里问甚么青红皂白,教小的选剥了衣裳,汲湍中清水洗净,
算计要细切细锉,着酥合香油煎吃。忽闻得报声“祸事”,老大着惊,问是何故。
小妖道:“大门前有一个毛脸雷公嘴的和尚嚷道:大王摄了他师父来,教快送出去,
免吾等性命;不然,就要掀翻窝巢,教我们都化为脓血哩!”那老妖听说,个个心
惊道:“才拿了这厮,还不曾问他个姓名来历。小的们,且把衣服与他穿了,带过
来审他一审,端是何人,何自而来也。”

众妖一拥上前,把唐僧解了索,穿了衣服,推至座前,唬得唐僧战兢兢的跪在
下面,只叫:“大王,饶命,饶命!”三个妖精,异口同声道:“你是那方来的和尚?
怎么见佛像不躲,却冲撞我的云路?”唐僧磕头道:“贫僧是东土大唐驾下差来的,
前往天竺国大雷音寺拜佛祖取经的。因到金平府慈云寺打斋,蒙那寺僧留过元宵看
灯。正在金灯桥上,见大王显现佛像,贫僧乃肉眼凡胎,见佛就拜,故此冲撞大王
云路。”

那妖精道:“你那东土到此,路程甚远;一行共有几众,都叫甚名字,快实实
供来,我饶你性命。”唐僧道:“贫僧俗名陈玄奘,自幼在金山寺为僧。后蒙唐皇敕
赐在长安洪福寺为僧官。又因魏徵丞相梦斩泾河老龙,唐王游地府,回生阳世,开
设水陆大会,超度阴魂,蒙唐王又选赐贫僧为坛主,大阐都纲。幸观世音菩萨出现,
指化贫僧,说西天大雷音寺有三藏真经,可以超度亡者升天,差贫僧来取,因赐号
三藏,即倚唐为姓,所以人都呼我为唐三藏。我有三个徒弟,大的个姓孙,名悟空
行者,乃齐天大圣归正。”群妖闻得此名,着了一惊道:“这个齐天大圣,可是五百
年前大闹天宫的?”唐僧道:“正是,正是。第二个姓猪,名悟能八戒,乃天蓬大
元帅转世。第三个姓沙,名悟净和尚,乃卷帘大将临凡。”三个妖王听说,个个心
惊道:“早是不曾吃他。小的们,且把唐僧将铁链锁在后面,待拿他三个徒弟来凑
吃。”遂点了一群山牛精、水牛精、黄牛精,各持兵器,走出门,掌了号头,摇旗
擂鼓。

三个妖披挂整齐,都到门外喝道:“是谁人敢在我这里吆喝!”行者闪在石崖上,
仔细观看。那妖精生得:

彩面环睛,二角峥嵘。尖尖四只耳,灵窍闪光明。一体花纹如彩画,满身锦绣
若蜚英。第一个,头顶狐裘花帽暖,一脸昂毛热气腾;第二个,身挂轻纱飞烈焰,
四蹄花莹玉玲玲;第三个,威雄声吼如雷振,獠牙尖利赛银针。个个勇而猛,手持
三样兵:一个使钺斧,一个大刀能;但看第三个,肩上横担挞藤。
又见那七长八短、七肥八瘦的大大小小妖精,都是些牛头鬼怪,各执枪棒。有三面
大旗,旗上明明书着“辟寒大王”、“辟暑大王”、“辟尘大王”。孙行者看了一会,
忍耐不得,上前高叫道:“泼贼怪!认得老孙么?”那妖喝道,“你是那闹天宫的孙
悟空?真个是‘闻名不曾见面,见面羞杀天神’!你原来是这等个猢狲儿,敢说大话!”
行者大怒,骂道:“我把你这个偷灯油的贼!油嘴妖怪,不要胡谈!快还我师父来!”
赶近前,轮铁棒就打。那三个老妖,举三般兵器,急架相迎。这一场在山凹中好杀:

钺斧钢刀挞藤,猴王一棒敢来迎。辟寒辟暑辟尘怪,认得齐天大圣名。棒起
致令神鬼怕,斧来刀砍乱飞腾。好一个混元有法真空像!抵住三妖假佛形。那三个
偷油润鼻今年犯,务捉钦差驾下僧。这个因师不惧山程远,那个为嘴常年设献灯。
乒乓只听刀斧响,劈朴惟闻棒有声。冲冲撞撞三攒一,架架遮遮各显能。一朝斗至
天将晚,不知那个亏输那个赢。
孙行者一条棒与那三个妖魔斗经百五十合,天色将晚,胜负未分,只见那辟尘大王
把挞藤闪一闪,跳过阵前,将旗摇了一摇,那伙牛头怪簇拥上前,把行者围在垓
心,各轮兵器,乱打将来。行者见事不谐,唿喇的纵起筋斗云,败阵而走。那妖更
不来赶,招回群妖,安排些晚食,众各吃了。也叫小妖送一碗与唐僧,只待拿住孙
行者等才要整治。那师父一则长斋,二则愁苦,哭啼啼的未敢沾唇不题。

却说行者驾云回至慈云寺内,叫声:“师弟。”那八戒、沙僧正自盼望商量,听
得叫时,一齐出接道:“哥哥,如何去这一日方回?端的师父下落何如?”行者笑道:
“昨夜闻风而赶,至天晓,到一山,不见。幸四值功曹传信道:那山叫做青龙山,
山中有一玄英洞。洞中有三个妖精,唤做辟寒大王、辟暑大王、辟尘大王。原来积
年在此偷油,假变佛像,哄了金平府官员人等。今年遇见我们,他不知好歹,反连
师父都摄去。老孙审得此情,吩咐功曹等众暗中保护师父,我寻近门前叫骂。那三
怪齐出,都像牛头鬼形。大的个使钺斧,第二个使大刀,第三个使藤棍。后引一窝
子牛头鬼怪,摇旗擂鼓,与老孙斗了一日,杀个手平。那妖王摇动旗,小妖都来,
我见天晚,恐不能取胜,所以驾筋斗回来也。”八戒道:“那里想是酆都城鬼王弄喧。”
沙僧道:“你怎么就猜道是酆都城鬼王弄喧?”八戒笑道:“哥哥说是牛头鬼怪,故
知之耳。”行者道:“不是,不是!若论老孙看那怪,是三只犀牛成的精。”八戒道:
“若是犀牛,且拿住他,锯下角来,倒值好几两银子哩!”

正说处,众僧道:“孙老爷可吃晚斋?”行者道:“方便吃些儿,不吃也罢。”
众僧道:“老爷征战这一日,岂不饥了?”行者笑道:“这日把儿那里便得饥!老孙
曾五百年不吃饮食哩!”众僧不知是实,只以为说笑。须臾拿来,行者也吃了;道:
“且收拾睡觉,待明日我等都去相持,拿住妖王,庶可救师父也。”沙僧在旁道:“哥
哥说那里话!常言道:‘停留长智。’那妖精倘或今晚不睡,把师父害了,却如之何?
不若如今就去,嚷得他措手不及,方才好救师父。少迟,恐有失也。”八戒闻言,
抖擞神威道:“沙兄弟说得是!我们都趁此月光去降魔耶!”行者依言,即吩咐寺僧:
“看守行李、马匹。待我等把妖精捉来,对本府刺史证其假佛,免却灯油,以苏概
县小民之困,却不是好?”众僧领诺,称谢不已。他三个遂纵起祥云,出城而去。
正是那:
懒散无拘禅性乱,灾危有分道心蒙。

毕竟不知此去胜败何如,且听下回分解。