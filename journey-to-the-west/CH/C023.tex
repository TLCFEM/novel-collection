\chapter{三藏不忘本~四圣试禅心}

诗曰:
奉法西来道路赊,秋风淅淅落霜花。
乖猿牢锁绳休解,劣马勤兜鞭莫加。
木母金公原自合,黄婆赤子本无差。
咬开铁弹真消息,般若波罗到彼家。

这回书,盖言取经之道,不离了一身务本之道也。却说他师徒四众,了悟真如,
顿开尘锁,自跳出性海流沙,浑无挂碍,径投大路西来。历遍了青山绿水,看不尽
野草闲花。真个也光阴迅速,又值九秋。但见了些:
枫叶满山红,黄花耐晚风。
老蝉吟渐懒,愁蟋思无穷。
荷破青纨扇,橙香金弹丛。
可怜数行雁,点点远排空。

正走处,不觉天晚。三藏道:“徒弟,如今天色又晚,却往那里安歇?”行者
道:“师父说话差了。出家人餐风宿水,卧月眠霜,随处是家。又问那里安歇,何
也?”猪八戒道:“哥啊,你只知道你走路轻省,那里管别人累坠?自过了流沙河,
这一向爬山过岭,身挑着重担;老大难挨也!须是寻个人家,一则化些茶饭,二则
养养精神,才是个道理。”行者道:“呆子,你这般言语,似有报怨之心。还像在高
老庄,倚懒不求福的自在,恐不能也。既是秉正沙门,须是要吃辛受苦,才做得徒
弟哩。”八戒道:“哥哥,你看这担行李多重?”行者道:“兄弟,自从有了你与沙
僧,我又不曾挑着,那知多重?”八戒道:“哥啊,你看看数儿么:

四片黄藤篾,长短八条绳。又要防阴雨,毡包三四层。匾担还愁滑,两头钉上
钉。铜镶铁打九环杖,篾丝藤缠大斗篷。
似这般许多行李,难为老猪一个逐日家担着走,偏你跟师父做徒弟,拿我做长工!”
行者笑道:“呆子,你和谁说哩?”八戒道:“哥哥,与你说哩。”行者道:“错和我
说了。老孙只管师父好歹,你与沙僧,专管行李、马匹。但若怠慢了些儿,孤拐上
先是一顿粗棍!”八戒道:“哥啊,不要说打,打就是以力欺人。我晓得你的尊性高
傲,你是定不肯挑;但师父骑的马,那般高大肥盛,只驮着老和尚一个,教他带几
件儿,也是弟兄之情。”

行者道:“你说他是马哩!他不是凡马,本是西海龙王敖闰之子,唤名龙马三太
子。只因纵火烧了殿上明珠,被他父亲告了忤逆,身犯天条,多亏观音菩萨救了他
的性命;他在那鹰愁陡涧,久等师父,又幸得菩萨亲临,却将他退鳞去角,摘了项
下珠,才变做这匹马,愿驮师父往西天拜佛。这个都是各人的功果,你莫攀他。”
那沙僧闻言道:“哥哥,真个是龙么?”行者道:“是龙。”八戒道:“哥啊,我闻得
古人云:‘龙能喷云嗳雾,播土扬沙:有巴山岭的手段,有翻江搅海的神通。’怎
么他今日这等慢慢而走?”行者道:“你要他快走,我教他快走个儿你看。”好大圣,
把金箍棒一,万道彩云生。那马看见拿棒,恐怕打来,慌得四只蹄疾如飞电,
飕的跑将去了。那师父手软勒不住,尽他劣性,奔上山崖,才大达步走。师父喘
息始定,抬头远见一簇松阴,内有几间房舍,着实轩昂。但见:

门垂翠柏,宅近青山。几株松冉冉,数茎竹斑斑。篱边野菊凝霜艳,桥畔幽兰
映水丹。粉泥墙壁,砖砌围圜。高堂多壮丽,大厦甚清安。牛羊不见无鸡犬,想是
秋收农事闲。

那师父正按辔徐观,又见悟空兄弟方到。悟净道:“师父不曾跌下马来么?”
长老骂道:“悟空这泼猴,他把马儿惊了,早是我还骑得住哩!”行者陪笑道:“师
父莫骂我,都是猪八戒说马行迟,故此着他快些。”那呆子因赶马,走急了些儿,
喘气嘘嘘,口里唧唧哝哝的闹道:“罢了!罢了!见自肚别腰松,担子沉重,挑不上
来,又弄我奔奔波波的赶马!”长老道:“徒弟啊,你且看那壁厢,有一座庄院,我
们却好借宿去也。”行者闻言,急抬头举目而看,果见那半空中庆云笼罩,瑞霭遮
盈。情知定是佛仙点化,他却不敢泄漏天机,只道:“好,好,好!我们借宿去来。”

长老连忙下马。见一座门楼,乃是垂莲象鼻,画栋雕梁。沙僧歇了担子。八戒
牵了马匹道:“这个人家,是过当的富实之家。”行者就要进去。三藏道:“不可,
你我出家人,各自避些嫌疑,切莫擅入。且自等他有人出来,以礼求宿,方可。”
八戒拴了马,斜倚墙根之下。三藏坐在石鼓上。行者、沙僧坐在台基边。久无人出。

行者性急,跳起身入门里看处:原来有向南的三间大厅,帘栊高控。屏门上,
挂一轴寿山福海的横披画;两边金漆柱上,贴着一幅大红纸的春联,上写着:
丝飘弱柳平桥晚,
雪点香梅小院春。
正中间,设一张退光黑漆的香几,几上放一个古铜兽炉。上有六张交椅。两山头挂
着四季吊屏。

行者正然偷看处,忽听得后门内有脚步之声,走出一个半老不老的妇人来,娇
声问道:“是甚么人,擅入我寡妇之门?”慌得个大圣喏喏连声道:“小僧是东土大
唐来的,奉旨向西方拜佛求经。一行四众,路过宝方,天色已晚。特奔老菩萨檀府,
告借一宵。”那妇人笑语相迎道:“长老,那三位在那里?请来。”行者高声叫道:“师
父,请进来耶。”三藏才与八戒、沙僧牵马挑担而入。只见那妇人出厅迎接。八戒
饧眼偷看,你道他怎生打扮:

穿一件织金官绿丝袄,上罩着浅红比甲;系一条结彩鹅黄锦绣裙,下映着高
底花鞋。时样髻皂纱漫,相衬着二色盘龙发;宫样牙梳朱翠晃,斜簪着两股赤金
钗。云鬓半苍飞凤翅,耳环双坠宝珠排;脂粉不施犹自美,风流还似少年才。

那妇人见了他三众,更加欣喜,以礼邀入厅房。一一相见礼毕,请各叙坐看茶。
那屏风后,忽有一个丫髻垂丝的女童,托着黄金盘、白玉盏,香茶喷暖气,异果散
幽香。那人绰彩袖,春笋纤长;擎玉盏,传茶上奉;对他们一一拜了。

茶毕,又吩咐办斋。三藏启手道:“老菩萨,高姓?贵地是甚地名?”妇人道:
“此间乃西牛贺洲之地。小妇人娘家姓贾,夫家姓莫。幼年不幸,公姑早亡,与丈
夫守承祖业。有家资万贯,良田千顷。夫妻们命里无子,止生了三个女孩儿。前年
大不幸,又丧了丈夫。小妇居孀,今岁服满。空遗下田产家业,再无个眷族亲人,
只是我娘女们承领。欲嫁他人,又难舍家业。适承长老下降,想是师徒四众。小妇
娘女四人,意欲坐山招夫,四位恰好。不知尊意肯否如何。”三藏闻言,推聋妆哑,
瞑目宁心,寂然不答。

那妇人道:“舍下有水田三百余顷,旱田三百余顷,山场果木三百余顷;黄水
牛有一千余只,骡马成群,猪羊无数;东南西北,庄堡草场,共有六七十处;家下
有八九年用不着的米谷,十来年穿不着的绫罗;一生有使不着的金银;胜强似那锦
帐藏春,说甚么金钗两行;你师徒们若肯回心转意,招赘在寒家,自自在在,享用
荣华,却不强如往西劳碌?”那三藏也只是如痴如蠢,默默无言。

那妇人道:“我是丁亥年三月初三日酉时生。故夫比我年大三岁,我今年四十
五岁。大女儿名真真,今年二十岁;次女名爱爱,今年十八岁;三小女名怜怜,今
年十六岁;俱不曾许配人家。虽是小妇人丑陋,却幸小女俱有几分颜色,女工针指,
无所不会。因是先夫无子,即把他们当儿子看养。小时也曾教他读些儒书,也都晓
得些吟诗作对。虽然居住山庄,也不是那十分粗俗之类,料想也配得过列位长老,
若肯放开怀抱,长发留头,与舍下做个家长,穿绫着锦,胜强如那瓦钵缁衣,雪鞋
云笠!”

三藏坐在上面,好便似雷惊的孩子,雨淋的虾蟆;只是呆呆挣挣,翻白眼儿打
仰。那八戒闻得这般富贵,这般美色,他却心痒难挠;坐在那椅子上,一似针戳屁
股,左扭右扭的,忍耐不住。走上前,扯了师父一把道:“师父!这娘子告诵你话,
你怎么佯佯不睬?好道也做个理会是。”那师父猛抬头,咄的一声,喝退了八戒道:
“你这个孽畜!我们是个出家人,岂以富贵动心,美色留意,成得个甚么道理!”

那妇人笑道:“可怜,可怜,出家人有何好处?”三藏道:“女菩萨,你在家人,
却有何好处?”那妇人道:“长老请坐,等我把在家人好处,说与你听。怎见得?有
诗为证,诗曰:
春裁方胜着新罗,夏换轻纱赏绿荷;
秋有新香糯酒,冬来暖阁醉颜酡。
四时受用般般有,八节珍羞件件多;
衬锦铺绫花烛夜,强如行脚礼弥陀。”
三藏道:“女菩萨,你在家人享荣华,受富贵,有可穿,有可吃,儿女团圆,果然
是好;但不知我出家的人,也有一段好处。怎见得?有诗为证,诗曰:
出家立志本非常,推倒从前恩爱堂。
外物不生闲口舌,身中自有好阴阳。
功完行满朝金阙,见性明心返故乡。
胜似在家贪血食,老来坠落臭皮囊。”

那妇人闻言,大怒道:“这泼和尚无礼!我若不看你东土远来,就该叱出。我倒
是个真心实意,要把家缘招赘汝等,你倒反将言语伤我。你就是受了戒,发了愿,
永不还俗,好道你手下人,我家也招得一个。你怎么这般执法?”

三藏见他发怒,只得者者谦谦,叫道:“悟空,你在这里罢。”行者道:“我从
小儿不晓得干那般事,教八戒在这里罢。”八戒道:“哥啊,不要栽人么。大家从长
计较。”三藏道:“你两个不肯,便教悟净在这里罢。”沙僧道:“你看师父说的话。
弟子蒙菩萨劝化,受了戒行,等候师父;自蒙师父收了我,又承教诲;跟着师父还
不上两月,更不曾进得半分功果,怎敢图此富贵!宁死也要往西天去,决不干此欺
心之事。”那妇人见他们推辞不肯,急抽身转进屏风,扑的把腰门关上。师徒们撇
在外面,茶饭全无,再没人出。

八戒心中焦燥,埋怨唐僧道:“师父忒不会干事,把话通说杀了。你好道还活
着些脚儿,只含糊答应,哄他些斋饭吃了,今晚落得一宵快活;明日肯与不肯,在
乎你我了。似这般关门不出,我们这清灰冷灶,一夜怎过!”

悟净道:“二哥,你在他家做个女婿罢。”八戒道:“兄弟,不要栽人。从长计
较。”行者道:“计较甚的?你要肯,便就教师父与那妇人做个亲家,你就做个倒踏
门的女婿。他家这等有财有宝,一定倒陪妆奁,整治个会亲的筵席。我们也落些受
用。你在此间还俗,却不是两全其美?”八戒道:“话便也是这等说,却只是我脱
俗又还俗,停妻再娶妻了。”

沙僧道:“二哥原来是有嫂子的?”行者道:“你还不知他哩,他本是乌斯藏高
老儿庄高太公的女婿。因被老孙降了,他也曾受菩萨戒行,没及奈何,被我捉他来
做个和尚,所以弃了前妻,投师父往西拜佛。他想是离别的久了,又想起那个勾当。
却才听见这个勾当,断然又有此心。呆子,你与这家子做了女婿罢。只是多拜老孙
几拜,我不检举你就罢了。”

那呆子道:“胡说!胡说!大家都有此心,独拿老猪出丑。常言道:‘和尚是色中
饿鬼。’那个不要如此?都这们扭扭捏捏的拿班儿,把好事都弄得裂了。这如今茶水
不得见面,灯火也无人管,虽熬了这一夜,但那匹马明日又要驮人,又要走路,再
若饿上这一夜,只好剥皮罢了。你们坐着,等老猪去放放马来。”那呆子虎急急的,
解了缰绳,拉出马去。

行者道:“沙僧,你且陪师父坐这里,等老孙跟他去,看他往那里放马。”三藏
道:“悟空,你看便去看他,但只不可只管嘲他了。”行者道:“我晓得。”这大圣走
出厅房,摇身一变,变作个红蜻蜓儿,飞出前门,赶上八戒。

那呆子拉着马,有草处且不教吃草,嗒嗒嗤嗤的,赶着马,转到后门首去。只
见那妇人,带了三个女子,在后门外闲立着,看菊花儿耍子。他娘女们看见八戒来
时,三个女儿闪将进去。那妇人伫立门首道:“小长老那里去?”这呆子丢了缰绳,
上前唱个喏,道声“娘!我来放马的。”那妇人道:“你师父忒弄精细。在我家招了
女婿,却不强似做挂搭僧,往西跄路?”八戒笑道:“他们是奉了唐王的旨意,不
敢有违君命,不肯干这件事。刚才都在前厅上栽我,我又有些奈上祝下的,只恐娘
嫌我嘴长耳大。”那妇人道:“我也不嫌,只是家下无个家长,招一个倒也罢了;但
恐小女儿有些儿嫌丑。”八戒道:“娘,你上复令爱,不要这等拣汉。想我那唐僧,
人才虽俊,其实不中用。我丑自丑,有几句口号儿。”妇人道:“你怎的说么?”八
戒道:“我

虽然人物丑,勤紧有些功。若言千顷地,不用使牛耕。只消一顿钯,布种及时
生。没雨能求雨,无风会唤风。房舍若嫌
矮,起上二三层。地下不扫扫一扫,阴沟不通通一通。家长里短诸般事,踢天弄井
我皆能。”
那妇人道:“既然干得家事,你再去与你师父商量商量看,不尴尬,便招你罢。”八
戒道:“不用商量:他又不是我的生身父母,干与不干,都在于我。”妇人道:“也
罢,也罢,等我与小女说。”看他闪进去,扑的掩上后门。八戒也不放马,将马拉
向前来。

怎知孙大圣已一一尽知,他转翅飞来,现了本相,先见唐僧道:“师父,悟能
牵马来了。”长老道:“马若不牵,恐怕撒欢走了。”行者笑将起来,把那妇人与八
戒说的勾当,从头说了一遍。三藏也似信不信的。

少时间,见呆子拉将马来拴下。长老道:“你马放了?”八戒道:“无甚好草,
没处放马。”行者道:“没处放马,可有处牵马么?”呆子闻得此言,情知走了消息,
也就垂头扭颈,努嘴皱眉,半晌不言。

又听得呀的一声,腰门开了,有两对红灯,一副提壶,香云霭霭,环叮叮,
那妇人带着三个女儿,走将出来,叫真真、爱爱、怜怜,拜见那取经的人物。那女
子排立厅中,朝上礼拜。果然也生得标致。但见他:

一个个蛾眉横翠,粉面生春。妖娆倾国色,窈窕动人心。花钿显现多娇态,绣
带飘迥绝尘。半含笑处樱桃绽,缓步行时兰麝喷。满头珠翠,颤巍巍无数宝钗簪;
遍体幽香,娇滴滴有花金缕细。说甚么楚娃美貌,西子娇容?真个是九天仙女从天
降,月里嫦娥出广寒。
那三藏合掌低头,孙大圣佯佯不睬,少沙僧转背回身。你看那猪八戒,眼不转睛,
淫心紊乱,色胆纵横,扭捏出悄语,低声道:“有劳仙子下降。娘,请姐姐们去耶。”
那三个女子,转入屏风,将一对纱灯留下。妇人道:“四位长老,可肯留心,着那
个配我小女么?”悟净道:“我们已商议了,着那个姓猪的招赘门下。”八戒道:“兄
弟,不要栽我,还从众计较。”行者道:“还计较甚么?你已是在后门首说合的停停
当当,‘娘’都叫了,又有甚么计较?师父做个男亲家,这婆儿做个女亲家,等老孙
做个保亲,沙僧做个媒人。也不必看通书,今朝是个天恩上吉日,你来拜了师父,
进去做了女婿罢。”八戒道:“弄不成,弄不成!那里好干这个勾当!”

行者道:“呆子,不要者嚣。你那口里‘娘’也不知叫了多少,又是甚么弄不
成。快快的应成,带携我们吃些喜酒,也是好处。”他一只手揪着八戒,一只手扯
住妇人道:“亲家母,带你女婿进去。”那呆子脚儿趄趄的,要往那里走。那妇人即
唤童子:“展抹桌椅,铺排晚斋,管待三位亲家。我领姑夫房里去也。”一壁厢又吩
咐庖丁排筵设宴,明晨会亲。那几个童子,又领命讫。他三众吃了斋,急急铺铺,
都在客座里安歇不题。

却说那八戒跟着丈母,行入里面,一层层也不知多少房舍,磕磕撞撞,尽都是
门槛绊脚。呆子道:“娘,慢些儿走。我这里边路生,你带我带儿。”那妇人道:“这
都是仓房、库房、碾房各房,还不曾到那厨房边哩。”八戒道:“好大人家!”磕磕
撞撞,转湾抹角,又走了半会,才是内堂房屋。那妇人道:“女婿,你师兄说今朝
是天恩上吉日,就教你招进来了;却只是仓卒间,不曾请得个阴阳,拜堂撒帐,你
可朝上拜八拜儿罢。”八戒道:“娘,娘说得是。你请上坐,等我也拜几拜,就当拜
堂,就当谢亲,两当一儿,却不省事?”他丈母笑道:“也罢,也罢,果然是个省
事干家的女婿。我坐着,你拜么。”

咦!满堂中银烛辉煌,这呆子朝上礼拜,拜毕,道:“娘,你把那个姐姐配我哩?”
他丈母道:“正是这些儿疑难:我要把大女儿配你,恐二女怪;要把二女配你,恐
三女怪;欲将三女配你,又恐大女怪,所以终疑未定。”八戒道:“娘,既怕相争,
都与我罢;省得闹闹吵吵,乱了家法。”他丈母道:“岂有此理,你一人就占我三个
女儿不成!”八戒道:“你看娘说的话。那个没有三房四妾?就再多几个,你女婿也
笑纳了。我幼年间,也曾学得个熬战之法,管情一个个伏侍得他欢喜。”那妇人道:
“不好,不好!我这里有一方手帕,你顶在头上,遮了脸,撞个天婚,教我女儿从
你跟前走过,你伸开手扯倒那个就把那个配了你罢。”呆子依言,接了手帕,顶在
头上。有诗为证,诗曰:
痴愚不识本原由,色剑伤身暗自休。
从来信有周公礼,今日新郎顶盖头。

那呆子顶裹停当。道:“娘,请姐姐们出来么。”他丈母叫:“真真、爱爱、怜
怜,都来撞天婚,配与你女婿。”只听得环响亮,兰麝馨香,似有仙子来往,那
呆子真个伸手去捞人。两边乱扑,左也撞不着,右也撞不着。来来往往,不知有多
少女子行动,只是莫想捞着一个。东扑抱着柱科,西扑摸着板壁。两头跑晕了,立
站不稳,只是打跌。前来蹬着门扇,后去汤着砖墙。磕磕撞撞,跌得嘴肿头青。坐
在地下。喘气的道:“娘啊,你女儿这等乖滑得紧,捞不着一个,奈何,奈何!”

那妇人与他揭了盖头道:“女婿,不是我女儿乖滑,他们大家谦让,不肯招你。”
八戒道:“娘啊,既是他们不肯招我啊,你招了我罢。”那妇人道:“好女婿呀!这等
没大没小的,连丈母也都要了!我这三个女儿,心性最巧。他一人结了一个珍珠嵌
锦汗衫儿。你若穿得那个的,就教那个招你罢。”八戒道:“好,好,好!把三件儿
都拿来我穿了看;若都穿得,就教都招了罢。”那妇人转进房里,止取出一件来,
递与八戒。那呆子脱下青锦布直裰,取过衫儿,就穿在身上;还未曾系上带子,扑
的一,跌倒在地。原来是几条绳紧紧绷住。那呆子疼痛难禁。这些人早已不见了。

却说三藏、行者、沙僧一觉睡醒,不觉的东方发白。忽睁睛抬头观看,那里得
那大厦高堂,也不是雕梁画栋,一个个都睡在松柏林中。慌得那长老忙呼行者。沙
僧道:“哥哥,罢了,罢了,我们遇着鬼了!”孙大圣心中明白,微微的笑道:“怎
么说?”长老道:“你看我们睡在那里耶!”行者道:“这松林下落得快活,但不知
那呆子在那里受罪哩。”长老道:“那个受罪?”行者笑道:“昨日这家子娘女们,
不知是那里菩萨,在此显化我等,想是半夜里去了,只苦了猪八戒受罪。”三藏闻
言,合掌顶礼。又只见那后边古柏树上,飘飘荡荡的,挂着一张简帖儿。沙僧急去
取来与师父看时,却是八句颂子云:
黎山老母不思凡,南海菩萨请下山。
普贤文殊皆是客,化成美女在林间。
圣僧有德还无俗,八戒无禅更有凡。
从此静心须改过,若生怠慢路途难!

那长老、行者、沙僧正然唱念此颂,只听得林深处高声叫道:“师父啊,绷杀
我了!救我一救,下次再不敢了!”三藏道:“悟空,那叫唤的可是悟能么?”沙僧
道:“正是。”行者道:“兄弟,莫睬他,我们去罢。”三藏道:“那呆子虽是心性愚
顽,却只是一味直,倒也有些膂力,挑得行李;还看当日菩萨之念,救他随我们
去罢。料他以后再不敢了。”那沙和尚却卷起铺盖,收拾了担子;孙大圣解缰牵马,
引唐僧入林寻看。咦!这正是:
从正修持须谨慎,扫除爱欲自归真。

毕竟不知那呆子凶吉如何,且听下回分解。