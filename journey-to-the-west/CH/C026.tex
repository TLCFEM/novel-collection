\chapter{孙悟空三岛求方~观世音甘泉活树}

诗曰:
处世须存心上刃,修身切记寸边而。
常言刃字为生意,但要三思戒怒欺。
上士无争传亘古,圣人怀德继当时。
刚强更有刚强辈,究竟终成空与非。

却说那镇元大仙用手搀着行者道:“我也知道你的本事,我也闻得你的英名,
只是你今番越理欺心,纵有腾那,脱不得我手。我就和你讲到西天,见了你那佛祖,
也少不得还我人参果树。你莫弄神通。”行者笑道:“你这先生,好小家子样!若要
树活,有甚疑难!早说这话,可不省了一场争竞?”大仙道:“不争竞,我肯善自饶
你!”行者道:“你解了我师父,我还你一颗活树如何?”大仙道:“你若有此神通,
医得树活,我与你八拜为交,结为兄弟。”行者道:“不打紧,放了他们,老孙管教
还你活树。”

大仙谅他走不脱,即命解放了三藏、八戒、沙僧。沙僧道:“师父啊,不知师
兄捣得是甚么鬼哩。”八戒道:“甚么鬼,这叫做‘当面人情鬼’!树死了,又可医
得活!他弄个光皮散儿好看,者着求医治树,单单了脱身走路,还顾得你和我哩!”
三藏道:“他决不敢撒了我们。我们问他那里求医去。”遂叫道:“悟空,你怎么哄
了仙长,解放我等?”行者道:“老孙是真言实语,怎么哄他?”三藏道:“你往何
处去求方?”行者道:“古人云:‘方从海上来。’我今要上东洋大海,遍游三岛十
洲,访问仙翁圣老,求一个起死回生之法,管教医得他树活。”三藏道:“此去几时
可回?”行者道:“只消三日。”三藏道:“既如此,就依你说,与你三日之限。三
日里来便罢;若三日之外不来,我就念那话儿经了。”行者道:“遵命,遵命。”

你看他急整虎皮裙,出门来对大仙道:“先生放心,我就去就来。你却要好生
伏侍我师父,逐日家三茶六饭,不可欠缺。若少了些儿,老孙回来和你算帐,先捣
塌你的锅底。衣服禳了,与他浆洗浆洗。脸儿黄了些儿,我不要;若瘦了些儿,不
出门。”那大仙道:“你去,你去,定不教他忍饿。”

好猴王,急纵斗云,别了五庄观,径上东洋大海。在半空中,快如掣电,疾
如流星,早到蓬莱仙境。按云头,仔细观看。真个好去处!有诗为证,诗曰:
大地仙乡列圣曹,蓬莱分合镇波涛。
瑶台影蘸天心冷,巨阙光浮海面高。
五色烟霞含玉籁,九霄星月射金鳌。
西池王母常来此,奉祝三仙几次桃。
那行者看不尽仙景,径入蓬莱。正然走处,见白云洞外,松阴之下,有三个老儿围
棋:观局者是寿星,对局者是福星、禄星。行者上前叫道:“老弟们,作揖了。”那
三星见了,拂退棋枰,回礼道:“大圣何来?”行者道:“特来寻你们耍子。”寿星
道:“我闻大圣弃道从释,脱性命保护唐僧往西天取经,遂日奔波山路,那些儿得
闲,却来耍子?”行者道:“实不瞒列位说,老孙因往西方,行在半路,有些儿阻
滞,特来小事欲干,不知肯否?”福星道:“是甚地方?是何阻滞?乞为明示,吾好
裁处。”行者道:“因路过万寿山五庄观有阻。”三老惊讶道:“五庄观是镇元大仙的
仙宫。你莫不是把他人参果偷吃了?”行者笑道:“偷吃了能值甚么?”三老道:“你
这猴子,不知好歹。那果子闻一闻,活三百六十岁;吃一个,活四万七千年:叫做
‘万寿草还丹’。我们的道,不及他多矣!他得之甚易,就可与天齐寿;我们还要养
精、炼气、存神,调和龙虎,捉坎填离,不知费多少工夫。你怎么说他的能值甚紧?
天下只有此种灵根!”行者道:“灵根,灵根,我已弄了他个断根哩!”三老惊道:“怎
的断根?”行者道:“我们前日在他观里,那大仙不在家,只有两个小童,接待了
我师父,却将两个人参果奉与我师。我师不认得,只说是三朝未满的孩童,再三不
吃。那童子就拿去吃了,不曾让得我们。是老孙就去偷了他三个,我三兄弟吃了。
那童子不知高低,贼前贼后的骂个不住。是老孙恼了,把他树打了一棍,推倒在地,
树上果子全无,桠开叶落,根出枝伤,已枯死了。不想那童子关住我们,又被老孙
扭开锁走了。次日清晨,那先生回家赶来,问答间,语言不和,遂与他赌斗;被他
闪一闪,把袍袖展开,一袖子都笼去了。绳缠索绑,拷问鞭敲,就打了一日。是夜
又逃了,他又赶上,依旧笼去。他身无寸铁,只是把个麈尾遮架。我兄弟这等三般
兵器,莫想打得着他。这一番仍旧摆布,将布裹漆了我师父与两师弟,却将我下油
锅。我又做了个脱身本事走了,把他锅都打破。他见拿我不住,尽有几分醋我。是
我又与他好讲,教他放了我师父、师弟,我与他医树管活,两家才得安宁。我想着
‘方从海上来’,故此特游仙境,访三位老弟。有甚医树的方儿,传我一个,急救
唐僧脱苦。”

三星闻言,心中也闷,道:“你这猴儿,全不识人。那镇元子乃地仙之祖;我
等乃神仙之宗,你虽得了天仙,还是太乙散数,未入真流,你怎么脱得他手?若是
大圣打杀了走兽飞禽,蜾虫鳞长,只用我黍米之丹,可以救活;那人参果乃仙木之
根,如何医治?没方,没方!”那行者见说无方,却就眉峰双锁,额蹙千痕。

福星道:“大圣,此处无方,他处或有,怎么就生烦恼?”行者道:“无方别访,
果然容易;就是游遍海角天涯,转透三十六天,亦是小可;只是我那唐长老法严量
窄,止与了我三日期限。三日以外不到,他就要念那紧箍儿咒哩。”三星笑道:“好,
好,好!若不是这个法儿拘束你,你又钻天了。”寿星道:“大圣放心,不须烦恼。
那大仙虽称上辈,却也与我等有识。一则久别,不曾拜望;二来是大圣的人情:如
今我三人同去望他一望,就与你道达此情,教那唐和尚莫念紧箍儿咒,休说三日五
日,只等你求得方来,我们才别。”行者道:“感激,感激!就请三位老弟行行,我
去也。”大圣辞别三星不题。

却说这三星驾起祥光,即往五庄观而来。那观中合众人等,忽听得长天鹤唳,
原来是三老光临。但见那:

盈空蔼蔼祥光簇,霄汉纷纷香馥郁。彩雾千条护羽衣,轻云一朵擎仙足。青鸾
飞,丹凤,袖引香风满地扑。拄杖悬龙喜笑生,皓髯垂玉胸前拂。童颜欢悦更无
忧,壮体雄威多有福。执星筹,添海屋,腰挂葫芦并宝篆。万纪千旬福寿长,十洲
三岛随缘宿。常来世上送千祥,每向人间增百福。概乾坤,荣福禄,福寿无疆今喜
得。三老乘祥谒大仙,福堂和气皆无极。
那仙童看见,即忙报道:“师父,海上三星来了。”镇元子正与唐僧师弟闲叙,闻报,
即降阶奉迎。那八戒见了寿星,近前扯住,笑道:“你这肉头老儿,许久不见,还
是这般脱洒,帽儿也不带个来。”遂把自家一个僧帽,扑的套在他头上,扑着手呵
呵大笑道:“好,好,好!真是‘加冠进禄’也!”那寿星将帽子掼了,骂道:“你这
个夯货,老大不知高低!”八戒道:“我不是夯货,你等真是奴才!”福星道:“你倒
是个夯货,反敢骂人是奴才!”八戒又笑道:“既不是人家奴才,好道叫做‘添寿’、
‘添福’、‘添禄’?”

那三藏喝退了八戒,急整衣拜了三星。那三星以晚辈之礼见了大仙,方才叙坐。
坐定,禄星道:“我们一向久阔尊颜,有失恭敬。今因孙大圣搅扰仙山,特来相见。”
大仙道:“孙行者到蓬莱去的?”寿星道:“是,因为伤了大仙的丹树,他来我处求
方医治。我辈无方,他又到别处求访;但恐违了圣僧三日之限,要念紧箍儿咒。我
辈一来奉拜,二来讨个宽限。”三藏闻言,连声应道:“不敢念,不敢念。”

正说处,八戒又跑进来,扯住福星,要讨果子吃。他去袖里乱摸,腰里乱吞,
不住的揭他衣服搜检。三藏笑道:“那八戒是甚么规矩!”八戒道:“不是没规矩,
此叫做‘番番是福’。”三藏又叱令出去。那呆子出门,瞅着福星,眼不转睛的发
狠。福星道:“夯货!我那里恼了你来,你这等恨我?”八戒道:“不是恨你,这叫
‘回头望福’。”那呆子出得门来,只见一个小童,拿了四把茶匙,方去寻锺取果看
茶,被他一把夺过,跑上殿,拿着小磬儿,用手乱敲乱打,两头玩耍。大仙道:“这
个和尚,越发不尊重了!”八戒笑道:“不是不尊重,这叫做‘四时吉庆’。”

且不说八戒打诨乱缠。却表行者纵祥云离了蓬莱,又早到方丈仙山。这山真好
去处。有诗为证,诗曰:
方丈巍峨别是天,太元宫府会神仙。
紫台光照三清路,花木香浮五色烟。
金凤自多蕊阙,玉膏谁逼灌芝田?
碧桃紫李新成熟,又换仙人信万年。
那行者按落云头,无心玩景。正走处,只闻得香风馥馥,玄鹤声鸣,那壁厢有个神
仙。但见:

盈空万道霞光现,彩雾飘光不断。丹凤衔花也更鲜,青鸾飞舞声娇艳。福如
东海寿如山,貌似小童身体健。壶隐洞天不老丹,腰悬与日长生篆。人间数次降祯
祥,世上几番消厄愿。武帝曾宣加寿龄,瑶池每赴蟠桃宴。教化众僧脱俗缘,指开
大道明如电。也曾跨海祝千秋,常去灵山参佛面。圣号东华大帝君,烟霞第一神仙
眷。
孙行者觌面相迎,叫声:“帝君,起手了。”那帝君慌忙回礼道:“大圣,失迎。请
荒居奉茶。”遂与行者搀手而入。果然是贝阙仙宫,看不尽瑶池琼阁。方坐待茶,
只见翠屏后转出一个童儿。他怎生打扮:

身穿道服飘霞烁,腰束丝绦光错落。头戴纶巾布斗星,足登芒履游仙岳。炼元
真,脱本壳,功行成时遂意乐。识破原流精气神,主人认得无虚错。逃名今喜寿无
疆,甲子周天管不着。转回廊,登宝阁,天上蟠桃三度摸。缥缈香云出翠屏,小仙
乃是东方朔。
行者见了,笑道:“这个小贼在这里哩!帝君处没有桃子你偷吃!”东方朔朝上进礼,
答道:“老贼,你来这里怎的?我师父没有仙丹你偷吃。”帝君叫道:“曼倩休乱言,
看茶来也。”

曼倩原是东方朔的道名。他急入里取茶二杯,饮讫。行者道:“老孙此来,有
一事奉干,未知允否?”帝君道:“何事?自当领教。”行者道:“近因保唐僧西行,
路过万寿山五庄观,因他那小童无状,是我一时发怒,把他人参果树推倒,因此阻
滞,唐僧不得脱身,特来尊处求赐一方医治,万望慨然。”

帝君道:“你这猴子,不管一二,到处里闯祸。那五庄观镇元子,圣号与世同
君,乃地仙之祖。你怎么就冲撞出他?他那人参果树,乃草还丹。你偷吃了,尚说
有罪;却又连树推倒,他肯干休?”行者道:“正是呢。我们走脱了,被他赶上,
把我们就当汗巾儿一般,一袖子都笼了去;所以阁气。没奈何,许他求方医治,故
此拜求。”帝君道:“我有一粒‘九转太乙还丹’,但能治世间生灵,却不能医树。
树乃水土之灵,天滋地润。若是凡间的果木,医治还可;这万寿山乃先天福地,五
庄观乃贺洲洞天,人参果又是天开地辟之灵根,如何可治?无方,无方!”

行者道:“既然无方,老孙告别。”帝君仍欲留奉玉液一杯,行者道:“急救事
紧,不敢久滞。”遂驾云复至瀛洲海岛。也好去处。有诗为证,诗曰:
珠树玲珑照紫烟,瀛洲宫阙接诸天。
青山绿水琪花艳,玉液锟铁石坚。
五色碧鸡啼海日,千年丹凤吸朱烟。
世人罔究壶中景,象外春光亿万年。
那大圣至瀛洲,只见那丹崖珠树之下,有几个皓发皤髯之辈,童颜鹤鬓之仙,在那
里着棋饮酒,谈笑讴歌。真个是:

祥云光满,瑞霭香浮。彩鸾鸣洞口,玄鹤舞山头。碧藕水桃为按酒,交梨火枣
寿千秋。一个个丹诏无闻,仙符有籍;逍遥随浪荡,散淡任清幽。周天甲子难拘管,
大地乾坤只自由。献果玄猿,对对参随多美爱;衔花白鹿,双双拱伏甚绸缪。
那些老儿,正然洒乐。这行者厉声高叫道:“带我耍耍儿便怎的!”众仙见了,急忙
趋步相迎。有诗为证,诗曰:
人参果树灵根折,大圣访仙求妙诀。
缭绕丹霞出宝林,瀛洲九老来相接。
行者认得是九老,笑道:“老兄弟们自在哩!”九老道:“大圣当年若存正,不闹天
宫,比我们还自在哩。如今好了,闻你归真向西拜佛,如何得暇至此?”行者将那
医树求方之事,具陈了一遍。九老也大惊道:“你也忒惹祸,惹祸!我等实是无方。”
行者道:“既是无方,我且奉别。”

九老又留他饮琼浆,食碧藕。行者定不肯坐,止立饮了他一杯浆,吃了一块藕,
急急离了瀛洲,径转东洋大海。早望见落伽山不远,遂落下云头,直到普陀岩上。
见观音菩萨在紫竹林中与诸天大神、木叉、龙女,讲经说法。有诗为证,诗曰:
海主城高瑞气浓,更观奇异事无穷。
须知隐约千般外,尽出希微一品中。
四圣授时成正果,六凡听后脱樊笼。
少林别有真滋味,花果馨香满树红。

那菩萨早已看见行者来到,即命守山大神去迎。那大神出林来,叫声“孙悟空,
那里去?”行者抬头喝道:“你这个熊罴,我是你叫的悟空!当初不是老孙饶了你,
你已此做了黑风山的尸鬼矣。今日跟了菩萨,受了善果,居此仙山,常听法教,你
叫不得我一声‘老爷’?”那黑熊真个得了正果,在菩萨处镇守普陀,称为大神,
是也亏了行者。他只得陪笑道:“大圣,古人云:‘君子不念旧恶。’只管题他怎的!
菩萨着我来迎你哩。”这行者就端肃尊诚,与大神到了紫竹林里,参拜菩萨。

菩萨道:“悟空,唐僧行到何处也?”行者道:“行到西牛贺洲万寿山了。”菩
萨道:“那万寿山有座五庄观。镇元大仙,你曾会他么?”行者顿首道:“因是在五
庄观,弟子不识镇元大仙,毁伤了他的人参果树,冲撞了他,他就困滞了我师父,
不得前进。”那菩萨情知,怪道:“你这泼猴,不知好歹!他那人参果树,乃天开地
辟的灵根;镇元子乃地仙之祖,我也让他三分;你怎么就打伤他树!”行者再拜道:
“弟子实是不知。那一日,他不在家,只有两个仙童,候待我等。是猪悟能晓得他
有果子,要一个尝新,弟子委偷了他三个,兄弟们分吃了。那童子知觉,骂我等无
已,是弟子发怒,遂将他树推倒。他次日回来赶上,将我等一袖子笼去,绳绑鞭抽,
拷打了一日。我等当夜走脱,又被他赶上,依然笼了。三番两次,其实难逃,已允
了与他医树。却才自海上求方,遍游三岛,众神仙都没有本事。弟子因此志心朝礼,
特拜告菩萨。伏望慈悯,俯赐一方,以救唐僧早早西去。”菩萨道:“你怎么不早来
见我,却往岛上去寻找?”

行者闻得此言,心中暗喜道:“造化了,造化了,菩萨一定有方也!”他又上前
恳求。菩萨道:“我这净瓶底的‘甘露水’,善治得仙树灵苗。”行者道:“可曾经验
过么?”菩萨道:“经验过的。”行者问:“有何经验?”菩萨道:“当年太上老君曾
与我赌胜:他把我的杨柳枝拔了去,放在炼丹炉里,炙得焦干,送来还我。是我拿
了插在瓶中,一昼夜,复得青枝绿叶,与旧相同。”行者笑道:“真造化了,真造化
了!烘焦了的尚能医活,况此推倒的,有何难哉!”菩萨吩咐大众:“看守林中,我
去去来。”遂手托净瓶,白鹦哥前边巧啭,孙大圣随后相从。有诗为证,诗曰:
玉毫金象世难论,正是慈悲救苦尊。
过去劫逢无垢佛,至今成得有为身。
几生欲海澄清浪,一片心田绝点尘。
甘露久经真妙法,管教宝树永长春。

却说那观里大仙与三老正然清话,忽见孙大圣按落云头,叫道:“菩萨来了,
快接,快接!”慌得那三星与镇元子共三藏师徒,一齐迎出宝殿。菩萨才住了祥云,
先与镇元子陪了话;后与三星作礼。礼毕上坐。那阶前,行者引唐僧、八戒、沙僧
都拜了。那观中诸仙,也来拜见。行者道:“大仙不必迟疑,趁早儿陈设香案,请
菩萨替你治那甚么果树去。”大仙躬身谢菩萨道:“小可的勾当,怎么敢劳菩萨下
降?”菩萨道:“唐僧乃我之弟子,孙悟空冲撞了先生,理当赔偿宝树。”三老道:
“既如此,不须谦讲了。请菩萨都到园中去看看。”

那大仙即命设具香案,打扫后园,请菩萨先行。三老随后。三藏师徒与本观众
仙,都到园内观看时,那棵树倒在地下,土开根现,叶落枝枯。菩萨叫:“悟空,
伸手来。”那行者将左手伸开。菩萨将杨柳枝,蘸出瓶中甘露,把行者手心里画了
一道起死回生的符字,教他放在树根之下,但看水出为度。那行者捏着拳头,往那
树根底下揣着;须臾,有清泉一汪。菩萨道:“那个水不许犯五行之器,须用玉瓢
舀出,扶起树来,从头浇下,自然根皮相合,叶长芽生,枝青果出。”行者道:“小
道士们,快取玉瓢来。”镇元子道:“贫道荒山,没有玉瓢,只有玉茶盏、玉酒杯,
可用得么?”菩萨道:“但是玉器,可舀得水的便罢,取将来看。”大仙即命小童子
取出有二三十个茶盏,四五十个酒盏,却将那根下清泉舀出。行者、八戒、沙僧,
扛起树来,扶得周正,拥上土,将玉器内甘泉,一瓯瓯捧与菩萨。菩萨将杨柳枝细
细洒上,口中又念着经咒。不多时,洒净那舀出之水,只见那树果然依旧青绿叶阴
森,上有二十三个人参果。清风、明月二童子道:“前日不见了果子时,颠倒只数
得二十二个;今日回生,怎么又多了一个?”行者道:“‘日久见人心’。前日老孙
只偷了三个,那一个落下地来,土地说这宝遇土而入,八戒只嚷我打了偏手,故走
了风信,只缠到如今,才见明白。”

菩萨道:“我方才不用五行之器者,知道此物与五行相畏故耳。”那大仙十分欢
喜,急令取金击子来,把果子敲下十个,请菩萨与三老复回宝殿,一则谢劳,二来
做个“人参果会”。众小仙遂调开桌椅,铺设丹盘,请菩萨坐了上面正席,三老左
席,唐僧右席,镇元子前席相陪,各食了一个。有诗为证,诗曰:
万寿山中古洞天,人参一熟九千年。
灵根现出芽枝损,甘露滋生果叶全。
三老喜逢皆旧契,四僧幸遇是前缘。
自今会服人参果,尽是长生不老仙。
此时菩萨与三老各吃了一个,唐僧始知是仙家宝贝,也吃了一个。悟空三人,亦各
吃一个。镇元子陪了一个。本观仙众分吃了一个。行者才谢了菩萨回上普陀岩,送
三星径转蓬莱岛。镇元子却又安排蔬酒,与行者结为兄弟。这才是不打不成相识,
两家合了一家。师徒四众,喜喜欢欢,天晚歇了。那长老才是:
有缘吃得草还丹,长寿苦捱妖怪难。

毕竟到明日如何作别,且听下回分解。