\chapter{荆棘岭悟能努力~木仙庵三藏谈诗}

话表祭赛国王谢了唐三藏师徒获宝擒怪之恩。所赠金玉,分毫不受。却命当驾
官照依四位常穿的衣服,各做两套,鞋袜各做两双,绦环各做两条,外备干粮烘炒,
倒换了通关文牒,大排銮驾,并文武多官、满城百姓、伏龙寺僧人,大吹大打,送
四众出城。约有二十里,先辞了国王。众人又送二十里辞回。伏龙寺僧人,送有五
六十里不回。有的要同上西天,有的要修行伏侍。行者见都不肯回去,遂弄个手段,
把毫毛拔了三四十根,吹口仙气,叫:“变!”都变作斑斓猛虎,拦住前路,哮吼踊
跃。众僧方惧,不敢前进。大圣才引师父策马而去。少时间,去得远了。众僧人放
声大哭,都喊:“有恩有义的老爷!我等无缘,不肯度我们也!”

且不说众僧啼哭。却说师徒四众,走上大路,却才收回毫毛,一直西去。正是
时序易迁,又早冬残春至,不暖不寒,正好逍遥行路。忽见一条长岭,岭顶上是路。
三藏勒马观看,那岭上荆棘丫叉,薜萝牵绕。虽是有道路的痕迹,左右却都是荆刺
棘针。唐僧叫:“徒弟,这路怎生走得?”行者道:“怎么走不得?”又道:“徒弟
啊,路痕在下,荆棘在上,只除是蛇虫伏地而游,方可去了;若你们走,腰也难伸,
教我如何乘马?”八戒道:“不打紧,等我使出钯柴手来,把钉钯分开荆棘,莫说
乘马,就抬轿也包你过去。”三藏道:“你虽有力,长远难熬。却不知有多少远近,
怎生费得这许多精神!”行者道:“不须商量,等我去看看。”将身一纵,跳在半空
看时,一望无际。真个是:

匝地远天,凝烟带雨。夹道柔茵乱,漫山翠盖张。密密搓搓初发叶,攀攀扯扯
正芬芳。遥望不知何所尽,近观一似绿云茫。蒙蒙茸茸,郁郁苍苍。风声飘索索,
日影映煌煌。那中间有松有柏还有竹,多梅多柳更多桑。薜萝缠古树,藤葛绕垂杨。
盘团似架,联络如床。有处花开真布锦,无端卉发远生香。为人谁不遭荆棘,那见
西方荆棘长!
行者看罢多时,将云头按下道:“师父,这去处远哩!”三藏问:“有多少远?”行
者道:“一望无际,似有千里之遥。”三藏大惊道:“怎生是好?”沙僧笑道:“师父
莫愁,我们也学烧荒的,放上一把火,烧绝了荆棘过去。”八戒道:“莫乱谈!烧荒
的须在十来月,草衰木枯,方好引火。如今正是蕃盛之时,怎么烧得!”行者道:“就
是烧得,也怕人子。”三藏道:“这般怎生得度?”八戒笑道:“要得度,还依我。”

好呆子,捻个诀,念个咒语,把腰躬一躬,叫“长!”就长了有二十丈高下的
身躯;把钉钯幌一幌,教“变!”就变了有三十丈长短的钯柄;拽开步,双手使钯,
将荆棘左右搂开:“请师父跟我来也!”三藏见了甚喜,即策马紧随。后面沙僧挑着
行李,行者也使铁棒拨开。这一日未曾住手;行有百十里,将次天晚,见有一块空
阔之处。当路上有一通石碣,上有三个大字,乃“荆棘岭”;下有两行十四个小字,
乃“荆棘蓬攀八百里,古来有路少人行”。八戒见了,笑道:“等我老猪与他添上两
句:‘自今八戒能开破,直透西方路尽平!’”三藏欣然下马道:“徒弟啊,累了你也!
我们就在此住过了今宵,待明日天光再走。”八戒道:“师父莫住,趁此天色晴明,
我等有兴,连夜搂开路走他娘!”那长老只得相从。

八戒上前努力。师徒们,人不住手,马不停蹄,又行了一日一夜,却又天色晚
矣。那前面蓬蓬结结,又闻得风敲竹韵,飒飒松声。却好又有一段空地,中间乃是
一座古庙。庙门之外,有松柏凝青,桃梅斗丽。三藏下马,与三个徒弟同看。只见:
岩前古庙枕寒流,落目荒烟锁废丘。
白鹤丛中深岁月,绿芜台下自春秋。
竹摇青疑闻语,鸟弄余音似诉愁。
鸡犬不通人迹少,闲花野蔓绕墙头。
行者看了道:“此地少吉多凶,不宜久坐。”沙僧道:“师兄差疑了。似这杳无人烟
之处,又无个怪兽妖禽,怕他怎的?”

说不了,忽见一阵阴风,庙门后,转出一个老者,头戴角巾,身穿淡服,手持
拐杖,足踏芒鞋,后跟着一个青脸獠牙,红须赤身鬼使,头顶着一盘面饼,跪下道:
“大圣,小神乃荆棘岭土地。知大圣到此,无以接待,特备蒸饼一盘,奉上老师父,
各请一餐。此地八百里,更无人家,聊吃些儿充饥。”八戒欢喜,上前舒手,就欲
取饼。不知行者端详已久,喝一声:“且住,这厮不是好人,休得无礼!你是甚么土
地,来诳老孙!看棍!”那老者见他打来,将身一转,化作一阵阴风,呼的一声,把
个长老摄将起来,飘飘荡荡,不知摄去何所。慌得那大圣没跟寻处,八戒、沙僧俱
相顾失色,白马亦只自惊吟。三兄弟连马四口,恍恍惚惚,远望高张,并无一毫下
落,前后找寻不题。

却说那老者同鬼使,把长老抬到一座烟霞石屋之前,轻轻放下。与他携手相搀
道:“圣僧休怕。我等不是歹人,乃荆棘岭十八公是也。因风清月霁之宵,特请你
来会友谈诗,消遣情怀故耳。”那长老却才定性,睁眼仔细观看。真个是:

漠漠烟云去所,清清仙境人家。正好洁身修炼,堪宜种竹栽花。每见翠岩来鹤,
时闻青沼鸣蛙。更赛天台丹灶,仍期华岳明霞。说甚耕云钓月,此间隐逸堪夸。坐
久幽怀如海,朦胧月上窗纱。

三藏正自点看,渐觉月明星朗,只听得人语相谈。都道:“十八公请得圣僧来
也。”长老抬头观看,乃是三个老者:前一个霜姿丰采,第二个绿鬓婆娑,第三个
虚心黛色。各各面貌、衣服俱不相同,都来与三藏作礼。长老还了礼,道:“弟子
有何德行,敢劳列位仙翁下爱?”十八公笑道:“一向闻知圣僧有道,等待多时,
今幸一遇。如果不吝珠玉,宽坐叙怀,足见禅机真派。”三藏躬身道:“敢问仙翁尊
号?”十八公道:“霜姿者号孤直公,绿鬓者号凌空子,虚心者号拂云叟。老拙号
曰劲节。”三藏道:“四翁尊寿几何?”孤直公道:
“我岁今经千岁古,撑天叶茂四时春。
香枝郁郁龙蛇状,碎影重重霜雪身。
自幼坚刚能耐老,从今正直喜修真。
乌栖凤宿非凡辈,落落森森远俗尘。”
凌空子笑道:
“吾年千载傲风霜,高干灵枝力自刚。
夜静有声如雨滴,秋晴荫影似云张。
盘根已得长生诀,受命尤宜不老方。
留鹤化龙非俗辈,苍苍爽爽近仙乡。”
拂云叟笑道:
“岁寒虚度有千秋,老景潇然清更幽。
不杂嚣尘终冷淡,饱经霜雪自风流。
七贤作侣同谈道,六逸为朋共唱酬。
戛玉敲金非琐琐,天然情性与仙游。”
劲节十八公笑道:
“我亦千年约有余,苍然贞秀自如如。
堪怜雨露生成力,借得乾坤造化机。
万壑风烟惟我盛,四时洒落让吾疏。
盖张翠影留仙客,博弈调琴讲道书。”
三藏称谢道:“四位仙翁,俱享高寿,但劲节翁又千岁余矣。高年得道,丰采清奇,
得非汉时之‘四皓’乎?”四老道:“承过奖,承过奖!吾等非四皓,乃深山之‘四
操’也。敢问圣僧,妙龄几何?”三藏合掌躬身答曰:
“四十年前出母胎,未产之时命已灾。
逃生落水随波滚,幸遇金山脱本骸。
养性看经无懈怠,诚心拜佛敢俄捱?
今蒙皇上差西去,路遇仙翁下爱来。”
四老俱称道:“圣僧自出娘胎,即从佛教,果然是从小修行,真中正有道之上僧也。
我等幸接台颜,敢求大教。望以禅法指教一二,足慰生平。”长老闻言,慨然不惧,
即对众言曰:

“禅者,静也;法者,度也。静中之度,非悟不成。悟者,洗心涤虑,脱俗离
尘是也。夫人身难得,中土难生,正法难遇:全此三者,幸莫大焉。至德妙道,渺
漠希夷,六根六识,遂可扫除。菩提者,不死不生,无余无欠,空色包罗,圣凡俱
遣。访真了元始钳锤,悟实了牟尼手段。发挥象罔,踏碎涅。必须觉中觉了悟中
悟,一点灵光全保护。放开烈焰照婆娑,法界纵横独显露。至幽微,更守固,玄关
口说谁人度?我本元修大觉禅,有缘有志方记悟。”

四老侧耳受了,无边喜悦。一个个稽首皈依,躬身拜谢道:“圣僧乃禅机之悟
本也!”

拂云叟道:“禅虽静,法虽度,须要性定心诚。纵为大觉真仙,终坐无生之道。
我等之玄,又大不同也。”三藏云:“道乃非常,体用合一,如何不同?”拂云叟笑
云:

“我等生来坚实,体用比尔不同。感天地以生身,蒙雨露而滋色。笑傲风霜,
消磨日月。一叶不雕,千枝节操。似这话不叩冲虚。你执持梵语。道也者,本安中
国,反来求证西方。空费了草鞋,不知寻个甚么?石狮子剜了心肝,野狐涎灌彻骨
髓。忘本参禅,妄求佛果,都似我荆棘岭葛藤谜语,萝浑言。此般君子,怎生接
引?这等规模,如何印授?必须要检点见前面目,静中自有生涯。没底竹篮汲水,无
根铁树生花。灵宝峰头牢着脚,归来雅会上龙华。”
三藏闻言,叩头拜谢。十八公用手搀扶。孤直公将身扯起。凌空子打个哈哈道:“拂
云之言,分明漏泄。圣僧请起,不可尽信。我等趁此月明,原不为讲论修持,且自
吟哦逍遥,放荡襟怀也。”拂云叟笑指石屋道:“若要吟哦,且入小庵一茶,何如?”

长老真个欠身,向石屋前观看。门上有三个大字,乃“木仙庵”。遂此同入,
又叙了坐次。忽见那赤身鬼使,捧一盘茯苓膏,将五盏香汤奉上。四老请唐僧先吃,
三藏惊疑,不敢便吃。那四老一齐享用,三藏却才吃了两块。各饮香汤收去。三藏
留心偷看,只见那里玲珑光彩,如月下一般:
水自石边流出,香从花里飘来。
满座清虚雅致,全无半点尘埃。
那长老见此仙境,以为得意,情乐怀开,十分欢喜。忍不住念了一句道:
“禅心似月迥无尘。”
劲节老笑而即联道:
“诗兴如天青更新。”
孤直公道:
“好句漫裁抟锦绣。”
凌空子道:
“佳文不点唾奇珍。”
拂云叟道:
“六朝一洗繁华尽,四始重删雅颂分。”
三藏道:“弟子一时失口,胡谈几字,诚所谓‘班门弄斧’。适闻列仙之言,清新飘
逸,真诗翁也。”劲节老道:“圣僧不必闲叙。出家人全始全终。既有起句,何无结
句?望卒成之。”三藏道:“弟子不能,烦十八公结而成篇为妙。”劲节道:“你好心
肠!你起的句,如何不肯结果?悭吝珠玑,非道理也。”三藏只得续后二句云:
“半枕松风茶未熟,吟怀潇洒满腔春。”

十八公道:“好个‘吟怀潇洒满腔春’!”孤直公道:“劲节,你深知诗味,所以
只管咀嚼。何不再起一篇?”十八公亦慨然不辞道:“我却是顶针字起:
春不荣华冬不枯,云来雾往只如无。”
凌空子道:“我亦体前顶针二句:
无风摇拽婆娑影,有客欣怜福寿图。”
拂云叟亦顶针道:
“图似西山坚节老,清如南国没心夫。”
孤直公亦顶针道:
“夫因侧叶称梁栋,台为横柯作宪乌。”

长老听了,赞叹不已道:“真是《阳春》《白雪》,浩气冲霄!弟子不才,敢再起
两句。”孤直公道:“圣僧乃有道之士,大养之人也。不必再相联句,请赐教全篇,
庶我等亦好勉强而和。”三藏无已,只得笑吟一律曰:
“杖锡西来拜法王,愿求妙典远传扬。
金芝三秀诗坛瑞,宝树千花莲蕊香。
百尺竿头须进步,十方世界立行藏。
修成玉象庄严体,极乐门前是道场。”
四老听毕,俱极赞扬。十八公道:“老拙无能,大胆搀越,也勉和一首。”云:
“劲节孤高笑木王,灵椿不似我名扬。
山空百丈龙蛇影,泉泌千年琥珀香。
解与乾坤生气概,喜因风雨化行藏。
衰残自愧无仙骨,惟有苓膏结寿场。”
孤直公道:“此诗起句豪雄,联句有力,但结句自谦太过矣。堪羡,堪羡!老拙也和
一首。”云:
“霜姿常喜宿禽王,四绝堂前大器扬。
露重珠缨蒙翠盖,风轻石齿碎寒香。
长廊夜静吟声细,古殿秋阴淡影藏。
元日迎春曾献寿,老来寄傲在山场。”
凌空子笑而言曰:“好诗!好诗!真个是月胁天心,老拙何能为和?但不可空过,也须
扯淡几句。”曰:
“梁栋之材近帝王,太清宫外有声扬。
晴轩恍若来青气,暗壁寻常度翠香。
壮节凛然千古秀,深根结矣九泉藏。
凌云势盖婆娑影,不在群芳艳丽场。”
拂云叟道:“三公之诗,高雅清淡,正是放开锦绣之囊也。我身无力,我腹无才,
得三公之教,茅塞顿开。无已,也打油几句,幸勿哂焉。”诗曰:
“淇澳园中乐圣王,渭川千亩任分扬。
翠筠不染湘娥泪,班箨堪传汉史香。
霜叶自来颜不改,烟梢从此色何藏?
子猷去世知音少,亘古留名翰墨场。”

三藏道:“众仙老之诗,真个是吐凤喷珠,游夏莫赞。厚爱高情,感之极矣。
但夜已深沉,三个小徒,不知在何处等我。意者弟子不能久留,敢此告回寻访,尤
无穷之至爱也。望老仙指示归路。”四老笑道:“圣僧勿虑。我等也是千载奇逢。况
天光晴爽,虽夜深却月明如昼,再宽坐坐,待天晓自当远送过岭,高徒一定可相会
也。”

正话间,只见石屋之外,有两个青衣女童,挑一对绛纱灯笼,后引着一个仙女。
那仙女拈着一枝杏花,笑吟吟进门相见。那仙女怎生模样?他生得:

青姿妆翡翠,丹脸赛胭脂。星眼光还彩,蛾眉秀又齐。下衬一条五色梅浅红裙
子,上穿一件烟里火比甲轻衣。弓鞋弯凤嘴,绫袜锦拖泥。妖娆娇似天台女,不亚
当年俏妲姬。
四老欠身问道:“杏仙何来?”那女子对众道了万福,道:“知有佳客在此赓酬,特
来相访。敢求一见。”十八公指着唐僧道:“佳客在此,何劳求见!”三藏躬身,不
敢言语。那女子叫:“快献茶来。”又有两个黄衣女童,捧一个红漆丹盘,盘内有六
个细磁茶盂,盂内设几品异果,横担着匙儿,提一把白铁嵌黄铜的茶壶,壶内香茶
喷鼻。斟了茶,那女子微露春葱,捧磁盂先奉三藏,次奉四老,然后一盏,自取而
陪。

凌空子道:“杏仙为何不坐?”那女子方才去坐。茶毕,欠身问道:“仙翁今宵
盛乐,佳句请教一二如何?”拂云叟道:“我等皆鄙俚之言,惟圣僧真盛唐之作,
甚可嘉羡。”那女子道:“如不吝教,乞赐一观。”四老即以长老前诗后诗并禅法论,
宣了一遍。那女子满面春风,对众道:“妾身不才,不当献丑。但聆此佳句,似不
可虚也,勉强将后诗奉和一律如何?”遂朗吟道:
“上盖留名汉武王,周时孔子立坛场。
董仙爱我成林积,孙楚曾怜寒食香。
雨润红姿娇且嫩,烟蒸翠色显还藏。
自知过熟微酸意,落处年年伴麦场。”
四老闻诗,人人称贺。都道:“清雅脱尘,句内包含春意。好个‘雨润红姿娇且嫩’!
‘雨润红姿娇且嫩’!”那女子笑而悄答道:“惶恐,惶恐!适闻圣僧之章,诚然锦心
绣口。如不吝珠玉,赐教一阕如何?”唐僧不敢答应。那女子渐有见爱之情,挨挨
轧轧,渐近坐边,低声悄语,呼道:“佳客莫者,趁此良宵,不耍子待要怎的?人生
光景,能有几何?”十八公道:“杏仙尽有仰高之情,圣僧岂可无俯就之意?如不见
怜,是不知趣了也。”孤直公道:“圣僧乃有道有名之士,决不苟且行事。如此样举
措,是我等取罪过了。污人名,坏人德,非远达也。果是杏仙有意,可教拂云叟与
十八公做媒,我与凌空子保亲,成此姻眷,何不美哉!”

三藏听言,遂变了颜色,跳起来高叫道:“汝等皆是一类邪物,这般诱我!当时
只以砥砺之言,谈玄谈道可也;如今怎么以美人局来骗害贫僧,是何道理!”四老
见三藏发怒,一个个咬指担惊,再不复言。那赤身鬼使,暴躁如雷道:“这和尚好
不识抬举!我这姐姐,那些儿不好?他人材俊雅,玉质娇姿,不必说那女工针指,只
这一段诗才,也配得过你。你怎么这等推辞!休错过了!孤直公之言甚当。如果不可
苟合,待我再与你主婚。”三藏大惊失色。凭他们怎么胡谈乱讲,只是不从。鬼使
又道:“你这和尚,我们好言好语,你不听从,若是我们发起村野之性,还把你摄
了去,教你和尚不得做,老婆不得娶,却不枉为人一世也?”那长老心如金石,坚
执不从。暗想道:“我徒弟们不知在那里寻我哩!”说一声,止不住眼中堕泪。那女
子陪着笑,挨至身边,翠袖中取出一个蜜合绫汗巾儿,与他揩泪,道:“佳客勿得
烦恼。我与你倚玉偎香,耍子去来。”长老“咄”的一声吆喝,跳起身来就走;被
那些人扯扯拽拽,嚷到天明。

忽听得那里叫声:“师父!师父!你在那方言语也?”原来那孙大圣与八戒、沙
僧,牵着马,挑着担,一夜不曾住脚,穿荆度棘,东寻西找;却好半云半雾的,过
了八百里荆棘岭西下,听得唐僧吆喝,却就喊了一声。那长老挣出门来,叫声:“悟
空,我在这里哩。快来救我!快来救我!”那四老与鬼使,那女子与女童,幌一幌,
都不见了。

须臾间,八戒、沙僧俱到边前道:“师父,你怎么得到此也?”三藏扯住行者
道:“徒弟啊,多累了你们了!昨日晚间见的那个老者,言说土地送斋一事,是你喝
声要打,他就把我抬到此方。他与我携手相搀,走入门,又见三个老者,来此会我,
俱道我做‘圣僧’。一个个言谈清雅,极善吟诗。我与他赓和相攀,觉有夜半时候,
又见一个美貌女子,执灯火,也来这里会我,吟了一首诗,称我做‘佳客’。因见
我相貌,欲求配偶,我方省悟。正不从时,又被他做媒的做媒,保亲的保亲,主婚
的主婚,我立誓不肯。正欲挣着要走,与他嚷闹,不期你们到了。一则天明,二来
还是怕你,只才还扯扯拽拽,忽然就不见了。”行者道:“你既与他叙话谈诗,就不
曾问他个名字?”三藏道:“我曾问他之号。那老者唤做十八公,号劲节;第二个
号孤直公;第三个号凌空子;第四个号拂云叟;那女子,人称他做杏仙。”八戒道:
“此物在于何处?才往那方去了?”三藏道:“去向之方,不知何所;但只谈诗之处,
去此不远。”

他三人同师父看处,只见一座石崖,崖上有“木仙庵”三字。三藏道:“此间
正是。”行者仔细观之,却原来是一株大桧树,一株老柏,一株老松,一株老竹。
竹后有一株丹枫。再看崖那边,还有一株老杏,二株腊梅,二株丹桂。行者笑道:
“你可曾看见妖怪?”八戒道:“不曾。”行者道:“你不知。就是这几株树木在此
成精也。”八戒道:“哥哥怎得知成精者是树?”行者道:“十八公乃松树,孤直公
乃柏树,凌空子乃桧树,拂云叟乃竹竿,赤身鬼乃枫树,杏仙即杏树,女童即丹桂、
腊梅也。”八戒闻言,不论好歹,一顿钉钯,三五长嘴,连拱带筑,把两颗腊梅、
丹桂、老杏、枫杨俱挥倒在地,果然那根下俱鲜血淋漓。三藏近前扯住道:“悟能,
不可伤了他!他虽成了气候,却不曾伤我。我等找路去罢。”行者道:“师父不可惜
他。恐日后成了大怪,害人不浅也。”那呆子索性一顿钯,将松、柏、桧、竹一齐
皆筑倒,却才请师父上马,顺大路一齐西行。

毕竟不知前去如何,且听下回分解。