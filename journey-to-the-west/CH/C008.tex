\chapter{我佛造经传极乐~观音奉旨上长安}

试问禅关,参求无数,往往到头虚老。磨砖作镜,积雪为粮,迷了几多年少?
毛吞大海,芥纳须弥,金色头陀微笑。悟时超十地三乘,凝滞了四生六道。

谁
听得、绝想崖前,无阴树下,杜宇一声春晓?曹溪路险,鹫岭云深,此处故人音杳。
千丈冰崖,五叶莲开,古殿帘垂香袅。那时节,识破源流,便见龙王三宝。

这一篇词,名《苏武慢》。话表我佛如来,辞别了玉帝,回至雷音宝刹,但见
那三千诸佛、五百阿罗、八大金刚、无边菩萨,一个个都执着幢幡宝盖,异宝仙花,
摆列在灵山仙境,娑罗双林之下接迎。如来驾住祥云,对众道:“我以甚深般若,
遍观三界。根本性原,毕竟寂灭,同虚空相,一无所有。殄伏乖猴,是事莫识,名
生死始,法相如是。”说罢,放舍利之光,满空有白虹四十二道,南北通连。大众
见了,皈身礼拜。少顷间,聚庆云彩雾,登上品莲台,端然坐下。那三千诸佛、五
百罗汉、八金刚、四菩萨,合掌近前礼毕,问曰“闹天宫搅乱蟠桃者,何也?”如
来道:“那厮乃花果山产的一妖猴,罪恶滔天,不可名状;概天神将,俱莫能降伏;
虽二郎捉获,老君用火煅炼,亦莫能伤损。我去时,正在雷将中间,扬威耀武,卖
弄精神;被我止住兵戈,问他来历,他言有神通,会变化,又驾筋斗云,一去十万
八千里。我与他打了个赌赛,他出不得我手,却将他一把抓住,指化五行山,封压
他在那里。玉帝大开金阙瑶宫,请我坐了首席,立安天大会谢我,却方辞驾而回。”
大众听言喜悦,极口称扬。谢罢,各分班而退,各执乃事,共乐天真。果然是:

瑞霭漫天竺,虹光拥世尊。西方称第一,无相法王门。常见玄猿献果,麋鹿衔
花;青鸾舞,彩凤鸣;灵龟捧寿,仙鹤噙芝。安享净土祇园,受用龙宫法界。日日
花开,时时果熟。习静归真,参禅果正。不灭不生,不增不减。烟霞缥缈随来往,
寒暑无侵不记年。

诗曰:
去来自在任优游,也无恐怖也无愁。
极乐场中俱坦荡,大千之处没春秋。

佛祖居于灵山大雷音宝刹之间,一日,唤聚诸佛、阿罗、揭谛、菩萨、金刚、
比丘僧、尼等众曰:“自伏乖猿安天之后,我处不知年月,料凡间有半千年矣。今
值孟秋望日,我有一宝盆,盆中具设百样奇花,千般异果等物,与汝等享此‘盂兰
盆会’,如何?”概众一个个合掌,礼佛三匝领会。如来却将宝盆中花果品物,着
阿傩捧定,着迦叶布散。大众感激,各献诗伸谢。

福诗曰:
福星光耀世尊前,福纳弥深远更绵。
福德无疆同地久,福缘有庆与天连。
福田广种年年盛,福海洪深岁岁坚。
福满乾坤多福荫,福增无量永周全。

禄诗曰:
禄重如山彩凤鸣,禄随时泰祝长庚。
禄添万斛身康健,禄享千钟世太平。
禄俸齐天还永固,禄名似海更澄清。
禄恩远继多瞻仰,禄爵无边万国荣。

寿诗曰:
寿星献彩对如来,寿域光华自此开。
寿果满盘生瑞霭,寿花新采插莲台。
寿诗清雅多奇妙,寿曲调音按美才。
寿命延长同日月,寿如山海更悠哉。

众菩萨献毕。因请如来明示根本,指解源流。那如来微开善口,敷演大法,宣
扬正果,讲的是三乘妙典,五蕴《楞严》。但见那天龙围绕,花雨缤纷。正是:
禅心朗照千江月,真性清涵万里天。

如来讲罢,对众言曰:“我观四大部洲,众生善恶,各方不一:东胜神洲者,
敬天礼地,心爽气平;北俱芦洲者,虽好杀生,只因糊口,性拙情疏,无多作践;
我西牛贺洲者,不贪不杀,养气潜灵,虽无上真,人人固寿;但那南赡部洲者,贪
淫乐祸,多杀多争,正所谓口舌凶场,是非恶海。我今有三藏真经,可以劝人为善。”
诸菩萨闻言,合掌皈依。向佛前问曰:“如来有那三藏真经?”如来曰:“我有法一
藏,谈天;论一藏,说地;经一藏,度鬼。三藏共计三十五部,该一万五千一百四
十四卷,乃是修真之经,正善之门。我待要送上东土,叵耐那方众生愚蠢,毁谤真
言,不识我法门之旨要,怠慢了瑜迦之正宗。怎么得一个有法力的,去东土寻一个
善信,教他苦历千山,询经万水,到我处求取真经,永传东土,劝化众生,却乃是
个山大的福缘,海深的善庆。谁肯去走一遭来?”当有观音菩萨,行近莲台,礼佛
三匝道:“弟子不才,愿上东土寻一个取经人来也。”诸众抬头观看,那菩萨:

理圆四德,智满金身。缨络垂珠翠,香环结宝明。乌云巧叠盘龙髻,绣带轻飘
彩凤翎。碧玉纽,素罗袍,祥光笼罩;锦绒裙,金落索,瑞气遮迎。眉如小月,眼
似双星。玉面天生喜,朱唇一点红。净瓶甘露年年盛,斜插垂杨岁岁青。解八难,
度群生,大慈悯:故镇太山,居南海,救苦寻声,万称万应,千圣千灵。兰心欣紫
竹,蕙性爱香藤。他是落伽山上慈悲主,潮音洞里活观音。

如来见了,心中大喜道:“别个是也去不得,须是观音尊者,神通广大,方可
去得。”菩萨道:“弟子此去东土,有甚言语吩咐?”如来道:“这一去,要踏看路
道,不许在霄汉中行,须是要半云半雾;目过山水,谨记程途远近之数,叮咛那取
经人。但恐善信难行,我与你五件宝贝。”即命阿傩、迦叶,取出“锦袈裟”一
领,“九环锡杖”一根,对菩萨言曰:“这袈裟、锡杖,可与那取经人亲用。若肯坚
心来此,穿我的袈裟,免堕轮回;持我的锡杖,不遭毒害。”这菩萨皈依拜领。如
来又取出三个箍儿,递与菩萨道:“此宝唤做‘紧箍儿’,虽是一样三个,但只是用
各不同。我有‘金、紧、禁’的咒语三篇。假若路上撞见神通广大的妖魔,你须是
劝他学好,跟那取经人做个徒弟。他若不伏使唤,可将此箍儿与他戴在头上,自然
见肉生根。各依所用的咒语念一念,眼胀头痛,脑门皆裂,管教他入我门来。”

那菩萨闻言,踊跃作礼而退。即唤惠岸行者随行。那惠岸使一条浑铁棍,重有
千斤,只在菩萨左右,作一个降魔的大力士。菩萨遂将锦袈裟,作一个包裹,令
他背了。菩萨将金箍藏了,执了锡杖,径下灵山。这一去,有分教:
佛子还来归本愿,金蝉长老裹檀。

那菩萨到山脚下,有玉真观金顶大仙在观门首接住,请菩萨献茶。菩萨不敢久
停,曰:“今领如来法旨,上东土寻取经人去。”大仙道:“取经人几时方到?”菩
萨道:“未定,约摸二三年间,或可至此。”遂辞了大仙,半云半雾,约记程途。有
诗为证。诗曰:
万里相寻自不言,却云谁得意难全?
求人忽若浑如此,是我平生岂偶然?
传道有方成妄语,说明无信也虚传。
愿倾肝胆寻相识,料想前头必有缘。
师徒二人正走间,忽然见弱水三千,乃是流沙河界。菩萨道:“徒弟呀,此处却是
难行。取经人浊骨凡胎,如何得渡?”惠岸道:“师父,你看河有多远?”那菩萨
停立云步看时,只见:

东连沙碛,西抵诸番;南达乌戈,北通鞑靼。径过有八百里遥,上下有千万里
远。水流一似地翻身,浪滚却如山耸背,洋洋浩浩,漠漠茫茫,十里遥闻万丈洪。
仙槎难到此,莲叶莫能浮。衰草斜阳流曲浦,黄云影日暗长堤。那里得客商来往?
何曾有渔叟依栖?平沙无雁落,远岸有猿啼。只是红蓼花蘩知景色,白香细任依
依。

菩萨正然点看,只见那河中,泼剌一声响,水波里跳出一个妖魔来,十分丑
恶。他生得:

青不青,黑不黑,晦气色脸;长不长,短不短,赤脚筋躯。眼光闪烁,好似灶
底双灯;口角丫叉,就如屠家火钵。獠牙撑剑刃,红发乱蓬松。一声叱咤如雷吼,
两脚奔波似滚风。
那怪物手执一根宝杖,走上岸就捉菩萨,却被惠岸掣浑铁棒挡住,喝声“休走!”
那怪物就持宝杖来迎。两个在流沙河边,这一场恶杀,真个惊人:

木叉浑铁棒,护法显神通;怪物降妖杖,努力逞英雄。双条银蟒河边舞,一对
神僧岸上冲。那一个威镇流沙施本事,这一个力保观音建大功。那一个翻波跃浪,
这一个吐雾喷风。翻波跃浪乾坤暗,吐雾喷风日月昏。那个降妖杖,好便似出山的
白虎;这个浑铁棒,却就如卧道的黄龙。那个使将来,寻蛇拨草;这个丢开去,扑
鹞分松。只杀得昏漠漠,星辰灿烂;雾腾腾,天地朦胧。那个久住弱水惟他狠,这
个初出灵山第一功。
他两个来来往往,战上数十合,不分胜负。那怪物架住了铁棒道:“你是那里和尚,
敢来与我抵敌?”木叉道:“我是托塔天王二太子木叉惠岸行者。今保我师父往东
土寻取经人去。你是何怪,敢大胆阻路?”那怪方才醒悟道:“我记得你跟南海观
音在紫竹林中修行,你为何来此?”木叉道:“那岸上不是我师父?”

怪物闻言,连声喏喏;收了宝杖,让木叉揪了去,见观音纳头下拜。告道:“菩
萨,恕我之罪,待我诉告。我不是妖邪,我是灵霄殿下侍銮舆的卷帘大将。只因在
蟠桃会上,失手打碎了玻璃盏,玉帝把我打了八百,贬下界来,变得这般模样。又
教七日一次,将飞剑来穿我胸胁百余下方回,故此这般苦恼。没奈何,饥寒难忍,
三二日间,出波涛寻一个行人食用;不期今日无知,冲撞了大慈菩萨。”菩萨道:“你
在天有罪,既贬下来,今又这等伤生,正所谓罪上加罪。我今领了佛旨,上东土寻
取经人。你何不入我门来,皈依善果,跟那取经人做个徒弟,上西天拜佛求经?我
教飞剑不来穿你。那时节功成免罪,复你本职,心下如何?”那怪道:“我愿皈正
果。”又向前道:“菩萨,我在此间吃人无数,向来有几次取经人来,都被我吃了。
凡吃的人头,抛落流沙,竟沉水底。这个水,鹅毛也不能浮。惟有九个取经人的骷
髅,浮在水面,再不能沉。我以为异物,将索儿穿在一处,闲时拿来顽耍。这去,
但恐取经人不得到此,却不是反误了我的前程也?”菩萨曰:“岂有不到之理?你可
将骷髅儿挂在头项下,等候取经人,自有用处。”怪物道:“既然如此,愿领教诲。”
菩萨方与他摩顶受戒,指沙为姓,就姓了沙;起个法名,叫做个沙悟净。当时入了
沙门,送菩萨过了河,他洗心涤虑,再不伤生,专等取经人。

菩萨与他别了,同木叉径奔东土。行了多时,又见一座高山,山上有恶气遮漫,
不能步上。正欲驾云过山,不觉狂风起处,又闪上一个妖魔。他生得又甚凶险。但
见他:

卷脏莲蓬吊搭嘴,耳如蒲扇显金睛。獠牙锋利如钢锉,长嘴张开似火盆。金盔
紧系腮边带,勒甲丝绦蟒退鳞。手执钉钯龙探爪,腰挎弯弓月半轮。纠纠威风欺太
岁,昂昂志气压天神。
他撞上来,不分好歹,望菩萨举钉钯就筑。被木叉行者挡住,大喝一声道:“那泼
怪,休得无礼!看棒!”妖魔道:“这和尚不知死活!看钯!”两个在山底下,一冲一
撞,赌斗输赢。真个好杀:

妖魔凶猛,惠岸威能。铁棒分心捣,钉钯劈面迎。播土扬尘天地暗,飞砂走石
鬼神惊。九齿钯,光耀耀,双环响;一条棒,黑悠悠,两手飞腾。这个是天王太
子,那个是元帅精灵;一个在普陀为护法,一个在山洞作妖精。这场相遇争高下,
不知那个亏输那个赢。

他两个正杀到好处,观世音在半空中,抛下莲花,隔开钯杖。怪物见了心惊,
便问:“你是那里和尚,敢弄甚么眼前花儿哄我?”木叉道:“我把你个肉眼凡胎的
泼物!我是南海菩萨的徒弟。这是我师父抛来的莲花,你也不认得哩!”那怪道:“南
海菩萨,可是扫三灾救八难的观世音么?”木叉道:“不是他是谁?”怪物撇了钉
钯,纳头下礼道:“老兄,菩萨在那里?累烦你引见一引见。”木叉仰面指道:“那不
是?”怪物朝上磕头,厉声高叫道:“菩萨,恕罪,恕罪!”

观音按下云头,前来问道:“你是那里成精的野豕,何方作怪的老彘,敢在此
间挡我?”那怪道:“我不是野豕,亦不是老彘,我本是天河里天蓬元帅。只因带
酒戏弄嫦娥,玉帝把我打了二千锤,贬下尘凡。一灵真性,竟来夺舍投胎,不期错
了道路,投在个母猪胎里,变得这般模样。是我咬杀母猪,可死群彘,在此处占了
山场,吃人度日。不期撞着菩萨,万望拔救,拔救。”菩萨道:“此山叫做甚么山?”
怪物道:“叫做福陵山。山中有一洞,叫做云栈洞。洞里原有个卵二姐。他见我有
些武艺,招我做了家长,又唤做‘倒门’。不上一年,他死了,将一洞的家当,
尽归我受用。在此日久年深,没有个赡身的勾当,只是依本等吃人度日。万望菩萨
恕罪。”菩萨道:“古人云:‘若要有前程,莫做没前程。’你既上界违法,今又不改
凶心,伤生造孽,却不是二罪俱罚?”那怪道:“前程,前程,若依你,教我嗑风!
常言道:‘依着官法打杀,依着佛法饿杀。’去也,去也,还不如捉个行人,肥腻腻
的吃他家娘!管甚么二罪,三罪,千罪,万罪!”菩萨道:“‘人有善愿,天必从之。’
汝若肯归依正果,自有养身之处。世有五谷,尽能济饥,为何吃人度日?”

怪物闻言,似梦方觉。向菩萨施礼道:“我欲从正,奈何‘获罪于天,无所祷
也’!”菩萨道:“我领了佛旨,上东土寻取经人。你可跟他做个徒弟,往西天走一
遭来,将功折罪,管教你脱离灾瘴。”那怪满口道:“愿随!愿随!”菩萨才与他摩顶
受戒,指身为姓,就姓了猪;替他起个法名,就叫做猪悟能。遂此领命归真,持斋
把素,断绝了五荤三厌,专候那取经人。

菩萨却与木叉,辞了悟能,半兴云雾前来。正走处,只见空中有一条玉龙叫唤。
菩萨近前问曰:“你是何龙,在此受罪?”那龙道:“我是西海龙王敖闰之子。因纵
火烧了殿上明珠,我父王表奏天庭,告了忤逆。玉帝把我吊在空中,打了三百,不
日遭诛。望菩萨搭救搭救。”观音闻言,即与木叉撞上南天门里。早有邱、张二天
师接着,问道:“何往?”菩萨道:“贫僧要见玉帝一面。”二天师即忙上奏。玉帝
遂下殿迎接。菩萨上前礼毕道:“贫僧领佛旨上东土寻取经人,路遇孽龙悬吊,特
来启奏,饶他性命,赐与贫僧,教他与取经人做个脚力。”玉帝闻言,即传旨赦宥,
差天将解放,送与菩萨。菩萨谢恩而出。这小龙叩头谢活命之恩,听从菩萨使唤。
菩萨把他送在深涧之中,只等取经人来,变做白马,上西方立功。小龙领命潜身不
题。

菩萨带引木叉行者过了此山,又奔东土。行不多时,忽见金光万道,瑞气千条。
木叉道:“师父,那放光之处,乃是五行山了,见有如来的‘压帖’在那里。”菩萨
道:“此却是那搅乱蟠桃会大闹天宫的齐天大圣,今乃压在此也。”木叉道:“正是,
正是。”师徒俱上山来,观看帖子,乃是“嘛呢叭”六字真言。菩萨看罢,
叹惜不已,作诗一首,诗曰:
堪叹妖猴不奉公,当年狂妄逞英雄。
欺心搅乱蟠桃会,大胆私行兜率宫。
十万军中无敌手,九重天上有威风。
自遭我佛如来困,何日舒伸再显功!

师徒们正说话处,早惊动了那大圣。大圣在山根下,高叫道:“是那个在山上
吟诗,揭我的短哩?”菩萨闻言,径下山来寻看。只见那石崖之下,有土地、山神、
监押大圣的天将,都来拜接了菩萨,引至那大圣面前。看时,他原来压于石匣之中,
口能言,身不能动。菩萨道:“姓孙的,你认得我么?”大圣睁开火眼金睛,点着
头儿高叫道:“我怎么不认得你。你好的是那南海普陀落伽山救苦救难大慈大悲南
无观世音菩萨。承看顾!承看顾!我在此度日如年,更无一个相知的来看我一看。你
从那里来也?”菩萨道:“我奉佛旨,上东土寻取经人去,从此经过,特留残步看
你。”大圣道:“如来哄了我,把我压在此山,五百余年了,不能展挣。万望菩萨方
便一二,救我老孙一救!”菩萨道:“你这厮罪业弥深,救你出来,恐你又生祸害,
反为不美。”大圣道:“我已知悔了。但愿大慈悲指条门路,情愿修行。”这才是:
人心生一念,天地尽皆知。
善恶若无报,乾坤必有私。

那菩萨闻得此言,满心欢喜。对大圣道:“圣经云:‘出其言善,则千里之外应
之;出其言不善,则千里之外违之。’你既有此心,待我到了东土大唐国寻一个取
经的人来,教他救你。你可跟他做个徒弟,秉教伽持,入我佛门,再修正果,如何?”
大圣声声道:“愿去!愿去!”菩萨道:“既有善果,我与你起个法名。”大圣道:“我
已有名了,叫做孙悟空。”菩萨又喜道:“我前面也有二人归降,正是‘悟’字排行。
你今也是‘悟’字,却与他相合,甚好,甚好。这等也不消叮嘱,我去也。”那大
圣见性明心归佛教,这菩萨留情在意访神僧。

他与木叉离了此处,一直东来,不一日就到了长安大唐国。敛雾收云,师徒们
变作两个疥癞游僧,入长安城里,早不觉天晚。行至大市街旁,见一座土地神祠,
二人径入,唬得那土地心慌,鬼兵胆战。知是菩萨,叩头接入。那土地又急跑报与
城隍、社令及满长安各庙神祇,都知是菩萨,参见告道:“菩萨,恕众神接迟之罪。”
菩萨道:“汝等切不可走漏一毫消息。我奉佛旨,特来此处寻访取经人。借你庙宇,
权住几日,待访着真僧即回。”众神各归本处,把个土地赶在城隍庙里暂住,他师
徒们隐遁真形。

毕竟不知寻出那个取经人来,且听下回分解。