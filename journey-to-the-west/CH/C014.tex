\chapter{心猿归正~六贼无踪}

诗曰:

佛即心兮心即佛,心佛从来皆要物。若知无物又无心,便是真如法身佛。法身
佛,没模样,一颗圆光涵万象。无体之体即真体,无相之相即实相。非色非空非不
空,不来不向不回向。无异无同无有无,难舍难取难听望。内外灵光到处同,一佛
国在一沙中。一粒沙含大千界,一个身心万法同。知之须会无心诀,不染不滞为净
业。善恶千端无所为,便是南无释迦叶。

却说那刘伯钦与唐三藏惊惊慌慌,又闻得叫声“师父来也”。众家僮道:“这叫
的必是那山脚下石匣中老猿。”太保道:“是他!是他!”三藏问:“是甚么老猿?”
太保道:“这山旧名五行山;因我大唐王征西定国,改名两界山。先年间曾闻得老
人家说:‘王莽篡汉之时,天降此山,下压着一个神猴,不怕寒暑,不吃饮食,自
有土神监押,教他饥餐铁丸,渴饮铜汁;自昔到今,冻饿不死。’这叫必定是他。
长老莫怕。我们下山去看来。”三藏只得依从,牵马下山。行不数里,只见那石匣
之间,果有一猴,露着头,伸着手,乱招手道:“师父,你怎么此时才来?来得好,
来得好!救我出来,我保你上西天去也!”这长老近前细看,你道他是怎生模样:

尖嘴缩腮,金睛火眼。头上堆苔藓,耳中生薜萝。鬓边少
发多青草,颔下无须有绿莎。眉间土,鼻凹泥,十分狼狈;指头粗,手掌厚,尘垢
余多。还喜得眼睛转动,喉舌声和。语言虽利便,身体莫能那。正是五百年前孙大
圣,今朝难满脱天罗。

刘太保诚然胆大,走上前来,与他拔去了鬓边草,颔下莎,问道:“你有甚么
说话?”那猴道:“我没话说,教那个师父上来,我问他一问。”三藏道:“你问我
甚么?”那猴道:“你可是东土大王差往西天取经去的么?”三藏道:“我正是,你
问怎么?”那猴道:“我是五百年前大闹天宫的齐天大圣;只因犯了诳上之罪,被
佛祖压于此处。前者有个观音菩萨,领佛旨意,上东土寻取经人。我教他救我一救,
他劝我再莫行凶,归依佛法,尽殷勤保护取经人,往西方拜佛,功成后自有好处。
故此昼夜提心,晨昏吊胆,只等师父来救我脱身。我愿保你取经,与你做个徒弟。”

三藏闻言,满心欢喜道:“你虽有此善心,又蒙菩萨教诲,愿入沙门,只是我
又没斧凿,如何救得你出?”那猴道:“不用斧凿,你但肯救我,我自出来也。”三
藏道:“我自救你,你怎得出来?”那猴道:“这山顶上有我佛如来的金字压帖。你
只上山去将帖儿揭起,我就出来了。”三藏依言,回头央浼刘伯钦道:“太保啊,我
与你上山走一遭。”伯钦道:“不知真假何如!”那猴高叫道:“是真!决不敢虚谬!”

伯钦只得呼唤家僮,牵了马匹。他却扶着三藏,复上高山。攀藤附葛,只行到
那极巅之处,果然见金光万道,瑞气千条,有块四方大石,石上贴着一封皮,却是
“嘛呢叭”六个金字。三藏近前跪下,朝石头,看着金字,拜了几拜,望西
祷祝道:“弟子陈玄奘,特奉旨意求经,果有徒弟之分,揭得金字,救出神猴,同
证灵山;若无徒弟之分,此辈是个凶顽怪物,哄赚弟子,不成吉庆,便揭不得起。”
祝罢又拜。拜毕,上前将六个金字,轻轻揭下。只闻得一阵香风,劈手把“压帖儿”
刮在空中,叫道:“吾乃监押大圣者。今日他的难满,吾等回见如来,缴此封皮去
也。”吓得个三藏与伯钦一行人,望空礼拜。径下高山,又至石匣边,对那猴道:“揭
了压帖矣,你出来么。”那猴欢喜,叫道:“师父,你请走开些,我好出来。莫惊了
你。”

伯钦听说,领着三藏,一行人回东即走。走了五七里远近,又听得那猴高叫道:
“再走!再走!”三藏又行了许远,下了山,只闻得一声响亮,真个是地裂山崩。众
人尽皆悚惧。只见那猴早到了三藏的马前,赤淋淋跪下,道声“师父,我出来也!”
对三藏拜了四拜,急起身,与伯钦唱个大喏道:“有劳大哥送我师父,又承大哥替
我脸上薅草。”谢毕,就去收拾行李,扣背马匹。那马见了他,腰软蹄矬,战兢兢
的立站不住。盖因那猴原是弼马温,在天上看养龙马的,有些法则,故此凡马见他
害怕。

三藏见他意思,实有好心,真个像沙门中的人物,便叫:“徒弟啊,你姓甚么?”
猴王道:“我姓孙。”三藏道:“我与你起个法名,却好呼唤。”猴王道:“不劳师父
盛意,我原有个法名,叫做孙悟空。”三藏欢喜道:“也正合我们的宗派。你这个模
样,就像那小头陀一般,我再与你起个混名,称为行者,好么?”悟空道:“好,
好,好!”自此时又称为孙行者。

那伯钦见孙行者一心收拾要行,却转身对三藏唱个喏道:“长老,你幸此间收
得个好徒,甚喜,甚喜。此人果然去得。我却告回。”三藏躬身作礼相谢道:“多有
拖步,感激不胜。回府多多致意令堂老夫人,令荆夫人,贫僧在府多扰,容回时踵
谢。”伯钦回礼,遂此两下分别。

却说那孙行者请三藏上马,他在前边,背着行李,赤条条,拐步而行。不多时,
过了两界山,忽然见一只猛虎,咆哮剪尾而来。三藏在马上惊心。行者在路旁欢喜
道:“师父莫怕他。他是送衣服与我的。”放下行李,耳朵里拔出一个针儿,迎着风,
幌一幌,原来是个碗来粗细一条铁棒。他拿在手中,笑道:“这宝贝,五百余年不
曾用着他,今日拿出来挣件衣服儿穿穿。”你看他拽开步,迎着猛虎,道声“业畜!
那里去!”那只虎蹲着身,伏在尘埃,动也不敢动动。却被他照头一棒,就打的脑
浆迸万点桃红,牙齿喷几珠玉块,唬得那陈玄奘滚鞍落马,咬指道声“天那!天那!
刘太保前日打的斑斓虎,还与他斗了半日;今日孙悟空不用争持,把这虎一棒打得
稀烂,正是‘强中更有强中手’!”

行者拖将虎来道:“师父略坐一坐,等我脱下他的衣服来,穿了走路。”三藏道:
“他那里有甚衣服?”行者道:“师父莫管我,我自有处置。”好猴王,把毫毛拔下
一根,吹口仙气,叫“变!”变作一把牛耳尖刀,从那虎腹上挑开皮,往下一剥,
剥下个囫囵皮来;剁去了爪甲,割下头来,割个四四方方一块虎皮,提起来,量了
一量道:“阔了些儿。一幅可作两幅。”拿过刀来,又裁为两幅。收起一幅,把一幅
围在腰间,路旁揪了一条葛藤,紧紧束定,遮了下体道:“师父,且去,且去!到了
人家,借些针线,再缝不迟。”他把条铁棒,捻一捻,依旧像个针儿,收在耳里,
背着行李,请师父上马。

两个前进,长老在马上问道:“悟空,你才打虎的铁棒,如何不见?”行者笑
道:“师父,你不晓得。我这棍,本是东洋大海龙宫里得来的,唤做‘天河镇底神
珍铁’,又唤做‘如意金箍棒’。当年大反天宫,甚是亏他。随身变化,要大就大,
要小就小。刚才变做一个绣花针儿模样,收在耳内矣。但用时,方可取出。”三藏
闻言暗喜。又问道:“方才那只虎见了你,怎么就不动动?让自在打他,何说?”悟
空道:“不瞒师父说:莫道是只虎,就是一条龙,见了我也不敢无礼。我老孙颇有
降龙伏虎的手段,翻江搅海的神通;见貌辨色,聆音察理;大之则量于宇宙,小之
则摄于毫毛;变化无端,隐显莫测。剥这个虎皮,何为稀罕?见到那疑难处,看展
本事么!”三藏闻得此言,愈加放怀无虑,策马前行。师徒两个走着路,说着话,
不觉得太阳星坠。但见:

焰焰斜辉返照,天涯海角归云。千山鸟雀噪声频,觅宿投
林成阵。野兽双双对对,回窝族族群群。一钩新月破黄昏,万点明星光晕。
行者道:“师父走动些,天色晚了。那壁厢树木森森,想必是人家庄院,我们赶早
投宿去来。”三藏果策马而行,径奔人家。

到了庄院前下马,行者撇了行李,走上前,叫声“开门!开门!”那里面有一老
者,扶筇而出;唿喇的开了门,看见行者这般恶相,腰系着一块虎皮,好似个雷公
模样,唬得脚软身麻,口出谵语道:“鬼来了!鬼来了!”三藏近前搀住,叫道:“老
施主,休怕。他是我贫僧的徒弟,不是鬼怪。”老者抬头,见了三藏的面貌清奇,
方然立定。问道:“你是那寺里来的和尚,带这恶人上我门来?”三藏道:“我贫僧
是唐朝来的,往西天拜佛求经。适路过此间,天晚,特造檀府借宿一宵,明早不犯
天光就行。万望方便一二。”老者道:“你虽是个唐人,那个恶的,却非唐人。”悟
空厉声高呼道:“你这个老儿全没眼色!唐人是我师父,我是他徒弟!我也不是甚‘糖
人,蜜人’,我是齐天大圣。你们这里人家,也有认得我的。我也曾见你来。”那老
者道:“你在那里见我?”悟空道:“你小时不曾在我面前扒柴?不曾在我脸上挑
菜?”老者道:“这厮胡说!你在那里住?我在那里住?我来你面前扒柴、挑菜!”悟
空道:“我儿子便胡说!你是认不得我了,我本是这两界山石匣中的大圣。你再认认
看。”老者方才省悟道:“你倒有些像他;但你是怎么得出来的?”悟空将菩萨劝善,
令我等待唐僧揭帖脱身之事,对那老者细说了一遍。老者却才下拜,将唐僧请到里
面,即唤老妻与儿女都来相见,具言前事,个个欣喜。又命看茶。茶罢,问悟空道:
“大圣啊,你也有年纪了?”悟空道:“你今年几岁了?”老者道:“我痴长一百三
十岁了。”行者道:“还是我重子重孙哩!我那生身的年纪,我不记得是几时;但只
在这山脚下,已五百余年了。”老者道:“是有,是有。我曾记得祖公公说,此山乃
从天降下,就压了一个神猴。只到如今,你才脱体。我那小时见你,是你头上有草,
脸上有泥,还不怕你;如今脸上无了泥,头上无了草,却像瘦了些,腰间又苫了一
块大虎皮,与鬼怪能差多少?”

一家儿听得这般话说,都呵呵大笑。这老儿颇贤,即令安排斋饭。饭后,悟空
道:“你家姓甚?”老者道:“舍下姓陈。”三藏闻言,即下来起手道:“老施主,与
贫僧是华宗。”行者道:“师父,你是唐姓,怎的和他是华宗?”三藏道:“我俗家
也姓陈,乃是唐朝海州弘农郡聚贤庄人氏。我的法名叫做陈玄奘。只因我大唐太宗
皇帝赐我做御弟三藏,指唐为姓,故名唐僧也。”那老者见说同姓,又十分欢喜。
行者道:“老陈,左右打搅你家。我有五百多年不洗澡了,你可去烧些汤来,与我
师徒们洗浴洗浴,一发临行谢你。”那老儿即令烧汤拿盆,掌上灯火。师徒浴罢,
坐在灯前。行者道:“老陈,还有一事累你,有针线借我用用。”那老儿道:“有,
有,有。”即教妈妈取针线来,递与行者。行者又有眼色:见师父洗浴,脱下一件
白布短小直裰未穿,他即扯过来披在身上,却将那虎皮脱下,联接一处,打一个马
面样的折子,围在腰间,勒了藤条,走到师父面前道:“老孙今日这等打扮,比昨
日如何?”三藏道:“好,好,好!这等样,才像个行者。”三藏道:“徒弟,你不嫌
残旧,那件直裰儿,你就穿了罢。”悟空唱个喏道:“承赐,承赐!”他又去寻些草
料喂了马。此时各各事毕,师徒与那老儿,亦各归寝。

次早,悟空起来,请师父走路。三藏着衣,教行者收拾铺盖行李。正欲告辞,
只见那老儿,早具脸汤,又具斋饭。斋罢,方才起身。三藏上马,行者引路。不觉
饥餐渴饮,夜宿晓行,又值初冬时候。但见那:

霜雕红叶千林瘦,岭上几株松柏秀。未开梅蕊散香幽,暖短昼,小春候,菊残
荷尽山茶茂。寒桥古树争枝斗,曲涧涓涓泉水溜。淡云欲雪满天浮,朔风骤,牵衣
袖,向晚寒威人怎受?
师徒们正走多时,忽见路旁唿哨一声,闯出六个人来,各执长枪短剑,利刃强弓,
大咤一声道:“那和尚!那里走!赶早留下马匹,放下行李,饶你性命过去!”唬得那
三藏魂飞魄散,跌下马来,不能言语。行者用手扶起道:“师父放心,没些儿事。
这都是送衣服送盘缠与我们的。”三藏道:“悟空,你想有些耳闭?他说教我们留马
匹、行李,你倒问他要甚么衣服盘缠?”行者道:“你管守着衣服行李马匹,待老
孙与他争持一场,看是何如。”三藏道:“好手不敌双拳,双拳不如四手。他那里六
条大汉,你这般小小的一个人儿,怎么敢与他争持?”

行者的胆量原大,那容分说,走上前来,叉手当胸,对那六个人施礼道:“列
位有甚么缘故,阻我贫僧的去路?”那人道:“我等是剪径的大王,行好心的山主。
大名久播,你量不知。早早的留下东西,放你过去;若道半个‘不’字,教你碎尸
粉骨!”行者道:“我也是祖传的大王,积年的山主,却不曾闻得列位有甚大名。”
那人道:“你是不知,我说与你听:一个唤做眼看喜,一个唤做耳听怒,一个唤做
鼻嗅爱,一个唤作舌尝思,一个唤作意见欲,一个唤作身本忧。”悟空笑道:“原来
是六个毛贼!你却不认得我这出家人是你的主人公,你倒来挡路。把那打劫的珍宝
拿出来,我与你作七分儿均分,饶了你罢!”那贼闻言,喜的喜,怒的怒,爱的爱,
思的思,欲的欲,忧的忧。一齐上前乱嚷道:“这和尚无礼!你的东西全然没有,转
来和我等要分东西!”他轮枪舞剑,一拥前来,照行者劈头乱砍,乒乒乓乓,砍有
七八十下。悟空停立中间,只当不知。那贼道:“好和尚!真个的头硬!”行者笑道:
“将就看得过罢了,你们也打得手困了,却该老孙取出个针儿来耍耍。”那贼道:“这
和尚是一个行针灸的郎中变的。我们又无病症,说甚么动针的话!”

行者伸手去耳朵里拔出一根绣花针儿,迎风一幌,却是一条铁棒,足有碗来粗
细。拿在手中道:“不要走!也让老孙打一棍儿试试手!”唬得这六个贼四散逃走,
被他拽开步,团团赶上,一个个尽皆打死。剥了他的衣服,夺了他的盘缠,笑吟吟
走将来道:“师父请行,那贼已被老孙剿了。”三藏道:“你十分撞祸!他虽是剪径的
强徒,就是拿到官司,也不该死罪;你纵有手段,只可退他去便了,怎么就都打死?
这却是无故伤人的性命,如何做得和尚?出家人‘扫地恐伤蝼蚁命,爱惜飞蛾纱罩
灯。’你怎么不分皂白,一顿打死?全无一点慈悲好善之心!早还是山野中无人查考;
若到城市,倘有人一时冲撞了你,你也行凶,执着棍子,乱打伤人,我可做得白客,
怎能脱身?”悟空道:“师父,我若不打死他,他却要打死你哩。”三藏道:“我这
出家人,宁死决不敢行凶。我就死,也只是一身,你却杀了他六人,如何理说?此
事若告到官,就是你老子做官,也说不过去。”行者道:“不瞒师父说,我老孙五百
年前,据花果山称王为怪的时节,也不知打死多少人;假似你说这般到官,倒也得
些状告是。”三藏道:“只因你没收没管,暴横人间,欺天诳上,才受这五百年前之
难。今既入了沙门,若是还像当时行凶,一味伤生,去不得西天,做不得和尚!忒
恶!忒恶!”

原来这猴子一生受不得人气。他见三藏只管绪绪叨叨,按不住心头火发道:“你
既是这等,说我做不得和尚,上不得西天,不必恁般绪恶我,我回去便了!”那
三藏却不曾答应,他就使一个性子,将身一纵,说一声“老孙去也”三藏急抬头,
早已不见。只闻得呼的一声,回东而去。撇得那长老孤孤零零,点头自叹,悲怨不
已,道:“这厮这等不受教诲!我但说他几句,他怎么就无形无影的,径回去了?罢,
罢,罢!也是我命里不该招徒弟,进人口!如今欲寻他无处寻,欲叫他叫不应,去来!
去来!”正是舍身拚命归西去,莫倚旁人自主张。

那长老只得收拾行李,捎在马上,也不骑马,一只手拄着锡杖,一只手揪着缰
绳,凄凄凉凉,往西前进。行不多时,只见山路前面,有一个年高的老母,捧一件
绵衣,绵衣上有一顶花帽。三藏见他来得至近,慌忙牵马,立于右侧让行。那老母
问道:“你是那里来的长老,孤孤凄凄独行于此?”三藏道:“弟子乃东土大唐奉圣
旨往西天拜活佛求真经者。”老母道:“西方佛乃大雷音寺天竺国界,此去有十万八
千里路。你这等单人独马,又无个伴侣,又无个徒弟,你如何去得!”三藏道:“弟
子日前,收得一个徒弟,他性泼凶顽,是我说了他几句,他不受教,遂渺然而去也。”
老母道:“我有这一领绵布直裰,一顶嵌金花帽。原是我儿子用的。他只做了三日
和尚,不幸命短身亡。我才去他寺里,哭了一场,辞了他师父,将这两件衣帽拿来,
做个忆念。长老啊,你既有徒弟,我把这衣帽送了你罢。”三藏道:“承老母盛赐;
但只是我徒弟已走了,不敢领受。”老母道:“他那厢去了?”三藏道:“我听得呼
的一声,他回东去了。”老母道:“东边不远,就是我家,想必往我家去了。我那里
还有一篇咒儿,唤做‘定心真言’;又名做‘紧箍儿咒’。你可暗暗的念熟,牢记心
头,再莫泄漏一人知道。我去赶上他,叫他还来跟你,你却将此衣帽与他穿戴。他
若不服你使唤,你就默念此咒,他再不敢行凶,也再不敢去了。”

三藏闻言,低头拜谢。那老母化一道金光,回东而去。三藏情知是观音菩萨授
此真言,急忙撮土焚香,望东恳恳礼拜。拜罢,收了衣帽,藏在包袱中间。却坐于
路旁,诵习那定心真言。来回念了几遍,念得烂熟,牢记心胸不题。

却说那悟空别了师父,一筋斗云,径转东洋大海。按住云头,分开水道,径至
水晶宫前。早惊动龙王出来迎接。接至宫里坐下,礼毕。龙王道:“近闻得大圣难
满,失贺!想必是重整仙山,复归古洞矣。”悟空道:“我也有此心性;只是又做了
和尚了。”龙王道:“做甚和尚?”行者道:“我亏了南海菩萨劝善,教我正果,随
东土唐僧,上西方拜佛,皈依沙门,又唤为行者了。”龙王道:“这等真是可贺!可
贺!这才叫做改邪归正,惩创善心。既如此,怎么下西去,复东回何也?”行者笑
道:“那是唐僧不识人性。有几个毛贼剪径,是我将他打死,唐僧就绪绪叨叨,说
了我若干的不是。你想,老孙可是受得闷气的?是我撇了他,欲回本山,故此先来
望你一望,求钟茶吃。”龙王道:“承降,承降!”当时龙子、龙孙即捧香茶来献。

茶毕,行者回头一看,见后壁上挂著一幅《圯桥进履》的画儿。行者道:“这
是甚么景致?”龙王道:“大圣在先,此事在后,故你不认得。这叫做‘圯桥三进
履’。”行者道:“怎的是‘三进履’?”龙王道:“此仙乃是黄石公。此子乃是汉世
张良。石公坐在圯桥上,忽然失履于桥下,遂唤张良取来。此子即忙取来,跪献于
前。如此三度,张良略无一毫倨傲怠慢之心,石公遂爱他勤谨,夜授天书,着他扶
汉。后果然运筹帷幄之中,决胜千里之外。太平后,弃职归山,从赤松子游,悟成
仙道。大圣,你若不保唐僧,不尽勤劳,不受教诲,到底是个妖仙,休想得成正果。”
悟空闻言,沉吟半晌不语。龙王道:“大圣自当裁处,不可图自在,误了前程。”悟
空道:“莫多话,老孙还去保他便了。龙王欣喜道:“既如此,不敢久留,请大圣早
发慈悲,莫要疏久了你师父。”行者见他催促请行,急耸身,出离海藏,驾着云,
别了龙王。

正走,却遇着南海菩萨。菩萨道:“孙悟空,你怎么不受教诲,不保唐僧,来
此处何干?”慌得个行者在云端里施礼道:“向蒙菩萨善言,果有唐朝僧到,揭了
压帖,救了我命,跟他做了徒弟。他却怪我凶顽,我才子闪了他一闪,如今就去保
他也。”菩萨道:“赶早去,莫错过了念头。”言毕,各回。

这行者,须臾间看见唐僧在路旁闷坐。他上前道:“师父!怎么不走路?还在此
做甚?”三藏抬头道:“你往那里去来?教我行又不敢行,动又不敢动,只管在此等
你。”行者道:“我往东洋大海老龙王家讨茶吃吃。”三藏道:“徒弟啊,出家人不要
说谎。你离了我,没多一个时辰,就说到龙王家吃茶?”行者笑道:“不瞒师父说:
我会驾筋斗云,一个筋斗有十万八千里路,故此得即去即来。”三藏道:“我略略的
言语重了些儿,你就怪我,使个性子丢了我去。像你这有本事的,讨得茶吃;像我
这去不得的,只管在此忍饿。你也过意不去呀!”行者道:“师父,你若饿了,我便
去与你化些斋吃。”三藏道:“不用化斋。我那包袱里,还有些干粮,是刘太保母亲
送的,你去拿钵盂寻些水来,等我吃些儿走路罢。”

行者去解开包袱,在那包裹中间见有几个粗面烧饼,拿出来递与师父。又见那
光艳艳的一领绵布直裰,一顶嵌金花帽,行者道:“这衣帽是东土带来的?”三藏
就顺口儿答应道:“是我小时穿戴的。这帽子若戴了,不用教经,就会念经;这衣
服若穿了,不用演礼,就会行礼。”行者道:“好师父,把与我穿戴了罢。”三藏道:
“只怕长短不一,你若穿得,就穿了罢。”行者遂脱下旧白布直裰,将绵布直裰穿
上——也就是比量着身体裁的一般——把帽儿戴上。

三藏见他戴上帽子,就不吃干粮,却默默的念那紧箍咒一遍。行者叫道:“头
痛!头痛!”那师父不住的又念了几遍,把个行者痛得打滚,抓破了嵌金的花帽。三
藏又恐怕扯断金箍,住了口不念。不念时,他就不痛了。伸手去头上摸摸,似一条
金线儿模样,紧紧的勒在上面,取不下,揪不断,已此生了根了。他就耳里取出针
儿来,插入箍里,往外乱捎。三藏又恐怕他捎断了,口中又念起来,他依旧生痛,
痛得竖蜻蜓,翻筋斗,耳红面赤,眼胀身麻。那师父见他这等,又不忍不舍,复住
了口,他的头又不痛了。行者道:“我这头,原来是师父咒我的。”三藏道:“我念
的是紧箍经,何曾咒你?”行者道:“你再念念看。”三藏真个又念。行者真个又痛,
只教:“莫念,莫念!念动我就痛了!这是怎么说?”三藏道:“你今番可听我教诲
了?”行者道:“听教了!”“你再可无礼了?”行者道:“不敢了!”

他口里虽然答应,心上还怀不善,把那针儿幌一幌,碗来粗细,望唐僧就欲下
手,慌得长老口中又念了两三遍,这猴子跌倒在地,丢了铁棒,不能举手,只教:
“师父,我晓得了!再莫念,再莫念!”三藏道:“你怎么欺心,就敢打我?”行者
道:“我不曾敢打,我问师父,你这法儿是谁教你的?”三藏道:“是适间一个老母
传授我的。”行者大怒道:“不消讲了!这个老母,坐定是那个观世音!他怎么那等害
我,等我上南海打他去!”三藏道:“此法既是他授与我,他必然先晓得了。你若寻
他,他念起来,你却不是死了?”行者见说得有理,真个不敢动身,只得回心,跪
下哀告道:“师父,这是他奈何我的法儿,教我随你西去。我也不去惹他,你也莫
当常言,只管念诵。我愿保你,再无退悔之意了。”三藏道:“既如此,伏侍我上马
去也。”那行者才死心塌地,抖擞精神,束一束绵布直裰,扣背马匹,收拾行李,
奔西而进。

毕竟这一去,后面又有甚话说,且听下回分解。