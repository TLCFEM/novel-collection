\chapter{唐三藏路阻火焰山~孙行者一调芭蕉扇}

若干种性本来同,海纳无穷。千思万虑终成妄,般般色色和融。有日功完行满,
圆明法性高隆。休教差别走西东,紧锁牢笼。收来安放丹炉内,炼得金乌一样红。
朗朗辉辉娇艳,任教出入乘龙。

话表三藏遵菩萨教旨,收了行者,与八戒、沙僧剪断二心,锁笼猿马,同心戮
力,赶奔西天。说不尽光阴似箭,日月如梭。历过了夏月炎天,却又值三秋霜景。
但见那:

薄云断绝西风紧,鹤鸣远岫霜林锦。光景正苍凉,山长水更长。征鸿来北塞,
玄鸟归南陌。客路怯孤单,衲衣容易寒。
师徒四众,进前行处,渐觉热气蒸人。三藏勒马道:“如今正是秋天,却怎返有热
气?”八戒道:“原来不知。西方路上有个斯哈哩国,乃日落之处,俗呼为‘天尽
头’。若到申酉时,国王差人上城,擂鼓吹角,混杂海沸之声。日乃太阳真火,落
于西海之间,如火淬水,接声滚佛;若无鼓角之声混耳,即振杀城中小儿。此地热
气蒸人,想必到日落之处也。”大圣听说,忍不住笑道:“呆子莫乱谈!若论斯哈哩
国,正好早哩。似师父朝三暮二的,这等担阁,就从小至老,老了又小,老小三生,
也还不到。”八戒道:“哥啊,据你说,不是日落之处,为何这等酷热?”沙僧道:
“想是天时不正,秋行夏令故也。”

他三个正都争讲,只见那路旁有座庄院,乃是红瓦盖的房舍,红砖砌的垣墙,
红油门扇,红漆板榻,一片都是红的。三藏下马道:“悟空,你去那人家问个消息,
看那炎热之故何也。”

大圣收了金箍棒,整肃衣裳,扭捏作个斯文气象,绰下大路,径至门前观看。
那门里忽然走出一个老者,但见他:

穿一领黄不黄、红不红的葛布深衣;戴一顶青不青、皂不皂的篾丝凉帽。手中
拄一根弯不弯、直不直、暴节竹杖;足下踏一双新不新、旧不旧、鞋。面似
红铜,须如白练。两道寿眉遮碧眼,一张口露金牙。
那老者猛抬头,看见行者,吃了一惊,拄着竹杖,喝道:“你是那里来的怪人?在我
这门首何干?”行者答礼道:“老施主,休怕我。我不是甚么怪人。贫僧是东土大
唐钦差上西方求经者。师徒四人,适至宝方,见天气蒸热,一则不解其故,二来不
知地名,特拜问指教一二。”那老者却才放心,笑云:“长老勿罪。我老汉一时眼花,
不识尊颜。”行者道:“不敢。”老者又问:“令师在那条路上?”行者道:“那南首
大路上立的不是!”老者教:“请来,请来。”行者欢喜,把手一招,三藏即同八戒、
沙僧、牵白马,挑行李近前,都对老者作礼。

老者见三藏丰姿标致,八戒、沙僧相貌奇稀,又惊又喜;只得请入里坐,教小
的们看茶,一壁厢办饭。三藏闻言,起身称谢道:“敢问公公:贵处遇秋,何返炎
热?”老者道:“敝地唤做火焰山。无春无秋,四季皆热。”三藏道:“火焰山却在
那边?可阻西去之路?”老者道:“西方却去不得。那山离此有六十里远,正是西方
必由之路,却有八百里火焰,四周围寸草不生。若过得山,就是铜脑盖,铁身躯,
也要化成汁哩。”三藏闻言,大惊失色,不敢再问。

只见门外一个少年男子,推一辆红车儿,住在门旁,叫声“卖糕!”大圣拔根
毫毛,变个铜钱,问那人买糕。那人接了钱,不论好歹,揭开车儿上衣裹,热气腾
腾,拿出一块糕递与行者。行者托在手中,好似火盆里的灼炭,煤炉内的红钉。你
看他左手倒在右手,右手换在左手,只道:“热,热,热!难吃,难吃!”那男子笑
道:“怕热,莫来这里。这里是这等热。”行者道:“你这汉子,好不明理。常言道:
‘不冷不热,五谷不结。’他这等热得很,你这糕粉,自何而来?”那人道:“若知
糕粉米,敬求铁扇仙。”行者道:“铁扇仙怎的?”那人道:“铁扇仙有柄‘芭蕉扇’。
求得来,一扇息火,二扇生风,三扇下雨,我们就布种,及时收割,故得五谷养生;
不然,诚寸草不能生也。”

行者闻言,急抽身走入里面,将糕递与三藏道:“师父放心,且莫隔年焦着,
吃了糕,我与你说。”长老接糕在手,向本宅老者道:“公公请糕。”老者道:“我家
的茶饭未奉,敢吃你糕?”行者笑道:“老人家,茶饭倒不必赐。我问你:铁扇仙
在那里住?”老者道:“你问他怎的?”行者道:“适才那卖糕人说,此仙有柄‘芭
蕉扇’。求将来,一扇息火,二扇生风,三扇下雨,你这方布种收割,才得五谷养
生。我欲寻他讨来扇息火焰山过去,且使这方依时收种,得安生也。”老者道:“固
有此说;你们却无礼物,恐那圣贤不肯来也。”三藏道:“他要甚礼物?”老者道:
“我这里人家,十年拜求一度。四猪四羊,花红表里,异香时果,鸡鹅美酒,沐浴
虔诚,拜到那仙山,请他出洞,至此施为。”行者道:“那山坐落何处?唤甚地名?有
几多里数?等我问他要扇子去。”老者道:“那山在西南方,名唤翠云山。山中有一
仙洞,名唤芭蕉洞。我这里众信人等去拜仙山,往回要走一月,计有一千四百五六
十里。”行者笑道:“不打紧,就去就来。”那老者道:“且住,吃些茶饭,办些干粮,
须得两人做伴。那路上没有人家,又多狼虎,非一日可到。莫当耍子。”行者笑道:
“不用,不用!我去也!”说一声,忽然不见。那老者慌张道:“爷爷呀!原来是腾云
驾雾的神人也!”

且不说这家子供奉唐僧加倍。却说那行者霎时径到翠云山,按住祥光,正自找
寻洞口,忽然闻得丁丁之声,乃是山林内一个樵夫伐木。行者即趋步至前,又闻得
他道:
“云际依依认旧林,断崖荒草路难寻。
西山望见朝来雨,南涧归时渡处深。”
行者近前作礼道:“樵哥,问讯了。”那樵子撇了柯斧,答礼道:“长老何往?”行
者道:“敢问樵哥,这可是翠云山?”樵子道:“正是。”行者道:“有个铁扇仙的芭
蕉洞,在何处?”樵子笑道:“这芭蕉洞虽有,却无个铁扇仙,只有个铁扇公主,
又名罗刹女。”行者道:“人言他有一柄芭蕉扇,能熄得火焰山,敢是他么?”樵子
道:“正是,正是。这圣贤有这件宝贝,善能熄火,保护那方人家,故此称为铁扇
仙。我这里人家用不着他,只知他叫做罗刹女,乃大力牛魔王妻也。”

行者闻言,大惊失色。心中暗想道:“又是冤家了!当年伏了红孩儿,说是这厮
养的。前在那解阳山破儿洞遇他叔子,尚且不肯与水,要作报仇之意;今又遇他父
母,怎生借得这扇子耶?”樵子见行者沉思默虑,嗟叹不已,便笑道:“长老,你
出家人,有何忧疑?这条小路儿向东去,不上五六里,就是芭蕉洞。休得心焦。”行
者道:“不瞒樵哥说。我是东土唐朝差往西天求经的唐僧大徒弟。前年在火云洞,
曾与罗刹之子红孩儿有些言语,但恐罗刹怀仇不与,故生忧疑。”樵子道:“大丈夫
鉴貌辨色,只以求扇为名,莫认往时之溲话,管情借得。”行者闻言,深深唱个大
喏道:“谢樵哥教诲。我去也。”

遂别了樵夫,径至芭蕉洞口。但见那两扇门紧闭牢关,洞外风光秀丽。好去处!
正是那:

山以石为骨。石作土之精。烟霞含宿润,苔藓助新青。嵯峨势耸欺蓬岛,幽静
花香若海瀛。几树乔松栖野鹤,数株衰
柳语山莺。诚然是千年古迹,万载仙踪。碧梧鸣彩凤,活水隐苍龙。曲径荜萝垂挂,
石梯藤葛攀笼。猿啸翠岩忻月上,鸟啼高树喜晴空。两林竹荫凉如雨,一径花浓没
绣绒。时见白云来远岫,略无定体漫随风。
行者上前叫:“牛大哥,开门!开门!”呀的一声,洞门开了,里边走出一个毛儿女,
手中提着花篮,肩上担着锄子,真个是:

一身蓝缕无妆饰,满面精神有道心。
行者上前迎着,合掌道:“女童,累你转报公主一声。我本是取经的和尚,在西方
路上,难过火焰山,特来拜借芭蕉扇一用。”那毛女道:“你是那寺里和尚?叫甚名
字?我好与你通报。”行者道:“我是东土来的,叫做孙悟空和尚。”

那毛女即便回身,转于洞内,对罗刹跪下道:“奶奶,洞门外有个东土来的孙
悟空和尚,要见奶奶,拜求芭蕉扇,过火焰山一用。”那罗刹听见“孙悟空”三字,
便似撮盐入火,火上烧油;骨都都红生脸上,恶狠狠怒发心头。口中骂道:“这泼
猴,今日来了!”叫:“丫鬟!取披挂,拿兵器来!”随即取了披挂,拿两口青锋宝剑,
整束出来。行者在洞外闪过,偷看怎生打扮。只见他:

头裹团花手帕,身穿纳锦云袍。腰间双束虎筋绦,微露绣裙偏绡。凤嘴弓鞋三
寸,龙须膝裤金销。手提宝剑怒声高,凶比月婆容貌。
那罗刹出门,高叫道:“孙悟空何在?”行者上前,躬身施礼道:“嫂嫂,老孙在此
奉揖。”罗刹咄的一声道:“谁是你的嫂嫂!那个要你奉揖!”行者道:“尊府牛魔王,
当初曾与老孙结义,乃七兄弟之亲。今闻公主是牛大哥令正,安得不以嫂嫂称之!”
罗刹道:“你这泼猴!既有兄弟之亲,如何坑陷我子?”行者佯问道:“令郎是谁?”
罗刹道:“我儿是号山枯松涧火云洞圣婴大王红孩儿,被你倾了。我们正没处寻你
报仇,你今上门纳命,我肯绕你!”行者满脸陪笑道:“嫂嫂原来不察理,错怪了老
孙。你令郎因是捉了师父,要蒸要煮,幸亏了观音菩萨收他去,救出我师。他如今
现在菩萨处做善财童子,实受了菩萨正果,不生不灭,不垢不净,与天地同寿,日
月同庚。你倒不谢老孙保命之恩,返怪老孙,是何道理!”罗刹道:“你这个巧嘴的
泼猴!我那儿虽不伤命,再怎生得到我的跟前,几时能见一面?”行者笑道:“嫂嫂
要见令郎,有何难处?你且把扇子借我,扇息了火,送我师父过去,我就到南海菩
萨处请他来见你,就送扇子还你,有何不可!那时节,你看他可曾损伤一毫。如有
些须之伤,你也怪得有理;如比旧时标致,还当谢我。”

罗刹道:“泼猴!少要饶舌,伸过头来,等我砍上几剑!若受得疼痛,就借扇子
与你;若忍耐不得,教你早见阎君!”行者叉手向前,笑道:“嫂嫂切莫多言。老孙
伸着光头,任尊意砍上多少,但没气力便罢。是必借扇子用用。”那罗刹不容分说,
双手轮剑,照行者头上乒乒乓乓,砍有十数下,这行者全不认真。罗刹害怕,回头
要走。行者道:“嫂嫂,那里去?快借我使使!”那罗刹道:“我的宝贝原不轻借。”
行者道:“既不肯借,吃你老叔一棒!”

好猴王,一只手扯住,一只手去耳内掣出棒来,幌一幌,有碗来粗细。那罗刹
挣脱手,举剑来迎。行者随又轮棒便打。两个在翠云山前,不论亲情,却只讲仇隙。
这一场好杀:

裙钗本是修成怪,为子怀仇恨泼猴。行者虽然生狠怒,因师路阻让娥流。先言
拜借芭蕉扇,不展骁雄耐性柔。罗刹无知轮剑砍,猴王有意说亲由。女流怎与男儿
斗,到底男刚压女流。这个金箍铁棒多凶猛,那个霜刃青锋甚紧稠。劈面打,照头
丢,恨苦相持不罢休。左挡右遮施武艺,前迎后架骋奇谋。却才斗到沉酣处,不觉
西方坠日头。罗刹忙将真扇子,一扇挥
动鬼神愁。
那罗刹女与行者相持到晚,见行者棒重,却又解数周密,料斗他不过,即便取出芭
蕉扇,幌一幌,一扇阴风,把行者得无影无形,莫想收留得住。这罗刹得胜回归。

那大圣飘飘荡荡,左沉不能落地,右坠不得存身。就如旋风翻败叶,流水淌残
花。滚了一夜,直至天明,方才落在一座山上,双手抱住一块峰石。定性良久,仔
细观看,却才认得是小须弥山。大圣长叹一声道:“好利害妇人!怎么就把老孙送到
这里来了?我当年曾记得在此处告求灵吉菩萨降黄风怪救我师父。那黄风岭至此直
南上有三千余里,今在西路转来,乃东南方隅,不知有几万里。等我下去问灵吉菩
萨一个消息,好回旧路。”

正踌躇间,又听得钟声响亮,急下山坡,径至禅院。那门前道人认得行者的形
容,即入里面报道:“前年来请菩萨去降黄风怪的那个毛脸大圣又来了。”

菩萨知是悟空,连忙下宝座相迎,入内施礼道:“恭喜!取经来耶?”悟空答道:
“正好未到!早哩,早哩!”灵吉道:“既未曾得到雷音,何以回顾荒山?”行者道:
“自上年蒙盛情降了黄风怪,一路上,不知历过多少苦楚。今到火焰山,不能前进,
询问土人,说有个铁扇仙芭蕉扇,得火灭,老孙特去寻访。原来那仙是牛魔王的
妻,红孩儿的母。他说我把他儿子做了观音菩萨的童子,不得常见,跟我为仇,不
肯借扇,与我争斗。他见我的棒重难撑,遂将扇子把我一,得我悠悠荡荡,直
至于此,方才落住。故此轻造禅院,问个归路。此处到火焰山,不知有多少里数?”
灵吉笑道:“那妇人唤名罗刹女,又叫做铁扇公主。他的那芭蕉扇本是昆仑山后,
自混沌开辟以来,天地产成的一个灵宝,乃太阴之精叶,故能灭火气。假若着人,
要飘八万四千里,方息阴风。我这山到火焰山,只有五万余里。此还是大圣有留云
之能,故止住了。若是凡人,正好不得住也。”行者道:“利害,利害!我师父却怎
生得度那方?”灵吉道:“大圣放心。此一来,也是唐僧的缘法,合教大圣成功。”
行者道:“怎见成功?”灵吉道:“我当年受如来教旨,赐我一粒‘定风丹’,一柄
‘飞龙杖’。飞龙杖已降了风魔。这定风丹尚未曾见用,如今送了大圣,管教那厮
你不动,你却要了扇子,息火,却不就立此功也!”行者低头作礼,感谢不尽。
那菩萨即于衣袖中取出一个锦袋儿,将那一粒定风丹与行者安在衣领里边,将针线
紧紧缝了。送行者出门道:“不及留款。往西北上去,就是罗刹的山场也。”

行者辞了灵吉,驾筋斗云,径返翠云山,顷刻而至。使铁棒打着洞门叫道:“开
门,开门!老孙来借扇子使使哩!”慌得那门里女童即忙来报:“奶奶,借扇子的又
来了!”罗刹闻言,心中悚惧道:“这泼猴真有本事!我的宝贝着人,要去八万四
千里方能停止;他怎么才吹去就回来也?这番等我一连他两三,教他找不着归
路!”

急纵身,结束整齐,双手提剑,走出门来道:“孙行者!你不怕我,又来寻死!”
行者笑道:“嫂嫂勿得悭吝,是必借我使使。保得唐僧过山,就送还你。我是个志
诚有余的君子,不是那借物不还的小人。”罗刹又骂道:“泼猢狲!好没道理,没分
晓!夺子之仇,尚未报得;借扇之意,岂得如心!你不要走,吃我老娘一剑!”大圣
公然不惧,使铁棒劈手相迎。他两个往往来来,战经五七回合,罗刹女手软难轮,
孙行者身强善敌。他见事势不谐,即取扇子,望行者了一,行者巍然不动。行
者收了铁棒,笑吟吟的道:“这番不比那番!任你怎么来,老孙若动一动,就不算
汉子!”那罗刹又两,果然不动。罗刹慌了,急收宝贝,转回走入洞里,将门
紧紧关上。

行者见他闭了门,却就弄个手段,拆开衣领,把定风丹噙在口中,摇身一变,
变作一个虫儿,从他门隙处钻进。只见罗刹叫道:“渴了,渴了!快拿茶来!”
近侍女童,即将香茶一壶,沙沙的满斟一碗,冲起茶沫漕漕。行者见了欢喜,嘤的
一翅,飞在茶沫之下。

那罗刹渴极,接过茶,两三气都喝了。行者已到他肚腹之内,现原身厉声高叫
道:“嫂嫂,借扇子我使使!”罗刹大惊失色,叫:“小的们,关了前门否?”俱说:
“关了。”他又说:“既关了门,孙行者如何在家里叫唤?”女童道:“在你身上叫
哩。”罗刹道:“孙行者,你在那里弄术哩?”行者道:“老孙一生不会弄术,都是
些真手段,实本事,已在尊嫂尊腹之内耍子,已见其肺肝矣。我知你也饥渴了,我
先送你个坐碗儿解渴!”却就把脚往下一登。那罗刹小腹之中,疼痛难禁,坐于地
下叫苦。行者道:“嫂嫂休得推辞,我再送你个点心充饥!”又把头往上一顶。那罗
刹心痛难禁,只在地上打滚,疼得他面黄唇白,只叫:“孙叔叔饶命!”

行者却才收了手脚道:“你才认得叔叔么?我看牛大哥情上,且饶你性命。快将
扇子拿来我使使。”罗刹道:“叔叔,有扇,有扇,你出来拿了去!”行者道:“拿扇
子我看了出来。”罗刹即叫女童拿一柄芭蕉扇,执在旁边。行者探到喉咙之上见了
道:“嫂嫂,我既饶你性命,不在腰肋之下搠个窟窿出来,还自口出。你把口张三
张儿。”那罗刹果张开口。行者还作个虫,先飞出来,丁在芭蕉扇上。那罗刹
不知,连张三次,叫:“叔叔出来罢。”行者化原身,拿了扇子,叫道:“我在此间
不是?谢借了,谢借了!”拽开步,往前便走。小的们连忙开了门,放他出洞。

这大圣拨转云头,径回东路。霎时按落云头,立在红砖壁下。八戒见了欢喜道:
“师父,师兄来了!来了!”三藏即与本庄老者同沙僧出门接着,同至舍内。把芭蕉
扇靠在旁边道:“老官儿,可是这个扇子?”老者道:“正是,正是!”唐僧喜道:“贤
徒有莫大之功。求此宝贝,甚劳苦了。”行者道:“劳苦倒也不说。那铁扇仙,你道
是谁?那厮原来是牛魔王的妻,红孩儿的母,名唤罗刹女,又唤铁扇公主。我寻到
洞外借扇,他就与我讲起仇隙,把我砍了几剑。是我使棒吓他,他就把扇子了我
一下,飘飘荡荡,直刮到小须弥山。幸见灵吉菩萨,送了我一粒定风丹,指与归路,
复至翠云山。又见罗刹女,罗刹女又使扇子,我不动,他就回洞。是老孙变作一
个虫,飞入洞去。那厮正讨茶吃,是我又钻在茶沫之下,到他肚里,做起手脚。
他疼痛难禁,不住口的叫我做叔叔饶命,情愿将扇借与我,我却饶了他,拿将扇来。
待过了火焰山,仍送还他。”三藏闻言,感谢不尽。师徒们俱拜辞老者。

一路西来,约行有四十里远近,渐渐酷热蒸人。沙僧只叫:“脚底烙得慌!”八
戒又道:“爪子烫得痛!”马比寻常又快。只因地热难停,十分难进。行者道:“师
父且请下马。兄弟们莫走。等我息了火,待风雨之后,地土冷些,再过山去。”
行者果举扇,径至火边,尽力一,那山上火光烘烘腾起;再一,更着百倍;又
一,那火足有千丈之高,渐渐烧着身体。行者急回,已将两股毫毛烧净,径跑至
唐僧面前叫:“快回去,快回去!火来了,火来了!”

那师父爬上马,与八戒、沙僧,复东来有二十余里,方才歇下,道:“悟空,
如何了呀!”行者丢下扇子道:“不停当,不停当,被那厮哄了!”三藏听说,愁促
眉尖,闷添心上,止不住两泪交流,只道:“怎生是好!”八戒道:“哥哥,你急急
忙忙叫回去是怎么说?”行者道:“我将扇子了一下,火光烘烘;第二扇,火气
愈盛;第三扇,火头飞有千丈之高。若是跑得不快,把毫毛都烧尽矣!”八戒笑道:
“你常说雷打不伤,火烧不损,如今何又怕火?”行者道:“你这呆子,全不知事!
那时节用心防备,故此不伤;今日只为息火光,不曾捻避火诀,又未使护身法,
所以把两股毫毛烧了。”沙僧道:“似这般火盛,无路通西,怎生是好?”八戒道:
“只拣无火处走便罢。”三藏道:“那方无火?”八戒道:“东方、南方、北方,俱
无火。”又问:“那方有经?”八戒道:“西方有经。”三藏道:“我只欲往有经处去
哩!”沙僧道:“有经处有火,无火处无经,诚是进退两难!”

师徒们正自胡谈乱讲,只听得有人叫道:“大圣不须烦恼,且来吃些斋饭再议。”
四众回看时,见一老人,身披飘风氅,头顶偃月冠,手持龙头杖,足踏铁靴,后
带着一个雕嘴鱼腮鬼,鬼头上顶着一个铜盆,盆内有些蒸饼糕糜,黄粮米饭,在于
西路下躬身道:“我本是火焰山土地。知大圣保护圣僧,不能前进,特献一斋。”行
者道:“吃斋小可,这火光几时灭得,让我师父过去?”土地道:“要灭火光,须求
罗刹女借芭蕉扇。”行者去路旁拾起扇子道:“这不是?那火光越越着,何也?”
土地看了,笑道:“此扇不是真的,被他哄了。”行者道:“如何方得真的?”那土
地又控背躬身,微微笑道:“若还要借芭蕉扇,须是寻求大力王。”

毕竟不知大力王有甚缘故,且听下回分解。