\chapter{尸魔三戏唐三藏~圣僧恨逐美猴王}

却说三藏师徒,次日天明,收拾前进。那镇元子与行者结为兄弟,两人情投意
合,决不肯放;又安排管待,一连住了五六日。那长老自服了草还丹,真似脱胎换
骨,神爽体健。他取经心重,那里肯淹留,无已,遂行。

师徒别了上路,早见一座高山。三藏道:“徒弟,前面有山险峻,恐马不能前,
大家须仔细仔细。”行者道:“师父放心,我等自然理会。”好猴王,他在那马前,
横担着棒,剖开山路,上了高崖,看不尽:

峰岩重叠,涧壑湾环。虎狼成阵走,麂鹿作群行。无数獐钻簇簇,满山狐兔
聚丛丛。千尺大蟒,万丈长蛇;大蟒喷愁雾,长蛇吐怪风。道旁荆棘牵漫,岭山松
楠秀丽。薜萝满目,芳草连天。影落沧溟北,云开斗柄南。万古常含元气老,千峰
巍列日光寒。
那长老马上心惊,孙大圣布施手段,舞着铁棒,哮吼一声,唬得那狼虫颠窜,虎豹
奔逃。

师徒们入此山,正行到嵯峨之处,三藏道:“悟空,我这一日,肚中饥了,你
去那里化些斋吃。”行者陪笑道:“师父好不聪明。这等半山之中,前不巴村,后不
着店,有钱也没买处,教往那里寻斋?”三藏心中不快,口里骂道:“你这猴子,
想你在两界山,被如来压在石匣之内,口能言,足不能行;也亏我救你性命,摩顶
受戒,做了我的徒弟。怎么不肯努力,常怀懒惰之心!”行者道:“弟子亦颇殷勤,
何尝懒惰?”三藏道:“你既殷勤,何不化斋我吃?我肚饥怎行?况此地山岚瘴气,
怎么得上雷音?”行者道:“师父休怪,少要言语。我知你尊性高傲,十分违慢了
你,便要念那话儿咒。你下马稳坐,等我寻那里有人家处化斋去。”

行者将身一纵,跳上云端里,手搭凉篷,睁眼观看。可怜西方路甚是寂寞,更
无庄堡人家;正是多逢树木,少见人烟去处。看多时,只见正南上有一座高山。那
山向阳处,有一片鲜红的点子。行者按下云头道:“师父,有吃的了。”那长老问甚
东西。行者道:“这里没人家化饭,那南山有一片红的,想必是熟透了的山桃,我
去摘几个来你充饥。”三藏喜道:“出家人若有桃子吃,就为上分了。快去!”行者
取了钵盂,纵起祥光,你看他斗幌幌,冷气飕飕,须臾间,奔南山摘桃不题。

却说常言有云:“山高必有怪,岭峻却生精。”果然这山上有一个妖精。孙大圣
去时,惊动那怪。他在云端里,踏着阴风,看见长老坐在地下,就不胜欢喜道:“造
化,造化!几年家人都讲东土的唐和尚取‘大乘’,他本是金蝉子化身,十世修行的
原体。有人吃他一块肉,长寿长生。真个今日到了。”那妖精上前就要拿他,只见
长老左右手下有两员大将护持,不敢拢身。他说两员大将是谁?说是八戒、沙僧。
八戒、沙僧,虽没甚么大本事,然八戒是天蓬元帅,沙僧是卷帘大将。他的威气尚
不曾泄,故不敢拢身。妖精说:“等我且戏他戏,看怎么说。”

好妖精,停下阴风,在那山凹里,摇身一变,变做个月貌花容的女儿,说不尽
那眉清目秀,齿白唇红,左手提着一个青砂儿,右手提着一个绿磁瓶儿,从西向
东,径奔唐僧:
圣僧歇马在山岩,忽见裙钗女近前。
翠袖轻摇笼玉笋,湘裙斜拽显金莲。
汗流粉面花含露,尘拂蛾眉柳带烟。
仔细定睛观看处,看看行至到身边。
三藏见了,叫:“八戒,沙僧,悟空才说这里旷野无人,你看那里不走出一个人来
了?”八戒道:“师父,你与沙僧坐着,等老猪去看看来。”那呆子放下钉钯,整整
直裰,摆摆摇摇,充作个斯文气象,一直的觌面相迎。真个是远看未实,近看分明。
那女子生得:

冰肌藏玉骨,衫领露酥胸。柳眉积翠黛,杏眼闪银星。月样容仪俏,天然性格
清。体似燕藏柳,声如莺啭林。半放海棠笼晓日,才开芍药弄春晴。
那八戒见他生得俊俏,呆子就动了凡心,忍不住胡言乱语。叫道:“女菩萨,往那
里去?手里提着是甚么东西?”——分明是个妖怪,他却不能认得。——那女子连
声答应道:“长老,我这青里是香米饭,绿瓶里是炒面筋。特来此处无他故,
因还誓愿要斋僧。”

八戒闻言,满心欢喜。急抽身,就跑了个猪颠风,报与三藏道:“师父!‘吉人
自有天报!’师父饿了,教师兄去化斋,那猴子不知那里摘桃儿耍子去了。桃子吃
多了,也有些嘈人,又有些下坠。你看那不是个斋僧的来了?”唐僧不信道:“你
这个夯货胡缠!我们走了这向,好人也不曾遇着一个,斋僧的从何而来!”八戒道:
“师父,这不到了?”

三藏一见,连忙跳起身来,合掌当胸道:“女菩萨,你府上在何处住?是甚人家?
有甚愿心,来此斋僧?”分明是个妖精,那长老也不认得。那妖精见唐僧问他来历,
他立地就起个虚情,花言巧语,来赚哄道:“师父,此山叫做蛇回兽怕的白虎岭。
正西下面是我家。我父母在堂,看经好善,广斋方上远近僧人。只因无子,求神作
福;生了奴奴,欲扳门第,配嫁他人,又恐老来无倚,只得将奴招了一个女婿,养
老送终。”三藏闻言道:“女菩萨,你语言差了。圣经云:‘父母在,不远游;游必
有方。’你既有父母在堂,又与你招了女婿,有愿心,教你男子还,便也罢,怎么
自家在山行走?又没个侍儿随从。这个是不遵妇道了。”

那女子笑吟吟,忙陪俏语道:“师父,我丈夫在山北凹里,带几个客子锄田。
这是奴奴煮的午饭,送与那些人吃的。只为五黄六月,无人使唤,父母又年老,所
以亲身来送。忽遇三位远来,却思父母好善,故将此饭斋僧。如不弃嫌,愿表芹献。”
三藏道:“善哉,善哉!我有徒弟摘果子去了,就来,我不敢吃;假如我和尚吃了你
饭,你丈夫晓得骂你,却不罪坐贫僧也?”

那女子见唐僧不肯吃,却又满面春生道:“师父啊,我父母斋僧,还是小可;
我丈夫更是个善人,一生好的是修桥补路,爱老怜贫。但听见说这饭送与师父吃了,
他与我夫妻情上,比寻常更是不同。”三藏也只是不吃。旁边子恼坏了八戒。那呆
子努着嘴,口里埋怨道:“天下和尚也无数,不曾像我这个老和尚罢软!现成的饭,
三分儿,倒不吃,只等那猴子来,做四分才吃!”他不容分说,一嘴把个子拱倒,
就要动口。

只见那行者自南山顶上,摘了几个桃子,托着钵盂,一筋斗,点将回来;睁火
眼金睛观看,认得那女子是个妖精,放下钵盂,掣铁棒,当头就打。唬得个长老用
手扯住道:“悟空!你走将来打谁?”行者道:“师父,你面前这个女子,莫当做个
好人,他是个妖精,要来骗你哩!”三藏道:“你这猴头,当时倒也有些眼力,今日
如何乱道!这女菩萨有此善心,将这饭要斋我等,你怎么说他是个妖精?”行者笑
道:“师父,你那里认得。老孙在水帘洞里做妖魔时,若想人肉吃,便是这等:或
变金银,或变庄台,或变醉人,或变女色。有那等痴心的,爱上我,我就迷他到洞
里,尽意随心,或蒸或煮受用。吃不了,还要晒干了防天阴哩!师父,我若来迟,
你定入他套子,遭他毒手!”那唐僧那里肯信,只说是个好人。行者道:“师父,我
知道你了。你见他那等容貌,必然动了凡心。若果有此意,叫八戒伐几棵树来,沙
僧寻些草来,我做木匠,就在这里搭个窝铺,你与他圆房成事,我们大家散了,却
不是件事业?何必又跋涉,取甚经去!”那长老原是个软善的人,那里吃得他这句言
语,羞得个光头彻耳通红。

三藏正在此羞惭,行者又发起性来,掣铁棒,望妖精劈脸一下。那怪物有些手
段,使个“解尸法”,见行者棍子来时,他却抖擞精神,预先走了,把一个假尸首
打死在地下。唬得个长老战战兢兢,口中作念道:“这猴着然无礼!屡劝不从,无故
伤人性命。”行者道:“师父莫怪,你且来看看这子里是甚东西。”沙僧搀着长老,
近前看时,那里是甚香米饭,却是一子拖尾巴的长蛆;也不是面筋,却是几个青
蛙、癞虾蟆,满地乱跳。长老才有三分儿信了。怎禁猪八戒气不忿,在旁漏八分儿
唆嘴道:“师父,说起这个女子,他是此间农妇,因为送饭下田,路遇我等,却怎
么栽他是个妖怪?哥哥的棍重,走将来试手打他一下,不期就打杀了;怕你念甚么
紧箍儿咒,故意的使个障眼法儿,变做这等样东西,演幌你眼,使不念咒哩。”

三藏自此一言,就是晦气到了:果然信那呆子撺唆,手中捻诀,口里念咒。行
者就叫:“头疼,头疼!莫念,莫念!有话便说。”唐僧道:“有甚话说!出家人时时常
要方便,念念不离善心,扫地恐伤蝼蚁命,爱惜飞蛾纱罩灯。你怎么步步行凶?打
死这个无故平人,取将经来何用?你回去罢!”行者道:“师父,你教我回那里去?”
唐僧道:“我不要你做徒弟。”行者道:“你不要我做徒弟,只怕你西天路去不成。”
唐僧道:“我命在天,该那个妖精蒸了吃,就是煮了也算不过。终不然你救得我的
大限?你快回去!”行者道:“师父,我回去便也罢了,只是不曾报得你的恩哩。”唐
僧道:“我与你有甚恩?”那大圣闻言,连忙跪下叩头道:“老孙因大闹天宫,致下
了伤身之难,被我佛压在两界山;幸观音菩萨与我受了戒行,幸师父救脱吾身;若
不与你同上西天,显得我‘知恩不报非君子,万古千秋作骂名’。”原来这唐僧是个
慈悯的圣僧。他见行者哀告,却也回心转意道:“既如此说,且饶你这一次。再休
无礼。如若仍前作恶,这咒语颠倒就念二十遍!”行者道:“三十遍也由你,只是我
不打人了。”却才伏侍唐僧上马,又将摘来桃子奉上。唐僧在马上也吃了几个,权
且充饥。

却说那妖精,脱命升空。原来行者那一棒不曾打杀妖精,妖精出神去了。他在
那云端里,咬牙切齿,暗恨行者道:“几年只闻得讲他手段,今日果然话不虚传。
那唐僧已此不认得我,将要吃饭。若低头闻一闻儿,我就一把捞住,却不是我的人
了。不期被他走来,弄破我这勾当,又几乎被他打了一棒。若饶了这个和尚,诚然
是劳而无功也。我还下去戏他一戏。”

好妖精,按落阴云,在那前山坡下,摇身一变,变作个老妇人,年满八旬,手
拄着一根弯头竹杖,一步一声的哭着走来。八戒见了,大惊道:“师父!不好了!那
妈妈儿来寻人了!”唐僧道:“寻甚人?”八戒道:“师兄打杀的,定是他女儿。这
个定是他娘寻将来了。”行者道:“兄弟莫要胡说!那女子十八岁,这老妇有八十岁,
怎么六十多岁还生产?断乎是个假的,等老孙去看来。”好行者,拽开步,走近前观
看,那怪物:

假变一婆婆,两鬓如冰雪。走路慢腾腾,行步虚怯怯。弱体瘦伶仃,脸如枯菜
叶。颧骨望上翘,嘴唇往下别。老年不比少年时,满脸都是荷叶折。

行者认得他是妖精,更不理论,举棒照头便打。那怪见棍子起时,依然抖擞,
又出化了元神,脱真儿去了;把个假尸首又打死在山路之下。

唐僧一见,惊下马来,睡在路旁,更无二话,只是把紧箍儿咒颠倒足足念了二
十遍。可怜把个行者头,勒得似个亚腰儿葫芦,十分疼痛难忍,滚将来哀告道:“师
父,莫念了!有甚话说了罢!”唐僧道:“有甚话说,出家人耳听善言,不堕地狱。
我这般劝化你,你怎么只是行凶,把平人打死一个,又打死一个,此是何说?”行
者道:“他是妖精。”唐僧道:“这个猴子胡说!就有这许多妖怪!你是个无心向善之
辈,有意作恶之人,你去罢!”行者道:“师父又教我去?回去便也回去了,只是一
件不相应。”唐僧道:“你有甚么不相应处?”八戒道:“师父,他要和你分行李哩。
跟着你做了这几年和尚,不成空着手回去?你把那包袱里的甚么旧褊衫,破帽子,
分两件与他罢。”

行者闻言,气得暴跳道:“我把你这个尖嘴的夯货!老孙一向秉教沙门,更无一
毫嫉妒之意,贪恋之心,怎么要分甚么行李?”唐僧道:“你既不嫉妒贪恋,如何
不去?”行者道:“实不瞒师父说。老孙五百年前,居花果山水帘洞大展英雄之际,
收降七十二洞邪魔,手下有四万七千群怪,头戴的是紫金冠,身穿的是赭黄袍,腰
系的是蓝田带,足踏的是步云履,手执的是如意金箍棒,着实也曾为人。自从涅
罪度,削发秉正沙门,跟你做了徒弟,把这个‘金箍儿’勒在我头上,若回去,却
也难见故乡人。师父果若不要我,把那个松箍儿咒念一念,退下这个箍子,交付与
你,套在别人头上,我就快活相应了。也是跟你一场。莫不成这些人意儿也没有了?”
唐僧大惊道:“悟空,我当时只是菩萨暗受一卷紧箍儿咒,却没有甚么松箍儿咒。”
行者道:“若无松箍儿咒,你还带我去走走罢。”长老又没奈何道:“你且起来,我
再饶你这一次,却不可再行凶了。”行者道:“再不敢了,再不敢了。”又伏侍师父
上马,剖路前进。

却说那妖精,原来行者第二棍也不曾打杀他。那怪物在半空中,夸奖不尽道:
“好个猴王,着然有眼!我那般变了去,他也还认得我。这些和尚,他去得快,若
过此山,西下四十里,就不伏我所管了。若是被别处妖魔捞了去,好道就笑破他人
口,使碎自家心。我还下去戏他一戏。”好妖怪,按耸阴风,在山坡下摇身一变,
变做一个老公公,真个是:
白发如彭祖,苍髯赛寿星。
耳中鸣玉磬,眼里幌金星。
手拄龙头拐,身穿鹤氅轻。
数珠掐在手,口诵南无经。
唐僧在马上见了,心中欢喜道:“阿弥陀佛,西方真是福地!那公公路也走不上来,
逼法的还念经哩。”八戒道:“师父,你且莫要夸奖。那个是祸的根哩。”唐僧道:“怎
么是祸根?”八戒道:“行者打杀他的女儿,又打杀他的婆子,这个正是他的老儿
寻将来了。我们若撞在他的怀里呵,师父,你便偿命,该个死罪;把老猪为从,问
个充军;沙僧喝令,问个摆站;那行者使个遁法走了,却不苦了我们三个顶缸?”
行者听见道:“这个呆根,这等胡说,可不唬了师父?等老孙再去看看。”

他把棍藏在身边,走上前,迎着怪物,叫声“老官儿,往那里去?怎么又走路
又念经?”那妖精错认了定盘星,把孙大圣也当做个等闲的,遂答道:“长老啊,
我老汉祖居此地,一生好善斋僧,看经念佛。命里无儿,止生得一个小女,招了个
女婿。今早送饭下田,想是遭逢虎口。老妻先来找寻,也不见回去。全然不知下落,
老汉特来寻看。果然是伤残他命,也没奈何,将他骸骨收拾回去,安葬茔中。”行
者笑道:“我是个做虎的祖宗,你怎么袖子里笼了个鬼儿来哄我?你瞒了诸人,瞒
不过我。我认得你是个妖精!”那妖精唬得顿口无言。行者掣出棒来,自忖思道:“若
要不打他,显得他倒弄个风儿;若要打他,又怕师父念那话儿咒语。”又思量道:“不
打杀他,他一时间抄空儿把师父捞了去,却不又费心劳力去救他?还打的是!就一棍
子打杀他,师父念起那咒,常言道:‘虎毒不吃儿。’凭着我巧言花语,嘴伶舌便,
哄他一哄,好道也罢了。”

好大圣,念动咒语,叫当坊土地、本处山神道:“这妖精三番来戏弄我师父,
这一番却要打杀他。你与我在半空中作证,不许走了。”众神听令,谁敢不从,都
在云端里照应。那大圣棍起处,打倒妖魔,才断绝了灵光。

那唐僧在马上,又唬得战战兢兢,口不能言。八戒在旁边又笑道:“好行者!风
发了!只行了半日路,倒打死三个人!”唐僧正要念咒,行者急到马前,叫道:“师
父,莫念,莫念!你且来看看他的模样。”却是一堆粉骷髅在那里。唐僧大惊道:“悟
空,这个人才死了,怎么就化作一堆骷髅?”行者道:“他是个潜灵作怪的僵尸,
在此迷人败本;被我打杀,他就现了本相。他那脊梁上有一行字,叫做‘白骨夫人’。”
唐僧闻说,倒也信了;怎禁那八戒旁边唆嘴道:“师父,他的手重棍凶,把人打死,
只怕你念那话儿,故意变化这个模样,掩你的眼泪哩!”唐僧果然耳软,又信了他,
随复念起。行者禁不得疼痛,跪于路旁,只叫“莫念!莫念!有话快说了罢!”唐僧
道:“猴头,还有甚说话!山家人行善,如春园之草,不见其长,日有所增,行恶之
人,如磨刀之石,不见其损,日有所亏。你在这荒郊野外,一连打死三人,还是无
人检举,没有对头;倘到城市之中,人烟凑集之所,你拿了那哭丧棒,一时不知好
歹,乱打起人来,撞出大祸,教我怎的脱身?你回去罢!”行者道:“师父错怪了我
也。这厮分明是个妖魔,他实有心害你。我倒打死他,替你除了害,你却不认得,
反信了那呆子谗言冷语,屡次逐我。常言道:‘事不过三。’我若不去,真是个下流
无耻之徒。我去,我去,去便去了,只是你手下无人。”唐僧发怒道:“这泼猴越发
无礼!看起来,只你是人,那悟能、悟净,就不是人?”

那大圣一闻得说他两个是人,止不住伤情凄惨,对唐僧道声“苦啊!你那时节,
出了长安,有刘伯钦送你上路;到两界山,救我出来,投拜你为师,我曾穿古洞,
入深林,擒魔捉怪,收八戒,得沙僧,吃尽千辛万苦;今日昧着惺惺使糊涂,只教
我回去:这才是‘鸟尽弓藏,兔死狗烹’!罢,罢,罢!但只是多了那紧箍儿咒”。
唐僧道:“我再不念了。”行者道:“这个难说:若到那毒魔苦难处不得脱身,八戒、
沙僧救不得你,那时节,想起我来,忍不住又念诵起来,就是十万里路,我的头也
是疼的;假如再来见你,不如不作此意。”

唐僧见他言言语语,越添恼怒,滚鞍下马来,叫沙僧包袱内取出纸笔,即于涧
下取水,石上磨墨,写了一纸贬书,递于行者道:“猴头,执此为照!再不要你做徒
弟了!如再与你相见,我就堕了阿鼻地狱!”行者连忙接了贬书道:“师父,不消发
誓,老孙去罢。”他将书折了,留在袖中,却又软款唐僧道:“师父,我也是跟你一
场,又蒙菩萨指教;今日半涂而废,不曾成得功果,你请坐,受我一拜,我也去得
放心。”唐僧转回身不睬,口里唧唧哝哝的道:“我是个好和尚,不受你歹人的礼!”
大圣见他不睬,又使个身外法,把脑后毫毛拔了三根,吹口仙气,叫“变”!即变
了三个行者,连本身四个,四面围住师父下拜。那长老左右躲不脱,好道也受了一
拜。

大圣跳起来,把身一抖,收上毫毛,却又吩咐沙僧道:“贤弟,你是个好人,
却只要留心防着八戒言语,途中更要仔细。倘一时有妖精拿住师父,你就说老
孙是他大徒弟:西方毛怪,闻我的手段,不敢伤我师父。”唐僧道:“我是个好和尚,
不题你这歹人的名字。你回去罢。”那大圣见长老三番两覆,不肯转意回心,没奈
何才去。你看他:
噙泪叩头辞长老,含悲留意嘱沙僧。
一头拭迸坡前草,两脚蹬翻地上藤。
上天下地如轮转,跨海飞山第一能。
顷刻之间不见影,霎时疾返旧途程。
你看他忍气别了师父,纵筋斗云,径回花果山水帘洞去了。独自个凄凄惨惨,忽闻
得水声聒耳。大圣在那半空里看时,原来是东洋大海潮发的声响。一见了,又想起
唐僧,止不住腮边泪坠,停云住步,良久方去。

毕竟不知此去反复如何,且听下回分解。