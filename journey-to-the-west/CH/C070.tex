\chapter{妖魔宝放烟沙火~悟空计盗紫金铃}

却说那孙行者抖擞神威,持着铁棒,踏祥光,起在空中,迎面喝道:“你是那
里来的邪魔,待往何方猖獗!”那怪物厉声高叫道:“吾党不是别人,乃麒麟山獬豸
洞赛太岁大王爷爷部下先锋。今奉大王令,到此取宫女二名,伏侍金圣娘娘。你是
何人,敢来问我!”行者道:“吾乃齐天大圣孙悟空。因保东土唐僧西天拜佛,路过
此国,知你这伙邪魔欺主,特展雄才,治国祛邪。正没处寻你,却来此送命!”那
怪闻言,不知好歹,展长枪就刺行者。行者举铁棒劈面相迎。在半空里这一场好杀:

棍是龙宫镇海珍,枪乃人间转炼铁。凡兵怎敢比仙兵,擦着些儿神气泄。大圣
原来太乙仙,妖精本是邪魔孽。鬼祟焉能近正人,一正之时邪就灭。那个弄风播土
唬皇王,这个踏雾腾云遮日月。丢开架手赌输赢,无能谁敢夸豪杰!还是齐天大圣
能,乒乓一棍枪先折。
那妖精被行者一铁棒把根枪打做两截,慌得顾性命,拨转风头,径往西方败走。

行者且不赶他,按下云头,来至避妖楼地穴之处,叫道:“师父,请同陛下出
来。怪物已赶去矣。”那唐僧才扶着君王,同出穴外。见满天清朗,更无妖邪之气。

那皇帝即至酒席前,自己拿壶把盏,满斟金杯,奉与行者道:“神僧,权谢,
权谢!”这行者接杯在手,还未回言,只听得朝门外有官来报:“西门上火起了!”
行者闻说,将金杯连酒望空一撇,当的一声响,那个金杯落地。君王着了忙,躬
身施礼道:“神僧,恕罪,恕罪!是寡人不是了!礼当请上殿拜谢,只因有这方便酒
在此,故就奉耳。神僧却把杯子撇了,却不是有见怪之意?”行者笑道:“不是这
话,不是这话。”少顷间,又有官来报:“好雨呀!才西门上起火,被一场大雨,把
火灭了。满街上流水,尽都是酒气。”行者又笑道:“陛下,你见我撇杯,疑有见怪
之意,非也。那妖败走西方,我不曾赶他,他就放起火来。这一杯酒,却是我灭了
妖火,救了西城里外人家,岂有他意!”

国王更十分欢喜加敬。即请三藏四众,同上宝殿,就有推位让国之意。行者笑
道:“陛下,才那妖精,他称是赛太岁部下先锋,来此取宫女的。他如今战败而回,
定然报与那厮。那厮定要来与我相争。我恐他一时兴师帅众,未免又惊伤百姓,恐
唬陛下。欲去迎他一迎,就在那半空中擒了他,取回圣后。但不知向那方去,这里
到他那山洞有多少远近?”国王道:“寡人曾回圣后。但不知向那方去,这里到他
那山洞有多少远近?”国王道:“寡人曾差‘夜不收’军马到那里探听声息,往来
要行五十余日。坐落南方,约有三千余里。”行者闻言,叫:“八戒、沙僧,护持在
此,老孙去来。”国王扯住道:“神僧且从容一日,待安排些干粮烘炒,与你些盘缠
银两,选一匹快马,方才可去。”行者笑道:“陛下说得是巴山转岭步行之话。我老
孙不瞒你说,似这三千里路,斟酒在钟不冷,就打个往回。”国王道:“神僧,你不
要怪我说。你这尊貌,却像个猿猴一般,怎生有这等法力会走路也?”行者道:

“我身虽是猿猴数,自幼打开生死路。遍访明师把道传,山前修炼无朝暮。倚
天为顶地为炉,两般药物团乌兔。采取阴阳水火交,时间顿把玄关悟。全仗天罡搬
运功,也凭斗柄迁移步。退炉进火最依时,抽铅添汞相交顾。攒簇五行造化生,
合和四象分时度。二气归于黄道间,三家会在金丹路。悟通法律归四肢,本来筋斗
如神助。一纵纵过太行山,一打打过凌云渡。何愁峻岭几千重,不怕长江百十数。
只因变化没遮拦,一打十万八千路!”
那国王见说,又惊又喜,笑吟吟捧着一杯御酒递与行者道:“神僧远劳,进此一杯
引意。”这大圣一心要去降妖,那里有心吃酒,只叫:“且放下,等我去了回来再饮。”
好行者,说声去,唿哨一声,寂然不见。那一国君臣,皆惊讶不题。

却说行者将身一纵,早见一座高山,阻住雾角。即按云头,立在那巅峰之上。
仔细观看,好山:

冲天占地,碍日生云:冲天处,尖峰矗矗;占地处,远脉迢迢;碍日的,乃岭
头松郁郁;生云的,乃岸下石磷磷。松郁郁,四时八节常青;石磷磷,万载千年不
改。林中每听夜猿啼,涧内常闻妖蟒过。山禽声咽咽,山兽吼呼呼。山獐山鹿,成
双作对纷纷走;山鸦山鹊,打阵攒群密密飞。山草山花看不尽,山桃山果映时新。
虽然倚险不堪行,却是妖仙隐逸处。
这大圣看看不厌,正欲找寻洞口,只见那山凹里烘烘火光飞出,霎时间,扑天红焰,
红焰之中冒出一股恶烟,比火更毒。好烟!但见那:

火光迸万点金灯,火焰飞千条红虹。那烟不是灶筒烟,不是草木烟,烟却有五
色:青红白黑黄。熏着南天门外柱,燎着灵霄殿上梁。烧得那窝中走兽连皮烂,林
内飞禽羽尽光。但看这烟如此恶,怎入深山伏怪王!
大圣正自恐惧,又见那山中迸出一道沙来。好沙,真个是遮天蔽日!你看:
纷纷遍天涯,邓邓浑浑大地遮。
细尘到处迷人目,粗灰满谷滚芝麻。
采药仙僮迷失伴,打柴樵子没寻家。
手中就有明珠现,时间刮得眼生花。

这行者只顾看玩,不觉沙灰飞入鼻内,痒斯斯的,打了两个喷嚏,即回头伸手,
在岩下摸了两个鹅卵石,塞住鼻子;摇身一变,变作一个攒火的鹞子,飞入烟火中
间,蓦了几蓦,却就没了沙灰,烟火也息了。急现本象下来。又看时,只听得丁丁
东东的,一个铜锣声响。却道:“我走错了路也!这里不是妖精住处。锣声似铺兵之
锣。想是通国的大路,有铺兵去下文书。且等老孙去问他一问。”

正走处,忽见是个小妖儿,担着黄旗,背着文书,敲着锣儿,急走如飞而来。
行者笑道:“原来是这厮打锣。他不知送的是甚么书信,等我听他一听。”好大圣,
摇身一变,变做个猛虫儿,轻轻的飞在他书包之上。只听得那妖精敲着锣,绪绪聒
聒的自念自诵道:“我家大王,忒也心毒。三年前到朱紫国强夺了金圣皇后,一向
无缘,未得沾身,只苦了要来的宫女顶缸。两个来,弄杀了;四个来,也弄杀了。
前年要了,去年又要,今年还要,却撞个对头来了。那个要宫女的先锋,被个甚么
孙行者打败了,不发宫女。我大王因此发怒,要与他国争持,教我去下甚么战书。
这一去,那国王不战则可,战必不利。我大王使烟火飞沙,那国王君臣百姓等,莫
想一个得活。那时,我等占了他的城池,大王称帝,我等称臣,虽然也有个大小官
爵,只是天理难容也!”

行者听了,暗喜道:“妖精也有存心好的。似他后边这两句话说,‘天理难容’,
却不是个好的?但只说金圣皇后一向无缘,未得沾身,此话却不解其意。等我问他
一问。”嘤的一声,一翅飞离了妖精,转向前路,有十数里地,摇身一变,又变做
一个道童:
头挽双抓髻,身穿百衲衣。
手敲鱼鼓简,口唱道情词。
转山坡,迎着小妖,打个起手道:“长官,那里去?送的是甚么公文?”那妖物就像
认得他的一般。住了锣槌,笑嘻嘻的还礼道:“我大王差我到朱紫国下战书的。”行
者接口问道:“朱紫国那话儿,可曾与大王配合哩?”小妖道:“自前年摄得来,当
时就有一个神仙,送一件五彩仙衣与金圣宫妆新。他自穿了那衣,就浑身上下都生
了针刺,我大王摸也不敢摸他一摸。但挽着些儿,手心就痛,不知是甚缘故。自始
至今,尚未沾身。早间差先锋去要宫女伏侍,被一个甚么孙行者战败了。大王奋怒,
所以教我去下战书,明日与他交战也。”行者道:“怎的大王却着恼呵?”小妖道:
“正在那里着恼哩。你去与他唱个道情词儿解解闷也好。”

行者拱手抽身就走。那妖依旧敲锣前行。行者就行起凶来,掣出棒,复转身,
望小妖脑后一下,可怜就打得头烂血流浆迸出,皮开颈折命倾之!收了棍子,却又
自悔道:“急了些儿,不曾问他叫做甚么名字,罢了!”却去取下他的战书,藏于袖
内;将他黄旗、铜锣,藏在路旁草里;因扯着脚要往涧下时,只听当的一声,腰
间露出一个镶金的牙牌。牌上有字,写道:

心腹小校一名,有来有去。五短身材,挞脸,无须。长川悬挂,无牌即假。
行者笑道:“这厮名字叫做有来有去,这一棍子,打得‘有去无来’也!”将牙牌解
下,带在腰间,欲要下尸骸;却又思量起烟火之毒,且不敢寻他洞府,即将棍子
举起,着小妖胸前捣了一下,挑在空中,径回本国,且当报一个头功。你看他自思
自念,唿哨一声,到了国界。

那八戒在金銮殿前,正护持着王、师,忽回头看见行者半空中将个妖精挑来,
他却怨道:“嗳!不打紧的买卖!早知老猪去拿来,却不算我一功?”说未毕,行者
按落云头,将妖精在阶下。八戒跑上去,就筑了一钯道:“此是老猪之功!”行者
道:“是你甚功?”八戒道:“莫赖我!我有证见!你不看一钯筑了九个眼子哩!”行
者道:“你看看可有头没头。”八戒笑道:“原来是没头的!我道如何筑他也不动动
儿。”行者道:“师父在那里?”八戒道:“在殿里与王叙话哩。”行者道:“你且去
请他出来。”八戒急上殿,点点头。三藏即便起身下殿,迎着行者。行者将一封战
书,揣在三藏袖里道:“师父收下,且莫与国王看见。”

说不了,那国王也下殿,迎着行者道:“神僧孙长老来了!拿妖之事如何?”行
者用手指道:“那阶下不是妖精,被老孙打杀了也?”国王见了道:“是便是个妖尸,
却不是赛太岁。赛太岁寡人亲见他两次:身长丈八,膊阔五停;面似金光,声如霹
雳;那里是这般鄙矮。”行者笑道:“陛下认得。果然不是。这是一个报事的小妖,
撞见老孙,却先打死,挑回来报功。”国王大喜道:“好,好,好!该算头功!寡人这
里常差人去打探,更不曾得个的实。似神僧一出,就捉了一个回来,真神通也!”
叫:“看暖酒来,与长老贺功!”

行者道:“吃酒还是小事。我问陛下,金圣宫别时,可曾留下个甚么表记?你与
我些儿。”那国王听说“表记”二字,却似刀剑剜心,忍不住失声泪下,说道:
“当年佳节庆朱明,太岁凶妖发喊声。
强夺御妻为压寨,寡人献出为苍生。
更无会话并离话,那有长亭共短亭!
表记香囊全没影,至今撇我苦伶仃!”
行者道:“陛下在迩,何以为脑?那娘娘既无表记,他在宫内,可有甚么心爱之物,
与我一件也罢。”国王道:“你要怎的?”行者道:“那妖王实有神通。我见他放烟、
放火、放沙,果是难收。纵收了,又恐娘娘见我面生,不肯跟我回国。须是得他平
日心爱之物一件,他方信我,我好带他回来。为此故要带去。”国王道:“昭阳宫里,
梳妆阁上,有一双黄金宝串,原是金圣宫手上带的。只因那日端午,要缚五色彩线,
故此褪下,不曾带上。此乃是他心爱之物。如今现收在减妆盒里。寡人见他遭此离
别,更不忍见;一见即如见他玉容,病又重几分也。”行者道:“且休题这话。且将
金串取来。如舍得,都与我拿去;如不舍,只拿一只去也。”国王遂命玉圣宫取出。
取出即递与国王。国王见了,叫了几声“知疼着热的娘娘”,遂递与行者。行者接
了,套在胳膊上。

好大圣,不吃得功酒,且驾筋斗云,唿哨一声,又至麒麟山上。无心玩景,径
寻洞府而去。正行时,只听得人语喧嚷,即伫立凝睛观看。原来那獬豸洞口把门的
大小头目,约摸有五百名,在那里:

森森罗列,密密挨排:森森罗列执干戈,映日光明;密密挨排展旌旗,迎风飘
闪。虎将熊师能变化,豹头彪帅弄精神。苍狼多猛烈,獭象更骁雄。狡兔乖獐轮剑
戟,长蛇大蟒挎刀弓。猩猩能解人言语,引阵安营识汛风。
行者见了,不敢前进,抽身径转旧路。你道他抽身怎么?不是怕他。他却至那打死
小妖之处,寻出黄旗、铜锣、迎风捏诀,想象腾那,即摇身一变,变做那有来有去
的模样,乒乓敲着锣,大踏步,一直前来,径撞至獬豸洞。正欲看看洞景,只闻得
猩猩出语道:“有来有去,你回来了?”行者只得答应道:“来了。”猩猩道:“快走!
大王爷爷正在剥皮亭上等你回话哩。”行者闻言,拽开步,敲着锣,径入前门里看
处,原来是悬崖削壁石屋虚堂,左右有琪花瑶草,前后多古柏乔松。不觉又至二门
之内,忽抬头见一座八窗明亮的亭子,亭子中间有一张戗金的交椅,椅子上端坐着
一个魔王,真个生得恶象。但见他:
幌幌霞光生顶上,威威杀气迸胸前。
口外獠牙排利刃,鬓边焦发放红烟。
嘴上髭须如插箭,遍体昂毛似迭毡。
眼突铜铃欺太岁,手持铁杵若摩天。

行者见了,公然傲慢那妖精,更不循一些儿礼法。调转脸,朝着外,只管敲锣。
妖王问道:“你来了?”行者不答。又问:“有来有去,你来了?”也不答应。妖王
上前扯住道:“你怎么到了家还筛锣?问之又不答,何也?”行者把锣往地下一掼道:
“甚么‘何也,何也’!我说我不去,你却教我去。行到那厢,只见无数的人马列
成阵势,见了我,就都叫:‘拿妖精!拿妖精!’把我揪揪扯扯,拽拽扛扛,拿进城
去,见了那国王,国王便教‘斩了’,幸亏那两班谋士道:‘两家相争,不斩来使。’
把我饶了。收了战书,又押出城外,对军前打了三十顺腿,放我来回话。他那里不
久就要来此与你交战哩。”妖王道:“这等说,是你吃亏了。怪不道问你更不言语。”
行者道:“却不是怎的?只为护疼,所以不曾答应。”妖王道:“那里有多少人马?”
行者道:“我也唬昏了,又吃他打怕了,那里曾查他人马数目!只见那里森森兵器摆
列着:

弓箭刀枪甲与衣,干戈剑戟开缨旗。剽枪月铲兜鍪铠,大斧团牌铁蒺藜。长闷
棍,短窝槌,钢叉铳及头盔。打扮得鞋护顶并胖祆,简鞭袖弹与铜锤。”
那王听了笑道:“不打紧,不打紧!似这般兵器,一火皆空。你且去报与金圣娘娘得
知,教他莫恼。今早他听见我发狠,要去战斗,他就眼泪汪汪的不干。你如今去说
那里人马骁勇,必然胜我,且宽他一时之心。”

行者闻言,十分欢喜道:“正中老孙之意!”你看他偏是路熟,转过角门,穿过
厅堂。那里边尽都是高堂大厦,更不似前边的模样。直到后面宫里,远见彩门壮丽,
乃是金圣娘娘住处。直入里面看时,有两班妖狐、妖鹿,一个个都妆成美女之形,
侍立左右。正中间坐着那个娘娘,手托着香腮,双眸滴泪,果然是:

玉容娇嫩,美貌妖娆。懒梳妆,散鬓堆鸦;怕打扮,钗环不戴。面无粉,冷淡
了胭脂;发无油,蓬松了云鬓。努樱唇,紧咬银牙;皱蛾眉,泪淹星眼。一片心,
只忆着朱紫君王;一时间,恨不离天罗地网。诚然是:自古红颜多薄命,恹恹无语
对东风!
行者上前打了个问讯道:“接喏。”那娘娘道:“这泼村怪,十分无状!想我在那朱紫
国中,与王同享荣华之时,那太师宰相见了,就俯伏尘埃,不敢仰视。这野怪怎么
叫声‘接喏’?是那里来的这般村泼?”众侍婢上前道:“太太息怒。他是大王爷爷
心腹的小校,唤名有来有去。今早差下战书的是他。”娘娘听说,忍怒问曰:“你下
战书,可曾到朱紫国界?”行者道:“我持书直至城里,到于金銮殿,面见君王,
已讨回音来也。”娘娘道:“你面君,君有何言?”行者道:“那君王敌战之言,与
排兵布阵之事,才与大王说了。只是那君王有思想娘娘之意,有一句合心的话儿,
特来上禀。奈何左右人众,不是说处。”

娘娘闻言,喝退两班狐鹿。行者掩上宫门,把脸一抹,现了本象。对娘娘道:
“你休怕我。我是东土大唐差往大西天天竺国雷音寺见佛求经的和尚。我师父是唐
王御弟唐三藏。我是他大徒弟孙悟空。因过你国倒换关文,见你君臣出榜招医,是
我大施三折之肱,把他相思之病治好了。排宴谢我,饮酒之间,说出你被妖摄来,
我会降龙伏虎,特请我来捉怪,救你回国。那战败先锋是我,打死小妖也是我。我
见他门外凶狂,是我变作有来有去模样,舍身到此,与你通信。”那娘娘听说,沉
吟不语。行者取出宝串,双手奉上道:“你若不信,看此物何来。”娘娘一见垂泪。
下座拜谢道:“长老,你果是救得我回朝,没齿不忘大恩!”

行者道:“我且问你,他那放火,放烟,放沙的,是件甚么宝贝?”娘娘道:“那
里是甚宝贝!乃是三个金铃。他将头一个幌一幌,有三百丈火光烧人;第二个幌一
幌,有三百丈烟光熏人;第三个幌一幌,有三百丈黄沙迷人。烟、火还不打紧,只
是黄沙最毒。若钻入人鼻孔,就伤了性命。”行者道:“利害!利害!我曾经着,打了
两个嚏喷,却不知他的铃儿放在何处?”娘娘道:“他那肯放下,只是带在腰间,
行住坐卧,再不离身。”行者道:“你若有意于朱紫国,还要相会国王,把那烦恼忧
愁,都且权解,使出个风流喜悦之容,与他叙个夫妻之情,教他把铃儿与你收贮。
待我取便偷了,降了这妖怪,那时节,好带你回去,重谐鸾凤,共享安宁也。”那
娘娘依言。

这行者还变作心腹小校,开了宫门,唤进左右侍婢。娘娘叫:“有来有去,快
往前亭,请你大王来,与他说话。”好行者,应了一声,即至剥皮亭,对妖精道:“大
王,圣宫娘娘有请。”妖王欢喜道:“娘娘常时只骂,怎么今日有请?”行者道:“那
娘娘问朱紫国王之事,是我说:‘他不要你了,他国中另扶了皇后。’娘娘听说,故
此没了想头,方才命我来奉请。”妖王大喜道:“你却中用。待我剿除了他国,封你
为个随朝的太宰。”

行者顺口谢恩,疾与妖王来至后宫门首。那娘娘欢容迎接,就去用手相搀。那
妖王喏喏而退道:“不敢!不敢!多承娘娘下爱,我怕手痛,不敢相傍。”娘娘道:“大
王请坐,我与你说。”妖王道:“有话但说不妨。”娘娘道:“我蒙大王辱爱,今已三
年,未得共枕同衾。也是前世之缘,做了这场夫妻;谁知大王有外我之意,不以夫
妻相待。我想着当时在朱紫国为后,外邦凡进贡之宝,君看毕,一定与后收之。你
这里更无甚么宝贝,左右穿的是貂裘,吃的是血食,那曾见绫锦金珠!只一味铺皮
盖毯。或者就有些宝贝,你因外我,也不教我看见,也不与我收着。且如闻得你有
三个铃铛,想就是件宝贝,你怎么走也带着,坐也带着?你就拿与我收着,待你用
时取出,未为不可。此也是做夫妻一场,也有个心腹相托之意。如此不相托付,非
外我而何?”

妖王大笑陪礼道:“娘娘怪得是,怪得是。宝贝在此,今日就当付你收之。”便
即揭衣取宝。行者在旁,眼不转睛,看着那怪揭起两三层衣服,贴身带着三个铃儿。
他解下来,将些绵花塞了口儿,把一块豹皮作一个包袱儿包了,递与娘娘道:“物
虽微贱,却要用心收藏,切不可摇幌着他。”娘娘接过手道:“我晓得。安在这妆台
之上,无人摇动。”叫:“小的们,安排酒来,我与大王交欢会喜,饮几杯儿。”众
侍婢闻言,即铺排果菜,摆上些獐鹿兔之肉,将椰子酒斟来奉上。那娘娘做出妖
娆之态,哄着精灵。

孙行者在旁取事,但挨挨摸摸,行近妆台,把三个金铃轻轻拿过,慢慢移步,
溜出宫门,径离洞府。到了剥皮亭前,无人处,展开豹皮幅子看时,中间一个,有
茶钟大;两头两个,有拳头大。他不知利害,就把绵花扯了。只闻得当的一欢响,
骨都都的迸出烟火黄沙,急收不住,满亭中烘烘火起。唬得那把门精怪,一拥撞入
后宫,惊动了妖王,慌忙教:“去救火!救火!”出来看时,原来是有来有去拿了金
铃儿哩。妖王上前喝道:“好贱奴!怎么偷了我的金铃宝贝,在此胡弄!”叫:“拿来!
拿来!”那门前虎将、熊师、豹头、彪帅、獭象、苍狼、乖獐、狡兔、长蛇、大蟒、
猩猩,帅众妖一齐攒簇。

那行者慌了手脚,丢了金铃,现出本象。掣出金箍如意棒,撒开解数,往前乱
打。那妖王收了宝贝,传号令,教:“关了前门!”众妖听了,关门的关门,打仗的
打仗。

那行者难得脱身,收了棒,摇身一变,变作个痴苍蝇儿,钉在那无火处石壁上。
众妖寻不见。报道:“大王,走了贼也!走了贼也!”妖王问:“可曾自门里走出去?”
众妖都说:“前门紧锁牢拴在此,不曾走出。”妖王只说:“仔细搜寻!”有的取水泼
火,有的仔细搜寻,更无踪迹。妖王怒道:“是个甚么贼子,好大胆,变作有来有
去的模样,进来见我回话,又跟在身边,乘机盗我宝贝!早是不曾拿将出去,若拿
出山头,见了天风,怎生是好?”虎将上前道:“大王的洪福齐天,我等的气数不
尽,故此知觉了。”熊师上前道:“大王,这贼不是别人,定是那战败先锋的那个孙
悟空。想必路上遇着有来有去,伤了性命,夺了黄旗、铜锣、牙牌,变作他的模样,
到此欺骗了大王也。”妖王道:“正是,正是,见得有理!”叫:“小的们,仔细搜求
防避,切莫开门放出走了!”这才是个有分教:
弄巧翻成拙,作耍却为真。

毕竟不知孙行者怎么脱得妖门,且听下回分解。