\chapter{色邪淫戏唐三藏~性正修持不坏身}

却说孙大圣与猪八戒正要使法定那些妇女,忽闻得风响处,沙僧嚷闹,急回头
时,不见了唐僧。行者道:“是甚人来抢师父去了?”沙僧道:“是一个女子,弄阵
旋风,把师父摄了去也。”行者闻言,唿哨跳在云端里,用手搭凉篷,四下里观看。
只见一阵灰尘,风滚滚,往西北上去了。急回头叫道:“兄弟们,快驾云同我赶师
父去来!”八戒与沙僧,即把行囊捎在马上,响一声,都跳在半空里去。慌得那西
梁国君臣女辈,跪在尘埃,都道:“是白日飞升的罗汉,我主不必惊疑。唐御弟也
是个有道的禅僧,我们都有眼无珠,错认了中华男子,枉费了这场神思。请主公上
辇回朝也。”女王自觉惭愧,多官都一齐回国不题。

却说孙大圣兄弟三人腾空踏雾,望着那阵旋风,一直赶来,前至一座高山,只
见灰尘息静,风头散了,更不知怪向何方。兄弟们按落云雾,找路寻访,忽见一壁
厢,青石光明,却似个屏风模样。三人牵着马转过石屏,石屏后有两扇石门,门上
有六个大字,乃是“毒敌山琵琶洞”。八戒无知,上前就使钉钯筑门。行者急止住
道:“兄弟莫忙。我们随旋风赶便赶到这里,寻了这会,方遇此门,又不知深浅如
何。倘不是这个门儿,却不惹他见怪?你两个且牵了马,还转石屏前立等片时,待
老孙进去打听打听,察个有无虚实,却好行事。”沙僧听说,大喜道:“好,好,好!
正是粗中有细,果然急处从宽。”他二人牵马回头。

孙大圣显个神通,捻着诀,念个咒语,摇身一变,变作蜜蜂儿,真个轻巧!你
看他:
翅薄随风软,腰轻映日纤。
嘴甜曾觅蕊,尾利善降蟾。
酿蜜功何浅,投衙礼自谦。
如今施巧计,飞舞入门檐。
行者自门瑕处钻将进去,飞过二层门里,只见正当中花亭子上端坐着一个女怪,左
右列几个彩衣绣服,丫髻两鬏的女童,都欢天喜地,正不知讲论甚么。这行者轻轻
的飞上去,钉在那花亭格子上,侧耳才听,又见两个总角蓬头女子,捧两盘热腾腾
的面食,上亭来道:“奶奶,一盘是人肉馅的荤馍馍,一盘是邓沙馅的素馍馍。”那
女怪笑道:“小的们,搀出唐御弟来。”几个彩衣绣服的女童,走向后房,把唐僧扶
出。那师父面黄唇白,眼红泪滴。行者在暗中嗟叹道:“师父中毒了!”

那怪走下亭,露春葱十指纤纤,扯住长老道:“御弟宽心。我这里虽不是西梁
女国的宫殿,不比富贵奢华,其实却也清闲自在,正好念佛看经。我与你做个道伴
儿,真个是百岁和谐也。”三藏不语。那怪道:“且休烦恼。我知你在女国中赴宴之
时,不曾进得饮食。这里荤素面饭两盘,凭你受用些儿压惊。”

三藏沉思默想道:“我待不说话,不吃东西,此怪比那女王不同,女王还是人
身,行动以礼;此怪乃是妖神,恐为加害,奈何?……我三个徒弟,不知我困陷在
于这里,倘或加害,却不枉丢性命?……”以心问心,无计所奈,只得强打精神,
开口道:“荤的何如?素的何如?”女怪道:“荤的是人肉馅馍馍,素的是邓沙馅馍
馍。”三藏道:“贫僧吃素。”那怪笑道:“女童,看热茶来,与你家长爷爷吃素馍馍。”
一女童,果捧着香茶一盏,放在长老面前。那怪将一个素馍馍劈破,递与三藏。三
藏将个荤馍馍囫囵递与女怪。女怪笑道:“御弟,你怎么不劈破与我?”三藏合掌
道:“我出家人,不敢破荤。”那女怪道:“你出家人不敢破荤,怎么前日在子母河
边吃水高,今日又好吃邓沙馅?”三藏道:“水高船去急,沙陷马行迟。”

行者在格子眼听着两个言语相攀,恐怕师父乱了真性,忍不住,现了本相,掣
铁棒喝道:“孽畜无礼!”那女怪见了,口喷一道烟光,把花亭子罩住,教:“小的
们,收了御弟!”他却拿一柄三股钢叉,跳出亭门,骂道:“泼猴惫懒!怎么敢私入
吾家,偷窥我容貌!不要走!吃老娘一叉!”这大圣使铁棒架住,且战且退。

二人打出洞外。那八戒、沙僧,正在石屏前等候,忽见他两个争持,慌得八戒
将白马牵过道:“沙僧,你只管看守行李、马匹,等老猪去帮打帮打。”好呆子,双
手举钯,赶上前叫道:“师兄靠后,让我打这泼贼!”那怪见八戒来,他又使个手段,
“呼了”一声,鼻中出火,口内生烟,把身子抖了一抖,三股叉飞舞冲迎。那女怪
也不知有几只手,没头没脸的滚将来。这行者与八戒,两边攻住。那怪道:“孙悟
空,你好不识进退!我便认得你,你是不认得我。你那雷音寺里佛如来,也还怕我
哩。量你这两个毛人,到得那里!都上来,一个个仔细看打!”这一场怎见得好战:

女怪威风长,猴王气概兴。天蓬元帅争功绩,乱举钉钯要显能。那一个手多叉
紧烟光绕,这两个性急兵强雾气腾。女怪只因求配偶,男僧怎肯泄元精!阴阳不对
相持斗,各逞雄才恨苦争。阴静养荣思动动,阳收息卫爱清清。致令两处无和睦,
叉钯铁棒赌输赢。这个棒有力,钯更能,女怪钢叉丁对丁。毒敌山前三不让,琵琶
洞外两无情。那一个喜得唐僧谐凤侣,这两个必随长老取真经。惊天动地来相战,
只杀得日月无光星斗更!
三个斗罢多时,不分胜负。那女怪将身一纵,使出个倒马毒桩,不觉的把大圣头皮
上扎了一下。行者叫声:“苦啊!”忍耐不得,负痛败阵而走。八戒见事不谐,拖着
钯彻身而退。那怪得了胜,收了钢叉。

行者抱头,皱眉苦面,叫声“利害!利害!”八戒到跟前问道:“哥哥,你怎么
正战到好处,却就叫苦连天的走了?”行者抱着头,只叫:“疼,疼,疼!”沙僧道:
“想是你头风发了?”行者跳道:“不是,不是!”八戒道:“哥哥,我不曾见你受
伤,却头疼,何也?”行者哼哼的道:“了不得,了不得!我与他正然打处,他见我
破了他的叉势,他就把身子一纵,不知是件甚么兵器,着我头上扎了一下,就这般
头疼难禁;故此败了阵来。”八戒笑道:“只这等静处常夸口,说你的头是修炼过的。
却怎么就不禁这一下儿?”行者道:“正是。我这头,自从修炼成真,盗食了蟠桃
仙酒,老子金丹;大闹天宫时,又被玉帝差大力鬼王、二十八宿,押赴斗牛宫外处
斩,那些神将使刀斧锤剑,雷打火烧;及老子把我安于八卦炉,锻炼四十九日,俱
未伤损。今日不知这妇人用的是甚么兵器,把老孙头弄伤也!”沙僧道:“你放了手,
等我看看。莫破了!”行者道:“不破,不破!”八戒道:“我去西梁国讨个膏药你贴
贴。”行者道:“又不肿不破,怎么贴得膏药?”八戒笑道:“哥啊,我的胎前产后
病倒不曾有,你倒弄了个脑门痈了。”沙僧道:“二哥且休取笑。如今天色晚矣,大
哥伤了头,师父又不知死活,怎的是好!”

行者哼道:“师父没事。我进去时,变作蜜蜂儿,飞入里面,见那妇人坐在花
亭子上。少顷,两个丫鬟,捧两盘馍馍:一盘是人肉馅,荤的;一盘是邓沙馅,素
的。又着两个女童扶师父出来吃一个压惊,又要与师父做甚么道伴儿。师父始初不
与那妇人答话,也不吃馍馍;后见他甜言美语,不知怎么,就开口说话,却说吃素
的。那妇人就将一个素的劈开,递与师父。师父将个囫囵荤的递与那妇人。妇人道:
‘怎不劈破?’师父道:‘出家人不敢破荤。’那妇人道:‘既不破荤,前日怎么在
子母河边饮水高,今日又好吃邓沙馅?’师父不解其意,答他两句道:‘水高船去
急,沙陷马行迟。’我在格子上听见,恐怕师父乱性,便就现了原身,掣棒就打。
他也使神通,喷出烟雾,叫‘收了御弟’,就轮钢叉,与老孙打出洞来也。”沙僧听
说,咬指道:“这泼贼也不知从那里就随将我们来,把上项事都知道了!”八戒道:
“这等说,便我们安歇不成?莫管甚么黄昏半夜,且去他门上索战,嚷嚷闹闹,搅
他个不睡,莫教他捉弄了我师父。”行者道:“头疼,去不得!”沙僧道:“不须索战。
一则师兄头痛;二来我师父是个真僧,决不以色空乱性。且就在山坡下,闭风处,
坐这一夜,养养精神,待天明再作理会。”遂此,三个弟兄,拴牢白马,守护行囊,
就在坡下安歇不题。

却说那女怪放下凶恶之心,重整欢愉之色,叫:“小的们,把前后门都关紧了。”
又使两个支更,防守行者。但听门响,即时通报。却又教:“女童,将卧房收拾齐
整,掌烛焚香,请唐御弟来,我与他交欢。”遂把长老从后边搀出。那女怪弄出十
分娇媚之态,携定唐僧道:“常言‘黄金未为贵,安乐值钱多’。且和你做会夫妻儿,
耍子去也。”

这长老咬定牙关,声也不透。欲待不去,恐他生心害命,只得战兢兢,跟着他
步入香房。却如痴如哑,那里抬头举目,更不曾看他房里是甚床铺幔帐,也不知有
甚箱笼梳妆。那女怪说出的雨意云情,亦漠然无听。好和尚,真是那:

目不视恶色,耳不听淫声。他把这锦绣娇容如粪土,金珠美貌若灰尘。一生只
爱参禅,半步不离佛地。那里会惜玉怜香,只晓得修真养性。那女怪,活泼泼,春
意无边;这长老,死丁丁,禅机有在。一个似软玉温香,一个如死灰槁木。那一个,
展鸳衾,淫兴浓浓;这一人,束褊衫,丹心耿耿。那个要贴胸交股和鸾凤,这个要
面壁归山访达摩。女怪解衣,卖弄他肌香肤腻;唐僧敛衽,紧藏了糙肉粗皮。女怪
道:“我枕剩衾闲何不睡?”唐僧道:“我头光服异怎相陪!”那个道:“我愿作前朝
柳翠翠。”这个道:“贫僧不是月黎。”女怪道:“我美若西
施还袅娜。”唐僧道:“我越王因此久埋尸。”女怪道:“御弟,你记得‘宁教花下死,
做鬼也风流’?”唐僧道:“我的真阳为至宝,怎肯轻与你这粉骷髅……”
他两个散言碎语的,直斗到更深,唐长老全不动念。那女怪扯扯拉拉的不放,这师
父只是老老成成的不肯。直缠到有半夜时候,把那怪弄得恼了,叫:“小的们,拿
绳来!”可怜将一个心爱的人儿,一条绳,捆的像个猱狮模样。又教拖在房廊下去,
却吹灭银灯,各归寝处。

一夜无词,不觉的鸡声三唱。那山坡下孙大圣欠身道:“我这头疼了一会,到
如今也不疼不麻,只是有些作痒。”八戒笑道:“痒便再教他扎一下,何如?”行者
啐了一口道:“放,放,放!”八戒又笑道:“放,放,放!我师父这一夜倒浪,浪,
浪!”沙僧道:“且莫斗口。天亮了,快赶早儿捉妖怪去。”行者道:“兄弟,你只管
在此守马,休得动身。猪八戒跟我去。”

那呆子抖擞精神,束一束皂锦直裰,相随行者,各带了兵器,跳上山崖,径至
石屏之下。行者道:“你且立住。只怕这怪物夜里伤了师父,先等我进去打听打听。
倘若被他哄了,丧了元阳,真个亏了德行,却就大家散火;若不乱性情,禅心未动,
却好努力相持,打死精怪,救师西去。”八戒道:“你好痴哑!常言道:‘干鱼可好与
猫儿作枕头?’就不如此,就不如此,也要抓你几把是!”行者道:“莫胡疑乱说,
待我看去。”

好大圣,转石屏,别了八戒。摇身还变个蜜蜂儿,飞入门里。见那门里有两个
丫鬟,头枕着绑铃,正然睡哩。却到花亭子观看,那妖精原来弄了半夜,都辛苦了,
一个个都不知天晓,还睡着哩。

行者飞来后面,隐隐的只听见唐僧声唤。忽抬头,见那步廊下四马攒蹄捆着师
父。行者轻轻的钉在唐僧头上,叫:“师父。”唐僧认得声音,道:“悟空来了?快救
我命!”行者道:“夜来好事如何?”三藏咬牙道:“我宁死也不肯如此。”行者道:
“昨日我见他有相怜相爱之意,却怎么今日把你这般挫折?”三藏道:“他把我缠
了半夜,我衣不解带,身未沾床。他见我不肯相从,才捆我在此。你千万救我取经
去也!”

他师徒们正然回答,早惊醒了那个妖精。妖精虽是下狠,却还有流连不舍之意;
一觉翻身,只听见“取经去也”一句,他就滚下床来,厉声高叫道:“好夫妻不做,
却取甚么经去?”

行者慌了,撇却师父,急展翅,飞将出去,现了本相,叫声“八戒”。那呆子
转过石屏道:“那话儿成了否?”行者笑道:“不曾,不曾。老师父被他摩弄不从,
恼了,捆在那里。正与我诉说前情,那怪惊醒了,我慌得出来也。”八戒道:“师父
曾说甚来?”行者道:“他只说衣不解带,身未沾床。”八戒笑道:“好,好,好!还
是个真和尚!我们救他去!”

呆子粗鲁,不容分说,举钉钯,望他那石头门上尽力气一钯,唿喇喇筑做几块。
唬得那几个枕梆铃睡的丫鬟,跑至二层门外,叫声“开门!前门被昨日那两个丑男
人打破了!”那女怪正出房门,只见四五个丫鬟跑进去报道:“奶奶,昨日那两个丑
男人又来把前门已打碎矣。”那怪闻言,即忙叫:“小的们!快烧汤洗面梳妆!”叫:
“把御弟连绳抬在后房收了。等我打他去!”

好妖精,走出来,举着三股叉,骂道:“泼猴!野彘!老大无知,你怎敢打破我
门!”八戒骂道:“滥淫贱货!你倒困陷我师父,返敢硬嘴!我师父是你哄将来做老公
的?快快送出饶你!敢再说半个‘不’字,老猪一顿钯,连山也筑倒你的!”那妖精
那容分说,抖擞身躯,依前弄法鼻口内喷烟冒火,举钢叉就刺八戒。八戒侧身躲过,
着钯就筑。孙大圣使铁棒并力相帮。那怪又弄神通,也不知是几只手,左右遮拦。
交锋三五个回合,不知是甚兵器,把八戒嘴唇上,也又扎了一下。那呆子拖着钯,
侮着嘴,负痛逃生。行者却也有些醋他,虚丢一棒,败阵而走。那妖精得胜而回,
叫小的们搬石块垒叠了前门不题。

却说那沙和尚正在坡前放马,只听得那里猪哼。忽抬头,见八戒侮着嘴,哼将
来。沙僧道:“怎的说?”呆子哼道:“了不得,了不得!疼,疼,疼!”说不了,行
者也到跟前,笑道:“好呆子啊!昨日咒我是脑门痈,今日却也弄做个肿嘴瘟了!”
八戒哼道:“难忍难忍,疼得紧!利害,利害!”

三人正然难处,只见一个老妈妈儿,左手提着一个青竹篮儿,自南山路上挑菜
而来。沙僧道:“大哥,那妈妈来得近了,等我问他个信儿,看这个是甚妖精,是
甚兵器,这般伤人。”行者道:“你且住,等老孙问他去来。”行者急睁睛看,只见
头直上有祥云盖顶,左右有香雾笼身。行者认得,即叫:“兄弟们,还不来叩头,
那妈妈是菩萨来也!”慌得猪八戒忍疼下拜,沙和尚牵马躬身,孙大圣合掌跪下,
叫声“南无大慈大悲救苦救难灵感观世音菩萨。”

那菩萨见他们认得元光,即踏祥云,起在半空,现了真象。原来是鱼篮之象。
行者赶到空中,拜告道:“菩萨,恕弟子失迎之罪!我等努力救师,不知菩萨下降;
今遇魔难难收,万望菩萨搭救搭救!”菩萨道:“这妖精十分利害。他那三股叉是生
成的两只钳脚。扎人痛者,是尾上一个钩子,唤做‘倒马毒’。本身是个蝎子精。
他前者在雷音寺听佛谈经,如来见了,不合用手推他一把,他就转过钩子,把如来
左手中拇指上扎了一下。如来也疼难禁,即着金刚拿他。他却在这里。若要救得唐
僧,除是别告一位方好。我也是近他不得。”行者再拜道:“望菩萨指示指示,别告
那位去好,弟子即去请他也。”菩萨道:“你去东天门里光明宫告求昴日星官,方能
降伏。”言罢,遂化作一道金光,径回南海。

孙大圣才按云头,对八戒、沙僧道:“兄弟放心,师父有救星了。”沙僧道:“是
那里救星?”行者道:“才然菩萨指示,教我告请昴日星官。老孙去来。”八戒侮着
嘴哼道:“哥啊,就问星官讨些止疼的药饵来!”行者笑道:“不须用药,只似昨日
疼过夜就好了。”沙僧道:“不必烦叙,快早去罢。”

好行者,急忙驾筋斗云。须臾,到东天门外。忽见增长天王当面作礼道:“大
圣何往?”行者道:“因保唐僧西方取经,路遇魔障缠身,要到光明宫见昴日星官
走走。”忽又见陶、张、辛、邓四大元帅,也问何往。行者道:“要寻昴日星官去降
妖救师。”四元帅道:“星官今早奉玉帝旨意,上观星台巡札去了。”行者道:“可有
这话?”辛天君道:“小将等与他同下斗牛宫,岂敢说假?”陶天君道:“今已许久,
或将回矣。大圣还先去光明宫;如未回,再去观星台可也。”大圣遂喜,即别他们,
至光明宫门首,果是无人,复抽身就走,只见那壁厢有一行兵士摆列,后面星官来
了。

那星官还穿的是拜驾朝衣,一身金缕。但见他:
冠簪五岳金光彩,笏执山河玉色琼。
袍挂七星云,腰围八极宝环明。
叮当响如敲韵,迅速风声似摆铃。
翠羽扇开来昴宿,天香飘袭满门庭。
前行的兵士,看见行者立于光明宫外,急转身报道:“主公,孙大圣在这里也。”那
星官敛云雾整束朝衣,停执事分开左右,上前作礼道:“大圣何来?”行者道:“专
来拜烦救师父一难。”星官道:“何难?有何地方?”行者道:“在西梁国毒敌山琵琶
洞。”星官道:“那山洞有甚妖怪,却来呼唤小神?”行者道:“观音菩萨适才显化,
说是一个蝎子精。特举先生方能治得,因此来请。”星官道:“本欲回奏玉帝,奈大
圣至此,又感菩萨举荐,恐迟误事,小神不敢请献茶,且和你去降妖精,却再来回
旨罢。”

大圣闻言,即同出东天门,直至西梁国。望见毒敌山不远,行者指道:“此山
便是。”星官按下云头,同行者至石屏前山坡之下。沙僧见了道:“二哥起来,大哥
请得星官来了。”那呆子还侮着嘴道:“恕罪,恕罪!有病在身,不能行礼。”星官道:
“你是修行之人,何病之有?”八戒道:“早间与那妖精交战,被他着我唇上扎了
一下,至今还疼呀。”星官道:“你上来,我与你医治医治。”呆子才放了手,口里
哼哼道:“千万治治,待好了谢你。”那星官用手把嘴唇上摸了一摸,吹一口气,
就不疼了。呆子欢喜下拜道:“妙啊!妙啊!”行者笑道:“烦星官也把我头上摸摸。”
星官道:“你未遭毒,摸他何为?”行者道:“昨日也曾遭过,只是过了夜,才不疼;
如今还有些麻痒,只恐发天阴,也烦治治。”星官真个也把头上摸了一摸,吹口气,
也就解了余毒,不麻不痒了。八戒发狠道:“哥哥,去打那泼贱去!”星官道:“正
是,正是。你两个叫他出来,等我好降他。”

行者与八戒跳上山坡,又至石屏之后。呆子口里乱骂,手似捞钩,一顿钉钯,
把那洞门外垒叠的石块爬开;闯至一层门,又一钉钯,将二门筑得粉碎。慌得那门
里小妖飞报:“奶奶!那两个丑男人,又把二层门也打破了!”那怪正教解放唐僧,
讨素茶饭与他吃哩,听见打破二门,即便跳出花亭子,轮叉来刺八戒。八戒使钉钯
迎架。行者在旁,又使铁棒来打。那怪赶至身边,要下毒手,他两个识得方法,回
头就走。

那怪赶过石屏之后,行者叫声“昴宿何在?”只见那星官立于山坡上,现出本
相,原来是一只双冠子大公鸡,昂起头来,约有六七尺高,对着妖精叫一声,那怪
即时就现了本象,是个琵琶来大小的蝎子精。星官再叫一声,那怪浑身酥软,死在
坡前。有诗为证,诗曰:
花冠绣颈若团缨,爪硬距长目怒睛。
踊跃雄威全五德,峥嵘壮势羡三鸣。
岂如凡鸟啼茅屋,本是天星显圣名。
毒蝎枉修人道行,还原反本见真形。
八戒上前,一只脚住那怪的胸道:“孽畜!今番使不得倒马毒了!”那怪动也不
动,被呆子一顿钉钯,捣作一团烂酱。那星官复聚金光,驾云而去。行者与八戒、
沙僧朝天拱谢道:“有累,有累,改日赴宫拜酬。”

三人谢毕。却才收拾行李、马匹,都进洞里。见那大小丫鬟,两边跪下,拜道:
“爷爷,我们不是妖邪,都是西梁国女人,前者被这妖精摄来的。你师父在后边香
房里坐着哭哩。”行者闻言,仔细观看,果然不见妖气,遂入后边叫道:“师父!”
那唐僧见众齐来,十分欢喜道:“贤徒,累及你们了。那妇人何如也?”八戒道:“那
厮原是个大母蝎子。幸得观音菩萨指示,大哥去天宫里请得那昴日星官下降,把那
厮收伏。才被老猪筑做个泥了,方敢深入于此,得见师父之面。”唐僧谢之不尽。
又寻些素米、素面,安排了饮食,吃了一顿。把那些摄将来的女子赶下山,指与回
家之路。点上一把火,把几间房宇,烧毁罄尽。请唐僧上马,找寻大路西行。正是:
割断尘缘离色相,推干金海悟禅心。

毕竟不知几年上才得成真,且听下回分解。