\chapter{外道弄强欺正法~心猿显圣灭诸邪}

话说那国王见孙行者有呼龙使圣之法,即将关文用了宝印,便要递与唐僧,放
行西路。那三个道士,慌得拜倒在金銮殿上启奏。那皇帝即下龙位,御手忙搀道:
“国师今日行此大礼,何也?”道士说:“陛下,我等至此,匡扶社稷,保国安民,
苦历二十年来,今日这和尚弄法力,抓了丢去,败了我们声名,陛下以一场之雨,
就恕杀人之罪,可不轻了我等也?望陛下且留住他的关文,让我兄弟与他再赌一赌,
看是何如。”

那国王着实昏乱,东说向东,西说向西,真个收了关文,道:“国师,你怎么
与他赌?”虎力大仙道:“我与他赌坐禅。”国王道:“国师差矣。那和尚乃禅教出
身,必然先会禅机,才敢奉旨求经;你怎与他赌此?”大仙道:“我这坐禅,比常
不同:有一异名,教做‘云梯显圣’。”国王道:“何为‘云梯显圣’?”大仙道:“要
一百张桌子,五十张作一禅台,一张一张叠将起去,不许手攀而上,亦不用梯凳而
登,各驾一朵云头,上台坐下,约定几个时辰不动。”

国王见此有些难处,就便传旨问道:“那和尚,我国师要与你赌‘云梯显圣’
坐禅,那个会么?”行者闻言,沉吟不答。八戒道:“哥哥,怎么不言语?”行者
道:“兄弟,实不瞒你说。若是踢天弄井,搅海翻江,担山赶月,换斗移星,诸般
巧事,我都干得;就是砍头剁脑,剖腹剜心,异样腾那,却也不怕;但说坐禅,我
就输了。我那里有这坐性?你就把我锁在铁柱子上,我也要上下爬,莫想坐得住。”
三藏忽的开言道:“我会坐禅。”行者欢喜道:“却好,却好!可坐得多少时?”三藏
道:“我幼年遇方上禅僧讲道,那性命根本上,定性存神,在死生关里,也坐二三
个年头。”行者道:“师父若坐二三年,我们就不取经罢;多也不上二三个时辰,就
下来了。”三藏道:“徒弟呀,却是不能上去。”行者道:“你上前答应,我送你上去。”
那长老果然合掌当胸道:“贫僧会坐禅。”国王教传旨,立禅台。国家有倒山之力,
不消半个时辰,就设起两座台,在金銮殿左右。

那虎力大仙下殿,立于阶心,将身一纵,踏一朵席云,径上西边台上坐下。行
者拔一根毫毛,变做假象,陪着八戒、沙僧,立于下面,他却作五色祥云,把唐僧
撮起空中,径至东边台上坐下。他又敛祥光,变作一个虫,飞在八戒耳朵边道:
“兄弟,仔细看着师父,再莫与老孙替身说话。”那呆子笑道:“理会得,理会得!”

却说那鹿力大仙在绣墩上坐看多时,他两个在高台上,不分胜负,这道士就助
他师兄一功:将脑后短发,拔了一根,捻着一团,弹将上去,径至唐僧头上,变作
一个大臭虫,咬住长老。那长老先前觉痒,然后觉疼。原来坐禅的不许动手,动手
算输。一时间疼痛难禁,他缩着头,就着衣襟擦痒。八戒道:“不好了!师父羊儿风
发了。”沙僧道:“不是,是头风发了。”行者听见道:“我师父乃志诚君子,他说会
坐禅,断然会坐;说不会,只是不会。君子家,岂有谬乎?你两个休言,等我上去
看看。”

好行者,嘤的一声,飞在唐僧头上,只见有豆粒大小一个臭虫叮他师父。慌忙
用手捻下,替师父挠挠摸摸。那长老不疼不痒,端坐上面。行者暗想道:“和尚头
光,虱子也安不得一个,如何有此臭虫?……想是那道士弄的玄虚,害我师父。哈
哈!枉自也不见输赢,等老孙去弄他一弄!”这行者飞将去,金殿兽头上落下,摇身
一变,变作一条七寸长的蜈蚣,径来道士鼻凹里叮了一下。那道士坐不稳,一个筋
斗,翻将下去,几乎丧了性命;幸亏大小官员人多救起。国王大惊,即着当驾太师
领他往文华殿里梳洗去了。行者仍驾祥云,将师父驮下阶前,已是长老得胜。

那国王只教放行。鹿力大仙又奏道:“陛下,我师兄原有暗风疾,因到了高处,
冒了天风,旧疾举发,故令和尚得胜。且留下他,等我与他赌‘隔板猜枚’。”国王
道:“怎么叫做‘隔板猜枚’?”鹿力道:“贫道有隔板知物之法,看那和尚可能够。
他若猜得过我,让他出去;猜不着,凭陛下问拟罪名,雪我昆仲之恨,不污了二十
年保国之恩也。”

真个那国王十分昏乱,依此谗言。即传旨,将一朱红漆的柜子,命内官抬到宫
殿。教娘娘放上件宝贝。须臾抬出,放在白玉阶前,教僧道:“你两家各赌法力,
猜那柜中是何宝贝。”三藏道:“徒弟,柜中之物,如何得知?”行者敛祥光,还变
作虫,钉在唐僧头上道:“师父放心,等我去看看来。”好大圣,轻轻飞到柜上,
爬在那柜脚之下,见有一条板缝儿。他钻将进去,见一个红漆丹盘,内放一套宫衣,
乃是山河社稷袄,乾坤地理裙。用手拿起来,抖乱了,咬破舌尖上,一口血哨喷将
去,叫声“变!”即变作一件破烂流丢一口钟;临行又撒上一泡臊溺,却还从板缝
里钻出来,飞在唐僧耳朵上道:“师父,你只猜是破烂流丢一口钟。”三藏道:“他
教猜宝贝哩,流丢是件甚宝贝?”行者道:“莫管他,只猜着便是。”

唐僧进前一步,正要猜,那鹿力大仙道:“我先猜,那柜里是山河社稷袄,乾
坤地理裙。”唐僧道:“不是,不是,柜里是件破烂流丢一口钟。”国王道:“这和尚
无礼!敢笑我国中无宝,猜甚么流丢一口钟!”教:“拿了!”那两班校尉,就要动手,
慌得唐僧合掌高呼:
“陛下,且赦贫僧一时,待打开柜看。端的是宝,贫僧领罪;如不是宝,却不屈了
贫僧也?”国王教打开看。当驾官即开了,捧出丹盘来看,果然是件破烂流丢一口
钟。

国王大怒道:“是谁放上此物?”龙座后面,闪上三宫皇后道:“我主,是梓童
亲手放的山河社稷袄,乾坤地理裙,却不知怎么变成此物。”国王道:“御妻请退,
寡人知之。宫中所用之物,无非是缎绢绫罗,那有此甚么流丢?”教:“抬上柜来,
等朕亲藏一宝贝,再试如何。”

那皇帝即转后宫,把御花园里仙桃树上结得一个大桃子
——有碗来大小——摘下,放在柜内,又抬下叫猜。唐僧道:“徒弟啊,又来猜了。”
行者道:“放心,等我再去看看。”又嘤的一声,飞将去,还从板缝儿钻进去;见是
一个桃子,正合他意,即现了原身,坐在柜里,将桃子一顿口啃得干干净净,连两
边腮凹儿都啃净了,将核儿安在里面。仍变虫,飞将出去,钉在唐僧耳朵上道:
“师父,只猜是个桃核子。”长老道:“徒弟啊,休要弄我。先前不是口快,几乎拿
去典刑。这番须猜宝贝方好。桃核子是甚宝贝?”行者道:“休怕,只管赢他便了。”

三藏正要开言,听得那羊力大仙道:“贫道先猜,是一颗仙桃。”三藏猜道:“不
是桃,是个光桃核子。”那国王喝道:“是朕放的仙桃,如何是核?三国师猜着了。”
三藏道:“陛下,打开来看就是。”当驾官又抬上去打开,捧出丹盘,果然是一个核
子,皮肉俱无。国王见了,心惊道:“国师,休与他赌斗了,让他去罢。寡人亲手
藏的仙桃,如今只是一核子,是甚人吃了?想是有鬼神暗助他也。”八戒听说,与沙
僧微微冷笑道:“还不知他是会吃桃子的积年哩!”

正话间,只见那虎力大仙从文华殿梳洗了,走上殿道:“陛下,这和尚有搬运
抵物之术,抬上柜来,我破他术法,与他再猜。”国王道:“国师还要猜甚?”虎力
道:“术法只抵得物件,却抵不得人身。将这道童藏在里面,管教他抵换不得。”这
小童果藏在柜里,掩上柜盖,抬将下去,教:“那和尚再猜,这三番是甚宝贝。”三
藏道:“又来了!”行者道:“等我再去看看。”嘤的又飞去,钻入里面,见是一个小
童儿。好大圣,他却有见识。果然是:
腾那天下少,似这伶俐世间稀!

他就摇身一变,变作个老道士一般容貌。进柜里叫声“徒弟。”童儿道:“师父,
你从那里来的?”行者道:“我使遁法来的。”童儿道:“你来有么教诲?”行者道:
“那和尚看见你进柜来了,他若猜个道童,却不又输了?是特来和你计较计较,剃
了头,我们猜和尚罢。”童儿道:“但凭师父处治,只要我们赢他便了。若是再输与
他,不但低了声名,又恐朝廷不敬重了。”行者道:“说得是。我儿过来。赢了他,
我重重赏你。”将金箍棒就变作一把剃头刀,搂抱着那童儿,口里叫道:“乖乖,忍
着疼,莫放声,等我与你剃头。”须臾,剃下发来,窝作一团,塞在那柜脚纥络里。
收了刀儿,摸着他的光头道:“我儿,头便像个和尚,只是衣裳不趁。脱下来,我
与你变一变。”那道童穿的一领葱白色云头花绢绣锦沿边的鹤氅,真个脱下来,被
行者吹一口仙气,叫“变!”即变做一件土黄色的直裰儿,与他穿了。却又拔下两
根毫毛,变作一个木鱼儿,递在他手里道:“徒弟,须听着。但叫道童,千万莫出
去;若叫和尚,你就与我顶开柜盖,敲着木鱼,念一卷佛经钻出来,方得成功也。”
童儿道:“我只会念《三官经》、《北斗经》、《消灾经》,不会念佛家经。”行者道:“你
可会念佛?”童儿道:“阿弥陀佛,那个不会念?”行者道:“也罢,也罢,就念佛,
省得我又教你,切记着,我去也。”还变虫,钻出去,飞在唐僧耳轮边道:“师
父,你只猜是个和尚。”三藏道:“这番他准赢了。”行者道:“你怎么定得?”三藏
道:“经上有云:‘佛、法、僧三宝。’和尚却也是一宝。”

正说处,只见那虎力大仙道:“陛下,第三番是个道童。”只管叫,他那里肯出
来。三藏合掌道:“是个和尚。”八戒尽力高叫道:“柜里是个和尚!”那童儿忽的顶
开柜盖,敲着木鱼,念着佛,钻出来。喜得那两班文武,齐声喝采。唬得那三个道
士,口无言。国王道:“这和尚是有鬼神辅佐!怎么道士入柜,就变做和尚?纵有
待诏跟进去,也只剃得头便了,如何衣服也能趁体,口里又会念佛?国师啊!让他去
罢!”

虎力大仙道:“陛下,左右是‘棋逢对手,将遇良材。’贫道将锺南山幼时学的
武艺,索性与他赌一赌。”国王道:“有甚么武艺?”虎力道:“弟兄三个,都有些
神通。会砍下头来,又能安上;剖腹剜心,还再长完;滚油锅里,又能洗澡。”国
王大惊道:“此三事都是寻死之路!”虎力道:“我等有此法力,才敢出此朗言,断
要与他赌个才休。”那国王叫道:“东土的和尚,我国师不肯放你,还要与你赌砍头
剖腹,下滚油锅洗澡哩。”

行者正变作虫,往来报事。忽听此言,即收了毫毛,现出本相,哈哈大笑
道:“造化,造化!买卖上门了!”八戒道:“这三件都是丧性命的事,怎么说买卖上
门?”行者道:“你还不知我的本事。”八戒道:“哥哥,你只像这等变化腾那也够
了,怎么还有这等本事?”行者道:“我啊:
砍下头来能说话,剁了臂膊打得人。
扎去腿脚会走路,剖腹还平妙绝伦。
就似人家包匾食,一捻一个就囫囵。
油锅洗澡更容易,只当温汤涤垢尘。”
八戒、沙僧闻言,呵呵大笑。行者上前道:“陛下,小和尚会砍头。”国王道:“你
怎么会砍头?”行者道:“我当年在寺里修行,曾遇着一个方上禅和子,教我一个
砍头法,不知好也不好,如今且试试新。”国王笑道:“那和尚年幼不知事。砍头那
里好试新?头乃六阳之首,砍下即便死矣。”虎力道:“陛下,正要他如此,方才出
得我们之气。”那昏君信他言语,即传旨,教设杀场。

一声传旨,即有羽林军三千,摆列朝门之外。国王教:“和尚先去砍头。”行者
欣然应道:“我先去,我先去!”拱着手,高呼道:“国师,恕大胆,占先了。”拽回
头,往外就走。唐僧一把扯住道:“徒弟呀,仔细些。那里不是耍处。”行者道:“怕
他怎的!撒了手,等我去来。”

那大圣径至杀场里面,被刽子手挝住了,捆做一团。按在那土墩高处,只听喊
一声“开刀!”飕的把个头砍将下来。又被刽子手一脚踢了去,好似滚西瓜一般,
滚有三四十步远近。行者腔子中更不出血。只听得肚里叫声:“头来!”慌得鹿力大
仙见有这般手段,即念咒语,教本坊土地神祇:“将人头扯住,待我赢了和尚,奏
了国王,与你把小祠堂盖作大庙宇,泥塑像改作正金身。”原来那些土地神祇因他
有五雷法,也服他使唤,暗中真个把行者头按住了。行者又叫声:“头来!”那头一
似生根,莫想得动。行者心焦,捻着拳,挣了一挣,将捆的绳子就皆挣断,喝声:
“长!”飕的腔子内长出一个头来。唬得那刽子手,个个心惊;羽林军,人人胆战。
那监斩官急走入朝奏道:“万岁,那小和尚砍了头,又长出一颗来了。”八戒冷笑道:
“沙僧,那知哥哥还有这般手段。”沙僧道:“他有七十二般变化,就有七十二个头
哩。”

说不了,行者走来,叫声“师父。”三藏大喜道:“徒弟,辛苦么?”行者道:
“不辛苦,倒好耍子。”八戒道:“哥哥,可用刀疮药么?”行者道:“你是摸摸看,
可有刀痕?”那呆子伸手一摸,就笑得呆呆睁睁道:“妙哉,妙哉!却也长得完全,
截疤儿也没些儿!”

兄弟们正都欢喜,又听得国王叫领关文:“赦你无罪。快去!快去!”行者道:“关
文虽领,必须国师也赴曹砍砍头,也当试新去来。”国王道:“大国师,那和尚也不
肯放你哩。你与他赌胜,且莫唬了寡人。”虎力也只得去,被几个刽子手,也捆翻
在地,幌一幌,把头砍下,一脚也踢将去,滚了有三十余步,他腔子里也不出血,
也叫一声:“头来!”行者即忙拔下一根毫毛,吹口仙气,叫“变!”变作一条黄犬,
跑入场中,把那道士头,一口衔来,径跑到御水河边丢下不题。

却说那道士连叫三声,人头不到,怎似行者的手段,长不出来,腔子中,骨都
都红光迸出。可怜空有唤雨呼风法,怎比长生果正仙?须臾,倒在尘埃。众人观看,
乃是一只无头的黄毛虎。

那监斩官又来奏:“万岁,大国师砍下头来,不能长出,死在尘埃,是一只无
头的黄毛虎。”国王闻奏,大惊失色。目不转睛,看那两个道士。鹿力起身道:“我
师兄已是命到禄绝了,如何是只黄虎!这都是那和尚惫懒,使的掩样法儿,将我师
兄变作畜类!我今定不饶他,定要与他赌那剖腹剜心!”

国王听说,方才定性回神。又叫:“那和尚,二国师还要与你赌哩。”行者道:
“小和尚久不吃烟火食,前日西来,忽遇斋公家劝饭,多吃了几个馍馍;这几日腹
中作痛,想是生虫,正欲借陛下之刀,剖开肚皮,拿出脏腑,洗净脾胃,方好上西
天见佛。”国王听说,教:“拿他赴曹。”那许多人,搀的搀,扯的扯。行者展脱手
道:“不用人搀,自家走去。但一件,不许缚手,我好用手洗刷脏腑。”国王传旨,
教:“莫绑他手。”

行者摇摇摆摆,径至杀场。将身靠着大桩,解开衣带,露出肚腹。那刽子手将
一条绳套在他膊项上,一条绳札住他腿足,把一口牛耳短刀,幌一幌,着肚皮下一
割,搠个窟窿。这行者双手爬开肚腹,拿出肠脏来,一条条理够多时,依然安在里
面。照旧盘曲,捻着肚皮,吹口仙气,叫“长!”依然长合。国王大惊,将他那关
文捧在手中道:“圣僧莫误西行,与你关文去罢。”行者笑道:“关文小可,也请二
国师剖剖剜剜,何如?”国王对鹿力说:“这事不与寡人相干,是你要与他做对头
的。请去,请去。”鹿力道:“宽心,料我决不输与他。”

你看他也像孙大圣,摇摇摆摆,径入杀场,被刽子手套上绳,将牛耳短刀,唿
喇的一声,割开肚腹,他也拿出肝肠,用手理弄。行者即拔一根毫毛,吹口仙气,
叫“变!”即变作一只饿鹰,展开翅爪,飕的把他五脏心肝,尽情抓去,不知飞向
何方受用。这道士弄做一个空腔破肚淋漓鬼,少脏无肠浪荡魂。那刽子手蹬倒大桩,
拖尸来看,呀!原来是一只白毛角鹿!

慌得那监斩官又来奏道:“二国师晦气,正剖腹时,被一只饿鹰将脏腑肝肠都
刁去了,死在那里。原身是个白毛角鹿也。”国王害怕道:“怎么是个角鹿?”那羊
力大仙又奏道:“我师兄既死,如何得现兽形?这都是那和尚弄术法坐害我等。等我
与师兄报仇者。”国王道:“你有甚么法力赢他?”羊力道:“我与他赌下滚油锅洗
澡,”国王便教取一口大锅,满着香油,教他两个赌去。行者道:“多承下顾。小和
尚一向不曾洗澡,这两日皮肤燥痒,好歹荡荡去。”

那当驾官果安下油锅,架起干柴,燃着烈火,将油烧滚,教和尚先下去。行者
合掌道:“不知文洗,武洗?”国王道:“文洗如何?武洗如何?”行者道:“文洗不
脱衣服,似这般叉着手,下去打个滚,就起来,不许污坏了衣服,若有一点油腻算
输。武洗要取一张衣架,一条手巾,脱了衣服,跳将下去,任意翻筋斗,竖蜻蜓,
当耍子洗也。”国王对羊力说:“你要与他文洗,武洗?”羊力道:“文洗恐他衣服
是药炼过的,隔油。武洗罢。”行者又上前道:“恕大胆,屡次占先了。”你看他脱
了布直裰,褪了虎皮裙,将身一纵,跳在锅内,翻波斗浪,就似负水一般顽耍。

八戒见了,咬着指头,对沙僧道:“我们也错看了这猴子了!平时间言讪语,
斗他耍子,怎知他有这般真实本事!”他两个唧唧哝哝,夸奖不尽。行者望见,心
疑道:“那呆子笑我哩!正是‘巧者多劳拙者闲’。老孙这般舞弄,他倒自在。等我
作成他捆一绳,看他可怕。”正洗浴,打个水花,淬在油锅底上,变作个枣核钉儿,
再也不起来了。

那监斩官近前又奏:“万岁,小和尚被滚油烹死了。”国王大喜,教捞上骨骸来
看。刽子手将一把铁笊篱,在油锅里捞,原来那笊篱眼稀,行者变得钉小,往往来
来,从眼孔漏下去了,那里捞得着!又奏道:“和尚身微骨嫩,俱札化了。”

国王教:“拿三个和尚下去!”两边校尉,见八戒面凶,先揪翻,把背心捆了。
慌得三藏高叫:“陛下,赦贫僧一时。我那个徒弟,自从归教,历历有功;今日冲
撞国师,死在油锅之内,奈何先死者为神,我贫僧怎敢贪生!正是天下官员也管着
天下百姓。陛下若教臣死,臣岂敢不死。只望宽恩,赐我半盏凉浆水饭,三张纸马,
容到油锅边,烧此一陌纸,也表我师徒一念,那时再领罪也。”国王闻言道:“也是,
那中华人多有义气。”命取些浆饭、黄钱与他。果然取了,递与唐僧。

唐僧教沙和尚同去。行至阶下,有几个校尉,把八戒揪着耳朵,拉在锅边。三
藏对锅祝曰:“徒弟孙悟空!
自从受戒拜禅林,护我西来恩爱深。
指望同时成大道,何期今日你归阴!
生前只为求经意,死后还存念佛心。
万里英魂须等候,幽冥做鬼上雷音!”
八戒听见道:“师父,不是这般祝了。沙和尚,你替我奠浆饭,等我祷。”那呆子捆
在地下,气呼呼的道:

“闯祸的泼猴子,无知的弼马温!该死的泼猴子,油烹的弼马温!猴儿了帐,马
温断根!”

孙行者在油锅底上,听得那呆子乱骂,忍不住现了本相。赤淋淋的,站在油锅
底道:“馕糟的夯货,你骂那个哩!”唐僧见了道:“徒弟,唬杀我也!”沙僧道:“大
哥干净推佯死惯了!”慌得那两班文武,上前来奏道:“万岁,那和尚不曾死,又打
油锅里钻出来了。”监斩官恐怕虚诳朝廷,却又奏道:“死是死了,只是日期犯凶,
小和尚来显魂哩。”

行者闻言大怒,跳出锅来,揩了油腻,穿上衣服,掣出棒,挝过监斩官,着头
一下,打做了肉团,道:“我显甚么魂哩!”唬得多官连忙解了八戒,跪地哀告:“恕
罪!恕罪!”国王走下龙座。行者上殿扯住道:“陛下不要走,且教你三国师也下下
油锅去。”那皇帝战战兢兢道:“三国师,你救朕之命,快下锅去,莫教和尚打我。”

羊力下殿,照依行者脱了衣服,跳下油锅,也那般支吾洗浴。

行者放了国王,近油锅边,叫烧火的添柴,却伸手探了一把,呀!那滚油都冰
冷,心中暗想道:“我洗时滚热,他洗时却冷。我晓得了,这不知是那个龙王,在
此护持他哩。”急纵身跳在空中,念声“”字咒语,把那北海龙王唤来:“我把你
这个带角的蚯蚓,有鳞的泥鳅!你怎么助道士冷龙护住锅底,教他显圣赢我!”唬得
那龙王喏喏连声道:“敖顺不敢相助。大圣原来不知。这个孽畜,苦修行了一场,
脱得本壳,却只是五雷法真受,其余都了傍门,难归仙道。这个是他在小茅山学
来的‘大开剥’。那两个已是大圣破了他法,现了本相。这一个也是他自己炼的冷
龙,只好哄瞒世俗之人耍子,怎瞒得大圣!小龙如今收了他冷龙,管教他骨碎皮焦,
显什么手段。”行者道:“趁早收了,免打!”那龙王化一阵旋风,到油锅边,将冷
龙捉下海去不题。

行者下来,与三藏、八戒、沙僧立在殿前,见那道士在滚油锅里打挣,爬不出
来。滑了一跌,霎时间骨脱皮焦肉烂。

监斩官又来奏道:“万岁,三国师化了也。”那国王满眼垂泪,手扑着御案,
放声大哭道:
“人身难得果然难,不遇真传莫炼丹。
空有驱神咒水术,却无延寿保生丸。
圆明混,怎涅?徒用心机命不安。
早觉这般轻折挫,何如秘食稳居山!”
这正是:
点金炼汞成何济,唤雨呼风总是空!

毕竟不知师徒们怎的维持,且听下回分解。