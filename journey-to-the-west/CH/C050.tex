\chapter{情乱性从因爱欲~神昏心动遇魔头}

词曰:

心地频频扫,尘情细细除,莫教坑堑陷毗卢。本体常清净,方可论元初。

性
烛须挑剔,曹溪任吸呼,勿令猿马气声粗。昼夜绵绵息,方显是功夫。
这一首词,牌名《南柯子》,单道着唐僧脱却通天河寒冰之灾,踏白鼋负登彼岸。
四众奔西,正遇严冬之景,但见那林光漠漠烟中淡,山骨棱棱水外清。师徒们正当
行处,忽然又遇一山,阻住去道。路窄崖高,石多岭峻,人马难行。三藏在马上兜
住缰绳,叫声“徒弟”。时有孙行者引八戒、沙僧近前侍立道:“师父,有何吩咐?”
三藏道:“你看那前面山高,只恐有虎狼作怪,妖兽伤人,今番是必仔细!”行者道:
“师父放心莫虑。我等兄弟三人,性和意合,归正求真,使出荡怪降妖之法,怕甚
么虎狼妖兽!”三藏闻言,只得放怀前进。到于谷口,促马登崖,抬头观看,好山:

嵯峨矗矗,峦削巍巍:嵯峨矗矗冲霄汉,峦削巍巍碍碧空。怪石乱堆如坐虎,
苍松斜挂似飞龙。岭上鸟啼娇韵美,崖前梅放异香浓。涧水潺流出冷,巅云黯淡
过来凶。又见那飘飘雪,凛凛风,咆哮饿虎吼山中。寒鸦拣树无栖处,野鹿寻窝没
定踪。可叹行人难进步,皱眉愁脸把头蒙。

师徒四众,冒雪冲寒,战澌澌,行过那巅峰峻岭,远望见山凹中有楼台高耸,
房舍清幽。唐僧马上欣然道:“徒弟啊,这一日又饥又寒,幸得那山凹里有楼台房
舍,断乎是庄户人家,庵观寺院;且去化些斋饭,吃了再走。”

行者闻言,急睁睛看,只见那壁厢凶云隐隐,恶气纷纷,回首对唐僧道:“师
父,那厢不是好处。”三藏道:“见有楼台亭宇,如何不是好处?”行者笑道:“师
父啊,你那里知道?西方路上多有妖怪邪魔,善能点化庄宅。不拘甚么楼台房舍,
馆阁亭宇,俱能指化了哄人。你知道‘龙生九种’,内有一种名‘蜃’,蜃气放出,
就如楼阁浅池。若遇大江昏迷,蜃现此势。倘有乌鹊飞腾,定来歇翅。那怕你上万
论千,尽被他一气吞之。此意害人最重。那壁厢气色凶恶,断不可入。”

三藏道:“既不可入,我却着实饥了。”行者道:“师父果饥,且请下马,就在
这平处坐下,待我别处化些斋来你吃。”三藏依言下马。八戒采定缰绳,沙僧放下
行李,即去解开包裹,取出钵盂,递与行者。行者接钵盂在手,吩咐沙僧道:“贤
弟,却不可前进。好生保护师父稳坐于此,待我化斋回来,再往西去。”沙僧领诺。
行者又向三藏道:“师父,这去处少吉多凶,切莫要动身别往。老孙化斋去也。”唐
僧道:“不必多言,但要你快去快来。我在这里等你。”行者转身欲行,却又回来道:
“师父,我知你没甚坐性,我与你个安身法儿。”即取金箍棒,幌了一幌,将那平
地下周围画了一道圈子,请唐僧坐在中间;着八戒、沙僧侍立左右,把马与行李都
放在近身。对唐僧合掌道:“老孙画的这圈,强似那铜墙铁壁。凭他甚么虎豹狼虫,
妖魔鬼怪,俱莫敢近。但只不许你们走出圈外,只在中间稳坐,保你无虞;但若出
了圈儿,定遭毒手。千万,千万!至嘱,至嘱!”三藏依言,师徒俱端然坐下。

行者才起云头,寻庄化斋,一直南行,忽见那古树参天,乃一村庄舍。按下云
头,仔细观看,但只见:

雪欺衰柳,冰结方塘。疏疏修竹摇青,郁郁乔松凝翠。几间茅屋半装银,一座
小桥斜砌粉。篱边微吐水仙花,檐下长垂冰冻箸。飒飒寒风送异香,雪漫不见梅开
处。
行者随步观看庄景,只听得呀的一声,柴扉响处,走出一个老者,手拖藜杖,头顶
羊裘,身穿破衲,足踏蒲鞋,拄着杖,仰身朝天道:“西北风起,明日晴了。”说不
了,后边跑出一个哈巴狗儿来,望着行者,汪汪的乱吠。老者却才转过头来,看见
行者捧着钵盂,打个问讯道:“老这施主,我和尚是东土大唐钦差上西天拜佛求经
者。适路过宝方,我师父腹中饥馁,特造尊府募化一斋。”老者闻言,点头顿杖道:
“长老,你且休化斋,你走错路了。”行者道:“不错。”老者道:“往西天大路,在
那直北下。此间到那里有千里之遥,还不去找大路而行?”行者笑道:“正是直北
下。我师父现在大路上端坐,等我化斋哩。”那老者道:“这和尚胡说了。你师父在
大路上等你化斋,似这千里之遥,就会走路,也须得六七日;走回去又要六七日,
却不饿坏他也?”行者笑道:“不瞒老施主说。我才然离了师父,还不上一盏热茶
之时,却就走到此处。如今化了斋,还要趁去作午斋哩。”老者见说,心中害怕道:
“这和尚是鬼,是鬼!”急抽身往里就走。行者一把扯住道:“施主那里去?有斋快
化些儿。”老者道:“不方便,不方便,别转一家儿罢!”行者道:“你这施主,好不
会事!你说我离此有千里之遥,若再转一家,却不又有千里?真是饿杀我师父也。”
那老者道:“实不瞒你说。我家老小六七口,才淘了三升米下锅,还未曾煮熟。你
且到别处去转转再来。”行者道:“古人云:‘走三家不如坐一家。’我贫僧在此等一
等罢。”那老者见缠得紧,恼了,举藜杖就打。行者公然不惧,被他照光头上打了
七八下,只当与他拂痒。那老者道:“这是个撞头的和尚!”行者笑道:“老官儿,
凭你怎么打,只要记得杖数明白。一杖一升米,慢慢量来。”那老者闻言,急丢了
藜杖,跑进去把门关了,只嚷:“有鬼,有鬼!”慌得那一家儿战战兢兢,把前后门
俱关上。行者见他关了门,心中暗想:“这老贼才说淘米下锅,不知是虚是实。常
言道:‘道化贤良释化愚。’且等老孙进去看看。”好大圣,捻着诀,使个隐身遁法,
径走入厨中看处,果然那锅里气腾腾的,煮了半锅干饭。就把钵盂往里一,满满
的了一钵盂,即驾云回转不题。

却说唐僧坐在圈子里,等待多时,不见行者回来,欠身怅望道:“这猴子往那
里化斋去了!”八戒在旁笑道:“知他往那里耍子去来!化甚么斋,却教我们在此坐
牢!”三藏道:“怎么谓之坐牢?”八戒道:“师父,你原来不知。古人划地为牢。
他将棍子划个圈儿,强似铁壁铜墙,假如有虎狼妖兽来时,如何挡得他住?只好白
白的送与他吃罢了。”三藏道:“悟能,凭你怎么处治。”八戒道:“此间又不藏风,
又不避冷,若依老猪,只该顺着路,往西且行。师兄化了斋,驾了云,必然来快,
让他赶来。如有斋,吃了再走。如今坐了这一会,老大脚冷!”三藏闻此言,就是
晦气星进宫:遂依呆子,一齐出了圈外。沙僧牵了马,八戒担了担,那长老顺路步
行前进。

不一时,到了那楼阁之所,原来是坐北向南之家。门外八字粉墙,有一座倒垂
莲升斗门楼,都是五色装的。那门儿半开半掩。八戒就把马拴在门枕石鼓上。沙僧
歇了担子。三藏畏风,坐于门限之上。八戒道:“师父,这所在想是公侯之宅,相
辅之家。前门外无人,想必都在里面烘火。你们坐着,让我进去看看。”唐僧道:“仔
细耶!莫要冲撞了人家。”呆子道:“我晓得。自从归正禅门,这一向也学了些礼数,
不比那村莽之夫也。”

那呆子把钉钯撒在腰里,整一整青锦直裰,斯斯文文,走入门里。只见是三间
大厅,帘栊高控,静悄悄全无人迹,也无桌椅家火。转过屏门,往里又走,乃是一
座穿堂。堂后有一座大楼,楼上窗格半开,隐隐见一顶黄绫帐幔。呆子道:“想是
有人怕冷,还睡哩。”他也不分内外,拽步走上楼来,用手掀开看时,把呆子唬了
一个踵踵。原来那帐里,象牙床上,白媸媸的一堆骸骨,骷髅有巴斗大,腿挺骨
有四五尺长。呆子定了性,止不住腮边泪落,对骷髅点头叹云:“你不知是:
那代那朝元帅体,何邦何国大将军。
当时豪杰争强胜,今日凄凉露骨筋。
不见妻儿来侍奉,那逢士卒把香焚?
谩观这等真堪叹,可惜兴王霸业人。”
八戒正才感叹,只见那帐幔后有火光一幌。呆子道:“想是有侍奉香火之人在后面
哩。”急转步过帐观看,却是穿楼的窗扇透光。那壁厢有一张彩漆的桌子,桌子上
乱搭着几件锦绣绵衣。呆子提起来看时,却是三件纳锦背心儿。

他也不管好歹,拿下楼来,出厅房,径到门外道:“师父,这里全没人烟,是
一所亡灵之宅。老猪走进里面,直至高楼之上,黄绫帐内,有一堆骸骨。串楼旁有
三件纳锦的背心,被我拿来了,也是我们一程儿造化。此时天气寒冷,正当用处。
师父,且脱了褊衫,把他且穿在底下,受用受用,免得吃冷。”三藏道:“不可,不
可!律云:‘公取窃取皆为盗。’倘或有人知觉,赶上我们,到了当官,断然是一个
窃盗之罪。还不送进去与他搭在原处!我们在此避风坐一坐,等悟空来时走路。出
家人不要这等爱小。”八戒道:“四顾无人,虽鸡犬亦不知之,但只我们知道,谁人
告我?有何证见?就如拾到的一般,那里论甚么公取窃取也!”三藏道:“你胡做啊!
虽是人不知之,天何盖焉!玄帝垂训云:‘暗室亏心,神目如电。’趁早送去还他,
莫爱非礼之物。”

那呆子莫想肯听,对唐僧笑道:“师父啊,我自为人,也穿了几件背心,不曾
见这等纳锦的。你不穿,且待老猪穿一穿,试试新,晤晤脊背。等师兄来,脱了还
他走路。”沙僧道:“既如此说,我也穿一件儿。”两个齐脱了上盖直裰,将背心套
上。才紧带子,不知怎么立站不稳,扑的一跌。原来这背心儿赛过绑缚手,霎时间,
把他两个背剪手贴心捆了。慌得个三藏跌足报怨,急忙上前来解,那里便解得开?
三个人在那里喝之声不绝,却早惊动了魔头也。

话说那座楼房果是妖精点化的,终日在此拿人。他在洞里正坐,忽闻得怨恨之
声,急出门来看,果见捆住几个人了。妖魔即唤小妖,同到那厢,收了楼台房屋之
形,把唐僧搀住,牵了白马,挑了行李,将八戒、沙僧一齐捉到洞里。老妖魔登台
高坐,众小妖把唐僧推近台边,跪伏于地。妖魔问道:“你是那方和尚?怎么这般胆
大,白日里偷盗我的衣服?”三藏滴泪告曰:“贫僧是东土大唐钦差往西天取经的。
因腹中饥馁,着大徒弟去化斋未回,不曾依得他的言语,误撞仙庭避风。不期我这
两个徒弟爱小,拿出这衣物。贫僧决不敢坏心,当教送还本处。他不听吾言,要穿
此晤晤脊背,不料中了大王机会,把贫僧拿来。万望慈悯,留我残生,求取真经,
永注大王恩情,回东土千古传扬也!”

那妖魔笑道:“我这里常听得人言:有人吃了唐僧一块肉,发白还黑,齿落更
生。幸今日不请自来,还指望饶你哩!你那大徒弟叫做甚么名字?往何方化斋?”八
戒闻言,即开口称扬道:“我师兄乃五百年前大闹天宫齐天大圣孙悟空也。”

那妖魔听说是齐天大圣孙悟空,老大有些悚惧,口内不言,心中暗想道:“久
闻那厮神通广大,如今不期而会。”教:“小的们,把唐僧捆了;将那两个解下宝贝,
换两条绳子,也捆了。且抬在后边,待我拿住他大徒弟,一发刷洗,却好凑笼蒸吃。”
众小妖答应一声,把三人一齐捆了,抬在后边。将白马拴在槽头,行李挑在屋里。
众妖都磨兵器,准备擒拿行者不题。

却说孙行者自南庄人家摄了一钵盂斋饭,驾云回返旧路;径至山坡平处,按下
云头,早已不见唐僧,不知何往。棍划的圈子还在,只是人马都不见了。回看那楼
台处所,亦俱无矣,惟见山根怪石。行者心惊道:“不消说了!他们定是遭那毒手也!”
急依路看着马蹄,向西而赶。

行有五六里,正在凄怆之际,只闻得北坡外有人言语。看时,乃一个老翁,毡
衣苫体,暖帽蒙头,足下踏一双半新半旧的油靴,手持着一根龙头拐棒,后边跟一
个年幼的僮仆,折一枝腊梅花,自坡前念歌而走。

行者放下钵盂,觌面道个问讯,叫:“老公公,贫僧问讯了。”那老翁即便回礼
道:“长老那里来的?”行者道:“我们东土来的,往西天拜佛求经。一行师徒四众。
我因师父饥了,特去化斋,教他三众坐在那山坡平处相候。及回来不见,不知往那
条路上去了?动问公公,可曾看见?”老者闻言,呵呵冷笑道:“你那三众,可有一
个长嘴大耳的么?”行者道:“有,有,有!”又有一个晦气色脸的,牵着一匹白马,
领着一个白脸的胖和尚么?”行者道:“是!是,是!”老翁道:“你们走错路了。你
休寻他,各人顾命去也。”行者道:“那白脸者是我师父,那怪样者是我师弟。我与
他共发虔心,要往西天取经,如何不寻他去!”老翁道:“我才然从此过时,看见他
错走了路径,闯入妖魔口里去了。”行者道:“烦公公指教指教,是个甚么妖魔,居
于何方,我好上门取索他等,往西天去也。”老翁道:“这座山,叫做金山。山前
有个金洞。那洞中有个独角兕大王。那大王神通广大,威武高强。那三众此回断
没命了。你若去寻,只怕连你也难保,不如不去之为愈也。我也不敢阻你,也不敢
留你,只凭你心中度量。”

行者再拜称谢道:“多蒙公公指教。我岂有不寻之理!”把这斋饭倒与他,将这
空钵盂自家收拾。那老翁放下拐棒,接了钵盂,递与僮仆,现出本象,双双跪下,
叩头叫:“大圣,小神不敢隐瞒。我们两个就是此山山神、土地,在此候接大圣。
这斋饭连钵盂,小神收下,让大圣身轻好施法力。待救唐僧出难,将此斋还奉唐僧,
方显得大圣至恭至孝。”行者喝道:“你这毛鬼讨打!既知我到,何不早迎?却又这般
藏头露尾,是甚道理?”土地道:“大圣性急,小神不敢造次,恐犯威颜,故此隐
象告知。”行者息怒道:“你且记打!好生与我收着钵盂,待我拿那妖精去来!”土地、
山神遵领。

这大圣却才束一束虎筋绦,拽起虎皮裙,执着金箍棒,径奔山前,找寻妖洞。
转过山崖,只见那乱石磷磷,翠崖边有两扇石门,门外有许多小妖,在那里轮枪舞
剑。真个是:

烟云凝瑞,苔藓堆青。怪石列,崎岖曲道萦。猿啸鸟啼风景丽,鸾飞凤舞
若蓬瀛。向阳几树梅初放,弄暖千竿竹自青。陡崖之下,深涧之中;陡崖之下雪堆
粉,深涧之中水结冰。两林松柏千年秀,几簇山茶一样红。
这大圣观看不尽,拽开步径至门前,厉声高叫道:“那小妖,你快进去与你那洞主
说,我本是唐朝圣僧徒弟齐天大圣孙悟空。快教他送我师父出来,免教你等丧了性
命!”

那伙小妖,急入洞里报道:“大王,前面有一个毛脸勾嘴的和尚。称是齐天大
圣孙悟空,来要他师父哩。”那魔王闻得此言,满心欢喜道:“正要他来哩!我自离
了本宫,下降尘世,更不曾试试武艺。今日他来,必是个对手。”即命:“小的们取
出兵器。”那洞中大小群魔,一个个精神抖擞,即忙抬出一根丈二长的点钢枪,递
与老怪。老怪传令,教:“小的们,各要整齐。进前者赏,退后者诛!”众妖得令,
随着老怪,腾出门来。叫道:“那个是孙悟空?”行者在旁闪过,见那魔王生得好
不凶丑:

独角参差,双眸幌亮。顶上粗皮突,耳根黑肉光。舌长时搅鼻,口阔版牙黄。
毛皮青似靛,筋挛硬如钢。比犀难照水,象牯不耕荒。全无喘月犁云用,倒有欺天
振地强。两只焦筋蓝靛手,雄威直挺点钢枪。细看这等凶模样,不枉名称兕大王!
孙大圣上前道:“你孙外公在这里也!快早还我师父,两无毁伤!若道半个‘不’字,
我教你死无葬身之地!”那魔喝道:“我把你这个大胆泼猴精!你有些甚么手段,敢
出这般大言!”行者道:“你这泼物,是也不曾见我老孙的手段!”那妖魔道:“你师
父偷盗我的衣服,实是我拿住了,如今待要蒸吃。你是个甚么好汉,就敢上我的门
来取讨!”行者道:“我师父乃忠良正直之僧,岂有偷你甚么妖物之理?”妖魔道:
“我在山路边点化一座仙庄,你师父潜入里面,心爱情欲,将我三领纳锦绵装背心
儿偷穿在身,见有赃证,故此我才拿他。你今果有手段,即与我比势。假若三合敌
得我,饶了你师之命;如敌不过我,教你一路归阴!”

行者笑道:“泼物,不须讲口!但说比势,正合老孙之意。走上来,吃吾之棒!”
那怪物那怕甚么赌斗,挺钢枪劈面迎来。这一场好杀!你看那:

金箍棒举,长杆枪迎:金箍棒举,亮藿藿似电掣金蛇;长杆枪迎,明幌幌如龙
离黑海。那门前小妖擂鼓,排开阵势助威风;这壁厢大圣施功,使出纵横逞本事。
他那里一杆枪,精神抖擞;我这里一条棒,武艺高强。正是英雄相遇英雄汉,果然
对手才逢对手人。那魔王口喷紫气盘烟雾,这大圣眼放光华结绣云。只为大唐僧有
难,两家无义苦争抡。
他两个战经三十合,不分胜负。那魔王见孙悟空棍法齐整,一往一来,全无些破绽,
喜得他连声喝采道:“好猴儿,好猴儿!真个是那闹天宫的本事!”这大圣也爱他枪
法不乱,右遮左挡,甚有解数,也叫道:“好妖精,好妖精!果然是一个偷丹的魔头!”
二人又斗了一二十合。

那魔王把枪尖点地,喝令小妖齐来。那些泼怪,一个个拿刀弄杖,执剑轮枪,
把个孙大圣围在中间。行者公然不惧,只叫:“来得好,来得好!正合吾意!”使一
条金箍棒,前迎后架,东挡西除。那伙群妖,莫想肯退。行者忍不住焦躁,把金箍
棒丢将起去,喝声“变!”即变作千百条铁棒,好便似飞蛇走蟒,盈空里乱落下来。
那伙妖精见了,一个个魄散魂飞,抱头缩颈,尽往洞中逃命。老魔王唏唏冷笑道:
“那猴不要无礼,看手段!”即忙袖中取出一个亮灼灼白森森的圈子来,望空抛起,
叫声“着!”唿喇一下,把金箍棒收做一条,套将去了。弄得孙大圣赤手空拳,翻
筋斗逃了性命。那妖魔得胜回归洞,行者朦胧失主张。这正是:
道高一尺魔高丈,性乱情昏错认家。
可恨法身无坐位,当时行动念头差。

毕竟不知这番怎么结果,且听下回分解。