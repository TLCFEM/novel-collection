\chapter{官封弼马心何足~名注齐天意未宁}

那太白金星与美猴王,同出了洞天深处,一齐驾云而起。原来悟空筋斗云比众
不同,十分快疾,把个金星撇在脑后,先至南天门外。正欲收云前进,被增长天王
领着庞、刘、苟、毕、邓、辛、张、陶,一路大力天丁,枪刀剑戟,挡住天门,不
肯放进。猴王道:“这个金星老儿,乃奸诈之徒!既请老孙,如何教人动刀动枪,阻
塞门路?”正嚷间,金星倏到。悟空就觌面发狠道:“你这老儿,怎么哄我?被你说
奉玉帝招安旨意来请,却怎么教这些人阻住天门,不放老孙进去?”金星笑道:“大
王息怒。你自来未曾到此天堂,却又无名,众天丁又与你素不相识,他怎肯放你擅
入?等如今见了天尊,授了仙,注了官名,向后随你出入,谁复挡也?”悟空道:
“这等说,也罢,我不进去了。”金星又用手扯住道:“你还同我进去。”

将近天门,金星高叫道:“那天门天将,大小吏兵,放开路者。此乃下界仙人,
我奉玉帝圣旨,宣他来也。”那增长天王与众天丁俱才敛兵退避。猴王始信其言。
同金星缓步入里观看。真个是:

初登上界,乍入天堂。金光万道滚红霓,瑞气千条喷紫雾。只见那南天门,
碧沉沉,琉璃造就;明幌幌,宝玉妆成。两边摆数十员镇天元帅,一员员顶梁靠柱,
持铣拥旄;四下列十数个金甲神人,一个个执戟悬鞭,持刀仗剑。外厢犹可,入内
惊人。里壁厢有几根大柱,柱上缠绕着金鳞耀日赤须龙;又有几座长桥,桥上盘旋
着彩羽凌空丹顶凤。明霞幌幌映天光,碧
雾蒙蒙遮斗口。这天上有三十三座天宫,乃遣云宫、毗沙宫、五明宫、太阳宫、化
乐宫,……一宫宫脊吞金稳兽;又有七十二重宝殿,乃朝会殿、凌虚殿、宝光殿、
天王殿、灵官殿,……一殿殿柱列玉麒麟。寿星台上,有千千年不卸的名花;炼药
炉边,有万万载常青的瑞草。又至那朝圣楼前,绛纱衣,星辰灿烂;芙蓉冠,金璧
辉煌。玉簪珠履,紫绶金章。金钟撞动,三曹神表进丹墀;天鼓鸣时,万圣朝王参
玉帝。又至那灵霄宝殿,金钉攒玉户,彩凤舞朱门。复道回廊,处处玲珑剔透;三
檐四簇,层层龙凤翱翔。上面有个紫巍巍,明幌幌,圆丢丢,亮灼灼,大金葫芦顶;
下面有天妃悬掌扇,玉女捧仙巾。恶狠狠,掌朝的天将;气昂昂,护驾的仙卿。正
中间,琉璃盘内,放许多重重叠叠太乙丹;玛瑙瓶中,插几枝弯弯曲曲珊瑚树。正
是天宫异物般般有,世上如他件件无。金阙银銮并紫府,琪花瑶草暨琼葩。朝王玉
兔坛边过,参圣金乌着底飞。猴王有分来天境,不堕人间点污泥。

太白金星,领着美猴王,到于灵霄殿外。不等宣诏,直至御前,朝上礼拜。悟
空挺身在旁,且不朝礼,但侧耳以听金星启奏。金星奏道:“臣领圣旨,已宣妖仙
到了。”玉帝垂帘问曰:“那个是妖仙?”悟空却才躬身答应道:“老孙便是。”仙卿
们都大惊失色道:“这个野猴!怎么不拜伏参见,辄敢这等答应道:‘老孙便是’,却
该死了,该死了!”玉帝传旨道:“那孙悟空乃下界妖仙,初得人身,不知朝礼,且
姑恕罪。”众仙卿叫声“谢恩!”猴王却才朝上唱个大喏。玉帝宣文选武选仙卿,看
那处少甚官职,着孙悟空去除授。旁边转过武曲星君,启奏道:“天宫里各宫各殿,
各方各处,都不少官,只是御马监缺个正堂管事。”玉帝传旨道:“就除他做个‘弼
马温’罢。”众臣叫谢恩,他也只朝上唱个大喏。玉帝又差木德星官送他去御马监
到任。

当时猴王欢欢喜喜,与木德星官径去到任。事毕,木德回宫。他在监里,会聚
了监丞、监副、典簿、力士、大小官员人等,查明本监事务,止有天马千匹。乃是:

骅骝骐骥,纤离;龙媒紫燕,挟翼;银,飞黄;翻羽,
赤兔超光;逾辉弥景,腾雾胜黄;追风绝地,飞奔霄;逸飘赤电,铜爵浮云;骢
珑虎,绝尘紫鳞;四极大宛,八骏九逸,千里绝群:此等良马,一个个,嘶风逐
电精神壮,踏雾登云气力长。
这猴王查看了文簿,点明了马数。本监中典簿管征备草料;力士官管刷洗马匹、扎
草、饮水、煮料;监丞、监副辅佐催办;弼马昼夜不睡,滋养马匹。日间舞弄犹可,
夜间看管殷勤:但是马睡的,赶起来吃草;走的,捉将来靠槽。那些天马见了他,
泯耳攒蹄,都养得肉肥膘满。

不觉的半月有余。一朝闲暇,众监官都安排酒席,一则与他接风,一则与他贺
喜。正在欢饮之间,猴王忽停杯问曰:“我这‘弼马温’是个甚么官衔?”众曰:“官
名就是此了。”又问:“此官是个几品?”众道:“没有品从。”猴王道:“没品,想
是大之极也。”众道:“不大,不大,只唤做‘未入流’。”猴王道:“怎么叫做‘未
入流’?”众道:“末等。这样官儿,最低最小,只可与他看马。似堂尊到任之后,
这等殷勤,喂得马肥,只落得道声‘好’字;如稍有些羸,还要见责;再十分伤
损,还要罚赎问罪。”猴王闻此,不觉心头火起,咬牙大怒道:“这般藐视老孙!老
孙在那花果山,称王称祖,怎么哄我来替他养马?养马者,乃后生小辈,下贱之役,
岂是待我的?不做他,不做他,我将去也!”忽喇的一声,把公案推倒,耳中取出宝
贝,幌一幌,碗来粗细,一路解数,直打出御马监,径至南天门。众天丁知他受了
仙,乃是个弼马温,不敢阻当,让他打出天门去了。

须臾,按落云头,回至花果山上。只见那四健将与各洞妖王,在那里操演兵卒。
这猴王厉声高叫道:“小的们!老孙来了!”一群猴都来叩头,迎接进洞天深处,请
猴王高登宝位,一壁厢办酒接风。都道:“恭喜大王,上界去十数年,想必得意荣
归也?”猴王道:“我才半月有余,那里有十数年?”众猴道:“大王,你在天上,
不觉时辰。天上一日,就是下界一年哩。请问大王,官居何职?”猴王摇手道:“不
好说,不好说,活活的羞杀人!那玉帝不会用人,他见老孙这般模样,封我做个甚
么‘弼马温’,原来是与他养马,未入流品之类。我初到任时不知,只在御马监中
顽耍。及今日问我同寮,始知是这等卑贱。老孙心中大恼,推倒席面,不受官衔,
因此走下来了。”众猴道:“来得好,来得好!大王在这福地洞天之处为王,多少尊
重快乐,怎么肯去与他做马夫?教小的们快办酒来,与大王释闷!”

正饮酒欢会间,有人来报道:“大王,门外有两个独角鬼王,要见大王。”猴王
道:“教他进来。”那鬼王整衣跑入洞中,倒身下拜。美猴王问他:“你见我何干?”
鬼王道:“久闻大王招贤,无由得见;今见大王授了天,得意荣归,特献赭黄袍
一件,与大王称庆。肯不弃鄙贱,收纳小人,亦得效犬马之劳。”猴王大喜,将赭
黄袍穿起,众等欣然排班朝拜,即将鬼王封为前部总督先锋。鬼王谢恩毕,复启道:
“大王在天许久,所授何职?”猴王道:“玉帝轻贤,封我做个甚么‘弼马温’!”
鬼王听言,又奏道:“大王有此神通,如何与他养马?就做个‘齐天大圣’,有何不
可?”猴王闻说,欢喜不胜,连道几个“好,好,好!”教四健将:“就替我快置个
旌旗,旗上写‘齐天大圣’四大字,立竿张挂。自此以后,只称我为齐天大圣,不
许再称大王。亦可传与各洞妖王,一体知悉。”此不在话下。

却说那玉帝次日设朝,只见张天师引御马监监丞、监副在丹墀下拜奏道:“万
岁,新任弼马温孙悟空,因嫌官小,昨日反下天宫去了。”正说间,又见南天门外
增长天王领众天丁,亦奏道:“弼马温不知何故,走出天门去了。”玉帝闻言,即传
旨:“着两路神元,各归本职,朕遣天兵,擒拿此怪。”班部中闪上托塔李天王与哪
吒三太子,越班奏上道:“万岁,微臣不才,请旨降此妖怪。”玉帝大喜,即封托塔
天王李靖为降魔大元帅,哪吒三太子为三坛海会大神,即刻兴师下界。

李天王与哪吒叩头谢辞,径至本宫,点起三军,帅众头目,着巨灵神为先锋,
鱼肚将掠后,药叉将催兵。一霎时出南天门外,径来到花果山。选平阳处安了营寨,
传令教巨灵神挑战。巨灵神得令,结束整齐,轮着宣花斧,到了水帘洞外。只见那
洞门外,许多妖魔,都是些狼虫虎豹之类,丫丫叉叉,轮枪舞剑,在那里跳斗咆哮。
这巨灵神喝道:“那业畜!快早去报与弼马温知道,吾乃上天大将,奉玉帝旨意,到
此收伏;教他早早出来受降,免致汝等皆伤残也。”

那些怪,奔奔波波,传报洞中道:“祸事了!祸事了!”猴王问:“有甚祸事?”
众妖道:“门外有一员天将,口称大圣官衔,道:奉玉帝圣旨,来此收伏;教早早
出去受降,免伤我等性命。”猴王听说,教:“取我披挂来!”就戴上紫金冠,贯上
黄金甲,登上步云鞋,手执如意金箍棒,领众出门,摆开阵势。这巨灵神睁睛观看,
真好猴王:

身穿金甲亮堂堂,头戴金冠光映映。手举金箍棒一根,足踏云鞋皆相称。一双
怪眼似明星,两耳过肩查又硬。挺挺身才变化多,声音响亮如钟磬。尖嘴咨牙弼马
温,心高要做齐天圣。

巨灵神厉声高叫道:“那泼猴!你认得我么?”大圣听言,急问道:“你是那路
毛神?老孙不曾会你,你快报名来。”巨灵神道:“我把你那欺心的猢狲!你是认不得
我!我乃高上神霄托塔李天王部下先锋巨灵天将!今奉玉帝圣旨,到此收降你。你快
卸了装束,归顺天恩,免得这满山诸畜遭诛;若道半个‘不’字,教你顷刻化为齑
粉!”

猴王听说,心中大怒道:“泼毛神,休夸大口,少弄长舌!我本待一棒打死你,
恐无人去报信;且留你性命,快早回天,对玉皇说:他甚不用贤,老孙有无穷的本
事,为何教我替他养马?你看我这旌旗上字号。若依此字号升官,我就不动刀兵,
自然的天地清泰;如若不依,时间就打上灵霄宝殿,教他龙床定坐不成!”

这巨灵神闻此言,忽睁睛迎风观看,果见门外竖一高竿,竿上有旌旗一面,上
写着“齐天大圣”四大字。巨灵神冷笑三声道:“这泼猴,这等不知人事,辄敢无
状,你就要做齐天大圣!好好的吃吾一斧!”劈头就砍将去。那猴王正是会家不忙,
将金箍棒应手相迎。这一场好杀:

棒名如意,斧号宣花。他两个乍相逢,不知深浅;斧和棒,左右交加。一个暗
藏神妙,一个大口称夸。使动法,喷云嗳雾;展开手,播土扬沙。天将神通就有道,
猴王变化实无涯。棒举却如龙戏水,斧来犹似凤穿花。巨灵名望传天下,原来本事
不如他;大圣轻轻轮铁棒,着头一下满身麻。
巨灵神抵敌他不住,被猴王劈头一棒,慌忙将斧架隔,的一声,把个斧柄打做
两截,急撤身败阵逃生。猴王笑道:“脓包,脓包!我已饶了你,你快去报信!快去
报信!”

巨灵神回至营门,径见托塔天王,忙哈哈跪下道:“弼马温果是神通广大!末将
战他不得,败阵回来请罪。”李天王发怒道:“这厮锉吾锐气,推出斩之!”旁边闪
出哪吒太子,拜告:“父王息怒,且恕巨灵之罪,待孩儿出师一遭,便知深浅。”天
王听谏,且教回营待罪管事。

这哪吒太子,甲胄齐整,跳出营盘,撞至水帘洞外。那悟空正来收兵,见哪吒
来的勇猛。好太子:

总角才遮囟,披毛未苫肩。神奇多敏悟,骨秀更清妍。诚为天上麒麟子,果是
烟霞彩凤仙。龙种自然非俗相,妙龄端不
类尘凡。身带六般神器械,飞腾变化广无边。今受玉皇金口诏,敕封海会号三坛。
悟空迎近前来问曰:“你是谁家小哥?闯近吾门,有何事干?”哪吒喝道:“泼妖猴!
岂不认得我?我乃托塔天王三太子哪吒是也。今奉玉帝钦差,至此捉你。”悟空笑道:
“小太子,你的奶牙尚未退,胎毛尚未干,怎敢说这般大话?我且留你的性命,不
打你。你只看我旌旗上是甚么字号,拜上玉帝:是这般官衔,再也不须动众,我自
皈依;若是不遂我心,定要打上灵霄宝殿。”哪吒抬头看处,乃“齐天大圣”四字。
哪吒道:“这妖猴能有多大神通,就敢称此名号!不要怕!吃吾一剑!”悟空道:“我
只站下不动,任你砍几剑罢。”那哪吒奋怒,大喝一声,叫“变!”即变做三头六臂,
恶狠狠,手持着六般兵器,乃是斩妖剑、砍妖刀、缚妖索、降妖杵、绣球儿、火轮
儿,丫丫叉叉,扑面来打。悟空见了,心惊道:“这小哥倒也会弄些手段!莫无礼,
看我神通!”好大圣,喝声“变”也变做三头六臂;把金箍棒幌一幌,也变作三条;
六只手拿着三条棒架住。这场斗,真个是地动山摇,好杀也:

六臂哪吒太子,天生美石猴王,相逢真对手,正遇本源流。那一个蒙差来
下界,这一个欺心闹斗牛。斩妖宝剑锋芒快,砍妖刀狠鬼神愁;缚妖索子如飞蟒,
降妖大杵似狼头;火轮掣电烘烘艳,往往来来滚绣球。大圣三条如意棒,前遮后挡
运机谋。苦争数合无高下,太子心中不肯休。把那六件兵器多数变,百千万亿照头
丢。猴王不惧呵呵笑,铁棒翻腾自运筹。以一化千千化万,满空乱舞赛飞虬。唬得
各洞妖王都闭户,遍山鬼怪尽藏头。神兵怒气云惨惨,金箍铁棒响飕飕。那壁厢,
天丁呐喊人人怕;这壁厢,猴怪摇旗个个忧。发狠两家齐斗勇,不知那个刚强那个
柔。
三太子与悟空各骋神威,斗了个三十回合。那太子六般兵,变做千千万万;孙悟空
金箍棒,变作万万千千。半空中似雨点流星,不分胜负。原来悟空手疾眼快,正在
那混乱之时,他拔下一根毫毛,叫声“变!”就变做他的本相,手挺着棒,演着哪
吒;他的真身,却一纵,赶至哪吒脑后,着左膊上一棒打来。哪吒正使法间,听得
棒头风响,急躲闪时,不能措手,被他着了一下,负痛逃走;收了法,把六件兵器,
依旧归身,败阵而回。

那阵上李天王早已看见,急欲提兵助战。不觉太子倏至面前,战兢兢报道:“父
王!弼马温真个有本事!孩儿这般法力,也战他不过,已被他打伤膊也。”天王大惊
失色道:“这厮恁的神通,如何取胜?”太子道:“他洞门外竖一竿旗,上写‘齐天
大圣’四字,亲口夸称,教玉帝就封他做齐天大圣,万事俱休;若还不是此号,定
要打上灵霄宝殿哩!”天王道:“既然如此,且不要与他相持,且去上界,将此言回
奏,再多遣天兵,围捉这厮,未为迟也。”太子负痛,不能复战,故同天王回天启
奏不题。

你看那猴王得胜归山,那七十二洞妖王与那六弟兄,俱来贺喜。在洞天福地,
饮乐无比。他却对六弟兄说:“小弟既称齐天大圣,你们亦可以大圣称之。”内有牛
魔王忽然高叫道:“贤弟言之有理,我即称做个平天大圣。”蛟魔王道:“我称做复
海大圣。”鹏魔王道:“我称混天大圣。”狮王道:“我称移山大圣。”猕猴王道:“我
称通风大圣。”狨王道:“我称驱神大圣。”此时七大圣自作自为,自称自号,耍
乐一日,各散讫。

却说那李天王与三太子领着众将,直至灵霄宝殿。启奏道:“臣等奉圣旨出师
下界,收伏妖仙孙悟空,不期他神通广大,不能取胜,仍望万岁添兵剿除。”玉帝
道:“谅一妖猴,有多少本事,还要添兵?”太子又近前奏道:“望万岁赦臣死罪!
那妖猴使一条铁棒,先败了巨灵神,又打伤臣臂膊。洞门外立一竿旗,上书‘齐天
大圣’四字,道是封他这官职,即便休兵来投;若不是此官,还要打上灵霄宝殿也。”
玉帝闻言,惊讶道:“这妖猴何敢这般狂妄!着众将即刻诛之。”

正说间,班部中又闪出太白金星,奏道:“那妖猴只知出言,不知大小。欲加
兵与他争斗,想一时不能收伏,反又劳师。不若万岁大舍恩慈,还降招安旨意,就
教他做个齐天大圣。只是加他个空衔,有官无禄便了。”玉帝道:“怎么唤做‘有官
无禄’?”金星道:“名是齐天大圣,只不与他事管,不与他俸禄,且养在天壤之
间,收他的邪心,使不生狂妄,庶乾坤安靖,海宇得清宁也。”玉帝闻言道:“依卿
所奏。”即命降了诏书,仍着金星领去。

金星复出南天门,直至花果山水帘洞外观看。这番比前不同,威风凛凛,杀气
森森,各样妖精,无般不有。一个个都执剑拈枪,拿刀弄杖的,在那里咆哮跳跃。
一见金星,皆上前动手。金星道:“那众头目来!累你去报你大圣知之。吾乃上帝遣
来天使,有圣旨在此请他。”

众妖即跑入报道:“外面有一老者,他说是上界天使,有旨意请你。”悟空道:
“来得好,来得好!想是前番来的那太白金星。那次请我上界,虽是官爵不堪,却
也天上走了一次,认得那天门内外之路。今番又来,定有好意。”教众头目大开旗
鼓,摆队迎接。大圣即带引群猴,顶冠贯甲,甲上罩了赭黄袍,足踏云履,急出洞
门,躬身施礼,高叫道:“老星请进,恕我失迎之罪!”

金星趋步向前,径入洞内,面南立着道:“今告大圣,前者因大圣嫌恶官小,
躲离御马监,当有本监中大小官员奏了玉帝。玉帝传旨道:‘凡授官职,皆由卑而
尊,为何嫌小?’即有李天王领哪吒下界取战。不知大圣神通,故遭败北,回天奏
道:‘大圣立一竿旗,要做“齐天大圣”。’众武将还要支吾,是老汉力为大圣冒罪
奏闻,免兴师旅,请大王授。玉帝准奏,因此来请。”悟空笑道:“前番动劳,今
又蒙爱,多谢,多谢!但不知上天可有此‘齐天大圣’之官衔也?”金星道:“老汉
以此衔奏准,方敢领旨而来;如有不遂,只坐罪老汉便是。”

悟空大喜,恳留饮宴不肯,遂与金星纵着祥云,到南天门外。那些天丁天将,
都拱手相迎。径入灵霄殿下。金星拜奏道:“臣奉诏宣弼马温孙悟空已到。”玉帝道:
“那孙悟空过来。今宣你做个‘齐天大圣’,官品极矣,但切不可胡为。”这猴亦止
朝上唱个喏,道声谢恩。玉帝即命工干官张、鲁二班在蟠桃园右首,起一座齐天大
圣府,府内设个二司:一名安静司,一名宁神司。司俱有仙吏,左右扶持。又差五
斗星君送悟空去到任,外赐御酒二瓶,金花十朵,着他安心定志,再勿胡为。那猴
王信受奉行,即日与五斗星君到府,打开酒瓶,同众尽饮。送星官回转本宫,他才
遂心满意,喜地欢天,在于天宫快乐,无挂无碍。正是:
仙名永注长生,不堕轮回万古传。

毕竟不知向后如何,且听下回分解。