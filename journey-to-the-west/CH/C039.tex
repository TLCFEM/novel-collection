\chapter{一粒金丹天上得~三年故主世间生}

话说那孙大圣头痛难禁,哀告道:“师父,莫念!莫念!等我医罢!”长老问:“怎
么医?”行者道:“只除过阴司,查勘那个阎王家有他魂灵,请将来救他。”八戒道:
“师父莫信他。他原说不用过阴司,阳世间就能医活,方见手段哩。”那长老信邪
风,又念紧箍儿咒,慌得行者满口招承道:“阳世间医罢!阳世间医罢!”八戒道:“莫
要住!只管念,只管念!”行者骂道:“你这呆孽畜,撺道师父咒我哩!”八戒笑得打
跌道:“哥耶,哥耶,你只晓得捉弄我,不晓得我也捉弄你捉弄!”行者道:“师父,
莫念!莫念!待老孙阳世间医罢。”三藏道:“阳世间怎么医?”行者道:“我如今一
筋斗云,撞入南天门里,不进斗牛宫,不入灵霄殿,径到那三十三天之上,离恨天
宫兜率院内,见太上老君,把他‘九转还魂丹’求得一粒来,管取救活他也。”

三藏闻言,大喜道:“就去快来。”行者道:“如今有三更时候罢了,投到回来,
好天明了。只是这个人睡在这里,冷淡冷淡,不像个模样;须得举哀人看着他哭,
便才好哩。”八戒道:“不消讲,这猴子一定是要我哭哩。”行者道:“怕你不哭!你
若不哭,我也医不成!”八戒道:“哥哥,你自去,我自哭罢了。”行者道:“哭有几
样:若干着口喊,谓之嚎;扭搜出些眼泪儿来,谓之啕。又要哭得有眼泪,又要哭
得有心肠,才算着嚎啕痛哭哩。”

八戒道:“我且哭个样子你看看。”他不知那里扯个纸条,拈作一个纸拈儿,往
鼻孔里通了两通,打了几个涕喷,你看他眼泪汪汪,粘涎答答的,哭将起来。口里
不住的絮絮叨叨,数黄道黑,真个像死了人的一般。哭到那伤情之处,唐长老也泪
滴心酸。行者笑道:“正是那样哀痛,再不许住声。你这呆子哄得我去了,你就不
哭。我还听哩!若是这等哭便罢;若略住住声儿,定打二十个孤拐!”八戒笑道:“你
去,你去!我这一哭动头,有两日哭哩。”沙僧见他数落,便去寻几枝香来烧献。行
者笑道:“好,好,好!一家儿都有些敬意,老孙才好用功。”

好大圣,此时有半夜时分,别了他师徒三众,纵筋斗云,只入南天门里。果然
也不谒灵霄宝殿,不上那斗牛天宫,一路云光,径来到三十三天离恨天兜率宫中。
才入门,只见那太上老君正坐在那丹房中,与众仙童执芭蕉扇火炼丹哩。他见行
者来时,即吩咐看丹的童儿:“各要仔细。偷丹的贼又来也!”

行者作礼笑道:“老官儿,这等没搭撒。防备我怎的?我如今不干那样事了。”
老君道:“你那猴子,五百年前大闹天宫,把我灵丹偷吃无数,着小圣二郎捉拿上
界,送在我丹炉炼了四十九日,炭也不知费了多少。你如今幸得脱身,皈依佛果,
保唐僧往西天取经,前者在平顶山上降魔,弄刁难,不与我宝贝,今日又来做甚?”
行者道:“前日事,老孙更没稽迟,将你那五件宝贝当时交还,你反疑心怪我?”

老君道:“你不走路,潜入吾宫怎的?”行者道:“自别后,西过一方,名乌鸡
国。那国王被一妖精假妆道士,呼风唤雨,阴害了国王,那妖假变国王相貌,现坐
金銮殿上。是我师父夜坐宝林寺看经,那国王鬼魂参拜我师,敦请老孙与他降妖,
辨明邪正。正是老孙思无指实,与弟八戒夜入园中,打破花园,寻着埋藏之所,乃
是一眼八角琉璃井内。捞上他的尸首,容颜不改。到寺中见了我师,他发慈悲,着
老孙医救,不许去赴阴司里求索灵魂,只教在阳世间救治。我想着无处回生,特来
参谒。万望道祖垂怜,把‘九转还魂丹’借得一千丸儿,与我老孙,搭救他也。”
老君道:“这猴子胡说!甚么一千丸,二千丸,当饭吃哩!是那里土块的,这等容
易?咄!快去,没有!”行者笑道:“百十丸儿也罢。”老君道:“也没有。”行者道:“十
来丸也罢。”老君怒道:“这泼猴却也缠帐!没有,没有!出去,出去!”行者笑道:“真
个没有,我问别处去救罢。”老君喝道:“去,去,去!”这大圣拽转步,往前就走。

老君忽的寻思道:“这猴子惫懒哩,说去就去,只怕溜进来就偷。”即命仙童叫
回来道:“你这猴子,手脚不稳,我把这‘还魂丹’送你一丸罢。”行者道:“老官
儿,既然晓得老孙的手段,快把金丹拿出来,与我四六分分,还是你的造化哩;不
然,就送你个‘皮笊篱,——一捞个罄尽’。”那老祖取过葫芦来,倒吊过底子,倾
出一粒金丹,递与行者道:“止有此了。拿去,拿去!送你这一粒,医活那皇帝,只
算你的功果罢。”行者接了道:“且休忙,等我尝尝看。只怕是假的,莫被他哄了。”
扑的往口里一丢,慌得那老祖上前扯住,一把揪着顶瓜皮,着拳头,骂道:“这
泼猴若要咽下去,就直打杀了!”行者笑道:“嘴脸,小家子样,那个吃你的哩,能
值几个钱!虚多实少的。在这里不是?”原来那猴子颏下有嗉袋儿。他把那金丹噙
在嗉袋里,被老祖捻着道:“去罢,去罢!再休来此缠绕!”这大圣才谢了老祖,出
离了兜率天宫。

你看他千条瑞霭离瑶阙,万道祥云降世尘。须臾间,下了南天门,回到东观,
早见那太阳星上。按云头,径至宝林寺山门外,只听得八戒还哭哩。忽近前叫声:
“师父。”三藏喜道:“悟空来了,可有丹药?”行者道:“有。”八戒道:“怎么得
没有?他偷也去偷人家些来!”行者笑道:“兄弟,你过去罢,用不着你了。你揩揩
眼泪,别处哭去。”

教沙和尚:“取些水来我用。”沙僧急忙往后面井上,有个方便吊桶,即将半钵
盂水递与行者。行者接了水,口中吐出丹来,安在那皇帝唇里;两手扳开牙齿,用
一口清水,把金丹冲灌下肚。有半个时辰,只听他肚里呼呼的乱响,只是身体不能
转移。行者道:“师父,弄我金丹也不能救活,可是杀老孙么?”三藏道:“岂有
不活之理。似这般久死之尸,如何吞得水下?此乃金丹之仙力也。自金丹入腹,却
就肠鸣了;肠鸣乃血脉和动,但气绝不能回伸。莫说人在井里浸了三年,就是生铁
也上锈了。只是元气尽绝,得个人度他一口气便好。”那八戒上前就要度气,三藏
一把扯住道:“使不得!还教悟空来。”那师父甚有主张:原来猪八戒自幼儿伤生作
孽吃人,是一口浊气;惟行者从小修持,咬松嚼柏,吃桃果为生,是一口清气。这
大圣上前,把个雷公嘴,噙着那皇帝口唇,呼的一口气,吹入咽喉,度下重楼,转
明堂,径至丹田,从涌泉倒返泥垣宫。呼的一声响,那君王气聚神归,便翻身,
轮拳曲足,叫了一声“师父”!双膝跪在尘埃道:“记得昨夜鬼魂拜谒,怎知道今朝
天晓返阳神!”三藏慌忙搀起道:“陛下,不干我事,你且谢我徒弟。”行者笑道:“师
父说那里话?常言道:‘家无二主。’你受他一拜儿不亏。”

三藏甚不过意,搀起那皇帝来,同入禅堂。又与八戒、行者、沙僧拜见了,方
才按座。只见那本寺的僧人,整顿了早斋,却欲来奉献;忽见那个水衣皇帝,个个
惊张,人人疑说。孙行者跳出来道:“那和尚,不要这等惊疑。这本是乌鸡国王,
乃汝之真主也。三年前被怪害了性命,是老孙今夜救活。如今进他城去,要辨明邪
正。若有了斋,摆将来,等我们吃了走路。”众僧即奉献汤水,与他洗了面,换了
衣服。把那皇帝赭黄袍脱了,本寺僧官,将两领布直裰,与他穿了;解下蓝田带,
将一条黄丝绦子与他系了;褪下无忧履,与他一双旧僧鞋撒了;却才都吃了早斋,
扣背马匹。

行者问:“八戒,你行李有多重?”八戒道:“哥哥,这行李日逐挑着,倒也不
知有多重。”行者道:“你把那一担儿分为两担,将一担儿你挑着,将一担儿与这皇
帝挑。我们赶早进城干事。”八戒欢喜道:“造化!造化!当时驮他来,不知费了多少
力;如今医活了,原来是个替身。”

那呆子就弄玄虚,将行李分开,就问寺中取条匾担,轻些的自己挑了,重些的
教那皇帝挑着。行者笑道:“陛下,着你那般打扮,挑着担子,跟我们走走,可亏
你么?”那国王慌忙跪下道:“师父,你是我重生父母一般,莫说挑担,情愿执鞭
坠镫,伏侍老爷,同行上西天去也。”行者道:“不要你去西天。我内中有个缘故。
你只挑得四十里进城。待捉了妖精,你还做你的皇帝,我们还取我们的经也。”八
戒听言道:“这等说,他只挑四十里路,我老猪还是长工!”行者道:“兄弟,不要
胡说:趁早外边引路。”

真个八戒领那皇帝前行,沙僧伏侍师父上马,行者随后。只见那本寺五百僧人,
齐齐整整,吹打着细乐,都送出山门之外。行者笑道:“和尚们不必远送:但恐官
家有人知觉,泄漏我的事机,反为不美。快回去,快回去!但把那皇帝的衣服冠带,
整顿干净,或是今晚明早,送进城来,我讨些封赠赏赐谢你。”众僧依命各回讫。
行者搀开大步,赶上师父,一直前来。正是:
西方有诀好寻真,金木和同却炼神。
丹母空怀懂梦,婴儿长恨杌樗身。
必须井底求明主,还要天堂拜老君。
悟得色空还本性,诚为佛度有缘人。

师徒们在路上,那消半日,早望见城池相近。三藏道:“悟空,前面想是乌鸡
国了。”行者道:“正是,我们快赶进城干事。”那师徒进得城来,只见街市上人物
齐整,风光闹热,早又见凤阁龙楼,十分壮丽。有诗为证,诗曰:
海外宫楼如上邦,人间歌舞若前唐。
花迎宝扇红云绕,日照鲜袍翠雾光。
孔雀屏开香霭出,珍珠帘卷彩旗张。
太平景象真堪贺,静列多官没奏章。
三藏下马道:“徒弟啊,我们就此进朝倒换关文,省得又拢那个衙门费事。”行者道:
“说得有理。我兄弟们都进去,人多才好说话。”唐僧道:“都进去,莫要撒村,先
行了君臣礼,然后再讲。”行者道:“行君臣礼,就要下拜哩。”三藏道:“正是,要
行五拜三叩头的大礼。”行者笑道:“师父不济。若是对他行礼,诚为不智。你且让
我先走到里边,自有处置。等他若有言语,让我对答。我若拜,你们也拜;我若蹲,
你们也蹲。”

你看那惹祸的猴王,引至朝门,与阁门大使言道:“我等是东土大唐驾下差来,
上西天拜佛求经者。今到此倒换关文,烦大人转达,是谓不误善果。”那黄门官即
入端门,跪下丹墀,启奏道:“朝门外有五众僧人,言是东土唐国钦差上西天拜佛
求经。今至此倒换关文,不敢擅入,现在门外听宣。”

那魔王即令传宣。唐僧却同入朝门里面。那回生的国主随行。正行,忍不住腮
边堕泪,心中暗道:“可怜!我的铜斗儿江山,铁围的社稷,谁知被他阴占了!”行
者道:“陛下切莫伤感,恐走漏消息。这棍子在我耳朵里跳哩,如今决要见功。管
取打杀妖魔,扫荡邪物。这江山不久就还归你也。”那君王不敢违言,只得扯衣揩
泪,舍死相从,径来到金銮殿下。

又见那两班文武,四百朝官,一个个威严端肃,象貌轩昂。这行者引唐僧站立
在白玉阶前,挺身不动。那阶下众官,无不悚惧,道:“这和尚十分愚浊!怎么见我
王便不下拜,亦不开言呼祝?喏也不唱一个,好大胆无礼!”说不了,只听得那魔王
开口问道:“那和尚是那方来的?”行者昂然答道:“我是南赡部洲东土大唐国奉钦
差前往西域天竺国大雷音寺拜活佛求真经者。今到此方,不敢空度,特来倒换通关
文牒。”那魔王闻说,心中作怒道:“你东土便怎么!我不在你朝进贡,不与你国相
通,你怎么见吾抗礼,不行参拜!”行者笑道:“我东土古立天朝,久称上国,汝等
乃下土边邦。自古道:‘上邦皇帝,为父为君;下邦皇帝,为臣为子。’你倒未曾接
我,且敢争我不拜?”那魔王大怒,教文武官:“拿下这野和尚去!”说声叫“拿”,
你看那多官一齐踊跃。这行者喝了一声,用手一指,教:“莫来!”那一指,就使个
定身法,众官俱莫能行动。真个是校尉阶前如木偶,将军殿上似泥人。

那魔王见他定住了文武多官,急纵身,跳下龙床,就要来拿。猴王暗喜道:“好!
正合老孙之意。这一来就是个生铁铸的头,汤着棍子,也打个窟窿!”正动身,不
期旁边转出一个救命星来。你道是谁,原来是乌鸡国王的太子,急上前扯住那魔王
的朝服,跪在面前道:“父王息怒。”

妖精问:“孩儿怎么说?”太子道:“启父王得知。三年前闻得人说,有个东土
唐朝驾下钦差圣僧往西天拜佛求经,不期今日才来到我邦。父王尊性威烈,若将这
和尚拿去斩首,只恐大唐有日得此消息,必生嗔怒。你想那李世民自称王位,一统
江山,心尚未足,又兴过海征伐;若知我王害了他御弟圣僧,一定兴兵发马,来与
我王争敌。奈何兵少将微,那时悔之晚矣。父王依儿所奏,且把那四个和尚,问他
个来历分明,先定他一段不参王驾,然后方可问罪。”

这一篇,原来是太子小心,恐怕来伤了唐僧,故意留住妖魔,更不知行者安排
着要打。那魔王果信其言,立在龙床前面,大喝一声道:“那和尚是几时离了东土,
唐王因甚事着你求经?”行者昂然而答道:“我师父乃唐王御弟,号曰三藏。因唐
王驾下有一丞相,姓魏名徵,奉天条梦斩泾河老龙。大唐王梦游阴司地府,复得回
生之后,大开水陆道场,普度冤魂孽鬼。因我师父敷演经文,广运慈悲,忽得南海
观世音菩萨指教来西。我师父大发弘愿,情欣意美,报国尽忠,蒙唐王赐与文牒。
那时正是大唐贞观十三年九月望前三日。离了东土,前至两界山,收了我做大徒弟,
姓孙,名悟空行者;又到乌斯国界高家庄,收了二徒弟,姓猪,名悟能八戒;流沙
河界,又收了三徒弟,姓沙,名悟净和尚;前日在敕建宝林寺,又新收个挑担的行
童道人。”

魔王闻说,又没法搜检那唐僧,弄巧计盘诘行者,怒目问道:“那和尚,你起
初时,一个人离东土,又收了四众,那三僧可让,这一道难容。那行童断然是拐来
的。他叫做甚么名字?有度牒是无度牒?拿他上来取供。”唬得那皇帝战战兢兢道:“师
父啊!我却怎的供?”孙行者捻他一把道:“你休怕,等我替你供。”

好大圣,趋步上前,对怪物厉声高叫道:“陛下,这老道是一个喑痖之人,却
又有些耳聋。只因他年幼间曾走过西天,认得道路。他的一节儿起落根本,我尽知
之,望陛下宽恕,待我替他供罢。”魔王道:“趁早实实的替他供来,免得取罪。”
行者道:

“供罪行童年且迈,痴聋喑痖家私坏。祖居原是此间人,五载之前遭破败。天
无雨,民干坏,君王黎庶都斋戒。焚香沐浴告天公,万里全无云。百姓饥荒若
倒悬,锺南忽降全真怪。呼风唤雨显神通,然后暗将他命害。推下花园水井中,阴
侵龙位人难解。幸吾来,功果大,起死回生无挂碍。情愿皈依作行童,与僧同去朝
西界。假变君王是道人,道人转是真王代。”
那魔王在金銮殿上,闻得这一篇言语,唬得他心头撞小鹿,面上起红云。急抽身就
要走路,奈何手内无一兵器;转回头,只见一个镇殿将军,腰挎一口宝刀,被行者
使了定身法,直挺挺如痴如痖,立在那里,他近前,夺了这宝刀,就驾云头望空而
去。气得沙和尚爆躁如雷,猪八戒高声喊叫,埋怨行者是一个急猴子:“你就慢说
些儿,却不稳住他了?如今他驾云逃走,却往何处追寻?”行者笑道:“兄弟们且莫
乱嚷。我等叫那太子下来拜父,嫔后出来拜夫。”却又念个咒语,解了定身法。“教
那多官苏醒回来拜君,方知是真实皇帝。教诉前情,才见分晓,我再去寻他。”好
大圣,吩咐八戒、沙僧:“好生保护他君臣父子嫔后,与我师父!”只听说声去,就
不见形影。

他原来跳在九霄云里,睁眼四望,看那魔王哩。只见那畜果逃了性命,径往东
北上走哩。行者赶得将近,喝道:“那怪物,那里去!老孙来了也!”那魔王急回头,
掣出宝刀,高叫道:“孙行者,你好惫懒!我来占别人的帝位,与你无干,你怎么来
抱不平,泄漏我的机密!”行者呵呵笑道:“我把你大胆的泼怪!皇帝又许你做?你既
知我是老孙,就该远遁;怎么还刁难我师父,要取甚么供状!适才那供状是也不是?
你不要走,好汉吃我老孙这一棒!”那魔侧身躲过,掣宝刀劈面相还。他两个搭上
手,这一场好杀,真是:
猴王猛,魔王强,刀迎棒架敢相当。
一天云雾迷三界,只为当朝立帝王。

他两个战经数合,那妖魔抵不住猴王,急回头复从旧路跳入城里,闯在白玉阶
前两班文武丛中,摇身一变,即变得与唐三藏一般模样,并搀手,立在阶前。

这大圣赶上,就欲举棒来打,那怪道:“徒弟莫打,是我!”急掣棒要打那个唐
僧,却又道:“徒弟莫打,是我!”一样两个唐僧,实难辨认。“倘若一棒打杀妖怪
变的唐僧,这个也成了功果;假若一棒打杀我的真实师父,却怎么好!……”只得
停手,叫八戒、沙僧问道:“果然那一个是怪,那一个是我的师父?你指与我,我好
打他。”八戒道:“你在半空中相打相嚷,我瞥瞥眼就见两个师父,也不知谁真谁假。”

行者闻言,捻诀念声咒语,叫那护法诸天、六丁六甲、五方揭谛、四值功曹、
一十八位护驾伽蓝、当坊土地、本境山神道:“老孙至此降妖,妖魔变作我师父,
气体相同,实难辨认。汝等暗中知会者,请师父上殿,让我擒魔。”

原来那妖怪善腾云雾,听得行者言语,急撒手跳上金銮宝殿。这行者举起棒望
唐僧就打。可怜!若不是唤那几位神来,这一下,就是二十个唐僧,也打为肉酱!多
亏众神架住铁棒道:“大圣,那怪会腾云,先上殿去了。”行者赶上殿,他又跳将下
来扯住唐僧,在人丛里又混了一混,依然难认。

行者心中不快;又见那八戒在旁冷笑,行者大怒道:“你这夯货怎的?如今有两
个师父,你有得叫,有得应,有得伏侍哩,你这般欢喜得紧!”八戒笑道:“哥啊,
说我呆,你比我又呆哩!师父既不认得,何劳费力?你且忍些头疼,叫我师父念念那
话儿,我与沙僧各搀一个听着。若不会念的,必是妖怪,有何难也?”行者道:“兄
弟,亏你也。正是,那话儿只有三人记得。原是我佛如来心苗上所发,传与观世音
菩萨,菩萨又传与我师父,便再没人知道。也罢,师父,念念。”真个那唐僧就念
起来。那魔王怎么知得,口里胡哼乱哼。八戒道:“这哼的却是妖怪了!”他放了手,
举钯就筑。那魔王纵身跳起,踏着云头便走。

好八戒,喝一声,也驾云头赶上,慌得那沙和尚丢了唐僧,也掣出宝杖来打。
唐僧才停了咒语。孙大圣忍着头疼,着铁棒,赶在空中。呀!这一场,三个狠和
尚,围住一个泼妖魔。那魔王被八戒、沙僧使钉钯宝杖左右攻住了。行者笑道:“我
要再去,当面打他,他却有些怕我,只恐他又走了;等我老孙跳高些,与他个捣蒜
打,结果了他罢。”

这大圣纵祥光,起在九霄,正欲下个切手,只见那东北上,一朵彩云里面,厉
声叫道:“孙悟空,且休下手!”行者回头看处,原来文殊菩萨。急收棒,上前施礼
道:“菩萨,那里去?”文殊道:“我来替你收这个妖怪的。”行者谢道:“累烦了。”
那菩萨袖中取出照妖镜,照住了那怪的原身。行者才招呼八戒、沙僧齐来见了菩萨。
却将镜子里看处,那魔王生得好不凶恶:

眼似琉璃盏,头若炼炒缸。浑身三伏靛,四爪九秋霜。搭拉两个耳,一尾扫帚
长。青毛生锐气,红眼放金光。匾牙排玉板,圆须挺硬枪。镜里观真象,原是文殊
一个狮猁王。
行者道:“菩萨,这是你坐下的一个青毛狮子,却怎么走将来成精,你就不收服他?”
菩萨道:“悟空,他不曾走,他是佛旨差来的。”行者道:“这畜类成精,侵夺帝位,
还奉佛旨差来。似老孙保唐僧受苦,就该领几道敕书!”

菩萨道:“你不知道。当初这乌鸡国王,好善斋僧,佛差我来度他归西,早证
金身罗汉。因是不可原身相见,变做一种凡僧,问他化些斋供。被吾几句言语相难,
他不识我是个好人,把我一条绳捆了,送在那御水河中,浸了我三日三夜。多亏六
甲金身救我归西,奏与如来,如来将此怪令到此处推他下井,浸他三年,以报吾三
日水灾之恨。‘一饮一啄,莫非前定’。今得汝等来此,成了功绩。”

行者道:“你虽报了甚么‘一饮一啄’的私仇,但那怪物不知害了多少人也。”
菩萨道:“也不曾害人。自他到后,这三年间,风调雨顺,国泰民安,何害人之有?”
行者道:“固然如此,但只三宫娘娘,与他同眠同起,点污了他的身体,坏了多少
纲常伦理,还叫做不曾害人?”菩萨道:“点污他不得。他是个骟了的狮子。”八戒
闻言,走近前,就摸了一把。笑道:“这妖精真个是‘糟鼻子不吃酒——枉担其名’
了!”行者道:“既如此,收了去罢。若不是菩萨亲来,决不饶他性命。”那菩萨却
念个咒,喝道:“畜生,还不皈正,更待何时!”那魔王才现了原身。菩萨放莲花罩
定妖魔,坐在背上,踏祥光辞了行者。咦!
径转五台山上去,宝莲座下听谈经。

毕竟不知那唐僧师徒怎的出城,且听下回分解。