\chapter{寇员外喜待高僧~唐长老不贪富贵}

色色原无色,空空亦非空。静喧语默本来同,梦里何劳说梦。有用用中无用,
无功功里施功。还如果熟自然红,莫问如何修种。

话表唐僧师众,使法力,阻住那布金寺僧。僧见黑风过处,不见他师徒,以为
活佛临凡,磕头而回不题。他师徒们西行,正是春尽夏初时节:
清和天气爽,池沼芰荷生。
梅逐雨余熟,麦随风里成。
草香花落处,莺老柳枝轻。
江燕携雏习,山鸡哺子鸣。
斗南当日永,万物显光明。
说不尽那朝餐暮宿,转涧寻坡。在那平安路上,行经半月。前边又见一城垣相近。
三藏问道:“徒弟,此又是甚么去处?”行者道:“不知,不知。”八戒笑道:“这路
是你行过的,怎说不知?却是又有些儿跷蹊。故意推不认得,捉弄我们哩。”行者道:
“这呆子全不察理!这路虽是走过几遍,那时只在九霄空里,驾云而来,驾云而去,
何曾落在此地?事不关心,查他做甚,此所以不知。却有甚跷蹊,又捉弄你也?”

说话间,不觉已至边前。三藏下马,过吊桥,径入门里。长街上,只见廊下坐
着两个老儿叙话。三藏叫:“徒弟,你们在那街心里站住,低着头,不要放肆,等
我去那廊下,问个地方。”行者等果依言立住。长老近前合掌,叫声:“老施主,贫
僧问讯了。”

那二老正在那里闲讲闲论,说甚么兴衰得失,谁圣谁贤,当时的英雄事业,而
今安在,诚可谓大叹息。忽听得道声问讯,随答礼道:“长老有何话说?”三藏道:
“贫僧乃远方来拜佛祖的,适到宝方,不知是甚地名。那里有向善的人家,化斋一
顿?”老者道:“我敝处是铜台府。府后有一县,叫做地灵县。长老若要吃斋,不
须募化,过此牌坊,南北街,坐西向东者,有一个虎坐门楼,乃是寇员外家。他门
前有个‘万僧不阻’之牌。似你这远方僧,尽着受用。去,去,去!莫打断我们的
话头。”

三藏谢了。转身对行者道:“此处乃铜台府地灵县。那二老道:‘过此牌坊,南
北街,向东虎坐门楼,有个寇员外家,他门前有个“万僧不阻”之牌。’教我到他
家去吃斋哩。”沙僧道:“西方乃佛家之地,真个有斋僧的。此间既是府县,不必照
验关文,我们去化些斋吃了,就好走路。”长老与三人缓步长街,又惹得那市口里
人,都惊惊恐恐,猜猜疑疑的,围绕争看他们相貌。长老吩咐闭口,只教:“莫放
肆!莫放肆!”三人果低着头,不敢仰视。转过拐角,果见一条南北大街。

正行时,见一个虎坐门楼,门里边影壁上挂着一面大牌,书着“万僧不阻”四
字。三藏道:“西方佛地,贤者,愚者,俱无诈伪。那二老说时,我犹不信,至此
果如其言。”八戒村野,就要进去。行者道:“呆子且住。待有人出来,问及何如,
方好进去。”沙僧道:“大哥说得有理。恐一时不分内外,惹施主烦恼。”在门口歇
下马匹、行李。须臾间,有个苍头出来,提着一把秤,一只篮儿,猛然看见,慌的
丢了,倒跑进去报道:“主公!外面有四个异样僧家来也!”那员外拄着拐,正在天
井中闲走,口里不住的念佛,一闻报道,就丢了拐,出来迎接。见他四众,也不怕
丑恶,只叫:“请进,请进。”三藏谦谦逊逊,一同都入。转过一条巷子,员外引路,
至一座房里,说道:“此上手房宇,乃管待老爷们的佛堂、经堂、斋堂;下手的,
是我弟子老小居住。”三藏称赞不已。随取袈裟穿了拜佛,举步登堂观看,但见那:

香云,烛焰光辉。满堂中锦簇花攒,四下里金铺彩绚。朱红架,高挂紫金
钟;彩漆檠,对设花腔鼓。几对,绣成八宝;千尊佛,尽戗黄金。古铜炉,古铜
瓶,雕漆桌,雕漆盒。古铜炉内,常常不断沉檀;古铜瓶中,每有莲花现彩。雕漆
桌上五云鲜,雕漆盒中香瓣积。玻璃盏,净水澄清;琉璃灯,香油明亮。一声金磬,
响韵虚徐。真个是红尘不到赛珍楼,家奉佛堂欺上刹。
长老净了手,拈了香,叩头拜毕,却转回与员外行礼。员外道:“且住!请到经堂中
相见。”又见那:

方台竖柜,玉匣金函:方台竖柜,堆积着无数经文;玉匣金函,收贮着许多简
札。彩漆桌上,有纸墨笔砚,都是些精精致致的文房;椒粉屏前,有书画琴棋,尽
是些妙妙玄玄的真趣。放一口轻玉浮金之仙磬,挂一柄披风披月之龙髯。清气令人
神气爽,斋心自觉道心闲。
长老到此,正欲行礼,那员外又搀住道:“请宽佛衣。”三藏脱了袈裟,才与长老见
了。又请行者三人见了。又叫把马喂了,行李安在廊下,方问起居。三藏道:“贫
僧是东土大唐钦差,诣宝方谒灵山见佛祖求真经者。闻知尊府敬僧,故此拜见,求
一斋就行。”员外面生喜色,笑吟吟的道:“弟子贱名寇洪,字大宽,虚度六十四岁。
自四十岁上,许斋万僧,才做圆满。今已斋了二十四年,有一簿斋僧的帐目。连日
无事,把斋过的僧名算一算,已斋过九千九百九十六员。止少四众,不得圆满。今
日可可的天降老师四位,完足万僧之数,请留尊讳。好歹宽住月余,待做了圆满,
弟子着轿马送老师上山。此间到灵山只有八百里路,苦不远也。”三藏闻言,十分
欢喜,都就权且应承不题。

他那几个大小家僮,往宅里搬柴打水,取米面蔬菜,整治斋供,忽惊动员外妈
妈问道:“是那里来的僧,这等上紧?”僮仆道:“才有四位高僧,爹爹问他起居,
他说是东土大唐皇帝差来的,往灵山拜佛爷爷。到我们这里,不知有多少路程。爹
爹说是天降的,吩咐我们快整斋,供养他也。”那老妪听说也喜,叫丫鬟:“取衣服
来我穿,我也去看看。”僮仆道:“奶奶,只一位看得,那三位看不得,形容丑得狠
哩。”老妪道:“汝等不知。但形容丑陋,古怪清奇,必是天人下界。快先去报你爹
爹知道。”那僮仆跑至经堂,对员外道:“奶奶来了,要拜见东土老爷哩。”三藏听
见,即起身下座。

说不了,老妪已至堂前。举目见唐僧相貌轩昂,丰姿英伟。转面见行者三人模
样非凡,虽知他是天人下界,却也有几分悚惧,朝上跪拜。三藏急急还礼道:“有
劳菩萨错敬。”老妪问员外说道:“四位师父,怎不并坐?”八戒掬着嘴道:“我三
个是徒弟。”噫!他这一声,就如深山虎啸。那妈妈一发害怕。

正说处,又见一个家僮来报道:“两个叔叔也来了。”三藏急转身看时,原来是
两个少年秀才。那秀才走上经堂,对长老倒身下拜,慌得三藏急便还礼。员外上前
扯住道:“这是我两个小儿,唤名寇梁,寇栋,在书房里读书方回,来吃午饭。知
老师下降,故来拜也。”三藏喜道:“贤哉,贤哉!正是欲高门第须为善,要好儿孙
在读书。”二秀才启上父亲道:“这老爷是那里来的?”员外笑道:“来路远哩。南
赡部洲东土大唐皇帝钦差到灵山拜佛祖爷爷取经的。”秀才道:“我看《事林广记》
上,盖天下只有四大部洲。我们这里叫做西牛贺洲。还有个东胜神洲。想南赡部洲
至此,不知走了多少年代?”三藏笑道:“贫僧在路,耽阁的日子多,行的日子少。
常遭毒魔狠怪,万苦千辛。甚亏我三个徒弟保护。共计一十四遍寒暑,方得至宝方。”
秀才闻言,称奖不尽道:“真是神僧,真是神僧!”说未毕,又有个小的来请道:“斋
筵已摆,请老爷进斋。”员外着妈妈与儿子转宅,他却陪四众进斋堂吃斋。那里铺
设的齐整。但见:

金漆桌案,黑漆交椅。前面是五色高果,俱巧匠新装成的时样。第二行五盘小
菜,第三行五碟水果,第四行五大盘闲食。般般甜美,件件馨香。素汤米饭,蒸卷
馒头,辣辣爨爨热腾腾,尽皆可口,真足充肠。七八个僮仆往来奔奉,四五个庖丁
不住手。
你看那上汤的上汤,添饭的添饭。一往一来,真如流星赶月。这猪八戒一口一碗,
就是风卷残云。师徒们尽受用了一顿。长老起身,对员外谢了斋,就欲走路。那员
外拦住道:“老师,放心住几日儿。常言道:‘起头容易结梢难。’只等我做过了圆
满,方敢送程。”三藏见他心诚意恳,没奈何住了。

早经过五七遍朝夕,那员外才请了本处应佛僧二十四员,办做圆满道场。众僧
们写作有三四日,选定良辰,开启佛事。他那里与大唐的世情一般,却倒也:

大扬,铺设金容;齐秉烛,烧香供养。擂鼓敲铙,吹笙捻管。云锣儿,横笛
音清,也都是,尺工字样。打一回,吹一荡,朗言齐语开经藏。先安土地,次请神
将。发了文书,拜了佛像。谈一部《孔雀经》,句句消灾障;点一架药师灯,焰焰
辉光亮。拜水忏,解冤愆;讽《华严》,除诽谤。三乘妙法甚精勤,一二沙门皆一
样。
如此做了三昼夜,道场已毕。唐僧想着雷音,一心要去,又相辞谢。员外道:“老
师辞别甚急,想是连日佛事冗忙,多致简慢,有见怪之意。”三藏道:“深扰尊府,
不知何以为报,怎敢言怪!但只当时圣君送我出关,问几时可回,我就误答三年可
回。不期在路耽阁,今已十四年矣!取经未知有无,及回又得十二三年,岂不违背
圣旨?罪何可当!望老员外让贫僧前去,待取得经回,再造府久住些时,有何不可!”
八戒忍不住,高叫道:“师父忒也不从人愿!不近人情!老员外大家巨富,许下这等
斋僧之愿,今已圆满,又况留得至诚,须住年把,也不妨事;只管要去怎的?放了
这等现成好斋不吃,却往人家化募!前头有你甚老爷、老娘家哩?”长老“咄”的
喝了一声道:“你这夯货,只知要吃,更不管回向之因,正是那‘槽里吃食,胃里
擦痒’的畜生!汝等既要贪此嗔痴,明日等我自家去罢。”行者见师父变了脸,即揪
住八戒,着头打一顿拳,骂道:“呆子不知好歹,惹得师父连我们都怪了!”沙僧笑
道:“打得好,打得好!只这等不说话,还惹人嫌,且又插嘴!”那呆子气呼呼的,
立在旁边,再不敢言。员外见他师徒们生恼,只得满面陪笑道:“老师莫焦燥,今
日且少宽容,待明日我办些旗鼓,请几个邻里亲戚,送你们起程。”

正讲处,那老妪又出来道:“老师父,即蒙到舍,不必苦辞。今到几日了?”
三藏道:“已半月矣。”老妪道:“这半月算我员外的功德。老身也有些针线钱儿,
也愿斋老师父半月。”说不了,寇栋兄弟又出来道:“四位老爷,家父斋僧二十余年,
更不曾遇着好人,今幸圆满,四位下降,诚然是蓬屋生辉。学生年幼,不知因果,
常闻得有云:‘公修公得,婆修婆得,不修不得。’我家父、家母,各欲献芹者,正
是各求得些因果,何必苦辞?就是愚兄弟,也省得有些束修钱儿,也只望供养老爷
半月,方才送行。”三藏道:“令堂老菩萨盛情,已不敢领,怎么又承贤昆玉厚爱?
决不敢领。今朝定要起身,万勿见罪。不然,久违钦限,罪不容诛矣。”那老妪与
二子见他执一不住,便生起恼来道:“好意留他,他这等固执要去,要去便就去了
罢!只管劳叨甚么!”母子遂抽身进去。八戒忍不住口,又对唐僧道:“师父,不要
拿过了班儿。常言道:‘留得在,落得怪。’我们且住一个月儿,了了他母子的愿心
也罢了,只管忙怎的?”唐僧又咄了一声,喝道。那呆子就自家把嘴打了两下道:
“啐,啐,啐!”说道:“莫多话,又做声了!”行者与沙僧的笑在一边。唐僧
又怪行者道:“你笑甚么?”即捻诀要念紧箍儿咒,慌得个行者跪下道:“师父,我
不曾笑,我不曾笑!千万莫念,莫念!”

员外又见他师徒们渐生烦恼,再也不敢苦留,只叫:“老师不必吵闹,准于明
早送行。”遂此出了经堂,吩咐书办,写了百十个简帖儿,邀请邻里亲戚,明早奉
送唐朝老师西行。一壁厢又叫庖人安排饯行的筵宴;一壁厢又叫管办的做二十对彩
旗,觅一班吹鼓手乐人,南来寺里请一班和尚,东岳观里请一班道士,限明日巳时,
各项俱要整齐。众执事领命去讫。不多时,天又晚了。吃了晚斋,各归寝处。正是
那:
几点归鸦过别村,楼头钟鼓远相闻。
六街三市人烟静,万户千门灯火昏。
月皎风清花弄影,银河惨淡映星辰。
子规啼处更深矣,天籁无声大地钧。
当时三四更天气,各管事的家僮,尽皆早起,买办各项物件。你看那办筵席的,厨
上慌忙;置彩旗的,堂前吵闹;请僧道的,两脚奔波;叫鼓乐的,一身急纵;送简
帖的,东走西跑;备轿马的,上呼下应。这半夜,直嚷至天明,将巳时前后,各项
俱完,也只是有钱不过。

却表唐僧师徒们早起,又有那一班人供奉。长老吩咐收拾行李,扣备马匹。呆
子听说要走,又努嘴胖唇,唧唧哝哝,只得将衣钵收拾,找启高肩担子。沙僧刷
马匹,套起鞍辔伺候。行者将九环杖递在师父手里,他将通关文牒的引袋儿,挂在
胸前,只是一齐要走。员外又都请至后面大厂厅内。那里面又铺设了筵宴,比斋堂
中相待的更是不同。但见那:

帘幕高挂,屏围四绕。正中间,挂一幅寿山福海之图;两
壁厢,列四轴春夏秋冬之景。龙文鼎内香飘霭,鹊尾炉中瑞气生。看盘簇彩,宝妆
花色色鲜明;排桌堆金,狮仙糖齐齐摆列。阶前鼓舞按宫商,堂上果肴铺锦绣。素
汤素饭甚清奇,香酒香茶多美艳。虽然是百姓之家,却不亚王侯之宅。只听得一片
欢声,真个也惊天动地。
长老正与员外作礼,只见家僮来报:“客俱到了。”却是那请来的左邻、右舍、妻弟、
姨兄、姐夫、妹丈;又有那些同道的斋公,念佛的善友,一齐都向长老礼拜。拜毕,
各各叙坐。只见堂下面鼓瑟吹笙,堂上边弦歌酒宴。这一席盛宴,八戒留心,对沙
僧道:“兄弟,放怀放量吃些儿。离了寇家,再没这好丰盛的东西了!”沙僧笑道:
“二哥说那里话!常言道:‘珍馐百味,一饱便休。只有私房路,那有私房肚?’”
八戒道:“你也忒不济,不济!我这一顿尽饱吃了,就是三日也急忙不饿。”行者听
见道:“呆子,莫胀破了肚子!如今要走路哩!”

说不了,日将中矣。长老在上举箸,念《揭斋经》。八戒慌了,拿过添饭来,
一口一碗,又丢够有五六碗,把那馒头、卷儿、饼子、烧果,没好没歹的,满满笼
了两袖,才跟师父起身。长老谢了员外,又谢了众人,一同出门。你看那门外摆着
彩旗宝盖,鼓手乐人。又见那两班僧道方来,员外笑道:“列位来迟,老师去急,
不及奉斋,俟回来谢罢。”众等让叙道路,抬轿的抬轿,骑马的骑马,步行的步行,
都让长老四众前行。只闻得鼓乐喧天,旗蔽日,人烟凑集,车马骈填,都来看寇
员外迎送唐僧。这一场富贵,真赛过珠围翠绕,诚不亚锦帐藏春!

那一班僧,打一套佛曲;那一班道,吹一道玄音,俱送出府城之外。行至十里
长亭,又设着箪食壶浆,擎杯把盏,相饮而别。那员外犹不忍舍,噙着泪道:“老
师取经回来,是必到舍再住几日,以了我寇洪之心。”三藏感之不尽,谢之无已道:
“我若到灵山,得见佛祖,首表员外之大德。回时定踵门叩谢,叩谢。”说说话儿,
不觉的又有二三里路。长老恳切拜辞。那员外又放声大哭而转。这正是:
有愿斋僧归妙觉,无缘得见佛如来。

且不说寇员外送至十里长亭,同众回家。却说他师徒四众,行有四五十里之地,
天色将晚。长老道:“天晚了,何方借宿?”八戒挑着担,努着嘴道:“放了现成茶
饭不吃,清凉瓦屋不住,却要走甚么路,像抢丧踵魂的!如今天晚,倘下起雨来,
却如之何!”三藏骂道:“泼孽畜,又来报怨了!常言道:‘长安虽好,不是久恋之家。’
待我们有缘拜了佛祖,取得真经,那时回转大唐,奏过主公,将那御厨里饭,凭你
吃上几年,胀死你这孽畜,教你做个饱鬼!”那呆子吓吓的暗笑,不敢复言。

行者举目遥观,只见大路旁有几间房宇,急请师父道:“那里安歇,那里安歇。”
长老至前,见是一座倒塌的牌坊,坊上有一旧扁,扁上有落颜色积尘的四个大字,
乃“华光行院”。长老下了马道:“华光菩萨是火焰五光佛的徒弟。因剿除毒火鬼王,
降了职,化做五显灵官。此间必有庙祝。”遂一齐进去。但见廊房俱倒,墙壁皆倾,
更不见人之踪迹,只是些杂草丛菁。欲抽身而出,不期天上黑云盖顶,大雨淋漓。
没奈何,却在那破房之下,拣遮得风雨处,将身躲避。密密寂寂,不敢高声,恐有
妖邪知觉。坐的坐,站的站,苦捱了一夜未睡。咦!真个是:
泰极还生否,乐处又逢悲。

毕竟不知天晓向前去还是如何,且听下回分解。