\chapter{乱蟠桃大圣偷丹~反天宫诸神捉怪}

话表齐天大圣到底是个妖猴,更不知官衔品从,也不较俸禄高低,但只注名便
了。那齐天府下二司仙吏,早晚伏侍,只知日食三餐,夜眠一榻,无事牵萦,自由
自在。闲时节会友游宫,交朋结义。见三清,称个“老”字;逢四帝,道个“陛下”。
与那九曜星、五方将、二十八宿、四大天王、十二元辰、五方五老、普天星相、河
汉群神,俱只以弟兄相待,彼此称呼。今日东游,明日西荡,云去云来,行踪不定。

一日,玉帝早朝,班部中闪出许旌阳真人,囟启奏道:“今有齐天大圣,无
事闲游,结交天上众星宿,不论高低,俱称朋友。恐后闲中生事。不若与他一件事
管,庶免别生事端。”玉帝闻言,即时宣诏。那猴王欣欣然而至,道:“陛下,诏老
孙有何升赏?”玉帝道:“朕见你身闲无事,与你件执事。你且权管那蟠桃园,早
晚好生在意。”大圣欢喜谢恩,朝上唱喏而退。

他等不得穷忙,即入蟠桃园内查勘。本园中有个土地拦住,问道:“大圣何往?”
大圣道:“吾奉玉帝点差,代管蟠桃园,今来查勘也。”那土地连忙施礼,即呼那一
班锄树力士、运水力士、修桃力士、打扫力士都来见大圣磕头,引他进去。但见那:

夭夭灼灼,颗颗株株:夭夭灼灼花盈树,颗颗株株果压枝。果压枝头垂锦弹,
花盈树上簇胭脂。时开时结千年熟,无夏无冬万载迟。先熟的,酡颜醉脸;还生的,
带蒂青皮。凝烟
肌带绿,映日显丹姿。树下奇葩并异卉,四时不谢色齐齐。左右楼台并馆舍,盈空
常见罩云霓。不是玄都凡俗种,瑶池王母自栽培。
大圣看玩多时,问土地道:“此树有多少株数?”土地道:“有三千六百株:前面一
千二百株,花微果小,三千年一熟,人吃了成仙了道,体健身轻。中间一千二百株,
层花甘实,六千年一熟,人吃了霞举飞升,长生不老。后面一千二百株,紫纹缃核,
九千年一熟,人吃了与天地齐寿,日月同庚。”大圣闻言,欢喜无任。当日查明了
株树,点看了亭阁,回府。自此后,三五日一次赏玩,也不交友,也不他游。

一日,见那老树枝头,桃熟大半,他心里要吃个尝新。奈何本园土地、力士并
齐天府仙吏紧随不便。忽设一计道:“汝等且出门外伺候,让我在这亭上少憩片时。”
那众仙果退。只见那猴王脱了冠服,爬上大树,拣那熟透的大桃,摘了许多,就在
树枝上自在受用。吃了一饱,却才跳下树来,簪冠着服,唤众等仪从回府。迟三二
日,又去设法偷桃,尽他享用。

一朝,王母娘娘设宴,大开宝阁,瑶池中做“蟠桃胜会”,即着那红衣仙女、
青衣仙女、素衣仙女、皂衣仙女、紫衣仙女、黄衣仙女、绿衣仙女,各顶花篮,去
蟠桃园摘桃建会。七衣仙女直至园门首,只见蟠桃园土地、力士同齐天府二司仙吏,
都在那里把门。仙女近前道:“我等奉王母懿旨,到此摘桃设宴。”土地道:“仙娥
且住。今岁不比往年了,玉帝点差齐天大圣在此督理,须是报大圣得知,方敢开园。”
仙女道:“大圣何在?”土地道:“大圣在园内,因困倦,自家在亭子上睡哩。”仙
女道:“既如此,寻他去来,不可迟误。”

土地即与同进。寻至花亭不见,只有衣冠在亭,不知何往。四下里都没寻处。
原来大圣耍了一会,吃了几个桃子,变做二寸长的个人儿,在那大树梢头浓叶之下
睡着了。七衣仙女道:“我等奉旨前来,寻不见大圣,怎敢空回?”旁有仙使道:“仙
娥既奉旨来,不必迟疑。我大圣闲游惯了,想是出园会友去了。汝等且去摘桃。我
们替你回话便是。”

那仙女依言,入树林之下摘桃。先在前树摘了二篮,又在中树摘了三篮;到后
树上摘取,只见那树上花果稀疏,止有几个毛蒂青皮的。原来熟的都是猴王吃了。
七仙女张望东西,只见向南枝上止有一个半红半白的桃子。青衣女用手扯下枝来,
红衣女摘了,却将枝子望上一放。原来那大圣变化了,正睡在此枝,被他惊醒。大
圣即现本相,耳躲里掣出金箍棒,幌一幌,碗来粗细,咄的一声道:“你是那方怪
物,敢大胆偷摘我桃!”慌得那七仙女一齐跪下道:“大圣息怒。我等不是妖怪,乃
王母娘娘差来的七衣仙女,摘取仙桃,大开宝阁,做‘蟠桃胜会’。适至此间,先
见了本园土地等神,寻大圣不见。我等恐迟了王母懿旨,是以等不得大圣,故先在
此摘桃,万望恕罪。”大圣闻言,回嗔作喜道:“仙娥请起。王母开阁设宴,请的是
谁?”仙女道:“上会自有旧规。请的是西天佛老、菩萨、圣僧、罗汉,南方南极
观音,东方崇恩圣帝、十洲三岛仙翁,北方北极玄灵,中央黄极黄角大仙,这个是
五方五老。还有五斗星君,上八洞三清、四帝、太乙天仙等众,中八洞玉皇、九垒、
海岳神仙;下八洞幽冥教主、注世地仙。各宫各殿大小尊神,俱一齐赴蟠桃嘉会。”
大圣笑道:“可请我么?”仙女道:“不曾听得说。”大圣道:“我乃齐天大圣,就请
我老孙做个席尊,有何不可?”仙女道:“此是上会旧规,今会不知如何。”大圣道:
“此言也是,难怪汝等。你且立下,待老孙先去打听个消息,看可请老孙不请。”

好大圣,捻着诀,念声咒语,对众仙女道:“住,住,住!”这原来是个定身法,
把那七衣仙女,一个个睁睁,白着眼,都站在桃树之下。大圣纵朵祥云,跳出
园内,竟奔瑶池路上而去。正行时,只见那壁厢:

一天瑞霭光摇曳,五色祥云飞不绝。白鹤声鸣振九,紫芝色秀分千叶。中间
现出一尊仙,相貌昂然丰采别。神舞虹霓幌汉霄,腰悬宝无生灭。名称赤脚大罗
仙,特赴蟠桃添寿节。
那赤脚大仙觌面撞见大圣,大圣低头定计,赚哄真仙。他要暗去赴会,却问:“老
道何往?”大仙道:“蒙王母见招,去赴蟠桃嘉会。”大圣道:“老道不知。玉帝因
老孙筋斗云疾,着老孙五路邀请列位,先至通明殿下演礼,后方去赴宴。”大仙是
个光明正大之人,就以他的诳语作真。道:“常年就在瑶池演礼谢恩,如何先去通
明殿演礼,方去瑶池赴会?”无奈,只得拨转祥云,径往通明殿去了。

大圣驾着云,念声咒语,摇身一变,就变做赤脚大仙模样,前奔瑶池。不多时,
直至宝阁,按住云头,轻轻移步,走入里面。只见那里:

琼香缭绕,瑞霭缤纷。瑶台铺彩结,宝阁散氤氲。凤翥鸾翔形缥缈,金花玉萼
影浮沉。上排着九凤丹霞,八宝紫霓墩。五彩描金桌,千花碧玉盆。桌上有龙肝
和凤髓,熊掌与猩唇。珍馐百味般般美,异果嘉肴色色新。

那里铺设得齐齐整整,却还未有仙来。这大圣点看不尽,忽闻得一阵酒香扑鼻;忽
转头,见右壁厢长廊之下,有几个造酒的仙官,盘糟的力士,领几个运水的道人,
烧火的童子,在那里洗缸刷瓮,已造成了玉液琼浆,香醪佳酿。大圣止不住口角流
涎,就要去吃,奈何那些人都在这里。他就弄个神通,把毫毛拔下几根,丢入口中
嚼碎,喷将出去,念声咒语,叫“变!”即变做几个瞌睡虫,奔在众人脸上。你看
那伙人,手软头低,闭眉合眼,丢了执事,都去盹睡。大圣却拿了些百味八珍,佳
肴异品,走入长廊里面,就着缸,挨着瓮,放开量,痛饮一番。吃勾了多时,
醉了。自揣自摸道:“不好,不好!再过会,请的客来,却不怪我?一时拿住,怎生
是好?不如早回府中睡去也。”

好大圣,摇摇摆摆,仗着酒,任情乱撞,一会把路差了;不是齐天府,却是兜
率天宫。一见了,顿然醒悟道:“兜率宫是三十三天之上,乃离恨天太上老君之处,
如何错到此间?——也罢,也罢!一向要来望此老,不曾得来,今趁此残步,就望他
一望也好。”即整衣撞进去。那里不见老君,四无人迹。原来那老君与燃灯古佛在
三层高阁朱陵丹台上讲道,众仙童、仙将、仙官、仙吏,都侍立左右听讲。这大圣
直至丹房里面,寻访不遇,但见丹灶之旁,炉中有火。炉左右安放着五个葫芦,葫
芦里都是炼就的金丹。大圣喜道:“此物乃仙家之至宝。老孙自了道以来,识破了
内外相同之理,也要炼些金丹济人,不期到家无暇;今日有缘,却又撞着此物,趁
老子不在,等我吃他几丸尝新。”他就把那葫芦都倾出来,就都吃了,如吃炒豆相
似。

一时间丹满酒醒。又自己揣度道:“不好!不好!这场祸,比天还大;若惊动玉
帝,性命难存。走,走,走!不如下界为王去也!”他就跑出兜率宫,不行旧路,从
西天门,使个隐身法逃去。即按云头,回至花果山界。但见那旌旗闪灼,戈戟光辉,
原来是四健将与七十二洞妖王,在那里演习武艺。大圣高叫道:“小的们,我来也!”
众怪丢了器械,跪倒道:“大圣好宽心!丢下我等许久,不来相顾!”大圣道:“没多
时,没多时!”且说且行,径入洞天深处。四健将打扫安歇,叩头礼拜毕。俱道:“大
圣在天这百十年,实受何职?”大圣笑道:“我记得才半年光景,怎么就说百十年
话?”健将道:“在天一日,即在下方一年也。”大圣道:“且喜这番玉帝相爱,果
封做‘齐天大圣’,起一座齐天府,又设安静、宁神二司,司设仙吏侍卫。向后见
我无事,着我代管蟠桃园。近因王母娘娘设‘蟠桃大会’,未曾请我,是我不待他
请,先赴瑶池,把他那仙品、仙酒,都是我偷吃了。走出瑶池,踉踉误入老君
宫阙,又把他五个葫芦金丹也偷吃了。但恐玉帝见罪,方才走出天门来也。”

众怪闻言大喜。即安排酒果接风,将椰酒满斟一石碗奉上。大圣喝了一口,即
咨牙嘴道:“不好吃!不好吃!”崩、芭二将道:“大圣在天宫,吃了仙酒、仙肴,
是以椰酒不甚美口。常言道:‘美不美,乡中水。’”大圣道:“你们就是‘亲不亲,
故乡人。’我今早在瑶池中受用时,见那长廊之下,有许多瓶罐,都是那玉液琼浆。
你们都不曾尝着。待我再去偷他几瓶回来,你们各饮半杯,一个个也长生不老。”
众猴欢喜不胜。

大圣即出洞门,又翻一筋斗,使个隐身法,径至蟠桃会上。进瑶池宫阙,只见
那几个造酒、盘糟、运水、烧火的,还鼾睡未醒。他将大的从左右胁下挟了两个,
两手提了两个,即拨转云头回来,会众猴在于洞中,就做个“仙酒会”,各饮了几
杯,快乐不题。

却说那七衣仙女自受了大圣的定身法术,一周天方能解脱。各提花篮,回奏王
母,说道:“齐天大圣使术法困住我等,故此来迟。”王母问道:“汝等摘了多少蟠
桃?”仙女道:“只有两篮小桃,三篮中桃。至后面,大桃半个也无,想都是大圣
偷吃了。及正寻间,不期大圣走将出来,行凶拷打,又问设宴请谁。我等把上会事
说了一遍,他就定住我等,不知去向。直到如今,才得醒解回来。”

王母闻言,即去见玉帝,备陈前事。说不了,又见那造酒的一班人,同仙官等
来奏:“不知甚么人,搅乱了‘蟠桃大会’,偷吃了玉液琼浆,其八珍百味,亦俱偷
吃了。”又有四个大天师来奏上:“太上道祖来了。”玉帝即同王母出迎。老君朝礼
毕,道:“老道宫中,炼了些‘九转金丹’,伺候陛下做‘丹元大会’,不期被贼偷
去,特启陛下知之。”玉帝见奏,悚惧。少时,又有齐天府仙吏叩头道:“孙大圣不
守执事,自昨日出游,至今未转,更不知去向。”玉帝又添疑思。只见那赤脚大仙
又囟上奏道:“臣蒙王母诏昨日赴会,偶遇齐天大圣,对臣言万岁有旨,着他邀
臣等先赴通明殿演礼,方去赴会。臣依他言语,即返至通明殿外,不见万岁龙车凤
辇,又急来此俟候。”玉帝越发大惊道:“这厮假传旨意,赚哄贤卿,快着纠察灵官
缉访这厮踪迹!”

灵官领旨,即出殿遍访,尽得其详细。回奏道:“搅乱天宫者,乃齐天大圣也。”
又将前事尽诉一番。玉帝大恼。即差四大天王,协同李天王并哪吒太子,点二十八
宿、九曜星官、十二元辰、五方揭谛、四值功曹、东西星斗、南北二神、五岳四渎、
普天星相,共十万天兵,布一十八架天罗地网下界,去花果山围困,定捉获那厮处
治。众神即时兴师,离了天宫。这一去,但见那:

黄风滚滚遮天暗,紫雾腾腾罩地昏。只为妖猴欺上帝,致令众圣降凡尘。四大
天王,五方揭谛:四大天王权总制,五方揭谛调多兵。李托塔中军掌号,恶哪吒前
部先锋。罗星为头检点,计都星随后峥嵘。太阴星精神抖擞,太阳星照耀分明。
五行星偏能豪杰,九曜星最喜相争。元辰星子午卯酉,一个个都是大力天丁。五瘟
五岳东西摆,六丁六甲左右行。四渎龙神分上下,二十八宿密层层。角亢氐房为总
领,奎娄胃昴惯翻腾。斗牛女虚危室壁,心尾箕星个个能,井鬼柳星张翼轸,轮枪
舞剑显威灵。停云降雾临凡世,花果山前扎下营。

诗曰:
天产猴王变化多,偷丹偷酒乐山窝。
只因搅乱蟠桃会,十万天兵布网罗。

当时李天王传了令,着众天兵扎了营,把那花果山围得水泄不通。上下布了十
八架天罗地网,先差九曜恶星出战。九曜即提兵径至洞外,只见那洞外大小群猴跳
跃顽耍。星官厉声高叫道:“那小妖!你那大圣在那里?我等乃上界差调的天神,到
此降你这造反的大圣。教他快快来归降;若道半个‘不’字,教汝等一概遭诛!”
那小妖慌忙传入道:“大圣,祸事了!祸事了!外面有九个凶神,口称上界差来的天
神,收降大圣。”

那大圣正与七十二洞妖王,并四健将分饮仙酒,一闻此报,公然不理道:“‘今
朝有酒今朝醉,莫管门前是与非。’”说不了,一起小妖又跳来道:“那九个凶神,
恶言泼语,在门前骂战哩!”大圣笑道:“莫睬他。‘诗酒且图今日乐,功名休问几
时成。’”说犹未了,又一起小妖来报:“爷爷!那九个凶神已把门打破,杀进来也!”
大圣怒道:“这泼毛神,老大无礼!本待不与他计较,如何上门来欺我?”即命独角
鬼王,领帅七十二洞妖王出阵,老孙领四健将随后。那鬼王疾帅妖兵,出门迎敌,
却被九曜恶星一齐掩杀,抵住在铁板桥头,莫能得出。

正嚷间,大圣到了。叫一声“开路!”掣开铁棒,幌一幌,碗来粗细,丈二长
短,丢开架子,打将出来。九曜星那个敢抵,一时打退。那九曜星立住阵势道:“你
这不知死活的弼马温!你犯了十恶之罪,先偷桃,后偷酒,搅乱了蟠桃大会,又窃
了老君仙丹,又将御酒偷来此处享乐,你罪上加罪,岂不知之?”大圣笑道:“这
几桩事,实有,实有!但如今你怎么?”九曜星道:“吾奉玉帝金旨,帅众到此收降
你,快早皈依!免教这些生灵纳命。不然,就平了此山,掀翻了此洞也!”大圣大
怒道:“量你这些毛神,有何法力,敢出浪言。不要走,请吃老孙一棒!”这九曜星
一齐踊跃。那美猴王不惧分毫,轮起金箍棒,左遮右挡,把那九曜星战得筋疲力软,
一个个倒拖器械,败阵而走,急入中军帐下,对托塔天王道:“那猴王果十分骁勇!
我等战他不过,败阵来了。”

李天王即调四大天王与二十八宿,一路出师来斗。大圣也公然不惧,调出独角
鬼王、七十二洞妖王与四个健将,就于洞门外列成阵势。你看这场混战,好惊人也:

寒风飒飒,怪雾阴阴。那壁厢旌旗飞彩,这壁厢戈戟生
辉。滚滚盔明,层层甲亮:滚滚盔明映太阳,如撞天的银磬;层层甲亮砌岩崖,似
压地的冰山。大捍刀,飞云掣电,楮白枪,度雾穿云。方天戟,虎眼鞭,麻林摆列;
青铜剑,四明铲,密树排阵。弯弓硬弩雕翎箭,短棍蛇矛挟了魂。大圣一条如意棒,
翻来覆去战天神。杀得那空中无鸟过,山内虎狼奔;扬砂走石乾坤黑,播土飞尘宇
宙昏。只听兵兵扑扑惊天地,煞煞威威振鬼神。
这一场自辰时布阵,混杀到日落西山。那独角鬼王与七十二洞妖怪,尽被众天神捉
拿去了,止走了四健将与那群猴,深藏在水帘洞底。这大圣一条棒,抵住了四大天
神与李托塔、哪吒太子,俱在半空中,——杀彀多时,大圣见天色将晚,即拔毫毛
一把,丢在口中,嚼碎了,喷将出去,叫声“变”!就变了千百个大圣,都使的是
金箍棒,打退了哪吒太子,战败了五个天王。

大圣得胜,收了毫毛,急转身回洞,早又见铁板桥头,四个健将,领众叩迎那
大圣,哽哽咽咽大哭三声,又唏唏哈哈大笑三声。大圣道:“汝等见了我,又哭又
笑,何也?”四健将道:“今早帅众将与天王交战,把七十二洞妖王与独角鬼王,
尽被众神捉了,我等逃生,故此该哭;这见大圣得胜回来,未曾伤损,故此该笑。”
大圣道:“胜负乃兵家之常。古人云:‘杀人一万,自损三千。’况捉了去的头目乃
是虎豹、狼虫、獾獐、狐之类,我同类者未伤一个,何须烦恼?他虽被我使个分
身法杀退,他还要安营在我山脚下。我等且紧紧防守,饱食一顿,安心睡觉,养养
精神。天明看我使个大神通,拿这些天将,与众报仇。”四将与众猴将椰酒吃了几
碗,安心睡觉不题。

那四大天王收兵罢战,众各报功:有拿住虎豹的,有拿住狮象的,有拿住狼虫
狐的,更不曾捉着一个猴精。当时果又安辕营,下大寨,赏了得功之将,吩咐
了天罗地网之兵,各各提铃喝号,围困了花果山,专待明早大战。各人得令,一处
处谨守。此正是:

妖猴作乱惊天地,布网张罗昼夜看。

毕竟天晓后如何处治,且听下回分解。