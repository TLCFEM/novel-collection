\chapter{猪八戒助力败魔王~孙行者三调芭蕉扇}

话表牛魔王赶上孙大圣,只见他肩膊上掮着那柄芭蕉扇,怡颜悦色而行。魔王
大惊道:“猢狲原来把运用的方法儿也叨得来了。我若当面问他索取,他定然不
与。倘若扇我一扇,要去十万八千里远,却不遂了他意?我闻得唐僧在那大路上等
候。他二徒弟猪精,三徒弟沙流精,我当年做妖怪时,也曾会他。且变作猪精的模
样,返骗他一场。料猢狲以得意为喜,必不详细堤防。”

好魔王,他也有七十二变,武艺也与大圣一般,只是身子狼些,欠钻疾,不
活达些;把宝剑藏了,念个咒语,摇身一变,即变作八戒一般嘴脸,抄下路,当面
迎着大圣,叫道:“师兄,我来也!”

这大圣果然欢喜。古人云“得胜的猫儿欢似虎”也,只倚着强能,更不察来人
的意思。见是个八戒的模样,便就叫道:“兄弟,你往那里去?”牛魔王绰着经儿
道:“师父见你许久不回,恐牛魔王手段大,你斗他不过,难得他的宝贝,教我来
迎你的。”行者笑道:“不必费心,我已得了手了。”牛王又问道:“你怎么得的?”
行者道:“那老牛与我战经百十合,不分胜负。他就撇了我,去那乱石山碧波潭底,
与一伙蛟精、龙精饮酒。是我暗跟他去,变作个螃蟹,偷了他所骑的辟水金睛兽,
变了老牛的模样,径至芭蕉洞哄那罗刹女。那女子与老孙结了一场干夫妻,是老孙
设法骗将来的。”牛王道:“却是生受了。哥哥劳碌太甚,可把扇子我拿。”孙大圣
那知真假,也虑不及此,遂将扇子递与他。

原来那牛王,他知那扇子收放的根本,接过手,不知捻个甚么诀儿,依然小似
一片杏叶,现出本象。开言骂道:“泼猢狲!认得我么?”行者见了,心中自悔道:
“是我的不是了!”恨了一声,跌足高呼道:“咦!逐年家打雁,今却被小雁儿了
眼睛。”狠得他爆躁如雷,掣铁棒,劈头便打,那魔王就使扇子他一下;不知那
大圣先前变虫入罗刹女腹中之时,将定风丹噙在口里,不觉的咽下肚里,所以
五脏皆牢,皮骨皆固;凭他怎么,再也他不动。牛王慌了,把宝贝丢入口中,
双手轮剑就砍。那两个在那半空中这一场好杀:

齐天孙大圣,混世泼牛王,只为芭蕉扇,相逢各骋强。粗心大圣将人骗,大胆
牛王把扇诓。这一个,金箍棒起无情义;那一个,双刃青锋有智量。大圣施威喷彩
雾,牛王放泼吐毫光。齐斗勇,两不良,咬牙锉齿气昂昂。播土扬尘天地暗,飞砂
走石鬼神藏。这个说:“你敢无知返骗我!”那个说:“我妻许你共相将!”言村语泼,
性烈情刚。那个说:“你哄人妻女真该死!告到官司有罪殃!”伶俐的齐天圣,凶顽
的大力王,一心只要杀,更不待商量。棒打剑迎齐努力,有些松慢见阎王。

且不说他两个相斗难分。却表唐僧坐在途中,一则火气蒸人,二来心焦口渴,
对火焰山土地道:“敢问尊神,那牛魔王法力如何?”土地道:“那牛王神通不小,
法力无边,正是孙大圣的敌手。”三藏道:“悟空是个会走路的,往常家二千里路,
一霎时便回,怎么如今去了一日?断是与那牛王赌斗。”叫:“悟能,悟净!你两个,
那一个去迎你师兄一迎?倘或遇敌,就当用力相助,求得扇子来,解我烦躁,早早
过山,赶路去也。”八戒道:“今日天晚,我想着要去接他,但只是不认得积雷山路。”
土地道:“小神认得。且教卷帘将军与你师父做伴,我与你去来。”三藏大喜道:“有
劳尊神,功成再谢。”那八戒抖擞精神,束一束皂锦直裰,搴着钯,即与土地纵起
云雾,径回东方而去。

正行时,忽听得喊杀声高,狂风滚滚。八戒按住云头看时,原来孙行者与牛王
厮杀哩。土地道:“天蓬还不上前怎的?”呆子掣钉钯,厉声高叫道:“师兄,我来
也!”行者恨道:“你这夯货,误了我多少大事!”八戒道:“师父教我来迎你,因认
不得山路,商议良久,教土地引我,故此来迟;如何误了大事?”行者道:“不是
怪你来迟。这泼牛十分无礼!我向罗刹处弄得扇子来,却被这厮变作你的模样,口
称迎我,我一时欢悦,转把扇子递在他手,他却现了本像,与老孙在此比并,所以
误了大事也。”

八戒闻言大怒。举钉钯,当面骂道:“我把你这血皮胀的遭瘟!你怎敢变作你祖
宗的模样,骗我师兄,使我兄弟不睦!”你看他没头没脸的使钉钯乱筑。那牛王,
一则是与行者斗了一日,力倦神疲;二则是见八戒的钉钯凶猛,遮架不住,败阵就
走。只见那火焰山土地,帅领阴兵,当面挡住道:“大力王,且住手。唐三藏西天
取经,无神不保,无天不佑,三界通知,十方拥护。快将芭蕉扇来息火焰,教他
无灾无障,早过山去;不然,上天责你罪愆,定遭诛也。”牛王道:“你这土地,全
不察理!那泼猴夺我子,欺我妾,骗我妻,番番无道,我恨不得囫囵吞他下肚,化
作大便喂狗,怎么肯将宝贝借他!”

说不了,八戒赶上骂道:“我把你个结心癀!快拿出扇来,饶你性命!”那牛王
只得回头,使宝剑又战八戒。孙大圣举棒相帮。这一场在那里好杀:

成精豕,作怪牛,兼上偷天得道猴。禅性自来能战炼,必当用土合元由。钉钯
九齿尖还利,宝剑双锋快更柔。铁棒卷舒为主仗,土神助力结丹头。三家刑克相争
竞,各展雄才要运筹。捉牛耕地金钱长,唤豕归炉木气收。心不在焉何作道,神常
守舍要拴猴。胡乱嚷,苦相求,三般兵刃响搜搜。钯筑剑伤无好意,金箍棒起有因
由。只杀得星不光兮月不皎,一天寒雾
黑悠悠!
那魔王奋勇争强,且行且斗,斗了一夜,不分上下,早又天明。前面是他的积雷山
摩云洞口,他三个与土地、阴兵,又喧哗振耳,惊动那玉面公主,唤丫鬟看是那里
人嚷。只见守门小妖来报:“是我家爷爷与昨日那雷公嘴汉子并一个长嘴大耳的和
尚同火焰山土地等众厮杀哩!”玉面公主听言,即命外护的大小头目,各执枪刀助
力。前后点起七长八短,有百十余口。一个个卖弄精神,拈抢弄棒,齐告:“大王
爷爷,我等奉奶奶内旨,特来助力也!”牛王大喜道:“来得好,来得好!”众妖一
齐上前乱砍。八戒措手不及,倒拽着钯,败阵而走。大圣纵筋斗云,跳出重围。众
阴兵亦四散奔走。老牛得胜,聚众妖归洞,紧闭了洞门不题。

行者道:“这厮骁勇!自昨日申时前后,与老孙战起,直到今夜,未定输赢,却
得你两个来接力。如此苦斗半日一夜,他更不见劳困。才这一伙小妖,却又莽壮。
他将洞门紧闭不出,如之奈何?”八戒道:“哥哥,你昨日巳时离了师父,怎么到
申时才与他斗起?你那两三个时辰,在那里的?”行者道:“别你后,顷刻就到这座
山上,见一个女子,问讯,原来就是他爱妾玉面公主。被我使铁棒唬他一唬,他就
跑进洞,叫出那牛王来。与老孙言语,嚷了一会,又与他交手,斗了有一个时
辰。正打处,有人请他赴宴去了。是我跟他到那乱石山碧波潭底,变作一个螃蟹,
探了消息,偷了他辟水金睛兽,假变牛王模样,复至翠云山芭蕉洞,骗了罗刹女,
哄得他扇子。出门试演试演方法,把扇子弄长了,只是不会收小。正掮了走处,被
他假变做你的嘴脸,返骗了去。故此耽搁两三个时辰也。”

八戒道:“这正是俗语云:‘大海里翻了豆腐船,汤里来,水里去。’如今难得
他扇子,如何保得师父过山?且回去,转路走他娘罢!”土地道:“大圣休焦恼,天
蓬莫懈怠。但说转路,就是入了傍门,不成个修行之类,古语云‘行不由径’,岂
可转走?你那师父,在正路上坐着,眼巴巴只望你们成功哩!”行者发狠道:“正是,
正是!呆子莫要胡谈!土地说得有理。我们正要与他:

赌输赢,弄手段,等我施为地煞变。自到西方无对头,牛王本是心猿变。今番
正好会源流,断要相持借宝扇。趁清凉,息火焰,打破顽空参佛面。行满超升极乐
天,大家同赴龙华宴!”
那八戒听言,便生势力。殷勤道:

“是,是,是!去,去,去!管甚牛王会不会,木生在亥配为猪,牵转牛儿归土
类。申下生金本是猴,无刑无克多和气。用芭蕉,为水意,焰火消除成既济。昼夜
休离苦尽功,功完赶赴盂兰会。”

他两个领着土地、阴兵一齐上前,使钉钯,轮铁棒,乒乒乓乓,把一座摩云洞
的前门,打得粉碎。唬得那外护头目,战战兢兢,闯入里边报道:“大王!孙悟空率
众打破前门也!”

那牛王正与玉面公主备言其事,懊恨孙行者哩。听说打破前门,十分发怒,急
披挂,拿了铁棍,从里边骂出来道:“泼猢狲!你是多大个人儿,敢这等上门撒泼,
打破我门扇?”八戒近前乱骂道:“泼老剥皮!你是个甚样人物,敢量那个大小!不
要走!看钯!”牛王喝道:“你这个囔糟食的夯货,不见怎的!快叫那猴儿上来!”行
者道:“不知好歹的草!我昨日还与你论兄弟,今日就是仇人了!仔细吃吾一棒!”
那牛王奋勇而迎。这场比前番更胜。三个英雄,厮混在一处。好杀:

钉钯铁棒逞神威,同帅阴兵战老牺。牺牲独展凶强性,遍满同天法力恢。使钯
筑,着棍擂,铁棒英雄又出奇。三般兵器叮当响,隔架遮拦谁让谁?他道他为首,
我道我夺魁。土兵为
证难分解,木土相煎上下随。这两个说:“你如何不借芭蕉扇!”那一个道:“你焉
敢欺心骗我妻!赶妾害儿仇未报,敲门打户又惊疑!”这个说:“你仔细堤防如意棒,
擦着些儿就破皮!”那个说:“好生躲避钯头齿,一伤九孔血淋漓!”牛魔不怕施威
猛,铁棍高擎有见机。翻云覆雨随来往,吐雾喷风任发挥。恨苦这场都拚命,各怀
恶念喜相持。丢架手,让高低,前迎后挡总无亏。兄弟二人齐努力,单身一棍独施
为。卯时战到辰时后,战罢牛魔束手回。
他三个含死忘生,又斗有百十余合。八戒发起呆性,仗着行者神通,举钯乱筑。牛
王遮架不住,败阵回头,就奔洞门。却被土地、阴兵拦住洞门,喝道:“大力王,
那里走!吾等在此!”那老牛不得进洞,急抽身,又见八戒、行者赶来,慌得卸了盔
甲,丢了铁棍,摇身一变,变做一只天鹅,望空飞走。

行者看见,笑道:“八戒!老牛去了。”那呆子漠然不知,土地亦不能晓,一个
个东张西觑,只在积雷山前后乱找。行者指道:“那空中飞的不是?”八戒道:“那
是一只天鹅。”行者道:“正是老牛变的。”土地道:“既如此,却怎生么?”行者道:
“你两个打进此门,把群妖尽情剿除,拆了他的窝巢,绝了他的归路,等老孙与他
赌变化去。”那八戒与土地,依言攻破洞门不题。

这大圣收了金箍棒,捻诀念咒,摇身一变,变作一个海东青,飕的一翅,钻在
云眼里,倒飞下来,落在天鹅身上,抱住颈项眼。那牛王也知是孙行者变化,急
忙抖抖翅,变作一只黄鹰,返来海东青。行者又变作一个乌凤,专一赶黄鹰。牛
王识得,又变作一只白鹤,长唳一声,向南飞去。行者立定,抖抖翎毛,又变作一
只丹凤,高鸣一声。那白鹤见凤是鸟王,诸禽不敢妄动,刷的一翅,淬下山崖,将
身一变,变作一只香獐,乜乜些些,在崖前吃草。行者认得,也就落下翅来,变作
一只饿虎,剪尾跑蹄,要来赶獐作食。魔王慌了手脚,又变作一只金钱花斑的大豹,
要伤饿虎。行者见了,迎着风,把头一幌,又变作一只金眼狻猊,声如霹雳,铁额
铜头,复转身要食大豹。牛王着了急,又变作一个人熊,放开脚,就来擒那狻猊。
行者打个滚,就变作一只赖象,鼻似长蛇,牙如竹笋,撒开鼻子,要去卷那人熊。

牛王嘻嘻的笑了一笑,现出原身——一只大白牛:头如峻岭,眼若闪光。两只
角,似两座铁塔。牙排利刃。连头至尾,有千余丈长短;自蹄至背,有八百丈高下。
对行者高叫道:“泼猢狲!你如今将奈我何?”行者也就现了原身,抽出金箍棒来,
把腰一躬,喝声叫“长!”长得身高万丈,头如泰山,眼如日月,口似血池,牙似
门扇,手执一条铁棒,着头就打。那牛王硬着头,使角来触。这一场,真个是撼岭
摇山,惊天动地!有诗为证,诗曰:
道高一尺魔千丈,奇巧心猿用力降。
若得火山无烈焰,必须宝扇有清凉。
黄婆矢志扶元老,木母留情扫荡妖。
和睦五行归正果,炼魔涤垢上西方。
他两个大展神通,在半山中赌斗,惊得那过往虚空,一切神众与金头揭谛、六甲六
丁、一十八位护教伽蓝都来围困魔王。那魔王公然不惧,你看他东一头,西一头,
直挺挺,光耀耀的两只铁角,往来抵触;南一撞,北一撞,毛森森,筋暴暴的一条
硬尾,左右敲摇。孙大圣当面迎,众多神四面打,牛王急了,就地一滚,复本象,
便投芭蕉洞去。行者也收了法象,与众多神随后追袭。那魔王闯入洞里,闭门不出。
概众把一座翠云山围得水泄不通。

正都上门攻打,忽听得八戒与土地、阴兵嚷嚷而至。行者见了,问曰:“那摩
云洞事体如何?”八戒笑道:“那老牛的娘子,被我一钯筑死,剥开衣看,原来是
个玉面狸精。那伙群妖,俱是些驴、骡、犊、特、獾、狐、、獐、羊、虎、糜、
鹿等类。已此尽皆剿戮,又将他洞府房廊放火烧了。土地说他还有一处家小,住居
此山,故又来这里扫荡也。”行者道:“贤弟有功。可喜!可喜!老孙空与那老牛赌变
化,未曾得胜。他变做无大不大的白牛,我变了法天象地的身量。正和他抵触之间,
幸蒙诸神下降。围困多时,他却复原身,走进洞去矣。”八戒道:“那可是芭蕉洞么?”
行者道:“正是,正是!罗刹女正在此间。”八戒发狠道:“既是这般,怎么不打进去,
剿除那厮,问他要扇子,倒让他停留长智,两口儿叙情!”

好呆子,抖擞威风,举钯照门一筑,忽辣的一声,将那石崖连门筑倒了一边。
慌得那女童忙报:“爷爷!不知甚人把前门都打坏了!”牛王方跑进去,喘嘘嘘的,
正告诉罗刹女与孙行者夺扇子赌斗之事,闻报,心中大怒。就口中吐出扇子,递与
罗刹女。罗刹女接扇在手,满眼垂泪道:“大王!把这扇子送与那猢狲,教他退兵去
罢。”牛王道:“夫人啊,物虽小而恨则深。你且坐着,等我再和他比并去来。”

那魔重整披挂,又选两口宝剑,走出门来。正遇着八戒使钯筑门,老牛更不打
话,掣剑劈脸便砍。八戒举钯迎着,向后倒退了几步,出门来,早有大圣轮棒当头。
那牛魔即驾狂风,跳离洞府,又都在那翠云山上相持。众多神四面围绕,土地兵左
右攻击。这一场,又好杀哩:

云迷世界,雾罩乾坤。飒飒阴风砂石滚,巍巍怒气海波浑。重磨剑二口,复挂
甲全身。结冤深似海,怀恨越生嗔。你看齐天大圣因功绩,不讲当年老故人。八戒
施威求扇子,众神护法捉牛君。牛王双手无停息,左遮右挡弄精神。只杀得那过鸟
难飞皆敛翅,游鱼不跃尽潜鳞;鬼泣神嚎天地暗,龙愁虎怕日光昏!

那牛王拚命捐躯,斗经五十余合,抵敌不住,败了阵,往北就走。早有五台山
秘魔岩神通广大泼法金刚阻住,道:“牛魔,你往那里去!我等乃释迦牟尼佛祖差来,
布列天罗地网,至此擒汝也!”正说间,随后有大圣、八戒、众神赶来。那魔王慌
转身向南走;又撞着峨眉山清凉洞法力无量胜至金刚挡住,喝道:“吾奉佛旨在此,
正要拿住你也!”牛王心慌脚软,急抽身往东便走;却逢着须弥山摩耳崖卢沙门
大力金刚迎住道:“你老牛何往!我蒙如来密令,教来捕获你也!”牛王又悚然而退,
向西就走;又遇着昆仑山金霞岭不坏尊王永住金刚敌住,喝道:“这厮又将安走!我
领西天大雷音寺佛老亲言,在此把截,谁放你也!”那老牛心惊胆战,悔之不及。
见那四面八方都是佛兵天将,真个似罗网高张,不能脱命。正在仓惶之际,又闻得
行者帅众赶来,他就驾云头,望上便走。

却好有托塔李天王并哪吒太子,领鱼肚药叉、巨灵神将,幔住空中,叫道:“慢
来,慢来!吾奉玉帝旨意,特来此剿除你也!”牛王急了,依前摇身一变,还变做一
只大白牛,使两只铁角去触天王。天王使刀来砍。随后孙行者又到。哪吒太子厉声
高叫:“大圣,衣甲在身,不能为礼。愚父子昨日见佛如来,发檄奏闻玉帝,言唐
僧路阻火焰山,孙大圣难伏牛魔王,玉帝传旨,特差我父王领众助力。”行者道:“这
厮神通不小!又变作这等身躯,却怎奈何?”太子笑道:“大圣忽疑,你看我擒他。”

这太子即喝一声“变!”变得三头六臂,飞身跳在牛王背上,使斩妖剑望颈项
上一挥,不觉得把个牛头斩下。天王收刀,却才与行者相见。那牛王腔子里又钻出
一个头来,口吐黑气,眼放金光。被哪吒又砍一剑,头落处,又钻出一个头来。一
连砍了十数剑,随即长出十数个头。哪吒取出火轮儿挂在那老牛的角上,便吹真火,
焰焰烘烘,把牛王烧得张狂哮吼,摇头摆尾。才要变化脱身,又被托塔天王将照妖
镜照住本象,腾那不动,无计逃生,只叫“莫伤我命!情愿归顺佛家也!”哪吒道:
“既惜身命,快拿扇子出来!”牛王道:“扇子在我山妻处收着哩。”哪吒见说,将
缚妖索子解下,跨在他那颈项上,一把拿住鼻头,将索穿在鼻孔里,用手牵来。孙
行者却会聚了四大金刚、六丁六甲、护教伽蓝、托塔天王、巨灵神将并八戒、土地、
阴兵,簇拥着白牛,回至芭蕉洞口。

老牛叫道:“夫人,将扇子出来,救我性命!”罗刹听叫,急卸了钗环,脱了色
服,挽青丝如道姑,穿缟素似比丘,双手捧那柄丈二长短的芭蕉扇子,走出门;又
见有金刚众圣与天王父子,慌忙跪在地下,磕头礼拜道:“望菩萨饶我夫妻之命,
愿将此扇奉承孙叔叔成功去也!”行者近前接了扇,同大众共驾祥云,径回东路。

却说那三藏与沙僧立一会,坐一会盼望行者,许久不回,何等忧虑。忽见祥云
满空,瑞光满地,飘飘摇摇,盖众神行将近,这长老害怕道:“悟净!那壁厢是谁神
兵来也?”沙僧认得道:“师父啊,那是四大金刚、金头揭谛、六甲六丁、护教伽
蓝与过往众神。牵牛的是哪吒三太子。拿镜的是托塔李天王。大师兄执着芭蕉扇,
二师兄并土地随后,其余的都是护卫神兵。”三藏听说,换了卢帽,穿了袈裟,
与悟净拜迎众圣,称谢道:“我弟子有何德能,敢劳列位尊圣临凡也。”四大金刚道:
“圣僧喜了,十分功行将完。吾等奉佛旨差来助汝,汝当竭力修持,勿得须臾怠惰。”
三藏叩齿叩头,受身受命。

孙大圣执着扇子,行近山边,尽气力挥了一,那火焰山平平息焰,寂寂除光;
行者喜喜欢欢,又一扇,只闻得习习潇潇,清风微动;第三扇,满天云漠漠,细
雨落霏霏。有诗为证,诗曰:
火焰山遥八百程,火光大地有声名。
火煎五漏丹难熟,火燎三关道不清。
时借芭蕉施雨露,幸蒙天将助神功。
牵牛归佛休颠劣,水火相联性自平。
此时三藏解燥除烦,清心了意。四众皈依,谢了金刚,各转宝山。六丁六甲,升空
保护。过往神四散。天王、太子,牵牛径归佛地回缴。止有本山土地,押着罗刹
女,在旁伺候。

行者道:“那罗刹,你不走路,还立在此等甚?”罗刹跪道:“万望大圣垂慈,
将扇子还了我罢。”八戒喝道:“泼贱人,不知高低!饶了你的性命,就够了,还要
讨甚么扇子,我们拿过山去,不会卖钱买点心吃?费了这许多精神力气,又肯与你!
雨蒙蒙的,还不回去哩!”罗刹再拜道:“大圣原说扇息了火还我。今此一场,诚悔
之晚矣。只因不倜傥,致令劳师动众。我等也修成人道,只是未归正果。见今真身
现象归西,我再不敢妄作。愿赐本扇,从立自新,修身养命去也。”土地道:“大圣,
趁此女深知息火之法,断绝火根,还他扇子,小神居此苟安,拯救这方生民,求些
血食,诚为恩便。”行者道:“我当时问着乡人说:‘这山扇息火,只收得一年五谷,
便又火发。’如何治得除根?”罗刹道:“要是断绝火根,只消连四十九扇,永远
再不发了。”

行者闻言,执扇子,使尽筋力,望山头连四十九扇,那山上大雨淙淙。果然
是宝贝:有火处下雨,无火处天晴。他师徒们立在这无火处,不遭雨湿。坐了一夜,
次早才收拾马匹、行李,把扇子还了罗刹。又道:“老孙若不与你,恐人说我言而
无信。你将扇子回山,再休生事。看你得了人身,饶你去罢!”那罗刹接了扇子,
念个咒语,捏做个杏叶儿,噙在口里。拜谢了众圣,隐姓修行。后来也得了正果,
经藏中万古流名。罗刹、土地,俱感激谢恩,随后相送。行者、八戒、沙僧,保着
三藏遂此前进,真个是身体清凉,足下滋润。诚所谓:
坎离既济真元合,水火均平大道成。

毕竟不知几年才回东土,且听下回分解。