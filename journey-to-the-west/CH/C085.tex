\chapter{心猿妒木母~魔主计吞禅}

话说那国王早朝,文武多官俱执表章启奏道:“主公,望赦臣等失仪之罪。”国
王道:“众卿礼貌如常,有何失仪?”从卿道:“主公啊,不知何故,臣等一夜把头
发都没了。”国王执了这没头发之表,下龙床对群臣道:“果然不知何故。朕宫中大
小人等,一夜也尽没了头发。”君臣们都各汪汪滴泪道:“从此后,再不敢杀戮和尚
也。”

王复上龙位,众官各立本班。王又道:“有事出班来奏,无事卷帘散朝。”只见
那武班中闪出巡城总兵官,文班中走出东城兵马使,当阶叩头道:“臣蒙圣旨巡城,
夜来获得贼赃一柜,白马一匹。微臣不敢擅专,请旨定夺。”国王大喜道:“连柜取
来。”

二臣即退至本衙,点起齐整军士,将柜抬出。三藏在内,魂不附体道:“徒弟
们,这一到国王前,如何理说?”行者笑道:“莫嚷!我已打点停当了。开柜时,他
就拜我们为师哩。只教八戒不要争竞长短。”八戒道:“但只免杀,就是无量之福,
还敢争竞哩!”说不了,抬至朝外,入五凤楼,放在丹墀之下。

二臣请国王开看,国王即命打开。方揭了盖,猪八戒就忍不住往外一跳,唬得
那多官胆战,口不能言。又见孙行者搀出唐僧,沙和尚搬出行李。八戒见总兵官牵
着马,走上前,咄的一声道:“马是我的,拿过去!”吓得那官儿翻跟头,跌倒在地。
四众俱立在阶中。那国王看见是四个和尚,忙下龙床,宣召三宫妃后,下金銮宝殿,
同群臣拜问道:“长老何来?”三藏道:“是东土大唐驾下差往西方天竺国大雷音寺
拜活佛取真经的。”国王道:“老师远来,为何在这柜里安歇?”三藏道:“贫僧知
陛下有愿心杀和尚,不敢明投上国,扮俗人,夜至宝方饭店里借宿。因怕人识破原
身,故此在柜中安歇。不幸被贼偷出,被总兵捉获抬来。今得见陛下龙颜,所谓拨
云见日。望陛下赦放贫僧,海深恩便也!”国王道:“老师是天朝上国高僧,朕失迎
迓。朕常年有愿杀僧者,曾因僧谤了朕,朕许天愿,要杀一万和尚做圆满。不期今
夜归依,教朕等为僧。如今君臣后妃,发都剃落了,望老师勿吝高贤,愿为门下。”
八戒听言,呵呵大笑道:“既要拜为门徒,有何贽见之礼?”国王道:“师若肯从,
愿将国中财宝献上。”行者道:“莫说财宝,我和尚是有道之僧。你只把关文倒换了,
送我们出城,保你皇图永固,福寿长臻。”那国王听说,即着光禄寺大排筵宴。君
臣合同,拜归于一。即时倒换关文,求三藏改换国号。行者道:“陛下‘法国’之
名甚好,但只‘灭’字不通;自经我过,可改号‘钦法国’,管教你海晏河清千代
胜,风调雨顺万方安。”国王谢了恩。摆整朝銮驾,送唐僧四众出城西去。君臣们
秉善归真不题。

却说长老辞别了钦法国王,在马上欣然道:“悟空,此一法甚善,大有功也。”
沙僧道:“哥啊,是那里寻这许多整容匠,连夜剃这许多头?”行者把那施变化弄
神通的事说了一遍。师徒们都笑不合口。

正欢喜处,忽见一座高山阻路。唐僧勒马道:“徒弟们,你看这面前山势崔巍,
切须仔细!”行者笑道:“放心!放心!保你无事!”三藏道:“休言无事。我见那山峰
挺立,远远的有些凶气,暴云飞出,渐觉惊惶,满身麻木,神思不安。”行者笑道:
“你把乌巢禅师的《多心经》早已忘了。”三藏道:“我记得。”行者道:“你虽记得,
还有四句颂子,你却忘了哩。”三藏道:“那四句?”行者道:
“佛在灵山莫远求,灵山只在汝心头。
人人有个灵山塔,好向灵山塔下修。”
三藏道:“徒弟,我岂不知?若依此四句,千经万典,也只是修心。”行者道:“不消
说了。心净孤明独照,心存万境皆清。差错些儿成惰懈,千年万载不成功。但要一
片志诚,雷音只在眼下。似你这般恐惧惊惶,神思不安,大道远矣,雷音亦远矣。
且莫胡疑,随我去。”那长老闻言,心神顿爽,万虑皆休。

四众一同前进。不几步,到于山上。举目看时:

那山真好山,细看色班班。顶上云飘荡,崖前树影寒。飞禽淅沥,走兽凶顽。
林内松千干,峦头竹几竿。吼叫是苍狼夺食,咆哮是饿虎争餐。野猿长啸寻鲜果,
糜鹿攀花上翠岚。风洒洒,水潺潺,时闻幽鸟语间关。几处藤萝牵又扯,满溪瑶草
杂香兰。磷磷怪石,削削峰岩。狐成群走,猴猿作队顽。行客正愁多险峻,奈何
古道又湾还!
师徒们怯怯惊惊,正行之时,只听得呼呼一阵风起。三藏害怕道:“风起了!”行者
道:“春有和风,夏有熏风,秋有金风,冬有朔风:四时皆有风。风起怕怎的?”
三藏道:“这风来得甚急,决然不是天风。”行者道:“自古来,风从地起,云自山
出。怎么得个天风?”说不了,又见一阵雾起。那雾真个是:
漠漠连天暗,蒙蒙匝地昏。
日色全无影,鸟声无处闻。
宛然如混沌,仿佛似飞尘。
不见山头树,那逢采药人。
三藏一发心惊道:“悟空,风还未定,如何又这般雾起?”行者道:“且莫忙。请师
父下马,你兄弟二个在此保守,等我去看看是何吉凶。”

好大圣,把腰一躬,就到半空。用手搭在眉上,圆睁火眼,向下观之,果见那
悬岩边坐着一个妖精。你看他怎生模样:
炳炳文斑多采艳,昂昂雄势甚抖擞。
坚牙出口如钢钻,利爪藏蹄似玉钩。
金眼圆睛禽兽怕,银须倒竖鬼神愁。
张狂哮吼施威猛,嗳雾喷风运智谋。
又见逼左右手下有三四十个小妖摆列,他在那里逼法的喷风嗳雾。行者暗笑道:“我
师父也有些儿先兆。他说不是天风,果然不是,却是个妖精在这里弄喧儿哩。若老
孙使铁棒往下就打,这叫做‘捣蒜打’,打便打死了,只是坏了老孙的名头。”那行
者一生豪杰,再不晓得暗算计人。他道:“我且回去,照顾猪八戒照顾,教他来先
与这妖精见一仗。若是八戒有本事,打倒这妖,算他一功;若无手段,被这妖拿去,
等我再去救他,才好出名。他想道:“八戒有些躲懒,不肯出头,却只是有些口紧,
好吃东西。等我哄他一哄,看他怎么说。”

即时落下云头,到三藏前。三藏问道:“悟空,风雾处吉凶何如?”行者道:“这
会子明净了,没甚风雾。”三藏道:“正是,觉到退下些去了。”行者笑道:“师父,
我常时间还看得好,这番却看错了。我只说风雾之中恐有妖怪,原来不是。”三藏
道:“是甚么?”行者道:“前面不远,乃是一庄村。村上人家好善,蒸的白米干饭,
白面馍馍斋僧哩。这些雾,想是那些人家蒸笼之气,也是积善之应。”八戒听说,
认了真实,扯过行者,悄悄的道:“哥哥,你先吃了他的斋来的?”行者道:“吃不
多儿,因那菜蔬太咸了些,不喜多吃。”八戒道:“啐!凭他怎么咸,我也尽肚吃他
一饱!十分作渴,便回来吃水。”行者道:“你要吃么?”八戒道:“正是,我肚里有
些饥了,先要去吃些儿,不知如何?”行者道:“兄弟莫题。古书云:‘父在,子不
得自专。’师父又在此,谁敢先去?”八戒笑道:“你若不言语,我就去了。”行者
道:“我不言语,看你怎么得去。”

那呆子吃嘴的见识偏有,走上前,唱个大喏道:“师父,适才师兄说,前村里
有人家斋僧。你看这马,有些要打搅人家,便要草要料,却不费事?幸如今风雾明
净,你们且略坐坐,等我去寻些嫩草儿,先喂喂马,然后再往那家子化斋去罢。”
唐僧欢喜道:“好啊!你今日却怎肯这等勤谨?快去快来。”那呆子暗暗笑着便走。行
者赶上扯住道:“兄弟,他那里斋僧,只斋俊的,不斋丑的。”八戒道:“这等说,
又要变化是。”行者道:“正是。你变变儿去。”好呆子,他也有三十六般变化,走
到山凹里,捻着诀,念动咒语,摇身一变,变做个矮瘦和尚。手里敲个木鱼,口里
哼阿哼的,又不会念经,只哼的是“上大人……”。

却说那怪物收风敛雾,号令群妖,在于大路口上,摆开一个圈子阵,专等行客。
这呆子晦气,不多时,撞到当中,被群妖围住,这个扯住衣服,那个扯着丝绦,推
推拥拥,一齐下手。八戒道:“不要扯,等我一家家吃将来。”群妖道:“和尚,你
要吃甚的?”八戒道:“你们这里斋僧,我来吃斋的。”群妖道:“你想这里斋僧,
不知我这里专要吃僧。我们都是山中得道的妖仙,专要把你们和尚拿到家里,上蒸
笼蒸熟吃哩。你倒还想来吃斋!”八戒闻言,心中害怕;才报怨行者道:“这个弼马
温,其实惫懒!他哄我说是这村里斋僧,这里那得村庄人家,那里斋甚么僧,却原
来是些妖精!”那呆子被他扯急了,即便现出原身,腰间掣钉钯,一顿乱筑,筑退
那些小妖。

小妖急跑去报与老怪道:“大王,祸事了!”老怪道:“有甚祸事?”小妖道:“山
前来了一个和尚,且是生得干净。我说拿家来蒸他吃,若吃不了,留些儿防天阴,
不想他会变化。”老妖道:“变化甚的模样?”小妖道:“那里成个人相!长嘴大耳朵,
背后又有鬃。双手轮一根钉钯,没头没脸的乱筑,唬得我们跑回来报大王也。”老
怪道:“莫怕,等我去看。”轮着一条铁杵,走近前看时,见呆子果然丑恶。他生得:
碓嘴初长三尺零,獠牙觜出赛银钉。
一双圆眼光如电,两耳扇风唿唿声。
脑后鬃长排铁箭,浑身皮糙癞还青。
手中使件蹊跷物,九齿钉钯个个惊。
妖精硬着胆喝道:“你是那里来的,叫甚名字?快早说来,饶你性命!”八戒笑道:“我
的儿,你是也不认得你猪祖宗哩!上前来,说与你听:

巨口獠牙神力大,玉皇升我天蓬帅。掌管天河八万兵,天宫快乐多自在。只因
酒醉戏宫娥,那时就把英雄卖。一嘴拱倒斗牛宫,吃了王母灵芝菜。玉皇亲打二千
锤,把吾贬下三天界。教吾立志养元神,下方却又为妖怪。正在高庄喜结亲,命低
撞着孙兄在。金箍棒下受他降,低头才把沙门拜。背马挑包做夯工,前生少了唐僧
债。铁脚天蓬本姓猪,法名改作猪八戒。”
那妖精闻言,喝道:“你原来是唐僧的徒弟。我一向闻得唐僧的肉好吃,正要拿你
哩。你却撞得来,我肯饶你?不要走,看杵!”八戒道:“孽畜!你原来是个染博士出
身!”妖精道:“我怎么是染博士?”八戒道:“不是染博士,怎么会使棒槌?”那
怪那容分说,近前乱打。他两个在山凹里,这一场好杀:

九齿钉钯,一条铁棒。钯丢解数滚狂风,杵运机谋飞骤雨。一个是无名恶怪阻
山程,一个是有罪天蓬扶性主。性正何愁怪与魔,山高不得金生土。那个杵架犹如
蟒出潭,这个钯来却似龙离浦。喊声叱咤振山川,喝雄威惊地府。两个英雄各逞
能,舍身却把神通赌。
八戒长起威风,与妖精厮斗,那怪喝令小妖把八戒一齐围住不题。

却说行者在唐僧背后,忽失声冷笑。沙僧道:“哥哥冷笑,何也?”行者道:“猪
八戒真个呆呀!听见说斋僧,就被我哄去了。这早晚还不见回来:若是一顿钯打退
妖精,你看他得胜而回,争嚷功果;若战他不过,被他拿去,却是我的晦气,背前
面后,不知骂了多少弼马温哩!悟净,你休言语,等我去看看。”

好大圣,他也不使长老知道,悄悄的脑后拔了一根毫毛,吹口仙气,叫“变!”
即变做本身模样,陪着沙僧,随着长老。他的真身出个神,跳在空中观看,但见那
呆子被怪围绕,钉钯势乱,渐渐的难敌。

行者忍不住,按落云头,厉声高叫道:“八戒不要忙,老孙来了!”那呆子听得
是行者声音,仗着势,愈长威风,一顿钯,向前乱筑。那妖精抵敌不住,道:“这
和尚先前不济,这会子怎么又发起狠来。”八戒道:“我的儿,不可欺负我,我家里
人来也!”一发向前,没头没脸筑去。那妖精委架不住,领群妖败阵去了。行者见
妖精败去,他就不曾近前,拨转云头,径回本处,把毫毛一抖,收上身来。长老的
肉眼凡胎,那里认得。

不一时,呆子得胜,也自转来,累得那粘涎鼻涕,白沫生生,气的,走将
来,叫声:“师父!”长老见了,惊讶道:“八戒,你去打马草的,怎么这般狼狈回
来?想是山上人家有人看护,不容你打草么?”呆子放下钯,捶胸跌脚道:“师父,
莫要问,说起来就活活羞杀人!”长老道:“为甚么羞来?”八戒道:“师兄捉弄我!
他先头说风雾里不是妖精,没甚凶兆,是一庄村人家好善,蒸白米干饭、白面馍馍
斋僧的,我就当真,想着肚里饥了,先去吃些儿,假倚打草为名;岂知若干妖怪,
把我围了,苦战了这一会,若不是师兄的哭丧棒相助,我也莫想得脱罗网回来也!”
行者在旁笑道:“这呆子胡说!你若做了贼,就攀上一牢人。是我在这里看着师父,
何曾侧离?”长老道:“是啊,悟空不曾离我。”那呆子跳着嚷道:“师父,你不晓
得,他有替身!”长老道:“悟空,端的可有怪么?”行者瞒不过,躬身笑道:“是
有个把小妖儿,他不敢惹我们。八戒,你过来,一发照顾你照顾。我们既保师父,
走过险峻山路,就似行军的一般。”八戒道:“行军便怎的?”行者道:“你做个开
路将军,在前剖路。那妖精不来便罢,若来时,你与他赌斗。打倒妖精,算你的功
果。”八戒量着那妖精手段与他差不多,却说:“我就死在他手内也罢,等我先走!”
行者笑道:“这呆子先说晦气话,怎么得长进!”八戒道:“哥啊,你知道‘公子登
筵,不醉即饱;壮士临阵,不死带伤’?先说句错话儿,后便有威风。”行者欢喜,
即忙背了马,请师父骑上,沙僧挑着行李,相随八戒,一路入山不题。

却说那妖精帅几个败残的小妖,径回本洞,高坐在那石崖上,默默无言。洞中
还有许多看家的小妖,都上前问道:“大王常时出去,喜喜欢欢回来,今日如何烦
恼?”老妖道:“小的们,我往常出洞巡山,不管那里的人与兽,定捞几个来家,
养赡汝等;今日造化低,撞见一个对头。”小妖问:“是那个对头?”老妖道:“是
一个和尚,乃东土唐僧取经的徒弟,名唤猪八戒。我被他一顿钉钯,把我筑得败下
阵来。好恼啊!我这一向,常闻得人说,唐僧乃十世修行的罗汉,有人吃他一块肉,
可以延寿长生。不期他今日到我山里,正好拿住他蒸吃,不知他手下有这等徒弟!”

说不了,班部丛中闪上一个小妖,对老妖哽哽咽咽哭了三声,又嘻嘻哈哈的笑
了三声。老妖喝道:“你又哭又笑,何也?”小妖跪下道:“大王才说要吃唐僧,唐
僧的肉不中吃。”老妖道:“人都说吃他一块肉可以长生不老,与天同寿,怎么说他
不中吃?”小妖道:“若是中吃,也到不得这里,别处妖精,也都吃了。他手下有三
个徒弟哩。”老妖道:“你知是那三个?”小妖道:“他大徒弟是孙行者,三徒弟是
沙和尚。这个是他二徒弟猪八戒。”老妖道:“沙和尚比猪八戒如何?”小妖道:“也
差不多儿。”“那个孙行者比他如何?”小妖吐舌道:“不敢说!那孙行者神通广大,
变化多端!他五百年前曾大闹天宫,上方二十八宿、九曜星官、十二元辰、五卿四
相、东西星斗、南北二神、五岳四渎、普天神将,也不曾惹得他过,你怎敢要吃唐
僧?”老妖道:“你怎么晓得他这等详细?”小妖道:“我当初在狮驼岭狮驼洞与那
大王居住,那大王不知好歹,要吃唐僧,被孙行者使一条金箍棒,打进门来,可怜
就打得犯了骨牌名,都‘断么绝六’;还亏我有些见识,从后门走了,来到此处,
蒙大王收留。故此知他手段。”老妖听言,大惊失色。这正是“大将军怕谶语”。他
闻得自家人这等说,安得不惊。

正都在悚惧之际,又一个小妖上前道:“大王莫恼,莫怕。常言道:‘事从缓来。’
若是要吃唐僧,等我定个计策拿他。”老妖道:“你有何计?”小妖道:“我有个‘分
瓣梅花计’。”老妖道:“怎么叫做‘分瓣梅花计’?”小妖道:“如今把洞中大小群
妖,点将起来,千中选百,百中选十,十中只选三个,须是有能干,会变化的,都
变做大王的模样,顶大王之盔,贯大王之甲,执大王之杵,三处埋伏。先着一个战
猪八戒,再着一个战孙行者,再着一个战沙和尚:舍着三个小妖,调开他弟兄三个,
大王却在半空伸下拿云手去捉这唐僧,就如‘探囊取物’,就如‘鱼水盆内捻苍蝇’,
有何难哉!”老妖闻此言,满心欢喜,道:“此计绝妙,绝妙!这一去,拿不得唐僧
便罢;若是拿了唐僧,决不轻你,就封你做个前部先锋。”小妖叩头谢恩,叫点妖
怪。即将洞中大小妖精点起,果然选出三个有能的小妖,俱变做老妖,各执铁杵,
埋伏等待唐僧不题。

却说这唐长老无虑无忧,相随八戒上大路,行够多时,只见那路旁边扑禄的一
声响亮,跳出一个小妖,奔向前边,要捉长老。孙行者叫道:“八戒!妖精来了,何
不动手?”那呆子不认真假,掣钉钯赶上乱筑。那妖精使铁杵急架相迎。他两个一
往一来的,在山坡下正然赌斗,又见那草科里响一声,又跳出个怪来,就奔唐僧。
行者道:“师父!不好了!八戒的眼拙,放那妖道来拿你了,等老孙打他去!”急掣棒
迎上前喝道:“那里去!看棒!”那妖精更不打话,举杵来迎。他两个在草坡下一撞
一冲,正相持处,又听得山背后呼的风响,又跳出个妖精来,径奔唐僧。沙僧见了,
大惊道:“师父!大哥与二哥的眼都花了,把妖精放将来拿你了!你坐在马上,等老
沙拿他去!”这和尚也不分好歹,即掣杖,对面挡住那妖精铁杵,恨苦相持。吆吆
喝喝,乱嚷乱斗,渐渐的调远。那老怪在半空中,见唐僧独坐马上,伸下五爪钢钩,
把唐僧一把挝住。那师父丢了马,脱了镫,被妖精一阵风径摄去了。可怜!这正是:
禅性遭魔难正果,江流又遇苦灾星。

老妖按下风头,把唐僧拿到洞里,叫:“先锋!”那定计的小妖上前跪倒,口中
道:“不敢,不敢!”老妖道:“何出此言?大将军一言既出,如白染皂。当时说拿不
得唐僧便罢,拿了唐僧,封你为前部先锋。今日你果妙计成功,岂可失信于你?你
可把唐僧拿来,着小的们挑水刷锅,搬柴烧火,把他蒸一蒸,我和你都吃他一块肉,
以图延寿长生也。”

先锋道:“大王,且不可吃。”老怪道:“既拿来,怎么不可吃?”先锋道:“大
王吃了他不打紧,猪八戒也做得人情,沙和尚也做得人情,但恐孙行者那主子刮毒。
他若晓得是我们吃了,他也不来和我们厮打,他只把那金箍棒往山腰里一搠,搠个
窟窿,连山都掬倒了,我们安身之处也无之矣!”老怪道:“先锋,凭你有何高见?”
先锋道:“依着我,把唐僧送在后园,绑在树上,两三日不要与他饭吃,一则图他
里面干净;二则等他三人不来门前寻找,打听得他们回去了,我们却把他拿出来,
自自在在的受用,却不是好?”老怪笑道:“正是,正是!先锋说得有理!”

一声号令,把唐僧拿入后园,一条绳绑在树上。众小妖都去前面去听候。你看
那长老苦捱着绳缠索绑,紧缚牢拴,止不住腮边流泪,叫道:“徒弟呀!你们在那山
中擒怪,甚路里赶妖?我被泼魔捉来,此处受灾,何日相会?痛杀我也!”

正自两泪交流,只见对面树上有人叫道:“长老,你也进来了!”长老正了性道:
“你是何人?”那人道:“我是本山中的樵子,被那山主前日拿来,绑在此间,今
已三日,算计要吃我哩。”长老滴泪道:“樵夫啊,你死只是一身,无甚挂碍,我却
死得不甚干净。”樵子道:“长老,你是个出家人,上无父母,下无妻子,死便死了,
有甚么不干净?”长老道:“我本是东土往西天取经去的,奉唐朝太宗皇帝御旨拜
活佛,取真经,要超度那幽冥无主的孤魂。今若丧了性命,可不盼杀那君王,孤负
那臣子?那枉死城中,无限的冤魂,却不大失所望,永世不得超生;一场功果,尽
化作风尘,这却怎么得干净也?”樵子闻言,眼中堕泪道:“长老,你死也只如此,
我死又更伤情。我自幼失父,与母鳏居,更无家业,止靠着打柴为生。老母今年八
十三岁,只我一人奉养。倘若身丧,谁与他埋尸送老?苦哉,苦哉!痛杀我也!”长
老闻言,放声大哭道:“可怜,可怜!山人尚有思亲意,空教贫僧会念经!事君事亲,
皆同一理。你为亲恩,我为君恩。”正是那:
流泪眼观流泪眼,断肠人送断肠人。

且不言三藏身遭困苦。却说孙行者在草坡下战退小妖,急回来路旁边,不见了
师父,止存白马、行囊。慌得他牵马挑担,向山头找寻。咦!正是那:
有难的江流专遇难,降魔的大圣亦遭魔。

毕竟不知寻找师父下落如何,且听下回分解。