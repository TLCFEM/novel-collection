\chapter{婴儿问母知邪正~金木参玄见假真}

逢君只说受生因,便作如来会上人。
一念静观尘世佛,十方同看降威神。
欲知今日真明主,须问当年嫡母身。
别有世间曾未见,一行一步一花新。

却说那乌鸡国王太子,自别大圣,不多时,回至城中。果然不奔朝门,不敢报
传宣诏,径至后宰门首,见几个太监在那里把守。见太子来,不敢阻滞,让他进去
了。好太子,夹一夹马,撞入里面,忽至锦香亭下。只见那正宫娘娘坐在锦香亭上,
两边有数十个嫔妃掌扇,那娘娘倚雕栏儿流泪哩。

你道他流泪怎的?原来他四更时也做了一梦,记得一半,含糊了一半,沉沉思
想。这太子下马,跪于亭下。叫:“母亲!”那娘娘强整欢容,叫声“孩儿,喜呀,
喜呀!这二三年在前殿与你父王开讲,不得相见,我甚思量;今日如何得暇来看我
一面?诚万千之喜,诚万千之喜!孩儿,你怎么声音悲惨?你父王年纪高迈,有一日
龙归碧海,凤返丹霄,你就传了帝位,还有甚么不悦?”太子叩头道:
“母亲,我问你:即位登龙是那个?称孤道寡果何人?”娘娘闻言道:“这孩儿发风
了!做皇帝的是你父王,你问怎的?”太子叩头道:“万望母亲赦子无罪,敢问;不
赦,不敢问。”娘娘道:“子母家有何罪?赦你,赦你,快快说来。”太子道:“母亲,
我问你三年前夫妻宫里之事与后三年恩爱同否,如何?”

娘娘见说,魂飘魄散,急下亭抱起,紧搂在怀,眼中滴泪道:
“孩儿!我与你久不相见,怎么今日来宫问此?”太子发怒道:“母亲有话早说;不
说时,且误了大事。”娘娘才喝退左右,泪眼低声道:“这桩事,孩儿不问,我到九
泉之下,也不得明白。既问时,听我说:
三载之前温又暖,三年之后冷如冰。
枕边切切将言问,他说老迈身衰事不兴!”

太子闻言,撒手脱身,攀鞍上马。那娘娘一把扯住道:“孩儿,你有甚事,话
不终就走?”太子跪在面前道:“母亲,不敢说。今日早朝,蒙钦差架鹰逐犬,出
城打猎,偶遇东土驾下来的个取经圣僧,有大徒弟乃孙行者,极善降妖。原来我父
王死在御花园八角琉璃井内,这全真假变父王,侵了龙位。今夜三更,父王托梦,
请他到城捉怪。孩儿不敢尽信,特来问母。母亲才说出这等言语,必然是个妖精。”
那娘娘道:“儿啊,外人之言,你怎么就信为实?”太子道:“儿还不敢认实,父王
遗下表记与他了。”娘娘问是何物,太子袖中取出那金厢白玉,递与娘娘。那娘
娘认得是当时国王之宝,止不住泪如泉涌。叫声“主公!你怎么死去三年,不来见
我,却先见圣僧,后来见我?”太子道:“母亲,这话是怎的说?”娘娘道:“儿啊,
我四更时分,也做了一梦,梦见你父王水淋淋的,站在我跟前,亲说他死了,鬼魂
儿拜请了唐僧,降假皇帝,救他前身。记便记得是这等言语,只是一半儿不得分明。
正在这里狐疑,怎知今日你又来说这话,又将宝贝拿出。我且收下,你且去请那圣
僧急急为之。果然扫荡妖氛,辨明邪正,庶报你父王养育之恩也。”

太子急忙上马,出后宰门,躲离城池。真个是噙泪叩头辞国母,含悲顿首复唐
僧。不多时,出了城门,径至宝林寺山门前下马。众军士接着太子,又见红轮将坠。
太子传令,不许军士乱动。他又独自个入了山门,整束衣冠,拜请行者。

只见那猴王从正殿摇摇摆摆走来。那太子双膝跪下道:“师父,我来了。”行者
上前搀住道:“请起,你到城中,可曾问谁么?”太子道:“问母亲来。”将前言尽
说了一遍。行者微微笑道:“若是那般冷啊,想是个甚么冰冷的东西变的。不打紧,
不打紧,等我老孙与你扫荡。却只是今日晚了,不好行事。你先回去,待明早我来。”
太子跪地叩拜道:“师父,我只在此伺候,到明日同师父一路去罢。”行者道:“不
好,不好,若是与你一同入城,那怪物生疑,不说是我撞着你,却说是你请老孙,
却不惹他反怪你也?”太子道:“我如今进城,他也怪我。”行者道:“怪你怎么?”
太子道:“我自早朝蒙差,带领若干人马鹰犬出城,今一日更无一件野物,怎么见
驾?若问我个不才之罪,监陷里,你明日进城,却将何倚?况那班部中更没个相知
人也。”行者道:“这甚打紧?你肯早说时,却不寻下些等你。”

好大圣!你看他就在太子面前,显个手段,将身一纵,跳在云端里。捻着诀,
念一声“蓝净法界”的真言,拘得那山神、土地在半空中施礼道:“大圣,呼唤
小神,有何使令?”行者道:“老孙保护唐僧至此,欲拿邪魔,奈何那太子打猎无
物,不敢回朝;问汝等讨个人情,快将獐鹿兔,走兽飞禽,各寻些来,打发他回
去。”山神、土地闻言,敢不承命;又问各要几何。大圣道:“不拘多少,取些来便
罢。”那各神即着本处阴兵,刮一阵聚兽阴风,捉了些野鸡山雉,角鹿肥獐,狐獾
兔,虎豹狼虫,共有百千余只,献与行者。行者道:“老孙不要。你可把他都捻
就了筋,单摆在那四十里路上两旁,教那些人不纵鹰犬,拿回城去,算了汝等之功。”
众神依言,散了阴风,摆在左右。

行者才按云头,对太子道:“殿下请回,路上已有物了,你自收去。”太子见他
在半空中弄此神通,如何不信,只得叩头拜别。出山门传了令,教军士们回城。只
见那路旁果有无限的野物,军士们不放鹰犬,一个个俱着手擒捉,喝采,俱道是千
岁殿下的洪福,怎知是老孙的神功?你听凯歌声唱,一拥回城。

这行者保护了三藏。那本寺中的和尚,见他们与太子这样绸缪,怎不恭敬。却
又安排斋供,管待了唐僧,依然还歇在禅堂里。将近有一更时分,行者心中有事,
急睡不着。他一毂辘爬起来,到唐僧床前,叫:“师父。”此时长老还未睡哩。他晓
得行者会失惊打怪的,推睡不应。行者摸着他的光头,乱摇道:“师父怎睡着了?”
唐僧怒道:“这个顽皮!这早晚还不睡,吆喝甚么?”行者道:“师父,有一桩事儿,
和你计较计较。”长老道:“甚么事?”行者道:“我日间与那太子夸口,说我的手
段比山还高,比海还深,拿那妖精如探囊取物一般,伸了手去就拿将转来。却也睡
不着,想起来,有些难哩。”唐僧道:“你说难,便就不拿了罢。”行者道:“拿是还
要拿,只是理上不顺。”唐僧道:“这猴头乱说!妖精夺了人君位,怎么叫做理上不
顺!”行者道:“你老人家只知念经拜佛,打坐参禅,那曾见那萧何的律法?常言道:
‘拿贼拿赃。’那怪物做了三年皇帝,又不曾走了马脚,漏了风声。他与三宫妃后
同眠,又和两班文武共乐,我老孙就有本事拿住他,也不好定个罪名。”唐僧道:“怎
么不好定罪?”行者道:“他就是个没嘴的葫芦,也与你滚上几滚。他敢道:‘我是
乌鸡国王。有甚逆天之事,你来拿我?’将甚执照与他折辩?”唐僧道:“凭你怎
生裁处?”

行者笑道:“老孙的计已成了。只是干碍着你老人家,有些儿护短。”唐僧道:
“我怎么护短?”行者道:“八戒生得夯,你有些儿偏向他。”唐僧道:“我怎么向
他?”行者道:“你若不向他啊,且如今把胆放大些,与沙僧只在这里。待老孙与
八戒趁此时先入那乌鸡国城中,寻着御花园,打开琉璃井,把那皇帝尸首捞将上来,
包在我们包袱里。明日进城,且不管甚么倒换文牒,见了那怪,掣棍子就打。他但
有言语,就将骨榇与他看,说:‘你杀的是这个人!’却教太子上来哭父,皇后出来
认夫,文武多官见主,我老孙与兄弟们动手;这才是有对头的官事好打。”唐僧闻
言,暗喜道:“只怕八戒不肯去。”行者笑道:“如何?我说你护短。你怎么就知他不
肯去?你只像我叫你时不答应,半个时辰便了!我这去,但凭三寸不烂之舌,莫说是
猪八戒,就是‘猪九戒’,也有本事教他跟着我走。”唐僧道:“也罢,随你去叫他。”

行者离了师父,径到八戒床边。叫:“八戒,八戒!”那呆子是走路辛苦的人,
丢倒头,只情打呼,那里叫得醒。行者揪着耳朵,抓着鬃,把他一拉,拉起来,叫
声“八戒”。那呆子还打挣。行者又叫一声,呆子道:“睡了罢,莫顽,明日要走
路哩!”行者道:“不是顽,有一桩买卖,我和你做去。”八戒道:“甚么买卖?”行
者道:“你可曾听得那太子说么?”八戒道:“我不曾见面,不曾听见说甚么。”行
者道:“那太子告诵我说,那妖精有件宝贝,万夫不当之勇。我们明日进朝,不免
与他争敌;倘那怪执了宝贝,降倒我们,却不反成不美,我想着打人不过,不如先
下手。我和你去偷他的来,却不是好?”八戒道:“哥哥,你哄我去做贼哩。这个
买卖,我也去得,果是晓得实实的帮寸。我也与你讲个明白:偷了宝贝,降了妖精,
我却不奈烦甚么小家罕气的分宝贝,我就要了。”行者道:“你要作甚?”八戒道:
“我不如你们乖巧能言,人面前化得出斋来;老猪身子又夯,言语又粗,不能念经,
若到那无济无生处,可好换斋吃么?”行者道:“老孙只要图名,那里图甚宝贝,
就与你罢便了。”那呆子听见说都与他,他就满心欢喜,一毂辘爬将起来,套上衣
服,就和行者走路。这正是清酒红人面,黄金动道心。两个密密的开了门,躲离三
藏,纵祥光,径奔那城。

不多时到了,按落云头,只听得楼头方二鼓矣。行者道:“兄弟,二更时分了。”
八戒道:“正好,正好,人都在头觉里正浓睡也。”二人不奔正阳门,径到后宰门首,
只听得梆铃声响。行者道:“兄弟,前后门皆紧急,如何得入?”八戒道:“那见做
贼的从门里走么?瞒墙跳过便罢。”行者依言,将身一纵,跳上里罗城墙。八戒也跳
上去。二人潜入里面,找着门路,径寻那御花园。

正行时,只见有一座三檐白簇的门楼,上有三个亮灼灼的大字,映着那星月光
辉,乃是“御花园”。行者近前看了,有几重封皮,公然将锁门锈住了。即命八戒
动手,那呆子掣铁钯,尽力一筑,把门筑得粉碎。行者先举步入,忍不住跳将起
来,大呼小叫。唬得八戒上前扯住道:“哥呀,害杀我也!那见做贼的乱嚷,似这般
吆喝!惊醒了人,把我们拿住,送入官司,就不该死罪,也要解回原籍充军。”行者
道:“兄弟啊,你却不知我发急为何?你看这:

彩画雕栏狼狈,宝妆亭阁歪。莎汀蓼岸尽尘埋,芍药荼俱败。茉莉玫瑰香
暗,牡丹百合空开。芙蓉木槿草垓垓,异卉奇葩壅坏。

巧石山峰俱倒,池塘水涸鱼衰。青松紫竹似干柴,满路茸茸蒿艾。丹桂碧桃枝
损,海榴棠棣根歪。桥头曲径有苍苔,冷落花园境界!”

八戒道:“且叹他做甚?快干我们的买卖去来!”行者虽然感慨,却留心想起唐
僧的梦来,说芭蕉树下方是井。正行处,果见一株芭蕉,生得茂盛,比众花木不同。
真是:

一种灵苗秀,天生体性空。枝枝抽片纸,叶叶卷芳丛。翠缕千条细,丹心一点
红。凄凉愁夜雨,憔悴怯秋风。长养元丁力,栽培造化工。缄书成妙用,挥洒有奇
功。凤翎宁得似,鸾尾迥相同。薄露滴,轻烟淡淡笼。青阴遮户牖,碧影上帘
栊。不许栖鸿雁,何堪系玉骢。霜天形槁悴,月夜色朦胧。仅可消炎暑,犹宜避日
烘。愧无桃李色,冷落粉墙东。
行者道:“八戒,动手么!宝贝在芭蕉树下埋着哩。”那呆子双手举钯,筑倒了芭蕉,
然后用嘴一拱,拱了有三四尺深,见一块石板盖住。呆子欢喜道:“哥呀,造化了,
果有宝贝!是一片石板盖着哩!不知是坛儿盛着,是柜儿装着哩。”行者道:“你掀起
来看看。”那呆子果又一嘴,拱开看处,又见有霞光灼灼,白气明明。八戒笑道:“造
化!造化!宝贝放光哩!”又近前细看时,呀!原来是星月之光,映得那井中水亮。八
戒道:“哥呀,你但干事,便要留根。”行者道:“我怎留根?”八戒道:“这是一眼
井。你在寺里,早说是井中有宝贝,我却带将两条捆包袱的绳来,怎么作个法儿,
把老猪放下去;如今空手,这里面东西,怎么得下去上来耶?”行者道:“你下去
么?”八戒道:“正是要下去,只是没绳索。”行者笑道:“你脱了衣服,我与你个
手段。”八戒道:“有甚么好衣服?解了这直裰子就是了。”

好大圣,把金箍棒拿出来,两头一扯,叫“长!”足有七八丈长。教:“八戒,
你抱着一头儿,把你放下井去。”八戒道:“哥呀,放便放下去,若到水边,就住了
罢。”行者道:“我晓得。”那呆子抱着铁棒,被行者轻轻提将起来,将他放下去。
不多时,放至水边。八戒道:“到水了!”行者听见他说,却将棒往下一按。那呆子
扑通的一个没头蹲,丢了铁棒,便就负水,口里哺哺的嚷道:“这天杀的!我说到水
莫放,他却就把我一按!”行者掣上棒来。笑道:“兄弟,可有宝贝么?”八戒道:
“见甚么宝贝,只是一井水!”行者道:“宝贝沉在水底下哩。你下去摸一摸来。”
呆子真个深知水性,却就打个猛子,淬将下去。呀!那井底深得紧,他却着实又一
淬,忽睁眼见有一座牌楼,上有“水晶宫”三个字。八戒大惊道:“罢了,罢了,
错走了路了,下海来也!海内有个水晶宫,井里如何有之?”原来八戒不知此是
井龙王的水晶宫。

八戒正叙话处,早有一个巡水的夜叉,开了门,看见他的模样,急抽身进去报
道:“大王,祸事了!井上落一个长嘴大耳的和尚来了!赤淋淋的,衣服全无,还不
死,逼法说话哩。”那井龙王忽闻此言,心中大惊道:“这是天蓬元帅来也。昨夜夜
游神奉上敕旨,来取乌鸡国王魂灵去拜见唐僧,请齐天大圣降妖。这怕是齐天大圣、
天蓬元帅来了。却不可怠慢他,快接他去也。”

那龙王整衣冠,领众水族,出门来厉声高叫道:“天蓬元帅,请里面坐。”八戒
却才欢喜道:“原来是个故知。”那呆子不管好歹,径入水晶宫里。其实不知上下,
赤淋淋的,就坐在上面。龙王道:
“元帅,近闻你得了性命,皈依释教,保唐僧西天取经,如何得到此处?”八戒道:
“正为此说。我师兄孙悟空多多拜上,着我来问你取甚么宝贝哩。”龙王道:“可怜,
我这里怎么得个宝贝!比不得那江、河、淮、济的龙王,飞腾变化,便有宝贝。我
久困于此,日月且不能长见,宝贝果何自而来也?”八戒道:“不要推辞,有便拿
出来罢。”龙王道:“有便有一件宝贝,只是拿不出来;就元帅亲自来看看,何如?”
八戒道:“妙,妙,妙!须是看看来也。”

那龙王前走,这呆子随后。转过了水晶宫殿,只见廊庑下,横着一个六尺长
躯。龙王用手指定道:“元帅,那厢就是宝贝了。”八戒上前看了,呀!原来是个死
皇帝,戴着冲天冠,穿着赭黄袍,踏着无忧履,系着蓝田带,直挺挺睡在那厢。八
戒笑道:“难,难,难!算不得宝贝。想老猪在山为怪时,时常将此物当饭;且莫说
见的多少,吃也吃够无数,那里叫做甚么宝贝。”龙王道:“元帅原来不知。他本是
乌鸡国王的尸首;自到井中,我与他定颜珠定住,不曾得坏。你若肯驮他出去,见
了齐天大圣,假有起死回生之意啊,莫说宝贝,凭你要甚么东西都有。”八戒道:“既
这等说,我与你驮出去,只说把多少烧埋钱与我?”龙王道:“其实无钱。”八戒道:
“你好白使人?果然没钱,不驮!”龙王道:“不驮,请行。”八戒就走。龙王差两个
有力量的夜叉,把尸抬将出去,送到水晶宫门外,丢在那厢,摘了辟水珠,就有水
响。

八戒急回头看,不见水晶宫门,一把摸着那皇帝的尸首,慌得他脚软筋麻,撺
出水面,扳着井墙,叫道:“师兄!伸下棒来救我一救!”行者道:“可有宝贝么?”
八戒道:“那里有!只是水底下有一个井龙王,教我驮死人;我不曾驮,他就把我送
出门来,就不见那水晶宫了,只摸着那个尸首。唬得我手软筋麻,挣搓不动了!哥
呀,好歹救我救儿!”行者道:“那个就是宝贝,如何不驮上来?”八戒道:“知他
死了多少时了,我驮他怎的?”行者道:“你不驮,我回去耶。”八戒道:“你回那
里去?”行者道:“我回寺中,同师父睡觉去。”八戒道:“我就不去了?”行者道:
“你爬得上来,便带你去,爬不上来,便罢。”八戒慌了:“怎生爬得动!你想,城
墙也难上,这井肚子大,口儿小,壁陡的圈墙,又是几年不曾打水的井,团团都长
的是苔痕,好不滑也,教我怎爬?哥哥,不要失了兄弟们和气,等我驮上来罢。”
行者道:“正是。快快驮上来,我同你回去睡觉。”那呆子又一个猛子,淬将下去,
摸着尸首,拽过来,背在身上,撺出水面。扶井墙道:“哥哥,驮上来了。”那行者
睁睛看处,真个的背在身上。却才把金箍棒伸下井底,那呆子着了恼的人,张开口,
咬着铁棒,被行者轻轻的提将出来。

八戒将尸放下,捞过衣服穿了。行者看时,那皇帝容颜依旧,似生时未改分毫。
行者道:“兄弟啊,这人死了三年,怎么还容颜不坏?”八戒道:“你不知之。这井
龙王对我说,他使了定颜珠定住了,尸首未曾坏得。”行者道:“造化,造化!一则
是他的冤仇未报,二来该我们成功。兄弟快把他驮了去。”八戒道:“驮往那里去?”
行者道:“驮了去见师父。”八戒口中作念道:“怎的起,怎的起!好好睡觉的人,被
这猢狲花言巧语,哄我教做甚么买卖,如今却干这等事,教我驮死人!驮着他,腌
臭水淋将下来,污了衣服,没人与我浆洗。上面有几个补丁,天阴发潮,如何穿
么?”行者道:“你只管驮了去,到寺里,我与你换衣服。”八戒道:“不羞!连你穿
的也没有,又替我换!”行者道:“这般弄嘴,便不驮罢!”八戒道:“不驮!”——
“便伸过孤拐来,打二十棒!”八戒慌了道:“哥哥,那棒子重,若是打上二十,我
与这皇帝一般了。”行者道:“怕打时,趁早儿驮着走路!”八戒果然怕打。没好气,
把尸首拽将过来,背在身上,拽步出园就走。

好大圣,捻着诀,念声咒语,往巽地上吸一口气,吹将去,就是一阵狂风,把
八戒撮出皇宫内院,躲离了城池,息了风头,二人落地,徐徐却走将来。那呆子心
中暗恼,算计要恨报行者,道:“这猴子捉弄我,我到寺里也捉弄他捉弄,撺唆师
父,只说他医得活;医不活,教师父念紧箍儿咒,把这猴子的脑浆勒出来,方趁我
心!”走着路,再再寻思道:“不好!不好!若教他医人,却是容易:他去阎王家讨将
魂灵儿来,就医活了。只说不许赴阴司,阳世间就能医活,这法儿才好。”

说不了,却到了山门前,径直进去,将尸首丢在那禅堂门前,道:“师父,起
来看邪。”那唐僧睡不着,正与沙僧讲行者哄了八戒去久不回之事。忽听得他来叫
了一声,唐僧连忙起身道:“徒弟,看甚么?”八戒道:“行者的外公,教老猪驮将
来了。”行者道:“你这馕糟的呆子!我那里有甚么外公。”八戒道:“哥,不是你外
公,却教老猪驮他来怎么?也不知费了多少力了!”

那唐僧与沙僧开门看处,那皇帝容颜未改,似活的一般。长老忽然惨凄道:“陛
下,你不知那世里冤家,今生遇着他,暗丧其身,抛妻别子,致令文武不知,多官
不晓!可怜你妻子昏蒙,谁曾见焚香献茶?”忽失声泪如雨下。八戒笑道:“师父,
他死了可干你事?又不是你家父祖,哭他怎的!”三藏道:“徒弟啊,出家人慈悲为
本,方便为门。你怎的这等心硬?”八戒道:“不是心硬;师兄和我说来,他能医得
活。若是医不活,我也不驮他来了。”

那长老原来是一头水的,被那呆子摇动了,也便就叫:“悟空,若果有手段医
活这个皇帝,正是‘救人一命,胜造七级浮图。’我等也强似灵山拜佛。”行者道:
“师父,你怎么信这呆子乱谈!人若死了,或三七五七,尽七七日,受满了阳间罪
过,就转生去了。如今已死三年,如何救得!”三藏闻其言道:“也罢了。”八戒苦
恨不息。道:“师父,你莫被他瞒了。他有些夹脑风。你只念念那话儿,管他还你
一个活人。”真个唐僧就念紧箍儿咒,勒得那猴子眼胀头疼。

毕竟不知怎生医救,且听下回分解。