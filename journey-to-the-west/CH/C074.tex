\chapter{长庚传报魔头狠~行者施为变化能}

情欲原因总一般,有情有欲自如然。沙门修炼纷纷士,断欲忘情即是禅。须着
意,要心坚,一尘不染月当天。行功进步休教错,行满功完大觉仙。

话表三藏师徒们打开欲网,跳出情牢,放马西行。走多时,又是夏尽秋初,新
凉透体。但见那:
急雨收残暑,梧桐一叶惊。
萤飞莎径晚,蛩语月华明。
黄葵开映露,红蓼遍沙汀。
蒲柳先零落,寒蝉应律鸣。
三藏正然行处,忽见一座高山,峰插碧空,真个是摩星碍日。长老心中害怕,叫悟
空道:“我看前面这山,十分高耸,但不知有路通行否。”行者笑道:“师父说那里
话。自古道:‘山高自有客行路,水深自有渡船人。’岂无通达之理?可放心前去。”
长老闻言,喜笑花生,扬鞭策马而进,径上高岩。

行不数里,见一老者,鬓蓬松,白发飘搔;须稀朗,银丝摆动;项挂一串数珠
子,手持拐杖现龙头;远远的立在那山坡上高呼:“西进的长老,且暂住骅骝,紧
兜玉勒。这山上有一伙妖魔,吃尽了阎浮世上人,不可前进!”三藏闻言,大惊失
色。一是马的足下不平,二是坐个雕鞍不稳,扑的跌下马来,挣挫不动,睡在草里
哼哩。行者近前搀起道:“莫怕,莫怕!有我哩!”长老道:“你听那高岩上老者,报
道这山上有伙妖魔,吃尽阎浮世上人,谁敢去问他一个真实端的?”行者道:“你
且坐地,等我去问他。”三藏道:“你的相貌丑陋,言语粗俗,怕冲撞了他,问不出
个实信。”行者笑道:“我变个俊些儿的去问他。”三藏道:“你是变了我看。”好大
圣,捻着诀,摇身一变,变做个干干净净的小和尚儿,真个是目秀眉清,头圆脸正;
行动有斯文之气象,开口无俗类之言辞。抖一抖锦衣直裰,拽步上前。向唐僧道:
“师父,我可变得好么?”三藏见了大喜道:“变得好!”八戒道:“怎么不好!只是
把我们都比下去了。老猪就滚上二三年,也变不得这等俊俏!”

好大圣,躲离了他们,径直近前,对那老者躬身道:“老公公,贫僧问讯了。”
那老儿见他生得俊雅,年少身轻,待答不答的,还了他个礼,用手摸着他头儿,笑
嘻嘻问道:“小和尚,你是那里来的?”行者道:“我们是东土大唐来的,特上西天
拜佛求经。适到此间,闻得公公报道有妖怪,我师父胆小怕惧,着我来问一声:端
的是甚妖精,他敢这般短路!烦公公细说与我知之,我好把他贬解起身。”那老儿笑
道:“你这小和尚年幼,不知好歹,言不帮衬。那妖魔神通广大得紧,怎敢就说贬
解他起身!”行者笑道:“据你之言,似有护他之意,必定与他有亲,或是紧邻契友;
不然,怎么长他的威智,兴他的节概,不肯倾心吐胆说他个来历。”公公点头笑道:
“这和尚倒会弄嘴!想是跟你师父游方,到处儿学些法术,或者会驱缚魍魉,与人
家镇宅降邪,你不曾撞见十分狠怪哩!”行者道:“怎的狠?”公公道:“那妖精一
封书到灵山,五百阿罗都来迎接;一纸简上天宫,十一大曜个个相钦。四海龙曾与
他为友,八洞仙常与他作会。十地阎君以兄弟相称,社令、城隍以宾朋相爱。”

大圣闻言,忍不住呵呵大笑,用手扯着老者道:“不要说,不要说。那妖精与
我后生小厮为兄弟、朋友,也不见十分高作。若知是我小和尚来啊,他连夜就搬起
身去了!”公公道:“你这小和尚胡说!不当人子。那个神圣是你的后生小厮?”行
者笑道:“实不瞒你说。我小和尚祖居傲来国花果山水帘洞,姓孙,名悟空。当年
也曾做过妖精,干过大事。曾因会众魔,多饮了几杯酒睡着,梦中见二人将批勾我
去到阴司。一时怒发,将金箍棒打伤鬼判,唬倒阎王,几乎掀翻了森罗殿。吓得那
掌案的判官拿纸,十阎王佥名画字,教我饶他打,情愿与我做后生小厮。”那公公
闻说道:“阿弥陀佛!这和尚说了这过头话,莫想再长得大了。”行者道:“官儿,似
我这般大也够了。”公公道:“你年几岁了?”行者道:“你猜猜看。”老者道:“有
七八岁罢了。”行者笑道:“有一万个七八岁!我把旧嘴脸拿出来你看看,你即莫怪。”
公公道:“怎么又有个嘴脸?”行者道:“我小和尚有七十二副嘴脸哩。”

那公公不识窍,只管问他,他就把脸抹一抹,即现出本象,咨牙嘴,两股通
红,腰间系一条虎皮裙,手里执一根金箍棒,立在石崖之下,就像个活雷公。那老
者见了,吓得面容失色,腿脚酸麻,站不稳,扑的一跌;爬起来,又一个踵。大
圣上前道:“老官儿,不要虚惊。我等山恶人善。莫怕,莫怕!适间蒙你好意,报有
妖魔。委的有多少怪,一发累你说说,我好谢你。”那老儿战战兢兢,口不能言,
又推耳聋,一句不应。

行者见他不言,即抽身回坡。长老道:“悟空,你来了?所问如何?”行者笑道:
“不打紧,不打紧!西天有便有个把妖精儿,只是这里人胆小,把他放在心上。没
事,没事,有我哩!”长老道:“你可曾问他此处是甚么山,甚么洞,有多少妖怪,
那条路通得雷音?”八戒道:“师父,莫怪我说。若论赌变化,使捉掐,捉弄人,
我们三五个也不如师兄;若论老实,像师兄就摆一队伍,也不如我。”唐僧道:“正
是,正是,你还老实。”八戒道:“他不知怎么钻过头不顾尾的,问了两声,不尴不
尬的就跑回来了。等老猪去问他个实信来。”唐僧道:“悟能,你仔细着。”

好呆子,把钉钯撒在腰里,整一整皂直裰,扭扭捏捏,奔上山坡,对老者叫道:
“公公,唱喏了。”那老儿见行者回去,方拄着杖挣得起来,战战兢兢的要走,忽
见八戒,愈觉惊怕道:“爷爷呀!今夜做的甚么恶梦,遇着这伙恶人!为先的那和尚
丑便丑,还有三分人相;这个和尚,怎么这等个碓梃嘴,蒲扇耳朵,铁片脸,毛
颈项,一分人气儿也没有了!”八戒笑道:“你这老公公不高兴,有些儿好褒贬人。
你是怎的看我哩?丑便丑,奈看,再停一时就俊了。”那老者见他说出人话来,只得
开言问他:“你是那里来的?”八戒道:“我是唐僧第二个徒弟,法名叫做悟能八戒。
才自先问的,叫做悟空行者,是我师兄。师父怪他冲撞了公公,不曾问得实信,所
以特着我来拜问。此处果是甚山、甚洞,洞里果是甚妖精,那里是西去大路,烦尊
一指示指示。”老者道:“可老实么?”八戒道:“我生平不敢有一毫虚的。”老者道:
“你莫像才来的那个和尚走花弄水的胡缠。”八戒道:“我不像他。”

公公拄着杖,对八戒说:“此山叫做八百里狮驼岭。中间有座狮驼洞。洞里有
三个魔头。”八戒啐了一声:“你这老儿却也多心!三个妖魔,也费心劳力的来报遭
信!”公公道:“你不怕么?”八戒道:“不瞒你说。这三个妖魔,我师兄一棍就打
死一个,我一钯就筑死一个;我还有个师弟,他一降妖杖又打死一个:三个都打死,
我师父就过去了,有何难哉!”那老者笑道:“这和尚不知深浅!那三个魔头,神通
广大得紧哩!他手下小妖,南岭上有五千,北岭上有五千;东路口有一万,西路口
有一万;巡哨的有四五千,把门的也有一万;烧火的无数,打柴的也无数:共计算
有四万七八千。这都是有名字带牌儿的,专在此吃人。”

那呆子闻得此言,战兢兢跑将转来,相近唐僧,且不回话,放下钯,在那里出
恭。行者见了,喝道:“你不回话,却蹲在那里怎的?”八戒道:“唬出屎来了!如
今也不消说,赶早儿各自顾命去罢!”行者道:“这个呆根!我问信偏不惊恐,你去
问就这等慌张失智!”长老道:“端的何如?”八戒道:“这老儿说:此山叫做八百
里狮驼山。中间有座狮驼洞。洞里有三个老妖,有四万八千小妖,专在那里吃人。
我们若着他些山边儿,就是他口里食了。莫想去得!”

三藏闻言,战兢兢,毛骨悚然,道:“悟空,如何是好?”行者笑道:“师父放
心,没大事。想是这里有便有几个妖精,只是这里人胆小,把他就说出许多人,许
多大,所以自惊自怪。有我哩!”八戒道:“哥哥说的是那里话!我比你不同:我问
的是实,决无虚谬之言。满山满谷都是妖魔,怎生前进?”行者笑道:“呆子嘴脸,
不要虚惊!若论满山满谷之魔,只消老孙一路棒,半夜打个罄尽!”八戒道:“不羞,
不羞,莫说大话。那些妖精点卯也是七八日,怎么就打得罄尽?”行者道:“你说
怎样打?”八戒道:“凭你抓倒,捆倒,使定身法定倒,也没有这等快的。”行者笑
道:“不用甚么抓拿捆缚。我把这棍子两头一扯,叫‘长’!就有四十丈长短;幌一
幌,叫‘粗!’就有八丈围圆粗细。往山南一滚,滚杀五千;山北一滚,滚杀五千;
从东往西一滚,只怕四五万砑做肉泥烂酱!”八戒道:“哥哥,若是这等赶面打,或
者二更时也都了了。”沙僧在旁笑道:“师父,有大师兄恁样神通,怕他怎的!请上
马走啊。”唐僧见他们讲论手段,没奈何,只得宽心上马而走。

正行间,不见了那报信的老者。沙僧道:“他就是妖怪,故意狐假虎威的来传
报,恐唬我们哩。”行者道:“不要忙,等我去看看。”好大圣,跳上高峰,四顾无
迹,急转面,见半空中有彩霞幌亮,即纵云赶上看时,乃是太白金星。走到身边,
用手扯住,口口声声只叫他的小名道:“李长庚,李长庚,你好惫懒。有甚话,当
面来说便好;怎么装做个山林之老,魇样混我!”金星慌忙施礼道:“大圣,报信来
迟,乞勿罪,乞勿罪!这魔头果是神通广大,势要峥嵘,只看你挪移变化,乖巧机
谋,可便过去;如若怠慢些儿,其实难去。”行者谢道:“感激,感激。果然此处难
行,望老星上界与玉帝说声,借些天兵帮助老孙帮助。”金星道:“有,有,有,你
只口信带去,就是十万天兵,也是有的。”

大圣别了金星,按落云头,见了三藏道:“适才那个老儿,原是太白星来与我
们报信的。”长老合掌道:“徒弟,快赶上他,问他那里另有个路,我们转了去罢。”
行者道:“转不得。此山径过有八百里,四周围不知更有多少路哩。怎么转得?”
三藏闻言,止不住眼中流泪道:“徒弟,似此艰难,怎生拜佛!”行者道:“莫哭,
莫哭,一哭便脓包行了!他这报信,必有几分虚话,只是要我们着意留心,诚所谓
‘以告者,过也。’你且下马来坐着。”八戒道:“又有甚商议?”行者道:“没甚商
议。你且在这里用心保守师父。沙僧好生看守行李、马匹。等老孙先上岭打听打听,
看前后共有多少妖怪,拿住一个,问他个详细,教他写个执结,开个花名,把他老
老小小,一一查明,吩咐他关了洞门,不许阻路,却请师父静静悄悄的过去,方显
得老孙手段!”沙僧只教:“仔细,仔细!”行者笑道:“不消嘱咐。我这一去,就是
东洋大海也荡开路,就是铁裹银山也撞透门!”

好大圣,唿哨一声,纵筋斗云,跳上高峰。扳藤负葛,平山观看,那山里静悄
无人。忽失声道:“错了,错了!不该放这金星老儿去了。他原来恐唬我。这里那有
个甚么妖精!他就出来跳风顽耍,必定拈枪弄棒,操演武艺;如何没有一个?”正
自家揣度,只听得山背后,叮叮当当,辟辟剥剥,梆铃之声。急回头看处,原来是
个小妖儿,掮着一杆“令”字旗,腰间悬着铃子,手里敲着梆子,从北向南而走。
仔细看他,有一丈二尺的身子。行者暗笑道:“他必是个铺兵。想是送公文下报帖
的。且等我去听他一听,看他说些甚话。”

好大圣,捻着诀,念个咒,摇身一变,变做个苍蝇儿,轻轻飞在他帽子上,侧
耳听之。只见那小妖走上大路,敲着梆,摇着铃,口里作念道:“我等寻山的,各
人要谨慎提防孙行者:他会变苍蝇!”行者闻言,暗自惊疑道:“这厮看见我了;若
未看见,怎么就知我的名字,又知我会变苍蝇!”原来那小妖也不曾见他,只是那
魔头不知怎么就吩咐他这话,却是个谣言,着他这等胡念。行者不知,反疑他看见,
就要取出棒来打他,却又停住,暗想道:“曾记得八戒问金星时,他说老妖三个,
小妖有四万七八千名。似这小妖,再多几万,也不打紧,却不知这三个老魔有多大
手段。等我问他一问,动手不迟。”

好大圣!你道他怎么去问:跳下他的帽子来,钉在树头上,让那小妖先行几步,
急转身腾那,也变做个小妖儿,照依他敲着梆,摇着铃,掮着旗,一般衣服,只是
比他略长了三五寸,口里也那般念着,赶上前叫道:“走路的,等我一等。”那小妖
回头道:“你是那里来的?”行者笑道:“好人呀,一家人也不认得!”小妖道:“我
家没你呀。”行者道:“怎的没我?你认认看。”小妖道:“面生,认不得,认不得。”
行者道:“可知道面生。我是烧火的,你会得我少。”小妖摇头道:“没有,没有,
我洞里就是烧火的那些兄弟,也没有这个嘴尖的。”行者暗想道:“这个嘴好的变尖
了些了。”即低头,把手侮着嘴揉一揉道:“我的嘴不尖啊。”真个就不尖了。那小
妖道:“你刚才是个尖嘴,怎么揉一揉就不尖了?疑惑人子,大不好认。不是我一家
的,少会,少会!可疑,可疑!我那大王家法甚严,烧火的只管烧火,巡山的只管巡
山,终不然教你烧火,又教你来巡山?”行者口乖,就趁过来道:“你不知道。大
王见我烧得火好,就升我来巡山。”

小妖道:“也罢,我们这巡山的,一班有四十名,十班共四百名,各自年貌,
各自名色。大王怕我们乱了班次,不好点卯,一家与我们一个牌儿为号。你可有牌
儿?”行者只见他那般打扮,那般报事,遂照他的模样变了;因不曾看见他的牌儿,
所以身上没有。好大圣,更不说没有,就满口应承道:“我怎么没牌?但只是刚才领
的新牌。拿你的出来我看。”

那小妖那里知这个机括,即揭起衣服,贴身带着个金漆牌儿,穿条绒线绳儿,
扯与行者看看。行者见那牌背是个“威镇诸魔”的金牌,正面有三个真字,是“小
钻风”,他却心中暗想道:“不消说了!但是巡山的,必有个‘风’字坠脚。”便道:
“你且放下衣走过,等我拿牌儿你看。”即转身,插下手,将尾巴梢儿的小毫毛拔
下一根,捻他把,叫“变”!即变做个金漆牌儿,也穿上个绿绒绳儿,上书三个真
字,乃“总钻风”,拿出来,递与他看了。小妖大惊道:“我们都叫做个小钻风,偏
你又叫做个甚么‘总钻风’?”行者干事找绝,说话合宜,就道:“你实不知。大
王见我烧得火好,把我升个巡风;又与我个新牌,叫做‘总巡风’。教我管你这一
班四十名兄弟也。”那妖闻言,即忙唱喏道:“长官,长官,新点出来的,实是面生。
言语冲撞,莫怪!”行者还着礼笑道:“怪便不怪你,只是一件:见面钱却要哩。每
人拿出五两来罢。”小妖道:“长官不要忙,待我向南岭头会了我这一班的人,一总
打发罢。”行者道:“既如此,我和你同去。”那小妖真个前走,大圣随后相跟。

不数里,忽见一座笔峰。何以谓之笔峰?那山头上长出一条峰来,约有四五丈
高,如笔插在架上一般,故以为名。行者到边前,把尾巴掬一掬,跳上去,坐在峰
尖儿上。叫道:“钻风,都过来!”那些小钻风在下面躬身道:“长官,伺候。”行者
道:“你可知大王点我出来之故?”小妖道:“不知。”行者道:“大王要吃唐僧,只
怕孙行者神通广大,说他会变化,只恐他变作小钻风,来这里着路径,打探消息,
把我升作总钻风,来查勘你们这一班可有假的。”小钻风连声应道:“长官,我们俱
是真的。”行者道:“你既是真的,大王有甚本事,你可晓得?”小钻风道:“我晓
得。”行者道:“你晓得,快说来我听。如若说得合着我,便是真的;若说差了一些
儿,便是假的。我定拿去见大王处治。”

那小钻风见他坐在高处,弄獐弄智,呼呼喝喝的,没奈何,只得实说道:“我
大王神通广大,本事高强,一口曾吞了十万天兵。”行者闻说,吐出一声道:“你是
假的!”小钻风慌了道:“长官老爷,我是真的,怎么说是假的?”行者道:“你既
是真的,如何胡说!大王身子能有多大,一口都吞了十万天兵?”小钻风道:“长官
原来不知。我大王会变化:要大能撑天堂,要小就如菜子。因那年王母娘娘设蟠桃
大会,邀请诸仙,他不曾具柬来请,我大王意欲争天,被玉皇差十万天兵来降我大
王:是我大王变化法身,张开大口,似城门一般,用力吞将去,唬得众天兵不敢交
锋,关了南天门:故此是一口曾吞十万兵。”

行者闻言暗笑道:“若是讲手头之话,老孙也曾干过。”又应声道:“二大王有
何本事?”小钻风道:“二大王身高三丈,卧蚕眉,丹凤眼,美人声,匾担牙,鼻
似蛟龙。若与人争斗,只消一鼻子卷去,就是铁背铜身,也就魂亡魄丧!”行者道:
“鼻子卷人的妖精也好拿。”

又应声道:“三大王也有几多手段?”小钻风道:“我三大王不是凡间之怪物,
名号云程万里鹏,行动时,抟风运海,振北图南。随身有一件儿宝贝,唤做‘阴阳
二气瓶’。假若是把人装在瓶中,一时三刻,化为浆水。”

行者听说,心中暗惊道:“妖魔倒也不怕,只是仔细防他瓶儿。”又应声道:“三
个大王的本事,你倒也说得不差,与我知道的一样;但只是那个大王要吃唐僧哩?”
小钻风道:“长官,你不知道?”行者喝道:“我比你不知些儿!因恐汝等不知底细,
吩咐我来着实盘问你哩!”小钻风道:“我大大王与二大王久住在狮驼岭狮驼洞。三
大王不在这里住。他原住处离此西下有四百里远近。那厢有座城,唤做狮驼国。他
五百年前吃了这城国王及文武官僚,满城大小男女也尽被他吃了干净,因此上夺了
他的江山。如今尽是些妖怪。不知那一年打听得东土唐朝差一个僧人去西天取经,
说那唐僧乃十世修行的好人,有人吃他一块肉,就延寿长生不老;只因怕他一个徒
弟孙行者十分利害,自家一个难为,径来此处与我这两个大王结为兄弟,合意同心,
打伙儿捉那个唐僧也。”

行者闻言,心中大怒道:“这泼魔十分无礼!我保唐僧成正果,他怎么算计要吃
我的人!”恨一声,咬响钢牙,掣出铁棒,跳下高峰,把棍子望小妖头上砑了一砑,
可怜,就砑得像一个肉陀!自家见了,又不忍道:“咦!他倒是个好意,把些家常话
儿都与我说了,我怎么却这一下子就结果了他?也罢,也罢,左右是左右!”好大圣,
只为师父阻路,没奈何干出这件事来。就把他牌儿解下,带在自家腰里,将“令”
字旗掮在背上,腰间挂了铃,手里敲着梆子,迎风捻个诀,口里念个咒语,摇身一
变,变的就像小钻风模样;拽回步,径转旧路,找寻洞府,去打探那三个老妖魔的
虚实。这正是:
千般变化美猴王,万样腾那真本事!

闯入深山,依着旧路,正走处,忽听得人喊马嘶之声,即举目观之,原来是狮
驼洞口有万数小妖排列着枪刀剑戟,旗帜旌旄。这大圣心中暗喜道:“李长庚之言,
真是不妄,真是不妄!”原来这摆列的有些路数:二百五十名作一大队伍。他只见
有四十名杂彩长旗,迎风乱舞,就知有万名人马;却又自揣自度道:“老孙变作小
钻风,这一进去,那老魔若问我巡山的话,我必随机答应。倘或一时言语差讹,认
得我啊,怎生脱体?就要往外跑时,那伙把门的挡住,如何出得门去?要拿洞里妖王,
必先除了门前众怪!”

你道他怎么除得众怪?好大圣,想着:“那老魔不曾与我会面,就知我老孙的名
头,我且倚着我的这个名头,仗着威风,说些大话,吓他一吓看。果然中土众僧有
缘有分,取得经回,这一去,只消我几句英雄之言,就吓退那门前若干之怪:假若
众僧无缘无分,取不得真经啊,就是纵然说得莲花现,也除不得西方洞外精。”心
问口,口问心,思量此计,敲着梆,摇着铃,径直闯到狮驼洞口,早被前营上小妖
挡住道:“小钻风来了?”行者不应,低着头就走。

走至二层营里,又被小妖扯住道:“小钻风来了?”行者道:“来了。”众妖道:
“你今早巡风去,可曾撞见甚么孙行者么?”行者道:“撞见的。正在那里磨扛子
哩。”众妖害怕道:“他怎么个模样?磨甚么扛子?”行者道:“他蹲在那涧边,还似
个开路神;若站起来,好道有十数丈长!手里拿着一条铁棒,就似碗来粗细的一根
大扛子,在那石崖上抄一把水,磨一磨,口里又念着:‘扛子啊!这一向不曾拿你出
来显显神通,这一去就有十万妖精,也都替我打死!等我杀了那三个魔头祭你!’他
要磨得明了,先打死你门前一万妖精哩!”那些小妖闻得此言,一个个心惊胆战,
魂散魄飞。行者又道:“列位,那唐僧的肉也不多几斤,也分不到我处,我们替他
顶这个缸怎的!不如我们各自散一散罢。”众妖都道:“说得是。我们各自顾命去来。”
假若是些军民人等,服了圣化,就死也不敢走。原来此辈都是些狼虫虎豹,走兽飞
禽,呜的一声,都哄然而去了。这个倒不像孙大圣几句铺头话,却就如楚歌声吹散
了八千兵!行者暗自喜道:“好了,老妖是死了,闻言就走,怎敢觌面相逢?这进去
还似此言方好;若说差了,才这伙小妖有一两个倒走进去听见,却不走了风汛?……”
你看他:
存心来古洞,仗胆入深门。

毕竟不知见那个老魔头有甚吉凶,且听下回分解。