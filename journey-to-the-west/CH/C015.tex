\chapter{蛇盘山诸神暗佑~鹰愁涧意马收缰}

却说行者伏侍唐僧西进,行经数日,正是那腊月寒天,朔风凛凛,滑冻凌凌;
去的是些悬崖峭壁崎岖路,迭岭层峦险峻山。三藏在马上,遥闻唿喇喇水声聒耳,
回头叫:“悟空,是那里水响?”行者道:“我记得此处叫做蛇盘山鹰愁涧,想必
是涧里水响。”说不了,马到涧边,三藏勒缰观看。但见:
涓涓寒脉穿云过,湛湛清波映日红。
声摇夜雨闻幽谷,彩发朝霞眩太空。
千仞浪飞喷碎玉,一泓水响吼清风。
流归万顷烟波去,鸥鹭相忘没钓逢。
师徒两个正然看处,只见那涧当中响一声,钻出一条龙来,推波掀浪,撺出崖山,
就抢长老。慌得个行者丢了行李,把师父抱下马来,回头便走。那条龙就赶不上,
把他的白马连鞍辔一口吞下肚去,依然伏水潜踪。行者把师父送在那高阜上坐了,
却来牵马挑担,止存得一担行李,不见了马匹。他将行李担送到师父面前道:“师
父,那孽龙也不见踪影,只是惊走我的马了。”三藏道:“徒弟啊,却怎生寻得马着
么?”行者道:“放心,放心,等我去看来。”

他打个唿哨,跳在空中。火眼金睛,用手搭凉篷,四下里观看,更不见马的踪
迹。按落云头,报道:“师父,我们的马断乎是那龙吃了,四下里再看不见。”三藏
道:“徒弟呀,那厮能有多大口,却将那匹大马连鞍辔都吃了?想是惊张溜缰,走在
那山凹之中。你再仔细看看。”行者道:“你也不知我的本事。我这双眼,白日里常
看一千里路的吉凶。像那千里之内,蜻蜓儿展翅,我也看见,何期那匹大马,我就
不见!”三藏道:“既是他吃了,我如何前进!可怜啊!这万水千山,怎生走得!”说
着话,泪如雨落。

行者见他哭将起来,他那里忍得住暴燥,发声喊道:“师父莫要这等脓包形么!
你坐着!坐着!等老孙去寻着那厮,教他还我马匹便了!”三藏却才扯住道:“徒弟啊,
你那里去寻他?只怕他暗地里撺将出来,却不又连我都害了?那时节人马两亡,怎生
是好!”行者闻得这话,越加嗔怒,就叫喊如雷道:“你忒不济,不济!又要马骑,
又不放我去,似这般看着行李,坐到老罢!”

哏哏的吆喝,正难息怒,只听得空中有人言语,叫道:“孙大圣莫恼,唐御弟
休哭。我等是观音菩萨差来的一路神,特来暗中保取经者。”那长老闻言,慌忙
礼拜。行者道:“你等是那几个,可报名来,我好点卯。”众神道:“我等是六丁六
甲、五方揭谛、四值功曹、一十八位护教伽蓝,各各轮流值日听候。”行者道:“今
日先从谁起?”众揭谛道:“丁甲、功曹、伽蓝轮次。我五方揭谛,惟金头揭谛昼
夜不离左右。”行者道:“既如此,不当值者且退,留下六丁神将与日值功曹和众揭
谛保守着我师父。等老孙寻那涧中的孽龙,教他还我马来。”众神遵令。三藏才放
下心,坐在石崖之上,吩咐:“行者仔细。”行者道:“只管宽心。”好猴王,束一束
绵布直裰,撩起虎皮裙子,着金箍铁棒,抖擞精神,径临涧壑,半云半雾的,在
那水面上高叫道:“泼泥鳅,还我马来!还我马来!”

却说那龙吃了三藏的白马,伏在那涧底中间,潜灵养性。只听得有人叫骂索马,
他按不住心中火发,急纵身跃浪翻波,跳将上来道:“是那个敢在这里海口伤吾?”
行者见了他,大咤一声:“休走!还我马来!”轮着棍,劈头就打。那条龙张牙舞爪
来抓。他两个在涧边前这一场赌斗,果是骁雄。但见那:

龙舒利爪,猴举金箍。那个须垂白玉线,这个眼幌赤金灯。那个须下明珠喷彩
雾,这个手中铁棒舞狂风。那个是迷爷娘的业子,这个是欺天将的妖精。他两个都
因有难遭磨折,今要成功各显能。
来来往往,战罢多时,盘旋良久,那条龙力软筋麻,不能抵敌,打一个转身,又撺
于水内;深潜涧底,再不出头。被猴王骂詈不绝,他也只推耳聋。

行者没及奈何,只得回见三藏道:“师父,这个怪被老孙骂将出来,他与我赌
斗多时,怯战而走,只躲在水中间,再不出来了。”三藏道:“不知端的可是他吃了
我马?”行者道:“你看你说的话!不是他吃了,他还肯出来招声,与老孙犯对?”
三藏道:“你前日打虎时,曾说有降龙伏虎的手段,今日如何便不能降他?”原来
那猴子吃不得人急他。见三藏抢白了他这一句,他就发起神威道:“不要说,不要
说!等我与他再见个上下!”

这猴王拽开步,跳到涧边,使出那翻江搅海的神通,把一条鹰愁陡涧彻底澄清
的水,搅得似那九曲黄河泛涨的波。那孽龙在于深涧中,坐卧不宁,心中思想道:
“这才是福无双降,祸不单行。我才脱了天条死难,不上一年,在此随缘度日,又
撞着这般个泼魔,他来害我!”你看他越思越恼,受不得屈气,咬着牙,跳将出去,
骂道:“你是那里来的泼魔,这等欺我!”行者道:“你莫管我那里不那里,你只还
了马,我就饶你性命!”那龙道:“你的马是我吞下肚去,如何吐得出来!不还你,
便待怎的!”行者道:“不还马时看棍!只打杀你,偿了我马的性命便罢!”他两个又
在那山崖下苦斗。斗不数合,小龙委实难搪,将身一幌,变作一条水蛇儿,钻入草
科中去了。

猴王拿着棍,赶上前来,拨草寻蛇,那里得些影响。急得他三尸神咋,七窍烟
生,念了一声“”字咒语,即唤出当坊土地、本处山神,一齐来跪下道:“山神、
土地来见。”行者道:“伸过孤拐来,各打五棍见面,与老孙散散心!”二神叩头哀
告道:“望大圣方便,容小神诉告。”行者道:“你说甚么?”二神道:“大圣一向久
困,小神不知几时出来,所以不曾接得,万望恕罪。”行者道:“既如此,我且不打
你。我问你:鹰愁涧里,是那方来的怪龙?他怎么抢了我师父的白马吃了?”二神
道:“大圣自来不曾有师父,原来是个不伏天不伏地混元上真,如何得有甚么师父
的马来?”行者道:“你等是也不知。我只为那诳上的勾当,整受了这五百年的苦
难。今蒙观音菩萨劝善,着唐朝驾下真僧救出我来,教我跟他做徒弟,往西天去拜
佛求经。因路过此处,失了我师父的白马。”二神道:“原来是如此。这涧中自来无
邪,只是深陡宽阔,水光彻底澄清,鸦鹊不敢飞过;因水清照见自己的形影,便认
做同群之鸟,往往身掷于水内:故名‘鹰愁陡涧’。只是向年间,观音菩萨因为寻
访取经人去,救了一条玉龙,送他在此,教他等候那取经人,不许为非作歹,他只
是饥了时,上岸来扑些鸟鹊吃,或是捉些獐鹿食用。不知他怎么无知,今日冲撞了
大圣。”行者道:“先一次,他还与老孙侮手,盘旋了几合;后一次,是老孙叫骂,
他再不出。因此使了一个翻江搅海的法儿,搅混了他涧水,他就撺将上来,还要争
持。不知老孙的棍重,他遮架不住,就变做一条水蛇,钻在草里。我赶来寻他,却
无踪迹。”土地道:“大圣不知。这条涧千万个孔窍相通,故此这波澜深远。想是此
间也有一孔,他钻将下去。也不须大圣发怒,在此找寻;要擒此物,只消请将观世
音来,自然伏了。”

行者见说,唤山神、土地,同来见了三藏,具言前事。三藏道:“若要去请菩
萨,几时才得回来?我贫僧饥寒怎忍!”说不了,只听得暗空中有金头揭谛叫道:“大
圣,你不须动身,小神去请菩萨来也。”行者大喜,道声“有累,有累!快行,快行!”
那揭谛急纵云头,径上南海。行者吩咐山神、土地守护师父,日值功曹去寻斋供,
他又去涧边巡绕不题。

却说金头揭谛,一驾云,早到了南海。按祥光,直至落伽山紫竹林中,托那金
甲诸天与木叉惠岸转达,得见菩萨。菩萨道:“汝来何干?”揭谛道:“唐僧在蛇盘
山鹰愁陡涧失了马,急得孙大圣进退两难。及问本处土神,说是菩萨送在那里的孽
龙吞了,那大圣着小神来告请菩萨降这孽龙,还他马匹。”菩萨闻言道:“这厮本是
西海敖闰之子。他为纵火烧了殿上明珠,他父告他忤逆,天庭上犯了死罪,是我亲
见玉帝,讨他下来,教他与唐僧做个脚力。他怎么反吃了唐僧的马?这等说,等我
去来。”那菩萨降莲台,径离仙洞,与揭谛驾着祥光,过了南海而来。有诗为证。
诗曰:
佛说蜜多三藏经,菩萨扬善满长城。
摩诃妙语通天地,般若真言救鬼灵。
致使金蝉重脱壳,故令玄奘再修行。
只因路阻鹰愁涧,龙子归真化马形。

那菩萨与揭谛,不多时,到了蛇盘山。却在那半空里留住祥云,低头观看。只
见孙行者正在涧边叫骂。菩萨着揭谛唤他来。那揭谛按落云头,不经由三藏,直至
涧边,对行者道:“菩萨来也。”行者闻得,急纵云跳到空中,对他大叫道:“你这
个七佛之师,慈悲的教主,你怎么生方法儿害我!”菩萨道:“我把你这个大胆的马
流,村愚的赤尻!我倒再三尽意,度得个取经人来,叮咛教他救你性命,你怎么不
来谢我活命之恩,反来与我嚷闹?”行者道:“你弄得我好哩!你既放我出来,让我
逍遥自在耍子便了;你前日在海上迎着我,伤了我几句,教我来尽心竭力,伏侍唐
僧便罢了;你怎么送他一顶花帽,哄我戴在头上受苦?把这个箍子长在老孙头上,
又教他念一卷甚么‘紧箍儿咒’,着那老和尚念了又念,教我这头上疼了又疼,这
不是你害我也?”菩萨笑道:“你这猴子!你不遵教令,不受正果,若不如此拘系你,
你又诳上欺天,知甚好歹!再似从前撞出祸来,有谁收管?须是得这个魔头,你才肯
入我瑜伽之门路哩!”

行者道:“这桩事,作做是我的魔头罢;你怎么又把那有罪的孽龙,送在此处
成精,教他吃了我师父的马匹?此又是纵放歹人为恶,太不善也!”菩萨道:“那条
龙,是我亲奏玉帝,讨他在此,专为求经人做个脚力。你想那东土来的凡马,怎历
得这万水千山?怎到得那灵山佛地?须是得这个龙马,方才去得。”行者道:“像他这
般惧怕老孙,潜躲不出,如之奈何?”菩萨叫揭谛道:“你去涧边叫一声‘敖闰龙
王玉龙三太子,你出来,有南海菩萨在此。’他就出来了。”那揭谛果去涧边叫了两
遍。那小龙翻波跳浪,跳出水来,变作一个人像,踏了云头,到空中对菩萨礼拜道:
“向蒙菩萨解脱活命之恩,在此久等,更不闻取经人的音信。”菩萨指着行者道:“这
不是取经人的大徒弟?”小龙见了道:“菩萨,这是我的对头。我昨日腹中饥馁,
果然吃了他的马匹。他倚着有些力量,将我斗得力怯而回;又骂得我闭门不敢出来。
他更不曾提着一个‘取经’的字样。”行者道:“你又不曾问我姓甚名谁,我怎么就
说?”小龙道:“我不曾问你是那里来的泼魔?你嚷道:‘管甚么那里不那里,只还
我马来!’何曾说出半个‘唐’字!”菩萨道:“那猴头,专倚自强,那肯称赞别人?
今番前去,还有归顺的哩。若问时,先提起‘取经’的字来,却也不用劳心,自然
拱伏。”

行者欢喜领教。菩萨上前,把那小龙的项下明珠摘了,将杨柳枝蘸出甘露,往
他身上拂了一拂,吹口仙气,喝声叫“变!”那龙即变做他原来的马匹毛片。又将
言语吩咐道:“你须用心了还业障;功成后,超越凡龙,还你个金身正果。”那小龙
口衔着横骨,心心领诺。

菩萨教悟空领他去见三藏,“我回海上去也。”行者扯住菩萨不放道:“我不去
了,我不去了!西方路这等崎岖,保这个凡僧,几时得到?似这等多磨多折,老孙的
性命也难全,如何成得甚么功果!我不去了,我不去了!”菩萨道:“你当年未成人
道,且肯尽心修悟;你今日脱了天灾,怎么倒生懒惰?我门中以寂灭成真,须是要
信心正果;假若到了那伤身苦磨之处,我许你叫天天应,叫地地灵。十分再到那难
脱之际,我也亲来救你。你过来,我再赠你一般本事。”菩萨将杨柳叶儿,摘下三
个,放在行者的脑后,喝声“变!”即变做三根救命的毫毛,教他:“若到那无济无
主的时节,可以随机应变,救得你急苦之灾。”行者闻了这许多好言,才谢了大慈
大悲的菩萨。那菩萨香风绕绕,彩雾飘飘,径转普陀而去。

这行者才按落云头,揪着那龙马的顶鬃,来见三藏道:“师父,马有了也。”三
藏一见大喜道:“徒弟,这马怎么比前反肥盛了些?在何处寻着的?”行者道:“师父,
你还做梦哩!却才是金头揭谛请了菩萨来,把那涧里龙化作我们的白马。其毛片相
同,只是少了鞍辔,着老孙揪将来也。”三藏大惊道:“菩萨何在?待我去拜谢他。”
行者道:“菩萨此时已到南海,不耐烦矣。”三藏就撮土焚香,望南礼拜。拜罢,起
身即与行者收拾前进。行者喝退了山神、土地,吩咐了揭谛、功曹,却请师父上马。
三藏道:“那无鞍辔的马,怎生骑得?且待寻船渡过涧去,再作区处。”行者道:“这
个师父好不知时务!这个旷野山中,船从何来?这匹马,他在此久住,必知水势,就
骑着他做个船儿过去罢。”三藏无奈,只得依言,跨了马。行者挑着行囊。到了
涧边。

只见那上流头,有一个渔翁,撑着一个枯木的筏子,顺流而下。行者见了,用
手招呼道:“那老渔,你来,你来。我是东土取经去的。我师父到此难过,你来渡
他一渡。”渔翁闻言,即忙撑拢。行者请师父下了马,扶持左右。三藏上了筏子,
揪上马匹,安了行李。那老渔撑开筏子,如风似箭,不觉的过了鹰愁陡涧,上了西
岸。三藏教行者解开包袱,取出大唐的几文钱钞,送与老渔。老渔把筏子一篙撑开
道:“不要钱,不要钱。”向中流渺渺茫茫而去。三藏甚不过意,只管合掌称谢。行
者道:“师父休致意了。你不认得他?他是此涧里的水神。不曾来接得我老孙,老孙
还要打他哩。只如今免打就够了他的,怎敢要钱!”那师父也似信不信,只得又跨
着马,随着行者,径投大路,奔西而去。这正是:广大真如登彼岸,诚心了性上
灵山。同师前进,不觉的红日沉西,天光渐晚。但见:

淡云撩乱,山月昏蒙。满天霜色生寒,四面风声透体。孤鸟去时苍渚阔,落霞
明处远山低。疏林千树吼,空岭独猿啼。长途不见行人迹,万里归舟入夜时。
三藏在马上遥观,忽见路旁一座庄院。三藏道:“悟空,前面人家,可以借宿,明
早再行。”行者抬头看见道:“师父,不是人家庄院。”三藏道:“如何不是?”行者
道:“人家庄院,却没飞鱼稳兽之脊,这断是个庙宇庵院。”

师徒们说着话,早已到了门首。三藏下了马,只见那门上有三个大字,乃“里
社祠”,遂入门里。那里边有一个老者,顶挂着数珠儿,合掌来迎,叫声“师父请
坐。”三藏慌忙答礼,上殿去参拜了圣像。那老者即呼童子献茶。茶罢,三藏问老
者道:“此庙何为‘里社’?”老者道:“敝处乃西番哈国界。这庙后有一庄人家,
共发虔心,立此庙宇。里者,乃一乡里地;社者,乃一社土神。每遇春耕、夏耘、
秋收、冬藏之日,各办三牲花果,来此祭社,以保四时清吉,五谷丰登,六畜茂盛
故也。”三藏闻言,点头夸赞:“正是‘离家三里远,别是一乡风。’我那里人家,
更无此善。”老者却问:“师父仙乡是何处?”三藏道:“贫僧是东土大唐国,奉旨
意,上西天拜佛求经的。路过宝坊,天色将晚,特投圣祠,告宿一宵,天光即行。”
那老者十分欢喜,道了几声“失迎”,又叫童子办饭。三藏吃毕,谢了。

行者的眼乖,见他房檐下,有一条搭衣的绳子,走将去,一把扯断,将马脚系
住。那老者笑道:“这马是那里偷来的?”行者怒道:“你那老头子,说话不知高低!
我们是拜佛的圣僧,又会偷马!”老儿笑道:“不是偷的,如何没有鞍辔缰绳,却来
扯断我晒衣的索子?”三藏陪礼道:“这个顽皮,只是性燥。你要拴马,好生问老
人家讨条绳子,如何就扯断他的衣索?老先休怪,休怪。我这马,实不瞒你说,不
是偷的:昨日东来,至鹰愁陡涧,原有骑的一匹白马,鞍辔俱全。不期那涧里有条
孽龙,在彼成精,他把我的马,连鞍辔一口吞之。幸亏我徒弟有些本事,又感得观
音菩萨来涧边擒住那龙,教他就变做我原骑的白马,毛片俱同,驮我上西天拜佛。
今此过涧,未经一日,却到了老先的圣祠,还不曾置得鞍辔哩。”那老者道:“师父
休怪,我老汉作笑耍子,谁知你高徒认真。我小时也有几个村钱,也好骑匹骏马;
只因累岁屯,遭丧失火,到此没了下梢,故充为庙祝,侍奉香火。幸亏这后庄施
主家募化度日。我那里倒还有一副鞍辔,是我平日心爱之物,就是这等贫穷,也不
曾舍得卖了。才听老师父之言,菩萨尚且救护,神龙教他化马驮你,我老汉却不能
少有周济,明日将那鞍辔取来,愿送老师父,扣背前去,乞为笑纳。”三藏闻言,
称谢不尽。早又见童子拿出晚斋。斋罢,掌上灯,安了铺,各各寝歇。

至次早,行者起来道:“师父,那庙祝老儿,昨晚许我们鞍辔,问他要,不要
饶他。”说未了,只见那老儿,果擎着一副鞍辔,衬屉、缰笼之类,凡马上一切用
的,无不全备,放在廊下道:“师父,鞍辔奉上。”三藏见了,欢喜领受。教行者拿
了,背上马看,可相称否。行者走上前,一件件的取起看了,果然是些好物。有诗
为证。诗曰:
雕鞍彩晃柬银星,宝凳光飞金线明。
衬屉几层绒苫迭,牵缰三股紫丝绳。
辔头皮札团花粲,云扇描金舞兽形。
环嚼叩成磨炼铁,两垂蘸水结毛缨。
行者心中暗喜,将鞍辔背在马上,就似量着做的一般。三藏拜谢那老,那老慌忙搀
起道:“惶恐,惶恐,何劳致谢?”那老者也不再留,请三藏上马。那长老出得门
来,攀鞍上马。行者担着行李。那老儿复袖中取出一条鞭儿来,却是皮丁儿寸札的
香藤柄子,虎筋丝穿结的梢儿,在路旁拱手奉上道:“圣僧,我还有一条挽手儿,
一发送了你罢。”那三藏在马上接了道:“多承布施!多承布施!”

正打问讯,却早不见了那老儿。及回看那里社祠,是一片光地。只听得半空中
有人言语道:“圣僧,多简慢你。我是落伽山山神,土地,蒙菩萨差送鞍辔与汝等
的。汝等可努力西行,却莫一时怠慢。”慌得个三藏滚鞍下马,望空礼拜道:“弟子
肉眼凡胎,不识尊神尊面,望乞恕罪。烦转达菩萨,深蒙恩佑。”你看他只管朝天
磕头,也不计其数。路旁边活活的笑倒个孙大圣,孜孜的喜坏个美猴王,上前来扯
住唐僧道:“师父,你起来罢。他已去得远了,听不见你祷祝,看不见你磕头。只
管拜怎的?”长老道:“徒弟呀,我这等磕头,你也就不拜他一拜,且立在旁边,
只管哂笑,是何道理?”行者道:“你那里知道?像他这个藏头露尾的,本该打他一
顿;只为看菩萨面上,饶他打尽够了,他还敢受我老孙之拜?老孙自小儿做好汉,
不晓得拜人,就是见了玉皇大帝、太上老君,我也只是唱个喏便罢了。”三藏道:“不
当人子!莫说这空头话!快起来,莫误了走路。”那师父才起来收拾投西而去。

此去行有两个月太平之路,相遇的都是些虏虏、回回,狼虫虎豹。光阴迅速,
又值早春时候。但见山林锦翠色,草木发青芽;梅英落尽,柳眼初开。师徒们行玩
春光,又见太阳西坠。三藏勒马遥观,山凹里,有楼台影影,殿阁沉沉。三藏道:
“悟空,你看那里是甚么去处?”行者抬头看了道:“不是殿宇,定是寺院。我们
赶起些,那里借宿去。”三藏欣然从之,放开龙马,径奔前来。

毕竟不知此去是甚么去处,且听下回分解。