\chapter{妖邪假设小雷音~四众皆遭大厄难}

这回因果,劝人为善,切休作恶。一念生,神明照鉴,任他为作。拙蠢乖能君
怎学,两般还是无心药。趁生前有道正该修,莫浪泊。认根源,脱本壳。访长生,
须把捉。要时时明见,醍醐斟酌。贯彻三关填黑海,管教善者乘鸾鹤。那其间愍故
更慈悲,登极乐。

话表唐三藏一念虔诚,且休言天神保护,似这草木之灵,尚来引送,雅会一宵,
脱出荆棘针刺,再无萝攀缠。四众西进,行够多时,又值冬残,正是那三春之日:

物华交泰,斗柄回寅。草芽遍地绿,柳眼满堤青。一岭桃花红锦,半溪烟水
碧罗明。几多风雨,无限心情。日晒花心艳,燕衔苔蕊轻。山色王维画浓淡,鸟声
季子舌纵横。芳菲铺绣无人赏,蝶舞蜂歌却有情。
师徒们也自寻芳踏翠,缓随马步。正行之间,忽见一座高山,远望着与天相接。三
藏扬鞭指道:“悟空,那座山也不知有多少高,可便似接着青天,透冲碧汉。”行者
道:“古诗不云:‘只有天在上,更无山与齐。’但言山之极高,无可与他比并。岂
有接天之理!”八戒道:“若不接天,如何把昆仑山号为‘天柱’?”行者道:“你
不知。自古‘天不满西北’。昆仑山在西北乾位上,故有顶天塞空之意,遂名天柱。”
沙僧笑道:“大哥把这好话儿莫与他说。他听了去,又降别人。我们且走路。等上
了那山,就知高下也。”

那呆子赶着沙僧,厮耍厮斗。老师父马快如飞。须臾,到那山崖之边。一步步
往上行来,只见那山:

林中风飒飒,涧底水潺潺。鸦雀飞不过,神仙也道难。千崖万壑,亿曲百湾。
尘埃滚滚无人到,怪石森森不厌看。有处有云如水,是方是树鸟声繁。鹿衔芝去,
猿摘桃还。狐貉往来崖上跳,獐出入岭头顽。忽闻虎啸惊人胆,斑豹苍狼把路拦。
唐三藏一见心惊。孙行者神通广大,你看他一条金箍棒,哮吼一声,吓过了狼虫虎
豹,剖开路,引师父直上高山。行过岭头,下西平处,忽见祥光蔼蔼,彩雾纷纷,
有一所楼台殿阁,隐隐的钟磬悠扬。三藏道:“徒弟们,看是个甚么去处。”行者抬
头,用手搭凉篷,仔细观看,那壁厢好个所在!真个是:

珍楼宝座,上刹名方。谷虚繁地籁,境寂散天香。青松带雨遮高阁,翠竹留云
护讲堂。霞光缥缈龙宫显,彩色飘摇沙界长。朱栏玉户,画栋雕梁。谈经香满座,
语月当窗。鸟啼丹树内,鹤饮石泉旁。四围花发琪园秀,三面门开舍卫光。楼台
突兀门迎嶂,钟磬虚徐声韵长。窗开风细,帘卷烟茫。有僧情散淡,无俗意和昌。
红尘不到真仙境,静土招提好道场。
行者看罢,回复道:“师父,那去处是便是座寺院,却不知禅光瑞蔼之中,又有些
凶气,何也。观此景象,也似雷音,却又路道差池。我们到那厢,决不可擅入,恐
遭毒手。”唐僧道:“既有雷音之景,莫不就是灵山?你休误了我诚心,担搁了我来
意。”行者道:“不是,不是!灵山之路,我也走过几遍,那是这路途!”八戒道:“纵
然不是,也必有个好人居住。”沙僧道:“不必多疑。此条路未免从那门首过,是不
是一见可知也。”行者道:“悟净说得有理。”

那长老策马加鞭,至山门前,见“雷音寺”三个大字,慌得滚下马来,倒在地
下。口里骂道:“泼猢狲!害杀我也!现是雷音寺,还哄我哩!”行者陪笑道:“师父
莫恼,你再看看。山门上乃四个字,你怎么只念出三个来,倒还怪我?”长老战兢
兢的爬起来再看,真个是四个字,乃“小雷音寺”。三藏道:“就是小雷音寺,必定
也有个佛祖在内。经上言三千诸佛,想是不在一方:似观音在南海,普贤在峨眉,
文殊在五台。这不知是那一位佛祖的道场。古人云:‘有佛有经,无方无宝。’我们
可进去来。”行者道:“不可进去。此处少吉多凶。若有祸患,你莫怪我。”三藏道:
“就是无佛,也必有个佛像。我弟子心愿,遇佛拜佛,如何怪你。”即命八戒取袈
裟,换僧帽,结束了衣冠,举步前进。

只听得山门里有人叫道:“唐僧,你自东土来拜见我佛,怎么还这等怠慢?”
三藏闻言,即便下拜。八戒也磕头,沙僧也跪倒;惟大圣牵马,收拾行李,在后。
方入到二层门内,就见如来大殿。殿门外宝台之下,摆列着五百罗汉、三千揭谛、
四金刚、八菩萨、比丘尼、优婆塞、无数的圣僧、道者。真个也香花艳丽,瑞气缤
纷。慌得那长老与八戒、沙僧一步一拜,拜上灵台之间。行者公然不拜。又闻得莲
台座上厉声高叫道:“那孙悟空,见如来怎么不拜?”不知行者又仔细观看,见得
是假,遂丢了马匹、行囊,掣棒在手,喝道:“你这伙孽畜,十分胆大!怎么假倚佛
名,败坏如来清德!不要走!”双手轮棒,上前便打。只听得半空中叮当一声,撇下
一副金铙,把行者连头带足,合在金铙之内。慌得个猪八戒、沙和尚连忙使起钯杖,
就被些阿罗、揭谛、圣僧、道者一拥近前围绕。他两个措手不及,尽被拿了。将三
藏捉住,一齐都绳缠索绑,紧缚牢拴。

原来那莲花座上装佛祖者乃是个妖王,众阿罗等都是些小怪。遂收了佛祖体象,
依然现出妖身。将三众抬入后边收藏;把行者合在金铙之中,永不开放。只搁在宝
台之上,限三昼夜化为脓血。化后,才将铁笼蒸他三个受用。这正是:
碧眼猢儿识假真,禅机见象拜金身。
黄婆盲目同参礼,木母痴心共话论。
邪怪生强欺本性,魔头怀恶诈天人。
诚为道小魔头大,错入旁门枉费身。
那时群妖将唐僧三众收藏在后;把马拴在后边;把他的袈裟、僧帽安在行李担内,
亦收藏了。一壁厢严紧不题。

却说行者合在金铙里,黑洞洞的,燥得满身流汗,左拱右撞,不能得出。急得
他使铁棒乱打,莫想得动分毫。他心里没了算计,将身往外一挣,却要挣破那金铙;
遂捻着一个诀,就长有千百丈高,那金铙也随他身长,全无一些瑕缝光明。却又捻
诀把身子往下一小,小如芥菜子儿,那铙也就随身小了,更没些些孔窍。他又把铁
棒,吹口仙气,叫“变!”即变做竿一样,撑住金铙。他却把脑后毫毛,选长的,
拔下两根,叫“变!”即变做梅花头,五瓣钻儿,挨着棒下,钻有千百下,只钻得
苍苍响,再不钻动一些。

行者急了,却捻个诀,念一声“静法界,乾元亨利贞”的咒语。拘得那五
方揭谛、六丁六甲、一十八位护教伽蓝,都在金铙之外道:“大圣,我等俱保护着
师父,不教妖魔伤害,你又拘唤我等做甚?”行者道:“我那师父,不听我劝解,
就弄死他也不亏!但只你等怎么快作法将这铙钹掀开,放我出来,再作处治。这里
面不通光亮,满身暴燥,却不闷杀我也?”众神真个掀铙,就如长就的一般,莫想
揭得分毫。金头揭谛道:“大圣,这铙钹不知是件甚么宝贝,连上带下,合成一块。
小神力薄,不能掀动。”行者道:“我在里面,不知使了多少神通,也不得动。”

揭谛闻言,即着六丁神保护着唐僧,六甲神看守着金铙,众伽蓝前后照察;他
却纵起祥光,须臾间,闯入南天门里。不待宣召,直上灵霄宝殿之下,见玉帝,俯
伏启奏道:“主公,臣乃五方揭谛使。今有齐天大圣保唐僧取经,路遇一山,名小
雷音寺。唐僧错认灵山进拜,原来是妖魔假设,困陷他师徒,将大圣合在一副金铙
之内,进退无门,看看至死,特来启奏。”即传旨:“差二十八宿星辰,快去释厄降
妖。”

那星宿不敢少缓,随同揭谛,出了天门,至山门之内。有二更时分,那些大小
妖精,因获了唐僧,老妖俱犒赏了,各去睡觉。众星宿更不惊张,都到铙钹之外,
报道:“大圣,我等是玉帝差来二十八宿,到此救你。”行者听说大喜。便教:“动
兵器打破,老孙就出来了!”众星宿道:“不敢打。此物乃浑金之宝,打着必响;响
时惊动妖魔,却难救拔。等我们用兵器捎他。你那里但见有一些光处就走。”行者
道:“正是。”你看他们使枪的使枪,使剑的使剑,使刀的使刀,使斧的使斧;扛的
扛,抬的抬,掀的掀,捎的捎;弄到有三更天气,漠然不动,就是铸成了囫囵的一
般。那行者在里边,东张张,西望望,爬过来,滚过去,莫想看见一些光亮。

亢金龙道:“大圣啊,且休焦躁。观此宝定是个如意之物,断然也能变化。你
在那里面,于那合缝之处,用手摸着,等我使角尖儿拱进来,你可变化了,顺松处
脱身。”行者依言,真个在里面乱摸。这星宿把身变小了,那角尖儿就似个针尖一
样,顺着钹合缝口上,伸将进去。可怜用尽千斤之力,方能穿透里面。却将本身与
角使法象,叫“长,长,长!”角就长有碗来粗细。那钹口倒也不像金铸的,好似
皮肉长成的,顺着亢金龙的角,紧紧噙住,四下里更无一丝拔缝。

行者摸着他的角,叫道:“不济事!上下没有一毫松处!没奈何,你忍着些儿疼,
带我出去。”好大圣,即将金箍棒变作一把钢钻儿,将他那角尖上钻了一个孔窍,
把身子变得似个芥菜子儿,拱在那钻眼里蹲着,叫:“扯出角去!扯出角去!”这星
宿又不知费了多少力,方才拔出,使得力尽筋柔,倒在地下。

行者却自他角尖钻眼里钻出,现了原身,掣出铁棒,照铙钹当的一声打去,就
如崩倒铜山,咋开金铙。可惜把个佛门之器,打做个千百块散碎之金!唬得那二十
八宿惊张,五方揭谛发竖。大小群妖皆梦醒。

老妖王睡里慌张,急起来,披衣擂鼓,聚点群妖,各执器械。此时天将黎明。
一拥赶到宝台之下。只见孙行者与列宿围在碎破金铙之外,大惊失色,即令:“小
的们!紧关了前门,不要放出人去!”

行者听说,即携星众,驾云跳在九霄空里。那妖王收了碎金,排开妖卒,列在
山门外。妖王怀恨,没奈何披挂了,使一根短软狼牙棒,出营高叫:“孙行者!好男
子不可远走高飞,快向前与我交战三合!”行者忍不住,即引星众,按落云头,观
看那妖精怎生模样。但见他:

蓬着头,勒一条扁薄金箍;光着眼,簇两道黄眉的竖。悬胆鼻,孔窃开查;四
方口,牙齿尖利。穿一副叩结连环铠,勒一条生丝攒穗绦。脚踏乌喇鞋一对,手执
狼牙棒一根。此形似兽不如兽,相貌非人却似人。
行者挺着铁棒喝道:“你是个甚么怪物,擅敢假装佛祖,侵占山头,虚设小雷音寺!”
那妖王道:“这猴儿是也不知我的姓名,故来冒犯仙山。此处唤做小西天。因我修
行,得了正果,天赐与我的宝阁珍楼。我名乃是黄眉老佛。这里人不知,但称我为
黄眉大王、黄眉爷爷。一向久知你往西去,有些手段,故此设象显能,诱你师父进
来,要和你打个赌赛。如若斗得过我,饶你师徒,让汝等成个正果;如若不能,将
汝等打死,等我去见如来取经,果正中华也。”行者笑道:“妖精,不必海口!既要
赌,快上来领棒!”那妖王喜孜孜,使狼牙棒抵住。这一场好杀:

两条棒,不一样,说将起来有形状:一条短软佛家兵,一条坚硬藏海藏。都有
随心变化功,今番相遇争强壮。短软狼牙杂锦妆,坚硬金箍蛟龙象。若粗若细实可
夸,要短要长甚停
当。猴与魔,齐打仗,这场真个无虚诳。驯猴秉教作心猿,泼怪欺天弄假象。嗔嗔
恨恨各无情,恶恶凶凶都有样。那一个当头手起不放松,这一个架丢劈面难推让。
喷云照日昏,吐雾遮峰嶂。棒来棒去两相迎,忘生忘死因三藏。
看他两个斗经五十回合,不见输赢。那山门口,鸣锣擂鼓,众妖精呐喊摇旗。这壁
厢有二十八宿天兵共五方揭谛众圣,各掮器械,吆喝一声,把那魔头围在中间,吓
得那山门外群妖难擂鼓,战兢兢手软不敲锣。

老妖魔公然不惧,一只手使狼牙棒,架着众兵;一只手去腰间解下一条旧白布
搭包儿,往上一抛,“滑”的一声响,把孙大圣、二十八宿与五方揭谛,一搭包
儿通装将去,挎在肩上,拽步回身。众小妖个个欢然得胜而回。老妖教小的们取了
三五十条麻索,解开搭包,拿一个,捆一个。一个个都骨软筋麻,皮肤皱。捆了
抬去后边,不分好歹,俱掷之于地。妖王又命排筵畅饮,自旦至暮方散,各归寝处
不题。

却说孙大圣与众神捆至夜半,忽闻有悲泣之声。侧耳听时,却原来是三藏声音。
哭道:“悟空啊!我
自恨当时不听伊,致令今日受灾危。
金铙之内伤了你,麻绳捆我有谁知。
四众遭逢缘命苦,三千功行尽倾颓。
何由解得难,坦荡西方去复归!”
行者听言,暗自怜悯道:“那师父虽是未听吾言,今遭此毒,然于患难之中,还有
忆念老孙之意。趁此夜静妖眠,无人防备,且去解脱众等逃生也。”

好大圣,使了个遁身法,将身一小,脱下绳来,走近唐僧身边,叫声:“师父。”
长老认得声音,叫道:“你为何到此?”行者悄悄的把前项事告诉了一遍。长老甚
喜道:“徒弟,快救我一救!向后事,但凭你处,再不强了!”行者才动手,先解了
师父,放了八戒、沙僧,又将二十八宿、五方揭谛,个个解了,又牵过马来,教快
先走出去。

方出门,却不知行李在何处,又来找寻。亢金龙道:“你好重物轻人!既救了你
师父就够了,又还寻甚行李?”行者道:“人固要紧,衣钵尤要紧。包袱中有通关
文牒、锦袈裟、紫金钵盂,俱是佛门至宝,如何不要!”八戒道:“哥哥,你去找
寻,我等先去路上等你。”你看那星众,簇拥着唐僧,使个摄法,共弄神通,一阵
风,撮出垣围,奔大路,下了山坡,却屯于平处等候。

约有三更时分,孙大圣轻挪慢步,走入里面,原来一层层门户甚紧。他就爬上
高楼看时,窗牖皆关。欲要下去,又恐怕窗棂儿响,不敢推动。捻着诀,摇身一变,
变做一个仙鼠,俗名蝙蝠。你道他怎生模样:
头尖还似鼠,眼亮亦如之。
有翅黄昏出,无光白昼居。
藏身穿瓦穴,觅食扑蚊儿。
偏喜晴明月,飞腾最识时。
他顺着不封瓦口椽子之下,钻将进去。越门过户,到了中间看时,只见那第三重楼
窗之下,灼灼一道毫光,也不是灯烛之光,香火之光,又不是飞霞之光,掣电之
光。他半飞半跳,近于光前看时,却是包袱放光。那妖精把唐僧的袈裟脱了,不曾
折,就乱乱的在包袱之内。那袈裟本是佛宝,上边有如意珠、摩尼珠、红玛瑙、
紫珊瑚、舍利子、夜明珠,所以透的光彩。

他见了此衣钵,心中一喜,就现了本象,拿将过来,也不管担绳偏正,抬上肩,
往下就走。不期脱了一头,扑的落在楼板上,唿喇的一声响。噫!有这般事:可
可的老妖精在楼下睡觉,一声响,把他惊醒,跳起来,乱叫道:“有人了,有人了!”
那些大小妖都起来,点灯打火,一齐吆喝,前后去看。有的来报道:“唐僧走了!”
又有的来报道:“行者众人俱走了!”老妖急传号令,教:“拿!各门上谨慎!”行者
听言,恐又遭他罗网,挑不成包袱,纵筋斗,就跳出楼窗外走了。

那妖精前前后后,寻不着唐僧等。又见天色将明,取了棒,帅众来赶,只见那
二十八宿与五方揭谛等神,云雾腾腾,屯住山坡之下。妖王喝了一声:“那里去,
吾来也!”角木蛟急唤:“兄弟们!怪物来了!”亢金龙、女土蝠、房日兔、心月狐、
尾火虎、箕水豹、斗木獬、牛金牛、氐土貉、虚日鼠、危月燕、室火猪、壁水、
奎木狼、娄金狗、胃土彘、昴日鸡、毕月乌、觜火猴、参水猿、井木犴、鬼金羊、
柳土獐、星日马、张月鹿、翼火蛇、轸水蚓,领着金头揭谛、银头揭谛、六甲、六
丁等神、护教伽蓝,同八戒、沙僧,——不领唐三藏,丢了白龙马——各执兵器,
一拥而上。这妖王见了,呵呵冷笑,叫一声哨子,有四五千大小妖精,一个个威强
力胜,浑战在西山坡上。好杀:

魔头泼恶欺真性,真性温柔怎奈魔。百计施为难脱苦,千方妙用不能和。诸天
来拥护,众圣助干戈。留情亏木母,定志感黄婆。浑战惊天并振地,强争设网与张
罗。那壁厢摇旗呐喊,这壁厢擂鼓筛锣。枪刀密密寒光荡,剑戟纷纷杀气多。妖卒
凶还勇,神兵怎奈何。愁云遮日月,惨雾罩山河。苦苦拽来相战,皆因三藏拜弥
陀。

那妖精倍加勇猛,帅众上前掩杀。正在那不分胜败之际,只闻得行者叱咤一声
道:“老孙来了!”八戒迎着道:“行李如何?”行者道:“老孙的性命几乎难免,却
便说甚么行李!”沙僧执着宝杖道:“且休叙话,快去打妖精也!”那星宿、揭谛、
丁甲等神,被群妖围在垓心浑杀,老妖使棒来打他三个,这行者、八戒,沙僧丢开
棍杖,轮着钉钯抵住。真个是地暗天昏,不能取胜。只杀得太阳星西没山根,太阴
星东生海峤。那妖见天晚,打个哨子,教群妖各各留心,他却取出宝贝。孙行者看
得分明。那怪解下搭包,拿在手中。行者道声:“不好了,走啊!”他就顾不得八戒、
沙僧、诸天等众,一路筋斗,跳上九霄空里。众神、八戒、沙僧不解其意,被他抛
起去,又都装在里面,只是走了行者。那妖王收兵回寺,又教取出绳索,照旧绑了。
将唐僧、八戒、沙僧悬梁高吊,白马拴在后边,诸神亦俱绑缚,抬在地窑子内,封
了盖锁。那众妖遵依,一一收了不题。

却说行者跳在九霄,全了性命;见妖兵回转,不张旗号,已知众等遭擒。他却
按下祥光,落在那东山顶上,咬牙恨怪物,滴泪想唐僧,仰面朝天望,悲嗟忽失声。
叫道:“师父啊!你是那世里造下这难,今生里步步遇妖精。似这般苦楚难逃,
怎生是好!”独自一个,嗟叹多时,复又宁神思虑,以心问心道:“这妖魔不知是个
甚么搭包子,那般装得许多物件?如今将天神、天将,许多人又都装进去了。我待
求救于天,奈恐玉帝见怪。我记得有个北方真武,号曰荡魔天尊,他如今现在南赡
部洲武当山上,等我去请他来搭救师父一难。”正是:
仙道未成猿马散,心神无主五行枯。

毕竟不知此去端的如何,且听下回分解。