\chapter{金酬外护遭魔蛰~圣显幽魂救本原}

且不言唐僧等在华光破屋中,苦奈夜雨存身。却说铜台府地灵县城内有伙凶徒,
因宿娼、饮酒、赌博,花费了家私,无计过活,遂伙了十数人做贼,算道本城那家
是第一个财主,那家是第二个财主,去打劫些金银用度。内有一人道:“也不用缉
访,也不须算计,只有今日送那唐朝和尚的寇员外家,十分富厚。我们乘此夜雨,
街上人也不防备,火甲等也不巡逻,就此下手,劫他些资本,我们再去嫖赌儿耍子,
岂不美哉!”众贼欢喜,齐了心,都带了短刀、蒺藜、拐子、闷棍、麻绳、火把,
冒雨前来。打开寇家大门,呐喊杀入。慌得他家里,若大若小,是男是女,俱躲个
干净。妈妈儿躲在床底;老头儿闪在门后;寇梁、寇栋与着亲的几个儿女,都战战
兢兢的四散逃走顾命。那伙贼,拿着刀,点着火,将他家箱笼打开,把些金银宝贝,
首饰衣裳,器皿家火,尽情搜劫。那员外割舍不得,拚了命,走出门来,对众强人
哀告道:“列位大王,够你用的便罢,还留几件衣物与我老汉送终。”那众强人那容
分说,赶上前,把寇员外撩阴一脚,踢翻在地,可怜三魂渺渺归阴府,七魄悠悠别
世人!众贼得了手,走出寇家,顺城脚做了软梯,漫城墙一一系出,冒着雨连夜奔
西而去。那寇家僮仆,见贼退了,方才出头。及看时,老员外已死在地下。放声哭
道:“天呀!主人公已打死了!”众皆伏尸而哭,悲悲啼啼。

将四更时,那妈妈想恨唐僧等不受他的斋供,因为花扑扑的送他,惹出这场灾
祸,便生妒害之心,欲陷他四众。扶着寇梁道:“儿啊,不须哭了。你老子今日也
斋僧,明日也斋僧,岂知今日做圆满,斋着那一伙送命的僧也!”他兄弟道:“母亲,
怎么是送命的僧?”妈妈道:“贼势凶勇,杀进房来,我就躲在床下,战兢兢的留
心向灯火处看得明白。你说是谁?点火的是唐僧,持刀的是猪八戒,搬金银的是沙
和尚,打死你老子的是孙行者。”二子听言,认了真实道:“母亲既然看得明白,必
定是了。他四人在我家住了半月,将我家门户墙垣,窗巷道,俱看熟了,财动人
心,所以乘此夜雨,复到我家。既劫去财物,又害了父亲,此情何毒!待天明到府
里递失状坐名告他。”寇栋道:“失状如何写?”寇梁道:“就依母亲之言。”写道:
唐僧点着火,八戒叫杀人。
沙和尚劫出金银去,孙行者打死我父亲。
一家子吵吵闹闹,不觉天晓。一壁厢传请亲人,置办棺木;一壁厢寇梁兄弟,赴府
投词。原来这铜台府刺史正堂大人:

平生正直,素性贤良。少年向雪案攻书,早岁在金銮对策。常怀忠义之心,每
切仁慈之念。名扬青史播千年,龚、黄再见;声振黄堂传万古,卓、鲁重生。
当时坐了堂,发放了一应事务,即令抬出放告牌。这寇梁兄弟抱牌而入,跪倒高叫
道:“爷爷,小的们是告强盗得财,杀伤人命重情事。”刺史接上状去,看了这般这
的,如此如彼,即问道:“昨日有人传说,你家斋僧圆满,斋得四众高僧,乃东土
唐朝的罗汉,花扑扑的满街鼓乐送行,怎么却有这般事情?”寇梁等磕头道:“爷
爷,小的父亲寇洪,斋僧二十四年,因这四僧远来,恰足万僧之数;因此做了圆满,
留他住了半月。他就将路道、门窗都看熟了。当日送出,当晚复回,乘黑夜风雨,
遂明火执杖,杀进房来,劫去金银财宝,衣服首饰;又将父打死在地。望爷爷与小
民做主!”刺史闻言,即点起马步快手并民壮人役,共有百五十人,各执锋利器械,
出西门一直来赶唐僧四众。

却说他师徒们,在那华光行院破屋下挨至天晓。方才出门,上路奔西。可可的
那些强盗当夜打劫了寇家,系出城外,也向西方大路上,行经天晓,走过华光院西
去,有二十里远近,藏于山凹中,分拨金银等物。分还未了,忽见唐僧四众顺路而
来,众贼心犹不歇,指定唐僧道:“那不是昨日送行的和尚来了!”众贼笑道:“来
得好!来得好!我们也是干这般没天理的买卖。这些和尚缘路来,又在寇家许久,不
知身边有多少东西,我们索性去截住他,夺了盘缠,抢了白马凑分,却不是遂心满
意之事?”众贼遂持兵器,呐一声喊,跑上大路,一字儿摆开。叫道:“和尚,不
要走!快留下买路钱,饶你性命!牙迸半个‘不’字,一刀一个,决不留存!”唬得
个唐僧在马上乱战,沙僧与八戒心慌,对行者道:“怎的了,怎的了!苦奈得半夜雨
天,又早遇强徒断路,诚所谓‘祸不单行’也!”行者笑道:“师父莫怕,兄弟勿忧,
等老孙去问他一问。”

好大圣,束一束虎皮裙子,抖一抖锦布直裰,走近前,叉手当胸道:“列位是
做甚么的?”贼徒喝道:“这厮不知死活,敢来问我!你额颅下没眼,不认得我是大
王爷爷?快将买路钱来,放你过去!”行者闻言,满面陪笑道:“你原来是剪径的强
盗!”贼徒发狠叫:“杀了!”行者假假的惊恐道:“大王,大王!我是乡村中的和尚,
不会说话,冲撞莫怪,莫怪!若要买路钱,不要问那三个,只消问我。我是个管帐
的。凡有经钱、衬钱,那里化缘的、布施的,都在包袱中,尽是我管出入。那个骑
马的,虽是我的师父,他却只会念经,不管闲事,财色俱忘,一毫没有。那个黑脸
的,是我半路上收的个后生,只会养马。那个长嘴的,是我雇的长工,只会挑担。
你把三个放过去,我将盘缠、衣钵,尽情送你。”众贼听说:“这个和尚倒是个老实
头儿。既如此,饶了你命,教那三个丢下行李,放他过去。”

行者回头使个眼色,沙僧就丢了行李担子,与师父牵着马,同八戒往西径走。
行者低头打开包袱,就地挝把尘土,往上一洒,念个咒语,乃是个定身之法;喝一
声“住!”那伙贼共有三十来名,一个个咬着牙,睁着眼,撒着手,直直的站定,
莫能言语,不得动身。行者跳出路口,叫道:“师父!回来,回来!”八戒慌了道:“不
好,不好!师兄供出我们来了!他身上又无钱财,包袱里又无金银,必定是叫师父要
马哩。叫我们是剥衣服了。”沙僧笑道:“二哥莫乱说!大哥是个了得的。向者那般
毒魔狠怪,也能收服,怕这几个毛贼?他那里招呼,必有话说,快回去看看。”

长老听言,欣然转马,回至边前,叫道:“悟空,有甚事叫回来也?”行者道:
“你们看这些贼是怎的说?”八戒近前推着他,叫道:“强盗,你怎的不动弹了?”
那贼浑然无知,不言不语。八戒道:“好的痴哑了!”行者笑道:“是老孙使个定身
法定住也。”八戒道:“既定了身,未曾定口,怎么连声也不做?”行者道:“师父
请下马坐着。常言道:‘只有错拿,没有错放。’兄弟,你们把贼都扳翻倒,捆了,
教他供一个供状,看他是个雏儿强盗,把势强盗。”沙僧道:“没绳索哩。”行者即
拔下些毫毛,吹口仙气,变作三十条绳索,一齐下手,把贼扳翻,都四马攒蹄捆住,
却又念念解咒,那伙贼渐渐苏醒。

行者请唐僧坐在上首,他三人各执兵器喝道:“毛贼!你们一起有多少人?做了
几年买卖?打劫了有多少东西?可曾杀伤人口?还是初犯,却是二犯,三犯?”众贼
开口道:“爷爷饶命!”行者道:“莫叫唤!从实供来!”众贼道:“老爷,我们不是久
惯做贼的,都是好人家子弟。只因不才,吃酒赌钱,宿娼顽耍,将父祖家业,尽花
费了,一向无干,又无钱用。访知铜台府城中寇员外家资财豪富,昨日合伙,当晚
乘夜雨昏黑,就去打劫。劫的有些金银服饰,在这路北下山凹里正自分赃,忽见老
爷们来。内中有认得是寇员外送行的,必定身边有物;又见行李沉重,白马快走,
人心不足,故又来邀截。岂知老爷有大神通法力,将我们困住。万望老爷慈悲,收
去那劫的财物,饶了我的性命也!”

三藏听说是寇家劫的财物,猛然吃了一惊,慌忙站起道:“悟空,寇老员外十
分好善,如何招此灾厄?”行者笑道:“只为送我们起身,那等彩帐花幢,盛张鼓
乐,惊动了人眼目,所以这伙光棍就去下手他家。今又幸遇着我们,夺下他这许多
金银服饰。”三藏道:“我们扰他半月,感激厚恩,无以为报,不如将此财物护送他
家,却不是一件好事?”行者依言。即与八戒、沙僧,去山凹里取将那些赃物,收
拾了,驮在马上。又教八戒挑了一担金银,沙僧挑着自己行李。行者欲将这伙强盗
一棍尽情打死,又恐唐僧怪他伤人性命,只得将身一抖,收上毫毛。那伙贼松了手
脚,爬起来,一个个落草逃生而去。这唐僧转步回身,将财物送还员外,这一去,
却似飞蛾投火,反受其殃。有诗为证,诗曰:
恩将恩报人间少,反把恩慈变作仇。
下水救人终有失,三思行事却无忧。

三藏师徒们将着金银服饰拿转,正行处,忽见那枪刀簇簇而来。三藏大惊道:
“徒弟,你看那兵器簇拥相临,是甚好歹?”八戒道:“祸来了,祸来了!这是那放
去的强盗,他取了兵器,又伙了些人,转过路来与我们斗杀也!”沙僧道:“二哥,
那来的不是贼势。大哥,你仔细观之。”行者悄悄的向沙僧道:“师父的灾星又到了,
此必是官兵捕贼之意。”说不了,众兵卒至边前,撒开个圈子阵,把他师徒围住道:
“好和尚!打劫了人家东西,还在这里摇摆哩!”一拥上前,先把唐僧抓下马来,用
绳捆了;又把行者三人,也一齐捆了;穿上扛子,两个抬一个,赶着马,夺了担,
径转府城。只见那:

唐三藏,战战兢兢,滴泪难言;猪八戒,絮絮叨叨,心中报怨;沙和尚,囊突
突,意下踌躇;孙行者,笑唏唏,要施手段。
众官兵攒拥扛抬,须臾间,拿到城里。径自解上黄堂报道:“老爷,民快人等捕获
强盗来了!”那刺史端坐堂上,赏劳了民快,捡看了贼赃,当叫寇家领去。却将三
藏等提近厅前,问道:“你这起和尚,口称是东土远来,向西天拜佛,却原来是些
设法看门路,打家劫舍之贼!”三藏道:“大人容告:贫僧实不是贼,决不敢假,
随身现有通关文牒可照。只因寇员外家斋我等半月,情意深重,我等路遇强盗,夺
转打劫寇家的财物,因送还寇家报恩,不期民快人等捉获,以为是贼,实不是贼。
望大人详察。”刺史道:“你这厮见官兵捕获,却巧言报恩。既是路遇强盗,何不连
他捉来,报官报恩?如何只是你四众!你看!寇梁递得失状,坐名告你,你还敢展挣?”

三藏闻言,一似大海烹舟,魂飞魄丧。叫:“悟空,你何不上来折辨?”行者
道:“有赃是实,折辨何为!”刺史道:“正是啊!赃证现存,还敢抵赖?”叫手下:
“拿脑箍来,把这秃贼的光头箍他一箍,然后再打!”行者慌了,心中暗想道:“虽
是我师父该有此难,还不可教他十分受苦。”他见那皂隶们收拾索子,结脑箍,即
便开口道:“大人且莫箍那个和尚。昨夜打劫寇家,点火的也是我,持刀的也是我,
劫财的也是我,杀人的也是我。我是个贼头,要打只打我,与他们无干。但只不放
我便是。”刺史闻言,就教:“先箍起这个来。”皂隶们齐来上手,把行者套上脑箍,
收紧了一勒,扑的把索子断了。又结又箍,又扑的断了。一连箍了三四次,他
的头皮,皱也不曾皱一些儿。却又换索子再结时,只听得有人来报道:“老爷,都
下陈少保爷爷到了,请老爷出郭迎接。”那刺史即命刑房吏:“把贼收监,好生看辖。
待我接过上司,再行拷问。”刑房吏遂将唐僧四众,推进监门。八戒、沙僧将自己
行李担进随身。

三藏道:“徒弟,这是怎么起的?”行者笑道:“师父,进去,进去!这里边没
狗叫,倒好耍子!”可怜把四众捉将进去,一个个都推入辖床,扣拽了滚肚、敌脑、
攀胸。禁子们又来乱打。三藏苦痛难禁,只叫:“悟空!怎的好,怎的好!”行者道:
“他打是要钱哩。常言道:‘好处安身,苦处用钱。’如今与他些钱,便罢了。”三
藏道:“我的钱自何来?”行者道:“若没钱,衣物也是。把那袈裟与了他罢。”

三藏听说,就如刀刺其心。一时间见他打不过了,只得开言道:“悟空,随你
罢。”行者便叫:“列位长官,不必打了。我们担进来的那两个包袱中,有一件锦
袈裟,价值千金。你们解开拿了去罢。”众禁子听言,一齐动手,把两个包袱解看。
虽有几件布衣,虽有个引袋,俱不值钱。只见几层油纸包裹着一物,霞光焰焰,知
是好物。抖开看时,但只见:
巧妙明珠缀,稀奇佛宝攒。
盘龙铺绣结,飞凤锦沿边。
众皆争看,又惊动本司狱官。走来喝道:“你们在此嚷甚的?”禁子们跪道:“老爹
才子却提控送下四个和尚,乃是大伙强盗。他见我们打了他几下,把这两个包袱与
我。我们打开看时,见有此物,无可处置。若众人扯破分之,其实可惜,若独归一
人,众人无利。幸老爹来,凭老爹做个劈着。”狱官见了,乃是一件袈裟,又将别
项衣服,并引袋儿通检看了。又打开袋内关文一看,见有各国的宝印花押,道:“早
是我来看呀!不然,你们都撞出事来了。这和尚不是强盗。切莫动他衣物。待明日
太爷再审,方知端的。”众禁子听言,将包袱还与他,照旧包裹,交与狱官收讫。

渐渐天晚,听得楼头起鼓,火甲巡更。捱至四更三点,行者见他们都不呻吟,
尽皆睡着。他暗想道:“师父该有这一夜牢狱之灾。老孙不开口折辨,不使法力者,
盖为此耳。如今四更将尽,灾将满矣,我须去打点打点,天明好出牢门。”你看他
弄本事,将身小一小,脱出辖床,摇身一变,变做个蜢虫儿,从房檐瓦缝里飞出。
见那星光月皎,正是清和夜静之天,他认了方向,径飞向寇家门首。只见那街西下
一家儿灯火明亮。又飞近他门口看时,原来是个做豆腐的。见一个老头儿烧火,妈
妈儿挤浆。

那老儿忽的叫声:“妈妈,寇大官且是有子有财,只是没寿。我和他小时,同
学读书,我还大他五岁。他老子叫做寇铭,当时也不上千亩田地,放些租帐,也讨
不起。他到二十岁时,那铭老儿死了,他掌着家当,其实也是他一步好运。娶的妻
是那张旺之女,小名叫做穿针儿,却倒旺夫。自进他门,种田又收,放帐又起;买
着的有利,做着的赚钱,被他如今挣了有十万家私。他到四十岁上,就回心向善,
斋了万僧。不期昨夜被强盗踢死。可怜!今年才六十四岁,正好享用,何期这等向
善,不得好报,乃死于非命?可叹!可叹!”

行者一一听之,却早五更初点。他就飞入寇家,只见那堂屋里已停着棺材,材
头边点着灯,摆列着香烛花果,妈妈在旁啼哭;又见他两个儿子也来拜哭,两个媳
妇拿两盏饭儿供献。行者就钉在他材头上,咳嗽了一声。唬得那两个媳妇,查手舞
脚的往外跑;寇梁兄弟伏在地下,不敢动。只叫:“爹爹!,,!……”那妈
妈子胆大,把材头扑了一把道:“老员外,你活了?”行者学着那员外的声音道:“我
不曾活。”两个儿子一发慌了,不住的叩头垂泪,只叫:“爹爹!,,!”妈妈
子硬着胆,又问道:“员外,你不曾活,如何说话?”行者道:“我是阎王差鬼使押
将来家与你们讲说的。”……说道:“那张氏穿针儿枉口诳舌,陷害无辜。”那妈妈
子听见叫他小名,慌得跪倒磕头道:“好老儿啊!这等大年纪还叫我的小名儿!我那
些枉口诳舌,害甚么无辜?”行者喝道:“那里有个甚么‘唐僧点着火,八戒叫杀
人。沙僧劫出金银去,行者打死你父亲’?只因你诳言,把那好人受难:那唐朝四
位老师,路遇强徒,夺将财物,送来谢我,是何等好意!你却假捻失状,着儿子们
首官,官府又未细审;又如今把他们监禁,那狱神、土地、城隍俱慌了,坐立不宁,
报与阎王。阎王转差鬼使押解我来家,教你们趁早解放他去;不然,教我在家搅闹
一月,将合门老幼并鸡狗之类,一个也不存留!”寇梁兄弟又磕头哀告道:“爹爹请
回,切莫伤残老幼。待天明就去本府投递解状,愿认招回,只求存殁均安也。”行
者听了,即叫:“烧纸,我去呀!”他一家儿都来烧纸。

行者一翅飞起,径又飞至刺史住宅里面。低头观看,那房内里已有灯光,见刺
史已起来了。他就飞进中堂看时,只见中间后壁挂着一轴画儿,是一个官儿骑着一
匹点子马,有几个从人,打着一把青伞,搴着一张交床,更不识是甚么故事,行者
就钉在中间。忽然那刺史自房里出来,弯着腰梳洗。行者猛的里咳嗽一声,把刺史
唬得慌慌张张,走入房内。梳洗毕,穿了大衣,即出来对着画儿焚香祷告道:“伯
考姜公乾一神位。孝侄姜坤三蒙祖上德荫,忝中甲科,今叨受铜台府刺史,旦夕侍
奉香火不绝,为何今日发声?切勿为邪为祟,恐唬家众。”行者暗笑道:“此是他大
爷的神子!”却就绰着经儿叫道:“坤三贤侄,你做官虽承祖荫,一向清廉,怎的昨
日无知,把四个圣僧当贼,不审来音,囚于禁内!那狱神、土地、城隍不安,报与
阎君,阎君差鬼使押我来对你说,教你推情察理,快快解放他;不然,就教你去阴
司折证也。”刺史听说,心中悚惧道:“大爷请回,小侄升堂,当就释放。”行者道:
“既如此,烧纸来。我去见阎君回话。”刺史复添香烧纸拜谢。

行者又飞出来看时,东方早已发白。及飞到地灵县,又见那合县官却都在堂上。
他思道:“蜢虫儿说话,被人看见,露出马脚来不好。”他就半空中,改了个大法身,
从空里伸下一只脚来,把个县堂满。口中叫道:“众官听着:吾乃玉帝差来的浪
荡游神。说你这府监里屈打了取经的佛子,惊动三界诸神不安,教吾传说,趁早放
他;若有差池,教我再来一脚,先踢死合府县官,后死四境居民,把城池都踏为
灰烬!”概县官吏人等,慌得一齐跪倒,磕头礼拜道:“上圣请回。我们如今进府,
禀上府尊,即教放出。千万莫动脚,惊唬死下官。”行者才收了法身,仍变做个蜢
虫儿,从监房瓦缝儿飞入,依旧钻在辖床中间睡着。

却说那刺史升堂,才抬出投文牌去,早有寇梁兄弟,抱牌跪门叫喊。刺史着令
进来。二人将解状递上。刺史见了,发怒道:“你昨日递了失状,就与你拿了贼来,
你又领了赃去,怎么今日又来递解状?”二人滴泪道:“老爷,今夜小的父亲显魂
道:‘唐朝圣僧,原将贼徒拿住,夺获财物,放了贼去,好意将财物送还我家报恩,
怎么反将他当贼,拿在狱中受苦?狱中土地城隍俱不安,报了阎王,阎王差鬼使押
解我来教你赴府再告,释放唐僧,庶免灾咎:不然,老幼皆亡。’因此,特来递个
解词。望老爷方便,方便!”刺史听他说了这话,却暗想道:“他那父亲,乃是热尸
新鬼,显魂报应犹可;我伯父死去五六年了,却怎么今夜也来显魂,教我审放?看
起来必是冤枉。”

正忖度间,只见那地灵县知县等官,急急跑上堂,乱道:“老大人,不好了,
不好了!适才玉帝差浪荡游神下界,教你快放狱中好人。昨日拿的那些和尚,不是
强盗,都是取经的佛子。若少迟延,就要踢杀我等官员,还要把城池连百姓俱尽踏
为灰烬。”刺史又大惊失色,即叫刑房吏火速写牌提出。当时开了监门提出。八戒
愁道:“今日又不知怎的打哩。”行者笑道:“管你一下儿也不敢打。老孙俱已干办
停当。上堂切不可下跪,他还要下来请我们上坐。却等我问他要行李,要马匹。少
了一些儿,等我打他你看。”

说不了,已至堂口。那刺史、知县并府县大小官员,一见都下来迎接道:“圣
僧昨日来时,一则接上司忙迫,二则又见了所获之赃,未及细问端的。”唐僧合掌
躬身,又将前情细陈了一遍。众官满口认称,都道:“错了,错了!莫怪,莫怪!”
又问狱中可曾有甚疏失。行者近前努目睁看,厉声高叫道:“我的白马是堂上人得
了,行李是狱中人得了,快快还我!今日却该我拷较你们了,枉拿平人做贼,你们
该个甚罪?”府县官见他作恶,无一个不怕,即便叫收马的牵马来,收行李的取行
李来,一一交付明白。你看他三人一个个逞凶,众官只以寇家遮饰。三藏劝解了道:
“徒弟,是也不得明白。我们且到寇家去,一则吊问,二来与他对证对证,看是何
人见我做贼。”行者道:“说得是。等老孙把那死的叫起来,看是那个打他。”

沙僧就在府堂上把唐僧撮上马,喝喝,一拥而出。那些府县多官,也一一
俱到寇家。唬得那寇梁兄弟在门前不住的磕头,接进厅。只见他孝堂之中,一家儿
都在孝幔里啼哭。行者叫道:“那打诳语栽害平人的妈妈子,且莫哭!等老孙叫你老
公来,看他说是那个打死的,羞他一羞!”众官员只道孙行者说的是笑话。行者道:
“列位大人,略陪我师父坐坐。八戒、沙僧,好生保护。等我去了就来。”

好大圣,跳出门,望空就起。只见那遍地彩霞笼住宅,一天瑞气护元神。众等
方才认得是个腾云驾雾之仙,起死回生之圣。这里一一焚香礼拜不题。

那大圣一路筋斗云,直至幽冥地界,径撞入森罗殿上,慌得那:

十代阎君拱手接,五方鬼判叩头迎。千株剑树皆侧,万迭刀山尽坦平。枉死
城中魑魅化,奈河桥下鬼超生。正是那神光一照如天赦,黑暗阴司处处明。
十阎王接下大圣,相见了,问及何来何干。行者道:“铜台府地灵县斋僧的寇洪之
鬼,是那个收了?快点查来与我。”十阎王道:“寇洪善士,也不曾有鬼使勾他,他
自家到此,遇着地藏王的金衣童子,他引见地藏也。”行者即别了,径至翠云宫,
见地藏王菩萨。菩萨与他礼毕,具言前事。菩萨喜道:“寇洪阳寿,止该卦数,命
终,不染床席,弃世而来。我因他斋僧,是个善士,收他做个掌善缘簿子的案长。
既大圣来取,我再延他阳寿一纪,教他跟大圣去。”金衣童子遂领出寇洪。寇洪见
了行者,声声叫道:“老师,老师,救我一救!”行者道:“你被强盗踢死。此乃阴
司地藏王菩萨之处。我老孙特来取你到阳世间,对明此事。既蒙菩萨放回,又延你
阳寿一纪,待十二年之后,你再来也。”那员外顶礼不尽。

行者谢辞了菩萨,将他吹化为气,掉于衣袖之间,同去幽府,复返阳间。驾云
头,到了寇家。即唤八戒捎开材盖,把他魂灵儿推付本身。须臾间,透出气来活了。

那员外爬出材来,对唐僧四众磕头道:“师父,师父,寇洪死于非命,蒙师父
至阴司救活,乃再造之恩!”言谢不已。及回头,见各官罗列,即又磕头道:“列位
老爹都如何在舍?”那刺史道:“你儿子始初递失状,坐名告了圣僧,我即差人捕
获;不期圣僧路遇杀劫你家之贼,夺取财物,送还你家;是我下人误捉,未得详审,
当送监禁。今夜被你显魂,我先伯亦来家诉告;县中又蒙浪荡游神下界;一时就有
这许多显应,所以放出圣僧,圣僧却又去救活你也。”那员外跪道:“老爹,其实枉
了这四位圣僧!那夜有三十多名强盗,明火执杖,劫去家私,是我难舍,向贼理说,
不期被他一脚,撩阴踢死,与这四位何干!”叫过妻子来:“是谁人踢死,你等辄敢
妄告?请老爹定罪。”当时一家老小,只是磕头。刺史宽恩,免其罪过。寇洪教安排
筵宴,酬谢府县厚恩。个个未坐回衙。至次日,再挂斋僧牌,又款留三藏;三藏决
不肯住。却又请亲友,办旌幢,如前送行而去。咦!这正是:
地辟能存凶恶事,天高不负善心人。
逍遥稳步如来径,只到灵山极乐门。

毕竟不知见佛何如,且听下回分解。