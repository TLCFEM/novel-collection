\chapter{云栈洞悟空收八戒~浮屠山玄奘受心经}

却说那怪的火光前走,这大圣的彩霞随跟。正行处,忽见一座高山,那怪把红
光结聚,现了本相,撞入洞里,取出一柄九齿钉钯来战。行者喝一声道:“泼怪!你
是那里来的邪魔?怎么知道我老孙的名号?你有甚么本事,实实供来,饶你性命!”
那怪道:“是你也不知我的手段!上前来站稳着,我说与你听:我

自小生来心性拙,贪闲爱懒无休歇。不曾养性与修真,混沌迷心熬日月。忽然
闲里遇真仙,就把寒温坐下说。劝我回心莫堕凡,伤生造下无边孽。有朝大限命终
时,八难三途悔不喋。听言意转要修行,闻语心回求妙诀。有缘立地拜为师,指示
天关并地阙。得传九转大还丹,工夫昼夜无时辍。上至顶门泥丸宫,下至脚板涌泉
穴。周流肾水入华池,丹田补得温温热。婴儿姹女配阴阳,铅汞相投分日月。离龙
坎虎用调和,灵龟吸尽金乌血。三花聚顶得归根,五气朝元通透彻。功圆行满却飞
升,天仙对对来迎接。朗然足下彩云生,身轻体健朝金阙。玉皇设宴会群仙,各分
品级排班列。敕封元帅管天河,总督水兵称宪节。只因王母会蟠桃,开宴瑶池邀众
客。那时酒醉意昏沉,东倒西歪乱撒泼。逞雄撞入广寒宫,风流仙子来相接。见他
容貌挟人魂,旧日凡心难得灭。全无上下失尊卑,扯住嫦娥要陪歇。再三再四不依
从,东躲西藏心不悦。色胆如
天叫似雷,险些震倒天关阙。纠察灵官奏玉皇,那日吾当命运拙。广寒围困不通风,
进退无门难得脱。却被诸神拿住我,酒在心头还不怯。押赴灵霄见玉皇,依律问成
该处决。多亏太白李金星,出班俯亲言说。改刑重责二千锤,肉绽皮开骨将折。放
生遭贬出天关,福陵山下图家业。我因有罪错投胎,俗名唤做猪刚鬣。”
行者闻言道:“你这厮原来是天蓬水神下界。怪道知我老孙名号。”那怪道声:“哏!
你这诳上的弼马温,当年撞那祸时,不知带累我等多少,今日又来此欺人!不要无
礼,吃我一钯!”行者怎肯容情,举起棒,当头就打。他两个在那半山之中,黑夜
里赌斗。好杀:

行者金睛似闪电,妖魔环眼似银花。这一个口喷彩雾,那一个气吐红霞;气吐
红霞昏处亮,口喷彩雾夜光华。金箍棒,九齿钯,两个英雄实可夸:一个是大圣临
凡世,一个是元帅降天涯。那个因失威仪成怪物,这个幸逃苦难拜僧家。钯去好似
龙伸爪,棒迎浑若凤穿花。那个道:“你破人亲事如杀父!”这个道:“你强奸幼女
正该拿!”闲言语,乱喧哗,往往来来棒架钯。看看战到天将晓,那妖精两膊觉酸
麻。
他两个自二更时分,直斗到东方发白。那怪不能迎敌,败阵而逃,依然又化狂风,
径回洞里,把门紧闭,再不出头。行者在这洞门外看有一座石碣,上书“云栈洞”
三字;见那怪不出,天又大明,心却思量:“恐师父等候,且回去见他一见,再来
捉此怪不迟。”随踏云点一点,早到高老庄。

却说三藏与那诸老谈今论古,一夜无眠。正想行者不来,只见天井里,忽然站
下行者。行者收藏铁棒,整衣上厅。叫道:“师父,我来了。”慌得那诸老一齐下拜,
谢道:“多劳,多劳!”三藏问道:“悟空,你去这一夜,拿得妖精在那里?”行者
道:“师父,那妖不是凡间的邪祟,也不是山间的怪兽。他本是天蓬元帅临凡,只
因错投了胎,嘴脸像一个野猪模样,其实性灵尚存。他说以相为姓,唤名猪刚鬣。
是老孙从后宅里掣棒就打,他化一阵狂风走了。被老孙着风一棒,他就化道火光,
径转他那本山洞里,取出一柄九齿钉钯,与老孙战了一夜。适才天色将明,他怯战
而走,把洞门紧闭不出。老孙还要打开那门,与他见个好歹,恐师父在此疑虑盼望,
故先来回个信息。”

说罢,那老高上前跪下道:“长老,没及奈何,你虽赶得去了,他等你去后复
来,却怎区处?索性累你与我拿住,除了根,才无后患。我老夫不敢怠慢,自有重
谢:将这家财田地,凭众亲友写立文书,与长老平分。只是要剪草除根,莫教坏了
我高门清德。”

行者笑道:“你这老儿不知分限。那怪也曾对我说,他虽是食肠大,吃了你家
些茶饭,他与你干了许多好事。这几年挣了许多家资,皆是他之力量。他不曾白吃
了你东西,问你祛他怎的。据他说,他是一个天神下界,替你把家做活,又未曾害
了你家女儿。想这等一个女婿,也门当户对,不怎么坏了家声,辱了行止。当真的
留他也罢。”老高道:“长老,虽是不伤风化,但名声不甚好听。动不动着人就说:
‘高家招了一个妖怪女婿!’这句话儿教人怎当?”三藏道:“悟空,你既是与他做
了一场,一发与他做个竭绝,才见始终。”行者道:“我才试他一试耍子。此去一定
拿来与你们看。且莫忧愁。”叫:“老高,你还好生管待我师父,我去也。”

说声去,就无形无影的,跳到他那山上,来到洞口,一顿铁棍,把两扇门打得
粉碎。口里骂道:“那馕糠的夯货,快出来与老孙打么!”那怪正喘嘘嘘的,睡在洞
里。听见打得门响,又听见骂馕糠的夯货,他却恼怒难禁,只得拖着钯,抖擞精神,
跑将出来,厉声骂道:“你这个弼马温,着实惫懒!与你有甚相干,你把我大门打破?
你且去看看律条,打进大门而入,该个杂犯死罪哩!”行者笑道:“这个呆子!我就
打了大门,还有个辨处。像你强占人家女子,又没个三媒六证,又无些茶红酒礼,
该问个真犯斩罪哩!”那怪道:“且休闲讲,看老猪这钯!”行者使棍支住道:“你这
钯可是与高老家做园工筑地种菜的?有何好处怕你!”那怪道:“你错认了!这钯岂是
凡间之物?你且听我道来:

此是煅炼神冰铁,磨琢成工光皎洁。老君自己动钤锤,荧惑亲身添炭屑。五方
五帝用心机,六丁六甲费周折。造成九齿玉垂牙,铸就双环金坠叶。身妆六曜排五
星,体按四时依八节。短长上下定乾坤,左右阴阳分日月。六爻神将按天条,八卦
星辰依斗列。名为上宝逊金钯,进与玉皇镇丹阙。因我修成大罗仙,为吾养就长生
客。敕封元帅号天蓬,钦赐钉钯为御节。举起烈焰并毫光,落下猛风飘瑞雪。天曹
神将尽皆惊,地府阎罗心胆怯。人间那有这般兵,世上更无此等铁。随身变化可心
怀,任意翻腾依口诀。相携数载未曾离,伴我几年无日别。日食三餐并不丢,夜眠
一宿浑无撇。也曾佩去赴蟠桃,也曾带他朝帝阙。皆因仗酒却行凶,只为倚强便撒
泼。上天贬我降凡尘,下世尽我作罪孽。石洞心邪曾吃人,高庄情喜婚姻结。这钯
下海掀翻龙鼍窝,上山抓碎虎狼穴。诸般兵刃且休题,惟有吾当钯最切。相持取胜
有何难,赌斗求功不用说。何怕你铜头铁脑一身钢,钯到魂消神气泄!”

行者闻言,收了铁棒道:“呆子不要说嘴!老孙把这头伸在那里,你且筑一下儿,
看可能魂消气泄。”那怪真个举起钯,着气力筑将来。扑的一下,钻起钯的火光焰
焰,更不曾筑动一些儿头皮。唬得他手麻脚软,道声“好头!好头!”行者道:“你
是也不知。老孙因为闹天宫,偷了仙丹,盗了蟠桃,窃了御酒,被小圣二郎擒住,
押在斗牛宫前,众天神把老孙斧剁锤敲,刀砍剑刺,火烧雷打,也不曾损动分毫。
又被那太上老君拿了我去,放在八卦炉中,将神火煅炼,炼做个火眼金睛,铜头铁
臂。不信,你再筑几下,看看疼与不疼。”

那怪道:“你这猴子,我记得你闹天宫时,家住在东胜神洲傲来国花果山水帘
洞里,到如今久不闻名,你怎么来到这里,上门子欺我?莫敢是我丈人去那里请你
来的?”行者道:“你丈人不曾去请我。因是老孙改邪归正,弃道从僧,保护一个
东土大唐驾下御弟,叫做三藏法师,往西天拜佛求经,路过高庄借宿,那高老儿因
话说起,就请我救他女儿,拿你这馕糠的夯货!”

那怪一闻此言,丢了钉钯,唱个大喏道:“那取经人在那里?累烦你引见,引见。”
行者道:“你要见他怎的?”那怪道:“我本是观世音菩萨劝善,受了他的戒行,这
里持斋把素,教我跟随那取经人往西天拜佛求经,将功折罪,还得正果。教我等他,
这几年不闻消息。今日既是你与他做了徒弟,何不早说取经之事,只倚凶强,上门
打我?”行者道:“你莫诡诈,欺心软我,欲为脱身之计。果然是要保护唐僧,略
无虚假,你可朝天发誓,我才带你去见我师父。”那怪扑的跪下,望空似捣碓的一
般,只管磕头道:“阿弥陀佛,南无佛,我若不是真心实意,还教我犯了天条,劈
尸万段!”

行者见他赌咒发愿,道:“既然如此,你点把火来烧了你这住处,我方带你去。”
那怪真个搬些芦苇荆棘,点着一把火,将那云栈洞烧得像个破瓦窑。对行者道:“我
今已无挂碍了,你却引我去罢。”行者道:“你把钉钯与我拿着。”那怪就把钯递与
行者。行者又拔了一根毫毛,吹口仙气,叫“变!”即变做一条三股麻绳,走过来,
把手背绑剪了。那怪真个倒背着手,凭他怎么绑缚。却又揪着耳朵,拉着他,叫“快
走!快走!”那怪道:“轻着些儿!你的手重,揪得我耳根子疼。”行者道:“轻不成,
顾你不得,常言道:‘善猪恶拿。’只等见了我师父,果有真心,方才放你。”他两
个半云半雾的,径转高家庄来。有诗为证:
金性刚强能克木,心猿降得木龙归。
金从木顺皆为一,木恋金仁总发挥。
一主一宾无间隔,三交三合有玄微。
性情并喜贞元聚,同证西方话不违。

顷刻间,到了庄前。行者着他的钯,揪着他的耳道:“你看那厅堂上端坐的
是谁?乃吾师也。”那高氏诸亲友与老高,忽见行者把那怪背绑揪耳而来,一个个欣
然迎到天井中,道声“长老,长老!他正是我家的女婿”。那怪走上前,双膝跪下,
背着手,对三藏叩头,高叫道:“师父,弟子失迎。早知是师父住在我丈人家,我
就来拜接,怎么又受到许多泼折?”三藏道:“悟空,你怎么降得他来拜我?”行
者才放了手,拿钉钯柄儿打着,喝道:“呆子!你说么!”那怪把菩萨劝善事情,细
陈了一遍。

三藏大喜。便叫:“高太公,取个香案用用。”老高即忙抬出香案。三藏净了手
焚香,望南礼拜道:“多蒙菩萨圣恩!”那几个老儿也一齐添香礼拜。拜罢,三藏上
厅高坐,教:“悟空放了他绳。”行者才把身抖了一抖,收上身来,其缚自解。那怪
从新礼拜三藏,愿随西去。又与行者拜了,以先进者为兄,遂称行者为师兄。三藏
道:“既从吾善果,要做徒弟,我与你起个法名,早晚好呼唤。”他道:“师父,我
是菩萨已与我摩顶受戒,起了法名,叫做猪悟能也。”三藏笑道:“好,好,你师兄
叫做悟空,你叫做悟能,其实是我法门中的宗派。”悟能道:“师父,我受了菩萨戒
行,断了五荤三厌,在我丈人家持斋把素,更不曾动荤;今日见了师父,我开了斋
罢。”三藏道:“不可,不可!你既是不吃五荤三厌,我再与你起个别名,唤为八戒。”
那呆子欢欢喜喜道:“谨遵师命。”因此又叫做猪八戒。

高老见这等去邪归正,更十分喜悦。遂命家僮安排筵宴,酬谢唐僧。八戒上前
扯住老高道:“爷,请我拙荆出来拜见公公、伯伯,如何?”行者笑道:“贤弟,你
既入了沙门,做了和尚,从今后,再莫题起那‘拙荆’的话说。世间只有个火居道
士,那里有个火居的和尚?我们且来叙了坐次,吃顿斋饭,赶早儿往西天走路。”

高老儿摆了桌席,请三藏上坐。行者与八戒,坐于左右两旁。诸亲下坐。高老
把素酒开樽,满斟一杯,奠了天地,然后奉与三藏。三藏道:“不瞒太公说,贫僧
是胎里素,自幼儿不吃荤。”老高道:“因知老师清素,不曾敢动荤。此酒也是素的,
请一杯不妨。”三藏道:“也不敢用酒。酒是我僧家第一戒者。”悟能慌了道:“师父,
我自持斋,却不曾断酒。”悟空道:“老孙虽量窄,吃不上坛把,却也不曾断酒。”
三藏道:“既如此,你兄弟们吃些素酒也罢。只是不许醉饮误事。”遂而他两个接了
头钟。各人俱照旧坐下,摆下素斋。说不尽那杯盘之盛,品物之丰。

师徒们宴罢,老高将一红漆丹盘,拿出二百两散碎金银,奉三位长老为途中之
费;又将三领绵布褊衫,为上盖之衣。三藏道:“我们是行脚僧,遇庄化饭,逢处
求斋,怎敢受金银财帛?”行者近前,轮开手,抓了一把。叫:“高才,昨日累你
引我师父,今日招了一个徒弟,无物谢你,把这些碎金碎银,权作带领钱,拿了去
买草鞋穿。以后但有妖精,多作成我几个,还有谢你处哩。”高才接了,叩头谢赏。
老高又道:“师父们既不受金银,望将这粗衣笑纳,聊表寸心。”三藏又道:“我出
家人,若受了一丝之贿,千劫难修。只是把席上吃不了的饼果,带些去做干粮足矣。”
八戒在旁边道:“师父、师兄,你们不要便罢,我与他家做了这几年女婿,就是挂
脚粮也该三石哩。——丈人啊,我的直裰,昨晚被师兄扯破了,与我一件青锦袈裟;
鞋子绽了,与我一双好新鞋子。”高老闻言,不敢不与。随买一双新鞋,将一领褊
衫,换下旧时衣物。

那八戒摇摇摆摆,对高老唱个喏道:“上复丈母、大姨、二姨并姨夫、姑舅诸
亲:我今日去做和尚了,不及面辞,休怪。丈人啊,你还好生看待我浑家:只怕我
们取不成经时,好来还俗,照旧与你做女婿过活。”行者喝道:“夯货,却莫胡说!”
八戒道:“哥呵,不是胡说,只恐一时间有些儿差池,却不是和尚误了做,老婆误
了娶,两下里都耽搁了?”三藏道:“少题闲话,我们赶早儿去来。”遂此收拾了一
担行李,八戒担着;背了白马,三藏骑着;行者肩担铁棒,前面引路。一行三众,
辞别高老及众亲友,投西而去。有诗为证。诗曰:
满地烟霞树色高,唐朝佛子苦劳劳。
饥餐一钵千家饭,寒着千针一衲袍。
意马胸头休放荡,心猿乖劣莫教嚎。
情和性定诸缘合,月满金华是伐毛。

三众进西路途,有个月平稳。行过了乌斯藏界,猛抬头见一座高山。三藏停鞭
勒马道:“悟空、悟能,前面山高,须索仔细,仔细。”八戒道:“没事。这山唤做
浮屠山,山中有一个乌巢禅师,在此修行。老猪也曾会他。”三藏道:“他有些甚么
勾当?”八戒道:“他倒也有些道行。他曾劝我跟他修行,我不曾去罢了。”师徒们
说着话,不多时,到了山上。好山!但见那:

山南有青松碧桧,山北有绿柳红桃。闹聒聒,山禽对语;舞翩翩,仙鹤齐飞。
香馥馥,诸花千样色;青冉冉,杂草万般奇。涧下有滔滔绿水,崖前有朵朵祥云。
真个是景致非常幽雅处,寂然不见往来人。
那师父在马上遥观,见香桧树前,有一柴草窝。左边有麋鹿衔花,右边有山猴献果。
树梢头,有青鸾彩凤齐鸣,玄鹤锦鸡咸集。八戒指道:“那不是乌巢禅师!”三藏纵
马加鞭,直至树下。

却说那禅师见他三众前来,即便离了巢穴,跳下树来。三藏下马奉拜,那禅师
用手搀道:“圣僧请起。失迎,失迎。”八戒道:“老禅师,作揖了。”禅师惊问道:
“你是福陵山猪刚鬣,怎么有此大缘,得与圣僧同行?”八戒道:“前年蒙观音菩
萨劝善,愿随他做个徒弟。”禅师大喜道:“好,好,好!”又指定行者,问道:“此
位是谁?”行者笑道:“这老禅怎么认得他,倒不认得我?”禅师道:“因少识耳。”
三藏道:“他是我的大徒弟孙悟空。”禅师陪笑道:“欠礼,欠礼。”

三藏再拜,请问西天大雷音寺还在那里。禅师道:“远哩!远哩!只是路多虎豹,
难行。”三藏殷勤致意,再问:“路途果有多远?”禅师道:“路途虽远,终须有到
之日,却只是魔瘴难消。我有《多心经》一卷,凡五十四句,共计二百七十字。若
遇魔瘴之处,但念此经,自无伤害。”三藏拜伏于地恳求,那禅师遂口诵传之。经
云:

《摩诃般若波罗蜜多心经》。观自在菩萨,行深般若波罗蜜多,时照见五蕴皆
空,度一切苦厄。舍利子,色不异空,空不异色;色即是空,空即是色。受想行识,
亦复如是。舍利子,是诸法空相,不生不灭,不垢不净,不增不减。是故空中无色,
无受想行识,无眼耳鼻舌身意,无色声香味触法,无眼界,乃至无意识界,无无明,
亦无无明尽。乃至无老死,亦无老死尽。无苦寂灭道,无智亦无得。以无所得故,
菩提萨。依般若波罗蜜多故,心无挂碍;无挂碍故,无有恐怖;远离颠倒梦想,
究竟涅。三世诸佛,依般若波罗蜜多故,得阿耨多罗三藐三菩提。故知般若波罗
蜜多,是大神咒,是大明咒,是无上咒,是无等等咒,能除一切苦,真实不虚。故
说般若波罗蜜多咒,即说咒曰:“揭谛,揭谛!波罗揭谛,波罗僧揭谛!菩提萨婆诃!”
此时唐朝法师本有根源,耳闻一遍《多心经》,即能记忆,至今传世。此乃修真之
总经,作佛之会门也。

那禅师传了经文,踏云光,要上乌巢而去;被三藏又扯住奉告,定要问个西去
的路程端的。那禅师笑云:

“道路不难行,试听我吩咐:千山千水深,多瘴多魔处。若遇接天崖,放心休
恐怖。行来摩耳岩,侧着脚踪步。仔细黑
松林,妖狐多截路。精灵满国城,魔主盈山住。老虎坐琴堂,苍狼为主簿。狮象尽
称王,虎豹皆作御。野猪挑担子,水怪前头遇。多年老石猴,那里怀嗔怒。你问那
相识,他知西去路。”

行者闻言,冷笑道:“我们去,不必问他,问我便了。”三藏还不解其意。那禅
师化作金光,径上乌巢而去。长老往上拜谢。行者心中大怒,举铁棒望上乱捣,只
见莲花生万朵,祥雾护千层。行者纵有搅海翻江力,莫想挽着乌巢一缕藤。三藏见
了,扯住行者道:“悟空,这样一个菩萨,你捣他窝巢怎的?”行者道:“他骂了我
兄弟两个一场去了。”三藏道:“他讲的西天路径,何尝骂你?”行者道:“你那里
晓得?他说‘野猪挑担子’,是骂的八戒;‘多年老石猴’是骂的老孙。你怎么解得
此意?”八戒道:“师兄息怒。这禅师也晓得过去未来之事,但看他‘水怪前头遇’
这句话,不知验否。饶他去罢。”行者见莲花祥雾,近那巢边。只得请师父上马,
下山往西而去。那一去:
管教清福人间少,致使灾魔山里多。

毕竟不知前程端的如何,且听下回分解。