\chapter{禅主吞餐怀鬼孕~黄婆运水解邪胎}

德行要修八百,阴功须积三千。均平物我与亲冤,始合西天本愿。

魔兕刀兵不怯,空劳水火无愆。老君降伏却朝天,笑把青牛牵转。

话说那大路旁叫唤者谁?乃金山山神、土地,捧着紫金钵盂叫道:“圣僧啊,
这钵盂饭是孙大圣向好处化来的。因你等不听良言,误入妖魔之手,致令大圣劳苦
万端,今日方救得出。且来吃了饭,再去走路。莫孤负孙大圣一片恭孝之心也。”
三藏道:“徒弟,万分亏你,言谢不尽!早知不出圈痕,那有此杀身之害。”行者道:
“不瞒师父说。只因你不信我的圈子,却教你受别人的圈子。多少苦楚,可叹!可
叹!”八戒道:“怎么又有个圈子?”行者道:“都是你这孽嘴孽舌的夯货,弄师父
遭此一场大难!着老孙翻天覆地,请天兵水火与佛祖丹砂,尽被他使一个白森森的
圈子套去。如来暗示了罗汉,对老孙说出那妖的根原,才请老君来收伏,却是个青
牛作怪。”三藏闻言,感激不尽道:“贤徒,今番经此,下次定然听你吩咐。”遂此
四人分吃那饭。那饭热气腾腾的。行者道:“这饭多时了,却怎么还热?”土地跪
下道:“是小神知大圣功完,才自热来伺候。”须臾饭毕。收拾了钵盂,辞了土地、
山神。

那师父才攀鞍上马,过了高山。正是:
涤虑洗心皈正觉,餐风宿水向西行。
行够多时,又值早春天气。听了些:

紫燕呢喃,黄鹂:紫燕呢喃香嘴困,黄鹂巧音频。满地落红如布锦,
遍山发翠似堆茵。岭上青梅结豆,崖前古柏留云。野润烟光淡,沙暄日色曛。几处
园林花放蕊,阳回大地柳芽新。
正行处,忽遇一道小河,澄澄清水,湛湛寒波。唐长老勒过马观看,远见河那边有
柳阴垂碧,微露着茅屋几椽。行者遥指那厢道:“那里人家,一定是摆渡的。”三藏
道:“我见那厢也似这般,却不见船只,未敢开言。”八戒旋下行李,厉声高叫道:
“摆渡的!撑船过来!”连叫几遍,只见那柳阴里面,咿咿哑哑的,撑出一只船儿。
不多时,相近这岸。师徒们仔细看了那船儿,真个是:

短棹分波,轻桡泛浪:堂油漆彩,板满平仓。船头上铁缆盘窝,船后边舵
楼明亮。虽然是一苇之航,也不亚泛湖浮海。纵无锦缆牙墙,实有松桩桂楫。固不
如万里神舟,真可渡一河之隔。往来只在两崖边,出入不离古渡口。
那船儿须臾顶岸。有梢子叫云:“过河的,这里去。”三藏纵马近前看处,那梢子怎
生模样:

头裹锦绒帕,足踏皂丝鞋。身穿百纳绵裆袄,腰束千针裙布衫。手腕皮粗筋力
硬,眼花眉皱面容衰。声音娇细如莺啭,近观乃是老裙钗。
行者近于船边道:“你是摆渡的?”那妇人道:“是。”行者道:“梢公如何不在,却
着梢婆撑船?”妇人微笑不答,用手拖上跳板。沙和尚将行李挑上去,行者扶着师
父上跳,然后顺过船来,八戒牵上白马,收了跳板。那妇人撑开船,摇动桨,顷刻
间过了河。身登西岸,长老教沙僧解开包,取几文钱钞与他。妇人更不争多寡,将
缆拴在傍水的桩上,笑嘻嘻径入庄屋里去了。

三藏见那水清,一时口渴,便着八戒:“取钵盂,舀些水来我吃。”那呆子道:
“我也正要些儿吃哩。”即取钵盂,舀了一钵,递与师父。师父吃了有一少半,还
剩了多半,呆子接来,一气饮干,却伏侍三藏上马。

师徒们找路西行,不上半个时辰,那长老在马上呻吟道:“腹痛!”八戒随后道:
“我也有些腹痛!”沙僧道:“想是吃冷水了?”说未毕,师父声唤道:“疼的紧!”
八戒也道:“疼得紧!”他两个疼痛难禁,渐渐肚子大了。用手摸时,似有血团肉块,
不住的骨冗骨冗乱动。三藏正不稳便,忽然见那路旁有一村舍,树梢头挑着两个草
把。行者道:“师父,好了。那厢是个卖酒的人家。我们且去化他些热汤与你吃,
就问可有卖药的,讨贴药,与你治治腹痛。”

三藏闻言甚喜,却打白马。不一时,到了村舍门口下马。但只见那门儿外有一
个老婆婆,端坐在草墩上绩麻。行者上前,打个问讯道:“婆婆,贫僧是东土大唐
来的,我师父乃唐朝御弟。因为过河吃了河水,觉肚腹疼痛。”那婆婆喜哈哈的道:
“你们在那边河里吃水来?”行者道:“是,在此东边清水河吃的。”那婆婆欣欣的
笑道:“好耍子!好耍子!你都进来,我与你说。”

行者即搀唐僧,沙僧即扶八戒。两人声声唤唤,腆着肚子,一个个只疼得面黄
眉皱,入草舍坐下。行者只叫:“婆婆,是必烧些热汤与我师父。我们谢你。”那婆
婆且不烧汤,笑唏唏跑走后边,叫道:“你们来看,你们来看!”那里面,蹼蹼踏
的,又走出两三个半老不老的妇人,都来望着唐僧洒笑。行者大怒,喝了一声,把
牙一嗟,唬得那一家子跌跌,往后就走。行者上前,扯住那老婆子道:“快早
烧汤,我饶了你!”那婆子战兢兢的道:“爷爷呀,我烧汤也不济事,也治不得他两
个肚疼。你放了我,等我说。”行者放了他,他说:“我这里乃是西梁女国。我们这
一国尽是女人,更无男子,故此见了你们欢喜。你师父吃的那水不好了。那条河,
唤做子母河。我那国王城外,还有一座迎阳馆驿,驿门外有一个‘照胎泉’。我这
里人,但得年登二十岁以上,方敢去吃那河里水。吃水之后,便觉腹痛有胎。至三
日之后,到那迎阳馆照胎水边照去。若照得有了双影,便就降生孩儿。你师吃了子
母河水,以此成了胎气,也不日要生孩子。热汤怎么治得?”

三藏闻言,大惊失色道:“徒弟啊!似此怎了?”八戒扭腰撒胯的哼道:“爷爷
啊!要生孩子,我们却是男身!那里开得产门?如何脱得出来?”行者笑道:“古人云:
‘瓜熟自落。’若到那个时节,一定从胁下裂个窟窿,钻出来也。”八戒见说,战兢
兢,忍不得疼痛道:“罢了,罢了!死了,死了!”沙僧笑道:“二哥,莫扭,莫扭!
只怕错了养儿肠,弄做个胎前病。”那呆子越发慌了,眼中噙泪,扯着行者道:“哥
哥!你问这婆婆,看那里有手轻的稳婆,预先寻下几个,这半会一阵阵的动荡得紧,
想是摧阵疼。快了!快了!”沙僧又笑道:“二哥,既知摧阵疼,不要扭动,只恐挤
破浆泡耳。”

三藏哼着道:“婆婆啊,你这里可有医家?教我徒弟去买一贴堕胎药吃了,打下
胎来罢。”那婆子道:“就有药也不济事。只是我们这正南街上有一座解阳山,山中
有一个破儿洞,洞里有一眼‘落胎泉’。须得那泉里水吃一口,方才解了胎气。却
如今取不得水了,向年来了一个道人,称名如意真仙,把那破儿洞改作聚仙庵,护
住落胎泉水,不肯善赐与人;但欲求水者,须要花红表礼,羊酒果盘,志诚奉献,
只拜求得他一碗儿水哩。你们这行脚僧,怎么得许多钱财买办?但只可挨命,待时
而生产罢了。”行者闻得此言,满心欢喜道:“婆婆,你这里到那解阳山有几多路
程?”婆婆道:“有三十里。”行者道:“好了,好了!师父放心,待老孙取些水来你
吃。”

好大圣,吩咐沙僧道:“你好好细看着师父。若这家子无礼,侵哄师父,你拿
出旧时手段来,装虎唬他,等我取水去。”沙僧依命。只见那婆子端出一个大瓦
钵来,递与行者道:“拿这钵头儿去,是必多取些来,与我们留着用急。”行者真个
接了瓦钵,出草舍,纵云而去。那婆子才望空礼拜道:“爷爷呀,这和尚会驾云!”
才进去叫出那几个妇人来,对唐僧磕头礼拜,都称为罗汉菩萨。一壁厢烧汤办饭,
供奉唐僧不题。

却说那孙大圣筋斗云起,少顷间见一座山头,阻住云角,即按云光,睁睛看处,
好山!但见那:

幽花摆锦,野草铺蓝。涧水相连落,溪云一样闲。重重谷壑藤萝密,远远峰峦
树木蘩。鸟啼雁过,鹿饮猿攀。翠岱如屏嶂,青崖似髻鬟。尘埃滚滚真难到,泉石
涓涓不厌看。每见仙童采药去,常逢樵子负薪还。果然不亚天台景,胜似三峰西华
山!
这大圣正然观看那山不尽,又只见背阴处,有一所庄院,忽闻得犬吠之声。大圣下
山,径至庄所,却也好个去处。看那:
小桥通活水,茅舍倚青山。
村犬汪篱落,幽人自往还。

不时来至门首,见一个老道人,盘坐在绿茵之上。大圣放下瓦钵,近前道问讯。
那道人欠身还礼道:“那方来者?至小庵有何勾当?”行者道:“贫僧乃东土大唐钦
差西天取经者。因我师父误饮了子母河之水,如今腹疼肿胀难禁。问及土人,说是
结成胎气,无方可治。访得解阳山破儿洞有‘落胎泉’可以消得胎气,故此特来拜
见如意真仙,求些泉水,搭救师父。累烦老道指引指引。”那道人笑道:“此间就是
破儿洞,今改为聚仙庵了。我却不是别人,即是如意真仙老爷的大徒弟。你叫做甚
么名字?待我好与你通报。”行者道:“我是唐三藏法师的大徒弟,贱名孙悟空。”那
道人问曰:“你的花红、酒礼,都在那里?”行者道:“我是个过路的挂搭僧,不曾
办得来。”道人笑道:“你好痴呀!我老师父护住山泉,并不曾白送与人。你回去办
将礼来,我好通报。不然请回。莫想!莫想!”行者道:“人情大似圣旨。你去说我
老孙的名字,他必然做个人情,或者连井都送我也。”

那道人闻此言,只得进去通报。却见那真仙抚琴,只待他琴终,方才说道:“师
父,外面有个和尚,口称是唐三藏大徒弟孙悟空,欲求落胎泉水,救他师父。”那
真仙不听说便罢;一听得说个悟空名字,却就怒从心上起,恶向胆边生;急起身,
下了琴床,脱了素服,换上道衣,取一把如意钩子,跳出庵门。叫道:“孙悟空何
在?”行者转头,观见那真仙打扮:

头戴星冠飞彩艳,身穿金缕法衣红。足下云鞋堆锦绣,腰间宝带绕玲珑。一双
纳锦凌波袜,半露裙闪绣绒。手拿如意金钩子,利杆长若蟒龙。凤眼光明眉
竖,钢牙尖利口翻红。额下髯飘如烈火,鬓边赤发短蓬松。形容恶似温元帅,争奈
衣冠不一同。
行者见了,合掌作礼道:“贫僧便是孙悟空。”那先生笑道:“你真个是孙悟空,却
是假名托姓者?”行者道:“你看先生说话。常言道:‘君子行不更名,坐不改姓。’
我便是悟空。岂有假托之理?”先生道:“你可认得我么?”行者道:“我因归正释
门,秉诚僧教,这一向登山涉水,把我那幼时的朋友也都疏失,未及拜访,少识尊
颜。适间问道子母河西乡人家,言及先生乃如意真仙,故此知之。”那先生道:“你
走你的路,我修我的真,你来访我怎的?”行者道:“因我师父误饮了子母河水,
腹疼成胎,特来仙府,拜求一碗落胎泉水,救解师难也。”

那先生怒目道:“你师父可是唐三藏么?”行者道:“正是,正是。”先生咬牙
恨道:“你们可曾会着一个圣婴大王么?”行者道:“他是号山枯松涧火云洞红孩儿
妖怪的绰号。真仙问他怎的?”先生道:“是我之舍侄。我乃牛魔王的兄弟。前者
家兄处有信来报我,称说唐三藏的大徒弟孙悟空惫懒,将他害了。——我这里正没
处寻你报仇,你倒来寻我,还要甚么水哩!”行者陪笑道:“先生差了。你令兄也曾
与我做朋友,幼年间也曾拜七弟兄。但只是不知先生尊府,有失拜望。如今令侄得
了好处,现随着观音菩萨,做了善财童子,我等尚且不如,怎么反怪我也?”

先生喝道:“这泼猢狲!还弄巧舌!我舍侄还是自在为王好,还是与人为奴好?不
得无礼!吃我这一钩!”大圣使铁棒架住道:“先生莫说打的话,且与些泉水去也。”
那先生骂道:“泼猢狲!不知死活!如若三合敌得我,与你水去;敌不过,只把你剁
为肉酱,方与我侄子报仇。”大圣骂道:“我把你不识起倒的孽障!既要打,走上来
看棍!”那先生如意钩劈手相还。二人在聚仙庵好杀:

圣僧误食成胎水,行者来寻如意仙。那晓真仙原是怪,倚强护住落胎泉。及至
相逢讲仇隙,争持决不遂如然。言来语去成,意恶情凶要报冤。这一个因师伤
命来求水,那一个为侄亡身不与泉。如意钩强如蝎毒,金箍棒狠似龙巅。当胸乱刺
施威猛,着脚斜钩展妙玄。阴手棍丢伤处重,过肩钩起近头鞭。锁腰一棍鹰持雀,
压顶三钩螂捕蝉。往往来来争胜败,返返复复两回还。钩挛棒打无前后,不见输赢
在那边。
那先生与大圣战经十数合,敌不得大圣。这大圣越加猛烈,一条棒似滚滚流星,着
头乱打。先生败了筋力,倒拖着如意钩,往山上走了。

大圣不去赶他,却来庵内寻水。那个道人早把庵门关了。大圣拿着瓦钵,赶至
门前,尽力气一脚,踢破庵门,闯将进去。见那道人伏在井栏上,被大圣喝了一声,
举棒要打,那道人往后跑了。却才寻出吊桶来,正自打水,又被那先生赶到前边,
使如意钩子把大圣钩着脚一跌,跌了个嘴啃地。大圣爬起来,使铁棒就打。他却闪
在旁边,执着钩子道:“看你可取得我的水去!”大圣骂道:“你上来!你上来!我把
你这个孽障,直打杀你!”那先生也不上前拒敌,只是禁住了,不许大圣打水。大
圣见他不动,却使左手轮着铁棒,右手使吊桶,将索子才突鲁鲁的放下。他又来使
钩。大圣一只手撑持不得,又被他一钩钩着脚,扯了个踵,连井索通跌下井去了。
大圣道:“这厮却是无礼!”爬起来,双手轮棒,没头没脸的打将上去。那先生依然
走了,不敢迎敌。大圣又要去取水,奈何没有吊桶,又恐怕来钩扯,心中暗暗想道:
“且去叫个帮手来!”

好大圣,拨转云头,径至村舍门首,叫一声:“沙和尚。”那里边三藏忍痛呻吟,
猪八戒哼声不绝。听得叫唤,二人欢喜道:“沙僧啊,悟空来也。”沙僧连忙出门接
着道:“大哥,取水来了?”大圣进门,对唐僧备言前事。三藏滴泪道:“徒弟啊,
似此怎了?”大圣道:“我来叫沙兄弟与我同去。到那庵边,等老孙和那厮敌斗,
教沙僧乘便取水来救你。”三藏道:“你两个没病的都去了,丢下我两个有病的,教
谁伏侍?”那个老婆婆在旁道:“老罗汉只管放心。不须要你徒弟,我家自然看顾
伏侍你。你们早间到时,我等实有爱怜之意;却才见这位菩萨云来雾去,方知你是
罗汉菩萨。我家决不敢复害你。”

行者“咄”的一声道:“汝等女流之辈,敢伤那个?”老婆子笑道:“爷爷呀,
还是你们有造化,来到我家!若到第二家,你们也不得囫囵了!”八戒哼哼的道:“不
得囫囵,是怎么的?”婆婆道:“我一家儿四五口,都是有几岁年纪的,把那风月
事尽皆休了,故此不肯伤你。若还到第二家,老小众大,那年小之人,那个肯放过
你去!就要与你交合。假如不从,就要害你性命,把你们身上肉,都割了去做香袋
儿哩。”八戒道:“若这等,我决无伤。他们都是香喷喷的,好做香袋;我是个臊猪,
就割了肉去,也是臊的,故此可以无伤。”行者笑道:“你不要说嘴;省些力气,好
生产也。”那婆婆道:“不必迟疑,快求水去。”行者道:“你家可有吊桶?借个使使。”
那婆子即往后边取出一个吊桶,又窝了一条索子,递与沙僧。沙僧道:“带两条索
子去。恐一时井深要用。”

沙僧接了桶索,即随大圣出了村舍,一同驾云而去。那消半个时辰,却到解阳
山界。按下云头,径至庵外。大圣吩咐沙僧道:“你将桶索拿了,且在一边躲着,
等老孙出头索战。你待我两人交战正浓之时,你乘机进去,取水就走。”沙僧谨依
言命。

孙大圣掣了铁棒,近门高叫:“开门!开门!”那守门的看见,急入里通报道:“师
父,那孙悟空又来了也。”那先生心中大怒道:“这泼猴老大无状!一向闻他有些手
段,果然今日方知。他那条棒真是难敌。”道人道:“师父,他的手段虽高,你亦不
亚与他,正是个对手。”先生道:“前面两回,被他赢了。”道人道:“前两回虽赢,
不过是一猛之性;后面两次打水之时,被师父钩他两跌,却不是相比肩也?先既无
奈而去,今又复来,必然是三藏胎成身重,埋怨得紧,不得已而来也。决有慢他师
之心。管取我师决胜无疑。”

真仙闻言,喜孜孜满怀春意,笑盈盈一阵威风,挺如意钩子,走出门来喝道:
“泼猢狲!你又来作甚?”大圣道:“我来只是取水。”真仙道:“泉水乃吾家之井,
凭是帝王宰相,也须表礼羊酒来求,方才仅与些须;况你又是我的仇人,擅敢白手
来取?”大圣道:“真个不与?”真仙道:“不与,不与!”大圣骂道:“泼孽障!既
不与水,看棍!”丢一个架手,抢个满怀,不容说,着头便打。那真仙侧身躲过,
使钩子急架相还。这一场比前更胜。好杀:

金箍棒,如意钩,二人奋怒各怀仇。飞砂走石乾坤暗,播土扬尘日月愁。大圣
救师来取水,妖仙为侄不容求。两家齐努力,一处赌安休。咬牙争胜负,切齿定刚
柔。添机见,越抖擞,喷云嗳雾鬼神愁。朴朴兵兵钩棒响,喊声哮吼振山丘。狂风
滚滚催林木,杀气纷纷过斗牛。大圣愈争愈喜悦,真仙越打越绸缪。有心有意相争
战,不定存亡不罢休。
他两个在庵门外交手,跳跳舞舞的,斗到山坡之下,恨苦相持。不题。

却说那沙和尚提着吊桶,闯进门去,只见那道人在井边挡住道:“你是甚人,
敢来取水!”沙僧放下吊桶,取出降妖宝杖,不对话,着头便打。那道人躲闪不及,
把左臂膊打折,道人倒在地下挣命。沙僧骂道:“我要打杀你这孽畜,怎奈你是个
人身!我还怜你,饶你去罢!让我打水!”那道人叫天叫地的,爬到后面去了。沙僧
却才将吊桶向井中满满的打了一吊桶水,走出庵门,驾起云雾,望着行者喊道:“大
哥,我已取了水去也!饶他罢!饶他罢!”

大圣听得,方才使铁棒支住钩子道:“你听老孙说,我本待斩尽杀绝,争奈你
不曾犯法;二来看你令兄牛魔王的情上。先头来,我被钩了两下,未得水去。才然
来,我是个调虎离山计,哄你出来争战,却着我师弟取水去了。老孙若肯拿出本事
来打你,莫说你是一个甚么如意真仙,就是再有几个,也打死了。正是打死不如放
生,且饶你,教你活几年耳。已后再有取水者,切不可勒他。”那妖仙不识好歹,
演一演,就来钩脚;被大圣闪过钩头,赶上前,喝声:“休走!”那妖仙措手不及,
推了一个蹼辣,挣扎不起。大圣夺过如意钩来,折为两段;总拿着又一抉,抉作四
段,掷之于地道:“泼孽畜!再敢无礼么?”那妖仙战战兢兢,忍辱无言。这大圣笑
呵呵,驾云而起。有诗为证,诗曰:
真铅若炼须真水,真水调和真汞干。
真汞真铅无母气,灵砂灵药是仙丹。
婴儿枉结成胎象,土母施功不费难。
推倒旁门宗正教,心君得意笑容还。

大圣纵着祥光,赶上沙僧。得了真水,喜喜欢欢,回于本处。按下云头,径来
村舍。只见猪八戒腆着肚子,倚在门枋上哼哩。行者悄悄上前道:“呆子,几时占
房的?”呆子慌了道:“哥哥莫取笑。可曾有水来么?”行者还要耍他,沙僧随后
就到,笑道:“水来了!水来了!”三藏忍痛欠身道:“徒弟呀,累了你们也!”那婆
婆却也欢喜,几口儿都出礼拜道:“菩萨呀,却是难得,难得!”即忙取个花磁盏子,
舀了半盏儿,递与三藏道:“老师父,细细的吃;只消一口,就解了胎气。”八戒道:
“我不用盏子,连吊桶等我喝了罢。”那婆子道:“老爷爷,唬杀人罢了!若吃了这
吊桶水,好道连肠子肚子都化尽了!”吓得呆子不敢胡为,也只吃了半盏。

那里有顿饭之时,他两个腹中绞痛,只听“毂辘毂辘”三五阵肠鸣。肠鸣之后,
那呆子忍不住,大小便齐流。唐僧也忍不住要往静处解手。行者道:“师父啊,切
莫出风地里去。怕人子,一时冒了风,弄做个产后之疾。”那婆婆即取两个净桶来,
教他两个方便。须臾间,各行了几遍,才觉住了疼痛,渐渐的销了肿胀,化了那血
团肉块。

那婆婆家又煎些白米粥与他补虚。八戒道:“婆婆,我的身子实落,不用补虚。
且烧些汤水与我洗个澡,却好吃粥。”沙僧道:“哥哥,洗不得澡。坐月子的人弄了
水浆致病。”八戒道:“我又不曾大生,左右只是个小产,怕他怎的?洗洗儿干净。”
真个那婆子烧些汤与他两个净了手脚。唐僧才吃两盏儿粥汤,八戒就吃了十数碗,
还只要添。行者笑道:“夯货!少吃些!莫弄做个‘沙包肚’,不像模样。”八戒道:“没
事,没事,我又不是母猪,怕他做甚?”那家子真个又去收拾煮饭。

老婆婆对唐僧道:“老师父,把这水赐了我罢。”行者道:“呆子,不吃水了?”
八戒道:“我的肚腹也不疼了,胎气想是已行散了。洒然无事,又吃水何为?”行
者道:“既是他两个都好了,将水送你家罢。”那婆婆谢了行者,将余剩之水,装于
瓦罐之中,埋在后边地下,对众老小道:“这罐水,够我的棺材本也!”众老小无不
欢喜。整顿斋饭,调开桌凳,唐僧们吃了斋。消消停停,将息了一宿。

次日天明,师徒们谢了婆婆家,出离村舍。唐三藏攀鞍上马,沙和尚挑着行囊,
孙大圣前边引路,猪八戒拢了缰绳。这里才是:
洗净口孽身干净,销化凡胎体自然。

毕竟不知到国界中还有甚么理会,且听下回分解。