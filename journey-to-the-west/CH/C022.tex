\chapter{八戒大战流沙河~木叉奉法收悟净}

话说唐僧师徒三众,脱难前来,不一日,行过了八百黄风岭,进西却是一脉平
阳之地。光阴迅速,历夏经秋,见了些寒蝉鸣败柳,大火向西流。

正行处,只见一道大水狂澜,浑波涌浪。三藏在马上忙呼道:“徒弟,你看那
前边水势宽阔,怎不见船只行走,我们从那里过去?”八戒见了道:“果是狂澜,
无舟可渡。”那行者跳在空中,用手搭凉篷而看。他也心惊道:“师父啊,真个是难,
真个是难!这条河若论老孙去呵,只消把腰儿扭一扭,就过去了;若师父,诚千分
难渡,万载难行。”三藏道:“我这里一望无边,端的有多少宽阔?”行者道:“径
过有八百里远近。”八戒道:“哥哥怎的定得个远近之数?”行者道:“不瞒贤弟说,
老孙这双眼,白日里常看得千里路上的吉凶。却才在空中看出:此河上下不知多远,
但只见这径过足有八百里。”长老忧嗟烦恼,兜回马,忽见岸上有一通石碑。三众
齐来看时,见上有三个篆字,乃“流沙河”;腹上有小小的四行真字云:

八百流沙界,三千弱水深。
鹅毛飘不起,芦花定底沉。
师徒们正看碑文,只听得那浪涌如山,波翻若岭,河当中滑辣的钻出一个妖精,十
分凶丑:
一头红焰发蓬松,两只圆睛亮似灯。
不黑不青蓝靛脸,如雷如鼓老龙声。
身披一领鹅黄氅,腰束双攒露白藤。
项下骷髅悬九个,手持宝杖甚峥嵘。
那怪一个旋风,奔上岸来,径抢唐僧,慌得行者把师父抱住,急登高岸,回身走脱。
那八戒放下担子,掣出铁钯,望妖精便筑。那怪使宝杖架住。他两个在流沙河岸,
各逞英雄。这一场好斗:

九齿钯,降妖杖,二人相敌河岸上。这个是总督大天蓬,那个是谪下卷帘将。
昔年曾会在灵霄,今日争持赌猛壮。这一个钯去探爪龙,那一个杖架磨牙象。伸开
大四平,钻入迎风戗。这个没头没脸抓,那个无乱无空放。一个是久占流沙界吃人
精,一个是秉教迦持修行将。
他两个来来往往,战经二十回合,不分胜负。

那大圣护了唐僧,牵着马,守定行李,见八戒与那怪交战,就恨得咬牙切齿,
擦掌磨拳,忍不住要去打他,掣出棒来道:“师父,你坐着,莫怕。等老孙和他耍
耍儿来。”那师父苦留不住。他打个唿哨,跳到前边,原来那怪与八戒正战到好处,
难解难分。被行者轮起铁棒,望那怪着头一下,那怪急转身,慌忙躲过,径钻入流
沙河里。气得个八戒乱跳道:“哥啊,谁着你来的!那怪渐渐手慢,难架我钯,再不
上三五合,我就擒住他了!他见你凶险,败阵而逃,怎生是好!”行者笑道:“兄弟,
实不瞒你说:自从降了黄风怪,下山来,这个把月不曾耍棍,我见你和他战的甜美,
我就忍不住脚痒,故就跳将来耍耍的。那知那怪不识耍,就走了。”

他两个搀着手,说说笑笑转回见了唐僧。唐僧道:“可曾捉得妖怪?”行者道:
“那妖怪不奈战,败回钻入水去也。”三藏道:“徒弟,这怪久住于此,他知道浅深;
似这般无边的弱水,又没了舟楫,须是得个知水性的,引领引领才好哩。”行者道:
“正是这等说。常言道:‘近朱者赤,近墨者黑。’那怪在此,断知水性。我们如今
拿住他,且不要打杀,只教他送师父过河,再做理会。”八戒道:“哥哥不必迟疑,
让你先去拿他,等老猪看守师父。”行者笑道:“贤弟呀,这桩儿我不敢说嘴。水里
勾当,老孙不大十分熟。若是空走,还要捻诀,又念念‘避水咒’,方才走得;不
然,就要变化做甚么鱼虾蟹鳖之类,我才去得。若论赌手段,凭你在高山云里,干
甚么蹊跷异样事儿,老孙都会;只是水里的买卖,有些儿榔杭。”八戒道:“老猪当
年总督天河,掌管了八万水兵大众,倒学得知些水性,却只怕那水里有甚么眷族老
小,七窝八代的都来,我就弄他不过。一时不被他捞去耶?”行者道:“你若到他
水中与他交战,却不要恋战,许败不许胜,把他引将出来,等老孙下手助你。”八
戒道:“言得是,我去耶。”

说声去,就剥了青锦直裰,脱了鞋,双手舞钯,分开水路,使出那当年的旧手
段,跃浪翻波,撞将进去,径至水底之下,往前正走。

却说那怪败了阵回,方才喘定,又听得有人推得水响,忽起身观看,原来是八
戒执了钯推水。那怪举杖当面高呼道:“那和尚,那里走!仔细看打!”八戒使钯架
住道:“你是个甚么妖精,敢在此间挡路?”那妖道:“你是也不认得我。我不是那
妖魔鬼怪,也不是少姓无名。”八戒道:“你既不是邪妖鬼怪,却怎生在此伤生?你
端的甚么姓名,实实说来,我饶你性命。”那怪道:“我
自小生来神气壮,乾坤万里曾游荡。英雄天下显威名,豪杰人家做模样。万国
九州任我行,五湖四海从吾撞。皆因学道荡天涯,只为寻师游地旷。常年衣钵谨随
身,每日心神不可放。沿地云游数十遭,到处闲行百余趟。因此才得遇真人,引开
大道金光亮。先将婴儿姹女收,后把木母金公放。明堂肾水入华池,重楼肝火投心
脏。三千功满拜天颜,志心朝礼明华
向。玉皇大帝便加升,亲口封为卷帘将。南天门里我为尊,灵霄殿前吾称上。腰间
悬挂虎头牌,手中执定降妖杖。头顶金盔晃日光,身披铠甲明霞亮。往来护驾我当
先,出入随朝予在上。只因王母降蟠桃,设宴瑶池邀众将。失手打破玉玻璃,天神
个个魂飞丧。玉皇即便怒生嗔,却令掌朝左辅相:卸冠脱甲摘官衔,将身推在杀场
上。多亏赤脚大天仙,越班启奏将吾放。饶死回生不典刑,遭贬流沙东岸上。饱时
困卧此山中,饿去翻波寻食饷。樵子逢吾命不存,渔翁见我身皆丧。来来往往吃人
多,翻翻覆覆伤生瘴。你敢行凶到我门,今日肚皮有所望。莫言粗糙不堪尝,拿住
消停剁酱!”

八戒闻言大怒,骂道:“你这泼物,全没一些儿眼色!我老猪还掐出水沫儿来哩,
你怎敢说我粗糙,要剁酱!看起来,你把我认做个老走硝哩。休得无礼,吃你祖
宗这一钯!”那怪见钯来,使一个“凤点头”躲过。两个在水中打出水面,各人踏
浪登波。这一场赌斗,比前不同。你看那:

卷帘将,天蓬帅,各显神通真可爱。那个降妖宝杖着头轮,这个九齿钉钯随手
快。跃浪振山川,推波昏世界。凶如太岁撞幢幡,恶似丧门掀宝盖。这一个赤心凛
凛保唐僧,那一个犯罪滔滔为水怪。钯抓一下九条痕,杖打之时魂魄败。努力喜相
持,用心要赌赛。算来只为取经人,怒气冲天不忍耐。搅得那鲤鳜退鲜鳞,龟
鳖鼋鼍伤嫩盖;红虾紫蟹命皆亡,水府诸神朝上拜。只听得波翻浪滚似雷轰,日月
无光天地怪。
二人整斗有两个时辰,不分胜败。这才是铜盆逢铁帚,玉磬对金钟。

却说那大圣保着唐僧,立于左右,眼巴巴的望着他两个在水上争持,只是他不
好动手。只见那八戒虚幌一钯,佯输诈败,转回头往东岸上走。那怪随后赶来,将
近到了岸边,这行者忍耐不住,撇了师父,掣铁棒,跳到河边,望妖精劈头就打。
那妖物不敢相迎,飕的又钻入河内。八戒嚷道:“你这弼马温,彻是个急猴子!你再
缓缓些儿,等我哄他到了高处,你却阻住河边,教他不能回首呵,却不拿住他也;
他这进去,几时又肯出来?”行者笑道:“呆子,莫嚷,莫嚷,我们且回去见师父
去来。”

八戒却同行者到高岸上,见了三藏。三藏欠身道:“徒弟辛苦呀。”八戒道:“且
不说辛苦,只是降了妖精,送得你过河,方是万全之策。”三藏道:“你才与妖精交
战何如?”八戒道:“那妖的手段,与老猪是个对手。正战处,使一个诈败,他才
赶到岸上。见师兄举着棍子,他就跑了。”三藏道:“如此怎生奈何?”行者道:“师
父放心,且莫焦恼。如今天色又晚,且坐在这崖次之下,待老孙去化些斋饭来,你
吃了睡去,待明日再处。”八戒道:“说得是,你快去快来。”

行者急纵云跳起去,正到直北下人家化了一钵素斋,回献师父。

师父见他来得甚快,便叫:“悟空,我们去化斋的人家,求问他一个过河之策,
不强似与这怪争持?”行者笑道:“这家子远得狠哩!相去有五七千里之路。他那里
得知水性?问他何益?”八戒道:“哥哥又来扯谎了。五七千里路,你怎么这等去来
得快?”行者道:“你那里晓得,老孙的斗云,一纵有十万八千里。像这五七千
路,只消把头点上两点,把腰躬上一躬,就是个往回,有何难哉!”八戒道:“哥啊,
既是这般容易,你把师父背着,只消点点头,躬躬腰,跳过去罢了;何必苦苦的与
他厮战?”行者道:“你不会驾云?你把师父驮过去不是?”八戒道:“师父的骨肉
凡胎,重似泰山,我这驾云的,怎称得起?须是你的斗方可。”行者道:“我的
斗,好道也是驾云,只是去的有远近些儿。你是驮不动,我却如何驮得动?自古道:
‘遣泰山轻如芥子,携凡夫难脱红尘。’像这泼魔毒怪,使摄法,弄风头,却是扯
扯拉拉,就地而行,不能带得空中而去;像那样法儿,老孙也会使会弄;还有那隐
身法、缩地法,老孙件件皆知。但只是师父要穷历异邦,不能够超脱苦海,所以寸
步难行也。我和你只做得个拥护,保得他身在命在,替不得这些苦恼,也取不得经
来;就是有能先去见了佛,那佛也不肯把经善与你我:正叫做‘若将容易得,便作
等闲看。’”那呆子闻言,喏喏听受。遂吃了些无菜的素食,师徒们歇在流沙河东,
崖次之下。

次早,三藏道:“悟空,今日怎生区处?”行者道:“没甚区处,还须八戒下水。”
八戒道:“哥哥,你要图干净,只作成我下水。”行者道:“贤弟,这番我再不急性
了,只让你引他上来,我拦住河沿,不让他回去,务要将他擒了。”

好八戒,抹抹脸,抖擞精神,双手拿钯,到河沿,分开水路,依然又下至窝巢。
那怪方才睡醒,忽听推得水响,急回头睁睛看看。见八戒执钯下至,他跳出来,当
头阻住。喝道:“慢来,慢来,看杖!”八戒举钯架住道:“你是个甚么‘哭丧杖’,
断叫你祖宗看杖!”那怪道:“你这厮甚不晓得哩!我这

宝杖原来名誉大,本是月里梭罗派。吴刚伐下一枝来,鲁班制造工夫盖。里边
一条金趁心,外边万道珠丝玠。名称宝杖善降妖,永镇灵霄能伏怪。只因官拜大将
军,玉皇赐我随身带。或长或短任吾心,要细要粗凭意态。也曾护驾宴蟠桃,也曾
随朝居上界。值殿曾经众圣参,卷帘曾见诸仙拜。养成灵性一神兵,不是人间凡器
械。自从遭贬下天门,任意纵横游海外。不当大胆自称夸,天下枪刀难比赛。看你
那个锈钉钯,只好锄田与筑菜!”
八戒笑道:“我把你少打的泼物!且莫管甚么筑菜,只怕荡了一下儿,教你没处贴膏
药,九个眼子一齐流血!纵然不死,也是个到老的破伤风!”那怪丢开架手,在那水
底下,与八戒依然打出水面。这一番斗,比前果更不同。你看他:

宝杖轮,钉钯筑,言语不通非眷属。只因木母克刀圭,致令两下相战触。没输
赢,无反复,翻波淘浪不和睦。这个怒气怎含容?那个伤心难忍辱。钯来杖架逞英
雄,水滚流沙能恶毒。气昂昂,劳碌碌,多因三藏朝西域。钉钯老大凶,宝杖十分
熟。这个揪住要往岸上拖,那个抓来就将水里沃。声如霹雳动鱼龙,云暗天昏神鬼
伏。
这一场,来来往往,斗经三十回合,不见强弱。八戒又使个佯输计,拖了钯走。那
怪随后又赶来,拥波捉浪,赶至崖边。八戒骂道:“我把你这个泼怪!你上来!这高
处,脚踏实地好打!”那妖骂道:“你这厮哄我上去,又教那帮手来哩。你下来,还
在水里相斗。”原来那妖乖了,再不肯上岸,只在河沿与八戒闹吵。

却说行者见他不肯上岸,急得他心焦性爆,恨不得一把捉来;行者道:“师父,
你自坐下,等我与他个‘饿鹰雕食’。”就纵筋斗,跳在半空,刷的落下来,要抓那
妖。那妖正与八戒嚷闹,忽听得风响,急回头,见是行者落下云来,却又收了那杖,
一头淬下水,隐迹潜踪,渺然不见。行者伫立岸上,对八戒说:“兄弟呀,这妖也
弄得滑了。他再不肯上岸,如之奈何?”八戒道:“难,难,难!战不胜他!——就
把吃奶的气力也使尽了,只绷得个手平。”行者道:“且见师父去。”

二人又到高岸,见了唐僧,备言难捉。那长老满眼下泪道:“似此艰难,怎生
得渡!”行者道:“师父莫要烦恼。这怪深潜水底,其实难行。八戒,你只在此保守
师父,再莫与他厮斗,等老孙往南海走走去来。”八戒道:“哥呵,你去南海何干?”
行者道:“这取经的勾当,原是观音菩萨;及脱解我等,也是观音菩萨;今日路阻
流沙河,不能前进,不得他,怎生处治?等我去请他,还强如和这妖精相斗。”八戒
道:“也是,也是。师兄,你去时,千万与我上复一声:向日多承指教。”三藏道:
“悟空,若是去请菩萨,却也不必迟疑,快去快来。”

行者即纵筋斗云,径上南海。咦!那消半个时辰,早望见普陀山境。须臾间,
坠下筋斗,到紫竹林外,又只见那二十四路诸天,上前迎着道:“大圣何来?”行
者道:“我师有难,特来谒见菩萨。”诸天道:“请坐,容报。”那轮日的诸天,径至
潮音洞口报道:“孙悟空有事朝见。”菩萨正与捧珠龙女在宝莲池畔扶栏看花,闻报,
即转云岩,开门唤入。大圣端肃皈依参拜。

菩萨问曰:“你怎么不保唐僧?为甚事又来见我?”行者启上道:“菩萨,我师
父前在高老庄,又收了一个徒弟,唤名猪八戒,多蒙菩萨又赐法讳悟能。才行过黄
风岭,今至八百里流沙河,乃是弱水三千,师父已是难渡;河中又有个妖怪,武艺
高强,甚亏了悟能与他水面上大战三次,只是不能取胜,被他拦阻,不能渡河。因
此,特告菩萨,望垂怜悯,济渡他一济渡。”菩萨道:“你这猴子,又逞自满,不肯
说出保唐僧的话来么?”行者道:“我们只是要拿住他,教他送我师父渡河。水里
事,我又弄不得精细,只是悟能寻着他窝巢,与他打话。想是不曾说出取经的勾当。”
菩萨道:“那流沙河的妖怪,乃是卷帘大将临凡,也是我劝化的善信,教他保护取
经之辈。你若肯说出是东土取经人呵,他决不与你争持,断然归顺矣。”行者道:“那
怪如今怯战,不肯上崖,只在水里潜踪,如何得他归顺?我师如何得渡弱水?”

菩萨即唤惠岸,袖中取出一个红葫芦儿,吩咐道:“你可将此葫芦,同孙悟空
到流沙河水面上,只叫‘悟净’,他就出来了。先要引他归依了唐僧;然后把他那
九个骷髅穿在一处,按九宫布列,却把这葫芦安在当中,就是法船一只,能渡唐僧
过流沙河界。”惠岸闻言,谨遵师命,当时与大圣捧葫芦出了潮音洞,奉法旨辞了
紫竹林。有诗为证,诗曰:
五行匹配合天真,认得从前旧主人。
炼已立基为妙用,辨明邪正见原因。
金来归性还同类,木去求情共复沦。
二土全功成寂寞,调和水火没纤尘。

他两个,不多时,按落云头,早来到流沙河岸。猪八戒认得是木叉行者,引师
父上前迎接。那木叉与三藏礼毕,又与八戒相见。八戒道:“向蒙尊者指示,得见
菩萨,我老猪果遵法教,今喜拜了沙门。这一向在途中奔碌,未及致谢,恕罪,恕
罪。”行者道:“且莫叙阔。我们叫唤那厮去来。”三藏道:“叫谁?”行者道:“老
孙见菩萨,备陈前事。菩萨说:这流沙河的妖怪,乃是卷帘大将临凡;因为在天有
罪,堕落此河,忘形作怪。他曾被菩萨劝化,愿归师父往西天去的。但是我们不曾
说出取经的事情,故此苦苦争斗。菩萨今差木叉,将此葫芦,要与这厮结作法船,
渡你过去哩。”三藏闻言,顶礼不尽。对木叉作礼道:“万望尊者作速一行。”

那木叉捧定葫芦,半云半雾,径到了流沙河水面上,厉声高叫道:“悟净!悟净!
取经人在此久矣,你怎么还不归顺!”

却说那怪惧怕猴王,回于水底,正在窝中歇息。只听得叫他法名,情知是观音
菩萨;又闻得说“取经人在此”,他也不惧斧钺,急翻波伸出头来,又认得是木叉
行者。你看他笑盈盈,上前作礼道:“尊者失迎。菩萨今在何处?”木叉道:“我师
未来,先差我来吩咐你早跟唐僧做个徒弟。叫把你项下挂的骷髅与这个葫芦,按九
宫结做一只法船,渡他过此弱水。”悟净道:“取经人却在那里?”木叉用手指道:
“那东岸上坐的不是?”悟净看见了八戒道:“他不知是那里来的个泼物,与我整
斗了这两日,何曾言着一个取经的字儿?”又看见行者,道:“这个主子,是他的
帮手,好不利害!我不去了。”木叉道:“那是猪八戒,这是孙行者。俱是唐僧的徒
弟,俱是菩萨劝化的,怕他怎的?我且和你见唐僧去。”那悟净才收了宝杖,整一整
黄锦直裰,跳上岸来,对唐僧双膝跪下道:“师父,弟子有眼无珠,不认得师父的
尊容,多有冲撞,万望恕罪。”八戒道:“你这脓包,怎的早不皈依,只管要与我打?
是何说话!”行者笑道:“兄弟,你莫怪他,还是我们不曾说出取经的事样与姓名耳。”
长老道:“你果肯诚心皈依吾教么?”悟净道:“弟子向蒙菩萨教化,指河为姓,与
我起个法名,唤做沙悟净,岂有不从师父之理!”三藏道:“既如此,”叫:“悟空,
取戒刀来,与他落了发。”大圣依言,即将戒刀与他剃了头。又来拜了三藏,拜了
行者与八戒,分了大小。三藏见他行礼,真像个和尚家风,故又叫他做沙和尚。木
叉道:“既秉了迦持,不必叙烦,早与作法船去来。”

那悟净不敢怠慢,即将颈项下挂的骷髅取下,用索子结作九宫,把菩萨葫芦安
在当中,请师父下岸。那长老遂登法船,坐于上面,果然稳似轻舟。左有八戒扶持,
右有悟净捧托;孙行者在后面牵了龙马,半云半雾相跟;头直上又有木叉拥护;那
师父才飘然稳渡流沙河界,浪静风平过弱河。真个也如飞似箭,不多时,身登彼岸,
得脱洪波;又不拖泥带水,幸喜脚干手燥,清净无为,师徒们脚踏实地。那木叉按
祥云,收了葫芦。又只见那骷髅一时解化作九股阴风,寂然不见。三藏拜谢了木叉,
顶礼了菩萨。正是:
木叉径回东洋海,三藏上马却投西。

毕竟不知几时才得正果求经,且听下回分解。
