\chapter{外道施威欺正性~心猿获宝伏邪魔}

本性圆明道自通,翻身跳出网罗中。
修成变化非容易,炼就长生岂俗同?
清浊几番随运转,辟开数劫任西东。
逍遥万亿年无计,一点神光永注空。

此诗暗合孙大圣的道妙。他自得了那魔真宝,笼在袖中。喜道:“泼魔苦苦用
心拿我,诚所谓水中捞月;老孙若要擒你,就好似火上弄冰。”藏着葫芦,密密的
溜出门外,现了本相,厉声高叫道:
“精怪开门!”旁有小妖道:“你又是甚人,敢来吆喝?”行者道:“快报与你那老
泼魔,吾乃行者孙来也。”

那小妖急入里报道:“大王,门外有个甚么行者孙来了。”老魔大惊道:“贤弟,
不好了!惹动他一窝风了!幌金绳现拴着孙行者,葫芦里现装着者行孙,怎么又有个
甚么行者孙?想是他几个兄弟都来了。”二魔道:“兄长放心。我这葫芦装下一千人
哩。我才装了者行孙一个,又怕那甚么行者孙!等我出去看看,一发装来。”老魔道:
“兄弟仔细。”

你看那二魔拿着个假葫芦,还像前番,雄纠纠,气昂昂,走出门高呼道:“你
是那里人氏,敢在此间吆喝?”行者道:“你认不得我?

家居花果山,祖贯水帘洞。只为闹天宫,多时罢争竞。如今幸脱灾,弃道从僧
用。秉教上雷音,求经归觉正。相逢野泼
魔,却把神通弄。还我大唐僧,上西参佛圣。两家罢战争,各守平安境。休惹老孙
焦,伤残老性命!”
那魔道:“你且过来,我不与你相打,但我叫你一声,你敢应么?”行者笑道:“你
叫我,我就应了;我若叫你,你可应么?”那魔道:“我叫你,是我有个宝贝葫芦,
可以装人;你叫我,却有何物?”行者道:“我也有个葫芦儿。”那魔道:“既有,
拿出来我看。”行者就于袖中取出葫芦道:“泼魔,你看!”幌一幌,复藏在袖中,
恐他来抢。

那魔见了大惊道:“他葫芦是那里来的?怎么就与我的一般?纵是一根藤上结
的,也有个大小不同,偏正不一,却怎么一般无二?”他便正色叫道:“行者孙,
你那葫芦是那里来的?”行者委的不知来历,接过口来,就问他一句道:“你那葫
芦是那里来的?”那魔不知是个见识,只道是句老实言语,就将根本从头说出道:
“我这葫芦是混沌初分,天开地辟,有一位太上老祖,解化女娲之名,炼石补天,
普救阎浮世界;补到乾宫央地,见一座昆仑山脚下,有一缕仙藤,上结着这个紫金
红葫芦,却便是老君留下到如今者。”

大圣闻言,就绰了他口气道:“我的葫芦,也是那里来的。”魔头道:“怎见得?”
大圣道:“自清浊初开,天不满西北,地不满东南,太上道祖解化女娲,补完天缺,
行至昆仑山下,有根仙藤,藤结有两个葫芦。我得一个是雄的,你那个却是雌的。”
那怪道:“莫说雌雄;但只装得人的,就是好宝贝。”大圣道:“你也说得是,我就
让你先装。”

那怪甚喜,急纵身跳将起去,到空中,执着葫芦,叫一声:“行者孙。”大圣听
得,却就不歇气连应了八九声,只是不能装去。那魔坠将下来,跌脚捶胸道:“天
那!只说世情不改变哩!这样个宝贝,也怕老公,雌见了雄,就不敢装了!”

行者笑道:“你且收起,轮到老孙该叫你哩。”急纵筋斗,跳起去,将葫芦底儿
朝天,口儿朝地,照定妖魔,叫声“银角大王”。那怪不敢闭口,只得应了一声,
倏的装在里面,被行者贴上“太上老君急急如律令奉敕”的帖子。心中暗喜道:“我
的儿,你今日也来试试新了!”

他就按落云头,拿着葫芦,心心念念,只是要救师父,又往莲花洞口而来。那
山上都是些洼踏不平之路,况他又是个圈盘腿,拐呀拐的走着,摇的那葫芦里
索索,响声不绝。你道他怎么便有响声?原来孙大圣是熬炼过的身体,急切化他不
得;那怪虽也能腾云驾雾,不过是些法术,大端是凡胎未脱,到于宝贝里就化了。
行者还不当他就化了,笑道:“我儿子啊,不知是撒尿耶,不知是漱口哩。这是老
孙干过的买卖。不等到七八日,化成稀汁,我也不揭盖来看。忙怎的?有甚要紧?想
着我出来的容易,就该千年不看才好!”他拿着葫芦,说着话,不觉的到了洞口,
把那葫芦摇摇,一发响了。他道:“这个像发课的筒子响,倒好发课。等老孙发一
课,看师父甚么时才得出门。”你看他手里不住的摇,口里不住的念道:“周易文王、
孔子圣人、桃花女先生、鬼谷子先生。”

那洞里小妖看见道:“大王,祸事了!行者孙把二大王爷爷装在葫芦里发课哩!”
那老魔闻得此言,唬得魂飞魄散,骨软筋麻,扑的跌倒在地,放声大哭道:“贤弟
呀!我和你私离上界,转托尘凡,指望同享荣华,永为山洞之主;怎知为这和尚,
伤了你的性命,断吾手足之情!”满洞群妖,一齐痛哭。

猪八戒吊在梁上,听得他一家子齐哭,忍不住叫道:“妖精,你且莫哭,等老
猪讲与你听。先来的孙行者,次来的者行孙,后来的行者孙,返复三字,都是我师
兄一人。他有七十二变化,腾那进来,盗了宝贝,装了令弟。令弟已是死了,不必
这等扛丧,快些儿刷净锅灶,办些香蕈、磨菇、茶芽、竹笋、豆腐、面筋、木耳、
蔬菜,请我师徒们下来,与你令弟念卷《受生经》。”那老魔闻言,心中大怒道:“只
说猪八戒老实,原来甚不老实!他倒作笑话儿打觑我!”叫:小妖,“且休举哀,把
猪八戒解下来,蒸得稀烂,等我吃饱了,再去拿孙行者报仇”。沙僧埋怨八戒道:“好
么!我说教你莫多话,多话的要先蒸吃哩!”那呆子也尽有几分悚惧。旁一小妖道:
“大王,猪八戒不好蒸。”八戒道:“阿弥陀佛!是那位哥哥积阴德的?果是不好蒸。”
又有一个妖道:“将他皮剥了,就好蒸。”八戒慌了道:“好蒸,好蒸!皮骨虽然粗糙,
汤滚就烂。户!户!”

正嚷处,只见前门外一个小妖报道:“行者孙又骂上门来了!”那老魔又大惊道:
“这厮轻我无人!”叫:“小的们,且把猪八戒照旧吊起,查一查还有几件宝贝。”
管家的小妖道:“洞中还有三件宝贝哩。”老魔问:“是那三件?”管家的道:“还有
‘七星剑’、‘芭蕉扇’与‘净瓶’。”老魔道:“那瓶子不中用:原是叫人,人应了
就装得,转把个口诀儿教了那孙行者,倒把自家兄弟装去了。不用他,放在家里。
快将剑与扇子拿来。”那管家的即将两件宝贝献与老魔。老魔将芭蕉扇插在后项衣
领,把七星剑提在手中,又点起大小群妖,有三百多名,都教一个个拈枪弄棒,理
索轮刀。这老魔却顶盔贯甲,罩一领赤焰焰的丝袍。群妖摆出阵去,要拿孙大圣。
那孙大圣早已知二魔化在葫芦里面,却将他紧紧拴扣停当,撒在腰间,手持着金箍
棒,准备厮杀。只见那老妖红旗招展,跳出门来。却怎生打扮?

头上盔缨光焰焰,腰间带束彩霞鲜。身穿铠甲龙鳞砌,上罩红袍烈火然。圆眼
睁开光掣电,钢须飘起乱飞烟。七星宝剑轻提手,芭蕉扇子半遮肩。行似流云离海
岳,声如霹雳震山川。威风凛凛欺天将,怒帅群妖出洞前。

那老魔急令小妖摆开阵势。骂道:“你这猴子,十分无礼!害我兄弟,伤我手足,
着然可恨!”行者骂道:“你这讨死的怪物!你一个妖精的性命舍不得,似我师父、
师弟、连马四个生灵,平白的吊在洞里,我心何忍,情理何甘!快快的送将出来还
我,多多贴些盘费,喜喜欢欢打发老孙起身,还饶了你这个老妖的狗命!”那怪那
容分说,举宝剑劈头就砍。这大圣使铁棒举手相迎。这一场在洞门外好杀!咦!

金箍棒与七星剑,对撞霞光如闪电。悠悠冷气逼人寒,荡荡昏云遮岭堰。那个
皆因手足情,些儿不放善;这个只为取经僧,毫厘不容缓。两家各恨一般仇,二处
每怀生怒怨。只杀得天昏地暗鬼神惊,日淡烟浓龙虎战。这个咬牙锉玉钉,那个怒
目飞金焰。一来一往逞英雄,不住翻腾棒与剑。

这老魔与大圣战经二十回合,不分胜负。他把那剑梢一指,叫声“小妖齐来!”
那三百余精,一齐拥上,把行者围在垓心。好大圣,公然无惧,使一条棒,左冲右
撞,后抵前遮。那小妖都有手段,越打越上,一似绵絮缠身,搂腰扯腿,莫肯退后。
大圣慌了,即使个身外身法,将左胁下毫毛,拔了一把,嚼碎喷去,喝声叫“变!”
一根根都变做行者。你看他长的使棒,短的轮拳,再小的没处下手,抱着孤拐啃筋,
把那小妖都打得星落云散,齐声喊道:“大王啊,事不谐矣!难矣乎哉!满地盈山,
皆是孙行者了!”被这身外法把群妖打退,止撇得老魔围困中间,赶得东奔西走,
出路无门。

那魔慌了,将左手擎着宝剑,右手伸于项后,取出芭蕉扇子,望东南丙丁火,
正对离宫,唿喇的一扇子,将下来,只见那就地上,火光焰焰。原来这般宝贝,
平白地出火来。那怪物着实无情:一连了七八扇子,天炽地,烈火飞腾。好
火:

那火不是天上火,不是炉中火,也不是山头火,也不是灶底火,乃是五行中自
然取出的一点灵光火。这扇也不是凡间常有之物,也不是人工造就之物,乃是自开
辟混沌以来产成的珍宝之物。用此扇,此火,煌煌烨烨,就如电掣红绡;灼灼辉
辉,却似霞飞绛绮。更无一缕青烟,尽是满山赤焰,只烧得岭上松翻成火树,崖前
柏变作灯笼。那窝中走兽贪性命,西撞东
奔;这林内飞禽惜羽毛,高飞远举。这场神火飘空燎,只烧得石烂溪干遍地红!
大圣见此恶火,却也心惊胆颤;道声“不好了!我本身可处,毫毛不济,一落这火
中,岂不真如燎毛之易?”将身一抖,遂将毫毛收上身来。只将一根变作假身子,
避火逃灾,他的真身,捻着避火诀,纵筋斗,跳将起去,脱离了大火之中,径奔他
莲花洞里,想着要救师父。

急到门前,把云头按落。又见那洞门外有百十个小妖,都破头折脚,肉绽皮开。
原来都是他分身法打伤了的,都在这里声声唤唤,忍疼而立。大圣见了,按不住恶
性凶顽,轮起铁棒,一路打将进去。可怜把那苦炼人身的功果息,依然是块旧皮毛!

那大圣打绝了小妖,撞入洞里,要解师父,又见那内面有火光焰焰,唬得他手
慌脚忙道:“罢了!罢了!这火从后门口烧起来,老孙却难救师父也!”正悚惧处,仔
细看时,呀!原来不是火光,却是一道金光。他正了性,往里视之,乃羊脂玉净瓶
放光,却自心中欢喜道:“好宝贝耶!这瓶子曾是那小妖拿在山上放光,老孙得了,
不想那怪又复搜去;今日藏在这里,原来也放光。”你看他窃了这瓶子,喜喜欢欢,
且不救师父,急抽身往洞外而走。才出门,只见那妖魔提着宝剑,拿着扇子,从南
而来。孙大圣回避不及,被那老魔举剑劈头就砍。大圣急纵筋斗云,跳将起去,无
影无踪的逃了不题。

却说那怪到得门口,但见尸横满地,——就是他手下的群精——慌得仰天长叹,
止不住放声大哭道:“苦哉!痛哉!”有诗为证,诗曰:
可恨猿乖马劣顽,灵胎转托降尘凡。
只因错念离天阙,致使忘形落此山。
鸿雁失群情切切,妖兵绝族泪潺潺。
何时孽满开愆锁,返本还原上御关?
那老魔惭惶不已,一步一声,哭入洞内。只见那什物家火俱在,只落得静悄悄,没
个人形;悲切切,愈加凄惨。独自个坐在洞中,蹋伏在那石案之上,将宝剑斜倚案
边,把扇子插于肩后,昏昏默默睡着了。这正是“人逢喜事精神爽,闷上心来瞌睡
多。”

话说孙大圣拨转筋斗云,伫立山前,想着要救师父,把那净瓶儿牢扣腰间,径
来洞口打探。见那门开两扇,静悄悄的不闻消耗,随即轻轻移步,潜入里边。只见
那魔斜倚石案,呼呼睡着,芭蕉扇褪出肩衣,半盖着脑后,七星剑还斜倚案边;却
被他轻轻的走上前拔了扇子,急回头,呼的一声,跑将出去。原来这扇柄儿刮着那
怪的头发,早惊醒他。抬头看时,是孙行者偷了,急慌忙执剑来赶。那大圣早已跳
出门前,将扇子撒在腰间,双手轮开铁棒,与那魔抵敌。这一场好杀:

恼坏泼妖王,怒发冲冠志。恨不过挝来囫囵吞,难解心头气。恶口骂猢狲:“你
老大将人戏,伤我若干生,还来偷宝贝。这场决不容,定见存亡计!”大圣喝妖魔:
“你好不知趣!徒弟要与老孙争,累卵焉能击石碎?”宝剑来,铁棒去,两家更不
留仁义。一翻二复赌输赢,三转四回施武艺。盖为取经僧,灵山参佛位,致令金火
不相投,五行拨乱伤和气;扬威耀武显神通,走石飞沙弄本事。交锋渐渐日将晡,
魔头力怯先回避。
那老魔与大圣战经三四十合,天将晚矣,抵敌不住,败下阵来;径往西南上,投奔
压龙洞去不题。

这大圣才按落云头,闯入莲花洞里,解下唐僧与八戒、沙和尚来。他三人脱得
灾危,谢了行者,却问:“妖魔那里去了?”行者道:“二魔已装在葫芦里,想是这
会子已化了;大魔才然一阵战败,往西南压龙山去讫。概洞小妖,被老孙分身法打
死一半,还有些败残回的,又被老孙杀绝,方才得入此处,解放你们。”唐僧谢之
不尽道:“徒弟啊,多亏你受了劳苦!”行者笑道:“诚然劳苦。你们还只是吊着受
疼,我老孙再不曾住脚,比急递铺的铺兵还甚,反复里外,奔波无已。因是偷了他
的宝贝,方能平退妖魔。”猪八戒道:“师兄,你把那葫芦儿拿出来与我们看看。只
怕那二魔已化了也。”大圣先将净瓶解下,又将金绳与扇子取出,然后把葫芦儿拿
在手道:“莫看!莫看!他先曾装了老孙,被老孙漱口,哄得他扬开盖子,老孙方得
走了。我等切莫揭盖,只怕他也会弄喧走了。”师徒们喜喜欢欢,将他那洞中的米
面菜蔬寻出,烧刷了锅灶,安排些素斋吃了。饱餐一顿,安寝洞中,一夜无词。早
又天晓。

却说那老魔径投压龙山,会聚了大小女怪,备言打杀母亲,装了兄弟,绝灭妖
兵,偷骗宝贝之事。众女怪一齐大哭。哀痛多时道:“你等且休凄惨。我身边还有
这口七星剑,欲会汝等女兵,都去压龙山后,会借外家亲戚,断要拿住那孙行者报
仇。”

说不了,有门外小妖报道:“大王,山后老舅爷帅领若干兵卒来也。”老魔闻言,
急换了缟素孝服,躬身迎接。原来那老舅爷是他母亲之弟,名换狐阿七大王。因闻
得哨山的妖兵报道,他姐姐被孙行者打死,假变姐形,盗了外甥宝贝,连日在平顶
山拒敌。他却帅本洞妖兵二百余名,特来助阵;故此先拢姐家问信。才进门,见老
魔挂了孝服,二人大哭。哭久,老魔拜下,备言前事。那阿七大怒,即命老魔换了
孝服,提了宝剑,尽点女妖,合同一处,纵风云,径投东北而来。

这大圣却教沙僧整顿早斋,吃了走路。忽听得风声,走出门看,乃是一伙妖兵,
自西南上来。行者大惊,急抽身,忙呼八戒道:“兄弟,妖精又请救兵来也。”三藏
闻言,惊恐失色道:“徒弟,似此如何?”行者笑道:“放心!放心!把他这宝贝都拿
来与我。”大圣将葫芦、净瓶系在腰间,金绳笼于袖内,芭蕉扇插在肩后,双手轮
着铁棒,教沙僧保守师父,稳坐洞中;着八戒执钉钯,同出洞外迎敌。

那怪物摆开阵势,只见当头的是阿七大王。他生的玉面长髯,钢眉刀耳;头戴
金炼盔,身穿锁子甲,手执方天戟,高声骂道:“我把你个大胆的泼猴!怎敢这等欺
人!偷了宝贝,伤了眷族,杀了妖兵,又敢久占洞府!赶早儿一个个引颈受死,雪我
姐家之仇!”行者骂道:“你这伙作死的毛团,不识你孙外公的手段!不要走!领吾一
棒!”那怪物侧身躲过,使方天戟劈面相迎。两个在山头一来一往,战经三四回合,
那怪力软,败阵回走。行者赶来,却被老魔接住。又斗了三合,只见那狐阿七复转
来攻。这壁厢八戒见了,急掣九齿钯挡住。一个抵一个,战经多时,不分胜败。那
老魔喝了一声,众妖兵一齐围上。

却说那三藏坐在莲花洞里,听得喊声振地,便叫:“沙和尚,你出去看你师兄
胜负何如。”沙僧果举降妖杖出来,喝一声,撞将出去,打退群妖。阿七见事势不
利,回头就走;被八戒赶上,照背后一钯,就筑得九点鲜红往外冒,可怜一灵真性
赴前程。急拖来剥了衣服看处,原来也是个狐狸精。

那老魔见伤了他老舅,丢了行者,提宝剑,就劈八戒。八戒使钯架住。正赌斗
间,沙僧撞近前来,举杖便打。那妖抵敌不住,纵风云往南逃走。八戒、沙僧紧紧
赶来。大圣见了,急纵云跳在空中,解下净瓶,罩定老魔,叫声“金角大王”。那
怪只道是自家败残的小妖呼叫,就回头应了一声;飕的装将进去,被行者贴上“太
上老君急急如律令奉敕”的帖子。只见那七星剑坠落尘埃,也归了行者。八戒迎着
道:“哥哥,宝剑你得了,精怪何在?”行者笑道:“了了!已装在我这瓶儿里也。”
沙僧听说,与八戒十分欢喜。

当时通扫净诸邪,回至洞里,与三藏报喜道:“山已净,妖已无矣,请师父上
马走路。”三藏喜不自胜。师徒们吃了早斋,收拾了行李、马匹、奔西找路。

正行处,猛见路旁闪出一个瞽者,走上前扯住三藏马,道:“和尚,那里去?还
我宝贝来!”八戒大惊道:“罢了!这是老妖来讨宝贝了!”行者仔细观看,原来是太
上李老君,慌得近前施礼道:“老官儿,那里去?”那老祖急升玉局宝座,九霄空
里伫立,叫:“孙行者,还我宝贝。”大圣起到空中道:“甚么宝贝?”老君道:“葫
芦是我盛丹的,净瓶是我盛水的,宝剑是我炼魔的,扇子是我火的,绳子是我一
根勒袍的带。那两个怪:一个是我看金炉的童子,一个是我看银炉的童子。只因他
偷了我的宝贝,走下界来,正无觅处,却是你今拿住,得了功绩。”大圣道:“你这
老官儿,着实无礼。纵放家属为邪,该问个钤束不严的罪名。”老君道:“不干我事,
不可错怪了人。此乃海上菩萨问我借了三次,送他在此托化妖魔,看你师徒可有真
心往西去也。”大圣闻言,心中作念道:“这菩萨也老大惫懒!当时解脱老孙,教保
唐僧西去取经,我说路途艰涩难行,他曾许我到急难处亲来相救;如今反使精邪
害,语言不的,该他一世无夫!若不是老官儿亲来,我决不与他;既是你这等说,
拿去罢。”

那老君收得五件宝贝,揭开葫芦与净瓶盖口,倒出两股仙气,用手一指,仍化
为金银二童子,相随左右。只见那霞光万道。咦!
缥缈同归兜率院,逍遥直上大罗天。

毕竟不知此后又有甚事,孙大圣怎生保护唐僧,几时得到西天,且听下回分解。