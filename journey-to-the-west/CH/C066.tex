\chapter{诸神遭毒手~弥勒缚妖魔}

话表孙大圣无计可施,纵一朵祥云,驾筋斗,径转南赡部洲去拜武当山,参请
荡魔天尊,解释三藏、八戒、沙僧、天兵等众之灾。他在半空里无停止。不一日,
早望见祖师仙境,轻轻按落云头,定睛观看,好去处:

巨镇东南,中天神岳。芙蓉峰竦杰,紫盖岭巍峨。九江水尽荆扬远,百越山连
翼轸多。上有太虚之宝洞,朱陆之灵台。三十六宫金磬响,百千万客进香来。舜巡
禹祷,玉简金书。楼阁飞青鸟,幢幡摆赤裾。地设名山雄宇宙,天开仙境透空虚。
几树榔梅花正放,满山瑶草色皆舒。龙潜涧底,虎伏崖中。幽含如诉语,驯鹿近人
行。白鹤伴云栖老桧,青鸾丹凤向阳鸣。玉虚师相真仙地,金阙仁慈治世门。

上帝祖师,乃净乐国王与善胜皇后梦吞日光,觉而有孕,怀胎一十四个月,于
开皇元年甲辰之岁三月初一日午时降诞于王宫。那爷爷:

幼而勇猛,长而神灵。不统王位,惟务修行。父母难禁,弃舍皇宫。参玄入定,
在此山中。功完行满,白日飞升。玉皇敕号,真武之名。玄虚上应,龟蛇合形。周
天六合,皆称万灵。无幽不禁,无显不成。劫终劫始,剪伐魔精。

孙大圣玩着仙境景致,早来到一天门、二天门、三天门。却至太和宫外,忽见
那祥光瑞气之间,簇拥着五百灵官。那灵官上前迎着道:“那来的是谁?”大圣道:
“我乃齐天大圣孙悟空,要见师相。”众灵官听说,随报。祖师即下殿,迎到太和
宫。

行者作礼道:“我有一事奉劳。”问:“何事?”行者道:“保唐僧西天取经,路
遭险难。至西牛贺洲,有座山唤小西天,小雷音寺有一妖魔。我师父进得山门,见
有阿罗、揭谛、比丘、圣僧排列,以为真佛,倒身才拜,忽被他拿住绑了。我又失
于防闲,被他抛一副金铙,将我罩在里面,无纤毫之缝,口合如钳。甚亏金头揭谛
请奏玉帝,钦差二十八宿,当夜下界,掀揭不起。幸得亢金龙将角透入铙内,将我
度出,被我打碎金铙,惊醒怪物。赶战之间,又被撒一个白布搭包儿,将我与二十
八宿并五方揭谛,尽皆装去,复用绳捆了。是我当夜脱逃,救了星辰等众,与我唐
僧等。后为找寻衣钵,又惊醒那妖,与天兵赶战。那怪又拿出搭包儿,理弄之时,
我却知道前音,遂走了。众等被他依然装去。我无计可施,特来拜求师相一助力也。”

祖师道:“我当年威镇北方,统摄真武之位,剪伐天下妖邪,乃奉玉帝敕旨。
后又披发跣足,踏腾蛇神龟,领五雷神将、巨虬狮子、猛兽毒龙,收降东北方黑气
妖氛,乃奉元始天尊符召。今日静享武当山,安逸太和殿,一向海岳平宁,乾坤清
泰。奈何我南赡部洲并北俱芦洲之地,妖魔剪伐,邪鬼潜踪。今蒙大圣下降,不得
不行;只是上界无有旨意,不敢擅动干戈。假若法遣众神,又恐玉帝见罪;十分却
了大圣,又是我逆了人情。我谅着那西路上纵有妖邪,也不为大害。我今着龟、蛇
二将并五大神龙与你助力,管教擒妖精,救你师之难。”行者拜谢了祖师,即同龟、
蛇、龙神各带精锐之兵,复转西洲之界。不一日,到了小雷音寺,按下云头,径至
山门外叫战。

却说那黄眉大王聚众怪在宝阁下说:“孙行者这两日不来,又不知往何方去借
兵也。”说不了,只见前门上小妖报道:“行者引几个龙蛇龟相,在门外叫战!”妖
魔道:“这猴儿怎么得个龙蛇龟相?此等之类,却是何方来者?”随即披挂,走出山
门高叫:“汝等是那路龙神,敢来造吾仙境?”五龙、二将相貌峥嵘,精神抖擞,
喝道:“那泼怪!我乃武当山太和宫混元教主荡魔天尊之前五位龙神、龟、蛇二将。
今蒙齐天大圣相邀,我天尊符召,到此捕你这妖精,快送唐僧与天星等出来,免你
一死!不然,将这一山之怪,碎劈其尸;几间之房,烧为灰烬!”那怪闻言,心中大
怒道:“这畜生,有何法力,敢出大言!不要走!吃吾一棒!”这五条龙,翻云使雨;
那两员将,播土扬沙,各执枪刀剑戟,一拥而攻。孙大圣又使铁棒随后。这一场好
杀:

凶魔施武,行者求兵:凶魔施武,擅据珍楼施佛像;行者求兵,远参宝境借龙
神。龟蛇生水火,妖怪动刀兵。五龙奉旨来西路,行者因师在后收。剑戟光明摇彩
电,枪刀晃亮闪霓虹。这个狼牙棒,强能短软;那个金箍棒,随意如心。只听得
扑响声如爆竹,叮当音韵似敲金。水火齐来征怪物,刀兵共簇绕精灵。喊杀惊狼虎,
喧哗振鬼神。浑战正当无胜处,妖魔又取宝和珍。
行者帅五龙、二将,与妖魔战经半个时辰,那妖精即解下搭包在手。行者见了心惊,
叫道:“列位仔细!”那龙神、蛇、龟不知甚么仔细,一个个都停住兵,近前抵挡。
那妖精幌的一声,把搭包儿撇将起去;孙大圣顾不得五龙、二将,驾筋斗,跳在九
霄逃脱。他把个龙神、龟、蛇一搭包子又装将去了。妖精得胜回寺,也将绳捆了,
抬在地窑子里盖住不题。

你看那大圣落下云头,斜在山巅之上,没精没采,懊恨道:“这怪物十分利
害!”不觉的合着眼,似睡一般。猛听得有人叫道:“大圣,休推睡,快早上紧求救。
你师父性命,只在须臾间矣!”行者急睁睛跳起来看,原来是日值功曹。行者喝道:
“你这毛神,这向在那方贪图血食,不来点卯,今日却来惊我!伸过孤拐来,让老
孙打两棒解闷!”功曹慌忙施礼道:“大圣,你是人间之喜仙,何闷之有!我等早奉
菩萨旨令,教我等暗中护佑唐僧,乃同土地等神,不敢暂离左右,是以不得常来参
见。怎么反见责也?”行者道:“你既是保护,如今那众星、揭谛、伽蓝并我师等,
被妖精困在何方?受甚罪苦?”功曹道:“你师父、师弟,都吊在宝殿廊下;星辰等
众,都收在地窑之间受罪。这两日不闻大圣消息,却才见妖精又拿了神龙、龟、蛇,
又送在地窑里去了,方知是大圣请来之兵,小神特来寻大圣。大圣莫辞劳倦,千万
再急急去求救援。”

行者闻言及此,不觉对功曹滴泪道:“我如今愧上天宫,羞临海藏!怕问菩萨之
原由,愁见如来之玉像!才拿去者,乃真武师相之龟、蛇、五龙圣众。教我再无方
求救,奈何?”功曹笑道:“大圣宽怀。小神想起一处精兵,请来断然可降。适才
大圣至武当,是南赡部洲之地。这枝兵也在南赡部洲盱眙山城,即今泗州是也。
那里有个大圣国师王菩萨,神通广大。他手下有一个徒弟,唤名小张太子,还有四
大神将,昔年曾降伏水母娘娘。你今若去请他。他来施恩相助,准可捉怪救师也。”
行者心喜道:“你且去保护我师父,勿令伤他,待老孙去请也。”

行者纵起筋斗云,躲离怪处,直奔盱眙山。不一日,早到。细观,真好去处:

南近江津,北临淮水;东通海峤,西接封浮。山顶上有楼观峥嵘,山凹里有涧
泉浩涌。嵯峨怪石,秀乔松。百般果品应时新,千样花枝迎日放。人如蚁阵往来
多,船似雁行归去广。上边有瑞岩观、东岳宫、五显祠、龟山寺,钟韵香烟冲碧汉;
又有玻璃泉、五塔峪、八仙台、杏花园,山光树色映城。白云横不度,幽鸟倦还
鸣。说甚泰嵩衡华秀,此间仙景若蓬瀛。
大圣点玩不尽,径过了淮河,入城之内,到大圣禅寺山门外。又见那殿宇轩昂,
长廊彩丽,有一座宝塔峥嵘。真是:
插云倚汉高千丈,仰视金瓶透碧空。
上下有光凝宇宙,东西无影映帘栊。
风吹宝铎闻天乐,日映冰虬对梵宫。
飞宿灵禽时诉语,遥瞻淮水渺无穷。

行者且观且走,直至二层门下。那国师王菩萨早已知之,即与小张太子出门迎
迓。相见叙礼毕,行者道:“我保唐僧西天取经,路上有个小雷音寺,那里有个黄
眉怪,假充佛祖。我师父不辨真伪,就下拜,被他拿了。又将金铙把我罩了,幸亏
天降星辰救出。是我打碎金铙,与他赌斗,又将一个布搭包儿,把天神、揭谛、伽
蓝与我师父、师弟尽皆装了进去。我前去武当山请玄天上帝救援,他差五龙、龟、
蛇拿怪,又被他一搭包子装去。弟子无依无倚,故来拜请菩萨,大展威力,将那收
水母之神通,拯生民之妙用,同弟子去救师父一难!取得经回,永传中国,扬我佛
之智慧,兴般若之波罗也。”

国师王道:“你今日之事,诚我佛教之兴隆,理当亲去;奈时值初夏,正淮水
泛涨之时。新收了水猿大圣,那厮遇水即兴;恐我去后,他乘空生顽,无神可治。
今着小徒领四将和你去助力,炼魔收伏罢。”行者称谢。即同四将并小张太子,又
驾云回小西天。直至小雷音寺,小张太子使一条楮白枪,四大将轮四把锟剑,和
孙大圣上前骂战。小妖又去报知,那妖王复帅群妖,鼓噪而出道:“猢狲!你今又请
得何人来也?”说不了,小张太子指挥四将,上前喝道:“泼妖精!你面上无肉,不
认得我等在此!”妖王道:“是那方小将,敢来与他助力?”太子道:“吾乃泗州大
圣国师王菩萨弟子,帅领四大神将,奉令擒你!”妖王笑道:“你这孩儿有甚武艺,
擅敢到此轻薄?”太子道:“你要知我武艺,等我道来:

祖居西土流沙国,我父原为沙国王。自幼一身多疾苦,命干华盖恶星妨。因师
远慕长生诀,有分相逢舍药方。半粒丹砂祛病退,愿从修行不为王。学成不老同天
寿,容颜永似少年郎。也曾赶赴龙华会,也曾腾云到佛堂。捉雾拿风收水怪,擒龙
伏虎镇山场。抚民高立浮屠塔,静海深明舍利光。楮白枪尖能缚怪,淡缁衣袖把妖
降。如今静乐城内,大地扬名说小张!”

妖王听说,微微冷笑道:“那太子,你舍了国家,从那国师王菩萨,修的是甚
么长生不老之术?只好收捕淮河水怪。却怎么听信孙行者诳谬之言,千山万水,来
此纳命!看你可长生可不老也!”

小张闻言,心中大怒,缠枪当面便刺,四大将一拥齐攻,孙大圣使铁棒上前又
打。好妖精,公然不惧,轮着他那短软狼牙棒,左遮右架,直挺横冲。这场好杀:

小太子,楮白枪,四柄锟剑更强。悟空又使金箍棒,齐心围绕杀妖王。妖王
其实神通大,不惧分毫左右搪。狼牙棒是佛中宝,剑砍枪轮莫可伤。只听狂风声吼
吼,又观恶气混茫茫。那个有意思凡弄本事,这个专心拜佛取经章。几番驰骋,数
次张狂。喷云雾,闭三光,奋怒怀嗔各不良。多时三乘无上法,致令百艺苦相将。
概众争战多时,不分胜负。那妖精又解搭包儿。行者又叫:“列位仔细!”太子并众
等不知“仔细”之意。那怪“滑”的一声,把四大将与太子,一搭包又装将进去,
只是行者预先知觉走了,那妖王得胜回寺,又教取绳捆了,送在地窖,牢封固锁不
题。

这行者纵筋斗云,起在空中,见那怪回兵闭门,方才按下祥光,立于西山坡上,
怅望悲啼道:“师父啊!我
自从秉教入禅林,感荷菩萨脱难深。
保你西来求大道,相同辅助上雷音。
只言平坦羊肠路,岂料崔巍怪物侵。
百计千方难救你,东求西告枉劳心!”

大圣正当凄惨之时,忽见那西南上一朵彩云坠地,满山头大雨缤纷,有人叫道:
“悟空,认得我么?”行者急走前看处,那个人:
大耳横颐方面相,肩查腹满身躯胖。
一腔春意喜盈盈,两眼秋波光荡荡。
敞袖飘然福气多,芒鞋洒落精神壮。
极乐场中第一尊,南无弥勒笑和尚。
行者见了,连忙下拜道:“东来佛祖,那里去?弟子失回避了。万罪,万罪!”佛祖
道:“我此来,专为这小雷音妖怪也。”行者道:“多蒙老爷盛德大恩。敢问那妖是
那方怪物,何处精魔,不知他那搭包儿是件甚么宝贝,烦老爷指示指示。”佛祖道:
“他是我面前司磬的一个黄眉童儿。三月三日,我因赴元始会去,留他在宫看守,
他把我这几件宝贝拐来,假佛成精。那搭包儿是我的后天袋子,俗名唤做‘人种袋’。
那条狼牙棒是个敲磬的槌儿。”

行者听说,高叫一声道:“好个笑和尚!你走了这童儿,教他诳称佛祖,陷害老
孙,未免有个家法不谨之过!”弥勒道:“一则是我不谨,走失人口;二则是你师徒
们魔障未完:故此百灵下界,应该受难。我今来与你收他去也。”行者道:“这妖精
神通广大,你又无些兵器,何以收之?”弥勒笑道:“我在这山坡下,设一草庵,
种一田瓜果在此,你去与他索战。交战之时,许败不许胜,引他到我这瓜田里。我
别的瓜都是生的,你却变做一个大熟瓜。他来定要瓜吃,我却将你与他吃。吃下肚
中,任你怎么在内摆布他。那时等我取了他的搭包儿,装他回去。”行者道:“此计
虽妙,你却怎么认得变的熟瓜?他怎么就肯跟我来此?”弥勒笑道:“我为治世之尊,
慧眼高明,岂不认得你!凭你变作甚物,我皆知之。但恐那怪不肯跟来耳。我却教
你一个法术。”行者道:“他断然是以搭包儿装我,怎肯跟来!有何法术可来也?”
弥勒笑道:“你伸手来。”行者即舒左手,递将过去。弥勒将右手食指,蘸着口中神
水,在行者掌上写了一个“禁”字,教他捏着拳头,见妖精当面放手,他就跟来。

行者拳,欣然领教。一只手轮着铁棒,直至山门外,高叫道:“妖魔,你孙
爷爷又来了!可快出来,与你见个上下!”小妖又忙忙奔告。妖王问道:“他又领多
少兵来叫战?”小妖道:“别无甚兵,止他一个。”妖王笑道:“那猴儿计穷力竭,
无处求人,断然是送命来也。”

随又结束整齐,带了宝贝,举着那轻软狼牙棒,走出门来,叫道:“孙悟空,
今番挣挫不得了!”行者骂道:“泼怪物!我怎么挣挫不得?”妖王道:“我见你计穷
力竭,无处求人,独自个强来支持,如今拿住,再没个甚么神兵救拔,此所以说你
挣挫不得也。”行者道:“这怪不知死活!莫说嘴,吃吾一棒!”那妖王见他一只手轮
棒,忍不住笑道:“这猴儿,你看他弄巧!怎么一只手使棒支吾?”行者道:“儿子!
你禁不得我两只手打。若是不使搭包子,再着三五个,也打不过老孙这一只手!”
妖王闻言,道:“也罢,也罢!我如今不使宝贝,只与你实打,比个雌雄。”即举狼
牙棒,上前来斗。孙行者迎着面,把拳头一放,双手轮棒。那妖精着了禁,不思退
步,果然不弄搭包,只顾使棒来赶。行者虚幌一下,败阵就走。那妖精直赶到西山
坡下。

行者见有瓜田,打个滚,钻入里面,即变做一个大熟瓜,又熟又甜。那妖精停
身四望,不知行者那方去了。他却赶至庵边叫道:“瓜是谁人种的?”弥勒变作一
个种瓜叟,出草庵答道:“大王,瓜是小人种的。”妖王道:“可有熟瓜么?”弥勒
道:“有熟的。”妖王叫:“摘个熟的来,我解渴。”弥勒即把行者变的那瓜,双手递
与妖王。妖王更不察情,到此接过手,张口便啃。那行者乘此机会,一毂辘钻入咽
喉之下,等不得好歹,就弄手脚。抓肠蒯腹,翻根头,竖蜻蜓,任他在里面摆布。
那妖精疼得牙嘴,眼泪汪汪,把一块种瓜之地,滚得似个打麦之场,口中只叫:
“罢了,罢了,谁人救我一救!”弥勒却现了本象,嘻嘻笑叫道:“孽畜!认得我么?”
那妖抬头看见,慌忙跪倒在地,双手揉着肚子,磕头撞脑,只叫:“主人公!饶我命
罢,饶我命罢!再不敢了!”

弥勒上前,一把揪住,解了他的后天袋儿,夺了他的敲磬槌儿,叫:“孙悟空,
看我面上,饶他命罢。”行者十分恨苦,却又左一拳,右一脚,在里面乱掏乱捣。
那怪万分疼痛难忍,倒在地下。弥勒又道:“悟空,他也够了,你饶他罢。”行者才
叫:“你张大口,等老孙出来。”那怪虽是肚腹绞痛,还未伤心。俗语云:“人未伤
心不得死,花残叶落是根枯。”他听见叫张口,即便忍着疼,把口大张。行者方才
跳出,现了本象,急掣棒还要打时,早被佛祖把妖精装在袋里,斜跨在腰间。手执
着磬槌,骂道:“孽畜!金铙偷了那里去了?”那怪却只要怜生,在后天袋内哼哼
的道:“金铙是孙悟空打破了。”佛祖道:“铙破,还我金来。”那怪道:“碎金堆
在殿莲台上哩。”

那佛祖提着袋子,执着磬槌,嘻嘻笑叫道:“悟空,我和你去寻金还我。”行者
见此法力,怎敢违误。只得引佛上山,回至寺内,收取金碴。只见那山门紧闭。佛
祖使槌一指,门开入里看时,那些小妖,已得知老妖被擒,各自收拾囊底,都要逃
生四散。被行者见一个,打一个;见两个,打两个;把五七百个小妖,尽皆打死。
各现原身,都是些山精树怪,兽孽禽魔。佛祖将金收攒一处,吹口仙气,念声咒语,
即时返本还原,复得金铙一副。别了行者,驾祥云,径转极乐世界。

这大圣却才解下唐僧、八戒、沙僧。那呆子吊了几日,饿得慌了,且不谢大圣,
却就虾着腰,跑到厨房寻饭吃。原来那怪正安排了午饭,因行者索战,还未得吃。
这呆子看见,即吃了半锅,却拿出两钵头叫师父、师弟们各吃了两碗,然后才谢了
行者。问及妖怪原由。行者把先请祖师,龟、蛇,后请大圣借太子,并弥勒收降之
事,细陈了一遍。三藏闻言,谢之不尽,顶礼了诸天,道:“徒弟,这些神圣,困
于何所?”行者道:“昨日日值功曹对老孙说,都在地窖之内。”叫:“八戒,我与
你去解脱他等。”

那呆子得食力壮,抖擞精神,寻着他的钉钯,即同大圣到后面,打开地窖,将
众等解了绳,请出珍楼之下。三藏披了袈裟,朝上一一拜谢。这大圣才送五龙、二
将回武当;送小张太子与四将回城;后送二十八宿归天府;发放揭谛、伽蓝各回
境。师徒们却宽住了半日。喂饱了白马,收拾行囊,至次早登程。临行时,放上一
把火,将那些珍楼、宝座、高阁、讲堂,俱尽烧为灰烬。这里才:
无挂无牵逃难去,消灾消障脱身行。

毕竟不知几时才到大雷音,且听下回分解。