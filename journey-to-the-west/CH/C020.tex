\chapter{黄风岭唐僧有难~半山中八戒争先}

偈曰:

法本从心生,还是从心灭。生灭尽由谁,请君自辨别。既然皆己心,何用别人
说?只须下苦功,扭出铁中血。绒绳着鼻穿,挽定虚空结。拴在无为树,不使他颠
劣。莫认贼为子,心法都忘绝。休教他瞒我,一拳先打彻。现心亦无心,现法法也
辍。人牛不见时,碧天光皎洁。秋月一般圆,彼此难分别。

这一篇偈子,乃是玄奘法师悟彻了《多心经》,打开了门户。那长老常念常存,
一点灵光自透。

且说他三众,在路餐风宿水,带月披星,早又至夏景炎天。但见那:

花尽蝶无情叙,树高蝉有声喧。野蚕成茧火榴妍,沼内新荷出现。

那日正行时,忽然天晚,又见山路旁边,有一村舍。三藏道:“悟空,你看那
‘日落西山藏火镜,月升东海现冰轮’。幸而道旁有一人家,我们且借宿一宵,明
日再走。”八戒道:“说得是。我老猪也有些饿了,且到人家化些斋吃,有力气,好
挑行李。”行者道:“这个恋家鬼!你离了家几日,就生报怨!”八戒道:“哥啊,似
不得你这喝风呵烟的人。我从跟了师父这几日,长忍半肚饥,你可晓得?”三藏闻
之道:“悟能,你若是在家心重呵,不是个出家的了,你还回去罢。”那呆子慌得跪
下道:“师父,你莫听师兄之言。他有些赃埋人。我不曾报怨甚的,他就说我报怨。
我是个直肠的痴汉,我说道肚内饥了,好寻个人家化斋,他就骂我是恋家鬼。师父
啊,我受了菩萨的戒行,又承师父怜悯,情愿要伏侍师父往西天去,誓无退悔。这
叫做‘恨苦修行’。怎的说不是出家的话!”三藏道:“既是如此,你且起来。”那呆
子纵身跳起,口里絮絮叨叨的,挑着担子,只得死心塌地,跟着前来。

早到了路旁人家门首。三藏下马,行者接了缰绳,八戒歇了行李,都伫立绿荫
之下。三藏拄着九环锡杖,按按藤缠篾织斗篷,先奔门前,只见一老者,斜倚竹床
之上,口里嘤嘤的念佛。三藏不敢高言,慢慢的叫一声“施主,问讯了。”那老者
一骨鲁跳将起来,忙敛衣襟,出门还礼道:“长老,失迎。你自那方来的?到我寒门
何故?”三藏道:“贫僧是东土大唐和尚,奉圣旨,上雷音寺拜佛求经。适至宝方
天晚,意投檀府告借一宵,万祈方便方便。”那老儿摆手摇头道:“去不得。西天难
取经。要取经,往东天去罢。”三藏口中不语,意下沉吟:“菩萨指道西去,怎么此
老说往东行?东边那得有经?……”腼腆难言,半晌不答。

却说行者素性凶顽,忍不住,上前高叫道:“那老儿,你这们大年纪,全不晓
事。我出家人远来借宿,就把这厌钝的话虎唬我。十分你家窄狭,没处睡时,我们
在树底下好道也坐一夜,不打搅你。”那老者扯住三藏道:“师父,你倒不言语,你
那个徒弟,那般拐子脸,别颏腮,雷公嘴,红眼睛的一个痨病魔鬼,怎么反冲撞我
这年老之人!”行者笑道:“你这个老儿,忒也没眼色!似那俊刮些儿的,叫做中看
不中吃。想我老孙,虽小,颇结实,皮裹一团筋哩。”那老者道:“你想必有些手段。”
行者道:“不敢夸言,也将就看得过。”老者道:“你家居何处?因甚事削发为僧?”
行者道:“老孙祖贯东胜神洲海东傲来国花果山水帘洞居住。自小儿学做妖怪,称
名悟空。凭本事,挣了一个齐天大圣。只因不受天禄,大反天宫,惹了一场灾愆。
如今脱难消灾,转拜沙门,前求正果,保我这唐朝驾下的师父,上西天拜佛走遭,
怕甚么山高路险,水阔波狂!我老孙也捉得怪,降得魔。伏虎擒龙,踢天弄井,都
晓得些儿。倘若府上有甚么丢砖打瓦,锅叫门开,老孙便能安镇。”

那老儿听得这篇言语,哈哈笑道:“原来是个撞头化缘的熟嘴儿和尚。”行者道:
“你儿子便是熟嘴!我这些时,只因跟我师父走路辛苦,还懒说话哩。”那老儿道:
“若是你不辛苦,不懒说话,好道活活的聒杀我!你既有这样手段,西方也还去得,
去得。你一行几众?请至茅舍里安宿。”三藏道:“多蒙老施主不叱之恩。我一行三
众。”老者道:“那一众在那里?”行者指着道:“这老儿眼花,那绿荫下站的不是?”
老儿果然眼花,忽抬头细看,一见八戒这般嘴脸,就唬得一步一跌,往屋里乱跑,
只叫:“关门,关门!妖怪来了!”行者赶上扯住道:“老儿莫怕,他不是妖怪,是我
师弟。”老者战兢兢的道:“好,好,好!一个丑似一个的和尚!”八戒上前道:“老
官儿,你若以相貌取人,干净差了。我们丑自丑,却都有用。”那老者正在门前与
三个和尚相讲,只见那庄南边有两个少年人,带着一个老妈妈,三四个小男女,敛
衣赤脚,插秧而回。他看见一匹白马,一担行李,都在他家门首喧哗,不知是甚来
历,都一拥上前问道:“做甚么的?”八戒调过头来,把耳朵摆了几摆,长嘴伸了
一伸,吓得那些人东倒西歪,乱跄乱跌。慌得那三藏满口招呼道:“莫怕,莫怕!我
们不是歹人,我们是取经的和尚。”那老儿才出了门,搀着妈妈道:“婆婆起来,少
要惊恐。这师父,是唐朝来的,只是他徒弟脸嘴丑些,却也山恶人善。带男女们家
去。”那妈妈才扯着老儿,二少年领着儿女进去。

三藏却坐在他门楼里竹床之上,埋怨道:“徒弟呀,你两个相貌既丑,言语又
粗,把这一家儿吓得七损八伤,都替我身造罪哩!”八戒道:“不瞒师父说,老猪自
从跟了你,这些时俊了许多哩。若像往常在高老庄走时,把嘴朝前一掬,把耳两头
一摆,常吓杀二三十人哩。”行者笑道:“呆子不要乱说,把那丑也收拾起些。”三
藏道:“你看悟空说的话。相貌是生成的,你教他怎么收拾?”行者道:“把那个耙
子嘴,揣在怀里,莫拿出来;把那蒲扇耳,贴在后面,不要摇动,这就是收拾了。”
那八戒真个把嘴揣了,把耳贴了,拱着头,立于左右。行者将行李拿入门里,将白
马拴在桩上。

只见那老儿才引个少年,拿一个板盘儿,托三杯清茶来献。茶罢,又吩咐办斋。
那少年又拿一张有窟窿无漆水的旧桌,端两条破头折脚的凳子,放在天井中,请三
众凉处坐下。三藏方问道:“老施主,高姓?”老者道:“在下姓王。”——“有几
位令嗣?”——道:“有两个小儿,三个小孙。”三藏道:“恭喜,恭喜。”又问:“年
寿几何?”——道:“痴长六十一岁。”行者道:“好,好,好!花甲重逢矣。”三藏
复问道:“老施主,始初说西天经难取者,何也?”老者道:“经非难取,只是道中
艰涩难行。我们这向西去,只有三十里远近,有一座山,叫做八百里黄风岭。那山
中多有妖怪。故言难取者,此也。若论此位小长老,说有许多手段,却也去得。”
行者道:“不妨,不妨!有了老孙与我这师弟,任他是甚么妖怪,不敢惹我。”

正说处,又见儿子拿将饭来,摆在桌上,道声“请斋”。三藏就合掌讽起斋经。
八戒早已吞了一碗。长老的几句经还未了,那呆子又吃够三碗。行者道:“这个馕
糠!好道汤着饿鬼了!”那老王倒也知趣,见他吃得快,道:“这个长老,想着实饿
了,快添饭来。”那呆子真个食肠大:看他不抬头,一连就吃有十数碗。三藏、行
者俱各吃不上两碗。呆子不住,便还吃哩。老王道:“仓卒无肴,不敢苦劝,请再
进一。”三藏、行者俱道:“够了。”八戒道:“老儿滴答甚么,谁和你发课,说甚
么五爻六爻;有饭只管添将来就是。”呆子一顿,把他一家子饭都吃得罄尽,还只
说才得半饱。却才收了家火,在那门楼下,安排了竹床板铺睡下。

次日天晓,行者去背马,八戒去整担,老王又教妈妈整治些点心汤水管待,三
众方致谢告行。老者道:“此去倘路间有甚不虞,是必还来茅舍。”行者道:“老儿,
莫说哈话。我们出家人,不走回头路。”遂此策马挑担西行。

噫!这一去,果无好路朝西域,定有邪魔降大灾。三众前来,不上半日,果逢
一座高山。说起来,十分险峻。三藏马到临崖,斜挑宝观看,果然那:

高的是山,峻的是岭;陡的是崖,深的是壑;响的是泉,鲜的是花。那山高不
高,顶上接青霄;这涧深不深,底中见地府。山前面,有骨都都白云,屹嶝嶝怪石;
说不尽千丈万丈挟魂崖。崖后有,弯弯曲曲藏龙洞,洞中有叮叮当当滴水岩。又见
些丫丫叉叉带角鹿,泥泥痴痴看人獐;盘盘曲曲红鳞蟒,耍耍顽顽白面猿。至晚巴
山寻穴虎,带晓翻波出水龙,登的洞门唿喇喇响。草里飞禽,扑轳轳起;林中走兽,
掬行。猛然一阵狼虫过,吓得人心蹬蹬惊。正是那当倒洞当当倒洞,洞当当
倒洞当山;青岱染成千丈玉,碧纱笼罩万堆烟。
那师父缓促银骢,孙大圣停云慢步,猪悟能磨担徐行。正看那山,忽闻得一阵旋风
大作。三藏在马上心惊,道:“悟空,风起了!”行者道:“风却怕他怎的!此乃天家
四时之气,有何惧哉!”三藏道:“此风甚恶,比那天风不同。”行者道:“怎见得不
比天风?”三藏道:“你看这风:

巍巍荡荡飒飘飘,渺渺茫茫出碧霄。过岭只闻千树吼,入林但见万竿摇。岸边
摆柳连根动,园内吹花带叶飘。收网渔舟皆紧缆,落篷客艇尽抛锚。途半征夫迷失
路,山中樵子担难挑。仙果林间猴子散,奇花丛内鹿儿逃。崖前桧柏颗颗倒,涧下
松篁叶叶凋。播土扬尘沙迸迸,翻江搅海浪涛涛。”

八戒上前,一把扯住行者道:“师兄,十分风大!我们且躲一躲儿干净。”行者
笑道:“兄弟不济!风大时就躲,倘或亲面撞见妖精,怎的是好?”八戒道:“哥啊,
你不曾闻得‘避色如避仇,避风如避箭’哩!我们躲一躲,也不亏人。”行者道:“且
莫言语,等我把这风抓一把来闻一闻看。”八戒笑道:“师兄又扯空头谎了,风又好
抓得过来闻!就是抓得来,便也渍了去了。”行者道:“兄弟,你不知道老孙有个‘抓
风’之法。”

好大圣,让过风头,把那风尾抓过去闻了一闻,有些腥气,道:“果然不是好
风!这风的味道不是虎风,定是怪风。断乎有些蹊跷。”说不了,只见那山坡下,剪
尾跑蹄,跳出一只斑斓猛虎,慌得那三藏坐不稳雕鞍,翻根头跌下白马,斜倚在路
旁,真个是魂飞魄散。八戒丢了行李,掣钉钯,不让行者走上前,大喝一声道:“孽
畜!那里走!”赶将去,劈头就筑。那只虎直挺挺站将起来,把那前左爪轮起,抠住
自家的胸膛,往下一抓,滑剌的一声,把个皮剥将下来,站立道旁。你看他怎生恶
相!咦,那模样:

血津津的赤剥身躯,红媸媸的弯环腿足。火焰焰的两鬓蓬松,硬搠搠的双眉的
竖。白森森的四个钢牙,光耀耀的一双金眼。气昂昂的努力大哮,雄纠纠的厉声高
喊。
喊道:“慢来!慢来!吾党不是别人,乃是黄风大王部下的前路先锋。今奉大王严命,
在山巡逻,要拿几个凡夫去做案酒。你是那里来的和尚,敢擅动兵器伤我?”八戒
骂道:“我把你这个孽畜!你是认不得我!我等不是那过路的凡夫,乃东土大唐御弟
三藏之弟子,奉旨上西方拜佛求经者。你早早的远避他方,让开大路,休惊了我师
父,饶你性命;若似前猖獗,钯举处,却不留情!”那妖精那容分说,急近步,丢
一个架子,望八戒劈脸来抓。这八戒忙闪过,轮钯就筑。那怪手无兵器,下头就走,
八戒随后赶来。那怪到了山坡下,乱石丛中,取出两口赤铜刀,急轮起,转身来迎。
两个在这坡前,一往一来,一冲一撞的赌斗。那里孙行者搀起唐僧道:“师父,你
莫害怕。且坐住,等老孙去助助八戒,打倒那怪好走。”三藏才坐将起来,战兢兢
的,口里念着《多心经》不题。

那行者掣了铁棒,喝声叫“拿了!”此时八戒抖擞精神,那怪败下阵去。行者
道:“莫饶他!务要赶上!”他两个轮钉钯,举铁棒,赶下山来。那怪慌了手脚,使
个“金蝉脱壳计”,打个滚,现了原身,依然是一只猛虎。行者与八戒那里肯舍,
赶着那虎,定要除根。那怪见他赶得至近,却又抠着胸膛,剥下皮来,苫盖在那卧
虎石上,脱真身,化一阵狂风,径回路口。路口上那师父正念《多心经》,被他一
把拿住,驾长风摄将去了。可怜那三藏啊!江流注定多磨折,寂灭门中功行难。

那怪把唐僧擒来洞口,按住狂风,对把门的道:“你去报大王说,前路虎先锋
拿了一个和尚,在门外听令。”那洞主传令,教:“拿进来。”那虎先锋,腰撇着两
口赤铜刀,双手捧着唐僧,上前跪下道:“大王,小将不才,蒙钧令差往山上巡逻,
忽遇一个和尚,他是东土大唐驾下御弟三藏法师,上西方拜佛求经,被我擒来奉上,
聊具一馔。”

那洞主闻得此言,吃了一惊道:“我闻得前者有人传说:三藏法师乃大唐奉旨
意取经的神僧;他手下有一个徒弟,名唤孙行者,神通广大,智力高强。你怎么能
够捉得他来?”先锋道:“他有两个徒弟:先来的,便一柄九齿钉钯,他生得嘴长
耳大;又一个,使一根金箍铁棒,他生得火眼金睛。正赶着小将争持,被小将使一
个‘金蝉脱壳’之计,撤身得空,把这和尚拿来,奉献大王,聊表一餐之敬。”

洞主道:“且莫吃他着。”先锋道:“大王,见食不食,呼为劣蹶。”洞主道:“你
不晓得。吃了他不打紧,只恐怕他那两个徒弟上门吵闹,未为稳便。且把他绑在后
园定风桩上,待三五日,他两个不来搅扰,那时节,一则图他身子干净,二来不动
口舌,却不任我们心意?或煮或蒸,或煎或炒,慢慢的自在受用不迟。”先锋大喜道:
“大王深谋远虑,说得有理。”教:“小的们,拿了去。”

旁边拥上七八个绑缚手,将唐僧拿去,好便似鹰拿燕雀,索绑绳缠。这的是苦
命江流思行者,遇难神僧想悟能。道声:“徒弟啊!不知你在那山擒怪,何处降妖,
我却被魔头拿来,遭此毒害,几时再得相见!好苦啊!你们若早些儿来,还救得我命;
若十分迟了,断然不能保矣!”一边嗟叹,一边泪落如雨。

却说那行者、八戒,赶那虎下山坡,只见那虎跑倒了,塌伏在崖前。行者举棒,
尽力一打,转震得自己手疼。八戒复筑了一钯,并将钯齿迸起。原来是一张虎皮,
盖着一块卧虎石。行者大惊道:“不好了,不好了!中了他计也!”八戒道:“中他甚
计?”行者道:“这个叫做‘金蝉脱壳计’:他将虎皮苫在此,他却走了。我们且回
去看看师父,莫遭毒手。”两个急急转来,早已不见了三藏。行者大叫如雷道:“怎
的好!师父已被他擒去了!”八戒即便牵着马,眼中滴泪道:“天哪!天哪!却往那里
找寻!”行者抬着头跳道:“莫哭,莫哭!一哭就挫了锐气。横竖想只在此山,我们
寻寻去来。”

他两个果奔入山中,穿岗越岭,行够多时,只见那石崖之下,耸出一座洞府。
两人定步观瞻,果然凶险。但见那:

迭障尖峰,回峦古道。青松翠竹依依,绿柳碧梧冉冉。崖前有怪石双双,林内
有幽禽对对。涧水远流冲石壁,山泉细滴漫沙堤。野云片片,瑶草芊芊。妖狐狡兔
乱撺梭,角鹿香獐齐斗勇。劈崖斜挂万年藤,深壑半悬千岁柏。奕奕巍巍欺华岳,
落花啼鸟赛天台。
行者道:“贤弟,你可将行李歇在藏风山凹之间,撒放马匹,不要出头。等老孙去
他门首,与他赌斗。必须拿住妖精,方才救得师父。”八戒道:“不消吩咐,请快去。”

行者整一整直裰,束一束虎裙,掣了棒,撞至那门前,只见那门上有六个大字,
乃“黄风岭黄风洞”,却便丁字脚站定,执着棒,高叫道:“妖怪!趁早儿送我师父
出来,省得掀翻了你窝巢,平了你住处!”

那小怪闻言,一个个害怕,战兢兢的,跑入里面报道:“大王!祸事了!”那黄
风怪正坐间,问:“有何事?”小妖道:“洞门外来了一个雷公嘴毛脸的和尚,手持
着一根许大粗的铁棒,要他师父哩!”那洞主惊张,即唤虎先锋道:“我教你去巡山,
只该拿些山牛、野彘、肥鹿、胡羊,怎么拿那唐僧来!却惹他那徒弟来此闹吵,怎
生区处?”先锋道:“大王放心稳便,高枕勿忧,小将不才,愿带领五十个小妖校
出去,把那甚么孙行者拿来凑吃。”洞主道:“我这里除了大小头目,还有五七百名
小校,凭你选择,领多少去。只要拿住那行者,我们才自自在在吃那和尚一块肉,
情愿与你拜为兄弟;但恐拿他不得,反伤了你,那时休得埋怨我也。”

虎怪道:“放心!放心!等我去来。”果然点起五十名精壮小妖,擂鼓摇旗,缠两
口赤铜刀,腾出门来,厉声高叫道:“他是那里来的个猴和尚!敢在此间大呼小叫的
做甚?”行者骂道:“你这个剥皮的畜生!你弄甚么脱壳法儿,把我师父摄了,倒转
问我做甚!趁早好好送我师父出来,还饶你这个性命!”虎怪道:“你师父是我拿了,
要与我大王做顿下饭。你识起倒,回去吧!不然,拿住你,一齐凑吃,却不是‘买
一个又饶一个’?”行者闻言,心中大怒。迸迸,钢牙错啮;滴流流,火眼睁圆;
掣铁棒喝道:“你多大欺心,敢说这等大话,休走,看棍!”那先锋急持刀按住。这
一场果然不善,他两个各显威能。好杀:

那怪是个真鹅卵,悟空是个鹅卵石。赤铜刀架美猴王,浑如垒卵来击石。鸟鹊
怎与凤凰争?鹁鸽敢和鹰鹞敌?那怪喷风灰满山,悟空吐雾云迷日。来往不禁三五回,
先锋腰软全无力。转身败了要逃生,却被悟空抵死逼。

那虎怪撑持不住,回头就走。他原来在那洞主面前说了嘴,不敢回洞,径往山
坡上逃生。行者那里肯放,执着棒,只情赶来,呼呼吼吼,喊声不绝,却赶到那藏
风山凹之间。正抬头,见八戒在那里放马。八戒忽听见呼呼声喊,回头观看,乃是
行者赶败的虎怪,就丢了马,举起钯,刺斜着头一筑。可怜那先锋,脱身要跳黄丝
网,岂知又遇罩鱼人。却被八戒一钯,筑得九个窟窿鲜血冒,一头脑髓尽流干。有
诗为证,诗曰:
三五年前归正宗,持斋把素悟真空。
诚心要保唐三藏,初秉沙门立此功。

那呆子一脚住他的脊背,两手轮钯又筑。行者见了,大喜道:“兄弟,正是
这等!他领了几十个小妖,敢与老孙赌斗;被我打败了,他转不往洞跑,却跑来这
里寻死。亏你接着;不然,又走了。”八戒道:“弄风摄师父去的可是他?”行者道:
“正是,正是。”八戒道:“你可曾问他师父的下落么?”行者道:“这怪把师父拿
在洞里,要与他甚么鸟大王做下饭。是老孙恼了,就与他斗将这里来,却着你送了
性命。兄弟啊,这个功劳算你的。你可还守着马与行李,等我把这死怪拖了去,再
到那洞口索战。须是拿得那老妖,方才救得师父。”八戒道:“哥哥说得有理。你去,
你去。若是打败了这老妖,还赶将这里来,等老猪截住杀他。”好行者,一只手提
着铁棒,一只手拖着死虎,径至他洞口。正是:
法师有难逢妖怪,情性相和伏乱魔。

毕竟不知此去可降得妖怪,救得唐僧,且听下回分解。