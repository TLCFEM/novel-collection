\chapter{假合真形擒玉兔~真阴归正会灵元}

却说那唐僧忧忧愁愁,随着国王至后宫,只听得鼓乐喧天,随闻得异香扑鼻,
低着头,不敢仰视。行者暗里欣然,丁在那毗卢帽顶上,运神光,睁火眼金睛观看,
又只见那两班彩女,摆列的似蕊宫仙府,胜强似锦帐春风。真个是:

娉婷袅娜,玉质冰肌。一双双娇欺楚女,一对对美赛西施。云髻高盘飞彩凤,
娥眉微显远山低。笙簧杂奏,箫鼓频吹。宫商角征羽,抑扬高下齐。清歌妙舞常堪
爱,锦砌花团色色怡。
行者见师父全不动念,暗自里咂嘴夸称道:“好和尚,好和尚!身居锦绣心无爱,足
步琼瑶意不迷。”

少时,皇后、嫔妃簇拥着公主出鹊宫,一齐迎接,都道声:“我王万岁,万
万岁!”慌的个长老战战兢兢,莫知所措。行者早已知识,见那公主头顶上微露出
一点妖氛,却也不十分凶恶,即忙爬近耳朵叫道:“师父,公主是个假的。”长老道:
“是假的,却如何教他现相?”行者道:“使出法身,就此拿他也。”长老道:“不
可,不可!恐惊了主驾。且待君后退散,再使法力。”

那行者一生性急,那里容得,大咤一声,现了本相,赶上前,揪住公主骂道:
“好孽畜!你在这里弄假成真,只在此这等受用也尽够了,心尚不足,还要骗我师
父,破他的真阳,遂你的淫性哩!”唬得那国王呆呆挣挣,后妃跌跌爬爬,宫娥彩
女,无一个不东躲西藏,各顾性命。好便似:

春风荡荡,秋气潇潇:春风荡荡过园林,千花摆动;秋气潇潇来径苑,万叶飘
摇。刮折牡丹槛下,吹歪芍药卧栏边。沼岸芙蓉乱撼,台基菊蕊铺堆。海棠无力
倒尘埃,玫瑰有香眠野径。春风吹折芰荷,冬雪压歪梅嫩蕊。石榴花瓣,乱落在
内院东西;岸柳枝条,斜垂在皇宫南北。好花风雨一宵狂,无数残红铺地锦。
三藏一发慌了手脚,战兢兢抱住国王,只叫:“陛下,莫怕,莫怕!此是我顽徒使法
力,辨真假也。”

却说那妖精见事不谐,挣脱了手,解剥了衣裳,头,摇落了钗环首饰,即
跑到御花园土地庙里,取出一条碓嘴样的短棍,急转身来乱打行者。行者随即跟来,
使铁棒劈面相迎。他两个喝喝,就在花园斗起。后却大显神通,各驾云雾,杀
在空中。这一场:

金箍铁棒有名声,碓嘴短棍无人识。一个因取真经到此方,一个为爱奇花来住
迹。那怪久知唐圣僧,要求配合元精液,旧年摄去真公主,变作人身钦爱惜。今逢
大圣认妖氛,救援活命分虚实。短棍行凶着顶丢,铁棒施威迎面击。喧喧嚷嚷两相
持,云雾满天遮白日。
他两个杀在半空赌斗,吓得那满城中百姓心慌,尽朝里多官胆怕。长老扶着国王,
只叫:“休惊!请劝娘娘与众等莫怕。你公主是个假作真形的。等我徒弟拿住他,方
知好歹也。”那些妃子,有胆大的,把那衣服、钗环拿与皇后看了,道:“这是公主
穿的,戴的,今都丢下,精着身子,与那和尚在天上争打,必定是个妖邪。”此时
国王、后妃人等才正了性,望空仰视不题。

却说那妖精与大圣斗经半日,不分胜败。行者把棒丢起,叫一声:“变!”就以
一变十,以十变百,以百变千,半天里,好似蛇游蟒搅,乱打妖邪。妖邪慌了手脚,
将身一闪,化道清风,即奔碧空之上逃走。行者念声咒语,将铁棒收做一根,纵祥
光一直赶来。将近西天门,望见那旌旗灼,行者厉声高叫道:“把天门的,挡住
妖精,不要放他走了!”真个那天门上,有护国天王帅领着庞、刘、苟、毕四大元
帅,各展兵器拦阻。妖邪不能前进,急回头,舍死忘生,使短棍又与行者相持。

这大圣用心力轮铁棒,仔细迎着看时,见那短棍儿一头壮,一头细,却似舂碓
臼的杵头模样,叱咤一声,喝道:“孽畜!你拿的是甚么器械,敢与老孙抵敌!快早
降伏,免得这一棒打碎你的天灵!”那妖邪咬着牙道:“你也不知我这兵器!听我道:

仙根是段羊脂玉,磨琢成形不计年。混沌开时吾已得,洪蒙判处我当先。源流
非比凡间物,本性生来在上天。一体金光和四相,五行瑞气合三元。随吾久住蟾宫
内,伴我常居桂殿边。因为爱花垂世境,故来天竺假婵娟。与君共乐无他意,欲配
唐僧了宿缘。你怎欺心破佳偶,死寻赶战逞凶顽!这般器械名头大,在你金箍棒子
前。广寒宫里捣药杵,打人一下命归泉!”

行者闻说,呵呵冷笑道:“好孽畜啊!你既住在蟾宫之内,就不知老孙的手段?
你还敢在此支吾?快早现相降伏,饶你性命!”那怪道:“我认得你是五百年前大闹
天宫的弼马温,理当让你;但只是破人亲事,如杀父母之仇,故此情理不甘,要打
你欺天罔上的弼马温!”那大圣恼得是“弼马温”三字。他听得此言,心中大怒,
举铁棒劈面就打。那妖邪轮杵来迎。就于西天门前,发狠相持。这一场:

金箍棒,捣药杵,两般仙器真堪比。那个为结婚姻降世间,这个因保唐僧到这
里。原来是国王没正经,爱花引得妖邪
喜。致使如今恨苦争,两家都把顽心起。一冲一撞赌输赢,语言齐斗嘴,药杵
英雄世罕稀,铁棒神威还更美。金光湛湛幌天门,彩雾辉辉连地里。来往战经十数
回,妖邪力弱难搪抵。
那妖精与行者又斗了十数回,见行者的棒势紧密,料难取胜,虚丢一杵,将身幌一
幌,金光万道,径奔正南上败走。大圣随后追袭。忽至一座大山,妖精按金光,钻
入山洞,寂然不见。又恐他遁身回国,暗害唐僧,他认了这山的规模,返云头径转
国内。

此时有申时矣。那国王正扯着三藏,战战兢兢,只叫:“圣僧救我!”那些嫔妃、
皇后也正怆惶,只见大圣自云端里落将下来,叫道:“师父,我来也!”三藏道:“悟
空立住,不可惊了圣躬。我问你:假公主之事,端的如何?”行者立于鹊宫外,
叉手当胸道:“假公主是个妖邪。初时与他打了半日,他战不过我,化道清风,径
往天门上跑,是我喝天神挡住。他现了相,又与我斗到十数合,又将身化作金光,
败回正南上一座山上。我急追至山,无处寻觅,恐怕他来此害你,特地回顾也。”
国王听说,扯着唐僧问道:“既然假公主是个妖邪,我真公主在于何处?”行者应
声道:“待我拿住假公主,你那真公主自然来也。”那后妃等闻得此言,都解了恐惧,
一个个上前拜告道:“望圣僧救得我真公主来,分了明暗,必当重谢。”行者道:“此
间不是我们说话处,请陛下与我师出宫上殿,娘娘等各转各宫,召我师弟八戒、沙
僧来保护师父,我却好去降妖。一则分了内外,二则免我悬心。谨当辨明,以表我
一场心力。”国王依言,感谢不已。遂与唐僧携手出宫,径至殿上。众后妃各各回
宫。一壁厢教备素膳,一壁厢请八戒、沙僧。须臾间,二人早至。行者备言前事,
教他两个用心护持。这大圣纵筋斗云,飞空而去。那殿前多官,一个个望空礼拜不
题。

孙大圣径至正南方那座山上寻找。原来那妖邪败了阵,到此山,钻入窝中,将
门儿使石块挡塞,虚怯怯藏隐不出。行者寻一会不见动静,心甚焦恼,捻着诀,念
动真言,唤出那山中土地、山神审问。少时,二神至了,叩头道:“不知!不知!知
当远接。万望恕罪!”行者道:“我且不打你。我问你:这山叫做甚么名字?此处有
多少妖精?从实说来,饶你罪过。”二神告道:“大圣,此山唤做毛颖山。山中只有
三处兔穴。亘古至今,没甚妖精。乃五环之福地也。大圣要寻妖精,还是西天路上
去有。”行者道:“老孙到了西天天竺国,那国王有个公主被个妖精摄去,抛在荒野,
他就变做公主模样,戏哄国王,结彩楼,抛绣球,欲招驸马。我保唐僧至其楼下,
被他有心打着唐僧,欲为配偶,诱取元阳。是我识破,就于宫中现身捉获。他就脱
了人衣、首饰,使一条短棍,唤名捣药杵,与我斗了半日,他就化清风而去。被老
孙赶至西天门,又斗有十数合,他料不能胜,复化金光,逃至此处。如何不见?”

二神听说,即引行者去那三窟中寻找。始于山脚下窟边看处,亦有几个草兔儿,
也惊得走了。寻至绝顶上窟中看时,只见两块大石头,将窟门挡住。土地道:“此
间必是妖邪赶急钻进去也。”行者即使铁棒,捎开石块。那妖邪果藏在里面,呼的
一声,就跳将出来,举药杵来打。行者轮起铁棒架住,唬得那山神倒退,土地忙奔。
那妖邪口里囔囔突突的,骂着山神、土地道:“谁教你引着他往这里来找寻!”他支
支撑撑的,抵着铁棒,且战且退,奔至空中。

正在危急之际,却又天色晚了。这行者愈发狠性,下毒手,恨不得一棒打杀。
忽听得九霄碧汉之间,有人叫道:“大圣,莫动手!莫动手!棍下留情!”行者回头看
时,原来是太阴星君,后带着娥仙子,降彩云到于当面。慌得行者收了铁棒,躬
身施礼道:“老太阴,那里来的?老孙失回避了。”太阴道:“与你对敌的这个妖邪,
是我广寒宫捣玄霜仙药之玉兔也。他私自偷开玉关金锁,走出宫来,经今一载。我
算他目下有伤命之灾,特来救他性命。望大圣看老身饶他罢。”行者喏喏连声,只
道:“不敢,不敢!怪道他会使捣药杵!原来是个玉兔儿!老太阴不知,他摄藏了天竺
国王之公主,却又假合真形,欲破我圣僧师父之元阳。其情其罪,其实何甘!怎么
便可轻恕饶他?”太阴道:“你亦不知。那国王之公主,也不是凡人,原是蟾宫中
之素娥。十八年前,他曾把玉兔儿打了一掌,却就思凡下界。一灵之光,遂投胎于
国王正宫皇后之腹,当时得以降生。这玉兔儿怀那一掌之仇,故于旧年走出广寒,
抛素娥于荒野。但只是不该欲配唐僧。此罪真不可逭。幸汝留心,识破真假,却也
未曾伤损你师。万望看我面上,恕他之罪,我收他去也。”行者笑道:“既有这些因
果,老孙也不敢抗违。但只是你收了玉兔儿,恐那国王不信,敢烦太阴君同众仙妹
将玉兔儿拿到那厢,对国王明证明证。一则显老孙之手段,二来说那素娥下降之因
由,然后着那国王取素娥公主之身,以见显报之意也。”太阴君信其言,用手指定
妖邪,喝道:“那孽畜还不归正同来!”玉兔儿打个滚,现了原身。真个是:

缺唇尖齿,长耳稀须。团身一块毛如玉,展足千山蹄若飞。直鼻垂酥,果赛霜
华填粉腻;双睛红映,犹欺雪上点胭脂。伏在地,白穰穰一堆素练;伸开腰,白铎
铎一架银丝。几番家吸残清露瑶天晓,捣药长生玉杵奇。
那大圣见了,不胜欣喜,踏云光,向前引导。那太阴君领着众娥仙子,带着玉兔
儿,径转天竺国界。

此时正黄昏,看看月上。到城边,闻得谯楼上擂鼓。那国王与唐僧尚在殿内,
八戒、沙僧与多官都在阶前。方议退朝,只见正南上一片彩霞,光明如昼。众抬头
看处,又闻得孙大圣厉声高叫道:“天竺陛下,请出你那皇后嫔妃看者。这宝幢下
乃月宫太阴星君,两边的仙妹是月里嫦娥。这个玉兔儿却是你家的假公主,今现真
相也。”那国王急召皇后、嫔妃与宫娥、彩女等众,朝天礼拜。他和唐僧及多官亦
俱望空拜谢。满城中各家各户,也无一人不设香案,叩头念佛。

正此观看处,猪八戒动了欲心,忍不住,跳在空中,把霓裳仙子抱住道:“姐
姐,我与你是旧相识,我和你耍子儿去也。”行者上前,揪着八戒,打了两掌,骂
道:“你这个村泼呆子!此是甚么去处,敢动淫心!”八戒道:“拉闲散闷耍子而已!”
那太阴君令转仙幢与众嫦娥收回玉兔,径上月宫而去。

行者把八戒揪落尘埃。这国王在殿上谢了行者。又问前因道:“多感神僧大法
力捉了假公主,朕之真公主,却在何处所也?”行者道:“你那真公主也不是凡胎,
就是月宫里素娥仙子。因十八年前,他将玉兔儿打了一掌,就思凡下界,投胎在你
正宫腹内,生下身来。那玉兔儿怀恨前仇,所以于旧年间偷开玉关金锁走下来,把
素娥摄抛荒野,他却变形哄你。这段因果,是太阴君亲口才与我说的。今日既去其
假者,明日请御驾去寻其真者。”国王闻说,又心意惭惶,止不住腮边流泪道:“孩
儿!我自幼登基,虽城门也不曾出去,却教我那里去寻你也!”行者笑道:“不须烦
恼。你公主现在给孤布金寺里装风。今且各散,到天明我还你个真公主便是。”众
官又拜伏奏道:“我王且心宽。这几位神僧,乃腾云驾雾之神佛,必知未来过去之
因由。明日即烦神僧四众同去一寻,便知端的。”国王依言,即请至留春亭摆斋安
歇。此时已近二更。正是那:
铜壶滴漏月华明,金铎叮当风送声。
杜宇正啼春去半,落花无路近三更。
御园寂寞秋千影,碧落空孚银汉横。
三市六街无客走,一天星斗夜光晴。
当夜各寝不题。

这一夜,国王退了妖气,陡长精神,至五更三点,复出临朝。朝毕,命请唐僧
四众,议寻公主。长老随至,朝上行礼。大圣三人,一同打个问讯。国王欠身道:
“昨所云公主孩儿,敢烦神僧为一寻救。”

长老道:“贫僧前日自东来,行至天晚,见一座给孤布金寺,特进求宿,幸那
寺僧相待。当晚斋罢,步月闲行,行至布金旧园,观看基址,忽闻悲声入耳。询问
其由,本寺一老僧,年已百岁之外,他屏退左右,细细的对我说了一遍,道:‘悲
声者,乃旧年春深时,我正明性月,忽然一阵风生,就有悲怨之声。下榻到园基
上看处,乃是一个女子。询问其故,那女子道:“我是天竺国国王公主。因为夜间
玩月观花,被风刮至于此’”。那老僧多知人礼,即将公主锁在一间僻静房中。惟恐
本寺顽僧污染,只说是妖精被我锁住。公主识得此意,日间胡言乱语,讨些茶饭吃
了;夜深无人处,思量父母悲啼。那老僧也曾来国打听几番,见公主在宫无恙,所
以不敢声言举奏。因见我徒弟有些神通,那老僧千叮万嘱,教贫僧到此查访。不期
他原是蟾宫玉兔为妖,假合真形,变作公主模样。他却又有心要破我元阳。幸亏我
徒弟施威显法,认出真假。今已被太阴星收去。贤公主见在布金寺装风也。”

国王见说此详细,放声大哭。早惊动三宫六院,都来问及前因。无一人不痛哭
者。良久,国王又问:“布金寺离城多远?”三藏道:“只有六十里路。”国王遂传
旨:“着东西二宫守殿,掌朝太师卫国,朕同正宫皇后帅多官,四神僧,去寺取公
主也。”

当时摆驾,一行出朝。你看那行者就跳在空中,把腰一扭,先到了寺里。众僧
慌忙跪接道:“老爷去时,与众步行,今日何从天上下来?”行者笑道:“你那老师
在于何处?快叫他出来,排设香案接驾。天竺国王、皇后、多官与我师父都来了。”
众僧不解其意,即请出那老僧。老僧见了行者,倒身下拜道:“老爷,公主之事如
何?”行者把那假公主抛绣球,欲配唐僧,并赶捉赌斗,与太阴星收去玉兔之言,
备陈了一遍。那老僧又磕头拜谢。行者搀起道:“且莫拜,且莫拜。快安排接驾。”
众僧才知后房里锁得是个女子。一个个惊惊喜喜,便都设了香案,摆列山门之外,
穿了袈裟,撞起钟鼓等候。

不多时,圣驾早到。果然是:
缤纷瑞霭满天香,一座荒山倏被祥。
虹流千载清河海,电绕长春赛禹汤。
草林沾恩添秀色,野花得润有余芳。
古来长者留遗迹,今喜明君降宝堂。
国王到于山门之外,只见那众僧齐齐整整,俯伏接拜,又见孙行者立在中间,国王
道:“神僧何先到此?”行者笑道:“老孙把腰略扭一扭儿,就到了。你们怎么就走
这半日?”随后唐僧等俱到。长老引驾,到于后面房边,那公主还装风胡说。老僧
跪指道:“此房内就是旧年风吹来的公主娘娘。”国王即令开门。随即打开铁锁,开
了门。国王与皇后见了公主,认得形容,不顾秽污,近前一把搂抱道:“我的受苦
的儿啊!你怎么遭这等折磨,在此受罪!”真是父母子女相逢,比他人不同。三人抱
头大哭。哭了一会,叙毕离情,即令取香汤,教公主沐浴更衣,上辇回国。

行者又对国王拱手道:“老孙还有一事奉上。”国王答礼道:“神僧有事吩咐,
朕即从之。”行者道:“他这山,名为百脚山。近来说有蜈蚣成精,黑夜伤人,往来
行旅,甚为不便。我思蜈蚣惟鸡可以降伏,可选绝大雄鸡千只,撒放山中,除此毒
虫。就将此山名改换改换,赐文一道敕封,就当谢此僧存养公主之恩也。”国王甚
喜,领诺。随差官进城取鸡;又改山名为宝华山,仍着工部办料重修,赐与封召,
唤做“敕建宝华山给孤布金寺”,把那老僧封为“报国僧官”,永远世袭,赐俸三十
六石。僧众谢了恩,送驾回朝。公主入宫,各各相见。安排筵宴,与公主释闷贺喜。
后妃母子,复聚首团。国王君臣,亦共喜,饮宴一宵不题。

次早,国王传旨,召丹青图下圣僧四众喜容,供养在华夷楼上。又请公主新妆
重整,出殿谢唐僧四众救苦之恩。谢毕,唐僧辞王西去。那国王那里肯放,大设佳
宴,一连吃了五六日,着实好了呆子,尽力放开肚量受用。国王见他们拜佛心重,
苦留不住,遂取金银二百锭,宝贝各一盘奉谢。师徒们一毫不受。教摆銮驾,请老
师父登辇,差官远送。那后妃并臣民人等俱各叩谢不尽。及至前途,又见众僧叩送,
俱不忍相别。行者见送者不肯回去,无已,捻诀,往巽地上吹口仙气,一阵暗风,
把送的人都迷了眼目,方才得脱身而去。这正是:
沐净恩波归了性,出离金海悟真空。

毕竟不知前路如何,且听下回分解。