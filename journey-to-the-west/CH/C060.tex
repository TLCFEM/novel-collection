\chapter{牛魔王罢战赴华筵~孙行者二调芭蕉扇}

土地说:“大力王即牛魔王也。”行者道:“这山本是牛魔王放的火,假名火焰
山?”土地道:“不是,不是。大圣若肯赦小神之罪,方敢直言。”行者道:“你有
何罪?直说无妨。”土地道:“这火原是大圣放的。”行者怒道:“我在那里,你这等
乱谈!我可是放火之辈?”土地道:“是你也认不得我了。此间原无这座山,因大圣
五百年前,大闹天宫时,被显圣擒了,压赴老君,将大圣安于八卦炉内,煅炼之后
开鼎,被你蹬倒丹炉,落了几个砖来,内有余火,到此处化为火焰山。我本是兜率
宫守炉的道人。当被老君怪我失守,降下此间,就做了火焰山土地也。”猪八戒闻
言,恨道:“怪道你这等打扮,原来是道士变的土地!”

行者半信不信道:“你且说,早寻大力王何故?”土地道:“大力王乃罗刹女丈
夫。他这向撇了罗刹,现在积雷山摩云洞。有个万岁狐王。那狐王死了,遗下一个
女儿,叫做玉面公主。那公主有百万家私,无人掌管;二年前,访着牛魔王神通广
大,情愿倒陪家私,招赘为夫。那牛王弃了罗刹,久不回顾。若大圣寻着牛王,拜
求来此,方借得真扇。一则息火焰,可保师父前进;二来永除火患,可保此地生
灵;三者赦我归天,回缴老君法旨。”行者道:“积雷山坐落何处?到彼有多少程途?”
土地道:“在正南方。此间到彼,有三千余里。”行者闻言,即吩咐沙僧、八戒保护
师父。又教土地,陪伴勿回。随即忽的一声,渺然不见。

那里消半个时辰,早见一座高山凌汉。按落云头,停立巅峰之上观看,真是好
山:

高不高,顶摩碧汉;大不大,根扎黄泉。山前日暖,岭后风寒;山前日暖,有
三冬草木无知;岭后风寒,见九夏冰霜不化。龙潭接涧水长流,虎穴依崖花放早。
水流千派似飞琼,花放一心如布锦。湾环岭上湾环树,石外松。真个是,
高的山、峻的岭,陡的崖、深的涧,香的花,美的果,红的藤、紫的竹,青的松、
翠的柳:八节四时颜不改,千年万古色如龙。
大圣看够多时,步下尖峰,入深山,找寻路径。正自没个消息,忽见松阴下,有一
女子,手折了一枝香兰,袅袅娜娜而来。大圣闪在怪石之旁,定睛观看,那女子怎
生模样:

娇娇倾国色,缓缓步移莲。貌若王嫱,颜如楚女。如花解语,似玉生香。高髻
堆青碧鸦,双睛蘸绿横秋水。湘裙半露弓鞋小,翠袖微舒粉腕长。说甚么暮雨朝
云,真个是朱唇皓齿。锦江滑腻蛾眉秀,赛过文君与薛涛。
那女子渐渐走近石边,大圣躬身施礼,缓缓而言曰:“女菩萨何往?”那女子未曾
观看,听得叫问,却自抬头;忽见大圣的相貌丑陋,老大心惊,欲退难退,欲行难
行,只得战兢兢,勉强答道:“你是何方来者?敢在此间问谁?”大圣沉思道:“我
若说出取经求扇之事,恐这厮与牛王有亲,且只以假亲托意,来请魔王之言而答方
可。”那女子见他不语,变了颜色,怒声喝道:“你是何人,敢来问我!”大圣躬身
陪笑道:“我是翠云山来的,初到贵处,不知路径。敢问菩萨,此间可是积雷山?”
那女子道:“正是。”大圣道:“有个摩云洞,坐落何处?”那女子道:“你寻那洞做
甚?”大圣道:“我是翠云山芭蕉洞铁扇公主央来请牛魔王的。”

那女子一听铁扇公主请牛魔王之言,心中大怒,彻耳根子通红,泼口骂道:“这
贱婢,着实无知!牛王自到我家,未及二载,也不知送了他多少珠翠金银,绫罗缎
匹,年供柴,月供米,自自在在受用,还不识羞,又来请他怎的!”

大圣闻言,情知是玉面公主,故意子掣出铁棒大喝一声道:“你这泼贱,将家
私买住牛王,诚然是陪钱嫁汉!你倒不羞,却敢骂谁!”那女子见了,唬得魄散魂飞,
没好步乱金莲;战兢兢回头便走。这大圣吆吆喝喝,随后相跟。原来穿过松阴,
就是摩云洞口。女子跑进去,扑的把门关了。大圣却收了铁棒,咳咳停步看时,好
所在:

树林森密,崖削。薜萝阴冉冉,兰蕙味馨馨。流泉漱玉穿修竹,巧石知机
带落英。烟霞笼远岫,日月照云屏。龙吟虎啸,鹤唳莺鸣。一片清幽真可爱,琪花
瑶草景常明。不亚天台仙洞,胜如海上蓬瀛。

且不言行者这里观看景致。却说那女子跑得粉汗淋淋,唬得兰心吸吸,径入书
房里面。原来牛魔王正在那里静玩丹书。这女子没好气倒在怀里,抓耳挠腮,放声
大哭。牛王满面陪笑道:“美人,休得烦恼。有甚话说?”那女子跳天索地,口中
骂道:“泼魔害杀我也!”牛王笑道:“你为甚事骂我?”女子道:“我因父母无依,
招你护身养命。江湖中说你是条好汉,你原来是个惧内的庸夫!”牛王闻说,将女
子抱住道:“美人,我有那些不是处,你且慢慢说来,我与你陪礼。”女子道:“适
才我在洞外闲步花阴,折兰采蕙,忽有一个毛脸雷公嘴的和尚,猛地前来施礼,把
我吓了个呆挣。及定性问是何人,他说是铁扇公主央他来请牛魔王的。被我说了两
句,他倒骂了我一场,将一根棍子,赶着我打。若不是走得快些,几乎被他打死!
这不是招你为祸?害杀我也!”牛王闻言,却与他整容陪礼。温存良久,女子方才息
气。魔王却发狠道:“美人在上,不敢相瞒。那芭蕉洞虽是僻静,却清幽自在。我
山妻自幼修持,也是个得道的女仙,却是家门严谨,内无一尺之童,焉得有雷公嘴
的男子央来,这想是那里来的怪妖,或者假绰名声,至此访我。等我出去看看。”

好魔王,拽开步,出了书房,上大厅取了披挂,结束了。拿了一条混铁棍,出
门高叫道:“是谁人在我这里无状?”行者在旁,见他那模样,与五百年前又大不
同。只见:

头上戴一顶水磨银亮熟铁盔,身上贯一副绒穿锦绣黄金甲;足下踏一双卷尖粉
底麂皮靴,腰间束一条攒丝三股狮蛮带。一双眼光如明镜,两道眉艳似红霓。口若
血盆,齿排铜板。吼声响震山神怕,行动威风恶鬼慌。四海有名称混世,西方大力
号魔王。
这大圣整衣上前,深深的唱个大喏道:“长兄,还认得小弟么?”牛王答礼道:“你
是齐天大圣孙悟空么?”大圣道:“正是,正是,一向久别未拜。适才到此问一女
子,方得见兄。丰采果胜常,真可贺也!”牛王喝道:“且休巧舌!我闻你闹了天宫,
被佛祖降压在五行山下,近解脱天灾,保护唐僧西天见佛求经,怎么在号山枯松涧
火云洞把我小儿牛圣婴害了?正在这里恼你,你却怎么又来寻我?”大圣作礼道:“长
兄勿得误怪小弟。当时令郎捉住吾师,要食其肉,小弟近他不得,幸观音菩萨欲救
我师,劝他归正。现今做了善财童子,比兄长还高,享极乐之门堂,受逍遥之永寿,
有何不可,返怪我耶?”牛王骂道:“这个乖嘴的猢狲!害子之情,被你说过;你才
欺我爱妾,打上我门何也?”大圣笑道:“我因拜谒长兄不见,向那女子拜问,不
知就是二嫂嫂;因他骂了我几句,是小弟一时粗卤,惊了嫂嫂。望长兄宽恕宽恕!”
牛王道:“既如此说,我看故旧之情,饶你去罢。”

大圣道:“既蒙宽恩,感谢不尽;但尚有一事奉渎,万望周济周济。”牛王骂道:
“这猢狲不识起倒!饶了你,倒还不走,反来缠我!甚么周济周济!”大圣道:“实不
瞒长兄。小弟因保唐僧西进,路阻火焰山,不能前进。询问土人,知尊嫂罗刹女有
一柄芭蕉扇,欲求一用。昨到旧府,奉拜嫂嫂,嫂嫂坚执不借,是以特求长兄。望
兄长开天地之心,同小弟到大嫂处一行,千万借扇灭火焰,保得唐僧过山,即时
完璧。”

牛王闻言,心如火发。咬响钢牙骂道:“你说你不无礼,你原来是借扇之故,
一定先欺我山妻,山妻想是不肯,故来寻我,且又赶我爱妾!常言道:‘朋友妻,不
可欺;朋友妾,不可灭。’你既欺我妻,又灭我妾,多大无礼?上来吃我一棍!”大
圣道:“哥要说打,弟也不惧。但求宝贝,是我真心。万乞借我使使!”牛王道:“你
若三合敌得我,我着山妻借你;如敌不过,打死你,与我雪恨!”大圣道:“哥说得
是。小弟这一向疏懒,不曾与兄相会,不知这几年武艺比昔日如何,我兄弟们请演
演棍看。”这牛王那容分说,掣混铁棍,劈头就打。这大圣持金箍棒,随手相迎。
两个这场好斗:

金箍棒,混铁棍,变脸不以朋友论。那个说:“正怪你这猢狲害子情!”这个说:
“你令郎已得道休嗔恨!”那个说:“你无知怎敢上我门?”这个说:“我有因特地
来相问。”一个要求扇子保唐僧,一个不借芭蕉忒鄙吝。语去言来失旧情,举家无
义皆生忿。牛王棍起赛蛟龙,大圣棒迎神鬼遁。初时争斗在山前,后来齐驾祥云进。
半空之内显神通,五彩光中施妙运。两条棍响振天关,不见输赢皆傍寸。
这大圣与那牛王斗经百十回合,不分胜负。正在难解难分之际,只听得山峰上有人
叫道:“牛爷爷,我大王多多拜上,幸赐早临,好安座也。”牛王闻说,使混铁棍支
住金箍棒,叫道:“猢狲,你且住了,等我去一个朋友家赴会来者!”

言毕,按下云头,径至洞里。对玉面公主道:“美人,才那雷公嘴的男子乃孙
悟空猢狲,被我一顿棍打走了,再不敢来。你放心耍子。我到一个朋友处吃酒去也。”
他才卸了盔甲,穿一领鸦青剪绒袄子,走出门,跨上“辟水金睛兽”,着小的们看
守门庭,半云半雾,一直向西北方而去。

大圣在高峰上看着,心中暗想道:“这老牛不知又结识了甚么朋友,往那里去
赴会。等老孙跟他走走。”好行者,将身幌一幌,变作一阵清风赶上,随着同走。
不多时,到了一座山中,那牛王寂然不见。大圣聚了原身,入山寻看,那山中有一
面清水深潭,潭边有一座石碣,碣上有六个大字,乃“乱石山碧波潭。”大圣暗想
道:“老牛断然下水去了。水底之精,若不是蛟精,必是龙精、鱼精,或是龟鳖鼋
鼍之精。等老孙也下去看看。”

好大圣,捻着诀,念个咒语,摇身一变,变作一个螃蟹,不大不小的,有三十
六斤重。扑的跳在水中,径沉潭底。忽见一座玲珑剔透的牌楼,楼下拴着那个辟水
金睛兽。进牌楼里面,却就没水。大圣爬进去,仔细看时,只见那壁厢一派音乐之
声,但见:

朱宫贝阙,与世不殊。黄金为屋瓦,白玉作门枢。屏开玳瑁甲,槛砌珊瑚珠。
祥云瑞蔼辉莲座,上接三光下八衢。非是天宫并海藏,果然此处赛蓬壶。高堂设宴
罗宾主,大小官员冠冕珠。忙呼玉女棒牙,催唤仙娥调律吕。长鲸鸣,巨蟹舞,
鳖吹笙,鼍击鼓,骊颔之珠照樽俎。鸟篆之文列翠屏,须之帘挂廊庑。八音迭奏
杂仙韶,宫商响彻遏云霄。青头鲈妓抚瑶瑟,红眼马郎品玉箫。鳜婆顶献香獐脯,
龙女头簪金凤翘。吃的是,天厨八宝珍羞味;饮的是,紫府琼浆熟酝醪。
那上面坐的是牛魔王,左右有三四个蛟精,前面坐着一个老龙精,两边乃龙子、龙
孙、龙婆、龙女。正在那里觥筹交错之际,孙大圣一直走将上去,被老龙看见,即
命:“拿下那个野蟹来!”龙子、龙孙一拥上前,把大圣拿住。大圣忽作人言,只叫:
“饶命!饶命!”老龙道:“你是那里来的野蟹?怎么敢上厅堂,在尊客之前,横行乱
走?快早供来,免汝死罪!”好大圣,假捏虚言,对众供道:

“生自湖中为活,傍崖作窟权居。盖因日久得身舒,官受横行介士。踏草拖泥
落索,从来未习行仪。不知法度冒王威,伏望尊慈恕罪!”
座上众精闻言,都拱身对老龙作礼道:“蟹介士初入瑶宫,不知王礼,望尊公饶他
去罢。”老龙称谢了。众精即教:“放了那厮,且记打,外面伺候。”

大圣应了一声,往外逃命,径至牌楼之下。心中暗想道:“这牛王在此贪杯,
那里等得他散?就是散了,也不肯借扇与我。不如偷了他的金睛兽,变做牛魔王,
去哄那罗刹女,骗他扇子,送我师父过山为妙。”

好大圣,即现本像,将金睛兽解了缰绳,扑一把跨上雕鞍,径直骑出水底。到
于潭外,将身变作牛王模样。打着兽,纵着云,不多时,已至翠云山芭蕉洞口。叫
声:“开门!”那洞门里有两个女童,闻得声音开了门,看见是牛魔王嘴脸,即入报:
“奶奶,爷爷来家了。”那罗刹听言,忙整云鬟,急移莲步,出门迎接。这大圣下
雕鞍,牵进金睛兽;弄大胆,诓骗女佳人。罗刹女肉眼,认他不出,即携手而入。
着丫鬟设座看茶,一家子见是主公,无不敬谨。

须臾间,叙及寒温。“牛王”道:“夫人久阔。”罗刹道:“大王万福。”又云:“大
王宠幸新婚,抛撇奴家,今日是那阵风儿吹你来的?”大圣笑道:“非敢抛撇,只
因玉面公主招后,家事繁冗,朋友多顾,是以稽留在外,却也又治得一个家当了。”
又道:“近闻悟空那厮,保唐僧,将近火焰山界,恐他来问你借扇子。我恨那厮害
子之仇未报,但来时,可差人报我,等我拿他,分尸万段,以雪我夫妻之恨。”罗
刹闻言,滴泪告道:“大王,常言说:‘男儿无妇财无主,女子无夫身无主。’我的
性命,险些儿不着这猢狲害了!”大圣得故子,发怒骂道:“那泼猴几时过去了?”
罗刹道:“还未去。昨日到我这里借扇子,我因他害孩儿之故,披挂了,轮宝剑出
门,就砍那猢狲。他忍着疼,叫我做嫂嫂,说大王曾与他结义。”大圣道:“是,五
百年前曾拜为七兄弟。”罗刹道:“被我骂也不敢回言,砍也不敢动手,后被我一扇
子扇去;不知在那里寻得个定风法儿,今早又在门外叫唤。是我又使扇扇,莫想得
动。急轮剑砍时,他就不让我了。我怕他棒重,就走入洞里,紧关上门。不知他又
从何处,钻在我肚腹之内,险被他害了性命!是我叫他几声叔叔,将扇与他去也。”
大圣又假意捶胸道:“可惜,可惜!夫人错了,怎么就把这宝贝与那猢狲?恼杀我也!”

罗刹笑道:“大王息怒。与他的是假扇,但哄他去了。”大圣问:“真扇在于何
处?”罗刹道:“放心,放心!我收着哩。”叫丫鬟整酒接风贺喜。遂擎杯奉上道:“大
王,燕尔新婚,千万莫忘结发,且吃一杯乡中之水。”大圣不敢不接,只得笑吟吟,
举觞在手道:“夫人先饮。我因图治外产,久别夫人,早晚蒙护守家门,权为酬谢。”
罗刹复接杯斟起,递与大王道:“自古道:‘妻者,齐也。’夫乃养身之父,讲甚么
谢。”两人谦谦讲讲,方才坐下巡酒。大圣不敢破荤,只吃几个果子,与他言言语
语。

酒至数巡,罗刹觉有半酣,色情微动,就和孙大圣挨挨擦擦,搭搭拈拈;携着
手,俏语温存;并着肩,低声俯就。将一杯酒,你喝一口,我喝一口,却又哺果。
大圣假意虚情,相陪相笑;没奈何,也与他相倚相偎。果然是:

钓诗钩,扫愁帚,破除万事无过酒。男儿立节放襟怀,女子忘情开笑口。面赤
似夭桃,身摇如嫩柳。絮絮叨叨话语多,捻捻掐掐风情有。时见惊云鬟,又见轮尖
手。几番常把脚儿跷,数次每将衣袖抖。粉项自然低,蛮腰渐觉扭。合欢言语不曾
丢,酥胸半露松金钮。醉来真个玉山颓,饧眼摩娑几弄丑。

大圣见他这等酣然,暗自留心,挑斗道:“夫人,真扇子你收在那里?早晚仔细。
但恐孙行者变化多端,却又来骗去。”罗刹笑嘻嘻的,口中吐出,只有一个杏叶儿
大小,递与大圣道:“这个不是宝贝?”大圣接在手中,却又不信,暗想着:“这些
些儿,怎生得火灭?怕又是假的。”罗刹见他看着宝贝沉思,忍不住上前,将粉面
在行者脸上,叫道:“亲亲,你收了宝贝吃酒罢。只管出神想甚么哩?”大圣就
趁脚儿跷,问他一句道:“这般小小之物,如何得八百里火焰?”罗刹酒陶真性,
无忌惮,就说出方法道:“大王,与你别了二载,你想是昼夜贪欢,被那玉面公主
弄伤了神思;怎么自家的宝贝事情,也都忘了?只将左手大指头捻着那柄儿上第七
缕红丝,念一声‘咽嘘呵吸嘻吹呼’,即长一丈二尺长短。这宝贝变化无穷!那怕他
八万里火焰,可一而消也。”

大圣闻言,切切记在心上。却把扇儿也噙在口里,把脸抹一抹,现了本象。厉
声高叫道:“罗刹女!你看看我可是你亲老公!就把我缠了这许多丑勾当,不羞,不
羞!”那女子一见是孙行者,慌得推倒桌席,跌落尘埃,羞愧无比,只叫:“气杀我
也!气杀我也!”

这大圣,不管他死活,脱手,拽大步,径出了芭蕉洞。正是无心贪美色,得
意笑颜回。将身一纵,踏祥云,跳上高山,将扇子吐出来,演演方法。将左手大指
头捻着那柄上第七缕红丝,念了一声“咽嘘呵吸嘻吹呼”,果然长了有一丈二尺长
短。拿在手中,仔细看了又看,比前番假的果是不同,只见祥光幌幌,瑞气纷纷,
上有三十六缕红丝,穿经度络,表里相联。原来行者只讨了个长的方法,不曾讨他
个小的口诀,左右只是那等长短。没奈何,只得搴在肩上,找旧路而回,不题。

却说那牛魔王在碧波潭底与众精散了筵席,出得门来,不见了辟水金睛兽。老
龙王聚众精问道:“是谁偷放牛爷的金睛兽也?”众精跪下道:“没人敢偷。我等俱
在筵前供酒捧盘,供唱奏乐,更无一人在前。”老龙道:“家乐儿断乎不敢,可曾有
甚生人进来?”龙子、龙孙道:“适才安座之时,有个蟹精到此。那个便是生人。”
牛王闻说,顿然省悟道:“不消讲了!早间贤友着人邀我时,有个孙悟空保唐僧取经,
路遇火焰山难过,曾问我求借芭蕉扇。我不曾与他,他和我赌斗一场,未分胜负,
我却丢了他,径赴盛会。那猴子千般伶俐,万样机关,断乎是那厮变作蟹精,来此
打探消息,偷了我兽,去山妻处骗了那一把芭蕉扇儿也!”众精见说,一个个胆战
心惊,问道:“可是那大闹天宫的孙悟空么?”牛王道:“正是。列公若在西天路上,
有不是处,切要躲避他些儿。”老龙道:“似这般说,大王的骏骑,却如之何!”牛
王笑道:“不妨,不妨。列公各散,等我赶他去来!”

遂而分开水路,跳出潭底,驾黄云,径至翠云山芭蕉洞。只听得罗刹女跌脚捶
胸,大呼小叫。推开门,又见辟水金睛兽拴在下边,牛王高叫:“夫人,孙悟空那
厢去了?”众女童看见牛魔,一齐跪下道:“爷爷来了!”罗刹女扯住牛王,磕头撞
脑,口里骂道:“泼老天杀的!怎样这般不谨慎,着那猢狲偷了金睛兽,变作你的模
样,到此骗我!”牛王切齿道:“猢狲那厢去了?”罗刹捶着胸膛骂道:“那泼猴赚
了我的宝贝,现出原身走了!气杀我也!”牛王道:“夫人保重,勿得心焦。等我赶
上猢狲,夺了宝贝,剥了他皮,锉碎他骨,摆出他的心肝,与你出气!”叫:“拿兵
器来!”女童道:“爷爷的兵器,不在这里。”牛王道:“拿你奶奶的兵器来罢!”侍
婢将两把青锋宝剑捧出。牛王脱了那赴宴的鸦青绒袄,束一束贴身的小衣,双手绰
剑,走出芭蕉洞,径奔火焰山上赶来。正是那:

忘恩汉骗了痴心妇,烈性魔来近木叉人。

毕竟不知此去吉凶如何,且听下回分解。