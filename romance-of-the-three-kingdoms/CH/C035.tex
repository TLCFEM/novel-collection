\chapter{玄德南漳逢隐沦~单福新野遇英主}

却说蔡瑁方欲回城,赵云引军赶出城来。原来赵云正饮酒间,忽见人马动,急入内观
之,席上不见了玄德。云大惊,出投馆舍,听得人说:“蔡瑁引军望西赶去了。”云火急绰
枪上马,引着原带来三百军,奔出西门,正迎着蔡瑁,急问曰:“吾主何在?”瑁曰:“使
君逃席而去,不知何往。”赵云是谨细之人,不肯造次,即策马前行。遥望大溪,别无去
路,乃复回马,喝问蔡瑁曰:“汝请吾主赴宴,何故引着军马追来?”瑁曰:“九郡四十二
州县官僚俱在此,吾为上将,岂可不防护?”云曰:“汝逼吾主何去了?”瑁曰:“闻使君
匹马出西门,到此却又不见。”云惊疑不定,直来溪边看时,只见隔岸一带水迹。云暗忖
曰:“难道连马跳过了溪去?”令三百军四散观望,并不见踪迹。云再回马时,蔡瑁已入城
去了。云乃拿守门军士追问,皆说:“刘使君飞马出西门而去。”云再欲入城?又恐有埋
伏,遂急引军归新野。

却说玄德跃马过溪,似醉如痴,想:“此阔涧一跃而过,岂非天意!”迤逦望南漳策马
而行,日将沉西。正行之间,见一牧童跨于牛背上,口吹短笛而来。玄德叹曰:“吾不如
也!”遂立马观之。牧童亦停牛罢笛,熟视玄德,曰:“将军莫非破黄巾刘玄德否?”玄德
惊问曰:“汝乃村僻小童,何以知吾姓字!”牧童曰:“我本不知,因常侍师父,有客到
日,多曾说有一刘玄德,身长七尺五寸,垂手过膝,目能自顾其耳,乃当世之英雌,今观将
军如此模样,想必是也。”玄德曰:“汝师何人也?”牧童曰:“吾师覆姓司马,名徽,字
德操,颍川人也。道号水镜先生。”玄德曰:“汝师与谁为友?”小童曰:“与襄阳庞德
公、庞统为友。”玄德曰:“庞德公乃庞统何人?”童子曰:“叔侄也。庞德公字山民,长
俺师父十岁;庞统字士元,少俺师父五岁。一日,我师父在树上采桑,适庞统来相访,坐于
树下,共相议论,终日不倦。吾师甚爱庞统,呼之为弟。”玄德曰:“汝师今居何处?”牧
童遥指曰:“前面林中,便是庄院。”玄德曰:“吾正是刘玄德。汝可引我去拜见你师
父。”童子便引玄德,行二里余,到庄前下马,入至中门,忽闻琴声甚美。玄德教童子且休
通报,侧耳听之。琴声忽住而不弹。一人笑而出曰:“琴韵清幽,音中忽起高抗之调。必有
英雄窃听。”童子指谓玄德曰:“此即吾师水镜先生也。”玄德视其人,松形鹤骨,器宇不
凡。慌忙进前施礼,衣襟尚湿。水镜曰:“公今日幸免大难!”玄德惊讶不已。小童曰:
“此刘玄德也。”水镜请入草堂,分宾主坐定。玄德见架上满堆书卷,窗外盛栽松竹,横琴
于石床之上,清气飘然。水镜问曰:“明公何来?”玄德曰:“偶尔经由此地,因小童相
指,得拜尊颜,不胜万幸!”水镜笑曰:“公不必隐讳。公今必逃难至此。”玄德遂以襄阳
一事告之。水镜曰:“吾观公气色,已知之矣。”因问玄德曰:“吾久闻明公大名,何故至
今犹落魄不偶耶?”玄德曰:“命途多蹇,所以至此。”水镜曰:“不然。盖因将军左右不
得其人耳。”玄德曰:“备虽不才,文有孙乾、糜竺、简雍之辈,武有关、张、赵云之流,
竭忠辅相,颇赖其力。”水镜曰:“关、张、赵云,皆万人敌,惜无善用之之人。若孙乾、
糜竺辈,乃白面书生,非经纶济世之才也。”玄德曰:“备亦尝侧身以求山谷之遗贤,奈未
遇其人何!”水镜曰:“岂不闻孔子云十室之邑必有忠信,何谓无人?”玄德曰:“备愚昧
不识,愿赐指教。”水镜曰:“公闻荆襄诸郡小儿谣言乎?其谣曰:八九年间始欲衰,至十
三年无孑遗。到头天命有所归,泥中蟠龙向天飞。此谣始于建安初:建安八年,刘景升丧却
前妻,便生家乱,此所谓始欲衰也;无孑遗者,不久则景升将逝,文武零落无孑遗矣;天命
有归,龙向天飞,盖应在将军也。”玄德闻言惊谢曰:“备安敢当此!”水镜曰:“今天下
之奇才,尽在于此,公当往求之。”玄德急问曰:“奇才安在?果系何人?”水镜曰:“伏
龙、凤雏,两人得一,可安天下。”玄德曰:“伏龙、凤雏何人也?”水镜抚掌大笑曰:
“好!好!”玄德再问时,水镜曰:“天色已晚,将军可于此暂宿一宵,明日当言之。”即
命小童具饮馔相待,马牵入后院喂养。玄德饮膳毕,即宿于草堂之侧。玄德因思水镜之言,
寝不成寐。约至更深,忽听一人叩门而入,水镜曰:“元直何来?”玄德起床密听之,闻其
人答曰:“久闻刘景升善善恶恶,特往谒之。及至相见,徒有虚名,盖善善而不能用,恶恶
而不能去者也。故遗书别之,而来至此。”水镜曰:“公怀王佐之才,宜择人而事,奈何轻
身往见景升乎?且英雄豪杰,只在眼前,公自不识耳。”其人曰:“先生之言是也。”玄德
闻之大喜,暗忖此人必是伏龙、凤雏,即欲出见,又恐造次。候至天晓,玄德求见水镜,问
曰:“昨夜来者是谁?”水镜曰:“此吾友也。”玄德求与相见。水镜曰:“此人欲往投明
主,已到他处去了。”玄德请问其姓名。水镜笑曰:“好!好!”玄德再问:“伏龙、凤
雏,果系何人?”水镜亦只笑曰:“好!好!”玄德拜请水镜出山相助,同扶汉室。水镜
曰:“山野闲散之人,不堪世用。自有胜吾十倍者来助公,公宜访之。”正谈论间,忽闻庄
外人喊马嘶,小童来报:“有一将军,引数百人到庄来也。”玄德大惊,急出视之,乃赵云
也。玄德大喜。云下马入见曰:“某夜来回县,寻不见主公,连夜跟问到此。主公可作速回
县。只恐有人来县中厮杀。”玄德辞了水镜,与赵云上马,投新野来。行不数里,一彪人马
来到,视之,乃云长、翼德也。相见大喜。玄德诉说跃马檀溪之事,共相嗟讶。到县中,与
孙乾等商议。乾曰:“可先致书于景升,诉告此事。”玄德从其言,即令孙乾赍书至荆州。
刘表唤入问曰:“吾请玄德襄阳赴会,缘何逃席而去?”孙乾呈上书札,具言蔡瑁设谋相
害,赖跃马檀溪得脱。表大怒,急唤蔡瑁责骂曰:“汝焉敢害吾弟!”命推出斩之。蔡夫人
出,哭求免死,表怒犹未息。孙乾告曰:“若杀蔡瑁,恐皇叔不能安居于此矣。”表乃责而
释之,使长子刘琦同孙乾至玄德处请罪。

琦奉命赴新野,玄德接着,设宴相待。酒酣,琦忽然堕泪。玄德问其故。琦曰:“继母
蔡氏,常怀谋害之心;侄无计免祸,幸叔父指教。”玄德劝以小心尽孝,自然无祸。次日,
琦泣别。玄德乘马送琦出郭,因指马谓琦曰:“若非此马,吾已为泉下之人矣。”琦曰:
“此非马之力,乃叔父之洪福也。”说罢。相别。刘琦涕泣而去。

玄德回马入城,忽见市上一人,葛巾布袍,皂绦乌履,长歌而来。歌曰:“天地反覆
兮,火欲殂;大厦将崩兮,一木难扶。山谷有贤兮,欲投明主;明主求贤兮,却不知吾。”
玄德闻歌,暗思:“此人莫非水镜所言伏龙、凤雏乎?”遂下马相见,邀入县衙。问其姓
名,答曰:“某乃颍上人也,姓单,名福。久闻使君纳士招贤,欲来投托,未敢辄造;故行
歌于市,以动尊听耳。”玄德大喜,待为上宾。单福曰:“适使君所乘之马,再乞一观。”
玄德命去鞍牵于堂下。单福曰:“此非的卢马乎?虽是千里马,却只妨主,不可乘也。”玄
德曰:“已应之矣。”遂具言跃檀溪之事。福曰:“此乃救主,非妨主也;终必妨一主。某
有一法可禳。玄德曰:“愿闻禳法。”福曰:“公意中有仇怨之人,可将此马赐之;待妨过
了此人,然后乘之,自然无事。”玄德闻言变色曰:“公初至此,不教吾以正道,便教作利
己妨人之事,备不敢闻教。”福笑谢曰:“向闻使君仁德,未敢便信,故以此言相试耳。”
玄德亦改容起谢曰:“备安能有仁德及人,惟先生教之。”福曰:“吾自颍上来此,闻新野
之人歌曰‘新野牧,刘皇叔;自到此,民丰足。’可见使君之仁德及人也。”玄德乃拜单福
为军师,调练本部人马。

却说曹操自冀州回许都,常有取荆州之意,特差曹仁、李典并降将吕旷、吕翔等领兵三
万,屯樊城,虎视荆襄,就探看虚实。时吕旷、吕翔禀曹仁曰:“今刘备屯兵新野,招军买
马,积草储粮,其志不小,不可不早图之。吾二人自降丞相之后,未有寸功,愿请精兵五
千,取刘备之头,以献丞相。”曹仁大喜,与二吕兵五千,前往新野厮杀。

探马飞报玄德。玄德请单福商议。福曰:“既有敌兵,不可令其入境。可使关公引一军
从左而出,以敌来军中路;张飞引一军从右而出,以敌来军后路;公自引赵云出兵前路相
迎:敌可破矣。”玄德从其言,即差关、张二人去讫;然后与单福、赵云等,共引二千人马
出关相迎。

行不数里,只见山后尘头大起,吕旷、吕翔引军来到。两边各射住阵角。玄德出马于旗
门下,大呼曰:“来者何人,敢犯吾境?”吕旷出马曰:“吾乃大将吕旷也。奉丞相命,特
来擒汝!”玄德大怒,使赵云出马。二将交战,不数合,赵云一枪刺吕旷于马下。玄德麾军
掩杀,吕翔抵敌不住,引军便走。正行间,路傍一军突出,为首大将,乃关云长也;冲杀一
阵,吕翔折兵大半,夺路走脱。行不到十里,又一军拦住去路,为首大将,挺矛大叫:“张
翼德在此!”直取吕翔。翔措手不及,被张飞一矛刺中,翻身落马而死。余众四散奔走。玄
德合军追赶,大半多被擒获。玄德班师回县,重待单富,稿赏三军。

却说败军回见曹仁,报说:“二吕被杀,军士多被活捉。”曹仁大惊,与李典商议。典
曰:“二将欺敌而亡,今只宜按兵不动,申报丞相,起大兵来征剿,乃为上策。”仁曰:
“不然。今二将阵亡,死折许多军马,此仇不可不急报。量新野弹丸之地,何劳丞相大
军?”典曰:“刘备人杰也,不可轻视。”仁曰:“公何怯也!”典曰:“兵法云知彼知
己,百战百胜。某非怯战,但恐不能必胜耳。”仁怒曰:“公怀二心耶?吾必欲生擒刘
备!”典曰:“将军若去,某守樊城。”仁曰:“汝若不同去,真怀二心矣!”典不得已,
只得与曹仁点起二万五千军马,渡河投新野而来。正是:偏裨既有舆尸辱,主将重兴雪耻
兵。未知胜负何如,且听下文分解。