\chapter{兄逼弟曹植赋诗~侄陷叔刘封伏法}

却说曹丕闻曹彰提兵而来,惊问众官;一人挺身而出,愿往折服之。众视其人,乃谏议大夫贾逵也。曹丕大喜,即命贾逵前往。逵领命出城,迎见曹彰。彰问曰:“先王玺绶安在?”逵正色而言曰:“家有长子,国有储君。先王玺绶,非君侯之所宜问也。”彰默然无语,乃与贾逵同入城。至宫门前,逵问曰:“君侯此来,欲奔丧耶?欲争位耶?”彰曰:“吾来奔丧,别无异心。”逵曰:“既无异心,何故带兵入城?”彰即时叱退左右将士,只身入内,拜见曹丕。兄弟二人,相抱大哭。曹彰将本部军马尽交与曹丕。丕令彰回鄢陵自守,彰拜辞而去。

于是曹丕安居王位,改建安二十五年为延康元年;封贾诩为太尉,华歆为相国,王朗为御史大夫;大小官僚,尽皆升赏。谥曹操曰武王,葬于邺郡高陵,令于禁董治陵事。禁奉命到彼,只见陵屋中白粉壁上,图画关云长水淹七军擒获于禁之事:画云长俨然上坐,庞德愤怒不屈,于禁拜伏于地,哀求乞命之状。原来曹丕以于禁兵败被擒,不能死节,既降敌而复归,心鄙其为人,故先令人图画陵屋粉壁,故意使之往见以愧之。当下于禁见此画像,又羞又恼,气愤成病,不久而死。后人有诗叹曰:“三十年来说旧交,可怜临难不忠曹。知人未向心中识,画虎今从骨里描。”

却说华歆奏曹丕曰:“鄢陵侯已交割军马,赴本国去了;临淄侯植、萧怀侯熊,二人竟不来奔丧,理当问罪,丕从之,即分遣二使往二处问罪。不一日,萧怀使者回报:“萧怀侯曹熊惧罪,自缢身死。”丕令厚葬之,追赠萧怀王。又过了一日,临淄使者回报,说:“临淄侯日与丁仪、丁廙兄弟二人酣饮,悖慢无礼,闻使命至,临淄侯端坐不动;丁仪骂曰:昔者先王本欲立吾主为世子,被谗臣所阻;今王丧未远,便问罪于骨肉,何也?丁廙又曰:据吾主聪明冠世,自当承嗣大位,今反不得立。汝那庙堂之臣,何不识人才若此!临淄侯因怒,叱武士将臣乱棒打出。”

丕闻之,大怒,即令许褚领虎卫军三千,火速至临淄擒曹植等一千人来。褚奉命,引军至临淄城。守将拦阻,褚立斩之,直入城中,无一人敢当锋锐,径到府堂。只见曹植与丁仪、丁廙等尽皆醉倒。褚皆缚之,载于车上,并将府下大小属官,尽行拿解邺郡,听候曹丕发落。丕下令,先将丁仪、丁廙等尽行诛戳。丁仪字正礼,丁廙字敬礼,沛郡人,乃一时文士;及其被杀,人多惜之。

却说曹丕之母卞氏,听得曹熊缢死,心甚悲伤;忽又闻曹植被擒,其党丁仪等已杀,大惊。急出殿,召曹丕相见。丕见母出殿,慌来拜谒。卞氏哭谓丕曰:“汝弟植平生嗜酒疏狂,盖因自恃胸中之才,故尔放纵。汝可念同胞之情,存其性命。吾至九泉亦瞑目也。”丕曰:“儿亦深爱其才,安肯害他?今正欲戒其性耳。母亲勿忧。”

卞氏洒泪而入,丕出偏殿,召曹植入见。华歆问曰:“适来莫非太后劝殿下勿杀子建乎?”丕曰:“然。”歆曰:“子建怀才抱智,终非池中物;若不早除,必为后患。”丕曰:“母命不可违。”歆曰:“人皆言子建出口成章,臣未深信。主上可召入,以才试之。若不能,即杀之;若果能,则贬之,以绝天下文人之口。”丕从之。须臾,曹植入见,惶恐伏拜请罪。丕曰:“吾与汝情虽兄弟,义属君臣,汝安敢恃才蔑礼?昔先君在日,汝常以文章夸示于人,吾深疑汝必用他人代笔。吾今限汝行七步吟诗一首。若果能,则免一死;若不能,则从重治罪,决不姑恕!”植曰:“愿乞题目。”时殿上悬一水墨画,画着两只牛,斗于土墙之下,一牛坠井而亡。丕指画曰:“即以此画为题。诗中不许犯着二牛斗墙下,一牛坠井死字样。”植行七步,其诗已成。诗曰:“两肉齐道行,头上带凹骨。相遇块山下,郯起相搪突。二敌不俱刚,一肉卧土窟。非是力不如,盛气不泄毕。”曹丕及群臣皆惊。丕又曰:“七步成章,吾犹以为迟。汝能应声而作诗一首否?”植曰:“愿即命题。”丕曰:“吾与汝乃兄弟也。以此为题。亦不许犯着‘兄弟’字样。”植略不思索,即口占一首曰:“煮豆燃豆萁,豆在釜中泣,本是同根生,相煎何太急!”曹丕闻之,潸然泪下。其母卞氏,从殿后出曰:“兄何逼弟之甚耶?”丕慌忙离坐告曰:“国法不可废耳。”于是贬曹植为安乡侯。植拜辞上马而去。

曹丕自继位之后,法令一新,威逼汉帝,甚于其父。早有细作报入成都。汉中王闻之,大惊,即与文武商议曰:“曹操已死,曹丕继位,威逼天子,更甚于操。东吴孙权,拱手称臣。孤欲先伐东吴,以报云长之仇;次讨中原,以除乱贼。”言未毕,廖化出班,哭拜于地曰:“关公父子遇害,实刘封、孟达之罪。乞诛此二贼。”玄德便欲遣人擒之。孔明谏曰:“不可。且宜缓图之,急则生变矣。可升此二人为郡守,分调开去,然后可擒。”玄德从之,遂遣使升刘封去守绵竹。

原来彭羕与孟达甚厚,听知此事,急回家作书,遣心腹人驰报孟达。使者方出南门外,被马超巡视军捉获,解见马超。超审知此事,即往见彭羕。羕接入,置酒相待。酒至数巡,超以言挑之曰:“昔汉中王待公甚厚,今何渐薄也?”羕因酒醉,恨骂曰:“老革荒悖,吾必有以报之!”超又探曰:“某亦怀怨心久矣。”羕曰:“公起本部军,结连孟达为外合,某领川兵为内应,大事可图也。”超曰:“先生之言甚当。来日再议。”

超辞了彭羕,即将人与书解见汉中王,细言其事。玄德大怒,即令擒彭羕下狱,拷问其情。羕在狱中,悔之无及。玄德问孔明曰:“彭羕有谋反之意,当何以治之?”孔明曰:“羕虽狂士,然留之久必生祸。”于是玄德赐彭羕死于狱。

羕既死,有人报知孟达。达大惊,举止失措。忽使命至,调刘封回守绵竹去讫。孟达慌请上庸、房陵都尉申耽、申仪弟兄二人商议曰:“我与法孝直同有功于汉中王;今孝直已死,而汉中王忘我前功,乃欲见害,为之奈何?“耽曰:“某有一计,使汉中王不能加害于公。”达大喜,急问何计。耽曰:“吾弟兄欲投魏久矣,公可作一表,辞了汉中王,投魏王曹丕,丕必重用。吾二人亦随后来降也。”达猛然省悟,即写表一通,付与来使;当晚引五十余骑投魏去了。

使命持表回成都,奏汉中王,言孟达投魏之事。先主大怒。览其表曰:“臣达伏惟殿下将建伊、吕之业,追桓、文之功,大事草创,假势吴、楚,是以有为之士,望风归顺。臣委质以来,愆戾山积;臣犹自知,况于君乎?今王朝英俊鳞集,臣内无辅佐之器,外无将领之才,列次功臣,诚足自愧!臣闻范蠡识微,浮于五湖;舅犯谢罪,逡巡河上。夫际会之间,请命乞身,何哉?欲洁去就之分也。况臣卑鄙,无元功巨勋,自系于时,窃慕前贤,早思远耻。昔申生至孝,见疑于亲;子胥至忠,见诛于君;蒙恬拓境而被大刑,乐毅破齐而遭谗佞。臣每读其书,未尝不感慨流涕;而亲当其事,益用伤悼!迩者,荆州覆败,大臣失节,百无一还;惟臣寻事,自致房陵、上庸,而复乞身,自放于外。伏想殿下圣恩感悟,愍臣之心,悼臣之举。臣诚小人,不能始终。知而为之,敢谓非罪?臣每闻交绝无恶声,去臣无怨辞,臣过奉教于君子,愿君王勉之,臣不胜惶恐之至!”玄德看毕,大怒曰:“匹夫叛吾,安敢以文辞相戏耶!”即欲起兵擒之。孔明曰:“可就遣刘封进兵,令二虎相并;刘封或有功,或败绩,必归成都,就而除之,可绝两害。玄德从之,遂遣使到绵竹,传谕刘封。封受命,率兵来擒孟达。却说曹丕正聚文武议事,忽近臣奏曰:“蜀将孟达来降。”丕召入问曰:“汝此来,莫非诈降乎?”达曰:“臣为不救关公之危,汉中王欲杀臣,因此惧罪来降,别无他意。”!曹丕尚未准信,忽报刘封引五万兵来取襄阳,单搦孟达厮杀。丕曰:“汝既是真心,便可去襄阳取刘封首级来,孤方准信。”达曰:“臣以利害说之,不必动兵,令刘封亦来降也。”丕大喜,遂加孟达为散骑常侍、建武将军、平阳亭侯,领新城太守,去守襄阳、樊城。原来夏侯尚、徐晃已先在襄阳,正将收取上庸诸部。孟达到了襄阳,与二将礼毕,探得刘封离城五十里下寨。达即修书一封,使人赍赴蜀寨招降刘封。刘封览书大怒曰:“此贼误吾叔侄之义,又间吾父子之亲,使吾为不忠不孝之人也!”遂扯碎来书,斩其使,次日,引军前来搦战。

孟达知刘封扯书斩使,勃然大怒,亦领兵出迎。两阵对圆,封立马于门旗下。以刀指骂曰:“背国反贼,安敢乱言!”孟达曰:“汝死已临头上,还自执迷不省!”封大怒,拍马轮刀,直奔孟达。战不三合,达败走,封乘虚追杀二十余里,一声喊起,伏兵尽出,左边夏侯尚杀来,右边徐晃杀来,孟达回身复战。三军夹攻,刘封大败而走,连夜奔回上庸,背后魏兵赶来。刘封到城下叫门,城上乱箭射下。申耽在敌楼上叫曰:“吾已降了魏也!”封大怒,欲要攻城,背后追军将至,封立脚不住,只得望房陵而奔,见城上已尽插魏旗。申仪在敌楼上将旗一飐,城后一彪军出,旗上大书“右将军徐晃”。封抵敌不住,急望西川而走。晃乘势追杀。刘封部下只剩得百余骑。到了成都,入见汉中王,哭拜于地,细奏前事。玄德怒曰:“辱子有何面目复来见吾!”封曰:“叔父之难,非儿不救,因孟达谏阻故耳。”玄德转怒曰:“汝须食人食、穿人衣,非土木偶人!安可听谗贼所阻!”命左右推出斩之。汉中王既斩刘封,后闻孟达招之,毁书斩使之事,心中颇悔;又哀痛关公,以致染病。因此按兵不动。

且说魏王曹丕,自即王位,将文武官僚,尽皆升赏;遂统甲兵三十万,南巡沛国谯县,大飨先茔。乡中父老,扬尘遮道,奉觞进酒,效汉高祖还沛之事。人报大将军夏侯惇病危,丕即还邺郡。时惇已卒,不为挂孝,以厚礼殉葬。

是岁八月间,报称石邑县凤凰来仪,临淄城麒麟出现,黄龙现于邺郡。于是中郎将李伏、太史丞许芝商议:种种瑞徵,乃魏当代汉之兆,可安排受禅之礼,令汉帝将天下让于魏王。遂同华歆、王朗、辛毗、贾诩、刘廙、刘晔、陈矫、陈群、桓阶等一班文武官僚,四十余人,直入内殿,来奏汉献帝,请禅位于魏王曹丕。正是:魏家社稷今将建,汉代江山忽已移。未知献帝如何回答,且看下文分解。