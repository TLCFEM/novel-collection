\chapter{讨魏国武侯再上表~破曹兵姜维诈献书}

却说蜀汉建兴六年秋九月,魏都督曹休被东吴陆逊大破于石亭,车仗马匹,军资器械,
并皆罄尽,休惶恐之甚,气忧成病,到洛阳,疽发背而死。魏主曹睿敕令厚葬。司马懿引兵
还、众将接入问曰:“曹都督兵败,即元帅之干系,何故急回耶?”懿曰:“吾料诸葛亮知
吾兵败,必乘虚来取长安。倘陇西紧急,何人救之?吾故回耳。”众皆以为惧怯,哂笑而
退。

却说东吴遣使致书蜀中,请兵伐魏,并言大破曹休之事:一者显自己威风,二者通和会
之好。后主大喜,令人持书至汉中,报知孔明。时孔明兵强马壮,粮草丰足,所用之物,一
切完备,正要出师。听知此信,即设宴大会诸将,计议出师。忽一阵大风,自东北角上而
起,把庭前松树吹折。众皆大惊。孔明就占一课,曰:“此风主损一大将!”诸将未信。正
饮酒间,忽报镇南将军赵云长子赵统、次子赵广,来见丞相。孔明大惊,掷杯于地曰:“子
龙休矣!”二子入见,拜哭曰:“某父昨夜三更病重而死。”孔明跌足而哭曰:“子龙身
故,国家损一栋梁,吾去一臂也!”众将无不挥涕。孔明令二子入成都面君报丧。后主闻云
死,放声大哭曰“朕昔年幼,非子龙则死于乱军之中矣!”即下诏追赠大将军,谥封顺平
侯,敕葬于成都锦屏山之东;建立庙堂,四时享祭。后人有诗曰:“常山有虎将,智勇匹关
张。汉水功勋在,当阳姓字彰。两番扶幼主,一念答先皇。青史书忠烈,应流百世芳。”

却说后主思念赵云昔日之功,祭葬甚厚;封赵统为虎贲中郎,赵广为牙门将,就令守
坟。二人辞谢而去。忽近臣奏曰:“诸葛丞相将军马分拨已定,即日将出师伐魏。”后主问
在朝诸臣,诸臣多言未可轻动。后主疑虑未决。忽奏丞相令杨仪赍出师表至。后主宜入,仪
呈上表章。后主就御案上拆视,其表曰:“先帝虑汉贼不两立,王业不偏安,故托臣以讨贼
也。以先帝之明,量臣之才,故知臣伐贼,才弱敌强也。然不伐贼,王业亦亡。惟坐而待
亡,孰与伐之?是故托臣而弗疑也。臣受命之日,寝不安席,食不甘味;思惟北征,宜先入
南:故五月渡沪,深入不毛,并日而食。——臣非不自惜也,顾王业不可偏安于蜀都,故冒
危难以奉先帝之遗意。而议者谓为非计。今贼适疲于西,又务于东,兵法“乘劳”:此进趋
之时也。谨陈其事如左:高帝明并日月,谋臣渊深,然涉险被创,危然后安;今陛下未及高
帝,谋臣不如良、平,而欲以长策取胜,坐定天下,此臣之未解一也。刘繇、王朗,各据州
郡,论安言计,动引圣人,群疑满腹,众难塞胸;今岁不战,明年不征,使孙权坐大,遂并
江东,此臣之未解二也。曹操智计,殊绝于人,其用兵也,仿佛孙、吴,然困于南阳,险于
乌巢,危于祁连,逼于黎阳,几败北山,殆死潼关,然后伪定一时耳;况臣才弱,而欲以不
危而定之,此臣之未解三也。曹操五攻昌霸不下,四越巢湖不成,任用李服而李服图之,委
任夏侯而夏侯败亡,先帝每称操为能,犹有此失;况臣驽下,何能必胜,此臣之未解四也。
自臣到汉中,中间期年耳,然丧赵云、阳群、马玉、阎芝、丁立、白寿、刘郃、邓铜等,及
曲长屯将七十余人,突将无前,賨、叟、青羌,散骑武骑一千余人,此皆数十年之内,所纠
合四方之精锐,非一州之所有;若复数年,则损三分之二也。——当何以图敌,此臣之未解
五也。今民穷兵疲,而事不可息;事不可息,则住与行,劳费正等;而不及今图之,欲以一
州之地,与贼持久,此臣之未解六也。夫难平者,事也。昔先帝败军于楚,当此之时,曹操
拊手,谓天下已定。——然后先帝东连吴、越,西取巴、蜀,举兵北征,夏侯授首,此操之
失计,而汉事将成也。——然后吴更违盟,关羽毁败,秭归蹉跌,曹丕称帝,凡事如是,难
可逆见。臣鞠躬尽瘁,死而后已;至于成败利钝,非臣之明所能逆睹也。”后主览表甚喜,
即敕令孔明出师。孔明受命,起三十万精兵,令魏延总督前部先锋,径奔陈仓道口而来。早
有细作报入洛阳。司马懿奏知魏主,大会文武商议。大将军曹真出班奏曰:“臣昨守陇西,
功微罪大,不胜惶恐。今乞引大军往擒诸葛亮。臣近得一员大将,使六十斤大刀,骑千里征
马宛马,开两石铁胎弓,暗藏三个流星锤,百发百中,有万夫不当之勇,乃陇西狄道人,姓
王,名双,字子全。臣保此人为先锋。”睿大喜,便召王双上殿。视之,身长九尺,面黑睛
黄,熊腰虎背。睿笑曰:“朕得此大将,有何虑哉!”遂赐锦袍金甲,封为虎威将军、前部
大先锋。曹真为大都督。真谢恩出朝,遂引十五万精兵,会合郭淮、张郃,分道守把隘口。
却说蜀兵前队哨至陈仓,回报孔明,说:“陈仓口已筑起一城,内有大将郝昭守把,深沟高
垒,遍排鹿角,十分谨严;不如弃了此城,从太白岭鸟道出祁山甚便。”孔明曰:“陈仓正
北是街亭;必得此城,方可进兵。”命魏延引兵到城下,四面攻之。连日不能破。魏延复来
告孔明,说城难打。孔明大怒,欲斩魏延。忽帐下一人告曰:“某虽无才,随丞相多年,未
尝报效。愿去陈仓城中,说郝昭来降,不用张弓只箭。”众视之,乃部曲靳祥也。孔明曰:
“汝用何言以说之?”祥曰:“郝昭与某,同是陇西人氏,自幼交契。某今到彼,以利害说
之,必来降矣。”孔明即令前去。

靳祥骤马径到城下,叫曰:“郝伯道故人靳祥来见。”城上人报知郝昭。昭令开门放
入,登城相见。昭问曰:“故人因何到此?”祥曰:“吾在西蜀孔明帐下,参赞军机,待以
上宾之礼。特令某来见公,有言相告。”昭勃然变色曰:“诸葛亮乃我国仇敌也!吾事魏,
汝事蜀,各事其主,昔时为昆仲,今时为仇敌!汝再不必多言,便请出城!”靳祥又欲开
言,郝昭已出敌楼上了。魏军急催上马,赶出城外。祥回头视之,见昭倚定护心木栏杆。祥
勒马以鞭指之曰:“伯道贤弟,何太情薄耶?”昭曰:“魏国法度,兄所知也。吾受国恩,
但有死而已,兄不必下说词。早回见诸葛亮,教快来攻城,吾不惧也!”

祥回告孔明曰:“郝昭未等某开言,便先阻却。”孔明曰:“汝可再去见他,以利害说
之。”祥又到城下,请郝昭相见。昭出到敌楼上。祥勒马高叫曰:“伯道贤弟,听吾忠言:
汝据守一孤城,怎拒数十万之众?今不早降,后悔无及!且不顺大汉而事奸魏,抑何不知天
命、不辨清浊乎?愿伯道思之。”郝昭大怒,拈弓搭箭,指靳祥而喝曰:“吾前言已定,汝
不必再言!可速退!吾不射汝!”

靳祥回见孔明,具言郝昭如此光景。孔明大怒曰:“匹夫无礼太甚!岂欺吾无攻城之具
耶?”随叫土人问曰:“陈仓城中,有多少人马?”土人告曰:“虽不知的数,约有三千
人。”孔明笑曰:“量此小城,安能御我!休等他救兵到,火速攻之!”于是军中起百乘云
梯,一乘上可立十数人,周围用木板遮护。军士各把短梯软索,听军中擂鼓,一齐上城。郝
昭在敌楼上,望见蜀兵装起云梯,四面而来,即令三千军各执火箭,分布四面;待云梯近
城,一齐射之。孔明只道城中无备,故大造云梯,令三军鼓噪呐喊而进;不期城上火箭齐
发,云梯尽着,梯上军士多被烧死,城上矢石如雨,蜀兵皆退。孔明大怒曰:“汝烧吾云
梯,吾却用冲车之法!”于是连夜安排下冲车。次日,又四面鼓嗓呐喊而进。郝昭急命运石
凿眼,用葛绳穿定飞打,冲车皆被打折。孔明又令人运土填城壕,教廖化引三千锹钁军,从
夜间掘地道,暗入城去。郝昭又于城中掘重壕横截之。如此昼夜相攻,二十余日,无计可
破。

孔明正在营中忧闷,忽报:“东边救兵到了,旗上书:‘魏先锋大将王双’。”孔明问
曰:“谁可迎之?”魏延出曰:“某愿往。”孔明曰:“汝乃先锋大将,未可轻出。”又
问:“谁敢迎之?”裨将谢雄应声而出。孔明与三千军去了。孔明又问曰:“谁敢再去?”
裨将龚起应声要去。孔明亦与三千兵去了。孔明恐城内郝昭引兵冲出,乃把人马退二十里下
寨。

却说谢雄引军前行,正遇王双;战不三合,被双一刀劈死。蜀兵败走,双随后赶来。龚
起接着,交马只三合,办被双所斩。败兵回报孔明。孔明大惊,忙令廖化、王平、张嶷三人
出迎。两阵对圆,张嶷出马,王平、廖化压住阵角。王双纵马来与张嶷交马,数合不分胜
负。双诈败便走,嶷随后赶去。王平见张嶷中计,忙叫曰:“休赶!”嶷急回马时,王双流
星锤早到,正中其背。巍伏鞍而走,双回马赶来。王平、廖化截住,救得张嶷回阵。王双驱
兵大杀一阵,蜀兵折伤甚多。巍吐血数口,回见孔明,说:“王双英雄无敌;如今将二万兵
就陈仓城外下寨,四围立起排栅,筑起重城,深挖壕堑,守御甚严。”孔明见折二将,张嶷
又被打伤,即唤姜维曰:“陈仓道口这条路不可行。别求何策?”维曰:“陈仓城池坚固,
郝昭守御甚密,又得王双相助,实不可取。不若令一大将,依山傍水,下寨固守;再令良将
守把要道,以防街亭之攻;却统大军去袭祁山,某却如此如此用计,可捉曹真也。”孔明从
其言,即令王平,李恢,引二枝兵守街亭小路;魏延引一军守陈仓口。马岱为先锋,关兴、
张苞为前后救应使,从小径出斜谷望祁山进发。却说曹真因思前番被司马懿夺了功劳,因此
到洛阳分调郭淮、孙礼东西守把;又听的陈仓告急,已令王双去救。闻知王双斩将立功,大
喜,乃令中护军大将费耀,权摄前部总督,诸将各自守把隘口。忽报山谷中捉得细作来见。
曹真令押入,跪于帐前。其人告曰:“小人不是奸细,有机密来见都督,误被伏路军捉来,
乞退左右。”真乃教去其缚,左右暂退。其人曰:“小人乃姜伯约心腹人也。蒙本官遣送密
书。”真曰:“书安在?”其人于贴肉衣内取出呈上。真拆视曰:“罪将姜维百拜,书呈大
都督曹麾下:维念世食魏禄,忝守边城;叨窃厚恩,无门补报。昨日误遭诸葛亮之计,陷身
于巅崖之中。思念旧国,何日忘之!今幸蜀兵西出,诸葛亮甚不相疑。赖都督亲提大兵而
来:如遇敌人,可以诈败;维当在后,以举火为号,先烧蜀人粮草,却以大兵翻身掩之,则
诸葛亮可擒也。非敢立功报国,实欲自赎前罪。倘蒙照察,速赐来命。”曹真看毕,大喜
曰:“天使吾成功也!”遂重赏来人,便令回报,依期会合。真唤费耀商议曰:“今姜维暗
献密书,令吾如此如此。”耀曰:“诸葛亮多谋,姜维智广,或者是诸葛亮所使,恐其中有
诈。”真曰:“他原是魏人,不得已而降蜀,又何疑乎?”耀曰:“都督不可轻去,只守定
本寨。某愿引一军接应姜维。如成,功尽归都督;倘有奸计,某自支当。”真大喜,遂令费
耀引五万兵,望斜谷而进。行了两三程,屯下军马,令人哨探。当日申时分,回报:“斜谷
道中,有蜀兵来也。”耀忙催兵进。蜀兵未及交战先退。耀引兵追之,蜀兵又来。方欲对
阵,蜀兵又退:如此者三次,俄延至次日申时分。魏军一日一夜,不曾敢歇,只恐蜀兵攻
击。方欲屯军造饭,忽然四面喊声大震,鼓角齐鸣,蜀兵漫山遍野而来。门旗开处,闪出一
辆四轮车,孔明端坐其中,令人请魏军主将答话。耀纵马而出,遥见孔明,心中暗喜,回顾
左右曰:“如蜀兵掩至,便退后走。若见山后火起,却回身杀去,自有兵来相应。”分付
毕,跃马出呼曰:“前者败将,今何敢又来!”孔明曰:“唤汝曹真来答话!”耀骂曰:
“曹都督乃金枝玉叶,安肯与反贼相见耶!”孔明大怒,把羽扇一招,左有马岱,右有张
嶷,两路兵冲出。魏兵便退。行不到三十里,望见蜀兵背后火起,喊声不绝。费耀只道号
火,便回身杀来。蜀兵齐退。耀提刀在前,只望喊处追赶。将次近火,山路中鼓角喧天、喊
声震地,两军杀出:左有关兴,右有张苞。山上矢石如雨,往下射来。魏兵大败。费耀知是
中计,急退军望山谷中而走,人马困乏。背后关兴引生力军赶来,魏兵自相践踏及落涧身死
者,不知其数。

耀逃命而走,正遇山坡口一彪军,乃是姜维。耀大骂曰:“反贼无信!吾不幸误中汝奸
计也!”维笑曰:“吾欲擒曹真,误赚汝矣!速下马受降!”耀骤马夺路,望山谷中而走。
忽见谷口火光冲天,背后追兵又至。耀自刎身死,余众尽降。孔明连夜驱兵,直出祁山前下
寨,收住军马,重赏姜维。维曰:“某恨不得杀曹真也!”孔明亦曰:“可惜大计小用
矣。”

却说曹真听知折了费耀,悔之不及,遂与郭淮商议退兵之策。于是孙礼、辛毗星夜具表
申奏魏主,言蜀兵又出祁山,曹真损兵折将,势甚危急。睿大惊,即召司马懿入内曰:“曹
真损兵折将,蜀兵又出祁山。卿有何策,可以退之?”懿曰:“臣已有退诸葛亮之计。不用
魏军扬武耀威,蜀兵自然走矣。”正是:已见子丹无胜术,全凭仲达有良谋。未知其计如
何,且看下文分解。