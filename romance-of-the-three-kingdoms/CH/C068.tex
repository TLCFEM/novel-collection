\chapter{甘宁百骑劫魏营~左慈掷杯戏曹操}

却说孙权在濡须口收拾军马,忽报曹操自汉中领兵四十万前来救合淝。孙权与谋士计
议,先拨董袭、徐盛二人领五十只大船,在濡须口埋伏;令陈武带领人马,往来江岸巡哨。
张昭曰:“今曹操远来,必须先挫其锐气。”权乃问帐下曰:“曹操远来,谁敢当先破敌,
以挫其锐气?”凌统出曰:“某愿往。”权曰:“带多少军去?”统曰:“三千人足矣。”
甘宁曰:“只须百骑,便可破敌,何必三千!”凌统大怒。两个就在孙权面前争竞起来。权
曰:“曹军势大,不可轻敌。”乃命凌统带三千军出濡须口去哨探,遇曹兵,便与交战。凌
统领命,引着三千人马,离濡须坞。尘头起处,曹兵早到。先锋张辽与凌统交锋,斗五十
合,不分胜败。孙权恐凌统有失,令吕蒙接应回营。甘宁见凌统回,即告权曰:“宁今夜只
带一百人马去劫曹营;若折了一人一骑,也不算功。”孙权壮之,乃调拨帐下一百精锐马兵
付宁;又以酒五十瓶,羊肉五十斤,赏赐军士。甘宁回到营中,教一百人皆列坐,先将银碗
斟酒,自吃两碗,乃语百人曰:“今夜奉命劫寨,请诸公各满饮一觞,努力向前。”众人闻
言,面面相觑。甘宁见众人有难色,乃拔剑在手,怒叱曰:“我为上将,且不惜命;汝等何
得迟疑!”众人见甘宁作色,皆起拜曰:“愿效死力。”甘宁将酒肉与百人共饮食尽,约至
二更时候取白鹅翎一百根,插于盔上为号;都披甲上马,飞奔曹操寨边,拔开鹿角,大喊一
声,杀入寨中,径奔中军来杀曹操。原来中军人马,以车仗伏路穿连,围得铁桶相似,不能
得进。甘宁只将百骑,左冲右突。曹兵惊慌,正不知敌兵多少,自相扰乱。那甘宁百骑,在
营内纵横驰骤,逢着便杀。各营鼓噪,举火如星,喊声大震。甘宁从寨之南门杀出,无人敢
当。孙权令周泰引一枝兵来接应。甘宁将百骑回到濡须。操兵恐有埋伏,不敢追袭。后人有
诗赞曰:“鼙鼓声喧震地来,吴师到处鬼神哀!百翎直贯曹家寨,尽说甘宁虎将才。”甘宁
引百骑到寨,不折一人一骑;至营门,令百人皆击鼓吹笛,口称“万岁”,欢声大震。孙权
自来迎接。甘宁下马拜伏。权扶起,携宁手曰:“将军此去,足使老贼惊骇。非孤相舍,正
欲观卿胆耳!”即赐绢千匹,利刀百口。宁拜受讫,遂分赏百人。权语诸将曰:“孟德有张
辽,孤有甘兴霸,足以相敌也。”

次日,张辽引兵搦战。凌统见甘宁有功,奋然曰:“统愿敌张辽。”权许之。统遂领兵
五千,离濡须。权自引甘宁临阵观战。对阵圆处,张辽出马,左有李典,右有乐进。凌统纵
马提刀,出至阵前。张辽使乐进出迎。两个斗到五十合,未分胜败。曹操闻知,亲自策马到
门旗下来看,见二将酣斗,乃令曹休暗放冷箭。曹休便闪在张辽背后,开弓一箭,正中凌统
坐下马,那马直立起来,把凌统掀翻在地。乐进连忙持枪来刺。枪还未到,只听得弓弦响
处,一箭射中乐进面门,翻身落马。两军齐出,各救一将回营,鸣金罢战。凌统回寨中拜谢
孙权。权曰:“放箭救你者,甘宁也。”凌统乃顿首拜宁曰:“不想公能如此垂恩!”自此
与甘宁结为生死之交,再不为恶。且说曹操见乐进中箭,令自到帐中调治。次日,分兵五路
来袭濡须:操自领中路;左一路张辽,二路李典;右一路徐晃,二路庞德。每路各带一万人
马,杀奔江边来。时董袭、徐盛二将,在楼船上见五路军马来到,诸军各有惧色。徐盛曰:
“食君之禄,忠君之事,何惧哉!”遂引猛士数百人,用小船渡过江边,杀入李典军中去
了。董袭在船上,令众军擂鼓呐喊助威。忽然江上猛风大作,白浪掀天,波涛汹涌。军士见
大船将覆,争下脚舰逃命。董袭仗剑大喝曰:“将受君命,在此防贼,怎敢弃船而去!”立
斩下船军士十余人。须臾,风急船覆,董袭竟死于江口水中。徐盛在李典军中,往来冲突。

却说陈武听得江边厮杀,引一军来,正与庞德相遇,两军混战。孙权在濡须坞中,听得
曹兵杀到江边,亲自与周泰引军前来助战。正见徐盛在李典军中搅做一团厮杀,便麾军杀入
接应。却被张辽、徐晃两枝军,把孙权困在垓心。曹操上高阜处看见孙权被围,急令许诸纵
马持刀杀入军中,把孙权军冲作两段,彼此不能相救。

却说周泰从军中杀出,到江边,不见了孙权,勒回马,从外又杀入阵中,问本部军:
“主公何在?”军人以手指兵马厚处,曰:“主公被围甚急!”周泰挺身杀入,寻见孙权。
泰曰:“主公可随泰杀出。”于是泰在前,权在后,奋力冲突。泰到江边,回头又不见孙
权,乃复翻身杀入围中,又寻见孙权。权曰:“弓弩齐发,不能得出,如何?”泰曰:“主
公在前,某在后,可以出围。”孙权乃纵马前行。周泰左右遮护,身被数枪,箭透重铠,救
得孙权。到江边,吕蒙引一枝水军前来接应下船。权曰:“吾亏周泰三番冲杀,得脱重围。
但徐盛在垓心,如何得脱?”周泰曰:“吾再救去。”遂轮枪复翻身杀入重围之中,救出徐
盛。二将各带重伤。吕蒙教军士乱箭射住岸上兵,救二将下船。却说陈武与庞德大战,后面
又无应兵,被庞德赶到峪口,树林丛密;陈武再欲回身交战,被树株抓往袍袖,不能迎敌,
为庞德所杀。曹操见孙权走脱了,自策马驱兵,赶到江边对射。吕蒙箭尽,正慌间,忽对江
一宗船到,为首一员大将,乃是孙策女婿陆逊,自引十万兵到;一阵射退曹兵,乘势登岸追
杀曹兵,复夺战马数千匹,曹兵伤者,不计其数,大败而回。于乱军中寻见陈武尸首,孙权
知陈武已亡,董袭又沉江而死,哀痛至切,令人入水中寻见董袭尸首,与陈武尸一齐厚葬
之。又感周泰救护之功,设宴款之。权亲自把盏,抚其背,泪流满面,曰:“卿两番相救,
不惜性命,被枪数十,肤如刻画,孤亦何心不待卿以骨肉之恩、委卿以兵马之重乎!卿乃孤
之功臣,孤当与卿共荣辱、同休戚也。”言罢,令周泰解衣与众将观之:皮肉肌肤,如同刀
剜,盘根遍体。孙权手指其痕,一一问之。周泰具言战斗被伤之状。一处伤令吃一觥酒。是
日,周泰大醉。权以青罗伞赐之,令出入张盖,以为显耀。权在濡须,与操相拒月余,不能
取胜。张昭,顾雍上言:“曹操势大,不可力取;若与久战,大损士卒:不若求和安民为
上。”孙权从其言,令步骘往曹营求和,许年纳岁贡。操见江南急未可下,乃从之,令:
“孙权先撤人马,吾然后班师。”步骘回覆,权只留蒋钦、周泰守濡须口,尽发大兵上船回
秣陵。操留曹仁、张辽屯合淝,班师回许昌。文武众官皆议立曹操为魏王。尚书崔琰力言不
可。众官曰:“汝独不见荀文若乎?”琰大怒曰:“时乎,时乎!会当有变,任自为之!”
有与琰不和者,告知操。操大怒,收琰下狱问之。琰虎目虬髯,只是大骂曹操欺君奸贼。廷
尉白操,操令杖杀崔琰在狱中。后人有赞曰:“清河崔琰,天性坚刚;虬髯虎目,铁石心
肠;奸邪辟易,声节显昂;忠于汉主,千古名扬!”

建安二十一年夏五月,群臣表奏献帝,颂魏公曹操功德,极天际地,伊、周莫及,宜进
爵为王。献帝即令钟繇草诏,册立曹操为魏王。曹操假意上书三辞。诏三报不许,操乃拜命
受魏王之爵,冕十二旒,乘金根车,驾六马,用天子车服銮仪,出警入跸,于邺郡盖魏王
宫,议立世子。操大妻丁夫人无出。妾刘氏生子曹昂,因征张绣时死于宛城。卞氏所生四
子:长曰丕,次曰彰,三曰植,四曰熊。于是黜丁夫人,而立卞氏为魏王后。第三子曹植,
字子建,极聪明,举笔成章,操欲立之为后嗣。长子曹丕,恐不得立,乃问计于中大夫贾
诩。诩教如此如此。自是但凡操出征,诸子送行,曹植乃称述功德,发言成章;惟曹丕辞
父,只是流涕而拜,左右皆感伤。于是操疑植乖巧,诚心不及丕也。丕又使人买嘱近侍,皆
言丕之德。操欲立后嗣,踌躇不定,乃问贾诩曰:“孤欲立后嗣,当立谁?”贾诩不答,操
问其故,诩曰:“正有所思,故不能即答耳。”操曰:“何所思?”诩对曰:“思袁本初、
刘景升父子也。”操大笑,遂立长子曹丕为王世子。

冬十月,魏王宫成,差人往各处收取奇花异果,栽植后苑。有使者到吴地,见了孙权,
传魏王令旨,再往温州取柑子。时孙权正尊让魏王,便令人于本城选了大柑子四十余担,星
夜送往邺郡。至中途,挑担役夫疲困,歇于山脚下,见一先生,眇一目,跛一足,头戴白藤
冠,身穿青懒衣,来与脚夫作礼,言曰:“你等挑担劳苦,贫道都替你挑一肩何如?”众人
大喜。于是先生每担各挑五里。但是先生挑过的担儿都轻了。众皆惊疑。先生临去,与领柑
子官说:“贫道乃魏王乡中故人,姓左,名慈,字元放,道号乌角先生。如你到邺郡,可说
左慈申意。”遂拂袖而去。

取柑人至邺郡见操,呈上柑子。操亲剖之,但只空壳,内并无肉。操大惊,问取柑人。
取柑人以左慈之事对。操未肯信,门吏忽报:“有一先生,自称左慈,求见大王。”操召
入。取柑人曰:“此正途中所见之人。”操叱之曰:“汝以何妖术,摄吾佳果?”慈笑曰:
“岂有此事!”取柑剖之,内皆有肉,其味甚甜。但操自剖者,皆空壳。操愈惊,乃赐左慈
坐而问之。慈索酒肉,操令与之,饮酒五斗不醉,肉食全羊不饱。操问曰:“汝有何术,以
至于此?”慈曰:“贫道于西川嘉陵峨嵋山中,学道三十年,忽闻石壁中有声呼我之名;及
视,不见。如此者数日。忽有天雷震碎石壁,得天书三卷,名曰《遁甲天书》。上卷名‘天
遁’,中卷名‘地遁’,下卷名‘人遁’。天遁能腾云跨风,飞升太虚;地遁能穿山透石;
人遁能云游四海,藏形变身,飞剑掷刀,取人首级。大王位极人臣,何不退步,跟贫道往峨
嵋山中修行?当以三卷天书相授。”操曰:“我亦久思急流勇退,奈朝廷未得其人耳。”慈
笑曰:“益州刘玄德乃帝室之胄,何不让此位与之?不然,贫道当飞剑取汝之头也。”操大
怒曰:“此正是刘备细作!”喝左右拿下。慈大笑不止。操令十数狱卒,捉下拷之。狱卒着
力痛打,看左慈时,却齁齁熟睡,全无痛楚。操怒,命取大枷,铁钉钉了,铁锁锁了,送入
牢中监收,令人看守。只见枷锁尽落,左慈卧于地上,并无伤损。连监禁七日,不与饮食。
及看时,慈端坐于地上,面皮转红。狱卒报知曹操,操取出问之。慈曰:“我数十年不食,
亦不妨;日食千羊,亦能尽。”操无可奈何。

是日,诸官皆至王宫大宴。正行酒间,左慈足穿木履,立于筵前。众官惊怪。左慈曰:
“大王今日水陆俱备,大宴群臣,四方异物极多,内中欠少何物,贫道愿取之。”操曰:
“我要龙肝作羹,汝能取否?”慈曰:“有何难哉!”取墨笔于粉墙上画一条龙,以袍袖一
拂,龙腹自开。左慈于龙腹中提出龙肝一副,鲜血尚流。操不信,叱之曰:“汝先藏于袖中
耳!”慈曰:“即今天寒,草木枯死;大王要甚好花,随意所欲。”操曰:“吾只要牡丹
花。”慈曰:“易耳。”令取大花盆放筵前。以水噀之。顷刻发出牡丹一株,开放双花。众
官大惊,邀慈同坐而食。少刻,庖人进鱼脍。慈曰:“脍必松江鲈鱼者方美,”操曰:“千
里之隔,安能取之?”慈曰:“此亦何难取!”教把钓竿来,于堂下鱼池中钓之。顷刻钓出
数十尾大鲈鱼,放在殿上。操曰:“吾池中原有此鱼。”慈曰:“大王何相欺耶?天下鲈鱼
只两腮,惟松江鲈鱼有四腮:此可辨也。”众官视之,果是四腮。慈曰:“烹松江鲈鱼,须
紫芽姜方可。”操曰:“汝亦能取之否?”慈曰:“易耳。”令取金盆一个,慈以衣覆之。
须臾,得紫芽姜满盆,进上操前。操以手取之,忽盆内有书一本,题曰《孟德新书》。操取
视之,一字不差。操大疑,慈取桌上玉杯,满斟佳酿进操曰:“大王可饮此酒,寿有千
年。”操曰:“汝可先饮。”慈遂拔冠上玉簪,于杯中一画,将酒分为两半;自饮一半,将
一半奉操。操叱之。慈掷杯于空中,化成一白鸠,绕殿而飞。众官仰面视之,左慈不知所
往。左右忽报:“左慈出宫门去了。”操曰:“如此妖人,必当除之!否则必将为害。”遂
命许褚引三百铁甲军追擒之。

褚上马引军赶至城门,望见左慈穿木履在前,慢步而行。褚飞马追之,却只追不上。直
赶到一山中,有牧羊小童,赶着一群羊而来,慈走入羊群内。褚取箭射之,慈即不见。褚尽
杀群羊而回。牧羊小童守羊而哭,忽见羊头在地上作人言,唤小童曰:“汝可将羊头都凑在
死羊腔子上。”小童大惊,掩面而走。忽闻有人在后呼曰:“不须惊走,还汝活羊。”小童
回顾,见左慈已将地上死羊凑活,赶将来了。小童急欲问时,左慈已拂袖而去。其行如飞,
倏忽不见。

小童归告主人,主人不敢隐讳,报知曹操。操画影图形,各处捉拿左慈。三日之内,城
里城外,所捉眇一目、跛一足、白藤冠、青懒衣、穿木履先生,都一般模样者,有三四百
个。哄动街市。操令众将,将猪羊血泼之,押送城南教场。曹操亲自引甲兵五百人围住,尽
皆斩之。人人颈腔内各起一道青气,到上天聚成一处,化成一个左慈,向空招白鹤一只骑
坐,拍手大笑曰:“土鼠随金虎,奸雄一旦休!”操令众将以弓箭射之。忽然狂风大作,走
石扬沙;所斩之尸,皆跳起来,手提其头,奔上演武厅来打曹操。文官武将,掩面惊倒,各
不相顾。正是:奸雄权势能倾国,道士仙机更异人。未知曹操性命如何,且看下文分解。