\chapter{贾文和料敌决胜~夏侯惇拔矢啖睛}

却说贾诩料知曹操之意,便欲将计就计而行,乃谓张绣曰:“某在城上见曹操绕城而观
者三日。他见城东南角砖土之色,新旧不等,鹿角多半毁坏,意将从此处攻进,却虚去西北
上积草,诈为声势,欲哄我撤兵守西北,彼乘夜黑必爬东南角而进也。绣曰:“然则奈
何?”诩曰:“此易事耳。来日可今精壮之兵,饱食轻装,尽蒙于东南房屋内,却教百姓假
扮军士,虚守西北。夜间任他在东南角上爬城。俟其爬进城时,一声炮响,伏兵齐起,操可
擒矣。”绣喜,从其计。

早有探马报曹操,说张绣尽撤兵在西北角上,呐喊守城,东南却甚空虚。操曰:“中吾
计矣!”遂命军中密备锹钁爬城器具。日间只引军攻西北角。至二更时分,却领精兵于东南
角上爬过壕去,砍开鹿角。城中全无动静,众军一齐拥入。只听得一声炮响,伏兵四起。曹
军急退,背后张绣亲驱勇壮杀来。曹军大败,退出城外,奔走数十里。张绣直杀至天明方收
军入城。曹操计点败军,折兵五万余人,失去辎重无数。吕虔、于禁俱各被伤。却说贾诩见
操败走,急劝张绣遗书刘表,使起兵截其后路。表得书,即欲起兵。忽探马报孙策屯兵湖
口。蒯良曰:“策屯兵湖口,乃曹操之计也。今操新败,若不乘势击之,后必有患。”表乃
令黄祖坚守隘口,自己统兵至安众县截操后路;一面约会张绣。绣知表兵已起,即同贾诩引
兵袭操。

且说操军缓缓而行,至襄城,到清水,操忽于马上放声大哭。众惊问其故,操曰:“吾
思去年于此地折了吾大将典韦,不由不哭耳!”因即下令屯住军马,大设祭筵,吊奠典韦亡
魂。操亲自拈香哭拜,三军无不感叹。祭典韦毕,方祭侄曹安民及长子曹昂,并祭阵亡军
士;连那匹射死的大宛马,也都致祭。次日,忽荀彧差人报说:“刘表助张绣屯兵安众,截
吾归路。”操答彧书曰:“吾日行数里,非不知贼来追我;然吾计划已定,若到安众,破绣
必矣。君等勿疑。”便催军行至安众县界。刘表军已守险要,张绣随后引军赶来。操乃令众
军黑夜凿险开道,暗伏奇兵。及天色微明,刘表、张绣军会合,见操兵少,疑操遁去,俱引
兵入险击之。操纵奇兵出,大破两家之兵。曹兵出了安众隘口,于隘外下塞。刘表、张绣各
整败兵相见。表曰:“何期反中曹操奸计!”绣曰:“容再图之。”于是两军集于安众。且
说荀彧探知袁绍欲兴兵犯许都,星夜驰书报曹操。操得书心慌,即日回兵。细作报知张绣,
绣欲追之。贾诩曰:“不可追也,追之必败。”刘表曰:“今日不追,坐失机会矣。”力劝
绣引军万余同往追之。约行十余里,赶上曹军后队。曹军奋力接战,绣、表两军大败而还。
绣谓诩曰:“不用公言,果有此败。”诩曰:“今可整兵再往追之。”绣与表俱曰:“今已
败,奈何复追?”诩曰:“今番追去,必获大胜;如其不然,请斩吾首。”绣信之。刘表疑
虑,不肯同往。绣乃自引一军往追。操兵果然大败,军马辎重,连路散弃而走。绣正往前追
赶。忽山后一彪军拥出。绣不敢前追,收军回安众。刘表问贾诩曰:“前以精兵追退兵,而
公曰必败;后以败卒击胜兵,而公曰必克:究竟悉如公言。何其事不同而皆验也?愿公明教
我。”诩曰:“此易知耳。将军虽善用兵,非曹操敌手。操军虽败,必有劲将为后殿,以防
追兵;我兵虽锐,不能敌之也:故知必败。夫操之急于退兵者,必因许都有事;既破我追军
之后,必轻车速回,不复为备;我乘其不备而更追之:故能胜也。”刘表、张绣俱服其高
见。诩劝表回荆州,绣守襄城,以为唇齿。两军各散。且说曹操正行间,闻报后军为绣所
追,急引众将回身救应,只见绣军已退。败兵回告操曰:“若非山后这一路人马阻住中路,
我等皆被擒矣。”操急问何人。那人绰枪下马,拜见曹操,乃镇威中郎将,江夏平春人,姓
李,名通,字文达。操问何来。通曰:“近守汝南,闻丞相与张绣、刘表战,特来接应。”
操喜,封之为建功侯,守汝南西界,以防表、绣。李通拜谢而去。操还许都,表奏孙策有
功,封为讨逆将军,赐爵吴侯,遣使赍诏江东,谕令防剿刘表。

操回府,众官参见毕,荀彧问曰:“丞相缓行至安众,何以知必胜贼兵?”操曰:“彼
退无归路,必将死战,吾缓诱之而暗图之,是以知其必胜也。”荀彧拜服。郭嘉入,操曰:
“公来何暮也?”嘉袖出一书,白操曰:“袁绍使人致书丞相,言欲出兵攻公孙瓒,特来借
粮借兵。”操曰:“吾闻绍欲图许都,今见吾归,又别生他议。”遂拆书观之。见其词意骄
慢,乃问嘉曰:“袁绍如此无状,吾欲讨之,恨力不及,如何?”嘉曰:“刘、项之不敌,
公所知也。高祖惟智胜,项羽虽强,终为所擒。今绍有十败,公有十胜,绍兵虽盛,不足惧
也:绍繁礼多仪,公体任自然,此道胜也;绍以逆动,公以顺率,此义胜也;桓、灵以来,
政失于宽,绍以宽济,公以猛纠,此治胜也;绍外宽内忌,所任多亲戚,公外简内明,用人
惟才,此度胜也;绍多谋少决,公得策辄行,此谋胜也;绍专收名誉,公以至诚待人,此德
胜也;绍恤近忽远,公虑无不周,此仁胜也;绍听谗惑乱,公浸润不行,此明胜也;绍是非
混淆,公法度严明,此文胜也;绍好为虚势,不知兵要,公以少克众,用兵如神,此武胜
也。公有此十胜,于以败绍无难矣。”操笑曰:“如公所言,孤何足以当之!”荀彧曰:
“郭奉孝十胜十败之说,正与愚见相合。绍兵虽众,何足惧耶!”嘉曰:“徐州吕布,实心
腹大患。今绍北征公孙瓒,我当乘其远出,先取吕布,扫除东南,然后图绍,乃为上计;否
则我方攻绍,布必乘虚来犯许都,为害不浅也。”操然其言,遂议东征吕布。荀彧曰:“可
先使人往约刘备,待其回报,方可动兵。”操从之,一面发书与玄德,一面厚遣绍使,奏封
绍为大将军、太尉,兼都督冀、青、幽、并四州,密书答之云:“公可讨公孙瓒。吾当相
助。”绍得书大喜,便进兵攻公孙瓒。

且说吕布在徐州,每当宾客宴会之际,陈珪父子必盛称布德。陈宫不悦,乘间告布曰:
“陈珪父子面谀将军,其心不可测,宜善防之。”布怒叱曰:“汝无端献谗,欲害好人
耶?”宫出叹曰:“忠言不入,吾辈必受殃矣!”意欲弃布他往,却又不忍;又恐被人嗤
笑。乃终日闷闷不乐。一日,带领数骑去小沛地面围猎解闷,忽见官道上一骑驿马,飞奔前
去。宫疑之,弃了围场,引从骑从小路赶上,问曰:“汝是何处使命?”那使者知是吕布部
下人,慌不能答。陈宫令搜其身,得玄德回答曹操密书一封。宫即连人与书,拿见吕布。布
问其故。来使曰:“曹丞相差我往刘豫州处下书,今得回书,不知书中所言何事。”布乃拆
书细看。书略曰:“奉明命欲图吕布,敢不夙夜用心。但备兵微将少,不敢轻动。丞相兴大
师,备当为前驱。谨严兵整甲,专待钧命。”

吕布见了,大骂曰:“操贼焉敢如此!”遂将使者斩首。先使陈宫、臧霸、结连泰山寇
孙观、吴敦、尹礼、昌稀,东取山东兖州诸郡。令高顺、张辽取沛城,攻玄德。令宋宪、魏
续西取汝、颍。布自总中军为三路救应。

且说高顺等引兵出徐州,将至小沛,有人报知玄德。玄德急与众商议。孙乾曰:“可速
告急于曹操。”玄德曰:“谁可去许都告急?”阶下一人出曰:“某愿往。”视之,乃玄德
同乡人,姓简,名雍,字宪和,现为玄德幕宾。玄德即修书付简雍,使星夜赴许都求援;一
面整顿守城器具。玄德自守南门,孙乾守北门,云长守西门,张飞守东门,令糜竺与其弟糜
芳守护中军。原来糜竺有一妹,嫁与玄德为次妻。玄德与他兄弟有郎舅之亲,故令其守中军
保护妻小。高顺军至,玄德在敌楼上问曰:“吾与奉先无隙,何故引兵至此?”顺曰:“你
结连曹操,欲害吾主,今事已露,何不就缚!”言讫,便麾军攻城。玄德闭门不出。次日,
张辽引兵攻打西门。云长在城上谓之曰:“公仪表非俗,何故失身于贼?”张辽低头不语。
云长知此人有忠义之气,更不以恶言相加,亦不出战。辽引兵退至东门,张飞便出迎战。早
有人报知关公。关公急来东门看时,只见张飞方出城,张辽军已退。飞欲追赶,关公急召入
城。飞曰:“彼惧而退,何不追之。”关公曰:“此人武艺不在你我之下。因我以正言感
之,颇有自悔之心,故不与我等战耳。”飞乃悟,只令士卒坚守城门,更不出战。

却说简雍至许都见曹操,具言前事。操即聚众谋士议曰:“吾欲攻吕布,不忧袁绍掣
肘,只恐刘表、张绣议其后耳。”荀攸曰:“二人新破,未敢轻动。吕布骁勇,若更结连袁
术,纵横淮、泗,急难图矣。”郭嘉曰:“今可乘其初叛,众心未附,疾往击之。”操从其
言。即命夏侯惇与夏侯渊、吕虔、李典领兵五万先行,自统大军陆续进发,简雍随行。早有
探马报知高顺。顺飞报吕布。布先令侯成、郝萌、曹性引二百余骑接应高顺,使离沛城三十
里去迎曹军,自引大军随后接应。玄德在小沛城中见高顺退去,知是曹家兵至,乃只留孙乾
守城,糜竺、糜芳守家,自己却与关、张二公,提兵尽出城外,分头下寨,接应曹军。却说
夏侯惇引军前进,正与高顺军相遇,便挺枪出马搦战。离顺迎敌。两马相交,战有四五十
合,高顺抵敌不住,败下阵来。惇纵马追赶,顺绕阵而走。惇不舍,亦绕阵追之。阵上曹性
看见,暗地拈弓搭箭,觑得亲切,一箭射去,正中夏侯惇左目。惇大叫一声,急用手拔箭,
不想连眼珠拨出,乃大呼曰:“父精母血,不可弃也!”遂纳于口内啖之,仍复挺枪纵马,
直取曹性。性不及提防,早被一枪搠透面门,死于马下。两边军士见者,无不骇然。夏侯惇
既杀曹性,纵马便回。高顺从背后赶来,麾军齐上,曹兵大败。夏侯渊救护其兄而走。吕
虔、李典将败军退去济北下寨。高顺得胜,引军回击玄德。恰好吕布大军亦至,布与张辽、
高顺分兵三路,来攻玄德、关、张三寨,正是:啖睛猛将虽能战,中箭先锋难久持。未知玄
德胜负如何,且听下文分解。