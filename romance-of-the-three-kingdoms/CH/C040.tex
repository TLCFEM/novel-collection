\chapter{蔡夫人议献荆州~诸葛亮火烧新野}

却说玄德问孔明求拒曹兵之计。孔明曰:“新野小县,不可久居,近闻刘景升病在危
笃,可乘此机会,取彼荆州为安身之地,庶可拒曹操也。”玄德曰:“公言甚善;但备受景
升之恩,安忍图之!”孔明曰:“今若不取,后悔何及!”玄德曰:“吾宁死,不忍作负义
之事。”孔明曰:“且再作商议。”

却说夏侯惇败回许昌,自缚见曹操,伏地请死。操释之。惇曰:“惇遭诸葛亮诡计,用
火攻破我军。”操曰:“汝自幼用兵,岂不知狭处须防火攻?”惇曰:“李典、于禁曾言及
此,悔之不及!”操乃赏二人。惇曰:“刘备如此猖狂,真腹心之患也,不可不急除。”操
曰:“吾所虑者,刘备、孙权耳;余皆不足介意,今当乘此时扫平江南。”便传令起大兵五
十万,令曹仁、曹洪为第一队,张辽、张郃为第二队。夏侯渊、夏侯惇为第三队,于禁、李
典为第四队,操自领诸将为第五队:每队各引兵十万。又令许褚为折冲将军,引兵三千为先
锋。选定建安十三年秋七月丙午日出师。

太中大夫孔融谏曰:“刘备,刘表皆汉室宗亲,不可轻伐;孙权虎踞六郡,且有大江之
险,亦不易取,今丞相兴此无义之师,恐失天下之望。”操怒曰:“刘备、刘表、孙权皆逆
命之臣,岂容不讨!”遂叱退孔融,下令:“如有再谏者,必斩。”孔融出府,仰天叹曰:
“以至不仁伐至仁,安得不败乎!”时御史大夫郗虑家客闻此言,报知郗虑,虑常被孔融侮
慢,心正恨之,乃以此言入告曹操,且曰:“融平日每每狎侮丞相,又与祢衡相善,衡赞融
曰仲尼不死,融赞衡曰颜回复生。向者祢衡之辱丞相,乃融使之也。”操大怒,遂命廷尉捕
捉孔融。融有二子,年尚少,时方在家,对坐弈棋。左右急报曰:“尊君被廷尉执去,将斩
矣!二公子何不急避?”二子曰:“破巢之下,安有完卵乎?”言未已,廷尉又至,尽收融
家小并二子,皆斩之,号令融尸于市。京兆脂习伏尸而哭。操闻之,大怒,欲杀之。荀彧
曰:“彧闻脂习常谏融曰:公刚直太过,乃取祸之道,今融死而来哭,乃义人也,不可
杀。”操乃止,习收融父子尸首,皆葬之。后人有诗赞孔融曰:“孔融居北海,豪气贯长
虹:坐上客长满,樽中酒不空;文章惊世俗,谈笑侮王公。史笔褒忠直,存官纪太中。”曹
操既杀孔融,传令五队军马次第起行,只留荀彧等守许昌。

却说荆州刘表病重,使人请玄德来托孤。玄德引关、张至荆州见刘表。表曰:“我病已
入膏肓,不久便死矣,特托孤于贤弟。我子无才,恐不能承父业,我死之后,贤弟可自领荆
州。”玄德泣拜曰:“备当竭力以辅贤侄,安敢有他意乎!”正说间,人报曹操自统大兵
至。玄德急辞刘表,星夜回新野。刘表病中闻此信,吃惊不小,商议写遗嘱,令玄德辅佐长
子刘琦为荆州之主。蔡夫人闻之大怒,关上内门;使蔡瑁、张允二人把住外门。时刘琦在江
夏,知父病危,来至荆州探病,方到外门,蔡瑁当住曰:“公子奉父命镇守江夏,其任至
重;今擅离职守,倘东吴兵至,如之奈何?若入见主公,主公必生嗔怒,病将转增,非孝
也。宜速回。”刘琦立于门外,大哭一场,上马仍回江夏。刘表病势危笃,望刘琦不来;至
八月戊申日,大叫数声而死。后人有诗叹刘表曰:“昔闻袁氏居河朔,又见刘君霸汉阳。总
为牝晨致家累,可怜不久尽销亡!”

刘表既死,蔡夫人与蔡瑁、张允商议,假写遗嘱,令次子刘琮为荆州之主,然后举哀报
丧。时刘琮年方十四岁,颇聪明,乃聚众言曰:“吾父弃世,吾兄现在江夏,更有叔父玄德
在新野。汝等立我为主。倘兄与叔兴兵问罪,如何解释?”众官未及对,幕官李珪答曰:
“公子之言甚善。今可急发哀书至江夏,请大公子为荆州之主,就命玄德一同理事:北可以
敌曹操,南可以拒孙权。此万全之策也。”蔡瑁叱曰:“汝何人,敢乱言以逆主公遗命!”
李珪大骂曰:“汝内外朋谋,假称遗命,废长立幼,眼见荆襄九郡,送于蔡氏之手!故主有
灵,必当殛汝!”蔡瑁大怒,喝令左右推出斩之。李珪“至死大骂不绝。于是蔡瑁遂立刘琮
为主。蔡氏宗族,分领荆州之兵;命治中邓义、别驾刘先守荆州;蔡夫人自与刘琮前赴襄阳
驻扎,以防刘琦、刘备。就葬刘表之柩于襄阳城东汉阳之原,竟不讣告刘琦与玄德。

刘琮至襄阳,方才歇马,忽报曹操引大军径望襄阳而来。琮大惊,遂请蒯越、蔡瑁等商
议。东曹掾傅巽进言曰:“不特曹操兵来为可忧;今大公子在江夏,玄德在新野,我皆未往
报丧,若彼兴兵问罪,荆襄危矣。巽有一计,可使荆襄之民,安如泰山,又可保全主公名
爵。”琮曰:“计将安出?”巽曰:“不如将荆襄九郡,献与曹操,操必重待主公也。”琮
叱曰:“是何言也!孤受先君之基业,坐尚未稳,岂可便弃之他人?”蒯越曰:“傅公悌之
言是也。夫逆顺有大体,强弱有定势。今曹操南征北讨,以朝廷为名,主公拒之,其名不
顺。且主公新立,外患未宁,内忧将作。荆襄之民,闻曹兵至,未战而胆先寒,安能与之敌
哉?”琮曰:“诸公善言,非我不从;但以先君之业,一旦弃与他人,恐贻笑于天下耳。”

言未已,一人昂然而进曰:“傅公悌、蒯异度之言甚善,何不从之?”众视之,乃山阳
高平人,姓王,名粲,字仲宣。粲容貌瘦弱,身材短小;幼时往见中郎蔡邕,时邕高朋满
座,闻粲至,倒履迎之。宾客皆惊曰:“蔡中郎何独敬此小子耶?”邕曰:“此子有异才,
吾不如也。”粲博闻强记,人皆不及:尝观道旁碑文一过,便能记诵;观人弈棋,棋局乱,
粲复为摆出,不差一子。又善算术。其文词妙绝一时。年十七,辟为黄门侍郎,不就。后因
避乱至荆襄,刘表以为上宾。当日谓刘琮曰:“将军自料比曹公何如?”琮曰:“不如
也。”粲曰:“曹公兵强将勇,足智多谋;擒吕布于下邳,摧袁绍于官渡,逐刘备于陇右,
破乌桓于白狼:枭除荡定者,不可胜计。今以大军南下荆襄,势难抵敌。傅、蒯二君之谋,
乃长策也。将军不可迟疑,致生后悔。”琮曰:“先生见教极是。但须禀告母亲知道。”只
见蔡夫人从屏后转出,谓琮曰:“既是仲宣、公悌、异度三人所见相同,何必告我。”于是
刘琮意决,便写降书,令宋忠潜地往曹操军前投献。宋忠领命,直至宛城,接着曹操,献上
降书。操大喜,重赏宋忠,分付教刘琮出城迎接,便着他永为荆州之主。

宋忠拜辞曹操,取路回荆襄。将欲渡江,忽见一枝人马到来,视之,乃关云长也。宋忠
回避不迭,被云长唤住,细问荆州之事。忠初时隐讳;后被云长盘问不过,只得将前后事
情,——实告。云长大惊,随捉宋忠至新野见玄德,备言其事。玄德闻之大哭。张飞曰:
“事已如此,可先斩宋忠,随起兵渡江,夺了襄阳,杀了蔡氏、刘琮,然后与曹操交战。”
玄德曰:“你且缄口。我自有斟酌。”乃叱宋忠曰:“你知众人作事,何不早来报我?今虽
斩汝无益于事。可速去。”忠拜谢,抱头鼠窜而去。玄德正忧闷间,忽报公子刘琦差伊籍到
来。玄德感伊籍昔日相救之恩,降阶迎之,再三称谢。籍曰:“大公子在江夏,闻荆州已
故,蔡夫人与蔡瑁等商议,不来报丧,竟立刘琮为主。公子差人往襄阳探听,回说是实;恐
使君不知,特差某赍哀书呈报,并求使君尽起麾下精兵,同往襄阳问罪。”玄德看书毕,谓
伊籍曰:“机伯只知刘琮僭立,更不知刘琮已将荆襄九郡献与曹操矣!”籍大惊曰:“使君
从何知之?”玄德具言拿获宋忠之事。籍曰:“若如此,使君不如以吊丧为名,前赴襄阳,
诱刘琮出迎,就便擒下,诛其党类,则荆州属使君矣。”孔明曰:“机伯之言是也。主公可
从之。”玄德垂泪曰:“吾兄临危托孤于我,今若执其子而夺其地,异日死于九泉之下,何
面目复见吾兄乎?”孔明曰:“如不行此事,今曹兵已至宛城,何以拒敌?”玄德曰:“不
如走樊城以避之。”

正商议间,探马飞报曹兵已到博望了。玄德慌忙发付伊籍回江夏整顿军马,一面与孔明
商议拒敌之计。孔明曰:“主公且宽心。前番一把火,烧了夏侯惇大半人马;今番曹军又
来,必教他中这条计。我等在新野住不得了,不如早到樊城去。”便差人四门张榜,晓谕居
民:“无问老幼男女,愿从者,即于今日皆跟我往樊城暂避,不可自误。”差孙乾往河边调
拨船只,救济百姓;差糜竺护送各官家眷到樊城。一面聚诸将听令,先教云长引一千军去白
河上流头埋伏。各带布袋,多装沙土,遏住白河之水,至来日三更后,只听下流头人喊马
嘶,急取起布袋,放水淹之,却顺水杀将下来接应。又唤张飞引一千军去博陵渡口埋伏。此
处水势最慢,曹军被淹,必从此逃难,可便乘势杀来接应。又唤赵云引军三千,分为四队,
自领一队伏于东门外,其三队分伏西、南、北三门,却先于城内人家屋上,多藏硫黄焰硝引
火之物。曹军入城,必安歇民房。来日黄昏后,必有大风;但看风起,便令西、南、北三门
伏军尽将火箭射入城去;待城中火势大作,却于城外呐喊助威,只留东门放他出走。汝却于
东门外从后击之。天明会合关、张二将,收军回樊城。再令糜芳、刘封二人带二千军。一半
红旗,一半青旗,去新野城外三十里鹊尾坡前屯住。一见曹军到,红旗军走在左,青旗军走
在右。他心疑必不敢追。汝二人却去分头埋伏。只望城中火起,便可追杀败兵,然后却来白
河上流头接应。孔明分拨已定,乃与玄德登高了望,只候捷音。却说曹仁、曹洪引军十万为
前队,前面已有许褚引三千铁甲军开路,浩浩荡荡,杀奔新野来。是日午牌时分,来到鹊尾
坡,望见坡前一簇人马,尽打青、红旗号,许褚催军向前。刘封、糜芳分为四队,青、红旗
各归左右。许褚勒马,教且休进:“前面必有伏兵。我兵只在此处住下。”许褚一骑马飞报
前队曹仁。曹仁曰:“此是疑兵,必无埋伏。可速进兵。我当催军继至。”许褚复回坡前,
提兵杀入。至林下追寻时,不见一人。时日已坠西。许褚方欲前进,只听得山上大吹大擂。
抬头看时,只见山顶上一簇旗,旗丛中两把伞盖:左玄德,右孔明,二人对坐饮酒。许褚大
怒,引军寻路上山。山上擂木炮石打将下来,不能前进。又闻山后喊声大震。欲寻路厮杀,
天色已晚。曹仁领兵到,教且夺新野城歇马。军士至城下时,只见四门大开。曹兵突人,并
无阻当,城中亦不见一人,竟是一座空城了。曹洪曰:“此是势孤计穷,故尽带百姓逃窜去
了。我军权且在城安歇,来日平明进兵。”此时各军走乏,都已饥饿,皆去夺房造饭。曹
仁、曹洪就在衙内安歇。初更已后,狂风大作。守门军士飞报火起。曹仁曰:“此必军士造
饭不小心,遗漏之火,不可自惊。”说犹未了,接连几次飞报,西、南、北三门皆火起。曹
仁急令众将上马时,满县火起,上下通红。是夜之火,更胜前日博望烧屯之火。后人有诗叹
曰:“奸雄曹操守中原,九月南征到汉川。风伯怒临新野县,祝融飞下焰摩天。”曹仁引众
将突烟冒火,寻路奔走,闻说东门无火,急急奔出东门。军士自相践踏,死者无数。曹仁等
方才脱得火厄,背后一声喊起,赵云引军赶来混战,败军各逃性命,谁肯回身厮杀。正奔走
间,糜芳引一军至,又冲杀一阵。曹仁大败,夺路而走,刘封又引一军截杀一阵。到四更时
分,人困马乏,军士大半焦头烂额;奔至白河边,喜得河水不甚深,人马都下河吃水:人相
喧嚷,马尽嘶鸣。

却说云长在上流用布袋遏住河水,黄昏时分,望见新野火起;至四更,忽听得下流头人
喊马嘶,急令军士一齐掣起布袋,水势滔天,望下流冲去,曹军人马俱溺于水中,死者极
多。曹仁引众将望水势慢处夺路而走。行到博陵渡口,只听喊声大起,一军拦路,当先大
将,乃张飞也,大叫:“曹贼快来纳命!”曹军大惊。正是:城内才看红焰吐,水边又遇黑
风来。未知曹仁性命如何,且看下文分解。