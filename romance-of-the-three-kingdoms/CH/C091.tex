\chapter{祭泸水汉相班师~伐中原武侯上表}

却说孔明班师回国,孟获率引大小洞主酋长及诸部落,罗拜相送。前军至泸水,时值九
月秋天,忽然阴云布合,狂风骤起;兵不能渡,回报孔明。孔明遂问孟获,获曰:“此水原
有猖神作祸,往来者必须祭之。”孔明曰:“用何物祭享?”获曰:“旧时国中因猖神作
祸,用七七四十九颗人头并黑牛白羊祭之,自然风恬浪静,更兼连年丰稔。”孔明曰:“吾
今事已平定,安可妄杀一人?”遂自到泸水岸边观看。果见阴风大起,波涛汹涌,人马皆
惊。孔明甚疑,即寻土人问之。土人告说:“自丞相经过之后,夜夜只闻得水边鬼哭神号。
自黄昏直至天晓,哭声不绝。瘴烟之内,阴鬼无数。因此作祸,无人敢渡。”孔明曰:“此
乃我之罪愆也。前者马岱引蜀兵千余,皆死于水中;更兼杀死南人,尽弃此处。狂魂怨鬼,
不能解释,以致如此。吾今晚当亲自往祭。”土人曰:“须依旧例,杀四十九颗人头为祭,
则怨鬼自散也。”孔明曰:“本为人死而成怨鬼,岂可又杀生人耶?吾自有主意。”唤行厨
宰杀牛马;和面为剂,塑成人头,内以牛羊等肉代之,名曰馒头。当夜于泸水岸上,设香
案,铺祭物,列灯四十九盏,扬幡招魂;将馒头等物,陈设于地。三更时分,孔明金冠鹤
氅,亲自临祭,令董厥读祭文。其文曰:“维大汉建兴三年秋九月一日,武乡侯、领益州
牧、丞相诸葛亮,谨陈祭仪,享于故殁王事蜀中将校及南人亡者阴魂曰:我大汉皇帝,威胜
五霸,明继三王。昨自远方侵境,异俗起兵;纵虿尾以兴妖,盗狼心而逞乱。我奉王命,问
罪遐荒;大举貔貅,悉除蝼蚁;雄军云集,狂寇冰消;才闻破竹之声,便是失猿之势。但士
卒儿郎,尽是九州豪杰;官僚将校,皆为四海英雄:习武从戎,投明事主,莫不同申三令,
共展七擒;齐坚奉国之诚,并效忠君之志。何期汝等偶失兵机,缘落奸计:或为流矢所中,
魂掩泉台;或为刀剑所伤,魄归长夜:生则有勇,死则成名,今凯歌欲还,献俘将及。汝等
英灵尚在,祈祷必闻:随我旌旗,逐我部曲,同回上国,各认本乡,受骨肉之蒸尝,领家人
之祭祀;莫作他乡之鬼,徒为异域之魂。我当奏之天子,使汝等各家尽沾恩露,年给衣粮,
月赐廪禄。用兹酬答,以慰汝心。至于本境土神,南方亡鬼,血食有常,凭依不远;生者既
凛天威,死者亦归王化,想宜宁帖,毋致号啕。聊表丹诚,敬陈祭祀。呜呼,哀哉!伏惟尚
飨!”读毕祭文,孔明放声大哭,极其痛切,情动三军,无不下泪。孟获等众,尽皆哭泣。
只见愁云怨雾之中,隐隐有数千鬼魂,皆随风而散。于是孔明令左右将祭物尽弃于泸水之
中。次日,孔明引大军俱到泸水南岸,但见云收雾散,风静浪平。蜀兵安然尽渡泸水,果然
鞭敲金镫响,人唱凯歌还。行到永昌,孔明留王伉、吕凯守四郡;发付孟获领众自回,嘱其
勤政驭下,善抚居民,勿失农务。孟获涕泣拜别而去。

孔明自引大军回成都。后主排銮驾出郭三十里迎接,下辇立于道傍,以侯孔明。孔明慌
下车伏道而言曰:“臣不能速平南方,使主上怀忧,臣之罪也。”后主扶起孔明,并车而
回,设太平筵会,重赏三军。自此远邦进贡来朝者二百余处。孔明奏准后主,将殁于王事者
之家,一一优恤。人心欢悦,朝野清平。却说魏主曹丕,在位七年,即蜀汉建兴四年也。丕
先纳夫人甄氏,即袁绍次子袁熙之妇,前破邺城时所得。后生一子,名睿,字元仲,自幼聪
明,不甚爱之。后丕又纳安平广宗人郭永之女为贵妃,甚有颜色;其父尝曰:“吾女乃女中
之王也。”故号为女王。自丕纳为贵妃,因甄夫人失宠,郭贵妃欲谋为后,却与幸臣张韬商
议。时丕有疾,韬乃诈称于甄夫人宫中掘得桐木偶人,上书天子年月日时,为魇镇之事。丕
大怒,遂将甄夫人赐死,立郭贵妃为后。因无出,养曹睿为己子。虽甚爱之,不立为嗣。

睿年至十五岁,弓马熟娴。当年春二月,丕带睿出猎。行于山坞之间,赶出子母二鹿,
丕一箭射倒母鹿,回观小鹿驰于曹睿马前。丕大呼曰:“吾儿何不射之?”睿在马上泣告
曰:“陛下已杀其母,臣安忍复杀其子也。”丕闻之,掷弓于地曰:“吾儿真仁德之主
也!”于是遂封睿为平原王。

夏五月,丕感寒疾,医治不痊,乃召中军大将军曹真、镇军大将军陈群、抚军大将军司
马懿三人入寝宫。丕唤曹睿至,指谓曹真等曰:“今朕病已沉重,不能复生。此子年幼,卿
等三人可善辅之,勿负朕心。”三人皆告曰:“陛下何出此言?臣等愿竭力以事陛下,至千
秋万岁。”丕曰:“今年许昌城门无故自崩,乃不祥之兆,朕故自知必死也。”正言间,内
侍奏征东大将军曹休入宫问安。丕召入谓曰:“卿等皆国家柱石之臣也,若能同心辅朕之
子,朕死亦瞑目矣!”言讫,堕泪而薨。时年四十岁,在位七年。于是曹真、陈群、司马
懿、曹休等,一面举哀,一面拥立曹睿为大魏皇帝。谥父丕为文皇帝,谥母甄氏为文昭皇
后。封钟繇为太傅,曹真为大将军,曹休为大司马,华歆为太尉,王朗为司徒,陈群为司
空,司马懿为骠骑大将军。其余文武官僚,各各封赠。大赦天下。时雍、凉二州缺人守把,
司马懿上表乞守西凉等处。曹睿从之,遂封懿提督雍、凉等处兵马。领诏去讫。

早有细作飞报入川。孔明大惊曰:“曹丕已死,孺子曹睿即位,余皆不足虑:司马懿深
有谋略,今督雍、凉兵马,倘训练成时,必为蜀中之大患。不如先起兵伐之。”参军马谡
曰:“今丞相平南方回,军马疲敝,只宜存恤,岂可复远征?某有一计,使司马懿自死于曹
睿之手,未知丞相钧意允否?”孔明问是何计,马谡曰:“司马懿虽是魏国大臣,曹睿素怀
疑忌。何不密遣人往洛阳、邺郡等处,布散流言,道此人欲反;更作司马懿告示天下榜文,
遍贴诸处。使曹睿心疑,必然杀此人也。”孔明从之,即遣人密行此计去了。

却说邺城门上。忽一日见贴下告示一道。守门者揭了,来奏曹睿。睿观之,其文曰:
“骠骑大将军总领雍、凉等处兵马事司马懿,谨以信义布告天下:昔太祖武皇帝,创立基
业,本欲立陈思王子建为社稷主;不幸奸谗交集,岁久潜龙。皇孙曹睿,素无德行,妄自居
尊,有负太祖之遗意。今吾应天顺人,克日兴师,以慰万民之望。告示到日,各宜归命新
君。如不顺者,当灭九族!先此告闻,想宜知悉。”

曹睿览毕,大惊失色,急问群臣。太尉华歆奏曰:“司马懿上表乞守雍、凉,正为此
也。先时太祖武皇帝尝谓臣曰:司马懿鹰视狼顾,不可付以兵权;久必为国家大祸。今日反
情已萌,可速诛之。”王朗奏曰:“司马懿深明韬略,善晓兵机,素有大志;若不早除,久
必为祸。”睿乃降旨,欲兴兵御驾亲征。忽班部中闪出大将军曹真奏曰:“不可。文皇帝托
孤于臣等数人,是知司马仲达无异志也。今事未知真假,遽尔加兵,乃逼之反耳。或者蜀、
吴奸细行反间之计,使我君臣自乱,彼却乘虚而击,未可知也。陛下幸察之。”睿曰:“司
马懿若果谋反,将奈何?”真曰:“如陛下心疑,可仿汉高伪游云梦之计。御驾幸安邑,司
马懿必然来迎;观其动静,就车前擒之,可也。”睿从之,遂命曹真监国,亲自领御林军十
万,径到安邑。司马懿不知其故,欲令天子知其威严,乃整兵马,率甲士数万来迎。近臣奏
曰:“司马懿果率兵十余万,前来抗拒,实有反心矣。”睿慌命曹休先领兵迎之。司马懿见
兵马前来,只疑车驾亲至,伏道而迎。曹休出曰:“仲达受先帝托孤之重,何故反耶?”懿
大惊失色,汗流遍体,乃问其故。休备言前事。懿曰:“此吴、蜀奸细反间之计,欲使我君
臣自相残害,彼却乘虚而袭。某当自见天子辨之。”遂急退了军马,至睿车前俯伏泣奏曰:
“臣受先帝托孤之重,安敢有异心?必是吴、蜀之奸计。臣请提一旅之师,先破蜀,后伐
吴,报先帝与陛下,以明臣心。”睿疑虑未决。华歆奏曰:“不可付之兵权。可即罢归田
里。”睿依言,将司马懿削职回乡,命曹休总督雍;凉军马。曹睿驾回洛阳。却说细作探知
此事,报入川中。孔明闻之大喜曰:“吾欲伐魏久矣,奈有司马懿总雍、凉之兵。今既中计
遭贬,吾有何忧!”次日,后主早朝,大会官僚,孔明出班,上《出师表》一道。表曰:
“臣亮言:先帝创业未半,而中道崩殂;今天下三分,益州罢敝,此诚危急存亡之秋也。然
侍卫之臣,不懈于内;忠志之士,忘身于外者:盖追先帝之殊遇,欲报之于陛下也。诚宜开
张圣听,以光先帝遗德,恢弘志士之气;不宜妄自菲薄,引喻失义,以塞忠谏之路也。宫中
府中,俱为一体;陟罚臧否,不宜异同。若有作奸犯科,及为忠善者,宜付有司,论其刑
赏,以昭陛下平明之治;不宜偏私,使内外异法也。侍中、侍郎郭攸之、费祎、董允等,此
皆良实,志虑忠纯,是以先帝简拔以遗陛下。愚以为宫中之事,事无大小,悉以咨之,然后
施行,必得裨补阙漏,有所广益。将军向宠,性行淑均,晓畅军事,试用之于昔日,先帝称
之曰能,是以众议举宠以为督。愚以为营中之事,事无大小,悉以咨之,必能使行阵和穆,
优劣得所也。亲贤臣,远小人,此先汉所以兴隆也;亲小人,远贤臣,此后汉所以倾颓也。
先帝在时,每与臣论此事,未尝不叹息痛恨于桓、灵也!侍中、尚书、长史、参军,此悉贞
亮死节之臣也,愿陛下亲之、信之,则汉室之隆,可计日而待也。臣本布衣,躬耕南阳,苟
全性命于乱世,不求闻达于诸侯。先帝不以臣卑鄙,猥自枉屈,三顾臣于草庐之中,谘臣以
当世之事,由是感激,遂许先帝以驱驰。后值倾覆,受任于败军之际,奉命于危难之间:尔
来二十有一年矣。先帝知臣谨慎,故临崩寄臣以大事也。受命以来,夙夜忧虑,恐付托不
效,以伤先帝之明;故五月渡泸,深入不毛。今南方已定,甲兵已足,当奖帅三军,北定中
原,庶竭弩钝,攘除奸凶,兴复汉室,还于旧都:此臣所以报先帝而忠陛下之职分也。至于
斟酌损益,进尽忠言,则攸之、祎、允之任也。愿陛下托臣以讨贼兴复之效,不效则治臣之
罪,以告先帝之灵;若无兴复之言,则责攸之、祎、允等之咨,以彰其慢。陛下亦宜自谋,
以谘诹善道,察纳雅言,深追先帝遗诏。臣不胜受恩感激!今当远离,临表涕泣,不知所
云。”

后主览表曰:“相父南征,远涉艰难;方始回都,坐未安席;今又欲北征,恐劳神
思。”孔明曰:“臣受先帝托孤之重,夙夜未尝有怠。今南方已平,可无内顾之忧;不就此
时讨贼,恢复中原,更待何日?”忽班部中太史谯周出奏曰:“臣夜观天象,北方旺气正
盛,星曜倍明,未可图也。”乃顾孔明曰:“丞相深明天文,何故强为?”孔明曰:“天道
变易不常,岂可拘执?吾今且驻军马于汉中,观其动静而后行。”谯周苦谏不从。于是孔明
乃留郭攸之、董允、费祎等为侍中,总摄宫中之事。又留向宠为大将,总督御林军马;蒋琬
为参军;张裔为长史,掌丞相府事;杜琼为谏议大夫;杜微、杨洪为尚书;孟光、来敏为祭
酒;尹默、李譔为博士;郤正、费诗为秘书;谯周为太史。内外文武官僚一百余员,同理蜀
中之事。

孔明受诏归府,唤诸将听令:前督部——镇北将军、领丞相司马、凉州刺史、都亭侯魏
延;前军都督——领扶风太守张翼;牙门将——裨将军王平;后军领兵使——安汉将军、领
建宁太守李恢,副将——定远将军、领汉中太守吕义;兼管运粮左军领兵使——平北将军、
陈仓侯马岱,副将——飞卫将军廖化;右军领兵使——奋威将军、博阳亭侯马忠,抚戎将
军、关内侯张嶷;行中军师——车骑大将军、都乡侯刘琰;中监军——扬武将军邓芝;中参
军——安远将军马谡;前将军——都亭侯袁綝;左将军——高阳侯吴懿;右将军——
玄都侯高翔;后将军——安乐侯吴班;领长史——绥军将军杨仪;前将军——征南将军
刘巴;前护军——偏将军、汉城亭侯许允;左护军——笃信中郎将丁咸;右护军——偏将军
刘敏;后护军——典军中郎将官雝;行参军——昭武中郎将胡济;行参军——谏议将军阎
晏;行参军——偏将军爨习;行参军——裨将军杜义,武略中郎将杜祺,绥戎都尉盛勃;
从事——武略中郎将樊岐;典军书记——樊建;丞相令史——
董厥;帐前左护卫使——龙骧将军关兴;右护卫使——虎翼将军张苞。——以上一应官
员,都随着平北大都督、丞相、武乡侯、领益州牧、知内外事诸葛亮。分拨已定,又檄李严
等守川口以拒东吴。选定建兴五年春三月丙寅日,出师伐魏。

忽帐下一老将,厉声而进曰:“我虽年迈,尚有廉颇之勇,马援之雄。此二古人皆不服
老,何故不用我耶?”众视之,乃赵云也。孔明曰:“吾自平南回都,马孟起病故,吾甚惜
之,以为折一臂也。今将军年纪已高,倘稍有参差,动摇一世英名,减却蜀中锐气。”云厉
声曰:“吾自随先帝以来,临阵不退,遇敌则先。大丈夫得死于疆场者,幸也,吾何恨焉?
愿为前部先锋!”孔明再三苦劝不住。云曰:“如不教我为先锋,就撞死于阶下!”孔明
曰:“将军既要为先锋,须得一人同去。”言未尽,一人应曰:“某虽不才,愿助老将军先
引一军前去破敌。”孔明视之,乃邓芝也。孔明大喜,即拨精兵五千。副将十员,随赵云、
邓芝去讫。

孔明出师,后主引百官送于北门外十里。孔明辞了后主,旌旗蔽野,戈戟如林,率军望
汉中迤逦进发。却说边庭探知此事,报入洛阳。是日曹睿设朝,近臣奏曰:“边官报称:诸
葛亮率领大兵三十余万,出屯汉中,令赵云、邓芝为前部先锋,引兵入境。”睿大惊,问群
臣曰:“谁可为将,以退蜀兵?”忽一人应声而出曰:“臣父死于汉中,切齿之恨,未尝得
报。今蜀兵犯境,臣愿引本部猛将,更乞陛下赐关西之兵,前往破蜀,上为国家效力,下报
父仇,臣万死不恨!”众视之,乃夏侯渊之子夏侯楙也。楙字子休,其性最急,又最吝,自
幼嗣与夏侯惇为子。后夏侯渊为黄忠所斩,曹操怜之,以女清河公主招楙为驸马,因此朝中
钦敬。虽掌兵权,未尝临阵。当时自请出征,曹睿即命为大都督,调关西诸路军马前去迎
敌。司徒王朗谏曰:“不可。夏侯驸马素不曾经战,今付以大任,非其所宜。更兼诸葛亮足
智多谋,深通韧略,不可轻敌。”夏侯楙叱曰:“司徒莫非结连诸葛亮,欲为内应耶?吾自
幼从父学习韬略,深通兵法。汝何欺我年幼?吾若不生擒诸葛亮,誓不回见天子!”王朗等
皆不敢言。夏侯楙辞了魏主,星夜到长安,调关西诸路军马二十余万,来敌孔明。正是:欲
秉白旄摩将士,却教黄吻掌兵权。未知胜负如何,且看下文分解。