\chapter{夺冀州袁尚争锋~决漳河许攸献计}

却说袁尚自斩史涣之后,自负其勇,不待袁谭等兵至,自引兵数万出黎阳,与曹军前队相迎。张辽当先出马,袁尚挺枪来战,不三合,架隔遮拦不住,大败而走。张辽乘势掩杀,袁尚不能主张,急急引军奔回冀州。

袁绍闻袁尚败回,又受了一惊,旧病复发,吐血数斗,昏倒在地。刘夫人慌救入卧内,病势渐危。刘夫人急请审配、逢纪,直至袁绍榻前,商议后事。绍但以手指而不能言。刘夫人曰:“尚可继后嗣否?”绍点头。审配便就榻前写了遗嘱。绍翻身大叫一声,又吐血斗余而死。后人有诗曰:“累世公卿立大名,少年意气自纵横。空招俊杰三千客,漫有英雄百万兵。羊质虎皮功不就,凤毛鸡胆事难成。更怜一种伤心处,家难徒延两弟兄。”袁绍既死,审配等主持丧事。刘夫人便将袁绍所爱宠妾五人尽行杀害;又恐其阴魂于九泉之下再与绍相见,乃髡其发,刺其面,毁其尸:其妒恶如此。袁尚恐宠妾家属为害,并收而杀之。审配、逢纪立袁尚为大司马将军,领冀、青、幽、并四州牧,遣使报丧。此时袁谭已发兵离青州,知父死,便与郭图、辛评商议。图曰:“主公不在冀州,审配、逢纪必立显甫为主矣。当速行。”辛评曰:“审、逢二人,必预定机谋。今若速往,必遭其祸。”袁谭曰:“若此当何如?”郭图曰:“可屯兵城外,观其动静。某当亲往察之。”谭依言。郭图遂入冀州,见袁尚。礼毕,尚问:“兄何不至?”图曰:“因抱病在军中,不能相见。”尚曰:“吾受父亲遗命,立我为主,加兄为车骑将军。目下曹军压境,请兄为前部,吾随后便调兵接应也。”图曰:“军中无人商议良策,愿乞审正南、逢元图二人为辅。”尚曰:“吾亦欲仗此二人早晚画策,如何离得!”图曰:“然则于二人内遣一人去,何如?”尚不得已,乃令二人拈阄,拈着者便去。逢纪拈着,尚即命逢纪赍印绶,同郭图赴袁谭军中。纪随图至谭军,见谭无病,心中不安,献上印绶。谭大怒,欲斩逢纪。郭图密谏曰:“今曹军压境,且只款留逢纪在此,以安尚心。待破曹之后,却来争冀州不迟。”

谭从其言。即时拔寨起行,前至黎阳,与曹军相抵。谭遣大将汪昭出战,操遣徐晃迎敌。二将战不数合,徐晃一刀斩汪昭于马下。曹军乘势掩杀,谭军大败。谭收败军入黎阳,遣人求救于尚。尚与审配计议,只发兵五千余人相助。曹操探知救军已到,遣乐进、李典引兵于半路接着,两头围住尽杀之。袁谭知尚止拨兵五千,又被半路坑杀,大怒,乃唤逢纪责骂。纪曰:“容某作书致主公,求其亲自来救。”谭即令纪作书,遣人到冀州致袁尚,与审配共议。配曰:“郭图多谋,前次不争而去者,为曹军在境也。今若破曹,必来争冀州矣。不如不发救兵,借操之力以除之。”尚从其言,不肯发兵。使者回报,谭大怒,立斩逢纪,议欲降曹。早有细作密报袁尚。尚与审配议曰:“使谭降曹,并力来攻,则冀州危矣。”乃留审配并大将苏由固守冀州,自领大军来黎阳救谭。尚问军中谁敢为前部,大将吕旷、吕翔兄弟二人愿去。尚点兵三万,使为先锋,先至黎阳。谭闻尚自来,大喜,遂罢降曹之议。谭屯兵城中,尚屯兵城外,为掎角之势。

不一日,袁熙、高干皆领军到城外,屯兵三处,每日出兵与操相持。尚屡败,操兵屡胜。至建安八年春二月,操分路攻打,袁谭、袁熙、袁尚、高干皆大败,弃黎阳而走。操引兵追至冀州,谭与尚入城坚守;熙与于离城三十里下寨,虚张声势。操兵连日攻打不下。郭嘉进曰:“袁氏废长立幼,而兄弟之间,权力相并,各自树党,急之则相救,缓之则相争;不如举兵南向荆州,征讨刘表,以候袁氏兄弟之变;变成而后击之,可一举而定也。”操善其言,命贾诩为太守,守黎阳;曹洪引兵守官渡。操引大军向荆州进兵。

谭、尚听知曹军自退,遂相庆贺。袁熙、高干各自辞去。袁谭与郭图、辛评议曰:“我为长子,反不能承父业;尚乃继母所生,反承大爵:心实不甘。”图曰:“主公可勒兵城外,只做请显甫、审配饮酒,伏刀斧手杀之,大事定矣。”谭从其言。适别驾王修自青州来,谭将此计告之。修曰:“兄弟者,左右手也。今与他人争斗,断其右手,而曰我必胜,安可得乎?夫弃兄弟而不亲,天下其谁亲之?彼谗人离间骨肉,以求一朝之利,原塞耳勿听也。”谭怒,叱退王修,使人去请袁尚。尚与审配商议。配曰:“此必郭图之计也。主公若往,必遭奸计;不如乘势攻之。”袁尚依言,便披挂上马,引兵五万出城。袁谭见袁尚引军来,情知事泄,亦即披挂上马,与尚交锋。尚见谭大骂。谭亦骂曰:“汝药死父亲,篡夺爵位,今又来杀兄耶!”二人亲自交锋,袁谭大败。尚亲冒矢石,冲突掩杀。

谭引败军奔平原,尚收兵还。袁谭与郭图再议进兵,令岑璧为将,领兵前来。尚自引兵出冀州。两阵对圆,旗鼓相望。璧出骂阵;尚欲自战,大将吕旷,拍马舞刀,来战岑璧。二将战无数合,旷斩岑璧于马下。谭兵又败,再奔平原。审配劝尚进兵,追至平原。谭抵挡不住,退入平原,坚守不出。尚三面围城攻打。谭与郭图计议。图曰:“今城中粮少,彼军方锐,势不相敌。愚意可遣人投降曹操,使操将兵攻冀州,尚必还救。将军引兵夹击之,尚可擒矣。若操击破尚军,我因而敛其军实以拒操。操军远来,粮食不继,必自退去。我可以仍据冀州,以图进取也。”谭从其言,问曰:“何人可为使?”图曰:“辛评之弟辛毗,字佐治,见为平原令。此人乃能言之士,可命为使。”谭即召辛毗,毗欣然而至。谭修书付毗,使三千军送毗出境。毗星夜赍书往见曹操,时操屯军西平伐刘表,表遣玄德引兵为前部以迎之。未及交锋,辛毗到操寨。见操礼毕,操问其来意,毗具言袁谭相求之意,呈上书信。操看书毕,留辛毗于寨中,聚文武计议。程昱曰:“袁谭被袁尚攻击太急,不得已而来降,不可准信。”吕虔、满宠亦曰:“丞相既引兵至此,安可复舍表而助谭?”荀攸曰:“三公之言未善。以愚意度之:天下方有事,而刘表坐保江、汉之间,不敢展足,其无四方之志可知矣。袁氏据四州之地,带甲数十万,若二子和睦,共守成业,天下事未可知也;今乘其兄弟相攻,势穷而投我,我提兵先除袁尚,后观其变,并灭袁谭,天下定矣。此机会不可失也。”操大喜,便邀辛毗饮酒,谓之曰:“袁谭之降,真耶诈耶?袁尚之兵,果可必胜耶?”毗对曰:“明公勿问真与诈也,只论其势可耳。袁氏连年丧败,兵革疲于外,谋臣诛于内;兄弟谗隙,国分为二;加之饥馑并臻,天灾人困:无问智愚,皆知土崩瓦解,此乃天灭袁氏之时也。今明公提兵攻邺,袁尚不还救,则失巢穴;若还救,则谭踵袭其后。以明公之威,击疲惫之众,如迅风之扫秋叶也。不此之图,而伐荆州;荆州丰乐之地,国和民顺,未可摇动。况四方之患,莫大于河北;河北既平,则霸业成矣。愿明公详之。”操大喜曰:“恨与辛佐治相见之晚也!”即日督军还取冀州。玄德恐操有谋,不跟追袭,引兵自回荆州。

却说袁尚知曹军渡河,急急引军还邺,命吕旷、吕翔断后。袁谭见尚退军,乃大起平原军马,随后赶来。行不到数十里,一声炮响,两军齐出:左边吕旷,右边吕翔,兄弟二人截住袁潭。谭勒马告二将曰:“吾父在日,吾并未慢待二将军,今何从吾弟而见逼耶?”二将闻言,乃下马降谭。谭曰:“勿降我,可降曹承相。”二将因随谭归营。谭候操军至,引二将见操。操大喜,以女许谭为妻,即令吕旷、吕翔为媒。谭请操攻取冀州。操曰:“方今粮草不接,搬运劳苦,我济河,遏淇水入白沟,以通粮道,然后进兵。”令谭且居平原。操引军退屯黎阳,封吕旷、吕翔为列侯,随军听用。

郭图谓袁谭曰:“曹操以女许婚,恐非真意。今又封赏吕旷、吕翔,带去军中,此乃牢笼河北人心。后必终为我祸。主公可刻将军印二颗,暗使人送与二吕,令作内应。待操破了袁尚,可乘便图之。”谭依言,遂刻将军印二颗,暗送与二吕。二吕受讫,径将印来禀曹操。操大笑曰:“谭暗送印者,欲汝等为内助,待我破袁尚之后,就中取事耳。汝等且权受之,我自有主张。”自此曹操便有杀谭之心。

且说袁尚与审配商议:“今曹兵运粮入白沟,必来攻冀州,如之奈何?”配曰:“可发檄使武安长尹楷屯毛城,通上党运粮道;令沮授之子沮鹄守邯郸,遥为声援。主公可进兵平原,急攻袁谭。先绝袁谭,然后破曹。”袁尚大喜,留审配与陈琳守冀州,使马延、张顗二将为先锋,连夜起兵攻打平原。

谭知尚兵来近,告急于操。操曰:“吾今番必得冀州矣。”正说间,适许攸自许昌来;闻尚又攻谭,入见操曰:“丞相坐守于此,岂欲待天雷击杀二袁乎?”操笑曰:“吾已料定矣。”遂令曹洪先进兵攻邺,操自引一军来攻尹楷。兵临本境,楷引军来迎。楷出马,操曰:“许仲康安在?”许褚应声而出,纵马直取尹楷。楷措手不及,被许褚一刀斩于马下,余众奔溃。操尽招降之,即勒兵取邯郸。沮鹄进兵来迎。张辽出马,与鹄交锋。战不三合,鹄大败,辽从后追赶。两马相离不远,辽急取弓射之,应弦落马。操指挥军马掩杀,众皆奔散。

于是操引大军前抵冀州。曹洪已近城下。操令三军绕城筑起土山,又暗掘地道以攻之。审配设计坚守,法令甚严,东门守将冯礼,因酒醉有误巡警,配痛责之。冯礼怀恨,潜地出城降操。操问破城之策,礼曰:“突门内土厚,可掘地道而入。”操便命冯礼引三百壮士,夤夜掘地道而入。却说审配自冯礼出降之后,每夜亲自登城点视军马。当夜在突门阁上,望见城外无灯火。配曰:“冯礼必引兵从地道而入也。”急唤精兵运石击突闸门;门闭,冯礼及三百壮士,皆死于土内。操折了这一场,遂罢地道之计,退军于洹水之上,以候袁尚回兵。袁尚攻平原,闻曹操已破尹楷、沮鹄,大军围困冀州,乃掣兵回救。部将马延曰:“从大路去,曹操必有伏兵;可取小路,从西山出滏水口去劫曹营,必解围也。”尚从其言,自领大军先行,令马延与张顗断后。早有细作去报曹操。操曰:“彼若从大路上来,吾当避之:若从西山小路而来,一战可擒也。吾料袁尚必举火为号,令城中接应。吾可分兵击之。”于是分拨已定。却说袁尚出滏水界口,东至阳平,屯军阳平亭,离冀州十七里,一边靠着滏水。尚令军士堆积柴薪干草,至夜焚烧为号;遣主簿李孚扮作曹军都督,直至城下。大叫:“开门!”审配认得是李孚声音,放入城中,说:“袁尚已陈兵在阳平亭,等候接应,若城中兵出,亦举火为号。”配教城中堆草放火,以通音信。孚曰:“城中无粮,可发老弱残兵并妇人出降;彼必不为备,我即以兵继百姓之后出攻之。”配从其论。

次日,城上竖起白旗,上写“冀州百姓投降。”操曰:“此是城中无粮,教老弱百姓出降,后必有兵出也。”操教张辽、徐晃各引三千军来,伏于两边。操自乘马、张麾盖至城下、果见城门开处,百姓扶老携幼,手持白旗而出。百姓才出尽,城中兵突出。操教将红旗一招,张辽、徐晃两路兵齐出乱杀,城中兵只得复回。操自飞马赶来,到吊桥边,城中弩箭如雨,射中操盔,险透其顶。众将急救回阵。操更衣换马,引众将来攻尚寨,尚自迎敌。时各路军马一齐杀至,两军混战,袁尚大败。尚引败兵退往西山下寨,令人催取马延、张顗军来。不知曹操已使吕旷、吕翔去招安二将。二将随二吕来降,操亦封为列侯。即日进兵攻打西山,先使二吕、马延、张顗截断袁尚粮道。尚情知西山守不住,夜走滥口。安营未定,四下火光并起,伏兵齐出,人不及甲,马不及鞍。尚军大溃,退走五十里,势穷力极,只得遣豫州刺史阴夔至操营请降。操佯许之,却连夜使张辽、徐晃去劫寨。尚尽弃印绶、节钺、衣甲、辎重,望中山而逃。操回军攻冀州。许攸献计曰:“何不决漳河之水以淹之?”操然其计,先差军于城外掘壕堑,周围四十里。审配在城上见操军在城外掘堑,却掘得甚浅。配暗笑曰:“此欲决漳河之水以灌城耳。壕深可灌,如此之浅,有何用哉!”遂不为备。当夜曹操添十倍军士并力发掘,比及天明,广深二丈,引漳水灌之,城中水深数尺。更兼粮绝,军士皆饿死。辛毗在城外,用枪挑袁尚印绶衣服,招安城内之人。审配大怒,将辛毗家屋老小八十余口,就于城上斩之,将头掷下。辛毗号哭不已。审配之侄审荣,素与辛毗相厚,见辛毗家属被害,心中怀忿,乃密写献门之书,拴于箭上,射下城来。军士拾献辛毗,毗将书献操。操先下令:如入冀州,休得杀害袁氏一门老小;军民降者免死。次日天明,审荣大开西门,放曹兵入。辛毗跃马先入,军将随后,杀入冀州。审配在东南城楼上,见操军已入城中,引数骑下城死战,正迎徐晃交马。徐晃生擒审配,绑出城来。路逢辛毗,毗咬牙切齿,以鞭鞭配首曰:“贼杀才!今日死矣!”配大骂:“辛毗贼徒!引曹操破我冀州,我恨不杀汝也!”徐晃解配见操。操曰:“汝知献门接我者乎?”配曰:“不知。”操曰:“此汝侄审荣所献也。”配怒曰:“小儿不行,乃至于此!”操曰:“昨孤至城下,何城中弩箭之多耶?”配曰:“恨少!恨少!”操曰:“卿忠于袁氏,不容不如此。今肯降吾否?”配曰:“不降!不降”辛毗哭拜于地曰:“家属八十余口,尽遭此贼杀害。愿丞相戮之,以雪此恨!”配曰:“吾生为袁氏臣,死为袁氏鬼,不似汝辈谗谄阿谀之贼!可速斩我!”操教牵出。临受刑,叱行刑者曰:“吾主在北,不可使我面南而死!”乃向北跪,引颈就刃。后人有诗叹曰:“河北多名士,谁如审正南:命因昏主丧,心与古人参。忠直言无隐,廉能志不贪。临亡犹北面,降者尽羞惭。”审配既死,操怜其忠义,命葬于城北。

众将请曹操入城。操方欲起行,只见刀斧手拥一人至,操视之,乃陈琳也。操谓之曰:“汝前为本初作檄,但罪状孤可也;何乃辱及祖父耶?”琳答曰:“箭在弦上,不得不发耳。”左右劝操杀之;操怜其才,乃赦之,命为从事。

却说操长子曹丕,字子桓,时年十八岁。丕初生时,有云气一片,其色青紫,圆如车盖,覆于其室,终日不散。有望气者,密谓操曰:“此天子气也。令嗣贵不可言!”丕八岁能属文,有逸才,博古通今,善骑射,好击剑。时操破冀州,不随父在军中,先领随身军,径投袁绍家,下马拔剑而入。有一将当之曰:“丞相有命,诸人不许入绍府。”不叱退,提剑入后堂。见两个妇人相抱而哭,不向前欲杀之。正是:四世公侯已成梦,一家骨肉又遭殃。未知性命如何,且听下文分解。