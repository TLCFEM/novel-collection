\chapter{孔明定计捉张任~杨阜借兵破马超}

却说张飞问计于严颜,颜曰:“从此取雒城,凡守御关隘,都是老夫所管,官军皆出于掌握之中。今感将军之恩,无可以报,老夫当为前部,所到之处,尽皆唤出拜降。”张飞称谢不已。于是严颜为前部,张飞领军随后。凡到之处,尽是严颜所管,都唤出投降。有迟疑未决者,颜曰:“我尚且投降,何况汝乎?”自是望风归顺,并不曾厮杀一场。

却说孔明已将起程日期申报玄德,教都会聚雒城。玄德与众官商议:“今孔明、翼德分两路取川,会于雒城,同入成都。水陆舟车,已于七月二十日起程,此时将及待到。今我等便可进兵。”黄忠曰:“张任每日来搦战,见城中不出,彼军懈怠,不做准备,今日夜间分兵劫寨,胜如白昼厮杀。”玄德从之,教黄忠引兵取左,魏延引兵取右,玄德取中路。当夜二更,三路军马齐发。张任果然不做准备。汉军拥入大寨,放起火来,烈焰腾空。蜀兵奔走,连夜直赶到雒城,城中兵接应入去。玄德还中路下寨;次日,引兵直到雒城,围住攻打。张任按兵不出。攻到第四日,玄德自提一军攻打西门,令黄忠、魏延在东门攻打,留南门北门放军行走。原来南门一带都是山路,北门有涪水:因此不围。张任望见玄德在西门,骑马往来,指挥打城,从辰至未,人马渐渐力乏。张任教吴兰、雷铜二将引兵出北门,转东门,敌黄忠、魏延;自己却引军出南门,转西门,单迎玄德。城内尽拨民兵上城,擂鼓助喊。却说玄德见红日平西,教后军先退。军士方回身,城上一片声喊起,南门内军马突出。张任径来军中捉玄德,玄德军中大乱。黄忠、魏延又被吴兰、雷铜敌住。两下不能相顾。玄德敌不住张任,拨马往山僻小路而走。张任从背后追来,看看赶上。玄德独自一人一马。张任引数骑赶来。玄德正望前尽力加鞭而行,忽山路一军冲来。玄德马上叫苦曰:“前有伏兵,后有追兵,天亡我也!”只见来军当头一员大将,乃是张飞。原来张飞与严颜正从那条路上来,望见尘埃起,知与川兵交战。张飞当先而来,正撞着张任,便就交马。战到十余合,背后严颜引兵大进。张任火速回身。张飞直赶到城下。张任退入城,拽起吊桥。张飞回见玄德曰:“军师溯江而来,尚且未到,反被我夺了头功。”玄德曰:“山路险阻,如何无军阻当,长驱大进,先到于此?”张飞曰:“于路关隘四十五处,皆出老将严颜之功,因此于路并不曾费分毫之力。”遂把义释严颜之事,从头说了一遍,引严颜见玄德。玄德谢曰:“若非老将军,吾弟安能到此?”即脱身上黄金锁子甲以赐之。严颜拜谢。正待安排宴饮,忽闻哨马回报:“黄忠、魏延和川将吴兰、雷铜交锋,城中吴懿、刘璝又引兵助战,两下夹攻,我军抵敌不住,魏、黄二将败阵投东去了。”张飞听得,便请玄德分兵两路,杀去救援。于是张飞在左,玄德在右,杀奔前来。吴懿、刘璝见后面喊声起,慌退入城中。吴兰、雷铜只顾引兵追赶黄忠、魏延,却被玄德、张飞截住归路。黄忠、魏延又回马转攻。吴兰、雷铜料敌不住,只得将本部军马前来投降。玄德准其降,收兵近城下寨。却设张任失了二将,心中忧虑。吴懿、刘璝曰:“兵势甚危,不决一死战,如何得兵退?一面差人去成都见主公告急,一面用计敌之。”张任曰:“吾来日领一军搦战,诈败,引转城北;城内再以一军冲出,截断其中:可获胜也。”吴懿曰:“刘将军相辅公子守城,我引兵冲出助战。”约会已定。次日,张任引数千人马,摇旗呐喊,出城搦战。张飞上马出迎,更不打话,与张任交锋。战不十余合,张任诈败,绕城而走。张飞尽力追之。吴懿一军截住,张任引军复回,把张飞围在垓心,进退不得。正没奈何,只见一队军从江边杀出。当先一员大将,挺枪跃马,与吴懿交锋;只一合,生擒吴懿,战退敌军,救出张飞。视之,乃赵云也。飞问:“军师何在?”云曰:“军师已至,想此时已与主公相见了也。”二人擒吴懿回寨。张任自退入东门去了。

张飞、赵云回寨中,见孔明、简雍、蒋琬已在帐中。飞下马来参军师。孔明惊问曰:“如何得先到?”玄德具述义释严颜之事。孔明贺曰:“张将军能用谋,皆主公之洪福也。”赵云解吴懿见玄德。玄德曰:“汝降否?”吴懿曰:“我既被捉,如何不降?”玄德大喜,亲解其缚。孔明问:“城中有几人守城?”吴懿曰:“有刘季玉之子刘循,辅将刘璝、张任。刘璝不打紧;张任乃蜀郡人,极有胆略,不可轻敌。”孔明曰:“先捉张任,然后取雒城。”问:“城东这座桥名为何桥?”吴懿曰:“金雁桥。”孔明遂乘马至桥边,绕河看了一遍,回到寨中,唤黄忠、魏延听令曰:“离金雁桥南五六里,两岸都是芦苇蒹葭,可以埋伏。魏延引一千枪手伏于左,单戳马上将;黄忠引一千刀手伏于右,单砍坐下马。杀散彼军,张任必投山东小路而来。张翼德引一千军伏在那里,就彼处擒之。”又唤赵云伏于金雁桥北:“待我引张任过桥,你便将桥拆断,却勒兵于桥北,遥为之势,使张任不敢望北走,退投南去,却好中计。”调遣已定,军师自去诱敌。

却说刘璋差卓鹰、张翼二将,前至雒城助战。张任教张翼与刘璝守城,自与卓膺为前后二队,任为前队,膺为后队,出城退敌。孔明引一队不整不齐军,过金雁桥来,与张任对阵。孔明乘四轮车,纶巾羽扇而出,两边百余骑簇捧,遥指张任曰:“曹操以百万之众,闻吾之名,望风而走;今汝何人,敢不投降?”张任看见孔明军伍不齐,在马上冷笑曰:“人说诸葛亮用兵如神,原来有名无实!”把枪一招,大小军校齐杀过来。孔明弃了四轮车,上马退走过桥。张任从背后赶来。过了金雁桥,见玄德军在左,严颜军在右,冲杀将来。张任知是计,急回军时,桥已拆断了;欲投北去,只见赵云一军隔岸摆开,遂不敢投北,径往南绕河而走。走不到五七里,早到芦苇丛杂处。魏延一军从芦中忽起,都用长枪乱戳。黄忠一军伏在芦苇里,用长刀只剁马蹄。马军尽倒,皆被执缚,步军那里敢来?张任引数十骑望山路而走,正撞着张飞。张任方欲退走,张飞大喝一声,众军齐上,将张任活捉了。原来卓膺见张任中计,已投赵云军前降了,一发都到大寨。玄德赏了卓膺。张飞解张任至。孔明亦坐于帐中。玄德谓张任曰:“蜀中诸将,望风而降,汝何不早投降?”张任睁目怒叫曰:“忠臣岂肯事二主乎?”玄德曰:“汝不识天时耳。降即免死。”任曰:“今日便降,久后也不降!可速杀我!”玄德不忍杀之。张任厉声高骂。孔明命斩之以全其名。后人有诗赞曰:“烈士岂甘从二主,张君忠勇死犹生。高明正似天边月,夜夜流光照雒城。”玄德感叹不已,令收其尸首,葬于金雁桥侧,以表其忠。次日,令严颜、吴懿等一班蜀中降将为前部。直至雒城,大叫:“早开门受降,免一城生灵受苦!”刘璝在城上大骂。严颜方待取箭射之,忽见城上一将,拔剑砍翻刘璝,开门投降。玄德军马入雒城,刘循开西门走脱,投成都去了。玄德出榜安民。杀刘璝者,乃武阳人张翼也。

玄德得了雒城,重赏诸将。孔明曰:“雒城已破,成都只在目前;惟恐外州郡不宁,可令张翼、吴懿引赵云抚外水江阳、犍为等处所属州郡,令严颜、卓膺引张飞抚巴西德阳所属州郡,就委官按治平靖,即勒兵回成都取齐。”张飞、赵云领命,各自引兵去了。孔明问:“前去有何处关隘?”蜀中降将曰:“止绵竹有重兵守御;若得绵竹,成都唾手可得。”孔明便商议进兵。法正曰:“雒城既破,蜀中危矣。主公欲以仁义服众,且勿进兵。某作一书上刘璋,陈说利害,璋自然降矣。”孔明曰:“孝直之言最善。”便令写书遣人径往成都。

却说刘循逃回见父,说雒城已陷,刘璋慌聚众官商议。从事郑度献策曰:“今刘备虽攻城夺地,然兵不甚多,士众未附,野谷是资,军无辎重。不如尽驱巴西梓潼民,过涪水以西。其仓鹰野谷,尽皆烧除,深沟高垒,静以待之。彼至请战,勿许。久无所资,不过百日,彼兵自走。我乘虚击之,备可擒也。”刘璋曰:“不然。吾闻拒敌以安民,未闻动民以备敌也。此言非保全之计。”正议间,人报法正有书至。刘璋唤入。呈上书。璋拆开视之。其略曰:“昨蒙遣差结好荆州,不意主公左右不得其人,以致如此。今荆州眷念旧情,不忘族谊。主公若得幡然归顺,量不薄待。望三思裁示。”刘璋大怒,扯毁其书,大骂:“法正卖主求荣,忘恩背义之贼!”逐其使者出城。即时遣妻弟费观,提兵前去守把绵竹。费观举保南阳人姓李,名严,字正方,一同领兵。

当下费观、李严点三万军来守绵竹。益州太守董和,字幼宰,南郡枝江人也,上书与刘璋,请往汉中借兵。璋曰:“张鲁与吾世仇,安肯相救?”和曰:“虽然与我有仇,刘备军在雒城,势在危急,唇亡则齿寒,若以利害说之,必然肯从。”璋乃修书遣使前赴汉中。却说马超自兵败入羌,二载有余,结好羌兵,攻拔陇西州郡。所到之处,尽皆归降;惟冀城攻打不下。刺史韦康,累遣人求救于夏侯渊。渊不得曹操言语,未敢动兵。韦康见救兵不来,与众商议:“不如投降马超。”参军杨阜哭谏曰:“超等叛君之徒,岂可降之?”康曰:“事势至此,不降何待?”阜苦谏不从。韦康大开城门,投拜马超。超大怒曰:“汝今事急请降,非真心也!”将韦康四十余口尽斩之,不留一人。有人言杨阜劝韦康休降,可斩之,超曰:“此人守义,不可斩也。”复用杨阜为参军。阜荐梁宽、赵衢二人,超尽用为军官。

杨阜告马超曰:阜妻死于临洮,乞告两个月假,归葬其妻便回。马超从之。杨阜过历城,来见抚彝将军姜叙。叙与阜是姑表兄弟:叙之母是阜之姑,时年已八十二。当日,杨阜入姜叙内宅,拜见其姑,哭告曰:“阜守城不能保,主亡不能死,愧无面目见姑。马超叛君,妄杀郡守,一州士民,无不恨之。今吾兄坐据历城,竟无讨贼之心,此岂人臣之理乎?”言罢,泪流出血。叙母闻言,唤姜叙入,责之曰:“韦使君遇害,亦尔之罪也。”又谓阜曰:“汝既降人,且食其禄,何故又兴心讨之?”阜曰:“吾从贼者,欲留残生,与主报冤也。”叙曰:“马超英勇,急难图之。”阜曰:“有勇无谋,易图也。吾已暗约下梁宽、赵衢。兄若肯兴兵,二人必为内应。”叙母曰:“汝不早图,更待何时,谁不有死,死于忠义,死得其所也。勿以我为念。汝若不听义山之言,吾当先死,以绝汝念。”

叙乃与统兵校尉尹奉、赵昂商议。原来赵昂之子赵月,现随马超为裨将。赵昂当日应允,归见其妻王氏曰:“吾今日与姜叙、杨阜、尹奉一处商议,欲报韦康之仇。吾想子赵月现随马超,今若兴兵,超必先杀吾子,奈何?”其妻厉声曰:“雪君父之大耻,虽丧身亦不惜,何况一子乎!君若顾子而不行,吾当先死矣!”赵昂乃决。次日一同起兵。姜叙、杨阜屯历城,尹奉、赵昂屯祁山。王氏乃尽将首饰资帛,亲自往祁山军中,赏劳军士,以励其众。

马超闻姜叙、杨阜会合尹奉、赵昂举事,大怒,即将赵月斩之;令庞德、马岱尽起军马,杀奔历城来。姜叙、杨阜引兵出。两阵圆处,杨阜、姜叙衣白袍而出,大骂曰:“叛君无义之贼!”马超大怒,冲将过来,两军混战。姜叙、杨卓如何抵得马超,大败而走。马超驱兵赶来。背后喊声起处,尹奉、赵昂杀来。超急回时,两下夹攻,首尾不能相顾。正斗间,刺斜里大队军马杀来。原来是夏侯渊得了曹操军令,正领军来破马超。超如何当得三路军马,大败奔回。

走了一夜,比及平明,到得翼城叫门时,城上乱箭射下。梁宽、赵衢立在城上,大骂马超;将马超妻杨氏从城上一刀砍了,撇下尸首来;又将马超幼子三人,并至亲十余口,都从城上一刀一个,剁将下来。超气噎塞胸,几乎坠下马来。背后夏侯渊引兵追赶。超见势大,不取恋战,与庞德、马岱杀开一条路走。前面又撞见姜叙、杨阜,杀了一阵;冲得过去,又撞着尹奉、赵昂,杀了一阵;零零落落,剩得五六十骑,连夜奔走,四更前后,走到历城下,守门者只道姜叙兵回,大开门接入。超从城南门边杀起,尽洗城中百姓。至姜叙宅,拿出老母。母全无惧色,指马超而大骂。超大怒,自取剑杀之。尹奉、赵昂全家老幼,亦尽被马超所杀。昂妻王氏因在军中,得免于难。次日,夏侯渊大军至,马超弃城杀出,望西而逃。行不得二十里,前面一军摆开,为首的是杨阜。超切齿而恨,拍马挺枪刺之。阜宗弟七人,一齐来助战。马岱、庞德敌住后军。宗弟七人,皆被马超杀死。阜身中五枪,犹然死战。后面夏侯渊大军赶来,马超遂走。只有庞德、马岱五七骑后随而去。夏侯渊自行安抚陇西诸州人民,令姜叙等各各分守,用车载杨阜赴许都,见曹操。操封阜为关内侯。阜辞曰:“阜无捍难之功,又无死难之节,于法当诛,何颜受职?”操嘉之,卒与之爵。却说马超与庞德、马岱商议,径往汉中投张鲁。张鲁大喜,以为得马超,则西可以吞益州,东可以拒曹操,乃商议欲以女招超为婿。大将杨柏谏曰:“马超妻子遭惨祸,皆超之贻害也。主公岂可以女与之?”鲁从其言,遂罢招婿之议。或以杨柏之言,告知马超。超大怒,有杀杨柏之意。杨柏知之,与兄杨松商议,亦有图马超之心。正值刘璋遣使求救于张鲁,鲁不从。忽报刘璋又遣黄权到。权先来见杨松,说:“东西两川,实为唇齿;西川若破,东川亦难保矣。今若肯相救,当以二十州相酬。”松大喜,即引黄权来见张鲁,说唇齿利害,更以二十州相谢。鲁喜其利,从之。巴西阎圃谏曰:“刘璋与主公世仇,今事急求救,诈许割地,不可从也。”忽阶下一人进曰:“某虽不才,愿乞一旅之师,生擒刘备。务要割地以还。”正是:方看真主来西蜀,又见精兵出汉中。未知其人是谁,且看下文分解。