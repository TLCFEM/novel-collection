\chapter{祢正平裸衣骂贼~吉太医下毒遭刑}

却说曹操欲斩刘岱、王忠。孔融谏曰:“二人本非刘备敌手,若斩之,恐失将士之
心。”操乃免其死,黜罢爵禄。欲自起兵伐玄德。孔融曰:“方今隆冬盛寒,未可动兵,待
来春未为晚也。可先使人招安张绣、刘表,然后再图徐州。”操然其言,先遣刘晔往说张
绣。晔至襄城,先见贾诩,陈说曹公盛德。诩乃留晔于家中。次日来见张绣,说曹公遣刘晔
招安之事。正议间,忽报袁绍有使至。绣命入。使者呈上书信。绣览之,亦是招安之意。诩
问来使曰:“近日兴兵破曹操,胜负何如?”使曰:“隆冬寒月,权且罢兵。今以将军与荆
州刘表俱有国士之风,故来相请耳。”诩大笑曰:“汝可便回见本初,道汝兄弟尚不能容,
何能容天下国士乎!”当面扯碎书,叱退来使。

张绣曰:“方今袁强曹弱;今毁书叱使,袁绍若至,当如之何?”诩曰:“不如去从曹
操。”绣曰:“吾先与操有仇,安得相容?”诩曰:“从操其便有三:夫曹公奉天子明诏,
征伐天下,其宜从一也;绍强盛,我以少从之,必不以我为重,操虽弱,得我必喜,其宜从
二也;曹公王霸之志,必释私怨,以明德于四海,其宜从三也。愿将军无疑焉。”绣从其
言,请刘晔相见。晔盛称操德,且曰:“丞相若记旧怨,安肯使某来结好将军乎?”绣大
喜,即同贾诩等赴许都投降。绣见操,拜于阶下。操忙扶起,执其手曰:“有小过失,勿记
于心。”遂封绣为扬武将军,封贾诩为执金吾使。

操即命绣作书招安刘表。贾诩进曰:“刘景升好结纳名流,今必得一有文名之士往说
之,方可降耳。”操问荀攸曰:“谁人可去?”攸曰:“孔文举可当其任。”操然之。攸出
见孔融曰:“丞相欲得一有文名之士,以备行人之选。公可当此任否?”融曰:“吾友祢
衡,字正平,其才十倍于我。此人宜在帝左右,不但可备行人而已。我当荐之天子。”于是
遂上表奏帝。其文曰:“臣闻洪水横流,帝思俾乂;旁求四方,以招贤俊。昔世宗继统,将
弘基业;畴咨熙载,群士响臻。陛下睿圣,纂承基绪,遭遇厄运,劳谦日昃;维岳降神,异
人并出。窃见处士平原祢衡:年二十四,字正平,淑质贞亮,英才卓跞。初涉艺文,升堂睹
奥;目所一见,辄诵之口,耳所暂闻,不忘于心;性与道合,思若有神;弘羊潜计,安世默
识,以衡准之,诚不足怪。忠果正直,志怀霜雪;见善若惊,嫉恶若仇;任座抗行,史鱼厉
节,殆无以过也。鸷鸟累百,不如一鹗;使衡立朝,必有可观。飞辩骋词,溢气坌涌;解疑
释结,临敌有余。昔贾谊求试属国,诡系单于;终军欲以长缨,牵制劲越:弱冠慷慨,前世
美之。近日路粹、严象,亦用异才,擢拜台郎。衡宜与为比。如得龙跃天衢,振翼云汉,扬
声紫微,垂光虹蜺,足以昭近署之多士,增四门之穆穆。钧天广乐,必有奇丽之观;帝室皇
居,必蓄非常之宝。若衡等辈,不可多得。激楚、阳阿,至妙之容,掌伎者之所贪;飞兔、
騕袅,绝足奔放,良、乐之所急也。臣等区区,敢不以闻?陛下笃慎取士,必须效试,乞令
衡以褐衣召见。如无可观采,臣等受面欺之罪。”帝览表,以付曹操。操遂使人召衡至。礼
毕,操不命坐。祢衡仰天叹曰:“天地虽阔,何无一人也!”操曰:“吾手下有数十人,皆
当世英雄,何谓无人?”衡曰:“愿闻。”操曰:“荀彧、荀攸、郭嘉、程昱,机深智远,
虽萧何、陈平不及也。张辽、许褚、李典、乐进,勇不可当,虽岑彭、马武不及也。吕虔、
满宠为从事,于禁、徐晃为先锋;夏侯惇天下奇才,曹子孝世间福将。安得无人?”衡笑
曰:“公言差矣!此等人物,吾尽识之:荀彧可使吊丧问疾,荀攸可使看坟守墓,程昱可使
关门闭户,郭嘉可使白词念赋,张辽可使击鼓鸣金,许褚可使牧牛放马,乐进可使取状读
招,李典可使传书送檄,吕虔可使磨刀铸剑,满宠可使饮酒食糟,于禁可使负版筑墙,徐晃
可使屠猪杀狗;夏侯惇称为完体将军,曹子孝呼为要钱太守。其余皆是衣架、饭囊、酒桶、
肉袋耳!”操怒曰:“汝有何能?”衡曰:“天文地理,无一不通;三教九流,无所不晓;
上可以致君为尧、舜,下可以配德于孔、颜。岂与俗子共论乎!”时止有张辽在侧,掣剑欲
斩之。操曰:“吾正少一鼓吏;早晚朝贺宴享,可令祢衡充此职。”衡不推辞,应声而去。
辽曰:“此人出言不逊,何不杀之?”操曰:“此人素有虚名,远近所闻。今日杀之,天下
必谓我不能容物。彼自以为能,故令为鼓吏以辱之。”来日,操于省厅上大宴宾客,令鼓吏
挝鼓。旧吏云:“挝鼓必换新衣。”衡穿旧衣而入。遂击鼓为《渔阳三挝》。音节殊妙,渊
渊有金石声。坐客听之,莫不慷慨流涕。左右喝曰:“何不更衣!”衡当面脱下旧破衣服,
裸体而立,浑身尽露。坐客皆掩面。衡乃徐徐着裤,颜色不变。操叱曰:“庙堂之上,何太
无礼?”衡曰:“欺君罔上乃谓无礼。吾露父母之形,以显清白之体耳!”操曰:“汝为清
白,谁为污浊?”衡曰:“汝不识贤愚,是眼浊也;不读诗书,是口浊也;不纳忠言,是耳
浊也;不通古今,是身浊也;不容诸侯,是腹浊也;常怀篡逆,是心浊也!吾乃天下名士,
用为鼓吏,是犹阳货轻仲尼,臧仓毁孟子耳!欲成王霸之业,而如此轻人耶?”

时孔融在坐,恐操杀衡,乃从容进曰:“祢衡罪同胥靡,不足发明王之梦。”操指衡而
言曰:“令汝往荆州为使。如刘表来降,便用汝作公卿。”衡不肯往。操教备马三匹,令二
人扶挟而行;却教手下文武,整酒于东门外送之。荀彧曰:“如祢衡来,不可起身。”衡
至,下马入见,众皆端坐。衡放声大哭。荀彧问曰:“何为而哭?”衡曰:“行于死柩之
中,如何不哭?”众皆曰:“吾等是死尸,汝乃无头狂鬼耳!”衡曰:“吾乃汉朝之臣,不
作曹瞒之党,安得无头?”众欲杀之。荀彧急止之曰:“量鼠雀之辈,何足汗刀!”衡曰:
“吾乃鼠雀,尚有人性;汝等只可谓之蜾虫!”众恨而散。

衡至荆州,见刘表毕,虽颂德,实讥讽。表不喜,令去江夏见黄祖。或问表曰:“祢衡
戏谑主公,何不杀之?”表曰:“祢衡数辱曹操,操不杀者,恐失人望;故令作使于我,欲
借我手杀之,使我受害贤之名也。吾今遣去见黄祖,使曹操知我有识。”众皆称善。时袁绍
亦遣使至。表问众谋士曰:“袁本初又遣使来,曹孟德又差祢衡在此,当从何便?”从事中
郎将韩嵩进曰:“今两雄相持,将军若欲有为,乘此破敌可也。如其不然,将择其善者而从
之。今曹操善能用兵,贤俊多归,其势必先取袁绍,然后移兵向江东,恐将军不能御;莫若
举荆州以附操,操必重待将军矣。”表曰:“汝且去许都,观其动静,再作商议。”嵩曰:
“君臣各有定分。嵩今事将军,虽赴汤蹈火,一唯所命。将军若能上顺天子,下从曹公,使
嵩可也;如持疑未定,嵩到京师,天子赐嵩一官,则嵩为天子之臣,不复为将军死矣。”表
曰:“汝且先往观之。吾别有主意。”

嵩辞表,到许都见操。操遂拜嵩为侍中,领零陵太守。荀彧曰:“韩嵩来观动静,未有
微功,重加此职,祢衡又无音耗,丞相遣而不问,何也?”操曰:“祢衡辱吾太甚,故借刘
表手杀之,何必再问?”遂遣韩嵩回荆州说刘表。

嵩回见表,称颂朝廷盛德,劝表遣子入侍,表大怒曰:“汝怀二心耶!”欲斩之。嵩大
叫曰:“将军负嵩,焉不负将军!”蒯良曰:“嵩未去之前,先有此言矣。”刘表遂赦之。

人报黄祖斩了祢衡,表问其故,对曰:“黄祖与祢衡共饮,皆醉。祖问衡曰:‘君在许
都有何人物?’衡曰:‘大儿孔文举,小儿杨德祖。除此二人,别无人物。’祖曰:‘似我
何如?’衡曰:‘汝似庙中之神,虽受祭祀,恨无灵验!’祖大怒曰:“汝以我为土木偶人
耶!’遂斩之。衡至死骂不绝口,”刘表闻衡死,亦嗟呀不已,令葬于鹦鹉洲边。后人有诗
叹曰:“黄祖才非长者俦,祢衡珠碎此江头。今来鹦鹉洲边过,惟有无情碧水流。”却说曹
操知祢衡受害,笑曰:“腐儒舌剑,反自杀矣!”因不见刘表来降,便欲兴兵问罪。荀彧谏
曰:“袁绍未平,刘备未灭,而欲用兵江汉,是犹舍心腹而顺手足也。可先灭袁绍,后灭刘
备,江汉可一扫而平矣。”操从之。

且说董承自刘玄德去后,日夜与王子服等商议,无计可施。建安五年,元旦朝贺,见曹
操骄横愈甚,感愤成疾。帝知国舅染病,令随朝太医前去医治。此医乃洛阳人,姓吉,名
太,字称平,人皆呼为吉平,当时名医也。平到董承府用药调治,旦夕不离;常见董承长吁
短叹,不敢动问。

时值元宵,吉平辞去,承留住,二人共饮。饮至更余,承觉困倦,就和衣而睡。忽报王
子服等四人至,承出接入。服曰:“大事谐矣!”承曰:“愿闻其说。”服曰:“刘表结连
袁绍,起兵五十万,共分十路杀来。马腾结连韩遂,起西凉军七十二万,从北杀来。曹操尽
起许昌兵马,分头迎敌,城中空虚。若聚五家僮仆,可得千余人。乘今夜府中大宴,庆赏元
宵,将府围住,突入杀之。不可失此机会!”承大喜,即唤家奴各人收拾兵器,自己披挂绰
枪上马,约会都在内门前相会,同时进兵。夜至二鼓,众兵皆到。董承手提宝剑,徒步直
入,见操设宴后堂,大叫:“操贼休走!”一剑剁去,随手而倒。霎时觉来,乃南柯一梦,
口中犹骂“操贼”不止。

吉平向前叫曰:“汝欲害曹公乎?”承惊惧不能答。吉平曰:“国舅休慌。某虽医人,
未尝忘汉。某连日见国舅嗟叹,不敢动问。恰才梦中之言,已见真情,幸勿相瞒。倘有用某
之处,虽灭九族,亦无后悔!”承掩面而哭曰:“只恐汝非真心!”平遂咬下一指为誓。承
乃取出衣带诏,令平视之;且曰:“今之谋望不成者,乃刘玄德、马腾各自去了,无计可
施,因此感而成疾。”平曰:“不消诸公用心。操贼性命,只在某手中。”承问其故。平
曰:“操贼常患头风,痛入骨髓;才一举发,便召某医治。如早晚有召,只用一服毒药,必
然死矣,何必举刀兵乎?”承曰:“若得如此,救汉朝社稷者,皆赖君也!”时吉平辞归。
承心中暗喜,步入后堂,忽见家奴秦庆童同侍妾云英在暗处私语。承大怒,唤左右捉下,欲
杀之。夫人劝免其死,各人杖脊四十,将庆童锁于冷房。庆童怀恨,夤夜将铁锁扭断,跳墙
而出,径入曹操府中,告有机密事。操唤入密室问之。庆童云:“王子服、吴子兰、种辑、
吴硕、马腾五人在家主府中商议机密,必然是谋丞相。家主将出白绢一段,不知写着甚的。
近日吉平咬指为誓,我也曾见。”曹操藏匿庆童于府中,董承只道逃往他方去了,也不追
寻。

次日,曹操诈患头风,召吉平用药。平自思曰:“此贼合休!”暗藏毒药入府。操卧于
床上,令平下药。平曰:“此病可一服即愈。”教取药罐,当面煎之。药已半干,平已暗下
毒药,亲自送上。操知有毒,故意迟延不服。平曰:“乘热服之,少汗即愈。”操起曰:
“汝既读儒书,必知礼义:君有疾饮药,臣先尝之;父有疾饮药,子先尝之。汝为我心腹之
人,何不先尝而后进?”平曰:“药以治病,何用人尝?”平知事已泄,纵步向前,扯住操
耳而灌之。操推药泼地,砖皆迸裂。

操未及言,左右已将吉平执下。操曰:“吾岂有疾,特试汝耳!汝果有害我之心!”遂
唤二十个精壮狱卒,执平至后园拷问。操坐于亭上,将平缚倒于地。吉平面不改容,略无惧
怯。操笑曰:“量汝是个医人,安敢下毒害我?必有人唆使你来。你说出那人,我便饶
你。”平叱之曰:“汝乃欺君罔上之贼,天下皆欲杀汝,岂独我乎!”操再三磨问。平怒
曰:“我自欲杀汝,安有人使我来?今事不成,惟死而已!”操怒,教狱卒痛打。打到两个
时辰,皮开肉裂,血流满阶。操恐打死,无可对证,令狱卒揪去静处,权且将息。

传令次日设宴,请众大臣饮酒。惟董承托病不来。王子服等皆恐操生疑,只得俱至。操
于后堂设席。酒行数巡,曰:“筵中无可为乐,我有一人,可为众官醒酒。”教二十个狱
卒:“与吾牵来!”须臾,只见一长枷钉着吉平,拖至阶下。操曰:“众官不知,此人连结
恶党,欲反背朝廷,谋害曹某;今日天败,请听口词。”操教先打一顿,昏绝于地,以水喷
面。吉平苏醒,睁目切齿而骂曰:“操贼!不杀我,更待何时!”操曰:“同谋者先有六
人。与汝共七人耶?”平只是大骂。王子服等四人面面相觑,如坐针毡。操教一面打,一面
喷。平并无求饶之意。操见不招,且教牵去。

众官席散,操只留王子服等四人夜宴。四人魂不附体,只得留待。操曰:“本不相留,
争奈有事相问。汝四人不知与董承商议何事?”子服曰:“并未商议甚事。”操曰:“白绢
中写着何事?”子服等皆隐讳。操教唤出庆童对证。子服曰:“汝于何处见来?”庆童曰:
“你回避了众人,六人在一处画字,如何赖得?”子服曰:“此贼与国舅侍妾通奸,被责诬
主,不可听也。”操曰:“吉平下毒,非董承所使而谁?”子服等皆言不知。操曰:“今晚
自首,尚犹可恕:若待事发,其实难容!”子服等皆言并无此事。操叱左右将四人拿住监
禁。

次日,带领众人径投董承家探病。承只得出迎。操曰:“缘何夜来不赴宴?”承曰:
“微疾未痊,不敢轻出。”操曰:“此是忧国家病耳。”承愕然。操曰:“国舅知吉平事
乎?”承曰:“不知。”操冷笑曰:“国舅如何不知?”唤左右:“牵来与国舅起病。”承
举措无地。须臾,二十狱卒推吉平至阶下。吉平大骂:“曹操逆贼!”操指谓承曰:“此人
曾攀下王子服等四人,吾已拿下廷尉。尚有一人,未曾捉获。”因问平曰:“谁使汝来药
我?可速招出!”平曰:“天使我来杀逆贼!”操怒教打。身上无容刑之处。承在座视之,
心如刀割。操又问平曰:“你原有十指,今如何只有九指?”平曰:“嚼以为誓,誓杀国
贼!”操教取刀来,就阶下截去其九指,曰:“一发截了,教你为誓!”平曰:“尚有口可
以吞贼,有舌可以骂贼!”操令割其舌。平曰:“且勿动手。吾今熬刑不过,只得供招。可
释吾缚。”操曰:“释之何碍?”遂命解其缚。平起身望阙拜曰:“臣不能为国家除贼,乃
天数也!”拜毕,撞阶而死。操令分其肢体号令。时建安五年正月也。史官有诗曰:“汉朝
无起色,医国有称平:立誓除奸党,捐躯报圣明。极刑词愈烈,惨死气如生。十指淋漓处,
千秋仰异名。”

操见吉平已死,教左右牵过秦庆童至面前。操曰:“国舅认得此人否?”承大怒曰:
“逃奴在此,即当诛之!”操曰:“他首告谋反,今来对证,谁敢诛之?”承曰:“丞相何
故听逃奴一面之说?”操曰:“王子服等吾已擒下,皆招证明白,汝尚抵赖乎?”即唤左右
拿下,命从人直入董承卧房内,搜出衣带诏并义状。操看了,笑曰:“鼠辈安敢如此!”遂
命:“将董承全家良贱,尽皆监禁,休教走脱一个。”操回府以诏状示众谋士商议,要废献
帝,更立新君。正是:数行丹诏成虚望,一纸盟书惹祸殃。未知献帝性命如何,且听下文分
解。