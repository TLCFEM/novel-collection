\chapter{驱巨兽六破蛮兵~烧藤甲七擒孟获}

却说孔明放了孟获等一干人,杨锋父子皆封官爵,重赏洞兵。杨锋等拜谢而去。孟获等
连夜奔回银坑洞。那洞外有三江:乃是泸水、甘南水、西城水。三路水会合,故为三江。其
洞北近平坦三百余里,多产万物。洞西二百里,有盐井。西南二百里,直抵泸、甘。正南三
百里,乃是梁都洞,洞中有山,环抱其洞;山上出银矿,故名为银坑山。山中置宫殿楼台,
以为蛮王巢穴。其中建一祖庙,名曰“家鬼”。四时杀牛宰马享祭,名为“卜鬼”。每年常
以蜀人并外乡之人祭之。若人患病,不肯服药,只祷师巫,名为“药鬼”。其处无刑法,但
犯罪即斩。有女长成,却于溪中沐浴,男女自相混淆,任其自配,父母不禁,名为“学
艺”。年岁雨水均调,则种稻谷;倘若不熟,杀蛇为羹,煮象为饭。每方隅之中,上户号曰
“洞主”,次曰“酋长”。每月初一、十五两日,皆在三江城中买卖,转易货物。其风俗如
此。

却说孟获在洞中,聚集宗党千余人,谓之曰:“吾屡受辱于蜀兵,立誓欲报之。汝等有
何高见?”言未毕,一人应曰:“吾举一人,可破诸葛亮。”众视之,乃孟获妻弟,现为八
番部长,名曰带来洞主。获大喜,急问何人。带来洞主曰:“此去西南八纳洞,洞主木鹿大
王,深通法术:出则骑象,能呼风唤雨,常有虎豹豺狼、毒蛇恶蝎跟随。手下更有三万神
兵,甚是英勇。大王可修书具礼,某亲往求之。此人若允,何惧蜀兵哉!”获忻然,令国舅
赍书而去。却令朵思大王守把三江城,以为前面屏障。却说孔明提兵直至三江城,遥望见此
城三面傍江,一面通旱;即遣魏延、赵云同领一军,于旱路打城。军到城下时,城上弓弩齐
发:原来洞中之人,多习弓弩,一弩齐发十矢,箭头上皆用毒药;但有中箭者,皮肉皆烂,
见五脏而死。赵云、魏延不能取胜,回见孔明,言药箭之事。孔明自乘小车,到军前看了虚
实,回到寨中,令军退数里下寨。蛮兵望见蜀兵远退,皆大笑作贺,只疑蜀兵惧怯而退,因
此夜间安心稳睡,不去哨探。却说孔明约军退后,即闭寨不出。一连五日,并无号令。黄昏
左侧,忽起微风。孔明传令曰:“每军要衣襟一幅,限一更时分应点。无者立斩。”诸将皆
不知其意,众军依令预备。初更时分,又传令曰:“每军衣襟一幅,包土一包。无者立
斩。”众军亦不知其意,只得依令预备。孔明又传令曰:“诸军包土,俱在三江城下交割。
先到者有赏。”众军闻令,皆包净土,飞奔城下。孔明令积土为蹬道,先上城者为头功。于
是蜀兵十余万,并降兵万余,将所包之土,一齐弃于城下。一霎时,积土成山,接连城上。
一声暗号,蜀兵皆上城。蛮兵急放弩时,大半早被执下,余者弃城而走。朵思大王死于乱军
之中。蜀将督军分路剿杀。孔明取了三江城,所得珍宝,皆赏三军。败残蛮兵逃回见孟获
说:“朵思大王身死。失了三江城。”获大惊。正虑之间,人报蜀兵已渡江,现在本洞前下
寨。孟获甚是慌张。忽然屏风后一人大笑而出曰:“既为男子,何无智也?我虽是一妇人,
愿与你出战。”获视之,乃妻祝融夫人也。夫人世居南蛮,乃祝融氏之后;善使飞刀,百发
百中。孟获起身称谢。夫人忻然上马,引宗党猛将数百员、生力洞兵五万,出银坑宫阙,来
与蜀兵对敌。方才转过洞口,一彪军拦住:为首蜀将,乃是张嶷。蛮兵见之,却早两路摆
开。祝融夫人背插五口飞刀,手挺丈八长标,坐下卷毛赤兔马。张嶷见之,暗暗称奇。二人
骤马交锋。战不数合,夫人拨马便走。张嶷赶去,空中一把飞刀落下。嶷急用手隔,正中左
臂,翻身落马。蛮兵发一声喊,将张嶷执缚去了。马忠听得张嶷被执,急出救时,早被蛮兵
捆住。望见祝融夫人挺标勒马而立,忠忿怒向前去战,坐下马绊倒,亦被擒了。都解入洞中
来见孟获。获设席庆贺。夫人叱刀斧手推出张嶷、马忠要斩。获止曰:“诸葛亮放吾五次,
今番若杀彼将,是不义也。且囚在洞中,待擒住诸葛亮,杀之未迟。”夫人从其言,笑饮作
乐。

却说败残兵来见孔明,告知其事。孔明即唤马岱、赵云、魏延三人受计,各自领军前
去。次日,蛮兵报入洞中,说赵云搦战。祝融夫人即上马出迎。二人战不数合,云拨马便
走。夫人恐有埋伏,勒兵而回。魏延又引军来搦战,夫人纵马相迎。正交锋紧急,延诈败而
逃,夫人只不赶。次日,赵云又引军来搦战,夫人领洞兵出迎。二人战不数合,云诈败而
走,夫人按标不赶。欲收兵回洞时,魏延引军齐声辱骂,夫人急挺标来取魏延。延拨马便
走。夫人忿怒赶来,延骤马奔入山僻小路。忽然背后一声响亮,延回头视之,夫人仰鞍落
马:原来马岱埋伏在此,用绊马索绊倒。就里擒缚,解投大寨而来。蛮将洞兵皆来救时,赵
云一阵杀散。孔明端坐于帐上,马岱解祝融夫人到,孔明急令武士去其缚,请在别帐赐酒压
惊,遣使往告孟获,欲送夫人换张嶷、马忠二将。

孟获允诺,即放出张嶷、马忠,还了孔明。孔明遂送夫人入洞。孟获接入,又喜又恼。
忽报八纳洞主到。孟获出洞迎接,见其人骑着白象,身穿金珠缨络,腰悬两口大刀,领着一
班喂养虎豹豺狼之士,簇拥而入。获再拜哀告,诉说前事。木鹿大王许以报仇。获大喜,设
宴相待。次日,木鹿大王引本洞兵带猛兽而出。赵云、魏延听知蛮兵出,遂将军马布成阵
势。二将并辔立于阵前视之,只见蛮兵旗帜器械皆别:人多不穿衣甲,尽裸身赤体,面目丑
陋;身带四把尖刀;军中不鸣鼓角,但筛金为号;木鹿大王腰挂两把宝刀,手执蒂钟,身骑
白象,从大旗中而出。赵云见了,谓魏延曰:“我等上阵一生,未尝见如此人物。”二人正
沉吟之际,只见木鹿大王口中不知念甚咒语,手摇蒂钟。忽然狂风大作,飞砂走石,如同骤
雨;一声画角响,虎豹豺狼,毒蛇猛兽,乘风而出,张牙舞爪,冲将过来。蜀兵如何抵当,
往后便退。蛮兵随后追杀,直赶到三江界路方回。赵云、魏延收聚败兵,来孔明帐前请罪,
细说此事。孔明笑曰:“非汝二人之罪。吾未出茅庐之时,先知南蛮有驱虎豹之法。吾在蜀
中已办下破此阵之物也:随军有二十辆车,俱封记在此。今日且用一半;留下一半,后有别
用。”遂令左右取了十辆红油柜车到帐下,留十辆黑油柜车在后。众皆不知其意。孔明将柜
打开,皆是木刻彩画巨兽,俱用五色绒线为毛衣,钢铁为牙爪,一个可骑坐十人。孔明选了
精壮军士一千余人,领了一百,口内装烟火之物,藏在军中。次日,孔明驱兵大进,布于洞
口。蛮兵探知,入洞报与蛮王。木鹿大王自谓无敌,即与孟获引洞兵而出。孔明纶巾羽扇,
身衣道袍,端坐于车上。孟获指曰:“车上坐的便是诸葛亮!若擒住此人,大事定矣!”木
鹿大王口中念咒,手摇蒂钟。顷刻之间,狂风大作,猛兽突出。孔明将羽扇一摇,其风便回
吹彼阵中去了,蜀阵中假兽拥出。蛮洞真兽见蜀阵巨兽口吐火焰,鼻出黑烟,身摇铜铃,张
牙舞爪而来,诸恶兽不敢前进,皆奔回蛮洞,反将蛮兵冲倒无数。孔明驱兵大进,鼓角齐
鸣,望前追杀。木鹿大王死于乱军之中。洞内孟获宗党,皆弃宫阙,扒山越岭而走。孔明大
军占了银坑洞。

次日,孔明正要分兵缉擒孟获,忽报:“蛮王孟获妻弟带来洞主,因劝孟获归降,获不
从,今将孟获并祝融夫人及宗党数百余人尽皆擒来,献与丞相。”孔明听知,即唤张嶷、马
忠,分付如此如此。二将受了计,引二千精壮兵,伏于两廊。孔明即令守门将,俱放进来。
带来洞主引刀斧手解孟获等数百人,拜于殿下。孔明大喝曰:“与吾擒下!”两廊壮兵齐
出,二人捉一人,尽被执缚。孔明大笑曰:“量汝些小诡计,如何瞒得过我!汝见二次俱是
本洞人擒汝来降,吾不加害;汝只道吾深信,故来诈降,欲就洞中杀吾!”喝令武士搜其身
畔,果然各带利刀。孔明问孟获曰:“汝原说在汝家擒住,方始心服;今日如何?”获曰:
“此是我等自来送死,非汝之能也。吾心未服。”孔明曰:“吾擒住六番,尚然不服,欲待
何时耶?”获曰:“汝第七次擒住,吾方倾心归服,誓不反矣。”孔明曰:“巢穴已破,吾
何虑哉!”令武士尽去其缚,叱之曰:“这番擒住,再若支吾,必不轻恕!”孟获等抱头鼠
窜而去。

却说败残蛮兵有千余人,大半中伤而逃,正遇蛮王孟获。获收了败兵,心中稍喜,却与
带来洞主商议曰:“吾今洞府已被蜀兵所占,今投何地安身?”带来洞主曰:“止有一国可
以破蜀。”获喜曰:“何处可去?”带来洞主曰:“此去东南七百里,有一国,名乌戈国。
国主兀突骨,身长丈二,不食五谷,以生蛇恶兽为饭;身有鳞甲,刀箭不能侵。其手下军
士,俱穿藤甲;其藤生于山涧之中,盘于石壁之上;国人采取,浸于油中,半年方取出晒
之;晒干复浸,凡十余遍,却才造成铠甲;穿在身上,渡江不沉,经水不湿,刀箭皆不能
入:因此号为藤甲军。今大王可往求之。若得彼相助,擒诸葛亮如利刀破竹也。”孟获大
喜,遂投乌戈国,来见兀突骨。其洞无宇舍,皆居土穴之内。孟获入洞,再拜哀告前事。兀
突骨曰:“吾起本洞之兵,与汝报仇。”获欣然拜谢。于是兀突骨唤两个领兵俘长:一名土
安,一名奚泥,起三万兵,皆穿藤甲,离乌戈国望东北而来。行至一江,名桃花水,两岸有
桃树,历年落叶于水中,若别国人饮之尽死,惟乌戈国人饮之,倍添精神。兀突骨兵至桃花
渡口下寨,以待蜀兵。

却说孔明令蛮人哨探孟获消息,回报曰:“孟获请乌戈国主,引三万藤甲军,现屯于桃
花渡口。孟获又在各番聚集蛮兵,并力拒战。”孔明听说,提兵大进,直至桃花渡口。隔岸
望见蛮兵,不类人形,甚是丑恶;又问土人,言说即日桃叶正落,水不可饮。孔明退五里下
寨,留魏延守寨。

次日,乌戈国主引一彪藤甲军过河来,金鼓大震。魏延引兵出迎。蛮兵卷地而至。蜀兵
以弩箭射到藤甲之上,皆不能透,俱落于地;刀砍枪刺,亦不能入。蛮兵皆使利刀钢叉,蜀
兵如何抵当,尽皆败走。蛮兵不赶而回。魏延复回,赶到桃花渡口,只见蛮兵带甲渡水而
去;内有困乏者,将甲脱下,放在水面,以身坐其上而渡。魏延急回大寨,来禀孔明,细言
其事。孔明请吕凯并土人问之。凯曰:“某素闻南蛮中有一乌戈国,无人伦者也。更有藤甲
护身,急切难伤。又有桃叶恶水,本国人饮之,反添精神;别国人饮之即死:如此蛮方,纵
使全胜,有何益焉?不如班师早回。”孔明笑曰:“吾非容易到此,岂可便去!吾明日自有
平蛮之策。”于是令赵云助魏延守寨,且休轻出。次日,孔明令土人引路,自乘小车到桃花
渡口北岸山僻去处,遍观地理。山险岭峻之处,车不能行,孔明弃车步行。忽到一山,望见
一谷,形如长蛇,皆光峭石壁,并无树木,中间一条大路。孔明问土人曰:“此谷何名?”
土人答曰:“此处名为盘蛇谷。出谷则三江城大路,谷前名塔郎甸。”孔明大喜曰:“此乃
天赐吾成功于此也!”遂回旧路,上车归寨,唤马岱分付曰:“与汝黑油柜车十辆,须用竹
竿千条,柜内之物,如此如此。可将本部兵去把住盘蛇谷两头,依法而行。与汝半月限,一
切完备。至期如此施设。倘有走漏,定按军法。”马岱受计而去。又唤赵云分付曰:“汝去
盘蛇谷后,三江大路口如此守把。所用之物,克日完备。”赵云受计而去。又唤魏延分付
曰:“汝可引本部兵去桃花渡口下寨。如蛮兵渡水来敌,汝便弃了寨,望白旗处而走。限半
个月内,须要连输十五阵,弃七个寨栅。若输十四阵,也休来见我。”魏延领命,心中不
乐,怏怏而去。孔明又唤张翼另引一军,依所指之处,筑立寨栅去了;却令张嶷、马忠引本
洞所降千人,如此行之。各人都依计而行。却说孟获与乌戈国主兀突骨曰:“诸葛亮多有巧
计,只是埋伏。今后交战,分付三军:但见山谷之中,林木多处,不可轻进。”兀突骨曰:
“大王说的有理。吾已知道中国人多行诡计。今后依此言行之。吾在前面厮杀;汝在背后教
道。”两人商议已定。忽报蜀兵在桃花渡口北岸立起营寨。兀突骨即差二俘长引藤甲军渡了
河,来与蜀兵交战。不数合,魏延败走。蛮兵恐有埋伏,不赶自回。次日,魏延又去立了营
寨。蛮兵哨得,又引众军渡过河来战。延出迎之。不数合,延败走。蛮兵追杀十余里,见四
下并无动静,便在蜀寨中屯住。次日,二俘长请兀突骨到寨,说知此事。兀突骨即引兵大
进,将魏延追一阵。蜀兵皆弃甲抛戈而走,只见前有白旗。延引败兵,急奔到白旗处,早有
一寨,就寨中屯住。兀突骨驱兵追至,魏延引兵弃寨而走。蛮兵得了蜀寨。次日,又望前追
杀。魏延回兵交战,不三合又败,只看白旗处而走,又有一寨,延就寨屯住。次日,蛮兵又
至。延略战又走。蛮兵占了蜀寨。

话休絮烦,魏延且战且走,已败十五阵,连弃七个营寨。蛮兵大进追杀。兀突骨自在军
前破敌,于路但见林木茂盛之处,便不敢进;却使人远望,果见树阴之中,旌旗招飐。兀突
骨谓孟获曰:“果不出大王所料。”孟获大笑曰:“诸葛亮今番被吾识破!大王连日胜了他
十五阵,夺了七个营寨,蜀兵望风而走。诸葛亮已是计穷;只此一进,大事定矣!”兀突骨
大喜,遂不以蜀兵为念。至第十六日,魏延引败残兵,来与藤甲军对敌,兀突骨骑象当先,
头戴日月狼须帽,身披金珠缨络,两肋下露出生鳞甲,眼目中微有光芒,手指魏延大骂。延
拨马便走。后面蛮兵大进。魏延引兵转过了盘蛇谷,望白旗而走。兀突骨统引兵众,随后追
杀。兀突骨望见山上并无草木,料无埋伏,放心追杀。赶到谷中,见数十辆黑油柜车在当
路。蛮兵报曰:“此是蜀兵运粮道路,因大王兵至,撇下粮车而走。”兀突骨大喜,催兵追
赶。将出谷口,不见蜀兵,只见横木乱石滚下,垒断谷口。兀突骨令兵开路而进,忽见前面
大小车辆,装载干柴,尽皆火起。兀突骨忙教退兵,只闻后军发喊,报说谷口已被干柴垒
断,车中原来皆是火药,一齐烧着。兀突骨见无草木,心尚不慌,令寻路而走。只见山上两
边乱丢火把,火把到处,地中药线皆着,就地飞起铁炮。满谷中火光乱舞,但逢藤甲,无有
不着。将兀突骨并三万藤甲军,烧得互相拥抱,死于盘蛇谷中。孔明在山上往下看时,只见
蛮兵被火烧的伸拳舒腿,大半被铁炮打的头脸粉碎,皆死于谷中,臭不可闻。孔明垂泪而叹
曰:“吾虽有功于社稷,必损寿矣!”左右将士,无不感叹。

却说孟获在寨中,正望蛮兵回报。忽然千余人笑拜于寨前,言说:“乌戈国兵与蜀兵大
战,将诸葛亮围在盘蛇谷中了。特请大王前去接应。我等皆是本洞之人,不得已而降蜀;今
知大王前到,特来助战。”孟获大喜,即引宗党并所聚番人,连夜上马;就令蛮兵引路。方
到盘蛇谷时,只见火光甚起,臭气难闻。获知中计,急退兵时,左边张嶷,右边马忠,两路
军杀出。获方欲抵敌,一声喊起,蛮兵中大半皆是蜀兵,将蛮王宗党并聚集的番人,尽皆擒
了。孟获匹马杀出重围,望山径而走。

正走之间,见山凹里一簇人马,拥出一辆小车;车中端坐一人,纶巾羽扇,身衣道袍,
乃孔明也。孔明大喝曰:“反贼孟获!今番如何?”获急回马走。旁边闪过一将,拦住去
路,乃是马岱。孟获措手不及,被马岱生擒活捉了。此时王平、张翼已引一军赶到蛮寨中,
将祝融夫人并一应老小皆活捉而来。

孔明归到寨中,升帐而坐,谓众将曰:“吾今此计,不得已而用之,大损阴德。我料敌
人必算吾于林木多处埋伏,吾却空设旌旗,实无兵马,疑其心也。吾令魏文长连输十五阵
者,坚其心也。吾见盘蛇谷止一条路,两壁厢皆是光石,并无树木,下面都是沙土,因令马
岱将黑油柜安排于谷中,车中油柜内,皆是预先造下的火炮,名曰‘地雷’,一炮中藏九
炮,三十步埋之,中用竹竿通节,以引药线;才一发动,山损石裂。吾又令赵子龙预备草
车,安排于谷中。又于山上准备大木乱石。却令魏延赚兀突骨并藤甲军入谷,放出魏延,即
断其路,随后焚之。吾闻:‘利于水者必不利于火。’藤甲虽刀箭不能入,乃油浸之物,见
火必着。蛮兵如此顽皮,非火攻安能取胜?使乌戈国之人不留种类者,是吾之大罪也!”众
将拜伏曰:“丞相天机,鬼神莫测也!”孔明令押过孟获来。孟获跪于帐下。孔明令去其
缚,教且在别帐与酒食压惊。孔明唤管酒食官至坐榻前,如此如此,分付而去。却说孟获与
祝融夫人并孟优、带来洞主、一切宗党在别帐饮酒。忽一人人帐谓孟获曰:“丞相面羞,不
欲与公相见。特令我来放公回去,再招人马来决胜负。公今可速去。”孟获垂泪言曰:“七
擒七纵,自古未尝有也。吾虽化外之人,颇知礼义,直如此无羞耻乎?”遂同兄弟妻子宗党
人等,皆匍匐跪于帐下,肉袒谢罪曰:“丞相天威,南人不复反矣!”孔明曰:“公今服
乎?”获泣谢曰:“某子子孙孙皆感覆载生成之恩,安得不服!”孔明乃请孟获上帐,设宴
庆贺,就令永为洞主。所夺之地,尽皆退还。孟获宗党及诸蛮兵,无不感戴,皆欣然跳跃而
去。后人有诗赞孔明曰:“羽扇纶巾拥碧幢,七擒妙策制蛮王。至今溪洞传威德,为选高原
立庙堂。”

长史费祎入谏曰:“今丞相亲提士卒,深入不毛,收服蛮方;目今蛮王既已归服,何不
置官吏,与孟获一同守之?”孔明曰:“如此有三不易:留外人则当留兵,兵无所食,一不
易也;蛮人伤破,父兄死亡,留外人而不留兵,必成祸患,二不易也;蛮人累有废杀之罪,
自有嫌疑,留外人终不相信,三不易也。今吾不留人,不运粮,与相安于无事而已。”众人
尽服。于是蛮方皆感孔明恩德,乃为孔明立生祠,四时享祭,皆呼之为慈父;各送珍珠金
宝、丹漆药材、耕牛战马,以资军用,誓不再反。南方已定。

却说孔明犒军已毕,班师回蜀,令魏延引本部兵为前锋。延引兵方至泸水,忽然阴云四
合,水面上一阵狂风骤起,飞沙走石,军不能进。延退兵回报孔明。孔明遂请孟获问之。正
是:塞外蛮人方帖服,水边鬼卒又猖狂。未知孟获所言若何,且看下文分解。