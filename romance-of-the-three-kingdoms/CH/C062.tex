\chapter{取涪关杨高授首~攻雒城黄魏争功}

却说张昭献计曰:“且休要动兵。若一兴师,曹操必复至。不如修书二封:一封与刘
璋,言刘备结连东吴,共取西川,使刘璋心疑而攻刘备;一封与张鲁,教进兵向荆州来。着
刘备首尾不能救应。我然后起兵取之,事可谐矣。”权从之,即发使二处去讫。且说玄德在
葭萌关日久,甚得民心。忽接得孔明文书。知孙夫人已回东吴。又闻曹操兴兵犯濡须,乃与
庞统议曰:“曹操击孙权,操胜必将取荆州,权胜亦必取荆州矣。为之奈何?”庞统曰:
“主公勿忧。有孔明在彼,料想东吴不敢犯荆州。主公可驰书去刘璋处,只推曹操攻击孙
权,权求救于荆州。吾与孙权唇齿之邦,不容不相援。张鲁自守之贼,决不敢来犯界。吾今
欲勒兵回荆州,与孙权会同破曹操,奈兵少粮缺。望推同宗之谊,速发精兵三、四万,行粮
十万斛相助。请勿有误。若得军马钱粮,却另作商议。”

玄德从之,遣人往成都。来到关前,杨怀、高沛闻知此事,遂教高沛守关,杨怀同使者
入成都,见刘璋呈上书信。刘璋看毕,问杨怀为何亦同来。杨怀曰:“专为此书而来。刘备
自从入川,广布恩德,以收民心,其意甚是不善。今求军马钱粮,切不可与。如若相助,是
把薪助火也。”刘璋曰:“吾与玄德有兄弟之情,岂可不助?”一人出曰:“刘备枭雄,久
留于蜀而不遣,是纵虎入室矣。今更助之以军马钱粮,何异与虎添翼乎?”众视其人,乃零
陵烝阳人,姓刘名巴,字子初。刘璋闻刘巴之言,犹豫未决。黄权又复苦谏。璋乃量拨老弱
军四千,米一万斛,发书遣使报玄德。仍令杨怀、高沛紧守关隘。刘璋使者到葭萌关见玄
德,呈上回书。玄德大怒曰:“吾为汝御敌,费力劳心。汝今积财吝赏,何以使士卒效命
乎?”遂扯毁回书,大骂而起。使者逃回成都。庞统曰:“主公只以仁义为重,今日毁书发
怒,前情尽弃矣。”玄德曰:“如此,当若何?”庞统曰:“某有三条计策,请主公自择而
行。”

玄德问:“那三条计?”统曰:“只今便选精兵,昼夜兼道径袭成都:此为上计。杨
怀、高沛乃蜀中名将,各仗强兵拒守关隘;今主公佯以回荆州为名,二将闻知,必来相送;
就送行处,擒而杀之,夺了关隘,先取涪城,然后却向成都:此中计也。退还白帝,连夜回
荆州,徐图进取:此为下计。若沉吟不去,将至大困,不可救矣。”玄德曰:“军师上计太
促,下计太缓;中计不迟不疾,可以行之。”

于是发书致刘璋,只说曹操令部将乐进引兵至青泥镇,众将抵敌不住,吾当亲往拒之,
不及面会,特书相辞。书至成都,张松听得说刘玄德欲回荆州,只道是真心,乃修书一封,
欲令人送与玄德,却值亲兄广汉太守张肃到,松急藏书于袖中,与肃相陪说话。肃见松神情
恍惚,心中疑惑。松取酒与肃共饮。献酬之间,忽落此书于地,被肃从人拾得。席散后,从
人以书呈肃。肃开视之。书略曰:“松昨进言于皇叔,并无虚谬,何乃迟迟不发?逆取顺
守,古人所贵。今大事已在掌握之中,何故欲弃此而回荆州乎?使松闻之,如有所失。书呈
到日,疾速进兵。松当为内应,万勿自误!”张肃见了,大惊曰:“吾弟作灭门之事,不可
不首。”连夜将书见刘璋,具言弟张松与刘备同谋,欲献西川。刘璋大怒曰:“吾平日未尝
薄待他,何故欲谋反!”遂下令捉张松全家,尽斩于市。后人有诗叹曰:“一览无遗世所
稀,谁知书信泄天机。未观玄德兴王业,先向成都血染衣。”

刘璋既斩张松,聚集文武商议曰:“刘备欲夺吾基业,当如之何?”黄权曰:“事不宜
迟。即便差人告报各处关隘,添兵把守,不许放荆州一人一骑入关。”璋从其言,星夜驰檄
各关去讫。却说玄德提兵回涪城,先令人报上涪水关,请杨怀,高沛出关相别。杨、高二将
闻报,商议曰:“玄德此回若何?”高沛曰:“玄德合死。我等各藏利刃在身,就送行处刺
之,以绝吾主之患。”杨怀曰:“此计大妙。”二人只带随行二百人,出关送行,其余并留
在关上。

玄德大军尽发。前至涪水之上,庞统在马上谓玄德曰:“杨怀、高沛若欣然而来,可提
防之;若彼不来,便起兵径取其关,不可迟缓。”正说间,忽起一阵旋风,把马前“帅”字
旗吹倒。玄德问庞统曰:“此何兆也?”统曰:“此警报也,杨怀、高沛二人必有行刺之
意,宜善防之。”玄德乃身披重铠,自佩宝剑防备。人报杨、高二将前来送行。玄德令军马
歇定。庞统分付魏延、黄忠:“但关上来的军士,不问多少,马步军兵,一个也休放回。”
二将得令而去。

却说杨怀、高沛二人身边各藏利刃,带二百军兵,牵羊送酒,直至军前。见并无准备,
心中暗喜,以为中计。入至帐下、见玄德正与庞统坐于帐中。二将声喏曰:“闻皇叔远回,
特具薄礼相送。”遂进酒劝玄德。玄德曰:“二将军守关不易,当先饮此杯。”二将饮酒
毕,玄德曰:“吾有密事与二将军商议,闲人退避。”遂将带来二百人尽赶出中军。玄德叱
曰:“左右与吾捉下二贼!”帐后刘封、关平应声而出。杨、高二人急待争斗,刘封、关平
各捉住一人。玄德喝曰:“吾与汝主是同宗兄弟,汝二人何故同谋,离间亲情?”庞统叱左
右搜其身畔,果然各搜出利刃一口。统便喝斩二人;玄德还犹未决,统曰:“二人本意欲杀
吾主,罪不容诛。”遂叱刀斧手斩杨怀、高沛于帐前。黄忠、魏延早将二百从人,先自捉
下,不曾走了一个。玄德唤入,各赐酒压惊。玄德曰:“杨怀、高沛离间吾兄弟,又藏利刃
行刺,故行诛戮。尔等无罪,不必惊疑。”众各拜谢。庞统曰:“吾今即用汝等引路,带吾
军取关。各有重赏。”众皆应允。是夜二百人先行,大军随后。前军至关下叫曰:“二将军
有急事回,可速开关。”城上听得是自家军,即时开关。大军一拥而入,兵不血刃,得了涪
关。蜀兵皆降。玄德各加重赏,遂即分兵前后守把。次日劳军,设宴于公厅。玄德酒酣,顾
庞统曰:“今日之会,可为乐乎?”庞统曰:“伐人之国而以为乐,非仁者之兵也。”玄德
曰:“吾闻昔日武王伐纣,作乐象功,此亦非仁者之兵欤?汝言何不合道理?可速退!”庞
统大笑而起。左右亦扶玄德入后堂。睡至半夜,酒醒。左右以逐庞统之言告知玄德。玄德大
悔;次早穿衣升堂,请庞统谢罪曰:“昨日酒醉,言语触犯,幸勿挂怀。”庞统谈笑自若。
玄德曰:“昨日之言,惟吾有失。”庞统曰:“君臣俱失,何独主公?”玄德亦大笑,其乐
如初。

却说刘璋闻玄德杀了杨、高二将,袭了涪水关,大惊曰:“不料今日果有此事!”遂聚
文武,问退兵之策。黄权曰:“可连夜遣兵屯雒县,塞住咽喉之路。刘备虽有精兵猛将,不
能过也。”璋遂令刘璝、泠苞、张任、邓贤点五万大军,星夜往守雒县,以拒刘备。四将行
兵之次,刘璝曰:“吾闻锦屏山中有一异人,道号紫虚上人,知人生死贵贱。吾辈今日行
军,正从锦屏山过。何不试往问之?”张任曰:“大丈夫行兵拒敌,岂可问于山野之人
乎?”璝曰:“不然。圣人云:至诚之道,可以前知。吾等问于高明之人,当趋吉避凶。”
于是四人引五六十骑至山下,问径樵夫。樵夫指高山绝顶上,便是上人所居。四人上山至庵
前,见一道童出迎。问了姓名,引入庵中。只见紫虚上人坐于蒲墩之上。四人下拜,求问前
程之事。紫虚上人曰:“贫道乃山野废人,岂知休咎?”刘璝再三拜问,紫虚遂命道童取纸
笔,写下八句言语,付与刘璝。其文曰:“左龙右凤,飞入西川。雏凤坠地,卧龙升天。一
得一失,天数当然。见机而作,勿丧九泉。”刘璝又问曰:“我四人气数如何?”紫虚上人
曰:“定数难逃,何必再问!”璝又请问时,上人眉垂目合,恰似睡着的一般,并不答应。
四人下山。刘璝曰:“仙人之言,不可不信。”张任曰:“此狂叟也,听之何益。”遂上马
前行。

既至雒县,分调人马,守把各处关隘口。刘璝曰:“雒城乃成都之保障,失此则成都难
保。吾四人公议,着二人守城,二人去雒县前面,依山傍险,扎下两个寨子,勿使敌兵临
城。”泠苞、邓贤曰:“某愿往结寨。”刘璝大喜,分兵二万,与泠、邓二人,离城六十里
下寨。刘璝、张任守护雒城。

却说玄德既得涪水关,与庞统商议进取雒城。人报刘璋拨四将前来,即日泠苞、邓贤领
二万军离城六十里,扎下两个大寨。玄德聚众将问曰:“谁敢建头功,去取二将寨栅?”老
将黄忠应声出曰:“老夫愿往。”玄德曰:“老将军率本部人马,前至雒城,如取得泠苞、
邓贤营寨,必当重赏。”

黄忠大喜,即领本部兵马,谢了要行。忽帐下一人出曰:“老将军年纪高大,如何去
得?小将不才愿往。”玄德视之,乃是魏延。黄忠曰:“我已领下将令,你如何敢搀越?”
魏延曰:“老者不以筋骨为能。吾闻泠苞、邓贤乃蜀中名将,血气方刚。恐老将军近他不
得,岂不误了主公大事?因此愿相替,本是好意。”黄忠大怒曰:“汝说吾老,敢与我比试
武艺么?”魏延曰:“就主公之前,当面比试。赢得的便去,何如?”黄忠遂趋步下阶,便
叫小校将刀来!玄德急止之曰:“不可!吾今提兵取川,全仗汝二人之力。今两虎相斗,必
有一伤。须误了我大事。吾与你二人劝解,休得争论。”庞统曰:“汝二人不必相争。即今
泠苞、邓贤下了两个营寨。今汝二人自领本部军马,各打一寨。如先夺得者,便为头功。”
于是分定黄忠打泠苞寨,魏延打邓贤寨。二人各领命去了。庞统曰:“此二人去,恐于路上
相争,主公可自引军为后应。”玄德留庞统守城,自与刘封、关平引五千军随后进发。

却说黄忠归寨,传令来日四更造饭,五更结束,平明进兵,取左边山谷而进。魏延却暗
使人探听黄忠甚时起兵。探事人回报:“来日四更造饭,五更起兵。”魏延暗喜,分付众军
士二更造饭,三更起兵,平明要到邓贤寨边。军士得令,都饱餐一顿,马摘铃,人衔枚,卷
旗束甲,暗地去劫寨。三更前后,离寨前进。到半路,魏延马上寻思:“只去打邓贤寨,不
显能处,不如先去打泠苞寨,却将得胜兵打邓贤寨。两处功劳,都是我的。”就马上传令,
教军士都投左边山路里去。天色微明,离泠苞寨不远,教军士少歇,排搠金鼓旗幡、枪刀器
械。早有伏路小军飞报入寨,泠苞已有准备了。一声炮响,三军上马,杀将出来。魏延纵马
提刀,与泠苞接战。二将交马,战到三十合,川兵分两路来袭汉军。汉军走了半夜,人马力
乏,抵当不住,退后便走。魏延听得背后阵脚乱,撇了泠苞,拨马回走。川兵随后赶来,汉
军大败。走不到五里,山背后鼓声震地,邓贤引一彪军从山谷里截出来,大叫:“魏延快下
马受降!”魏延策马飞奔,那马忽失前蹄,引足跪地,将魏延掀将下来。邓贤马奔到,挺枪
来刺魏延。枪未到处,弓弦响,邓贤倒撞下马。后面泠苞方欲来救,一员大将,从山坡上跃
马而来,厉声大叫:“老将黄忠在此!”舞刀直取泠苞。泠苞抵敌不住,望后便走。黄忠乘
势追赶,川兵大乱。

黄忠一枝军救了魏延,杀了邓贤,直赶到寨前。泠苞回马与黄忠再战。不到十余合,后
面军马拥将上来,泠苞只得弃了左寨,引败军来投右寨。只见寨中旗帜全别,泠苞大惊。兜
住马看时,当头一员大将,金甲锦袍,乃是刘玄德,左边刘封,右边关平,大喝道:“寨子
吾已夺下,汝欲何往?”原来玄德引兵从后接应,便乘势夺了邓贤寨子。泠苞两头无路,取
山僻小径,要回雒城。行不到十里,狭路伏兵忽起,搭钩齐举,把泠苞活捉了。原来却是魏
延自知犯罪,无可解释,收拾后军,令蜀兵引路,伏在这里,等个正着。用索缚了泠苞,解
投玄德寨来。却说玄德立起免死旗,但川兵倒戈卸甲者,并不许杀害,如伤者偿命;又谕众
降兵曰:“汝川人皆有父母妻子,愿降者充军,不愿降者放回。”于是欢声动地。黄忠安下
寨脚,径来见玄德,说魏延违了军令,可斩之。玄德急召魏延,魏延解泠苞至。玄德曰:
“延虽有罪,此功可赎。”令魏延谢黄忠救命之恩,今后毋得相争。魏延顿首伏罪。玄德重
赏黄忠,使人押泠苞到帐下,玄德去其缚,赐酒压惊,问曰:“汝肯降否?”泠苞曰:“既
蒙免死,如何不降?刘璝、张任与某为生死之交;若肯放某回去,当即招二人来降,就献雒
城。”玄德大喜,便赐衣服鞍马,令回雒城。魏延曰:“此人不可放回。若脱身一去,不复
来矣。”玄德曰:“吾以仁义待人,人不负我。”

却说泠苞得回雒城,见刘璝、张任,不说捉去放回,只说:“被我杀了十余人,夺得马
匹逃回。”刘璝忙遣人往成都求救。刘璋听知折了邓贤,大惊,慌忙聚众商议。长子刘循进
曰:“儿愿领兵前去守雒城。”璋曰:“既吾儿肯去,当遣谁人为辅?”一人出曰:“某愿
往”璋视之,乃舅氏吴懿也。璋曰:“得尊舅去最好。谁可为副将?”吴懿保吴兰、雷铜二
人为副将,点二万军马来到雒城。刘璝、张任接着,具言前事。吴懿曰:“兵临城下,难以
拒敌,汝等有何高见?”泠苞曰:“此间一带,正靠涪江,江水大急;前面寨占山脚,其形
最低。某乞五千军,各带锹锄前去,决涪江之水,可尽淹死刘备之兵也。”吴懿从其计,即
令泠苞前往决水,吴兰、雷铜引兵接应。泠苞领命,自去准备决水器械。

却说玄德令黄忠、魏延各守一寨,自回涪城,与军师庞统商议。细作报说:“东吴孙权
遣人结好东川张鲁,将欲来攻葭萌关。”玄德惊曰:“若葭萌关有失,截断后路,吾进退不
得,当如之何?”庞统谓孟达曰:“公乃蜀中人,多知地理,去守葭萌关如何?”达曰:
“某保一人与某同去守关,万无一失。”玄德问何人。达曰:“此人曾在荆州刘表部下为中
郎将,乃南郡枝江人,姓霍,名峻,字仲邈。”玄德大喜,即时遣孟达、霍峻守葭萌关去
了。庞统退归馆舍,门吏忽报:“有客特来相访。”统出迎接,见其人身长八尺,形貌甚
伟;头发截短,披于颈上;衣服不甚齐整。统问曰:“先生何人也?”其人不答,径登堂仰
卧床上。统甚疑之。再三请问。其人曰:“且消停,吾当与汝说知天下大事。”统闻之愈
疑,命左右进酒食。其人起而便食,并无谦逊;饮食甚多,食罢又睡。统疑惑不定,使人请
法正视之,恐是细作。法正慌忙到来。统出迎接,谓正曰:“有一人如此如此。”法正曰:
“莫非彭永言乎?”升阶视之。其人跃起曰:“孝直别来无慈!正是:只为川人逢旧识,遂
令涪水息洪流。毕竟此人是谁,且看下文分解。