\chapter{追汉军王双受诛~袭陈仓武侯取胜}

却说司马懿奏曰:“臣尝奏陛下,言孔明必出陈仓,故以郝昭守之,今果然矣。彼若从陈仓入寇,运粮甚便。今幸有郝昭、王双守把,不敢从此路运粮。其余小道,搬运艰难。臣算蜀兵行粮止有一月,利在急战。我军只宜久守。陛下可降诏,令曹真坚守诸路关隘,不要出战。不须一月,蜀兵自走。那时乘虚而击之,诸葛亮可擒也。”睿欣然曰:“卿既有先见之明,何不自引一军以袭之?”懿曰:“臣非惜身重命,实欲存下此兵,以防东吴陆逊耳。孙权不久必将僭号称尊;如称尊号,恐陛下伐之,定先入寇也:臣故欲以兵待之。”正言间,忽近臣奏曰:“曹都督奏报军情。”懿曰:“陛下可即令人告戒曹真:凡追赶蜀兵,必须观其虚实,不可深入重地,以中诸葛亮之计。”睿即时下诏,遣太常卿韩暨持节告戒曹真:“切不可战,务在谨守;只待蜀兵退去,方才击之。”司马懿送韩暨于城外,嘱之曰:“吾以此功让与子丹;公见子丹,休言是吾所陈之意,只道天子降诏,教保守为上。追赶之人,大要仔细,勿遣性急气躁者追之。”暨辞去。

却说曹真正升帐议事,忽报天子遣太常卿韩暨持节至。真出寨接入,受诏已毕,退与郭淮、孙礼计议。淮笑曰:“此乃司马仲达之见也。”真曰:“此见若何?”淮曰:“此言深识诸葛亮用兵之法。久后能御蜀兵者,必仲达也。”真曰:“倘蜀兵不退,又将如何?”淮曰:“可密令人去教王双,引兵于小路巡哨,彼自不敢运粮。待其粮尽兵退,乘势追击,可获全胜。”孙礼曰:“某去祁山虚妆做运粮兵,车上尽装干柴茅草,以硫黄焰硝灌之,却教人虚报陇西运粮到。若蜀人无粮,必然来抢。待人其中,放火烧车,外以伏兵应之,可胜矣。”真喜曰:“此计大妙!”即令孙礼引兵依计而行。又遣人教王双引兵于小路上巡哨,郭淮引兵提调箕谷、街亭,令诸路军马守把险要。真又令张辽子张虎为先锋,乐进子乐綝为副先锋,同守头营,不许出战。却说孔明在祁山寨中,每日今人挑战,魏兵坚守不出。孔明唤姜维等商议曰:“魏兵坚守不出,是料吾军中无粮也。今陈仓转运不通,其余小路盘涉艰难,吾算随军粮草,不敷一月用度,如之奈何?”正踌躇间,忽报:“陇西魏军运粮数千车于祁山之西,运粮官乃孙礼也。”孔明曰:“其人如何?”有魏人告曰:“此人曾随魏主出猎于大石山,忽惊起一猛虎,直奔御前,孙礼下马拔剑斩之。从此封为上将军。乃曹真心腹人也。”孔明笑曰:“此是魏将料吾乏粮,故用此计:车上装载者,必是茅草引火之物。吾平生专用火攻,彼乃欲以此计诱我耶?彼若知吾军去劫粮车,必来劫吾寨矣。可将计就计而行。”遂唤马岱分付曰:“汝引三千军径到魏兵屯粮之所,不可入营,但于上风头放火。若烧着车仗,魏兵必来围吾寨。”又差马忠、张嶷各引五千兵在外围住,内外夹攻。三人受计去了。又唤关兴、张苞分付曰:“魏兵头营接连四通之路。今晚若西山火起,魏兵必来劫吾营。汝二人却伏于魏寨左右,只等他兵出寨,汝二人便可劫之。”又唤吴班、吴懿分付曰:“汝二人各引一军伏于营外。如魏兵到,可截其归路。”孔明分拨已毕,自在祁山上凭高而坐。

魏兵探知蜀兵要来劫粮,慌忙报与孙礼。礼令人飞报曹真。真遣人去头营分付张虎、乐綝:“看今夜山西火起,蜀兵必来救应。可以出军,如此如此。”二将受计,令人登楼专看号火。却说孙礼把军伏于山西,只待蜀兵到。是夜二更,马岱引三千兵来,人皆衔枚,马尽勒口,径到山西。见许多车仗,重重叠叠,攒绕成营,车仗虚插旌旗。正值西南风起,岱令军士径去营南放火,车仗尽着,火光冲天。孙礼只道蜀兵到魏寨内放号火,急引兵一齐掩至。背后鼓角喧天,两路兵杀来:乃是马忠、张嶷,把魏军围在垓心。孙礼大惊。又听的魏军中喊声起,一彪军从火光边杀来,乃是马岱。内外夹攻,魏兵大败。火紧风急,人马乱窜,死者无数。孙礼引中伤军,突烟冒火而走。却说张虎在营中,望见火光,大开寨门,与乐綝尽引人马,杀奔蜀寨来,寨中却不见一人。急收军回时,吴班、吴懿两路兵杀出,断其归路。张、乐二将急冲出重围,奔回本寨,只见土城之上,箭如飞蝗,原来却被关兴、张苞袭了营寨。魏兵大败,皆投曹真寨来。方欲入寨,只见一彪败军飞奔而来,乃是孙礼;遂同入寨见真,各言中计之事。真听知,谨守大寨,更不出战。蜀兵得胜,回见孔明。孔明令人密授计与魏延,一面教拔寨齐起。杨仪曰:“今已大胜,挫尽魏兵锐气,何故反欲收军?”孔明曰:“吾兵无粮,利在急战。今彼坚守不出,吾受其病矣。彼今虽暂时兵败,中原必有添益;若以轻骑袭吾粮道,那时要归不能。今乘魏兵新败,不敢正视蜀兵,便可出其不意,乘机退去。所忧者但魏延一军,在陈仓道口拒住王双,急不能脱身;吾已令人授以密计,教斩王双,使魏人不敢来追。只今后队先行。”当夜,孔明只留金鼓守在寨中打更。一夜兵已尽退,只落空营。却说曹真正在寨中忧闷,忽报左将军张郃领军到。郃下马入帐,谓真曰:“某奉圣旨,特来听调。”真曰:“曾别仲达否?”郃曰:“仲达分付云:吾军胜,蜀兵必不便去;若吾军败,蜀兵必即去矣。今吾军失利之后,都督曾往哨探蜀兵消息否?”真曰:“未也。”于是即令人往探之,果是虚营,只插着数十面旌旗,兵已去了二日也。曹真懊悔无及。

且说魏延受了密计,当夜二更拔寨,急回汉中。早有细作报知王双。双大驱军马,并力追赶。追到二十余里,看看赶上,见魏延旗号在前,双大叫曰:“魏延休走!”蜀兵更不回头。双拍马赶来。背后魏兵叫曰:“城外寨中火起,恐中敌人奸计。”双急勒马回时,只见一片火光冲天,慌令退军。行到山坡左侧,忽一骑马从林中骤出,大喝曰:“魏延在此!”王双大惊,措手不及,被延一刀砍于马下。魏兵疑有埋伏,四散逃走。延手下止有三十骑人马,望汉中缓缓而行。后人有诗赞曰:“孔明妙算胜孙庞,耿若长星照一方。进退行兵神莫测,陈仓道口斩王双。”原来魏延受了孔明密计:先教存下三十骑,伏于王双营边;只待王双起兵赶时,却去他营中放火;待他回寨,出其不意,突出斩之。魏延斩了王双,引兵回到汉中见孔明,交割了人马。孔明设宴大会,不在话下。

且说张郃追蜀兵不上,回到寨中。忽有陈仓城郝昭差人申报,言王双被斩,曹真闻知,伤感不已,因此忧成疾病,遂回济阳;命郭淮、孙礼、张郃守长安诸道。

却说吴王孙权设朝,有细作人报说:“蜀诸葛丞相出兵两次,魏都督曹真兵损将亡。”于是群臣皆劝吴王兴师伐魏,以图中原。权犹疑未决。张昭奏曰:“近闻武昌东山,凤凰来仪;大江之中,黄龙屡现。主公德配唐、虞,明并文、武,可即皇帝位,然后兴兵。”多官皆应曰:“子布之言是也。”遂选定夏四月丙寅日,筑坛于武昌南郊。是日,群臣请权登坛即皇帝位,改黄武八年为黄龙元年。谥父孙坚为武烈皇帝,母吴氏为武烈皇后,兄孙策为长沙桓王。立子孙登为皇太子。命诸葛瑾长子诸葛恪为太子左辅,张昭次子张体为太子右弼。

恪字元逊,身长七尺,极聪明,善应对。权甚爱之。年六岁时,值东吴筵会,恪随父在座。权见诸葛瑾面长,乃令人牵一驴来,用粉笔书其面曰:“诸葛子瑜”。众皆大笑。恪趋至前,取粉笔添二字于其下曰:“诸葛子瑜之驴”。满座之人,无不惊讶。权大喜,遂将驴赐之。又一日,大宴官僚,权命恪把盏。巡至张昭面前,昭不饮,曰:“此非养老之礼也。”权谓恪曰:“汝能强子布饮乎?”恪领命,乃谓昭曰:“昔姜尚父年九十,秉旄仗钺,未尝言老。今临阵之日,先生在后;饮酒之日,先生在前:何谓不养老也?”昭无言可答,只得强饮。权因此爱之,故命辅太子。张昭佐吴王,位列三公之上,故以其子张休为太子右弼。又以顾雍为丞相,陆逊为上将军,辅太子守武昌。权复还建业。群臣共议伐魏之策。张昭奏曰:“陛下初登宝位,未可动兵。只宜修文偃武,增设学校,以安民心;遣使入川,与蜀同盟,共分天下,缓缓图之。”权从其言,即令使命星夜入川,来见后主。礼毕,细奏其事。后主闻知,遂与群臣商议。众议皆谓孙权僭逆,宜绝其盟好。蒋琬曰:“可令人问于丞相。”后主即遣使到汉中问孔明。孔明曰:“可令人赍礼物入吴作贺,乞遣陆逊兴师伐魏。魏必命司马懿拒之。懿若南拒东吴,我再出祁山,长安可图也。”后主依言,遂令太尉陈震,将名马、玉带、金珠、宝贝,入吴作贺。

震至东吴,见了孙权,呈上国书。权大喜,设宴相待,打发回蜀。权召陆逊入,告以西蜀约会兴兵伐魏之事。逊曰:“此乃孔明惧司马懿之谋也。既与同盟,不得不从。今却虚作起兵之势,遥与西蜀为应。待孔明攻魏急,吾可乘虚取中原也。”即时下令,教荆襄各处都要训练人马,择日兴师。

却说陈震回到汉中,报知孔明。孔明尚忧陈仓不可轻进,先令人去哨探。回报说:“陈仓城中郝昭病重。”孔明曰:“大事成矣。”遂唤魏延、姜维分付曰:“汝二人领五千兵,星夜直奔陈仓城下;如见火起,并力攻城。”二人俱未深信,又来告曰:“何日可行?”孔明曰:“三日都要完备;不须辞我,即便起行。”二人受计去了。又唤关兴、张苞至,附耳低言,如此如此。二人各受密计而去。且说郭淮闻郝昭病重,乃与张郃商议曰:“郝昭病重,你可速去替他。我自写表申奏朝廷,别行定夺。”张郃引着三千兵,急来替郝昭。时郝昭病危,当夜正呻吟之间,忽报蜀军到城下了。昭急令人上城守把。时各门上火起,城中大乱。昭听知惊死。蜀兵一拥入城。

却说魏延、姜维领兵到陈仓城下看时,并不见一面旗号,又无打更之人。二人惊疑,不敢攻城。忽听得城上一声炮响,四面旗帜齐竖。只见一人纶巾羽扇,鹤氅道袍,大叫曰:“汝二人来的迟了!”二人视之,乃孔明也。二人慌忙下马,拜伏于地曰:“丞相真神计也!”孔明令放入城,谓二人曰:“吾打探得郝昭病重,吾令汝三日内领兵取城,此乃稳众人之心也。吾却令关兴、张苞,只推点军,暗出汉中。吾即藏于军中,星夜倍道径到城下,使彼不能调兵。吾早有细作在城内放火、发喊相助,令魏兵惊疑不定。兵无主将,必自乱矣。吾因而取之,易如反掌。兵法云:出其不意,攻其无备。正谓此也。”魏延、姜维拜伏。孔明怜郝昭之死,令彼妻小扶灵柩回魏,以表其忠。孔明谓魏延、姜维曰:“汝二人且莫卸甲,可引兵去袭散关。把关之人,若知兵到,必然惊走。若稍迟便有魏兵至关,即难攻矣。”魏延、姜维受命,引兵径到散关。把关之人,果然尽走。二人上关才要卸甲,遥见关外尘头大起,魏兵到来。二人相谓曰:“丞相神算,不可测度!”急登楼视之,乃魏将张郃也。二人乃分兵守住险道。张郃见蜀兵把住要路,遂令退军。魏延随后追杀一阵,魏兵死者无数,张郃大败而去。延回到关上,令人报知孔明。

孔明先自领兵,出陈仓斜谷,取了建威。后面蜀兵陆续进发。后主又命大将陈式来助。孔明驱大兵复出祁出。安下营寨,孔明聚众言曰:“吾二次出祁山,不得其利,今又到此,吾料魏人必依旧战之地,与吾相敌。彼意疑我取雍、郿二处,必以兵拒守;吾观阴平、武都二郡,与汉连接,若得此城,亦可分魏兵之势。何人敢取之?”姜维曰:“某愿往。”王平应曰:“某亦愿往。”孔明大喜,遂令姜维引兵一万取武都,王平引兵一万取阴平。二人领兵去了。

再说张郃回到长安,见郭淮、孙礼,说:“陈仓已失,郝昭已亡,散关亦被蜀兵夺了。今孔明复出祁山,分道进兵。”淮大惊曰:“若如此,必取雍、郿矣!”乃留张郃守长安,令孙礼保雍城。淮自引兵星夜来郿城守御,一面上表入洛阳告急。

却说魏主曹睿设朝,近臣奏曰:“陈仓城已失,郝昭已亡,诸葛亮又出祁山,散关亦被蜀兵夺了。”睿大惊。忽又奏满宠等有表,说:“东吴孙权僭称帝号,与蜀同盟。今遣陆逊在武昌训练人马,听候调用。只在旦夕,必入寇矣。”睿闻知两处危急,举止失措,甚是惊慌。此时曹真病未痊,即召司马懿商议。懿奏曰:“以臣愚意所料,东吴必不举兵。”睿曰:“卿何以知之?”懿曰:“孔明尝思报猇亭之仇,非不欲吞吴也,只恐中原乘虚击彼,故暂与东吴结盟。陆逊亦知其意,故假作兴兵之势以应之,实是坐观成败耳。陛下不必防吴,只须防蜀。”睿曰:“卿真高见!”遂封懿为大都督,总摄陇西诸路军马,令近臣取曹真总兵将印来。懿曰:“臣自去取之。”

遂辞帝出朝,径到曹真府下,先令人入府报知,懿方进见。问病毕,懿曰:“东吴、西蜀会合,兴兵入寇,今孔明又出祁山下寨,明公知之乎?”真惊讶曰:“吾家人知我病重,不令我知。似此国家危急,何不拜仲达为都督,以退蜀兵耶?”懿曰:“某才薄智浅,不称其职。”真曰:“取印与仲达。”懿曰:“都督少虑。某愿助一臂之力,只不敢受此印也。”真跃起曰:“如仲达不领此任,中国必危矣!吾当抱病见帝以保之!懿曰:“天子已有恩命,但懿不敢受耳。”真大喜曰:“仲达今领此任,可退蜀兵。”懿见真再三让印,遂受之,入内辞了魏主,引兵往长安来与孔明决战。正是:旧帅印为新帅取,两路兵惟一路来。未知胜负如何,且看下文分解。