\chapter{曹操煮酒论英雄~关公赚城斩车胄}

却说董承等问马腾曰:“公欲用何人?”马腾曰:“见有豫州牧刘玄德在此,何不求
之?”承曰:“此人虽系皇叔,今正依附曹操,安肯行此事耶?”腾曰:“吾观前日围场之
中,曹操迎受众贺之时,云长在玄德背后,挺刀欲杀操,玄德以目视之而止。玄德非不欲图
操,恨操牙爪多,恐力不及耳。公试求之,当必应允。”吴硕曰:“此事不宜太速,当从容
商议。”众皆散去。次日黑夜里,董承怀诏,径往玄德公馆中来。门吏入报,玄德迎出,请
入小阁坐定。关、张侍立于侧。玄德曰:“国舅夤夜至此,必有事故。”承曰:“白日乘马
相访,恐操见疑,故黑夜相见。”玄德命取酒相待。承曰:“前日围场之中,云长欲杀曹
操,将军动目摆头而退之,何也?”玄德失惊曰:“公何以知之?”承曰:“人皆不见,某
独见之。”玄德不能隐讳,遂曰:“舍弟见操僭越,故不觉发怒耳。”承掩面而哭曰:“朝
廷臣子,若尽如云长,何忧不太平哉!”玄德恐是曹操使他来试探,乃佯言曰:“曹丞相治
国,为何忧不太平?”承变色而起曰:“公乃汉朝皇叔,故剖肝沥胆以相告,公何诈也?”
玄德曰:“恐国舅有诈,故相试耳。”于是董承取衣带诏令观之,玄德不胜悲愤。又将义状
出示,上止有六位:一,车骑将军董承;二,工部侍郎王子服;三,长水校尉种辑;四,议
郎吴硕;五,昭信将军吴子兰;六,西凉太守马腾。玄德曰:“公既奉诏讨贼,备敢不效犬
马之劳。”承拜谢,便请书名。玄德亦书“左将军刘备”,押了字,付承收讫。承曰:“尚
容再请三人,共聚十义,以图国贼,”玄德曰:“切宜缓缓施行,不可轻泄。”共议到五
更,相别去了。

玄德也防曹操谋害,就下处后园种菜,亲自浇灌,以为韬晦之计。关、张二人曰:“兄
不留心天下大事,而学小人之事,何也?”玄德曰:“此非二弟所知也。”二人乃不复言。

一日,关、张不在,玄德正在后园浇菜,许褚、张辽引数十人入园中曰:“丞相有命,
请使君便行。”玄德惊问曰:“有甚紧事?”许褚曰:“不知。只教我来相请。”玄德只得
随二人入府见操。操笑曰:“在家做得好大事!”?得玄德面如土色。操执玄德手,直至后
园,曰:“玄德学圃不易!”玄德方才放心,答曰:“无事消遣耳。”操曰:“适见枝头梅
子青青,忽感去年征张绣时,道上缺水,将士皆渴;吾心生一计,以鞭虚指曰:‘前面有梅
林。’军士闻之,口皆生唾,由是不渴。今见此梅,不可不赏。又值煮酒正熟,故邀使君小
亭一会。”玄德心神方定。随至小亭,已设樽俎:盘置青梅,一樽煮酒。二人对坐,开怀畅
饮。酒至半酣,忽阴云漠漠,聚雨将至。从人遥指天外龙挂,操与玄德凭栏观之。操曰:
“使君知龙之变化否?”玄德曰:“未知其详。”操曰:“龙能大能小,能升能隐;大则兴
云吐雾,小则隐介藏形;升则飞腾于宇宙之间,隐则潜伏于波涛之内。方今春深,龙乘时变
化,犹人得志而纵横四海。龙之为物,可比世之英雄。玄德久历四方,必知当世英雄。请试
指言之。”玄德曰:“备肉眼安识英雄?”操曰:“休得过谦。”玄德曰:“备叨恩庇,得
仕于朝。天下英雄,实有未知。”操曰:“既不识其面,亦闻其名。”玄德曰:“淮南袁
术,兵粮足备,可为英雄?”操笑曰:“冢中枯骨,吾早晚必擒之!”玄德曰:“河北袁
绍,四世三公,门多故吏;今虎踞冀州之地,部下能事者极多,可为英雄?“操笑曰:“袁
绍色厉胆薄,好谋无断;干大事而惜身,见小利而忘命:非英雄也。玄德曰:“有一人名称
八俊,威镇九州:刘景升可为英雄?”操曰:“刘表虚名无实,非英雄也。”玄德曰:“有
一人血气方刚,江东领袖——孙伯符乃英雄也?”操曰:“孙策藉父之名,非英雄也。”玄
德曰:“益州刘季玉,可为英雄乎?”操曰:“刘璋虽系宗室,乃守户之犬耳,何足为英
雄!”玄德曰:“如张绣、张鲁、韩遂等辈皆何如?”操鼓掌大笑曰:“此等碌碌小人,何
足挂齿!”玄德曰:“舍此之外,备实不知。”操曰:“夫英雄者,胸怀大志,腹有良谋,
有包藏宇宙之机,吞吐天地之志者也。”玄德曰:“谁能当之?”操以手指玄德,后自指,
曰:“今天下英雄,惟使君与操耳!”玄德闻言,吃了一惊,手中所执匙箸,不觉落于地
下。时正值天雨将至,雷声大作。玄德乃从容俯首拾箸曰:“一震之威,乃至于此。”操笑
曰:“丈夫亦畏雷乎?”玄德曰:“圣人迅雷风烈必变,安得不畏?”将闻言失箸缘故,轻
轻掩饰过了。操遂不疑玄德。后人有诗赞曰:“勉从虎穴暂趋身,说破英雄惊杀人。巧借闻
雷来掩饰,随机应变信如神。”

天雨方住,见两个人撞入后园,手提宝剑,突至亭前,左右拦挡不住。操视之,乃关、
张二人也。原来二人从城外射箭方回,听得玄德被许褚、张辽请将去了,慌忙来相府打听;
闻说在后园,只恐有失,故冲突而入。却见玄德与操对坐饮酒。二人按剑而立。操问二人何
来。云长曰:“听知丞相和兄饮酒,特来舞剑,以助一笑。”操笑曰:“此非鸿门会,安用
项庄、项伯乎?”玄德亦笑。操命:“取酒与二樊哙压惊。”关、张拜谢。须臾席散,玄德
辞操而归。云长曰:“险些惊杀我两个!”玄德以落箸事说与关、张。关、张问是何意。玄
德曰:“吾之学圃,正欲使操知我无大志;不意操竟指我为英雄,我故失惊落箸。又恐操生
疑,故借惧雷以掩饰之耳。”关、张曰:“兄真高见!”

操次日又请玄德。正饮间,人报满宠去探听袁绍而回。操召入问之。宠曰:“公孙瓒已
被袁绍破了。”玄德急问曰:“愿闻其详。”宠曰:“瓒与绍战不利,筑城围圈,圈上建
楼,高十丈,名曰易京楼,积粟三十万以自守。战士出入不息,或有被绍围者,众请救之。
瓒曰:‘若救一人,后之战者只望人救,不肯死战矣。’遂不肯救。因此袁绍兵来,多有降
者。瓒势孤,使人持书赴许都求救,不意中途为绍军所获。瓒又遗书张燕,暗约举火为号,
里应外合。下书人又被袁绍擒住,却来城外放火诱敌。瓒自出战,伏兵四起,军马折其大
半。退守城中,被袁绍穿地直入瓒所居之楼下,放起火来。瓒无走路,先杀妻子,然后自
缢,全家都被火焚了。今袁绍得了瓒军,声势甚盛。绍弟袁术在淮南骄奢过度,不恤军民,
众皆背反。术使人归帝号于袁绍。绍欲取玉玺,术约亲自送至,见今弃淮南欲归河北。若二
人协力,急难收复。乞丞相作急图之。”玄德闻公孙瓒已死,追念昔日荐己之恩,不胜伤
感;又不知赵子龙如何下落,放心不下。因暗想曰:“我不就此时寻个脱身之计,更待何
时?”遂起身对操曰:“术若投绍,必从徐州过,备请一军就半路截击,术可擒矣。”操笑
曰:“来日奏帝,即便起兵。”次日,玄德面奏君。操令玄德总督五万人马,又差朱灵、路
昭二人同行。玄德辞帝,帝泣送之。

玄德到寓,星夜收拾军器鞍马,挂了将军印,催促便行。董承赶出十里长亭来送。玄德
曰:“国舅宁耐。某此行必有以报命。”承曰:“公宜留意,勿负帝心。”二人分别。关、
张在马上问曰:“兄今番出征,何故如此慌速?”玄德曰:“吾乃笼中鸟、网中鱼,此一行
如鱼入大海、鸟上青霄,不受笼网之羁绊也!”因命关、张催朱灵、路昭军马速行。

时郭嘉、程昱考较钱粮方回,知曹操已遣玄德进兵徐州,慌入谏曰:“丞相何故令刘备
督军?”操曰:“欲截袁术耳。”程昱曰:“昔刘备为豫州牧时,某等请杀之,丞相不听;
今日又与之兵:此放龙入海,纵虎归山也。后欲治之,其可得乎?”郭嘉曰:“丞相纵不杀
备,亦不当使之去。古人云:一日纵敌,万世之患。望丞相察之。”操然其言,遂令许褚将
兵五百前往,务要追玄德转来。许褚应诺而去。

却说玄德正行之间,只见后面尘头骤起,谓关、张曰:“此必曹兵追至也。”遂下了营
寨,令关、张各执军器,立于两边。许褚至,见严兵整甲,乃下马入营见玄德。玄德曰:
“公来此何干?”褚曰:“奉丞相命,特请将军回去,别有商议。”玄德曰:“将在外,君
命有所不受。吾面过君,又蒙丞相钧语。今别无他议,公可速回,为我禀覆丞相。”许褚寻
思:“丞相与他一向交好,今番又不曾教我来厮杀,只得将他言语回覆,另候裁夺便了。”
遂辞了玄德,领兵而回。回见曹操,备述玄德之言。操犹豫未决。程昱、郭嘉曰:“备不肯
回兵,可知其心变矣。”操曰:“我有朱灵、路昭二人在彼,料玄德未必敢心变。况我既遣
之,何可复悔?”遂不复追玄德。后人有诗叹玄德曰:“束兵秣马去匆匆,心念天言衣带
中。撞破铁笼逃虎豹,顿开金锁走蛟龙。”却说马腾见玄德已去,边报又急,亦回西凉州去
了。玄德兵至徐州,刺史车胄出迎。公宴毕,孙乾、糜竺等都来参见。玄德回家探视老小,
一面差人探听袁术。探子回报:“袁术奢侈太过,雷薄、陈兰皆投嵩山去了。术势甚衰,乃
作书让帝号于袁绍。绍命人召术,术乃收拾人马、宫禁御用之物,先到徐州来。”玄德知袁
术将至,乃引关、张、朱灵、路昭五万军出,正迎着先锋纪灵至。张飞更不打话,直取纪
灵。斗无十合,张飞大喝一声,刺纪灵于马下,败军奔走。袁术自引军来斗。玄德分兵三
路:朱灵、路昭在左,关、张在右,玄德自引兵居中,与术相见,在门旗下责骂曰:“汝反
逆不道,吾今奉明诏前来讨汝!汝当束手受降,免你罪犯。”袁术骂曰:“织席编屦小辈,
安敢轻我!”麾兵赶来。玄德暂退,让左右两路军杀出。杀得术军尸横遍野,血流成渠;兵
卒逃亡,不可胜计。又被嵩山雷薄、陈兰劫去钱粮草料。欲回寿春,又被群盗所袭,只得住
于江亭。止有一千余众,皆老弱之辈。时当盛暑,粮食尽绝,只剩麦三十斛,分派军士。家
人无食,多有饿死者。术嫌饭粗,不能下咽,乃命庖人取蜜水止渴。庖人曰:“止有血水,
安有蜜水!”术坐于床上,大叫一声,倒于地下,吐血斗余而死。时建安四年六月也。后人
有诗曰:汉末刀兵起四方,无端袁术太猖狂,不思累世为公相,便欲孤身作帝王。强暴枉夸
传国玺,骄奢妄说应天祥。渴思蜜水无由得,独卧空床呕血亡。”袁术已死,侄袁胤将灵柩
及妻子奔庐江来,被徐璆尽杀之。璆夺得玉玺,赴许都献于曹操。操大喜,封徐璆为高陵太
守。此时玉玺归操。

却说玄德知袁术已丧,写表申奏朝廷,书呈曹操,令朱灵、路昭回许都,留下军马保守
徐州;一面亲自出城,招谕流散人民复业。

且说朱灵、路昭回许都见曹操,说玄德留下军马。操怒,欲斩二人。荀彧曰:“权归刘
备,二人亦无奈何。”操乃赦之。彧又曰:“可写书与车胄就内图之。”操从其计,暗使人
来见车胄,传曹操钧旨。胄随即请陈登商议此事。登曰:“此事极易。今刘备出城招民,不
日将还;将军可命军士伏于瓮城边,只作接他,待马到来,一刀斩之;某在城上射住后军,
大事济矣。”胄从之。陈登回见父陈珪,备言其事。珪命登先往报知玄德。登领父命,飞马
去报,正迎着关、张,报说如此如此。原来关、张先回,玄德在后。张飞听得,便要去厮
杀。云长曰:“他伏瓮城边待我,去必有失。我有一计,可杀车胄:乘夜扮作曹军到徐州,
引车胄出迎,袭而杀之。”飞然其言。那部下军原有曹操旗号,衣甲都同。当夜三更,到城
边叫门。城上问是谁,众应是曹丞相差来张文远的人马。报知车胄,胄急请陈登议曰:“若
不迎接,诚恐有疑;若出迎之,又恐有诈。”胄乃上城回言:“黑夜难以分辨,平明了相
见。”城下答应:“只恐刘备知道,疾快开门!”车胄犹豫未定,城外一片声叫开门。车胄
只得披挂上马,引一千军出城;跑过吊桥,大叫:“文远何在?”火光中只见云长提刀纵马
直迎车胄,大叫曰:“匹夫安敢怀诈,欲杀吾兄!”车胄大惊,战未数合,遮拦不住,拨马
便回。到吊桥边,城上陈登乱箭射下,车胄绕城而走。云长赶来,手起一刀,砍于马下,割
下首级提回,望城上呼曰:“反贼车胄,吾已杀之;众等无罪,投降免死!”诸军倒戈投
降,军民皆安。云长将胄头去迎玄德,具言车胄欲害之事,今已斩首。玄德大惊曰:“曹操
若来。如之奈何?”云长曰:“弟与张飞迎之。”玄德懊悔不已,遂入徐州。百姓父老,伏
道而接。玄德到府,寻张飞,飞已将车胄全家杀尽。玄德曰:“杀了曹操心腹之人,如何肯
休?”陈登曰:“某有一计,可退曹操。”正是:既把孤身离虎穴,还将妙计息狼烟。不知
陈登说出甚计来,且听下文分解。