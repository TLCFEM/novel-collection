\chapter{锺会分兵汉中道~武侯显圣定军山}

却说司马昭谓西曹掾邵悌曰:“朝臣皆言蜀未可伐,是其心怯;若使强战,必败之道也。今钟会独建伐蜀之策,是其心不怯;心不怯,则破蜀必矣。蜀既破,则蜀人心胆已裂;败军之将,不可以言勇;亡国之大夫,不可以图存。会即有异志,蜀人安能助之乎?至若魏人得胜思归,必不从会而反,更不足虑耳。此言乃吾与汝知之,切不可泄漏。”邵悌拜服。

却说钟会下寨已毕,升帐大集诸将听令。时有监军卫瓘,护军胡烈,大将田续、庞会、田章、爰青、丘建、夏侯咸、王买、皇甫闿、句安等八十余员。会曰:“必须一大将为先锋,逢山开路,遇水叠桥。谁敢当之?”一人应声曰:“某愿往。”会视之,乃虎将许褚之子许仪也。众皆曰:“非此人不可为先锋。”会唤许仪曰:“汝乃虎体猿班之将。父子有名;今众将亦皆保汝。汝可挂先锋印,领五千马军、一千步军,径取汉中。兵分三路:汝领中路,出斜谷;左军出骆谷;右军出子午谷。此皆崎岖山险之地,当今军填平道路,修理桥梁,凿山破石,勿使阻碍。如违必按军法。”许仪受命,领兵而进。钟会随后提十万余众,星夜起程。

却说邓艾在陇西,既受伐蜀之诏,一面令司马望往遏羌人,又遣雍州刺史诸葛绪,天水太守王颀,陇西太守牵弘,金城太守杨欣,各调本部兵前来听令。比及军马云集,邓艾夜作一梦:梦见登高山,望汉中,忽于脚下迸出一泉,水势上涌。须臾惊觉,浑身汗流;遂坐而待旦,乃召护卫爰邵问之。邵素明《周易》,艾备言其梦,邵答曰:“《易》云:山上有水曰蹇。蹇卦者:‘利西南,不利东北。’孔子云:‘蹇利西南,往有功也;不利东北,其道穷也。’将军此行,必然克蜀;但可惜蹇滞不能还。”艾闻言,愀然不乐。忽钟会檄文至,约艾起兵,于汉中取齐。艾遂遣雍州刺史诸葛绪,引兵一万五千,先断姜维归路;次遣天水太守王颀,引兵一万五千,从左攻沓中;陇西太守牵弘,引一万五千人,从右攻沓中;又遣金城太守杨欣,引一万五千人,于甘松邀姜维之后。艾自引兵三万,往来接应。却说钟会出师之时,有百官送出城外,旌旗蔽日,铠甲凝霜,人强马壮,威风凛然。人皆称羡,惟有相国参军刘寔,微笑不语。太尉王祥见寔冷笑,就马上握其手而问曰:“钟、邓二人,此去可平蜀乎?”寔曰:“破蜀必矣。但恐皆不得还都耳。”王祥问其故,刘寔但笑而不答。祥遂不复问。

却说魏兵既发,早有细作入沓中报知姜维。维即具表申奏后主:“请降诏遣左车骑将军张翼领兵守护阳安关,右车骑将军廖化领兵守阴平桥:这二处最为要紧,若失二处,汉中不保矣。一面当遣使入吴求救。臣一面自起沓中之兵拒敌。”时后主改景耀六年为炎兴元年,日与宦官黄皓在宫中游乐。忽接姜维之表,即召黄皓问曰:“今魏国遣钟会、邓艾大起人马,分道而来,如之奈何?”皓奏曰:“此乃姜维欲立功名,故上此表。陛下宽心,勿生疑虑。臣闻城中有一师婆,供奉一神,能知吉凶,可召来问之。”后主从其言,于后殿陈设香花纸烛、享祭礼物,令黄皓用小车请入宫中,坐于龙床之上。后主焚香祝毕,师婆忽然披发跣足,就殿上跳跃数十遍,盘旋于案上。皓曰:“此神人降矣。陛下可退左右,亲祷之。”后主尽退侍臣,再拜祝之。师婆大叫曰:“吾乃西川土神也。陛下欣乐太平,何为求问他事?数年之后,魏国疆土亦归陛下矣。陛下切勿忧虑。”言讫,昏倒于地,半晌方苏。后主大喜,重加赏赐。自此深信师婆之说,遂不听姜维之言,每日只在宫中饮宴欢乐。姜维累申告急表文,皆被黄皓隐匿,因此误了大事。却说钟会大军,迤逦望汉中进发。前军先锋许仪,要立头功,先领兵至南郑关。仪谓部将曰:“过此关即汉中矣。关上不多人马,我等便可奋力抢关。”众将领命,一齐并力向前。原来守关蜀将卢逊,早知魏兵将到,先于关前木桥左右,伏下军士,装起武侯所遗十矢连弩;比及许仪兵来抢关时,一声梆子响处,矢石如雨。仪急退时,早射倒数十骑。魏兵大败。仪回报钟会。会自提帐下甲士百余骑来看,果然箭弩一齐射下。会拨马便回,关上卢逊引五百军杀下来。会拍马过桥,桥上土塌,陷住马蹄,争些儿掀下马来。马挣不起,会弃马步行;跑下桥时,卢逊赶上,一枪刺来,却被魏兵中荀恺回身一箭,射卢逊落马。钟会麾众乘势抢关,关上军士因有蜀兵在关前,不敢放箭,被钟会杀散,夺了山关。即以荀恺为护军,以全副鞍马铠甲赐之。

会唤许仪至帐下,责之曰:“汝为先锋,理合逢山开路,遇水叠桥,专一修理桥梁道路,以便行军。吾方才到桥上,陷住马蹄,几乎堕桥;若非荀恺,吾已被杀矣!汝既违军令,当按军法!”叱左右推出斩之。诸将告曰:“其父许褚有功于朝廷,望都督恕之。”会怒曰:“军法不明,何以令众?”遂令斩首示众。诸将无不骇然。时蜀将王含守乐城,蒋斌守汉城,见魏兵势大,不敢出战,只闭门自守。钟会下令曰:“兵贵神速,不可少停。”乃令前军李辅围乐城,护军荀恺围汉城,自引大军取阳安关。守关蜀将傅佥与副将蒋舒商议战守之策,舒曰:“魏兵甚众,势不可当,不如坚守为上。”佥曰:“不然。魏兵远来,必然疲困,虽多不足惧。我等若不下关战时,汉、乐二城休矣。”蒋舒默然不答。忽报魏兵大队已至关前,蒋、傅二人至关上视之。钟会扬鞭大叫曰:“吾今统十万之众到此,如早早出降,各依品级升用;如执迷不降,打破关隘,玉石俱焚!”傅佥大怒,令蒋舒把关,自引三千兵杀下关来。钟会便走,魏兵尽退。佥乘势追之,魏兵复合。佥欲退入关时,关上已竖起魏家旗号,只见蒋舒叫曰:“吾已降了魏也!”佥大怒,厉声骂曰:“忘恩背义之贼,有何面目见天下人乎!”拨回马复与魏兵接战。魏兵四面合来,将傅佥围在垓心。佥左冲右突,往来死战,不能得脱;所领蜀兵,十伤八九。佥乃仰天叹曰:“吾生为蜀臣,死亦当为蜀鬼!”乃复拍马冲杀,身被数枪,血盈袍铠;坐下马倒,佥自刎而死。后人有诗叹曰:“一日抒忠愤,千秋仰义名。宁为傅佥死,不作蒋舒生。”

钟会得了阳安关,关内所积粮草、军器极多,大喜,遂犒三军。是夜,魏兵宿于阳安城中,忽闻西南上喊声大震。钟会慌忙出帐视之,绝无动静。魏军一夜不敢睡。次夜三更,西南上喊声又起。钟会惊疑,向晓,使人探之。回报曰:“远哨十余里,并无一人。”会惊疑不定,乃自引数百骑,俱全装惯带,望西南巡哨。前至一山,只见杀气四面突起,愁云布合,雾锁山头。会勒住马,问向导官曰:“此何山也?”答曰:“此乃定军山,昔日夏侯渊殁于此处。”会闻之,怅然不乐,遂勒马而回。转过山坡,忽然狂风大作,背后数千骑突出,随风杀来。会大惊,引众纵马而走。诸将坠马者,不计其数。及奔到阳安关时,不曾折一人一骑,只跌损面目,失了头盔。皆言曰:“但见阴云中人马杀来,比及近身,却不伤人,只是一阵旋风而已。”会问降将蒋舒曰:“定军山有神庙乎?”舒曰:“并无神庙,惟有诸葛武侯之墓。”会惊曰:“此必武侯显圣也。吾当亲往祭之。”次日,钟会备祭礼,宰太牢,自到武侯墓前再拜致祭。祭毕,狂风顿息,愁云四散。忽然清风习习,细雨纷纷。一阵过后,天色晴朗。魏兵大喜,皆拜谢回营。是夜,钟会在帐中伏几而寝,忽然一阵清风过处,只见一人,纶巾羽扇,身衣鹤氅,素履皂绦,面如冠玉,唇若抹朱,眉清目朗,身长八尺,飘飘然有神仙之概。其人步入帐中,会起身迎之曰:“公何人也?”其人曰:“今早重承见顾。吾有片言相告:虽汉祚已衰,天命难违,然两川生灵,横罹兵革,诚可怜悯。汝入境之后,万勿妄杀生灵。”言讫,拂袖而去。会欲挽留之,忽然惊醒,乃是一梦。会知是武侯之灵,不胜惊异。于是传令前军,立一白旗,上书“保国安民”四字;所到之处,如妄杀一人者偿命。于是汉中人民,尽皆出城拜迎。会一一抚慰,秋毫无犯。后人有诗赞曰:“数万阴兵绕定军,致令钟会拜灵神。生能决策扶刘氏,死尚遗言保蜀民。”

却说姜维在沓中,听知魏兵大至,传檄廖化、张翼、董厥提兵接应;一面自分兵列将以待之。忽报魏兵至,维引兵迎之。魏阵中为首大将乃天水太守王颀也。颀出马大呼曰:“吾今大兵百万,上将千员,分二十路而进,已到成都。汝不思早降,犹欲抗拒,何不知天命耶!”维大怒,挺枪纵马,直取王颀。战不三合,颀大败而走。姜维驱兵追杀至二十里,只听得金鼓齐鸣,一枝兵摆开,旗上大书“陇西太守牵弘”字样。维笑曰:“此等鼠辈,非吾敌手!”遂催兵追之。又赶到十里,却遇邓艾倾兵杀到。两军混战。维抖擞精神,与艾战有十余合,不分胜负,后面锣鼓又鸣。维急退时,后军报说:“甘松诸寨,尽被金城太守杨欣烧毁了。”维大惊,急令副将虚立旗号,与邓艾相拒。维自撤后军,星夜来救甘松,正遇杨欣。欣不敢交战,望山路而走。维随后赶来。将至山岩下,岩上木石如雨,维不能前进。比及回到半路,蜀兵已被邓艾杀败。魏兵大队而来,将姜维围住。

维引众骑杀出重围,奔入大寨坚守,以待救兵。忽然流星马到,报说:“钟会打破阳安关,守将蒋舒归降,傅佥战死,汉中已属魏矣。乐城守将王含,汉城守将蒋斌,知汉中已失,亦开门而降。胡济抵敌不住,逃回成都求援去了。”维大惊,即传令拔寨。

是夜兵至疆川口,前面一军摆开,为首魏将,乃是金城太守杨欣。维大怒,纵马交锋,只一合,杨欣败走,维拈弓射之,连射三箭皆不中。维转怒,自折其弓,挺枪赶来。战马前失,将维跌在地上。杨欣拨回马来杀姜维。维跃起身,一枪刺去,正中杨欣马脑。背后魏兵骤至,救欣去了。维骑上从马,欲待追时,忽报后面邓艾兵到。维首尾不能相顾,遂收兵要夺汉中。哨马报说:“雍州刺史诸葛绪已断了归路。”维乃据山险下寨。魏兵屯于阴平桥头。维进退无路,长叹曰:“天丧我也!”副将宁随曰:“魏兵虽断阴平桥头,雍州必然兵少,将军若从孔函谷,径取雍州,诸葛绪必撤阴平之兵救雍州,将军却引兵奔剑阁守之,则汉中可复矣。”维从之,即发兵入孔函谷,诈取雍州。细作报知诸葛绪。绪大惊曰:“雍州是吾合守之地,倘有疏失,朝廷必然问罪。”急撤大兵从南路去救雍州,只留一枝兵守桥头。姜维入北道,约行三十里,料知魏兵起行,乃勒回兵,后队作前队,径到桥头,果然魏兵大队已去,只有些小兵把桥,被维一阵杀散,尽烧其寨栅。诸葛绪听知桥头火起,复引兵回,姜维兵已过半日了,因此不敢追赶。却说姜维引兵过了桥头,正行之间,前面一军来到,乃左将军张翼、右将军廖化也。维问之,翼曰:“黄皓听信师巫之言,不肯发兵。翼闻汉中已危,自起兵来,时阳安关已被钟会所取。今闻将军受困,特来接应。”遂合兵一处,前赴白水关。化曰:“今四面受敌,粮道不通,不如退守剑阁,再作良图。”维疑虑未决。忽报钟会、邓艾分兵十余路杀来。维欲与翼、化分兵迎之。化曰:“白水地狭路多,非争战之所,不如且退去救剑阁可也;若剑阁一失,是绝路矣。”维从之,遂引兵来投剑阁。将近关前,忽然鼓角齐鸣,喊声大起,旌旗遍竖,一枝军把住关口。正是:汉中险峻已无有,剑阁风波又忽生。未知何处之兵,且看下文分解。