\chapter{关云长单刀赴会~伏皇后为国捐生}

却说孙权要索荆州。张昭献计曰:“刘备所倚仗者,诸葛亮耳。其兄诸葛瑾今仕于吴,何不将瑾老小执下,使瑾入川告其弟,令劝刘备交割荆州:‘如其不还,必累及我老小。’亮念同胞之情,必然应允。”权曰:“诸葛瑾乃诚实君子,安忍拘其老小?”昭曰:“明教知是计策,自然放心。”权从之,召诸葛瑾老小,虚监在府;一面修书,打发诸葛瑾往西川去。

不致日,早到成都,先使人报知玄德。玄德问孔明曰:“令兄此来为何?”孔明曰:“来索荆州耳。”玄德曰:“何以答之?”孔明曰:“只须如此如此。”计会已定,孔明出郭接瑾。不到私宅,径入宾馆。参拜毕,瑾放声大哭。亮曰:“兄长有事但说。何故发哀?”瑾曰:“吾一家老小休矣!”亮曰:“莫非为不还荆州乎?因弟之故,执下兄长老小,弟心何安?兄休忧虑,弟自有计还荆州便了。”

瑾大喜,即同孔明入见玄德,呈上孙权书。玄德看了,怒曰:“孙权既以妹嫁我,却乘我不在荆州,竟将妹子潜地取去,情理难容!我正要大起川兵,杀下江南,报我之恨,却还想来索荆州乎!”孔明哭拜于地,曰:“吴侯执下亮兄长老小,倘若不还,吾兄将全家被戮。兄死,亮岂能独生?望主公看亮之面,将荆州还了东吴,全亮兄弟之情!”玄德再三不肯,孔明只是哭求。玄德徐徐曰:“既如此,看军师面,分荆州一半还之:将长沙、零陵、桂阳三郡与他。”亮曰:“既蒙见允,便可写书与云长令交割三郡。”玄德曰:“子瑜到彼,须用善言求吾弟。吾弟性如烈火,吾尚惧之。切宜仔细。”

瑾求了书,辞了玄德,别了孔明,登途径到荆州。云长请入中堂,宾主相叙。瑾出玄德书曰:“皇叔许先以三郡还东吴,望将军即日交割,令瑾好回见吾主。”云长变色曰:“吾与吾兄桃园结义,誓共匡扶汉室。荆州本大汉疆土,岂得妄以尺寸与人?将在外,君命有所不受。虽吾兄有书来,我却只不还。”瑾曰:“今吴侯执下瑾老小,若不得荆州,必将被诛。望将军怜之!”云长曰:“此是吴侯谲计,如何瞒得我过!”瑾曰:“将军何太无面目?”云长执剑在手曰:“休再言!此剑上并无面目!”关平告曰:“军师面上不好看,望父亲息怒。”云长曰:“不看军师面上,教你回不得东吴!”

瑾满面羞惭,急辞下船,再往西川见孔明。孔明已自出巡去了。瑾只得再见玄德,哭告云长欲杀之事。玄德曰:“吾弟性急,极难与言。子瑜可暂回,容吾取了东川、汉中诸郡,调云长往守之,那时方得交付荆州。”

瑾不得已,只得回东吴见孙权,具言前事。孙权大怒曰:“子瑜此去,反覆奔走,莫非皆是诸葛亮之计?”瑾曰:“非也。吾弟亦哭告玄德,方许将三郡先还,又无奈云长恃顽不肯,”孙权曰:“既刘备有先还三郡之言,便可差官前去长沙、零陵、桂阳三郡赴任,且看如何。”瑾曰:“主公所言极善。”权乃令瑾取回老小,一面差官往三郡赴任。不一日,三郡差去官吏,尽被逐回,告孙权曰:“关云长不肯相容,连夜赶逐回吴。迟后者便要杀。”

孙权大怒,差人召鲁肃责之曰:“子敬昔为刘备作保,借吾荆州;今刘备已得西川,不肯归还,子敬岂得坐视?”肃曰:“肃已思得一计,正欲告主公。”权问:“何计?”肃曰:“今屯兵于陆口,使人请关云长赴会。若云长肯来,以善言说之;如其不从,伏下刀斧手杀之。如彼不肯来,随即进兵,与决胜负,夺取荆州便了。”孙权曰:“正合吾意。可即行之。”阐泽进曰:“不可,关云长乃世之虎将,非等闲可及。恐事不谐,反遭其害。”孙权怒曰:“若如此,荆州何日可得!”便命鲁肃速行此计。肃乃辞孙权,至陆口,召吕蒙、甘宁商议,设宴于陆口寨外临江亭上,修下请书,选帐下能言快语一人为使,登舟渡江。江口关平问了,遂引使者入荆州,叩见云长,具道鲁肃相邀赴会之意,呈上请书。云长看书毕,谓来人曰:“既子敬相请,我明日便来赴宴。汝可先回。”

使者辞去。关平曰:“鲁肃相邀,必无好意;父亲何故许之?”云长笑曰:“吾岂不知耶?此是诸葛瑾回报孙权,说吾不肯还三郡,故令鲁肃屯兵陆口,邀我赴会,便索荆州。吾若不往,道吾怯矣。吾来日独驾小舟,只用亲随十余人,单刀赴会,看鲁肃如何近我!”平谏曰:“父亲奈何以万金之躯,亲蹈虎狼之穴?恐非所以重伯父之寄托也。”云长曰:“吾于千枪万刃之中,矢石交攻之际,匹马纵横,如入无人之境;岂忧江东群鼠乎!”马良亦谏曰:“鲁肃虽有长者之风,但今事急,不容不生异心。将军不可轻往。”云长曰:“昔战国时赵人蔺相如,无缚鸡之力,于渑池会上,觑秦国君臣如无物;况吾曾学万人敌者乎!既已许诺,不可失信。”良曰:“纵将军去,亦当有准备。”云长曰:“只教吾儿选快船十只,藏善水军五百,于江上等候。看吾认旗起处,便过江来。”平领命自去准备。却说使者回报鲁肃,说云长慨然应允,来日准到。肃与吕蒙商议:“此来若何?”蒙曰:“彼带军马来,某与甘宁各人领一军伏于岸侧,放炮为号,准备厮杀;如无军来,只于庭后伏刀斧手五十人,就筵间杀之。”计会已定。次日,肃令人于岸口遥望。辰时后,见江面上一只船来,梢公水手只数人,一面红旗,风中招飐,显出一个大“关”字来。船渐近岸,见云长青巾绿袍,坐于船上;傍边周仓捧着大刀;八九个关西大汉,各跨腰刀一口。鲁肃惊疑,接入庭内。叙礼毕,入席饮酒,举杯相劝,不敢仰视。云长谈笑自若。

酒至半酣,肃曰:“有一言诉与君侯,幸垂听焉:昔日令兄皇叔,使肃于吾主之前,保借荆州暂住,约于取川之后归还。今西川已得,而荆州未还,得毋失信乎?”云长曰:“此国家之事,筵间不必论之。”肃曰:“吾主只区区江东之地,而肯以荆州相借者,为念君侯等兵败远来,无以为资故也。今已得益州,则荆州自应见还;乃皇叔但肯先割三郡,而君侯又不从,恐于理上说不去。”云长曰:“乌林之役,左将军亲冒矢石,戮力破敌,岂得徒劳而无尺土相资?今足下复来索地耶?”肃曰:“不然。君侯始与皇叔同败于长坂,计穷力竭,将欲远窜,吾主矜念皇叔身无处所,不爱土地,使有所托足,以图后功;而皇叔愆德隳好,已得西川,又占荆州,贪而背义,恐为天下所耻笑。惟君侯察之。”云长曰:“此皆吾兄之事,非某所宜与也。”肃曰:“某闻君侯与皇叔桃园结义,誓同生死。皇叔即君侯也,何得推托乎?”云长未及回答,周仓在阶下厉声言曰:“天下土地,惟有德者居之。岂独是汝东吴当有耶!”云长变色而起,夺周仓所捧大刀,立于庭中,目视周仓而叱曰:“此国家之事,汝何敢多言!可速去!”仓会意,先到岸口,把红旗一招。关平船如箭发,奔过江东来。云长右手提刀,左手挽住鲁肃手,佯推醉曰:“公今请吾赴宴,莫提起荆州之事。吾今已醉,恐伤故旧之情。他日令人请公到荆州赴会,另作商议。”鲁肃魂不附体,被云长扯至江边。吕蒙、甘宁各引本部军欲出,见云长手提大刀,亲握鲁肃,恐肃被伤,遂不敢动。云长到船边,却才放手,早立于船首,与鲁肃作别。肃如痴似呆,看关公船已乘风而去。后人有诗赞关公曰:“藐视吴臣若小儿,单刀赴会敢平欺。当年一段英雄气,尤胜相如在渑池。”云长自回荆州。鲁肃与吕蒙共议:“此计又不成,如之奈何?”蒙曰:“可即申报主公,起兵与云长决战。”肃即时使人申报孙权。权闻之大怒,商议起倾国之兵,来取荆州。忽报:“曹操又起三十万大军来也!”权大惊,且教鲁肃休惹荆州之兵,移兵向合淝、濡须,以拒曹操。

却说操将欲起程南征,参军傅干,字彦材,上书谏操。书略曰:“干闻用武则先威,用文则先德;威德相济,而后王业成。往者天下大乱,明公用武攘之,十平其九;今未承王命者,吴与蜀耳。吴有长江之险,蜀有崇山之阻,难以威胜。愚以为且宜增修文德,按甲寝兵,息军养士,待时而动。今若举数十万之众,顿长江之滨,倘贼凭险深藏,使我士马不得逞其能,奇变无所用其权,则天威屈矣。惟明公详察焉。”曹操览之,遂罢南征,兴设学校,延礼文士。于是侍中王粲、杜袭、卫凯、和洽四人,议欲尊曹操为魏王。中书令荀攸曰:“不可。丞相官至魏公,荣加九锡,位已极矣。今又进升王位,于理不可。”曹操闻之,怒曰:“此人欲效荀彧耶!”荀攸知之,忧愤成疾,卧病十数日而卒,亡年五十八岁。操厚葬之,遂罢魏王事。一日,曹操带剑入宫,献帝正与伏后共坐。伏后见操来,慌忙起身。帝见曹操,战栗不已。操曰:“孙权、刘备各霸一方,不尊朝廷,当如之何?”帝曰:“尽在魏公裁处,”操怒曰:“陛下出此言,外人闻之,只道吾欺君也。”帝曰:“君若肯相辅则幸甚;不尔,愿垂恩相舍。”操闻言,怒目视帝,恨恨而出。左右或奏帝曰:“近闻魏公欲自立为王,不久必将篡位。”帝与伏后大哭。后曰:“妾父伏完常有杀操之心,妾今当修书一封,密与父图之”。帝曰:“昔董承为事不密,反遭大祸;今恐又泄漏,朕与汝皆休矣!”后曰:“旦夕如坐针毡,似此为人,不如早亡!妾看宦官中之忠义可托者,莫如穆顺,当令寄此书。”乃即召穆顺入屏后,退去左右近侍。帝后大哭告顺曰:“操贼欲为魏王,早晚必行篡夺之事。朕欲令后父伏完密图此贼,而左右之人,俱贼心腹,无可托者。欲汝将皇后密书,寄与伏完。量汝忠义,必不负朕。”顺泣曰:“臣感陛下大恩,敢不以死报!臣即请行。”后乃修书付顺。顺藏书于发中,潜出禁宫,径至伏完宅,将书呈上。完见是伏后亲笔,乃谓穆顺曰:“操贼心腹甚众,不可遽图。除非江东孙权、西川刘备,二处起兵于外,操必自往。此时却求在朝忠义之臣,一同谋之。内外夹攻,庶可有济。”顺曰:“皇丈可作书覆帝后,求密诏,暗遣人往吴、蜀二处,令约会起兵,讨贼救主。”伏完即取纸写书付顺。顺乃藏于头髻内,辞完回宫。

原来早有人报知曹操。操先于宫门等候。穆顺回遇曹操,操问:“那里去来?”顺答曰:“皇后有病,命求医去。”操曰:“召得医人何在?”顺曰:“还未召至。”操喝左右,遍搜身上,并无夹带,放行。忽然风吹落其帽。操又唤回,取帽视之,遍观无物,还帽令戴。穆顺双手倒戴其帽。操心疑,令左右搜其头发中,搜出伏完书来。操看时,书中言欲结连孙、刘为外应。操大怒,执下穆顺于密室问之,顺不肯招。操连夜点起甲兵三千,围住伏完私宅,老幼并皆拿下;搜出伏后亲笔之书,随将伏氏三族尽皆下狱。平明,使御林将军郗虑持节入宫,先收皇后玺绶。是日,帝在外殿,见郗虑引三百甲兵直入。帝问曰:“有何事?”虑曰:“奉魏公命收皇后玺。”帝知事泄,心胆皆碎。虑至后宫,伏后方起。虑便唤管玺绶人索取玉玺而出。伏后情知事发,便于殿后椒房内夹壁中藏躲。少顷,尚书令华歆引五百甲兵入到后殿,问宫人:伏后何在?”宫人皆推不知。歆教甲兵打开朱户,寻觅不见;料在壁中,便喝甲士破壁搜寻。歆亲自动手揪后头髻拖出。后曰:“望免我一命!”歆叱曰:“汝自见魏公诉去!”后披发跣足,二甲士推拥而出。原来华歆素有才名,向与邴原、管宁相友善。时人称三人为一龙:华歆为龙头,邴原为龙腹,管宁为龙尾。一日,宁与歆共种园蔬,锄地见金。宁挥锄不顾;歆拾而视之,然后掷下。又一日,宁与歆同坐观书,闻户外传呼之声,有贵人乘轩而过。宁端坐不动,歆弃书往观。宁自此鄙歆之为人,遂割席分坐,不复与之为友。后来管宁避居辽东,常戴白帽,坐卧一楼,足不履地,终身不肯仕魏;而歆乃先事孙权,后归曹操,至此乃有收捕伏皇后一事。后人有诗叹华歆曰:“华歆当日逞凶谋,破壁生将母后收。助虐一朝添虎翼,骂名千载笑龙头!”又有诗赞管宁曰:“辽东传有管宁楼,人去楼空名独留。笑杀子鱼贪富贵,岂如白帽自风流。”

且说华歆将伏后拥至外殿。帝望见后,乃下殿抱后而哭。歆曰:“魏公有命,可速行!”后哭谓帝曰:“不能复相活耶?”帝曰:“我命亦不知在何时也!”甲士拥后而去,帝捶胸大恸。见郗虑在侧,帝曰:“郗公!天下宁有是事乎!”哭倒在地。郗虑令左右扶帝入宫。华歆拿伏后见操。操骂曰:“吾以诚心待汝等,汝等反欲害我耶!吾不杀汝,汝必杀我!”喝左右乱棒打死。随即入宫,将伏后所生二子,皆鸩杀之。当晚将伏完、穆顺等宗族二百余口,皆斩于市。朝野之人,无不惊骇。时建安十九年十一月也。后人有诗叹曰:“曹瞒凶残世所无,伏完忠义欲何如。可怜帝后分离处,不及民间妇与夫!”

献帝自从坏了伏后,连日不食。操入曰:“陛下无忧,臣无异心。臣女已与陛下为贵人,大贤大孝,宜居正宫。”献帝安敢不从。于建安二十年正月朔,就庆贺正旦之节,册立曹操女曹贵人为正宫皇后。群下莫敢有言。

此时曹操威势日甚。会大臣商议收吴灭蜀之事。贾诩曰:“须召夏侯惇、曹仁二人回,商议此事。”操即时发使,星夜唤回。夏侯惇未至,曹仁先到,连夜便入府中见操。操方被酒而卧,许褚仗剑立于堂门之内,曹仁欲入,被许褚当住。曹仁大怒曰:“吾乃曹氏宗族,汝何敢阻当耶?”许褚曰:“将军虽亲,乃外藩镇守之官;许褚虽疏,现充内侍。主公醉卧堂上,不敢放入。”仁乃不敢入。曹操闻之,叹曰:“许褚真忠臣也!”不数日,夏侯惇亦至,共议征伐。惇曰:“吴、蜀急未可攻,宜先取汉中张鲁,以得胜之兵取蜀,可一鼓而下也。”曹操曰:“正合吾意。”遂起兵西征。正是:方逞凶谋欺弱主,又驱劲卒扫偏邦。未知后事如何,且看下文分解。