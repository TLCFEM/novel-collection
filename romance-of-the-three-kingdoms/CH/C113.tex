\chapter{丁奉定计斩孙綝~姜维斗阵破邓艾}

却说姜维恐救兵到,先将军器车仗,一应军需,步兵先退,然后将马军断后。细作报知邓艾。艾笑曰:“姜维知大将军兵到,故先退去。不必追之,追则中彼之计也。”乃令人哨探,回报果然骆谷道狭之处,堆积柴草,准备要烧追兵。众皆称艾曰:“将军真神算也!”遂遣使赍表奏闻。于是司马昭大喜,又加赏邓艾。却说东吴大将军孙綝,听知全端、唐咨等降魏,勃然大怒,将各人家眷,尽皆斩之。吴主孙亮,时年方十六,见綝杀戮太过,心甚不然。一日出西苑,因食生梅,令黄门取蜜。须臾取至,见蜜内有鼠粪数块,召藏吏责之。藏吏叩首曰:“臣封闭甚严,安有鼠粪?”亮曰:“黄门曾向尔求蜜食否?”藏吏曰:“黄门于数日前曾求蜜食,臣实不敢与。”亮指黄门曰:“此必汝怒藏吏不与尔蜜,故置粪于蜜中,以陷之也。”黄门不服。亮曰:“此事易知耳。若粪久在蜜中,则内外皆湿,若新在蜜中,则外湿内燥。”命剖视之,果然内燥,黄门服罪。亮之聪明,大抵如此。虽然聪明,却被孙綝把持,不能主张,綝令弟威远将军孙据入苍龙宿卫,武卫将军孙恩、偏将军孙干、长水校尉孙闿分屯诸营。

一日,吴主孙亮闷坐,黄门侍郎全纪在侧,纪乃国舅也。亮因泣告曰:“孙綝专权妄杀,欺朕太甚;今不图之,必为后患。”纪曰:“陛下但有用臣处,臣万死不辞。”亮曰:“卿可只今点起禁兵,与将军刘丞各把城门,朕自出杀孙綝。但此事切不可令卿母知之,卿母乃綝之姊也。倘若泄漏,误朕匪轻。”纪曰:“乞陛下草诏与臣。临行事之时,臣将诏示众,使綝手下人皆不敢妄动。”亮从之,即写密诏付纪。纪受诏归家,密告其父全尚。尚知此事,乃告妻曰:“三日内杀孙綝矣。”妻曰:“杀之是也。”口虽应之,却私令人持书报知孙綝。綝大怒,当夜便唤弟兄四人,点起精兵,先围大内;一面将全尚、刘丞并其家小俱拿下。比及平明,吴主孙亮听得宫门外金鼓大震,内侍慌入奏曰:“孙綝引兵围了内苑。”亮大怒,指全后骂曰:“汝父兄误我大事矣!”乃拔剑欲出。全后与侍中近臣,皆牵其衣而哭,不放亮出。孙綝先将全尚、刘丞等杀讫,然后召文武于朝内,下令曰:“主上荒淫久病,昏乱无道,不可以奉宗庙,今当废之。汝诸文武,敢有不从者,以谋叛论!”众皆畏俱,应曰:“愿从将军之令。”尚书桓彝大怒,从班部中挺然而出,指孙綝大骂曰:“今上乃聪明之主,汝何取出此乱言!吾宁死不从贼臣之命!”綝大怒,自拔剑斩之,即入内指吴主孙亮骂曰:“无道昏君!本当诛戮以谢天下!看先帝之面,废汝为会稽王,吾自选有德者立之!”叱中书郎李崇夺其玺绶,令邓程收之。亮大哭而去。后人有诗叹曰:“乱贼诬伊尹,奸臣冒霍光。可怜聪明主,不得莅朝堂。”

孙綝遣宗正孙楷、中书郎董朝,往虎林迎请琅琊王孙休为君。休字子烈,乃孙权第六子也,在虎林夜梦乘龙上天,回顾不见龙尾,失惊而觉。次日,孙楷、董朝至,拜请回都。行至曲阿,有一老人,自称姓干,名休,叩头言曰:“事久必变,愿殿下速行。”休谢之。行至布塞亭,孙恩将车驾来迎。休不敢乘辇,乃坐小车而入。百官拜迎道傍,休慌忙下车答礼。孙綝出令扶起,请入大殿,升御座即天子位。休再三谦让,方受玉玺。文官武将朝贺已毕,大赦天下,改元永安元年;封孙綝为丞相、荆州牧;多官各有封赏;又封兄之子孙皓为乌程侯。孙綝一门五侯,皆典禁兵,权倾人主。吴主孙休,恐其内变,阳示恩宠,内实防之。綝骄横愈甚。

冬十二月,奉牛酒入宫上寿,吴主孙休不受,綝怒,乃以牛酒诣左将军张布府中共饮。酒酣,乃谓布曰:“吾初废会稽王时,人皆劝吾为君。吾为今上贤,故立之。今我上寿而见拒,是将我等闲相待。吾早晚教你看!”布闻言,唯唯而已。次日,布入宫密奏孙休。休大惧,日夜不安。数日后,孙綝遣中书郎孟宗,拨与中营所管精兵一万五千,出屯武昌;又尽将武库内军器与之。于是,将军魏邈、武卫士施朔二人密奏孙休曰:“綝调兵在外,又搬尽武库内军器,早晚必为变矣。”休大惊,急召张布计议。布奏曰:“老将丁奉,计略过人,能断大事,可与议之。”休乃召奉入内,密告其事。奉奏曰:“陛下无忧。臣有一计,为国除害。”休问何计,奉曰:“来朝腊日,只推大会群臣,召綝赴席,臣自有调遣。”休大喜。奉同魏邈、施朔掌外事,张布为内应。

是夜,狂风大作,飞沙走石,将老树连根拔起。天明风定,使者奉旨来请孙綝入宫赴会。孙綝方起床,平地如人推倒,心中不悦。使者十余人,簇拥入内。家人止之曰:“一夜狂风不息,今早又无故惊倒,恐非吉兆,不可赴会。”綝曰:“吾弟兄共典禁兵,谁敢近身!倘有变动,于府中放火为号。”嘱讫,升车出内。吴主孙休忙下御座迎之,请綝高坐。酒行数巡,众惊曰:“宫外望有火起!”綝便欲起身。休止之曰:“丞相稳便。外兵自多,何足惧哉?”言未毕,左将军张布拔剑在手,引武士三十余人,抢上殿来,口中厉声而言曰:“有诏擒反贼孙綝!”綝急欲走时,早被武士擒下。綝叩头奏曰:“愿徙交州归田里。”休叱曰:“尔何不徙滕胤、吕据、王惇耶?”命推下斩之。于是张布牵孙綝下殿东斩讫。从者皆不敢动。布宣诏曰:“罪在孙綝一人,余皆不问。”众心乃安。布请孙休升五凤楼。丁奉、魏邈、施朔等,擒孙綝兄弟至,休命尽斩于市。宗党死者数百人,灭其三族,命军士掘开孙峻坟墓,戮其尸首。将被害诸葛恪、滕胤、吕据、王惇等家,重建坟墓,以表其忠。其牵累流远者,皆赦还乡里。丁奉等重加封赏。

驰书报入成都。后主刘禅遣使回贺,吴使薛珝答礼。珝自蜀中归,吴主孙休问蜀中近日作何举动。珝奏曰:“近日中常侍黄皓用事,公卿多阿附之。入其朝,不闻直言;经其野,民有菜色。所谓燕雀处堂,不知大厦之将焚者也。”休叹曰:“若诸葛武侯在时,何至如此乎!”于是又写国书,教人赍入成都,说司马昭不日篡魏,必将侵吴、蜀以示威,彼此各宜准备。姜维听得此信,忻然上表,再议出师伐魏。时蜀汉景耀元年冬,大将军姜维以廖化、张翼为先锋,王含、蒋斌为左军,蒋舒,傅佥为右军,胡济为合后,维与夏侯霸总中军,共起蜀兵二十万,拜辞后主,径到汉中。与夏侯霸商议,当先攻取何地。霸曰:“祁山乃用武之地,可以进兵,故丞相昔日六出祁山,因他处不可出也。”维从其言,遂令三军并望祁山进发,至谷口下寨。时邓艾正在祁山寨中,整点陇右之兵。忽流星马报到,说蜀兵现下三寨于谷口。艾听知,遂登高看了,回寨升帐,大喜曰:“不出吾之所料也!”原来邓艾先度了地脉,故留蜀兵下寨之地;地中自祁山寨直至蜀寨,早挖了地道,待蜀兵至时,于中取事。此时姜维至谷口分作三寨,地道正在左寨之中,乃王含、蒋斌下寨之处。邓艾唤子邓忠,与师纂各引一万兵,为左右冲击;却唤副将郑伦,引五百掘子军,于当夜二更,径从地道直至左营,于帐后地下拥出。

却说王含、蒋斌因立寨未定,恐魏兵来劫寨,不敢解甲而寝。忽闻中军大乱,急绰兵器上的马时,寨外邓忠引兵杀到。内外夹攻,王、蒋二将奋死抵敌不住,弃寨而走。姜维在帐中听得左寨中大喊,料道有内应外合之兵,遂急上马,立于中军帐前,传令曰:“如有妄动者斩!便有敌兵到营边,休要问他,只管以弓弩射之!”一面传示右营,亦不许妄动。果然魏兵十余次冲击,皆被射回。只冲杀到天明,魏兵不敢杀入。邓艾收兵回寨,乃叹曰:“姜维深得孔明之法!兵在夜而不惊,将闻变而不乱:真将才也!”次日,王含、蒋斌收聚败兵,伏于大寨前请罪。维曰:“非汝等之罪,乃吾不明地脉之故也,”又拨军马,令二将安营讫。却将伤死身尸,填于地道之中,以土掩之。令人下战书单搦邓艾来日交锋。艾忻然应之。次日,两军列于祁山之前。维按武侯八阵之法,依天、地、风、云、鸟、蛇、龙、虎之形,分布已定。邓艾出马,见维布成八卦,乃亦布之,左右前后,门户一般。维持枪纵马大叫曰:“汝效吾排八阵,亦能变阵否?”艾笑曰:“汝道此阵只汝能布耶?吾既会布阵,岂不知变阵!”艾便勒马入阵,令执法官把旗左右招飐,变成八八六十四个门户;复出阵前曰:“吾变法若何?”维曰:“虽然不差,汝敢与吾八阵相围么?”艾曰:“有何不敢!”两军各依队伍而进。艾在中军调遣。两军冲突,阵法不曾错动。姜维到中间,把旗一招,忽然变成长蛇卷地阵,将邓艾困在垓心,四面喊声大震。艾不知其阵,心中大惊。蜀兵渐渐逼近,艾引众将冲突不出。只听得蜀兵齐叫曰:“邓艾早降!”艾仰天长叹曰:“我一时自逞其能,中姜维之计矣!”忽然西北角上一彪军杀入,艾见是魏兵,遂乘势杀出。救邓艾者,乃司马望也。比及救出邓艾时,祁山九寨,皆被蜀兵所夺。艾引败兵,退于渭水南下寨。艾谓望曰:“公何以知此阵法而救出我也?”望曰:“吾幼年游学于荆南,曾与崔州平、石广元为友,讲论此阵。今日姜维所变者,乃长蛇卷地阵也。若他处击之,必不可破。吾见其头在西北,故从西北击之,自破矣。”艾谢曰:“我虽学得阵法,实不知变法。公既知此法,来日以此法复夺祁山寨栅,如何?”望曰:“我之所学,恐瞒不过姜维。”艾曰:“来日公在阵上与他斗阵法,我却引一军暗袭祁山之后。两下混战。可夺旧寨也。”于是令郑伦为先锋,艾自引军袭山后;一面令人下战书,搦姜维来日斗阵法。维批回去讫,乃谓众将曰:“吾受武侯所传密书,此阵变法共三百六十五样,按周天之数。今搦吾斗阵法,乃班门弄斧耳!但中间必有诈谋,公等知之乎?”廖化曰:“此必赚我斗阵法,却引一军袭我后也。”维笑曰:“正合我意。”即令张翼、廖化,引一万兵去山后埋伏。

次日,姜维尽拔九寨之兵,分布于祁山之前。司马望引兵离了渭南,径到祁山之前,出马与姜维答话。维曰:“汝请吾斗阵法,汝先布与吾看。”望布成了八卦。维笑曰:“此即吾所布八阵之法也,汝今盗袭,何足为奇!”望曰:“汝亦窃他人之法耳!”维曰:“此阵凡有几变?”望笑曰:“吾既能布,岂不会变?此阵有九九八十一变。”维笑曰:“汝试变来。”望入阵变了数番,复出阵曰:“汝识吾变否?”维笑曰:“吾阵法按周天三百六十五变。汝乃井底之蛙,安知玄奥乎!”望自知有此变法,实不曾学全,乃勉强折辩曰:“吾不信,汝试变来。”维曰:“汝教邓艾出来,吾当布与他看。”望曰:“邓将军自有良谋,不好阵法。”维大笑曰:“有何良谋!不过教汝赚吾在此布阵,他却引兵袭吾山后耳!”望大惊,恰欲进兵混战,被维以鞭梢一指,两翼兵先出,杀的那魏兵弃甲抛戈,各逃性命。却说邓艾催督先锋郑伦来袭山后。伦刚转过山角,忽然一声炮响,鼓角喧天,伏兵杀出:为首大将。乃廖化也。二人未及答话,两马交处,被廖化一刀,斩郑伦于马下。邓艾大惊,急勒兵退时,张翼引一军杀到。两下夹攻,魏兵大败。艾舍命突出,身被四箭。奔到谓南寨时,司马望亦到。二人商议退兵之策。望曰:“近日蜀主刘禅,宠幸中贵黄皓,日夜以酒色为乐。可用反间计召回姜维,此危可解。”艾问众谋士曰:“谁可入蜀交通黄皓?”言未毕,一人应声曰:“某愿往。”艾视之,乃襄阳党均也。艾大喜,即令党均赍金珠宝物,径到成都结连黄皓,布散流言,说姜维怨望天子,不久投魏。于是成都人人所说皆同。黄皓奏知后主,即遣人星夜宣姜维入朝。却说姜维连日搦战,邓艾坚守不出。维心中甚疑。忽使命至。诏维入朝。维不知何事,只得班师回朝。邓艾、司马望知姜维中计,遂拔渭南之兵,随后掩杀。正是:乐毅伐齐遭间阻,岳飞破敌被谗回。未知胜负如何,且看下文分解。