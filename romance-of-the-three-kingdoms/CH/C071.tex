\chapter{占对山黄忠逸待劳~据汉水赵云寡胜众}

却说孔明分付黄忠:“你既要去,吾教法正助你。凡事计议而行。吾随后拨人马来接应。”黄忠应允,和法正领本部兵去了。孔明告玄德曰:“此老将不着言语激他,虽去不能成功。他今既去,须拨人马前去接应。”乃唤赵云:“将一枝人马,从小路出奇兵接应黄忠:若忠胜,不必出战;倘忠有失,即去救应。”又遣刘封、孟达:“领三千兵于山中险要去处,多立旌旗,以壮我兵之声势,令敌人惊疑。”三人各自领兵去了。又差人往下辨,授计与马超,令他如此而行。又差严颜往巴西阆中守隘,替张飞、魏延来同取汉中。

却说张郃与夏侯尚来见夏侯渊,说:“天荡山已失,折了夏侯德、韩浩。今闻刘备亲自领兵来取汉中,可速奏魏王,早发精兵猛将,前来策应。”夏侯渊便差人报知曹洪。洪星夜前到许昌,禀知曹操。操大惊,急聚文武,商议发兵救汉中。长史刘晔进曰:“汉中若失,中原震动。大王休辞劳苦,必须亲自征讨。”操自悔曰:“恨当时不用卿言,以致如此!”忙传令旨,起兵四十万亲征。时建安二十三年秋七月也。

曹操兵分三路而进:前部先锋夏侯惇,操自领中军,使曹休押后,三军陆续起行。操骑白马金鞍,玉带锦衣;武士手执大红罗销金伞盖,左右金瓜银钺,镫棒戈矛,打日月龙凤旌旗;护驾龙虎官军二万五千,分为五队,每队五千,按青、黄、赤、白、黑五色,旗幡甲马,并依本色:光辉灿烂,极其雄壮。兵出潼关,操在马上望见一簇林木,极其茂盛,问近侍曰:“此何处也?”答曰:“此名蓝田。林木之间,乃蔡邕庄也。今邕女蔡琰,与其夫董祀居此。”原来操素与蔡邕相善。先时其女蔡琰,乃卫仲道之妻;后被北方掳去,于北地生二子,作《胡笳十八拍》,流入中原。操深怜之,使人持千金入北方赎之。左贤王惧操之势,送蔡琰还汉。操乃以琰配与董祀为妻。当日到庄前,因想起蔡邕之事,令军马先行,操引近侍百余骑,到庄门下马。时董祀出仕于外,止有蔡琰在家,琰闻操至,忙出迎接。操至堂,琰起居毕,侍立于侧。操偶见壁间悬一碑文图轴,起身观之。问于蔡琰,琰答曰:“此乃曹娥之碑也。昔和帝时,上虞有一巫者,名曹旰,能婆婆乐神;五月五日,醉舞舟中,堕江而死。其女年十四岁,绕江啼哭七昼夜,跳入波中;后五日,负父之尸浮于江面;里人葬之江边。上虞令度尚奏闻朝廷,表为孝女。度尚令邯郸淳作文镌碑以记其事。时邯郸淳年方十三岁,文不加点,一挥而就,立石墓侧,时人奇之。妾父蔡邕闻而往观,时日已暮,乃于暗中以手摸碑文而读之,索笔大书八字于其背。后人镌石,并镌此八字。”操读八字云:“黄绢幼妇,外孙齑臼。”操问琰曰:“汝解此意否?”琰曰:“虽先人遗笔,妾实不解其意。”操回顾众谋士曰:“汝等解否?”众皆不能答。于内一人出曰:“某已解其意。”操视之,乃主簿杨修也。操曰:“卿且勿言,容吾思之。”遂辞了蔡琰,引众出庄。上马行三里,忽省悟,笑谓修曰:“卿试言之。”修曰:“此隐语耳。黄绢乃颜色之丝也:色傍加丝,是绝字。幼妇者,少女也:女傍少字,是妙字。外孙乃女之子也:女傍子字,是好字。齑臼乃受五辛之器也:受傍辛字,是辞字。总而言之,是绝妙好辞四字。”操大惊曰:“正合孤意!”众皆叹羡杨修才识之敏。不一日,军至南郑。曹洪接着,备言张郃之事。操曰:“非郃之罪,胜负乃兵家常事耳。”洪曰:“目今刘备使黄忠攻打定军山,夏侯渊知大王兵至,固守未曾出战。”操曰:“若不出战,是示懦也。”便差人持节到定军山,教夏侯渊进兵。刘晔谏曰:“渊性太刚,恐中奸计。”操乃作手书与之。使命持节到渊营,渊接入。使者出书,渊拆视之。略曰:“凡为将者,当以刚柔相济,不可徒恃其勇。若但任勇,则是一夫之敌耳。吾今屯大军于南郑,欲观卿之妙才,勿辱二字可也。”夏侯渊览毕大喜。打发使命回讫,乃与张郃商议曰:“今魏王率大兵屯于南郑,以讨刘备。吾与汝久守此地,岂能建立功业?来日吾出战,务要生擒黄忠。”张郃曰:“黄忠谋勇兼备,况有法正相助,不可轻敌。此间山路险峻,只宜坚守。”渊曰:“若他人建了功劳,吾与汝有何面目见魏王耶?汝只守山,吾去出战。”遂下令曰:“谁敢出哨诱敌?”夏侯尚曰:“吾愿往。”渊曰:“汝去出哨,与黄忠交战,只宜输,不宜赢。吾有妙计,如此如此。”尚受令,引三千军离定军山大寨前行。

却说黄忠与法正引兵屯于定军山口,累次挑战,夏侯渊坚守不出;欲要进攻,又恐山路危险,难以料敌,只得据守。是日,忽报山上曹兵下来搦战。黄忠恰待引军出迎,牙将陈式曰:“将军休动,某愿当之。”忠大喜,遂令陈式引军一千,出山口列阵。夏侯尚兵至,遂与交锋。不数合,尚诈败而走。式赶去,行到半路,被两山上擂木炮石,打将下来,不能前进。正欲回时,背后夏侯渊引兵突出,陈式不能抵当,被夏侯渊生擒回寨。部卒多降。有败军逃得性命,回报黄忠,说陈式被擒。忠慌与法正商议,正曰:“渊为人轻躁,恃勇少谋。可激劝士卒,拔寨前进,步步为营,诱渊来战而擒之:此乃反客为主之法。”忠用其谋,将应有之物,尽赏三军,欢声满谷,愿效死战。黄忠即日拔寨而进,步步为营;每营住数日,又进。渊闻之,欲出战。张郃曰:“此乃反客为主之计,不可出战,战则有失。”渊不从,令夏侯尚引数千兵出战,直到黄忠寨前。忠上马提刀出迎,与夏侯尚交马,只一合,生擒夏侯尚归寨。余皆败走,回报夏侯渊。

渊急使人到黄忠寨,言愿将陈式来换夏侯尚。忠约定来日阵前相换。次日,两军皆到山谷阔处,布成阵势。黄忠、夏侯渊各立马于本阵门旗之下。黄忠带着夏侯尚,夏侯渊带着陈式,各不与袍铠,只穿蔽体薄衣。一声鼓响,陈式、侯夏尚各望本阵奔回。夏侯尚比及到阵门时,被黄忠一箭,射中后心。尚带箭而回。渊大怒,骤马径取黄忠。忠正要激渊厮杀。两将交马,战到二十余合,曹营内忽然鸣金收兵。渊慌拨马而回,被忠乘势杀了一阵。渊回阵问押阵官:“为何鸣金?”答曰:“某见山凹中有蜀兵旗幡数处,恐是伏兵,故急招将军回。”渊信其说,遂坚守不出。

黄忠逼到定军山下,与法正商议。正以手指曰:“定军山西,巍然有一座高山,四下皆是险道。此山上足可下视定军山之虚实。将军若取得此山,定军山只在掌中也。”忠仰见山头稍平,山上有些少人马。是夜二更,忠引军士鸣金击鼓,直杀上山顶。此山有夏侯渊部将杜袭守把,止有数百余人。当时见黄忠大队拥上,只得弃山而走。忠得了山顶,正与定军山相对。法正曰:“将军可守在半山,某居山顶。待夏侯渊兵至,吾举白旗为号,将军却按兵勿动;待他倦怠无备,吾却举起红旗,将军便下山击之:以逸待劳,必当取胜。”忠大喜,从其计。却说杜袭引军逃回,见夏侯渊,说黄忠夺了对山。渊大怒曰:“黄忠占了对山,不容我不出战。”张郃谏曰:“此乃法正之谋也。将军不可出战,只宜坚守。”渊曰:“占了吾对山,观吾虚实,如何不出战?”郃苦谏不听。渊分军围住对山,大骂挑战。法正在山上举起白旗;任从夏侯渊百般辱骂,黄忠只不出战。午时以后,法正见曹兵倦怠,锐气已堕,多下马坐息,乃将红旗招展,鼓角齐鸣,喊声大震,黄忠一马当先,驰下山来,犹如天崩地塌之势。夏侯渊措手不及,被黄忠赶到麾盖之下,大喝一声,犹如雷吼。渊未及相迎,黄忠宝刀已落,连头带肩,砍为两段。后人有诗赞黄忠曰:“苍头临大敌,皓首逞神威。力趁雕弓发,风迎雪刃挥。雄声如虎吼,骏马似龙飞。献馘功勋重,开疆展帝畿。”黄忠斩了夏侯渊,曹兵大溃,各自逃生。黄忠乘势去夺定军山,张郃领兵来迎。忠与陈式两下夹攻,混杀一阵,张郃败走。忽然山傍闪出一彪人马,当住去路;为首一员大将,大叫:“常山赵子龙在此!”张郃大惊,引败军夺路望定军山而走。只见前面一枝兵来迎,乃杜袭也。袭曰:“今定军山已被刘封、孟达夺了。”郃大惊,遂与杜袭引败兵到汉水扎营;一面令人飞报曹操。

操闻渊死,放声大哭,方悟管辂所言:“三八纵横”,乃建安二十四年也,“黄猪遇虎”,乃岁在己亥正月也;“定军之南”,乃定军山之南也;“伤折一股”,乃渊与操有兄弟之亲情也。操令人寻管辂时,不知何处去了。操深恨黄忠,遂亲统大军,来定军山与夏侯渊报仇,令徐晃作先锋。行到汉水,张郃、杜袭接着曹操。二将曰:“今定军山已失,可将米仓山粮草移于北山寨中屯积,然后进兵。”曹操依允。

却说黄忠斩了夏侯渊首级,来葭萌关上见玄德献功。玄德大喜,加忠为征西大将军,设宴庆贺。忽牙将张著来报说:“曹操自领大军二十万,来与夏侯渊报仇。目今郃在米仓山搬运粮草,移于汉水北山脚下。”孔明曰:“今操引大兵至此,恐粮草不敷,故勒兵不进;若得一人深入其境,烧其粮草,夺其辎重,则操之锐气挫矣。”黄忠曰:“老夫愿当此任。”孔明曰:“操非夏侯渊之比,不可轻敌。”玄德曰:“夏侯渊虽是总帅,乃一勇夫耳,安及张郃?若斩得张郃,胜斩夏侯渊十倍也。”忠奋然曰:“吾愿往斩之。”孔明曰:“你可与赵子龙同领一枝兵去;凡事计议而行,看谁立功。”忠应允便行。孔明就令张著为副将同去。云谓忠曰:“今操引二十万众,分屯十营,将军在主公前要去夺粮,非小可之事。将军当用何策?”忠曰:“看我先去,如何?”云曰:“等我先去。”忠曰:“我是主将,你是副将,如何先争?”云曰:“我与你都一般为主公出力,何必计较?我二人拈阄,拈着的先去。”忠依允。当时黄忠拈着先去。云曰:“既将军先去,某当相助。可约定时刻。如将军依时而还,某按兵不动;若将军过时而不还,某即引军来接应。”忠曰:“公言是也。”于是二人约定午时为期。云回本寨,谓部将张翼曰:“黄汉升约定明日去夺粮草,若午时不回,我当往助。吾营前临汉水,地势危险;我若去时,汝可谨守寨栅,不可轻动。”张翼应诺。

却说黄忠回到寨中,谓副将张著曰;“我斩了夏侯渊,张郃丧胆;吾明日领命去劫粮草,只留五百军守营。你可助吾。今夜三更,尽皆饱食;四更离营,杀到北山脚下,先捉张郃,后劫粮草。”张著依令。当夜黄忠领人马在前,张著在后,偷过汉水,直到北山之下。东方日出,见粮积如山。有些少军士看守,见蜀兵到,尽弃而走。黄忠教马军一齐下马,取柴堆于米粮之上。正欲放火,张郃兵到,与忠混战一处。曹操闻知,急令除晃接应。晃领兵前进,将黄忠困于垓心。张著引三百军走脱,正要回寨,忽一枝兵撞出,拦住去路;为首大将,乃是文聘;后面曹兵又至,把张著围住。

却说赵云在营中,看看等到午时,不见忠回,急忙披挂上马,引三千军向前接应;临行,谓张翼曰:“汝可坚守营寨。两壁厢多设弓弩,以为准备。”翼连声应诺。云挺枪骤马直杀往前去。迎头一将拦路,乃文聘部将慕容烈也,拍马舞刀来迎赵云;被云手起一枪刺死。曹兵败走。云直杀入重围,又一枝兵截住;为首乃魏将焦炳。云喝问曰:“蜀兵何在?”炳曰:“已杀尽矣!”云大怒,骤马一枪,又刺死焦炳。杀散余兵,直至北山之下,见张郃、徐晃两人围住黄忠,军士被困多时。云大喝一声,挺枪骤马,杀入重围,左冲右突,如入无人之境。那枪浑身上下,若舞梨花;遍体纷纷,如飘瑞雪。张郃、徐晃心惊胆战,不敢迎敌。云救出黄忠,且战且走;所到之处,无人敢阻。操于高处望见,惊问众将曰:“此将何人也?”有识者告曰:“此乃常山赵子龙也。”操曰:“昔日当阳长坂英雄尚在!”急传令曰:“所到之处,不许轻敌。”赵云救了黄忠,杀透重围,有军士指曰:“东南上围的,必是副将张著。”云不回本寨,遂望东南杀来。所到之处,但见“常山赵云”四字旗号,曾在当阳长坂知其勇者,互相传说,尽皆逃窜。云又救了张著。曹操见云东冲西突,所向无前,莫敢迎敌,救了黄忠,又救了张著,奋然大怒,自领左右将士来赶赵云。云已杀回本寨。部将张翼接着,望见后面尘起,知是曹兵追来,即谓云曰:“追兵渐近,可令军士闭上寨门,上敌楼防护。”云喝曰:“休闭寨门!汝岂不知吾昔在当阳长坂时,单枪匹马,觑曹兵八十三万如草芥!今有军有将,又何惧哉!”遂拨弓弩手于寨外壕中埋伏;将营内旗枪,尽皆倒偃,金鼓不鸣。云匹马单枪,立于营门之外。却说张郃、徐晃领兵追至蜀寨,天色已暮;见寨中偃旗息鼓,又见赵云匹马单枪,立于营外,寨门大开,二将不敢前进。正疑之间,曹操亲到,急催督众军向前。众军听令,大喊一声,杀奔营前;见赵云全然不动,曹兵翻身就回。赵云把枪一招,壕中弓弩齐发。时天色昏黑,正不知蜀兵多少。操先拨回马走。只听得后面喊声大震,鼓角齐鸣,蜀兵赶来。曹兵自相践踏,拥到汉水河边,落水死者,不知其数。赵云、黄忠、张著各引兵一枝,追杀甚急。操正奔走间,忽刘封、孟达率二枝兵,从米仓山路杀来,放火烧粮草。操弃了北山粮草,忙回南郑。徐晃、张郃扎脚不住,亦弃本寨而走。赵云占了曹寨,黄忠夺了粮草,汉水所得军器无数,大获胜捷,差人去报玄德。玄德遂同孔明前至汉水,问赵云的部卒曰:“子龙如何厮杀?”军士将子龙救黄忠、拒汉水之事,细述一遍。玄德大喜,看了山前山后险峻之路,欣然谓孔明曰:“子龙一身都是胆也!”后人有诗赞曰:“昔日战长坂,威风犹未减。突阵显英雄,被围施勇敢。鬼哭与神号,天惊并地惨。常山赵子龙,一身都是胆!”于是玄德号子龙为虎威将军,大劳将士,欢宴至晚。忽报曹操复遣大军从斜谷小路而进,来取汉水。玄德笑曰:“操此来无能为也。我料必得汉水矣。”乃率兵于汉水之西以迎之。曹操命徐晃为先锋,前来决战。帐前一人出曰:“某深知地理,愿助徐将军同去破蜀。”操视之,乃巴西宕渠人也,姓王,名平,字子均;现充牙门将军。操大喜,遂命王平为副先锋,相助徐晃。操屯兵于定军山北。徐晃、王平引军至汉水,晃令前军渡水列阵。平曰:“军若渡水,倘要急退,如之奈何?”晃曰:“昔韩信背水为阵,所谓致之死地而后生也。”平曰:“不然。昔者韩信料敌人无谋而用此计;今将军能料赵云、黄忠之意否?”晃曰:“汝可引步军拒敌,看我引马军破之。”遂令搭起浮桥,随即过河来战蜀兵。正是:魏人妄意宗韩信,蜀相那知是子房。未知胜负如何,且看下文分解。