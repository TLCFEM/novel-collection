\chapter{关云长刮骨疗毒~吕子明白衣渡江}

却说曹仁见关公落马,即引兵冲出城来;被关平一阵杀回,救关公归寨,拔出臂箭。原来箭头有药,毒已入骨,右臂青肿,不能运动。关平慌与众将商议曰:“父亲若损此臂,安能出敌?不如暂回荆州调理。”于是与众将入帐见关公。公问曰:“汝等来有何事?”众对曰:“某等因见君侯右臂损伤,恐临敌致怒,冲突不便。众议可暂班师回荆州调理。”公怒曰:“吾取樊城,只在目前;取了樊城,即当长驱大进,径到许都,剿灭操贼,以安汉室。岂可因小疮而误大事?汝等敢慢吾军心耶!”平等默然而退。众将见公不肯退兵,疮又不痊,只得四方访问名医。忽一日,有人从江东驾小舟而来,直至寨前。小校引见关平。平视其人:方巾阔服,臂挽青囊;自言姓名,乃沛国谯郡人,姓华,名伦,字元化。因闻关将军乃天下英雄,今中毒箭,特来医治。平曰:“莫非昔日医东吴周泰者乎?”佗曰:“然。”平大喜,即与众将同引华佗入帐见关公。时关公本是臂疼,恐慢军心,无可消遣,正与马良弈棋;闻有医者至,即召入。礼毕,赐坐。茶罢,佗请臂视之。公袒下衣袍,伸臂令佗看视。佗曰:“此乃弩箭所伤,其中有乌头之药,直透入骨;若不早治,此臂无用矣。”公曰:“用何物治之?”佗曰:“某自有治法,但恐君侯惧耳。”公笑曰:“吾视死如归,有何惧哉?”佗曰:“当于静处立一标柱,上钉大环,请君侯将臂穿于环中,以绳系之,然后以被蒙其首。吾用尖刀割开皮肉,直至于骨,刮去骨上箭毒,用药敷之,以线缝其口,方可无事。但恐君侯惧耳。”公笑曰:“如此,容易!何用柱环?”令设酒席相待。

公饮数杯酒毕,一面仍与马良弈棋,伸臂令佗割之。佗取尖刀在手,令一小校捧一大盆于臂下接血。佗曰:“某便下手,君侯勿惊。”公曰:“任汝医治,吾岂比世间俗子惧痛者耶!”佗乃下刀,割开皮肉,直至于骨,骨上已青;佗用刀刮骨,悉悉有声。帐上帐下见者,皆掩面失色。公饮酒食肉,谈笑弈棋,全无痛苦之色。须臾,血流盈盆。佗刮尽其毒,敷上药,以线缝之。公大笑而起,谓众将曰:“此臂伸舒如故,并无痛矣。先生真神医也!”佗曰:“某为医一生,未尝见此。君侯真天神也!”后人有诗曰:“治病须分内外科,世间妙艺苦无多。神威罕及惟关将,圣手能医说华佗。”

关公箭疮既愈,设席款谢华佗。佗曰:“君侯箭疮虽治,然须爱护。切勿怒气伤触。过百日后,平复如旧矣。”关公以金百两酬之。佗曰:“某闻君侯高义,特来医治,岂望报乎!”坚辞不受,留药一帖,以敷疮口,辞别而去。

却说关公擒了于禁,斩了庞德,威名大震,华夏皆惊。探马报到许都,曹操大惊,聚文武商议曰:“某素知云长智勇盖世,今据荆襄,如虎生翼。于禁被擒,庞德被斩,魏兵挫锐;倘彼率兵直至许都,如之奈何?孤欲迁都以避之。”司马懿谏曰:“不可。于禁等被水所淹,非战之故;于国家大计,本无所损。今孙、刘失好,云长得志,孙权必不喜;大王可遣使去东吴陈说利害,令孙权暗暗起兵蹑云长之后,许事平之日,割江南之地以封孙权,则樊城之危自解矣。”主簿蒋济曰:“仲达之言是也。今可即发使往东吴,不必迁都动众。”操依允,遂不迁都;因叹谓诸将曰:“于禁从孤三十年,何期临危反不如庞德也!今一面遣使致书东吴,一面必得一大将以当云长之锐。”言未毕,阶下一将应声而出曰:“某愿往。”操视之,乃徐晃也。操大喜,遂拨精兵五万,令徐晃为将,吕建副之,克日起兵,前到阳陵坡驻扎;看东南有应,然后征进。

却说孙权接得曹操书信,览毕,欣然应允,即修书发付使者先回,乃聚文武商议。张昭曰:“近闻云长擒于禁,斩庞德,威震华夏,操欲迁都以避其锋。今樊城危急,遣使求救,事定之后,恐有反覆。”权未及发言,忽报吕蒙乘小舟自陆口来,有事面禀。权召入问之,蒙曰:“今云长提兵围樊城,可乘其远出,袭取荆州。”权曰:“孤欲北取徐州,如何?”蒙曰:“今操远在河北,未暇东顾,徐州守兵无多,往自可克;然其地势利于陆战,不利水战,纵然得之,亦难保守。不如先取荆州,全据长江,别作良图。”权曰:“孤本欲取荆州,前言特以试卿耳。卿可速为孤图之。孤当随后便起兵也。”

吕蒙辞了孙权,回至陆口,早有哨马报说:“沿江上下,或二十里,或三十里,高阜处各有烽火台。”又闻荆州军马整肃,预有准备,蒙大惊曰:“若如此,急难图也。我一时在吴侯面前劝取荆州,今却如何处置?”寻思无计,乃托病不出,使人回报孙权。权闻吕蒙患病,心甚怏怏。陆逊进言曰:“吕子明之病,乃诈耳,非真病也。”权曰:“伯言既知其诈,可往视之。”陆逊领命,星夜至陆口寨中,来见吕蒙,果然面无病色。逊曰:“某奉吴侯命,敬探子明贵恙。”蒙曰:“贱躯偶病,何劳探问。”逊曰:“吴侯以重任付公,公不乘时而动,空怀郁结,何也?”蒙目视陆逊,良久不语。逊又曰:“愚有小方,能治将军之疾,未审可用否?”蒙乃屏退左右而问曰:“伯言良方,乞早赐教。”逊笑曰:“子明之疾,不过因荆州兵马整肃,沿江有烽火台之备耳。予有一计,令沿江守吏,不能举火;荆州之兵,束手归降,可乎?”蒙惊谢曰:“伯言之语,如见我肺腑。愿闻良策。”陆逊曰:“云长倚恃英雄,自料无敌,所虑者惟将军耳。将军乘此机会,托疾辞职,以陆口之任让之他人,使他人卑辞赞美关公,以骄其心,彼必尽撤荆州之兵,以向樊城。若荆州无备,用一旅之师,别出奇计以袭之,则荆州在掌握之中矣。”蒙大喜曰:“真良策也!”

由是吕蒙托病不起,上书辞职。陆逊回见孙权,具言前计。孙权乃召吕蒙还建业养病。蒙至,入见权,权问曰:“陆口之任,昔周公谨荐鲁子敬以自代,后子敬又荐卿自代,今卿亦须荐一才望兼隆者,代卿为妙。”蒙曰:“若用望重之人,云长必然提备。陆逊意思深长,而未有远名,非云长所忌;若即用以代臣之任,必有所济。”权大喜,即日拜陆逊为偏将军、右都督,代蒙守陆口。逊谢曰:“某年幼无学,恐不堪重任。”权曰:“子明保卿,必不差错。卿毋得推辞。”逊乃拜受印绶,连夜往陆口;交割马步水三军已毕,即修书一封,具名马、异锦、酒礼等物,遣使赍赴樊城见关公。

时公正将息箭疮,按兵不动。忽报:“江东陆口守将吕蒙病危,孙权取回调理,近拜陆逊为将,代吕蒙守陆口。今逊差人赍书具礼,特来拜见。”关公召入,指来使而言曰:“仲谋见识短浅,用此孺子为将!”来使伏地告曰:“陆将军呈书备礼:一来与君侯作贺,二来求两家和好。幸乞笑留。”公拆书视之,书词极其卑谨。关公览毕,仰面大笑,令左右收了礼物,发付使者回去。使者回见陆逊曰:“关公欣喜,无复有忧江东之意。”

逊大喜,密遣人探得关公果然撤荆州大半兵赴樊城听调,只待箭疮痊可,便欲进兵。逊察知备细,即差人星夜报知孙权,孙权召吕蒙商议曰:“今云长果撤荆州之兵,攻取樊城,便可设计袭取荆州。卿与吾弟孙皎同引大军前去,何如?”孙皎字叔明,乃孙权叔父孙静之次子也。蒙曰:“主公若以蒙可用则独用蒙;若以叔明可用则独用叔明。岂不闻昔日周瑜、程普为左右都督,事虽决于瑜,然普自以旧臣而居瑜下,颇不相睦;后因见瑜之才,方始敬服?今蒙之才不及瑜,而叔明之亲胜于普,恐未必能相济也。”

权大悟,遂拜吕蒙为大都督,总制江东诸路军马;令孙皎在后接应粮草。蒙拜谢,点兵三万,快船八十余只,选会水者扮作商人,皆穿白衣,在船上摇橹,却将精兵伏于【舟冓】【舟鹿】船中。次调韩当、蒋钦、朱然、潘璋、周泰、徐盛、丁奉等七员大将,相继而进。其余皆随吴侯为合后救应。一面遣使致书曹操,令进兵以袭云长之后;一面先传报陆逊,然后发白衣人,驾快船往浔阳江去。昼夜趱行,直抵北岸。江边烽火台上守台军盘问时,吴人答曰:“我等皆是客商,因江中阻风,到此一避。”随将财物送与守台军士。军士信之,遂任其停泊江边。约至二更,【舟冓】【舟鹿】中精兵齐出,将烽火台上官军缚倒,暗号一声,八十余船精兵俱起,将紧要去处墩台之军,尽行捉入船中,不曾走了一个。于是长驱大进,径取荆州,无人知觉。将至荆州,吕蒙将沿江墩台所获官军,用好言抚慰,各各重赏,令赚开城门,纵火为号。众军领命,吕蒙便教前导。比及半夜,到城下叫门。门吏认得是荆州之兵,开了城门。众军一声喊起,就城门里放起号火。吴兵齐入,袭了荆州。吕蒙便传令军中:“如有妄杀一人,妄取民间一物者,定按军法。”原任官吏,并依旧职。将关公家属另养别宅,不许闲人搅扰。一面遣人申报孙权。

一日大雨,蒙上马引数骑点看四门。忽见一人取民间箸笠以盖铠甲,蒙喝左右执下问之,乃蒙之乡人也。蒙曰:“汝虽系我同乡,但吾号令已出,汝故犯之,当按军法。”其人泣告曰:“其恐雨湿官铠,故取遮盖,非为私用。乞将军念同乡之情!”蒙曰:“吾固知汝为覆官铠,然终是不应取民间之物。”叱左右推下斩之。枭首传示毕,然后收其尸首,泣而葬之。自是三军震肃。不一日,孙权领众至。吕蒙出郭迎接入衙。权慰劳毕,仍命潘浚为治中,掌荆州事;监内放出于禁,遣归曹操;安民赏军,设宴庆贺。权谓吕蒙曰:“今荆州已得,但公安傅士仁、南郡糜芳,此二处如何收复?”言未毕,忽一人出曰:“不须张弓只箭,某凭三寸不烂之舌,说公安傅士仁来降,可乎?”众视之,乃虞翻也。权曰:“仲翔有何良策,可使傅士仁归降?”翻曰:“某自幼与士仁交厚;今若以利害说之,彼必归矣。”权大喜,遂令虞翻领五百军,径奔公安来。

却说傅士仁听知荆州有失,急令闭城坚守。虞翻至,见城门紧闭,遂写书拴于箭上,射入城中。军士拾得,献与傅士仁。士仁拆书视之,乃招降之意。览毕,想起“关公去日恨吾之意,不如早降。”即令大开城门,请虞翻入城。二人礼毕,各诉旧情。翻说吴侯宽洪大度,礼贤下土;士仁大喜,即同虞翻赍印绶来荆州投降。孙权大悦,仍令去守公安。吕蒙密谓权曰:“今云长未获,留士仁于公安,久必有变;不若使往南郡招糜芳归降。”权乃召傅士仁谓曰:“糜芳与卿交厚,卿可招来归降,孤自当有重赏。”傅士仁慨然领诺,遂引十余骑,径投南郡招安糜芳。正是:今日公安无守志,从前王甫是良言。未知此去如何,且看下文分解。