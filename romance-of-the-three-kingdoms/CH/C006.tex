\chapter{焚金阙董卓行凶~匿玉玺孙坚背约}

却说张飞拍马赶到关下,关上矢石如雨,不得进而回。八路诸侯,同请玄德、关、张贺
功,使人去袁绍寨中报捷。绍遂移檄孙坚,令其进兵。坚引程普、黄盖至袁术寨中相见。坚
以杖画地曰:“董卓与我,本无仇隙。今我奋不顾身,亲冒矢石,来决死战者,上为国家讨
贼,下为将军家门之私;而将军却听谗言,不发粮草,致坚败绩,将军何安?”术惶恐无
言,命斩进谗之人,以谢孙坚。

忽人报坚曰:“关上有一将,乘马来寨中,要见将军。”坚辞袁术,归到本寨,唤来问
时,乃董卓爱将李傕。坚曰:“汝来何为?”傕曰:“丞相所敬者,惟将军耳。今特使傕来
结亲:丞相有女,欲配将军之子。”坚大怒,叱曰:“董卓逆天无道,荡覆王室,吾欲夷其
九族,以谢天下,安肯与逆贼结亲耶!吾不斩汝,汝当速去,早早献关,饶你性命!倘若迟
误,粉骨碎身!”李傕抱头鼠窜,回见董卓,说孙坚如此无礼。卓怒,问李儒。儒曰:“温
侯新败,兵无战心。不若引兵回洛阳,迁帝于长安,以应童谣。近日街市童谣曰:西头一个
汉,东头一个汉。鹿走入长安,方可无斯难。臣思此言‘西头一个汉’,乃应高祖旺于西都
长安,传一十二帝;‘东头一个汉’,乃应光武旺于东都洛阳,今亦传一十二帝。天运合
回。丞相迁回长安,方可无虞。”卓大喜曰:“非汝言,吾实不悟。”遂引吕布星夜回洛
阳,商议迁都。聚文武于朝堂,卓曰:“汉东都洛阳,二百余年,气数已衰。吾观旺气实在
长安,吾欲奉驾西幸。汝等各宜促装。”司徒杨彪曰:“关中残破零落。今无故捐宗庙,弃
皇陵,恐百姓惊动。天下动之至易,安之至难。望丞相监察。”卓怒曰:“汝阻国家大计
耶?”太尉黄琬曰:“杨司徒之言是也。往者王莽篡逆,更始赤眉之时,焚烧长安,尽为瓦
砾之地;更兼人民流移,百无一二。今弃宫室而就荒地,非所宜也。”卓曰:“关东贼起,
天下播乱。长安有崤函之险;更近陇右,木石砖瓦,克日可办,宫室营造,不须月余。汝等
再休乱言。”司徒荀爽谏曰:“丞相若欲迁都,百姓骚动不宁矣。”卓大怒曰:“吾为天下
计,岂惜小民哉!”即日罢杨彪、黄琬、荀爽为庶民。卓出上车,只见二人望车而揖,视
之,乃尚书周毖、城门校尉伍琼也。卓问有何事,毖曰:“今闻丞相欲迁都长安,故来谏
耳。”卓大怒曰:“我始初听你两个,保用袁绍;今绍已反,是汝等一党!”叱武士推出都
门斩首。遂下令迁都,限来日便行。李儒曰:“今钱粮缺少,洛阳富户极多,可籍没入官。
但是袁绍等门下,杀其宗党而抄其家赀,必得巨万。”卓即差铁骑五千、遍行捉拿洛阳富
户,共数千家,插旗头上大书“反臣逆党”,尽斩于城外,取其金赀。

李傕、郭汜尽驱洛阳之民数百万口,前赴长安。每百姓一队,间军一队,互相拖押;死
于沟壑者,不可胜数。又纵军士淫人妻女,夺人粮食;啼哭之声,震动天地。如有行得迟
者,背后三千军催督,军手执白刃,于路杀人。

卓临行,教诸门放火,焚烧居民房屋,并放火烧宗庙宫府。南北两宫,火焰相接;长乐
宫庭,尽为焦土。又差吕布发掘先皇及后妃陵寝,取其金宝。军士乘势掘官民坟冢殆尽。董
卓装载金珠缎匹好物数千余车,劫了天子并后妃等,竟望长安去了。却说卓将赵岑,见卓已
弃洛阳而去,便献了汜水关。孙坚驱兵先入。玄德、关、张杀入虎牢关,诸侯各引军入。

且说孙坚飞奔洛阳,遥望火焰冲天,黑烟铺地,二三百里,并无鸡犬人烟;坚先发兵救
灭了火,令众诸侯各于荒地上屯住军马。曹操来见袁绍曰:“今董贼西去,正可乘势追袭;
本初按兵不动,何也?”绍曰:“诸兵疲困,进恐无益。”操曰:“董贼焚烧宫室,劫迁天
子,海内震动,不知所归:此天亡之时也,一战而天下定矣。诸公何疑而不进?”众诸侯皆
言不可轻动。操大怒曰:“竖子不足与谋!”遂自引兵万余,领夏侯惇、夏侯渊、曹仁、曹
洪、李典、乐进,星夜来赶董卓。

且说董卓行至荥阳地方,太守徐荣出接。李儒曰:“丞相新弃洛阳,防有追兵。可教徐
荣伏军荥阳城外山坞之旁,若有兵追来,可竟放过;待我这里杀败,然后截住掩杀。令后来
者不敢复追。”卓从其计,又令吕布引精兵遏后。布正行间,曹操一军赶上。吕布大笑曰:
“不出李儒所料也!”将军马摆开。曹操出马,大叫:“逆贼!劫迁天子,流徙百姓,将欲
何往?”吕布骂曰:“背主懦夫,何得妄言!”夏侯惇挺枪跃马,直取吕布。战不数合,李
傕引一军,从左边杀来,操急令夏侯渊迎敌。右边喊声又起,郭汜引军杀到,操急令曹仁迎
敌。三路军马,势不可当。夏侯惇抵敌吕布不住,飞马回阵。布引铁骑掩杀,操军大败,回
望荥阳而走。走至一荒山脚下,时约二更,月明如昼。方才聚集残兵,正欲埋锅造饭,只听
得四围喊声,徐荣伏兵尽出。曹操慌忙策马,夺路奔逃,正遇徐荣,转身便走。荣搭上箭,
射中操肩膊。操带箭逃命,踅过山坡。两个军士伏于草中,见操马来,二枪齐发,操马中枪
而倒。操翻身落马,被二卒擒住。只见一将飞马而来,挥刀砍死两个步军,下马救起曹操。
操视之,乃曹洪也。操曰:“吾死于此矣,贤弟可速去!”洪曰:“公急上马!洪愿步
行。”操曰:“贼兵赶上,汝将奈何?”洪曰:“天下可无洪,不可无公。”操曰:“吾若
再生,汝之力也。”操上马,洪脱去衣甲,拖刀跟马而走。约走至四更余,只见前面一条大
河,阻住去路,后面喊声渐近。操曰:“命已至此,不得复活矣!”洪急扶操下马,脱去袍
铠,负操渡水。才过彼岸,追兵已到,隔水放箭。操带水而走。比及天明,又走三十余里,
土冈下少歇。忽然喊声起处,一彪人马赶来:却是徐荣从上流渡河来追。操正慌急间,只见
夏侯惇、夏侯渊引数十骑飞至,大喝:“徐荣无伤吾主!”徐荣便奔夏侯惇,惇挺枪来迎。
交马数合,惇刺徐荣于马下,杀散余兵。随后曹仁、李典、乐进各引兵寻到,见了曹操,忧
喜交集;聚集残兵五百余人,同回河内。卓兵自往长安。却说众诸侯分屯洛阳。孙坚救灭宫
中余火,屯兵城内,设帐于建章殿基上。坚令军士扫除宫殿瓦砾。凡董卓所掘陵寝。尽皆掩
闭。于太庙基上,草创殿屋三间,请众诸侯立列圣神位,宰太牢祀之。祭毕,皆散。坚归寨
中,是夜星月交辉,乃按剑露坐,仰观天文。见紫微垣中白气漫漫,坚叹曰:“帝星不明,
贼臣乱国,万民涂炭,京城一空!”言讫,不觉泪下。

傍有军士指曰:“殿南有五色毫光起于井中,”坚唤军士点起火把,下井打捞。捞起一
妇人尸首,虽然日久,其尸不烂:宫样装束,项下带一锦囊。取开看时,内有朱红小匣,用
金锁锁着。启视之,乃一玉玺:方圆四寸,上镌五龙交纽;傍缺一角,以黄金镶之;上有篆
文八字云:“受命于天,既寿永昌。”坚得玺,乃问程普。普曰:“此传国玺也。此玉是昔
日卞和于荆山之下,见凤凰栖于石上,载而进之楚文王。解之,果得玉。秦二十六年,令良
工琢为玺,李斯篆此八字于其上。二十八年,始皇巡狩至洞庭湖。风浪大作,舟将覆,急投
玉玺于湖而止。至三十六年,始皇巡狩至华阴,有人持玺遮道,与从者曰:‘持此还祖
龙。’言讫不见,此玺复归于秦。明年,始皇崩。后来子婴将玉玺献与汉高祖。后至王莽篡
逆,孝元皇太后将玺打王寻、苏献,崩其一角,以金镶之。光武得此宝于宜阳,传位至今。
近闻十常侍作乱,劫少帝出北邙,回宫失此宝。今天授主公,必有登九五之分。此处不可久
留,宜速回江东,别图大事。”坚曰:“汝言正合吾意。明日便当托疾辞归。”商议已定,
密谕军士勿得泄漏。

谁想数中一军,是袁绍乡人,欲假此为进身之计,连夜偷出营寨,来报袁绍。绍与之赏
赐,暗留军中。次日,孙坚来辞袁绍曰:“坚抱小疾,欲归长沙,特来别公。”绍笑曰:
“吾知公疾乃害传国玺耳。”坚失色曰:“此言何来?”绍曰:“今兴兵讨贼,为国除害。
玉玺乃朝廷之宝,公既获得,当对众留于盟主处,候诛了董卓,复归朝廷。今匿之而去,意
欲何为?”坚曰:“玉玺何由在吾处?”绍曰:“建章殿井中之物何在?”坚曰:“吾本无
之,何强相逼?”绍曰:“作速取出,免自生祸。”坚指天为誓曰:“吾若果得此宝,私自
藏匿,异日不得善终,死于刀箭之下!”众诸侯曰:“文台如此说誓,想必无之。”绍唤军
士出曰:“打捞之时,有此人否?”坚大怒,拔所佩之剑,要斩那军士。绍亦拔剑曰:“汝
斩军人,乃欺我也。”绍背后颜良、文丑皆拔剑出鞘。坚背后程普、黄盖、韩当亦掣刀在
手。众诸侯一齐劝住。坚随即上马,拔寨离洛阳而去。绍大怒,遂写书一封,差心腹人连夜
往荆州,送与刺史刘表,教就路上截住夺之。

次日,人报曹操追董卓,战于荥阳,大败而回。绍令人接至寨中,会众置酒,与操解
闷。饮宴间,操叹曰:“吾始兴大义,为国除贼。诸公既仗义而来,操之初意,欲烦本初引
河内之众,临孟津、酸枣;诸将固守成皋,据敖仓,塞轘辕、太谷,制其险要;公路率南阳
之军,驻丹、析,入武关,以震三辅。皆深沟高垒,勿与战,益为疑兵,示天下形势。以顺
诛逆,可立定也。今迟疑不进,大失天下之望。操窃耻之!”绍等无言可对。既而席散,操
见绍等各怀异心,料不能成事,自引军投扬州去了。公孙瓒谓玄德、关、张曰:“袁绍无能
为也,久必有变。吾等且归。”遂拔寨北行。至平原,令玄德为平原相,自去守地养军。兖
州太守刘岱,问东郡太守乔瑁借粮。瑁推辞不与,岱引军突入瑁营,杀死乔瑁,尽降其众。
袁绍见众人各自分散,就领兵拔寨,离洛阳,投关东去了。

却说荆州刺史刘表,字景升,山阳高平人也,乃汉室宗亲;幼好结纳,与名士七人为
友,时号“江夏八俊”。那七人:汝南陈翔,字仲麟;同郡范滂,字孟博;鲁国孔昱,字世
元;渤海范康,字仲真,山阳檀敷,字文友;同郡张俭,字元节;南阳岑咥,字公孝。刘表
与此七人为友;有延平人蒯良、蒯越,襄阳人蔡瑁为辅。当时看了袁绍书,随令蒯越、蔡瑁
引兵一万来截孙坚。坚军方到,蒯越将阵摆开,当先出马。孙坚问曰:“蒯异度何故引兵截
吾去路?”越曰:“汝既为汉臣,如何私匿传国之宝?可速留下,放汝归去!”坚大怒,命
黄盖出战。蔡瑁舞刀来迎。斗到数合,盖挥鞭打瑁正中护心镜。瑁拨回马走,孙坚乘势杀过
界口。山背后金鼓齐鸣、乃刘表亲自引军来到。孙坚就马上施礼曰:“景升何故信袁绍之
书,相逼邻郡?”表曰:“汝匿传国玺,将欲反耶?”坚曰:“吾若有此物,死于刀箭之
下!”表曰:“汝若要我听信,将随军行李,任我搜看。”坚怒曰:“汝有何力,敢小觑
我!”方欲交兵,刘表便退。坚纵马赶去,两山后伏兵齐起,背后蔡瑁、蒯越赶来,将孙坚
困在垓心。正是:玉玺得来无用处,反因此宝动刀兵。毕竟孙坚怎地脱身,且听下文分解。