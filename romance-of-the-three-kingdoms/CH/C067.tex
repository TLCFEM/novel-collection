\chapter{曹操平定汉中地~张辽威震逍遥津}

却说曹操兴师西征,分兵三队:前部先锋夏侯渊;张郃;操自领诸将居中;后部曹仁、夏侯惇,押运粮草。早有细作报入汉中来。张鲁与弟张卫,商议退敌之策。卫曰:“汉中最险无如阳平关;可于关之左右,依山傍林,下十余个寨栅,迎敌曹兵。兄在汉宁,多拨粮草应付。”张鲁依言,遣大将杨昂、杨任,与其弟即日起程。军马到阳平关,下寨已定。夏侯渊、张郃前军随到,闻阳平关已有准备,离关一十五里下寨。是夜,军士疲困,各自歇息。忽寨后一把火起,杨昂、杨任两路兵杀来劫寨。夏侯渊、张郃急上得马,四下里大兵拥入,曹兵大败,退见曹操。操怒曰:“汝二人行军许多年,岂不知兵若远行疲困,可防劫寨?如何不作准备?”欲斩二人,以明军法。众官告免。操次日自引兵为前队,见山势险恶,林木丛杂,不知路径,恐有伏兵,即引军回寨,谓许褚、徐晃二将曰:“吾若知此处如此险恶,必不起兵来。”许褚曰:“兵已至此,主公不可惮劳。”次日,操上马,只带许褚、徐晃二人,来看张卫寨栅。三匹马转过山坡,早望见张卫寨栅。操扬鞭遥指,谓二将曰:“如此坚固,急切难下!”言未已,背后一声喊起,箭如雨发。杨昂、杨任分两路杀来。操大惊。许褚大呼曰:“吾当敌贼!徐公明善保主公。”说罢,提刀纵马向前,力敌二将。杨昂、杨任不能当许褚之勇,回马退去,其余不敢向前。徐晃保着曹操奔过山坡,前面又一军到;看时,却是夏侯渊;张郃二将,听得喊声,故引军杀来接应。于是杀退杨昂、杨任,救得曹操回寨。操重赏四将。

自此两边相拒五十余日,只不交战。曹操传令退军。贾诩曰:“贼势未见强弱,主公何故自退耶?”操曰:“吾料贼兵每日提备,急难取胜。吾以退军为名,使贼懈而无备,然后分轻骑抄袭其后,必胜贼矣。”贾诩曰:“丞相神机,不可测也。”于是令夏侯渊;张郃分兵两路,各引轻骑三千,取小路抄阳平关后。曹操一面引大军拔寨尽起。杨昂听得曹兵退,请杨任商议,欲乘势击之。杨任曰:“操诡计极多,未知真实,不可追赶。”杨昂曰:“公不往,吾当自去。”杨任苦谏不从。杨昂尽提五寨军马前进,只留些少军士守寨。

是日,大雾迷漫,对面不相见。杨昂军至半路,不能行,权且扎住。却说夏侯渊一军抄过山后,见重雾垂空,又闻人语马嘶,恐有伏兵,急催人马行动,大雾中误走到杨昂寨前。守寨军士,听得马蹄响,只道是杨昂兵回,开门纳之。曹军一拥而入,见是空寨,便就寨中放起火来。五寨军士,尽皆弃寨而走。比及雾散,杨任领兵来救,与夏侯渊战不数合,背后张郃兵到。杨任杀条大路,奔回南郑。杨昂待要回时,已被夏侯渊、张郃两个占了寨栅。背后曹操大队军马赶来。两下夹攻,四边无路。杨昂欲突阵而出,正撞着张郃。两个交手,被张郃杀死。败兵回投阳平关,来见张卫。原来卫知二将败走,诸营已失,半夜弃关,奔回去了。曹操遂得阳平关并诸寨。张卫、杨任回见张鲁。卫言二将失了隘口,因此守关不住。张鲁大怒,欲斩杨任。任曰:“某曾谏杨昂,休追操兵。他不肯听信,故有此败。任再乞一军前去挑战,必斩曹操。如不胜,甘当军令。”张鲁取了军令状。杨任上马,引二万军离南郑下寨。却说曹操提军将进,先令夏侯渊领五千军,往南郑路上哨探,正迎着杨任军马,两军摆开。任遣部将昌奇出马,与渊交锋;战不三合,被渊一刀斩于马下。杨任自挺枪出马,与渊战三十余合,不分胜负。渊佯败而走,任从后追来;被渊用拖刀计,斩于马下。军士大败而回。曹操知夏侯渊斩了杨任,即时进兵,直抵南郑下寨。张鲁慌聚文武商议。阎圃曰:“某保一人,可敌曹操手下诸将。”鲁问是谁。圃曰:“南安庞德,前随马超投主公;后马超往西川,庞德卧病不曾行。现今蒙主公恩养,何不令此人去?”

张鲁大喜,即召庞德至,厚加赏劳;点一万军马,令庞德出。离城十余里,与曹兵相对,庞德出马搦战。曹操在渭桥时,深知庞德之勇,乃嘱诸将曰:“庞德乃西凉勇将,原属马超;今虽依张鲁,未称其心。吾欲得此人。汝等须皆与缓斗,使其力乏,然后擒之。”张郃先出,战了数合便退。夏侯渊也战数合退了。徐晃又战三五合也退了。临后许褚战五十余合亦退。庞德力战四将,并无惧怯。各将皆于操前夸庞德好武艺。曹操心中大喜,与众将商议:“如何得此人投降?”贾诩曰:“某知张鲁手下,有一谋士杨松。其人极贪贿赂。今可暗以金帛送之,使谮庞德于张鲁,便可图矣。”操曰:“何由得人入南郑?”诩曰:“来日交锋,诈败佯输,弃寨而走,使庞德据我寨。我却于夤夜引兵劫寨,庞德必退入城。却选一能言军士,扮作彼军,杂在阵中,便得入城。”操听其计,选一精细军校,重加赏赐,付与金掩心甲一副,今披在贴肉,外穿汉中军士号衣,先于半路上等候。

次日,先拨夏侯渊;张郃两枝军,远去埋伏;却教徐晃挑战,不数合败走。庞德招军掩杀,曹兵尽退。庞德却夺了曹操寨栅。见寨中粮草极多,大喜,即时申报张鲁;一面在寨中设宴庆贺。当夜二更之后,忽然三路火起:正中是徐晃、许褚,左张郃,右夏侯渊。三路军马,齐来劫寨。庞德不及提备,只得上马冲杀出来,望城而走。背后三路兵追来。庞德急唤开城门,领兵一拥而入。

此时细作已杂到城中,径投杨松府下谒见,具说:“魏公曹丞相久闻盛德,特使某送金甲为信。更有密书呈上。”松大喜,看了密书中言语,谓细作曰:“上覆魏公,但请放心。某自有良策奉报。”打发来人先回,便连夜入见张鲁,说庞德受了曹操贿赂,卖此一阵。张鲁大怒,唤庞德责骂,欲斩之。阎圃苦谏。张鲁曰:“你来日出战,不胜必斩!”庞德抱恨而退。次日,曹兵攻城,庞德引兵冲出。操令许褚交战。褚诈败,庞德赶来。操自乘马于山坡上唤曰:“庞令明何不早降?”庞德寻思:“拿住曹操,抵一千员上将!”遂飞马上坡。一声喊起,天崩地塌,连人和马,跌入陷坑内去;四壁钩索一齐上前,活捉了庞德,押上坡来。曹操下马,叱退军士,亲释其缚,问庞德肯降否。庞德寻思张鲁不仁,情愿拜降。曹操亲扶上马,共回大寨,故意教城上望见。人报张鲁,德与操并马而行。鲁益信杨松之言为实。次日,曹操三面竖立云梯,飞炮攻打。张鲁见其势已极,与弟张卫商议。卫曰:“放火尽烧仓廪府库,出奔南山,去守巴中可也。”杨松曰:“不如开门投降。”张鲁犹豫不定。卫曰:“只是烧了便行。”张鲁曰:“我向本欲归命国家,而意未得达;今不得已而出奔,仓廪府库,国家之有,不可废也。”遂尽封锁。是夜二更,张鲁引全家老小,开南门杀出。曹操教休追赶;提兵入南郑,见鲁封闭库藏,心甚怜之。遂差人往巴中,劝使投降。张鲁欲降,张卫不肯。杨松以密书报操,便教进兵,松为内应。操得书,亲自引兵往巴中。张鲁使弟卫领兵出敌,与许褚交锋;被褚斩于马下。败军回报张鲁,鲁欲坚守。杨松曰:“今若不出,坐而待毙矣。某守城,主公当亲与决一死战。”鲁从之。阎圃谏鲁休出。鲁不听,遂引军出迎。未及交锋,后军已走。张鲁急退,背后曹兵赶来。鲁到城下,杨松闭门不开。张鲁无路可走,操从后追至,大叫:“何不早降!”鲁乃下马投拜。操大喜;念其封仓库之心,优礼相待,封鲁为镇南将军。阎圃等皆封列侯。于是汉中皆平。曹操传令各郡分设太守,置都尉,大赏士卒。惟有杨松卖主求荣,即命斩之于市曹示众。后人有诗叹曰:“妨贤卖主逞奇功,积得金银总是空。家未荣华身受戮,令人千载笑杨松!”

曹操已得东川,主簿司马懿进曰:“刘备以诈力取刘璋,蜀人尚未归心。今主公已得汉中,益州震动。可速进兵攻之,势必瓦解。智者贵于乘时,时不可失也。”曹操叹曰:人苦不知足,既得陇,复望蜀耶?”刘晔曰:“司马仲达之言是也。若少迟缓,诸葛亮明于治国而为相,关、张等勇冠三军而为将,蜀民既定,据守关隘,不可犯矣。”操曰:“士卒远涉劳苦,且宜存恤。”遂按兵不动。却说西川百姓,听知曹操已取东川,料必来取西川,一日之间,数遍惊恐。玄德请军师商议。孔明曰:“亮有一计。曹操自退。”玄德问何计。孔明曰:“曹操分军屯合淝,惧孙权也。今我若分江夏、长沙、桂阳三郡还吴,遣舌辩之士,陈说利害,令吴起兵袭合淝,牵动其势,操必勒兵南向矣。”玄德问:“谁可为使?”伊籍曰:“某愿往。”玄德大喜,遂作书具礼,令伊籍先到荆州,知会云长,然后入吴。

到秣陵,来见孙权,先通了姓名。权召籍入。籍见权礼毕,权问曰:“汝到此何为?”籍曰:“昨承诸葛子瑜取长沙等三郡,为军师不在,有失交割,今传书送还。所有荆州南郡、零陵,本欲送还;被曹操袭取东川,使关将军无容身之地。今合淝空虚,望君侯起兵攻之,使曹操撤兵回南。吾主若取了东川,即还荆州全土。”权曰:“汝且归馆舍,容吾商议。”伊籍退出,权问计于众谋士。张昭曰:“此是刘备恐曹操取西川,故为此谋。虽然如此,可因操在汉中。乘势取合淝,亦是上计。”权从之,发付伊籍回蜀去讫,便议起兵攻操:令鲁肃收取长沙、江夏、桂阳三郡,屯兵于陆口,取吕蒙、甘宁回;又去余杭取凌统回。不一日,吕蒙、甘宁先到。蒙献策曰:“现今曹操令庐江太守朱光,屯兵于皖城,大开稻田,纳谷于合淝,以充军实。今可先取皖城,然后攻合淝。”权曰:“此计甚合吾意。”遂教吕蒙、甘宁为先锋,蒋钦、潘璋为合后,权自引周泰、陈武、董袭、徐盛为中军。时程普、黄盖、韩当在各处镇守,都未随征。却说军马渡江,取和州,径到皖城。皖城太守朱光,使人往合淝求救;一面固守城池,坚壁不出。权自到城下看时,城上箭如雨发,射中孙权麾盖。权回寨,问众将曰:“如何取得皖城?”董袭曰:“可差军士筑起土山攻之。”徐盛曰:“可竖云梯,造虹桥,下观城中而攻之。”吕蒙曰:“此法皆费日月而成,合淝救军一至,不可图矣。今我军初到,士气方锐,正可乘此锐气,奋力攻击。来日平明进兵,午未时便当破城。”权从之。次日五更饭毕,三军大进。城上矢石齐下。甘宁手执铁链,冒矢石而上。朱光令弓弩手齐射,甘宁拨开箭林,一链打倒朱光。吕蒙亲自擂鼓。士卒皆一拥而上,乱刀砍死朱光。余众多降,得了皖城,方才辰时。张辽引军至半路,哨马回报皖城已失。辽即回兵归合淝。

孙权入皖城,凌统亦引军到。权慰劳毕,大犒三军,重赏吕蒙,甘宁诸将,设宴庆功。吕蒙逊甘宁上坐,盛称其功劳。酒至半酣,凌统想起甘宁杀父之仇,又见吕蒙夸美之,心中大怒,瞪目直视良久,忽拔左右所佩之剑,立于筵上曰:“筵前无乐,看吾舞剑。”甘宁知其意,推开果桌起身,两手取两枝戟挟定,纵步出曰:“看我筵前使戟。”吕蒙见二人各无好意,便一手挽牌,一手提刀,立于其中曰:“二公虽能,皆不如我巧也。”说罢,舞起刀牌,将二人分于两下。早有人报知孙权。权慌跨马,直至筵前。众见权至,方各放下军器。权曰:“吾常言二人休念旧仇,今日又何如此?”凌统哭拜于地。孙权再三劝止。至次日,起兵进取合淝,三军尽发。

张辽为失了皖城,回到合淝,心中愁闷。忽曹操差薛悌送木匣一个,上有操封,傍书云:“贼来乃发。”是日报说孙权自引十万大军,来攻合淝。张辽便开匣观之。内书云:“若孙权至,张、李二将军出战,乐将军守城。”张辽将教帖与李典、乐进观之。乐进曰:“将军之意若何?”张辽曰:“主公远征在外,吴兵以为破我必矣。今可发兵出迎,奋力与战,折其锋锐,以安众心,然后可守也。”李典素与张辽不睦,闻辽此言,默然不答。乐进见李典不语,便道:“贼众我寡,难以迎敌,不如坚守。”张辽曰:“公等皆是私意,不顾公事。吾今自出迎敌,决一死战。”便教左右备马。李典慨然而起曰:“将军如此,典岂敢以私憾而忘公事乎?愿听指挥。”张辽大喜曰:“既曼成肯相助,来日引一军于逍遥津北埋伏:待吴兵杀过来,可先断小师桥,吾与乐文谦击之。”李典领命,自去点军埋伏。却说孙权令吕蒙、甘宁为前队,自与凌统居中,其余诸将陆续进发,望合淝杀来。吕蒙、甘宁前队兵进,正与乐进相迎。甘宁出马与乐进交锋,战不数合,乐进诈败而走。甘宁招呼吕蒙一齐引军赶去。孙权在第二队,听得前军得胜,催兵行至逍遥津北,忽闻连珠炮响,左边张辽一军杀来,右边李典一军杀来。孙权大惊,急令人唤吕蒙、甘宁回救时,张辽兵已到。凌统手下,止有三百余骑,当不得曹军势如山倒。凌统大呼曰:“主公何不速渡小师桥!”言未毕,张辽引二千余骑,当先杀至。凌统翻身死战。孙权纵马上桥,桥南已折丈余,并无一片板。孙权惊得手足无措。牙将谷利大呼曰:“主公可约马退后,再放马向前,跳过桥去。”孙权收回马来有三丈余远,然后纵辔加鞭,那马一跳飞过桥南。后人有诗曰:“的卢当日跳檀溪,又见吴侯败合淝。退后着鞭驰骏骑,逍遥津上玉龙飞。”孙权跳过桥南,徐盛、董袭驾舟相迎。凌统、谷利抵住张辽。甘宁、吕蒙引军回救,却被乐进从后追来,李典又截住厮杀,吴兵折了大半。凌统所领三百余人,尽被杀死。统身中数枪,杀到桥边,桥已折断,绕河而逃。孙权在舟中望见,急令董袭棹舟接之,乃得渡回。吕蒙、甘宁皆死命逃过河南。这一阵杀得江南人人害怕;闻张辽大名,小儿也不敢夜啼。众将保护孙权回营。权乃重赏凌统、谷利,收军回濡须,整顿船只,商议水陆并进;一面差人回江南,再起人马来助战。却说张辽闻孙权在濡须将欲兴兵进取,恐合淝兵少难以抵敌,急令薛悌星夜往汉中,报知曹操,求请救兵。操同众官议曰:“此时可收西川否?”刘晔曰:“今蜀中稍定,已有提备,不可击也。不如撤兵去救合淝之急,就下江南。”操乃留夏侯渊守汉中定军山隘口,留张郃守蒙头岩等隘口。其余军兵拔寨都起,杀奔濡须坞来。正是:铁骑甫能平陇右,旌旄又复指江南。未知胜负如何,且看下文分解。