\chapter{赵子龙力斩五将~诸葛亮智取三城}

却说孔明率兵前至沔阳,经过马超坟墓,乃令其弟马岱挂孝,孔明亲自祭之。祭毕,回
到寨中,商议进兵。忽哨马报道:“魏主曹睿遣驸马夏侯楙,调关中诸路军马,前来拒
敌。”魏延上帐献策曰:“夏侯楙乃膏粱子弟,懦弱无谋。延愿得精兵五干,取路出褒中,
循秦岭以东,当子午谷而投北,不过十日,可到安长。夏侯楙若闻某骤至,必然弃城望横门
邸阁而走。某却从东方而来,丞相可大驱士马,自斜谷而进。如此行之,则咸阳以西,一举
可定也。”孔明笑曰:“此非万全之计也。汝欺中原无好人物,倘有人进言,于山僻中以兵
截杀,非惟五千人受害,亦大伤锐气。决不可用。”魏延又曰:“丞相兵从大路进发,彼必
尽起关中之兵,于路迎敌,则旷日持久,何时而得中原?”孔明曰:“吾从陇右取平坦大
路,依法进兵,何忧不胜!”遂不用魏延之计。魏延怏怏不悦。孔明差人令赵云进兵。却说
夏侯楙在长安聚集诸路军马。时有西凉大将韩德,善使开山大斧,有万夫不当之勇,引西羌
诸路兵八万到来;见了夏侯楙,楙重赏之,就遣为先锋。德有四子,皆精通武艺,弓马过
人:长子韩瑛,次子韩瑶,三子韩琼,四子韩班。韩德带四子并西羌兵八万,取路至凤鸣
山,正遇蜀兵。两阵对圆。韩德出马,四子列于两边。德厉声大骂曰:“反国之贼,安敢犯
吾境界!”赵云大怒,挺枪纵马,单搦韩德交战。长子韩瑛,跃马来迎;战不三合,被赵云
一枪刺死于马下。次子韩瑶见之,纵马挥刀来战。赵云施逞旧日虎威,抖擞精神迎战。瑶抵
敌不住。三子韩琼,急挺方天戟骤马前来夹攻。云全然不惧,枪法不乱。四子韩琪,见二兄
战云不下,也纵马抡两口日月刀而来,围住赵云。云在中央独战三将。少时,韩琪中枪落
马,韩阵中偏将急出救去。云拖枪便走。韩琼按戟,急取弓箭射之,连放三箭,皆被云用枪
拨落。琼大怒,仍绰方天戟纵马赶来;却被云一箭射中面门,落马而死,韩瑶纵马举宝刀便
砍赵云。云弃枪于地,闪过宝刀,生擒韩瑶归阵,复纵马取枪杀过阵来。韩德见四子皆丧于
赵云之手,肝胆皆裂,先走入阵去。西凉兵素知赵云之名,今见其英勇如昔,谁敢交锋?赵
云马到处,阵阵倒退。赵云匹马单枪,往来冲突,如入无人之境。后人有诗赞曰:“忆昔常
山赵子龙,年登七十建奇功。独诛四将来冲阵,犹似当阳救主雄。”

邓芝见赵云大胜,率蜀兵掩杀,西凉兵大败而走。韩德险被赵云擒住,弃甲步行而逃。
云与邓芝收军回寨。芝贺曰:“将军寿已七旬,英勇如昨。今日阵前力斩四将,世所罕
有!”云曰:“丞相以吾年迈,不肯见用,吾故聊以自表耳。”遂差人解韩瑶,申报捷书,
以达孔明。

却说韩德引败军回见夏侯楙,哭告其事。楙自统兵来迎赵云。探马报入蜀寨,说夏侯楙
引兵到。云上马绰枪,引千余军,就凤鸣山前摆成阵势。当日,夏侯楙戴金盔,坐白马,手
提大砍刀,立在门旗之下。见赵云跃马挺枪,往来驰骋,楙欲自战。韩德曰:“杀吾四子之
仇,如何不报!”纵马轮开山大斧,直取赵云。云奋怒挺枪来迎;战不三合,枪起处,刺死
韩德于马下,急拨马直取夏侯楙。楙慌忙闪入本阵。邓芝驱兵掩杀,魏兵又折一阵,退十余

里下寨。楙连夜与众将商议曰:“吾久闻赵云之名,未尝见面;今日年老,英雄尚在,方信
当阳长坂之事。似此无人可敌,如之奈何?”参军程武,乃程昱之子也,进言曰:“某料赵
云有勇无谋,不足为虑。来日都督再引兵出,先伏两军于左右;都督临阵先退,诱赵云到伏
兵处;都督却登山指挥四面军马,重叠围住,云可擒矣。”楙从其言,遂遣董禧引三万军伏
于左,薛则引三万军伏于右。二人埋伏已定。次日,夏侯楙复整金鼓旗幡,率兵而进。赵
云、邓芝出迎。芝在马上谓赵云曰:“昨夜魏兵大败而走,今日复来,必有诈也。老将军防
之。”子龙曰:“量此乳臭小儿,何足道哉!吾今日必当擒之!”便跃马而出。魏将潘遂出
迎,战不三合,拨马便走。赵云赶去,魏阵中八员将一齐来迎。放过夏侯楙先走,八将陆续
奔走。赵云乘势追杀,邓芝引兵继进。赵云深入重地,只听得四面喊声大震。邓芝急收军退
回,左有董禧,右有薛则,两路兵杀到。邓芝兵少,不能解救。赵云被困在垓心,东冲西
突,魏兵越厚。时云手下止有千余人,杀到山坡之下,只见夏侯楙在山上指挥三军。赵云投
东则望东指,投西则望西指,因此赵云不能突围,乃引兵杀上山来。半山中擂木炮石打将下
来,不能上山。赵云从辰时杀至酉时,不得脱走,只得下马少歇,且待月明再战。却才卸甲
而坐,月光方出,忽四下火光冲天,鼓声大震,矢石如雨,魏兵杀到,皆叫曰:“赵云早
降!”云急上马迎敌。四面军马渐渐逼近,八方弩箭交射甚急,人马皆不能向前。云仰天叹
曰:“吾不服老,死于此地矣!”忽东北角上喊声大起,魏兵纷纷乱窜,一彪军杀到,为首
大将持丈八点钢矛,马项下挂一颗人头。云视之,乃张苞也。苞见了赵云,言曰:“丞相恐
老将军有夫,特遣某引五千兵接应。闻老将军被困,故杀透重围。正遇魏将薛则拦路,被某
杀之。”云大喜,即与张苞杀出西北角来。只见魏兵弃戈奔走:一彪军从外呐喊杀人,为首
大将提偃月青龙刀,手挽人头。云视之,乃关兴也。兴曰:“奉丞相之命,恐老将军有失,
特引五千兵前来接应。却才阵上逢着魏将董禧,被吾一刀斩之,枭首在此。丞相随后便到
也。”云曰:“二将军已建奇功,何不趁今日擒住夏侯楙,以定大事?”张苞闻言,遂引兵
去了。兴曰:“我也干功去。”遂亦引兵去了。云回顾左右曰:“他两个是吾子侄辈,尚且
争先干功;吾乃国家上将,朝廷旧臣,反不如此小儿耶?吾当舍老命以报先帝之恩!”于是
引兵来捉夏侯楙。当夜三路兵夹攻,大破魏军一阵。邓芝引兵接应,杀得尸横遍野,血流成
河。夏侯楙乃无谋之人,更兼年幼,不曾经战,见军大乱,遂引帐下骁将百余人,望南安郡
而走。众军因见无主,尽皆逃窜。兴、苞二将闻夏侯楙望南安郡去了,连夜赶来。楙走入城
中,令紧闭城门,驱兵守御。兴、苞二人赶到,将城围住;赵云随后也到:三面攻打。少
时,邓芝亦引兵到。一连围了十日,攻打不下。

忽报丞相留后军住沔阳,左军屯阳平,右军屯石城,自引中军来到。赵云、邓芝、关
兴、张苞皆来拜问孔明,说连日攻城不下。孔明遂乘小车亲到城边周围看了一遍,回寨升帐
而坐。众将环立听令。孔明曰:“此郡壕深城峻,不易攻也。吾正事不在此城,汝等如只久
攻,倘魏兵分道而出,以取汉中,吾军危矣。”邓芝曰:“夏侯楙乃魏之驸马,若擒此人,
胜斩百将。今困于此,岂可弃之而去?”孔明曰:“吾自有计。此处西连天水郡,北抵安定
郡,二处太守,不知何人?”探卒答曰:“天水太守马遵,安定太守崔谅。”孔明大喜,乃
唤魏延受计,如此如此;又唤关兴、张苞受计,如此如此;又唤心腹军士二人受计,如此行
之。各将领命,引兵而去。孔明却在南安城外,令军运柴草堆于城下,口称烧城。魏兵闻
知,皆大笑不惧。却说安定太守崔谅,在城中闻蜀兵围了南安,困住夏侯楙,十分慌惧,即
点军马约共四千,守住城池。忽见一人自正南而来,口称有机密事。崔谅唤入问之,答曰:
“某是夏侯都督帐下心腹将裴绪。今奉都督将令,特来求救于天水、安定二郡。南安甚急,
每日城上纵火为号,专望二郡救兵,并不见到;因复差某杀出重围,来此告急。可星夜起兵
为外应。都督若见二郡兵到,却开城门接应也。”谅曰:“有都督文书否?”绪贴肉取出,
汗已湿透;略教一视,急令手下换了乏马,便出城望天水而去。不二日,又有报马到,告天
水太守已起兵救援南安去了,教安定早早接应。崔谅与府官商议。多官曰:“若不去救,失
了南安,送了夏侯驸马,皆我两郡之罪也:只得救之。”谅即点起人马,离城而去,只留文
官守城。

崔谅提兵向南安大路进发,遥望见火光冲天,催兵星夜前进,离南安尚有五十余里,忽
闻前后喊声大震,哨马报道:“前面关兴截住去路,背后张苞杀来!”安定之兵,四下逃
窜。谅大惊,乃领手下百余人,往小路死战得脱,奔回安定。方到城壕边,城上乱箭射下
来。蜀将魏延在城上叫曰:“吾已取了城也!何不早降?”原来魏延扮作安定军,夤夜赚开
城门,蜀兵尽入,因此得了安定。

崔谅慌投天水郡来。行不到一程,前面一彪军摆开。大旗之下,一人纶巾羽扇,道袍鹤

氅,端坐于车上。谅视之,乃孔明也,急拨回马走。关兴、张苞两路兵追到,只叫:“早
降!”崔谅见四面皆是蜀兵,不得已遂降,同归大寨。孔明以上宾相待。孔明曰:“南安太
守与足下交厚否?”谅曰:“此人乃杨阜之族弟杨陵也;与某邻郡,交契甚厚。”孔明曰:
“今欲烦足下入城,说杨陵擒夏侯楙,可乎?”谅曰:“丞相若令某去,可暂退军马,容某
入城说之。”孔明从其言,即时传令,教四面军马各退二十里下寨。崔谅匹马到城边叫开城
门,入到府中,与杨陵礼毕,细言其事。陵曰:“我等受魏主大恩,安忍背之?可将计就计
而行。”遂引崔谅到夏侯楙处,备细说知。楙曰:“当用何计?”杨陵曰:“只推某献城
门,赚蜀兵入,却就城中杀之。”崔谅依计而行,出城见孔明,说:“杨陵献城门,放大军
入城,以擒夏侯楙。杨陵本欲自捉,因手下勇士不多,未敢轻动。”孔明曰:“此事至易:
今有足下原降兵百余人,于内暗藏蜀将扮作安定军马,带入城去、先伏于夏侯楙府下;却暗
约杨陵,待半夜之时,献开城门,里应外合。”崔谅暗思:“若不带蜀将去,恐孔明生疑。
且带入去,就内先斩之,举火为号,赚孔明入来,杀之可也。”因此应允。孔明嘱曰:“吾
遣亲信将关兴、张苞随足下先去,只推救军杀入城中,以安夏侯楙之心;但举火,吾当亲入
城去擒之。”时值黄昏,关兴、张苞受了孔明密计,披挂上马,各执兵器,杂在安定军中,
随崔谅来到南安城下。杨陵在城上撑起悬空板,倚定护心栏,问曰:“何处军马?”崔谅
曰:“安定救军来到。”谅先射一号箭上城,箭上带着密书曰:“今诸葛亮先遣二将,伏于
城中,要里应外合;且不可惊动,恐泄漏计策。待入府中图之。”杨陵将书见了夏侯楙,细
言其事。楙曰:“既然诸葛亮中计,可教刀斧手百余人,伏于府中。如二将随崔太守到府下
马,闭门斩之;却于城上举火,赚诸葛亮入城。伏兵齐出,亮可擒矣。”

安排已毕,杨陵回到城上言曰:“既是安定军马,可放入城。”关兴跟崔谅先行,张苞
在后。杨陵下城,在门边迎接。兴手起刀落,斩杨陵于马下。崔谅大惊,急拨马奔到吊桥
边,张苞大喝曰:“贼子休走!汝等诡计,如何瞒得丞相耶!”手起一枪,刺崔谅于马下。
关兴早到城上,放起火来。四面蜀兵齐入。夏侯楙措手不及,开南门并力杀出。一彪军拦
住,为首大将,乃是王平;交马只一合,生擒夏侯楙于马上,余皆杀死。孔明入南安,招谕
军民,秋毫无犯。众将各各献功。孔明将夏侯楙囚于车中。邓芝问曰:“丞相何故知崔谅诈
也?”孔明曰:“吾已知此人无降心,故意使入城。彼必尽情告与夏侯楙,欲将计就计而
行。吾见来情,足知其诈,复使二将同去,以稳其心。此人若有真心,必然阻当;彼忻然同
去者,恐吾疑也。他意中度二将同去,赚入城内杀之未迟;又令吾军有托,放心而进。吾已
暗嘱二将,就城门下图之。城内必无准备,吾军随后便到。此出其不意也。”众将拜服。孔
明曰:“赚崔谅者,吾使心腹人诈作魏将裴绪也。吾又去赚天水郡,至今未到,不知何故。
今可乘势取之。”乃留吴懿守南安,刘琰守安定,替出魏延军马去取天水郡。

却说天水郡太守马遵,听知夏侯楙困在南安城中,乃聚文武官商议。功曹梁绪、主簿尹
赏、主记梁虔等曰:“夏侯驸马乃金枝玉叶,倘有疏虞,难逃坐视之罪。太守何不尽起本部
兵以救之?”马遵正疑虑间,忽报夏侯驸马差心腹将裴绪到。绪入府,取公文付马遵,说:
“都督求安定、天水两郡之兵,星夜救应。”言讫,匆匆而去。次日又有报马到,称说:
“安定兵已先去了,教太守火急前来会合。”

马遵正欲起兵,忽一人自外而入曰:“太守中诸葛亮之计矣!”众视之,乃天水冀人
也,姓姜名维,字伯约。父名囧,昔日曾为天水郡功曹,因羌人乱,没于王事。维自幼博览
群书,兵法武艺,无所不通;奉母至孝,郡人敬之;后为中郎将,就参本郡军事。当日姜维
谓马遵曰:“近闻诸葛亮杀败夏侯楙,困于南安,水泄不通,安得有人自重围之中而出?又
且裴绪乃无名下将,从不曾见;况安定报马,又无公文,以此察之,此人乃蜀将诈称魏将。
赚得太守出城,料城中无备,必然暗伏一军于左近,乘虚而取天水也,”马遵大悟曰:“非
伯约之言,则误中奸计矣!”维笑曰:“太守放心。某有一计,可擒诸葛亮,解南安之
危。”正是:运筹又遇强中手,斗智还逢意外人。未知其计如何,且看下文分解。