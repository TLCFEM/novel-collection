\chapter{斩蔡阳兄弟释疑~会古城主臣聚义}

却说关公同孙乾保二嫂向汝南进发,不想夏侯惇领三百余骑,从后追来。孙乾保车仗前
行。关公回身勒马按刀问曰:“汝来赶我,有失丞相大度。”夏侯惇曰:“丞相无明文传
报,汝于路杀人,又斩吾部将,无礼太甚!我特来擒你,献与丞相发落!”言讫,便拍马挺
枪欲斗。

只见后面一骑飞来,大叫:“不可与云长交战!”关公按辔不动。来使于怀中取出公
文,谓夏侯惇曰:“丞相敬爱关将军忠义,恐于路关隘拦截,故遣某特赍公文,遍行诸
处。”惇曰:“关某于路杀把关将士,丞相知否?”来使曰:“此却未知。”惇曰:“我只
活捉他去见丞相,待丞相自放他。”关公怒曰:“吾岂惧汝耶!”拍马持刀,直取夏侯惇。
惇挺枪来迎。两马相交,战不十合,忽又一骑飞至,大叫:“二将军少歇!”惇停枪问来使
曰:“丞相叫擒关某乎?”使者曰:“非也。丞相恐守关诸将阻挡关将军,故又差某驰公文
来放行。”惇曰:“丞相知其于路杀人否?”使者曰:“未知。”惇曰:“既未知其杀人,
不可放去。”指挥手下军士,将关公围住。关公大怒,舞刀迎战。两个正欲交锋,阵后一人
飞马而来,大叫:“云长、元让,休得争战!”众视之,乃张辽也。二人各勒住马。张辽近
前言曰:“奉丞相钧旨:因闻知云长斩关杀将,恐于路有阻,特差我传谕各处关隘,任便放
行。”惇曰:“秦琪是蔡阳之甥。他将秦琪托付我处,今被关某所杀,怎肯干休?”辽曰:
“我见蔡将军,自有分解。既丞相大度,教放云长去,公等不可废丞相之意。”夏侯惇只得
将军马约退。辽曰:“云长今欲何往?”关公曰:“闻兄长又不在袁绍处,吾今将遍天下寻
之。”辽曰:“既未知玄德下落,且再回见丞相,若何?”关公笑曰:“安有是理!文远回
见丞相,幸为我谢罪。”说毕,与张辽拱手而别。于是张辽与夏侯惇领军自回。

关公赶上车仗,与孙乾说知此事。二人并马而行。行了数日,忽值大雨滂沱,行装尽
湿。遥望山冈边有一所庄院,关公引着车仗,到彼借宿。庄内一老人出迎。关公具言来意。
老人曰:“某姓郭,名常,世居于此。久闻大名,幸得瞻拜。”遂宰羊置酒相待,请二夫人
于后堂暂歇。郭常陪关公、孙乾于草堂饮酒。一边烘焙行李,一边喂养马匹。至黄昏时候,
忽见一少年,引数人入庄,径上草堂。郭常唤曰:“吾儿来拜将军。”因谓关公曰:“此愚
男也。”关公问何来。常曰:“射猎方回。”少年见过关公,即下堂去了。常流泪言曰:
“老夫耕读传家,止生此子,不务本业,惟以游猎为事。是家门不幸也!”关公曰:“方今
乱世,若武艺精熟,亦可以取功名,何云不幸?”常曰:“他若肯习武艺,便是有志之人。
今专务游荡,无所不为:老夫所以忧耳!”关公亦为叹息。

至更深,郭常辞出。关公与孙乾方欲就寝,忽闻后院马嘶人叫。关公急唤从人,却都不
应,乃与孙乾提剑往视之。只见郭常之子倒在地上叫唤,从人正与庄客厮打。公问其故。从
人曰:“此人来盗赤兔马,被马踢倒。我等闻叫唤之声,起来巡看,庄客们反来厮闹。”公
怒曰:“鼠贼焉敢盗吾马!”恰待发作,郭常奔至告曰:“不肖子为此歹事,罪合万死!奈
老妻最怜爱此子,乞将军仁慈宽恕!”关公曰:“此子果然不肖,适才老翁所言,真知子莫
若父也。我看翁面,且姑恕之。”遂分付从人看好了马,喝散庄客,与孙乾回草堂歇息。

次日,郭常夫妇出拜于堂前,谢曰:“犬子冒渎虎威,深感将军恩恕。”关公令唤出:
“我以正言教之。”常曰:“他于四更时分,又引数个无赖之徒,不知何处去了。”关公谢
别郭常,奉二嫂上车,出了庄院,与孙乾并马,护着车仗,取山路而行。不及三十里,只见
山背后拥出百余人,为首两骑马:前面那人,头裹黄巾,身穿战袍;后面乃郭常之子也。黄
巾者曰:“我乃天公将军张角部将也!来者快留下赤兔马,放你过去!”关公大笑曰:“无
知狂贼!汝既从张角为盗,亦知刘、关、张兄弟三人名字否?”黄巾者曰:“我只闻赤面长
髯者名关云长,却未识其面。汝何人也?”公乃停刀立马,解开须囊,出长髯令视之。其人
滚鞍下马,脑揪郭常之子拜献于马前。关公问其姓名。告曰:“某姓裴,名元绍。自张角死
后,一向无主,啸聚山林,权于此处藏伏。今早这厮来报:有一客人,骑一匹千里马,在我
家投宿。特邀某来劫夺此马。不想却遇将军。”郭常之子拜伏乞命。关公曰:“吾看汝父之
面,饶你性命!”郭子抱头鼠窜而去。

公谓元绍曰:“汝不识吾面,何以知吾名?”元绍曰:“离此二十里有一卧牛山。山上
有一关西人,姓周,名仓,两臂有千斤之力,板肋虬髯,形容甚伟;原在黄巾张宝部下为
将,张宝死,啸聚山林。他多曾与某说将军盛名,恨无门路相见。”关公曰:“绿林中非豪
杰托足之处。公等今后可各去邪归正,勿自陷其身。”元绍拜谢。

正说话间,遥望一彪人马来到。元绍曰:“此必周仓也。”关公乃立马待之。果见一
人,黑面长身,持枪乘马,引众而至;见了关公,惊喜曰:“此关将军也!”疾忙下马,俯
伏道傍曰:“周仓参拜。”关公曰:“壮士何处曾识关某来?”仓曰:“旧随黄巾张宝时,
曾识尊颜;恨失身贼党,不得相随。今日幸得拜见。愿将军不弃,收为步卒,早晚执鞭随
镫,死亦甘心!”公见其意甚诚,乃谓曰:“汝若随我,汝手下人伴若何?”仓曰:“愿从
则俱从;不愿从者,听之可也。”于是众人皆曰:“愿从。”关公乃下马至车前禀问二嫂。
甘夫人曰:“叔叔自离许都,于路独行至此,历过多少艰难,未尝要军马相随。前廖化欲相
投,叔既却之,今何独容周仓之众耶?我辈女流浅见,叔自斟酌。”公曰:“嫂嫂之言是
也。”遂谓周仓曰:“非关某寡情,奈二夫人不从。汝等且回山中,待我寻见兄长,必来相
招。”周仓顿首告曰:“仓乃一粗莽之夫,失身为盗;今遇将军,如重见天日,岂忍复错
过!若以众人相随为不便,可令其尽跟裴元绍去。仓只身步行,跟随将军,虽万里不辞
也!”关公再以此言告二嫂。甘夫人曰:“一二人相从,无妨于事。”公乃令周仓拨人伴随
裴元绍去。元绍曰:“我亦愿随关将军。”周仓曰:“汝若去时,人伴皆散;且当权时统
领。我随关将军去,但有住扎处,便来取你。”元绍怏怏而别。

周仓跟着关公,往汝南进发。行了数日,遥见一座山城。公问土人:“此何处也?”土
人曰:“此名古城。数月前有一将军,姓张,名飞,引数十骑到此,将县官逐去,占住古
城,招军买马,积草屯粮。今聚有三五千人马,四远无人敢敌。”关公喜曰:“吾弟自徐州
失散,一向不知下落,谁想却在此!”乃令孙乾先入城通报,教来迎接二嫂。

却说张飞在芒砀山中,住了月余,因出外探听玄德消息,偶过古城。入县借粮;县官不
肯,飞怒,因就逐去县官,夺了县印,占住城池,权且安身。当日孙乾领关公命,入城见
飞。施礼毕,具言:“玄德离了袁绍处,投汝南去了。今云长直从许都送二位夫人至此,请
将军出迎。”张飞听罢,更不回言,随即披挂持矛上马,引一千余人,径出北门。孙乾惊
讶,又不敢问,只得随出城来。关公望见张飞到来,喜不自胜,付刀与周仓接了,拍马来
迎。只见张飞圆睁环眼,倒竖虎须,吼声如雷,挥矛向关公便搠。关公大惊,连忙闪过,便
叫:“贤弟何故如此?岂忘了桃园结义耶?”飞喝曰:“你既无义,有何面目来与我相
见!”关公曰:“我如何无义?”飞曰:“你背了兄长,降了曹操,封侯赐爵。今又来赚
我!我今与你拼个死活!”关公曰:“你原来不知!我也难说。现放着二位嫂嫂在此,贤弟
请自问。”二夫人听得,揭帘而呼曰:“三叔何故如此?”飞曰:“嫂嫂住着。且看我杀了
负义的人,然后请嫂嫂入城。”甘夫人曰:“二叔因不知你等下落,故暂时栖身曹氏。今知
你哥哥在汝南,特不避险阻,送我们到此。三叔休错见了。”糜夫人曰:“二叔向在许都,
原出于无奈。”飞曰:“嫂嫂休要被他瞒过了!忠臣宁死而不辱。大丈夫岂有事二主之
理!”关公曰:“贤弟休屈了我。”孙乾曰:“云长特来寻将军。”飞喝曰:“如何你也胡
说!他那里有好心,必是来捉我!”关公曰:“我若捉你,须带军马来。”飞把手指曰:
“兀的不是军马来也!”关公回顾,果见尘埃起处,一彪人马来到。风吹旗号,正是曹军。
张飞大怒曰:“今还敢支吾么?”挺丈八蛇矛便搠将来。关公急止之曰:“贤弟且住。你看
我斩此来将,以表我真心。”飞曰:“你果有真心,我这里三通鼓罢。便要你斩来将!”关
公应诺。须臾,曹军至。为首一将,乃是蔡阳,挺刀纵马大喝曰:“你杀吾外甥秦琪,却原
来逃在此!吾奉丞相命,特来拿你!”关公更不打话,举刀便砍。张飞亲自擂鼓。只见一通
鼓未尽,关公刀起处,蔡阳头已落地。众军士俱走。关公活捉执认旗的小卒过来,问取来
由。小卒告说:“蔡阳闻将军杀了他外甥,十分忿怒,要来河北与将军交战。丞相不肯,因
差他往汝南攻刘辟。不想在这里遇着将军。”关公闻言,教去张飞前告说其事。飞将关公在
许都时事细问小卒;小卒从头至尾,说了一遍,飞方才信。

正说间,忽城中军士来报:“城南门外有十数骑来的甚紧,不知是甚人。”张飞心中疑
虑,便转出南门看时,果见十数骑轻弓短箭而来。见了张飞,滚鞍下马。视之,乃糜竺、糜
芳也。飞亦下马相见。竺曰:“自徐州失散,我兄弟二人逃难回乡。使人远近打听,知云长
降了曹操,主公在于河北;又闻简雍亦投河北去了。只不知将军在此。昨于路上遇见一伙客
人,说有一姓张的将军,如此模样,今据古城。我兄弟度量必是将军,故来寻访。幸得相
见!”飞曰:“云长兄与孙乾送二嫂方到,已知哥哥下落。”二糜大喜,同来见关公,并参
见二夫人。飞遂迎请二嫂入城。至衙中坐定,二夫人诉说关公历过之事,张飞方才大哭,参
拜云长。二糜亦俱伤感。张飞亦自诉别后之事,一面设宴贺喜。

次日,张飞欲与关公同赴汝南见玄德。关公曰:“贤弟可保护二嫂,暂住此城,待我与
孙乾先去探听兄长消息。”飞允诺。关公与孙乾引数骑奔汝南来。刘辟、龚都接着,关公便
问:“皇叔何在?”刘辟曰:“皇叔到此住了数日,为见军少,复往河北袁本初处商议去
了。”关公怏怏不乐。孙乾曰:“不必忧虑。再苦一番驱驰,仍往河北去报知皇叔,同至古
城便了。”关公依言,辞了刘辟、龚都,回至古城,与张飞说知此事。张飞便欲同至河北。
关公曰:“有此一城,便是我等安身之处,未可轻弃。我还与孙乾同往袁绍处,寻见兄长,
来此相会。贤弟可坚守此城。”飞曰:“兄斩他颜良、文丑,如何去得?”关公曰:“不
妨。我到彼当见机而变。”遂唤周仓问曰:“卧牛山裴元绍处,共有多少人马?”仓曰:
“约有四五百。”关公曰:“我今抄近路去寻兄长。汝可往卧牛山招此一枝人马,从大路上
接来。”仓领命而去。

关公与孙乾只带二十余骑投河北来,将至界首,乾曰:“将军未可轻入,只在此间暂
歇。待某先入见皇叔,别作商议。”关公依言,先打发孙乾去了,遥望前村有一所庄院,便
与从人到彼投宿。庄内一老翁携杖而出,与关公施礼。公具以实告。老翁曰:“某亦姓关,
名定。久闻大名,幸得瞻谒。”遂命二子出见,款留关公,并从人俱留于庄内。

且说孙乾匹马入冀州见玄德,具言前事。玄德曰:“简雍亦在此间,可暗请来同议。”
少顷,简雍至,与孙乾相见毕,共议脱身之计。雍曰:“主公明日见袁绍,只说要往荆州,
说刘表共破曹操,便可乘机而去。”玄德曰:“此计大妙!但公能随我去否?”雍曰:“某
亦自有脱身之计。”商议已定。次日,玄德入见袁绍,告曰:“刘景升镇守荆襄九郡,兵精
粮足,宜与相约,共攻曹操。”绍曰:“吾尝遣使约之,奈彼未肯相从。”玄德曰:“此人
是备同宗,备往说之,必无推阻。”绍曰:“若得刘表,胜刘辟多矣。”遂命玄德行。绍又
曰:“近闻关云长已离了曹操,欲来河北;吾当杀之,以雪颜良、文丑之恨!”玄德曰:
“明公前欲用之,吾故召之。今何又欲杀之耶?且颜良、文丑比之二鹿耳,云长乃一虎也:
失二鹿而得一虎,何恨之有?”绍笑曰:“吾实爱之,故戏言耳。公可再使人召之,令其速
来。”玄德曰:“即遣孙乾往召之可也。”绍大喜从之。玄德出,简雍进曰:“玄德此去,
必不回矣。某愿与偕往:一则同说刘表,二则监住玄德。”绍然其言,便命简雍与玄德同
行。郭图谏绍曰:“刘备前去说刘辟,未见成事;今又使与简雍同往荆州,必不返矣。”绍
曰:“汝勿多疑,简雍自有见识。”郭图嗟呀而出。却说玄德先命孙乾出城,回报关公;一
面与简雍辞了袁绍,上马出城。行至界首,孙乾接着,同往关定庄上。关公迎门接拜,执手
啼哭不止。关定领二子拜于草堂之前。玄德问其姓名。关公曰:“此人与弟同姓,有二子:
长子关宁,学文;次子关平,学武。”关定曰:“今愚意欲遣次子跟随关将军,未识肯容纳
否?”玄德曰:“年几何矣?”定曰:“十八岁矣。”玄德曰:“既蒙长者厚意,吾弟尚未
有子,今即以贤郎为子,若何?”关定大喜,便命关平拜关公为父,呼玄德为伯父。玄德恐
袁绍追之,急收拾起行。关平随着关公,一齐起身。关定送了一程自回。关公教取路往卧牛
山来。正行间,忽见周仓引数十人带伤而来。关公引他见了玄德。问其何故受伤,仓曰:
“某未至卧牛山之前,先有一将单骑而来,与裴元绍交锋,只一合,刺死裴元绍,尽数招降
人伴,占住山寨。仓到彼招诱人伴时,止有这几个过来,余者俱惧怕,不敢擅离。仓不忿,
与那将交战,被他连胜数次,身中三枪。因此来报主公。”玄德曰:“此人怎生模样?姓甚
名谁?”仓曰:“极其雄壮,不知姓名。”于是关公纵马当先,玄德在后,径投卧牛山来。
周仓在山下叫骂,只见那将全副披挂,持枪骤马,引众下山。玄德早挥鞭出马大叫曰:“来
者莫非子龙否?”那将见了玄德,滚鞍下马,拜伏道旁。原来果然是赵子龙。玄德、关公俱
下马相见,问其何由至此。云曰:“云自别使君,不想公孙瓒不听人言,以致兵败自焚,袁
绍屡次招云,云想绍亦非用人之人,因此未往。后欲至徐州投使君,又闻徐州失守,云长已
归曹操,使君又在袁绍处。云几番欲来相投,只恐袁绍见怪。四海飘零,无容身之地。前偶
过此处,适遇裴元绍下山来欲夺吾马,云因杀之,借此安身。近闻翼德在古城,欲往投之,
未知真实。今幸得遇使君!”玄德大喜,诉说从前之事。关公亦诉前事。玄德曰:“吾初见
子龙,便有留恋不舍之情。今幸得相遇!”云曰:“云奔走四方,择主而事,未有如使君
者。今得相随,大称平生。虽肝脑涂地,无恨矣。”当日就烧毁山寨,率领人众,尽随玄德
前赴古城。张飞、糜竺、糜芳迎接入城,各相拜诉。二夫人具言云长之事,玄德感叹不已。
于是杀牛宰马,先拜谢天地,然后遍劳诸军。玄德见兄弟重聚,将佐无缺,又新得了赵云,
关公又得了关平、周仓二人,欢喜无限,连饮数日。后人有诗赞之曰:“当时手足似瓜分,
信断音稀杳不闻。今日君臣重聚义,正如龙虎会风云。”时玄德、关、张、赵云、孙乾、简
雍、糜竺、糜芳、关平、周仓部领马步军校共四五千人。玄德欲弃了古城去守汝南,恰好刘
辟、龚都差人来请。于是遂起军往汝南驻扎,招军买马,徐图征进,不在话下。

且说袁绍见玄德不回,大怒,欲起兵伐之。郭图曰:“刘备不足虑。曹操乃劲敌也,不
可不除。刘表虽据荆州,不足为强。江东孙伯符威镇三江,地连六郡,谋臣武士极多,可使
人结之,共攻曹操。”绍从其言,即修书遣陈震为使,来会孙策。正是:只因河北英雄去,
引出江东豪杰来。未知其事如何,且听下文分解。