\chapter{孔明用智激周瑜~孙权决计破曹操}

却说吴国太见孙权疑惑不决,乃谓之曰:“先姊遗言云:‘伯符临终有言:内事不决问张昭,外事不决问周瑜。’今何不请公瑾问之?”权大喜,即遣使往鄱阳请周瑜议事。原来周瑜在鄱阳湖训练水师,闻曹操大军至汉上,便星夜回柴桑郡议军机事。使者未发,周瑜已先到。鲁肃与瑜最厚,先来接着,将前项事细述一番。周瑜曰:“子敬休忧,瑜自有主张。今可速请孔明来相见。”鲁肃上马去了。

周瑜方才歇息,忽报张昭、顾雍、张纮、步骘四人来相探。瑜接入堂中坐定,叙寒温毕。张昭曰:“都督知江东之利害否?”瑜曰:“未知也。”昭曰:“曹操拥众百万,屯于汉上,昨传檄文至此,欲请主公会猎于江夏。虽有相吞之意,尚未露其形。昭等劝主公且降之,庶免江东之祸。不想鲁子敬从江夏带刘备军师诸葛亮至此,彼因自欲雪愤,特下说词以激主公。子敬却执迷不悟。正欲待都督一决。”瑜曰:“公等之见皆同否?”顾雍等曰:“所议皆同。”瑜曰:“吾亦欲降久矣。公等请回,明早见主公,自有定议。”昭等辞去。

少顷,又报程普、黄盖、韩当等一班战将来见。瑜迎入,各问慰讫。程普曰:“都督知江东早晚属他人否?”瑜曰:“未知也。”普曰:“吾等自随孙将军开基创业,大小数百战,方才战得六郡城池。今主公听谋士之言,欲降曹操,此真可耻可惜之事!吾等宁死不辱。望都督劝主公决计兴兵,吾等愿效死战。”瑜曰:“将军等所见皆同否?”黄盖忿然而起,以手拍额曰:“吾头可断,誓不降曹!”众人皆曰:“吾等都不愿降!”瑜曰:“吾正欲与曹操决战,安肯投降!将军等请回。瑜见主公,自有定议。”程普等别去。

又未几,诸葛瑾、吕范等一班儿文官相候。瑜迎入,讲礼方毕,诸葛瑾曰:“舍弟诸葛亮自汉上来,言刘豫州欲结东吴,共伐曹操,文武商议未定。因舍弟为使,瑾不敢多言,专候都督来决此事。”瑜曰:“以公论之若何?”瑾曰:“降者易安,战者难保。”周瑜笑曰:“瑜自有主张。来日同至府下定议。”瑾等辞退。忽又报吕蒙、甘宁等一班儿来见。瑜请入,亦叙谈此事。有要战者,有要降者,互相争论。瑜曰:“不必多言,来日都到府下公议。”众乃辞去。周瑜冷笑不止。

至晚,人报鲁子敬引孔明来拜。瑜出中门迎入。叙礼毕,分宾主而坐。肃先问瑜曰:“今曹操驱众南侵,和与战二策,主公不能决,一听于将军。将军之意若何?”瑜曰:“曹操以天子为名,其师不可拒。且其势大,未可轻敌。战则必败,降则易安。吾意已决。来日见主公,便当遣使纳降。”鲁肃愕然曰:“君言差矣!江东基业,已历三世,岂可一旦弃于他人?伯符遗言,外事付托将军。今正欲仗将军保全国家,为泰山之靠,奈何从懦夫之议耶?”瑜曰:“江东六郡,主灵无限;若罹兵革之祸,必有归怨于我,故决计请降耳。”肃曰:“不然。以将军之英雄,东吴之险固,操未必便能得志也。”

二人互相争辩,孔明只袖手冷笑。瑜曰:“先生何故哂笑?”孔明曰:“亮不笑别人,笑子敬不识时务耳。”肃曰:“先生如何反笑我不识时务?”孔明曰:“公瑾主意欲降操,甚为合理。”瑜曰:“孔明乃识时务之士,必与吾有同心。”肃曰:“孔明,你也如何说此?”孔明曰:“操极善用兵,天下莫敢当。向只有吕布、袁绍、袁术、刘表敢与对敌。今数人皆被操灭,天下无人矣。独有刘豫州不识时务,强与争衡;今孤身江夏,存亡未保。将军决计降曹,可以保妻子,可以全富贵。国祚迁移,付之天命,何足惜哉!”鲁肃大怒曰:“汝教吾主屈膝受辱于国贼乎!”孔明曰:“愚有一计:并不劳牵羊担酒,纳土献印;亦不须亲自渡江;只须遣一介之使,扁舟送两个人到江上。操一得此两人,百万之众,皆卸甲卷旗而退矣。”瑜曰:“用何二人,可退操兵?”孔明曰:“江东去此两人,如大木飘一叶,太仓减一粟耳;而操得之,必大喜而去。”瑜又问:“果用何二人?”孔明曰:“亮居隆中时,即闻操于漳河新造一台,名曰铜雀,极其壮丽;广选天下美女以实其中。操本好色之徒,久闻江东乔公有二女,长曰大乔,次曰小乔,有沉鱼落雁之容,闭月羞花之貌。操曾发誓曰:吾一愿扫平四海,以成帝业;一愿得江东二乔,置之铜雀台,以乐晚年,虽死无恨矣。今虽引百万之众,虎视江南,其实为此二女也。将军何不去寻乔公,以千金买此二女,差人送与曹操,操得二女,称心满意,必班师矣。此范蠡献西施之计,何不速为之?”瑜曰:“操欲得二乔,有何证验?”孔明曰:“曹操幼子曹植,字子建,下笔成文。操尝命作一赋,名曰《铜雀台赋》。赋中之意,单道他家合为天子,誓取二乔。”瑜曰:“此赋公能记否?”孔明曰:“吾爱其文华美,尝窃记之。”瑜曰:“试请一诵。”孔明即时诵《铜雀台赋》云:“从明后以嬉游兮,登层台以娱情。见太府之广开兮。观圣德之所营。建高门之嵯峨兮,浮双阙乎太清。立中天之华观兮,连飞阁乎西城。临漳水之长流兮,望园果之滋荣。立双台于左右兮,有玉龙与金凤。揽二乔于东南兮,乐朝夕之与共。俯皇都之宏丽兮,瞰云霞之浮动。欣群才之来萃兮,协飞熊之吉梦。仰春风之和穆兮,听百鸟之悲鸣。天云垣其既立兮,家愿得乎双逞,扬仁化于宇宙兮,尽肃恭于上京。惟桓文之为盛兮,岂足方乎圣明?休矣!美矣!惠泽远扬。翼佐我皇家兮,宁彼四方。同天地之规量兮,齐日月之辉光。永贵尊而无极兮,等君寿于东皇。御龙旂以遨游兮,回鸾驾而周章。恩化及乎四海兮,嘉物阜而民康。愿斯台之永固兮,乐终古而未央!”

周瑜听罢,勃然大怒,离座指北而骂曰:“老贼欺吾太甚!”孔明急起止之曰:“昔单于屡侵疆界,汉天子许以公主和亲,今何惜民间二女乎?”瑜曰:“公有所不知:大乔是孙伯符将军主妇,小乔乃瑜之妻也。”孔明佯作惶恐之状,曰:“亮实不知。失口乱言,死罪!死罪!”瑜曰:“吾与老贼誓不两立!”孔明曰:“事须三思免致后悔。”瑜曰:“吾承伯符寄托,安有屈身降操之理?适来所言,故相试耳。吾自离鄱阳湖,便有北伐之心,虽刀斧加头,不易其志也!望孔明助一臂之力,同破曹贼。”孔明曰:“若蒙不弃,愿效犬马之劳,早晚拱听驱策。”瑜曰:“来日入见主公,便议起兵。”孔明与鲁肃辞出,相别而去。次日清晨,孙权升堂。左边文官张昭、顾雍等三十余人;右边武官程普、黄盖等三十余人:衣冠济济,剑佩锵锵,分班侍立。少顷,周瑜入见。礼毕,孙权问慰罢,瑜曰:“近闻曹操引兵屯汉上,驰书至此,主公尊意若何?”权即取檄文与周瑜看。瑜看毕,笑曰:“老贼以我江东无人,敢如此相侮耶!”权曰:“君之意若何?”瑜曰:“主公曾与众文武商议否?”权曰:“连日议此事:有劝我降者,有劝我战者。吾意未定,故请公瑾一决。”瑜曰:“谁劝主公降?”权曰:“张子布等皆主其意。”瑜即问张昭曰:“愿闻先生所以主降之意。”昭曰:“曹操挟天子而征四方,动以朝廷为名;近又得荆州,威势越大。吾江东可以拒操者,长江耳。今操艨艟战舰,何止千百?水陆并进,何可当之?不如且降,更图后计。”瑜曰:“此迂儒之论也!江东自开国以来,今历三世,安忍一旦废弃?”权曰:“若此,计将安出?”瑜曰:“操虽托名汉相,实为汉贼。将军以神武雄才,仗父兄余业,据有江东,兵精粮足,正当横行天下,为国家除残去暴,奈何降贼耶?且操今此来,多犯兵家之忌:北土未平,马腾、韩遂为其后患,而操久于南征,一忌也;北军不熟水战,操舍鞍马,仗舟楫,与东吴争衡,二忌也;又时值隆冬盛寒,马无藁草,三忌也;驱中国士卒,远涉江湖,不服水土,多生疾病,四忌也。操兵犯此数忌,虽多必败。将军擒操,正在今日。瑜请得精兵数万人,进屯夏口,为将军破之!”权矍然起曰:“老贼欲废汉自立久矣,所惧二袁、吕布、刘表与孤耳。今数雄已灭,惟孤尚存。孤与老贼,誓不两立!卿言当伐,甚合孤意。此天以卿授我也。”瑜曰:“臣为将军决一血战,万死不辞。只恐将军狐疑不定。”权拔佩剑砍面前奏案一角曰:“诸官将有再言降操者,与此案同!”言罢,便将此剑赐周瑜,即封瑜为大都督,程普为副都督,鲁肃为赞军校尉。如文武官将有不听号令者,即以此剑诛之。瑜受了剑,对众言曰:“吾奉主公之命,率众破曹。诸将官吏来日俱于江畔行营听令。如迟误者,依七禁令五十四斩施行。”言罢,辞了孙权,起身出府。众文武各无言而散。周瑜回到下处,便请孔明议事。孔明至。瑜曰:“今日府下公议已定,愿求破曹良策。”孔明曰:“孙将军心尚未稳,不可以决策也。”瑜曰:“何谓心不稳?”孔明曰:“心怯曹兵之多,怀寡不敌众之意。将军能以军数开解,使其了然无疑,然后大事可成。”瑜曰:“先生之论甚善。”乃复入见孙权。权曰:“公瑾夜至,必有事故。”瑜曰:“来日调拨军马,主公心有疑否?”权曰“但忧曹操兵多,寡不敌众耳。他无所疑。”瑜笑曰:“瑜特为此来开解主公。主公因见操檄文,言水陆大军百万,故怀疑惧,不复料其虚实。今以实较之:彼将中国之兵,不过十五六万,且已久疲;所得袁氏之众,亦止七八万耳,尚多怀疑未服。夫以久疲之卒,御狐疑之众,其数虽多,不足畏也。瑜得五万兵,自足破之。愿主公勿以为虑。”权抚瑜背曰:“公瑾此言,足释吾疑。子布无谋,深失孤望;独卿及子敬,与孤同心耳。卿可与子敬、程普即日选军前进。孤当续发人马,多载资粮,为卿后应。卿前军倘不如意,便还就孤。孤当亲与操贼决战,更无他疑。”周瑜谢出,暗忖曰:“孔明早已料着吴侯之心。其计画又高我一头。久必为江东之患,不如杀之。乃令人连夜请鲁肃入帐,言欲杀孔明之事。肃曰:“不可。今操贼未破,先杀贤士,是自去其助也。”瑜曰:“此人助刘备,必为江东之患。”肃曰:“诸葛瑾乃其亲兄,可令招此人同事东吴,岂不妙哉?”瑜善其言。

次日平明,瑜赴行营,升中军帐高坐。左右立刀斧手,聚集文官武将听令。原来程普年长于瑜,今瑜爵居其上,心中不乐:是日乃托病不出,令长子程咨自代。瑜令众将曰:“王法无亲,诸君各守乃职。方今曹操弄权,甚于董卓:囚天子于许昌。屯暴兵于境上。吾今奉命讨之,诸君幸皆努力向前。大军到处,不得扰民。赏劳罚罪,并不徇纵。”令毕,即差韩当、黄盖为前部先锋,领本部战船,即日起行,前至三江口下寨,别听将令;蒋钦、周泰为第二队;凌统、潘璋为第三队;太史慈、吕蒙为第四队;陆逊、董袭为第五队;吕范、朱治为四方巡警使,催督六郡官军,水陆并进,克期取齐。调拨已毕,诸将各自收拾船只军器起行。程咨回见父程普,说周瑜调兵,动止有法。普大惊曰:“吾素欺周郎懦弱,不足为将;今能如此,真将才也!我如何不服!”遂亲诣行营谢罪。瑜亦逊谢。次日,瑜请诸葛瑾,谓曰:“令弟孔明有王佐之才,如何屈身事刘备?今幸至江东,欲烦先生不惜齿牙余论,使令弟弃刘备而事东吴,则主公既得良辅,而先生兄弟又得相见,岂不美哉?先生幸即一行。”瑾曰:“瑾自至江东,愧无寸功。今都督有命,敢不效力。”即时上马,径投驿亭来见孔明。孔明接入,哭拜,各诉阔情。瑾泣曰:“弟知伯夷、叔齐乎?”孔明暗思:“此必周郎教来说我也。”遂答曰:“夷、齐古之圣贤也。”瑾曰:“夷、齐虽至饿死首阳山下,兄弟二人亦在一处。我今与你同胞共乳,乃各事其主,不能旦暮相聚。视夷、齐之为人,能无愧乎?”孔明曰:“兄所言者,情也;弟所守者,义也。弟与兄皆汉人。今刘皇叔乃汉室之胄,兄若能去东吴,而与弟同事刘皇叔,则上不愧为汉臣,而骨肉又得相聚,此情义两全之策也。不识兄意以为何如?”瑾思曰:“我来说他,反被他说了我也。”遂无言回答,起身辞去。回见周瑜,细述孔明之言。瑜曰:“公意若何?”瑾曰:“吾受孙将军厚恩,安肯相背!”瑜曰:“公既忠心事主,不必多言。吾自有伏孔明之计。”正是:智与智逢宜必合,才和才角又难容。毕竟周瑜定何计伏孔明,且看下回分解。