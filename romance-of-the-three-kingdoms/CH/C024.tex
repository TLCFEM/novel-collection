\chapter{国贼行凶杀贵妃~皇叔败走投袁绍}

却说曹操见了衣带诏,与众谋士商议,欲废却献帝,更择有德者立之。程昱谏曰:“明
公所以能威震四方,号令天下者,以奉汉家名号故也,今诸侯未平,遽行废立之事,必起兵
端矣。”操乃止。只将董承等五人,并其全家老小,押送各门处斩。死者共七百余人。城中
官民见者,无不下泪。后人有诗叹董承曰:“密诏传衣带,天言出禁门。当年曾救驾,此日
更承恩。忧国成心疾,除奸入梦魂。忠贞千古在,成败复谁论。”又有叹王子服等四人诗
曰:“书名尺素矢忠谋,慷慨思将君父酬。赤胆可怜捐百口,丹心自是足千秋。”

且说曹操既杀了董承等众人,怒气未消,遂带剑入宫,来弑董贵妃。贵妃乃董承之妹,
帝幸之,已怀孕五月。当日帝在后宫,正与伏皇后私论董承之事至今尚无音耗。忽见曹操带
剑入宫,面有怒容,帝大惊失色。操曰:“董承谋反,陛下知否?”帝曰:“董卓已诛
矣。”操大声曰:“不是董卓!是董承!”帝战栗曰:“朕实不知。”操曰:“忘了破指修
诏耶?”帝不能答。操叱武士擒董妃至。帝告曰:“董妃有五月身孕,望丞相见怜。”操
曰:“若非天败,吾已被害。岂得复留此女,为吾后患!”伏后告曰:“贬于冷宫,待分娩
了,杀之未迟。”操曰:“欲留此逆种,为母报仇乎?”董妃泣告曰:“乞全尸而死,勿令
彰露。”操令取白练至面前。帝泣谓妃曰:“卿于九泉之下,勿怨朕躬!”言讫,泪下如
雨。伏后亦大哭。操怒曰:“犹作儿女态耶!”叱武士牵出,勒死于宫门之外。后人有诗叹
董妃曰:“春殿承恩亦枉然,伤哉龙种并时捐。堂堂帝主难相救,掩面徒看泪涌泉。”操谕
监宫官曰:“今后但有外戚宗族,不奉吾旨,辄入宫门者,斩,守御不严,与同罪。”又拨
心腹人三千充御林军,令曹洪统领,以为防察。

操谓程昱曰:“今董承等虽诛,尚有马腾、刘备,亦在此数,不可不除。”昱曰:“马
腾屯军西凉,未可轻取;但当以书慰劳,勿使生疑,诱入京师,图之可也。刘备现在徐州,
分布掎角之势,亦不可轻敌。况今袁绍屯兵官渡,常有图许都之心。若我一旦东征,刘备势
必求救于绍。绍乘虚来袭,何以当之?”操曰:“非也。备乃人杰也,今若不击,待其羽翼
既成。急难图矣。袁绍虽强,事多怀疑不决,何足忧乎!”正议间,郭嘉自外而入。操问
曰:“吾欲东征刘备,奈有袁绍之忧,如何?”嘉曰:“绍性迟而多疑,其谋士各相妒忌,
不足忧也。刘备新整军兵,众心未服,丞相引兵东征,一战可定矣。”操大喜曰:“正合吾
意。”遂起二十万大军,分兵五路下徐州。细作探知,报入徐州。孙乾先往下邳报知关公,
随至小沛报知玄德,玄德与孙乾计议曰:“此必求救于袁绍,方可解危。”于是玄德修书一
封,遣孙乾至河北。乾乃先见田丰,具言其事,求其引进。丰即引孙乾入见绍,呈上书信。
只见绍形容憔悴,衣冠不整。丰曰:“今日主公何故如此?绍曰:“我将死矣!”丰曰:
“主公何出此言?”绍曰:“吾生五子,惟最幼者极快吾意;今患疥疮,命已垂绝。吾有何
心更论他事乎?”丰曰:“今曹操东征刘玄德,许昌空虚,若以义兵乘虚而入,上可以保天
子,下可以救万民。此不易得之机会也,惟明公裁之。”绍曰:“吾亦知此最好,奈我心中
恍惚,恐有不利。”丰曰:“何恍惚之有?”绍曰:“五子中惟此子生得最异,倘有疏虞,
吾命休矣。”遂决意不肯发兵,乃谓孙乾曰:“汝回见玄德,可言其故。倘有不如意,可来
相投,吾自有相助之处。”田丰以杖击地曰:“遭此难遇之时,乃以婴儿之病,失此机会!
大事去矣,可痛惜哉!”跌足长叹而出。

孙乾见绍不肯发兵,只得星夜回小沛见玄德,具说此事。玄德大惊曰:“似此如之奈
何?”张飞曰:“兄长勿忧。曹兵远来,必然困乏;乘其初至,先去劫寨,可破曹操。”玄
德曰:“素以汝为一勇夫耳。前者捉刘岱时,颇能用计;今献此策,亦中兵法。”乃从其
言,分兵劫寨。

且说曹操引军往小沛来。正行间,狂风骤至,忽听一声响亮,将一面牙旗吹折。操便令
军兵且住,聚众谋士问吉凶。荀彧曰:“风从何方来?吹折甚颜色旗?”操曰:“风自东南
方来,吹折角上牙旗,旗乃青红二色。”彧曰:“不主别事,今夜刘备必来劫寨。”操点
头。忽毛玠入见曰:“方才东南风起,吹折青红牙旗一面。主公以为主何吉凶?”操曰:
“公意若何?”毛玠曰:“愚意以为今夜必主有人来劫寨。”后人有诗叹曰:“吁嗟帝胄势
孤穷,全仗分兵劫寨功。争奈牙旗折有兆,老天何故纵奸雄?”操曰:“天报应我,当即防
之。”遂分兵九队,只留一队向前虚扎营寨,余众八面埋伏。

是夜月色微明。玄德在左,张飞在右,分兵两队进发;只留孙乾守小沛。且说张飞自以
为得计,领轻骑在前,突入操寨,但见零零落落,无多人马,四边火光大起,喊声齐举。飞
知中计,急出寨外。正东张辽、正西许褚、正南于禁、正北李典、东南徐晃、西南乐进,东
北夏侯惇、西北夏侯渊,八处军马杀来。张飞左冲右突,前遮后当;所领军兵原是曹操手下
旧军,见事势已急,尽皆投降去了。飞正杀间,逢着徐晃大杀一阵,后面乐进赶到。飞杀条
血路突围而走,只有数十骑跟定。欲还小沛,去路已断,欲投徐州、下邳,又恐曹军截住;
寻思无路,只得望芒砀山而去。

却说玄德引军劫寨,将近寨门,忽然喊声大震,后面冲出一军,先截去了一半人马。夏
侯惇又到。玄德突围而走,夏侯渊又从后赶来。玄德回顾,止有三十余骑跟随;急欲奔还小
沛,早望见小沛城中火起,只得弃了小沛;欲投徐州、下邳,又见曹军漫山塞野,截住去
路。玄德自思无路可归,想:“袁绍有言,‘倘不如意,可来相投’,今不若暂往依栖,别
作良图。”遂望青州路而走,正逢李典拦住。玄德匹马落荒望北而逃,李典掳将从骑去了。

且说玄德匹马投青州,日行三百里,奔至青州城下叫门。门吏问了姓名,来报刺史。刺
史乃袁绍长子袁谭。谭素敬玄德,闻知匹马到来,即便开门相迎,接入公廨,细问其故。玄
德备言兵败相投之意。谭乃留玄德于馆驿中住下,发书报父袁绍;一面差本州人马,护送玄
德。至平原界口,袁绍亲自引众出邺郡三十里迎接玄德。玄德拜谢,绍忙答礼曰:“昨为小
儿抱病,有失救援,于心怏怏不安。今幸得相见,大慰平生渴想之思。”玄德曰:“孤穷刘
备,久欲投于门下,奈机缘未遇。今为曹操所攻,妻子俱陷,想将军容纳四方之士,故不避
羞惭,径来相投。望乞收录。誓当图报。”绍大喜,相待甚厚,同居冀州。且说曹操当夜取
了小沛,随即进兵攻徐州。糜竺、简雍守把不住,只得弃城而走。陈登献了徐州。曹操大军
入城,安民已毕,随唤众谋士议取下邳。荀彧曰:“云长保护玄德妻小,死守此城。若不速
取。恐为袁绍所窃。”操曰:“吾素爱云长武艺人材,欲得之以为己用,不若令人说之使
降。”郭嘉曰:“云长义气深重,必不肯降。若使人说之,恐被其害。”帐下一人出曰:
“某与关公有一面之交,愿往说之。”众视之,乃张辽也。程昱曰:“文远虽与云长有旧,
吾观此人,非可以言词说也。某有一计,使此人进退无路,然后用文远说之,彼必归丞相
矣。”正是:整备窝弓射猛虎,安排香饵钓鳌鱼。未知其计若何,且听下文分解。