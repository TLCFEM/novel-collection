\chapter{司马懿占北原渭桥~诸葛亮造木牛流马}

却说谯周官居太史,颇明天文;见孔明又欲出师,乃奏后主曰:“臣今职掌司天台,但有祸福,不可不奏:近有群鸟数万,自南飞来,投于汉水而死,此不祥之兆;臣又观天象,见奎星躔于太白之分,盛气在北,不利伐魏;又成都人民,皆闻柏树夜哭:有此数般灾异,丞相只宜谨守,不可妄动。”孔明曰:“吾受先帝托孤之重,当竭力讨贼,岂可以虚妄之灾氛,而废国家大事耶!”遂命有司设太牢祭于昭烈之庙,涕泣拜告曰:“臣亮五出祁山,未得寸土,负罪非轻!今臣复统全师,再出祁山,誓竭力尽心,剿灭汉贼,恢复中原,鞠躬尽瘁,死而后已!”祭毕,拜辞后主,星夜至汉中,聚集诸将,商议出师。忽报关兴病亡。孔明放声大哭,昏倒于地,半晌方苏。众将再三劝解,孔明叹曰:“可怜忠义之人,天不与以寿”我今番出师,又少一员大将也!”后人有诗叹曰:“生死人常理,蜉蝣一样空。但存忠孝节,何必寿乔松。”

孔明引蜀兵三十四万,分五路而进,令姜维、魏延为先锋,皆出祁山取齐;令李恢先运粮草于斜谷道口伺候。

却说魏国因旧岁有青龙自摩坡井内而出,改为青龙元年;此时乃青龙二年春二月也。近臣奏曰:“边官飞报蜀兵三十余万,分五路复出祁山。魏主曹睿大惊,急召司马懿至,谓曰:“蜀人三年不曾入寇;今诸葛亮又出祁山,如之奈何?”懿奏曰:“臣夜观天象,见中原旺气正盛,奎星犯太白,不利于西川。今孔明自负才智,逆天而行,乃自取败亡也。臣托陛下洪福,当往破之。但愿保四人同去。”睿曰:“卿保何人?”懿曰:“夏侯渊有四子:长名霸,字仲权;次名威,字季权;三名惠,字稚权;四名和,字义权。霸、威二人,弓马熟娴;惠、和二人,谙知韬略:此四人常欲为父报仇。臣今保夏侯霸、夏侯威为左右先锋,夏侯惠;夏侯和为行军司马,共赞军机,以退蜀兵。”睿曰:“向者夏侯楙驸马违误军机,失陷了许多人马,至今羞惭不回。今此四人,亦与楙同否?”懿曰:“此四人非夏侯楙所可比也。”睿乃从其请,即命司马懿为大都督,凡将士悉听量才委用,各处兵马皆听调遣。

懿受命,辞朝出城。睿又以手诏赐懿曰:“卿到渭滨,宜坚壁固守,勿与交锋。蜀兵不得志,必诈退诱敌,卿慎勿追。待彼粮尽,必将自走,然后乘虚攻之,则取胜不难,亦免军马疲劳之苦:计莫善于此也。”司马懿顿首受诏,即日到长安,聚集各处军马共四十万,皆来渭滨下寨;又拨五万军,于渭水上搭起九座浮桥,令先锋夏侯霸、夏侯威过渭水安营;又于大营之后东原,筑起一城,以防不虞。

懿正与众将商议间,忽报郭淮、孙礼来见。懿迎入,礼毕,淮曰:“今蜀兵现在祁山,倘跨渭登原,接连北山,阻绝陇道,大可虞也。”懿曰:“所言甚善。公可就总督陇西军马,据北原下寨,深沟高垒,按兵休动;只待彼兵粮尽,方可攻之。”郭淮、孙礼领命,引兵下寨去了。

却说孔明复出祁山,下五个大寨,按左、右、中、前、后;自斜谷直至剑阁,一连又下十四个大寨,分屯军马,以为久计。每日令人巡哨。忽报郭淮、孙礼领陇西之兵,于北原下寨。孔明谓诸将曰:“魏兵于北原安营者,惧吾取此路,阻绝陇道也。吾今虚攻北原,却暗取渭滨。令人扎木筏百余只,上载草把,选惯熟水手五千人驾之。我夤夜只攻北原,司马懿必引兵来救。彼若少败,我把后军先渡过岸去,然后把前军下于筏中。休要上岸,顺水取浮桥放火烧断,以攻其后。吾自引一军去取前营之门。若得渭水之南,则进兵不难矣。”诸将遵令而行。早有巡哨军飞报司马懿。懿唤诸将议曰:“孔明如此设施,其中有计:彼以取北原为名,顺水来烧浮桥,乱吾后,却攻吾前也。”即传令与夏侯霸、夏侯威曰:“若听得北原发喊,便提兵于渭水南山之中,待蜀兵至击之。”又令张虎、乐綝,引二千弓弩手伏于渭水浮桥北岸:“若蜀兵乘木筏顺水而来,可一齐射之,休令近桥。”又传令郭淮、孙礼曰:“孔明来北原暗渡渭水,汝新立之营,人马不多,可尽伏于半路。若蜀兵于午后渡水,黄昏时分,必来攻汝。汝诈败而走,蜀兵必追。汝等皆以弓弩射之。吾水陆并进。若蜀兵大至,只看吾指挥而击之。”各处下令已毕,又令二子司马师、司马昭,引兵救应前营。懿自引一军救北原。

却说孔明令魏延、马岱引兵渡渭水攻北原;令吴班、吴懿引木筏兵去烧浮桥;令王平、张嶷为前队,姜维、马忠为中队,廖化、张翼为后队:兵分三路,去攻渭水旱营。是日午时,人马离大寨,尽渡渭水,列成阵势,缓缓而行。却说魏延、马岱将近北原,天色已昏。孙礼哨见,便弃营而走。魏延知有准备,急退军时,四下喊声大震:左有司马懿,右有郭淮,两路兵杀来。魏延、马岱奋力杀出,蜀兵多半落于水中,余众奔逃无路。幸得吴懿兵杀来,救了败兵过岸拒住。吴班分一半兵撑筏顺水来烧浮桥,却被张虎、乐綝在岸上乱箭射住。吴班中箭,落水而死。余军跳水逃命,木筏尽被魏兵夺去。此时王平、张嶷,不知北原兵败,直奔到魏营,已有二更天气,只听得喊声四起。王平谓张嶷曰:“军马攻打北原,未知胜负。渭南之寨,现在面前,如何不见一个魏兵?莫非司马懿知道了,先作准备也?我等且看浮桥火起,方可进兵。”二人勒住军马,忽背后一骑马来报,说:“丞相教军马急回。北原兵、浮桥兵,俱失了。”王平、张嶷大惊,急退军时,却被魏兵抄在背后,一声炮响,一齐杀来,火光冲天。王平、张嶷引兵相迎,两军混战一场。平、嶷二人奋力杀出,蜀兵折伤大半。孔明回到祁山大寨,收聚败兵,约折了万余人,心中忧闷。忽报费祎自成都来见丞相。孔明请入。费祎礼毕,孔明曰:“吾有一书,正欲烦公去东吴投递,不知肯去否?”祎曰:“丞相之命,岂敢推辞?”孔明即修书付费祎去了。祎持书径到建业,入见吴主孙权,呈上孔明之书。权拆视之,书略曰:“汉室不幸,王纲失纪,曹贼篡逆,蔓延及今。亮受昭烈皇帝寄托之重,敢不竭力尽忠:今大兵已会于祁山,狂寇将亡于渭水。伏望陛下念同盟之义,命将北征,共取中原,同分天下。书不尽言,万希圣听!”权览毕,大喜,乃谓费祎曰:“朕久欲兴兵,未得会合孔明。今既有书到,即日朕自亲征,入居巢门,取魏新城;再令陆逊、诸葛瑾等屯兵于江夏、沔口取襄阳;孙韶、张承等出兵广陵取淮阳等处:三处一齐进军,共三十万,克日兴师。”费祎拜谢曰:“诚如此,则中原不日自破矣!”权设宴款待费祎。饮宴间,权问曰:“丞相军前,用谁当先破敌?”祎曰:“魏延为首。”权笑曰:“此人勇有余。而心不正。若一朝无孔明,彼必为祸。孔明岂未知耶?”祎曰:“陛下之言极当!臣今归去,即当以此言告孔明。”遂拜辞孙权,回到祁山,见了孔明,具言吴主起大兵三十万,御驾亲征,兵分三路而进。孔明又问曰:“吴主别有所言否?”费祎将论魏延之语告之。孔明叹曰:“真聪明之主也!吾非不知此人。为惜其勇,故用之耳。”祎曰:“丞相早宜区处。”孔明曰:“吾自有法。”祎辞别孔明,自回成都。

孔明正与诸将商议征进,忽报有魏将来投降。孔明唤入问之,答曰:“某乃魏国偏将军郑文也。近与秦朗同领人马,听司马懿调用,不料懿徇私偏向,加秦朗为前将军,而视文如草芥,因此不平,特来投降丞相。愿赐收录。”言未已,人报秦朗引兵在寨外,单搦郑文交战。孔明曰:“此人武艺比汝若何?”郑文曰:“某当立斩之。”孔明曰:“汝若先杀秦朗,吾方不疑。”郑文欣然上马出营,与秦朗交锋。孔明亲自出营视之。只见秦朗挺枪大骂曰:“反贼盗我战马来此,可早早还我!”言讫,直取郑文。文拍马舞刀相迎,只一合,斩秦朗于马下。魏军各自逃走。郑文提首级入营。孔明回到帐中坐定,唤郑文至,勃然大怒,叱左右:“推出斩之!”郑文曰:“小将无罪!”孔明曰:“吾向识秦朗;汝今斩者,并非秦朗。安敢欺我!”文拜告曰:“此实秦朗之弟秦明也。”孔明笑曰:“司马懿令汝来诈降,于中取事,却如何瞒得我过!若不实说,必然斩汝!”郑文只得诉告其实是诈降,泣求免死。孔明曰:“汝既求生,可修书一封,教司马懿自来劫营,吾便饶汝性命。若捉住司马懿,便是汝之功,还当重用。”郑文只得写了一书,呈与孔明。孔明令将郑文监下。樊建问曰:“丞相何以知此人诈降?”孔明曰:“司马懿不轻用人。若加秦朗为前将军,必武艺高强;今与郑文交马只一合,便为文所杀,必不是秦朗也。以故知其诈。”众皆拜服。孔明选一舌辩军士,附耳分付如此如此。军士领命,持书径来魏寨,求见司马懿。懿唤入,拆书看毕,问曰:“汝何人也?”答曰:“某乃中原人,流落蜀中:郑文与某同乡。今孔明因郑文有功,用为先锋。郑文特托某来献书,约于明日晚间,举火为号,望乞都督尽提大军前来劫寨,郑文在内为应。”司马懿反覆诘问,又将来书仔细检看,果然是实;即赐军士酒食,分付曰:“本日二更为期,我自来劫寨。大事若成,必重用汝。”军士拜别,回到本寨告知孔明。孔明仗剑步罡,祷祝已毕,唤王平、张嶷公付如此如此;又唤马忠、马岱分付如此如此;又唤魏延分付如此如此。孔明自引数十人,坐于高山之上,指挥众军。却说司马懿见了郑文之书,便欲引二子提大兵来劫蜀寨。长子司马师谏曰:“父亲何故据片纸而亲入重地?倘有疏虞,如之奈何?不如令别将先去,父亲为后应可也。”懿从之,遂令秦朗引一万兵,去劫蜀寨,懿自引兵接应。是夜初更,风清月朗;将及二更时分,忽然阴云四合,黑气漫空,对面不见。懿大喜曰:“天使我成功也!”于是人尽衔枚,马皆勒口,长驱大进。秦朗当先,引万兵直杀入蜀寨中,并不见一人。朗知中计,忙叫退兵。四下火把齐明,喊声震地:左有王平、张嶷,右有马岱、马忠,两路兵杀来。秦朗死战,不能得出。背后司马懿见蜀寨火光冲天,喊声不绝,又不知魏兵胜负,只顾催兵接应,望火光中杀来。忽然一声喊起,鼓角喧天,火炮震地:左有魏延,右有姜维,两路杀出。魏兵大败,十伤八九,四散逃奔。此时秦朗所引一万兵,都被蜀兵围住,箭如飞蝗。秦朗死于乱军之中。司马懿引败兵奔入本寨。

三更以后,天复清朗。孔明在山头上鸣金收军。原来二更时阴云暗黑,乃孔明用遁甲之法;后收兵已了,天复清朗,乃孔明驱六丁六甲扫荡浮云也。

当下孔明得胜回寨,命将郑文斩了,再议取渭南之策。每日令兵搦战,魏军只不出迎。孔明自乘小车,来祁山前、渭水东西,踏看地理。忽到一谷口,见其形如葫芦之状,内中可容千余人;两山又合一谷,可容四五百人;背后两山环抱,只可通一人一骑。孔明看了,心中大喜,问向导官曰:“此处是何地名?”答曰:“此名上方谷,又号葫芦谷。”孔明回到帐中,唤裨将杜睿、胡忠二人,附耳授以密计。令唤集随军匠作一千余人,入葫芦谷中,制造木牛流马应用;又令马岱领五百兵守住谷口。孔明嘱马岱曰:“匠作人等,不许放出;外人不许放入。吾还不时自来点视。捉司马懿之计,只在此举。切不可走漏消息。”马岱受命而去。杜睿等二人在谷中监督匠作,依法制造。孔明每日往来指示。

忽一日,长史杨仪入告曰:“即今粮米皆在剑阁,人夫牛马,搬运不便,如之奈何?”孔明笑曰:“吾已运谋多时也。前者所积木料,并西川收买下的大木,教人制造木牛流马,搬运粮米,甚是便利。牛马皆不水食,可以昼夜转运不绝也。”众皆惊曰:“自古及今,未闻有木牛流马之事。不知丞相有何妙法,造此奇物?”孔明曰:“吾已令人依法制造,尚未完备。吾今先将造木牛流马之法,尺寸方圆,长短阔狭,开写明白,汝等视之。”众大喜。孔明即手书一纸,付众观看。众将环绕而视。造木牛之法云:“方腹曲头,一脚四足;头入领中,舌着于腹。载多而行少:独行者数十里,群行者二十里。曲者为牛头,双者为牛脚,横者为牛领,转者为牛足,覆者为牛背,方者为牛腹,垂者为牛舌,曲者为牛肋,刻者为牛齿,立者为牛角,细者为牛鞅,摄者为牛鞦轴。牛仰双辕,人行六尺,牛行四步。每牛载十人所食一月之粮,人不大劳,牛不饮食。”造流马之法云:“肋长三尺五寸,广三寸,厚二寸二分:左右同。前轴孔分墨去头四寸,径中二寸。前脚孔分墨二寸,去前轴孔四寸五分,广一寸。前杠孔去前脚孔分墨二寸七分,孔长二寸,广一寸。后轴孔去前杠分墨一尺五分,大小与前同。后脚孔分墨去后轴孔三寸五分,大小与前同。后杠孔去后脚孔分墨二寸七分,后载克去后杠孔分墨四寸五分。前杠长一尺八寸,广二寸,厚一寸五分。后杠与等。板方囊二枚,厚八分,长二尺七寸,高一尺六寸五分,广一尺六寸:每枚受米二斛三斗。从上杠孔去肋下七寸:前后同。上杠孔去下杠孔分墨一尺三寸,孔长一寸五分,广七分:八孔同。前后四脚广二寸,厚一寸五分。形制如象,靬长四寸,径面四寸三分。孔径中三脚杠,长二尺一寸,广一寸五分,厚一寸四分,同杠耳。”众将看了一遍,皆拜伏曰:“丞相真神人也!”

过了数日,木牛流马皆造完备,宛然如活者一般;上山下岭,各尽其便。众军见之,无不欣喜。孔明令右将军高翔,引一千兵驾着木牛流马,自剑阁直抵祁山大寨,往来搬运粮草,供给蜀兵之用。后人有诗赞曰:“剑关险峻驱流马,斜谷崎岖驾木牛。后世若能行此法,输将安得使人愁?”

却说司马懿正忧闷间,忽哨马报说:“蜀兵用木牛流马转运粮草。人不大劳,牛马不食。”懿大惊曰:“吾所以坚守不出者,为彼粮草不能接济,欲待其自毙耳。今用此法,必为久远之计,不思退矣。如之奈何?”急唤张虎、乐綝二人分付曰:“汝二人各引五百军,从斜谷小路抄出;待蜀兵驱过木牛流马,任他过尽,一齐杀出;不可多抢,只抢三五匹便回。”

二人依令,各引五百军,扮作蜀兵,夜间偷过小路,伏在谷中,果见高翔引兵驱木牛流马而来。将次过尽,两边一齐鼓噪杀出。蜀兵措手不及,弃下数匹,张虎、乐綝欢喜,驱回本寨。司马懿看了,果然进退如活的一般,乃大喜曰:“汝会用此法,难道我不会用!”便令巧匠百余人,当面拆开,分付依其尺寸长短厚薄之法,一样制造木牛流马。不消半月,造成二千余只,与孔明所造者一般法则,亦能奔走。遂令镇远将军岑威,引一千军驱驾木牛流马,去陇西搬运粮草,往来不绝。魏营军将,无不欢喜。

却说高翔回见孔明,说魏兵抢夺木牛流马各五六匹去了。孔明笑曰:“吾正要他抢去。我只费了几匹木牛流马,却不久便得军中许多资助也。”诸将问曰:“丞相何以知之?”孔明曰:“司马懿见了木牛流马,必然仿我法度,一样制造。那时我又有计策。”数日后,人报魏兵也会造木牛流马,往陇西搬运粮草。孔明大喜曰:“不出吾之算也。”便唤王平分付曰:“汝引一千兵,扮作魏人,星夜偷过北原,只说是巡粮军,径到运粮之所,将护粮之人尽皆杀散;却驱木牛流马而回,径奔过北原来:此处必有魏兵追赶,汝便将木牛流马口内舌头扭转,牛马就不能行动,汝等竟弃之而走,背后魏兵赶到,牵拽不动,打抬不去。吾再有兵到,汝却回身再将牛马舌扭过来,长驱大行。魏兵必疑为怪也!”王平受计引兵而去。

孔明又唤张嶷分付曰:“汝引五百军,都扮作六丁六甲神兵,鬼头兽身,用五彩涂面,妆作种种怪异之状;一手执绣旗,一手仗宝剑;身挂葫芦,内藏烟火之物,伏于山傍。待木牛流马到时,放起烟火,一齐拥出,驱牛马而行。魏人见之,必疑是神鬼,不敢来追赶。”张嶷受计引兵而去。孔明又唤魏延、姜维分付曰:“汝二人同引一万兵,去北原寨口接应木牛流马,以防交战。”又唤廖化、张翼分付曰:“汝二人引五千兵,去断司马懿来路。”又唤马忠、马岱分付曰:“汝二人引二千兵去渭南搦战。”六人各各遵令而去。

且说魏将岑威引军驱木牛流马,装载粮米,正行之间,忽报前面有兵巡粮。岑威令人哨探,果是魏兵,遂放心前进。两军合在一处。忽然喊声大震,蜀兵就本队里杀起,大呼:“蜀中大将王平在此!”魏兵措手不及,被蜀兵杀死大半。岑威引败兵抵敌,被王平一刀斩了,余皆溃散。王平引兵尽驱木牛流马而回。败兵飞奔报入北原寨内。郭淮闻军粮被劫,疾忙引军来救。王平令兵扭转木牛流马舌头,皆弃于道上,且战且走。郭淮教且莫追,只驱回木牛流马。众军一齐驱赶,却那里驱得动?郭淮心中疑惑,正无奈何,忽鼓角喧天,喊声四起,两路兵杀来,乃魏延、姜维也。王平复引兵杀回。三路夹攻,郭淮大败而走。王平令军士将牛马舌头,重复扭转,驱赶而行。郭淮望见,方欲回兵再追,只见山后烟云突起,一队神兵拥出,一个个手执旗剑,怪异之状,驱驾木牛流马如风拥而去。郭淮大惊曰:“此必神助也!”众军见了,无不惊畏,不敢追赶。却说司马懿闻北原兵败,急自引军来救。方到半路,忽一声炮响,两路兵自险峻处杀出,喊声震地。旗上大书汉将张翼、廖化。司马懿见了大惊。魏军着慌,各自逃窜。正是:路逢神将粮遭劫,身遇奇兵命又危。未知司马懿怎地抵敌,且看下文分解。