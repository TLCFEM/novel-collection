\chapter{刘玄德携民渡江~赵子龙单骑救主}

却说张飞因关公放了上流水,遂引军从下流杀将来,截住曹仁混杀。忽遇许褚,便与交
锋;许褚不敢恋战,夺路走脱。张飞赶来,接着玄德、孔明,一同沿河到上流。刘封、糜芳
已安排船只等候,遂一齐渡河,尽望樊城而去,孔明教将船筏放火烧毁。却说曹仁收拾残
军,就新野屯住,使曹洪去见曹操,具言失利之事。操大怒曰:“诸葛村夫,安敢如此;”
催动三军,漫山塞野,尽至新野下寨。传令军士一面搜山,一面填塞白河。令大军分作八
路,一齐去取樊城。刘晔曰:“丞相初至襄阳,必须先买民心,今刘备尽迁新野百姓入樊
城,若我兵径进,二县为齑粉矣;不如先使人招降刘备。备即不降,亦可见我爱民之心;若
其来降,则荆州之地,可不战而定也。”操从其言,便问:“谁可为使?”刘晔曰:“徐庶
与刘备至厚,今现在军中,何不命他一往?”操曰:“他去恐不复来。”晔曰:“他若不
来,贻笑于人矣。丞相勿疑。”操乃召徐庶至,谓曰:“我本欲踏平樊城,奈怜众百姓之
命。公可往说刘备:如肯来降,免罪赐爵;若更执迷,军民共戮,玉石俱焚。吾知公忠义,
故特使公往。愿勿相负。”

徐庶受命而行。至樊城,玄德、孔明接见,共诉旧日之情。庶曰:“曹操使庶来招降使
君,乃假买民心也,今彼分兵八路,填白河而进。樊城恐不可守,宜速作行计。”玄德欲留
徐庶。庶谢曰:“某若不还,恐惹人笑。今老母已丧,抱恨终天。身虽在彼,誓不为设一
谋,公有卧龙辅佐,何愁大业不成。庶请辞。”玄德不敢强留。

徐庶辞回,见了曹操,言玄德并无降意。操大怒,即日进兵。玄德问计于孔明。孔明
曰:“可速弃樊城,取襄阳暂歇。”玄德曰:“奈百姓相随许久,安忍弃之?”孔明曰:
“可令人遍告百姓:有愿随者同去,不愿者留下。”先使云长往江岸整顿船只,令孙乾、简
雍在城中声扬曰:“今曹兵将至,孤城不可久守,百姓愿随者,便同过江。”两县之民,齐
声大呼曰:“我等虽死,亦愿随使君!”即日号泣而行。扶老携幼,将男带女,滚滚渡河,
两岸哭声不绝。玄德于船上望见,大恸曰:“为吾一人而使百姓遭此大难,吾何生哉!”欲
投江而死,左右急救止。闻者莫不痛哭。船到南岸,回顾百姓,有未渡者,望南而哭。玄德
急令云长催船渡之,方才上马。

行至襄阳东门,只见城上遍插旌旗,壕边密布鹿角,玄德勒马大叫曰:“刘琮贤侄,吾
但欲救百姓,并无他念。可快开门。”刘琮闻玄德至,惧而不出。蔡瑁、张允径来敌楼上,
叱军士乱箭射下。城外百姓,皆望敌楼而哭。城中忽有一将,引数百人径上城楼,大喝:
“蔡瑁、张允卖国之贼!刘使君乃仁德之人,今为救民而来投,何得相拒!”众视其人,身
长八尺,面如重枣;乃义阳人也,姓魏,名延,字文长。当下魏延轮刀砍死守门将士,开了
城门,放下吊桥,大叫:“刘皇叔快领兵入城,共杀卖国之贼!”张飞便跃马欲入,玄德急
止之曰:“休惊百姓!”魏延只管招呼玄德军马入城。只见城内一将飞马引军而出,大喝:
“魏延无名小卒,安敢造乱!认得我大将文聘么!”魏延大怒,挺枪跃马,便来交战。两下
军兵在城边混杀,喊声大震。玄德曰:“本欲保民,反害民也!吾不愿入襄阳!”孔明曰:
“江陵乃荆州要地,不如先取江陵为家。”玄德曰:“正合吾心。”于是引着百姓,尽离襄
阳大路,望江陵而走。襄阳城中百姓,多有乘乱逃出城来,跟玄德而去。魏延与文聘交战,
从已至未,手下兵卒皆已折尽。延乃拨马而逃,却寻不见玄德,自投长沙太守韩玄去了。

却说玄德同行军民十余万,大小车数千辆,挑担背包者不计其数,路过刘表之墓,玄德
率众将拜于墓前,哭告曰:“辱弟备无德无才,负兄寄托之重,罪在备一身,与百姓无干。
望兄英灵,垂救荆襄之民!”言甚悲切,军民无不下泪。忽哨马报曰:“曹操大军已屯樊
城,使人收拾船筏,即日渡江赶来也。”众将皆曰:“江陵要地,足可拒守。今拥民众数
万,日行十余里,似此几时得至江陵?倘曹兵到,如何迎敌?不如暂弃百姓,先行为上。”
玄德泣曰:“举大事者必以人为本。今人归我,奈何弃之?”百姓闻玄德此言,莫不伤感。
后人有诗赞之曰:“临难仁心存百姓,登舟挥泪动三军。至今凭吊襄江口,父老犹然忆使
君。”却说玄德拥着百姓,缓缓而行。孔明曰:“追兵不久即至。可遣云长往江夏求救于公
子刘琦。教他速起兵乘船会于江陵。”玄德从之,即修书令云长同孙乾领五百军往江夏求
救;令张飞断后;赵云保护老小;其余俱管顾百姓而行。每日只走十余里便歇。却说曹操在
樊城,使人渡江至襄阳,召刘琮相见。琮惧怕不敢往见。蔡瑁、张允请行。王威密告琮曰:
“将军既降,玄德又走,曹操必懈弛无备。愿将军奋整奇兵,设于险处击之,操可获矣。获
操则威震天下,中原虽广,可传檄而定。此难遇之机,不可失也。”琮以其言告蔡瑁。瑁叱
王威曰:“汝不知天命,安敢妄言!”威怒骂曰:“卖国之徒,吾恨不生啖汝肉!”瑁欲杀
之,蒯越劝止。

瑁遂与张允同至樊城,拜见曹操。瑁等辞色甚是谄佞。操问:“荆州军马钱粮,今有多
少?”瑁曰:“马军五万,步军十五万,水军八万:共二十八万。钱粮大半在江陵;其余各
处,亦足供给一载。”操曰:“战船多少?原是何人管领?”瑁曰:“大小战船,共七千余
只,原是瑁等二人掌管。”操遂加瑁为镇南侯、水军大都督,张允为助顺侯、水军副都督。
二人大喜拜谢。操又曰:“刘景升既死,其子降顺,吾当表奏天子,使永为荆州之主。”二
人大喜而退。荀攸曰:“蔡瑁,张允乃谄佞之徒,主公何遂加以如此显爵,更教都督水军
乎?”操笑曰:“吾岂不识人!止因吾所领北地之众,不习水战,故且权用此二人;待成事
之后,别有理会。”

却说蔡瑁、张允归见刘琮,具言:“曹操许保奏将军永镇荆襄。”琮大喜!次日,与母
蔡夫人赍捧印缓兵符,亲自渡江拜迎曹操。操抚慰毕,即引随征军将,进屯襄阳城外。蔡
瑁、张允令襄阳百姓焚香拜接。曹操俱用好言抚谕。入城至府中坐定,即召蒯越近前,抚慰
曰:“吾不喜得荆州,喜得异度也。”遂封蒯越为江陵太守樊城侯;傅巽、王粲等皆为关内
侯;而以刘琮为青州刺史,便教起程。琮闻命大惊,辞曰:“琮不愿为官,愿守父母乡
土。”操曰:“青州近帝都,教你随朝为官,免在荆襄被人图害。”琮再三推辞,曹操不
准。琮只得与母蔡夫人同赴青州。只有故将王威相随,其余官员俱送至江口而回。操唤于禁
嘱咐曰:“你可引轻骑追刘琮母子杀子,以绝后患。”于禁得令,领众赶上,大喝曰:“我
奉丞相令,教来杀汝母子!可早纳下首级!”蔡夫人抱刘琮而大哭。于禁喝令军士下手。王
威忿怒,奋力相斗,竟被众军所杀。军士杀死刘琮及蔡夫人,于禁回报曹操,操重赏于禁。
便使人往隆中搜寻孔明妻小,却不知去向。原来孔明先已令人搬送至三江内隐避矣。操深恨
之。襄阳既定,荀攸进言曰:“江陵乃荆襄重地,钱粮极广。刘备若据此地,急难动摇。”
操曰:“孤岂忘之!”随命于襄阳诸将中,选一员引军开道。诸将中却独不见文聘。操使人
寻问,方才来见。操曰:“汝来何迟?”对曰:“为人臣而不能使其主保全境土,心实悲
惭,无颜早见耳。”言讫,欷歔流涕。操曰:“真忠臣也!”除江夏太守,赐爵关内侯,便
教引军开道。探马报说:“刘备带领百姓,日行止十数里,计程只有三百余里。”操教各部
下精选五千铁骑,星夜前进,限一日一夜,赶上刘备。大军陆续随后而进。

却说玄德引十数万百姓、三千余军马,一程程挨着往江陵进发。赵云保护老小,张飞断
后。孔明曰:“云长往江夏去了,绝无回音,不知若何。”玄德曰:“敢烦军师亲自走一
遭。刘琦感公昔日之教,今若见公亲至,事必谐矣。”孔明允诺,便同刘封引五百军先往江
夏求救去了。

当日玄德自与简雍、糜竺、糜芳同行。正行间,忽然一阵狂风就马前刮起,尘土冲天,
平遮红日。玄德惊曰:“此何兆也?”简雍颇明阴阳,袖占一课,失惊曰:“此大凶之兆
也。应在今夜。主公可速弃百姓而走。”玄德曰:“百姓从新野相随至此,吾安忍弃之?”
雍曰:“主公若恋而不弃,祸不远矣。”玄德问:“前面是何处?”左右答曰:“前面是当
阳县。有座山名为景山。”玄德便教就此山扎住。

时秋末冬初,凉风透骨;黄昏将近,哭声遍野。至四更时分,只听得西北喊声震地而
来。玄德大惊,急上马引本部精兵二千余人迎敌。曹兵掩至,势不可当。玄德死战。正在危
迫之际,幸得张飞引军至,杀开一条血路,救玄德望东而走。文聘当先拦住,玄德骂曰:
“背主之贼,尚有何面目见人!”文聘羞惭满面,引兵自投东北去了。张飞保着玄德,且战
且走。奔至天明,闻喊声渐渐远去,玄德方才歇马。看手下随行人,止有百余骑;百姓、老
小并糜竺、糜芳、简雍、赵云等一干人,皆不知下落。玄德大哭曰:“十数万生灵,皆因恋
我,遭此大难;诸将及老小,皆不知存亡:虽土木之人,宁不悲乎!”正凄惶时,忽见糜芳
面带数箭,踉跄而来,口言:“赵子龙反投曹操去了也!”玄德叱曰:“子龙是我故交,安
肯反乎?”张飞曰:“他今见我等势穷力尽,或者反投曹操,以图富贵耳!”玄德曰:“子
龙从我于患难,心如铁石,非富贵所能动摇也。”糜芳曰:“我亲见他投西北去了。”张飞
曰:“待我亲自寻他去。若撞见时,一枪刺死!”玄德曰:“休错疑了。岂不见你二兄诛颜
良、文丑之事乎?子龙此去,必有事故。吾料子龙必不弃我也。”张飞那里肯听,引二十余
骑,至长坂桥。见桥东有一带树木,飞生一计:教所从二十余骑,都砍下树枝,拴在马尾
上,在树林内往来驰骋,冲起尘土,以为疑兵。飞却亲自横矛立马于桥上,向西而望。

却说赵云自四更时分,与曹军厮杀,往来冲突,杀至天明,寻不见玄德,又失了玄德老
小,云自思曰:“主公将甘、糜二夫人与小主人阿斗,托付在我身上;今日军中失散,有何
面目去见主人?不如去决一死战,好歹要寻主母与小主人下落!”回顾左右,只有三四十骑
相随。云拍马在乱军中寻觅,二县百姓号哭之声震天动地;中箭着枪抛男弃女而走者不计其
数。赵云正走之间,见一人卧在草中,视之,乃简雍也。云急问曰:“曾见两位主母否?”
雍曰:“二主母弃了车仗,抱阿斗而走。我飞马赶去,转过山坡,被一将刺了一枪,跌下马
来,马被夺了去。我争斗不得,故卧在此。”云乃将从骑所骑之马,借一匹与简雍骑坐;又
着二卒扶护简雍先去报与主人:“我上天入地,好歹寻主母与小主人来。如寻不见,死在沙
场上也!”

说罢,拍马望长坂坡而去。忽一人大叫:“赵将军那里去?”云勒马问曰:“你是何
人?”答曰:“我乃刘使君帐下护送车仗的军士,被箭射倒在此。”赵云便问二夫人消息。
军士曰:“恰才见甘夫人披头跣足,相随一伙百姓妇女,投南而走。”云见说,也不顾军
士,急纵马望南赶去。只见一伙百姓,男女数百人,相携而走。”云大叫曰:“内中有甘夫
人否?”夫人在后面望见赵云,放声大哭。云下马插枪而泣曰:“使主母失散,云之罪也!
糜夫人与小主人安在?”甘夫人曰:“我与糜夫人被逐,弃了车仗,杂于百姓内步行,又撞
见一枝军马冲散。糜夫人与阿斗不知何往。我独自逃生至此。”

正言间,百姓发喊,又撞出一枝军来。赵云拔枪上马看时,面前马上绑着一人,乃糜竺
也。背后一将,手提大刀,引着千余军。乃曹仁部将淳于导,拿住糜竺,正要解去献功。赵
云大喝一声,挺枪纵马,直取淳于导。导抵敌不住,被云一枪刺落马下,向前救了糜竺,夺
得马二匹。云请甘夫人上马,杀开条大路,直送至长坂城。只见张飞横矛立马于桥上,大
叫:“子龙!你如何反我哥哥?”云曰:“我寻不见主母与小主人,因此落后,何言反
耶?”飞曰:“若非简雍先来报信,我今见你,怎肯干休也!”云曰:“主公在何处?”飞
曰:“只在前面不远。”云谓糜竺曰:“糜子仲保甘夫人先行,待我仍往寻糜夫人与小主人
去。”言罢,引数骑再回旧路。

正走之间,见一将手提铁枪,背着一口剑,引十数骑跃马而来。赵云更不打话,直取那
将。交马只一合,把那将一枪刺倒,从骑皆走。原来那将乃曹操随身背剑之将夏侯恩也。曹
操有宝剑二口:一名“倚天”,一名“青釭”;倚天剑自佩之,青釭剑令夏侯恩佩之。那青釭
剑砍铁如泥,锋利无比。当时夏侯恩自恃勇力,背着曹操,只顾引人抢夺掳掠。不想撞着赵
云,被他一枪刺死,夺了那口剑,看靶上有金嵌“青釭”二字,方知是宝剑也。云插剑提
枪,复杀入重围,回顾手下从骑,已没一人,只剩得孤身。云并无半点退心,只顾往来寻
觅;但逢百姓,便问糜夫人消息。忽一人指曰:“夫人抱着孩儿,左腿上着了枪,行走不
得,只在前面墙缺内坐地。”

赵云听了,连忙追寻。只见一个人家,被火烧坏土墙,糜夫人抱着阿斗,坐于墙下枯井
之傍啼哭。云急下马伏地而拜。夫人曰:“妾得见将军,阿斗有命矣。望将军可怜他父亲飘
荡半世,只有这点骨血。将军可护持此子,教他得见父面,妾死无恨!”云曰:“夫人受
难,云之罪也。不必多言,请夫人上马。云自步行死战,保夫人透出重围。”糜夫人曰:
“不可!将军岂可无马!此子全赖将军保护。妾已重伤,死何足惜!望将军速抱此子前去,
勿以妾为累也。”云曰:“喊声将近,追兵已至,请夫人速速上马。”糜夫人曰:“妾身委
实难去。休得两误。”乃将阿斗递与赵云曰:“此子性命全在将军身上!”赵云三回五次请
夫人上马,夫人只不肯上马。四边喊声又起。云厉声曰:“夫人不听吾言,追军若至,为之
奈何?”糜夫人乃弃阿斗于地,翻身投入枯井中而死。后人有诗赞之曰:“战将全凭马力
多,步行怎把幼君扶?拚将一死存刘嗣,勇决还亏女丈夫。”赵云见夫人已死,恐曹军盗
尸,便将土墙推倒,掩盖枯井。掩讫,解开勒甲绦,放下掩心镜,将阿斗抱护在怀,绰枪上
马。早有一将,引一队步军至,乃曹洪部将晏明也,持三尖两刃刀来战赵云。不三合,被赵
云一枪刺倒,杀散众军,冲开一条路。正走间,前面又一枝军马拦路。当先一员大将,旗号
分明,大书河间张郃。云更不答话,挺枪便战。约十余合,云不敢恋战,夺路而走。背后张郃
赶来,云加鞭而行,不想趷跶一声,连马和人,颠入土坑之内。张郃挺枪来刺,忽然一道红
光,从土坑中滚起,那匹马平空一跃,跳出坑外。后人有诗曰:“红光罩体困龙飞,征马冲
开长坂围。四十二年真命主,将军因得显神威。”张郃见了,大惊而退。赵云纵马正走,背
后忽有二将大叫:“赵云休走!”前面又有二将,使两般军器,截住去路:后面赶的是马
延、张顗,前面阻的是焦触、张南,都是袁绍手下降将。赵云力战四将,曹军一齐拥至。云
乃拔青釭剑乱砍,手起处,衣甲平过,血如涌泉。杀退众军将,直透重围。却说曹操在景山
顶上,望见一将,所到之处,威不可当,急问左右是谁。曹洪飞马下山大叫曰:“军中战将
可留姓名!”云应声曰:“吾乃常山赵子龙也!”曹洪回报曹操。操曰:“真虎将也!吾当
生致之。”遂令飞马传报各处:“如赵云到,不许放冷箭,只要捉活的。”因此赵云得脱此
难;此亦阿斗之福所致也。这一场杀:赵云怀抱后主,直透重围,砍倒大旗两面,夺槊三
条;前后枪刺剑砍,杀死曹营名将五十余员。后人有诗曰:“血染征袍透甲红,当阳谁敢与
争锋!古来冲阵扶危主,只有常山赵子龙。”

赵云当下杀透重围,已离大阵,血满征袍。正行间,山坡下又撞出两枝军,乃夏侯惇部
将钟缙、钟绅兄弟二人,一个使大斧,一个使画戟,大喝:“赵云快下马受缚!”正是:
“才离虎窟愈生去,又遇龙潭鼓浪来。毕竟子龙怎地脱身,且听下回分解。