\chapter{文鸯单骑退雄兵~姜维背水破大敌}

却说魏正元二年正月,扬州都督、镇东将军、领淮南军马毋丘俭,字仲恭,河东闻喜人
也。闻司马师擅行废立之事,心中大怒。长子毋丘甸曰:“父亲官居方面,司马师专权废
主,国家有累卵之危,安可宴然自守?”俭曰:“吾儿之言是也。”遂请刺史文钦商议。钦
乃曹爽门下客,当日闻俭相请,即来参谒。俭邀入后堂,礼毕,说话间,俭流泪不止。钦问
其故,俭曰:“司马师专权废主,天地反覆,安得不伤心乎!”钦曰:“都督镇守方面,若
肯仗义讨贼,钦愿舍死相助。钦中子文淑,小字阿鸯,有万夫不当之勇,常欲杀司马师兄
弟,与曹爽报仇,今可令为先锋。”俭大喜,即时酹酒为誓。二人诈称太后有密诏,令淮南
大小官兵将士,皆入寿春城,立一坛于西,宰白马歃血为盟,宣言司马师大逆不道,今奉太
后密诏,令尽起淮南军马,仗义讨贼。众皆悦服。俭提六万兵,屯于项城。文钦领兵二万在
外为游兵,往来接应。俭移檄诸郡,令各起兵相助。却说司马师左眼肉瘤,不时痛痒,乃命
医官割之,以药封闭,连日在府养病;忽闻淮南告急,乃请太尉王肃商议。肃曰:“昔关云
长威震华夏,孙权令吕蒙袭取荆州,抚恤将士家属,因此关公军势瓦解,今淮南将士家属,
皆在中原,可急抚恤,更以兵断其归路:必有土崩之势矣。”师曰:“公言极善。但吾新割
目瘤,不能自往。若使他人,心又不稳。”时中书侍郎钟会在侧,进言曰:“淮楚兵强,其
锋甚锐;若遣人领兵去退,多是不利。倘有疏虞,则大事废矣。”师蹶然起曰:“非吾自
在,不可破贼!”遂留弟司马昭守洛阳,总摄朝政。师乘软舆,带病东行。令镇东将军诸葛
诞,总督豫州诸军,从安风津取寿春;又令征东将军胡遵,领青州诸军,出谯、宋之地,绝
其归路;又遣荆州刺史、监军王基,领前部兵,先取镇南之地。师领大军屯于襄阳,聚文武
于帐下商议。光禄勋郑袤曰:“毋丘俭好谋而无断,文钦有勇而无智。今大军出其不意,
江、淮之卒锐气正盛,不可轻敌;只宜深沟高垒,以挫其锐。此亚夫之长策也。”监军王基
曰:“不可。淮南之反,非军民思乱也;皆因毋丘俭势力所逼,不得已而从之。若大军一
临,必然瓦解。”师曰:“此言甚妙。”遂进兵于濦水之上,中军屯于濦桥。基曰:“南
顿极好屯兵,可提兵星夜取之。若迟则毋丘俭必先至矣。”师遂令王基领前部兵来南顿城下
寨。

却说毋丘俭在项城,闻知司马师自来,乃聚众商议。先锋葛雍曰:“南顿之地,依山傍
水,极好屯兵;若魏兵先占,难以驱遣,可速取之。”俭然其言,起兵投南顿来。正行之
间,前面流星马报说,南顿已有人马下寨。俭不信,自到军前视之,果然旌旗遍野,营寨齐
整。俭回到军中,无计可施。忽哨马飞报:“东吴孙峻提兵渡江袭寿春来了。”俭大惊曰:
“寿春若失,吾归何处!”是夜退兵于项城。

司马师见毋丘俭军退,聚多官商议。尚书傅嘏曰:“今俭兵退者,忧吴人袭寿春也。必
回项城分兵拒守。将军可令一军取乐嘉城,一军取项城,一军取寿春,则淮南之卒必退矣。
兖州刺史邓艾,足智多谋;若领兵径取乐嘉,更以重兵应之,破贼不难也。”师从之,急遣
使持檄文,教邓艾起兖州之兵破乐嘉城。师随后引兵到彼会合。

却说毋丘俭在项城,不时差人去乐嘉城哨探,只恐有兵来。请文钦到营共议,钦曰:
“都督勿忧。我与拙子文鸯,只消五千兵,取保乐嘉城。”俭大喜。钦父子引五千兵投乐嘉
来。前军报说:“乐嘉城西,皆是魏兵,约有万余。遥望中军,白旄黄钺,皂盖朱幡,簇拥
虎帐,内竖一面锦绣帅字旗,必是司马师也,安立营寨,尚未完备。”时文鸯悬鞭立于父
侧,闻知此语,乃告父曰:“趁彼营寨未成,可分兵两路,左右击之,可全胜也。”钦曰:
“何时可去?”鸯曰:“今夜黄昏,父引二千五百兵,从城南杀来;儿引二千五百兵,从城
北杀来:三更时分,要在魏寨会合。”钦从之,当晚分兵两路。且说文鸯年方十八岁,身长
八尺,全装惯甲,腰悬钢鞭,绰枪上马,遥望魏寨而进。是夜,司马师兵到乐嘉,立下营
寨,等邓艾未至。师为眼下新割肉瘤,疮口疼痛,卧于帐中,令数百甲士环立护卫。三更时
分,忽然寨内喊声大震,人马大乱。师急问之,人报曰:“一军从寨北斩围直入,为首一
将,勇不可当!”师大惊,心如火烈,眼珠从肉瘤疮口内迸出,血流遍地,疼痛难当;又恐
有乱军心,只咬被头而忍,被皆咬烂。原来文鸯军马先到,一拥而进,在寨中左冲右突;所
到之处,人不敢当,有相拒者,枪搠鞭打,无不被杀。鸯只望父到,以为外应,并不见来。
数番杀到中军,皆被弓弩射回。鸯直杀到天明,只听得北边鼓角喧天。鸯回顾从者曰:“父
亲不在南面为应,却从北至,何也?”鸯纵马看时,只见一军行如猛风,为首一将,乃邓艾
也,跃马横刀,大呼曰:“反贼休走!”鸯大怒,挺枪迎之。战有五十合,不分胜败。正斗
间,魏兵大进,前后夹攻,鸯部下兵乃各自逃散,只文鸯单人独马,冲开魏兵,望南而走。
背后数百员魏将,抖擞精神,骤马追来;将至乐嘉桥边,看看赶上。鸯忽然勒回马大喝一
声,直冲入魏将阵中来;钢鞭起处,纷纷落马,各各倒退。鸯复缓缓而行。魏将聚在一处,
惊讶曰:“此人尚敢退我等之众耶!可并力追之!”于是魏将百员,复来追赶。鸯勃然大怒
曰:“鼠辈何不惜命也!”提鞭拨马,杀入魏将丛中,用鞭打死数人,复回马缓辔而行。魏
将连追四五番,皆被文鸯一人杀退。后人有诗曰:“长坂当年独拒曹,子龙从此显英豪。乐
嘉城内争锋处,又见文鸯胆气高。”原来文钦被山路崎岖,迷入谷中,行了半夜,比及寻路
而出,天色已晓,文鸯人马不知所向,只见魏兵大胜。钦不战而退。魏兵乘势追杀,钦引兵
望寿春而走。

却说魏殿中校尉尹大目,乃曹爽心腹之人,因爽被司马懿谋杀,故事司马师,常有杀师
报爽之心;又素与文钦交厚。今见师眼瘤突出,不能动止,乃入帐告曰:“文钦本无反心,
今被毋丘俭逼迫,以致如此。某去说之,必然来降。”师从之。大目顶盔惯甲,乘马来赶文
钦;看看赶上,乃高声大叫曰:“文刺史见尹大目么?”钦回头视之,大目除盔放于鞍鞒之
前,以鞭指曰:“文刺史何不忍耐数日也?”此是大目知师将亡,故来留钦。钦不解其意,
厉声大骂,便欲开弓射之。大目大哭而回。钦收聚人马奔寿春时,已被诸葛诞引兵取了;欲
复回项城时,胡遵、王基、邓艾三路兵皆到。钦见势危,遂投东吴孙峻去了。却说毋丘俭在
项城内,听知寿春已失,文钦势败,城外三路兵到,俭遂尽撤城中之兵出战。正与邓艾相
遇,俭令葛雍出马,与艾交锋,不一合,被艾一刀斩之,引兵杀过阵来。毋丘俭死战相拒。
江淮兵大乱。胡遵、王基引兵四面夹攻。毋丘俭敌不住,引十余骑夺路而走。前至慎县城
下,县令宋白开门接入,设席待之。俭大醉,被宋白令人杀了,将头献与魏兵。于是淮南平
定。司马师卧病不起,唤诸葛诞入帐,赐以印绶,加为镇东大将军,都督扬州诸路军马;一
面班师回许昌。师目痛不止,每夜只见李丰、张缉、夏侯玄三人立于榻前。师心神恍惚,自
料难保,遂令人往洛阳取司马昭到。昭哭拜于床下。师遗言曰:“吾今权重,虽欲卸肩,不
可得也。汝继我为之,大事切不可轻托他人,自取灭族之祸。”言讫,以印绶付之,泪流满
面。昭急欲问时,师大叫一声,眼睛迸出而死。时正元二年二月也。于是司马昭发丧,申奏
魏主曹髦。

髦遣使持诏到许昌,即命暂留司马昭屯军许昌,以防东吴。昭心中犹豫未决。钟会曰:
“大将军新亡,人心未定,将军若留守于此。万一朝廷有变,悔之何及?”昭从之,即起兵
还屯洛水之南。髦闻之大惊。太尉王肃奏曰:“昭既继其兄掌大权,陛下可封爵以安之。”
髦遂命王肃持诏,封司马昭为大将军、录尚书事。昭入朝谢恩毕。自此,中外大小事情,皆
归于昭。却说西蜀细作哨知此事,报入成都。姜维奏后主曰:“司马师新亡,司马昭初握重
权,必不敢擅离洛阳。臣请乘间伐魏,以复中原。”后主从之,遂命姜维兴师伐魏。维到汉
中,整顿人马。征西大将军张翼曰:“蜀地浅狭,钱粮鲜薄,不宜远征;不如据险守分,恤
军爱民:此乃保国之计也。”维曰:“不然。昔丞相未出茅庐,已定三分天下,然且六出祁
山以图中原;不幸半途而丧,以致功业未成。今吾既受丞相遗命,当尽忠报国以继其志,虽
死而无恨也。今魏有隙可乘,不就此时伐之,更待何时?”夏侯霸曰:“将军之言是也。可
将轻骑先出枹罕。若得洮西南安,则诸郡可定。”张翼曰:“向者不克而还,皆因军出甚迟
也。兵法云:攻其无备,出其不意。今若火速进兵,使魏人不能提防,必然全胜矣。”

于是姜维引兵五万,望枹罕进发。兵至洮水,守边军士报知雍州刺史王经、征西将军陈
泰。王经先起马步兵七万来迎。姜维分付张翼如此如此,又分付夏侯霸如此如此:二人领计
去了;维乃自引大军背洮水列阵。王经引数员牙将出而问曰:“魏与吴、蜀,已成鼎足之
势;汝累次入寇,何也?”维曰:“司马师无故废主,邻邦理宜问罪,何况仇敌之国乎?”
经回顾张明、花永、刘达、朱芳四将曰:“蜀兵背水为阵。败则皆没于水矣。姜维骁勇,汝
四将可战之。彼若退动,便可追击。”四将分左右而出,来战姜维。维略战数合,拨回马望
本阵中便走。王经大驱士马,一齐赶来。维引兵望着洮水而走;将次近水,大呼将士曰:
“事急矣!诸将何不努力!”众将一齐奋力杀回,魏兵大败。张翼、夏侯霸抄在魏兵之后,
分两路杀来,把魏兵困在垓心。维奋武扬威,杀入魏军之中,左冲右突,魏兵大乱,自相践
踏,死者大半,逼入洮水者无数,斩首万余,垒尸数里。王经引败兵百骑,奋力杀出,径往
狄道城而走;奔入城中,闭门保守。

姜维大获全功,犒军已毕,便欲进兵攻打狄道城。张翼谏曰:“将军功绩已成,威声大
震,可以止矣。今若前进,倘不如意,正如画蛇添足也。”维曰:“不然。向者兵败,尚欲
进取,纵横中原;今日洮水一战,魏人胆裂,吾料狄道唾手可得。汝勿自堕其志也。”张翼
再三劝谏,维不从,遂勒兵来取狄道城。却说雍州征西将军陈泰,正欲起兵与王经报兵败之
仇,忽兖州刺史邓艾引兵到。泰接着,礼毕,艾曰:“今奉大将军之命,特来助将军破
敌。”泰问计于邓艾,艾曰:“洮水得胜,若招羌人之众,东争关陇,传檄四郡:此吾兵之
大患也。今彼不思如此,却图狄道城;其城垣坚固,急切难攻,空劳兵费力耳。吾今陈兵于
项岭,然后进兵击之,蜀兵必败矣。”陈泰曰:“真妙论也!”遂先拨二十队兵,每队五十
人,尽带旌旗、鼓角、烽火之类,日伏夜行,去狄道城东南高山深谷之中埋伏;只待兵来,
一齐鸣鼓吹角为应,夜则举火放炮以惊之。调度已毕,专候蜀兵到来。于是陈泰、邓艾,各
引二万兵相继而进。却说姜维围住狄道城,令兵八面攻之,连攻数日不下,心中郁闷,无计
可施。是日黄昏时分,忽三五次流星马报说:“有两路兵来,旗上明书大字:一路是征西将
军陈泰,一路是兖州刺史邓艾。”维大惊,遂请夏侯霸商议。霸曰:“吾向尝为将军言:邓
艾自幼深明兵法,善晓地理。今领兵到,颇为劲敌。”维曰:“彼军远来,我休容他住脚,
便可击之。”乃留张翼攻城,命夏侯霸引兵迎陈泰。维自引兵来迎邓艾。行不到五里,忽然
东南一声炮响,鼓角震地,火光冲天。维纵马看时,只见周围皆是魏兵旗号。维大惊曰:
“中邓艾之计矣!”遂传令教夏侯霸、张翼各弃狄道而退。于是蜀兵皆退于汉中。维自断
后,只听得背后鼓声不绝,维退入剑阁之时,方知火鼓二十余处,皆虚设也。维收兵退屯于
钟提。

且说后主因姜维有洮西之功,降诏封维为大将军。维受了职,上表谢恩毕,再议出师伐
魏之策。正是:成功不必添蛇足,讨贼犹思奋虎威。不知此番北伐如何,且看下文分解。