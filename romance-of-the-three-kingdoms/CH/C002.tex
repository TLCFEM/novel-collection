\chapter{张翼德怒鞭督邮~何国舅谋诛宦竖}

且说董卓字仲颖,陇西临洮人也,官拜河东太守,自来骄傲。当日怠慢了玄德,张飞性发,便欲杀之。玄德与关公急止之曰;“他是朝廷命官,岂可擅杀?”飞曰:“若不杀这厮,反要在他部下听令,其实不甘!二兄要便住在此,我自投别处去也!”玄德曰:“我三人义同生死,岂可相离?不若都投别处去便了。”飞曰:“若如此,稍解吾恨。”

于是三人连夜引军来投朱儁。儁待之甚厚,合兵一处,进讨张宝。是时曹操自跟皇甫嵩讨张梁,大战于曲阳。这里朱儁进攻张宝。张宝引贼众八九万,屯于山后。儁令玄德为其先锋,与贼对敌。张宝遣副将高升出马搦战,玄德使张飞击之。飞纵马挺矛,与升交战,不数合,刺升落马。玄德麾军直冲过去。张宝就马上披发仗剑,作起妖法。只见风雷大作,一股黑气从天而降,黑气中似有无限人马杀来。玄德连忙回军,军中大乱。败阵而归,与朱儁计议。儁曰:“彼用妖术,我来日可宰猪羊狗血,令军士伏于山头;候贼赶来,从高坡上泼之,其法可解。”玄德听令,拨关公、张飞各引军一千,伏于山后高冈之上,盛猪羊狗血并秽物准备。次日,张宝摇旗擂鼓,引军搦战,玄德出迎。交锋之际,张宝作法,风雷大作,飞砂走石,黑气漫天,滚滚人马,自天而下。玄德拨马便走,张宝驱兵赶来。将过山头,关、张伏军放起号炮,秽物齐泼。但见空中纸人草马,纷纷坠地;风雷顿息,砂石不飞。

张宝见解了法,急欲退军。左关公,右张飞,两军都出,背后玄德、朱儁一齐赶上,贼兵大败。玄德望见“地公将军”旗号,飞马赶来,张宝落荒而走。玄德发箭,中其左臂。张宝带箭逃脱,走入阳城,坚守不出。

朱儁引兵围住阳城攻打,一面差人打探皇甫嵩消息。探子回报,具说:“皇甫嵩大获胜捷,朝廷以董卓屡败,命嵩代之。嵩到时,张角已死;张梁统其众,与我军相拒,被皇甫嵩连胜七阵,斩张梁于曲阳。发张角之棺,戮尸枭首,送往京师。余众俱降。朝廷加皇甫嵩为车骑将军,领冀州牧。皇甫嵩又表奏卢植有功无罪,朝廷复卢植原官。曹操亦以有功,除济南相,即日将班师赴任。”朱儁听说,催促军马,悉力攻打阳城。贼势危急,贼将严政刺杀张宝,献首投降。朱儁遂平数郡,上表献捷。时又黄巾余党三人:赵弘、韩忠、孙仲,聚众数万,望风烧劫,称与张角报仇。朝廷命朱儁即以得胜之师讨之。儁奉诏,率军前进。时贼据宛城,儁引兵攻之,赵弘遣韩忠出战。儁遣玄德、关、张攻城西南角。韩忠尽率精锐之众,来西南角抵敌。朱儁自纵铁骑二千,径取东北角。贼恐失城,急弃西南面回。玄德从背后掩杀,贼众大败,奔入宛城。朱儁分兵四面围定。城中断粮,韩忠使人出城投降。儁不许。玄德曰:“昔高祖之得天下,盖为能招降纳顺;公何拒韩忠耶?”儁曰:“彼一时,此一时也。昔秦项之际,天下大乱,民无定主,故招降赏附,以劝来耳。今海内一统,惟黄巾造反;若容其降,无以劝善。使贼得利恣意劫掠,失利便投降:此长寇之志,非良策也。”玄德曰:“不容寇降是矣。今四面围如铁桶,贼乞降不得,必然死战。万人一心,尚不可当,况城中有数万死命之人乎?不若撤去东南,独攻西北。贼必弃城而走,无心恋战,可即擒也。”儁然之,随撤东南二面军马,一齐攻打西北。韩忠果引军弃城而奔。儁与玄德、关、张率三军掩杀,射死韩忠,余皆四散奔走。正追赶间,赵弘、孙仲引贼众到,与儁交战。儁见弘势大,引军暂退。弘乘势复夺宛城。儁离十里下寨。方欲攻打,忽见正东一彪人马到来。为首一将,生得广额阔面,虎体熊腰;吴郡富春人也,姓孙,名坚,字文台,乃孙武子之后。年十七岁时,与父至钱塘,见海贼十余人,劫取商人财物,于岸上分赃。坚谓父曰:“此贼可擒也。”遂奋力提刀上岸,扬声大叫,东西指挥,如唤人状。贼以为官兵至,尽弃财物奔走。坚赶上,杀一贼。由是郡县知名,荐为校尉。后会稽妖贼许昌造反,自称“阳明皇帝”,聚众数万;坚与郡司马招募勇士千余人,会合州郡破之,斩许昌并其子许韶。刺史臧旻上表奏其功,除坚为盐渎丞,又除盱眙丞、下邳丞。今见黄巾寇起,聚集乡中少年及诸商旅,并淮泗精兵一千五百余人,前来接应。

朱儁大喜,便令坚攻打南门,玄德打北门,朱儁打西门,留东门与贼走。孙坚首先登城,斩贼二十余人,贼众奔溃。赵弘飞马突槊,直取孙坚。坚从城上飞身夺弘槊,刺弘下马;却骑弘马,飞身往来杀贼。孙仲引贼突出北门,正迎玄德,无心恋战,只待奔逃。玄德张弓一箭,正中孙仲,翻身落马。朱儁大军随后掩杀,斩首数万级,降者不可胜计。南阳一路,十数郡皆平。儁班师回京,诏封为车骑将军,河南尹。儁表奏孙坚、刘备等功。坚有人情,除别郡司马上任去了。惟玄德听候日久,不得除授,三人郁郁不乐,上街闲行,正值郎中张钧车到。玄德见之,自陈功绩。钧大惊,随入朝见帝曰:“昔黄巾造反,其原皆由十常侍卖官鬻爵,非亲不用,非仇不诛,以致天下大乱。今宜斩十常侍,悬首南郊,遣使者布告天下,有功者重加赏赐,则四海自清平也。”十常侍奏帝曰:“张钧欺主。”帝令武士逐出张钧。十常侍共议:“此必破黄巾有功者,不得除授,故生怨言。权且教省家铨注微名,待后却再理会未晚。”因此玄德除授定州中山府安喜县尉,克日赴任。

玄德将兵散回乡里,止带亲随二十余人,与关、张来安喜县中到任。署县事一月,与民秋毫无犯,民皆感化。到任之后,与关、张食则同桌,寝则同床。如玄德在稠人广坐,关、张侍立,终日不倦。到县未及四月,朝廷降诏,凡有军功为长吏者当沙汰。玄德疑在遣中。适督邮行部至县,玄德出郭迎接,见督邮施礼。督邮坐于马上,惟微以鞭指回答。关、张二公俱怒。及到馆驿,督邮南面高坐,玄德侍立阶下。良久,督邮问曰:“刘县尉是何出身?”玄德曰:“备乃中山靖王之后;自涿郡剿戮黄巾,大小三十余战,颇有微功,因得除今职。”督邮大喝曰:“汝诈称皇亲,虚报功绩!目今朝廷降诏,正要沙汰这等滥官污吏!”玄德喏喏连声而退。归到县中,与县吏商议。吏曰:“督邮作威,无非要贿赂耳。”玄德曰:“我与民秋毫无犯,那得财物与他?”次日,督邮先提县吏去,勒令指称县尉害民。玄德几番自往求免,俱被门役阻住,不肯放参。

却说张飞饮了数杯闷酒,乘马从馆驿前过,见五六十个老人,皆在门前痛哭。飞问其故,众老人答曰:“督邮逼勒县吏,欲害刘公;我等皆来苦告,不得放入,反遭把门人赶打!”张飞大怒,睁圆环眼,咬碎钢牙,滚鞍下马,径入馆驿,把门人那里阻挡得住,直奔后堂,见督邮正坐厅上,将县吏绑倒在地。飞大喝:“害民贼!认得我么?”督邮未及开言,早被张飞揪住头发,扯出馆驿,直到县前马桩上缚住;攀下柳条,去督邮两腿上着力鞭打,一连打折柳条十数枝。玄德正纳闷间,听得县前喧闹,问左右,答曰:“张将军绑一人在县前痛打。”玄德忙去观之,见绑缚者乃督邮也。玄德惊问其故。飞曰:“此等害民贼,不打死等甚!”督邮告曰:“玄德公救我性命!”玄德终是仁慈的人,急喝张飞住手。傍边转过关公来,曰:“兄长建许多大功,仅得县尉,今反被督邮侮辱。吾思枳棘丛中,非栖鸾凤之所;不如杀督邮,弃官归乡,别图远大之计。”玄德乃取印绶,挂于督邮之颈,责之曰:据汝害民,本当杀却;今姑饶汝命。吾缴还印绶,从此去矣。”督邮归告定州太守,太守申文省府,差人捕捉。玄德、关、张三人往代州投刘恢。恢见玄德乃汉室宗亲,留匿在家不题。

却说十常侍既握重权,互相商议:但有不从己者,诛之。赵忠、张让差人问破黄巾将士索金帛,不从者奏罢职。皇甫嵩、朱儁皆不肯与,赵忠等俱奏罢其官。帝又封赵忠等为车骑将军,张让等十三人皆封列侯。朝政愈坏,人民嗟怨。于是长沙贼区星作乱;渔阳张举、张纯反:举称天子,纯称大将军。表章雪片告急,十常侍皆藏匿不奏。

一日,帝在后园与十常侍饮宴,谏议大夫刘陶,径到帝前大恸。帝问其故。陶曰:“天下危在旦夕,陛下尚自与阉宦共饮耶!”帝曰:“国家承平,有何危急?”陶曰:“四方盗贼并起,侵掠州郡。其祸皆由十常侍卖官害民,欺君罔上。朝廷正人皆去,祸在目前矣!”十常侍皆免冠跪伏于帝前曰:“大臣不相容,臣等不能活矣!愿乞性命归田里,尽将家产以助军资。”言罢痛哭。帝怒谓陶曰:“汝家亦有近侍之人,何独不容朕耶?”呼武士推出斩之。刘陶大呼:“臣死不惜!可怜汉室天下,四百余年,到此一旦休矣!”

武士拥陶出,方欲行刑,一大臣喝住曰:“勿得下手,待我谏去。”众视之,乃司徒陈耽,径入宫中来谏帝曰:“刘谏议得何罪而受诛?”帝曰:“毁谤近臣,冒渎朕躬。”耽曰:“天下人民,欲食十常侍之肉,陛下敬之如父母,身无寸功,皆封列侯;况封谞等结连黄巾,欲为内乱:陛下今不自省,社稷立见崩摧矣!”帝曰:“封谞作乱,其事不明。十常侍中,岂无一二忠臣?”陈耽以头撞阶而谏。帝怒,命牵出,与刘陶皆下狱。是夜,十常侍即于狱中谋杀之;假帝诏以孙坚为长沙太守,讨区星,不五十日,报捷,江夏平,诏封坚为乌程侯。

封刘虞为幽州牧,领兵往渔阳征张举、张纯。代州刘恢以书荐玄德见虞。虞大喜,令玄德为都尉,引兵直抵贼巢,与贼大战数日,挫动锐气。张纯专一凶暴,士卒心变,帐下头目刺杀张纯,将头纳献,率众来降。张举见势败,亦自缢死。渔阳尽平。刘虞表奏刘备大功,朝廷赦免鞭督邮之罪,除下密丞,迁高堂尉。公孙瓒又表陈玄德前功,荐为别部司马,守平原县令。玄德在平原,颇有钱粮军马,重整旧日气象。刘虞平寇有功,封太尉。中平六年夏四月,灵帝病笃,召大将军何进入宫,商议后事。那何进起身屠家;因妹入宫为贵人,生皇子辩,遂立为皇后。进由是得权重任。帝又宠幸王美人,生皇子协。何后嫉妒,鸩杀王美人。皇子协养于董太后宫中。董太后乃灵帝之母,解渎亭侯刘苌之妻也。初因桓帝无子,迎立解渎亭侯之子,是为灵帝。灵帝入继大统,遂迎养母氏于宫中,尊为太后。董太后尝劝帝立皇子协为太子。帝亦偏爱协,欲立之。当时病笃,中常侍蹇硕奏曰:“若欲立协,必先诛何进,以绝后患。”帝然其说,因宣进入宫。进至宫门,司马潘隐谓进曰:“不可入宫。蹇硕欲谋杀公。”进大惊,急归私宅,召诸大臣,欲尽诛宦官。座上一人挺身出曰:“宦官之势,起自冲、质之时;朝廷滋蔓极广,安能尽诛?倘机不密,必有灭族之祸:请细详之。”进视之,乃典军校尉曹操也。进叱曰:“汝小辈安知朝廷大事!”正踌躇间,潘隐至,言:“帝已崩。今赛硕与十常侍商议,秘不发丧,矫诏宣何国舅入宫,欲绝后患,册立皇子协为帝。”说未了,使命至,宣进速入,以定后事。操曰:“今日之计,先宜正君位,然后图贼。”进曰:“谁敢与吾正君讨贼?”一人挺身出曰:“愿借精兵五千,斩关入内,册立新君,尽诛阉竖,扫清朝廷,以安天下!”进视之,乃司徒袁逢之子,袁隗之侄:名绍,字本初,现为司隶校尉。何进大喜,遂点御林军五千。绍全身披挂。何进引何顒、荀攸、郑泰等大臣三十余员,相继而入,就灵帝柩前,扶立太子辩即皇帝位。

百官呼拜已毕,袁绍入宫收蹇硕。硕慌走入御园,花阴下为中常侍郭胜所杀。硕所领禁军,尽皆投顺。绍谓何进曰:“中官结党。今日可乘势尽诛之。”张让等知事急,慌入告何后曰:“始初设谋陷害大将军者,止赛硕一人,并不干臣等事。今大将军听袁绍之言,欲尽诛臣等,乞娘娘怜悯!”何太后曰:“汝等勿忧,我当保汝。”传旨宣何进入。太后密谓曰:“我与汝出身寒微,非张让等,焉能享此富贵?今蹇硕不仁,既已伏诛,汝何听信人言,欲尽诛宦官耶?”何进听罢,出谓众官曰:“蹇硕设谋害我,可族灭其家。其余不必妄加残害。”袁绍曰:“若不斩草除根,必为丧身之本。”进曰:“吾意已决,汝勿多言。”众官皆退。次日,太后命何进参录尚书事,其余皆封官职。董太后宣张让等入宫商议曰:“何进之妹,始初我抬举他。今日他孩儿即皇帝位,内外臣僚,皆其心腹:威权太重,我将如何?”让奏曰:“娘娘可临朝,垂帘听政;封皇子协为王;加国舅董重大官,掌握军权;重用臣等:大事可图矣。”董太后大喜。次日设朝,董太后降旨,封皇子协为陈留王,董重为骠骑将军,张让等共预朝政。何太后见董太后专权,于宫中设一宴,请董太后赴席。酒至半酣,何太后起身捧杯再拜曰:“我等皆妇人也,参预朝政,非其所宜。昔吕后因握重权,宗族千口皆被戮。今我等宜深居九重;朝廷大事,任大臣元老自行商议,此国家之幸也。愿垂听焉。”董后大怒曰:“汝鸩死王美人,设心嫉妒。今倚汝子为君,与汝兄何进之势,辄敢乱言!吾敕骠骑断汝兄首,如反掌耳!”何后亦怒曰:“吾以好言相劝,何反怒耶?”董后曰:“汝家屠沽小辈,有何见识!”两宫互相争竞,张让等各劝归宫。何后连夜召何进入宫,告以前事。何进出,召三公共议。来早设朝,使廷臣奏董太后原系藩妃,不宜久居宫中,合仍迁于河间安置,限日下即出国门。一面遣人起送董后;一面点禁军围骠骑将军董重府宅,追索印绶。董重知事急,自刎于后堂。家人举哀,军士方散。张让、段珪见董后一枝已废,遂皆以金珠玩好结构何进弟何苗并其母舞阳君,令早晚入何太后处,善言遮蔽:因此十常侍又得近幸。

六月,何进暗使人鸩杀董后于河间驿庭,举柩回京,葬于文陵。进托病不出。司隶校尉袁绍入见进曰:“张让、段珪等流言于外,言公鸩杀董后,欲谋大事。乘此时不诛阉宦,后必为大祸。昔窦武欲诛内竖,机谋不密,反受其殃。今公兄弟部曲将吏,皆英俊之士;若使尽力,事在掌握。此天赞之时,不可失也。”进曰:“且容商议。”左右密报张让,让等转告何苗,又多送贿赂。苗入奏何后云:“大将军辅佐新君,不行仁慈,专务杀伐。今无端又欲杀十常侍,此取乱之道也。”后纳其言。少顷,何进入白后,欲诛中涓。何后曰:“中官统领禁省,汉家故事。先帝新弃天下,尔欲诛杀旧臣,非重宗庙也。”进本是没决断之人,听太后言,唯唯而出。袁绍迎问曰:“大事若何?”进曰:“太后不允,如之奈何?”绍曰:“可召四方英雄之士,勒兵来京,尽诛阉竖。此时事急,不容太后不从。”进曰:“此计大妙!”便发檄至各镇,召赴京师。主薄陈琳曰:“不可!俗云:掩目而捕燕雀,是自欺也,微物尚不可欺以得志,况国家大事乎?今将军仗皇威,掌兵要,龙骧虎步,高下在心:若欲诛宦官,如鼓洪炉燎毛发耳。但当速发雷霆,行权立断,则天人顺之。却反外檄大臣,临犯京阙,英雄聚会,各怀一心:所谓倒持干戈,授人以柄,功必不成,反生乱矣。”何进笑曰:“此懦夫之见也!”傍边一人鼓掌大笑曰:“此事易如反掌,何必多议!”视之,乃曹操也。正是:欲除君侧宵人乱,须听朝中智士谋。不知曹操说出甚话来,且听下文分解。