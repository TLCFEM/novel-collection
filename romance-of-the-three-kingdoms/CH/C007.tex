\chapter{袁绍磐河战公孙~孙坚跨江击刘表}

却说孙坚被刘表围住,亏得程普、黄盖、韩当三将死救得脱,折兵大半,夺路引兵回江东。自此孙坚与刘表结怨。

且说袁绍屯兵河内,缺少粮草。冀州牧韩馥,遣人送粮以资军用。谋士逢纪说绍曰:“大丈夫纵横天下,何待人送粮为食!冀州乃钱粮广盛之地,将军何不取之?”绍曰:“未有良策。”纪曰:“可暗使人驰书与公孙瓒,令进兵取冀州,约以夹攻,瓒必兴兵。韩馥无谋之辈,必请将军领州事;就中取事,唾手可得。”绍大喜,即发书到瓒处。瓒得书,见说共攻冀州,平分其地,大喜,即日兴兵。

绍却使人密报韩馥。馥慌聚荀谌、辛评二谋士商议。谌曰:“公孙瓒将燕、代之众,长驱而来,其锋不可当。兼有刘备、关、张助之,难以抵敌。今袁本初智勇过人,手下名将极广,将军可请彼同治州事,彼必厚待将军,无患公孙瓒矣。”韩馥即差别驾关纯去请袁绍。长史耿武谏曰:“袁绍孤客穷军,仰我鼻息,譬如婴儿在股掌之上,绝其乳哺,立可饿死。奈何欲以州事委之?此引虎入羊群也。”馥曰:“吾乃袁氏之故吏,才能又不如本初。古者择贤者而让之,诸君何嫉妒耶?”耿武叹曰:“冀州休矣!”于是弃职而去者三十余人。独耿武与关纯伏于城外,以待袁绍。

数日后,绍引兵至。耿武、关纯拔刀而出,欲刺杀绍。绍将颜良立斩耿武,文丑砍死关纯。绍入冀州,以馥为奋威将军,以田丰、沮授、许攸、逢纪分掌州事,尽夺韩馥之权。馥懊悔无及,遂弃下家小,匹马往投陈留太守张邈去了。

却说公孙瓒知袁绍已据冀州,遣弟公孙越来见绍,欲分其地。绍曰:“可请汝兄自来,吾有商议。”越辞归。行不到五十里,道旁闪出一彪军马,口称:“我乃董丞相家将也!”乱箭射死公孙越。从人逃回见公孙瓒,报越已死。瓒大怒曰:“袁绍诱我起兵攻韩馥,他却就里取事;今又诈董卓兵射死吾弟,此冤如何不报!”尽起本部兵,杀奔冀州来。

绍知瓒兵至,亦领军出。二军会于磐河之上:绍军于磐河桥东,瓒军于桥西。瓒立马桥上,大呼曰:“背义之徒,何敢卖我!”绍亦策马至桥边,指瓒曰:“韩馥无才,愿让冀州于吾,与尔何干?”瓒曰:“昔日以汝为忠义,推为盟主;今之所为,真狼心狗行之徒,有何面目立于世间!”袁绍大怒曰:“谁可擒之?”言未毕,文丑策马挺枪,直杀上桥。公孙瓒就桥边与文丑交锋。战不到十余合,瓒抵挡不住,败阵而走。文丑乘势追赶。瓒走入阵中,文丑飞马径入中军,往来冲突。瓒手下健将四员,一齐迎战;被文丑一枪,刺一将下马,三将俱走。文丑直赶公孙瓒出阵后,瓒望山谷而逃。文丑骤马厉声大叫:“快下马受降!”瓒弓箭尽落,头盔堕地;披发纵马,奔转山坡;其马前失,瓒翻身落于坡下。文丑急捻枪来刺。忽见草坡左侧转出个少年将军,飞马挺枪,直取文丑,公孙瓒扒上坡去,看那少年:生得身长八尺,浓眉大眼,阔面重颐,威风凛凛,与文丑大战五六十合,胜负未分。瓒部下救军到,文丑拨回马去了。那少年也不追赶。瓒忙下土坡,问那少年姓名。那少年欠身答曰:“某乃常山真定人也,姓赵,名云,字子龙。本袁绍辖下之人。因见绍无忠君救民之心,故特弃彼而投麾下,不期于此处相见。”瓒大喜,遂同归寨,整顿甲兵。次日,瓒将军马分作左右两队,势如羽翼。马五千余匹,大半皆是白马。因公孙瓒曾与羌人战,尽选白马为先锋,号为白马将军;羌人但见白马便走,因此白马极多。袁绍令颜良、文丑为先锋,各引弓弩手一千,亦分作左右两队;令在左者射公孙瓒右军,在右者射公孙瓒左军。再令麴义引八百弓手,步兵一万五千,列于阵中。袁绍自引马步军数万,于后接应。公孙瓒初得赵云,不知心腹,令其另领一军在后。遣大将严纲为先锋。瓒自领中军,立马桥上,傍竖大红圈金线帅字旗于马前。从辰时擂鼓,直到巳时,绍军不进。麴义令弓手皆伏于遮箭牌下,只听炮响发箭。严纲鼓噪呐喊,直取麴义。义军见严纲兵来,都伏而不动;直到来得至近,一声炮响,八百弓弩手一齐俱发。纲急待回,被麴义拍马舞刀,斩于马下,瓒军大败。左右两军,欲来救应,都被颜良、文丑引弓弩手射住。绍军并进,直杀到界桥边。麴义马到,先斩执旗将,把绣旗砍倒。公孙瓒见砍倒绣旗,回马下桥而走。麴义引军直冲到后军,正撞着赵云,挺枪跃马,直取麴义。战不数合,一枪刺麴义于马下。赵云一骑马飞入绍军,左冲右突,如入无人之境。公孙瓒引军杀回,绍军大败。

却说袁绍先使探马看时,回报麴义斩将搴旗,追赶败兵;因此不作准备,与田丰引着帐下持戟军士数百人,弓箭手数十骑,乘马出观,呵呵大笑曰:“公孙瓒无能之辈!”正说之间,忽见赵云冲到面前。弓箭手急待射时,云连刺数人,众军皆走。后面瓒军团团围裹上来。田丰慌对绍曰:“主公且于空墙中躲避!”绍以兜鍪扑地,大呼曰:“大丈夫愿临阵斗死,岂可入墙而望活乎!”众军士齐心死战,赵云冲突不入,绍兵大队掩至,颜良亦引军来到,两路并杀。赵云保公孙瓒杀透重围,回到界桥。绍驱兵大进,复赶过桥,落水死者,不计其数。

袁绍当先赶来,不到五里,只听得山背后喊声大起,闪出一彪人马,为首三员大将,乃是刘玄德、关云长、张翼德。因在平原探知公孙瓒与袁绍相争,特来助战。当下三匹马,三般兵器,飞奔前来,直取袁绍。绍惊得魂飞天外,手中宝刀坠于马下,忙拨马而逃,众人死救过桥。公孙瓒亦收军归寨。玄德、关、张动问毕,瓒曰:“若非玄德远来救我,几乎狼狈。”教与赵云相见。玄德甚相敬爱,便有不舍之心。

却说袁绍输了一阵,坚守不出。两军相拒月余,有人来长安报知董卓。李儒对卓曰:“袁绍与公孙瓒,亦当今豪杰。现在磐河厮杀,宜假天子之诏,差人往和解之。二人感德,必顺太师矣。”卓曰:“善。”次日便使太傅马日磾、太仆赵岐,赍诏前去。二人来至河北,绍出迎于百里之外,再拜奉诏。次日,二人至瓒营宣谕,瓒乃遣使致书于绍,互相讲和。二人自回京复命。瓒即日班师,又表荐刘玄德为平原相。玄德与赵云分别,执手垂泪,不忍相离。云叹曰:“某曩日误认公孙瓒为英雄;今观所为,亦袁绍等辈耳!”玄德曰:“公且屈身事之,相见有日。”洒泪而别。

却说袁术在南阳,闻袁绍新得冀州,遣使来求马千匹。绍不与,术怒。自此兄弟不睦。又遣使往荆州,问刘表借粮二十万,表亦不与。术恨之,密遣人遗书于孙坚,使伐刘表。其书略曰:“前者刘表截路,乃吾兄本初之谋也。今本初又与表私议欲袭江东。公可速兴兵伐刘表,吾为公取本初,二仇可报。公取荆州,吾取冀州,切勿误也!”坚得书曰:“叵耐刘表昔日断吾归路,今不乘时报恨,更待何年!”聚帐下程普、黄盖、韩当等商议。程普曰:“袁术多诈,未可准信。”坚曰:“吾自欲报仇,岂望袁术之助乎?”便差黄盖先来江边安排战船,多装军器粮草,大船装载战马,克日兴师。江中细作探知,来报刘表。表大惊,急聚文武将士商议。蒯良曰:“不必忧虑。可令黄祖部领江夏之兵为前驱,主公率荆襄之众为援。孙坚跨江涉湖而来,安能用武乎?”表然之,令黄祖设备,随后便起大军。却说孙坚有四子,皆吴夫人所生:长子名策,字伯符;次子名权,字仲谋;三子名翊,字叔弼;四子名匡,字季佐。吴夫人之妹,即为孙坚次妻,亦生一子一女:子名朗,字早安;女名仁。坚又过房俞氏一子,名韶,字公礼。坚有一弟,名静,字幼台。坚临行,静引诸子列拜于马前而谏曰:“今董卓专权,天子懦弱,海内大乱,各霸一方;江东方稍宁,以一小恨而起重兵,非所宜也。愿兄详之。”坚曰:“弟勿多言。吾将纵横天下,有仇岂可不报!”长子孙策曰:“如父亲必欲往,儿愿随行。”坚许之,遂与策登舟,杀奔樊城。

黄祖伏弓弩手于江边,见船傍岸,乱箭俱发。坚令诸军不可轻动,只伏于船中来往诱之;一连三日,船数十次傍岸。黄祖军只顾放箭,箭已放尽。坚却拔船上所得之箭,约十数万。当日正值顺风,坚令军士一齐放箭。岸上支吾不住,只得退走。坚军登岸,程普、黄盖分兵两路,直取黄祖营寨。背后韩当驱兵大进。三面夹攻,黄祖大败,弃却樊城,走入邓城。坚令黄盖守住船只,亲自统兵追袭。黄祖引军出迎,布阵于野。坚列成阵势,出马于门旗之下。孙策也全副披挂,挺枪立马于父侧。黄祖引二将出马,一个是江夏张虎,一个是襄阳陈生。黄祖扬鞭大骂:“江东鼠贼,安敢侵犯汉室宗亲境界!”便令张虎搦战。坚阵内韩当出迎。两骑相交,战二十余合,陈主见张虎力怯,飞马来助。孙策望见,按住手中枪,扯弓搭箭,正射中陈生面门,应弦落马。张虎见陈生坠地,吃了一惊,措手不及,被韩当一刀,削去半个脑袋。程普纵马直来阵前捉黄祖。黄祖弃却头盔、战马,杂于步军内逃命。孙坚掩杀败军,直到汉水,命黄盖将船只进泊汉江。

黄祖聚败军,来见刘表,备言坚势不可当。表慌请蒯良商议。良曰:“目今新败,兵无战心;只可深沟高垒,以避其锋;却潜令人求教于袁绍,此围自可解也。”蔡瑁曰:“子柔之言,直拙计也。兵临城下,将至壕边,岂可束手待毙!某虽不才,愿请军出城,以决一战。”刘表许之。蔡瑁引军万余,出襄阳城外,于岘山布阵。孙坚将得胜之兵,长驱大进。蔡瑁出马。坚曰:“此人是刘表后妻之兄也,谁与吾擒之?”程普挺铁脊矛出马,与蔡瑁交战。不到数合,蔡瑁败走。坚驱大军,杀得尸横遍野。蔡瑁逃入襄阳。蒯良言瑁不听良策,以致大败,按军法当斩。刘表以新娶其妹,不肯加刑。

却说孙坚分兵四面,围住襄阳攻打。忽一日,狂风骤起,将中军帅字旗竿吹折。韩当曰:“此非吉兆,可暂班师。”坚曰:“吾屡战屡胜,取襄阳只在旦夕;岂可因风折旗竿,遽尔罢兵!”遂不听韩当之言,攻城愈急。蒯良谓刘表曰:“某夜观天象,见一将星欲坠。以分野度之,当应在孙坚。主公可速致书袁绍,求其相助。”刘表写书,问谁敢突围而出。健将吕公,应声愿往。蒯良曰:“汝既敢去,可听吾计:与汝军马五百,多带能射者冲出阵去,即奔岘山。他必引军来赶,汝分一百人上山,寻石子准备;一百人执弓弩伏于林中。但有追兵到时,不可径走;可盘旋曲折,引到埋伏之处,矢石俱发。若能取胜,放起连珠号炮,城中便出接应。如无追兵,不可放炮,趱程而去。今夜月不甚明,黄昏便可出城。”

吕公领了计策,拴束军马。黄昏时分,密开东门,引兵出城。孙坚在帐中,忽闻喊声,急上马引三十余骑,出营来看。军士报说:“有一彪人马杀将出来,望岘山而去。”坚不会诸将,只引三十余骑赶来。吕公已于山林丛杂去处,上下埋伏。坚马快,单骑独来,前军不远。坚大叫:“休走!”吕公勒回马来战孙坚。交马只一合,吕公便走,闪入山路去。坚随后赶入,却不见了吕公。坚方欲上山,忽然一声锣响,山上石子乱下,林中乱箭齐发。坚体中石、箭,脑浆迸流,人马皆死于岘山之内;寿止三十七岁。

吕公截住三十骑,并皆杀尽,放起连珠号炮。城中黄祖、蒯越、蔡瑁分头引兵杀出,江东诸军大乱。黄盖听得喊声震天,引水军杀来,正迎着黄祖。战不两合,生擒黄祖。程普保着孙策,急待寻路,正遇吕公。程普纵马向前,战不到数合,一矛刺吕公于马下。两军大战,杀到天明,各自收车。

刘表军自入城。孙策回到汉水,方知父亲被乱箭射死,尸首已被刘表军士扛抬入城去了,放声大哭。众军俱号泣。策曰:“父尸在彼,安得回乡!”黄盖曰:“今活捉黄祖在此,得一人入城讲和,将黄祖去换主公尸首。”言未毕,军吏桓阶出曰:“某与刘表有旧,愿入城为使。”策许之。桓阶入城见刘表,具说其事。表曰:“文台尸首、吾已用棺木盛贮在此。可速放回黄祖,两家各罢兵,再休侵犯。”桓阶拜谢欲行,阶下蒯良出曰:“不可!不可!吾有一言,今江东诸军片甲不回。请先斩桓阶,然后用计。”正是:追敌孙坚方殒命,求和桓阶又遭殃。未知桓阶性命如何,且听下文分解。