\chapter{诸葛亮乘雪破羌兵~司马懿克日擒孟达}

却说郭淮谓曹真曰:“西羌之人,自太祖时连年入贡,文皇帝亦有恩惠加之;我等今可
据住险阻,遣人从小路直入羌中求救,许以和亲,羌人必起兵袭蜀兵之后。吾却以大兵击
之,首尾夹攻,岂不大胜?”真从之,即遣人星夜驰书赴羌。

却说西羌国王彻里吉,自曹操时年年入贡;手下有一文一武:文乃雅丹丞相,武乃越吉
元帅。时魏使赍金珠并书到国,先来见雅丹丞相,送了礼物,具言求救之意。雅丹引见国
王,呈上书礼。彻里吉览了书,与众商议。雅丹曰:“我与魏国素相往来,今曹都督求救,
且许和亲,理合依允。”彻里吉从其言,即命雅丹与越吉元帅起羌兵一十五万,皆惯使弓
弩、枪刀、蒺藜、飞锤等器;又有战车,用铁叶裹钉,装载粮食军器什物:或用骆驼驾车,
或用骡马驾车,号为铁车兵。二人辞了国王,领兵直扣西平关。守关蜀将韩祯,急差人赍文
报知孔明。孔明闻报,问众将曰:“谁敢去退羌兵?”张苞、关兴应曰:“某等愿往。”孔
明曰:“汝二人要去,奈路途不熟。”遂唤马岱曰:“汝素知羌人之性,久居彼处,可作向
导。”便起精兵五万,与兴、苞二人同往。兴、苞等引兵而去。行有数日,早遇羌兵。关兴
先引百余骑登山坡看时,只见羌兵把铁车首尾相连,随处结寨;车上遍排兵器,就似城池一
般。兴睹之良久,无破敌之策,回寨与张苞、马岱商议。岱曰:“且待来日见阵,观看虚
实,另作计议。”次早,分兵三路:关兴在中,张苞在左,马岱在右,三路兵齐进。羌兵阵
里,越吉元帅手挽铁锤,腰悬宝雕弓,跃马奋勇而出。关兴招三路兵径进。忽见羌兵分在两
边,中央放出铁车,如潮涌一般,弓弩一齐骤发。蜀兵大败,马岱、张苞两军先退;关兴一
军,被羌兵一裹,直围入西北角上去了。

兴在垓心,左冲右突,不能得脱;铁车密围,就如城池。蜀兵你我不能相顾。兴望山谷
中寻路而走。看看天晚,但见一簇皂旗,蜂拥而来,一员羌将,手提铁锤大叫曰:“小将休
走!吾乃越吉元帅也!”关兴急走到前面,尽力纵马加鞭,正遇断涧,只得回马来战越吉。
兴终是胆寒,抵敌不住,望涧中而逃;被越吉赶到,一铁锤打来,兴急闪过,正中马胯。那
马望涧中便倒,兴落于水中。忽听得一声响处,背后越吉连人带马,平白地倒下水来。兴就
水中挣起看时,只见岸上一员大将,杀退羌兵。兴提刀待砍越吉,吉跃水而走。关兴得了越
吉马,牵到岸上,整顿鞍辔,绰刀上马。只见那员将,尚在前面追杀羌兵。兴自思此人救我
性命,当与相见,遂拍马赶来。看看至近,只见云雾之中,隐隐有一大将,面如重枣,眉若
卧蚕,绿袍金铠,提青龙刀,骑赤兔马,手绰美髯,分明认得是父亲关公。兴大惊。忽见关
公以手望东南指曰:“吾儿可速望此路去。吾当护汝归寨。”言讫不见。关兴望东南急走。
至半夜,忽一彪军到,乃张苞也,问兴曰:“你曾见二伯父否?”兴曰:“你何由知之?”
苞曰:“我被铁车军追急,忽见伯父自空而下,惊退羌兵,指曰:‘汝从这条路去救吾
儿。’因此引军径来寻你。’关兴亦说前事,共相嗟异。二人同归寨内。马岱接着,对二人
说:“此军无计可退。我守住寨栅,你二人去禀丞相,用计破之。”于是兴、苞二人,星夜
来见孔明,备说此事。孔明随命赵云、魏延各引一军埋伏去讫;然后点三万军,带了姜维、
张冀、关兴、张苞,亲自来到马岱寨中歇定。次日上高阜处观看,见铁车连络不绝,人马纵
横,往来驰骤。孔明曰:“此不难破也。”唤马岱、张冀分付如此如此。二人去了,乃唤姜
维曰:“伯约知破车之法否?”维曰:“羌人惟恃一勇力,岂知妙计乎?”孔明笑曰:“汝
知吾心也。今彤云密布,朔风紧急,天将降雪,吾计可施矣。”便令关兴、张苞二人引兵埋
伏去讫;令姜维领兵出战:但有铁车兵来,退后便走;寨口虚立旌旗,不设军马。准备已
定。

是时十二月终,果然天降大雪。姜维引军出,越吉引铁车兵来。姜维即退走。羌兵赶到
寨前,姜维从寨后而去。羌兵直到寨外观看,听得寨内鼓琴之声,四壁皆空竖旌旗,急回报
越吉。越吉心疑,未敢轻进。雅丹丞相曰:“此诸葛亮诡计,虚设疑兵耳。可以攻之。”越
吉引兵至寨前,但见孔明携琴上车,引数骑入寨,望后而走。羌兵抢入寨栅,直赶过山口,
见小车隐隐转入林中去了。雅丹谓越吉曰:“这等兵虽有埋伏,不足为惧。”遂引大兵追
赶。又见姜维兵俱在雪地之中奔走。越吉大怒,催兵急追。山路被雪漫盖,一望平坦。正赶
之间,忽报蜀兵自山后而出。雅丹曰:“纵有些小伏兵,何足惧哉!”只顾催趱兵马,往前
进发。忽然一声响,如山崩地陷,羌兵俱落于坑堑之中;背后铁车正行得紧溜,急难收止,
并拥而来,自相践踏。后兵急要回时,左边关兴、右边张苞,两军冲出,万弩齐发;背后姜
维、马岱、张冀三路兵又杀到。铁车兵大乱。越吉元帅望后面山谷中而逃,正逢关兴;交马
只一合,被兴举刀大喝一声,砍死于马下。雅丹丞相早被马岱活捉,解投大寨来。羌兵四散
逃窜。孔明升帐,马岱押过雅丹来。孔明叱武士去其缚,赐酒压惊,用好言抚慰。雅丹深感
其德。孔明曰:“吾主乃大汉皇帝,今命吾讨贼,尔如何反助逆?吾今放汝回去,说与汝
主:吾国与尔乃邻邦,永结盟好,勿听反贼之言。”遂将所获羌兵及车马器械,尽给还雅
丹,俱放回国。众皆拜谢而去。孔明引三军连夜投祁山大寨而来,命关兴、张苞引军先行;
一面差人赍表奏报捷音。

却说曹真连日望羌人消息,忽有伏路军来报说:“蜀兵拔寨收拾起程。”郭淮大喜曰:
“此因羌兵攻击,故尔退去。”遂分两路追赶。前面蜀兵乱走,魏兵随后追袭。先锋曹遵正
赶之间,忽然鼓声大震,一彪军闪出,为首大将乃魏延也,大叫曰:“反贼休走!”曹遵大
惊,拍马交锋;不三合,被魏延一刀斩于马下。副先锋朱赞引兵追赶,忽然一彪军闪出,为
首大将乃赵云也。朱赞措手不及,被云一枪刺死。曹真、郭淮见西路先锋有失,欲收兵回;
背后喊声大震,鼓角齐鸣:关兴、张苞两路兵杀出,围了曹真、郭淮,痛杀一阵。曹、郭二
人,引败兵冲路走脱。蜀兵全胜,直追到渭水,夺了魏寨。曹真折了两个先锋,哀伤不已;
只得写本申朝,乞拨援兵。

却说魏主曹睿设朝,近臣奏曰:“大都督曹真,数败于蜀,折了两个先锋,羌兵又折了
无数,其势甚急,今上表求救,请陛下裁处。”睿大惊,急问退军之策。华歆奏曰:“须是
陛下御驾亲征,大会诸侯,人皆用命,方可退也。不然,长安有失,关中危矣!”太傅钟繇
奏曰:“凡为将者,智过于人,则能制人。孙子云:知彼知己,百战百胜。臣量曹真虽久用
兵,非诸葛亮对手。臣以全家良贱,保举一人,可退蜀兵。未知圣意准否?”睿曰:“卿乃
大老元臣,有何贤士,可退蜀兵,早召来与朕分忧。”钟繇奏曰:“向者,诸葛亮欲兴师犯
境,但惧此人,故散流言,使陛下疑而去之,方敢长驱大进。今若复用之,则亮自退矣。”
睿问何人。繇曰:“骠骑大将军司马懿也。”睿叹曰:“此事朕亦悔之。今仲达现在何
地?”繇曰:“近闻仲达在宛城闲住。”睿即降诏,遣使持节,复司马懿官职,加为平西都
督,就起南阳诸路军马,前赴长安。睿御驾亲征,令司马懿克日到彼聚会。使命星夜望宛城
去了。

却说孔明自出师以来,累获全胜,心中甚喜;正在祁山寨中,会聚议事,忽报镇守永安
宫李严令子李丰来见。孔明只道东吴犯境,心甚惊疑,唤入帐中问之。丰曰:“特来报
喜。”孔明曰:“有何喜?”丰曰:“昔日孟达降魏,乃不得已也。彼时曹不爱其才,时以
骏马金珠赐之,曾同辇出入,封为散骑常侍,领新城太守,镇守上庸、金城等处,委以西南
之任。自不死后,曹睿即位,朝中多人嫉妒,孟达日夜不安,常谓诸将曰:‘我本蜀将,势
逼于此。’今累差心腹人,持书来见家父,教早晚代禀丞相:前者五路下川之时,曾有此
意;今在新城,听知丞相伐魏,欲起金城、新城、上庸三处军马,就彼举事,径取洛阳:丞
相取长安,两京大定矣。今某引来人并累次书信呈上。”孔明大喜,厚赏李丰等。

忽细作人报说:“魏主曹睿,一面驾幸长安;一面诏司马懿复职,加为平西都督,起本
处之兵,于长安聚会。”孔明大惊。参军马谡曰:“量曹睿何足道!若来长安,可就而擒
之。丞相何故惊讶?”孔明曰:“吾岂惧曹睿耶?所患者惟司马懿一人而已。今孟达欲举大

事,若遇司马懿,事必败矣。达非司马懿对手,必被所擒。孟达若死,中原不易得也。”马
谡曰:“何不急修书,令孟达提防?’孔明从之,即修书令来人星夜回报孟达。却说孟达在
新城,专望心腹人回报。一日,心腹人到来,将孔明回书呈上。孟达拆封视之。书略曰:
“近得书,足知公忠义之心,不忘故旧,吾甚喜慰。若成大事,则公汉朝中兴第一功臣也。
然极宜谨密,不可轻易托人。慎之!戒之!近闻曹睿复诏司马懿起宛、洛之兵,若闻公举
事,必先至矣。须万全提备,勿视为等闲也。”孟达览毕,笑曰:“人言孔明心多,今观此
事可知矣。”乃具回书,令心腹人来答孔明。孔明唤入帐中。其人呈上回书。孔明拆封视
之。书曰:“适承钧教,安敢少怠。窃谓司马懿之事,不必惧也:宛城离洛阳约八百里,至
新城一千二百里。若司马懿闻达举事,须表奏魏主。往复一月间事,达城池已固,诸将与三
军皆在深险之地。司马懿即来,达何惧哉?丞相宽怀,惟听捷报!”

孔明看毕,掷书于地而顿足曰:“孟达必死于司马懿之手矣!”马谡问曰:“丞相何谓
也?”孔明曰:“兵法云,攻其不备,出其不意。岂容料在一月之期?曹睿既委任司马懿,
逢寇即除,何待奏闻?若知孟达反,不须十日,兵必到矣,安能措手耶?”众将皆服。孔明
急令来人回报曰:“若未举事,切莫教同事者知之;知则必败。”其人拜辞,归新城去了。

却说司马懿在宛城闲住,闻知魏兵累败于蜀,乃仰天长叹。懿长子司马师,字子元;次
子司马昭,字子尚:二人素有大志,通晓兵书。当日侍立于侧,见懿长叹,乃问曰:“父亲
何为长叹?”懿曰:“汝辈岂知大事耶?”司马师曰:“莫非叹魏主不用乎?”司马昭笑
曰:“早晚必来宣召父亲也。”言未已,忽报天使持节至。懿听诏毕,遂调宛城诸路军马。
忽又报金城太守申仪家人,有机密事求见。懿唤入密室问之,其人细说孟达欲反之事。更有
孟达心腹人李辅并达外甥邓贤,随状出首。司马懿听毕,以手加额曰:“此乃皇上齐天之洪
福也!诸葛亮兵在祁山,杀得内外人皆胆落;今天子不得已而幸长安,若旦夕不用吾时,孟
达一举,两京休矣!此贼必通谋诸葛亮。吾先擒之,诸葛亮定然心寒,自退兵也。”长子司
马师曰:“父亲可急写表申奏天子。”懿曰:“若等圣旨,往复一月之间,事无及矣。”即
传令教人马起程,一日要行二日之路,如迟立斩;一面令参军梁畿赍檄星夜去新城,教孟达
等准备征进,使其不疑。梁畿先行,懿随后发兵。行了二日,山坡下转出一军,乃是右将军
徐晃。晃下马见懿,说:“天子驾到长安,亲拒蜀兵,今都督何往?”懿低言曰:“今孟达
造反,吾去擒之耳。”晃曰:“某愿为先锋。”懿大喜,合兵一处。徐晃为前部,懿在中
军,二子押后。又行了二日,前军哨马捉住孟达心腹人,搜出孔明回书,来见司马懿。懿
曰:“吾不杀汝,汝从头细说。”其人只得将孔明、孟达往复之事,一一告说。懿看了孔明
回书,大惊曰:“世间能者所见皆同。吾机先被孔明识破。幸得天子有福,获此消息:孟达
今无能为矣。”遂星夜催军前行。

却说孟达在新城,约下金城太守申仪、上庸太守申耽,克日举事。耽仪二人佯许之,每
日调练军马,只待魏兵到,便为内应;却报孟达言:军器粮草,俱未完备,不敢约期起事。
达信之不疑。忽报参军梁畿来到,孟达迎入城中。畿传司马懿将今日:“司马都督今奉天子
诏,起诸路军以退蜀兵。太守可集本部军马听候调遣。”达问曰:“都督何日起程?”畿
曰:“此时约离宛城,望长安去了。”达暗喜曰:“吾大事成矣!”遂设宴待了梁畿,送出
城外,即报申耽、申仪知道,明日举事,换上大汉旗号,发诸路军马,径取洛阳。忽报:
“城外尘土冲天,不知何处兵来。”孟达登城视之,只见一彪军,打着“右将军徐晃”旗
号,飞奔城下。达大惊,急扯起吊桥。徐晃坐下马收拾不住,直来到壕边,高叫曰:“反贼
孟达,早早受降!”达大怒,急开弓射之,正中徐晃头额,魏将救去。城上乱箭射下,魏兵
方退。孟达恰待开门追赶,四面旌旗蔽日,司马懿兵到。达仰天长叹曰:“果不出孔明所料
也!”于是闭门坚守。却说徐晃被孟达射中头额,众军救到寨中,取了箭头,令医调治;当
晚身死,时年五十九岁。司马懿令人扶柩还洛阳安葬。次日,孟达登城遍视,只见魏兵四面
围得铁桶相似。达行坐不安,惊疑未定,忽见两路兵自外杀来,旗上大书“申耽”、“申
仪”。孟达只道是救军到,忙引本部兵大开城门杀出。耽、仪大叫曰:“反贼休走!早早受
死!”达见事变,拨马望城中便走,城上乱箭射下。李辅、邓贤二人在城上大骂曰:“吾等
已献了城也!”达夺路而走,申耽赶来。达人困马乏,措手不及,被申耽一枪刺于马下,枭
其首级。余军皆降。李辅、邓贤大开城门,迎接司马懿入城。抚民劳军已毕,遂遣人奏知魏
主曹睿。睿大喜,教将孟达首级去洛阳城市示众;加申耽、申仪官职,就随司马懿征进;命
李辅、邓贤守新城、上庸。却说司马懿引兵到长安城外下寨。懿入城来见魏主。睿大喜曰:
“朕一时不明,误中反间之计,悔之无及。今达造反,非卿等制之,两京休矣!”懿奏曰:
“臣闻申仪密告反情,意欲表奏陛下,恐往复迟滞,故不待圣旨,星夜而去。若待奏闻,则
中诸葛亮之计也。”言罢,将孔明回孟达密书奉上。睿看毕,大喜曰:“卿之学识,过于
孙、吴矣!”赐金钺斧一对,后遇机密重事,不必奏闻,便宜行事。就令司马懿出关破蜀。
懿奏曰:“臣举一大将,可为先锋。”睿曰:“卿举何人?”懿曰:“右将军张郃,可当此
任。”睿笑曰:“朕正欲用之。”遂命张郃为前部先锋,随司马懿离长安来破蜀兵。正是:
既有谋臣能用智,又求猛将助施威。未知胜负如何,且看下文分解。