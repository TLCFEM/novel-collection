\chapter{小霸王怒斩于吉~碧眼儿坐领江东}

却说孙策自霸江东,兵精粮足。建安四年,袭取庐江,败刘勋,使虞翻驰檄豫章,豫章太守华歆投降。自此声势大振,乃遣张纮往许昌上表献捷。曹操知孙策强盛,叹曰:“狮儿难与争锋也!”遂以曹仁之女许配孙策幼弟孙匡,两家结婚。留张纮在许昌。孙策求为大司马,曹操不许。策恨之,常有袭许都之心。于是吴郡太守许贡,乃暗遣使赴许都上书于曹操。其略曰:“孙策骁勇,与项籍相似。朝廷宜外示荣宠,召在京师;不可使居外镇,以为后患。”使者赍书渡江,被防江将士所获,解赴孙策处。策观书大怒,斩其使,遣人假意请许贡议事。贡至,策出书示之,叱曰:“汝欲送我于死地耶!”命武士绞杀之。贡家属皆逃散。有家客三人,欲为许贡报仇,恨无其便。一日,孙策引军会猎于丹徒之西山,赶起一大鹿,策纵马上山逐之。正赶之间,只见树林之内有三个人持枪带弓面立。策勒马问曰:“汝等何人?”答曰:“乃韩当军士也。在此射鹿。”策方举辔欲行,一人拈枪望策左腿便刺。策大惊,急取佩剑从马上砍去,剑刃忽坠,止存剑靶在手。一人早拈弓搭箭射来,正中孙策面颊。策就拔面上箭,取弓回射放箭之人,应弦面倒。那二人举枪向孙策乱搠,大叫曰:“我等是许贡家客,特来为主人报仇!”策别无器械,只以弓拒之,且拒且走。二人死战不退。策身被数枪,马亦带伤。正危急之时,程普引数人至。孙策大叫:“杀贼!“程普引众齐上,将许贡家客砍为肉泥。看孙策时,血流满面,被伤至重,乃以刀割抱,裹其伤处,救回吴会养病。后人有诗赞许家三客曰:“孙郎智勇冠江湄,射猎山中受困危。许客三人能死义,杀身豫让未为奇。”却说孙策受伤而回,使人寻请华伦医治。不想华佗已往中原去了,止有徒弟在吴,命其治疗。其徒曰:“箭头有药,毒已入骨。须静养百日,方可无虞。若怒气冲激,其疮难治。”孙策为人最是性急,恨不得即日便愈。将息到二十余日,忽闻张纮有使者自许昌回,策唤问之。使者曰:“曹操甚惧主公;其帐下谋士,亦俱敬服;惟有郭嘉不服。”策曰:“郭嘉曾有何说?”使者不敢言。策怒,固问之。使者只得从实告曰:“郭嘉曾对曹操言主公不足惧也:轻而无备,性急少谋,乃匹夫之勇耳,他日必死于小人之手。”策闻言,大怒曰:“匹夫安敢料吾!吾誓取许昌!”遂不待疮愈,便欲商议出兵。张昭谏曰:“医者戒主公百日休动,今何因一时之忿,自轻万金之躯?”正话间,忽报袁绍遣使陈震至。策唤入问之。震具言袁绍欲结东吴为外应,共攻曹操。策大喜,即日会诸将于城楼上,设宴款待陈震。饮酒之间,忽见诸将互相耳语,纷纷下楼。策怪问何故,左右曰:“有于神仙者,今从楼下过,诸将欲往拜之耳。”策起身凭栏观之,见一道人,身披鹤氅,手携藜杖,立于当道,百姓俱焚香伏道而拜。策怒曰:“是何妖人?快与我擒来!”左右告曰:“此人姓于,名吉,寓居东方,往来吴会,普施符水,救人万病,无有不验。当世呼为神仙,未可轻渎。”策愈怒,喝令:“速速擒来!违者斩!”

左右不得已,只得下楼,拥于吉至楼上。策叱曰:“狂道怎敢煽惑人心!”于吉曰:“贫道乃琅琊宫道士,顺帝时曾入山采药,得神书于阳曲泉水上,号曰《太平青领道》,凡百余卷,皆治人疾病方术。贫道得之,惟务代天宣化,普救万人,未曾取人毫厘之物,安得煽惑人心?”策曰:“汝毫不取人,衣服饮食,从何而得?汝即黄巾张角之流,今若不诛,必为后患!”叱左右斩之。张昭谏曰:“于道人在江东数十年,并无过犯,不可杀害。”策曰:“此等妖人,君杀之,何异屠猪狗!”众官皆苦谏,陈震亦劝。策怒未息,命且囚于狱中。众官俱散。陈震自归馆驿安歇。孙策归府,早有内侍传说此事与策母吴太夫人知道。夫人唤孙策入后堂,谓曰:“吾闻汝将于神仙下于缧绁。此人多曾医人疾病,军民敬仰,不可加害。”策曰:“此乃妖人,能以妖术惑众,不可不除!”夫人再三劝解。策曰:“母亲勿听外人妄言,儿自有区处。乃出唤狱吏取于吉来问。原来狱吏皆敬信于吉,吉在狱中时,尽去其枷锁;及策唤取,方带枷锁而出。策访知大怒,痛责狱吏,仍将于吉械系下狱。张昭等数十人,连名作状,拜求孙策,乞保于神仙。策曰:“公等皆读书人,何不达理?昔交州刺史张津,听信邪教,鼓瑟焚香,常以红帕裹头,自称可助出军之威,后竟为敌军所杀。此等事甚无益,诸君自未悟耳。吾欲杀于吉,正思禁邪觉迷也。”

吕范曰:“某素知于道人能祈风祷雨。方今天旱,何不令其祈雨以赎罪?”策曰:“吾且看此妖人若何。”遂命于狱中取出于吉,开其枷锁,令登坛求雨。吉领命,即沐浴更衣,取绳自缚于烈日之中。百姓观者,填街塞巷。于吉谓众人曰:“吾求三尺甘霖,以救万民,然我终不免一死。”众人曰:“若有灵验,主公必然敬服。”于吉曰:“气数至此,恐不能逃。”少顷,孙策亲至坛中下令:“若午时无雨,即焚死于吉。”先令人堆积干柴伺候。将及午时,狂风骤起。风过处,四下阴云渐合。策曰:“时已近午,空有阴云,而无甘雨,正是妖人!”叱左右将于吉扛上柴堆,四下举火,焰随风起。忽见黑烟一道,冲上空中,一声响喨,雷电齐发,大雨如注。顷刻之间,街市成河,溪涧皆满,足有三尺甘雨。于吉仰卧于柴堆之上,大喝一声,云收雨住,复见太阳。于是众官及百姓,共将于吉扶下柴堆,解去绳索,再拜称谢。孙策见官民俱罗拜于水中,不顾衣服,乃勃然大怒,叱曰:“晴雨乃天地之定数,妖人偶乘其便,你等何得如此惑乱!”掣宝剑令左右速斩于吉。众官力谏,策怒曰:“尔等皆欲从于吉造反耶!”众官乃不敢复言。策叱武士将于吉一刀斩头落地。只见一道青气,投东北去了。策命将其尸号令于市,以正妖妄之罪。

是夜风雨交作,及晓,不见了于吉尸首。守尸军士报知孙策。策怒,欲杀守尸军士。忽见一人,从堂前徐步而来,视之,却是于吉。策大怒,正欲拔剑斫之,忽然昏倒于地。左右急救入卧内,半晌方苏。吴太夫人来视疾,谓策曰:“吾儿屈杀神仙,故招此祸。”策笑曰:“儿自幼随父出征,杀人如麻,何曾有为祸之理?今杀妖人,正绝大祸,安得反为我祸?”夫人曰:“因汝不信,以致如此;今可作好事以禳之。”策曰:“吾命在天,妖人决不能为祸。何必禳耶!”夫人料劝不信,乃自令左右暗修善事禳解。是夜二更,策卧于内宅,忽然阴风骤起,灯灭而复明。灯影之下,见于吉立于床前。策大喝曰:“吾平生誓诛妖妄,以靖天下!汝既为阴鬼,何敢近我!”取床头剑掷之,忽然不见。吴太夫人闻之,转生忧闷。策乃扶病强行,以宽母心。母谓策曰:“圣人云:‘鬼神之为德,其盛矣乎!’又云:‘祷尔于上下神袛。’鬼神之事,不可不信。汝屈杀于先生,岂无报应?吾已令人设醮于郡之玉清观内,汝可亲往拜祷,自然安妥。”

策不敢违母命,只得勉强乘轿至玉清观。道士接入,请策焚香,策焚香而不谢。忽香炉中烟起不散,结成一座华盖,上面端坐着于吉。策怒,唾骂之;走离殿宇,又见于吉立于殿门首,怒目视策。策顾左右曰:“汝等见妖鬼否?”左右皆云未见。策愈怒,拔佩剑望于吉掷去,一人中剑而倒。众视之,乃前日动手杀于吉之小卒,被剑斫入脑袋,七窍流血而死。策命扛出葬之。比及出观,又见于吉走入观门来。策曰:“此观亦藏妖之所也!”遂坐于观前,命武士五百人拆毁之。武士方上屋揭瓦,却见于吉立于屋上,飞瓦掷地。策大怒,传令逐出本观道士,放火烧毁殿宇。火起处,又见于吉立于火光之中。策怒归府,又见于吉立于府门前。策乃不入府,随点起三军,出城外下寨,传唤众将商议,欲起兵助袁绍夹攻曹操。众将俱曰:“主公玉体违和,未可轻动。且待平愈,出兵未迟。”是夜孙策宿于寨内,又见于吉披发而来。策于帐中叱喝不绝。次日,吴太夫人传命,召策回府。策乃归见其母。夫人见策形容憔悴,泣曰:“儿失形矣!”策即引镜自照,果见形容十分瘦损,不觉失惊,顾左右曰:“吾奈何憔悴至此耶!”言未已,忽见于吉立于镜中。策拍镜大叫一声,金疮迸裂,昏绝于地。夫人令扶入卧内。须臾苏醒,自叹曰:“吾不能复生矣!”

随召张昭等诸人,及弟孙权,至卧榻前,嘱付曰:“天下方乱,以吴越之众,三江之固,大可有为。子布等幸善相吾弟。”乃取印绶与孙权曰:“若举江东之众,决机于两阵之间,与天下争衡,卿不如我;举贤任能,使各尽力以保江东,我不如卿。卿宜念父兄创业之艰难,善自图之!”权大哭,拜受印绶。策告母曰:“儿天年已尽,不能奉慈母。今将印绶付弟,望母朝夕训之。父兄旧人,慎勿轻怠。”母哭曰:“恐汝弟年幼,不能任大事,当复如何?”策曰:“弟才胜儿十倍,足当大任。倘内事不决,可问张昭;外事不决,可问周瑜。恨周瑜不在此,不得面嘱之也!”又唤诸弟嘱曰:“吾死之后,汝等并辅仲谋。宗族中敢有生异心者,众共诛之;骨肉为逆,不得入祖坟安葬。”诸弟泣受命。又唤妻乔夫人谓曰:“吾与汝不幸中途相分,汝须孝养尊姑。早晚汝妹入见,可嘱其转致周郎,尽心辅佐吾弟,休负我平日相知之雅。”言讫,瞑目而逝。年止二十六岁。后人有诗赞曰:“独战东南地,人称小霸王。运筹如虎踞,决策似鹰扬。威镇三江靖,名闻四海香。临终遗大事,专意属周郎。”

孙策既死,孙权哭倒于床前。张昭曰:“此非将军哭时也。宜一面治丧事,一面理军国大事。”权乃收泪。张昭令孙静理会丧事,请孙权出堂,受众文武谒贺。孙权生得方颐大口,碧眼紫髯。昔汉使刘琬入吴,见孙家诸昆仲,因语人曰:“吾遍观孙氏兄弟,虽各才气秀达,然皆禄祚不终。惟仲谋形貌奇伟,骨格非常,乃大贵之表,又亨高寿,众皆不及也。”

且说当时孙权承孙策遗命,掌江东之事。经理未定,人报周瑜自巴丘提兵回吴。权曰:“公瑾已回,吾无忧矣。”原来周瑜守御巴丘。闻知孙策中箭被伤,因此回来问候;将至吴郡,闻策已亡,故星夜来奔丧。当下周瑜哭拜于孙策灵柩之前。吴太夫人出,以遗嘱之语告瑜,瑜拜伏于地曰:“敢不效犬马之力,继之以死!”少顷,孙权入。周瑜拜见毕,权曰:“愿公无忘先兄遗命。”瑜顿首曰:“愿以肝脑涂地,报知己之恩。”权曰:“今承父兄之业,将何策以守之?”瑜曰:“自古得人者昌,失人者亡。为今之计,须求高明远见之人为辅,然后江东可定也。”权曰:“先兄遗言:内事托子布,外事全赖公瑾。”瑜曰:“子布贤达之士,足当大任。瑜不才,恐负倚托之重,愿荐一人以辅将军。”权问何人。瑜曰:“姓鲁,名肃,字子敬,临淮东川人也。此人胸怀韬略,腹隐机谋。早年丧父,事母至孝。其家极富,尝散财以济贫乏。瑜为居巢长之时,将数百人过临淮,因乏粮,闻鲁肃家有两囷米,各三千斛,因往求助。肃即指一囷相赠,其慷慨如此。平生好击剑骑射,寓居曲阿。祖母亡,还葬东城。其友刘子扬欲约彼往巢湖投郑宝,肃尚踌躇未往。今主公可速召之。”权大喜,即命周瑜往聘。

瑜奉命亲往,见肃叙礼毕,具道孙权相慕之意。肃曰:“近刘子扬约某往巢湖,某将就之。”瑜曰:“昔马援对光武云:当今之世,非但君择臣,臣亦择君。今吾孙将军亲贤礼士,纳奇录异,世所罕有。足下不须他计,只同我往投东吴为是。”

肃从其言,遂同周瑜来见孙权。权甚敬之,与之谈论,终日不倦。一日,众官皆散,权留鲁肃共饮,至晚同榻抵足而卧。夜半,权问肃曰:“方今汉室倾危,四方纷扰;孤承父兄余业,思为桓、文之事,君将何以教我?”肃曰:“昔汉高祖欲尊事义帝而不获者,以项羽为害也。今之曹操可比项羽,将军何由得为桓、文乎?肃窃料汉室不可复兴,曹操不可卒除。为将军计,惟有鼎足江东以观天下之衅。今乘北方多务,剿除黄祖,进伐刘表,竟长江所极而据守之;然后建号帝王,以图天下:此高祖之业也。”权闻言大喜,披衣起谢。次日厚赠鲁肃,并将衣服帏帐等物赐肃之母。

肃又荐一人见孙权:此人博学多才,事母至孝;覆姓诸葛,名瑾,字子瑜,琅琊南阳人也。权拜之为上宾。瑾劝权勿通袁绍,且顺曹操,然后乘便图之。权依言,乃遣陈震回,以书绝袁绍。却说曹操闻孙策已死,欲起兵下江南。侍御史张纮谏曰:“乘人之丧而伐之,既非义举;若其不克,弃好成仇:不如因而善遇之。”操然其说,乃即奏封孙权为将军,兼领会稽太守;即令张纮为会稽都尉,赍印往江东。孙权大喜,又得张纮回吴,即命与张昭同理政事。张纮又荐一人于孙权:此人姓顾,名雍,字元叹,乃中郎蔡邕之徒;其为人少言语,不饮酒,严厉正大。权以为丞,行太守事。自是孙权威震江东,深得民心。且说陈震回见袁绍,具说:“孙策已亡,孙权继立。曹操封之为将军,结为外应矣。”袁绍大怒,遂起冀、青、幽、并等处人马七十余万,复来攻取许昌。正是:江南兵革方休息,冀北干戈又复兴。未知胜负若何,且听下文分解。