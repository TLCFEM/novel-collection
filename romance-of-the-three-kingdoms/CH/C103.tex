\chapter{上方谷司马受困~五丈原诸葛禳星}

却说司马懿被张翼、廖化一阵杀败,匹马单枪,望密林间而走。张翼收住后军,廖化当先追赶。看看赶上,懿着慌,绕树而转。化一刀砍去,正砍在树上;及拔出刀时,懿已走出林外。廖化随后赶出,却不知去向,但见树林之东,落下金盔一个。廖化取盔捎在马上,一直望东追赶。原来司马懿把金盔弃于林东,却反向西走去了。廖化追了一程,不见踪迹,奔出谷口,遇见姜维,同回寨见孔明。张嶷早驱木牛流马到寨,交割已毕,获粮万余石。廖化献上金盔,录为头功。魏延心中不悦,口出怨言。孔明只做不知。

且说司马懿逃回寨中,心甚恼闷。忽使命赍诏至,言东吴三路入寇,朝廷正议命将抵敌,令懿等坚守勿战。懿受命已毕,深沟高垒,坚守不出。

却说曹睿闻孙权分兵三路而来,亦起兵三路迎之:令刘劭引兵救江夏,田豫引兵救襄阳,睿自与满宠率大军救合淝。满宠先引一军至巢湖口,望见东岸战船无数,旌旗整肃。宠入军中奏魏主曰:“吴人必轻我远来,未曾提备;今夜可乘虚劫其水寨,必得全胜。”魏主曰:“汝言正合朕意。”即令骁将张球领五千兵,各带火具,从湖口攻之;满宠引兵五千,从东岸攻之。是夜二更时分,张球、满宠各引军悄悄望湖口进发;将近水寨,一齐呐喊杀入。吴兵慌乱,不战而走;被魏军四下举火,烧毁战船、粮草、器具不计其数。诸葛瑾率败兵逃走沔口。魏兵大胜而回。

次日,哨军报知陆逊。逊集诸将议曰:“吾当作表申奏主上,请撤新城之围,以兵断魏军归路,吾率众攻其前:彼首尾不敌,一鼓可破也。”众服其言。陆逊即具表,遣一小校密地赍往新城。小校领命,赍着表文,行至渡口,不期被魏军伏路的捉住,解赴军中见魏主曹睿。睿搜出陆逊表文,览毕,叹曰:“东吴陆逊真妙算也!”遂命将吴卒监下,令刘劭谨防孙权后兵。却说诸葛瑾大败一阵,又值暑天,人马多生疾病;乃修书一封,令人转达陆逊,议欲撤兵还国。逊看书毕,谓来人曰:“拜上将军:吾自有主意。”使者回报诸葛瑾。瑾问:“陆将军作何举动?”使者曰:“但见陆将军催督众人于营外种豆菽,自与诸将在辕门射戏。”瑾大惊,亲自往陆逊营中,与逊相见,问曰:“今曹睿亲来,兵势甚盛,都督何以御之?”逊曰:“吾前遣人奉表于主上,不料为敌人所获。机谋既泄,彼必知备;与战无益,不如且退。已差人奉表约主上缓缓退兵矣。”瑾曰:“都督既有此意,即宜速退,何又迟延?”逊曰:“吾军欲退,当徐徐而动。今若便退,魏人必乘势追赶:此取败之道也。足下宜先督船只诈为拒敌之意,吾悉以人马向襄阳而进,为疑敌之计,然后徐徐退归江东,魏兵自不敢近耳。”瑾依其计,辞逊归本营,整顿船只,预备起行。陆逊整肃部伍,张扬声势,望襄阳进发。

早有细作报知魏主,说吴兵已动,须用提防。魏将闻之,皆要出战。魏主素知陆逊之才,谕众将曰:“陆逊有谋,莫非用诱敌之计?不可轻进。”众将乃止。数日后,哨卒报来:“东吴三路兵马皆退矣。”魏主未信,再令人探之,回报果然尽退。魏主曰:“陆逊用兵,不亚孙、吴。东南未可平也。”因敕诸将,各守险要,自引大军屯合淝,以伺其变。

却说孔明在祁山,欲为久驻之计,乃令蜀兵与魏民相杂种田:军一分,民二分,并不侵犯,魏民皆安心乐业。司马师入告其父曰:“蜀兵劫去我许多粮米,今又令蜀兵与我民相杂屯田于渭滨,以为久计:似此真为国家大患。父亲何不与孔明约期大战一场,以决雌雄?”懿曰:“吾奉旨坚守,不可轻动。”正议间,忽报魏延将着元帅前日所失金盔,前来骂战。众将忿怒,俱欲出战。懿笑曰:“圣人云:小不忍则乱大谋。但坚守为上。”诸将依令不出。魏延辱骂良久方回。孔明见司马懿不肯出战,乃密令马岱造成木栅,营中掘下深堑,多积干柴引火之物;周围山上,多用柴草虚搭窝铺,内外皆伏地雷。置备停当,孔明附耳嘱之曰:“可将葫芦谷后路塞断,暗伏兵于谷中。若司马懿追到,任他入谷,便将地雷干柴一齐放起火来。”又令军士昼举七星号带于谷口,夜设七盏明灯于山上,以为暗号。马岱受计引兵而去。孔明又唤魏延分付曰:“汝可引五百兵去魏寨讨战,务要诱司马懿出战。不可取胜,只可诈败。懿必追赶,汝却望七星旗处而入;若是夜间,则望七盏灯处而走。只要引得司马懿入葫芦谷内,吾自有擒之之计。”魏延受计,引兵而去。孔明又唤高翔分付曰:“汝将木牛流马或二三十为一群,或四五十为一群,各装米粮,于山路往来行走。如魏兵抢去,便是汝之功。”高翔领计,驱驾木牛流马去了。孔明将祁山兵一一调去,只推屯田;分付:“如别兵来战,只许诈败;若司马懿自来,方并力只攻渭南,断其归路。”孔明分拨已毕,自引一军近上方谷下营。

且说夏侯惠、夏侯和二人入寨告司马懿曰:“今蜀兵四散结营,各处屯田,以为久计;若不趁此时除之,纵令安居日久,深根固蒂,难以摇动。”懿曰:“此必又是孔明之计。”二人曰:“都督若如此疑虑,寇敌何时得灭?我兄弟二人,当奋力决一死战,以报国恩。”懿曰:“既如此,汝二人可分头出战。”遂令夏侯惠、夏侯和各引五千兵去讫。懿坐待回音。

却说夏侯惠、夏侯和二人分兵两路,正行之间,忽见蜀兵驱木牛流马而来。二人一齐杀将过去,蜀兵大败奔走,木牛流马尽被魏兵抢获,解送司马懿营中。次日又劫掳得人马百余。亦解赴大寨。懿将解到蜀兵,诘审虚实。蜀兵告曰:“孔明只料都督坚守不出,尽命我等四散屯田,以为久计。不想却被擒获。”懿即将蜀兵尽皆放回。夏侯和曰:“何不杀之?”懿曰:“量此小卒,杀之无益。放归本寨,令说魏将宽厚仁慈,释彼战心:此吕蒙取荆州之计也。“遂传令今后凡有擒到蜀兵,俱当善遣之。仍重赏有功将吏。诸将皆听令而去。

却说孔明令高翔佯作运粮,驱驾木牛流马,往来于上方谷内;夏侯惠等,不时截杀,半月之间,连胜数阵。司马懿见蜀兵屡败,心中欢喜。一日,又擒到蜀兵数十人。懿唤至帐下问曰:“孔明今在何处?”众告曰:“诸葛丞相不在祁山,在上方谷西十里下营安住。今每日运粮屯于上方谷。”懿备细问了,即将众人放去;乃唤诸将分付曰:“孔明今不在祁山,在上方谷安营。汝等于明日,可一齐并力攻取祁山大寨。吾自引兵来接应。”众将领命,各各准备出战。司马师曰:“父亲何故反欲攻其后?”懿曰:“祁山乃蜀人之根本,若见我兵攻之,各营必尽来救;我却取上方谷烧其粮草,使彼首尾不接:必大败也。”司马师拜服。懿即发兵起行,令张虎、乐綝各引五千兵,在后救应。且说孔明正在山上,望见魏兵或三五千一行,或一二千一行,队伍纷纷,前后顾盼,料必来取祁山大寨,乃密传令众将:“若司马懿自来,汝等便往劫魏寨,夺了渭南。”众将各各听令。却说魏兵皆奔祁山寨来,蜀兵四下一齐呐喊奔走,虚作救应之势。司马懿见蜀兵都去救祁山寨,便引二子并中军护卫人马,杀奔上方谷来。魏延在谷口,只盼司马懿到来;忽见一枝魏兵杀到,延纵马向前视之,正是司马懿。延大喝曰:“司马懿休走!”舞刀相迎。懿挺枪接战。不上三合,延拨回马便走,懿随后赶来。延只望七星旗处而走。懿见魏延只一人,军马又少,放心追之;令司马师在左,司马昭在右,懿自居中,一齐攻杀将来。魏延引五百兵皆退入谷中去。懿追到谷口,先令人入谷中哨探。回报谷内并无伏兵,山上皆是草房。懿曰:“此必是积粮之所也。”遂大驱士马,尽入谷中。懿忽见草房上尽是干柴,前面魏延已不见了。懿心疑,谓二子曰:“倘有兵截断谷口,如之奈何?”言未已,只听得喊声大震,山上一齐丢下火把来,烧断谷口。魏兵奔逃无路。山上火箭射下,地雷一齐突出,草房内干柴都着,刮刮杂杂,火势冲天。司马懿惊得手足无措,乃下马抱二子大哭曰:“我父子三人皆死于此处矣!”正哭之间,忽然狂风大作,黑气漫空,一声霹雳响处,骤雨倾盆。满谷之火,尽皆浇灭:地雷不震,火器无功。司马懿大喜曰:“不就此时杀出,更待何时!”即引兵奋力冲杀。张虎、乐綝亦各引兵杀来接应。马岱军少,不敢追赶。司马懿父子与张虎、乐綝合兵一处,同归渭南大寨,不想寨栅已被蜀兵夺了。郭淮、孙礼正在浮桥上与蜀兵接战。司马懿等引兵杀到,蜀兵退去。懿烧断浮桥,据住北岸。

且说魏兵在祁山攻打蜀寨,听知司马懿大败,失了渭南营寨,军心慌乱;急退时,四面蜀兵冲杀将来,魏兵大败,十伤八九,死者无数,余众奔过渭北逃生。孔明在山上见魏延诱司马懿入谷,一霎时火光大起,心中甚喜,以为司马懿此番必死。不期天降大雨,火不能着,哨马报说司马懿父子俱逃去了。孔明叹曰:“谋事在人,成事在天。不可强也!”后人有诗叹曰:“谷口风狂烈焰飘,何期骤雨降青霄。武侯妙计如能就,安得山河属晋朝!”

却说司马懿在渭北寨内传令曰:“渭南寨栅,今已失了。诸将如再言出战者斩。”众将听令,据守不出。郭淮入告曰:“近日孔明引兵巡哨,必将择地安营。”懿曰:“孔明若出武功,依山而东,我等皆危矣;若出渭南,西止五丈原,方无事也。”令人探之,回报果屯五丈原。司马懿以手加额曰:“大魏皇帝之洪福也!”遂令诸将:“坚守勿出,彼久必自变。”

且说孔明自引一军屯于五丈原,累令人搦战,魏兵只不出。孔明乃取巾帼并妇人缟素之服,盛于大盒之内,修书一封,遣人送至魏寨。诸将不敢隐蔽,引来使入见司马懿。懿对众启盒视之,内有巾帼妇人之衣,并书一封。懿拆视其书,略曰:“仲达既为大将,统领中原之众,不思披坚执锐,以决雌雄,乃甘窟守土巢,谨避刀箭,与妇人又何异哉!今遣人送巾帼素衣至,如不出战,可再拜而受之。倘耻心未泯,犹有男子胸襟,早与批回,依期赴敌。”司马懿看毕,心中大怒,乃佯笑曰:“孔明视我为妇人耶!”即受之,令重待来使。懿问曰:“孔明寝食及事之烦简若何?”使者曰:“丞相夙兴夜寐,罚二十以上皆亲览焉。所啖之食,日不过数升。”懿顾谓诸将曰:“孔明食少事烦,其能久乎?”

使者辞去,回到五丈原,见了孔明,具说:“司马懿受了巾帼女衣,看了书札,并不嗔怒,只问丞相寝食及事之烦简,绝不提起军旅之事。某如此应对,彼言:食少事烦,岂能长久?”孔明叹曰:“彼深知我也!”主簿杨顒谏曰:“某见丞相常自校簿书,窃以为不必。夫为治有体,上下不可相侵。譬之治家之道,必使仆执耕,婢典爨,私业无旷,所求皆足,其家主从容自在,高枕饮食而已。若皆身亲其事,将形疲神困,终无一成。岂其智之不如婢仆哉?失为家主之道也。是故古人称:坐而论道,谓之三公;作而行之,谓之士大夫。昔丙吉忧牛喘,而不问横道死人;陈平不知钱谷之数,曰:自有主者。今丞相亲理细事,汗流终日岂不劳乎?司马懿之言,真至言也。”孔明泣曰:“吾非不知。但受先帝托孤之重,惟恐他人不似我尽心也!”众皆垂泪。自此孔明自觉神思不宁。诸将因此未敢进兵。却说魏将皆知孔明以巾帼女衣辱司马懿,懿受之不战。众将不忿,入帐告曰:“我等皆大国名将,安忍受蜀人如此之辱!即请出战,以决雌雄。”懿曰:“吾非不敢出战而甘心受辱也。奈天子明诏,令坚守勿动。今若轻出,有违君命矣。”众将俱忿怒不平。懿曰:“汝等既要出战,待我奏准天子,同力赴敌,何如?”众皆允诺。懿乃写表遣使,直至合淝军前,奏闻魏主曹睿。睿拆表览之。表略曰:“臣才薄任重,伏蒙明旨,令臣坚守不战,以待蜀人之自敝;奈今诸葛亮遗臣以巾帼,待臣如妇人,耻辱至甚!臣谨先达圣聪:旦夕将效死一战,以报朝廷之恩,以雪三军之耻。臣不胜激切之至!”睿览讫,乃谓多官曰:“司马懿坚守不出,今何故又上表求战?”卫尉辛毗曰:“司马懿本无战心,必因诸葛亮耻辱,众将忿怒之故,特上此表,欲更乞明旨,以遏诸将之心耳。”睿然其言,即令辛毗持节至渭北寨传谕,令勿出战。司马懿接诏入帐,辛毗宣谕曰:“如再有敢言出战者,即以违旨论。”众将只得奉诏。懿暗谓辛毗曰:“公真知我心也!”于是令军中传说:魏主命辛毗持节,传谕司马懿勿得出战。蜀将闻知此事,报与孔明。孔明笑曰:“此乃司马懿安三军之法也。”姜维曰:“丞相何以知之?”孔明曰:“彼本无战心;所以请战者,以示武于众耳。岂不闻:将在外,君命有所不受。安有千里而请战者乎?此乃司马懿因将士忿怒,故借曹睿之意,以制众人。今又播传此言,欲懈我军心也。”

正论间,忽报费祎到。孔明请入问之,祎曰:“魏主曹睿闻东吴三路进兵,乃自引大军至合淝,令满宠、田豫、刘劭分兵三路迎敌。满宠设计尽烧东吴粮草战具,吴兵多病。陆逊上表于吴王,约会前后夹攻,不意赍表人中途被魏兵所获,因此机关泄漏,吴兵无功而退。”孔明听知此信,长叹一声,不觉昏倒于地;众将急救,半晌方苏。孔明叹曰:“吾心昏乱,旧病复发,恐不能生矣!”

是夜,孔明扶病出帐,仰观天文,十分惊慌;入帐谓姜维曰:“吾命在旦夕矣!”维曰:“丞相何出此言?”孔明曰:“吾见三台星中,客星倍明,主星幽隐,相辅列曜,其光昏暗:天象如此,吾命可知!”维曰:“天象虽则如此,丞相何不用祈禳之法挽回之?”孔明曰:“吾素谙祈禳之法,但未知天意若何。汝可引甲士四十九人,各执皂旗,穿皂衣,环绕帐外;我自于帐中祈禳北斗。若七日内主灯不灭,吾寿可增一纪;如灯灭,吾必死矣。闲杂人等,休教放入。凡一应需用之物,只令二小童搬运。”姜维领命,自去准备。

时值八月中秋,是夜银河耿耿,玉露零零,旌旗不动,刁斗无声。姜维在帐外引四十九人守护。孔明自于帐中设香花祭物,地上分布七盏大灯,外布四十九盏小灯,内安本命灯一盏。孔明拜祝曰:“亮生于乱世,甘老林泉;承昭烈皇帝三顾之恩,托孤之重,不敢不竭犬马之劳,誓讨国贼。不意将星欲坠,阳寿将终。谨书尺素,上告穹苍:伏望天慈,俯垂鉴听,曲延臣算,使得上报君恩,下救民命,克复旧物,永延汉祀。非敢妄祈,实由情切。”拜祝毕,就帐中俯伏待旦。次日,扶病理事,吐血不止。日则计议军机,夜则步罡踏斗。

却说司马懿在营中坚守,忽一夜仰观天文,大喜,谓夏侯霸曰:“吾见将星失位,孔明必然有病,不久便死。你可引一千军去五丈原哨探。若蜀人攘乱,不出接战,孔明必然患病矣。吾当乘势击之。”霸引兵而去。孔明在帐中祈禳已及六夜,见主灯明亮,心中甚喜。姜维入帐,正见孔明披发仗剑,踏罡步斗,压镇将星。忽听得寨外呐喊,方欲令人出问,魏延飞步入告曰:“魏兵至矣!”延脚步急,竟将主灯扑灭。孔明弃剑而叹曰!“死生有命,不可得而禳也!”魏延惶恐,伏地请罪;姜维忿怒,拔剑欲杀魏延。正是:万事不由人做主,一心难与命争衡。未知魏延性命如何,且看下文分解。