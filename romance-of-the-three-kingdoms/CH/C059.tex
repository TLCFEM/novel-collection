\chapter{许褚裸衣斗马超~曹操抹书间韩遂}

却说当夜两兵混战,直到天明,各自收兵。马超屯兵渭口,日夜分兵,前后攻击。曹操
在渭河内将船筏锁链作浮桥三条,接连南岸。曹仁引军夹河立寨,将粮草车辆穿连,以为屏
障。马超闻之,教军士各挟草一束,带着火种,与韩遂引军并力杀到寨前,堆积草把,放起
烈火。操兵抵敌不住,弃寨而走。车乘、浮桥,尽被烧毁。西凉兵大胜,截住渭河。曹操立
不起营寨,心中忧惧。荀攸曰:“可取渭河沙土筑起土城,可以坚守。”操拨三万军担土筑
城。马超又差庞德、马岱各引五百马军,往来冲突;更兼沙土不实,筑起便倒,操无计可
施。时当九月尽,天气暴冷,彤云密布,连日不开。曹操在寨中纳闷。忽人报曰:“有一老
人来见丞相,欲陈说方略。”操请入。见其人鹤骨松姿,形貌苍古。问之,乃京兆人也,隐
居终南山,姓娄,名子伯,道号梦梅居士。操以客礼待之。子伯曰:“丞相欲跨渭安营久
矣,今何不乘时筑之?”操曰:“沙土之地,筑垒不成。隐士有何良策赐教?”子伯曰:
“丞相用兵如神,岂不知天时乎?连日阴云布合,朔风一起,必大冻矣。风起之后,驱兵士
运土泼水,比及天明,土城已就。”操大悟,厚赏子伯。子伯不受而去。

是夜北风大作。操尽驱兵士担土泼水;为无盛水之具,作缣囊盛水浇之,随筑随冻。比
及天明,沙水冻紧,土城已筑完。细作报知马超。超领兵观之,大惊,疑有神助。次日,集
大军呜鼓而进。操自乘马出营,止有许褚一人随后。操扬鞭大呼曰:“孟德单骑至此,请马
超出来答话。”超乘马挺枪而出。操曰:“汝欺我营寨不成,今一夜天已筑就,汝何不早
降!”马超大怒,意欲突前擒之,见操背后一人,睁圆怪眼,手提钢刀,勒马而立。超疑是
许褚,乃扬鞭问曰:“闻汝军中有虎侯,安在哉?”许褚提刀大叫曰:“吾即谯郡许褚
也!”目射神光,威风抖擞。超不敢动,乃勒马回。操亦引许褚回寨。两军观之,无不骇
然。操谓诸将曰:“贼亦知仲康乃虎侯也!”自此军中皆称褚为虎侯,许褚曰:“某来日必
擒马超。”操曰:“马超英勇,不可轻敌。”褚曰:“某誓与死战!”即使人下战书,说虎
侯单搦马超来日决战。超接书大怒曰:“何敢如此相欺耶!”即批次日誓杀虎痴。

次日,两军出营布成阵势。超分庞德为左翼,马岱为右翼,韩遂押中军。超挺枪纵马,
立于阵前,高叫:“虎痴快出!”曹操在门旗下回顾众将曰:“马超不减吕布之勇!”言未
绝,许褚拍马舞刀而出。马超挺枪接战。斗了一百余合,胜负不分。马匹困乏,各回军中,
换了马匹,又出阵前。又斗一百余合,不分胜负。许褚性起,飞回阵中,卸了盔甲,浑身筋
突,赤体提刀,翻身上马,来与马超决战。两军大骇。两个又斗到三十余合,褚奋威举刀便
砍马超。超闪过,一枪望褚心窝刺来。褚弃刀将枪挟住。两个在马上夺枪。许诸力大,一声
响,拗断枪杆,各拿半节在马上乱打。操恐褚有失,遂令夏侯渊、曹洪两将齐出夹攻。庞
德、马岱见操将齐出,麾两翼铁骑,横冲直撞,混杀将来。操兵大乱。许褚臂中两箭。诸将
慌退入寨。马超直杀到壕边,操兵折伤大半。操令坚闭休出。马超回至渭口,谓韩遂曰:
“吾见恶战者莫如许褚,真虎痴也!”

却说曹操料马超可以计破,乃密令徐晃、朱灵尽渡河西结营,前后夹攻。一日,操于城
上见马超引数百骑,直临寨前,往来如飞。操观良久,掷兜鍪于地曰:“马儿不死,吾无葬
地矣!”夏侯渊听了,心中气忿,厉声曰:“吾宁死于此地,誓灭马贼!”遂引本部千余
人,大开寨门,直赶去。操急止不住,恐其有失,慌自上马前来接应。马超见曹兵至,乃将
前军作后队,后队作先锋,一字儿摆开。夏侯渊到,马超接往厮杀。超于乱军中遥见曹操,
就撇了夏侯渊,直取曹操。操大惊,拨马而走。曹兵大乱。

正追之际,忽报操有一军,已在河西下了营寨,超大惊,无心追赶,急收军回寨,与韩
遂商议,言:“操兵乘虚已渡河西,吾军前后受敌,如之奈何?”部将李堪曰:“不如割地
请和,两家且各罢兵,捱过冬天,到春暖别作计议。”韩遂曰:“李堪之言最善,可从
之。”

超犹豫未决。杨秋、侯选皆劝求和,于是韩遂遣杨秋为使,直往操寨下书,言割地请和
之事。操曰:“汝且回寨,吾来日使人回报。”杨秋辞去。贾诩入见操曰:“丞相主意若
何?”操曰:“公所见若何?”诩曰:“兵不厌诈,可伪许之;然后用反间计,令韩、马相
疑,则一鼓可破也。”操抚掌大喜曰:“天下高见,多有相合。文和之谋,正吾心中之事
也。”于是遣人回书,言:“待我徐徐退兵,还汝河西之地。”一面教搭起浮桥,作退军之
意。马超得书,谓韩遂曰:“曹操虽然许和,奸雄难测。倘不准备,反受其制。超与叔父轮
流调兵,今日叔向操,超向徐晃;明日超向操,叔向徐晃:分头提备,以防其诈。”韩遂依
计而行。

早有人报知曹操。操顾贾诩曰:“吾事济矣!”问:“来日是谁合向我这边?”人报
曰:“韩遂。”次日,操引众将出营,左右围绕,操独显一骑于中央。韩遂部卒多有不识操
者,出阵观看。操高叫曰:“汝诸军欲观曹公耶?吾亦犹人也,非有四目两口,但多智谋
耳。”诸军皆有惧色。操使人过阵谓韩遂曰:“丞相谨请韩将军会话。”韩遂即出阵;见操
并无甲仗,亦弃衣甲,轻服匹马而出。二人马头相交,各按辔对语。操曰:“吾与将军之
父,同举孝廉,吾尝以叔事之。吾亦与公同登仕路,不觉有年矣。将军今年妙龄几何?”韩
遂答曰:“四十岁矣。”操曰:“往日在京师,皆青春年少,何期又中旬矣!安得天下清平
共乐耶!”只把旧事细说,并不提起军情。说罢大笑,相谈有一个时辰,方回马而别,各自
归寨。早有人将此事报知马超。超忙来问韩遂曰:“今日曹操阵前所言何事?”遂曰:“只
诉京师旧事耳。”超曰:“安得不言军务乎?”遂曰:“曹操不言,吾何独言之?”超心甚
疑,不言而退。

却说曹操回寨,谓贾诩曰:“公知吾阵前对语之意否?”诩曰:“此意虽妙,尚未足间
二人。某有一策,令韩、马自相仇杀。”操问其计。贾诩曰:“马超乃一勇之夫,不识机
密。丞相亲笔作一书,单与韩遂,中间朦胧字样,于要害处,自行涂抹改易,然后封送与韩
遂,故意使马超知之。超必索书来看。若看见上面要紧去处,尽皆改抹,只猜是韩遂恐超知
甚机密事,自行改抹,正合着单骑会语之疑;疑则必生乱。我更暗结韩遂部下诸将,使互相
离间,超可图矣。”操曰:“此计甚妙。”随写书一封,将紧要处尽皆改抹,然后实封,故
意多遣从人送过寨去,下了书自回。果然有人报知马超。超心愈疑,径来韩遂处索书看。韩
遂将书与超。超见上面有改抹字样,问遂曰:“书上如何都改抹糊涂?”遂曰:“原书如
此,不知何故。”超曰:“岂有以草稿送与人耶?必是叔父怕我知了详细,先改抹了。”遂
曰:“莫非曹操错将草稿误封来了。”超曰:“吾又不信。曹操是精细之人,岂有差错?吾
与叔父并力杀贼,奈何忽生异心?”遂曰:“汝若不信吾心,来日吾在阵前赚操说话,汝从
阵内突出,一枪刺杀便了。”超曰:“若如此,方见叔父真心。”两人约定。次日,韩遂引
侯选、李堪、梁兴、马玩、杨秋五将出阵。马超藏在门影里。韩遂使人到操寨前,高叫:
“韩将军请丞相攀话。”操乃令曹洪引数十骑径出阵前与韩遂相见。马离数步,洪马上欠身
言曰:“夜来丞相拜意将军之言,切莫有误。”言讫便回马。超听得大怒,挺枪骤马,便刺
韩遂。五将拦住,劝解回寨。遂曰:“贤侄休疑,我无歹心。”马超那里肯信,恨怨而去。
韩遂与五将商议曰:“这事如何解释?”杨秋曰:“马超倚仗武勇,常有欺凌主公之心,便
胜得曹操,怎肯相让?以某愚见,不如暗投曹公,他日不失封侯之位。”遂曰:“吾与马腾
结为兄弟,安忍背之?”杨秋曰:“事已至此,不得不然。”遂曰:“谁可以通消息?”杨
秋曰:“某愿往。”遂乃写密书,遣杨秋径来操寨,说投降之事。操大喜,许封韩遂为西凉
侯、杨秋为西凉太守。其余皆有官爵。约定放火为号,共谋马超。杨秋拜辞,回见韩遂,备
言其事:“约定今夜放火,里应外合。”遂大喜,就令军士于中军帐后堆积干柴,五将各悬
刀剑听候,韩遂商议,欲设宴赚请马超,就席图之,犹豫未去。不想马超早已探知备细,便
带亲随数人,仗剑先行,令庞德、马岱为后应。超潜步入韩遂帐中,只见五将与韩遂密语,
只听得杨秋口中说道:“事不宜迟,可速行之!”超大怒,挥剑直入,大喝曰:“群贼焉敢
谋害我!”众皆大惊。超一剑望韩遂面门剁去,遂慌以手迎之,左手早被砍落。五将挥刀齐
出。超纵步出帐外,五将围绕混杀。超独挥宝剑,力敌五将。剑光明处,鲜血溅飞:砍翻马
玩,剁倒梁兴,三将各自逃生。超复入帐中来杀韩遂时,已被左右救去。帐后一把火起,各
寨兵皆动。超连忙上马,庞德、马岱亦至,互相混战。超领军杀出时,操兵四至:前有许
褚,后有徐晃,左有夏侯渊,右有曹洪。西凉之兵,自相并杀。超不见了庞德、马岱,乃引
百余骑,截于渭桥之上。天色微明,只见李堪领一军从桥下过,超挺枪纵马逐之。李堪拖枪
而走。恰好于禁从马超背后赶来。禁开弓射马超。超听得背后弦响,急闪过,却射中前面李
堪,落马而死。超回马来杀于禁,禁拍马走了。超回桥上住扎。操兵前后大至,虎卫军当
先,乱箭夹射马超。超以枪拨之,矢皆纷纷落地。超令从骑往来突杀。争奈曹兵围裹坚厚,
不能冲出。超于桥上大喝一声,杀入河北,从骑皆被截断。超独在阵中冲突,却被暗弩射倒
坐下马,马超堕于地上,操军逼合。正在危急,忽西北角上一彪军杀来,乃庞德、马岱也。
二人救了马超,将军中战马与马超骑了,翻身杀条血路,望西北而走。曹操闻马超走脱,传
令诸将:“无分晓夜,务要赶到马儿。如得首级者,千金赏,万户侯;生获者封大将军。”
众将得令,各要争功,迤逦追袭。马超顾不得人马困乏,只顾奔走。从骑渐渐皆散。步兵走
不上者,多被擒去。止剩得三十余骑,与庞德、马岱望陇西临洮而去。

曹操亲自追至安定,知马超去远,方收兵回长安。众将毕集。韩遂已无左手,做了残疾
之人,操教就于长安歇马,授西凉侯之职。杨秋、侯选皆封列侯,令守渭口。下令班师回许
都。凉州参军杨阜,字义山,径来长安见操。操问之,杨阜曰:“马超有吕布之勇,深得羌
人之心。今丞相若不乘势剿绝,他日养成气力,陇上诸郡,非复国家之有也。望丞相且休回
兵。”操曰:“吾本欲留兵征之,奈中原多事,南方未定,不可久留。君当为孤保之。”阜
领诺,又保荐韦康为凉州刺史,同领兵屯冀城,以防马超。阜临行,请于操曰:“长安必留
重兵以为后援。”操曰:“吾已定下,汝但放心。”阜辞而去。

众将皆问曰:“初贼据潼关,渭北道缺,丞相不从河东击冯翊,而反守潼关,迁延日
久,而后北渡,立营固守,何也?”操曰:“初贼守潼关,若吾初到,便取河东,贼必以各
寨分守诸渡口,则河西不可渡矣。吾故盛兵皆聚于潼关前,使贼尽南守,而河西不准备,故
徐晃、朱灵得渡也。吾然后引兵北渡,连车树栅为甬道,筑冰城,欲贼知吾弱,以骄其心,
使不准备。吾乃巧用反间,畜士卒之力,一旦击破之。正所谓疾雷不及掩耳。兵之变化,固
非一道也。”众将又请问曰:“丞相每闻贼加兵添众,则有喜色,何也?”操曰:“关中边
远,若群贼各依险阻,征之非一二年不可平复;今皆来聚一处,其众虽多,人心不一,易于
离间,一举可灭:吾故喜也。”众将拜曰:“丞相神谋,众不及也;”操曰:“亦赖汝众文
武之力。”遂重赏诸军。留夏侯渊屯兵长安,所得降兵,分拨各部。夏侯渊保举冯翊高陵
人,姓张,名既,字德容,为京兆尹,与渊同守长安。操班师回都。献帝排銮驾出郭迎接。
诏操“赞拜不名,入朝不趋,剑履上殿”:如汉相萧何故事。自此威震中外。这消息播入汉
中,早惊动了汉宁太守张鲁。原来张鲁乃沛国丰人。其祖张陵在西川鹄鸣山中造作道书以惑
人,人皆敬之。陵死之后,其子张衡行之。百姓但有学道者,助米五斗。世号“米贼”。张
衡死,张鲁行之。鲁在汉中自号为“师君”;其来学道者皆号为“鬼卒”;为首者号为“祭
酒”;领众多者号为“治头大祭酒”。务以诚信为主,不许欺诈。如有病者,即设坛使病人
居于静室之中,自思已过,当面陈首,然后为之祈祷;主祈祷之事者,号为“奸令祭洒”。
祈祷之法,书病人姓名,说服罪之意,作文三通,名为“三官手书”:一通放于山顶以奏
天,一通埋于地以奏地,一通沉于水以申水官。如此之后,但病痊可,将米五斗为谢。又盖
义舍:舍内饭米、柴火、肉食齐备,许过往人量食多少,自取而食;多取者受天诛。境内有
犯法者,必恕三次;不改者,然后施刑。所在并无官长,尽属祭酒所管。如此雄据汉中之地
已三十年。国家以为地远不能征伐,就命鲁为镇南中郎将,领汉宁太守,通进贡而已。当年
闻操破西凉之众,威震天下,乃聚众商议曰:“西凉马腾遭戮,马超新败,曹操必将侵我汉
中。我欲自称汉宁王,督兵拒曹操,诸君以为何如?”阎圃曰:“汉川之民户出十万余众,
财富粮足,四面险固;今马超新败,西凉之民,从子午谷奔入汉中者,不下数万。愚意益州
刘璋昏弱,不如先取西川四十一州为本,然后称王未迟。”张鲁大喜,遂与弟张卫商议起
兵。早有细作报入川中。

却说益州刘璋,字季玉,即刘焉之子,汉鲁恭王之后。章帝元和中,徙封竟陵,支庶因
居于此。后焉官至益州牧,兴平元年患病疽而死,州大吏赵韪等,共保璋为益州牧。璋曾杀
张鲁母及弟,因此有仇。璋使庞羲为巴西太守,以拒张鲁。时笼羲探知张鲁欲兴兵取川,急
报知刘璋。璋平生懦弱,闻得此信,心中大忧,急聚众官商议。忽一人昂然而出曰:“主公
放心。某虽不才,凭三寸不烂之舌,使张鲁不敢正眼来觑西川。”正是:只因蜀地谋臣进,
致引荆州豪杰来。未知此人是谁,且看下文分解。