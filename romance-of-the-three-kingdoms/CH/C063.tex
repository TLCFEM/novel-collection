\chapter{诸葛亮痛哭庞统~张翼德义释严颜}

却说法正与那人相见,各抚掌而笑。庞统问之。正曰:“此公乃广汉人,姓彭,名羕,
字永言,蜀中豪杰也。因直言触忤刘璋,被璋髡钳为徒隶,因此短发。”统乃以宾礼待之,
问羕从何而来。羕曰:“吾特来救汝数万人性命,见刘将军方可说。”法正忙报玄德。玄德
亲自谒见,请问其故。羕曰:“将军有多少军马在前寨?”玄德实告:“有魏延、黄忠在
彼。”羕曰:“为将之道,岂可不知地理乎?前寨紧靠涪江,若决动江水,前后以兵塞之,
一人无可逃也。”玄德大悟。彭羕曰:“罡星在西方,太白临于此地,当有不吉之事,切宜
慎之。”玄德即拜彭羕为幕宾,使人密报魏延、黄忠,教朝暮用心巡警,以防决水。黄忠、
魏延商议:二人各轮一日,如遇敌军到来,互相通报。却说泠苞见当夜风雨大作,引了五千
军,径循江边而进,安排决江。只听得后面喊声乱起,泠苞知有准备,急急回军。前面魏延
引军赶来,川兵自相践踏。泠苞正奔走间,撞着魏延。交马不数合,被魏延活捉去了。比及
吴兰、雷铜来接应时,又被黄忠一军杀退。魏延解泠苞到涪关。玄德责之曰:“吾以仁义相
待,放汝回去,何敢背我!今次难饶!”将泠苞推出斩之,重赏魏延。玄德设宴管待彭羕,
忽报荆州诸葛亮军师特遣马良奉书至此。玄德召入问之。马良礼毕曰:“荆州平安,不劳主
公忧念。”遂呈上军师书信。玄德拆书观之,略曰:“亮夜算太乙数,今年岁次癸巳,罡星
在西方;又观乾象,太白临于雒城之分:主将帅身上多凶少吉。切宜谨慎。”玄德看了书,
便教马良先回。玄德曰:“吾将回荆州,去论此事。”庞统暗思:“孔明怕我取了西川,成
了功,故意将此书相阻耳。”乃对玄德曰:“统亦算太乙数,已知罡星在西,应主公合得西
川,别不主凶事。统亦占天文,见太白临于雒城,先斩蜀将泠苞,已应凶兆矣。主公不可疑
心,可急进兵。”

玄德见庞统再三催促,乃引军前进。黄忠同魏延接入寨去。庞统问法正曰:“前至雒
城,有多少路?”法正画地作图。玄德取张松所遗图本对之,并无差错。法正言:“山北有
条大路,正取雒城东门;山南有条小路,却取雒城西门:两条路皆可进兵。”庞统谓玄德
曰:“统令魏延为先锋,取南小路而进;主公令黄忠作先锋,从山北大路而进:并到雒城取
齐。”玄德曰:“吾自幼熟于弓马,多行小路。军师可从大路去取东门,吾取西门。”庞统
曰:“大路必有军邀拦,主公引兵当之。统取小路。”玄德曰:“军师不可。吾夜梦一神
人,手执铁棒击吾右臂,觉来犹自臂疼。此行莫非不佳。”庞统曰:“壮士临阵,不死带
伤,理之自然也。何故以梦寐之事疑心乎?”玄德曰:“吾所疑者,孔明之书也。军师还守
涪关,如何?”庞统大笑曰:“主公被孔明所惑矣:彼不欲令统独成大功,故作此言以疑主
公之心。心疑则致梦,何凶之有?统肝脑涂地,方称本心。主公再勿多言,来早准行。”

当日传下号令,军士五更造饭,平明上马。黄忠、魏延领军先行。玄德再与庞统约会,
忽坐下马眼生前失,把庞统掀将下来。玄德跳下马,自来笼住那马。玄德曰:“军师何故乘
此劣马?”庞统曰:“此马乘久,不曾如此。”玄德曰:“临阵眼生,误人性命。吾所骑白
马,性极驯熟,军师可骑,万无一失。劣马吾自乘之。”遂与庞统更换所骑之马。庞统谢
曰:“深感主公厚恩,虽万死亦不能报也。”遂各上马取路而进。玄德见庞统去了,心中甚
觉不快,怏怏而行。

却说雒城中吴懿、刘璝听知折了泠苞,遂与众商议。张任曰:“城东南山僻有一条小
路,最为要紧,某自引一军守之。诸公紧守雒城,勿得有失。”忽报汉兵分两路前来攻城。
张任急引三千军,先来抄小路埋伏。见魏延兵过,张任教尽放过去,休得惊动。后见庞统军
来,张任军士遥指军中大将:“骑白马者必是刘备。”张任大喜,传令教如此如此。

却说庞统迤逦前进,抬头见两山逼窄,树木丛杂;又值夏末秋初,枝叶茂盛。庞统心下
甚疑,勒住马问:“此处是何地?”数内有新降军士,指道:“此处地名落凤坡。”庞统惊
曰:“吾道号凤雏,此处名落凤坡,不利于吾。”令后军疾退。只听山坡前一声炮响,箭如
飞蝗,只望骑白马者射来。可怜庞统竟死于乱箭之下。时年止三十六岁。后人有诗叹曰:
“古岘相连紫翠堆,士元有宅傍山隈。儿童惯识呼鸠曲,闾巷曾闻展骥才。预计三分平刻
削,长驱万里独徘徊。谁知天狗流星坠,不使将军衣锦回。”先是东南有童谣云:“一凤并
一龙,相将到蜀中。才到半路里,凤死落坡东。风送雨,雨随风,隆汉兴时蜀道通,蜀道通
时只有龙。”

当日张任射死庞统,汉军拥塞,进退不得,死者大半。前军飞报魏延。魏延忙勒兵欲
回,奈山路逼窄,厮杀不得。又被张任截断归路,在高阜处用强弓硬弩射来。魏延心慌。有
新降蜀兵曰:“不如杀奔雒城下,取大路而进。”延从其言,当先开路,杀奔雒城来。尘埃
起处,前面一军杀至,乃雒城守将吴兰、雷铜也;后面张任引兵追来:前后夹攻,把魏延围
在垓心。魏延死战不能得脱。但见吴兰、雷铜后军自乱,二将急回马去救。魏延乘势赶去,
当先一将,舞刀拍马,大叫:“文长,吾特来救汝!”视之,乃老将黄忠也。两下夹攻,杀
败吴、雷二将,直冲至雒城之下。刘瓒引兵杀出,却得玄德在后当住接应。黄忠、魏延翻身
便回。玄德军马比及奔到寨中,张任军马又从小路里截出。刘璝、吴兰、雷铜当先赶来。玄
德守不住二寨,且战且走,奔回涪关。蜀兵得胜,迤逦追赶。玄德人困马乏,那里有心厮
杀,且只顾奔走。将近涪关,张任一军追赶至紧。幸得左边刘封,右边关平,二将领三万生
力军截出,杀退张任;还赶二十里,夺回战马极多。

玄德一行军马,再入涪关,问庞统消息。有落凤坡逃得性命的军士,报说军师连人带
马,被乱箭射死于坡前。玄德闻言,望西痛哭不已,遥为招魂设祭。诸将皆哭。黄忠曰:
“今番折了庞统军师,张任必然来攻打涪关,如之奈何?不若差人往荆州,请诸葛军师来商
议收川之计。”正说之间,人报张任引军直临城下搦战。黄忠、魏延皆要出战。玄德曰:
“锐气新挫,宜坚守以待军师来到。”黄忠、魏延领命,只谨守城池。玄德写一封书,教关
平分付:“你与我往荆州请军师去。”关平领了书,星夜往荆州来。玄德自守涪关,并不出
战。

却说孔明在荆州,时当七夕佳节,大会众官夜宴,共说收川之事。只见正西上一星,其
大如斗,从天坠下,流光四散。孔明失惊,掷杯于地,掩面哭曰:“哀哉!痛哉”众官慌问
其故。孔明曰:“吾前者算今年罡星在西方,不利于军师;天狗犯于吾军,太白临于雒城,
已拜书主公,教谨防之。谁想今夕西方星坠,庞士元命必休矣!”言罢,大哭曰:“今吾主
丧一臂矣!”众官皆惊,未信其言。孔明曰:“数日之内,必有消息。”是夕酒不尽欢而
散。

数日之后,孔明与云长等正坐间,人报关平到,众官皆惊。关平入,呈上玄德书信。孔
明视之,内言本年七月初七日,庞军师被张任在落凤坡前箭射身故。孔明大哭,众官无不垂
泪。孔明曰:“既主公在涪关进退两难之际,亮不得不去。”云长曰:“军师去,谁人保守
荆州?荆州乃重地,干系非轻。”孔明曰:“主公书中虽不明言其人,吾已知其意了。”乃
将玄德书与众官看曰:“主公书中,把荆州托在吾身上,教我自量才委用。虽然如此,今教
关平赍书前来,其意欲云长公当此重任。云长想桃园结义之情,可竭力保守此地,责任非
轻,公宜勉之。”云长更不推辞,慨然领诺。孔明设宴,交割印绶。云长双手来接。孔明擎
着印曰:“这干系都在将军身上。”云长曰:“大丈夫既领重任,除死方休。”孔明见云长
说个“死”字,心中不悦;欲待不与,其言已出。孔明曰:“倘曹操引兵来到,当如之
何?”云长曰:“以力拒之。”孔明又曰:“倘曹操、孙权,齐起兵来,如之奈何?”云长
曰:“分兵拒之。”孔明曰:“若如此,荆州危矣。吾有八个字,将军牢记,可保守荆
州。”云长问:“那八个字?”孔明曰:“北拒曹操,东和孙权。”云长曰:“军师之言,
当铭肺腑。”

孔明遂与了印绶,令文官马良、伊籍、向朗、糜竺,武将糜芳、廖化、关平、周仓,一
班儿辅佐云长,同守荆州。一面亲自统兵入川。先拨精兵一万,教张飞部领,取大路杀奔巴
州、雒城之西,先到者为头功。又拨一枝兵,教赵云为先锋,溯江而上,会于雒城。孔明随
后引简雍、蒋琬等起行。那蒋琬字公琰,零陵湘乡人也,乃荆襄名士,现为书记。

当日孔明引兵一万五千,与张飞同日起行。张飞临行时,孔明嘱付曰:“西川豪杰甚
多,不可轻敌。于路戒约三军,勿得掳掠百姓,以失民心。所到之处,并宜存恤,勿得恣逞
鞭挞士卒。望将军早会雒城,不可有误。”

张飞欣然领诺,上马而去。迤逦前行,所到之处,但降者秋毫无犯。径取汉川路,前至
巴郡。细作回报:“巴郡太守严颜,乃蜀中名将,年纪虽高,精力未衰,善开硬弓,使大
刀,有万夫不当之勇:据住城郭,不竖降旗。”张飞教离城十里下寨,差人入城去:“说与
老匹夫,早早来降,饶你满城百姓性命;若不归顺,即踏平城郭,老幼不留!”

却说严颜在巴郡,闻刘璋差法正请玄德入川,拊心而叹曰:“此所谓独坐穷山,引虎自
卫者也!”后闻玄德据住涪关,大怒,屡欲提兵往战,又恐这条路上有兵来。当日闻知张飞
兵到,便点起本部五六千人马,准备迎敌。或献计曰:“张飞在当阳长坂,一声喝退曹兵百
万之众。曹操亦闻风而避之,不可轻敌。今只宜深沟高垒,坚守不出。彼军无粮,不过一
月,自然退去。更兼张飞性如烈火,专要鞭挞士卒;如不与战,必怒;怒则必以暴厉之气待
其军士:军心一变,乘势击之,张飞可擒也。”严颜从其言,教军士尽数上城守护。忽见一
个军士,大叫:“开门!”严颜教放入问之。那军士告说是张将军差来的,把张飞言语依直
便说。严颜大怒,骂:“匹夫怎敢无礼!吾严将军岂降贼者乎!借你口说与张飞!”唤武士
把军人割下耳鼻,却放回寨。军人回见张飞,哭告严颜如此毁骂。张飞大怒,咬牙睁目,披
挂上马,引数百骑来巴郡城下搦战。城上众军百般痛骂。张飞性急,几番杀到吊桥,要过护
城河,又被乱箭射回。到晚全无一个人出,张飞忍一肚气还寨。次日早晨,又引军去搦战。
那严颜在城敌楼上,一箭射中张飞头盔。飞指而恨曰:“若拿住你这老匹夫,我亲自食你
肉!”到晚又空回。第三日,张飞引了军,沿城去骂。原来那座城子是个山城,周围都是乱
山,张飞自乘马登出,下视城中。见军士尽皆披挂,分列队伍,伏在城中,只是不出;又见
民夫来来往往,搬砖运石,相助守城。张飞教马军下马,步军皆坐,引他出敌,并无动静。
又骂了一日,依旧空回。张飞在寨中自思:“终日叫骂,彼只不出,如之奈何?”猛然思得
一计,教众军不要前去搦战,都结束了在寨中等候;却只教三五十个军士,直去城下叫骂。
引严颜军出来,便与厮杀。张飞磨拳擦掌,只等敌军来。小军连骂了三日,全然不出。张飞
眉头一纵,又生一计,传令教军士四散砍打柴草,寻觅路径,不来搦战。严颜在城中,连日
不见张飞动静,心中疑惑,着十数个小军,扮作张飞砍柴的军,潜地出城,杂在军内,入山
中探听。

当日诸军回寨。张飞坐在寨中,顿足大骂:“严颜老匹夫!枉气杀我!”只见帐前三四
个人说道:“将军不须心焦:这几日打探得一条小路,可以偷过巴郡。”张飞故意大叫曰:
“既有这个去处,何不早来说?”众应曰:“这几日却才哨探得出。”张飞曰:“事不宜
迟,只今二更造饭,趁三更明月,拔寨都起,人衔枚,马去铃,悄悄而行。我自前面开路,
汝等依次而行。”传了令便满寨告报。探细的军听得这个消息,尽回城中来,报与严颜。颜
大喜曰:“我算定这匹夫忍耐不得。你偷小路过去,须是粮草辎重在后;我截住后路,你如
何得过?好无谋匹夫,中我之计!”即时传令:教军士准备赴敌,今夜二更也造饭,三更出
城,伏于树木丛杂去处。只等张飞过咽喉小路去了,车仗来时,只听鼓响,一齐杀出。传了
号令,看看近夜,严颜全军尽皆饱食,披挂停当,悄悄出城,四散伏住,只听鼓响:严颜自
引十数裨将,下马伏于林中。约三更后,遥望见张飞亲自在前,横矛纵马,悄悄引军前进。
去不得三四里,背后车仗人马、陆续进发。严颜看得分晓,一齐擂鼓,四下伏兵尽起。正来
抢夺车仗、背后一声锣响,一彪军掩到,大喝:“老贼休走!我等的你恰好!”严颜猛回头
看时,为首一员大将,豹头环眼,燕颌虎须,使丈八矛,骑深乌马:乃是张飞。四下里锣声
大震,众军杀来。严颜见了张飞,举手无措,交马战不十合,张飞卖个破绽,严颜一刀砍
来,张飞闪过,撞将入去,扯住严颜勒甲绦,生擒过来,掷于地下;众军向前,用索绑缚住
了。原来先过去的是假张飞。料道严颜击鼓为号,张飞却教鸣金为号:金响诸军齐到。川兵
大半弃甲倒戈而降。

张飞杀到巴郡城下,后军已自入城。张飞叫休杀百姓,出榜安民。群刀手把严颜推至。
飞坐于厅上,严颜不肯下跪。飞怒目咬牙大叱曰:“大将到此,何为不降,而敢拒敌?”严
颜全无惧色,回叱飞曰:“汝等无义,侵我州郡!但有断头将军,无降将军!”飞大怒,喝
左右斩来。严颜喝曰:“贼匹夫!砍头便砍,何怒也?”张飞见严颜声音雄壮,面不改色,
乃回嗔作喜,下阶喝退左右,亲解其缚,取衣衣之,扶在正中高坐,低头便拜曰:“适来言
语冒渎,幸勿见责。吾素知老将军乃豪杰之士也。”严颜感其恩义,乃降。后人有诗赞严颜
曰:“白发居西蜀,清名震大邦。忠心如皎月,浩气卷长江。宁可断头死,安能屈膝降?巴
州年老将,天下更无双。”又有赞张飞诗曰:“生获严颜勇绝伦,惟凭义气服军民。至今庙
貌留巴蜀,社酒鸡豚日日春。”张飞请问入川之计。严颜曰:“败军之将,荷蒙厚恩,无可
以报,愿施犬马之劳,不须张弓只箭,径取成都。”正是:只因一将倾心后,致使连城唾手
降。未知其计如何,且看下文分解。