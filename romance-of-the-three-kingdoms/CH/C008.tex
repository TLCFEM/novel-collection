\chapter{王司徒巧使连环计~董太师大闹凤仪亭}

却说蒯良曰:“今孙坚已丧,其子皆幼。乘此虚弱之时,火速进军,江东一鼓可得。若还尸罢兵,容其养成气力,荆州之患也。”表曰:“吾有黄祖在彼营中,安忍弃之?”良曰:“舍一无谋黄祖而取江东,有何不可?”表曰:“吾与黄祖心腹之交,舍之不义。”遂送桓阶回营,相约以孙坚尸换黄祖。

孙策换回黄祖,迎接灵柩,罢战回江东,葬父于曲阿之原。丧事已毕,引军居江都,招贤纳士,屈己待人,四方豪杰,渐渐投之。不在话下。

却说董卓在长安,闻孙坚已死,乃曰:“吾除却一心腹之患也!”问:“其子年几岁矣?”或答曰十七岁,卓遂不以为意。自此愈加骄横,自号为“尚父”,出入僭天子仪仗;封弟董晃为左将军、鄠侯,侄董璜为侍中,总领禁军。董氏宗族,不问长幼,皆封列侯。离长安城二百五十里,别筑郿坞,役民夫二十五万人筑之:其城郭高下厚薄一如长安,内盖宫室,仓库屯积二十年粮食;选民间少年美女八百人实其中,金玉、彩帛、珍珠堆积不知其数;家属都住在内。卓往来长安,或半月一回,或一月一回,公卿皆候送于横门外;卓常设帐于路,与公卿聚饮。一日,卓出横门,百官皆送,卓留宴,适北地招安降卒数百人到。卓即命于座前,或断其手足,或凿其眼睛,或割其舌,或以大锅煮之。哀号之声震天,百官战慄失箸,卓饮食谈笑自若。又一日,卓于省台大会百官,列坐两行。酒至数巡,吕布径入,向卓耳边言不数句,卓笑曰:“原来如此。”命吕布于筵上揪司空张温下堂。百官失色。不多时,侍从将一红盘,托张温头入献。百官魂不附体。卓笑曰:“诸公勿惊。张温结连袁术,欲图害我,因使人寄书来,错下在吾儿奉先处。故斩之。公等无故,不必惊畏。”众官唯唯而散。

司徒王允归到府中,寻思今日席间之事,坐不安席。至夜深月明,策杖步入后园,立于荼蘼架侧,仰天垂泪。忽闻有人在牡丹亭畔,长吁短叹。允潜步窥之,乃府中歌伎貂蝉也。其女自幼选入府中,教以歌舞,年方二八,色伎俱佳,允以亲女待之。是夜允听良久,喝曰:“贱人将有私情耶?”貂蝉惊跪答曰:“贱妾安敢有私!”允曰:“汝无所私,何夜深于此长叹?”蝉曰:“容妾伸肺腑之言。”允曰:“汝勿隐匿,当实告我。”蝉曰:“妾蒙大人恩养,训习歌舞,优礼相待,妾虽粉身碎骨,莫报万一。近见大人两眉愁锁,必有国家大事,又不敢问。今晚又见行坐不安,因此长叹。不想为大人窥见。倘有用妾之处,万死不辞!”允以杖击地曰:“谁想汉天下却在汝手中耶!随我到画阁中来。”貂蝉跟允到阁中,允尽叱出妇妾,纳貂蝉于坐,叩头便拜。貂蝉惊伏于地曰:“大人何故如此?”允曰:“汝可怜汉天下生灵!”言讫,泪如泉涌。貂蝉曰:“适间贱妾曾言:但有使令,万死不辞。”允跪而言曰:“百姓有倒悬之危,君臣有累卵之急,非汝不能救也。贼臣董卓,将欲篡位;朝中文武,无计可施。董卓有一义儿,姓吕,名布,骁勇异常。我观二人皆好色之徒,今欲用连环计,先将汝许嫁吕布,后献与董卓;汝于中取便,谍间他父子反颜,令布杀卓,以绝大恶。重扶社稷,再立江山,皆汝之力也。不知汝意若何?”貂蝉曰:“妾许大人万死不辞,望即献妾与彼。妾自有道理。”允曰:“事若泄漏,我灭门矣。”貂蝉曰:“大人勿忧。妾若不报大义,死于万刃之下!”允拜谢。

次日,便将家藏明珠数颗,令良匠嵌造金冠一顶,使人密送吕布。布大喜,亲到王允宅致谢。允预备嘉肴美馔;候吕布至,允出门迎迓,接入后堂,延之上坐。布曰:“吕布乃相府一将,司徒是朝廷大臣,何故错敬?”允曰:“方今天下别无英雄,惟有将军耳。允非敬将军之职,敬将军之才也。”布大喜。允殷勤敬酒,口称董太师并布之德不绝。布大笑畅饮。允叱退左右,只留侍妾数人劝酒。酒至半酣,允曰:“唤孩儿来。”少顷,二青衣引貂蝉艳妆而出。布惊问何人。允曰:“小女貂蝉也。允蒙将军错爱,不异至亲,故令其与将军相见。”便命貂蝉与吕布把盏。貂蝉送酒与布。两下眉来眼去。允佯醉曰:“孩儿央及将军痛饮几杯。吾一家全靠着将军哩。”布请貂蝉坐,貂蝉假意欲入。允曰:“将军吾之至友,孩儿便坐何妨。”貂蝉便坐于允侧。吕布目不转睛的看。又饮数杯,允指蝉谓布曰:“吾欲将此女送与将军为妾,还肯纳否?”布出席谢曰:“若得如此,布当效犬马之报!”允曰:“早晚选一良辰,送至府中。”布欣喜无限,频以目视貂蝉。貂蝉亦以秋波送情。少顷席散,允曰:“本欲留将军止宿,恐太师见疑。”布再三拜谢而去。过了数日,允在朝堂,见了董卓,趁吕布不在侧,伏地拜请曰:“允欲屈太师车骑,到草舍赴宴,未审钧意若何?”卓曰:“司徒见招,即当趋赴。”允拜谢归家,水陆毕陈,于前厅正中设座,锦绣铺地,内外各设帏幔。次日晌午,董卓来到。允具朝服出迎,再拜起居。卓下车,左右持戟甲士百余,簇拥入堂,分列两傍。允于堂下再拜,卓命扶上,赐坐于侧。允曰:“太师盛德巍巍,伊、周不能及也。”卓大喜。进酒作乐,允极其致敬。天晚酒酣,允请卓入后堂。卓叱退甲士。允捧觞称贺曰:“允自幼颇习天文,夜观乾象,汉家气数已尽。太师功德振于天下,若舜之受尧,禹之继舜,正合天心人意。”卓曰:“安敢望此!”允曰:“自古有道伐无道,无德让有德,岂过分乎!”卓笑曰:“若果天命归我,司徒当为元勋。”允拜谢。堂中点上画烛,止留女使进酒供食。允曰:“教坊之乐,不足供奉;偶有家伎,敢使承应。”卓曰:“甚妙。”允教放下帘栊,笙簧缭绕,簇捧貂蝉舞于帘外。有词赞之曰:“原是昭阳宫里人,惊鸿宛转掌中身,只疑飞过洞庭春。按彻《梁州》莲步稳,好花风袅一枝新,画堂香暖不胜春。”又诗曰:“红牙催拍燕飞忙,一片行云到画堂。眉黛促成游子恨,脸容初断故人肠。榆钱不买千金笑,柳带何须百宝妆。舞罢隔帘偷目送,不知谁是楚襄王。”舞罢,卓命近前。貂蝉转入帘内,深深再拜。卓见貂蝉颜色美丽,便问:“此女何人?”允曰:“歌伎貂蝉也。”卓曰:“能唱否?”允命貂蝉执檀板低讴一曲。正是:“一点樱桃启绛唇,两行碎玉喷阳春。丁香舌吐衠钢剑,要斩奸邪乱国臣。”卓称赏不已。允命貂蝉把盏。卓擎杯问曰:“青春几何?”貂蝉曰:“贱妾年方二八。”卓笑曰:“真神仙中人也!”允起曰:“允欲将此女献上太师,未审肯容纳否?”卓曰:“如此见惠,何以报德?”允曰:“此女得侍太师,其福不浅。”卓再三称谢。允即命备毡车,先将貂蝉送到相府。卓亦起身告辞。允亲送董卓直到相府,然后辞回。

乘马而行,不到半路,只见两行红灯照道,吕布骑马执戟而来,正与王允撞见,便勒住马,一把揪住衣襟,厉声问曰:“司徒既以貂蝉许我,今又送与太师,何相戏耶?”允急止之曰:“此非说话处,且请到草舍去。”布同允到家,下马入后堂。叙礼毕,允曰:“将军何故怪老夫?”布曰:“有人报我,说你把毡车送貂蝉入相府,是何意故?”允曰:“将军原来不知!昨日太师在朝堂中,对老夫说:‘我有一事,明日要到你家。’允因此准备小宴等候。太师饮酒中间,说:‘我闻你有一女,名唤貂蝉,已许吾儿奉先。我恐你言未准,特来相求,并请一见。’老夫不敢有违,随引貂蝉出拜公公。太师曰:‘今日良辰,吾即当取此女回去,配与奉先。’将军试思:太师亲临,老夫焉敢推阻?”布曰:“司徒少罪。布一时错见,来日自当负荆。”允曰:“小女颇有妆奁,待过将军府下,便当送至。”布谢去。次日,吕布在府中打听,绝不闻音耗。径入堂中,寻问诸侍妾。侍妾对曰:“夜来太师与新人共寝,至今未起。”布大怒,潜入卓卧房后窥探。时貂蝉起于窗下梳头,忽见窗外池中照一人影,极长大,头戴束发冠;偷眼视之,正是吕布。貂蝉故蹙双眉,做忧愁不乐之态,复以香罗频拭眼泪。吕布窥视良久,乃出;少顷,又入。卓己坐于中堂,见布来,问曰:“外面无事乎?”布曰:“无事。”侍立卓侧。卓方食,布偷目窃望,见绣帘内一女子往来观觑,微露半面,以目送情。布知是貂蝉,神魂飘荡。卓见布如此光景,心中疑忌,曰:“奉先无事且退。”布怏怏而出。

董卓自纳貂蝉后,为色所迷,月余不出理事。卓偶染小疾,貂蝉衣不解带,曲意逢迎,卓心意喜。吕布入内问安,正值卓睡。貂蝉于床后探半身望布,以手指心,又以手指董卓,挥泪不止。布心如碎。卓朦胧双目,见布注视床后,目不转睛;回身一看,见貂蝉立于床后。卓大怒,叱布曰:“汝敢戏吾爱姬耶!”唤左右逐出,今后不许入堂。吕布怒恨而归,路遇李儒,告知其故。儒急入见卓曰:“太师欲取天下,何故以小过见责温侯?倘彼心变,大事去矣。”卓曰:“奈何?”儒曰:“来朝唤入,赐以金帛,好言慰之,自然无事。”卓依言。次日,使人唤布入堂,慰之曰:“吾前日病中,心神恍惚,误言伤汝,汝勿记心。”随赐金十斤,锦二十匹。布谢归,然身虽在卓左右,心实系念貂蝉。

卓疾既愈,入朝议事。布执戟相随,见卓与献帝共谈,便乘间提戟出内门,上马径投相府来;系马府前,提戟入后堂,寻见貂蝉。蝉曰:“汝可去后园中凤仪亭边等我。”布提戟径往,立于亭下曲栏之傍。良久,见貂蝉分花拂柳而来,果然如月宫仙子,——泣谓布曰:“我虽非王司徒亲女,然待之如已出。自见将军,许侍箕帚。妾已生平愿足。谁想太师起不良之心,将妾淫污,妾恨不即死;止因未与将军一诀,故且忍辱偷生。今幸得见,妾愿毕矣!此身已污,不得复事英雄;愿死于君前,以明妾志!”言讫,手攀曲栏,望荷花池便跳。吕布慌忙抱住,泣曰:“我知汝心久矣!只恨不能共语!”貂蝉手扯布曰:“妾今生不能与君为妻,愿相期于来世。”布曰:“我今生不能以汝为妻,非英雄也!”蝉曰:“妾度日如年,愿君怜而救之。”布曰:“我今愉空而来,恐老贼见疑,必当速去。”蝉牵其衣曰:“君如此惧怕老贼,妾身无见天日之期矣!”布立住曰:“容我徐图良策。”语罢,提戟欲去。貂蝉曰:“妾在深闺,闻将军之名,如雷灌耳,以为当世一人而已;谁想反受他人之制乎!”言讫,泪下如雨。布羞惭满面,重复倚戟,回身搂抱貂蝉,用好言安慰。两个偎偎倚倚,不忍相离。

却说董卓在殿上,回头不见吕布,心中怀疑,连忙辞了献帝,登车回府;见布马系于府前;问门吏,吏答曰:“温侯入后堂去了。”卓叱退左右,径入后堂中,寻觅不见;唤貂蝉,蝉亦不见。急问侍妾,侍妾曰:“貂蝉在后园看花。”卓寻入后园,正见吕布和貂蝉在凤仪亭下共语,画戟倚在一边。卓怒,大喝一声。布见卓至,大惊,回身便走。卓抢了画戟,挺着赶来。吕布走得快,卓肥胖赶不上,掷戟刺布。布打戟落地。卓拾戟再赶,布已走远。卓赶出园门,一人飞奔前来,与卓胸膛相撞,卓倒于地。正是:冲天怒气高千丈,仆地肥躯做一堆。未知此人是谁,且听下文分解。