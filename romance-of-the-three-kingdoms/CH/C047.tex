\chapter{阚泽密献诈降书~庞统巧授连环计}

却说阚泽字德润,会稽山阴人也;家贫好学,与人佣工,尝借人书来看,看过一遍,更不遗忘;口才辨给,少有胆气。孙权召为参谋,与黄盖最相善。盖知其能言有胆,故欲使献诈降书。泽欣然应诺曰:“大丈夫处世,不能立功建业,不几与草木同腐乎!公既捐躯报主,泽又何惜微生!”黄盖滚下床来,拜而谢之。泽曰:“事不可缓,即今便行。”盖曰:“书已修下了。”泽领了书,只就当夜扮作渔翁,驾小舟,望北岸而行。

是夜寒星满天。三更时候,早到曹军水寨。巡江军士拿住,连夜报知曹操。操曰:“莫

非是奸细么?”军士曰:“只一渔翁,自称是东吴参谋阚泽,有机密事来见。”操便教引将入来。军士引阚泽至,只见帐上灯烛辉煌,曹操凭几危坐,问曰:“汝既是东吴参谋,来此何干?”泽曰:“人言曹丞相求贤若渴,今观此问,甚不相合。黄公覆,汝又错寻思了也!”操曰:“吾与东吴旦夕交兵,汝私行到此,如何不问?”泽曰:“黄公覆乃东吴三世旧臣,今被周瑜于众将之前,无端毒打,不胜忿恨。因欲投降丞相,为报仇之计,特谋之于我。我与公覆,情同骨肉,径来为献密书。未知丞相肯容纳否?”操曰:“书在何处?”阚泽取书呈上。

操拆书,就灯下观看。书略曰:“盖受孙氏厚恩,本不当怀二心。然以今日事势论之:用江东六郡之卒,当中国百万之师,众寡不敌,海内所共见也。东吴将吏,无有智愚,皆知其不可。周瑜小子,偏怀浅戆,自负其能,辄欲以卵敌石;兼之擅作威福,无罪受刑,有功不赏。盖系旧臣,无端为所摧辱,心实恨之!伏闻丞相诚心待物,虚怀纳士,盖愿率众归降,以图建功雪耻。粮草军仗,随船献纳。泣血拜白,万勿见疑。”曹操于几案上翻覆将书看了十余次,忽然拍案张目大怒曰:“黄盖用苦肉计,令汝下诈降书,就中取事,却敢来戏侮我耶!”便教左右推出斩之。左右将阚泽簇下。泽面不改容,仰天大笑。操教牵回,叱曰:“吾已识破奸计,汝何故哂笑?”泽曰:“吾不笑你。吾笑黄公覆不识人耳。”操曰:“何不识人?”泽曰:“杀便杀,何必多问!”操曰:“吾自幼熟读兵书,深知奸伪之道。汝这条计,只好瞒别人,如何瞒得我!”泽曰:“你且说书中那件事是奸计?”操曰:“我说出你那破绽,教你死而无怨:你既是真心献书投降,如何不明约几时?你今有何理说?”阚泽听罢,大笑曰:“亏汝不惶恐,敢自夸熟读兵书!还不及早收兵回去!倘若交战,必被周瑜擒矣!无学之辈!可惜吾屈死汝手!”操曰:“何谓我无学?”泽曰:“汝不识机谋,不明道理,岂非无学?”操曰:“你且说我那几般不是处?”泽曰:“汝无待贤之礼,吾何必言!但有死而已。”操曰:“汝若说得有理,我自然敬服。”泽曰:“岂不闻背主作窃,不可定期?倘今约定日期,急切下不得手,这里反来接应,事必泄漏。但可觑便而行,岂可预期相订乎?汝不明此理,欲屈杀好人,真无学之辈也!”操闻言,改容下席而谢曰:“某见事不明,误犯尊威,幸勿挂怀。”泽曰:“吾与黄公覆,倾心投降,如婴儿之望父母,岂有诈乎!”操大喜曰:“若二人能建大功,他日受爵,必在诸人之上。”泽曰:“某等非为爵禄而来,实应天顺人耳。”操取酒待之。

少顷,有人入帐,于操耳边私语。操曰:“将书来看。”其人以密书呈上。操观之,颜色颇喜。阚泽暗思:“此必蔡中、蔡和来报黄盖受刑消息,操故喜我投降之事为真实也。”操曰:“烦先生再回江东,与黄公覆约定,先通消息过江,吾以兵接应。”泽曰:“某已离江东,不可复还。望丞相别遣机密人去。”操曰:“若他人去,事恐泄漏。”泽再三推辞;良久,乃曰:“若去则不敢久停,便当行矣。”操赐以金帛,泽不受。辞别出营,再驾扁舟,重回江东,来见黄盖,细说前事。盖曰:“非公能辩,则盖徒受苦矣。”泽曰;“吾今去甘宁寨中,探蔡中、蔡和消息。”盖曰:“甚善。”泽至宁寨,宁接入,泽曰:“将军昨为救黄公覆,被周公瑾所辱,吾甚不平。”宁笑而不答。正话间,蔡和、蔡中至。泽以目送甘宁,宁会意,乃曰:“周公瑾只自恃其能,全不以我等为念。我今被辱,羞见江左诸人!”说罢,咬牙切齿,拍案大叫。泽乃虚与宁耳边低语。宁低头不言,长叹数声。蔡和、蔡中见宁、泽皆有反意,以言挑之曰:“将军何故烦恼?先生有何不平?”泽曰:“吾等腹中之苦,汝岂知耶!”蔡和曰:“莫非欲背吴投曹耶?”阚泽失色,甘宁拔剑而起曰:“吾事已为窥破,不可不杀之以灭口!”蔡和、蔡中慌曰:“二公勿忧。吾亦当以心腹之事相告。”宁曰:“可速言之!”蔡和曰:“吾二人乃曹公使来诈降者。二公若有归顺之心,吾当引进。”宁曰:“汝言果真?”二人齐声曰;“安敢相欺!”宁佯喜曰;“若如此,是天赐其便也!”二蔡曰:“黄公覆与将军被辱之事,吾已报知丞相矣。”泽曰:“吾已为黄公覆献书丞相,今特来见兴霸,相约同降耳。”宁曰:“大丈夫既遇明主,自当倾心相投。”于是四人共饮,同论心事。二蔡即时写书,密报曹操,说“甘宁与某同为内应。”阚泽另自修书,遣人密报曹操,书中具言:黄盖欲来,未得其便;但看船头插青牙旗而来者,即是也。

却说曹操连得二书,心中疑惑不定,聚众谋士商议曰:“江左甘宁,被周瑜所辱,愿为内应;黄盖受责,令阚泽来纳降:俱未可深信。谁敢直入周瑜寨中,探听实信?”蒋干进曰:“某前日空往东吴,未得成功,深怀惭愧。今愿舍身再往,务得实信,回报丞相。”操大喜,即时令蒋干上船。干驾小舟,径到江南水寨边,便使人传报。周瑜听得干又到,大喜曰:“吾之成功,只在此人身上!”遂嘱付鲁肃:“请庞士元来,为我如此如此。”原来襄阳庞统,字士元,因避乱寓居江东,鲁肃曾荐之于周瑜。统未及往见,瑜先使肃问计于统曰:“破曹当用何策?”统密谓肃曰:“欲破曹兵,须用火攻;但大江面上,一船着火,余船四散;除非献连环计,教他钉作一处,然后功可成也。”肃以告瑜,瑜深服其论,因谓肃曰:“为我行此计者,非庞士元不可。”肃曰:“只怕曹操奸猾,如何去得?”周瑜沉吟未决。正寻思没个机会,忽报蒋干又来。瑜大喜,一面分付庞统用计;一面坐于帐上,使人请干。

干见不来接,心中疑虑,教把船于僻静岸口缆系,乃入寨见周瑜。瑜作色曰:“子翼何故欺吾太甚?”蒋干笑曰:“吾想与你乃旧日弟兄,特来吐心腹事,何言相欺也?”瑜曰:“汝要说我降,除非海枯石烂!前番吾念旧日交情,请你痛饮一醉,留你共榻;你却盗吾私书,不辞而去,归报曹操,杀了蔡瑁、张允,致使吾事不成。今日无故又来,必不怀好意!吾不看旧日之情,一刀两段!本待送你过去,争奈吾一二日间,便要破曹贼;待留你在军中,又必有泄漏。”便教左右:“送子翼往西山庵中歇息。待吾破了曹操,那时渡你过江未迟。”蒋干再欲开言,周瑜已入帐后去了。

左右取马与蒋干乘坐,送到西山背后小庵歇息,拨两个军人伏侍。干在庵内,心中忧闷,寝食不安。是夜星露满天,独步出庵后,只听得读书之声。信步寻去,见山岩畔有草屋数椽,内射灯光。干往窥之,只见一人挂剑灯前,诵孙、吴兵书。干思:“此必异人也。”叩户请见。其人开门出迎,仪表非俗。干问姓名,答曰:“姓庞,名统,字士元。”干曰:“莫非凤雏先生否?”统曰:“然也。”干喜曰:“久闻大名,今何僻居此地?”答曰:“周瑜自恃才高,不能容物,吾故隐居于此。公乃何人?”干曰:“吾蒋干也。”统乃邀入草庵,共坐谈心。干曰:“以公之才,何往不利?如肯归曹,干当引进。”统曰:“吾亦欲离江东久矣。公既有引进之心,即今便当一行。如迟则周瑜闻之,必将见害。”于是与干连夜下山,至江边寻着原来船只,飞棹投江北。

既至操寨,干先入见,备述前事。操闻凤雏先生来,亲自出帐迎入,分宾主坐定,问曰:“周瑜年幼,恃才欺众,不用良谋。操久闻先生大名,今得惠顾,乞不吝教诲。”统曰:“某素闻丞相用兵有法,今愿一睹军容。”操教备马,先邀统同观旱寨。统与操并马登高而望。统曰:“傍山依林,前后顾盼,出入有门,进退曲折,虽孙、吴再生,穰苴复出,亦不过此矣。”操曰:“先生勿得过誉,尚望指教。”于是又与同观水寨。见向南分二十四座门,皆有艨艟战舰,列为城郭,中藏小船,往来有巷,起伏有序,统笑曰:“丞相用兵如此,名不虚传!”因指江南而言曰:“周郎,周郎!克期必亡!”操大喜。回寨,请入帐中,置酒共饮,同说兵机。统高谈雄辩,应答如流。操深敬服,殷勤相待。统佯醉曰:“敢问军中有良医否?”操问何用。统曰:“水军多疾,须用良医治之。”时操军因不服水土,俱生呕吐之疾,多有死者,操正虑此事;忽闻统言,如何不问?统曰:“丞相教练水军之法甚妙,但可惜不全。”操再三请问。统曰:“某有一策,使大小水军,并无疾病,安稳成功。”操大喜,请问妙策。统曰:“大江之中,潮生潮落,风浪不息;北兵不惯乘舟,受此颠播,便生疾病。若以大船小船各皆配搭,或三十为一排,或五十为一排,首尾用铁环连锁,上铺阔板,休言人可渡,马亦可走矣,乘此而行,任他风浪潮水上下,复何惧哉?”曹操下席而谢曰:“非先生良谋,安能破东吴耶!”统曰:“愚浅之见,丞相自裁之。”操即时传令,唤军中铁匠,连夜打造连环大钉,锁住船只。诸军闻之,俱各喜悦。后人有诗曰:“赤壁鏖兵用火攻,运筹决策尽皆同。若非庞统连环计,公瑾安能立大功?”

庞统又谓操曰:“某观江左豪杰,多有怨周瑜者;某凭三寸舌,为丞相说之,使皆来降。周瑜孤立无援,必为丞相所擒。瑜既破,则刘备无所用矣。”操曰:“先生果能成大功,操请奏闻天子,封为三公之列。”统曰:“某非为富贵,但欲救万民耳。丞相渡江,慎勿杀害。”操曰:“吾替天行道,安忍杀戮人民!”统拜求榜文,以安宗族。操曰:“先生家属,现居何处?”统曰:“只在江边。若得此榜,可保全矣。”操命写榜佥押付统。统拜谢曰:“别后可速进兵,休待周郎知觉。”操然之。统拜别,至江边,正欲下船,忽见岸上一人,道袍竹冠,一把扯住统曰:“你好大胆!黄盖用苦肉计,阚泽下诈降书,你又来献连环计:只恐烧不尽绝!你们把出这等毒手来,只好瞒曹操,也须瞒我不得!”?得庞统魂飞魄散。正是:莫道东南能制胜,谁云西北独无人?毕竟此人是谁,且看下文分解。