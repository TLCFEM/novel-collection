\chapter{荆州城公子三求计~博望坡军师初用兵}

却说孙权督众攻打夏口,黄祖兵败将亡,情知守把不住,遂弃江夏,望荆州而走。甘宁
料得黄祖必走荆州,乃于东门外伏兵等候。祖带数十骑突出东门,正走之间,一声喊起,甘
宁拦住。祖于马上谓宁曰:“我向日不曾轻待汝,今何相逼耶?”宁叱曰:“吾昔在江夏,
多立功绩,汝乃以劫江贼待我,今日尚有何说!”黄祖自知难免,拨马而走。甘宁冲开士
卒,直赶将来,只听得后面喊声起处,又有数骑赶来。宁视之,乃程普也。宁恐普来争功,
慌忙拈弓搭箭,背射黄祖,祖中箭翻身落马;宁枭其首级,回马与程普合兵一处,回见孙
权,献黄祖首级。权命以木匣盛贮,待回江东祭献于亡父灵前。重赏三军,升甘宁为都尉。
商议欲分兵守江夏。张昭曰:“孤城不可守,不如且回江东。刘表知我破黄祖,必来报仇;
我以逸待劳,必败刘表;表败而后乘势攻之,荆襄可得也。”权从其言,遂弃江夏,班师回
江东。

苏飞在槛车内,密使人告甘宁求救。宁曰:“飞即不言,吾岂忘之?”大军既至吴会,
权命将苏飞袅首,与黄祖首级一同祭献。甘宁乃入见权,顿首哭告曰:“某向日若不得苏
飞,则骨填沟壑矣,安能效命将军麾下哉?今飞罪当诛,某念其昔日之恩情,愿纳还官爵,
以赎飞罪。”权曰:“彼既有恩于君,吾为君赦之。但彼若逃去奈何?宁曰:“飞得免诛
戮,感恩无地,岂肯走乎!若飞去,宁愿将首级献于阶下。”权乃赦苏飞,止将黄祖首级祭
献。祭毕设宴,大会文武庆功。

正饮酒间,忽见座上一人大哭而起,拔剑在手,直取甘宁。宁忙举坐椅以迎之。权惊视
其人,乃凌统也,因甘宁在江夏时,射死他父亲凌操,今日相见,故欲报仇。权连忙劝住,
谓统曰:“兴霸射死卿父,彼时各为其主,不容不尽力。今既为一家人,岂可复理旧仇?万
事皆看吾面。”凌统即头大哭曰:“不共戴天之仇,岂容不报!”权与众官再三劝之,凌统
只是怒目而视甘宁。权即日命甘宁领兵五千、战船一百只,往夏口镇守,以避凌统。宁拜
谢,领兵自往夏口去了。权又加封凌统为承烈都尉。统只得含恨而止。东吴自此广造战船,
分兵守把江岸;又命孙静引一枝军守吴会;孙权自领大军,屯柴桑;周瑜日于鄱阳湖教练水
军,以备攻战。

话分两头。却说玄德差人打探江东消息,回报:“东吴已攻杀黄祖,现今屯兵柴桑。”
玄德便请孔明计议。正话间,忽刘表差人来请玄德赴荆州议事。孔明曰:“此必因江东破了
黄祖,故请主公商议报仇之策也。某当与主公同往,相机而行,自有良策。”玄德从之,留
云长守新野,令张飞引五百人马跟随往荆州来。玄德在马上谓孔明曰:“今见景升,当若何
对答?”孔明曰:“当先谢襄阳之事。他若令主公去征讨江东,切不可应允,但说容归新
野,整顿军马。”玄德依言。

来到荆州,馆驿安下,留张飞屯兵城外,玄德与孔明入城见刘表。礼毕,玄德请罪于阶
下。表曰:“吾已悉知贤弟被害之事。当时即欲斩蔡瑁之首,以献贤弟;因众人告危,故姑
恕之。贤弟幸勿见罪。”玄德曰:“非干蔡将军之事,想皆下人所为耳。”表曰:“今江夏
失守,黄祖遇害,故请贤弟共议报复之策。”玄德曰:“黄祖性暴,不能用人,故致此祸。
今若兴兵南征,倘曹操北来,又当奈何?”表曰:“吾今年老多病,不能理事,贤弟可来助
我。我死之后,弟便为荆州之主也。”玄德曰:“兄何出此言!量备安敢当此重任。”孔明
以目视玄德。玄德曰:“容徐思良策。”遂辞出。

回至馆驿,孔明曰:“景升欲以荆州付主公,奈何却之?”玄德曰:“景升待我,恩礼
交至,安忍乘其危而夺之?”孔明叹曰:“真仁慈之主也!”正商论间,忽报公子刘琦来
见。玄德接入。琦泣拜曰:“继母不能相容,性命只在旦夕,望叔父怜而救之。”玄德曰:
“此贤侄家事耳,奈何问我?”孔明微笑。玄德求计于孔明,孔明曰:“此家事,亮不敢与
闻。”少时,玄德送琦出,附耳低言曰:“来日我使孔明回拜贤侄,可如此如此,彼定有妙
计相告。”琦谢而去。

次日,玄德只推腹痛,乃浼孔明代往回拜刘琦。孔明允诺,来至公子宅前下马,入见公
子。公子邀入后堂。茶罢,琦曰:“琦不见容于继母,幸先生一言相救。”孔明曰:“亮客
寄于此,岂敢与人骨肉之事?倘有漏泄,为害不浅。”说罢,起身告辞。琦曰:“既承光
顾,安敢慢别。”乃挽留孔明入密室共饮。饮酒之间,琦又曰:“继母不见容,乞先生一言
救我。”孔明曰:“此非亮所敢谋也。”言讫,又欲辞去。琦曰:“先生不言则已,何便欲
去?”孔明乃复坐。琦曰:“琦有一古书,请先生一观。”乃引孔明登一小楼,孔明曰:
“书在何处?”琦泣拜曰:“继母不见容,琦命在旦夕,先生忍无一言相救乎?”孔明作色
而起,便欲下楼,只见楼梯已撤去。琦告曰:“琦欲求教良策,先生恐有泄漏,不肯出言;
今日上不至天,下不至地,出君之口,入琦之耳:可以赐教矣。”孔明曰:“疏不间亲,亮
何能为公子谋?琦曰:“先生终不幸教琦乎!琦命固不保矣,请即死于先生之前。”乃掣剑
欲自刎。孔明止之曰:“已有良策。”琦拜曰:“愿即赐教。”孔明曰:“公子岂不闻申
生、重耳之事乎?申生在内而亡,重耳在外而安。今黄祖新亡,江夏乏人守御,公子何不上
言,乞屯兵守江夏,则可以避祸矣。”琦再拜谢教,乃命人取梯迭孔明下楼。孔明辞别,回
见玄德,具言其事。玄德大喜。

次日,刘琦上言,欲守江夏。刘表犹豫未决,请玄德共议。玄德曰:“江夏重地,固非
他人可守,正须公子自往。东南之事,兄父子当之;西北之事,备愿当之。”表曰:“近闻
曹操于邺郡作玄武池以练水军,必有南征之意,不可不防。”玄德曰“备已知之,兄勿忧
虑。”遂拜辞回新野。刘表令刘琦引兵三千往江夏镇守。却说曹操罢三公之职,自以丞相兼
之。以毛玠为东曹掾,崔琰为西曹掾,司马懿为文学掾。懿字仲达,河内温人也。颍川太守
司马隽之孙,京兆尹司马防之子,主簿司马朗之弟也。自是文官大备,乃聚武将商议南征。
夏侯惇进曰:“近闻刘备在新野,每日教演士卒,必为后患,可早图之。”操即命夏侯惇为
都督,于禁、李典、夏侯兰、韩浩为副将,领兵十万,直抵博望城,以窥新野。荀彧谏曰:
“刘备英雄,今更兼诸葛亮为军师,不可轻敌。”惇曰:“刘备鼠辈耳,吾必擒之。”徐庶
曰:“将军勿轻视刘玄德。今玄德得诸葛亮为辅,如虎生翼矣。”操曰:“诸葛亮何人
也?”庶曰:亮字孔明,道号卧龙先生。有经天纬地之才,出鬼入神之计,真当世之奇才,
非可小觑。”操曰:“比公若何?”庶曰:“庶安敢比亮?庶如萤火之光,亮乃皓月之明
也。”夏侯惇曰:“元直之言谬矣。吾看诸葛亮如草芥耳,何足惧哉!吾若不一阵生擒刘
备,活捉诸葛,愿将首级献与丞相。”操曰:“汝早报捷书,以慰吾心。”惇奋然辞曹操,
引军登程。却说玄德自得孔明,以师礼待之。关、张二人不悦,曰:“孔明年幼,有甚才
学?兄长待之太过!又未见他真实效验!”玄德曰:“吾得孔明,犹鱼之得水也。两弟勿复
多言。”关、张见说,不言而退,一日,有人送蠫牛尾至。玄德取尾亲自结帽。孔明入见,
正色曰:“明公无复有远志,但事此而已耶?”玄德投帽于地而谢曰:“吾聊假此以忘忧
耳。”孔明曰:“明公自度比曹操若何?”玄德曰:“不如也。”孔明曰:“明公之众,不
过数千人,万一曹兵至,何以迎之?”玄德曰:“吾正愁此事,未得良策。”孔明曰:“可
速招募民兵,亮自教之,可以待敌。”玄德遂招新野之民,得三千人。孔明朝夕教演阵法。

忽报曹操差夏侯惇引兵十万,杀奔新野来了。张飞闻知,谓云长曰:“可着孔明前去迎
敌便了。”正说之间,玄德召二人入,谓曰:”夏侯惇引兵到来,如何迎敌?”张飞曰:
“哥哥何不使水去?”玄德曰:“智赖孔明,勇须二弟,何可推调?”关、张出,玄德请孔
明商议。孔明曰:“但恐关、张二人不肯听吾号令;主公若欲亮行兵,乞假剑印。”玄德便
以剑印付孔明,孔明遂聚集众将听令。张飞谓云长曰:“且听令去,看他如何调度。”孔明
令曰:“博望之左有山,名曰豫山;右有林,名曰安林:可以埋伏军马。云长可引一千军往
豫山埋伏,等彼军至,放过休敌;其辎重粮草,必在后面,但看南面火起,可纵兵出击,就
焚其粮草。翼德可引一千军去安林背后山谷中埋伏,只看南面火起,便可出,向博望城旧屯
粮草处纵火烧之。关平、刘封可引五百军,预备引火之物,于博望坡后两边等候,至初更兵
到,便可放火矣。”又命:“于樊城取回赵云,令为前部,不要赢,只要输,主公自引一军
为后援。各须依计而行,勿使有失。”云长曰:“我等皆出迎敌,未审军师却作何事?”孔
明曰:“我只坐守县城。”张飞大笑曰:“我们都去厮杀,你却在家里坐地,好自在!”孔
明曰:“剑印在此,违令者斩!”玄德曰:“岂不闻运筹帷幄之中,决胜千里之外?二弟不
可违令。”张飞冷笑而去。云长曰:“我们且看他的计应也不应,那时却来问他未迟。”二
人去了。众将皆未知孔明韬略,今虽听令,却都疑惑不定。孔明谓玄德曰:“主公今日可便
引兵就博望山下屯住。来日黄昏,敌军必到,主公便弃营而走;但见火起,即回军掩杀。亮
与糜竺、糜芳引五百军守县。”命孙乾、简雍准备庆喜筵席,安排功劳簿伺候。派拨已毕,
玄德亦疑惑不定。

却说夏侯惇与于禁等引兵至博望,分一半精兵作前队,其余尽护粮车而行。时当秋月,
商飙徐起。人马趱行之间,望见前面尘头忽起。惇便将人马摆开,问向导官曰:“此向是何
处?”答曰:“前面便是博望城,后面是罗川口。”惇令于禁、李典押住阵脚,亲自出马阵
前。遥望军马来到,惇忽然大笑。众问:“将军为何而笑?”惇曰:“吾笑徐元直在丞相面
前,夸诸葛亮为天人;今观其用兵,乃以此等军马为前部,与吾对敌,正如驱犬羊与虎豹斗
耳!吾于丞相前夸口。要活捉刘备、诸葛亮,今必应吾言矣。”遂自纵马向前。赵云出马。
惇骂曰:“汝等随刘备,如孤魂随鬼耳!”云大怒,纵马来战。两马相交,不数合,云诈败
而走。夏侯惇从后追赶。云约走十余里,回马又战。不数合又走。韩浩拍马向前谏曰:“赵
云诱敌,恐有埋伏。”惇曰:“敌军如此,虽十面埋伏,吾何惧哉!”遂不听浩言,直赶至
博望坡。一声炮响,玄德自引军冲将过来,接应交战。夏侯惇笑谓韩浩曰:“此即埋伏之兵
也!吾今晚不到新野,誓不罢兵!”乃催军前进。玄德、赵云退后便走,时天色已晚,浓云
密布,又无月色;昼风既起,夜风愈大。夏侯惇只顾催军赶杀。于禁、李典赶到窄狭处,两
边都是芦苇。典谓禁曰:“欺敌者必败。南道路狭,山川相逼。树木丛杂,倘彼用火攻,奈
何?”禁曰:“君言是也。吾当往前为都督言之;君可止住后军。”李典便勒回马,大叫:
“后军慢行!”人马走发,那里拦当得住?于禁骤马大叫:“前军都督且住!”夏侯惇正走
之间,见于禁从后军奔来,便问何故。禁曰:“南道路狭,山川相逼,树木丛杂,可防火
攻。”夏侯惇猛省,即回马令军马勿进。言未已,只听背后喊声震起,早望见一派火光烧
着,随后两边芦苇亦着。一霎时,四面八方,尽皆是火;又值风大,火势愈猛。曹家人马,
自相践踏,死者不计其数。赵云回军赶杀,夏侯惇冒烟突火而走。且说李典见势头不好,急
奔回博望城时,火光中一军拦住。当先大将,乃关云长也。李典纵马混战,夺路而走。于禁
见粮草车辆,都被火烧,便投小路奔逃去了。夏侯兰、韩浩来救粮草,正遇张飞。战不数
合,张飞一枪刺夏侯兰于马下。韩浩夺路走脱。直杀到天明,却才收军。杀得尸横遍野,血
流成河。后人有诗曰:“博望相持用火攻,指挥如意笑谈中。直须惊破曹公胆,初出茅庐第
一功!”夏侯惇收拾残军,自回许昌。却说孔明收军。关、张二人相谓曰:“孔明真英杰
也!”行不数里,见糜竺、糜芳引军簇拥着一辆小车。车中端坐一人,乃孔明也。关、张下
马拜伏于车前。须臾,玄德、赵云、刘封、关平等皆至,收聚众军,把所获粮草辎重,分赏
将士,班师回新野,新野百姓望尘遮道而拜,曰:“吾属生全,皆使君得贤人之力也!”孔
明回至县中,谓玄德曰:“夏侯惇虽败去,曹操必自引大军来。”玄德曰:“似此如之奈
何?”孔明曰:“亮有一计,可敌曹军。”正是:破敌未堪息战马,避兵又必赖良谋。未知
其计若何,且看下回分解。