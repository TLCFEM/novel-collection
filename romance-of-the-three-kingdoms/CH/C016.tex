\chapter{吕奉先射戟辕门~曹孟德败师淯水}

却说杨大将献计欲攻刘备。袁术曰:“计将安出?”大将曰:“刘备屯军小沛,虽然易
取,奈吕布虎踞徐州,前次许他金帛粮马,至今未与,恐其助备;今当令人送与粮食,以结
其心,使其按兵不动,则刘备可擒。先擒刘备,后图吕布,徐州可得也。”术喜,便具粟二
十万斛,令韩胤赍密书往见吕布。吕布甚喜,重待韩胤。胤回告袁术,术遂遣纪灵为大将,
雷薄、陈兰为副将,统兵数万,进攻小沛。玄德闻知此信,聚众商议。张飞要出战。孙韩
曰:“今小沛粮寡兵微,如何抵敌?可修书告急于吕布。”张飞曰:“那厮如何肯来!”玄
德曰:“乾之言善。”遂修书与吕布。书略曰:“伏自将军垂念,令备于小沛容身,实拜云
天之德。今袁术欲报私仇,遣纪灵领兵到县,亡在旦夕,非将军莫能救。望驱一旅之师,以
救倒悬之急,不胜幸甚!”吕布看了书,与陈宫计议曰:“前者袁术送粮致书,盖欲使我不
救玄德也。今玄德又来求救。吾想玄德屯军小沛,未必遂能为我害;若袁术并了玄德,则北
连泰山诸将以图我,我不能安枕矣:不若救玄德。”遂点兵起程。

却说纪灵起兵长驱大进,已到沛县东南,扎下营寨。昼列旌旗,遮映山川;夜设火鼓,
震明天地。玄德县中,止有五千余人,也只得勉强出县,布阵安营。忽报吕布引兵离县一
里、西南上扎下营寨。纪灵知吕布领兵来救刘备,急令人致书于吕布,责其无信。布笑曰:
“我有一计,使袁、刘两家都不怨我。”乃发使往纪灵、刘备寨中,请二人饮宴。玄德闻布
相请,即便欲往。关、张曰:“兄长不可去。吕布必有异心。”玄德曰:“我待彼不薄,彼
必不害我。”遂上马而行。关、张随往,到吕布寨中,入见。布曰:“吾今特解公之危。异
日得志,不可相忘!”玄德称谢。布请玄德坐。关、张按剑立于背后。人报纪灵到,玄德大
惊,欲避之。布曰:“吾特请你二人来会议,勿得生疑。”玄德未知其意,心下不安。

纪灵下马入寨,却见玄德在帐上坐,大惊,抽身便回。左右留之不住。吕布向前一把扯
回,如提童稚。灵曰:“将军欲杀纪灵耶?”布曰:“非也。”灵曰:“莫非杀大耳儿
乎?”布曰:“亦非也。”灵曰:“然则为何?”布曰:“玄德与布乃兄弟也,今为将军所
困,故来救之。”灵曰:“若此则杀灵也?”布曰:“无有此理。布平生不好斗,惟好解
斗。吾今为两家解之。”灵曰:“请问解之之法?”布曰:“我有一法,从天所决。”乃拉
灵入帐与玄德相见。二人各怀疑忌。布乃居中坐,使灵居左,备居右,且教设宴行酒。酒行
数巡,布曰:“你两家看我面上,俱各罢兵。”玄德无语。灵曰:“吾奉主公之命,提十万
之兵,专捉刘备,如何罢得?”张飞大怒,拔剑在手。叱曰:“吾虽兵少,觑汝辈如儿戏
耳!你比百万黄巾何如?你敢伤我哥哥!”关公急止之曰:“且看吕将军如何主意,那时各
回营寨厮杀未迟。”吕布曰:“我请你两家解斗,须不教你厮杀!”这边纪灵不忿,那边张
飞只要厮杀。布大怒,教左右:“取我戟来,布提画戟在手,纪灵、玄德尽皆失色。布曰:
“我劝你两家不要厮杀,尽在天命。”令左右接过画戟,去辕门外远远插定。乃回顾纪灵、
玄德曰:“辕门离中军一百五十步,吾若一箭射中戟小枝,你两家罢兵,如射不中,你各自
回营,安排厮杀。有不从吾言者,并力拒之。”纪灵私忖:“戟在一百五十步之外,安能便
中?且落得应允。待其不中,那时凭我厮杀。”便一口许诺。玄德自无不允。布都教坐,再
各饮一杯酒。酒毕,布教取弓箭来。玄德暗祝曰:“只愿他射得中便好!”只见吕布挽起袍
袖,搭上箭,扯满弓,叫一声:“着!”正是:弓开如秋月行天,箭去似流星落地,一箭正
中画戟小枝。帐上帐下将校,齐声喝采。后人有诗赞之曰:“温侯神射世间稀,曾向辕门独
解危。落日果然欺后羿,号猿直欲胜由基。虎筋弦响弓开处,雕羽翅飞箭到时。豹子尾摇穿
画戟,雄兵十万脱征衣。”

当下吕布射中画戟小枝,呵呵大笑,掷弓于地,执纪灵、玄德之手曰:“此天令你两家
罢兵也!”喝教军士:“斟酒来!”各饮一大觥。”玄德暗称惭愧。纪灵默然半响,告布
曰:“将军之言,不敢不听;奈纪灵回去,主人如何肯信?”布曰:“吾自作书复之便
了。”酒又数巡,纪灵求书先回。布谓玄德曰:“非我则公危矣。玄德拜谢,与关、张回。
次日,三处军马都散。不说玄德入小沛,吕布归徐州。却说纪灵回淮南见袁术,说吕布辕门
射就解和之事,呈上书信。袁术大怒曰:“吕布受吾许多粮米,反以此儿戏之事,偏护刘
备。吾当自提重兵,亲征刘备,兼讨吕布!”纪灵曰:“主公不可造次。吕布勇力过人,兼
有徐州之地;若布与备首尾相连,不易图也。吴闻布妻严氏有一女,年已及笄。主公有一
子,可令人求亲于布,布若嫁女于主公,必杀刘备:此乃疏不间亲之计也。”袁术从之,即
日遣韩胤为媒,赍礼物往徐州求亲。

胤到徐州见布,称说:“主公仰慕将军,欲求令爱为儿妇,永结秦晋之好。”布入谋于
妻严氏。原来吕布有二妻一妾:先娶严氏为正妻,后娶貂蝉为妾;及居小沛时,又娶曹豹之
女为次妻。曹氏先亡无出,貂蝉亦无所出,惟严氏生一女,布最钟爱。当下严氏对布曰:
“吾闻袁公路久镇淮南,兵多粮广,早晚将为天子。若成大事,则吾女有后妃之望。只不知
他有几子?”布曰:“止有一子。”妻曰:“既如此,即当许之。纵不为皇后,吾徐州亦无
忧矣。”布意遂决,厚款韩胤,许了亲事。韩胤回报袁术。术即备聘礼,仍令韩胤送至徐
州。吕布受了、设席相待,留于馆驿安歇。

次日,陈宫竟往馆驿内拜望韩胤。讲礼毕,坐定。宫乃叱退左右,对胤曰:“谁献此
计,教袁公与奉先联姻?意在取刘玄德之头乎?”胤失惊,起谢曰:“乞公台勿泄!”宫
曰:“吾自不泄,只恐其事若迟,必被他人识破,事将中变。”胤曰:“然则奈何?”愿公
教之。”宫曰:“吾见奉先,使其即日送女就亲,何如?”胤大喜,称谢曰:“若如此,袁
公感佩明德不浅矣!”宫遂辞别韩胤。入见吕布曰:“闻公女许嫁袁公路,甚善。但不知于
何日结亲?”布曰:“尚容徐议。”宫曰:“古者自受聘成婚之期,各有定例:天子一年,
诸侯半年,大夫一季,庶民一月。”布曰:“袁公路天赐国室,早晚当为帝,今从天子例,
可乎?”宫曰:“不可。”布曰:“然则仍从诸侯例?”宫曰:“亦不可。”布曰:“然则
将从卿大夫例矣?”宫曰:“亦不可。”布笑曰:“公岂欲吾依庶民例耶?”宫曰:“非
也”。布曰:“然则公意欲如何?”宫曰:“方今天下诸侯,互相争雄;今公与袁公路结
亲,诸侯保无有嫉妒者乎?”若复远择吉期,或竟乘我良辰,伏兵半路以夺之,如之奈何?
为今之计:不许便休;既已许之。当趁诸侯未知之时,即便送女到寿春,另居别馆,然后择
吉成亲,万无一失也。”布喜曰:“公台之言甚当。”遂入告严氏。连夜具办妆奁,收拾宝
马香车,令宋宪、魏续一同韩胤送女前去。鼓乐喧天,送出城外。

时陈元龙之父陈珪,养老在家,闻鼓乐之声,遂问左右。左右告以故。珪曰:“此乃疏
不间亲之计也。玄德危矣。”遂扶病来见吕布。布曰:“大夫何来?”珪曰:“闻将军死
至,特来吊丧。”布惊曰:“何出此言?”珪曰:“前者袁公路以金帛送公,欲杀刘玄德,
而公以射戟解之;今忽来求亲,其意盖欲以公女为质,随后就来攻玄德而取小沛。小沛亡,
徐州危矣。且彼或来借粮,或来借兵:公若应之,是疲于奔命,而又结怨于人;若其不允,
是弃亲而启兵端也。况闻袁术有称帝之意,是造反也。彼若造反,则公乃反贼亲属矣,得无
为天下所不容乎?”布大惊曰:“陈宫误我!”急命张辽引兵,追赶至三十里之外,将女抢
归;连韩胤都拿回监禁,不放归去。却令人回复袁术,只说女儿妆奁未备,俟备毕便自送
来。陈珪又说吕布,使解韩胤赴许都。布犹豫未决。

忽人报:“玄德在小沛招军买马,不知何意。”布曰:“此为将者本分事,何足为
怪。”正话间,宋宪、魏续至,告布曰:“我二人奉明公之命,往山东买马,买得好马三百
余匹;回至沛县界首,被强寇劫去一半。打听得是刘备之弟张飞,诈妆出贼,抢劫马匹去
了。”吕布听了大怒,随即点兵往小沛来斗张飞。玄德闻知大惊,慌忙领兵出迎。两阵圆
处,玄德出马曰:“兄长何故领兵到此?”布指骂曰:“我辕门射戟,救你大难,你何故夺
我马匹?”玄德曰:“备因缺马,令人四下收买,安敢夺兄马匹。”布曰:你便使张飞夺了
我好马一百五十匹,尚自抵赖!”张飞挺枪出马曰:“是我夺了你好马!你今待怎么?”布
骂曰:“环眼贼!你累次渺视我!”飞曰:“我夺你马你便恼,你夺我哥哥的徐州便不说
了!”布挺戟出马来战张飞,飞亦挺枪来迎。两个酣战一百余合,未见胜负。玄德恐有疏
失,急鸣金收军入城。吕布分军四面围定。玄德唤张飞责之曰:“都是你夺他马匹,惹起事
端!如今马匹在何处?”飞曰:“都寄在各寺院内。”玄德随令人出城,至吕布营中,说情
愿送还马匹,两相罢兵。布欲从之。陈宫曰:“今不杀刘备,久后必为所害。”布听之,不
从所请,攻城愈急。玄德与糜竺、孙乾商议。孙乾曰:“曹操所恨者,吕布也。不若弃城走
许都,投奔曹操,借军破布,此为上策。”玄德曰:“谁可当先破围而出?”飞曰:“小弟
情愿死战!”玄德令飞在前,云长在后;自居于中,保护老小。当夜三更,乘着月明,出北
门而走。正遇宋宪、魏续,被翼德一阵杀退,得出重围。后而张辽赶来,关公敌住。吕布见
玄德去了,也不来赶,随即入城安民,令高顺守小沛,自己仍回徐州去了。

却说玄德前奔许都,到城外下寨,先使孙乾来见曹操,言被吕布追逼。特来相投。操
曰:“玄德与吾,兄弟也。”便请入城相见。次日,玄德留关、张在城外,自带孙乾、糜竺
入见操。操待以上宾之礼。玄德备诉吕布之事,操曰:“布乃无义之辈,吾与贤弟并力诛
之。”玄德称谢。操设宴相待,至晚送出。荀彧入见曰:“刘备,英雄也。今不早图,后必
为患。”操不答。彧出,郭嘉入。操曰:“荀彧劝我杀玄德,当如何?”嘉曰:“不可。主
公兴义兵,为百姓除暴,惟仗信义以招俊杰,犹惧其不来也;今玄德素有英雄之名,以困穷
而来投,若杀之,是害贤也。天下智谋之士,闻而自疑,将裹足不前,主公谁与定天下乎?
夫除一人之患,以阻四海之望:安危之机不可不察。”操大喜曰:“君言正合吾心。”次
日,即表荐刘备领豫州牧。程昱谏曰:“刘备终不为人之下,不如早图之。”操曰:“方今
正用英雄之时,不可杀一人而失天下之心。此郭奉孝与吾有同见也。”遂不听昱言,以兵三
千、粮万斛送与玄德,使往豫州到任。进兵屯小沛,招集原散之兵,攻吕布。玄德至豫州,
令人约会曹操。操正欲起兵,自往征吕布,忽流星马报说张济自关中引兵攻南阳,为流矢所
中而死;济侄张绣统其众,用贾诩为谋士,结连刘表,屯兵宛城,欲兴兵犯阙夺驾。操大
怒,欲兴兵讨之,又恐吕布来侵许都,乃问计于荀彧。彧曰:“此易事耳。吕布无谋之辈,
见利必喜;明公可遣使往徐州,加官赐赏,令与玄德解和。布喜,则不思远图矣。”操曰:
“善。”遂差奉军都尉王则,赍官诰并和解书,往徐州去讫。一面起兵十五万,亲讨张绣。
分军三路而行,以夏侯惇为先锋。军马至淯水下寨。贾诩劝张绣曰:“操兵势大,不可与
敌,不如举众投降。”张绣从之,使贾诩至操寨通款。操见诩应对如流,甚爱之,效用为谋
士。诩曰:“某昔从李傕,得罪天下;今从张绣,言听计从,不忍弃之。”乃辞去。次日引
绣来见操,操待之甚厚。引兵入宛城屯扎,余军分屯城外,寨栅联络十余里。一住数日,绣
每日设宴请操。

一日操醉,退入寝所,私问左右曰:“此城中有妓女否?”操之兄子曹安民,知操意,
乃密对曰:“昨晚小侄窥见馆舍之侧,有一妇人,生得十分美丽,问之,即绣叔张济之妻
也。”操闻言,便令安民领五十甲兵往取之。须臾,取到军中。操见之,果然美丽。问其
姓,妇答曰:“妾乃张济之妻邹氏也。”操曰:“夫人识吾否?”邹氏曰:“久闻丞相威
名,今夕幸得瞻拜。”操曰:“吾为夫人故,特纳张绣之降;不然灭族矣。”邹氏拜曰:
“实感再生之恩。”操曰:“今日得见夫人,乃天幸也。今宵愿同枕席,随吾还都,安享富
贵,何如?”邹氏拜谢。是夜,共宿于帐中。邹氏曰:“久住城中,绣必生疑,亦恐外人议
论。”操曰:“明日同夫人去寨中住。”次日,移于城外安歇,唤典韦就中军帐房外宿卫。
他人非奉呼唤,不许辄入。因此,内外不通。操每日与邹氏取乐,不想归期。

张绣家人密报绣。绣怒曰:“操贼辱我太甚!”便请贾诩商议。诩曰:“此事不可泄
漏。来日等操出帐议事,如此如此。”次日,操坐帐中,张绣入告曰:“新降兵多有逃亡
者,乞移屯中军。”操许之。绣乃移屯其军。分为四寨,刻期举事。因畏典韦勇猛,急切难
近,乃与偏将胡车儿商议。那故车儿力能负五百斤,日行七百里,亦异人也。当下献计于绣
曰:“典韦之可畏者,双铁戟耳。主公明日可请他来吃酒,使尽醉而归。那时某便混入他跟
来军士数内,偷入帐房,先盗其戟,此人不足畏矣。”绣甚喜,预先准备弓箭、甲兵,告示
各寨。至期,令贾诩致意请典韦到寨,殷勤待酒。至晚醉归,胡车儿杂在众人队里,直入大
寨。是夜曹操于帐中与邹氏饮酒,忽听帐外人言马嘶。操使人观之。回报是张绣军夜巡,操
乃不疑。时近二更,忽闻寨内呐喊,报说草车上火起。操曰:“军人失火,勿得惊动。”须
臾,四下里火起。操始着忙,急唤典韦。韦方醉卧,睡梦中听得金鼓喊杀之声,便跳起身
来,却寻不见了双戟。时敌兵已到辕门,韦急掣步卒腰刀在手。只见门首无数军马,各抵长
枪,抢入寨来。韦奋力向前,砍死二十余人。马军方退,步军又到,两边枪如苇列。韦身无
片甲,上下被数十枪,兀自死战。刀砍缺不堪用,韦即弃刀,双手提着两个军人迎敌,击死
者八九人,群贼不敢近,只远远以箭射之,箭如骤雨。韦犹死拒寨门。争奈寨后贼军已入,
韦背上又中一枪,乃大叫数声,血流满地而死。死了半晌,还无一人敢从前门而入者。

却说曹操赖典韦当住寨门,乃得从寨后上马逃奔,只有曹安民步随。操右臂中了一箭,
马亦中了三箭。亏得那马是大宛良马,熬得痛,走得快。刚刚走到清水河边,贼兵追至,安
民被砍为肉泥。操急骤马冲波过河,才上得岸,贼兵一箭射来,正中马眼,那马扑地倒了。
操长子曹昂,即以己所乘之马奉操。操上马急奔。曹昂却被乱箭射死。操乃走脱。路逢诸
将,收集残兵。时夏侯惇所领青州之兵,乘势下乡,劫掠民家,平虏校尉于禁,即将本部军
于路剿杀,安抚乡民。青州兵走回,迎操泣拜于地,言于禁造反,赶杀青州军马。操大惊。
须臾,夏侯惇、许褚、李典;乐进都到。操言于禁造反,可整兵迎之,却说于禁见操等俱
到,乃引军射住阵角,凿堑安营。或告之曰:“青州军言将军造反,今丞相已到,何不分
辩,乃先立营寨耶?”于禁曰:“今贼追兵在后,不时即至;若不先准备,何以拒敌?分辩
小事,退敌大事。”

安营方毕,张绣军两路杀至。于禁身先出寨迎敌。绣急退兵。左右诸将,见于禁向前,
各引兵击之,绣军大败,追杀百余里。绣势穷力孤,引败兵投刘表去了。曹操收军点将,于
禁入见,备言青州之兵,肆行劫掠,大失民望,某故杀之。操曰:“不告我,先下寨,何
也?”禁以前言对。操曰:“将军在匆忙之中,能整兵坚垒,任谤任劳,使反败为胜,虽古
之名将,何以加兹!”乃赐以金器一副,封益寿亭侯;赍夏侯惇治兵不严之过。又设祭祭典
韦,操亲自哭而奠之,顾谓诸将曰:“吾折长子、爱侄,俱无深痛;独号泣典韦也!”众皆
感叹,次日下令班师。不说曹操还兵许都。且说王则赍诏至徐州,布迎接入府,开读诏书:
封布为平东将军,特赐印绶。又出操私书,王则在吕布面前极道曹公相敬之意。布大喜。忽
报袁术遣人至,布唤入问之。使言:“袁公早晚即皇帝位,立东宫,催取皇妃早到淮南。”
布大怒曰:“反贼焉敢如此!”遂杀来使,将韩胤用枷钉了,遣陈登赍谢表,解韩胤一同王
则上许都来谢恩。且答书于操,欲求实授徐州牧。操知布绝婚袁术,大喜,遂斩韩胤于市
曹。陈登密谏操曰:“吕布,豺狼也,勇而无谋,轻于去就,宜早图之。”操曰:“吾素知
吕布狼子野心,诚难久养。非公父子莫能究其情,公当与吾谋之。”登曰:“丞相若有举
动,某当为内应。”操喜,表赠陈珪秩中二千石,登为广陵太守。登辞回,操执登手曰:
“东方之事,便以相付。”登点头允诺。回徐州见吕布,布问之,登言:“父赠禄,某为太
守。”布大怒曰:“汝不为吾求徐州牧,而乃自求爵禄!汝父教我协同曹公,绝婚公路,今
吾所求,终无一获;而汝父子俱各显贵,吾为汝父子所卖耳!”遂拔剑欲斩之。登大笑曰:
“将军何其不明之甚也!”布曰:“吾何不明?”登曰:“吾见曹公,言养将军譬如养虎,
当饱其肉,不饱则将噬人。曹公笑曰:“不如卿言。吾待温侯,如养鹰耳:狐兔未息,不敢
先饱,饥则为用,饱则飏去。某问谁为狐兔,曹公曰:“淮南袁术;江东孙策、冀州袁绍、
荆襄刘表、益州刘璋、汉中张鲁,皆狐兔也。布掷剑笑曰:“曹公知我也!”正说话间,忽
报袁术军取徐州。吕布闻言失惊。正是:秦晋未谐吴越斗,婚姻惹出甲兵来。毕竟后事如
何,且听下文分解。