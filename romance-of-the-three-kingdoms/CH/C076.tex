\chapter{徐公明大战沔水~关云长败走麦城}

却说糜芳闻荆州有失,正无计可施。忽报公安守将傅士仁至,芳忙接入城,问其事故。
士仁曰:“吾非不忠。势危力困,不能支持,我今已降东吴。将军亦不如早降。”芳曰:
“吾等受汉中王厚恩,安忍背之?“士仁曰:“关公去日,痛恨吾二人;倘一日得胜而回,
必无轻恕。公细察之。”芳曰:“吾兄弟久事汉中王,岂可一朝相背?”正犹豫间,忽报关
公遣使至,接入厅上。使者曰:“关公军中缺粮,特来南郡、公安二处取白米十万石,令二
将军星夜解去军前交割。如迟立斩。”芳大惊,顾谓傅士仁曰:“今荆州已被东吴所取,此
粮怎得过去?”士仁厉声曰:“不必多疑!”遂拔剑斩来使于堂上。芳惊曰:“公如何斩
之?”士仁曰:“关公此意,正要斩我二人。我等安可束手受死?公今不早降东吴,必被关
公所杀。”正说间,忽报吕蒙引兵杀至城下。芳大惊,乃同傅士仁出城投降。蒙大喜,引见
孙权。权重赏二人。安民已毕,大犒三军。

时曹操在许都,正与众谋士议荆州之事,忽报东吴遣使奉书至。操召人,使者呈上书
信。操拆视之,书中具言吴兵将袭荆州,求操夹攻云长;且嘱勿泄漏,使云长有备也。操与
众谋士商议,主簿董昭曰:“今樊城被困,引颈望救,不如令人将书射入樊城,以宽军心;
且使关公知东吴将袭荆州。彼恐荆州有失,必速退兵,却令徐晃乘势掩杀,可获全功。”操
从其谋,一面差人催徐晃急战;一面亲统大兵,径往洛阳之南阳陵坡驻扎,以救曹仁。

却说徐晃正坐帐中,忽报魏王使至。晃接入问之,使曰:“今魏王引兵,已过洛阳;令
将军急战关公,以解樊城之困。”正说间,探马报说:“关平屯兵在偃城,廖化屯兵在四
冢:前后一十二个寨栅,连络不绝。”晃即差副将徐商、吕建假着徐晃旗号,前赴偃城与关
平交战。晃却自引精兵五百,循沔水去袭偃城之后。且说关平闻徐晃自引兵至,遂提本部兵
迎敌。两阵对圆,关平出马,与徐商交锋,只三合,商大败而走;吕建出战,五六合亦败
走。平乘胜追杀二十余里,忽报城中火起。平知中计,急勒兵回救偃城。正遇一彪军摆开,
徐晃立马在门旗下,高叫曰:“关平贤侄,好不知死!汝荆州已被东吴夺了,犹然在此狂
为!”平大怒,纵马轮刀,直取徐晃;不三四合,三军喊叫,偃城中火光大起。平不敢恋
战,杀条大路,径奔四冢寨来。廖化接着。化曰:“人言荆州已被吕蒙袭了,军心惊慌,如
之奈何?”平曰:“此必讹言也。军士再言者斩之。”

忽流星马到,报说正北第一屯被徐晃领兵攻打。平曰:“若第一屯有失,诸营岂得安
宁?此间皆靠沔水,贼兵不敢到此。吾与汝同去救第一屯。”廖化唤部将分付曰:“汝等坚
守营寨,如有贼到,即便举火。”部将曰:“四冢寨鹿角十重,虽飞鸟亦不能入,何虑贼
兵!”于是关平、廖化尽起四冢寨精兵,奔至第一屯住扎。关平看见魏兵屯于浅山之上,谓
廖化曰:“徐晃屯兵,不得地利,今夜可引兵劫寨。”化曰:“将军可分兵一半前去,某当
谨守本寨。”

是夜,关平引一枝兵杀入魏寨,不见一人。平知是计,火速退时,左边徐商,右边吕
建,两下夹攻。平大败回营,魏兵乘势追杀前来,四面围住。关平、廖化支持不住,弃了第
一屯,径投四冢寨来。早望见寨中火起。急到寨前,只见皆是魏兵旗号。关平等退兵,忙奔
樊城大路而走。前面一军拦住,为首大将,乃是徐晃也。平、化二人奋力死战,夺路而走,
回到大寨,来见关公曰:“今徐晃夺了偃城等处;又兼曹操自引大军,分三路来救樊城;多
有人言荆州已被吕蒙袭了。”关公喝曰:“此敌人讹言,以乱我军心耳!东吴吕蒙病危,孺
子陆逊代之,不足为虑!”

言未毕,忽报徐晃兵至。公令备马。平谏曰:“父体未痊,不可与敌。”公曰:“徐晃
与吾有旧,深知其能;若彼不退,吾先斩之,以警魏将。”遂披挂提刀上马,奋然而出。魏
军见之,无不惊惧。公勒马问曰:“徐公明安在?”魏营门旗开处,徐晃出马,欠身而言
曰:“自别君侯,倏忽数载,不想君侯须发已苍白矣!忆昔壮年相从,多蒙教诲,感谢不
忘。今君侯英风震于华夏,使故人闻之,不胜叹羡!兹幸得一见,深慰渴怀。”公曰:“吾
与公明交契深厚,非比他人;今何故数穷吾儿耶?”晃回顾众将,厉声大叫曰:“若取得云
长首级者,重赏千金!”公惊曰:“公明何出此言?”晃曰:“今日乃国家之事,某不敢以
私废公。”言讫,挥大斧直取关公。公大怒,亦挥刀迎之。战八十余合,公虽武艺绝伦,终
是右臂少力。关平恐公有失,火急鸣金,公拨马回寨。忽闻四下里喊声大震。原来是樊城曹
仁闻曹操救兵至,引军杀出城来,与徐晃会合,两下夹攻,荆州兵大乱。关公上马,引众将
急奔襄江上流头。背后魏兵追至。关公急渡过襄江,望襄阳而奔。忽流星马到,报说:“荆
州已被吕蒙所夺,家眷被陷。”关公大惊。不敢奔襄阳,提兵投公安来。探马又报:“公安
傅士仁已降东吴了。”关公大怒。忽催粮人到,报说:“公安傅士仁往南郡,杀了使命,招
糜芳都降东吴去了。”

关公闻言,怒气冲塞,疮口迸裂,昏绝于地。众将救醒,公顾谓司马王甫曰:“悔不听
足下之言,今日果有此事!”因问:“沿江上下,何不举火?”探马答曰:“吕蒙使水手尽
穿白衣,扮作客商渡江,将精兵伏于【舟冓】【舟鹿】之中,先擒了守台士卒,因此不得举
火。”公跌足叹曰:“吾中奸贼之谋矣!有何面目见兄长耶!”管粮都督赵累曰:“今事急
矣,可一面差人往成都求救,一面从旱路去取荆州。”关公依言,差马良、伊籍赍文三道,
星夜赴成都求救;一面引兵来取荆州,自领前队先行,留廖化、关平断后。却说樊城围解,
曹仁引众将来见曹操,泣拜请罪。操曰:“此乃天数,非汝等之罪也。”操重赏三军,亲至
四冢寨周围阅视,顾谓众将曰:“荆州兵围堑鹿角数重,徐公明深入其中,竟获全功。孤用
兵三十余年,未敢长驱径入敌围。公明真胆识兼优者也!”众皆叹服。操班师还于摩陂驻扎。
徐晃兵至,操亲出寨迎之,见晃军皆按队伍而行,并无差乱。操大喜曰:“徐将军真有周亚
夫之风矣!”遂封徐晃为平南将军,同夏侯尚守襄阳,以遏关公之师。操因荆州未定,就屯
兵于摩陂,以候消息。却说关公在荆州路上,进退无路,谓赵累曰:“目今前有吴兵,后有
魏兵,吾在其中,救兵不至,如之奈何?”累曰:“昔吕蒙在陆口时,尝致书君侯,两家约
好,共诛操贼,今却助操而袭我,是背盟也。君侯暂驻军于此,可差人遗书吕蒙责之,看彼
如何对答。”关公从其言,遂修书遣使赴荆州来。

却说吕蒙在荆州,传下号令:凡荆州诸郡,有随关公出征将士之家,不许吴兵搅扰,按
月给与粮米;有患病者,遣医治疗。将士之家,感其恩惠,安堵不动。忽报关公使至,吕蒙
出郭迎接入城,以宾礼相待。使者呈书与蒙。蒙看毕,谓来使曰:“蒙昔日与关将军结好,
乃一己之私见;今日之事,乃上命差遣,不得自主。烦使者回报将军,善言致意。”遂设宴
款待,送归馆驿安歇。于是随征将士之家,皆来问信;有附家书者,有口传音信者,皆言家
门无恙,衣食不缺。

使者辞别吕蒙,蒙亲送出城。使者回见关公,具道吕蒙之语,并说:“荆州城中,君侯
宝眷并诸将家属,俱各无恙,供给不缺。”公大怒曰:“此奸贼之计也!我生不能杀此贼,
死必杀之,以雪吾恨!”喝退使者。使者出寨,众将皆来探问家中之事;使者具言各家安
好,吕蒙极其恩恤,并将书信传送各将。各将欣喜,皆无战心。

关公率兵取荆州,军行之次,将士多有逃回荆州者。关公愈加恨怒,遂催军前进。忽然
喊声大震,一彪军拦住,为首大将,乃蒋钦也,勒马挺枪大叫曰:“云长何不早降!”关公
骂曰:“吾乃汉将,岂降贼乎!”拍马舞刀,直取蒋钦。不三合,钦败走。关公提刀追杀二
十余里,喊声忽起,左边山谷中韩当领军冲出,右边山谷中周泰引军冲出,蒋钦回马复战,
三路夹攻。关公急撒军回走。行无数里,只见南山冈上人烟聚集,一面白旗招飐,上写“荆
州土人”四字,众人都叫本处人速速投降。关公大怒,欲上冈杀之。山崦内又有两军撞出:
左边丁奉,右边徐盛;并合蒋钦等三路军马,喊声震地,鼓角喧天,将关公困在核心。手下
将士,渐渐消疏。比及杀到黄昏,关公遥望四山之上,皆是荆州土兵,呼兄唤弟,觅子寻
爷,喊声不住。军心尽变,皆应声而去。关公止喝不住,部从止有三百余人。杀至三更,正
东上喊声连天,乃是关平、廖化分两路兵杀入重围,救出关公。关平告曰:“军心乱矣,必
得城池暂屯,以待援兵。麦城虽小,足可屯扎。”关公从之,催促残军前至麦城,分兵紧守
四门,聚将士商议。赵累曰:“此处相近上庸,现有刘封、孟达在彼把守,可速差人往求救
兵。若得这枝军马接济,以待川兵大至,军心自安矣。”

正议间,忽报吴兵已至,将城四面围定。公问曰:“谁敢突围而出,往上庸求救?”廖
化曰:“某愿往。”关平曰:“我护送汝出重围。”关公即修书付廖化藏于身畔。饱食上
马,开门出城。正遇吴将丁奉截往。被关平奋力冲杀,奉败走,廖化乘势杀出重围。投上庸
去了。关平入城,坚守不出。

且说刘封、孟达自取上庸,太守申耽率众归降,因此汉中王加刘封为副将军,与孟达同
守上庸。当日探知关公兵败,二人正议间,忽报廖化至。

封令请人问之。化曰:“关公兵败,现困于麦城,被围至急。蜀中援兵,不能旦夕即
至。特命某突围而出,来此求救。望二将军速起上庸之兵,以救此危。倘稍迟延,公必陷
矣。”封曰:“将军且歇,容某计议。”

化乃至馆驿安歇,专候发兵。刘封谓孟达曰:“叔父被困,如之奈何?”达曰:“东吴
兵精将勇;且荆州九郡,俱已属彼,止有麦城,乃弹丸之地;又闻曹操亲督大军四五十万,
屯于摩陂:量我等山城之众,安能敌得两家之强兵?不可轻敌。”封曰:“吾亦知之。奈关
公是吾叔父,安忍坐视而下救乎?”达笑曰:“将军以关公为叔,恐关公未必以将军为侄
也。某闻汉中王初嗣将军之时,关公即不悦。后汉中王登位之后,欲立后嗣,问于孔明,孔
明曰:‘此家事也,问关、张可矣,’汉中王遂遣人至荆州问关公,关公以将军乃螟蛉之
子,不可僭立,劝汉中王远置将军于上庸山城之地,以杜后患。此事人人知之,将军岂反不
知耶?何今日犹沾沾以叔侄之义,而欲冒险轻动乎?”封曰:“君言虽是,但以何词却
之?”达曰:“但言山城初附,民心未定,不敢造次兴兵,恐失所守。”封从其言。次日,
请廖化至,言此山城初附之所,未能分兵相救。化大惊,以头叩地曰:“若如此,则关公休
矣!”达曰:“我今即往,一杯之水,安能救一车薪之火乎?将军速回,静候蜀兵至可
也。”化大恸告求,刘封、孟达皆拂袖而入。廖化知事不谐,寻思须告汉中王求救,遂上马
大骂出城,望成都而去。

却说关公在麦城盼望上庸兵到,却不见动静;手下止有五六百人,多半带伤;城中无
粮,甚是苦楚。忽报城下一人教休放箭,有话来见君侯。公令放入,问之,乃诸葛瑾也。礼
毕茶罢,瑾曰:“今奉吴侯命,特来劝谕将军。自古道识时务者为俊杰,今将军所统汉上九
郡,皆已属他人类;止有孤城一区,内无粮草,外无救兵,危在旦夕。将军何不从瑾之言,
归顺吴侯,复镇荆襄,可以保全家眷。幸君侯熟思之。”关公正色而言曰:“吾乃解良一武
夫,蒙吾主以手足相待,安肯背义投敌国乎?城若破,有死而已。玉可碎而不可改其白,竹
可焚而不可毁其节,身虽殒,名可垂于竹帛也。汝勿多言,速请出城,吾欲与孙权决一死
战!”瑾曰:“吴侯欲与君侯结秦晋之好,同力破曹,共扶汉室,别无他意。君侯何执迷如
是?”言未毕,关平拔剑而前,欲斩诸葛瑾。公止之曰:“彼弟孔明在蜀,佐汝伯父,今若
杀彼,伤其兄弟之情也。”遂令左右逐出诸葛瑾。瑾满面羞惭,上马出城,回见吴侯曰:
“关公心如铁石,不可说也。”孙权曰:“真忠臣也!似此如之奈何?’吕范曰:“某请卜
其休咎。”权即令卜之。范揲蓍成象,乃“地水师卦”,更有玄武临应,主敌人远奔。权问
吕蒙曰:“卦主敌人远奔,卿以何策擒之?”蒙笑曰:“卦象正合某之机也。关公虽有冲天
之翼,飞不出吾罗网矣!”正是:龙游沟壑遭虾戏,凤入牢笼被鸟欺。毕竟吕蒙之计若何,
且看下文分解。