\chapter{丁奉雪中奋短兵~孙峻席间施密计}

却说姜维正走,遇着司马师引兵拦截。原来姜维取雍州之时,郭淮飞报入朝,魏主与司马懿商议停当,懿遣长子司马师引兵五万,前来雍州助战;师听知郭淮敌退蜀兵,师料蜀兵势弱,就来半路击之。直赶到阳平关,却被姜维用武侯所传连弩法,于两边暗伏连弩百余张,一弩发十矢,皆是药箭,两边弩箭齐发,前军连人带马射死不知其数。司马师于乱军之中,逃命而回。却说麴山城中蜀将句安,见援兵不至,乃开门降魏。姜维折兵数万,领败兵回汉中屯扎。司马师自还洛阳。至嘉平三年秋八月,司马懿染病,渐渐沉重,乃唤二子至榻前嘱曰:“吾事魏历年,官授太傅,人臣之位极矣;人皆疑吾有异志,吾尝怀恐惧。吾死之后,汝二人善理国政。慎之!慎之!”言讫而亡。长子司马师,次子司马昭,二人申奏魏主曹芳。芳厚加祭葬,优锡赠谥;封师为大将军,总领尚书机密大事,昭为骠骑上将军。却说吴主孙权,先有太子孙登,乃徐夫人所生,于吴赤乌四年身亡,遂立次子孙和为太子,乃琅琊王夫人所生。和因与全公主不睦,被公主所谮,权废之,和忧恨而死,又立三子孙亮为太子,乃潘夫人所生。此时陆逊、诸葛瑾皆亡,一应大小事务,皆归于诸葛恪。太元元年秋八月初一日,忽起大风,江海涌涛,平地水深八尺。吴主先陵所种松柏,尽皆拔起,直飞到建业城南门外,倒卓于道上。权因此受惊成病。至次年四月内,病势沉重,乃召太傅诸葛恪、大司马吕岱至榻前,嘱以后事。嘱讫而薨。在位二十四年,寿七十一岁,乃蜀汉延熙十五年也。后人有诗曰:“紫髯碧眼号英雄,能使臣僚肯尽忠。二十四年兴大业,龙盘虎踞在江东。”

孙权既亡,诸葛恪立孙亮为帝,大赦天下,改元建兴元年;谥权曰大皇帝,葬于蒋陵。早有细作探知其事,报入洛阳。司马师闻孙权已死,遂议起兵伐吴。尚书傅嘏曰:“吴有长江之险,先帝屡次征伐,皆不遂意;不如各守边疆,乃为上策。”师曰:“天道三十年一变,岂得常为鼎峙乎?吾欲伐吴。”昭曰:“今孙权新亡,孙亮幼懦,其隙正可乘也。”遂令征南大将军王昶引兵十万攻南郡,征东将军胡遵引兵十万攻东兴,镇南都督毋丘俭引兵十万攻武昌:三路进发。又遣弟司马昭为大都督,总领三路军马。

是年冬十二月,司马昭兵至东吴边界,屯住人马,唤王昶、胡遵、毋丘俭到帐中计议曰:“东吴最紧要处,惟东兴郡也。今他筑起大堤,左右又筑两城,以防巢湖后面攻击,诸公须要仔细。”遂令王昶、毋丘俭各引一万兵,列在左右:“且勿进发;待取了东兴郡,那时一齐进兵。”昶、俭二人受令而去。昭又令胡遵为先锋,总领三路兵前去:“先搭浮桥,取东兴大堤;若夺得左右二城,便是大功。”遵领兵来搭浮桥。

却说吴太傅诸葛恪,听知魏兵三路而来,聚众商议。平北将军丁奉曰:“东兴乃东吴紧要处所,若有失,则南郡、武昌危矣。”恪曰:“此论正合吾意。公可就引三千水兵从江中去,吾随后令吕据、唐咨、留赞各引一万马步兵,分三路来接应。但听连珠炮响,一齐进兵。吾自引大兵后至。”丁奉得令,即引三千水兵,分作三十只船,望东兴而来。

却说胡遵渡过浮桥,屯军于堤上,差桓嘉、韩综攻打二城。左城中乃吴将全端守把,右城中乃吴将留略守把。此二城高峻坚固,急切攻打不下。全、留二人见魏兵势大,不敢出战,死守城池。胡遵在徐塘下寨。时值严寒,天降大雪,胡遵与众将设席高会。忽报水上有三十只战船来到。遵出寨视之,见船将次傍岸,每船上约有百人。遂还帐中,谓诸将曰:“不过三千人耳,何足惧哉!”只令部将哨探,仍前饮酒。

丁奉将船一字儿抛在水上,乃谓部将曰:“大丈夫立功名,取富贵,正在今日!”遂令众军脱去衣甲,卸了头盔,不用长枪大戟,止带短刀。魏兵见之大笑,更不准备。忽然连珠炮响了三声,丁奉扯刀当先,一跃上岸。众军皆拔短刀,随奉上岸,砍入魏寨,魏兵措手不及。韩综急拔帐前大戟迎之,早被丁奉抢入怀内,手起刀落,砍翻在地。桓嘉从左边转出,忙绰枪刺丁奉,被奉挟住枪杆。嘉弃枪而走,奉一刀飞去,正中左肩,嘉望后便倒。奉赶上,就以枪刺之。三千吴兵,在魏寨中左冲右突。胡遵急上马夺路而走。魏兵齐奔上浮桥,浮桥已断,大半落水而死;杀倒在雪地者,不知其数。车仗马匹军器,皆被吴兵所获。司马昭、王昶、毋丘俭听知东兴兵败,亦勒兵而退。却说诸葛恪引兵至东兴,收兵赏劳了毕,乃聚诸将曰:“司马昭兵败北归,正好乘势进取中原。”遂一面遣人赍书入蜀,求姜维进兵攻其北,许以平分天下;一面起大兵二十万,来伐中原。临行时,忽见一道白气,从地而起,遮断三军,对面不见。蒋延曰:“此气乃白虹也,主丧兵之兆。太傅只可回朝,不可伐魏。”恪大怒曰:“汝安敢出不利之言,以慢吾军心!”叱武士斩之。众皆告免,恪乃贬蒋延为庶人,仍催兵前进。丁奉曰:“魏以新城为总隘口,若先取得此城,司马师破胆矣。”恪大喜,即趱兵直至新城。守城牙门将军张特,见吴兵大至,闭门坚守。恪令兵四面围定。早有流星马报入洛阳。主簿虞松告司马师曰:“今诸葛恪困新城,且未可与战。吴兵远来,人多粮少,粮尽自走矣。待其将走,然后击之,必得全胜。但恐蜀兵犯境,不可不防。”师然其言,遂令司马昭引一军助郭淮防姜维;毋丘俭、胡遵拒住吴兵。

却说诸葛恪连月攻打新城不下,下令众将:“并力攻城,怠慢者立斩。”于是诸将奋力攻打。城东北角将陷。张特在城中定下一计:乃令一舌辩之士,赍捧册籍,赴吴寨见诸葛恪,告曰:“魏国之法:若敌人困城,守城将坚守一百日,而无救兵至,然后出城降敌者,家族不坐罪。今将军围城已九十余日;望乞再容数日,某主将尽率军民出城投降。今先具册

籍呈上。”恪深信之,收了军马,遂不攻城。原来张特用缓兵之计,哄退吴兵,遂拆城中房屋,于破城处修补完备,乃登城大骂曰:“吾城中尚有半年之粮,岂肯降吴狗耶!尽战无妨!”恪大怒,催兵打城。城上乱箭射下。恪额上正中一箭,翻身落马。诸将救起还寨,金疮举发。众军皆无战心;又因天气亢炎,军士多病。恪金疮稍可,欲催兵攻城。营吏告曰:“人人皆病,安能战乎?”恪大怒曰:“再说病者斩之!”众军闻知,逃者无数。忽报都督蔡林引本部军投魏去了。恪大惊,自乘马遍视各营,果见军士面色黄肿,各带病容。遂勒兵还吴。早有细作报知毋丘俭。俭尽起大兵,随后掩杀。

吴兵大败而归,恪甚羞惭,托病不朝。吴主孙亮自幸其宅问安,文武官僚皆来拜见。恪恐人议论,先搜求众官将过失,轻则发遣边方,重则斩首示众。于是内外官僚,无不悚惧。又令心腹将张约、朱恩管御林军。以为牙爪。却说孙峻字子远,乃孙坚弟孙静曾孙,孙恭之子也;孙权存日,甚爱之,命掌御林军马。今闻诸葛恪令张约、朱恩二人掌御林军,夺其权,心中大怒。太常卿滕胤,素与诸葛恪有隙,乃乘间说峻曰:“诸葛恪专权恣虐,杀害公卿,将有不臣之心。公系宗室,何不早图之?”峻曰:“我有是心久矣;今当即奏天子,请旨诛之。”于是孙峻、滕胤入见吴主孙亮,密奏其事。亮曰:“朕见此人,亦甚恐怖;常欲除之,未得其便。今卿等果有忠义,可密图之。”胤曰:“陛下可设席召恪,暗伏武士于壁衣中,掷杯为号,就席间杀之,以绝后患。”亮从之。

却说诸葛恪自兵败回朝,托病居家,心神恍惚。一日,偶出中堂,忽见一人穿麻挂孝而入。恪叱问之,其人大惊无措。恪令拿下拷问,其人告曰:“某因新丧父亲,入城请僧追荐;初见是寺院而入,却不想是太傅之府。却怎生来到此处也?”恪大怒,召守门军士问之。军士告曰:“某等数十人,皆荷戈把门,未尝暂离,并不见一人入来。”恪大怒,尽数斩之。是夜,恪睡卧不安,忽听得正堂中声响如霹雳。恪自出视之,见中梁折为两段。恪惊归寝室,忽然一阵阴风起处,见所杀披麻人与守门军士数十人,各提头索命。恪惊倒在地,良久方苏。次早洗面,闻水甚血臭。恪叱侍婢,连换数十盆,皆臭无异。恪正惊疑间,忽报天子有使至,宣太傅赴宴。

恪令安排车仗。方欲出府,有黄犬衔住衣服,嘤嘤作声,如哭之状。恪怒曰:“犬戏我也!”叱左右逐去之,遂乘车出府。行不数步,见车前一道白虹,自地而起,如白练冲天而去。恪甚惊怪,心腹将张约进车前密告曰;“今日宫中设宴,未知好歹,主公不可轻入。”恪听罢,便令回车。行不到十余步,孙峻、滕胤乘马至车前曰:“太傅何故便回?”恪曰:“吾忽然腹痛,不可见天子。”胤曰:“朝廷为太傅军回,不曾面叙,故特设宴相召,兼议大事。太傅虽感贵恙,还当勉强一行。”恪从其言,遂同孙峻、滕胤入宫,张约亦随入。

恪见吴主孙亮,施礼毕,就席而坐。亮命进酒,恪心疑,辞曰:“病躯不胜杯酌。”孙峻曰:“太傅府中常服药酒,可取饮乎?”恪曰:“可也。”遂令从人回府取自制药酒到,恪方才放心饮之。酒至数巡,吴主孙亮托事先起。孙峻下殿,脱了长服,着短衣,内披环甲,手提利刃,上殿大呼曰:“天子有诏诛逆贼!”诸葛恪大惊,掷杯于地,欲拔剑迎之,头已落地。张约见峻斩恪,挥刀来迎。峻急闪过,刀尖伤其左指。峻转身一刀,砍中张约右臂。武士一齐拥出,砍倒张约,剁为肉泥。孙峻一面令武士收恪家眷,一面令人将张约并诸葛恪尸首,用芦席包裹,以小车载出,弃于城南门外石子岗乱冢坑内。却说诸葛恪之妻正在房中心神恍惚,动止不宁,忽一婢女入房。恪妻问曰:“汝遍身如何血臭?”其婢忽然反目切齿,飞身跳跃,头撞屋梁,口中大叫:“吾乃诸葛恪也!被奸贼孙峻谋杀!”恪合家老幼,惊惶号哭。不一时,军马至,围住府第,将恪全家老幼,俱缚至市曹斩首。时吴建兴二年冬十月也。昔诸葛瑾存日,见恪聪明尽显于外,叹曰:“此子非保家之主也!”又魏光禄大夫张缉,曾对司马师曰:“诸葛恪不久死矣。”师问其故,缉曰:“威震其主,何能久乎?”至此果中其言。却说孙峻杀了诸葛恪,吴主孙亮封峻为丞相、大将军、富春侯,总督中外诸军事。自此权柄尽归孙峻矣。

且说姜维在成都,接得诸葛恪书,欲求相助伐魏,遂入朝,奏准后主,复起大兵,北伐中原。正是:一度兴师未奏绩,两番讨贼欲成功。未知胜负如何,且看下文分解。