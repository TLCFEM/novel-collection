\chapter{玄德进位汉中王~云长攻拔襄阳郡}

却说曹操退兵至斜谷,孔明料他必弃汉中而走,故差马超等诸将,分兵十数路,不时攻劫。因此操不能久住;又被魏延射了一箭,急急班师。三军锐气堕尽。前队才行,两下火起,乃是马超伏兵追赶。曹兵人人丧胆。操令军士急行,晓夜奔走无停;直至京兆,方始安心。

且说玄德命刘封、孟达、王平等,攻取上庸诸郡,申耽等闻操已弃汉中而走,遂皆投降,玄德安民已定,大赏三军,人心大悦。于是众将皆有推尊玄德为帝之心;未敢径启,却来禀告诸葛军师,孔明曰:“吾意已有定夺了。”随引法正等入见玄德,曰:“今曹操专权,百姓无主;主公仁义著于天下,今已抚有两川之地,可以应天顺人,即皇帝位,名正言顺,以讨国贼。事不宜迟,便请择吉。”玄德大惊曰:“军师之言差矣。刘备虽然汉之宗室,乃臣子也;若为此事,是反汉矣。”孔明曰:“非也。方今天下分崩,英雄并起,各霸一方,四海才德之士,舍死亡生而事其上者,皆欲攀龙附凤,建立功名也。今主公避嫌守义,恐失众人之望。愿主公熟思之。”玄德曰:“要吾僭居尊位,吾必不敢。可再商议长策。”诸将齐言曰:“主公若只推却,众心解矣。”孔明曰:“主公平生以义为本,未肯便称尊号。今有荆襄、两川之地,可暂为汉中王。”玄德曰:“汝等虽欲尊吾为王,不得天子明诏,是僭也。”孔明曰:“今宜从权,不可拘执常理。”张飞大叫曰:“异姓之人,皆欲为君何况哥哥乃汉朝宗派!莫说汉中王,就称皇帝,有何不可!”玄德叱曰:“汝勿多言!”孔明曰:“主公宜从权变,先进位汉中王,然后表奏天子,未为迟也。”

玄德再三推辞不过,只得依允。建安二十四年秋七月,筑坛于沔阳,方圆九里,分布五方,各设旌旗仪仗。群臣皆依次序排列。许靖、法正请玄德登坛,进冠冕玺绶讫,面南而坐,受文武官员拜贺为汉中王。子刘禅,立为王世子。封许靖为太傅,法正为尚书令;诸葛亮为军师,总理军国重事。封关羽、张飞、赵云、马超、黄忠为五虎大将,魏延为汉中太守。其余各拟功勋定爵。玄德既为汉中王,遂修表一道,差人赍赴许都。表曰:“备以具臣之才,荷上将之任,总督三军,奉辞于外;不能扫除寇难,靖匡王室,久使陛下圣教陵迟,六合之内,否而未泰:惟忧反侧,疢如疾首。曩者董卓,伪为乱阶。自是之后,群凶纵横,残剥海内。赖陛下圣德威临,人臣同应,或忠义奋讨,或上天降罚,暴逆并殪,以渐冰消。惟独曹操,久未枭除,侵擅国权,恣心极乱。臣昔与车骑将军董承,图谋讨操,机事不密,承见陷害。臣播越失据,忠义不果,遂得使操穷凶极逆:主后戮杀,皇子鸩害。虽纠合同盟,念在奋力;懦弱不武,历年未效。常恐殒没,辜负国恩;寤寐永叹,夕惕若厉。今臣群僚以为:在昔虞书敦叙九族,庶明励翼;帝王相传,此道不废;周监二代,并建诸姬,实赖晋、郑夹辅之力;高祖龙兴,尊王子弟,大启九国,卒斩诸吕,以安大宗。今操恶直丑正,实繁有徒,包藏祸心,篡盗已显;既宗室微弱,帝族无位,斟酌古式,依假权宜:上臣为大司马、汉中王。臣伏自三省:受国厚恩,荷任一方,陈力未效,所获已过,不宜复忝高位,以重罪谤。群僚见逼,迫臣以义。臣退惟寇贼不枭,国难未已;宗庙倾危,社稷将坠:诚臣忧心碎首之日。若应权通变,以宁静圣朝,虽赴水火,所不得辞。辄顺众议,拜受印玺,以崇国威。仰惟爵号,位高宠厚;俯思报效,忧深责重。惊怖惕息,如临于谷。敢不尽力输诚,奖励六师,率齐群义,应天顺时,以宁社稷。谨拜表以闻。”

表到许都,曹操在邺郡闻知玄德自立汉中王,大怒曰:“织席小儿,安敢如此!吾誓灭之!”即时传令,尽起倾国之兵,赴两川与汉中王决雌雄。一人出班谏曰:“大王不可因一时之怒,亲劳车驾远征。臣有一计,不须张弓只箭,令刘备在蜀自受其祸;待其兵衰力尽,只须一将往征之,便可成功。”操视其人,乃司马懿也。操喜问曰:“仲达有何高见?”懿曰:“江东孙权,以妹嫁刘备,而又乘间窃取回去;刘备又据占荆州不还:彼此俱有切齿之恨。今可差一舌辩之士,赍书往说孙权,使兴兵取荆州;刘备必发两川之兵以救荆州。那时大王兴兵去取汉川,令刘备首尾不能相救,势必危矣。”操大喜,即修书令满宠为使,星夜投江东来见孙权。

权知满宠到,遂与谋士商议。张昭进曰:“魏与吴本无仇;前因听诸葛之说词,致两家连年征战不息,生灵遭其涂炭。今满伯宁来,必有讲和之意,可以礼接之。”权依其言,令众谋士接满宠入城相见。礼毕,权以宾礼待宠。宠呈上操书,曰:“吴、魏自来无仇,皆因刘备之故,致生衅隙。魏王差某到此,约将军攻取荆州,魏王以兵临汉川,首尾夹击。破刘之后,共分疆土,誓不相侵。”孙权览书毕,设筵相待满宠,送归馆舍安歇。权与众谋士商议。顾雍曰:“虽是说词,其中有理。今可一面送满宠回,约会曹操,首尾相击;一面使人过江探云长动静,方可行事。”诸葛瑾曰:“某闻云长自到荆州,刘备娶与妻室,先生一子,次生一女。其女尚幼,未许字人。某愿往与主公世子求婚。若云长肯许,即与云长计议共破曹操;若云长不肯,然后助曹取荆州。”孙权用其谋,先送满宠回许都;却遣诸葛瑾为使,投荆州来。入城见云长,礼毕。云长曰:“子瑜此来何意?”瑾曰:“特来求结两家之好:吾主吴侯有一子,甚聪明;闻将军有一女,特来求亲。两家结好,并力破曹。此诚美事,请君侯思之。”云长勃然大怒曰:“吾虎女安肯嫁犬子乎!不看汝弟之面,立斩汝首!再休多言!”遂唤左右逐出。瑾抱头鼠窜,回见吴侯;不敢隐匿,遂以实告。权大怒曰:“何太无礼耶!”便唤张昭等文武官员,商议取荆州之策。步骘曰:“曹操久欲篡汉,所惧者刘备也;今遣使来令吴兴兵吞蜀,此嫁祸于吴也。”权曰:“孤亦欲取荆州久矣。”骘曰:“今曹仁现屯兵于襄阳、樊城,又无长江之险,旱路可取荆州;如何不取,却令主公动兵?只此便见其心。主公可遣使去许都见操,令曹仁旱路先起兵取荆州,云长必掣荆州之兵而取樊城。若云长一动,主公可遣一将,暗取荆州,一举可得矣。”权从其议,即时遣使过江,上书曹操,陈说此事。操大喜,发付使者先回,随遣满宠往樊城助曹仁,为参谋官,商议动兵;一面驰檄东吴,令领兵水路接应,以取荆州。

却说汉中王令魏延总督军马,守御东川。遂引百官回成都。差官起造宫庭,又置馆舍,自成都至白水,共建四百余处馆舍亭邮。广积粮草。多造军器,以图进取中原。细作人探听得曹操结连东吴,欲取荆州,即飞报入蜀。汉中王忙请孔明商议。孔明曰:“某已料曹操必有此谋;然吴中谋士极多,必教操令曹仁先兴兵矣。”汉中王曰:“依此如之奈何?”孔明曰:“可差使命就送官诰与云长,令先起兵取樊城,使敌军胆寒,自然瓦解矣。”汉中王大喜,即差前部司马费诗为使,赍捧诰命投荆州来。云长出郭,迎接入城。至公廨礼毕,云长问曰:“汉中王封我何爵?”诗曰:“五虎大将之首。”云长问:“那五虎将?”诗曰:“关、张、赵、马、黄是也。”云长怒曰:“翼德吾弟也;孟起世代名家;子龙久随吾兄,即吾弟也:位与吾相并,可也。黄忠何等人,敢与吾同列?大丈夫终不与老卒为伍?”遂不肯受印。诗笑曰:“将军差矣。昔萧何、曹参与高祖同举大事,最为亲近,而韩信乃楚之亡将也;然信位为王,居萧、曹之上,未闻萧、曹以此为怨。今汉中王虽有五虎将之封,而与将军有兄弟之义,视同一体。将军即汉中王,汉中王即将军也。岂与诸人等哉?将军受汉中王厚恩,当与同休戚、共祸福,不宜计较官号之高下。愿将军熟思之。”云长大悟,乃再拜曰:“某之不明,非足下见教,几误大事。”即拜受印绶。

费诗方出王旨,令云长领兵取樊城。云长领命,即时便差傅士仁、糜芳二人为先锋,先引一军于荆州城外屯扎;一面设宴城中,款待费诗。饮至二更,忽报城外寨中火起。云长急披挂上马,出城看时,乃是傅士仁、糜芳饮酒,帐后遗火,烧着火炮,满营撼动,把军器粮草,尽皆烧毁。云长引兵救扑,至四更方才火灭。云长入城,召傅士仁、糜芳责之曰:“吾令汝二人作先锋,不曾出师,先将许多军器粮草烧毁,火炮打死本部军人。如此误事,要你二人何用?”叱令斩之。费诗告曰:“未曾出师,先斩大将,于军不利。可暂免其罪。”云长怒气不息,叱二人曰:“吾不看费司马之面,必斩汝二人之首!”乃唤武士各杖四十,摘去先锋印绶,罚糜芳守南郡,傅士仁守公安;且曰:“若吾得胜回来之日,稍有差池,二罪俱罚!”二人满面羞惭,喏喏而去。

云长便令廖化为先锋,关平为副将,自总中军,马良、伊籍为参谋,一同征进。先是,有胡华之子胡班,到荆州来投降关公;公念其旧日相救之情,甚爱之;令随费诗入川,见汉中王受爵。费诗辞别关公,带了胡班,自回蜀中去了。

且说关公是日祭了“帅”字大旗,假寐于帐中。忽见一猪,其大如牛,浑身黑色,奔入帐中,径咬云长之足。云长大怒,急拔剑斩之,声如裂帛。霎然惊觉,乃是一梦。便觉左足阴阴疼痛,心中大疑。唤关平至,以梦告之。平对曰:“猪亦有龙象。龙附足,乃升腾之意,不必疑忌。”云长聚多官于帐下,告以梦兆。或言吉祥者,或言不祥者,众论不一。云长曰:“吾大丈夫,年近六旬,即死何憾!”正言间,蜀使至,传汉中王旨,拜云长为前将军,假节钺,都督荆襄九郡事。云长受命讫,众官拜贺曰:“此足见猪龙之瑞也。”于是云长坦然不疑,遂起兵奔襄阳大路而来。

曹仁正在城中,忽报云长自领兵来。仁大惊,欲坚守不出,副将翟元曰:“今魏王令将军约会东吴取荆州;今彼自来,是送死也,何故避之!”参谋满宠谏曰:“吾素知云长勇而有谋,未可轻敌。不如坚守,乃为上策。”骁将夏侯存曰:“此书生之言耳。岂不闻水来土掩,将至兵迎?我军以逸待劳,自可取胜。”曹仁从其言,令满宠守樊城,自领兵来迎云长。

云长知曹兵来,唤关平、廖化二将,受计而往。与曹兵两阵对圆,廖化出马搦战。翟元出迎。二将战不多时,化诈败,拨马便走,翟元从后追杀,荆州兵退二十里。次日,又来搦战。夏侯存、翟元一齐出迎,荆州兵又败,又追杀二十余里。忽听得背后喊声大震,鼓角齐鸣。曹仁急命前军速回,背后关平、廖化杀来,曹兵大乱。曹仁知是中计,先掣一军飞奔襄阳;离城数里,前面绣旗招飐,云长勒马横刀,拦住去路。曹仁胆战心惊,不敢交锋,望襄阳斜路而走。云长不赶。须臾,夏侯存军至,见了云长,大怒,便与云长交锋,只一合,被云长砍死。翟元便走,被关平赶上,一刀斩之。乘势追杀,曹兵大半死于襄江之中。曹仁退守樊城。

云长得了襄阳,赏军抚民。随军司马王甫曰:“将军一鼓而下襄阳,曹兵虽然丧胆,然以愚意论之:今东吴吕蒙屯兵陆口,常有吞并荆州之意;倘率兵径取荆州,如之奈何?”云长曰:“吾亦念及此。汝便可提调此事:去沿江上下,或二十里,或三十里,选高阜处置一烽火台,每台用五十军守之;倘吴兵渡江,夜则明火,昼则举烟为号。吾当亲往击之。”王甫曰:“糜芳、傅士仁守二隘口,恐不竭力;必须再得一人以总督荆州。”云长曰:“吾已差治中潘浚守之,有何虑焉?”甫曰:“潘浚平生多忌而好利,不可任用。可差军前都督粮料官赵累代之。赵累为人忠城廉直。若用此人,万无一失。”云长曰:“吾素知潘浚为人。今既差定,不必更改。赵累现掌粮料,亦是重事。汝勿多疑,只与我筑烽火台去。”王甫怏怏拜辞而行。云长令关平准备船只渡襄江,攻打樊城。

却说曹仁折了二将,退守樊城,谓满宠曰:“不听公言,兵败将亡,失却襄阳,如之奈何?”宠曰:“云长虎将,足智多谋,不可轻敌,只宜坚守。”正言间,人报云长渡江而来,攻打樊城。仁大惊,宠曰:“只宜坚守。”部将吕常奋然曰:“某乞兵数千,愿当来军于襄江之内。”宠谏曰:“不可。”吕常怒曰:“据汝等文官之言,只宜坚守,何能退敌?岂不闻兵法云:军半渡可击。今云长军半渡襄江,何不击之?若兵临城下,将至壕边,急难抵当矣。”仁即与兵二千,令吕常出樊城迎战。吕常来至江口,只见前面绣旗开处,云长横刀出马。吕常却欲来迎,后面众军见云长神威凛凛,不战先走,吕常喝止不住。云长混杀过来,曹兵大败,马步军折其大半,残败军奔入樊城。曹仁急差人求救,使命星夜至长安,将书呈上曹操,言:“云长破了襄阳,现围樊城甚急。望拨大将前来救援。”曹操指班部内一人而言曰:“汝可去解樊城之围。”其人应声而出。众视之,乃于禁也。禁曰:“某求一将作先锋,领兵同去。”操又问众人曰:“谁敢作先锋?”一人奋然出曰:“某愿施犬马之劳,生擒关某,献于麾下。”操观之大喜。正是:未见东吴来伺隙,先看北魏又添兵。未知此人是谁,且看下文分解。