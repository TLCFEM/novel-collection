\chapter{议温明董卓叱丁原~馈金珠李肃说吕布}

且说曹操当日对何进曰:“宦官之祸,古今皆有;但世主不当假之权宠,使至于此。若欲治罪,当除元恶,但付一狱吏足矣,何必纷纷召外兵乎?欲尽诛之,事必宣露。吾料其必败也。”何进怒曰:“孟德亦怀私意耶?”操退曰:“乱天下者,必进也。”进乃暗差使命,赍密诏星夜往各镇去。

却说前将军、鳌乡侯、西凉刺史董卓,先为破黄巾无功,朝议将治其罪,因贿赂十常侍幸免;后又结托朝贵,遂任显官,统西州大军二十万,常有不臣之心。是时得诏大喜,点起军马,陆续便行;使其婿中郎将牛辅;守住陕西,自己却带李傕、郭汜、张济、樊稠等提兵望洛阳进发。

卓婿谋士李儒曰:“今虽奉诏,中间多有暗味。何不差人上表,名正言顺,大事可图。”卓大喜,遂上表。其略曰:“窃闻天下所以乱逆不止者,皆由黄门常侍张让等侮慢天常之故。臣闻扬汤止沸,不如去薪;溃痈虽痛,胜于养毒。臣敢鸣钟鼓入洛阳,请除让等。社稷幸甚!天下幸甚!”何进得表,出示大臣。侍御史郑泰谏曰:“董卓乃豺狼也,引入京城,必食人矣。”进曰:“汝多疑,不足谋大事。”卢植亦谏曰:“植素知董卓为人,面善心狠;一入禁庭,必生祸患。不如止之勿来,免致生乱。”进不听,郑泰、卢植皆弃官而去。朝廷大臣,去者大半。进使人迎董卓于渑池,卓按兵不动。

张让等知外兵到,共议曰:“此何进之谋也;我等不先下手,皆灭族矣。”乃先伏刀斧手五十人于长乐宫嘉德门内,入告何太后曰:“今大将军矫诏召外兵至京师,欲灭臣等,望娘娘垂怜赐救。”太后曰:“汝等可诣大将军府谢罪。”让曰:“若到相府,骨肉齑粉矣。望娘娘宣大将军入宫谕止之。如其不从,臣等只就娘娘前请死。”太后乃降诏宣进。

进得诏便行。主簿陈琳谏曰:“太后此诏,必是十常侍之谋,切不可去。去必有祸。”进曰:“太后诏我,有何祸事?”袁绍曰:“今谋已泄,事已露,将军尚欲入宫耶?”曹操曰:“先召十常侍出,然后可入。”进笑曰:“此小儿之见也。吾掌天下之权,十常侍敢待如何?”绍曰:“公必欲去,我等引甲士护从,以防不测。”于是袁绍、曹操各选精兵五百,命袁绍之弟袁术领之。袁术全身披挂,引兵布列青琐门外。绍与操带剑护送何进至长乐宫前。黄门传懿旨云:“太后特宣大将军,余人不许辄入。”将袁绍、曹操等都阻住宫门外。

何进昂然直入。至嘉德殿门,张让、段珪迎出,左右围住,进大惊。让厉声责进曰:“董后何罪,妄以鸩死?国母丧葬,托疾不出!汝本屠沽小辈,我等荐之天子,以致荣贵;不思报效,欲相谋害,汝言我等甚浊,其清者是谁?”进慌急,欲寻出路,宫门尽闭,伏甲齐出,将何进砍为两段。后人有诗叹之曰;“汉室倾危天数终,无谋何进作三公。几番不听忠臣谏,难免宫中受剑锋。”

让等既杀何进,袁绍久不见进出,乃于宫门外大叫曰:“请将军上车!”让等将何进首级从墙上掷出,宣谕曰:“何进谋反,已伏诛矣!其余胁从,尽皆赦宥。”袁绍厉声大叫:“阉官谋杀大臣!诛恶党者前来助战!”何进部将吴匡,便于青琐门外放起火来。袁术引兵突入宫庭,但见阉官,不论大小,尽皆杀之。袁绍、曹操斩关入内。赵忠、程旷、夏恽、郭胜四个被赶至翠花楼前,剁为肉泥。宫中火焰冲天。张让、段珪、曹节、侯览将太后及太子并陈留王劫去内省,从后道走北宫。时卢植弃官未去,见宫中事变,擐甲持戈,立于阁下。遥见段珪拥逼何后过来,植大呼曰:“段珪逆贼,安敢劫太后!”段珪回身便走。太后从窗中跳出,植急救得免。吴匡杀入内庭,见何苗亦提剑出。匡大呼曰:“何苗同谋害兄,当共杀之!”众人俱曰:“愿斩谋兄之贼!”苗欲走,四面围定。砍为齑粉。绍复令军士分头来杀十常侍家属,不分大小,尽皆诛绝,多有无须者误被杀死。曹操一面救灭宫中之火,请何太后权摄大事,遣兵追袭张让等,寻觅少帝。

且说张让、段珪劫拥少帝及陈留王,冒烟突火,连夜奔走至北邙山。约二更时分,后面喊声大举,人马赶至;当前河南中部掾吏闵贡,大呼“逆贼休走!”张让见事急,遂投河而死。帝与陈留王未知虚实,不敢高声,伏于河边乱草之内。军马四散去赶,不知帝之所在。帝与王伏至四更,露水又下,腹中饥馁,相挤而哭;又怕人知觉,吞声草莽之中。陈留王曰:“此间不可久恋,须别寻活路。”于是二人以衣相结,爬上岸边。满地荆棘,黑暗之中,不见行路。正无奈何,忽有流萤千百成群,光芒照耀,只在帝前飞转。陈留王曰:“此天助我兄弟也!”遂随萤火而行,渐渐见路。行至五更,足痛不能行,山冈边见一草堆,帝与王卧于草堆之畔。草堆前面是一所庄院。庄主是夜梦两红日坠于庄后,惊觉,披衣出户,四下观望,见庄后草堆上红光冲天,慌忙往视,却是二人卧于草畔。庄主问曰:“二少年谁家之子?”帝不敢应。陈留王指帝曰:“此是当今皇帝,遭十常侍之乱,逃难到此。吾乃皇弟陈留王也。”庄主大惊,再拜曰:“臣先朝司徒崔烈之弟崔毅也。因见十常侍卖官嫉贤,故隐于此。”遂扶帝入庄,跪进酒食。却说闵贡赶上段珪,拿住问:“天子何在?”珪言:“已在半路相失,不知何往。”贡遂杀段珪,悬头于马项下,分兵四散寻觅;自己却独乘一马。随路追寻,偶至崔毅庄,毅见首级,问之,贡说详细,崔毅引贡见帝,君臣痛哭。贡曰:“国不可一日无君,请陛下还都。”崔毅庄上止有瘦马一匹,备与帝乘。贡与陈留王共乘一马。离庄而行,不到三里,司徒王允,太尉杨彪、左军校尉淳于琼、右军校尉赵萌、后军校尉鲍信、中军校尉袁绍,一行人众,数百人马,接着车驾。君臣皆哭。先使人将段珪首级往京师号令,另换好马与帝及陈留王骑坐,簇帝还京。先是洛阳小儿谣曰:“帝非帝,王非王,千乘万骑走北邙。”至此果应其谶。

车驾行不到数里,忽见旌旗蔽日,尘土遮天,一枝人马到来。百官失色,帝亦大惊。袁绍骤马出问:“何人?”绣旗影里,一将飞出,厉声问:“天子何在?”帝战栗不能言。陈留王勒马向前,叱曰:“来者何人?”卓曰:“西凉刺史董卓也。”陈留王曰:“汝来保驾耶,汝来劫驾耶?”卓应曰:“特来保驾。”陈留王曰:“既来保驾,天子在此,何不下马?”卓大惊,慌忙下马,拜于道左。陈留王以言抚慰董卓,自初至终,并无失语。卓暗奇之,已怀废立之意。是日还宫,见何太后,俱各痛哭。检点宫中,不见了传国玉玺。

董卓屯兵城外,每日带铁甲马军入城,横行街市,百姓惶惶不安。卓出入宫庭,略无忌惮。后军校尉鲍信,来见袁绍,言董卓必有异心,可速除之。绍曰:“朝廷新定,未可轻动。”鲍信见王允,亦言其事。允曰:“且容商议。”信自引本部军兵,投泰山去了。董卓招诱何进兄弟部下之兵,尽归掌握。私谓李儒曰:“吾欲废帝立陈留王,何如?”李儒曰:“今朝廷无主,不就此时行事,迟则有变矣。来日于温明园中,召集百官,谕以废立;有不从者斩之,则威权之行,正在今日。”卓喜。次日大排筵会,遍请公卿。公卿皆惧董卓,谁敢不到。卓待百官到了,然后徐徐到园门下马,带剑入席。酒行数巡,卓教停酒止乐,乃厉声曰:“吾有一言,众官静听。”众皆侧耳。卓曰:“天子为万民之主,无威仪不可以奉宗庙社稷。今上懦弱,不若陈留王聪明好学,可承大位。吾欲废帝,立陈留王,诸大臣以为何如?”诸官听罢,不敢出声。

座上一人推案直出,立于筵前,大呼:“不可!不可!汝是何人,敢发大语?天子乃先帝嫡子,初无过失,何得妄议废立!汝欲为篡逆耶?”卓视之,乃荆州刺史丁原也。卓怒叱曰:“顺我者生,逆我者死!”遂掣佩剑欲斩丁原。时李儒见丁原背后一人,生得器宇轩昂,威风凛凛,手执方天画戟,怒目而视。李儒急进曰:“今日饮宴之处,不可谈国政;来日向都堂公论未迟。”众人皆劝丁原上马而去。

卓问百官曰:“吾所言,合公道否?”卢植曰:“明公差矣。昔太甲不明,伊尹放之于桐宫;昌邑王登位方二十七日,造恶三千余条,故霍光告太庙而废之。今上虽幼,聪明仁智,并无分毫过失。公乃外郡刺史,素未参与国政,又无伊、霍之大才,何可强主废立之事?圣人云:有伊尹之志则可,无伊尹之志则篡也。”卓大怒,拔剑向前欲杀植。侍中蔡邕、议郎彭伯谏曰:“卢尚书海内人望,今先害之,恐天下震怖。”卓乃止。司徒王允曰:“废立之事,不可酒后相商,另日再议。”于是百官皆散。卓按剑立于园门,忽见一人跃马持戟,于园门外往来驰骤。卓问李儒:“此何人也?”儒曰:“此丁原义儿:姓吕,名布,字奉先者也。主公且须避之。”卓乃入园潜避。次日,人报丁原引军城外搦战。卓怒,引军同李儒出迎。两阵对圆,只见吕布顶束发金冠,披百花战袍,擐唐猊铠甲,系狮蛮宝带,纵马挺戟,随丁建阳出到阵前。建阳指卓骂曰:“国家不幸,阉官弄权,以致万民涂炭。尔无尺寸之功,焉敢妄言废立,欲乱朝廷!”董卓未及回言,吕布飞马直杀过来。董卓慌走,建阳率军掩杀。卓兵大败,退三十余里下寨,聚众商议。卓曰:“吾观吕布非常人也。吾若得此人,何虑天下哉!”帐前一人出曰:“主公勿忧。某与吕布同乡,知其勇而无谋,见利忘义。某凭三寸不烂之舌,说吕布拱手来降,可乎?”卓大喜,观其人,乃虎贲中郎将李肃也。卓曰:“汝将何以说之?”肃曰:“某闻主公有名马一匹,号曰赤兔,日行千里。须得此马,再用金珠,以利结其心。某更进说词,吕布必反丁原,来投主公矣。”卓问李儒曰:“此言可乎?”儒曰:“主公欲破天下,何惜一马!”卓欣然与之,更与黄金一千两、明珠数十颗、玉带一条。李肃赍了礼物,投吕布寨来。伏路军人围住。肃曰:“可速报吕将军,有故人来见。”军人报知,布命入见。肃见布曰:“贤弟别来无恙!”布揖曰:“久不相见,今居何处?”肃曰:“现任虎贲中郎将之职。闻贤弟匡扶社稷,不胜之喜。有良马一匹,日行千里,渡水登山,如履平地,名曰赤兔:特献与贤弟,以助虎威。”布便令牵过来看。果然那马浑身上下,火炭般赤,无半根杂毛;从头至尾,长一丈;从蹄至项,高八尺;嘶喊咆哮,有腾空入海之状。后人有诗单道赤兔马曰:“奔腾千里荡尘埃,渡水登山紫雾开。掣断丝缰摇玉辔,火龙飞下九天来。”布见了此马,大喜,谢肃曰:“兄赐此龙驹,将何以为报?”肃曰:“某为义气而来。岂望报乎!”布置酒相待。酒甜,肃曰:“肃与贤弟少得相见;令尊却常会来。”布曰:“兄醉矣!先父弃世多年,安得与兄相会?”肃大笑曰:“非也!某说今日丁刺史耳。”布惶恐曰:“某在丁建阳处,亦出于无奈。”肃曰:“贤弟有擎天驾海之才,四海孰不钦敬?功名富贵,如探囊取物,何言无奈而在人之下乎?”布曰:“恨不逢其主耳。”肃笑曰:“良禽择木而栖,贤臣择主而事。见机不早,悔之晚矣。”布曰:“兄在朝廷,观何人为世之英雄?”肃曰:“某遍观群臣,皆不如董卓。董卓为人敬贤礼士,赏罚分明,终成大业。”布曰:“某欲从之,恨无门路。”肃取金珠、玉带列于布前。布惊曰:“何为有此?”肃令叱退左右,告布曰:“此是董公久慕大名,特令某将此奉献。赤兔马亦董公所赠也。”布曰:“董公如此见爱,某将何以报之?”肃曰:“如某之不才,尚为虎贲中郎将;公若到彼,贵不可言。”布曰:“恨无涓埃之功,以为进见之礼。”肃曰:“功在翻手之间,公不肯为耳。”布沈吟良久曰:“吾欲杀丁原,引军归董卓,何如?”肃曰:“贤弟若能如此,真莫大之功也!但事不宜迟,在于速决。”布与肃约于明日来降,肃别去。

是夜二更时分,布提刀径入丁原帐中。原正秉烛观书,见布至,曰:“吾儿来有何事故?”布曰:“吾堂堂丈夫,安肯为汝子乎!”原曰:“奉先何故心变?”布向前,一刀砍下丁原首级,大呼左右:“丁原不仁,吾已杀之。肯从吾者在此,不从者自去!”军士散其大半。次日,布持丁原首级,往见李肃。肃遂引布见卓。卓大喜,置酒相待。卓先下拜曰:“卓今得将军,如旱苗之得甘雨也。”布纳卓坐而拜之曰:“公若不弃,布请拜为义父。”卓以金甲锦袍赐布,畅饮而散。卓自是威势越大,自领前将军事,封弟董旻为左将军、鄠侯,封吕布为骑都尉、中郎将、都亭侯。李儒劝卓早定废立之计。卓乃于省中设宴,会集公卿,令吕布将甲士千余,侍卫左右。是日,太傅袁隗与百官皆到。酒行数巡,卓按剑曰“今上暗弱,不可以奉宗庙;吾将依伊尹、霍光故事,废帝为弘农王,立陈留王为帝。有不从者斩!”群臣惶怖莫敢对。中军校尉袁绍挺身出曰:“今上即位未几,并无失德;汝欲废嫡立庶,非反而何?”卓怒曰:“天下事在我!我今为之,谁敢不从!汝视我之剑不利否?”袁绍亦拔剑曰:“汝剑利,吾剑未尝不利!”两个在筵上对敌。正是:丁原仗义身先丧,袁绍争锋势又危。毕竟袁绍性命如何,且听下文分解。