\chapter{诸葛亮智算华容~关云长义释曹操}

却说当夜张辽一箭射黄盖下水,救得曹操登岸,寻着马匹走时,军已大乱。韩当冒烟突火来攻水寨,忽听得士卒报道:“后梢舵上一人,高叫将军表字。”韩当细听,但闻高叫“义公救我?”当曰:“此黄公覆也!”急教救起。见黄盖负箭着伤,咬出箭杆,箭头陷在肉内。韩当急为脱去湿衣,用刀剜出箭头,扯旗束之,脱自己战袍与黄盖穿了,先令别船送回大寨医治。原来黄盖深知水性,故大寒之时,和甲堕江,也逃得性命。却说当日满江火滚,喊声震地。左边是韩当、蒋钦两军从赤壁西边杀来;右边是周泰、陈武两军从赤壁东边杀来;正中是周瑜、程普、徐盛、丁奉大队船只都到。火须兵应,兵仗火威。此正是:三江水战,赤壁鏖兵。曹军着枪中箭、火焚水溺者,不计其数。后人有诗曰:“魏吴争斗决雌雄,赤壁楼船一扫空。烈火初张照云海,周郎曾此破曹公。”又有一绝云:“山高月小水茫茫,追叹前朝割据忙。南士无心迎魏武,东风有意便周郎。”不说江中鏖兵。且说甘宁令蔡中引入曹寨深处,宁将蔡中一刀砍于马下,就草上放起火来。吕蒙遥望中军火起,也放十数处火,接应甘宁。潘璋、董袭分头放火呐喊,四下里鼓声大震。曹操与张辽引百余骑,在火林内走,看前面无一处不着。正走之间,毛玠救得文聘,引十数骑到。操令军寻路。张辽指道:“只有乌林地面,空阔可走。”操径奔乌林。正走间,背后一军赶到,大叫:“曹贼休走!”火光中现出吕蒙旗号。操催军马向前,留张辽断后,抵敌吕蒙。却见前面火把又起,从山谷中拥出一军,大叫:“凌统在此!”曹操肝胆皆裂。忽刺斜里一彪军到,大叫:“丞相休慌!徐晃在此!”彼此混战一场,夺路望北而走。忽见一队军马,屯在山坡前。徐晃出问,乃是袁绍手下降将马延、张顗,有三千北地军马,列寨在彼;当夜见满天火起,未敢转动,恰好接着曹操。操教二将引一千军马开路,其余留着护身。操得这枝生力军马,心中稍安。马延、张顗二将飞骑前行。不到十里,喊声起处,一彪军出。为首一将,大呼曰:“吾乃东吴甘兴霸也!”马延正欲交锋,早被甘宁一刀斩于马下;张顗挺枪来迎,宁大喝一声,顗措手不及,被宁手起一刀,翻身落马。后军飞报曹操。操此时指望合淝有兵救应;不想孙权在合淝路口,望见江中火光,知是我军得胜,便教陆逊举火为号,太史慈见了,与陆逊合兵一处,冲杀将来。操只得望彝陵而走。路上撞见张郃,操令断后。纵马加鞭,走至五更,回望火光渐远,操心方定,问曰:“此是何处?”左右曰:“此是乌林之西,宜都之北。”操见树木丛杂,山川险峻,乃于马上仰面大笑不止。诸将问曰:“丞相何故大笑?”操曰:“吾不笑别人,单笑周瑜无谋,诸葛亮少智。若是吾用兵之时,预先在这里伏下一军,如之奈何?”说犹未了,两边鼓声震响,火光竟天而起,惊得曹操几乎坠马。刺斜里一彪军杀出,大叫:“我赵子龙奉军师将令,在此等候多时了!”操教徐晃、张郃双敌赵云,自己冒烟突火而去。子龙不来追赶,只顾抢夺旗帜。曹操得脱。

天色微明,黑云罩地,东南风尚不息。忽然大雨倾盆,湿透衣甲。操与军士冒雨而行,诸军皆有饥色。操令军士往村落中劫掠粮食,寻觅火种。方欲造饭,后面一军赶到。操心甚慌。原来却是李典、许褚保护着众谋士来到,操大喜,令军马且行,问:“前面是那里地面?”人报:“一边是南彝陵大路,一边是北彝陵山路。”操问:“那里投南郡江陵去近?”军士禀曰:“取南彝陵过葫芦口去最便。”操教走南彝陵。行至葫芦口,军皆饥馁,行走不上,马亦困乏,多有倒于路者。操教前面暂歇。马上有带得锣锅的,也有村中掠得粮米的,便就山边拣干处埋锅造饭,割马肉烧吃。尽皆脱去湿衣,于风头吹晒;马皆摘鞍野放,咽咬草根。操坐于疏林之下,仰面大笑。众官问曰:“适来丞相笑周瑜、诸葛亮,引惹出赵子龙来,又折了许多人马。如今为何又笑?”操曰:“吾笑诸葛亮、周瑜毕竟智谋不足。若是我用兵时,就这个去处,也埋伏一彪军马,以逸待劳;我等纵然脱得性命,也不免重伤矣。彼见不到此,我是以笑之。”正说间,前军后军一齐发喊、操大惊,弃甲上马。众军多有不及收马者。早见四下火烟布合,山口一军摆开,为首乃燕人张翼德,横矛立马,大叫:“操贼走那里去!”诸军众将见了张飞,尽皆胆寒。许褚骑无鞍马来战张飞。张辽、徐晃二将,纵马也来夹攻。两边军马混战做一团。操先拨马走脱,诸将各自脱身。张飞从后赶来。操迤逦奔逃,追兵渐远,回顾众将多已带伤。

正行时,军士禀曰:“前面有两条路,请问丞相从那条路去?”操问:“那条路近?”军士曰:“大路稍平,却远五十余里。小路投华容道,却近五十余里;只是地窄路险,坑坎难行。”操令人上山观望,回报:“小路山边有数处烟起;大路并无动静。”操教前军便走华容道小路。诸将曰:“烽烟起处,必有军马,何故反走这条路?”操曰:“岂不闻兵书有云:虚则实之,实则虚之。诸葛亮多谋,故使人于山僻烧烟,使我军不敢从这条山路走,他却伏兵于大路等着。吾料已定,偏不教中他计!”诸将皆曰:“丞相妙算,人不可及。”遂勒兵走华容道。此时人皆饥倒,马尽困乏。焦头烂额者扶策而行,中箭着枪者勉强而走。衣甲湿透,个个不全;军器旗幡,纷纷不整:大半皆是彝陵道上被赶得慌,只骑得秃马,鞍辔衣服,尽皆抛弃。正值隆冬严寒之时,其苦何可胜言。

操见前军停马不进,问是何故。回报曰:“前面山僻路小,因早晨下雨,坑堑内积水不流,泥陷马蹄,不能前进。”操大怒,叱曰:“军旅逢山开路,遇水叠桥,岂有泥泞不堪行之理!”传下号令,教老弱中伤军士在后慢行,强壮者担土束柴,搬草运芦,填塞道路。务要即时行动,如违令者斩。众军只得都下马,就路旁砍伐竹木,填塞山路。操恐后军来赶,令张辽、许褚、徐晃引百骑执刀在手,但迟慢者便斩之。此时军已饿乏,众皆倒地,操喝令人马践踏而行,死者不可胜数。号哭之声,于路不绝。操怒曰:“生死有命,何哭之有!如再哭者立斩!”三停人马:一停落后,一停填了沟壑,一停跟随曹操。过了险峻,路稍平坦。操回顾止有三百余骑随后,并无衣甲袍铠整齐者。操催速行。众将曰:“马尽乏矣,只好少歇。”操曰:“赶到荆州将息未迟。”又行不到数里,操在马上扬鞭大笑。众将问:“丞相何又大笑?”操曰:“人皆言周瑜、诸葛亮足智多谋,以吾观之,到底是无能之辈。若使此处伏一旅之师,吾等皆束手受缚矣。”

言未毕,一声炮响,两边五百校刀手摆开,为首大将关云长,提青龙刀,跨赤兔马,截住去路。操军见了,亡魂丧胆,面面相觑。操曰:“既到此处,只得决一死战!”众将曰:“人纵然不怯,马力已乏,安能复战?”程昱曰:“某素知云长傲上而不忍下,欺强而不凌弱;恩怨分明,信义素著。丞相旧日有恩于彼,今只亲自告之,可脱此难。”操从其说,即纵马向前,欠身谓云长曰:“将军别来无恙!”云长亦欠身答曰:“关某奉军师将令,等候丞相多时。”操曰:“曹操兵败势危,到此无路,望将军以昔日之情为重。”云长曰:“昔日关某虽蒙丞相厚恩,然已斩颜良,诛文丑,解白马之围,以奉报矣。今日之事,岂敢以私废公?”操曰:“五关斩将之时,还能记否?大丈夫以信义为重。将军深明《春秋》,岂不知庾公之斯追子濯孺子之事乎?”云长是个义重如山之人,想起当日曹操许多恩义,与后来五关斩将之事,如何不动心?又见曹军惶惶,皆欲垂泪,一发心中不忍。于是把马头勒回,谓众军曰:“四散摆开。”这个分明是放曹操的意思。操见云长回马,便和众将一齐冲将过去。云长回身时,曹操已与众将过去了。云长大喝一声,众军皆下马,哭拜于地。云长愈加不忍。正犹豫间,张辽纵马而至。云长见了,又动故旧之情,长叹一声,并皆放去。后人有诗曰:“曹瞒兵败走华容,正与关公狭路逢。只为当初恩义重,放开金锁走蛟龙。”

曹操既脱华容之难。行至谷口,回顾所随军兵,止有二十七骑。比及天晚,已近南郡,火把齐明,一簇人马拦路。操大惊曰:“吾命休矣!”只见一群哨马冲到,方认得是曹仁军马。操才心安。曹仁接着,言:“虽知兵败,不敢远离,只得在附近迎接。”操曰:“几与汝不相见也!”于是引众入南郡安歇。随后张辽也到,说云长之德。操点将校,中伤者极多,操皆令将息。曹仁置酒与操解闷。众谋士俱在座。操忽仰天大恸。众谋士曰:“丞相于虎窟中逃难之时,全无惧怯;今到城中,人已得食,马已得料,正须整顿军马复仇,何反痛哭?”操曰:“吾哭郭奉孝耳!若奉孝在,决不使吾有此大失也!”遂捶胸大哭曰:“哀哉,奉孝!痛哉,奉孝!惜哉!奉孝!”众谋士皆默然自惭。次日,操唤曹仁曰:“吾今暂回许都,收拾军马,必来报仇。汝可保全南郡。吾有一计,密留在此,非急休开,急则开之。依计而行,使东吴不敢正视南郡。”仁曰:“合淝、襄阳,谁可保守?”操曰:“荆州托汝管领;襄阳吾已拨夏侯惇守把;合淝最为紧要之地,吾令张辽为主将,乐进、李典为副将,保守此地。但有缓急,飞报将来。”操分拨已定,遂上马引众奔回许昌。荆州原降文武各官,依旧带回许昌调用。曹仁自遣曹洪据守彝陵、南郡,以防周瑜。

却说关云长放了曹操,引军自回。此时诸路军马,皆得马匹、器械、钱粮,已回夏口;独云长不获一人一骑,空身回见玄德。孔明正与玄德作贺,忽报云长至。孔明忙离坐席,执杯相迎曰:“且喜将军立此盖世之功,与普天下除大害。合宜远接庆贺!”云长默然。孔明曰:“将军莫非因吾等不曾远接,故尔不乐?”回顾左右曰:“汝等缘何不先报?”云长曰:“关某特来请死。”孔明曰:“莫非曹操不曾投华容道上来?”云长曰:“是从那里来。关某无能,因此被他走脱。”孔明曰:“拿得甚将士来?”云长曰:“皆不曾拿。”孔明曰:“此是云长想曹操昔日之恩,故意放了。但既有军令状在此,不得不按军法。”遂叱武士推出斩之。正是:拚将一死酬知己,致令千秋仰义名。未知云长性命如何,且看下文分解。