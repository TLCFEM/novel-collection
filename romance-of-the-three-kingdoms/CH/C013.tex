\chapter{李傕郭汜大交兵~杨奉董承双救驾}

却说曹操大破吕布于定陶,布乃收集败残军马于海滨,众将皆来会集,欲再与曹操决战,陈宫曰:“今曹兵势大,未可与争。先寻取安身之地,那时再来未迟。”布曰:“吾欲再投袁绍,何如?”宫曰:“先使人往冀州探听消息,然后可去。”布从之。且说袁绍在冀州,闻知曹操与吕布相持,谋士审配进曰:“吕布,豺虎也:若得兖州,必图冀州。不若助操攻之,方可无患。”绍遂遣颜良将兵五万,往助曹操。细作探知这个消息,飞报吕布。布大惊,与陈宫商议。宫曰:“闻刘玄德新领徐州,可往投之。”布从其言,竟投徐州来。有人报知玄德。玄德曰:“布乃当今英勇之士,可出迎之。”糜竺曰:“吕布乃虎狼之徒,不可收留;收则伤人矣。”玄德曰:“前者非布袭兖州,怎解此郡之祸。今彼穷而投我,岂有他心!”张飞曰:“哥哥心肠忒好。虽然如此,也要准备。”

玄德领众出城三十里,接着吕布,并马入城。都到州衙厅上,讲礼毕,坐下。布曰:“某自与王司徒计杀董卓之后,又遭傕、汜之变,飘零关东,诸侯多不能相容。近因曹贼不仁,侵犯徐州,蒙使君力救陶谦,布因袭兖州以分其势;不料反堕奸计,败兵折将。今投使君,共图大事,未审尊意如何?”玄德曰:“陶使君新逝,无人管领徐州,因令备权摄州事。今幸将军至此,合当相让”遂将牌印送与吕布。吕布却待要接,只见玄德背后关、张二公各有怒色。布乃佯笑曰:“量吕布一勇夫,何能作州牧乎?”玄德又让。陈宫曰:“强宾不压主,请便君勿疑。”玄德方止。遂设宴相待,收拾宅院安下。次日,吕布回席请玄德,玄德乃与关、张同往。饮酒至半酣,布请玄德入后堂,关、张随入。布令妻女出拜玄德。玄德再三谦让。布曰:“贤弟不必推让。”张飞听了,瞋目大叱曰:“我哥哥是金枝玉叶,你是何等人,敢称我哥哥为贤弟!你来!我和你斗三百合!”玄德连忙喝住,关公劝飞出。玄德与吕布陪话曰:“劣弟酒后狂言,兄勿见责。”布默然无语。须臾席散。布送玄德出门,张飞跃马横枪而来,大叫:“吕布!我和你并三百合!”玄德急令关公劝止。

次日,吕布来辞玄德曰:“蒙使君不弃,但恐令弟辈不能相容。布当别投他处。”玄德曰:“将军若去,某罪大矣。劣弟冒犯,另日当今陪话。近邑小沛,乃备昔日屯兵之处。将军不嫌浅狭,权且歇马,如何?粮食军需,谨当应付。”吕布谢了玄德,自引军投小沛安身去了。玄德自去埋怨张飞不题。

却说曹操平了山东,表奏朝廷,加操为建德将军费亭侯。其时李傕自为大司马,郭汜自为大将军,横行无忌,朝廷无人敢言。太尉杨彪、大司农朱儁暗奏献帝曰:“今曹操拥兵二十余万,谋臣武将数十员,若得此人扶持社稷,剿除奸党,天下幸甚。”献帝泣曰:“朕被二贼欺凌久矣!若得诛之,诚为大幸!”彪奏曰:“臣有一计:先令二贼自相残害,然后诏曹操引兵杀之,扫清贼党,以安朝廷。”献帝曰:“计将安出?”彪曰:“闻郭汜之妻最妒,可令人于汜妻处用反间计,则二贼自相害矣。”帝乃书密诏付杨彪。彪即暗使夫人以他事入郭汜府,乘间告汜妻曰:“闻郭将军与李司马夫人有染,其情甚密。倘司马知之,必遭其害。夫人宜绝其往来为妙。”汜妻讶曰:“怪见他经宿不归!却干出如此无耻之事!非夫人言,妾不知也。当慎防之。”彪妻告归,汜妻再三称谢而别。过了数日,郭汜又将往李傕府中饮宴。妻曰:“傕性不测,况今两雄不并立,倘彼酒后置毒,妾将奈何?”汜不肯听,妻再三劝住。至晚间,傕使人送酒筵至。汜妻乃暗置毒于中,方始献入,汜便欲食。妻曰:“食自外来,岂可便食?”乃先与犬试之,犬立死。自此汜心怀疑。一日朝罢,李傕力邀郭汜赴家饮宴。至夜席散,汜醉而归,偶然腹痛。妻曰:“必中其毒矣!”急令将粪汁灌之,一吐方定。汜大怒曰:“吾与李共图大事,今无端欲谋害我,我不先发,必遭毒手。”遂密整本部甲兵,欲攻李傕。早有人报知傕。傕亦大怒曰:“郭阿多安敢如此!”遂点本部甲兵,来杀郭汜。两处合兵数万,就在长安城下混战,乘势掳掠居民。傕侄李暹引兵围住宫院,用车二乘,一乘载天子,一乘载伏皇后,使贾诩、左灵监押车驾;其余宫人内侍,并皆步走。拥出后宰门,正遇郭汜兵到,乱箭齐发,射死宫人不知其数。李傕随后掩杀,郭汜兵退,车驾冒险出城,不由分说,竟拥到李傕营中。郭汜领兵入官,尽抢掳宫嫔采女入营,放火烧宫殿。次日,郭汜知李傕劫了天子,领军来营前厮杀。帝后都受惊恐。后人有诗叹之曰:“光武中兴兴汉世,上下相承十二帝。桓灵无道宗社堕,阉臣擅权为叔季。无谋何进作三公,欲除社鼠招奸雄。豺獭虽驱虎狼入,西州逆竖生淫凶。王允赤心托红粉,致令董吕成矛盾。渠魁殄灭天下宁,谁知李郭心怀愤。神州荆棘争奈何,六宫饥馑愁干戈。人心既离天命去,英雄割据分山河。后王规此存兢业,莫把金瓯等闲缺。生灵糜烂肝脑涂,剩水残山多怨血。我观遗史不胜悲,今古茫茫叹黍离。人君当守苞桑戒,太阿谁执全纲维。

却说郭汜兵到,李傕出营接战。汜军不利,暂且退去。傕乃移帝后车驾于郿坞,使侄李暹监之,断绝内使,饮食不继,侍臣皆有饥色。帝令人问傕取米五斛,牛骨五具,以赐左右。傕怒曰:“朝夕上饭,何又他求?”乃以腐肉朽粮与之,皆臭不可食。帝骂曰:“逆贼直如此相欺!”侍中杨琦急奏曰:“傕性残暴。事势至此,陛下且忍之,不可撄其锋也。”帝乃低头无语,泪盈袍袖。忽左右报曰:“有一路军马,枪刀映日,金鼓震天,前来救驾。”帝教打听是谁,乃郭汜也。帝心转忧。只闻坞外喊声大起,原来李傕引兵出迎郭汜,鞭指郭汜而骂曰:“我待你不薄,你如何谋害我!”汜曰:“尔乃反贼,如何不杀你!”傕曰:“我保驾在此,何为反贼?”汜曰:“此乃劫驾,何为保驾?”傕曰:“不须多言!我两个各不许用军士,只自并输赢。赢的便把皇帝取去罢了。”二人便就阵前厮杀。战到十合。不分胜负。只见杨彪拍马而来,大叫:“二位将军少歇!老夫特邀众官,来与二位讲和。”傕、汜乃各自还营。

杨彪与朱儁会合朝廷官僚六十余人,先诣郭汜营中劝和。郭汜竟将众官尽行监下。众官曰:“我等为好而来,何乃如此相待?”汜曰:“李傕劫天子,偏我劫不得公卿!”杨彪曰:“一劫天子,一劫公卿,意欲何为?”汜大怒,便拔剑欲杀彪。中郎将杨密力劝,汜乃放了杨彪、朱儁,其余都监在营中。彪谓儁曰:“为社稷之臣,不能匡君救主,空生天地间耳!”言讫,相抱而哭,昏绝于地。儁归家成病而死。自此之后,傕、汜每日厮杀,一连五十余日,死者不知其数。

却说李傕平日最喜左道妖邪之术,常使女巫击鼓降神于军中。贾诩屡谏不听。侍中杨琦密奏帝曰:“臣观贾诩虽为李傕腹心,然实未尝忘君,陛下当与谋之。”正说之间,贾诩来到。帝乃屏退左右,泣谕诩曰:“卿能怜汉朝,救朕命乎?”诩拜伏于地曰:“固臣所愿也。陛下且勿言,臣自图之。”帝收泪而谢。少顷,李傕来见,带剑而入。帝面如土色。傕谓帝曰:“郭汜不臣,监禁公卿,欲劫陛下。非臣则驾被掳矣。”帝拱手称谢,傕乃出。时皇甫郦入见帝。帝知郦能言,又与李傕同乡,诏使往两边解和。郦奉诏,走至汜营说汜。汜曰:“如李傕送出天子,我便放出公卿。”郦即来见李傕曰:“今天子以某是西凉人,与公同乡,特令某来劝和二公。汜已奉诏,公意若何?”傕曰:“吾有败吕布之大功,辅政四年,多著勋绩,天下共知。郭阿多盗马贼耳,乃敢擅劫公卿,与我相抗,誓必诛之!君试观我方略士众,足胜郭阿多否?”郦答曰:“不然。昔有穷后羿恃其善射,不思患难,以致灭亡。近董太师之强,君所目见也,吕布受恩而反图之,斯须之间,头悬国门。则强固不足恃矣。将军身为上将,持钺仗节,子孙宗族,皆居显位,国恩不可谓不厚。今敦阿多劫公卿,而将军劫至尊,果谁轻谁重耶?”李傕大怒,拔剑叱曰:“天子使汝来辱我乎?我先斩汝头!”骑都尉场奉谏曰:今郭汜未除,而杀天使,则汜兴兵有名,诸侯皆助之矣。”贾诩亦力劝,傕怒少息。诩遂推皇甫郦出。郦大叫曰:“李傕不奉诏,欲弑君自立!”侍中胡邈急止之曰:“无出此言,恐于身不利。”郦叱之曰:“胡敬才!汝亦为朝廷之臣,如何附贼?君辱臣死,吾被李傕所杀,乃分也!”大骂不止。帝知之,急令皇甫郦回西凉。

却说李傕之军,大半是西凉人氏,更赖羌兵为助。却被皇甫郦扬言于西凉人曰:“李傕谋反,从之者即为贼党,后患不浅。”西凉人多有听郦之言,军心渐涣。傕闻郦言,大怒,差虎贲王昌追之。昌知郦乃忠义之士,竟不往追,只回报曰:“郦已不知何往矣。”贾诩又密谕羌人曰:“天子知汝等忠义,久战劳苦,密诏使汝还郡,后当有重赏。”羌人正怨李傕不与爵赏,遂听诩言,都引兵去。诩又密奏帝曰:“李傕贪而无谋,今兵散心怯,可以重爵饵之。”帝乃降诏,封傕为大司马。傕喜曰:“此女巫降神祈祷之力也!”遂重赏女巫,却不赏军将。骑都尉杨奉大怒,谓宋果曰:“吾等出生入死,身冒矢石,功反不及女巫耶!”宋果曰:“何不杀此贼,以救天子?”奉曰:“你于中军放火为号,吾当引兵外应。”二人约定是夜二更时分举事。不料其事不密,有人报知李傕。傕大怒,令人擒宋果先杀之。杨奉引兵在外,不见号火。李傕自将兵出,恰遇杨奉,就寨中混战到四更。奉不胜,引军投西安去了。李傕自此军势渐衰。更兼郭汜常来攻击,杀死者甚多。忽人来报:“张济统领大军,自陕西来到,欲与二公解和;声言如不从者,引兵击之。”傕便卖个人情,先遣人赴张济军中许和。郭汜亦只得许诺。张济上表,请天子驾幸弘农。帝喜曰:“朕思东都久矣。今乘此得还,乃万幸也!”诏封张济为骠骑将军。济进粮食酒肉,供给百官。汜放公卿出营。傕收拾车驾东行,遣旧有御林军数百,持戟护送。

銮舆过新丰,至霸陵,时值秋天,金风骤起。忽闻喊声大作,数百军兵来至桥上拦住车驾,厉声问曰:“来者何人?”侍中杨琦拍马上桥曰:“圣驾过此,谁敢拦阻?”有二将出曰:“吾等奉郭将军命,把守此桥,以防奸细。既云圣驾,须亲见帝,方可准信。”杨琦高揭珠帘。帝谕曰:“朕躬在此,卿何不退?”众将皆呼“万岁”,分于两边,驾乃得过。二将回报郭汜曰:“驾已去矣。”汜曰:“我正欲哄过张济,劫驾再入郿坞,你如何擅自放了过去?”遂斩二将,起兵赶来。车驾正到华阴县,背后喊声震天,大叫:“车驾且休动!”帝泣告大臣曰:“方离狼窝,又逢虎口,如之奈何?”众皆失色。贼军渐近。只听得一派鼓声,山背后转出一将,当先一面大旗,上书“大汉杨奉”四字,引军千余杀来。

原来杨奉自为李傕所败,便引军屯终南山下;今闻驾至,特来保护。当下列开阵势。汜将崔勇出马,大骂杨奉“反贼”。奉大怒,回顾阵中曰:“公明何在?”一将手执大斧,飞骤骅骝,直取崔勇。两马相交,只一合,斩崔勇于马下。杨奉乘势掩杀,汜军大败,退走二十余里。奉乃收军来见天子。帝慰谕曰:“卿救朕躬,其功不小!”奉顿首拜谢。帝曰:“适斩贼将者何人?”奉乃引此将拜于车下曰:“此人河东杨郡人,姓徐,名晃,字公明。”帝慰劳之。杨奉保驾至华阴驻跸。将军段煨,具衣服饮膳上献。是夜,天子宿于杨奉营中。

郭汜败了一阵,次日又点军杀至营前来。徐晃当先出马,郭汜大军八面围来,将天子、杨奉困在垓心。正在危急之中,忽然东南上喊声大震,一将引军纵马杀来。贼众奔溃。徐晃乘势攻击,大败汜军。那人来见天子,乃国戚董承也。帝哭诉前事。承曰:“陛下免忧。臣与杨将军誓斩二贼,以靖天下。”帝命早赴东都。连夜驾起,前幸弘农。

却说郭汜引败军回,撞着李傕,言:“杨奉、董承救驾往弘农去了。若到山东,立脚得牢,必然布告天下,令诸侯共伐我等。三族不能保矣。”傕曰:“今张济兵据长安,未可轻动。我和你乘间合兵一处,至弘农杀了汉君,平分天下,有何不可!”汜喜诺。二人合兵,于路劫掠,所过一空。杨奉、董承知贼兵远来,遂勒兵回,与贼大战于东涧。傕、汜二人商议:“我众彼寡,只可以混战胜之。”于是李在左,郭汜在右,漫山遍野拥来。杨奉、董承两边死战,刚保帝后车出;百官宫人,符册典籍,一应御用之物,尽皆抛弃。郭汜引军入弘农劫掠。承、奉保驾走陕北,傕、汜分兵赶来。

承、奉一面差人与傕、汜讲和,一面密传圣旨往河东,急召故白波帅韩暹、李乐、胡才三处军兵前来救应。那李乐亦是啸聚山林之贼,今不得已而召之。三处军闻天子赦罪赐官,如何不来;并拔本营军士,来与董承约会一齐,再取弘农。其时李傕、敦汜但到之处,劫掠百姓,老弱者杀之,强壮者充军;临敌则驱民兵在前,名曰:“敢死军”,贼势浩大,李乐军到,会于渭阳。郭汜令军士将衣服物件抛弃于道。乐军见衣服满地,争往取之,队伍尽失。傕、汜二军,四面混战,乐军大败。杨奉、董承遮拦不住,保驾北走,背后贼军赶来。李乐曰:“事急矣!请天子上马先行!”帝曰:“朕不可舍百官而去。”众皆号泣相随。胡才被乱军所杀。承、奉见贼追急,请天子弃车驾,步行到黄河岸边。李乐等寻得一只小舟作渡船。时值天气严寒,帝与后强扶到岸,边岸又高,不得下船,后面追兵将至。杨奉曰:“可解马疆绳接连,拴缚帝腰,放下船去。”人丛中国舅伏德挟白绢十数匹至,曰:“我于乱军中拾得此绢,可接连拽辇。”行军校尉尚弘用绢包帝及后,令众先挂帝往下放之,乃得下船。李乐仗剑立于船头上。后兄伏德,负后下船中。岸上有不得下船者,争扯船缆;李乐尽砍于水中。渡过帝后,再放船渡众人。其争渡者,皆被砍下手指,哭声震天。既渡彼岸,帝左右止剩得十余人。杨奉寻得牛车一辆,载帝至大阳。绝食,晚宿于瓦屋中,野老进粟饭,上与后共食,粗粝不能下咽。次日,诏封李乐为征北将军,韩暹为征东将军,起驾前行。有二大臣寻至,哭拜车前,乃太尉杨彪、太仆韩融也。帝后俱哭。韩融曰:“傕、汜二贼,颇信臣言;臣舍命去说二贼罢兵。陛下善保龙体。”韩融去了。李乐请帝入杨奉营暂歇。杨彪请帝都安邑县。驾至安邑,苦无高房,帝后都居于茅屋中;又无门关闭,四边插荆棘以为屏蔽。帝与大臣议事于茅屋之下,诸将引兵于篱外镇压。李乐等专权,百官稍有触犯,竟于帝前殴骂;故意送浊酒粗食与帝,帝勉强纳之。李乐、韩暹又连名保奏无徒、部曲、巫医、走卒二百余名,并为校尉、御史等官。刻印不及,以锥画之,全不成体统。却说韩融曲说傕、汜二贼。二贼从其言,乃放百官及宫人归。是岁大荒,百姓皆食枣菜,饿莩遍野。河内太守张杨献米肉,河东太守王邑献绢帛,帝稍得宁。董承、杨奉商议,一面差人修洛阳宫院,欲奉车驾还东都。李乐不从。董承谓李乐曰:“洛阳本天子建都之地,安邑乃小地面,如何容得车驾?今奉驾还洛阳是正理。”李乐曰:“汝等奉驾去,我只在此处住。”承、奉乃奉驾起程。李乐暗令人结连李傕、郭汜,一同劫驾。董承、杨奉、韩暹知其谋,连夜摆布军士,护送车驾前奔箕关。李乐闻知,不等傕、汜军到,自引本部人马前来追赶。四更左侧,赶到箕山下,大叫:“车驾休行!李傕、郭汜在此!”吓得献帝心惊胆战。山上火光遍起。正是:前番两贼分为二,今番三贼合为一。不知汉天子怎离此难,且听下文分解。