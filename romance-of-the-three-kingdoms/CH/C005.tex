\chapter{发矫诏诸镇应曹公~破关兵三英战吕布}

却说陈宫临欲下手杀曹操,忽转念曰:“我为国家跟他到此,杀之不义。不若弃而他往。”插剑上马,不等天明,自投东郡去了。操觉,不见陈宫,寻思:“此人见我说了这两句,疑我不仁,弃我而去;吾当急行,不可久留。”遂连夜到陈留,寻见父亲,备说前事;欲散家资,招募义兵。父言:“资少恐不成事。此间有孝廉卫弘,疏财仗义,其家巨富;若得相助,事可图矣。”操置酒张筵,拜请卫弘到家,告曰:“今汉室无主,董卓专权,欺君害民,天下切齿。操欲力扶社稷,恨力不足。公乃忠义之士,敢求相助!”卫弘曰:“吾有是心久矣,恨未遇英雄耳。既孟德有大志,愿将家资相助。”操大喜;于是先发矫诏,驰报各道,然后招集义兵,竖起招兵白旗一面,上书“忠义”二字。不数日间,应募之士,如雨骈集。

一日,有一个阳平卫国人,姓乐,名进,字文谦,来投曹操。又有一个山阳巨鹿人,姓李,名典,字曼成,也来投曹操。操皆留为帐前吏。又有沛国谯人夏侯惇,字元让,乃夏侯婴之后;自小习枪棒;年十四从师学武,有人辱骂其师,惇杀之,逃于外方;闻知曹操起兵,与其族弟夏侯渊两个,各引壮士千人来会。此二人本操之弟兄:操父曹嵩原是夏侯氏之子,过房与曹家,因此是同族。不数日,曹氏兄弟曹仁、曹洪各引兵千余来助。曹仁字子孝,曹洪字子廉:二人弓马熟娴,武艺精通。操大喜,于村中调练军马。卫弘尽出家财,置办衣甲旗幡。四方送粮食者,不计其数。

时袁绍得操矫诏,乃聚麾下文武,引兵三万,离渤海来与曹操会盟。操作檄文以达诸郡。檄文曰:“操等谨以大义布告天下:董卓欺天罔地,灭国弑君;秽乱宫禁,残害生灵;狼戾不仁,罪恶充积!今奉天子密诏,大集义兵,誓欲扫清华夏,剿戮群凶。望兴义师,共泄公愤;扶持王室,拯救黎民。檄文到日,可速奉行!”操发檄文去后,各镇诸侯皆起兵相应:第一镇,后将军南阳太守袁术。第二镇,冀州刺史韩馥。第三镇,豫州刺史孔伷。第四镇,兖州刺史刘岱。第五镇,河内郡太守王匡。第六镇,陈留太守张邈。第七镇,东郡太守乔瑁。第八镇,山阳太守袁遗。第九镇,济北相鲍信。第十镇,北海太守孔融。第十一镇,广陵太守张超。第十二镇,徐州刺史陶谦。第十三镇,西凉太守马腾。第十四镇,北平太守公孙瓒。第十五镇,上党太守张杨。第十六镇,乌程侯长沙太守孙坚。第十七镇,祁乡侯渤海太守袁绍。诸路军马,多少不等,有三万者,有一二万者,各领文官武将,投洛阳来。

且说北平太守公孙瓒,统领精兵一万五千,路经德州平原县。正行之间,遥见桑树丛中,一面黄旗,数骑来迎。瓒视之,乃刘玄德也。瓒问曰:“贤弟何故在此?”玄德曰:“旧日蒙兄保备为平原县令,今闻大军过此,将来奉候,就请兄长入城歇马。”瓒指关、张而问曰:“此何人也?”玄德曰:“此关羽、张飞,备结义兄弟也。”瓒曰:“乃同破黄巾者乎?”玄德曰:“皆此二人之力。”瓒曰:“今居何职?”玄德答曰:“关羽为马弓手,张飞为步弓手。”瓒叹曰:“如此可谓埋没英雄!今董卓作乱,天下诸侯共往诛之。贤弟可弃此卑官,一同讨贼,力扶汉室,若何?”玄德曰:“愿往。”张飞曰:“当时若容我杀了此贼,免有今日之事。”云长曰:“事已至此,即当收拾前去。”玄德、关、张引数骑跟公孙瓒来,曹操接着。众诸侯亦陆续皆至,各自安营下寨,连接二百余里。操乃宰牛杀马,大会诸侯,商议进兵之策。太守王匡曰:“今奉大义,必立盟主;众听约束,然后进兵。”操曰:“袁本初四世三公,门多故吏,汉朝名相之裔,可为盟主。”绍再三推辞,众皆曰非本初不可,绍方应允。次日筑台三层,遍列五方旗帜,上建白旄黄钺,兵符将印,请绍登坛。绍整衣佩剑,慨然而上,焚香再拜。其盟曰:“汉室不幸,皇纲失统。贼臣董卓,乘衅纵害,祸加至尊,虐流百姓。绍等惧社稷沦丧,纠合义兵,并赴国难。凡我同盟,齐心戮力,以致臣节,必无二志。有渝此盟,俾坠其命,无克遗育。皇天后土,祖宗明灵,实皆鉴之!”读毕歃血。众因其辞气慷慨,皆涕泗横流。歃血已罢,下坛。众扶绍升帐而坐,两行依爵位年齿分列坐定。操行酒数巡,言曰:“今日既立盟主,各听调遣,同扶国家,勿以强弱计较。”袁绍曰:“绍虽不才,既承公等推为盟主,有功必赏,有罪必罚。国有常刑,军有纪律。各宜遵守,勿得违犯。”众皆曰惟命是听。绍曰:“吾弟袁术总督粮草,应付诸营,无使有缺。更须一人为先锋,直抵汜水关挑战。余各据险要,以为接应。”

长沙太守孙坚出曰:“坚愿为前部。”绍曰:“文台勇烈,可当此任。”坚遂引本部人马杀奔汜水关来。守关将士,差流星马往洛阳丞相府告急。董卓自专大权之后,每日饮宴。李儒接得告急文书,径来禀卓。卓大惊,急聚众将商议。温侯吕布挺身出曰:“父亲勿虑。关外诸侯,布视之如草芥;愿提虎狼之师,尽斩其首,悬于都门。”卓大喜曰:“吾有奉先,高枕无忧矣!”言未绝,吕布背后一人高声出曰:“割鸡焉用牛刀?不劳温侯亲往。吾斩众诸侯首级,如探囊取物耳!”卓视之,其人身长九尺,虎体狼腰,豹头猿臂;关西人也,姓华,名雄。卓闻言大喜,加为骁骑校尉。拨马步军五万,同李肃、胡轸、赵岑星夜赴关迎敌。

众诸侯内有济北相鲍信,寻思孙坚既为前部,怕他夺了头功,暗拨其弟鲍忠,先将马步军三千,径抄小路,直到关下搦战。华雄引铁骑五百,飞下关来,大喝:“贼将休走!”鲍忠急待退,被华雄手起刀落,斩于马下,生擒将校极多。华雄遣人赍鲍忠首级来相府报捷,卓加雄为都督。

却说孙坚引四将直至关前。那四将?——第一个,右北平土垠人,姓程,名普,字德谋,使一条铁脊蛇矛;第二个,姓黄,名盖,字公覆,零陵人也,使铁鞭;第三个,姓韩,名当,字义公,辽西令支人也,使一口大刀;第四个,姓祖,名茂,字大荣,吴郡富春人也,使双刀。孙坚披烂银铠,裹赤帻,横古锭刀,骑花鬃马,指关上而骂曰:“助恶匹夫,何不早降!”华雄副将胡轸引兵五千出关迎战。程普飞马挺矛,直取胡轸。斗不数合,程普刺中胡轸咽喉,死于马下。坚挥军直杀至关前,关上矢石如雨。孙坚引兵回至梁东屯住,使人于袁绍处报捷,就于袁术处催粮。

或说术曰:“孙坚乃江东猛虎;若打破洛阳,杀了董卓,正是除狼而得虎也。今不与粮,彼军必散。”术听之,不发粮草。孙坚军缺食,军中自乱,细作报上关来。李肃为华雄谋曰:“今夜我引一军从小路下关,袭孙坚寨后,将军击其前寨,坚可擒矣。”雄从之,传令军士饱餐,乘夜下关。是夜月白风清。到坚寨时,已是半夜,鼓噪直进。坚慌忙披挂上马,正遇华雄。两马相交,斗不数合,后面李肃军到,竟天价放起火来。坚军乱窜。众将各自混战,止有祖茂跟定孙坚,突围而走。背后华雄追来。坚取箭,连放两箭,皆被华雄躲过。再放第三箭时,因用力太猛,拽折了鹊画弓,只得弃弓纵马而奔。祖茂曰:“主公头上赤帻射目,为贼所识认。可脱帻与某戴之。”坚就脱帻换茂盔,分两路而走。雄军只望赤帻者追赶,坚乃从小路得脱。祖茂被华雄追急,将赤帻挂于人家烧不尽的庭柱上,却入树林潜躲。华雄军于月下遥见赤帻,四面围定,不敢近前。用箭射之,方知是计,遂向前取了赤帻。祖茂于林后杀出,挥双刀欲劈华雄;雄大喝一声,将祖茂一刀砍于马下。杀至天明,雄方引兵上关。

程普、黄盖、韩当都来寻见孙坚,再收拾军马屯扎。坚为折了祖茂,伤感不已,星夜遣人报知袁绍。绍大惊曰:“不想孙文台败于华雄之手!”便聚众诸侯商议。众人都到,只有公孙瓒后至,绍请入帐列坐。绍曰:“前日鲍将军之弟不遵调遣,擅自进兵,杀身丧命,折了许多军士;今者孙文台又败于华雄:挫动锐气,为之奈何?”诸侯并皆不语。绍举目遍视,见公孙瓒背后立着三人,容貌异常,都在那里冷笑。绍问曰:“公孙太守背后何人?”瓒呼玄德出曰:“此吾自幼同舍兄弟,平原令刘备是也。”曹操曰:“莫非破黄巾刘玄德乎?”瓒曰:“然。”即令刘玄德拜见。瓒将玄德功劳,并其出身,细说一遍。绍曰:“既是汉室宗派,取坐来。”命坐。备逊谢。绍曰:“吾非敬汝名爵,吾敬汝是帝室之胄耳。”玄德乃坐于末位,关、张叉手侍立于后。忽探子来报:“华雄引铁骑下关,用长竿挑着孙太守赤帻,来寨前大骂搦战。”绍曰:“谁敢去战?”袁术背后转出骁将俞涉曰:“小将愿往。”绍喜,便著俞涉出马。即时报来:“俞涉与华雄战不三合,被华雄斩了。”众大惊。太守韩馥曰:“吾有上将潘凤,可斩华雄。”绍急令出战。潘凤手提大斧上马。去不多时,飞马来报:“潘凤又被华雄斩了。”众皆失色。绍曰:“可惜吾上将颜良、文丑未至!得一人在此,何惧华雄!”言未毕,阶下一人大呼出曰:“小将愿往斩华雄头,献于帐下!”众视之,见其人身长九尺,髯长二尺,丹凤眼,卧蚕眉,面如重枣,声如巨钟,立于帐前。绍问何人。公孙瓒曰:“此刘玄德之弟关羽也。”绍问现居何职。瓒曰:“跟随刘玄德充马弓手。”帐上袁术大喝曰:“汝欺吾众诸侯无大将耶?量一弓手,安敢乱言!与我打出!”曹操急止之曰:“公路息怒。此人既出大言,必有勇略;试教出马,如其不胜,责之未迟。”袁绍曰:“使一弓手出战,必被华雄所笑。”操曰:“此人仪表不俗,华雄安知他是弓手?”关公曰:“如不胜,请斩某头。”操教酾热酒一杯,与关公饮了上马。关公曰:“酒且斟下,某去便来。”出帐提刀,飞身上马。众诸侯听得关外鼓声大振,喊声大举,如天摧地塌,岳撼山崩,众皆失惊。正欲探听,鸾铃响处,马到中军,云长提华雄之头,掷于地上。其酒尚温。后人有诗赞之曰:“威镇乾坤第一功,辕门画鼓响冬冬。云长停盏施英勇,酒尚温时斩华雄。”曹操大喜。只见玄德背后转出张飞,高声大叫:“俺哥哥斩了华雄,不就这里杀入关去,活拿董卓,更待何时!”袁术大怒,喝曰:“俺大臣尚自谦让,量一县令手下小卒,安敢在此耀武扬威!都与赶出帐去!”曹操曰:“得功者赏,何计贵贱乎?”袁术曰:“既然公等只重一县令,我当告退。”操曰:“岂可因一言而误大事耶?”命公孙瓒且带玄德、关、张回寨。众官皆散。曹操暗使人赍牛酒抚慰三人。却说华雄手下败军,报上关来。李肃慌忙写告急文书,申闻董卓。卓急聚李儒、吕布等商议。儒曰:“今失了上将华雄,贼势浩大。袁绍为盟主,绍叔袁隗,现为太傅;倘或里应外合,深为不便,可先除之。请丞相亲领大军,分拨剿捕。”卓然其说,唤李催、郭汜领兵五百,围住太傅袁隗家,不分老幼,尽皆诛绝,先将袁隗首级去关前号令。

卓遂起兵二十万,分为两路而来:一路先令李傕、郭汜引兵五万,把住汜水关,不要厮杀;卓自将十五万,同李儒、吕布、樊稠、张济等守虎牢关。这关离洛阳五十里。军马到关,卓令吕布领三万军,去关前扎住大寨。卓自在关上屯住。

流星马探听得,报入袁绍大寨里来。绍聚众商议。操曰:“董卓屯兵虎牢,截俺诸侯中路,今可勒兵一半迎敌。”绍乃分王匡、乔瑁、鲍信、袁遗、孙融、张杨、陶谦、公孙瓒八路诸侯,往虎牢关迎敌。操引军往来救应。八路诸侯,各自起兵。河内太守王匡,引兵先到。吕布带铁骑三千,飞奔来迎。王匡将军马列成阵势,勒马门旗下看时,见吕布出阵:头戴三叉束发紫金冠,体挂西川红锦百花袍,身披兽面吞头连环铠,腰系勒甲玲珑狮蛮带;弓箭随身,手持画戟,坐下嘶风赤兔马:果然是“人中吕布,马中赤兔”!王匡回头问曰:“谁敢出战?”后面一将,纵马挺枪而出。匡视之,乃河内名将方悦。两马相交,无五合,被吕布一戟刺于马下,挺戟直冲过来。匡军大败,四散奔走。布东西冲杀,如入无人之境。幸得乔瑁、袁遗两军皆至,来救王匡,吕布方退。三路诸侯,各折了些人马,退三十里下寨。随后五路军马都至,一处商议,言吕布英雄,无人可敌。

正虑间,小校报来:“吕布搦战。”八路诸侯,一齐上马。军分八队,布在高冈。遥望吕布一簇军马,绣旗招飐,先来冲阵。上党太守张杨部将穆顺,出马挺枪迎战,被吕布手起一戟,刺于马下。众大惊。北海太守孔融部将武安国,使铁锤飞马而出。吕布挥戟拍马来迎。战到十余合,一戟砍断安国手腕,弃锤于地而走。八路军兵齐出,救了武安国。吕布退回去了。众诸侯回寨商议。曹操曰:“吕布英勇无敌,可会十八路诸侯,共议良策。若擒了吕布,董卓易诛耳。”

正议间,吕布复引兵搦战。八路诸侯齐出。公孙瓒挥槊亲战吕布。战不数合,瓒败走。吕布纵赤兔马赶来。那马日行千里,飞走如风。看看赶上,布举画戟望瓒后心便刺。傍边一将,圆睁环眼,倒竖虎须,挺丈八蛇矛,飞马大叫:“三姓家奴休走!燕人张飞在此!”吕布见了,弃了公孙瓒,便战张飞。飞抖擞精神,酣战吕布。连斗五十余合,不分胜负。云长见了,把马一拍,舞八十二斤青龙偃月刀,来夹攻吕布。三匹马丁字儿厮杀。战到三十合,战不倒吕布。刘玄德掣双股剑,骤黄鬃马,刺斜里也来助战。这三个围住吕布。转灯儿般厮杀。八路人马,都看得呆了。吕布架隔遮拦不定,看着玄德面上,虚刺一戟,玄德急闪。吕布荡开阵角,倒拖画戟,飞马便回。三个那里肯舍,拍马赶来。八路军兵,喊声大震,一齐掩杀。吕布军马望关上奔走;玄德、关、张随后赶来。古人曾有篇言语,单道着玄德、关、张三战吕布:“汉朝天数当桓灵,炎炎红日将西倾。奸臣董卓废少帝,刘协懦弱魂梦惊。曹操传檄告天下,诸侯奋怒皆兴兵。议立袁绍作盟主,誓扶王室定太平。温侯吕布世无比,雄才四海夸英伟。护躯银铠砌龙鳞,束发金冠簪雉尾。参差宝带兽平吞,错落锦袍飞凤起。龙驹跳踏起天风,画戟荧煌射秋水。出关搦战谁敢当?诸侯胆裂心惶惶。踊出燕人张冀德,手持蛇矛丈八枪。虎须倒竖翻金线,环眼圆睁起电光。酣战未能分胜败,阵前恼起关云长。青龙宝刀灿霜雪,鹦鹉战袍飞蛱蝶。马蹄到处鬼神嚎,目前一怒应流血。枭雄玄德掣双锋,抖擞天威施勇烈。三人围绕战多时,遮拦架隔无休歇。喊声震动天地翻,杀气迷漫牛斗寒。吕布力穷寻走路,遥望家山拍马还。倒拖画杆方天戟,乱散销金五彩幡。顿断绒绦走赤兔,翻身飞上虎牢关。”三人直赶吕布到关下,看见关上西风飘动青罗伞盖。张飞大叫:“此必董卓!追吕布有甚强处?不如先拿董贼,便是斩草除根!”拍马上关,来擒董卓。正是:擒贼定须擒贼首,奇功端的待奇人。未知胜负如何,且听下文分解。