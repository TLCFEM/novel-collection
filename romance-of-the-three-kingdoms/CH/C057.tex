\chapter{柴桑口卧龙吊丧~耒阳县凤雏理事}

却说周瑜怒气填胸,坠于马下,左右急救归船。军士传说:“玄德、孔明在前山顶上饮
酒取乐。”瑜大怒,咬牙切齿曰:“你道我取不得西川,吾誓取之!”正恨间,人报吴侯遣
弟孙瑜到。周瑜接入。具言其事。孙瑜曰:“吾奉兄命来助都督。”遂令催军前行。行至巴
丘,人报上流有刘封、关平二人领军截住水路。周瑜愈怒。忽又报孔明遣人送书至。周瑜拆
封视之。书曰:“汉军师中郎将诸葛亮,致书于东吴大都督公瑾先生麾下:亮自柴桑一别,
至今恋恋不忘。闻足下欲取西川,亮窃以为不可。益州民强地险,刘璋虽暗弱,足以自守。
今劳师远征,转运万里,欲收全功,虽吴起不能定其规,孙武不能善其后也。曹操失利于赤
壁,志岂须臾忘报仇哉?今足下兴兵远征,倘操乘虚而至,江南齑粉矣!亮不忍坐视,特此
告知。幸垂照鉴。”周瑜览毕,长叹一声,唤左右取纸笔作书上吴侯。乃聚众将曰:“吾非
不欲尽忠报国,奈天命已绝矣。汝等善事吴侯,共成大业。”言讫,昏绝。徐徐又醒,仰天
长叹曰:“既生瑜,何生亮!”连叫数声而亡。寿三十六岁。后人有诗叹曰:“赤壁遗雄
烈,青年有俊声。弦歌知雅意,杯酒谢良朋,曾谒三千斛,常驱十万兵。巴丘终命处,凭吊
欲伤情。”周瑜停丧于巴丘。众将将所遗书缄,遣人飞报孙权。权闻瑜死,放声大哭。拆视
其书,乃荐鲁肃以自代也。书略曰:“瑜以凡才,荷蒙殊遇,委任腹心,统御兵马,敢不竭
股肱之力,以图报效。奈死生不测,修短有命;愚志未展,微躯已殒,遗恨何极!方今曹操
在北,疆场未静;刘备寄寓,有似养虎;天下之事,尚未可知。此正朝士旰食之秋,至尊垂
虑之日也。鲁肃忠烈,临事不苟,可以代瑜之任。人之将死,其言也善。倘蒙垂鉴,瑜死不
朽矣。”孙权览毕,哭曰:“公瑾有王佐之才,今忽短命而死,孤何赖哉?既遗书特荐子
敬,孤敢不从之。”即日便命鲁肃为都督,总统兵马;一面教发周瑜灵柩回葬。却说孔明在
荆州,夜观天文,见将星坠地,乃笑曰:“周瑜死矣。”至晓,告于玄德。玄德使人探之,
果然死了。玄德问孔明曰:“周瑜既死,还当如何?”孔明曰:“代瑜领兵者,必鲁肃也。
亮观天象,将星聚于东方。亮当以吊丧为由。往江东走一遭,就寻贤士佐助主公。”玄德
曰:“只恐吴中将士加害于先生。”孔明曰:“瑜在之日,亮犹不惧;今瑜已死,又何患
乎?”乃与赵云引五百军,具祭礼,下船赴巴丘吊丧。于路探听得孙权已令鲁肃为都督,周
瑜灵柩已回柴桑。

孔明径至柴桑,鲁肃以礼迎接。周瑜部将皆欲杀孔明,因见赵云带剑相随,不敢下手。
孔明教设祭物于灵前,亲自奠酒,跪于地下,读祭文曰:“呜呼公瑾,不幸夭亡!修短故
天,人岂不伤?我心实痛,酹酒一觞;君其有灵,享我烝尝!吊君幼学,以交伯符;仗义疏
财,让舍以民。吊君弱冠,万里鹏抟;定建霸业,割据江南。吊君壮力,远镇巴丘;景升怀
虑,讨逆无忧。吊君丰度,佳配小乔;汉臣之婿,不愧当朝,吊君气概,谏阻纳质;始不垂
翅,终能奋翼。吊君鄱阳,蒋干来说;挥洒自如,雅量高志。吊君弘才,文武筹略;火攻破
敌,挽强为弱。想君当年,雄姿英发;哭君早逝,俯地流血。忠义之心,英灵之气;命终三
纪,名垂百世,哀君情切,愁肠千结;惟我肝胆,悲无断绝。昊天昏暗,三军怆然;主为哀
泣;友为泪涟。亮也不才,丐计求谋;助吴拒曹,辅汉安刘;掎角之援,首尾相俦,若存若
亡,何虑何忧?呜呼公瑾!生死永别!朴守其贞,冥冥灭灭,魂如有灵,以鉴我心:从此天
下,更无知音!呜呼痛哉!伏惟尚飨。”孔明祭毕,伏地大哭,泪如涌泉,哀恸不已。众将
相谓曰:“人尽道公瑾与孔明不睦,今观其祭奠之情,人皆虚言也。”鲁肃见孔明如此悲
切,亦为感伤,自思曰:“孔明自是多情,乃公瑾量窄,自取死耳。”后人有诗叹曰:“卧
龙南阳睡未醒,又添列曜下舒城。苍天既已生公瑾,尘世何须出孔明!”

鲁肃设宴款待孔明。宴罢,孔明辞回。方欲下船,只见江边一人道袍竹冠,皂绦素履,
一手揪住孔明大笑曰:“汝气死周郎,却又来吊孝,明欺东吴无人耶!”孔明急视其人,乃
凤雏先生庞统也。孔明亦大笑。两人携手登舟,各诉心事。孔明乃留书一封与统,嘱曰:
“吾料孙仲谋必不能重用足下。稍有不如意,可来荆州共扶玄德。此人宽仁厚德,必不负公
平生之所学。”统允诺而别,孔明自回荆州。

却说鲁肃送周瑜灵柩至芜湖,孙权接着,哭祭于前,命厚葬于本乡。瑜有两男一女,长
男循,次男胤,权皆厚恤之。鲁肃曰:“肃碌碌庸才,误蒙公瑾重荐,其实不称所职,愿举
一人以助主公。此人上通天文,下晓地理;谋略不减于管、乐,枢机可并于孙、吴。往日周
公瑾多用其言,孔明亦深服其智,现在江南,何不重用!”权闻言大喜,便问此人姓名。肃
曰:“此人乃襄阳人,姓庞,名统,字士元:道号凤雏先生。”权曰:“孤亦闻其名久矣。
今既在此,可即请来相见。”

于是鲁肃邀请庞统入见孙权。施礼毕。权见其人浓眉掀鼻,黑面短髯,形容古怪,心中
不喜。乃问曰:“公平生所学,以何为主?”统曰:“不必拘执,随机应变。”权曰:“公
之才学,比公瑾如何?”统笑曰:“某之所学,与公瑾大不相同。”权平生最喜周瑜,见统
轻之,心中愈不乐,乃谓统曰:“公且退。待有用公之时,却来相请。”统长叹一声而出。
鲁肃曰:“主公何不用庞士元?”权曰:“狂士也,用之何益!”肃曰:“赤壁鏖兵之时,
此人曾献连环策,成第一功。主公想必知之。”权曰:“此时乃曹操自欲钉船,未必此从之
功也,吾誓不用之。”

鲁肃出谓庞统曰:“非肃不荐足下,奈吴侯不肯用公。公且耐心。”统低头长叹不语。
肃曰:“公莫非无意于吴中乎?”统不答。肃曰:“公抱匡济之才,何往不利?可实对肃
言,将欲何往?”统曰:“吾欲投曹操去也。”肃曰:“此明珠暗投矣,可往荆州投刘皇
叔,必然重用。”统曰:“统意实欲如此,前言戏耳。”肃曰:“某当作书奉荐,公辅玄
德,必令孙、刘两家,无相攻击,同力破曹。”统曰:“此某平生之素志也。”乃求肃书。
径往荆州来见玄德。

此时孔明按察四郡未回,门吏传报:“江南名士庞统,特来相投。”玄德久闻统名,便
教请入相见。统见玄德,长揖不拜。玄德见统貌陋,心中亦不悦,乃问统曰:“足下远来不
易?”统不拿出鲁肃、孔明书投呈,但答曰:“闻皇叔招贤纳士,特来相投。”玄德曰:
“荆楚稍定,苦无闲职。此去东北一百三十里,有一县名耒阳县,缺一县宰,屈公任之,如
后有缺,却当重用。”统思:“玄德待我何薄!”欲以才学动之,见孔明不在,只得勉强相
辞而去。统到耒阳县,不理政事,终日饮酒为乐;一应钱粮词讼,并不理会。有人报知玄
德,言庞统将耒阳县事尽废。玄德怒曰:“竖儒焉敢乱吾法度!”遂唤张飞分付,引从人去
荆南诸县巡视:“如有不公不法者,就便究问。恐于事有不明处,可与孙乾同去。”张飞领
了言语,与孙乾前至耒阳县。军民官吏,皆出郭迎接,独不见县令。飞问曰:“县令何
在?”同僚覆曰:“庞县令自到任及今,将百余日,县中之事,并不理问,每日饮酒,自旦
及夜,只在醉乡。今日宿酒未醒,犹卧不起。”张飞大怒,欲擒之。孙乾曰:“庞士元乃高
明之人,未可轻忽。且到县问之。如果于理不当,治罪未晚。”飞乃入县,正厅上坐定,教
县令来见。统衣冠不整,扶醉而出。飞怒曰:“吾兄以汝为人,令作县宰,汝焉敢尽废县
事!”统笑曰:“将军以吾废了县中何事?”飞曰:“汝到任百余日,终日在醉乡,安得不
废政事?”统曰:“量百里小县,些小公事,何难决断!将军少坐,待我发落。”随即唤公
吏,将百余日所积公务,都取来剖断。吏皆纷然赍抱案卷上厅,诉词被告人等,环跪阶下。
统手中批判,口中发落,耳内听词,曲直分明,并无分毫差错。民皆叩首拜伏。

不到半日,将百余日之事,尽断毕了,投笔于地而对张飞曰:“所废之事何在!曹操、
孙权,吾视之若掌上观文,量此小县,何足介意!”飞大惊,下席谢曰:“先生大才,小子
失敬。吾当于兄长处极力举荐。”统乃将出鲁肃荐书。飞曰:“先生初见吾兄,何不将
出?”统曰:“若便将出,似乎专藉荐书来干谒矣。”飞顾谓孙乾曰:“非公则失一大贤
也。”遂辞统回荆州见玄德,具说庞统之才。玄德大惊曰:“屈待大贤,吾之过也!”飞将
鲁肃荐书呈上。玄德拆视之。书略曰:“庞士元非百里之才,使处治中、别驾之任,始当展
其骥足。如以貌取之,恐负所学,终为他人所用,实可惜也!”玄德看毕,正在嗟叹,忽报
孔明回。玄德接入,礼毕,孔明先明曰:“庞军师近日无恙否?”玄德曰:“近治耒阳县,
好酒废事。”孔明笑曰:“士元非百里之才,胸中之学,胜亮十倍。亮曾有荐书在士元处,
曾达主公否?”玄德曰:“今日方得子敬书,却未见先生之书。”孔明曰:“大贤若处小
任,往往以酒糊涂,倦于视事。”玄德曰:“若非吾弟所言,险失大贤。”随即令张飞往耒
阳县敬请庞统到荆州。玄德下阶请罪。统方将出孔明所荐之书。玄德看书中之意,言凤雏到
日,宜即重用。玄德喜曰:“昔司马德操言:‘伏龙、凤雏,两人得一,可安天下。’今吾
二人皆得,汉室可兴矣。”遂拜庞统为副军师中郎将,与孔明共赞方略,教练军士,听候征
伐。

早有人报到许昌,言刘备有诸葛亮、庞统为谋士,招军买马,积草屯粮,连结东吴,早
晚必兴兵北伐。曹操闻之,遂聚众谋士商议南征。荀攸进曰:“周瑜新死,可先取孙权,次
攻刘备。”操曰:“我若远征,恐马腾来袭许都。前在赤壁之时,军中有讹言,亦传西凉入
寇之事,今不可不防也。”荀攸曰:“以愚所见,不若降诏加马腾为征南将军,使讨孙权,
诱入京师,先除此人,则南征无患矣。”操大喜,即日遣人赍诏至西凉召马腾。却说腾字寿
成,汉伏波将军马援之后,父名肃,字子硕,桓帝时为天水兰干县尉;后失官流落陇西,与
羌人杂处,遂娶羌女生腾。腾身长八尺。体貌雄异,禀性温良,人多敬之。灵帝末年,羌人
多叛,腾招募民兵破之。初平中年,因讨贼有功,拜征西将军,与镇西将军韩遂为弟兄。当
日奉诏,乃与长子马超商议曰:“吾自与董承受衣带诏以来,与刘玄德约共讨贼,不幸董承
已死,玄德屡败。我又僻处西凉,未能协助玄德。今闻玄德已得荆州,我正欲展昔日之志,
而曹操反来召我,当是如何?”马超曰:“操奉天子之命以召父亲。今若不往,彼必以逆命
责我矣。当乘其来召,竟往京师,于中取事,则昔日之志可展也。”马腾兄子马岱谏曰:
“曹操心怀叵测,叔父若往,恐遭其害。”超曰:“儿愿尽起西凉之兵,随父亲杀入许昌,
为天下除害,有何不可?”腾曰:“汝自统羌兵保守西凉,只教次子马休、马铁并侄马岱随
我同往。曹操见有汝在西凉,又有韩遂相助,谅不敢加害于我也。”超曰:“父亲欲往,切
不可轻入京师。当随机应变,观其动静。”腾曰:“吾自有处,不必多虑。”

于是马腾乃引西凉兵五千,先教马休、马铁为前部,留马岱在后接应,迤逦望许昌而
来。离许昌二十里屯住军马。曹操听知马腾已到,唤门下侍郎黄奎分付曰:“目今马腾南
征,吾命汝为行军参谋,先至马腾寨中劳军,可对马腾说:西凉路远,运粮甚难,不能多带
人马。我当更遣大兵,协同前进。来日教他入城面君,吾就应付粮草与之。”奎领命,来见
马腾。腾置酒相待。奎酒半酣而言曰:“吾父黄琬死于李傕、郭汜之难,尝怀痛恨。不想今
日又遇欺君之贼!”腾曰:“谁为欺君之贼?”奎曰:“欺君者操贼也。公岂不知之,而问
我耶?”腾恐是操使来相探,急止之曰:“耳目较近,休得乱言。”奎叱曰:“公竟忘却衣
带诏乎!”腾见他说出心事,乃密以实情告之。奎曰:“操欲公入城面君,必非好意。公不
可轻入。来日当勒兵城下。待曹操出城点军,就点军处杀之,大事济矣。”二人商议已定。
黄奎回家,恨气未息。其妻再三问之,奎不肯言。不料其妾李春香、与奎妻弟苗泽私通。泽
欲得春香,正无计可施。妾见黄奎愤恨,遂对泽曰:“黄侍郎今日商议军情回,意甚愤恨,
不知为谁?”泽曰:“汝可以言挑之曰:“人皆说刘皇叔仁德,曹操奸雄,何也?看他说甚
言语。”是夜黄奎果到春香房中。妾以言挑之。奎乘醉言曰:“汝乃妇人,尚知邪正,何况
我乎?吾所恨者,欲杀曹操也!”妾曰:“若欲杀之,如何下手?”奎曰:“吾已约定马将
军,明日在城外点兵时杀之。”妾告于苗泽,泽报知曹操。操便密唤曹洪、许褚分付如此如
此;又唤夏侯渊、徐晃分付如此如此。各人领命去了,一面先将黄奎一家老小拿下。次日,
马腾领着西凉兵马,将次近城,只见前面一簇红旗,打着丞相旗号。马腾只道曹操自来点
军,拍马向前。忽听得一声炮响,红旗开处,弓弩齐发。一将当先,乃曹洪也。马腾急拨马
回时,两下喊声又起:左边许褚杀来,右边夏侯渊杀来,后面又是徐晃领兵杀至,截断西凉
军马,将马腾父子三人困在垓心。马腾见不是头,奋力冲杀。马铁早被乱箭射死。马休随着
马腾,左冲右突,不能得出。二人身带重伤,坐下马又被箭射倒。父子二人俱被执。曹操教
将黄奎与马腾父子,一齐绑至。黄奎大叫:“无罪!”操教苗泽对证。马腾大骂曰:“竖儒
误我大事!我不能为国杀贼,是乃天也!”操命牵出。马腾骂不绝口,与其子马休及黄奎,
一同遇害。后人有诗叹马腾曰:“父子齐芳烈,忠贞著一门,捐生图国难,誓死答君恩。嚼
血盟言在,诛奸义状存。西凉推世胄,不愧伏波孙!”苗泽告操曰:“不愿加赏,只求李春
香为妻。”操笑曰:“你为了一妇人,害了你姐夫一家,留此不义之人何用!”便教将苗
泽、李春香与黄奎一家老小并斩于市。观者无不叹息。后人有诗叹曰:“苗泽因私害荩臣,
春香未得反伤身。奸雄亦不相容恕,枉自图谋作小人。”

曹操教招安西凉兵马,谕之曰:“马腾父子谋反,不干众人之事。”一面使人分付把住
关隘,休教走了马岱。且说马岱自引一千兵在后。早有许昌城外逃回军士,报知马岱。岱大
惊,只得弃了兵马,扮作客商,连夜逃遁去了。曹操杀了马腾等,便决意南征。忽人报曰:
“刘备调练军马,收拾器械,将欲取川。”操惊曰:“若刘备收川,则羽翼成矣。将何以图
之?”言未毕,阶下一人进言曰:“某有一计,使刘备、孙权不能相顾,江南、西川皆归丞
相。”正是:西州豪杰方遭戮,南国英雄又受殃。未知献计者是谁,且看下文分解。