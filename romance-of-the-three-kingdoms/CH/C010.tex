\chapter{勤王室马腾举义~报父仇曹操兴师}

却说李、郭二贼欲弑献帝。张济、樊稠谏曰:“不可。今日若便杀之,恐众人不服,不
如仍旧奉之为主,赚诸侯入关,先去其羽翼,然后杀之,天下可图也。”李、郭二人从其
言,按住兵器。帝在楼上宣谕曰:“王允既诛,军马何故不退?”李傕、郭汜曰:“臣等有
功王室,未蒙赐爵,故不敢退军。”帝曰:“卿欲封何爵?”李、郭、张、樊四人各自写职
衔献上,勒要如此官品,帝只得从之。封李傕为车骑将军池阳侯领司隶校尉假节钺,郭汜为
后将军美阳侯假节钺,同秉朝政;樊稠为右将军万年侯,张济为骠骑将军平阳侯,领兵屯弘
农。其余李蒙、王方等,各为校尉。然后谢恩,领兵出城。又下令追寻董卓尸首,获得些零
碎皮骨,以香木雕成形体,安凑停当,大设祭祀,用王者衣冠棺椁,选择吉日,迁葬郿坞。
临葬之期,天降大雷雨,平地水深数尺,霹雳震开其棺,尸首提出棺外。李傕候晴再葬,是
夜又复如是。三次改葬,皆不能葬,零皮碎骨,悉为雷火消灭。天之怒卓。可谓甚矣!

且说李傕、郭汜既掌大权,残虐百姓;密遣心腹侍帝左右,观其动静。献帝此时举动荆
棘。朝廷官员,并由二贼升降。因采人望,特宣朱儁入朝封为太仆,同领朝政。一日,人报
西凉太守马腾;并州刺史韩遂二将引军十余万,杀奔长安来,声言讨贼。原来二将先曾使人
入长安,结连侍中马宇、谏议大夫种邵、左中郎将刘范三人为内应,共谋贼党。三人密奏献
帝,封马腾为征西将军、韩遂为镇西将军,各受密诏,并力讨贼。当下李傕、郭汜、张济、
樊稠闻二军将至,一同商议御敌之策。谋士贾诩曰:“二军远来,只宜深沟高垒,坚守以拒
之。不过百日,彼兵粮尽,必将自退,然后引兵追之,二将可擒矣。”李蒙、王方出曰:
“此非好计。愿借精兵万人,立斩马腾、韩遂之头,献于麾下。”贾诩曰:“今若即战,必
当败绩。”李蒙、王方齐声曰:“若吾二人败,情愿斩首;吾若战胜,公亦当输首级与
我。”诩谓李傕、郭汜曰:“长安西二百里盩厔山,其路险峻,可使张、樊两将军屯兵于
此,坚壁守之;待李蒙、王方自引兵迎敌,可也。”李傕、郭汜从其言,点一万五千人马与
李蒙、王方。二人忻喜而去,离长安二百八十里下寨。

西凉兵到,两个引军迎去。西凉军马拦路摆开阵势。马腾、韩遂联辔而出,指李蒙、王
方骂曰:“反国之贼!谁去擒之?”言未绝,只见一位少年将军,面如冠玉,眼若流星,虎
体猿臂,彪腹狼腰;手执长枪,坐骑骏马,从阵中飞出。原来那将即马腾之子马超,字孟
起,年方十七岁,英勇无敌。王方欺他年幼,跃马迎战。战不到数合,早被马超一枪刺于马
下。马超勒马便回。李蒙见王方刺死,一骑马从马超背后赶来。超只做不知。马腾在阵门下
大叫:“背后有人追赶!”声犹未绝,只见马超已将李蒙擒在马上。原来马超明知李蒙追
赶,却故意俄延;等他马近举枪刺来,超将身一闪,李蒙搠个空,两马相并,被马超轻舒猿
臂,生擒过去。军士无主,望风奔逃。马腾、韩遂乘势追杀,大获胜捷,直逼隘口下寨,把
李蒙斩首号令。李傕、郭汜听知李蒙、王方皆被马超杀了,方信贾诩有先见之明,重用其
计,只理会紧守关防,由他搦战,并不出迎。果然西凉军未及两月,粮草俱乏,商议回军。
恰好长安城中马宇家僮出首家主与刘范、种邵,外连马腾、韩遂,欲为内应等情。李傕、郭
汜大怒,尽收三家老少良贱斩于市,把三颗首级,直来门前号令。马腾、韩遂见军粮已尽,
内应又泄,只得拔寨退军。李傕、郭汜令张济引军赶马腾,樊稠引军赶韩遂,西凉军大败。
马超在后死战,杀退张济。樊稠去赶韩遂,看看赶上,相近陈仓,韩遂勒马向樊稠曰:“吾
与公乃同乡之人,今日何太无情?”樊稠也勒住马答道:“上命不可违!”韩遂曰:“吾此
来亦为国家耳,公何相逼之甚也?”樊稠听罢,拨转马头,收兵回寨,让韩遂去了。

不提防李傕之侄李别,见樊稠放走韩遂,回报其叔。李傕大怒,便欲兴兵讨樊稠。贾翊
曰:“目今人心未宁,频动干戈,深为不便;不若设一宴,请张济、樊稠庆功,就席间擒稠
斩之,毫不费力。”李傕大喜,便设宴请张济、樊稠。二将忻然赴宴。酒半阑,李傕忽然变
色曰:“樊稠何故交通韩遂,欲谋造反?”稠大惊,未及回言;只见刀斧手拥出,早把樊稠
斩首于案下。吓得张济俯伏于地。李傕扶起曰:“樊稠谋反,故尔诛之;公乃吾之心腹,何
须惊惧?”将樊稠军拨与张济管领。张济自回弘农去了。李傕、郭汜自战败西凉兵,诸侯莫
敢谁何。贾诩屡劝抚安百姓,结纳贤豪。自是朝廷微有生意。不想青州黄巾又起,聚众数十
万,头目不等,劫掠良民。太仆朱儁保举一人,可破群贼。李傕、郭汜问是何人。朱儁曰:
“要破山东群贼,非曹孟德不可。”李傕曰:“孟德今在何处?”儁曰:“现为东郡太守,
广有军兵。若命此人讨贼,贼可克日而破也。”李傕大喜,星夜草诏,差人赍往东郡,命曹
操与济北相鲍信一同破贼。操领了圣旨,会合鲍信,一同兴兵,击贼于寿阳。鲍信杀入重
地,为贼所害。操追赶贼兵,直到济北,降者数万。操即用贼为前驱,兵马到处,无不降
顺。不过百余日,招安到降兵三十余万、男女百余万口。操择精锐者,号为“青州兵”,其
余尽令归农。操自此威名日重。捷书报到长安,朝廷加曹操为镇东将军。操在兖州,招贤纳
士。有叔侄二人来投操:乃颍川颍阴人,姓荀,名彧,字文若,荀绲之子也;旧事袁绍,
今弃绍投操;操与语大悦,曰:“此吾之子房也!”遂以为行军司马。其侄荀攸,字公达,
海内名士,曾拜黄门侍郎,后弃官归乡,今与其叔同投曹操,操以为行军教授。荀彧曰:
“某闻兖州有一贤士,今此人不知何在。”操问是谁,彧曰:“乃东郡东阿人,姓程,名
昱,字仲德。”操曰:“吾亦闻名久矣。”遂遣人于乡中寻问。访得他在山中读书,操拜请
之。程昱来见,曹操大喜。昱谓荀彧曰:“某孤陋寡闻,不足当公之荐。公之乡人姓郭,名
嘉,字奉孝,乃当今贤士,何不罗而致之?”彧猛省曰:“吾几忘却!”遂启操徵聘郭嘉到
兖州,共论天下之事。郭嘉荐光武嫡派子孙,淮南成德人,姓刘,名晔,字子阳。操即聘晔
至。晔又荐二人:一个是山阳昌邑人,姓满,名宠,字伯宁;一个是武城人,姓吕,名虔,
字子恪。曹操亦素知这两个名誉,就聘为军中从事。满宠、吕虔共荐一人,乃陈留平邱人,
姓毛,名玠,字孝先。曹操亦聘为从事。

又有一将引军数百人,来投曹操:乃泰山巨平人,姓于,名禁,字文则。操见其人弓马
熟娴,武艺出众,命为点军司马。一日,夏侯惇引一大汉来见,操问何人,惇曰:“此乃陈
留人,姓典,名韦,勇力过人。旧跟张邈,与帐下人不和,手杀数十人,逃窜山中。惇出射
猎,见韦逐虎过涧,因收于军中。今特荐之于公。”操曰:“吾观此人容貌魁梧,必有勇
力。”惇曰:“他曾为友报仇杀人,提头直出闹市,数百人不敢近。只今所使两枝铁戟,重
八十斤,挟之上马,运使如飞。”操即令韦试之。韦挟戟骤马,往来驰骋。忽见帐下大旗为
风所吹,岌岌欲倒,众军士挟持不定;韦下马,喝退众军,一手执定旗杆,立于风中,巍然
不动。操曰:“此古之恶来也!”遂命为帐前都尉,解身上锦袄,及骏马雕鞍赐之。

自是曹操部下文有谋臣,武有猛将,威镇山东。乃遣泰山太守应劭,往瑯琊郡取父曹
嵩。嵩自陈留避难,隐居瑯琊;当日接了书信,便与弟曹德及一家老小四十余人,带从者百
余人,车百余辆,径望兖州而来。道经徐州,太守陶谦,字恭祖,为人温厚纯笃,向欲结纳
曹操,正无其由;知操父经过,遂出境迎接,再拜致敬,大设筵宴,款待两日。曹嵩要行,
陶谦亲送出郭,特差都尉张闿,将部兵五百护送。曹嵩率家小行到华、费间,时夏末秋初,
大雨骤至,只得投一古寺歇宿。寺僧接入。嵩安顿家小,命张闿将军马屯于两廊。众军衣
装,都被雨打湿,同声嗟怨。张闿唤手下头目于静处商议曰:“我们本是黄巾余党,勉强降
顺陶谦,未有好处。如今曹家辎重车辆无数,你们欲得富贵不难,只就今夜三更,大家砍将
入去,把曹嵩一家杀了,取了财物,同往山中落草。此计何如?”众皆应允。是夜风雨未
息,曹嵩正坐,忽闻四壁喊声大举。曹德提剑出看,就被搠死。曹嵩忙引一妾奔入方丈后,
欲越墙而走;妾肥胖不能出,嵩慌急,与妾躲于厕中,被乱军所杀。应劭死命逃脱,投袁绍
去了。张闿杀尽曹嵩全家,取了财物,放火烧寺,与五百人逃奔淮南去了。后人有诗曰:
“曹操奸雄世所夸,曾将吕氏杀全家。如今阖户逢人杀,天理循环报不差。”当下应劭部下
有逃命的军士,报与曹操。操闻之,哭倒于地。众人救起。操切齿曰:“陶谦纵兵杀吾父,
此仇不共戴天!吾今悉起大军,洗荡徐州,方雪吾恨!”遂留荀彧、程昱领军三万守鄄城、
范县、东阿三县,其余尽杀奔徐州来。夏侯惇、于禁、典韦为先锋。操令:但得城池,将城
中百姓,尽行屠戮,以雪父仇。当有九江太守边让,与陶谦交厚,闻知徐州有难,自引兵五
千来救。操闻之大怒,使夏侯惇于路截杀之。时陈宫为东郡从事,亦与陶谦交厚;闻曹操起
兵报仇,欲尽杀百姓,星夜前来见操。操知是为陶谦作说客,欲待不见,又灭不过旧恩,只
得请入帐中相见。宫曰:“今闻明公以大兵临徐州,报尊父之仇,所到欲尽杀百姓,某因此
特来进言。陶谦乃仁人君子,非好利忘义之辈;尊父遇害,乃张闿之恶,非谦罪也。且州县
之民,与明公何仇?杀之不祥。望三思而行。”操怒曰:“公昔弃我而去,今有何面目复来
相见?陶谦杀吾一家,誓当摘胆剜心,以雪吾恨!公虽为陶谦游说,其如吾不听何!”陈宫
辞出,叹曰:“吾亦无面目见陶谦也!”遂驰马投陈留太守张邈去了。

且说操大军所到之处,杀戮人民,发掘坟墓。陶谦在徐州,闻曹操起军报仇,杀戮百
姓,仰天恸哭曰:“我获罪于天,致使徐州之民,受此大难!”急聚众官商议。曹豹曰:
“曹兵既至,岂可束手待死!某愿助使君破之。”陶谦只得引兵出迎,远望操军如铺霜涌
雪,中军竖起白旗二面,大书报仇雪恨四字。军马列成阵势,曹操纵马出阵,身穿缟素,扬
鞭大骂。陶谦亦出马于门旗下,欠身施礼曰:“谦本欲结好明公,故托张闿护送。不想贼心
不改,致有此事。实不干陶谦之故。望明公察之。”操大骂曰:“老匹夫!杀吾父,尚敢乱
言!谁可生擒老贼?”夏侯惇应声而出。陶谦慌走入阵。夏侯惇赶来,曹豹挺枪跃马,前来
迎敌。两马相交,忽然狂风大作,飞沙走石,两军皆乱,各自收兵。

陶谦入城,与众计议曰:“曹兵势大难敌,吾当自缚往操营,任其剖割,以救徐州一郡
百姓之命。”言未绝,一人进前言曰:“府君久镇徐州,人民感恩。今曹兵虽众,未能即破
我城。府君与百姓坚守勿出;某虽不才,愿施小策,教曹操死无葬身之地!”众人大惊,便
问计将安出。正是:本为纳交反成怨,那知绝处又逢生。毕竟此人是谁,且听下文分解。