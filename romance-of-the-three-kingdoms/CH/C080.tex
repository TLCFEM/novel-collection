\chapter{曹丕废帝篡炎刘~汉王正位续大统}

却说华歆等一班文武,入见献帝。歆奏曰:“伏睹魏王,自登位以来,德布四方,仁及
万物,越古超今,虽唐、虞无以过此。群臣会议,言汉祚已终,望陛下效尧、舜之道,以山
川社稷,禅与魏王,上合天心,下合民意,则陛下安享清闲之福,祖宗幸甚!生灵幸甚!臣
等议定,特来奏请。”帝国奏大惊,半晌无言,觑百官而哭曰:“朕想高祖提三尺剑,斩蛇
起义,平秦灭楚,创造基业,世统相传,四百年矣。朕虽不才,初无过恶,安忍将祖宗大
业,等闲弃了?汝百官再从公计议。”华歆引李伏、许芝近前奏曰:“陛下若不信,可问此
二人。”李伏奏曰:“自魏王即位以来,麒麟降生,凤凰来仪,黄龙出现,嘉禾蔚生,甘露
下降。此是上天示瑞,魏当代汉之象也。”许芝又奏曰:“臣等职掌司天,夜观乾象,见炎
汉气数已终,陛下帝垦隐匿不明;魏国乾象,极天际地,言之难尽。更兼上应图谶,其谶
曰:鬼在边,委相连;当代汉,无可言。言在东,午在西;两日并光上下移。以此论之,陛
下可早禅位。鬼在边,委相连,是魏字也;言在东,午在西,乃许字也;两日并光上下移,
乃昌字也:此是魏在许昌应受汉禅也。愿陛下察之。”帝曰:“祥瑞图谶,皆虚妄之事;奈
何以虚妄之事,而遽欲朕舍祖宗之基业乎?”王朗奏曰:“自古以来,有兴必有废,有盛必
有衰,岂有不亡之国、不败之家乎?汉室相传四百余年,延至陛下,气数已尽,宜早退避,
不可迟疑;迟则生变矣。”帝大哭,入后殿去了。百官哂笑而退。

次日,官僚又集于大殿,令宦官入请献帝。帝忧惧不敢出。曹后曰:“百官请陛下设
朝,陛下何故推阻?”帝泣曰:“汝兄欲篡位,令百官相逼,朕故不出。”曹后大怒曰:
“吾兄奈何为此乱逆之事耶!”言未已,只见曹洪、曹休带剑而入,请帝出殿。曹后大骂
曰:“俱是汝等乱贼,希图富贵,共造逆谋!吾父功盖寰区,威震天下,然且不敢篡窃神
器。今吾兄嗣位未几,辄思篡汉,皇天必不祚尔!”言罢,痛哭入宫。左右侍者皆歔欷流
涕。曹洪、曹休力请献帝出殿。帝被逼不过,只得更衣出前殿。华歆奏曰:“陛下可依臣等
昨日之议,免遭大祸。”帝痛哭曰:“卿等皆食汉禄久矣;中间多有汉朝功臣子孙,何忍作
此不臣之事?”歆曰:“陛下若不从众议,恐旦夕萧墙祸起。非臣等不忠于陛下也。”帝
曰:“谁敢弑朕耶?”歆厉声曰:“天下之人,皆知陛下无人君之福,以致四方大乱!若非
魏王在朝,弑陛下者,何止一人?陛下尚不知恩报德,直欲令天下人共伐陛下耶?”帝大
惊,拂袖而起,王朗以目视华歆。歆纵步向前,扯住龙袍,变色而言曰:“许与不许,早发
一言!”帝战栗不能答,曹洪、曹休拔剑大呼曰:“符宝郎何在?”祖弼应声出曰:“符宝
郎在此!”曹洪索要玉玺。祖弼叱曰:“玉玺乃天子之宝,安得擅索!”洪喝令武士推出斩
之。祖弼大骂不绝口而死。后人有诗赞曰:“奸宄专权汉室亡,诈称禅位效虞唐。满朝百辟
皆尊魏,仅见忠臣符宝郎。”

帝颤栗不已。只见阶下披甲持戈数百余人,皆是魏兵。帝泣谓群臣曰:“朕愿将天下禅
于魏王,幸留残喘,以终天年。”贾诩曰:“魏王必不负陛下。陛下可急降诏,以安众
心。”帝只得令陈群草禅国之诏,令华歆赍捧诏玺,引百官直至魏王宫献纳。曹丕大喜。开
读诏曰:“朕在位三十二年,遭天下荡覆,幸赖祖宗之灵,危而复存。然今仰瞻天象,俯察
民心,炎精之数既终,行运在乎曹氏。是以前王既树神武之迹,今王又光耀明德,以应其
期。历数昭明,信可知矣。夫大道之行,天下为公;唐尧不私于厥子,而名播于无穷,朕窃
慕焉,今其追踵尧典,禅位于丞相魏王。王其毋辞!”

曹丕听毕,便欲受诏。司马懿谏曰:“不可。虽然诏玺已至,殿下宜且上表谦辞,以绝
天下之谤。”丕从之,令王朗作表,自称德薄,请别求大贤以嗣天位。帝览表,心甚惊疑,
谓群臣曰:“魏王谦逊,如之奈何?”华歆曰:“昔魏武王受王爵之时,三辞而诏不许,然
后受之,今陛下可再降诏,魏王自当允从。”帝不得已,又令桓阶草诏,遣高庙使张音,持
节奉玺至魏王宫。曹丕开读诏曰:“咨尔魏王,上书谦让。朕窃为汉道陵迟,为日已久;幸
赖武王操,德膺符运,奋扬神武,芟除凶暴,清定区夏。今王丕缵承前绪,至德光昭,声教
被四海,仁风扇八区;天之历数,实在尔躬。昔虞舜有大功二十,而放勋禅以天下;大禹有
疏导之绩,而重华禅以帝位。汉承尧运,有传圣之义,加顺灵袛,绍天明命,使行御史大夫
张音,持节奉皇帝玺绶。王其受之!”

曹丕接诏欣喜,谓贾诩曰:“虽二次有诏,然终恐天下后世,不免篡窃之名也。”诩
曰:“此事极易,可再命张音赍回玺绶,却教华歆令汉帝筑一坛,名受禅坛;择吉日良辰,
集大小公卿,尽到坛下,令天子亲奉玺绶,禅天下与王,便可以释群疑而绝众议矣。”丕大
喜,即令张音赍回玺绶,仍作表谦辞。音回奏献帝。帝问群臣曰:“魏王又让,其意若
何?”华歆奏曰:“陛下可筑一坛,名曰受禅坛,集公卿庶民,明白禅位;则陛下子子孙
孙,必蒙魏恩矣。”帝从之,乃遣太常院官,卜地于繁阳,筑起三层高坛,择于十月庚午日
寅时禅让。

至期,献帝请魏王曹丕登坛受禅,坛下集大小官僚四百余员,御林虎贲禁军三十余万,
帝亲捧玉玺奉曹丕。丕受之。坛下群臣跪听册曰:“咨尔魏王!昔者唐尧禅位于虞舜,舜亦
以命禹:天命不于常,惟归有德。汉道陵迟,世失其序;降及朕躬,大乱滋昏,群凶恣逆,
宇内颠覆。赖武王神武,拯兹难于四方,惟清区夏,以保绥我宗庙;岂予一人获乂,俾九服
实受其赐。今王钦承前绪,光于乃德;恢文武之大业,昭尔考之弘烈。皇灵降瑞,人神告
徵;诞惟亮采,师锡朕命。全曰尔度克协于虞舜,用率我唐典,敬逊尔位。於戏!天之历数
在尔躬,君其袛顺大礼,飨万国以肃承天命!”

读册已毕,魏王曹丕即受八般大礼,登了帝位。贾诩引大小官僚朝于坛下。改延康元年
为黄初元年。国号大魏。丕即传旨,大赦天下。谥父曹操为太祖武皇帝,华歆奏曰:“‘天
无二日,民无二主’。汉帝既禅天下,理宜退就藩服。乞降明旨,安置刘氏于何地?”言
讫,扶献帝跪于坛下听旨。丕降旨封帝为山阳公,即日便行。华歆按剑指帝,厉声而言曰:
“立一帝,废一帝,古之常道!今上仁慈,不忍加害,封汝为山阳公。今日便行,非宣召不
许入朝!”献帝含泪拜谢,上马而去。坛下军民人等见之,伤感不已。丕谓群臣曰:“舜、
禹之事,朕知之矣!”群臣皆呼万岁。后人观此受禅坛,有诗叹曰:“两汉经营事颇难,一
朝失却旧江山。黄初欲学唐虞事,司马将来作样看。”百官请曹丕答谢天地。丕方下拜,忽
然坛前卷起一阵怪风,飞砂走石,急如骤雨,对面不见;坛上火烛,尽皆吹灭。丕惊倒于坛
上,百官急救下坛,半晌方醒。侍臣扶入宫中,数日不能设朝。后病稍可,方出殿受群臣朝
贺。封华歆为司徒,王朗为司空;大小官僚,一一升赏。不疾未痊,疑许昌宫室多妖,乃自
许昌幸洛阳,大建宫室。

早有人到成都,报说曹丕自立为大魏皇帝,于洛阳盖造宫殿;且传言汉帝已遇害。汉中
王闻知,痛哭终日,下令百官挂孝,遥望设祭,上尊谥曰“孝愍皇帝”。玄德因此忧虑,致
染成疾,不能理事,政务皆托与孔明。

孔明与太傅许靖、光禄大夫谯周商议,言天下不可一日无君,欲尊汉中王为帝。谯周
曰:“近有祥风庆云之瑞;成都西北角有黄气数十丈,冲霄而起;帝星见于毕、胃、昴之
分,煌煌如月。此正应汉中王当即帝位,以继汉统,更复何疑?”于是孔明与许靖,引大小
官僚上表,请汉中王即皇帝位。汉中王览表,大惊曰:“卿等欲陷孤为不忠不义之人耶?”
孔明奏曰:“非也。曹丕篡汉自立,王上乃汉室苗裔,理合继统以延汉祀。”汉中王勃然变
色曰:“孤岂效逆贼所为!”拂袖而起,入于后宫。众官皆散。

三日后,孔明又引众官入朝,请汉中王出。众皆拜伏于前。许靖奏曰:“今汉天子已被
曹丕所弑,王上不即帝位,兴师讨逆,不得为忠义也。今天下无不欲王上为君,为孝愍皇帝
雪恨。若不从臣等所议,是失民望矣。”汉中王曰:“孤虽是景帝之孙,并未有德泽以布于
民;今一旦自立为帝,与篡窃何异!”孔明苦劝数次,汉中王坚执不从。

孔明乃设一计,谓众官曰:如此如此。于是孔明托病不出。汉中王闻孔明病笃,亲到府
中,直入卧榻边,问曰:“军师所感何疾?”孔明答曰:“忧心如焚,命不久矣!”汉中王
曰:“军师所忧何事?”连问数次,孔明只推病重,瞑目不答。汉中王再三请问。孔明喟然
叹曰:“臣自出茅庐,得遇大王,相随至今,言听计从;今幸大王有两川之地,不负臣夙昔
之言。目今曹丕篡位,汉祀将斩,文武官僚,咸欲奉大王为帝,灭魏兴刘,共图功名;不想
大王坚执不肯,众官皆有怨心,不久必尽散矣。若文武皆散,吴、魏来攻,两川难保。臣安
得不忧乎?”汉中王曰:“吾非推阻,恐天下人议论耳。”孔明曰:“圣人云:名不正则言
不顺,今大王名正言顺,有何可议?岂不闻天与弗取,反受其咎?”汉中王曰:“待军师病
可,行之未迟。”孔明听罢,从榻上跃然而起,将屏风一击,外面文武众官皆入,拜伏于地
曰:“王上既允,便请择日以行大礼。”汉中王视之,乃是太傅许靖、安汉将军糜竺、青衣
侯向举、阳泉侯刘豹、别驾赵祚、治中杨洪、议曹杜琼、从事张爽、太常卿赖恭、光禄卿黄
权、祭酒何宗、学士尹默、司业谯周、大司马殷纯、偏将军张裔、少府王谋、昭文博士伊
籍、从事郎秦宓等众也。

汉中王惊曰:“陷孤于不义,皆卿等也!”孔明曰:“王上既允所请,便可筑坛择吉,
恭行大礼。”即时送汉中王还宫,一面令博士许慈、谏议郎孟光掌礼,筑坛于成都武担之
南。诸事齐备,多官整设銮驾,迎请汉中王登坛致祭。谯周在坛上,高声朗读祭文曰:“惟
建安二十六年四月丙午朔,越十二日丁巳,皇帝备,敢昭告于皇天后土:汉有天下,历数无
疆。曩者王莽篡盗,光武皇帝震怒致诛,社稷复存。今曹操阻兵残忍,戮杀主后,罪恶滔
天;操子丕,载肆凶逆,窃据神器。群下将士,以为汉祀堕废,备宜延之,嗣武二祖,躬行
天罚。备惧无德忝帝位,询于庶民,外及遐荒君长,佥曰:天命不可以不答,祖业不可以久
替,四海不可以无主。率土式望,在备一人。备畏天明命,又惧高、光之业,将坠于地,谨
择吉日,登坛告祭,受皇帝玺绶,抚临四方。惟神飨祚汉家,永绥历服!”读罢祭文,孔明
率众官恭上玉玺。汉中王受了,捧于坛上,再三推辞曰:“备无才德,请择有才德者受
之。”孔明奏曰:“王上平定四海,功德昭于天下,况是大汉宗派,宜即正位。已祭告天
神,复何让焉!”文武各官,皆呼万岁。拜舞礼毕,改元章武元年。立妃吴氏为皇后,长子
刘禅为太子;封次子刘永为鲁王,三子刘理为梁王;封诸葛亮为丞相,许靖为司徒;大小官
僚,一一升赏。大赦天下。两川军民,无不欣跃。次日设朝,文武官僚拜毕,列为两班。先
主降诏曰:“朕自桃园与关、张结义,誓同生死。不幸二弟云长,被东吴孙权所害;若不报
仇,是负盟也。朕欲起倾国之兵,剪伐东吴,生擒逆贼,以雪此恨!”言未毕,班内一人拜
伏于阶下,谏曰:“不可。”先主视之,乃虎威将军赵云也。正是:君王未及行天讨,臣下
曾闻进直言。未知子龙所谏若何,且看下文分解。