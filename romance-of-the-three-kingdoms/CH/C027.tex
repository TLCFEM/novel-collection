\chapter{美髯公千里走单骑~汉寿侯五关斩六将}

却说曹操部下诸将中,自张辽而外,只有徐晃与云长交厚,其余亦皆敬服;独蔡阳不服关公,故今日闻其去,欲往追之。操曰:“不忘故主,来去明白,真丈夫也。汝等皆当效之。”遂叱退蔡阳,不令去赶。程昱曰:“丞相待关某甚厚,今彼不辞而去,乱言片楮,冒渎钧威,其罪大矣。若纵之使归袁绍,是与虎添翼也。不若追而杀了,以绝后患。”操曰:“吾昔已许之,岂可失信!彼各为其主,勿追也。”因谓张辽曰:“云长封金挂印,财贿不以动其心,爵禄不以移其志,此等人吾深敬之。想他去此不远,我一发结识他做个人情。汝可先去请住他,待我与他送行,更以路费征袍赠之,使为后日记念。”张辽领命,单骑先往。曹操引数十骑随后而来。

却说云长所骑赤兔马,日行千里,本是赶不上;因欲护送车仗,不敢纵马,按辔徐行。忽听背后有人大叫:“云长且慢行!”回头视之,见张辽拍马而至。关公教车仗从人,只管望大路紧行;自己勒住赤兔马,按定青龙刀,问曰:“文远莫非欲追我回乎?”辽曰:“非也。丞相知兄远行,欲来相送,特先使我请住台驾,别无他意。”关公曰:“便是丞相铁骑来,吾愿决一死战!”遂立马于桥上望之。见曹操引数十骑,飞奔前来,背后乃是许褚、徐晃、于禁、李典之辈。操见关公横刀立马于桥上,令诸将勒住马匹,左右排开。关公见众人手中皆无军器,方始放心。操曰:“云长行何太速?”关公于马上欠身答曰:“关某前曾禀过丞相。今故主在河北,不由某不急去。累次造府,不得参见,故拜书告辞,封金挂印,纳还丞相。望丞相勿忘昔日之言。”操曰:“吾欲取信于天下,安肯有负前言。恐将军途中乏用,特具路资相送。”一将便从马上托过黄金一盘。关公曰:“累蒙恩赐,尚有余资。留此黄金以赏将士。”操曰:“特以少酬大功于万一,何必推辞?”关公曰:“区区微劳,何足挂齿。”操笑曰:“云长天下义士,恨吾福薄,不得相留。锦袍一领,略表寸心。”令一将下马,双手捧袍过来。云长恐有他变,不敢下马,用青龙刀尖挑锦袍披于身上,勒马回头称谢曰:“蒙丞相赐袍,异日更得相会。”遂下桥望北而去。许褚曰:“此人无礼太甚,何不擒之?”操曰:“彼一人一骑,吾数十余人,安得不疑?吾言既出,不可追也。”曹操自引众将回城,于路叹想云长不已。

不说曹操自回。且说关公来赶车仗。约行三十里,却只不见。云长心慌,纵马四下寻之。忽见山头一人,高叫:“关将军且住!”云长举目视之,只见一少年,黄巾锦衣,持枪跨马,马项下悬着首级一颗,引百余步卒,飞奔前来。公问曰:“汝何人也?”少年弃枪下马,拜伏于地。云长恐是诈,勒马持刀问曰:“壮士,愿通姓名。”答曰:“吾本襄阳人,姓廖,名化,字元俭。因世乱流落江湖,聚众五百余人,劫掠为生。恰才同伴杜远下山巡哨,误将两夫人劫掠上山。吾问从者,知是大汉刘皇叔夫人,且闻将军护送在此,吾即欲送下山来。杜远出言不逊,被某杀之。今献头与将军请罪。”关公曰:“二夫人何在?”化曰:“现在山中。”关公教急取下山。不移时,百余人簇拥车仗前来。关公下马停刀,叉手于车前问候曰:“二嫂受惊否?”二夫人曰:“若非廖将军保全,已被杜远所辱。”关公问左右曰:“廖化怎生救夫人?”左右曰:“杜远劫上山去,就要与廖化各分一人为妻。廖化问起根由,好生拜敬,杜远不从,已被廖化杀了。”关公听言,乃拜谢廖化。廖化欲以部下人送关公。关公寻思此人终是黄巾余党,未可作伴,乃谢却之。廖化又拜送金帛,关公亦不受。廖化拜别,自引人伴投山谷中去了。云长将曹操赠袍事,告知二嫂,催促车仗前行。至天晚,投一村庄安歇。庄主出迎,须发皆白,问曰:“将军姓甚名谁?”关公施礼曰:“吾乃刘玄德之弟关某也。”老人曰:“莫非斩颜良、文丑的关公否?”公曰:“便是。”老人大喜,便请入庄。关公曰:“车上还有二位夫人。”老人便唤妻女出迎。二夫人至草堂上,关公叉手立于二夫人之侧。老人请公坐,公曰“尊嫂在上,安敢就坐!”老人乃令妻女请二夫人入内室款待,自于草堂款待关公。关公问老人姓名。老人曰:“吾姓胡,名华。桓帝时曾为议郎,致仕归乡。今有小儿胡班,在荣阳太守王植部下为从事。将军若从此处经过,某有一书寄与小儿。”关公允诺。次日早膳毕,请二嫂上车,取了胡华书信,相别而行,取路投洛阳来。前至一关,名东岭关。把关将姓孔,名秀,引五百军兵在岭上把守。当日关公押车仗上岭,军士报知孔秀,秀出关来迎。关公下马,与孔秀施礼。秀曰:“将军何往?”公曰:“某辞丞相,特往河北寻兄。”秀曰:“河北袁绍,正是丞相对头。将军此去,必有丞相文凭?”公曰:“因行期慌迫,不曾讨得。”秀曰:“既无文凭,待我差人禀过丞相,方可放行。”关公曰:“待去禀时,须误了我行程。”秀曰:“法度所拘,不得不如此。”关公曰:“汝不容我过关乎?”秀曰:“汝要过去,留下老小为质。”关公大怒,举刀就杀孔秀。秀退入关去,鸣鼓聚军,披挂上马,杀下关来,大喝曰:“汝敢过去么!”关公约退车仗,纵马提刀,竟不打话,直取孔秀。秀挺枪来迎。两马相交,只一合,钢刀起处,孔秀尸横马下。众军便走。关公曰:“军士休走。吾杀孔秀,不得已也,与汝等无干。借汝众军之口,传语曹丞相,言孔秀欲害我,我故杀之。”众军俱拜于马前。

关公即请二夫人车仗出关,望洛阳进发。早有军士报知洛阳太守韩福。韩福急聚众将商议。牙将孟坦曰:“既无丞相文凭,即系私行;若不阻挡,必有罪责。”韩福曰:“关公勇猛,颜良、文丑俱为所杀。今不可力敌,只须设计擒之。”孟坦曰:“吾有一计:先将鹿角拦定关口,待他到时,小将引兵和他交锋,佯败诱他来追,公可用暗箭射之。若关某坠马,即擒解许都,必得重赏。”商议停当,人报关公车仗已到。韩福弯弓插箭,引一千人马,排列关口,问:“来者何人?”关公马上欠身言曰:“吾汉寿亭侯关某,敢借过路。”韩福曰:“有曹丞相文凭否?”关公曰:“事冗不曾讨得。”韩福曰:“吾奉承相钧命,镇守此地,专一盘诘往来奸细。若无文凭,即系逃窜。”关公怒曰:“东岭孔秀,已被吾杀。汝亦欲寻死耶?”韩福曰:“谁人与我擒之?”孟坦出马,轮双刀来取关公。关公约退车仗,拍马来迎。孟坦战不三合,拨回马便走。关公赶来。孟坦只指望引诱关公,不想关公马快,早已赶上,只一刀,砍为两段。关公勒马回来,韩福闪在门首,尽力放了一箭,正射中关公左臂。公用口拔出箭,血流不住,飞马径奔韩福,冲散众军,韩福急走不迭,关公手起刀落,带头连肩,斩于马下;杀散众军,保护车仗。

关公割帛束住箭伤,于路恐人暗算,不敢久住,连夜投汜水关来。把关将乃并州人氏,姓卞,名喜,善使流星锤;原是黄巾余党,后投曹操,拨来守关。当下闻知关公将到,寻思一计:就关前镇国寺中,埋伏下刀斧手二百余人,诱关公至寺,约击盏为号,欲图相害。安排已定,出关迎接关公。公见卞喜来迎,便下马相见。喜曰:“将军名震天下,谁不敬仰!今归皇叔,足见忠义!”关公诉说斩孔秀、韩福之事。卞喜曰:“将军杀之是也。某见丞相,代禀衷曲。”关公甚喜,同上马过了汜水关,到镇国寺前下马。众僧鸣钟出迎。原来那镇国寺乃汉明帝御前香火院,本寺有僧三十余人。内有一僧,却是关公同乡人,法名普净。当下普净已知其意,向前与关公问讯,曰:“将军离蒲东几年矣?”关公曰:“将及二十年矣。”普净曰:“还认得贫僧否?”公曰:“离乡多年,不能相识。”普净曰:“贫僧家与将军家只隔一条河。”卞喜见普净叙出乡里之情,恐有走泄,乃叱之曰:“吾欲请将军赴宴,汝僧人何得多言!”关公曰:“不然。乡人相遇,安得不叙旧情耶?”普净请关公方丈待茶。关公曰:“二位夫人在车上,可先献茶。”普净教取茶先奉夫人,然后请关公入方丈。普净以手举所佩戒刀,以目视关公。公会意,命左右持刀紧随。

卞喜请关公于法堂筵席。关公曰:“卞君请关某,是好意,还是歹意?”卞喜未及回言,关公早望见壁衣中有刀斧手,乃大喝卞喜曰:“吾以汝为好人,安敢如此!”卞喜知事泄,大叫:“左右下手!”左右方欲动手,皆被关公拔剑砍之。卞喜下堂绕廊而走,关公弃剑执大刀来赶。卞喜暗取飞锤掷打关公。关公用刀隔开锤,赶将入去,一刀劈卞喜为两段。随即回身来看二嫂,早有军人围住,见关公来,四下奔走。关公赶散,谢普净曰:“若非吾师,已被此贼害矣。”普净曰:“贫僧此处难容,收拾衣钵,亦往他处云游也。后会有期,将军保重。”关公称谢,护送车仗,往荥阳进发。荥阳太守王植,却与韩福是两亲家;闻得关公杀了韩福,商议欲暗害关公,乃使人守住关口。待关公到时,王植出关,喜笑相迎。关公诉说寻兄之事。植曰:“将军于路驱驰,夫人车上劳困,且请入城,馆驿中暂歇一宵,来日登途未迟。”关公见王植意甚殷勤,遂请二嫂入城。馆驿中皆铺陈了当。王植请公赴宴,公辞不往;植使人送筵席至馆驿。关公因于路辛苦,请二嫂晚膳毕,就正房歇定;令从者各自安歇,饱喂马匹。关公亦解甲憩息。却说王植密唤从事胡班听令曰:“关某背丞相而逃,又于路杀太守并守关将校,死罪不轻!此人武勇难敌。汝今晚点一千军围住馆驿,一人一个火把,待三更时分,一齐放火;不问是谁,尽皆烧死!吾亦自引军接应。”胡班领命,便点起军士,密将干柴引火之物,搬于馆驿门首,约时举事。

胡班寻思:“我久闻关云长之名,不识如何模样,试往窥之。”乃至驿中,问驿吏曰:“关将军在何处?”答曰:“正厅上观书者是也。”胡班潜至厅前,见关公左手绰髯,于灯下凭几看书。班见了,失声叹曰:“真天人也!”公问何人,胡班入拜曰:“荥阳太守部下从事胡班。”关公曰:“莫非许都城外胡华之子否?”班曰:“然也。”公唤从者于行李中取书付班。班看毕,叹曰:“险些误杀忠良!”遂密告曰:“王植心怀不仁,欲害将军,暗令人四面围住馆驿,约于三更放火。今某当先去开了城门,将军急收拾出城。”

关公大惊,忙披挂提刀上马,请二嫂上车,尽出馆驿,果见军士各执火把听候。关公急来到城边,只见城门已开。关公催车仗急急出城。胡班还去放火。关公行不到数里,背后火把照耀,人马赶来。当先王植大叫:“关某休走!”关公勒马,大骂:“匹夫!我与你无仇,如何令人放火烧我?”王植拍马挺枪,径奔关公,被关公拦腰一刀,砍为两段。人马都赶散。关公催车仗速行,于路感胡班不已。

行至滑州界首,有人报与刘延。延引数十骑,出郭而迎。关公马上欠身而言曰:“太守别来无恙!”延曰:“公今欲何往?”公曰:“辞了丞相,去寻家兄。”延曰:“玄德在袁绍处,绍乃丞相仇人,如何容公去?”公曰:“昔日曾言定来。”延曰:“今黄河渡口关隘,夏侯惇部将秦琪据守,恐不容将军过渡。”公曰:“太守应付船只,若何?”延曰:“船只虽有,不敢应付。”公曰:“我前者诛颜良、文丑,亦曾与足下解厄。今日求一渡船而不与,何也?”延曰:“只恐夏侯惇知之,必然罪我。”关公知刘延无用之人,遂自催车仗前进。到黄河渡口,秦琪引军出问:“来者何人?”关公曰:“汉寿亭侯关某也。”琪曰:“今欲何往?”关公曰:“欲投河北去寻兄长刘玄德,敬来借渡。”琪曰:“丞相公文何在?”公曰:“吾不受丞相节制,有甚公文!”琪曰:“吾奉夏侯将军将令,守把关隘,你便插翅,也飞不过去!”关公大怒曰:“你知我于路斩戮拦截者乎?”琪曰:“你只杀得无名下将,敢杀我么?”关公怒曰:“汝比颜良、文丑若何?”秦琪大怒,纵马提刀,直取关公。二马相交,只一合,关公刀起,秦琪头落。关公曰:“当吾者已死,余人不必惊走。速备船只,送我渡河。”军士急撑舟傍岸。关公请二嫂上船渡河。渡过黄河,便是袁绍地方。关公所历关隘五处,斩将六员。后人有诗叹曰:“挂印封金辞汉相,寻兄遥望远途还。马骑赤兔行千里,刀偃青龙出五关。忠义慨然冲宇宙,英雄从此震江山。独行斩将应无敌,今古留题翰墨间。”

关公于马上自叹曰:“吾非欲沿途杀人,奈事不得已也。曹公知之,必以我为负恩之人矣。”正行间,忽见一骑自北而来,大叫:“云长少住!”关公勒马视之,乃孙乾也。关公曰:“自汝南相别,一向消息若何?”乾曰:“刘辟、龚都自将军回兵之后,复夺了汝南;遣某往河北结好袁绍,请玄德同谋破曹之计。不想河北将士,各相妒忌。田丰尚囚狱中;沮授黜退不用;审配、郭图各自争权;袁绍多疑,主持不定。某与刘皇叔商议,先求脱身之计。今皇叔已往汝南会合刘辟去了。恐将军不知,反到袁绍处,或为所害,特遣某于路迎接将来。幸于此得见。将军可速往汝南与皇叔相会。”关公教孙乾拜见夫人。夫人问其动静。孙乾备说袁绍二次欲斩皇叔,今幸脱身往汝南去了。夫人可与云长到此相会。二夫人皆掩面垂泪。关公依言,不投河北去,径取汝南来。正行之间,背后尘埃起处,一彪人马赶来,当先夏侯惇大叫:“关某休走!”正是:六将阻关徒受死,一军拦路复争锋。毕竟关公怎生脱身,且听下文分解。