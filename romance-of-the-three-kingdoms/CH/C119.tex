\chapter{假投降巧计成虚话~再受禅依样画葫芦}

却说钟会请姜维计议收邓艾之策。维曰:“可先令监军卫瓘收艾。艾若杀瓘,反情实
矣。将军却起兵讨之,可也。”会大喜,遂令卫瓘引数十人入成都,收邓艾父子。瓘手下人
止之曰:“此是钟司徒令邓征西杀将军,以正反情也。切不可行。”瓘曰:“吾自有计。”
遂先发檄文二三十道。其檄曰:“奉诏收艾,其余各无所问。若早来归,爵赏如先,敢有不
出者,灭三族。”随备槛车两乘,星夜望成都而来。

比及鸡鸣,艾部将见檄文者,皆来投拜于卫瓘马前。时邓艾在府中未起。瓘引数十人突
入大呼曰:“奉诏收邓艾父子!”艾大惊,滚下床来。瓘叱武士缚于车上。其子邓忠出问,
亦被捉下,缚于车上。府中将吏大惊,欲待动手抢夺,早望见尘头大起,哨马报说钟司徒大
兵到了。众各四散奔走。钟会与姜维下马入府,见邓艾父子已被缚,会以鞭挞邓艾之首而骂
曰:“养犊小儿,何敢如此!”姜维亦骂曰:“匹夫行险徼幸,亦有今日耶!”艾亦大骂。
会将艾父子送赴洛阳。会入成都,尽得邓艾军马,威声大震。乃谓姜维曰:“吾今日方趁平
生之愿矣!”维曰:“昔韩信不听蒯通之说,而有未央宫之祸;大夫种不从范蠡于五湖,卒
伏剑而死:斯二子者,其功名岂不赫然哉,徒以利害未明,而见机之不早也。今公大勋已
就,威震其主,何不泛舟绝迹,登峨嵋之岭,而从赤松子游乎?”会笑曰:“君言差矣。吾
年未四旬,方思进取,岂能便效此退闲之事?”维曰:“若不退闲,当早图良策。此则明公
智力所能,无烦老夫之言矣。”会抚掌大笑曰:“伯约知吾心也。”二人自此每日商议大
事。维密与后主书曰:“望陛下忍数日之辱,维将使社稷危而复安,日月幽而复明。必不使
汉室终灭也。”

却说钟会正与姜维谋反,忽报司马昭有书到。会接书。书中言:“吾恐司徒收艾不下,
自屯兵于长安;相见在近,以此先报。”会大惊曰:“吾兵多艾数倍,若但要我擒艾,晋公
知吾独能办之。今日自引兵来,是疑我也!”遂与姜维计议。维曰:“君疑臣则臣必死,岂
不见邓艾乎?”会曰:“吾意决矣!事成则得天下,不成则退西蜀,亦不失作刘备也。”维
曰:“近闻郭太后新亡,可诈称太后有遗诏,教讨司马昭,以正弑君之罪。据明公之才,中
原可席卷而定。”会曰:“伯约当作先锋。成事之后,同享富贵。”维曰:“愿效犬马微
劳,但恐诸将不服耳。”会曰:“来日元宵佳节,于故宫大张灯火,请诸将饮宴。如不从者
尽杀之。”维暗喜。次日,会、维二人请诸将饮宴。数巡后,会执杯大哭。诸将惊问其故,
会曰:“郭太后临崩有遗诏在此,为司马昭南阙弑君,大逆无道,早晚将篡魏,命吾讨之。
汝等各自佥名,共成此事。”众皆大惊,面面相觑。会拔剑出鞘曰:“违令者斩!”众皆恐
惧,只得相从。画字已毕,会乃困诸将于宫中,严兵禁守。维曰:“我见诸将不服,请坑
之。”会曰:“吾已令宫中掘一坑,置大棒数千;如不从者,打死坑之。”

时有心腹将丘建在侧。建乃护军胡烈部下旧人也,时胡烈亦被监在宫。建乃密将钟会所
言,报知胡烈。烈大惊,泣告曰:“吾儿胡渊领兵在外,安知会怀此心耶?汝可念何日之
情,透一消息,虽死无恨。”建曰:“恩主勿忧,容某图之。”遂出告会曰:“主公软监诸
将在内,水食不便,可令一人往来传递。”会素听丘建之言,遂令丘建监临。会分付曰:
“吾以重事托汝,休得泄漏。”建曰:“主公放心,某自有紧严之法。”建暗令胡烈亲信人
入内,烈以密书付其人。其人持书火速至胡渊营内,细言其事,呈上密书。渊大惊,遂遍示
诸营知之。众将大怒,急来渊营商议曰:“我等虽死,岂肯从反臣耶?”渊曰:“正月十八
日中,可骤入内,如此行之。”监军卫瓘深喜胡渊之谋,即整顿了人马,令丘建传与胡烈。
烈报知诸将。

却说钟会请姜维问曰:“吾夜梦大蛇数千条咬吾,主何吉凶?”维曰:“梦龙蛇者,皆
吉庆之兆也。”会喜,信其言,乃谓维曰:“器伏已备,放诸将出问之,若何?”维曰:
“此辈皆有不服之心,久必为害,不如乘早戮之。”会从之,即命姜维领武士往杀众魏将。
维领命,方欲行动,忽然一阵心疼,昏倒在地;左右扶起,半晌方苏。忽报宫外人声沸腾。
会方令人探时,喊声大震,四面八方,无限兵到。维曰:“此必是诸将作恶,可先斩之。”
忽报兵已入内。会令闭上殿门,使军士上殿屋以瓦击之,互相杀死数十人。宫外四面火起,
外兵砍开殿门杀入。会自掣剑立杀数人,却被乱箭射倒。众将枭其首。维拔剑上殿,往来冲
突,不幸心疼转加。维仰天大叫曰:“吾计不成,乃天命也!”遂自刎而死。时年五十九
岁。宫中死者数百人。卫瓘曰:“众军各归营所,以待王命。”魏兵争欲报仇,共剖维腹,
其胆大如鸡卵。众将又尽取姜维家属杀之。邓艾部下之人,见钟会、姜维已死,遂连夜去追
劫邓艾。早有人报知卫瓘。瓘曰:“是我捉艾;今若留他,我无葬身之地矣。”护军田续
曰:“昔邓艾取江油之时,欲杀续,得众官告免。今日当报此恨!”瓘大喜,遂遣田续引五
百兵赶至绵竹,正遇邓艾父子放出槛车,欲还成都。艾只道是本部兵到,不作准备;欲待问
时,被田续一刀斩之。邓忠亦死于乱军之中。后人有诗叹邓艾曰:“自幼能筹画,多谋善用
兵。凝眸知地理,仰面识天文。马到山根断,兵来石径分。功成身被害,魂绕汉江云。”又
有诗叹钟会曰:“髫年称早慧,曾作秘书郎。妙计倾司马,当时号子房。寿春多赞画,剑阁
显鹰扬。不学陶朱隐,游魂悲故乡。”又有诗叹姜维曰:“天水夸英俊,凉州产异才。系从
尚父出,术奉武侯来。大胆应无惧,雄心誓不回。成都身死日,汉将有余哀。”

却说姜维、钟会、邓艾已死,张翼等亦死于乱军之中。太子刘璇、汉寿亭侯关彝,皆被
魏兵所杀。军民大乱,互相践踏,死者不计其数。旬日后,贾充先至,出榜安民。方始宁
靖。留卫瓘守成都,乃迁后主赴洛阳。止有尚书令樊建、侍中张绍、光禄大夫谯周、秘书郎
郤正等数人跟随。廖化、董厥皆托病不起,后皆忧死。

时魏景元五年改为咸熙元年,春三月,吴将丁奉见蜀已亡,遂收兵还吴。中书丞华覈奏
吴主孙休曰:“吴、蜀乃唇齿也,唇亡则齿寒;臣料司马昭伐吴在即,乞陛下深加防御。”
休从其言,遂命陆逊子陆抗为镇东大将军,领荆州牧,守江口;左将军孙异守南徐诸处隘
口;又沿江一带,屯兵数百营,老将丁奉总督之,以防魏兵。

建宁太守霍戈闻成都不守,素服望西大哭三日。诸将皆曰:“既汉主失位,何不速降,
戈泣谓曰:“道路隔绝,未知吾主安危若何。若魏主以礼待之,则举城而降,未为晚也;万
一危辱吾主,则主辱臣死,何可降乎?”众然其言,乃使人到洛阳,探听后主消息去了。

且说后主至洛阳时,司马昭已自回朝。昭责后主曰:“公荒淫无道,废贤失政,理宜诛
戮。”后主面如土色,不知所为。文武皆奏曰:“蜀主既失国纪,幸早归降,宜赦之。”昭
乃封禅为安乐公,赐住宅,月给用度,赐绢万匹,僮婢百人。子刘瑶及群臣樊建、谯周、郤
正等,皆封侯爵。后主谢恩出内。昭因黄皓蠹国害民,令武士押出市曹,凌迟处死。时霍戈
探听得后主受封,遂率部下军士来降。次日,后主亲诣司马昭府下拜谢。昭设宴款待,先以
魏乐舞戏于前,蜀官感伤,独后主有喜色。昭令蜀人扮蜀乐于前,蜀官尽皆堕泪,后主嬉笑
自若。酒至半酣,昭谓贾充曰:“人之无情,乃至于此!虽使诸葛孔明在,亦不能辅之久
全,何况姜维乎?”乃问后主曰:“颇思蜀否?”后主曰:“此间乐,不思蜀也。”须臾,
后主起身更衣,郤正跟至厢下曰:“陛下如何答应不思蜀也?徜彼再问,可泣而答曰:先人
坟墓,远在蜀地,乃心西悲,无日不思。晋公必放陛下归蜀矣。”后主牢记入席。酒将微
醉,昭又问曰:“颇思蜀否?”后主如郤正之言以对,欲哭无泪,遂闭其目。昭曰:“何乃
似郤正语耶?”后主开目惊视曰:“诚如尊命。”昭及左右皆笑之。昭因此深喜后主诚实,
并不疑虑。后人有诗叹曰:“追欢作乐笑颜开,不念危亡半点哀。快乐异乡忘故国,方知后
主是庸才。”

却说朝中大臣因昭收川有功,遂尊之为王,表奏魏主曹奂。时奂名为天子,实不能主
张,政皆由司马氏,不敢不从,遂封晋公司马昭为晋王,谥父司马懿为宣王,兄司马师为景
王。昭妻乃王肃之女,生二子:长曰司马炎,人物魁伟,立发垂地,两手过膝,聪明英武,
胆量过人;次曰司马攸,情性温和,恭俭孝悌,昭甚爱之,因司马师无子,嗣攸以继其后。
昭常曰:“天下者,乃吾兄之天下也。”于是司马昭受封晋王,欲立攸为世子。山涛谏曰:
“废长立幼,违礼不祥。”贾充、何曾、裴秀亦谏曰:“长子聪明神武,有超世之才;人望
既茂,天表如此:非人臣之相也。”昭犹豫未决。太尉王祥、司空荀顗谏曰:“前代立少,
多致乱国。愿殿下思之。”昭遂立长子司马炎为世子。大臣奏称:“当年襄武县,天降一
人,身长二丈余,脚迹长三尺二寸,白发苍髯,着黄单衣;裹黄巾,挂藜头杖,自称曰:吾
乃民王也。今来报汝:天下换主,立见太平。如此在市游行三日,忽然不见。此乃殿下之瑞
也。殿下可戴十二旒冠冕,建天子旌旗,出警入跸,乘金根车,备六马,进王妃为王后,立
世子为太子。”昭心中暗喜;回到宫中,正欲饮食,忽中风不语。次日,病危,太尉王祥、
司徒何曾、司马荀顗及诸大臣入宫问安,昭不能言,以手指太子司马炎而死。时八月辛卯日
也。何曾曰:“天下大事,皆在晋王;可立太子为晋王,然后祭葬。”是日,司马炎即晋王
位,封何曾为晋丞相,司马望为司徒,石苞为骠骑将军,陈骞为车骑将军,谥父为文安葬已
毕,炎召贾充、裴秀入宫问曰:“曹操曾云:若天命在吾,吾其为周文王乎!果有此事
否?”充曰:“操世受汉禄,恐人议论篡逆之名,故出此言。乃明教曹丕为天子也。”炎
曰:“孤父王比曹操何如?”充曰:“操虽功盖华夏,下民畏其威而不怀其德。子丕继业,
差役甚重,东西驱驰,未有宁岁。后我宣王、景王,累建大功,布恩施德,天下归心久矣。
文王并吞西蜀,功盖寰宇。又岂操之可比乎?”炎曰:“曹丕尚绍汉统,孤岂不可绍魏统
耶?”贾充、裴秀二人再拜而奏曰:“殿下正当法曹丕绍汉故事,复筑受禅坛,布告天下,
以即大位。”炎大喜,次日带剑入内。此时,魏主曹奂连日不曾设朝,心神恍惚,举止失
措。炎直入后宫,奂慌下御榻而迎。炎坐毕,问曰:“魏之天下,谁之力也?”奂曰:“皆
晋王父祖之赐耳。”炎笑曰:“吾观陛下,文不能论道,武不能经邦。何不让有才德者主
之?”奂大惊,口噤不能言。傍有黄门侍郎张节大喝曰:“晋王之言差矣!昔日魏武祖皇
帝,东荡西除,南征北讨,非容易得此天下;今天子有德无罪,何故让与人耶?”炎大怒
曰:“此社稷乃大汉之社稷也。曹操挟天子以令诸侯,自立魏王,篡夺汉室。吾祖父三世辅
魏,得天下者,非曹氏之能,实司马氏之力也:四海咸知。吾今日岂不堪绍魏之天下乎?”
节又曰:“欲行此事,是篡国之贼也!”炎大怒曰:“吾与汉家报仇,有何不可!”叱武士
将张节乱瓜打死于殿下。奂泣泪跪告。炎起身下殿而去。奂谓贾充、裴秀曰:“事已急矣,
如之奈何?”充曰:“天数尽矣,陛下不可逆天,当照汉献帝故事,重修受禅坛,具大礼,
禅位与晋王:上合天心,下顺民情,陛下可保无虞矣。”

奂从之,遂令贾充筑受禅坛。以十二月甲子日,奂亲捧传国玺,立于坛上,大会文武。
后人有诗叹曰:“魏吞汉室晋吞曹,天运循环不可逃。张节可怜忠国死,一拳怎障泰山
高。”请晋王司马炎登坛,授与大礼。奂下坛,具公服立于班首。炎端坐于坛上。贾充、裴
秀列于左右,执剑,令曹奂再拜伏地听命。充曰:“自汉建安二十五年,魏受汉禅,已经四
十五年矣;今天禄永终,天命在晋。司马氏功德弥隆,极天际地,可即皇帝正位,以绍魏
统。封汝为陈留王,出就金墉城居止;当时起程,非宣诏不许入京。”奂泣谢而去。太傅司
马孚哭拜于奂前曰:“臣身为魏臣,终不背魏也。”炎见孚如此,封孚为安平王。孚不受而
退。是日,文武百官,再拜于坛下,山呼万岁。炎绍魏统,国号大晋,改元为泰始元年,大
赦天下。魏遂亡。后人有诗叹曰:“晋国规模如魏王,陈留踪迹似山阳。重行受禅台前事,
回首当年止自伤。

晋帝司马炎,追谥司马懿为宣帝,伯父司马师为景帝,父司马昭为文帝,立七庙以光祖
宗。那七庙?汉征西将军司马钧,钧生豫章太守司马量,量生颍川太守司马隽,隽生京兆尹
司马防,防生宣帝司马懿,懿生景帝司马师、文帝司马昭:是为七庙也。大事已定,每日设
朝计议伐吴之策。正是:汉家城郭已非旧,吴国江山将复更。未知怎生伐吴,且看下文分
解。