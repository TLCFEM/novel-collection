\chapter{玄德用计袭樊城~元直走马荐诸葛}

却说曹仁忿怒,遂大起本部之兵,星夜渡河,意欲踏平新野。且说单福得胜回县,谓玄
德曰:“曹仁屯兵樊城,今知二将被诛,必起大军来战。”玄德曰:“当何以迎之?”福
曰:“彼若尽提兵而来,樊城空虚,可乘间夺之。”玄德问计。福附耳低言如此如此。玄德
大喜,预先准备已定。忽报马报说:“曹仁引大军渡河来了。”单福曰:“果不出吾之
料。”遂请玄德出军迎敌。两阵对圆,赵云出马唤彼将答话。曹仁命李典出阵,与赵云交
锋。约战十数合,李典料敌不过,拨马回阵。云纵马追赶,两翼军射住,遂各罢兵归寨。李
典回见曹仁,言:“彼军精锐,不可轻敌,不如回樊城。”曹仁大怒曰:“汝未出军时,已
慢吾军心;今又卖阵,罪当斩首!”便喝刀斧手推出李典要斩;众将苦告方免。乃调李典领
后军,仁自引兵为前部。次日鸣鼓进军,布成一个阵势,使人问玄德曰:“识吾阵势?”单
福便上高处观看毕,谓玄德曰:“此八门金锁阵也。八门者:休、生、伤、杜、景、死、
惊、开。如从生门、景门、开门而入则吉;从伤门、惊门、休门而入则伤;从杜门、死们而
人则亡。今八门虽布得整齐,只是中间通欠主持。如从东南角上生门击人,往正西景门而
出,其阵必乱。”玄德传令,教军士把住阵角,命赵云引五百军从东南而入,径往西出。云
得令,挺枪跃马,引兵径投东南角上,呐喊杀入中军。曹仁便投北走。云不追赶,却突出西
门,又从西杀转东南角上来。曹仁军大乱。玄德麾军冲击,曹兵大败而退。单福命休追赶,
收军自回。却说曹仁输了一阵,方信李典之言;因复请典商议,言:“刘备军中必有能者,
吾阵竟为所破。”李典曰:“吾虽在此,甚忧樊城。”曹仁曰:“今晚去劫寨。如得胜,再
作计议;如不胜,便退军回樊城。”李典曰:“不可。刘备必有准备。”仁曰:“若如此多
疑,何以用兵!”遂不听李典之言。自引军为前队,使李典为后应,当夜二更劫寨。

却说单福正与玄德在寨中议事,忽信风骤起。福曰:“今夜曹仁必来劫寨。”玄德曰:
“何以敌之?”福笑曰:“吾已预算定了。”遂密密分拨已毕。至二更,曹仁兵将近寨,只
见寨中四围火起,烧着寨栅。曹仁知有准备,急令退军。赵云掩杀将来。仁不及收兵回寨,
急望北河而走。将到河边,才欲寻船渡河,岸上一彪军杀到:为首大将,乃张飞也。曹仁死
战,李典保护曹仁下船渡河。曹军大半淹死水中。曹仁渡过河面,上岸奔至樊城,令人叫
门。只见城上一声鼓响,一将引军而出,大喝曰:“吾已取樊城多时矣!”众惊视之,乃关
云长也。仁大惊,拨马便走。云长追杀过来。曹仁又折了好些军马,星夜投许昌。于路打
听,方知有单福为军师,设谋定计。不说曹仁败回许昌。且说玄德大获全胜,引军入樊城,
县令刘泌出迎。玄德安民已定。那刘泌乃长沙人,亦汉室宗亲,遂请玄德到家,设宴相待。
只见一人侍立于侧。玄德视其人器宇轩昂,因问泌曰:“此何人?”泌曰:“此吾之甥寇
封,本罗侯寇氏之子也;因父母双亡,故依于此。”玄德爱之,欲嗣为义子。刘泌欣然从
之,遂使寇封拜玄德为父,改名刘封。玄德带回,令拜云长、翼德为叔。云长曰:“兄长既
有子,何必用螟蛉?后必生乱。”玄德曰:“吾待之如子,彼必事吾如父,何乱之有!”云
长不悦。玄德与单福计议,令赵云引一千军守樊城。玄德领众自回新野。

却说曹仁与李典回许都,见曹操,泣拜于地请罪,具言损将折兵之事。操曰:“胜负乃
军家之常。但不知谁为刘备画策?”曹仁言是单福之计。操曰:“单福何人也?”程昱笑
曰:“此非单福也。此人幼好学击剑;中平末年,尝为人报仇杀人,披发涂面而走,为吏所
获;问其姓名不答,吏乃缚于车上,击鼓行于市,今市人识之,虽有识者不敢言,而同伴窃
解救之。乃更姓名而逃,折节向学,遍访名师,尝与司马徽谈论。此人乃颍川徐庶,字元
直。单福乃其托名耳。”操曰:“徐庶之才,比君何如?”昱曰:“十倍于昱。”操曰:
“惜乎贤士归于刘备!羽翼成矣?奈何?”昱曰:“徐庶虽在彼,丞相要用,召来不难。”
操曰:“安得彼来归?”昱曰:“徐庶为人至孝。幼丧其父,止有老母在堂。现今其弟徐康
已亡,老母无人侍养。丞相可使人赚其母至许昌,令作书召其子,则徐庶必至矣。”

操大喜,使人星夜前去取徐庶母。不一日取至,操厚待之。因谓之曰:“闻令嗣徐元
直,乃天下奇才也。今在新野,助逆臣刘备,背叛朝廷,正犹美玉落于汙泥之中,诚为可
惜。今烦老母作书,唤回许都,吾于天子之前保奏,必有重赏。”遂命左右捧过文房四宝,
令徐母作书。徐母曰:“刘备何如人也?”操曰:“沛郡小辈,妄称皇叔,全无信义,所谓
外君子而内小人者也。徐母厉声曰:“汝何虚诳之甚也!吾久闻玄德乃中山靖王之后,孝景
皇帝阁下玄孙,屈身下士,恭己待人,仁声素著,世之黄童、白叟、牧子、樵夫皆知其名:
真当世之英雄也。吾儿辅之,得其主矣。汝虽托名汉相,实为汉贼。乃反以玄德为逆臣,欲
使吾几背明投暗,岂不自耻乎!“言讫,取石砚便打曹操。操大怒,叱武士执徐母出,将斩
之。程昱急止之,入谏操曰:“徐母触忤丞相者,欲求死也。丞相若杀之,则招不义之名,
而成徐母之德。徐母既死,徐庶必死心助刘备以报仇矣;不如留之,使徐庶身心两处,纵使
助刘备,亦不尽力也。且留得徐母在,昱自有计赚徐庶至此,以辅丞相。”操然其言,遂不
杀徐母,送于别室养之。程昱日往问候,诈言曾与徐庶结为兄弟,待徐母如亲母;时常馈送
物件,必具手启。徐母因亦作手启答之。程昱赚得徐母笔迹,乃仿其字体,诈修家书一封,
差一心腹人,持书径奔新野县,寻问“单福”行幕。军士引见徐庶。庶知母有家书至,急唤
入问之。来人曰:“某乃馆下走卒,奉老夫人言语,有书附达。”庶拆封视之。书曰:“近
汝弟康丧,举目无亲。正悲凄间,不期曹丞相使人赚至许昌,言汝背反,下我于缧絏,赖程
昱等救免。若得汝降,能免我死。如书到日,可念劬劳之恩,星夜前来,以全孝道;然后徐
图归耕故园,免遭大祸。吾今命若悬丝,专望救援!更不多嘱。”徐庶览毕,泪如泉涌。持
书来见玄德曰:“某本颍川徐庶,字元直;为因逃难,更名单福。前闻刘景升招贤纳士,特
往见之;及与论事,方知是无用之人,故作书别之。夤夜至司马水镜庄上,诉说其事。水镜
深责庶不识主,因说刘豫州在此,何不事之?庶故作狂歌于市以动使君;幸蒙不弃,即赐重
用。争奈老母今被曹操奸计赚至许昌囚禁,将欲加害。老母手书来唤,庶不容不去。非不欲
效犬马之劳,以报使君;奈慈亲被执,不得尽力。今当告归,容图后会。”玄德闻言大哭
曰:“子母乃天性之亲,元直无以备为念。待与老夫人相见之后,或者再得奉教。”徐庶便
拜谢欲行。玄德曰:“乞再聚一宵,来日饯行。”孙乾密谓玄德曰:“元直天下奇才,久在
新野,尽知我军中虚实。今若使归曹操,必然重用,我其危矣。主公宜苦留之,切勿放去。
操见元直不去,必斩其母。元直知母死,必为母报仇。力攻曹操也。”玄德曰:“不可。使
人杀其母,而吾用其子,不仁也;留之不使去,以绝其子母之道,不义也。吾宁死,不为不
仁不义之事。”众皆感叹。

玄德请徐庶饮酒,庶曰:“今闻老母被囚,虽金波玉液不能下咽矣。”玄德曰:“备闻
公将去,如失左右手,虽龙肝凤髓,亦不甘味。”二人相对而泣,坐以待旦。诸将已于郭外
安排筵席饯行。玄德与徐庶并马出城,至长亭,下马相辞。玄德举杯谓徐庶曰:“备分浅缘
薄,不能与先生相聚。望先生善事新主,以成功名。”庶泣曰:“某才微智浅,深荷使君重
用。今不幸半途而别,实为老母故也。纵使曹操相逼,庶亦终身不设一谋。”玄德曰:“先
生既去,刘备亦将远遁山林矣。”庶曰:“某所以与使君共图王霸之业者,恃此方寸耳;今
以老母之故,方寸乱矣,纵使在此,无益于事。使君宜别求高贤辅佐,共图大业,何便灰心
如此?”玄德曰:“天下高贤,无有出先生右者。”庶曰:“某樗栎庸材,何敢当此重
誉。”临别,又顾谓诸将曰:“愿诸公善事使君,以图名垂竹帛,功标青史,切勿效庶之无
始终也。”诸将无不伤感。玄德不忍相离,送了一程,又送一程。庶辞曰:“不劳使君远
送,庶就此告别。”玄德就马上执庶之手曰:“先生此去,天各一方,未知相会却在何
日!”说罢,泪如雨下。庶亦涕泣而别。玄德立马于林畔,看徐庶乘马与从者匆匆而去。玄
德哭曰:“元直去矣!吾将奈何?”凝泪而望,却被一树林隔断。玄德以鞭指曰:“吾欲尽
伐此处树木。”众问何故。玄德曰:“因阻吾望徐元直之目也。”

正望间,忽见徐庶拍马而回。玄德曰:“元直复回,莫非无去意乎?”遂欣然拍马向前
迎问曰:“先生此回,必有主意。”庶勒马谓玄德曰:“某因心绪如麻,忘却一语:此间有
一奇士,只在襄阳城外二十里隆中。使君何不求之?”玄德曰:“敢烦元直为备请来相
见。”庶曰:“此人不可屈致,使君可亲往求之。若得此人,无异周得吕望、汉得张良
也。”玄德曰:“此人比先生才德何如?”庶曰:“以某比之,譬犹驽马并麒麟、寒鸦配鸾
凤耳。此人每尝自比管仲,乐毅;以吾观之,管、乐殆不及此人。此人有经天纬地之才,盖
天下一人也!”玄德喜曰:“愿闻此人姓名。”庶曰:“此人乃琅琊阳都人,覆姓诸葛,名
亮,字孔明,乃汉司隶校尉诸葛丰之后。其父名珪,字子贡,为泰山郡丞,早卒;亮从其叔
玄。玄与荆州刘景升有旧,因往依之,遂家于襄阳。后玄卒,亮与弟诸葛均躬耕于南阳。尝
好为《梁父吟》。所居之地有一冈,名卧龙冈,因自号为卧龙先生。此人乃绝代奇才,使君
急宜枉驾见之。若此人肯相辅佐,何愁天下不定乎!”玄德曰:“昔水镜先生曾为备言:
‘伏龙、凤雏,两人得一,可安天下。’今所云莫非即伏龙、凤雏乎?”庶曰:“凤雏乃襄
阳庞统也。伏龙正是诸葛孔明。”玄德踊跃曰:“今日方知伏龙、凤雏之语。何期大贤只在
目前!非先生言,备有眼如盲也!”后人有赞徐庶走马荐诸葛诗曰:“痛恨高贤不再逢,临
岐泣别两情浓。片言却似春雷震,能使南阳起卧龙。”徐庶荐了孔明,再别玄德,策马而
去。玄德闻徐庶之语,方悟司马德操之言,似醉方醒,如梦初觉。引众将回至新野,便具厚
币,同关、张前去南阳请孔明。

且说徐庶既别玄德,感其留恋之情,恐孔明不肯出山辅之,遂乘马直至卧龙冈下,入草
庐见孔明。孔明问其来意。庶曰:“庶本欲事刘豫州,奈老母为曹操所囚,驰书来召,只得
舍之而往。临行时,将公荐与玄德。玄德即日将来奉谒,望公勿推阻,即展平生之大才以辅
之,幸甚!”孔明闻言作色曰:“君以我为享祭之牺牲乎!”说罢,拂袖而入。庶羞惭而
退,上马趱程,赴许昌见母。正是:嘱友一言因爱主,赴家千里为思亲。未知后事若何,下
文便见。