\chapter{玄德智激孙夫人~孔明二气周公瑾}

却说玄德见孙夫人房中两边枪刀森列,侍婢皆佩剑,不觉失色。管家婆进曰:“贵人休得惊惧:夫人自幼好观武事,居常令侍婢击剑为乐,故尔如此。”玄德曰:“非夫人所观之事,吾甚心寒,可命暂去。”管家婆禀覆孙夫人曰:“房中摆列兵器,娇客不安,今且去之。”孙夫人笑曰:“厮杀半生,尚惧兵器乎!”命尽撤去,令侍婢解剑伏侍。当夜玄德与孙夫人成亲,两情欢洽。玄德又将金帛散给侍婢,以买其心,先教孙乾回荆州报喜。自此连日饮酒。国太十分爱敬。

却说孙权差人来柴桑郡报周瑜,说:“我母亲力主,已将吾妹嫁刘备。不想弄假成真。此事还复如何?”瑜闻大惊,行坐不安,乃思一计,修密书付来人持回见孙权。权拆书视之。书略曰:“瑜所谋之事,不想反覆如此。既已弄假成真,又当就此用计。刘备以枭雄之姿,有关、张、赵云之将,更兼诸葛用谋,必非久屈人下者。愚意莫如软困之于吴中:盛为筑宫室,以丧其心志;多送美色玩好,以娱其耳目;使分开关、张之情,隔远诸葛之契,各置一方,然后以兵击之,大事可定矣。今若纵之,恐蛟龙得云雨,终非池中物也。愿明公熟思之。”孙权看毕,以书示张昭。昭曰:“公瑾之谋,正合愚意。刘备起身微末,奔走天下,未尝受享富贵。今若以华堂大厦,子女金帛,令彼享用,自然疏远孔明、关、张等,使彼各生怨望,然后荆州可图也。主公可依公瑾之计而速行之。”权大喜,即日修整东府,广栽花木,盛设器用,请玄德与妹居住;又增女乐数十余人,并金玉锦绮玩好之物。国太只道孙权好意,喜不自胜。玄德果然被声色所迷,全不想回荆州。

却说赵云与五百军在东府前住,终日无事,只去城外射箭走马。看看年终。云猛省:“孔明分付三个锦囊与我,教我一到南徐,开第一个;住到年终,开第二个;临到危急无路之时,开第三个:于内有神出鬼没之计,可保主公回家。此时岁已将终,主公贪恋女色,并不见面,何不拆开第二个锦囊,看计而行?”遂拆开视之。原来如此神策。即日径到府堂,要见玄德。侍婢报曰:“赵子龙有紧急事来报贵人。”玄德唤入问之。云佯作失惊之状曰:“主公深居画堂,不想荆州耶?”玄德曰:“有甚事如此惊怪?”云曰:“今早孔明使人来报,说曹操要报赤壁鏖兵之恨,起精兵五十万,杀奔荆州,甚是危急,请主公便回。”玄德曰:“必须与夫人商议。”云曰:“若和夫人商议,必不肯教主公回。不如休说,今晚便好起程。迟则误事!”玄德曰:“你且暂退,我自有道理。”云故意催逼数番而出。玄德入见孙夫人,暗暗垂泪。孙夫人曰:“丈夫何故烦恼?”玄德曰:“念备一身飘荡异乡,生不能侍奉二亲,又不能祭祀宗祖,乃大逆不孝也。今岁旦在迩,使备悒怏不已。”孙夫人曰:“你休瞒我,我已听知了也!方才赵子龙报说荆州危急,你欲还乡,故推此意。”玄德跪而告曰:“夫人既知,备安敢相瞒。备欲不去,使荆州有失,被天下人耻笑;欲去,又舍不得夫人:因此烦恼。”夫人曰:“妾已事君,任君所之,妾当相随。”玄德曰:“夫人之心,虽则如此,争奈国太与吴侯安肯容夫人去?夫人若可怜刘备,暂时辞别。”言毕,泪如雨下。孙夫人劝曰:“丈夫休得烦恼。妾当苦告母亲,必放妾与君同去。”玄德曰:“纵然国太肯时,吴侯必然阻挡。”孙夫人沉吟良久,乃曰:“妾与君正旦拜贺时,推称江边祭祖,不告而去,若何?”玄德又跪而谢曰:“若如此,生死难忘!切勿漏泄。”两个商议已定。玄德密唤赵云分付:“正旦日,你先引军士出城,于官道等候。吾推祭祖,与夫人同走。”云领诺。

建安十五年春正月元旦,吴侯大会文武于堂上。玄德与孙夫人入拜国太。孙夫人曰:“夫主想父母宗祖坟墓,俱在涿郡,昼夜伤感不已。今日欲往江边,望北遥祭,须告母亲得知。”国太曰:“此孝道也,岂有不从?汝虽不识舅姑,可同汝夫前去祭拜,亦见为妇之礼。”孙夫人同玄德拜谢而出。

此时只瞒着孙权。夫人乘车,止带随身一应细软。玄德上马,引数骑跟随出城,与赵云相会。五百军士前遮后拥,离了南徐,趱程而行。当日,孙权大醉,左右近侍扶入后堂,文武皆散。比及众官探得玄德、夫人逃遁之时,天色已晚。要报孙权,权醉不醒。及至睡觉,已是五更。次日,孙权闻知走了玄德,急唤文武商议。张昭曰:“今日走了此人,早晚必生祸乱。可急追之。”孙权令陈武、潘璋选五百精兵,无分昼夜,务要赶上拿回。二将领命去了。

孙权深恨玄德,将案上玉砚摔为粉碎。程普曰:“主公空有冲天之怒,某料陈武、潘璋必擒此人不得。”权曰:“焉敢违我令!”普曰:“郡主自幼好观武事,严毅刚正,诸将皆惧。既然肯顺刘备,必同心而去。所追之将,若见郡主,岂肯下手?”权大怒,掣所佩之剑,唤蒋钦、周泰听令,曰:“汝二人将这口剑去取吾妹并刘备头来!违令者立斩!”蒋钦、周泰领命,随后引一千军赶来。

却说玄德加鞭纵辔,趱程而行;当夜于路暂歇两个更次,慌忙起行。看看来到柴桑界首,望见后面尘头大起,人报:“追兵至矣!”玄德慌问赵云曰:“追兵既至,如之奈何?”赵云曰:“主公先行,某愿当后。”转过前面山脚,一彪军马拦住去路。当先两员大将,厉声高叫曰:“刘备早早下马受缚!吾奉周都督将令,守候多时!”原来周瑜恐玄德走脱,先使徐盛、丁奉引三千军马于冲要之处扎营等候,时常令人登高遥望,料得玄德若投旱路,必经此道而过。当日徐盛、丁奉了望得玄德一行人到,各绰兵器截住去路。玄德惊慌勒回马问赵云曰:“前有拦截之兵,后有追赶之兵:前后无路,如之奈何?”云曰:“主公休慌。军师有三条妙计,多在锦囊之中。已拆了两个,并皆应验。今尚有第三个在此,分付遇危难之时,方可拆看。今日危急,当拆观之。”便将锦囊拆开,献与玄德。玄德看了,急来车前泣告孙夫人曰:“备有心腹之言,至此尽当实诉。”夫人曰:“丈夫有何言语,实对我说。”玄德曰:“昔日吴侯与周瑜同谋,将夫人招嫁刘备,实非为夫人计,乃欲幽困刘备而夺荆州耳。夺了荆州,必将杀备。是以夫人为香饵而钓备也。备不惧万死而来,盖知夫人有男子之胸襟,必能怜备。昨闻吴侯将欲加害,故托荆州有难,以图归计。幸得夫人不弃,同至于此。今吴侯又令人在后追赶,周瑜又使人于前截住,非夫人莫解此祸。如夫人不允,备请死于车前,以报夫人之德。”夫人怒曰:“吾兄既不以我为亲骨肉,我有何面目重相见乎!今日之危,我当自解。”于是叱从人推车直出,卷起车帘,亲喝徐盛、丁奉曰:“你二人欲造反耶?”徐、丁二将慌忙下马,弃了兵器,声喏于车前曰:“安敢造反。为奉周都督将令,屯兵在此专候刘备。”孙夫人大怒曰:“周瑜逆贼!我东吴不曾亏负你!玄德乃大汉皇叔,是我丈夫。我已对母亲、哥哥说知回荆州去。今你两个于山脚去处,引着军马拦截道路,意欲劫掠我夫妻财物耶?”徐盛、丁奉喏喏连声,口称:“不敢。请夫人息怒。这不干我等之事,乃是周都督的将令。”孙夫人叱曰:“你只怕周瑜,独不怕我?周瑜杀得你,我岂杀不得周瑜?”把周瑜大骂一场,喝令推车前进。徐盛、丁奉自思:“我等是下人。安敢与夫人违拗?”又见赵云十分怒气,只得把军喝住,放条大路教过去。

恰才行不得五六里,背后陈武、潘璋赶到。徐盛、丁奉备言其事。陈、潘二将曰:“你放他过去差了也。我二人奉吴侯旨意,特来追捉他回去。”于是四将合兵一处,趱程赶来。玄德正行间,忽听得背后喊声大起。玄德又告孙夫人曰:“后面追兵又到,如之奈何?”夫人曰:“丈夫先行,我与子龙当后。”玄德先引三百军,望江岸去了。子龙勒马于车傍,将士卒摆开,专候来将。四员将见了孙夫人,只得下马,叉手而立。夫人曰:“陈武、潘璋,来此何干?”二将答曰:“奉主公之命,请夫人、玄德回。”夫人正色叱曰:“都是你这伙匹夫,离间我兄妹不睦!我已嫁他人,今日归去,须不是与人私奔。我奉母亲慈旨,令我夫妇回荆州。便是我哥哥来,也须依礼而行。你二人倚仗兵威,欲待杀害我耶?”骂得四人面面相觑,各自寻思:“他一万年也只是兄妹。更兼国太作主;吴侯乃大孝之人,怎敢违逆母言?明日翻过脸来,只是我等不是。不如做个人情。”军中又不见玄德;但见赵云怒目睁眉,只待厮杀。因此四将喏喏连声而退。孙夫人令推车便行。徐盛曰:“我四人同去见周都督,告禀此事。”

四人犹豫未定。忽见一军如旋风而来,视之,乃蒋钦、周泰。二将问曰:“你等曾见刘备否?”四人曰:“早晨过去,已半日矣。”蒋钦曰:“何不拿下?”四人各言孙夫人发话之事。蒋钦曰:“便是吴侯怕道如此,封一口剑在此,教先杀他妹,后斩刘备。违者立斩!”四将曰:“去之已远,怎生奈何?”蒋钦曰:“他终是些步军,急行不上。徐、丁二将军可飞报都督,教水路棹快船追赶;我四人在岸上追赶:无问水旱之路,赶上杀了,休听他言语。”于是徐盛、丁奉飞报周瑜;蒋钦、周泰、陈武、潘璋四个领兵沿江赶来。

却说玄德一行人马,离柴桑较远,来到刘郎浦,心才稍宽。沿着江岸寻渡,一望江水弥漫,并无船只。玄德俯首沉吟。赵云曰:“主公在虎口中逃出,今已近本界,吾料军师必有调度,何用犹疑?”玄德听罢,蓦然想起在吴繁华之事,不觉凄然泪下。后人有诗叹曰:“吴蜀成婚此水浔,明珠步障屋黄金。谁知一女轻天下,欲易刘郎鼎峙心。”

玄德令赵云望前哨探船只,忽报后面尘土冲天而起。玄德登高望之,但见军马盖地而来,叹曰:“连日奔走,人困马乏,追兵又到,死无地矣!”看看喊声渐近。正慌急间,忽见江岸边一字儿抛着拖篷船二十余只。赵云曰:“天幸有船在此!何不速下,棹过对岸,再作区处!”玄德与孙夫人便奔上船。子龙引五百军亦都上船。只见船舱中一人纶巾道服,大笑而出,曰:“主公且喜!诸葛亮在此等候多时。”船中扮作客人的,皆是荆州水军。玄德大喜。不移时,四将赶到。孔明笑指岸上人言曰:“吾已算定多时矣。汝等回去传示周郎,教休再使美人局手段。”岸上乱箭射来,船已开的远了。蒋钦等四将,只好呆看。玄德与孔明正行间,忽然江声大震。回头视之,只见战船无数。帅字旗下,周瑜自领惯战水军,左有黄盖,右有韩当,势如飞马,疾似流星。看看赶上。孔明教棹船投北岸,弃了船,尽皆上岸而走,车马登程。周瑜赶到江边,亦皆上岸追袭。大小水军,尽是步行;止有为首官军骑马。周瑜当先,黄盖、韩当、徐盛、丁奉紧随。周瑜曰:“此处是那里?军士答曰:“前面是黄州界首。”望见玄德车马不远,瑜令并力追袭。正赶之间,一声鼓响,山崦内一彪刀手拥出,为首一员大将,乃关云长也。周瑜举止失措,急拨马便走;云长赶来,周瑜纵马逃命。正奔走间,左边黄忠,右边魏延,两军杀出。吴兵大败。周瑜急急下得船时,岸上军士齐声大叫曰:“周郎妙计安天下,陪了夫人又折兵!”瑜怒曰:“可再登岸决一死战!”黄盖、韩当力阻。瑜自思曰:“吾计不成,有何面目去见吴侯!”大叫一声,金疮迸裂,倒于船上。众将急救,却早不省人事。正是:两番弄巧翻成拙,此日含嗔却带羞。未知周郎性命如何,且看下文分解。