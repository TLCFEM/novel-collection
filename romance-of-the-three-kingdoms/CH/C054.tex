\chapter{吴国太佛寺看新郎~刘皇叔洞房续佳偶}

却说孔明闻鲁肃到,与玄德出城迎接,接到公廨,相见毕。肃曰:“主公闻令侄弃世,特具薄礼,遣某前来致祭。周都督再三致意刘皇叔、诸葛先生。”玄德、孔明起身称谢,收了礼物,置酒相待。肃曰:“前者皇叔有言:公子不在,即还荆州。今公子已去世,必然见还。不识几时可以交割?”玄德曰:“公且饮酒,有一个商议。”肃强饮数杯,又开言相问。玄德未及回答,孔明变色曰:“子敬好不通理,直须待人开口!自我高皇帝斩蛇起义,开基立业,传至于今;不幸奸雄并起,各据一方;少不得天道好还,复归正统。我主人乃中山靖王之后,孝景皇帝玄孙,今皇上之叔,岂不可分茅裂土?况刘景升乃我主之兄也,弟承兄业,有何不顺?汝主乃钱塘小吏之子,素无功德于朝廷;今倚势力,占据六郡八十一州,尚自贪心不足,而欲并吞汉土。刘氏天下,我主姓刘倒无分,汝主姓孙反要强争?且赤壁之战,我主多负勤劳,众将并皆用命,岂独是汝东吴之为?若非我借东南风,周郎安能展半筹之功?江南一破,休说二乔置于铜雀宫,虽公等家小,亦不能保。适来我主人不即答应者,以子敬乃高明之士,不待细说。何公不察之甚也!”一席话,说得鲁子敬缄口无言;半晌乃曰:“孔明之言,怕不有理;争奈鲁肃身上甚是不便。”孔明曰:“有何不便处?”肃曰:“昔日皇叔当阳受难时,是肃引孔明渡江,见我主公;后来周公瑾要兴兵取荆州,又是肃挡住;至说待公子去世还荆州,又是肃担承:今却不应前言,教鲁肃如何回覆?我主与周公瑾必然见罪。肃死不恨,只恐惹恼东吴,兴动干戈,皇叔亦不能安坐荆州,空为天下耻笑耳。”孔明曰:“曹操统百万之众,动以天子为名,吾亦不以为意,岂惧周郎一小儿乎!若恐先生面上不好看,我劝主人立纸文书,暂借荆州为本;待我主别图得城池之时,便交付还东吴。此论如何?”肃曰:“孔明待夺得何处,还我荆州?”孔明曰:“中原急未可图;西川刘璋闇弱,我主将图之。若图得西川,那时便还。”肃无奈,只得听从。玄德亲笔写成文书一纸,押了字。保人诸葛孔明也押了字。孔明曰:“亮是皇叔这里人,难道自家作保?烦子敬先生也押个字,回见吴侯也好看。”肃曰:“某知皇叔乃仁义之人,必不相负。”遂押了字,收了文书。宴罢辞回。玄德与孔明,送到船边。孔明嘱曰:“子敬回见吴侯,善言伸意,休生妄想。若不准我文书,我翻了面皮,连八十一州都夺了。今只要两家和气,休教曹贼笑话。”

肃作别下船而回,先到柴桑郡见周瑜。瑜问曰:“子敬讨荆州如何?”肃曰:“有文书在此。”呈与周瑜,瑜顿足曰:“子敬中诸葛之谋也!名为借地,实是混赖。他说取了西川便还,知他几时取西川?假如十年不得西川,十年不还?这等文书,如何中用,你却与他做保!他若不还时,必须连累足下,主公见罪奈何?”肃闻言,呆了半晌,曰:“恐玄德不负我。”瑜曰:“子敬乃诚实人也。刘备枭雄之辈,诸葛亮奸猾之徒,恐不似先生心地。”肃曰:“若此,如之奈何?”瑜曰:“子敬是我恩人,想昔日指囷相赠之情,如何不救你?你且宽心住数日,待江北探细的回,别有区处。”鲁肃跼蹐不安。

过了数日,细作回报:“荆州城中扬起布幡做好事,城外别建新坟,军士各挂孝。”瑜惊问曰:“没了甚人?”细作曰:“刘玄德没了甘夫人,即日安排殡葬。瑜谓鲁肃曰:“吾计成矣:使刘备束手就缚,荆州反掌可得!”肃曰:“计将安出?”瑜曰:“刘备丧妻,必将续娶。主公有一妹,极其刚勇,侍婢数百,居常带刀,房中军器摆列遍满,虽男子不及。我今上书主公,教人去荆州为媒,说刘备来入赘。赚到南徐,妻子不能勾得,幽囚在狱中,却使人去讨荆州换刘备。等他交割了荆州城池,我别有主意。于子敬身上,须无事也。”鲁肃拜谢。

周瑜写了书呈,选快船送鲁肃投南徐见孙权,先说借荆州一事,呈上文书。权曰:“你却如此糊涂!这样文书,要他何用!”肃曰:“周都督有书呈在此,说用此计,可得荆州。”权看毕,点头暗喜,寻思谁人可去。猛然省曰:“非吕范不可。”遂召吕范至,谓曰:“近闻刘玄德丧妇。吾有一妹,欲招赘玄德为婿,永结姻亲,同心破曹,以扶汉室。非子衡不可为媒,望即往荆州一言。”范领命,即日收拾船只,带数个从人,望荆州来。却说玄德自没了甘夫人,昼夜烦恼。一日,正与孔明闲叙,人报东吴差吕范到来。孔明笑曰:“此乃周瑜之计,必为荆州之故。亮只在屏风后潜听。但有甚说话,主公都应承了。留来人在馆驿中歇,别作商议。”

玄德教请吕范入。礼毕坐定,茶罢,玄德问曰:“子衡来,必有所谕?”范曰:“范近闻皇叔失偶,有一门好亲,故不避嫌,特来作媒。未知尊意若何?”玄德曰:“中年丧妻,大不幸也。骨肉未寒,安忍便议亲?”范曰:“人若无妻,如屋无梁,岂可中道而废人伦?吾主吴侯有一妹,美而贤,堪奉箕帚。若两家共结秦、晋之好,则曹贼不敢正视东南也。此事家国两便,请皇叔勿疑。但我国太吴夫人甚爱幼女,不肯远嫁,必求皇叔到东吴就婚。”玄德曰:“此事吴侯知否?”范曰:“不先禀吴侯,如何敢造次来说!”玄德曰:“吾年已半百,鬓发斑白;吴侯之妹,正当妙龄:恐非配偶。”范曰:“吴侯之妹,身虽女子,志胜男儿。常言:若非天下英雄,吾不事之。今皇叔名闻四海,正所谓淑女配君子,岂以年齿上下相嫌乎!”玄德曰:“公且少留,来日回报。”是日设宴相待,留于馆舍。

至晚,与孔明商议。孔明曰:“来意亮已知道了。适间卜易,得一大吉大利之兆。主公便可应允。先教孙乾和吕范回见吴侯,面许已定,择日便去就亲。”玄德曰:“周瑜定计欲害刘备,岂可以身轻入危险之地?”孔明大笑曰:“周瑜虽能用计,岂能出诸葛亮之料乎!略用小谋,使周瑜半筹不展;吴侯之妹,又属主公;荆州万无一失。”玄德怀疑未决。

孔明竟教孙乾往江南说合亲事。孙乾领了言语,与吕范同到江南,来见孙权。权曰:“吾愿将小妹招赘玄德,并无异心。”孙乾拜谢,回荆州见玄德,言:“吴侯专候主公去结亲。”玄德怀疑不敢往。孔明曰:“吾已定下三条计策,非子龙不可行也。”遂唤赵云近前,附耳言曰:“汝保主公入吴,当领此三个锦囊。囊中有三条妙计,依次而行。”即将三个锦囊,与云贴肉收藏,孔明先使人往东吴纳了聘,一切完备。

时建安十四年冬十月。玄德与赵长、孙乾取快船十只,随行五百余人,离了荆州,前往南徐进发。荆州之事,皆听孔明裁处。玄德心中怏怏不安。到南徐州,船已傍岸,云曰:“军师分付三条妙计,依次而行。今已到此,当先开第一个锦囊来看。”于是开囊看了计策。便唤五百随行军士,一一分付如此如此,众军领命而去,又教玄德先往见乔国老,那乔国老乃二乔之父,居于南徐。玄德牵羊担酒,先往拜见,说吕范为媒、娶夫人之事。随行五百军士,俱披红挂彩,入南徐买办物件,传说玄德入赘东吴,城中人尽知其事。孙权知玄德已到,教吕范相待,且就馆舍安歇。

却说乔国老既见玄德,便入见吴国太贺喜。国太曰:“有何喜事?”乔国老曰:“令爱已许刘玄德为夫人,今玄德已到,何故相瞒?”国太惊曰:“老身不知此事!”便使人请吴侯问虚实,一面先使人于城中探听。人皆回报:“果有此事。女婿已在馆驿安歇,五百随行军士都在城中买猪羊果品,准备成亲。做媒的女家是吕范,男家是孙乾,俱在馆驿中相待。”国太吃了一惊。少顷,孙权入后堂见母亲。国太捶胸大哭。权曰:“母亲何故烦恼?”国太曰:“你直如此将我看承得如无物!我姐姐临危之时,分付你甚么话来!”孙权失惊曰:“母亲有话明说,何苦如此?”国太曰:“男大须婚,女大须嫁,古今常理。我为你母亲,事当禀命于我。你招刘玄德为婿,如何瞒我?女儿须是我的!”权吃了一惊,问曰:“那里得这话来?”国太曰:“若要不知,除非莫为。满城百姓,那一个不知?你倒瞒我!”乔国老曰:“老夫已知多日了,今特来贺喜。”权曰:“非也。此是周瑜之计,因要取荆州,故将此为名,赚刘备来拘囚在此,要他把荆州来换;若其不从,先斩刘备。此是计策,非实意也。”国太大怒,骂周瑜曰:“汝做六郡八十一州大都督,直恁无条计策去取荆州,却将我女儿为名,使美人计!杀了刘备,我女便是望门寡,明日再怎的说亲?须误了我女儿一世!你们好做作!”乔国老曰:“若用此计,便得荆州,也被天下人耻笑。此事如何行得!”说得孙权默然无语。

国太不住口的骂周瑜。乔国老劝曰:“事已如此,刘皇叔乃汉室宗亲,不如真个招他为婿,免得出丑。”权曰:“年纪恐不相当。”国老曰:“刘皇叔乃当世豪杰,若招得这个女婿,也不辱了令妹。”国太曰:“我不曾认得刘皇叔。明日约在甘露寺相见:如不中我意,任从你们行事;若中我的意,我自把女儿嫁他!”孙权乃大孝之人,见母亲如此言语,随即应承,出外唤吕范,分付来日甘露寺方丈设宴,国太要见刘备。吕范曰:“何不令贾华部领三百刀斧手,伏于两廊;若国太不喜时,一声号举,两边齐出,将他拿下。”权遂唤贾华,分付预先准备,只看国太举动。却说乔国老辞吴国太归,使人去报玄德,言:“来日吴侯、国太亲自要见,好生在意!”玄德与孙乾、赵云商议。云曰:“来日此会,多凶少吉,云自引五百军保护。”次日,吴国太、乔国老先在甘露寺方丈里坐定。孙权引一班谋士,随后都到,却教吕范来馆驿中请玄德。玄德内披细铠,外穿棉袍,从人背剑紧随,上马投甘露寺来。赵云全装惯带,引五百军随行。来到寺前下马,先见孙权。权观玄德仪表非凡,心中有畏惧之意。二人叙礼毕,遂入方丈见国太。国太见了玄德,大喜,谓乔国老曰:“真吾婿也!”国老曰:“玄德有龙凤之姿,天日之表;更兼仁德布于天下:国太得此佳婿,真可庆也!”玄德拜谢,共宴于方丈之中。少刻,子龙带剑而入,立于玄德之侧。国太问曰:“此是何人?”玄德答曰:“常山赵子龙也。”国太曰:“莫非当阳长坂抱阿斗者乎?”玄德曰:“然。”国太曰:“真将军也!”遂赐以酒。赵云谓玄德曰:“却才某于廊下巡视,见房内有刀斧手埋伏,必无好意。可告知国太。”玄德乃跪于国太席前,泣而告曰:“若杀刘备,就此请诛。”国太曰:“何出此言?”玄德曰:“廊下暗伏刀斧手,非杀备而何?”国太大怒,责骂孙权:“今日玄德既为我婿,即我之儿女也。何故伏刀斧手于廊下!”权推不知,唤吕范问之;范推贾华;国太唤贾华责骂,华默然无言。国太喝令斩之。玄德告曰:“若斩大将,于亲不利,备难久居膝下矣。”乔国老也相劝。国太方叱退贾华。刀斧手皆抱头鼠窜而去。

玄德更衣出殿前,见庭下有一石块。玄德拔从者所佩之剑,仰天祝曰:“若刘备能勾回荆州,成王霸之业,一剑挥石为两段。如死于此地,剑剁石不开。”言讫,手起剑落,火光迸溅,砍石为两段。孙权在后面看见,问曰:“玄德公如何恨此石?”玄德曰:“备年近五旬,不能为国家剿除贼党,心常自恨。今蒙国太招为女婿,此平生之际遇也。恰才问天买卦,如破曹兴汉,砍断此石。今果然如此。”权暗思:“刘备莫非用此言瞒我?”亦掣剑谓玄德曰:“吾亦问天买卦。若破得曹贼,亦断此石。”却暗暗祝告曰:“若再取得荆州,兴旺东吴,砍石为两半!”手起剑落,巨石亦开。至今有十字纹“恨石”尚存。后人观此胜迹,作诗赞曰:“宝剑落时山石断,金环响处火光生,两朝旺气皆天数。从此乾坤鼎足成。”

二人弃剑,相携入席。又饮数巡,孙乾目视玄德,玄德辞曰:“备不胜酒力,告退。”孙权送出寺前,二人并立,观江山之景。玄德曰:“此乃天下第一江山也!”至今甘露寺牌上云:“天下第一江山”。后人有诗赞曰:“江山雨霁拥青螺,境界无忧乐最多。昔日英雄凝目处,岩崖依旧抵风波。”

二人共览之次,江风浩荡,洪波滚雪,白浪掀天。忽见波上一叶小舟,行于江面上,如行平地。玄德叹曰:“南人驾船,北人乘马,信有之也。”孙权闻言自思曰:“刘备此言,戏我不惯乘马耳。”乃令左右牵过马来,飞身上马,驰骤下山,复加鞭上岭,笑谓玄德曰:“南人不能乘马乎?”玄德闻言,撩衣一跃,跃上马背,飞走下山,复驰骋而上。二人立马于山坡之上,扬鞭大笑。至今此处名为“驻马坡”。后人有诗曰:“驰骤龙驹气概多,二人并辔望山河。东吴西蜀成王霸,千古犹存驻马坡。”当日二人并辔而回。南徐之民,无不称贺。

玄德自回馆驿,与孙乾商议。乾曰:“主公只是哀求乔国老,早早毕姻,免生别事。”次日,玄德复至乔国老宅前下马。国老接入,礼毕,茶罢,玄德告曰:“江左之人,多有要害刘备者,恐不能久居。”国老曰:“玄德宽心。吾为公告国太,令作护持。”玄德拜谢自回。乔国老入见国太,言玄德恐人谋害,急急要回。国太大怒曰:“我的女婿,谁敢害他!”即时便教搬入书院暂住,择日毕姻。玄德自入告国太曰:“只恐赵云在外不便,军士无人约束。”国太教尽搬入府中安歇,休留在馆驿中,免得生事。玄德暗喜。

数日之内,大排筵会,孙夫人与玄德结亲。至晚客散,两行红炬,接引玄德入房。灯光之下,但见枪刀簇满;侍婢皆佩剑悬刀,立于两傍。?得玄德魂不附体。正是:惊看侍女横刀立,疑是东吴设伏兵。毕竟是何缘故,且看下文分解。