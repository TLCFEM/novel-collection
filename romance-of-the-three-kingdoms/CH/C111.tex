\chapter{邓士载智败姜伯约~诸葛诞义讨司马昭}

却说姜维退兵屯于钟提,魏兵屯于狄道城外。王经迎接陈泰、邓艾入城,拜谢解围之事,设宴相待,大赏三军。泰将邓艾之功,申奏魏主曹髦,髦封艾为安西将军,假节,领护东羌校尉,同陈泰屯兵于雍、凉等处。邓艾上表谢恩毕,陈泰设席与邓艾作贺曰:“姜维夜遁,其力已竭,不敢再出矣。”艾笑曰:“吾料蜀兵必出有五。”泰问其故,艾曰:“蜀兵虽退,终有乘胜之势;吾兵终有弱败之实:其必出一也。蜀兵皆是孔明教演,精锐之兵,容易调遣;吾将不时更换,军又训练不熟:其必出二也。蜀人多以船行,吾军皆在旱地,劳逸不同;其必出三也。狄道、陇西、南安、祁山四处皆是守战之地;蜀人或声东击西,指南攻北,吾兵必须分头守把;蜀兵合为一处而来,以一分当我四分:其必出四也。若蜀兵自南安、陇西,则可取羌人之谷为食;若出祁山,则有麦可就食:其必出五也。”陈泰叹服曰;“公料敌如神,蜀兵何足虑哉!”于是陈泰与邓艾结为忘年之交。艾遂将雍、凉等处之兵,每日操练;各处隘口,皆立营寨,以防不测。

却说姜维在钟提大设筵宴,会集诸将,商议伐魏之事。令史樊建谏曰:“将军屡出,未获全功;今日洮西之捷,魏人已服威名,何故又欲出也?万一不利,前功尽弃。”维曰:“汝等只知魏国地宽人广,急不可得;却不知攻魏者有五可胜。”众问之,维答曰:“彼洮西一败,挫尽锐气,吾兵虽退,不曾损折:今若进兵,一可胜也。吾兵船载而进,不致劳困,彼兵皆从旱地来迎:二可胜也。吾兵久经训练之众,彼皆乌合之徒,不曾有法度:三可胜也。吾兵自出祁山,掠抄秋谷为食:四可胜也。彼兵须各守备,军力分开,吾兵一处而去,彼安能救:五可胜也。不在此时伐魏,更待何日耶?”夏侯霸曰:“艾年虽幼,而机谋深远;近封为安西将军之职,必于各处准备,非同往日矣。”维厉声曰:“吾何畏彼哉!公等休长他人锐气,灭自己威风!吾意已决,必先取陇西。”众不敢谏。维自领前部,令众将随后而进,于是蜀兵尽离钟提,杀奔祁山来。哨马报说魏兵已先在祁山立下九个寨栅。维不信,引数骑凭高望之,果见祁山九寨势如长蛇,首尾相顾。维回顾左右曰:“夏侯霸之言,信不诬矣。此寨形势绝妙。止吾师诸葛丞相能之;今观邓艾所为,不在吾师之下。”遂回本寨。唤诸将曰:“魏人既有准备,必知吾来矣。吾料邓艾必在此间。汝等可虚张吾旗号,据此谷口下寨;每日令百余骑出哨,每出哨一回,换一番衣甲、旗号、按青、黄、赤、白、黑五方旗帜相换。吾却提大兵偷出董亭,径袭南安去也。”遂令鲍素屯兵于祁山谷口。维尽率大兵,望南安进发。

却说邓艾知蜀兵出祁山,早与陈泰下寨准备;见蜀兵连日不来搦战,一日五番哨马出寨,或十里或十五里而回。艾凭高望毕。慌入帐与陈泰曰:“姜维不在此间,必取董亭袭南安去了。出寨哨马只是这几匹。更换衣甲,往来哨探,其马皆困乏,主将必无能者。陈将军可引一军攻之,其寨可破也。破了寨栅,便引兵袭董亭之路,先断姜维之后。吾当先引一军救南安,径取武城山。若先占此山头,姜维必取上邽。上邽有一谷,名曰段谷,地狭山险,正好埋伏。彼来争武城山时,吾先伏两军于段谷,破维必矣。”泰曰:“吾守陇西二三十年,未尝如此明察地理。公之所言,真神算也!公可速去,吾自攻此处寨栅。”于是邓艾引军星夜倍道而行,径到武城山;下寨已毕,蜀兵未到。即令子邓忠,与帐前校尉师篡,各引五千兵,先去段谷埋伏,如此如此而行。二人受计而去。艾令偃旗息鼓,以待蜀兵。却说姜维从董亭望南安而来,至武城山前,谓夏侯霸曰:“近南安有一山,名武城山;若先得了,可夺南安之势。只恐邓艾多谋,必先提防。”正疑虑间,忽然山上一声炮响,喊声大震,鼓角齐鸣,旌旗遍竖,皆是魏兵;中央风飘起一黄旗,大书邓艾字样。蜀兵大惊。山上数处精兵杀下,势不可当,前军大败。维急率中军人马去救时,魏兵已退。维直来武城山下搦邓艾战,山上魏兵并不下来。维令军士辱骂。至晚,方欲退军,山上鼓角齐鸣,却又不见魏兵下来。维欲上山冲杀,山上炮石甚严,不能得进。守至三更,欲回,山上鼓角又鸣,维移兵下山屯扎。比及令军搬运木石,方欲竖立为寨,山上鼓角又鸣,魏兵骤至。蜀兵大乱,自相践踏,退回旧寨。次日,姜维令军士运粮草车仗,至武城山,穿连排定,欲立起寨栅,以为屯兵之计。是夜二更,邓艾令五百人,各执火把,分两路下山,放火烧车仗。两兵混杀了一夜,营寨又立不成。

维复引兵退,再与夏侯霸商议曰:“南安未得,不如先取上邽。上邽乃南安屯粮之所;若得上邽,南安自危矣。”遂留霸屯于武城山,维尽引精兵猛将,径取上邽。行了一宿,将及天明,见山势狭峻,道路崎岖,乃问向导官曰:“此处何名?”答曰:“段谷。”维大惊曰:“其名不美:段谷者,断谷也。倘有人断其谷口,如之奈何?”正踌躇未决,忽前军来报:“山后尘头大起,必有伏兵。”维急令退兵。师篡、邓忠两军杀出,维且战且走,前面喊声大震,邓艾引兵杀到:三路夹攻,蜀兵大败。幸得夏侯霸引兵杀到,魏兵方退,救了姜维,欲再往祁山。霸曰:“祁山寨已被陈泰打破,鲍素阵亡,全寨人马皆退回汉中去了。”维不敢取董亭,急投山僻小路而回。后面邓艾急追,维令诸军前进,自为断后。正行之际,忽然山中一军突出,乃魏将陈泰也。魏兵一声喊起,将姜维困在垓心。维人马困乏,左冲右突,不能得出。荡寇将军张嶷,闻姜维受困,引数百骑杀入重围。维因乘势杀出。嶷被魏兵乱箭射死。维得脱重围,复回汉中,因感张嶷忠勇,殁于王事,乃表赠其子孙。于是,蜀中将士多有阵亡者,皆归罪于姜维。维照武侯街亭旧例,乃上表自贬为后将军,行大将军事。

却说邓艾见蜀兵退尽,乃与陈泰设宴相贺,大赏三军。泰表邓艾之功,司马昭遣使持节,加艾官爵,赐印绶;并封其子邓忠为亭侯。时魏主曹髦,改正元三年为甘露元年。司马昭自为天下兵马大都督,出入常令三千铁甲骁将前后簇拥,以为护卫;一应事务,不奏朝廷,就于相府裁处:自此常怀篡逆之心。有一心腹人,姓贾,名充,字公闾,乃故建威将军贾逵之子,为昭府下长史。充语昭曰:“今主公掌握大柄,四方人心必然未安;且当暗访,然后徐图大事。”昭曰:“吾正欲如此。汝可为我东行。只推慰劳出征军士为名,以探消息。”贾充领命,径到淮南,入见镇东大将军诸葛诞。诞字公休,乃琅琊南阳人,即武侯之族弟也;向事于魏,因武侯在蜀为相,因此不得重用;后武侯身亡,诞在魏历任重职,封高平侯。总摄两淮军马。当日,贾充托名劳军,至淮南见诸葛诞。诞设宴待之。酒至半酣,充以言挑诞曰:“近来洛阳诸贤,皆以主上懦弱,不堪为君。司马大将军三辈辅国,功德弥天,可以禅代魏统。未审钧意若何?”诞大怒曰:“汝乃贾豫州之子,世食魏禄,安敢出此乱言!”充谢曰:“某以他人之言告公耳。”诞曰:“朝廷有难,吾当以死报之。”充默然,次日辞归,见司马昭细言其事。昭大怒曰:“鼠辈安敢如此!”充曰:“诞在淮南,深得人心,久必为患,可速除之。”

昭遂暗发密书与扬州刺史乐綝。一面遣使赍诏征诞为司空。诞得了诏书,已知是贾充告变,遂捉来使拷问。使者曰:“此事乐綝知之。”诞曰:“他如何得知?”使者曰:“司马将军已令人到扬州送密书与乐綝矣。”诞大怒,叱左右斩了来使,遂起部下兵千人,杀奔扬州来。将至南门,城门已闭,吊桥拽起。诞在城下叫门,城上并无一人回答。诞大怒曰:“乐綝匹夫,安敢如此!”遂令将士打城。手下十余骁骑,下马渡壕,飞身上城,杀散军士,大开城门,于是诸葛诞引兵入城,乘风放火,杀至綝家。綝慌上楼避之。诞提剑上楼,大喝曰:“汝父乐进,昔日受魏国大恩!不思报本,反欲顺司马昭耶!”綝未及回言,为诞所杀。一面具表数司马昭之罪,使人申奏洛阳;一面大聚两淮屯田户口十余万,并扬州新降兵四万余人,积草屯粮,准备进兵;又令长史吴纲,送子诸葛靓入吴为质求援,务要合兵诛讨司马昭。

此时东吴丞相孙峻病亡,从弟孙綝辅政。綝字子通,为人强暴,杀大司马滕胤、将军吕据、王惇等,因此权柄皆归于綝。吴主孙亮,虽然聪明,无可奈何。于是吴纲将诸葛靓至石头城,入拜孙綝。綝问其故,纲曰:“诸葛诞乃蜀汉诸葛武侯之族弟也,向事魏国;今见司马昭欺君罔上,废主弄权,欲兴师讨之,而力不及,故特来归降。诚恐无凭,专送亲子诸葛靓为质。伏望发兵相助。”綝从其请,便遣大将全怿、全端为主将,于诠为合后,朱异、唐咨为先锋,文钦为向导,起兵七万,分三队而进。吴纲回寿春报知诸葛诞。诞大喜,遂陈兵准备。却说诸葛诞表文到洛阳,司马昭见了大怒,欲自往讨之。贾充谏曰:“主公乘父兄之基业,恩德未及四海,今弃天子而去,若一朝有变,悔之何及?不如奏请太后及天子一同出征,可保无虞。”昭喜曰:“此言正合吾意。”遂入奏太后曰:“诸葛诞谋反,臣与文武官僚,计议停当:请太后同天子御驾亲征,以继先帝之遗意。”太后畏惧,只得从之。次日,昭请魏主曹髦起程。髦曰:“大将军都督天下军马,任从调遣,何必朕自行也?”昭曰:“不然。昔日武祖纵横四海,文帝、明帝有包括宇宙之志,并吞八荒之心,凡遇大敌,必须自行。陛下正宜追配先君,扫清故孽。何自畏也?”髦畏威权,只得从之。昭遂下诏,尽起两都之兵二十六万,命镇南将军王基为正先锋,安东将军陈骞为副先锋,监军石苞为左军,兖州刺史州泰为右军,保护车驾,浩浩荡荡,杀奔淮南而来。

东吴先锋朱异,引兵迎敌。两军对圆,魏军中王基出马,朱异来迎。战不三合,朱异败走:唐咨出马,战不三合,亦大败而走。王基驱兵掩杀,吴兵大败,退五十里下寨,报入寿春城中。诸葛诞自引本部锐兵,会合文钦并二子文鸯、文虎,雄兵数万,来敌司马昭。正是:方见吴兵锐气堕。又看魏将劲兵来。未知胜负如何,且看下文分解。