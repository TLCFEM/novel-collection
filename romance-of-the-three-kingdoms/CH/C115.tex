\chapter{诏班师后主信谗~托屯田姜维避祸}

却说蜀汉景耀五年,冬十月,大将军姜维,差人连夜修了栈道,整顿军粮兵器,又于汉
中水路调拨船只。俱已完备,上表奏后主曰:“臣累出战,虽未成大功,已挫动魏人心胆。
今养兵日久,不战则懒,懒则致病。况今军思效死,将思用命。臣如不胜,当受死罪。”后
主览表,犹豫未决。谯周出班奏曰:“臣夜观天文,见西蜀分野,将星暗而不明。今大将军
又欲出师,此行甚是不利。陛下可降诏止之。”后主曰:“且看此行若何。果然有失,却当
阻之。”谯周再三苦谏不从,乃归家叹息不已,遂推病不出。

却说姜维临兴兵,乃问廖化曰:“吾今出师,誓欲恢复中原,当先取何处?”化曰:
“连年征伐,军民不宁;兼魏有邓艾,足智多谋,非等闲之辈:将军强欲行难为之事,此化
所以未敢专也。”维勃然大怒曰:“昔丞相六出祁山,亦为国也。吾今八次伐魏,岂为一己
之私哉?今当先取洮阳。如有逆吾者必斩!”遂留廖化守汉中,自同诸将提兵三十万,径取
洮阳而来。早有川口人报入祁山寨中。时邓艾正与司马望谈兵,闻知此信,遂令人哨探。回
报蜀兵尽从洮阳而出。司马望曰:“姜维多计,莫非虚取洮阳而实来取祁山乎?”邓艾曰:
“今姜维实出洮阳也。”望曰:“公何以知之?”艾曰:“向者姜维累出吾有粮之地,今洮
阳无粮,维必料吾只守祁山,不守洮阳,故径取洮阳;如得此城,屯粮积草,结连羌人,以
图久计耳。”望曰:“若此,如之奈何?”艾曰:“可尽撤此处之兵,分为两路去救洮阳。
离洮阳二十五里,有侯河小城,乃洮阳咽喉之地。公引一军伏于洮阳,偃旗息鼓,大开四
门,如此如此而行;我却引一军伏侯河,必获大胜也。”筹画已定,各各依计而行。只留偏
将师纂守祁山寨。

却说姜维令夏侯霸为前部,先引一军径取洮阳。霸提兵前进,将近洮阳,望见城上并无
一杆旌旗,四门大开。霸心下疑惑,未敢入城,回顾诸将曰:“莫非诈乎?”诸将曰:“眼
见得是空城,只有些小百姓,听知大将军兵到,尽弃城而走了。”霸未信,自纵马于城南视
之,只见城后老小无数,皆望西北而逃。霸大喜曰:“果空城也。”遂当先杀入,余众随后
而进。方到瓮城边,忽然一声炮响,城上鼓角齐鸣,旌旗遍竖,拽起吊桥。霸大惊曰:“误
中计矣!”慌欲退时,城上矢石如雨。可怜夏侯霸同五百军,皆死于城下。后人有诗叹曰:
“大胆姜维妙算长,谁知邓艾暗提防。可怜投汉夏侯霸,顷刻城边箭下亡。”司马望从城内
杀出,蜀兵大败而逃。随后姜维引接应兵到,杀退司马望,就傍城下寨。维闻夏侯霸射死,
嗟伤不已。是夜二更,邓艾自侯河城内,暗引一军潜地杀入蜀寨。蜀兵大乱,姜维禁止不
住。城上鼓角喧天,司马望引兵杀出。两下夹攻,蜀兵大败。维左冲右突,死战得脱,退二
十余里下寨。蜀兵两番败走之后,心中摇动。维与众将曰:“胜败乃兵家之常,今虽损兵折
将,不足为忧。成败之事,在此一举,汝等始终勿改。如有言退者立斩。”张翼进言曰:
“魏兵皆在此处,祁山必然空虚。将军整兵与邓艾交锋,攻打洮阳、侯河;某引一军取祁
山。取了祁山九寨,便驱兵向长安。此为上计。”维从之,即令张翼引后军径取祁山。

维自引兵到侯河搦邓艾交战。艾引军出迎。两军对圆,二人交锋数十余合,不分胜负,
各收兵回寨。次日,姜维又引兵挑战,邓艾按兵不出。姜维令军辱骂。邓艾寻思曰:“蜀人
被吾大杀一阵,全然不退,连日反来搦战:必分兵去袭祁山寨也。守寨将师纂,兵少智寡,
必然败矣。吾当亲往救之。”乃唤子邓忠分付曰:“汝用心守把此处,任他搦战,却勿轻
出。吾今夜引兵去祁山救应。”

是夜二更,姜维正在寨中设计,忽听得寨外喊声震地,鼓角喧天,人报邓艾引三千精兵
夜战。诸将欲出,维止之曰:“勿得妄动。”原来邓艾引兵至蜀寨前哨探了一遍,乘势去救
祁山,邓忠自入城去了。姜维唤诸将曰:“邓艾虚作夜战之势,必然去救祁山寨矣。”乃唤
傅佥分付曰:“汝守此寨,勿轻与敌。”嘱毕,维自引三千兵来助张翼。

却说张翼正到祁山攻打,守寨将师纂兵少,支持不住。看看待破,忽然邓艾兵至,冲杀
了一阵,蜀兵大败,把张翼隔在山后,绝了归路。正慌急之间,忽听的喊声大震,鼓角喧
天,只见魏兵纷纷倒退。左右报曰:“大将军姜伯约杀到!”翼乘势驱兵相应。两下夹攻,
邓艾折了一阵,急退上祁山寨不出。姜维令兵四面攻围。话分两头。却说后主在成都,听信
宦官黄皓之言,又溺于酒色,不理朝政。时有大臣刘琰妻胡氏,极有颜色;因入宫朝见皇
后,后留在宫中,一月方出。琰疑其妻与后主私通,乃唤帐下军士五百人,列于前,将妻绑
缚,令军以履挞其面数十,几死复苏。后主闻之大怒,令有司议刘琰罪。有司议得:“卒非
挞妻之人,面非受刑之地:合当弃市。”遂斩刘琰。自此命妇不许入朝。然一时官僚以后主
荒淫,多有疑怨者。于是贤人渐退,小人日进。时右将军阎宇,身无寸功,只因阿附黄皓,
遂得重爵;闻姜维统兵在祁山,乃说皓奏后主曰:“姜维屡战无功,可命阎宇代之。”后主
从其言,遣使赍诏,召回姜维。维正在祁山攻打寨栅,忽一日三道诏至,宣维班师。维只得
遵命,先令洮阳兵退,次后与张翼徐徐而退。邓艾在寨中,只听得一夜鼓角喧天,不知何
意。至平明,人报蜀兵尽退,止留空寨。艾疑有计,不敢追袭。姜维径到汉中,歇住人马,
自与使命入成都见后主。后主一连十日不朝。维心中疑惑。是日至东华门,遇见秘书郎郤
正。维问曰:“天子召维班师,公知其故否?”正笑曰:“大将军何尚不知?黄皓欲使阎宇
立功,奏闻朝廷,发诏取回将军。今闻邓艾善能用兵,因此寝其事矣。”维大怒曰:“我必
杀此宦竖!”郤正止之曰:“大将军继武侯之事,任大职重,岂可造次?倘若天子不容,反
为不美矣。”维谢曰:“先生之言是也。”次日,后主与黄皓在后园宴饮,维引数人径入。
早有人报知黄皓,皓急避于湖山之侧。维至亭下,拜了后主,泣奏曰:“臣困邓艾于祁山,
陛下连降三诏,召臣回朝,未审圣意为何?”后主默然不语。维又奏曰:“黄皓奸巧专权,
乃灵帝时十常侍也。陛下近则鉴于张让,远则鉴于赵高。早杀此人,朝廷自然清平,中原方
可恢复。”后主笑曰:“黄皓乃趋走小臣,纵使专权,亦无能为。昔者董允每切齿恨皓,朕
甚怪之。卿何必介意?”维叩头奏曰:“陛下今日不杀黄皓,祸不远也。”后主曰:“爱之
欲其生,恶之欲其死。卿何不容一宦官耶?”令近侍于湖山之侧,唤出黄皓至亭下,命拜姜
维伏罪。皓哭拜维曰:“某早晚趋侍圣上而已,并不干与国政。将军休听外人之言,欲杀某
也。某命系于将军,惟将军怜之!”言罢,叩头流涕。维忿忿而出,即往见郤正,备将此事
告之。正曰:“将军祸不远矣。将军若危,国家随灭!”维曰:“先生幸教我以保国安身之
策。正曰:“陇西有一去处,名曰沓中,此地极其肥壮。将军何不效武侯屯田之事,奏知天
子,前去沓中屯田?一者,得麦熟以助军实;二者,可以尽图陇右诸郡;三者,魏人不敢正
视汉中;四者,将军在外掌握兵权,人不能图,可以避祸:此乃保国安身之策也,宜早行
之。”维大喜,谢曰:“先生金玉之言也。”次日,姜维表奏后主,求沓中屯田,效武侯之
事。后主从之。维遂还汉中,聚诸将曰:“某累出师,因粮不足,未能成功。今吾提兵八
万,往沓中种麦屯田,徐图进取。汝等久战劳苦,今且敛兵聚谷,退守汉中;魏兵千里运
粮,经涉山岭,自然疲乏;疲乏必退:那时乘虚追袭。无不胜矣。”遂令胡济守汉寿城,王
含守乐城,蒋斌守汉城,蒋舒、傅佥同守关隘。分拨已毕,维自引兵八万,来沓中种麦,以
为久计。

却说邓艾闻姜维在沓中屯田,于路下四十余营,连络不绝,如长蛇之势。艾遂令细作相
了地形,画成图本,具表申奏。晋公司马昭见之,大怒曰:“姜维屡犯中原,不能剿除,是
吾心腹之患也。”贾充曰:“姜维深得孔明传授,急难退之。须得一智勇之将,往刺杀之,
可免动兵之劳。”从事中郎荀顗曰:“不然。今蜀主刘禅溺于酒色,信用黄皓,大臣皆有避
祸之心。姜维在沓中屯田,正避祸之计也。若令大将伐之,无有不胜,何必用刺客乎?”昭
大笑曰:“此言最善。吾欲伐蜀,谁可为将?”荀顗曰:“邓艾乃世之良材,更得钟会为副
将,大事成矣。”昭大喜曰:“此言正合吾意。”乃召钟会入而问曰:“吾欲令汝为大将,
去伐东吴,可乎?”会曰:“主公之意,本不欲伐吴,实欲伐蜀也。”昭大笑曰:“子诚识
吾心也。——

但卿往伐蜀,当用何策?”会曰:“某料主公欲伐蜀,已画图本在此。”昭展开视之,
图中细载一路安营下寨屯粮积草之处,从何而进,从何而退,——皆有法度。昭看了大喜
曰:“真良将也!卿与邓艾合兵取蜀,何如?”会曰:“蜀川道广,非一路可进;当使邓艾
分兵各进,可也。”

昭遂拜钟会为镇西将军,假节钺,都督关中人马,调遣青、徐、兖、豫、荆、扬等处;
一面差人持节令邓艾为征西将军,都督关外陇上,使约期伐蜀。次日,司马昭于朝中计议此
事,前将军邓敦曰:“姜维屡犯中原,我兵折伤甚多,只今守御,尚自未保;奈何深入山川
危险之地,自取祸乱耶?”昭怒曰:“吾欲兴仁义之师,伐无道之主,汝安敢逆吾意!”叱
武士推出斩之。须臾,呈邓敦首级于阶下。众皆失色。昭曰:“吾自征东以来,息歇六年,
治兵缮甲,皆已完备,欲伐吴、蜀久矣。今先定西蜀,乘顺流之势,水陆并进,并吞东吴;
此灭豸虎取虞之道也。吾料西蜀将士,守成都者八九万,守边境者不过四五万,姜维屯田者
不过六七万。今吾已令邓艾引关外陇右之兵十余万,绊住姜维于沓中,使不得东顾;遣钟会
引关中精兵二三十万,直抵骆谷,三路以袭汉中。蜀主刘禅昏暗,边城外破,士女内震。其
亡可必矣。”众皆拜服。

却说钟会受了镇西将军之印,起兵伐蜀。会恐机谋或泄,却以伐吴为名,令青、兖、
豫、荆、扬等五处各造大船;又遣唐咨于登、莱等州傍海之处,拘集海船。司马昭不知其
意,遂召钟会问之曰:“子从旱路收川,何用造船耶?”会曰:“蜀若闻我兵大进,必求救
于东吴也。故先布声势,作伐吴之状,吴必不敢妄动。一年之内,蜀已破,船已成,而伐
吴,岂不顺乎?”昭大喜,选日出师。时魏景元四年秋七月初三日,钟会出师。司马昭送之
于城外十里方回。西曹掾邵悌密谓司马昭曰:“今主公遣钟会领十万兵伐蜀,愚料会志大心
高,不可使独掌大权。”昭笑曰:“吾岂不知之?”悌曰:“主公既知,何不使人同领其
职?”昭言无数语,使邵悌疑心顿释。正是:方当士马驱驰日,早识将军跋扈心。未知其言
若何,且看下文分解。