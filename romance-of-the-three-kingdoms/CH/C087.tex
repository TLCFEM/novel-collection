\chapter{征南寇丞相大兴师~抗天兵蛮王初受执}

却说诸葛丞相在于成都,事无大小,皆亲自从公决断。两川之民,忻乐太平,夜不闭
户,路不拾遗。又幸连年大熟,老幼鼓腹讴歌,凡遇差徭,争先早办。因此军需器械应用之
物,无不完备;米满仓廒,财盈府库。

建兴三年,益州飞报:蛮王孟获,大起蛮兵十万,犯境侵掠。建宁太守雍闿,乃汉朝什
方侯雍齿之后,今结连孟获造反。牂牁郡太守朱褒、越嶲郡太守高定,二人献了城。止有永
昌太守王伉不肯反。现今雍闿、朱褒、高定三人部下人马,皆与孟获为向导官,攻打永昌
郡。今王伉与功曹吕凯,会集百姓,死守此城,其势甚急。孔明乃入朝奏后主曰:“臣观南
蛮不服,实国家之大患也。臣当自领大军,前去征讨。”后主曰“东有孙权,北有曹丕,今
相父弃朕而去,倘吴、魏来攻,如之奈何?”孔明曰:“东吴方与我国讲和,料无异心;若
有异心,李严在白帝城,此人可当陆逊也。曹丕新败,锐气已丧,未能远图;且有马超守把
汉中诸处关口,不必忧也。臣又留关兴、张苞等分两军为救应,保陛下万无一失。今臣先去
扫荡蛮方,然后北伐,以图中原,报先帝三顾之恩,托孤之重。”后主曰:“朕年幼无知,
惟相父斟酌行之。”言未毕,班部内一人出曰:“不可!不可!”众视之,乃南阳人也,姓
王,名连,字文仪,现为谏议大夫。连谏曰:“南方不毛之地,瘴疫之乡;丞相秉钧衡之重
任,而自远征,非所宜也。且雍闿等乃疥癣之疾,丞相只须遣一大将讨之,必然成功。”孔
明曰:“南蛮之地,离国甚远,人多不习王化,收伏甚难,吾当亲去征之。可刚可柔,别有
斟酌,非可容易托人。”

王连再三苦劝,孔明不从。是日,孔明辞了后主,令蒋琬为参军,费祎为长史,董厥、
樊建二人为掾史;赵云、魏延为大将,总督军马;王平、张翼为副将;并川将数十员:共起
川兵五十万,前望益州进发。忽有关公第三子关索,入军来见孔明曰:“自荆州失陷,逃难
在鲍家庄养病。每要赴川见先帝报仇,疮痕未合,不能起行。近已安痊,打探得系吴仇人已
皆诛戮,径来西川见帝,恰在途中遇见征南之兵,特来投见。”孔明闻之,嗟讶不已;一面
遣人申报朝廷,就令关索为前部先锋,一同征南。大队人马,各依队伍而行。饥餐渴饮,夜
住晓行;所经之处,秋毫无犯。

却说雍闿听知孔明自统大军而来,即与高定、朱褒商议,分兵三路:高定取中路,雍闿在
左,朱褒在右;三路各引兵五六万迎敌。于是高定令鄂焕为前部先锋。焕身长九尺,面貌丑
恶,使一枝方天戟,有万夫不当之勇:领本部兵,离了大寨,来迎蜀兵。却说孔明统大军已
到益州界分。前部先锋魏延,副将张翼、王平,才入界口,正遇鄂焕军马。两阵对圆,魏延
出马大骂曰:“反贼早早受降!”鄂焕拍马与魏延交锋。战不数合,延诈败走,焕随后赶
来。走不数里,喊声大震。张翼、王平两路军杀来,绝其后路。延复回,三员将并力拒战,
生擒鄂焕。解到大寨,入见孔明。孔明令去其缚,以酒食待之。问曰:“汝是何人部将?”
焕曰:“某是高定部将。”孔明曰:“吾知高定乃忠义之士,今为雍闿所惑,以致如此。吾
今放汝回去,令高太守早早归降,免遭大祸。”鄂焕拜谢而去,回见高定,说孔明之德。定
亦感激不已。次日,雍闿至寨。礼毕,闿曰:“如何得鄂焕回也?”定曰:“诸葛亮以义放
之。”闿曰:“此乃诸葛亮反间之计:欲令我两人不和,故施此谋也。”定半信不信,心中
犹豫。忽报蜀将搦战,闿自引三万兵出迎。战不数合,闿拨马便走。延率兵大进,追杀二十余
里。次日,雍闿又起兵来迎。孔明一连三日不出。至第四日,雍闿、高定分兵两路,来取蜀
寨。却说孔明令魏延两路伺候;果然雍闿、高定两路兵来,被伏兵杀伤大半,生擒者无数,
都解到大寨来。雍闿的人,囚在一边;高定的人,囚在一边。却令军士谣说:“但是高定的
人免死,雍闿的人尽杀。”众军皆闻此言。少时,孔明令取雍闿的人到帐前,问曰:“汝等皆
是何人部从?”众伪曰:“高定部下人也。”孔明教皆免其死,与酒食赏劳,令人送出界
首,纵放回寨。孔明又唤高定的人问之。众皆告曰:“吾等实是高定部下军士。”孔明亦皆
免其死,赐以酒食;却扬言曰:“雍闿今日使人投降,要献汝主并朱褒首级以为功劳,吾甚
不忍。汝等既是高定部下军,吾放汝等回去,再不可背反。若再擒来,决不轻恕。”

众皆拜谢而去;回到本寨,入见高定,说知此事。定乃密遣人去雍闿寨中探听,却有一
般放回的人,言说孔明之德;因此雍闿部军,多有归顺高定之心。虽然如此,高定心中不
稳,又令一人来孔明寨中探听虚实。被伏路军捉来见孔明。孔明故意认做雍闿的人,唤入帐
中问曰:“汝元帅既约下献高定、朱褒二人首级,因何误了日期?汝这厮不精细,如何做得
细作!”军士含糊答应。孔明以酒食赐之,修密书一封,付军士曰:“汝持此书付雍闿,教
他早早下手,休得误事。”细作拜谢而去,回见高定,呈上孔明之书,说雍闿如此如此。定
看书毕,大怒曰:“吾以真心待之,彼反欲害吾,情理难容!”使唤鄂焕商议。焕曰:“孔
明乃仁人,背之不祥。我等谋反作恶,皆雍闿之故;不如杀闿以投孔明。”定曰:“如何下
手?”焕曰:“可设一席,令人去请雍闿。彼若无异心,必坦然而来;若其不来,必有异
心。我主可攻其前,某伏于寨后小路候之;闿可擒矣。”高定从其言,设席请雍闿。闿果疑前
日放回军士之言,惧而不来。是夜高定引兵杀投雍闿寨中。原来有孔明放回免死的人,皆想
高定之德,乘时助战。雍闿军不战自乱。闿上马望山路而走。行不二里,鼓声响处,一彪军
出,乃鄂焕也:挺方天戟,骤马当先。雍闿措手不及,被焕一戟刺于马下,就枭其首级。闿部
下军士皆降高定。定引两部军来降孔明,献雍闿首级于帐下。孔明高坐于帐上,喝令左右推
转高定,斩首报来。定曰:“某感丞相大恩,今将雍闿首级来降,何故斩也?”孔明大笑
曰:“汝来诈降。敢瞒吾耶!”定曰:“丞相何以知吾诈降?”孔明于匣中取出一缄,与高
定曰:“朱褒已使人密献降书,说你与雍闿结生死之交,岂肯一旦便杀此人?吾故知汝诈
也。”定叫屈曰:“朱褒乃反间之计也。丞相切不可信!”孔明曰:“吾亦难凭一面之词。
汝若捉得朱褒,方表真心。”定曰:“丞相休疑。某去擒朱褒来见丞相,若何?”孔明曰:
“若如此,吾疑心方息也。”

高定即引部将鄂焕并本部兵,杀奔朱褒营来。比及离寨约有十里,山后一彪军到,乃朱
褒也。褒见高定军来,慌忙与高定答话。定大骂曰:“汝如何写书与诸葛丞相处,使反间之
计害吾耶?”褒目瞪口呆,不能回答。忽然鄂焕于马后转过,一戟刺朱褒于马下。定厉声而
言曰:“如不顺者皆戮之!”于是众军一齐拜降。定引两部军来见孔明,献朱褒首级于帐
下。孔明大笑曰:“吾故使汝杀此二贼,以表忠心。”遂命高定为益州太守,总摄三郡;令
鄂焕为牙将。三路军马已平。

于是永昌太守王伉出城迎接孔明。孔明入城已毕,问曰:“谁与公守此城,以保无
虞?”伉曰:“某今日得此郡无危者,皆赖永昌不韦人,姓吕,名凯,字季平。皆此人之
力。”孔明遂请目凯至。凯入见,礼毕。孔明曰:“久闻公乃永昌高士,多亏公保守此城。
今欲平蛮方,公有何高见?”吕凯遂取一图,呈与孔明曰:“某自历仕以来,知南人欲反久
矣,故密遣人入其境,察看可屯兵交战之处,画成一图,名曰《平蛮指掌图》。今敢献与明
公。明公试观之,可为征蛮之一助也。”孔明大喜,就用吕凯为行军教授,兼向导官。于是
孔明提兵大进,深入南蛮之境。正行军之次,忽报天子差使命至。孔明请入中军,但见一人
素袍白衣而进,乃马谡也——为兄马良新亡,因此挂孝。——谡曰:“奉主上敕命,赐众军
酒帛。”孔明接诏已毕,依命一一给散,遂留马谡在帐叙话。孔明问曰:“吾奉天子诏,削
平蛮方;久闻幼常高见,望乞赐教。”谡曰:“愚有片言,望丞相察之;南蛮恃其地远山
险,不服久矣;虽今日破之,明日复叛。丞相大军到彼,必然平服;但班师之日,必用北伐
曹丕;蛮兵若知内虚,其反必速。夫用兵之道:攻心为上,攻城为下;心战为上,兵战为
下。愿丞相但服其心足矣。”孔明叹曰:“幼常足知吾肺腑也!”于是孔明遂令马谡为参
军,即统大兵前进。却说蛮王孟获,听知孔明智破雍闿等,遂聚三洞元帅商议。第一洞乃金

环三结元帅,第二洞乃董荼那元帅,第三洞乃阿会喃元帅。三洞元帅入见孟获。获曰:“今
诸葛丞相领大军来侵我境界,不得不并力敌之。汝三人可分兵三路而进。如得胜者,便为洞
主。”于是分金环三结取中路,董荼那取左路,阿会喃取右路:各引五万蛮兵,依令而行。

却说孔明正在寨中议事,忽哨马飞报,说三洞元帅分兵三路到来。孔明听毕,即唤赵
云、魏延至,却都不分付;更唤王平、马忠至,嘱之曰:“今蛮兵三路而来,吾欲令子龙、
文长去;此二人不识地理,未敢用之。王平可往左路迎敌,马忠可往右路迎敌。吾却使子
龙、文长随后接应。今日整顿军马,来日平明进发。”二人听令而去。又唤张嶷、张翼分付
曰:“汝二人同领一军,往中路迎敌。今日整点军马,来日与王平、马忠约会而进。吾欲令
子龙、文长去取,奈二人不识地理,故未敢用之。”张嶷、张翼听令去了。

赵云、魏延见孔明不用,各有愠色。孔明曰:“吾非不用汝二人,但恐以中年涉险,为
蛮人所算,失其锐气耳。”赵云曰:“倘我等识地理,若何?”孔明曰:“汝二人只宜小
心,休得妄动。”二人怏怏而退。赵云请魏延到自己寨内商议曰:“吾二人为先锋,却说不
识地理而不肯用。今用此后辈,吾等岂不羞乎?”延曰:“吾二人只今就上马,亲去探之;
捉住土人,便教引进,以敌蛮兵,大事可成。”云从之,遂上马径取中路而来。方行不数
里,远远望见尘头大起。二人上山坡看时,果见数十骑蛮兵,纵马而来。二人两路冲出。蛮
兵见了,大惊而走。赵云、魏延各生擒几人,回到本寨,以酒食待之,却细问其故。蛮兵告
曰:“前面是金环三结元帅大寨,正在山口。寨边东西两路,却通五溪洞并董荼那、阿会喃
各寨之后。”

赵云、魏延听知此话,遂点精兵五千,教擒来蛮兵引路。比及起军时,已是二更天气;
月明星朗,趁着月色而行。刚到金环三结大寨之时,约有四更,蛮兵方起造饭,准备天明厮
杀。忽然赵云、魏延两路杀入,蛮兵大乱。赵云直杀入中军,正逢金环三结元帅;交马只一
合,被云一枪刺落马下,就枭其首级。余军溃散。魏延便分兵一半,望东路抄董荼那寨来。
赵云分兵一半,望西路抄阿会喃寨来。比及杀到蛮兵大寨之时,天已平明。先说魏延杀奔董
荼那寨来。董荼那听知寨后有军杀至,便引兵出寨拒敌。忽然寨前门一声喊起,蛮兵大乱。
原来王平军马早已到了。两下夹攻,蛮兵大败。董荼那夺路走脱,魏延追赶不上。却说赵云
引兵杀到阿会喃寨后之时,马忠已杀至寨前。两下夹攻,蛮兵大败,阿会喃乘乱走脱。各自
收军,回见孔明。孔明问曰:“三洞蛮兵,走了两洞之主;金环三结元帅首级安在?”赵云
将首级献功。众皆言曰:“董荼那、阿会喃皆弃马越岭而去,因此赶他不上。”孔明大笑
曰:“二人吾已擒下了。”赵、魏二人并诸将皆不信。少顷,张嶷解董荼那到,张翼解阿会
喃到。众皆惊讶。孔明曰:“吾观吕凯图本,已知他各人下的寨子,故以言激子龙、文长之
锐气,故教深入重地,先破金环三结,随即分兵左右寨后抄出,以王平、马忠应之。非子
龙、文长不可当此任也。吾料董荼那、阿会喃必从便径往山路而走,故遣张嶷、张翼以伏兵
待之,令关索以兵接应,擒此二人。”诸将皆拜伏曰:“丞相机算,神鬼莫测!”

孔明令押过董荼那、阿会喃至帐下,尽去其缚,以酒食衣服赐之,令各自归洞,勿得助
恶。二人泣拜,各投小路而去。孔明谓诸将曰:“来日孟获必然亲自引兵厮杀,便可就此擒
之。”乃唤赵云、魏延至,付与计策,各引五千兵去了。又唤王平、关索同引一军,授计而
去。孔明分拨已毕,坐于帐上待之。却说蛮王孟获在帐中正坐,忽哨马报来,说三洞元帅,
俱被孔明捉将去了;部下之兵,各自溃散。获大怒,遂起蛮兵迤逦进发,正遇王平军马。两
阵对圆,王平出马横刀望之:只见门旗开处,数百南蛮骑将两势摆开。中间孟获出马:头顶
嵌宝紫金冠,身披缨络红锦袍,腰系碾玉狮子带,脚穿鹰嘴抹绿靴,骑一匹卷毛赤兔马,悬
两口松纹镶宝剑,昂然观望,回顾左右蛮将曰:“人每说诸葛亮善能用兵;今观此阵,旌旗
杂乱,队伍交错;刀枪器械,无一可能胜吾者:始知前日之言谬也。早知如此,吾反多时
矣。谁敢去擒蜀将:以振军威?”言未尽,一将应声而出,名唤忙牙长;使一口截头大刀,
骑一匹黄骠马,来取王平。二将交锋,战不数合,王平便走。孟获驱兵大进,迤逦追赶。关
索略战又走,约退二十余里。孟获正追杀之间,忽然喊声大起,左有张嶷,右有张翼,两路
兵杀出,截断归路。王平、关索复兵杀回。前后夹攻,蛮兵大败。孟获引部将死战得脱,望
锦带山而逃。背后三路兵追杀将来。获正奔走之间,前面喊声大起,一彪军拦住:为首大将
乃常山赵子龙也。获见了大惊,慌忙奔锦带山小路而走。子龙冲杀一阵,蛮兵大败,生擒者
无数。孟获止与数十骑奔入山谷之中,背后追兵至近,前面路狭,马不能行,乃弃了马匹,
爬山越岭而逃。忽然山谷中一声鼓响,乃是魏延受了孔明计策,引五百步军,伏于此处,孟
获抵敌不住,被魏延生擒活捉了。从骑皆降。魏延解孟获到大寨来见孔明。孔明早已杀牛宰
羊,设宴在寨;却教帐中排开七重围子手,刀枪剑戟,灿若霜雪;又执御赐黄金钺斧,曲柄
伞盖,前后羽葆鼓吹,左右排开御林军,布列得十分严整。孔明端坐于帐上,只见蛮兵纷纷
穰穰,解到无数。孔明唤到帐中,尽去其缚,抚谕曰:“汝等皆是好百姓,不幸被孟获所
拘,今受惊?。吾想汝等父母、兄弟、妻子必倚门而望;若听知阵败,定然割肚牵肠,眼中
流血。吾今尽放汝等回去,以安各人父母、兄弟、妻子之心。”言讫,各赐酒食米粮而遣
之。蛮兵深感其恩,泣拜而去。孔明教唤武士押过孟获来。不移时,前推后拥,缚至帐前。
获跪与帐下。孔明曰:“先帝待汝不薄,汝何敢背反?”获曰:“两川之地,皆是他人所占
土地,汝主倚强夺之,自称为帝。吾世居此处,汝等无礼,侵我土地:何为反耶?”孔明
曰:“吾今擒汝,汝心服否?”获曰:“山僻路狭,误遭汝手,如何肯服!”孔明曰:“汝
既不服,吾放汝去,若何?”获曰:“汝放我回去,再整军马,共决雌雄;若能再擒吾,吾
方服也。”孔明即令去其缚。与衣服穿了,赐以酒食,给与鞍马,差人送出路,径望本寨而
去。正是:寇入掌中还放去,人居化外未能降。未知再来交战若何,且看下文分解。