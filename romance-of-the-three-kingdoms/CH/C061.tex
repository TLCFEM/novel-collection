\chapter{赵云截江夺阿斗~孙权遗书退老瞒}

却说庞统、法正二人,劝玄德就席间杀刘璋,西川唾手可得。玄德曰:“吾初入蜀中,恩信未立,此事决不可行。”二人再三说之,玄德只是不从。次日,复与刘璋宴于城中,彼此细叙衷曲,情好甚密。酒至半酣,庞统与法正商议曰:“事已至此,由不得主公了。”便教魏延登堂舞剑,乘势杀刘璋。延遂拔剑进曰:“筵间无以为乐,愿舞剑为戏。”庞统便唤众武士入,列于堂下,只待魏延下手。刘璋手下诸将,见魏延舞剑筵前,又见阶下武士手按刀靶,直视堂上,从事张任亦掣剑舞曰:“舞剑必须有对,某愿与魏将军同舞。”二人对舞于筵前。魏延目视刘封,封亦拔剑助舞。于是刘璝、泠苞、邓贤各掣剑出曰:“我等当群舞,以助一笑。”玄德大惊,急掣左右所佩之剑,立于席上曰:“吾兄弟相逢痛饮,并无疑忌。又非鸿门会上,何用舞剑?不弃剑者立斩!”刘璋亦叱曰:“兄弟相聚,何必带刀?”命侍卫者尽去佩剑。众皆纷然下堂。玄德唤诸将士上堂,以酒赐之,曰:“吾弟兄同宗骨血,共议大事,并无二心。汝等勿疑。”诸将皆拜谢。刘璋执玄德之手而泣曰:“吾兄之恩,誓不敢忘!”二人欢饮至晚而散。玄德归寨,责庞统曰:“公等奈何欲陷备于不义耶?今后断勿为此。”统嗟叹而退。却说刘璋归寨,刘璝等曰:“主公见今日席上光景乎?不如早回,免生后患。刘璋曰:“吾兄刘玄德,非比他人。”众将曰:“虽玄德无此心,他手下人皆欲吞并西川,以图富贵。”璋曰:“汝等无间吾兄弟之情。”遂不听,日与玄德欢叙。忽报张鲁整顿兵马,将犯葭萌关。刘璋便请玄德往拒之。玄德慨然领诺,即日引本部兵望葭萌关去了。众将劝刘璋令大将紧守各处关隘,以防玄德兵变。璋初时不从,后因众人苦劝,乃令白水都督杨怀、高沛二人,守把涪水关。刘璋自回成都。玄德到葭萌关,严禁军士,广施恩惠,以收民心。

早有细作报入东吴。吴侯孙权会文武商议。顾雍进曰:“刘备分兵远涉山险而去,未易往还。何不差一军先截川口,断其归路,后尽起东吴之兵,一鼓而下荆襄?此不可失之机会也。”权曰:“此计大妙!”正商议间,忽屏风后一人大喝而出曰:“进此计者可斩之!欲害吾女之命耶!”众惊视之,乃吴国太也。国太怒曰:“吾一生惟有一女,嫁与刘备。今若动兵,吾女性命如何!”因叱孙权曰:“汝掌父兄之业,坐领八十一州,尚自不足,乃顾小利而不念骨肉!”孙权喏喏连声,答曰:“老母之训,岂敢有违!”遂叱退众官。国太恨恨而入。孙权立于轩下,自思:“此机会一失,荆襄何日可得?”正沉吟间,只见张昭入问曰:“主公有何忧疑?”孙权曰:“正思适间之事。”张昭曰:“此极易也:今差心腹将一人,只带五百军。潜入荆州,下一封密书与郡主,只说国太病危,欲见亲女,取郡主星夜回东吴。玄德平生只有一子,就教带来。那时玄德定把荆州来换阿斗。如其不然,一任动兵,更有何碍?”权曰:“此计大妙!吾有一人,姓周,名善,最有胆量。自幼穿房入户,多随吾兄。今可差他去。”昭曰:“切勿漏泄。只此便令起行。”于是密遣周善将五百人,扮为商人,分作五船;更诈修国书,以备盘诘;船内暗藏兵器。周善领命,取荆州水路而来。船泊江边,善自入荆州,令门吏报孙夫人。夫人命周善入。善呈上密书。夫人见说国太病危,洒泪动问。周善拜诉曰:“国太好生病重,旦夕只是思念夫人。倘去得迟,恐不能相见。就教夫人带阿斗去见一面。”夫人曰:“皇叔引兵远出,我今欲回,须使人知会军师,方可以行。”周善曰:“若军师回言道:须报知皇叔,候了回命,方可下船,如之奈何?”夫人曰:“若不辞而去,恐有阻当。”周善曰:“大江之中,已准备下船只。只今便请夫人上车出城。”孙夫人听知母病危急,如何不慌?便将七岁孩子阿斗,载在车中;随行带三十余人,各跨刀剑,上马离荆州城,便来江边上船。府中人欲报时,孙夫人已到沙头镇,下在船中了。

周善方欲开船,只听得岸上有人大叫:“且休开船,容与夫人饯行!”视之,乃赵云也。原来赵云巡哨方回,听得这个消息,吃了一惊,只带四五骑,旋风般沿江赶来。周善手执长戈,大喝曰:“汝何人,敢当主母!”叱令军士一齐开船,各将军器出来,摆列在船上。风顺水急,船皆随流而去。赵云沿江赶叫:“任从夫人去。只有一句话拜禀。”周善不睬,只催船速进。赵云沿江赶到十余里,忽见江滩斜缆一只渔船在那里。赵云弃马执枪,跳上渔船。只两人驾船前来,望着夫人所坐大船追赶。周善教军士放箭。赵云以枪拨之,箭皆纷纷落水。离大船悬隔丈余,吴兵用枪乱刺。赵云弃枪在小船上,掣所佩青釭剑在手,分开枪搠,望吴船涌身一跳,早登大船。吴兵尽皆惊倒。赵云入舱中,见夫人抱阿斗于怀中,喝赵云曰:“何故无礼!”云插剑声喏曰:“主母欲何往?何故不令军师知会?”夫人曰:“我母亲病在危笃,无暇报知。”云曰:“主母探病,何故带小主人去?”夫人曰:“阿斗是吾子,留在荆州,无人看觑。”云曰:“主母差矣。主人一生,只有这点骨血,小将在当阳长坂坡百万军中救出,今日夫人却欲抱将去,是何道理?”夫人怒曰:“量汝只是帐下一武夫,安敢管我家事!”云曰:“夫人要去便去,只留下小主人。”夫人喝曰:“汝半路辄入船中,必有反意!”云曰:“若不留下小主人,纵然万死,亦不敢放夫人去。”夫人喝侍婢向前揪捽,被赵云推倒,就怀中夺了阿斗,抱出船头上。欲要傍岸,又无帮手;欲要行凶,又恐碍于道理:进退不得。夫人喝侍婢夺阿斗,赵云一手抱定阿斗,一手仗剑,人不敢近。周善在后梢挟住舵,只顾放船下水。风顺水急,望中流而去。赵云孤掌难鸣,只护得阿斗,安能移舟傍岸。

正在危急,忽见下流头港内一字儿使出十余只船来,船上磨旗擂鼓。赵云自思:“今番中了东吴之计!”只见当头船上一员大将,手执长矛,高声大叫:“嫂嫂留下侄儿去!”原来张飞巡哨,听得这个消息,急来油江夹口,正撞着吴船,急忙截住。当下张飞提剑跳上吴船。周善见张飞上船,提刀来迎,被张飞手起一剑砍倒,提头掷于孙夫人前。夫人大惊曰:“叔叔何故无礼?”张飞曰:“嫂嫂不以俺哥哥为重,私自归家,这便无礼!”夫人曰:“吾母病重,甚是危急,若等你哥哥回报,须误了我事。若你不放我回去,我情愿投江而死!”

张飞与赵云商议:“若逼死夫人,非为臣下之道。只护着阿斗过船去罢。”乃谓夫人曰:“俺哥哥大汉皇叔,也不辱没嫂嫂。今日相别,若思哥哥恩义,早早回来。”说罢,抱了阿斗,自与赵云回船,放孙夫人五只船去了。后人有诗赞子龙曰:“昔年救主在当阳,今日飞身向大江。船上吴兵皆胆裂,子龙英勇世无双!”又有诗赞翼德曰:“长坂桥边怒气腾,一声虎啸退曹兵。今朝江上扶危主,青史应传万载名。”

二人欢喜回船。行不数里,孔明引大队船只接来,见阿斗已夺回,大喜。三人并马而归。孔明自申文书往葭萌关,报知玄德。却说孙夫人回吴,具说张飞、赵云杀了周善,截江夺了阿斗。孙权大怒曰:“今吾妹已归,与彼不亲,杀周善之仇,如何不报!”唤集文武,商议起军攻取荆州。正商议调兵,忽报曹操起军四十万来报赤壁之仇。孙权大惊,且按下荆州,商议拒敌曹操。人报长史张纮辞疾回家,今已病故,有哀书上呈。权拆视之,书中劝孙权迁居秣陵,言秣陵山川有帝王之气,可速迁于此,以为万世之业。孙权览书大哭,谓众官曰:“张子纲劝吾迁居秣陵,吾如何不从!”即命迁治建业,筑石头城。吕蒙进曰:“曹操兵来,可于濡须水口筑坞以拒之。”诸将皆曰:“上岸击贼,跣足入船,何用筑城?”蒙曰:“兵有利钝,战无必胜。如猝然遇敌,步骑相促,人尚不暇及水,何能入船乎?”权曰:“人无远虑,必有近忧。子明之见甚远。”便差军数万筑濡须坞。晓夜并工,刻期告竣。

却说曹操在许都,威福日甚。长史董昭进曰:“自古以来,人臣未有如丞相之功者,虽周公、吕望,莫可及也。栉风沐雨,三十余年,扫荡群凶,与百姓除害,使汉室复存。岂可与诸臣宰同列乎?合受魏公之位,加九锡以彰功德。”你道那九锡?一,车马(大辂、戎辂各一。大辂,金车也。戎辂,兵车也。玄牡二驷,黄马八匹。)二,衣服(衮冕之服,赤舄副焉。衮冕,王者之服。赤舄,朱履也。)三,乐悬(乐悬,王者之乐也。)四,朱户(居以朱户,红门也。)五,纳陛(纳陛以登。陛,阶也。)六,虎贲(虎贲三百人,守门之军也。)七,鈇钺(鈇钺各一。鈇,即斧也。钺,斧属。)八,弓矢(彤弓一,彤矢百。彤,赤色也。【左玄右旅去方】弓十,【左玄右旅去方】矢千。【左玄右旅去方】,黑色也。)九,秬鬯圭瓒(秬鬯一卣,圭瓒副焉。秬,黑黍也。鬯,香酒,灌地以求神于阴。卣,中樽也。圭瓒,宗庙祭器,以祀先王也。)侍中荀彧曰:“不可。丞相本兴义兵,匡扶汉室,当秉忠贞之志,守谦退之节。君子爱人以德,不宜如此。”曹操闻言,勃然变色。董昭曰:“岂可以一人而阻众望?”遂上表请尊操为魏公,加九锡。荀彧叹曰:“吾不想今日见此事!”操闻,深恨之,以为不助己也。建安十七年冬十月,曹操兴兵下江南,就命荀彧同行。彧已知操有杀己之心,托病止于寿春。忽曹操使人送饮食一盒至。盒上有操亲笔封记。开盒视之,并无一物。彧会其意,遂服毒而亡。年五十岁。后人有诗叹曰:“文若才华天下闻,可怜失足在权门。后人休把留侯比,临没无颜见汉君。”其子荀恽,发哀书报曹操。操甚懊悔,命厚葬之,谥曰敬侯。

且说曹操大军至濡须,先差曹洪领三万铁甲马军,哨至江边。回报云:“遥望沿江一带,旗幡无数,不知兵聚何处。”操放心不下,自领兵前进,就濡须口排开军阵。操领百余人上山坡,遥望战船,各分队伍,依次摆列。旗分五色,兵器鲜明。当中大船上青罗伞下,坐着孙权。左右文武,侍立两边。操以鞭指曰:“生子当如孙仲谋!若刘景升儿子,豚犬耳!”忽一声响动,南船一齐飞奔过来。濡须坞内又一军出,冲动曹兵。曹操军马退后便走,止喝不住。忽有千百骑赶到山边,为首马上一人碧眼紫髯,众人认得正是孙权。权自引一队马军来击曹操。操大惊,急回马时,东吴大将韩当、周泰,两骑马直冲将上来。操背后许褚纵马舞刀,敌住二将,曹操得脱归寨。许褚与二将战三十合方回。操回寨,重赏许褚,责骂众将:“临敌先退,挫吾锐气!后若如此,尽皆斩首。”是夜二更时分,忽寨外喊声大震。操急上马,见四下里火起,却被吴兵劫入大寨。杀至天明,曹兵退五十余里下寨。操心中郁闷,闲看兵书。程昱曰:“丞相既知兵法,岂不知兵贵神速乎?丞相起兵,迁延日久,故孙权得以准备,夹濡须水口为坞,难于攻击。不若且退兵还许都,别作良图。”操不应。

程昱出。操伏几而卧,忽闻潮声汹涌,如万马争奔之状。操急视之,见大江中推出一轮红日,光华射目;仰望天上,又有两轮太阳对照。忽见江心那轮红日,直飞起来,坠于寨前山中,其声如雷。猛然惊觉,原来在帐中做了一梦。帐前军报道午时。曹操教备马,引五十余骑,径奔出寨,至梦中所见落日山边。正看之间,忽见一簇人马,当先一人,金盔金甲。操视之,乃孙权也。权见操至,也不慌忙,在山上勒住马,以鞭指操曰:“丞相坐镇中原,富贵已极,何故贪心不足,又来侵我江南?”操答曰:“汝为臣下,不尊王室。吾奉天子诏,特来讨汝!”孙权笑曰:“此言岂不羞乎?天下岂不知你挟天子令诸侯?吾非不尊汉朝,正欲讨汝以正国家耳。”操大怒,叱诸将上山捉孙权。忽一声鼓响,山背后两彪军出,右边韩当、周泰,左边陈武、潘璋。四员将带三千弓弩手乱射,矢如雨发。操急引众将回走。背后四将赶来甚急。赶到半路,许褚引众虎卫军敌住,救回曹操。吴兵齐奏凯歌,回濡须去了。操还营自思:“孙权非等闲人物。红日之应,久后必为帝王。”于是心中有退兵之意,又恐东吴耻笑,进退未决。两边又相拒了月余,战了数场,互相胜负。直至来年正月,春雨连绵,水港皆满,军士多在泥水之中,困苦异常。操心甚忧。当日正在寨中,与众谋士商议。或劝操收兵,或云目今春暖,正好相持,不可退归。操犹豫未定。

忽报东吴有使赍书到。操启视之。书略曰:“孤与丞相,彼此皆汉朝臣宰。丞相不思报国安民,乃妄动干戈,残虐生灵,岂仁人之所为哉?即日春水方生,公当速去。如其不然,复有赤壁之祸矣。公宜自思焉。”书背后又批两行云:“足下不死,孤不得安。”曹操看毕,大笑曰:“孙仲谋不欺我也。”重赏来使,遂下令班师,命庐江太守朱光镇守皖城,自引大军回许昌。孙权亦收军回秣陵。权与众将商议:“曹操虽然北去,刘备尚在葭萌关未还。何不引拒曹操之兵,以取荆州?”张昭献计曰:“且未可动兵。某有一计,使刘备不能再还荆州。”正是:孟德雄兵方退北,仲谋壮志又图南。不知张昭说出甚计来,且看下文分解。