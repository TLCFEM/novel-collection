\chapter{刘皇叔北海救孔融~吕温侯濮阳破曹操}

却说献计之人,乃东海朐县人,姓糜,名竺,字子仲。此人家世富豪,尝往洛阳买卖,乘车而回,路遇一美妇人,来求同载,竺乃下车步行,让车与妇人坐。妇人请竺同载。竺上车端坐,目不邪视。行及数里,妇人辞去;临别对竺曰:“我乃南方火德星君也,奉上帝教,往烧汝家。感君相待以礼,故明告君。君可速归,搬出财物。吾当夜来。”言讫不见。竺大惊,飞奔到家,将家中所有,疾忙搬出。是晚果然厨中火起,尽烧其屋。竺因此广舍家财,济贫拔苦。后陶谦聘为别驾从事。当日献计曰:“某愿亲往北海郡,求孔融起兵救援;更得一人往青州田楷处求救:若二处军马齐来,操必退兵矣。”谦从之,遂写书二封,问帐下谁人敢去青州求救。一人应声愿往。众视之,乃广陵人,姓陈,名登,字元龙。陶谦先打发陈元龙往青州去讫,然后命糜竺赍书赴北海,自己率众守城,以备攻击。

却说北海孔融,字文举,鲁国曲阜人也,孔子二十世孙,泰山都尉孔宙之子。自小聪明,年十岁时,往谒河南尹李膺,阍人难之,融曰:“我系李相通家。”及入见,膺问曰:“汝祖与吾祖何亲?”融曰:“昔孔子曾问礼于老子,融与君岂非累世通家?”膺大奇之。少顷,太中大夫陈炜至。膺指融曰:“此奇童也。”炜曰:“小时聪明,大时未必聪明。”融即应声曰:“如君所言,幼时必聪明者。”炜等皆笑曰:“此子长成,必当代之伟器也。”自此得名。后为中郎将,累迁北海太守。极好宾客,常曰:“座上客常满,樽中酒不空:吾之愿也。”在北海六年,甚得民心。当日正与客坐,人报徐州糜竺至。融请入见,问其来意,竺出陶谦书,言:“曹操攻围甚急,望明公垂救。”融曰:“吾与陶恭祖交厚,子仲又亲到此,如何不去?只是曹孟德与我无仇,当先遣人送书解和。如其不从,然后起兵。”竺曰:“曹操倚仗兵威,决不肯和。”融教一面点兵,一面差人送书。正商议间,忽报黄巾贼党管亥部领群寇数万杀奔前来。孔融大惊,急点本部人马,出城与贼迎战。管亥出马曰:“吾知北海粮广,可借一万石,即便退兵;不然,打破城池,老幼不留!”孔融叱曰:“吾乃大汉之臣,守大汉之地,岂有粮米与贼耶!”管亥大怒,拍马舞刀,直取孔融,融将宗宝挺枪出马;战不数合,被管亥一刀,砍宗宝于马下。孔融兵大乱,奔入城中。管亥分兵四面围城,孔融心中郁闷。糜竺怀愁,更不可言。次日,孔融登城遥望,贼势浩大,倍添忧恼。忽见城外一人挺枪跃马杀入贼阵,左冲右突,如入无人之境,直到城下,大叫“开门”。孔融不识其人,不敢开门。贼众赶到壕边,那人回身连搠十数人下马,贼众倒退,融急命开门引入。其人下马弃枪,径到城上,拜见孔融。融问其姓名,对曰:“某东莱黄县人也,覆姓太史,名慈,字子义。老母重蒙恩顾。某昨自辽东回家省亲,知贼寇城。老母说:‘屡受府君深恩,汝当往救。’某故单马而来。”孔融大喜。原来孔融与太史慈虽未识面,却晓得他是个英雄。因他远出,有老母住在离城二十里之外,融常使人遗以粟帛;母感融德,故特使慈来救。

当下孔融重待太史慈,赠与衣甲鞍马。慈曰:“某愿借精兵一千,出城杀贼。”融曰:“君虽英勇,然贼势甚盛,不可轻出。”慈曰:“老母感君厚德,特遣慈来;如不能解围,慈亦无颜见母矣。愿决一死战!”融曰:“吾闻刘玄德乃当世英雄,若请得他来相救,此围自解。只无人可使耳。”慈曰:“府君修书,某当急往。”融喜,修书付慈,慈擐甲上马,腰带弓矢,手持铁枪,饱食严装,城门开处,一骑飞出。近壕,贼将率众来战。慈连搠死数人,透围而出。管亥知有人出城,料必是请救兵的,便自引数百骑赶来,八面围定。慈倚住枪,拈弓搭箭,八面射之,无不应弦落马。贼众不敢来追。

太史慈得脱,星夜投平原来见刘玄德。施礼罢,具言孔北海被围求救之事,呈上书札。玄德看毕,问慈曰:“足下何人?”慈曰:“某太史慈,东海之鄙人也。与孔融亲非骨肉,比非乡党,特以气谊相投,有分忧共患之意。今管亥暴乱,北海被围,孤穷无告,危在旦夕。闻君仁义素著,能救人危急,故特令某冒锋突围,前来求救。”玄德敛容答曰:“孔北海知世间有刘备耶?”乃同云长、翼德点精兵三千,往北海郡进发。

管亥望见救军来到,亲自引兵迎敌;因见玄德兵少,不以为意。玄德与关、张、太史慈立马阵前,管亥忿怒直出。太史慈却待向前,云长早出,直取管亥。两马相交,众军大喊。量管亥怎敌得云长,数十合之间,青龙刀起,劈管亥于马下。太史慈、张飞两骑齐出,双枪并举,杀入贼阵。玄德驱兵掩杀。城上孔融望见太史慈与关、张赶杀贼众,如虎入羊群,纵横莫当,便驱兵出城。两下夹攻,大败群贼,降者无数,余党溃散。孔融迎接玄德入城,叙礼毕,大设筵宴庆贺。又引糜竺来见玄德,具言张闿杀曹嵩之事:“今曹操纵兵大掠,围住徐州,特来求救。”玄德曰:“陶恭祖乃仁人君子,不意受此无辜之冤。”孔融曰:“公乃汉室宗亲。今曹操残害百姓,倚强欺弱,何不与融同往救之?”玄德曰:“备非敢推辞,奈兵微将寡,恐难轻动。“孔融曰:“融之欲救陶恭祖,虽因旧谊,亦为大义。公岂独无仗义之心耶?”玄德曰:“既如此,请文举先行,容备去公孙瓒处,借三五千人马,随后便来。”融曰;“公切勿失信。”玄德曰:“公以备为何如人也?圣人云:自古皆有死,人无信不立。刘备借得军、或借不得军,必然亲至。”孔融应允,教糜竺先回徐州去报,融便收拾起程。太史慈拜谢曰:“慈奉母命前来相助,今幸无虞。有扬州刺史刘繇,与慈同郡,有书来唤,不敢不去。容图再见。”融以金帛相酬,慈不肯受而归。其母见之,喜曰:“我喜汝有以报北海也!”遂遣慈往扬州去了。不说孔融起兵。且说玄德离北海来见公孙瓒,具说欲救徐州之事。瓒曰:“曹操与君无仇,何苦替人出力?”玄德曰:“备已许人,不敢失信。”瓒曰:“我借与君马步军二千。”玄德曰:“更望借赵子龙一行。”瓒许之。玄德遂与关、张引本部三千人为前部,子龙引二千人随后,往徐州来。

却说糜竺回报陶谦,言北海又请得刘玄德来助;陈元龙也回报青州田楷欣然领兵来救;陶谦心安。原来孔融、田楷两路军马,惧怕曹兵势猛,远远依山下寨,未敢轻进。曹操见两路军到,亦分了军势,不敢向前攻城。

却说刘玄德军到,见孔融。融曰:“曹兵势大,操又善于用兵,未可轻战。且观其动静,然后进兵。”玄德曰:“但恐城中无粮,难以久持。备令云长、子龙领军四千,在公部下相助;备与张飞杀奔曹营,径投徐州去见陶使君商议。”融大喜,会合田楷,为掎角之势;云长、子龙领兵两边接应。是日玄德、张飞引一千人马杀入曹兵寨边。正行之间,寨内一声鼓响,马军步军,如潮似浪,拥将出来。当头一员大将,乃是于禁,勒马大叫:“何处狂徒!往那里去!”张飞见了,更不打话,直取于禁。两马相交,战到数合,玄德掣双股剑麾兵大进,于禁败走。张飞当前追杀,直到徐州城下。

城上望见红旗白字,大书“平原刘玄德”,陶谦急令开门。玄德入城,陶谦接着,共到府衙。礼毕,设宴相待,一壁劳军。陶谦见玄德仪表轩昂,语言豁达,心中大喜,便命糜竺取徐州牌印,让与玄德。玄德愕然曰:“公何意也?”谦曰:“今天下扰乱,王纲不振;公乃汉室宗亲,正宜力扶社稷。老夫年迈无能,情愿将徐州相让。公勿推辞。谦当自写表文,申奏朝廷。”玄德离席再拜曰:“刘备虽汉朝苗裔,功微德薄,为平原相犹恐不称职。今为大义,故来相助。公出此言,莫非疑刘备有吞并之心耶?若举此念,皇天不佑!”谦曰:“此老夫之实情也。”再三相让,玄德那里肯受。糜竺进曰:“今兵临城下,且当商议退敌之策。待事平之日,再当相让可也。”玄德曰:“备生遗书于曹操,劝令解和。操若不从,厮杀未迟。”于是传檄三寨,且执兵不动;遣人赍书以达曹操。

却说曹操正在军中,与诸将议事,人报徐州有战书到。操拆而观之,乃刘备书也。书略曰:“备自关外得拜君颜,嗣后天各一方,不及趋侍。向者,尊父曹侯,实因张闿不仁,以致被害,非陶恭祖之罪也。目今黄巾遗孽,扰乱于外;董卓余党,盘踞于内。愿明公先朝廷之急,而后私仇;撤徐州之兵,以救国难:则徐州幸甚,天下幸甚!”曹操看书,大骂:“刘备何人,敢以书来劝我!且中间有讥讽之意!”命斩来使,一面竭力攻城。郭嘉谏曰:“刘备远来救援,先礼后兵,主公当用好言答之,以慢备心;然后进兵攻城,城可破也。”操从其言,款留来使,候发回书。

正商议间,忽流星马飞报祸事。操问其故,报说吕布已袭破兖州,进据濮阳。原来吕布自遭李、郭之乱,逃出武关,去投袁术;术怪吕布反覆不定,拒而不纳。投袁绍,绍纳之,与布共破张燕于常山。布自以为得志,傲慢袁绍手下将士。绍欲杀之。布乃去投张杨,杨纳之。时庞舒在长安城中,私藏吕布妻小,送还吕布。李傕、郭汜知之,遂斩庞舒,写书与张杨,教杀吕布。布因弃张杨去投张邈。恰好张邈弟张超引陈宫来见张邈。宫说邈曰:“今天下分崩,英雄并起;君以千里之众,而反受制于人,不亦鄙乎!今曹操征东,兖州空虚;而吕布乃当世勇士,若与之共取兖州,霸业可图也。”张邈大喜,便令吕布袭破兖州,随据濮阳。止有鄄城、东阿、范县三处,被荀彧、程昱设计死守得全,其余俱破。曹仁屡战,皆不能胜,特此告急。操闻报大惊曰:“兖州有失,使吾无家可归矣,不可不亟图之!”郭嘉曰:“主公正好卖个人情与刘备,退军去复兖州。”操然之,即时答书与刘备,拔寨退兵。

且说来使回徐州,入城见陶谦,呈上书札,言曹兵已退。谦大喜,差人请孔融、田楷、云长、子龙等赴城大会。饮宴既毕,谦延玄德于上座,拱手对众曰:“老夫年迈,二子不才,不堪国家重任。刘公乃帝室之青,德广才高,可领徐州。老夫情愿乞闲养病。”玄德曰:“孔文举令备来救徐州,为义也。今无端据而有之,天下将以备为无义人矣。”糜竺曰:“今汉室陵迟,海宇颠覆,树功立业,正在此时。徐州殷富,户口百万,刘使君领此,不可辞也。”玄德曰:“此事决不敢应命。”陈登曰:“陶府君多病,不能视事,明公勿辞。”玄德曰:“袁公路四世三公,海内所归,近在寿春,何不以州让之?”孔融曰:“袁公路冢中枯骨,何足挂齿!今日之事,天与不取,悔不可追。”玄德坚执不肯。陶谦泣下曰:“君若舍我而去,我死不瞑目矣!”云长曰:“既承陶公相让,兄且权领州事。”张飞曰:“又不是我强要他的州郡;他好意相让,何必苦苦推辞!”玄德曰:“汝等欲陷我于不义耶?”陶谦推让再三,玄德只是不受。陶谦曰:“如玄德必不肯从,此间近邑,名曰小沛,足可屯军,请玄德暂驻军此邑,以保徐州。何如?”众皆劝玄德留小沛,玄德从之。陶谦劳军已毕,赵云辞去,玄德执手挥泪而别。孔融、田楷亦各相别,引军自回。玄德与关、张引本部军来至小沛,修葺城垣,抚谕居民。

却说曹操回军,曹仁接着,言吕布势大,更有陈宫为辅,兖州、濮阳已失,其鄄城、东阿、范县三处,赖荀彧、程昱二人设计相连,死守城郭。操曰:“吾料吕布有勇无谋,不足虑也。”教且安营下寨,再作商议。吕布知曹操回兵,已过滕县,召副将薛兰、李封曰:“吾欲用汝二人久矣。汝可引军一万,坚守兖州。吾亲自率兵,前去破曹。”二人应诺。陈宫急入见曰:“将军弃兖州,欲何往乎?”布曰:“吾欲屯兵濮阳,以成鼎足之势。”宫曰:“差分。薛兰必守兖州不住。——此去正南一百八十里,泰山路险,可伏精兵万人在彼。曹兵闻失兖州,必然倍道而进,待其过半,一击可擒也。”布曰:“吾屯濮阳,别有良谋,汝岂知之!”遂不用陈宫之言,而用薛兰守兖州而行。曹操兵行至泰山险路,郭嘉曰:“且不可进,恐此处有伏兵。”曹操笑曰:“吕布无谋之辈,故教薛兰守兖州,自往濮阳,安得此处有埋伏耶?教曹仁领一军围兖州,吾进兵濮阳,速攻吕布。”陈宫闻曹兵至近,乃献计曰:“今曹兵远来疲困,利在速战,不可养成气力。”布曰:“吾匹马纵横天下,何愁曹操!待其下寨,吾自擒之。”

却说曹操兵近濮阳,下住寨脚。次日,引众将出,陈兵于野。操立马于门旗下,遥望吕布兵到。阵圆处,吕布当先出马,两边排开八员健将:第一个雁门马邑人,姓张,名辽,字文远;第二个泰山华阴人,姓臧,名霸,字宣高。两将又各引三员健将:郝萌、曹性、成廉,魏续、宋宪、侯成。布军五万,鼓声大震。操指吕布而言曰:“吾与汝自来无仇,何得夺吾州郡?”布曰:“汉家城池,诸人有分,偏尔合得?”便叫臧霸出马搦战。曹军内乐进出迎。两马相交,双枪齐举。战到三十余合,胜负不分。夏侯惇拍马便出助战,吕布阵上张辽截住厮杀。恼得吕布性起,挺戟骤马,冲出阵来。夏侯惇、乐进皆走,吕布掩杀,曹军大败,退三四十里。布自收军。

曹操输了一阵,回寨与诸将商议。于禁曰:“某今日上山观望,濮阳之西,吕布有一寨,约无多军。今夜彼将谓我军败走,必不准备,可引兵击之;若得寨,布军必惧:此为上策。”操从其言,带曹洪、李典、毛玠、吕虔、于禁、典韦六将,选马步二万人,连夜从小路进发。

却说吕布于寨中劳军。陈宫曰:“西寨是个要紧去处,倘或曹操袭之,奈何?”布曰:“他今日输了一阵,如何敢来!”宫曰:“曹操是极能用兵之人,须防他攻我不备。”布乃拨高顺并魏续、侯成引兵往守西寨。

却说曹操于黄昏时分,引军至西寨,四面突入。寨兵不能抵挡,四散奔走,曹操夺了寨。将及四更,高顺方引军到,杀将入来。曹操自引军马来迎,正逢高顺,三军混战、将及天明,正西鼓声大震,人报吕布自引救军来了。操弃寨而走。背后高顺、魏续、侯成赶来;当头吕布亲自引军来到。于禁、乐进双战吕布不往。操望北而行。山后一彪军出:左有张辽,右有臧霸。操使吕虔、曹洪战之,不利。操望西而走。忽又喊声大震,一彪军至:郝萌、曹性、成廉、宋宪四将拦住去路。众将死战,操当先冲阵。梆子响处,箭如骤雨射将来。操不能前进,无计可脱,大叫:“谁人救我!”马军队里,一将踊出,乃典韦也,手挺双铁戟,大叫:“主公勿忧!”飞身下马,插住双戟,取短戟十数枝,挟在手中,顾从人曰:“贼来十步乃呼我!”遂放开脚步,冒箭前行。布军数十骑追至。从人大叫曰:“十步矣!”韦曰:“五步乃呼我!”从人又曰:“五步矣!”韦乃飞戟刺之,一戟一人坠马,并无虚发,立杀十数人。众皆奔走。韦复飞身上马,挺一双大铁戟,冲杀入去。郝、曹、成、宋四将不能抵挡,各自逃去。典韦杀散敌军,救出曹操。众将随后也到,寻路归寨。看看天色傍晚,背后喊声起处,吕布骤马提戟赶来,大叫:“操贼休走!”此时人困马乏,大家面面相觑,各欲逃生。正是:虽能暂把重围脱,只怕难当劲敌追。不知曹操性命如何,且听下文分解。