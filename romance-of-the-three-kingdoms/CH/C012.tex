\chapter{陶恭祖三让徐州~曹孟德大战吕布}

曹操正慌走间,正南上一彪军到,乃夏侯惇引军来救援,截住吕布大战。斗到黄昏时分,大雨如注,各自引军分散。操回寨,重赏典韦,加为领军都尉。

却说吕布到寨,与陈宫商议。宫曰:“濮阳城中有富户田氏,家僮千百,为一郡之巨室;可令彼密使人往操寨中下书,言‘吕温侯残暴不仁,民心大怨。今欲移兵黎阳,止有高顺在城内。可连夜进兵,我为内应’。操若来,诱之入城,四门放火,外设伏兵。曹操虽有经天纬地之才,到此安能得脱也?”吕布从其计,密谕田氏使人径到操寨。操因新败,正在踌躇,忽报田氏人到,呈上密书云:“吕布已往黎阳,城中空虚。万望速来,当为内应。城上插白旗,大书‘义’字,便是暗号。”操大喜曰:“天使吾得濮阳也!”重赏来人,一面收拾起兵。刘晔曰:“布虽无谋,陈宫多计。只恐其中有诈,不可不防。明公欲去,当分三军为三队:两队伏城外接应,一队入城,方可。”操从其言,分军三队,来至濮阳城下。

操先往观之,见城上遍竖旗幡,西门角上,有一“义”字白旗,心中暗喜。是日午牌,城门开处,两员将引军出战:前军侯成,后军高顺。操即使典韦出马,直取侯成。侯成抵敌不过,回马望城中走。韦赶到吊桥边,高顺亦拦挡不住,都退入城中去了。数内有军人乘势混过阵来见操,说是田氏之使,呈上密书。约云:“今夜初更时分,城上鸣锣为号,便可进兵。某当献门。”操拨夏侯惇引军在左,曹洪引军在右,自己引夏侯渊、李典、乐进、典韦四将,率兵入城。李典曰:“主公且在城外,容某等先入城去。”操喝曰:“我不自往,谁肯向前!”遂当先领兵直入。

时约初更,月光未上。只听得西门上吹赢壳声,喊声忽起,门上火把燎乱,城门大开,吊桥放落。曹操争先拍马而入。直到州衙,路上不见一人,操知是计,忙拨回马,大叫:“退兵!”州衙中一声炮响,四门烈火,轰天而起;金鼓齐鸣,喊声如江翻海沸。东巷内转出张辽,西巷内转出臧霸,夹攻掩杀。操走北门,道傍转出郝萌、曹性,又杀一阵。操急走南门,高顺、侯成拦住。典韦怒目咬牙,冲杀出去。高顺、侯成倒走出城。典韦杀到吊桥,回头不见了曹操,翻身复杀入城来,门下撞着李典。典韦问:“主公何在?”典曰:“吾亦寻不见。”韦曰:“汝在城外催救军,我入去寻主公。”李典去了。典韦杀入城中,寻觅不见;再杀出城壕边,撞着乐进。进曰:“主公何在?”韦曰:“我往复两遭:寻览不见。”进曰:“同杀入去救主!”两人到门边,城上火炮滚下,乐进马不能入。典韦冒烟突火,又杀入去,到处寻觅。

却说曹操见典韦杀出去了,四下里人马截来,不得出南门;再转北门,火光里正撞见吕布挺戟跃马而来。操以手掩面,加鞭纵马竟过。吕布从后拍马赶来,将戟于操盔上一击,问曰:“曹操何在?”操反指曰:“前面骑黄马者是他。”吕布听说,弃了曹操,纵马向前追赶。曹操拨转马头,望东门而走,正逢典韦。韦拥护曹操,杀条血路,到城门边,火焰甚盛,城上推下柴草,遍地都是火,韦用戟拨开,飞马冒烟突火先出。曹操随后亦出。方到门道边,城门上崩下一条火梁来,正打着曹操战马后胯,那马扑地倒了。操用手托梁推放地上,手臂须发,尽被烧伤。典韦回马来救,恰好夏侯渊亦到。两个同救起曹操,突火而出。操乘渊马,典韦杀条大路而走。直混战到天明,操方回寨。

众将拜伏问安,操仰面笑曰:“误中匹夫之计,吾必当报之!”郭嘉曰:“计可速发。”操曰:“今只将计就计:诈言我被火伤,已经身死。布必引兵来攻。我伏兵于马陵山中,候其兵半渡而击之,布可擒矣。”赢曰:“真良策也!”于是令军士挂孝发丧,诈言操死。早有人来濮阳报吕布,说曹操被火烧伤肢体,到寨身死。布随点起军马,杀奔马陵山来。将到操寨,一声鼓响,伏兵四起。吕布死战得脱,折了好些人马;败回濮阳,坚守不出。

是年蝗虫忽起,食尽禾稻。关东一境,每谷一斛,直钱五十贯,人民相食。曹操因军中粮尽,引兵回鄄城暂住。吕布亦引兵出屯山阳就食。因此二处权且罢兵。

却说陶谦在徐州,时年已六十三岁,忽然染病,看看沉重,请糜竺、陈登议事。竺曰:“曹兵之去,止为吕布袭兖州故也。今因岁荒罢兵,来春又必至矣。府君两番欲让位于刘玄德,时府君尚强健,故玄德不肯受;今病已沉重,正可就此而与之,玄德不肯辞矣。”谦大喜,使人来小沛:请刘玄德商议军务。玄德引关、张带数十骑到徐州,陶谦教请入卧内。玄德问安毕,谦曰:“请玄德公来,不为别事:止因老夫病已危笃,朝夕难保;万望明公可怜汉家城池为重,受取徐州牌印,老夫死亦瞑目矣!”玄德曰:“君有二子,何不传之?”谦曰:“长子商,次子应,其才皆不堪任。老夫死后,犹望明公教诲,切勿令掌州事。”玄德曰:“备一身安能当此大任?”谦曰:“某举一人,可为公辅:系北海人,姓孙,名乾,字公祐。此人可使为从事。”又谓糜竺曰:“刘公当世人杰,汝当善事之。”玄德终是推托,陶谦以手指心而死。众军举哀毕,即捧牌印交送玄德。玄德固辞。次日,徐州百姓,拥挤府前哭拜曰:“刘使君若不领此郡,我等皆不能安生矣!”关、张二公亦再三相劝。玄德乃许权领徐州事;使孙乾、糜竺为辅,陈登为幕官;尽取小沛军马入城,出榜安民;一面安排丧事。玄德与大小军士,尽皆挂孝,大设祭奠祭毕,葬于黄河之原。将陶谦遗表,申奏朝廷。操在鄄城,知陶谦已死,刘玄德领徐州牧,大怒曰:“我仇未报,汝不费半箭之功,坐得徐州!吾必先杀刘备,后戮谦尸,以雪先君之怨!”即传号令,克日起兵去打徐州。荀彧入谏曰:“昔高祖保关中,光武据河内,皆深根固本以制天下,进足以胜敌,退足以坚守,故虽有困,终济大业。明公本首事兖州,且河、济乃天下之要地,是亦昔之关中、河内也。今若取徐州,多留兵则不足用,少留兵则吕布乘虚寇之,是无兖州也。若徐州不得,明公安所归乎?今陶谦虽死,已有刘备守之。徐州之民,既已服备,必助备死战。明公弃兖州而取徐州,是弃大而就小,去本而求末,以安而易危也。愿熟思之。”操曰:“今岁荒乏粮,军士坐守于此,终非良策。”彧曰:“不如东略陈地,使军就食汝南、颍川。黄巾余党何仪、黄劭等,劫掠州郡,多有金帛、粮食、此等贼徒,又容易破;破而取其粮,以养三军,朝廷喜,百姓悦,乃顺天之事也。”

操喜,从之,乃留夏侯惇、曹仁守鄄城等处,自引兵先略陈地,次及汝、颍。黄巾何仪、黄劭知曹兵到,引众来迎,会于羊山。时贼兵虽众,都是狐群狗党,并无队伍行列。操令强弓硬弩射住,令典韦出马。何仪令副元帅出战,不三合,被典韦一戟刺于马下。操引众乘势赶过羊山下寨。次日,黄劭自引军来。阵圆处,一将步行出战,头裹黄巾,身披绿袄,手提铁棒,大叫:“我乃截天夜叉何曼也!谁敢与我厮斗?”曹洪见了,大喝一声,飞身下马,提刀步出。两下向阵前厮杀,四五十合,胜负不分。曹洪诈败而走,何曼赶来。洪用拖刀背砍计,转身一踅,砍中何曼,再复一刀杀死。李典乘势飞马直入贼阵。黄劭不及提备,被李典生擒活捉过来。曹兵掩杀贼众,夺其金帛、粮食无数。何仪势孤,引数百骑奔走葛陂。正行之间,山背后撞出一军。为头一个壮士,身长八尺,腰大十围,手提大刀,截住去路。何仪挺枪出迎,只一合,被那壮士活挟过去。余众着忙,皆下马受缚,被壮士尽驱入葛陂坞中。却说典韦追袭何仪到葛陂,壮士引军迎住。典韦曰:“汝亦黄巾贼耶?”壮士曰:“黄巾数百骑,尽被我擒在坞内!”韦曰:“何不献出?”壮士曰:“你若赢得手中宝刀,我便献出!”韦大怒,挺双戟向前来战。两个从辰至午,不分胜负,各自少歇。不一时,那壮士又出搦战,典韦亦出。直战到黄昏,各因马乏暂止。典韦手下军土,飞报曹操。操大惊,忙引众将来看。次日,壮士又出搦战。操见其人威风凛凛,心中暗喜,分付典韦,今日且诈败。韦领命出战;战到三十合,败走回阵,壮士赶到阵门中,弓弩射回。操急引军退五里,密使人掘下陷坑,暗伏钩手。次日,再令典韦引百余骑出。壮士笑曰:“败将何敢复来!”便纵马接战。典韦略战数合,便回马走。壮士只顾望前赶来,不提防连人带马,都落于陷坑之内,被钩手缚来见曹操。操下帐叱退军士,亲解其缚,急取衣衣之,命坐,问其乡贯姓名。壮士曰:“我乃谯国谯县人也,姓许,名褚,字仲康。向遭寇乱,聚宗族数百人,筑坚壁于坞中以御之。一日寇至,吾令众人多取石子准备,吾亲自飞石击之,无不中者,寇乃退去。又一日寇至,坞中无粮,遂与贼和,约以耕牛换米。米已送到,贼驱牛至坞外,牛皆奔走回还,被我双手掣二牛尾,倒行百余步。贼大惊,不敢取牛而走。因此保守此处无事。”操曰:“吾闻大名久矣,还肯降否?”褚曰:“固所意也。”遂招引宗族数百人俱降。操拜许褚为都尉,赏劳甚厚。随将何仪、黄劭斩讫。汝、颍悉平。

曹操班师,曹仁、夏侯惇接见,言近日细作报说:兖州薛兰、李封军士皆出掳掠,城邑空虚,可引得胜之兵攻之,一鼓可下。操遂引军径奔商州。薛兰、李封出其不意,只得引兵出城迎战。许褚曰:“吾愿取此二人,以为贽见之礼。”操大喜,遂令出战。李封使画戟,向前来迎。交马两合,许褚斩李封于马下。薛兰急走回阵,吊桥边李典拦住。薛兰不敢回城,引军投巨野而去;却被吕虔飞马赶来,一箭射于马下,军皆溃散。曹操复得兖州,程昱便请进兵取濮阳。操令许褚、典韦为先锋,夏侯惇、夏侯渊为左军,李典、乐进为右军,操自领中军,于禁、吕虔为合后。兵至濮阳,吕布欲自将出迎,陈宫谏:“不可出战。待众将聚会后方可。”吕布曰:“吾怕谁来?”遂不听宫言,引兵出阵,横戟大骂。许褚便出。斗二十合,不分胜负。操曰:“吕布非一人可胜。”便差典韦助战,两将夹攻;左边夏侯惇、夏侯渊,右边李典、乐进齐到,六员将共攻吕布。布遮拦不住,拨马回城。城上田氏,见布败回,急令人拽起吊桥。布大叫;“开门!”田氏曰:“吾已降曹将军矣。”布大骂,引军奔定陶而去。陈宫急开东门,保护吕布老小出城。操遂得濮阳,恕田氏旧日之罪。刘晔曰:“吕布乃猛虎也,今日困乏,不可少容。”操令刘晔等守濮阳,自己引军赶至定陶。时吕布与张邈、张超尽在城中,高顺、张辽、臧霸、侯成巡海打粮未回。操军至定陶,连日不战,引军退四十里下寨。正值济郡麦熟。操即令军割麦为食。细作报知吕布,布引军赶来。将近操寨,见左边一望林木茂盛,恐有伏兵而回。操知布军回去,乃谓诸将曰:“布疑林中有伏兵耳,可多插旌旗于林中以疑之。寨西一带长堤,无水,可尽伏精兵。明日吕布必来烧林,堤中军断其后,布可擒矣。”于是止留鼓手五十人于寨中擂鼓;将村中掳来男女在寨内呐喊。精兵多伏堤中。却说吕布回报陈宫。宫曰:“操多诡计,不可轻敌。”布曰:“吾用火攻,可破伏兵。”乃留陈宫、高顺守城。布次日引大军来,遥见林中有旗,驱兵大进,四面放火,竟无一人。欲投寨中,却闻鼓声大震。正自疑惑不定,忽然寨后一彪军出。吕布纵马赶来。炮响处,堤内伏兵尽出:夏侯惇、夏侯渊、许褚、典韦、李典、乐进骤马杀来。吕布料敌不过,落荒而走。从将成廉,被乐进一箭射死。布军三停去了二停,败卒回报陈宫,宫曰:“空城难守,不若急去。”遂与高顺保着吕布老小,弃定陶而走。曹操将得胜之兵,杀入城中,势如劈竹。张超自刎,张邈投袁术去了。山东一境,尽被曹操所得。安民修城,不在话下。

却说吕布正走,逢诸将皆回。陈宫亦已寻着。布曰:“吾军虽少,尚可破曹。”遂再引军来。正是:兵家胜败真常事,卷甲重来未可知。不知吕布胜负如何,且听下文分解。