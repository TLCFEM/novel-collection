\chapter{武乡侯四番用计~南蛮王五次遭擒}

却说孔明自驾小车,引数百骑前来探路。前有一河,名曰西洱河,水势虽慢,并无一只船筏。孔明令伐木为筏而渡,其木到水皆沉。孔明遂问吕凯,凯曰:“闻西洱河上流有一山,其山多竹,大者数围。可令人伐之,于河上搭起竹桥,以渡军马。”孔明即调三万人入山,伐竹数十万根,顺水放下,于河面狭处,搭起竹桥,阔十余丈。乃调大军于河北岸一字儿下寨,便以河为壕堑,以浮桥为门,垒土为城;过桥南岸,一字下三个大营,以待蛮兵。

却说孟获引数十万蛮兵,恨怒而来。将近西洱河,孟获引前部一万刀牌獠丁,直扣前寨搦战。孔明头戴纶巾,身披鹤氅,手执羽扇,乘驷马车,左右众将簇拥而出。孔明见孟获身穿犀皮甲,头顶朱红盔,左手挽牌,右手执刀,骑赤毛牛,口中辱骂;手下万余洞丁,各舞刀牌,往来冲突。孔明急令退回本寨,四面紧闭,不许出战。蛮兵皆裸衣赤身,直到寨门前叫骂。诸将大怒,皆来禀孔明曰:“某等情愿出寨决一死战!”孔明不许。诸将再三欲战,孔明止曰:“蛮方之人,不遵王化,今此一来,狂恶正盛,不可迎也;且宜坚守数日,待其猖獗少懈,吾自有妙计破之。”

于是蜀兵坚守数日。孔明在高阜处探之,窥见蛮兵已多懈怠,乃聚诸将曰:“汝等敢出战否?”众将欣然要出。孔明先唤赵云、魏延入帐,向耳畔低言,分付如此如此。二人受了计策先进。却唤王平、马忠入帐,受计去了。又唤马岱分付曰:“吾今弃此三寨,退过河北;吾军一退,汝可便拆浮桥,移于下流,却渡赵云、魏延军马过河来接应。”岱受计而去。又唤张翼曰:“吾军退去,寨中多设灯火。孟获知之,必来追赶,汝却断其后。”张翼受计而退。孔明只教关索护车。众军退去,寨中多设灯火。蛮兵望见,不敢冲突。

次日平明,孟获引大队蛮兵径到蜀寨之时,只见三个大寨,皆无人马,于内弃下粮草车仗数百余辆。孟优曰:“诸葛弃寨而走,莫非有计否?”孟获曰:“吾料诸葛亮弃辎重而去,必因国中有紧急之事:若非吴侵,定是魏伐。故虚张灯火以为疑兵,弃车仗而去也。可速追之,不可错过。”于是孟获自驱前部,直到西洱河边。望见河北岸上,寨中旗帜整齐如故,灿若云锦;沿河一带,又设锦城。蛮兵哨见,皆不敢进。获谓优曰:“此是诸葛亮惧吾追赶,故就河北岸少住,不二日必走矣。”遂将蛮兵屯于河岸;又使人去山上砍竹为筏,以备渡河;却将敢战之兵,皆移于寨前面。却不知蜀兵早已入自己之境。是日,狂风大起。四壁厢火明鼓响,蜀兵杀到。蛮兵獠丁,自相冲突,孟获大惊,急引宗族洞丁杀开条路,径奔旧寨。忽一彪军从寨中杀出,乃是赵云。获慌忙回西洱河,望山僻处而走。又一彪军杀出,乃是马岱。孟获只剩得数十个败残兵,望山谷中而逃。见南、北、西三处尘头火光,因此不敢前进,只得望东奔走,方才转过山口,见一大林之前,数十从人,引一辆小车;车上端坐孔明,呵呵大笑曰:“蛮王孟获!天败至此,吾已等候多时也!”获大怒,回顾左右曰:“吾遭此人诡计!受辱三次;今幸得这里相遇。汝等奋力前去,连人带车砍为粉碎!”数骑蛮兵,猛力向前。孟获当先呐喊,抢到大林之前,趷踏一声,踏了陷坑,一齐塌倒。大林之内,转出魏延,引数百军来,一个个拖出,用索缚定。孔明先到寨中,招安蛮兵,并诸甸酋长洞丁——此时大半皆归本乡去了——除死伤外,其余尽皆归降。孔明以酒肉相待,以好言抚慰,尽令放回。蛮兵皆感叹而去。少顷,张翼解孟优至。孔明诲之曰:“汝兄愚迷,汝当谏之。今被吾擒了四番,有何面目再见人耶!”孟优羞惭满面。伏地告求免死。孔明曰:“吾杀汝不在今日。吾且饶汝性命,劝谕汝兄。”令武士解其绳索,放起孟优。优泣拜而去。不一时,魏延解孟获至。孔明大怒曰:“你今番又被吾擒了,有何理说!”获曰:“吾今误中诡计,死不瞑目!”孔明叱武士推出斩之。获全无惧色,回顾孔明曰:“若敢再放吾回去,必然报四番之恨!”孔明大笑,令左右去其缚,赐酒压惊,就坐于帐中。孔明问曰:“吾今四次以礼相待,汝尚然不服,何也?”获曰:“吾虽是化外之人,不似丞相专施诡计,吾如何肯服?”孔明曰:“吾再放汝回去,复能战乎?”获曰:“丞相若再拿住吾,吾那时倾心降服,尽献本洞之物犒军,誓不反乱。”孔明即笑而遣之。获忻然拜谢而去。于是聚得诸洞壮丁数千人,望南迤逦而行。早望见尘头起处,一队兵到;乃是兄弟孟优,重整残兵,来与兄报仇。兄弟二人,抱头相哭,诉说前事。优曰:“我兵屡败,蜀兵屡胜,难以抵当。只可就山阴洞中,退避不出。蜀兵受不过暑气,自然退矣。”获问曰:“何处可避?”优曰:“此去西南有一洞,名曰秃龙洞。洞主朵思大王,与弟甚厚,可投之。”于是孟获先教孟优到秃龙洞,见了朵思大王。朵思慌引洞兵出迎,孟获入洞,礼毕,诉说前事。朵思曰:“大王宽心。若蜀兵到来,令他一人一骑不得还乡,与诸葛亮皆死于此处!”获大喜,问计于朵思。朵思曰:“此洞中止有两条路:东北上一路,就是大王所来之路,地势平坦,土厚水甜,人马可行;若以木石垒断洞口,虽有百万之众,不能进也。西北上有一条路,山险岭恶,道路窄狭;其中虽有小路,多藏毒蛇恶蝎;黄昏时分,烟瘴大起,直至已,午时方收,惟未、申、酉三时,可以往来;水不可饮,人马难行。此处更有四个毒泉:一名哑泉,其水颇甜,人若饮之,则不能言,不过旬日必死;二曰灭泉,此水与汤无异,人若沐浴,则皮肉皆烂,见骨必死;三曰黑泉,其水微清,人若溅之在身,则手足皆黑而死;四曰柔泉,其水如冰,人若饮之,咽喉无暖气,身躯软弱如绵而死。此处虫鸟皆无,惟有汉伏波将军曾到;自此以后,更无一人到此。今垒断东北大路,令大王稳居敝洞,若蜀兵见东路截断,必从西路而入;于路无水,若见此四泉,定然饮水,虽百万之众,皆无归矣。何用刀兵耶!”孟获大喜,以手加额曰:“今日方有容身之地!”又望北指曰:“任诸葛神机妙算,难以施设!四泉之水,足以报败兵之恨也!”自此,孟获、孟优终日与朵思大王筵宴。

却说孔明连日不见孟获兵出,遂传号令教大军离西洱河,望南进发。此时正当六月炎天,其热如火。有后人咏南方苦热诗曰:“山泽欲焦枯,火光覆太虚。不知天地外,暑气更何如!”又有诗曰:“赤帝施权柄,阴云不敢生。云蒸孤鹤喘,海热巨鳌惊。忍舍溪边坐?慵抛竹里行。如何沙塞客,擐甲复长征!”孔明统领大军,正行之际,忽哨马飞报:“孟获退往秃龙洞中不出,将洞口要路垒断,内有兵把守;山恶岭峻,不能前进。”孔明请吕凯问之,凯曰:“某曾闻此洞有条路,实不知详细。”蒋琬曰:“孟获四次遭擒,既已丧胆,安敢再出?况今天气炎热,军马疲乏,征之无益;不如班师回国。”孔明曰:“若如此,正中孟获之计也。吾军一退,彼必乘势追之。今已到此,安有复回之理!”遂令王平领数百军为前部;却教新降蛮兵引路,寻西北小径而入。前到一泉,人马皆渴,争饮此水。王平探有此路,回报孔明。比及到大寨之时,皆不能言,但指口而已。孔明大惊,知是中毒,遂自驾小车,引数十人前来看时,见一潭清水,深不见底,水气凛凛,军不敢试。孔明下车,登高望之,四壁峰岭,鸟雀不闻,心中大疑。忽望见远远山冈之上,有一古庙。孔明攀藤附葛而到,见一石屋之中,塑一将军端坐,旁有石碑,乃汉伏波将军马援之庙:因平蛮到此,土人立庙祀之。孔明再拜曰:“亮受先帝托孤之重,今承圣旨,到此平蛮;欲待蛮方既平,然后伐魏吞吴,重安汉室。今军士不识地理,误饮毒水,不能出声。万望尊神,念本朝恩义,通灵显圣,护佑三军!”祈祷已毕,出庙寻土人问之。隐隐望见对山一老叟扶杖而来,形容甚异。孔明请老叟入庙,礼毕,对坐于石上。孔明问曰:“丈者高姓?”老叟曰:“老夫久闻大国丞相隆名,幸得拜见。蛮方之人,多蒙丞相活命,皆感恩不浅。”孔明问泉水之故,老叟答曰:“军所饮水,乃哑泉之水也,饮之难言,数日而死。此泉之外,又有三泉:东南有一泉,其水至冷,人若饮水,咽喉无暖气,身躯软弱而死,名曰柔泉;正南有一泉,人若溅之在身,手足皆黑而死,名曰黑泉;西南有一泉,沸如热汤,人若浴之,皮肉尽脱而死,名曰灭泉。敝处有此四泉,毒气所聚,无药可治,又烟瘴甚起,惟未、申、酉三个时辰可往来;余者时辰,皆瘴气密布,触之即死。”

孔明曰:“如此则蛮方不可平矣。蛮方不平,安能并吞吴、魏,再兴汉室?有负先帝托孤之重,生不如死也!”老叟曰:“丞相勿忧。老夫指引一处,可以解之。”孔明曰:“老丈有何高见,望乞指教。”老叟曰:“此去正西数里,有一山谷,入内行二十里,有一溪名曰万安溪。上有一高士,号为万安隐者;此人不出溪有数十余年矣。其草庵后有一泉,名安乐泉。人若中毒,汲其水饮之即愈。有人或生疥癞,或感瘴气,于万安溪内浴之,自然无事,更兼庵前有一等草,名曰薤叶芸香。人若口含一叶,则瘴气不染。丞相可速往求之。”孔明拜谢,问曰:“承丈者如此活命之德,感刻不胜。愿闻高姓。”老叟入庙曰:“吾乃本处山神,奉伏波将军之命,特来指引。”言讫、喝开庙后石壁而入。孔明惊讶不已,再拜庙神,寻旧路上车,回到大寨。次日,孔明备信香、礼物,引王平及众哑军,连夜望山神所言去处,迤逦而进。入山谷小径,约行二十余里,但见长松大柏,茂竹奇花,环绕一庄;篱落之中,有数间茅屋,闻得馨香喷鼻。孔明大喜,到庄前扣户,有一小童出。孔明方欲通姓名,早有一人,竹冠草履,白袍皂绦,碧眼黄发,忻然出曰:“来者莫非汉丞相否?”孔明笑曰:“高士何以知之?”隐者曰:“久闻丞相大纛南征,安得不知!”遂邀孔明入草堂。礼毕,分宾主坐定。孔明告曰:“亮受昭烈皇帝托孤之重,今承嗣君圣旨,领大军至此,欲服蛮邦,使归王化。不期孟获潜入洞中,军士误饮哑泉之水。夜来蒙伏波将军显圣,言高士有药泉,可以治之。望乞矜念,赐神水以救众兵残生。”隐者曰:“量老夫山野废人,何劳丞相枉驾。此泉就在庵后。”教取来饮。于是童子引王平等一起哑军,来到溪边,汲水饮之;随即吐出恶涎,便能言语。童子又引众军到万安溪中沐浴。

隐者于庵中进柏子茶、松花菜,以待孔明。隐者告曰:“此间蛮洞多毒蛇恶蝎,柳花飘入溪泉之间,水不可饮;但掘地为泉,汲水饮之方可。”孔明求薤叶芸香,隐者令众军尽意采取:“各人口含一叶,自然瘴气不侵。”孔明拜求隐者姓名,隐者笑曰:“某乃孟获之兄孟节是也。”孔明愕然。隐者又曰:“丞相休疑,容伸片言:某一父母所生三人:长即老夫孟节,次孟获,又次孟优。父母皆亡。二弟强恶,不归王化。某屡谏不从,故更名改姓,隐居于此。今辱弟造反,又劳丞相深入不毛之地,如此生受,孟节合该万死,故先于丞相之前请罪。”孔明叹曰:“方信盗跖、下惠之事,今亦有之。”遂与孟节曰:“吾申奏天子,立公为王,可乎?”节曰:“为嫌功名而逃于此,岂复有贪富贵之意!”孔明乃具金帛赠之。孟节坚辞不受。孔明嗟叹不已,拜别而回。后人有诗曰:“高士幽栖独闭关,武侯曾此破诸蛮。至今古木无人境,犹有寒烟锁旧山。”

孔明回到大寨之中,令军士掘地取水。掘下二十余丈,并无滴水;凡掘十余处,皆是如此。军心惊慌。孔明夜半焚香告天曰:“臣亮不才,仰承大汉之福,受命平蛮。今途中乏水,军马枯渴。倘上天不绝大汉,即赐甘泉!若气运已终,臣亮等愿死于此处!”是夜祝罢,平明视之,皆得满井甘泉。后人有诗曰:“为国平蛮统大兵,心存正道合神明。耿恭拜井甘泉出,诸葛虔诚水夜生。”孔明军马既得甘泉,遂安然由小径直入秃龙洞前下寨。蛮兵探知,来报孟获曰:“蜀兵不染瘴疫之气,又无枯渴之患,诸泉皆不应。”朵思大王闻知不信,自与孟获来高山望之。只见蜀兵安然无事,大桶小担,搬运水浆,饮马造饭。朵思见之,毛发耸然,回顾孟获曰:“此乃神兵也!”获曰:“吾兄弟二人与蜀兵决一死战,就殒于军前,安肯束手受缚!”朵思曰:“若大王兵败,吾妻子亦休矣。当杀牛宰马,大赏洞丁,不避水火,直冲蜀寨,方可得胜。”于是大赏蛮兵。

正欲起程,忽报洞后迤西银冶洞二十一洞主杨锋引三万兵来助战。孟获大喜曰:“邻兵助我,我必胜矣!”即与朵思大王出洞迎接。杨锋引兵入曰:“吾有精兵三万,皆披铁甲,能飞山越岭,足以敌蜀兵百万;我有五子,皆武艺足备。愿助大王。”锋令五子入拜,皆彪躯虎体,威风抖擞。孟获大喜,遂设席相待杨锋父子。酒至半酣,锋曰:“军中少乐,吾随军有蛮姑,善舞刀牌,以助一笑。”获忻然从之。须臾,数十蛮姑,皆披发跣足,从帐外舞跳而入,群蛮拍手以歌和之。杨锋令二子把盏。二子举杯诣孟获、孟优前。二人接杯,方欲饮酒,锋大喝一声,二子早将孟获、孟优执下座来。朵思大王却待要走,已被杨锋擒了。蛮姑横截于帐上,谁敢近前。获曰:“免死狐悲,物伤其类。吾与汝皆是各洞之主,往日无冤,何故害我?”锋曰:“吾兄弟子侄皆感诸葛丞相活命之恩,无可以报。今汝反叛,何不擒献!”

于是各洞蛮兵,皆走回本乡。杨锋将孟获、孟优、朵思等解赴孔明寨来。孔明令入,杨锋等拜于帐下曰:“某等子侄皆感丞相恩德,故擒孟获、孟优等呈献。”孔明重赏之,令驱孟获入。孔明笑曰:“汝今番心服乎?”获曰:“非汝之能,乃吾洞中之人,自相残害,以致如此。要杀便杀,只是不服!”孔明曰:“汝赚吾入无水之地,更以哑泉、灭泉、黑泉、柔泉如此之毒,吾军无恙,岂非天意乎?汝何如此执迷?”获又曰:“吾祖居银坑山中,有三江之险,重关之固。汝若就彼擒之,吾当子子孙孙,倾心服事。”孔明曰:“吾再放汝回去,重整兵马,与吾共决胜负;如那时擒住,汝再不服,当灭九族。”叱左右去其缚,放起孟获。获再拜而去。孔明又将孟优并朵思大王皆释其缚,赐酒食压惊。二人悚惧,不敢正视。孔明令鞍马送回。正是:深临险地非容易,更展奇谋岂偶然!未知孟获整兵再来,胜负如何,且看下文分解。