\chapter{宴桃园豪杰三结义~斩黄巾英雄首立功}

滚滚长江东逝水,浪花淘尽英雄。是非成败转头空。
青山依旧在,几度夕阳红。
白发渔樵江渚上,惯看秋月春风。一壶浊酒喜相逢。
古今多少事,都付笑谈中。
——调寄《临江仙》

话说天下大势,分久必合,合久必分。周末七国分争,并入于秦。及秦灭之后,楚、汉
分争,又并入于汉。汉朝自高祖斩白蛇而起义,一统天下,后来光武中兴,传至献帝,遂分
为三国。推其致乱之由,殆始于桓、灵二帝。桓帝禁锢善类,崇信宦官。及桓帝崩,灵帝即
位,大将军窦武、太傅陈蕃共相辅佐。时有宦官曹节等弄权,窦武、陈蕃谋诛之,机事不
密,反为所害,中涓自此愈横。

建宁二年四月望日,帝御温德殿。方升座,殿角狂风骤起。只见一条大青蛇,从梁上飞
将下来,蟠于椅上。帝惊倒,左右急救入宫,百官俱奔避。须臾,蛇不见了。忽然大雷大
雨,加以冰雹,落到半夜方止,坏却房屋无数。建宁四年二月,洛阳地震;又海水泛溢,沿
海居民,尽被大浪卷入海中。光和元年,雌鸡化雄。六月朔,黑气十余丈,飞入温德殿中。
秋七月,有虹现于玉堂;五原山岸,尽皆崩裂。种种不祥,非止一端。帝下诏问群臣以灾异
之由,议郎蔡邕上疏,以为蜺堕鸡化,乃妇寺干政之所致,言颇切直。帝览奏叹息,因起更
衣。曹节在后窃视,悉宣告左右;遂以他事陷邕于罪,放归田里。后张让、赵忠、封谞、段
珪、曹节、侯览、蹇硕、程旷、夏恽、郭胜十人朋比为奸,号为“十常侍”。帝尊信张让,
呼为“阿父”。朝政日非,以致天下人心思乱,盗贼蜂起。

时巨鹿郡有兄弟三人,一名张角,一名张宝,一名张梁。那张角本是个不第秀才,因入
山采药,遇一老人,碧眼童颜,手执藜杖,唤角至一洞中,以天书三卷授之,曰:“此名
《太平要术》,汝得之,当代天宣化,普救世人;若萌异心,必获恶报。”角拜问姓名。老
人曰:“吾乃南华老仙也。”言讫,化阵清风而去。角得此书,晓夜攻习,能呼风唤雨,号
为“太平道人”。中平元年正月内,疫气流行,张角散施符水,为人治病,自称“大贤良
师”。角有徒弟五百余人,云游四方,皆能书符念咒。次后徒众日多,角乃立三十六方,大
方万余人,小方六七千,各立渠帅,称为将军;讹言:“苍天已死,黄天当立;岁在甲子,
天下大吉。”令人各以白土书“甲子”二字于家中大门上。青、幽、徐、冀、荆、扬、兖、
豫八州之人,家家侍奉大贤良师张角名字。角遣其党马元义,暗赍金帛,结交中涓封谞,以
为内应。角与二弟商议曰:“至难得者,民心也。今民心已顺,若不乘势取天下,诚为可
惜。”遂一面私造黄旗,约期举事;一面使弟子唐周,驰书报封谞。唐周乃径赴省中告变。
帝召大将军何进调兵擒马元义,斩之;次收封谞等一干人下狱。张角闻知事露,星夜举兵,
自称“天公将军”,张宝称“地公将军”,张梁称“人公将军”。申言于众曰:“今汉运将
终,大圣人出。汝等皆宜顺天从正,以乐太平。”四方百姓,裹黄巾从张角反者四五十万。
贼势浩大,官军望风而靡。何进奏帝火速降诏,令各处备御,讨贼立功。一面遣中郎将卢
植、皇甫嵩、朱儁,各引精兵、分三路讨之。

且说张角一军,前犯幽州界分。幽州太守刘焉,乃江夏竟陵人氏,汉鲁恭王之后也。当
时闻得贼兵将至,召校尉邹靖计议。靖曰:“贼兵众,我兵寡,明公宜作速招军应敌。”刘
焉然其说,随即出榜招募义兵。

榜文行到涿县,引出涿县中一个英雄。那人不甚好读书;性宽和,寡言语,喜怒不形于
色;素有大志,专好结交天下豪杰;生得身长七尺五寸,两耳垂肩,双手过膝,目能自顾其
耳,面如冠玉,唇若涂脂;中山靖王刘胜之后,汉景帝阁下玄孙,姓刘名备,字玄德。昔刘
胜之子刘贞,汉武时封涿鹿亭侯,后坐酎金失侯,因此遗这一枝在涿县。玄德祖刘雄,父刘
弘。弘曾举孝廉,亦尝作吏,早丧。玄德幼孤,事母至孝;家贫,贩屦织席为业。家住本县
楼桑村。其家之东南,有一大桑树,高五丈余,遥望之,童童如车盖。相者云:“此家必出
贵人。”玄德幼时,与乡中小儿戏于树下,曰:“我为天子,当乘此车盖。”叔父刘元起奇
其言,曰:“此儿非常人也!”因见玄德家贫,常资给之。年十五岁,母使游学,尝师事郑
玄、卢植,与公孙瓒等为友。

及刘焉发榜招军时,玄德年已二十八岁矣。当日见了榜文,慨然长叹。随后一人厉声言
曰:“大丈夫不与国家出力,何故长叹?”玄德回视其人,身长八尺,豹头环眼,燕颔虎
须,声若巨雷,势如奔马。玄德见他形貌异常,问其姓名。其人曰:“某姓张名飞,字翼
德。世居涿郡,颇有庄田,卖酒屠猪,专好结交天下豪杰。恰才见公看榜而叹,故此相
问。”玄德曰:“我本汉室宗亲,姓刘,名备。今闻黄巾倡乱,有志欲破贼安民,恨力不
能,故长叹耳。”飞曰:“吾颇有资财,当招募乡勇,与公同举大事,如何。”玄德甚喜,
遂与同入村店中饮酒。

正饮间,见一大汉,推着一辆车子,到店门首歇了,入店坐下,便唤酒保:“快斟酒来
吃,我待赶入城去投军。”玄德看其人:身长九尺,髯长二尺;面如重枣,唇若涂脂;丹凤
眼,卧蚕眉,相貌堂堂,威风凛凛。玄德就邀他同坐,叩其姓名。其人曰:“吾姓关名羽,
字长生,后改云长,河东解良人也。因本处势豪倚势凌人,被吾杀了,逃难江湖,五六年
矣。今闻此处招军破贼,特来应募。”玄德遂以己志告之,云长大喜。同到张飞庄上,共议
大事。飞曰:“吾庄后有一桃园,花开正盛;明日当于园中祭告天地,我三人结为兄弟,协
力同心,然后可图大事。”玄德、云长齐声应曰:“如此甚好。”

次日,于桃园中,备下乌牛白马祭礼等项,三人焚香再拜而说誓曰:“念刘备、关羽、
张飞,虽然异姓,既结为兄弟,则同心协力,救困扶危;上报国家,下安黎庶。不求同年同
月同日生,只愿同年同月同日死。皇天后土,实鉴此心,背义忘恩,天人共戮!”誓毕,拜
玄德为兄,关羽次之,张飞为弟。祭罢天地,复宰牛设酒,聚乡中勇士,得三百余人,就桃
园中痛饮一醉。来日收拾军器,但恨无马匹可乘。正思虑间,人报有两个客人,引一伙伴
当,赶一群马,投庄上来。玄德曰:“此天佑我也!”三人出庄迎接。原来二客乃中山大
商:一名张世平,一名苏双,每年往北贩马,近因寇发而回。玄德请二人到庄,置酒管待,
诉说欲讨贼安民之意。二客大喜,愿将良马五十匹相送;又赠金银五百两,镔铁一千斤,以
资器用。

玄德谢别二客,便命良匠打造双股剑。云长造青龙偃月刀,又名“冷艳锯”,重八十二
斤。张飞造丈八点钢矛。各置全身铠甲。共聚乡勇五百余人,来见邹靖。邹靖引见太守刘
焉。三人参见毕,各通姓名。玄德说起宗派,刘焉大喜,遂认玄德为侄。不数日,人报黄巾
贼将程远志统兵五万来犯涿郡。刘焉令邹靖引玄德等三人,统兵五百,前去破敌。玄德等欣
然领军前进,直至大兴山下,与贼相见。贼众皆披发,以黄巾抹额。当下两军相对,玄德出
马,左有云长,右有翼德,扬鞭大骂:“反国逆贼,何不早降!”程远志大怒,遣副将邓茂
出战。张飞挺丈八蛇矛直出,手起处,刺中邓茂心窝,翻身落马。程远志见折了邓茂,拍马
舞刀,直取张飞。云长舞动大刀,纵马飞迎。程远志见了,早吃一惊,措手不及,被云长刀
起处,挥为两段。后人有诗赞二人曰:英雄露颖在今朝,一试矛兮一试刀。初出便将威力
展,三分好把姓名标。

众贼见程远志被斩,皆倒戈而走。玄德挥军追赶,投降者不计其数,大胜而回。刘焉亲
自迎接,赏劳军士。次日,接得青州太守龚景牒文,言黄巾贼围城将陷,乞赐救援。刘焉与
玄德商议。玄德曰:“备愿往救之。”刘焉令邹靖将兵五千,同玄德、关、张,投青州来。
贼众见救军至,分兵混战。玄德兵寡不胜,退三十里下寨。

玄德谓关、张曰:“贼众我寡;必出奇兵,方可取胜。”乃分关公引一千军伏山左,张
飞引一千军伏山右,鸣金为号,齐出接应。次日,玄德与邹靖引军鼓噪而进。贼众迎战,玄
德引军便退。贼众乘势追赶,方过山岭,玄德军中一齐鸣金,左右两军齐出,玄德摩军回身
复杀。三路夹攻,贼众大溃。直赶至青州城下,太守龚景亦率民兵出城助战。贼势大败,剿
戮极多,遂解青州之围。后人有诗赞玄德曰:运筹决算有神功,二虎还须逊一龙。初出便能
垂伟绩,自应分鼎在孤穷。

龚景犒军毕,邹靖欲回。玄德曰:“近闻中郎将卢植与贼首张角战于广宗,备昔曾师事
卢植,欲往助之。”于是邹靖引军自回,玄德与关、张引本部五百人投广宗来。至卢植军
中,入帐施礼,具道来意。卢植大喜,留在帐前听调。

时张角贼众十五万,植兵五万,相拒于广宗,未见胜负。植谓玄德曰:“我今围贼在
此,贼弟张梁、张宝在颍川,与皇甫嵩、朱儁对垒。汝可引本部人马,我更助汝一千官军,
前去颍川打探消息,约期剿捕。”玄德领命,引军星夜投颍川来。

时皇甫嵩、朱儁领军拒贼,贼战不利,退入长社,依草结营。嵩与儁计曰:“贼依草结
营,当用火攻之。”遂令军士,每人束草一把,暗地埋伏。其夜大风忽起。二更以后,一齐
纵火,嵩与儁各引兵攻击贼寨,火焰张天,贼众惊慌,马不及鞍,人不及甲,四散奔走。

杀到天明,张梁、张宝引败残军士,夺路而走。忽见一彪军马,尽打红旗,当头来到,
截住去路。为首闪出一将,身长七尺,细眼长髯,官拜骑都尉,沛国谯郡人也,姓曹名操字
孟德。操父曹嵩,本姓夏侯氏,因为中常侍曹腾之养子,故冒姓曹。曹嵩生操,小字阿瞒,
一名吉利。操幼时,好游猎,喜歌舞,有权谋,多机变。操有叔父,见操游荡无度,尝怒
之,言于曹嵩。嵩责操。操忽心生一计,见叔父来,诈倒于地,作中风之状。叔父惊告嵩,
嵩急视之。操故无恙。嵩曰:“叔言汝中风,今已愈乎?”操曰:“儿自来无此病;因失爱
于叔父,故见罔耳。”嵩信其言。后叔父但言操过,嵩并不听。因此,操得恣意放荡。时人
有桥玄者,谓操曰:“天下将乱,非命世之才不能济。能安之者,其在君乎?”南阳何顒见
操,言:“汉室将亡,安天下者,必此人也。”汝南许劭,有知人之名。操往见之,问曰:
“我何如人?”劭不答。又问,劭曰:“子治世之能臣,乱世之奸雄也。”操闻言大喜。年
二十,举孝廉,为郎,除洛阳北部尉。初到任,即设五色棒十余条于县之四门,有犯禁者,
不避豪贵,皆责之。中常侍蹇硕之叔,提刀夜行,操巡夜拿住,就棒责之。由是,内外莫敢
犯者,威名颇震。后为顿丘令,因黄巾起,拜为骑都尉,引马步军五千,前来颍川助战。正
值张梁、张宝败走,曹操拦住,大杀一阵,斩首万余级,夺得旗幡、金鼓、马匹极多。张
梁、张宝死战得脱。操见过皇甫嵩、朱儁,随即引兵追袭张梁、张宝去了。

却说玄德引关、张来颍川,听得喊杀之声,又望见火光烛天,急引兵来时,贼已败散。
玄德见皇甫嵩、朱儁,具道卢植之意。嵩曰:“张梁、张宝势穷力乏,必投广宗去依张角。
玄德可即星夜往助。”玄德领命,遂引兵复回。到得半路,只见一簇军马,护送一辆槛车,
车中之囚,乃卢植也。玄德大惊,滚鞍下马,问其缘故。植曰:“我围张角,将次可破;因
角用妖术,未能即胜。朝廷差黄门左丰前来体探,问我索取贿赂。我答曰:‘军粮尚缺,安
有余钱奉承天使?’左丰挟恨,回奏朝廷,说我高垒不战,惰慢军心;因此朝廷震怒,遣中
郎将董卓来代将我兵,取我回京问罪。”张飞听罢,大怒,要斩护送军人,以救卢植。玄德
急止之曰:“朝廷自有公论,汝岂可造次?”军士簇拥卢植去了。关公曰:“卢中郎已被
逮,别人领兵,我等去无所依,不如且回涿郡。”玄德从其言,遂引军北行。行无二日,忽
闻山后喊声大震。玄德引关、张纵马上高冈望之,见汉军大败,后面漫山塞野,黄巾盖地而
来,旗上大书“天公将军”。玄德曰:“此张角也!可速战!”三人飞马引军而出。张角正
杀败董卓,乘势赴来,忽遇三人冲杀,角军大乱,败走五十余里。

三人救了董卓回寨。卓问三人现居何职。玄德曰:“白身。”卓甚轻之,不为礼。玄德
出,张飞大怒曰:“我等亲赴血战,救了这厮,他却如此无礼。若不杀之,难消我气!”便
要提刀入帐来杀董卓。正是:人情势利古犹今,谁识英雄是白身?安得快人如翼德,尽诛世
上负心人!毕竟董卓性命如何,且听下文分解。