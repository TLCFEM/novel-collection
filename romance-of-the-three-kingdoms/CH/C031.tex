\chapter{曹操仓亭破本初~玄德荆州依刘表}

却说曹操乘袁绍之败,整顿军马,迤逦追袭。袁绍幅巾单衣,引八百余骑,奔至黎阳北岸,大将蒋义渠出寨迎接。绍以前事诉与义渠。义渠乃招谕离散之众,众闻绍在,又皆蚁聚。军势复振,议还冀州。军行之次,夜宿荒山。绍于帐中闻远远有哭声,遂私往听之。却是败军相聚,诉说丧兄失弟,弃伴亡亲之苦,各各捶胸大哭,皆曰:“若听田丰之言,我等怎遭此祸!”绍大悔曰:“吾不听田丰之言,兵败将亡;今回去,有何面目见之耶!”次日,上马正行间,逢纪引军来接。绍对逢纪曰:“吾不听田丰之言,致有此败。吾今归去,羞见此人。”逢纪因谮曰:“丰在狱中闻主公兵败,抚掌大笑曰:果不出吾之料!”袁绍大怒曰:“竖儒怎敢笑我!我必杀之!”遂命使者赍宝剑先往冀州狱中杀田丰。

却说田丰在狱中。一日,狱吏来见丰曰:“与别驾贺喜!”丰曰:“何喜可贺?”狱吏曰:“袁将军大败而回,君必见重矣。”丰笑曰:“吾今死矣!”狱吏问曰:“人皆为君喜,君何言死也?”丰曰:“袁将军外宽而内忌,不念忠诚。若胜而喜,犹能赦我;今战败则羞,吾不望生矣。”狱吏未信。忽使者赍剑至,传袁绍命,欲取田丰之首,狱吏方惊。丰曰:“吾固知必死也。”狱吏皆流泪。丰曰:“大丈夫生于天地间,不识其主而事之,是无智也!今日受死,夫何足惜!”乃自刎于狱中。后人有诗曰:“昨朝沮授军中失,今日田丰狱内亡。河北栋梁皆折断,本初焉不丧家邦!”田丰既死,闻者皆为叹惜。

袁绍回冀州,心烦意乱,不理政事。其妻刘氏劝立后嗣。绍所生三子长子袁谭字显思,出守青州;次子袁熙字显奕,出守幽州;三子袁尚字显甫,是绍后妻刘氏所出,生得形貌俊伟,绍至爱之,因此留在身边。自官渡兵败之后,刘氏劝立尚为后嗣,绍乃与审配、逢纪、辛评、郭图四人商议、原来审、逢二人,向辅袁尚;辛、郭二人,向辅袁谭;四人各为其主。当下袁绍谓四人曰:“今外患未息,内事不可不早定,吾将议立后嗣:长子谭,为人性刚好杀;次子熙,为人柔懦难成;三子尚,有英雄之表,礼贤敬士,吾欲立之。公等之意若何?”郭图曰:“三子之中,谭为长,今又居外;主公若废长立幼,此乱萌也。今军威稍挫,敌兵压境,岂可复使父子兄弟自相争乱耶?主公且理会拒敌之策,立嗣之事,毋容多议。”袁绍踌躇未决。忽报袁熙引兵六万,自幽州来;袁谭引兵五万,自青州来;外甥高干亦引兵五万,自并州来:各至冀州助战。绍喜,再整人马来战曹操。时操引得胜之兵,陈列于河上,有土人箪食壶浆以迎之。操见父老数人,须发尽白,乃命入帐中赐坐,问之曰:“老丈多少年纪?”答曰:“欲近百岁矣。”操曰:“吾军士惊扰汝乡,吾甚不安。”父老曰:“桓帝时,有黄星见于楚、宋之分,辽东人殷馗善晓天文,夜宿于此,对老汉等言:黄星见于乾象,正照此间。后五十年,当有真人起于梁沛之间。今以年计之,整整五十年。袁本初重敛于民,民皆怨之。丞相兴仁义之兵,吊民伐罪,官渡一战,破袁绍百万之众,正应当时殷馗之言,兆民可望太平矣。”操笑曰:“何敢当老丈所言?”遂取酒食绢帛赐老人而遣之。号令三军:“如有下乡杀人家鸡犬者,如杀人之罪!”于是军民震服。操亦心中暗喜。人报袁绍聚四州之兵,得二三十万,前至仓亭下寨。操提兵前进,下寨已定。次日,两军相对,各布成阵势。操引诸将出阵,绍亦引三子一甥及文官武将出到阵前。操曰:“本初计穷力尽,何尚不思投降?直待刀临项上,悔无及矣!”绍大怒,回顾众将曰:“谁敢出马?”袁尚欲于父前逞能,便舞双刀,飞马出阵,来往奔驰。操指问众将曰:“此何人?”有识者答曰:“此袁绍三子袁尚也。”言未毕,一将挺枪早出。操视之,乃徐晃部将史涣也。两骑相交,不三合,尚拨马刺斜而走。史涣赶来,袁尚拈弓搭箭,翻身背射,正中史涣左目,坠马而死。袁绍见子得胜,挥鞭一指,大队人马拥将过来,混战大杀一场,各鸣金收军还寨。

操与诸将商议破绍之策。程昱献十面埋伏之计,劝操退军于河上,伏兵十队,诱绍追至河上,“我军无退路,必将死战,可胜绍矣。”操然其计。左右各分五队。左:一队夏侯惇,二队张辽,三队李典,四队乐进,五队夏侯渊;右:一队曹洪,二队张郃,三队徐晃,四队于禁,五队高览。中军许褚为先锋。次日,十队先进,埋伏左右已定。至半夜,操令许褚引兵前进,伪作劫寨之势。袁绍五寨人马,一齐俱起。许褚回军便走。袁绍引军赶来,喊声不绝;比及天明,赶至河上。曹军无去路,操大呼曰:“前无去路,诸军何不死战?”众军回身奋力向前。许褚飞马当先,力斩十数将。袁军大乱。袁绍退军急回,背后曹军赶来。正行间:一声鼓响,左边夏侯渊,右边高览,两军冲出。袁绍聚三子一甥,死冲血路奔走。又行不到十里,左边乐进,右边于禁杀出,杀得袁军尸横遍野,血流成渠。又行不到数里,左边李典,右边徐晃,两军截杀一阵。袁绍父子胆丧心惊,奔入旧寨。令三军造饭,方欲待食,左边张辽,右边张郃,径来冲寨。绍慌上马,前奔仓亭。人马困乏,欲待歇息,后面曹操大军赶来,袁绍舍命而走。正行之间,右边曹洪,左边夏侯惇,挡住去路。绍大呼曰:“若不决死战,必为所擒矣!”奋力冲突,得脱重围。袁熙、高干皆被箭伤。军马死亡殆尽。绍抱三子痛哭一场,不觉昏倒。众人急救,绍口吐鲜血不止,叹曰:“吾自历战数十场,不意今日狼狈至此!此天丧吾也!汝等各回本州,誓与曹贼一决雌雄!”便教辛评、郭图火急随袁谭前往青州整顿,恐曹操犯境;令袁熙仍回幽州,高干仍回并州:各去收拾人马,以备调用。袁绍引袁尚等入冀州养病,令尚与审配、逢纪暂掌军事。却说曹操自仓亭大胜,重赏三军;令人探察冀州虚实。细作回报:“绍卧病在床。袁尚、审配紧守城池。袁谭,袁熙、高干皆回本州。”众皆劝操急攻之。操曰:“冀州粮食极广,审配又有机谋,未可急拔。现今禾稼在田,恐废民业,姑待秋成后取之未晚。”正议间,忽荀彧有书到,报说:“刘备在汝南得刘辟、龚都数万之众。闻丞相提军出征河北,乃令刘辟守汝南,备亲自引兵乘虚来攻许昌。丞相可速回军御之。”操大惊,留曹洪屯兵河上,虚张声势。操自提大兵往汝南来迎刘备。却说玄德与关、张、赵云等,引兵欲袭许都。行近穰山地面,正遇曹兵杀来,玄德便于穰山下寨,军分三队:云长屯兵于东南角上,张飞屯兵于西南角上,玄德与赵云于正南立寨。曹操兵至,玄德鼓噪而出。操布成阵势,叫玄德打话。玄德出马于门旗下。操以鞭指骂曰:“吾待汝为上宾,汝何背义忘恩?”玄德曰:“汝托名汉相,实为国贼!吾乃汉室宗亲,奉天子密诏,来讨反贼!”遂于马上朗诵衣带诏。操大怒,教许褚出战。玄德背后赵云挺枪出马。二将相交三十合,不分胜负。忽然喊声大震,东南角上,云长冲突而来;西南角上,张飞引军冲突而来。三处一齐掩杀。曹军远来疲困,不能抵当,大败而走。玄德得胜回营。

次日,又使赵云搦战。操兵旬日不出。玄德再使张飞搦战,操兵亦不出。玄德愈疑。忽报龚都运粮至,被曹军围住,玄德急令张飞去救。忽又报夏侯惇引军抄背后径取汝南,玄德大惊曰:“若如此,吾前后受敌,无所归矣!”急遣云长救之。两军皆去。不一日,飞马来报夏侯惇已打破汝南,刘辟弃城而走,云长现今被围。玄德大惊。又报张飞去救龚都,也被围住了。玄德急欲回兵,又恐操兵后袭。忽报寨外许褚搦战。玄德不敢出战,候至天明,教军士饱餐,步军先起,马军后随,寨中虚传更点。玄德等离寨约行数里,转过土山,火把齐明,山头上大呼曰:“休教走了刘备!丞相在此专等!”玄德慌寻走路。赵云曰:“主公勿忧,但跟某来。”赵云挺枪跃马,杀开条路,玄德掣双股剑后随。正战间。许褚追至,与赵云力战。背后于禁、李典又到。玄德见势危,落荒而走。听得背后喊声渐远,玄德望深山僻路,单马逃生。

捱到天明,侧首一彪军冲出。玄德大惊,视之,乃刘辟引败军千余骑,护送玄德家小前来;孙乾。简雍,糜芳亦至,诉说:“夏侯惇军势甚锐,因此弃城而走。曹兵赶来,幸得云长挡住,因此得脱。”玄德曰:“不知云长今在何处?”刘辟曰:“将军且行,却再理会。”行到数里,一棒鼓响,前面拥出一彪人马。当先大将,乃是张邰,大叫:“刘备快下马受降!”玄德方欲退后,只见山头上红旗磨动,一军从山坞内拥出,为首大将,乃高览也。玄德两头无路,仰天大呼曰:“天何使我受此窘极耶!事势至此,不如就死!”欲拔剑自刎,刘辟急止之曰:“容某死战,夺路救君。”言讫,便来与高览交锋。战不三合,被高览一刀砍于马下。

玄德正慌,方欲自战,高览后军忽然自乱,一将冲阵而来,枪起处,高览翻身落马。视之,乃赵云也。玄德大喜。云纵马挺枪,杀散后队,又来前军独战张邰。邰与云战三十余合,拨马败走。云乘势冲杀,却被邰兵守住山隘,路窄不得出。正夺路间,只见云长、关平、周仓引三百军到。两下相攻,杀退张邰。各出隘口,占住山险下寨。玄德使云长寻觅张飞。原来张飞去救龚都,龚都已被夏侯渊所杀;飞奋力杀退夏侯渊,迤逦赶去,却被乐进引军围住。云长路逢败军,寻踪而去,杀退乐进,与飞同回见玄德。

人报曹军大队赶来,玄德教孙乾等保护老小先行。玄德与关、张、赵云在后,且战且走。操见玄德去远,收军不赶。玄德败军不满一千,狼狈而奔。前至一江,唤土人问之,乃汉江也。玄德权且安营。土人知是玄德,奉献羊酒,乃聚饮于沙滩之上。玄德叹曰:“诸君皆有王佐之才,不幸跟随刘备。备之命窘,累及诸君。今日身无立锥,诚恐有误诸君。君等何不弃备而投明主,以取功名乎?”众皆掩面而哭。云长曰:“兄言差矣。昔日高祖与项羽争天下,数败于羽;后九里山一战成功,而开四百年基业。胜负兵家之常,何可自隳其志!”孙乾曰:“成败有时,不可丧志。此离荆州不远。刘景升坐镇九郡,兵强粮足,更且与公皆汉室宗亲,何不往投之?”玄德曰:“但恐不容耳。”乾曰:“某愿先往说之,使景升出境而迎庄公”玄德大喜,便令孙乾星夜往荆州。到郡入见刘表,礼毕,刘表问曰:“公从玄德,何故至此?”乾曰:“刘使君天下英雄,虽兵微将寡,而志欲匡扶社稷。汝南刘辟、龚都素无亲故,亦以死报之。明公与使君,同为汉室之胄;今使君新败,欲往江东投孙仲谋。乾僭言曰:不可背亲而向疏。荆州刘将军礼贤下士,士归之如水之投东,何况同宗乎?因此使君特使乾先来拜白。惟明公命之。”表大喜曰:“玄德,吾弟也。久欲相会而不可得。今肯惠顾,实为幸甚!”蔡瑁谮曰:“不可。刘备先从吕布,后事曹操,近投袁绍,皆不克终,足可见其为人。今若纳之,曹操必加兵于我,枉动干戈。不如斩孙乾之首,以献曹操,操必重待主公也。”孙乾正色曰:“乾非惧死之人也。刘使君忠心为国,非曹操、袁绍、吕布等比。前此相从,不得已也。今闻刘将军汉朝苗裔,谊切同宗,故千里相投。尔何献谗而妒贤如此耶?”刘表闻言,乃叱蔡瑁曰:“吾主意已定,汝勿多言。”蔡瑁惭恨而出,刘表遂命孙乾先往报玄德,一面亲自出郭三十里迎接。玄德见表,执礼甚恭。表亦相待甚厚。玄德引关、张等拜见刘表,表遂与玄德等同入荆州,分拨院宅居住。却说曹操探知玄德已往荆州投奔刘表,便欲引兵攻之。程昱曰:“袁绍未除,而遽攻荆襄,倘袁绍从北而起,胜负未可知矣。不如还兵许都,养军蓄锐,待来年春暖,然后引兵先破袁绍,后取荆襄:南北之利,一举可收也。”操然其言,遂提兵回许都。至建安七年,春正月,操复商议兴兵。先差夏侯惇、满宠镇守汝南,以拒刘表;留曹仁、荀彧守许都:亲统大军前赴官渡屯扎。且说袁绍自旧岁感冒吐血症候,今方稍愈,商议欲攻许都。审配谏曰:“旧岁官渡,仓亭之败,军心未振;尚当深沟高垒,以养军民之力。”正议间,忽报曹操进兵官渡,来攻冀州。绍曰:“若候兵临城下,将至壕边,然后拒敌,事已迟矣。吾当自领大军出迎。”袁尚曰:“父亲病体未痊,不可远征。儿愿提兵前去迎敌。”绍许之,遂使人往青州取袁谭,幽州取袁熙,并州取高干:四路同破曹操。正是:才向汝南鸣战鼓,又从冀北动征鼙。未知胜负如何,且听下文分解。