\chapter{哭祖庙一王死孝~入西川二士争功}

却说后主在成都,闻邓艾取了绵竹,诸葛瞻父子已亡,大惊,急召文武商议。近臣奏
曰:“城外百姓,扶老携幼,哭声大震,各逃生命。”后主惊惶无措。忽哨马报到,说魏兵
将近城下。多官议曰:“兵微将寡,难以迎敌;不如早弃成都,奔南中七郡。其地险峻,可
以自守,就借蛮兵,再来克复未迟。”光禄大夫谯周曰:“不可。南蛮久反之人,平昔无
惠;今若投之,必遭大祸。”多官又奏曰:“蜀、吴既同盟,今事急矣,可以投之。”周又
谏曰:“自古以来,无寄他国为天子者。臣料魏能吞吴,吴不能吞魏。若称臣于吴,是一辱
也;若吴被魏所吞,陛下再称臣于魏,是两番之辱矣。不如不投吴而降魏。魏必裂土以封陛
下,则上能自守宗庙,下可以保安黎民。愿陛下思之。”后主未决,退入宫中。次日,众议
纷然。谯周见事急,复上疏诤之。后主从谯周之言,正欲出降;忽屏风后转出一人,厉声而
骂周曰:“偷生腐儒,岂可妄议社稷大事!自古安有降天子哉!”后主视之,乃第五子北地
王刘谌也。后主生七子:长子刘璿,次子刘瑶,三子刘琮,四子刘瓒,五子即北地王刘谌,
六子刘恂,七子刘璩。七子中惟谌自幼聪明,英敏过人,余皆儒善。后主谓谌曰:“今大臣
皆议当降,汝独仗血气之勇,欲令满城流血耶?”谌曰:“昔先帝在日,谯周未尝于预国
政;今妄议大事,辄起乱言,甚非理也。臣切料成都之兵,尚有数万;姜维全师,皆在剑
阁,若知魏兵犯阙,必来救应:内外攻击,可获大功。岂可听腐儒之言,轻废先帝之基业
乎?”后主叱之曰:“汝小儿岂识天时!”谌叩头哭曰:“若势穷力极,祸败将及,便当父
子君臣背城一战,同死社稷,以见先帝可也。奈何降乎!”后主不听。谌放声大哭曰:“先
帝非容易创立基业,今一旦弃之,吾宁死不辱也!”后主令近臣推出宫门,遂令谯周作降
书,遣私署侍中张绍、驸马都尉邓良同谯周赍玉玺来雒城请降。时邓艾每日令数百铁骑来成
都哨探。当日见立了降旗,艾大喜。不一时,张绍等至,艾令人迎入。三人拜伏于阶下,呈
上降款玉玺。艾拆降书视之,大喜,受下玉玺,重待张绍、谯周、邓良等。艾作回书,付三
人赍回成都,以安人心。三人拜辞邓艾,径还成都,入见后主,呈上回书,细言邓艾相待之
善。后主拆封视之,大喜,即遣太仆蒋显赍敕令姜维早降;遣尚书郎李虎,送文簿与艾:共
户二十八万,男女九十四万,带甲将士十万二千,官吏四万,仓粮四十余万,金银各二千
斤,锦绮彩绢各二十万匹。余物在库,不及具数。择十二月初一日,君臣出降。北地王刘谌
闻知,怒气冲天,乃带剑入宫。其妻崔夫人问曰:“大王今日颜色异常,何也?”谌曰:
“魏兵将近,父皇已纳降款,明日君巨出降,社稷从此殄灭。吾欲先死以见先帝于地下,不
屈膝于他人也!”崔夫人曰:“贤哉!贤哉!得其死矣!妾请先死,王死未迟。”谌曰:
“汝何死耶?”崔夫人曰:“王死父,妾死夫:其义同也。夫亡妻死,何必问焉!”言讫,
触柱而死。谌乃自杀其三子,并割妻头,提至昭烈庙中,伏地哭曰:“臣羞见基业弃于他
人,故先杀妻子,以绝挂念,后将一命报祖!祖如有灵,知孙之心!”大哭一场,眼中流

血,自刎而死。蜀人闻知,无不哀痛。后人有诗赞曰:“君臣甘屈膝,一子独悲伤。去矣西
川事,雄哉北地王!捐身酬烈祖,搔首泣穹苍。凛凛人如在,谁云汉已亡?”后主听知北地
王自刎,乃令人葬之。次日,魏兵大至。后主率太子诸王,及群臣六十余人,面缚舆榇,出
北门十里而降。邓艾扶起后主,亲解其缚,焚其舆榇,并车入城。后人有诗叹曰:“魏兵数
万入川来,后主偷生失自裁。黄皓终存欺国意,姜维空负济时才。全忠义士心何烈,守节王
孙志可哀。昭烈经营良不易,一朝功业顿成灰。”

于是成都之人,皆具香花迎接。艾拜后主为骠骑将军,其余文武,各随高下拜官;请后
主还宫,出榜安民,交割仓库。又令太常张峻、益州别驾张绍,招安各郡军民。又令人说姜
维归降。一面遣人赴洛阳报捷。艾闻黄皓奸险,欲斩之。皓用金宝赂其左右,因此得免。自
是汉亡。后人因汉之亡,有追思武侯诗曰:“鱼鸟犹疑畏简书,风云长为护储胥。徒令上将
挥神笔,终见降王走传车。管乐有才真不忝,关张无命欲何如!他年锦里经祠庙,梁父吟成
恨有余!”

且说太仆蒋显到剑阁,入见姜维,传后主敕命,言归降之事。维大惊失语。帐下众将听
知,一齐怨恨,咬牙怒目,须发倒竖,拔刀砍石大呼曰:“吾等死战,何故先降耶!”号哭
之声,闻数十里。维见人心思汉,乃以善言抚之曰:“众将勿忧。吾有一计,可复汉室。”
众皆求问。姜维与诸将附耳低言,说了计策。即于剑阁关遍竖降旗,先令人报入钟会寨中,
说姜维引张翼、廖化、董厥等来降。会大喜,令人迎接维入帐。会曰:“伯约来何迟也?”
维正色流涕曰:“国家全军在吾,今日至此,犹为速也。”会甚奇之,下座相拜。待为上
宾。维说会曰:“闻将军自淮南以来。算无遗策;司马氏之盛,皆将军之力,维故甘心俯
首。如邓士载,当与决一死战,安肯降之乎?”会遂折箭为誓,与维结为兄弟,情爱甚密,
仍令照旧领兵。维暗喜,遂令蒋显回成都去了。

却说邓艾封师纂为益州刺史,牵弘、王颀等各领州郡;又于绵竹筑台以彰战功,大会蜀
中诸官饮宴。艾酒至半酣,乃指众官曰:“汝等幸遇我,故有今日耳。若遇他将,必皆殄灭
矣。”多官起身拜谢。忽蒋显至,说姜维自降钟镇西了。艾因此痛恨钟会。遂修书令人赍赴
洛阳,致晋公司马昭。昭得书视之。书曰:“臣艾切谓兵有先声而后实者,今因平蜀之势以
乘吴,此席卷之时也。然大举之后,将士疲劳,不可便用;宜留陇右兵二万、蜀兵二万,煮
盐兴冶,并造舟船,预备顺流之计;然后发使,告以利害,吴可不征而定也。今宜厚待刘
禅,以致孙休;若便送禅来京,吴人必疑,则于向化之心不劝。且权留之于蜀,须来年冬月
抵京。今即可封禅为扶风王,锡以资财,供其左右,爵其子为公侯,以显归命之宠:则吴人
畏威怀德,望风而从矣。”司马昭览毕,深疑邓艾有自专之心,乃先发手书与卫瓘,随后降
封艾诏曰:“征西将军邓艾耀威奋武,深入敌境,使僭号之主,系颈归降;兵不逾时,战不
终日,云彻席卷,荡定巴、蜀;虽白起破强楚,韩信克劲赵,不足比勋也。其以艾为太尉,
增邑二万户,封二子为亭侯,各食邑千户。”邓艾受诏毕,监军卫瓘取出司马昭手书与艾。
书中说邓艾所言之事,须候奏报,不可辄行。艾曰:“将在外,君命有所不受。吾既奉诏专
征,如何阻当?”遂又作书,今来使赍赴洛阳。时朝中皆言邓艾必有反意,司马昭愈加疑
忌。忽使命回,呈上邓艾之书。昭拆封视之。书曰:“艾衔命西征,元恶既服,当权宜行
事,以安初附。若待国命,则往复道途,延引日月。《春秋》之义:大夫出疆,有可以安社
稷、利国家,专之可也。今吴未宾,势与蜀连,不可拘常以失事机。兵法:进不求名,退不
避罪。艾虽无古人之节,终不自嫌以损于国也。先此申状,见可施行。”

司马昭看毕大惊,忙与贾充计议曰:“邓艾恃功而骄,任意行事,反形露矣。如之奈
何?”贾充曰:“主公何不封钟会以制之?”昭从其议,遣使赍诏封会为司徒,就令卫瓘监
督两路军马,以手书付瓘,使与会伺察邓艾,以防其变。会接读诏书。诏曰:“镇西将军钟
会所向无敌,前无强梁,节制众城,网罗进逸;蜀之豪帅,面缚归命;谋无遗策,举无废
功。其以会为司徒,进封县侯,增邑万户,封子二人亭侯,邑各千户。”钟会既受封,即请
姜维计议曰:“邓艾功在吾之上,又封太尉之职;今司马公疑艾有反志,故令卫瓘为监军,
诏吾制之。伯约有何高见?”维曰:“愚闻邓艾出身微贱,幼为农家养犊,今侥幸自阴平斜
径,攀木悬崖,成此大功;非出良谋,实赖国家洪福耳。若非将军与维相拒于剑阁,艾安能
成此功耶?今欲封蜀主为扶风王,乃大结蜀人之心,其反情不言可见矣。晋公疑之是也。”
会深喜其言。维又曰:“请退左右,维有一事密告。”会令左右尽退。维袖中取一图与会,
曰:“昔日武侯出草庐时,以此图献先帝,且曰:益州之地,沃野千里,民殷国富,可为霸
业。先帝因此遂创成都。今邓艾至此,安得不狂?”会大喜,指问山川形势。维一一言之。
会又问曰:“当以何策除艾?”维曰:“乘晋公疑忌之际,当急上表,言艾反状;晋公必令
将军讨之。一举而可擒矣。”会依言,即遣人赍表进赴洛阳,言邓艾专权恣肆,结好蜀人,
早晚必反矣。于是朝中文武皆惊。会又今人于中途截了邓艾表文,按艾笔法,改写傲慢之
辞,以实己之语。

司马昭见了邓艾表章,大怒,即遣人到钟会军前,令会收艾;又遣贾充引三万兵入斜
谷,昭乃同魏主曹奂御驾亲征。西曹掾邵悌谏曰:“钟会之兵,多艾六倍,当今会收艾足
矣,何必明公自行耶?”昭笑曰:“汝忘了旧日之言耶?汝曾道会后必反。吾今此行,非为
艾,实为会耳。”悌笑曰“某恐明公忘之,故以相问。今既有此意,切宜秘之,不可泄
漏。”昭然其言,遂提大兵起程。时贾充亦疑钟会有变,密告司马昭。昭曰:“如遣汝,亦
疑汝耶?吾到长安,自有明白。”早有细作报知钟会,说昭已至长安。会慌请姜维商议收艾
之策。正是:才看西蜀收降将,又见长安动大兵。不知姜维以何策破艾,且看下文分解。