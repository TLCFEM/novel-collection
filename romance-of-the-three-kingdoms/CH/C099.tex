\chapter{诸葛亮大破魏兵~司马懿入寇西蜀}

蜀汉建兴七年夏四月,孔明兵在祁山,分作三寨,专候魏兵。却说司马懿引兵到长安,张郃接见,备言前事。懿令郃为先锋,戴陵为副将,引十万兵到祁山,于渭水之南下寨。郭淮、孙礼入寨参见。懿问曰:“汝等曾与蜀兵对阵否?”二人答曰:“未也。”懿曰:“蜀兵千里而来,利在速战;今来此不战,必有谋也。陇西诸路,曾有信息否?”淮曰:“已有细作探得各郡十分用心,日夜提防,并无他事。只有武都、阴平二处,未曾回报。”懿曰:“吾自差人与孔明交战。汝二人急从小路去救二郡,却掩在蜀兵之后,彼必自乱矣。”

二人受计,引兵五千,从陇西小路来救武都、阴平,就袭蜀兵之后。郭淮于路谓孙礼曰:“仲达比孔明如何?”礼曰:“孔明胜仲达多矣。”淮曰:“孔明虽胜,此一计足显仲达有过人之智。蜀兵如正攻两郡,我等从后抄到,彼岂不自乱乎?”正言间,忽哨马来报:“阴平已被王平打破了,武都已被姜维打破了。前离蜀兵不远。”礼曰:“蜀兵既已打破了城池,如何陈兵于外?必有诈也。不如速退。”郭淮从之。方传令教军退时,忽然一声炮响,山背后闪出一枝军马来,旗上大书:“汉丞相诸葛亮”,中央一辆四轮车,孔明端坐于上;左有关兴,右有张苞。孙、郭二人见之,大惊。孔明大笑曰:“郭淮、孙礼休走!司马懿之计,安能瞒得过吾?他每日令人在前交战,却教汝等袭吾军后。武都、阴平吾已取了。汝二人不早来降,欲驱兵与吾决战耶?”郭淮、孙礼听毕,大慌。忽然背后喊杀连天,王平、姜维引兵从后杀来。兴、苞二将又引军从前面杀来。两下夹攻,魏兵大败。郭、孙二人弃马爬山而走。张苞望见,骤马赶来;不期连人带马,跌入涧内,后军急忙救起,头已跌破。孔明令人送回成都养病。

却说郭、孙二人走脱,回见司马懿曰:“武都、阴平二郡已失。孔明伏于要路,前后攻杀,因此大败,弃马步行,方得逃回。”懿曰:“非汝等之罪,孔明智在吾先。可再引兵守把雍、郿二城,切勿出战。吾自有破敌之策。”二人拜辞而去。懿又唤张郃、戴陵分付曰:“今孔明得了武都、阴平,必然抚百姓以安民心,不在营中矣。汝二人各引一万精兵,今夜起身,抄在蜀兵营后,一齐奋勇杀将过来;吾却引军在前布阵,只待蜀兵势乱,吾大驱士马,攻杀进去:两军并力,可夺蜀寨也。若得此地山势,破敌何难?”二人受计引兵而去。

戴陵在左,张郃在右,各取小路进发,深入蜀兵之后。三更时分,来到大路,两军相遇,合兵一处,却从蜀兵背后杀来。行不到三十里,前军不行。张、戴二人自纵马视之,只见数百辆草车横截去路。郃曰:“此必有准备。可急取路而回。”才传令退军,只见满山火光齐明,鼓角大震,伏兵四下皆出,把二人围住。孔明在祁山上大叫曰:“戴陵、张郃可听吾言:司马懿料吾往武都、阴平抚民,不在营中,故令汝二人来劫吾寨,却中吾之计也。汝二人乃无名下将,吾不杀害,下马早降!”郃大怒,指孔明而骂曰:“汝乃山野村夫,侵吾大国境界,如何敢发此言!吾若捉住汝时,碎尸万段!”言讫,纵马挺枪,杀上山来。山上矢石如雨,郃不能上山,乃拍马舞枪,冲出重围,无人敢当。蜀兵困戴陵在垓心。郃杀出旧路,不见戴陵,即奋勇翻身又杀入重围,救出戴陵而回。孔明在山上,见郃在万军之中,往来冲突,英勇倍加,乃谓左右曰:“尝闻张翼德大战张郃,人皆惊惧。吾今日见之,方知其勇也。若留下此人,必为蜀中之害。吾当除之。”遂收军还营。

却说司马懿引兵布成阵势,只待蜀兵乱动,一齐攻之。忽见张郃、戴陵狼狈而来,告曰:“孔明先如此提防,因此大败而归。”懿大惊曰:“孔明真神人也!不如且退。”即传令教大军尽回本寨,坚守不出。且说孔明大胜,所得器械、马匹,不计其数,乃引大军回寨。每日令魏延挑战,魏兵不出。一连半月,不曾交兵。孔明正在帐中思虑,忽报天子遣侍中费祎赍诏至。孔明接入营中,焚香礼毕,开诏读曰:“街亭之役,咎由马谡;而君引愆,深自贬抑。重违君意,听顺所守。前年耀师,馘斩王双;今岁爱征,郭淮遁走;降集氏、羌,复兴二郡:威震凶暴,功勋显然。方今天下骚扰,元恶未枭,君受大任,干国之重,而久自抑损,非所以光扬洪烈矣。今复君丞相,君其勿辞!”孔明听诏毕,谓费祎曰:“吾国事未成,安可复丞相之职?”坚辞不受。祎曰:“丞相若不受职,拂了天子之意,又冷淡了将士之心。宜且权受。”孔明方才拜受。祎辞去。

孔明见司马懿不出,思得一计,传令教各处皆拔寨而起。当有细作报知司马懿,说孔明退兵了。懿曰:“孔明必有大谋,不可轻动。”张郃曰:“此必因粮尽而回,如何不追?”懿曰:“吾料孔明上年大收,今又麦熟,粮草丰足;虽然转运艰难,亦可支吾半载,安肯便走?彼见吾连日不战,故作此计引诱。可令人远远哨之。”军士探知,回报说:“孔明离此三十里下寨。”懿曰:“吾料孔明果不走。且坚守寨栅,不可轻进。”住了旬日,绝无音信,并不见蜀将来战。懿再令人哨探,回报说:“蜀兵已起营去了。”懿未信,乃更换衣服,杂在军中,亲自来看,果见蜀兵又退三十里下寨。懿回营谓张郃曰:“此乃孔明之计也,不可追赶。”又住了旬日,再令人哨探。回报说:“蜀兵又退三十里下寨。”郃曰:“孔明用缓兵之计,渐退汉中,都督何故怀疑,不早追之?郃愿往决一战!”懿曰:“孔明诡计极多,倘有差失,丧我军之锐气。不可轻进。”郃曰:“某去若败,甘当军令。”懿曰:“既汝要去,可分兵两枝:汝引一枝先行,须要奋力死战;吾随后接应,以防伏兵。汝次日先进,到半途驻扎,后日交战,使兵力不乏。”遂分兵已毕。

次日,张郃、戴陵引副将数十员、精兵三万,奋勇先进,到半路下寨。司马懿留下许多军马守寨,只引五千精兵,随后进发。原来孔明密令人哨探,见魏兵半路而歇。是夜,孔明唤众将商议曰:“今魏兵来追,必然死战,汝等须以一当十,吾以伏兵截其后:非智勇之将,不可当此任。”言毕,以目视魏延。延低头不语。王平出曰:“某愿当之。”孔明曰:“若有失,如何?”平曰:“愿当军令。”孔明叹曰:“王平肯舍身亲冒矢石,真忠臣也!虽然如此,奈魏兵分两枝前后而来,断吾伏兵在中;平纵然智勇,只可当一头,岂可分身两处?须再得一将同去为妙。怎奈军中再无舍死当先之人!”言未毕,一将出曰:“某愿往!”孔明视之,乃张翼也。孔明曰:“张郃乃魏之名将,有万夫不当之勇,汝非敌手。”翼曰:“若有失事,愿献首于帐下。”孔明曰:“汝既敢去,可与王平各引一万精兵伏于山谷中;只待魏兵赶上,任他过尽,汝等却引伏兵从后掩杀。若司马懿随后赶来,却分兵两头:张翼引一军当住后队,王平引一军截其前队。两军须要死战。吾自有别计相助。”二人受计引兵而去。

孔明又唤姜维、廖化分付曰:“与汝二人一个锦囊,引三千精兵,偃旗息鼓,伏于前山之上。如见魏兵围住王平、张翼,十分危急,不必去救,只开锦囊看视,自有解危之策。”二人受计引兵而去。又令吴班、吴懿、马忠、张嶷四将,附耳分付曰:“如来日魏兵到,锐气正盛,不可便迎,且战且走。只看关兴引兵来掠阵之时,汝等便回军赶杀,吾自有兵接应。”四将受计引兵而去。又唤关兴分付曰:“汝引五千精兵,伏于山谷;只看山上红旗飐动,却引兵杀出。”兴受计引兵而去。

却说张郃、戴陵领兵前来,骤如风雨。马忠、张嶷、吴懿、吴班四将接着,出马交锋。张郃大怒,驱兵追杀。蜀兵且战且走,魏兵追赶约有二十余里,时值六月天气,十分炎热,人马汗如泼水。走到五十里外,魏兵尽皆气喘。孔明在山上把红旗一招,关兴引兵杀出。马忠等四将,一齐引兵掩杀回来。张郃、戴陵死战不退。忽然喊声大震,两路军杀出,乃王平、张翼也。各奋勇追杀,截其后路。郃大叫众将曰:“汝等到此,不决一死战,更待何时!”魏兵奋力冲突,不得脱身。忽然背后鼓角喧天,司马懿自领精兵杀到。懿指挥众将,把王平、张翼围在垓心。翼大呼曰:“丞相真神人也!计已算定,必有良谋。吾等当决一死战!”即分兵两路:平引一军截住张郃、戴陵,翼引一军力当司马懿。两头死战,叫杀连天。姜维、廖化在山上探望,见魏兵势大,蜀兵力危,渐渐抵当不住。维谓化曰:“如此危急,可开锦囊看计。”二人拆开视之,内书云:“若司马懿兵来围王平、张翼至急,汝二人可分兵两枝,竟袭司马懿之营;懿必急退,汝可乘乱攻之。营虽不得,可获全胜。”二人大喜,即分兵两路,径袭司马懿营中而去。原来司马懿亦恐中孔明之计,沿途不住的令人传报。懿正催战间,忽流星马飞报,言蜀兵两路竟取大寨去了,懿大惊失色,乃谓众将曰:“吾料孔明有计,汝等不信,勉强追来,却误了大事!”即提兵急回。军心惶惶乱走。张翼随后掩杀,魏兵大败。张郃、戴陵见势孤,亦望山僻小路而走,蜀兵大胜。背后关兴引兵接应诸路。司马懿大败一阵,奔入寨时,蜀兵已自回去。懿收聚败军,责骂诸将曰:“汝等不知兵法,只凭血气之勇,强欲出战,致有此败。今后切不许妄动,再有不遵,决正军法!”众皆羞惭而退。这一阵,魏军死者极多,遗弃马匹器械无数。却说孔明收得胜军马入寨,又欲起兵进取。忽报有人自成都来,说张苞身死。孔明闻知,放声大哭,口中吐血,昏绝于地。众人救醒。孔明自此得病卧床不起。诸将无不感激。后人有诗叹曰:“悍勇张苞欲建功,可怜天不助英雄!武侯泪向西风洒,为念无人佐鞠躬。”

旬日之后,孔明唤董厥、樊建等入帐分付曰:“吾自觉昏沉,不能理事;不如且回汉中养病,再作良图。汝等切勿走泄:司马懿若知,必来攻击。”遂传号令,教当夜暗暗拔寨,皆回汉中。孔明去了五日,懿方得知,乃长叹曰:“孔明真有神出鬼没之计,吾不能及也!”于是司马懿留诸将在寨中,分兵守把各处隘口;懿自班师回。

却说孔明将大军屯于汉中,自回成都养病;文武官僚出城迎接,送入丞相府中,后主御驾自来问病,命御医调治,日渐痊可。建兴八年秋七月,魏都督曹真病可,乃上表说:“蜀兵数次侵界,屡犯中原,若不剿除,必为后患。今时值秋凉,人马安闲,正当征伐。臣愿与司马懿同领大军,径入汉中,殄灭奸党,以清边境。”魏主大喜,问侍中刘晔曰:“子丹劝朕伐蜀,若何?”晔奏曰:“大将军之言是也。今若不剿除,后必为大患。陛下便可行之。睿点头。晔出内回家,有众大臣相探,问曰:“闻天子与公计议兴兵伐蜀,此事如何?”晔应曰:“无此事也。蜀有山川之险,非可易图;空费军马之劳,于国无益。”众官皆默然而出。杨暨入内奏曰:“昨闻刘晔劝陛下伐蜀;今日与众臣议,又言不可伐:是欺陛下也。陛下何不召而问之?”睿即召刘晔入内问曰:“卿劝朕伐蜀;今又言不可,何也?”晔曰:“臣细详之,蜀不可伐。”睿大笑。少时,杨暨出内。晔奏曰:“臣昨日劝陛下伐蜀,乃国之大事,岂可妄泄于人?夫兵者,诡道也:事未发,切宜秘之。”睿大悟曰:“卿言是也。”自此愈加敬重。

旬日内,司马懿入朝,魏主将曹真表奏之事,逐一言之。懿奏曰:“臣料东吴未敢动兵,今日正可乘此去伐蜀。”睿即拜曹真为大司马、征西大都督,司马懿为大将军、征西副都督,刘晔为军师。三人拜辞魏主,引四十万大兵,前行至长安,径奔剑阁,来取汉中。其余郭淮、孙礼等,各取路而行。汉中人报入成都。此时孔明病好多时,每日操练人马,习学八阵之法,尽皆精熟,欲取中原;听得这个消息,遂唤张嶷、王平分付曰:“汝二人先引一千兵去守陈仓古道,以当魏兵;吾却提大兵便来接应。”二人告曰:“人报魏军四十万,诈称八十万,声势甚大,如何只与一千兵去守隘口?倘魏兵大至,何以拒之?”孔明曰:“吾欲多与,恐士卒辛苦耳。”嶷与平面面相觑,皆不敢去。孔明曰:“若有疏失,非汝等之罪。不必多言,可疾去。”二人又哀告曰:“丞相欲杀某二人,就此清杀,只不敢去。”孔明笑曰:“何其愚也!吾令汝等去,自有主见:吾昨夜仰观天文,见毕星廛于太阴之分,此月内必有大雨淋漓;魏兵虽有四十万,安敢深入山险之地?因此不用多军,决不受害。吾将大军皆在汉中安居一月,待魏兵退,那时以大兵掩之:以逸待劳,吾十万之众可胜魏兵四十万也。”二人听毕,方大喜,拜辞而去。孔明随统大军出汉中,传令教各处隘口,预备干柴草料细粮,俱够一月人马支用,以防秋雨;将大军宽限一月,先给衣食,伺候出征。却说曹真、司马懿同领大军,径到陈仓城内,不见一间房屋;寻土人问之,皆言孔明回时放火烧毁。曹真便要从陈仓道进发。懿曰:“不可轻进。我夜观天文,见毕星躔于太阴之分,此月内必有大雨;若深入重地,常胜则可。倘有疏虞,人马受苦,要退则难。且宜在城中搭起窝铺住扎,以防阴雨。”真从其言。未及半月,天雨大降,淋漓不止。陈仓城外,平地水深三尺,军器尽湿,人不得睡,昼夜不安。大雨连降三十日,马无草料,死者无数,军士怨声不绝。传入洛阳,魏主设坛,求晴不得。黄门侍郎王肃上疏曰:“前志有之;千里馈粮,士有饥色;樵苏后爨,师不宿饱。此谓平途之行军者也。又况于深入险阻,凿路而前,则其为劳,必相百也。今又加之以霖雨,山坂峻滑,众逼而不展,粮远而难继:实行军之大忌也。闻曹真发已逾月,而行方半谷,治道功大,战士悉作:是彼偏得以逸待劳,乃兵家之所惮也。言之前代,则武王伐纣,出关而复还;论之近事,则武、文征权,临江而不济:岂非顺天知时,通于权变者哉?愿陛下念水雨艰剧之故,休息士卒;后日有衅,乘时用之。所谓悦以犯难,民忘其死者也。”魏主览表,正在犹豫,杨阜、华歆亦上疏谏。魏主即下诏,遣使诏曹真、司马懿还朝。

却说曹真与司马懿商议曰:“今连阴三十日,军无战心,各有思归之意,如何禁止?”懿曰:“不如且回。”真曰:“倘孔明追来,怎生退之?”懿曰:“先伏两军断后,方可回兵。”正议间,忽使命来召。二人遂将大军前队作后队,后队作前队,徐徐而退。却说孔明计算一月秋雨将尽,天尚未晴,自提一军屯于城固,又传令教大军会于赤坡驻扎。孔明升帐唤众将言曰:“吾料魏兵必走,魏主必下诏来取曹真、司马懿兵回。吾若追之,必有准备;不如任他且去,再作良图。”忽王平令人报来,说魏兵已回。孔明分付来人,传与王平:“不可追袭。吾自有破魏兵之策。”正是:魏兵纵使能埋伏,汉相原来不肯追。未知孔明怎生破魏,且看下文分解。