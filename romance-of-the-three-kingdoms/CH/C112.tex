\chapter{救寿春于诠死节~取长城伯约鏖兵}

却说司马昭闻诸葛诞会合吴兵前来决战,乃召散骑长史裴秀、黄门侍郎钟会,商议破敌之策。钟会曰:“吴兵之助诸葛诞,实为利也;以利诱之,则必胜矣。”昭从其言,遂令石苞、州泰先引两军于石头城埋伏,王基、陈骞领精兵在后,却令偏将成倅引兵数万先去诱敌;又令陈俊引车仗牛马驴骡,装载赏军之物,四面聚集于阵中,如敌来则弃之。

是日,诸葛诞令吴将朱异在左,文钦在右,见魏阵中人马不整,诞乃大驱士马径进。成倅退走,诞驱兴掩杀,见牛马驴骡,遍满郊野;南兵争取,无心恋战。忽然一声炮响,两路兵杀来:左有石苞,右有州泰,诞大惊,急欲退时,王基、陈骞精兵杀到。诞兵大败。司马昭又引兵接应。诞引败兵奔入寿春,闭门坚守。昭令兵四面围困,并力攻城。

时吴兵退屯安丰,魏主车驾驻于项城。钟会曰:“今诸葛诞虽败,寿春城中粮草尚多,更有吴兵屯安丰以为掎角之势;今吾兵四面攻围,彼缓则坚守,急则死战;吴兵或乘势夹攻:吾军无益。不如三面攻之,留南门大路,容贼自走;走而击之,可全胜也。吴兵远来,粮必不继;我引轻骑抄在其后,可不战而自破矣。”昭抚会背曰:“君真吾之子房也!”遂令王基撤退南门之兵。却说吴兵屯于安丰,孙綝唤朱异责之曰:“量一寿春城不能救,安可并吞中原?如再不胜必斩!”朱异乃回本寨商议。于诠曰:“今寿春南门不围,某愿领一军从南门入去,助诸葛诞守城。将军与魏兵挑战,我却从城中杀出:两路夹攻,魏兵可破矣。”异然其言。于是全怿、全端、文钦等,皆愿入城。遂同于诠引兵一万,从南门而入城。魏兵不得将令,未敢轻敌,任吴兵入城,乃报知司马昭。昭曰:“此欲与朱异内外夹攻,以破我军也。”乃召王基、陈骞分付曰:“汝可引五千兵截断朱异来路,从背后击之。”二人领命而去。朱异正引兵来,忽背后喊声大震:左有王基,右有陈骞,两路军杀来。吴兵大败。朱异回见孙綝,綝大怒曰:“累败之将,要汝何用!”叱武士推出斩之。又责全端子全祎曰:“若退不得魏兵,汝父子休来见我!”于是孙綝自回建业去了。

钟会与昭曰:“今孙綝退去,外无救兵,城可围矣。”昭从之,遂催军攻围。全祎引兵欲入寿春,见魏兵势大,寻思进退无路,遂降司马昭。昭加祎为偏将军。祎感昭恩德,乃修家书与父全端,叔全怿,言孙綝不仁,不若降魏,将书射入城中。怿得祎书,遂与端引数千人开门出降。诸葛诞在城中忧闷,谋士蒋班、焦彝进言曰:“城中粮少兵多,不能久守,可率吴、楚之众,与魏兵决一死战。”诞大怒曰:“吾欲守,汝欲战,莫非有异心乎!再言必斩!”二人仰天长叹曰:“诞将亡矣!我等不如早降,免至一死!”是夜二更时分,蒋、焦二人逾城降魏,司马昭重用之。因此城中虽有敢战之士,不敢言战。诞在城中,见魏兵四下筑起土城以防淮水,只望水泛,冲倒土城,驱兵击之。不想自秋至冬,并无霖雨,淮水不泛。城中看看粮尽,文钦在小城内与二子坚守,见军士渐渐饿倒,只得来告诞曰:“粮皆尽绝,军士饿损,不如将北方之兵尽放出城,以省其食。”诞大怒曰:“汝教我尽去北军,欲谋我耶?”叱左右推出斩之。文鸯、文虎见父被杀,各拔短刀,立杀数十人,飞身上城,一跃而下,越壕赴魏寨投降。司马昭恨文鸯昔日单骑退兵之仇,欲斩之。钟会谏曰:“罪在文钦,今文钦已亡,二子势穷来归,若杀降将,是坚城内人之心也。”昭从之,遂召文鸯、文虎入帐,用好言抚慰,赐骏马锦衣,加为偏将军,封关内侯。二子拜谢,上马绕城大叫曰:“我二人蒙大将军赦罪赐爵,汝等何不早降!”城内人闻言,皆计议曰:“文鸯乃司马氏仇人,尚且重用,何况我等乎?”于是皆欲投降。诸葛诞闻之大怒,日夜自来巡城。以杀为威。

钟会知城中人心已变,乃入帐告昭曰:“可乘此时攻城矣。”昭大喜,遂激三军,四面云集,一齐攻打。守将曾宣献了北门,放魏兵入城。诞知魏兵已入;慌引麾下数百人,自城中小路突出;至吊桥边,正撞着胡奋,手起刀落,斩诞于马下,数百人皆被缚。王基引兵杀到西门,正遇吴将于诠。基大喝曰:“何不早降!”诠大怒曰:“受命而出,为人救难,既不能救,又降他人,义所不为也!”乃掷盔于地,大呼曰:“人生在世,得死于战场者,幸耳!”急挥刀死战三十余合,人困马乏,为乱军所杀。后人有诗赞曰:“司马当年围寿春,降兵无数拜车尘。东吴虽有英雄士,谁及于诠肯杀身!”

司马昭入寿春,将诸葛诞老小尽皆枭首,灭其三族。武士将所擒诸葛诞部卒数百人缚至。昭曰:“汝等降否?”众皆大叫曰:“愿与诸葛公同死,决不降汝!”昭大怒,叱武士尽缚于城外,逐一问曰:“降者免死。”并无一人言降。直杀至尽,终无一人降者。昭深加叹息不已,令皆埋之。后人有诗赞曰:“忠臣矢志不偷生,诸葛公休帐下兵,《薤露》歌声应未断,遗踪直欲继田横!”

却说吴兵大半降魏,裴秀告司马昭曰:“吴兵老小,尽在东南江、淮之地,今若留之,久必为变;不如坑之。”钟会曰:“不然。古之用兵者,全国为上,戮其元恶而已。若尽坑之,是不仁也。不如放归江南,以显中国之宽大。”昭曰:“此妙论也。”遂将吴兵尽皆放归本国。唐咨因惧孙綝,不敢回国,亦来降魏。昭皆重用,令分布三河之地。淮南已平。正欲退兵,忽报西蜀姜维引兵来取长城,邀截粮草。昭大惊,慌与多官计议退兵之策。时蜀汉延熙二十年,改为景耀元年。姜维在汉中,选川将两员,每日操练人马:一是蒋舒,一是傅佥。二人颇有胆勇,维甚爱之。忽报淮南诸葛诞起兵讨司马昭,东吴孙綝助之,昭大起两都之兵,将魏太后并魏主一同出征去了。维大喜曰:“吾今番大事济矣!”遂表奏后主,愿兴兵伐魏。中散大夫谯周听知,叹曰:“近来朝廷溺于酒色,信任中贵黄皓,不理国事,只图欢乐;伯约累欲征伐,不恤军士:国将危矣!”乃作《仇国论》一篇,寄与姜维。维拆封视之。论曰:“或问:古往能以弱胜强者,其术何如?曰:处大国无患者,恒多慢;处小国有忧者,恒思善。多慢则生乱;思善则生治,理之常也,故周文养民,以少取多;句践恤众,以弱毙强。此其术也。或曰:曩者楚强汉弱,约分鸿沟,张良以为民志既定则难动也,率兵追羽,终毙项氏;岂必由文王、句践之事乎?曰:商、周之际,王侯世尊,君臣久固。当此之时,虽有汉祖,安能仗剑取天下乎?及秦罢侯置守之后,民疲秦役,天下土崩,于是豪杰并争。今我与彼,皆传国易世矣,既非秦末鼎沸之时,实有六国并据之势,故可为文王,难为汉祖。时可而后动,数合而后举,故汤、武之师,不再战而克,诚重民劳而度时审也。如遂极武黩征,不幸遇难,虽有智者,不能谋之矣。”姜维看毕,大怒曰:“此腐儒之论也!”掷之于地,遂提川兵来取中原。乃问傅佥曰:“以公度之,可出何地?”佥曰:“魏屯粮草,皆在长城;今可径取骆谷,度沈岭,直到长城,先烧粮草,然后直取秦川,则中原指日可得矣。”维曰:“公之见与吾计暗合也。”即提兵径取骆谷,度沈岭,望长城而来。

却说长城镇守将军司马望,乃司马昭之族兄也。城内粮草甚多,人马却少。望听知蜀兵到,急与王真、李鹏二将,引兵离城二十里下寨。次日,蜀兵来到,望引二将出阵。姜维出马,指望而言曰:“今司马昭迁主于军中,必有李傕、郭汜之意也,吾今奉朝廷明命,前来问罪,汝当早降。若还愚迷,全家诛戮!”望大声而答曰:“汝等无礼,数犯上国,如不早退,令汝片甲不归!”言未毕,望背后王真挺枪出马,蜀阵中傅佥出迎。战不十合,佥卖个破绽,王真便挺枪来刺;傅佥闪过,活捉真于马上,便回本阵。李鹏大怒,纵马轮刀来救。佥故意放慢,等李鹏将近,努力掷真于地,暗掣四楞铁简在手;鹏赶上举刀待砍,傅佥偷身回顾,向李鹏面门只一简,打得眼珠迸出,死于马下。王真被蜀军乱枪刺死。姜维驱兵大进。司马望弃寨入城,闭门不出。维下令曰:“军士今夜且歇一宿,以养锐气。来日须要入城。”次日平明,蜀兵争先大进,一拥至城下,用火箭火炮打入城中。城上草屋一派烧着,魏兵自乱。维又令人取干柴堆满城下,一齐放火,烈焰冲天。城已将陷,魏兵在城内嚎啕痛哭,声闻四野。

正攻打之间,忽然背后喊声大震。维勒马回看,只见魏兵鼓噪摇旗,浩浩而来。维遂令后队为前队,自立于门旗下候之。只见魏阵中一小将,全装惯带,挺枪纵马而出,约年二十余岁,面如傅粉,唇似抹朱,厉声大叫曰:“认得邓将军否!”维自思曰:“此必是邓艾矣。”挺枪纵马来迎。二人抖擞精神,战到三四十合,不分胜负。那小将军枪法无半点放闲。维心中自思:“不用此计,安得胜乎?”便拨马望左边山路中而走。那小将骤马追来,维挂住了钢枪,暗取雕弓羽箭射之。那小将眼乖,早已见了,弓弦响处,把身望前一倒,放过羽箭。维回头看时,小将已到,挺枪来刺;维一闪,那枪从肋傍边过,被维挟住。那小将弃枪,望本阵而走。维嗟叹曰:“可惜!可惜!”再拨马赶来。追至阵门前,一将提刀而出曰:“姜维匹夫,勿赶吾儿!邓艾在此!”维大惊。原来小将乃艾之子邓忠也。维暗暗称奇;欲战邓艾,又恐马乏,乃虚指艾曰:“吾今日识汝父子也。各且收兵,来日决战。”艾见战场不利,亦勒马应曰:“既如此,各自收兵,暗算者非丈夫也。”于是两军皆退。邓艾据渭水下寨,姜维跨两山安营。艾见了蜀兵地理,乃作书与司马望曰:“我等切不可战,只宜固守。待关中兵至时,蜀兵粮草皆尽,三面攻之,无不胜也。今遣长子邓忠相助守城。”一面差人于司马昭处求救。

却说姜维令人于艾寨中下战书,约来日大战,艾佯应之。次日五更,维令三军造饭,平明布阵等候。艾营中偃旗息鼓,却如无人之状。维至晚方回。次日又令人下战书,责以失期之罪。艾以酒食待使,答曰:“微躯小疾,有误相持,明日会战。”次日,维又引兵来,艾仍前不出。如此五六番。傅佥谓维曰:“此必有谋也,宜防之。”维曰:“此必捱关中兵到,三面击我耳。吾今令人持书与东吴孙綝,使并力攻之。”忽探马报说:“司马昭攻打寿春,杀了诸葛诞,吴兵皆降。昭班师回洛阳。便欲引兵来救长城。”维大惊曰:“今番伐魏,又成画饼矣,不如且回。”正是:已叹四番难奏绩,又嗟五度未成功。未知如何退兵,且看下文分解。