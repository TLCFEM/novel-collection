\chapter{孙权降魏受九锡~先主征吴赏六军}

却说章武元年秋八月,先主起大军至夔关,驾屯白帝城。前队军马已出川口。近臣奏曰:“吴使诸葛瑾至。”先主传旨教休放入。黄权奏曰:“瑾弟在蜀为相,必有事而来。陛下何故绝之?当召入,看他言语。可从则从;如不可,则就借彼口说与孙权,令知问罪有名也。”先主从之,召瑾入城。瑾拜伏于地。先主问曰:“子瑜远来,有何事故?”瑾曰:“臣弟久事陛下,臣故不避斧钺,特来奏荆州之事。前者,关公在荆州时,吴侯数次求亲,关公不允。后关公取襄阳,曹操屡次致书吴侯,使袭荆州;吴侯本不肯许,因吕蒙与关公不睦,故擅自兴兵,误成大事,今吴侯悔之不及。此乃吕蒙之罪,非吴侯之过也。今吕蒙已死,冤仇已息。孙夫人一向思归。今吴侯令臣为使,愿送归夫人,缚还降将,并将荆州仍旧交还,永结盟好,共灭曹丕,以正篡逆之罪。”先主怒曰:“汝东吴害了朕弟,今日敢以巧言来说乎!”瑾曰:“臣请以轻重大小之事,与陛下论之:陛下乃汉朝皇叔,今汉帝已被曹丕篡夺,不思剿除;却为异姓之亲,而屈万乘之尊:是舍大义而就小义也。中原乃海内之地,两都皆大汉创业之方,陛下不取,而但争荆州:是弃重而取轻也。天下皆知陛下即位,必兴汉室,恢复山河;今陛下置魏不问,反欲伐吴:窃为陛下不取。”先主大怒曰:“杀吾弟之仇,不共戴天!欲朕罢兵,除死方休!不看丞相之面,先斩汝首!今且放汝回去,说与孙权:洗颈就戮!”诸葛瑾见先主不听,只得自回江南。

却说张昭见孙权曰:“诸葛子瑜知蜀兵势大,故假以请和为辞,欲背吴入蜀。此去必不回矣。”权曰:“孤与子瑜,有生死不易之盟;孤不负子瑜,子瑜亦不负孤。昔子瑜在柴桑时,孔明来吴,孤欲使子瑜留之。子瑜曰:弟已事玄德,义无二心;弟之不留,犹瑾之不往。其言足贯神明。今日岂肯降蜀乎?孤与子瑜可谓神交,非外言所得间也。”正言间,忽报诸葛瑾回。权曰:“孤言若何?”张昭满面羞惭而退。瑾见孙权,言先主不肯通和之意。权大惊曰:“若如此,则江南危矣!”阶下一人进曰:“某有一计,可解此危。”视之,乃中大夫赵咨也。权曰:“德度有何良策?”咨曰:“主公可作一表,某愿为使,往见魏帝曹丕,陈说利害,使袭汉中,则蜀兵自危矣。”权曰:“此计最善。但卿此去,休失了东吴气象。”咨曰:“若有些小差失,即投江而死,安有面目见江南人物乎!”

权大喜,即写表称臣,令赵咨为使。星夜到了许都,先见太尉贾诩等并大小官僚。次日早朝,贾诩出班奏曰:“东吴遣中大夫赵咨上表。”曹丕笑曰:“此欲退蜀兵故也。”即令召入。咨拜伏于丹墀。丕览表毕,遂问咨曰:“吴侯乃何如主也:”咨曰:“聪明、仁智、雄略之主也。”丕笑曰:“卿褒奖毋乃太甚?”咨曰:“臣非过誉也。吴侯纳鲁肃于凡品,是其聪也;拔吕蒙于行阵,是其明也;获于禁而不害,是其仁也;取荆州兵不血刃,是其智也;据三江虎视天下,是其雄也;屈身于陛下,是其略也:以此论之,岂不为聪明、仁智、雄略之主乎?”丕又问曰:“吴主颇知学乎?”咨曰:“吴主浮江万艘,带甲百万,任贤使能,志存经略;少有余闲,博览书传,历观史籍,采其大旨,不效书生寻章摘句而已。”丕曰:“朕欲伐吴,可乎?”咨曰:“大国有征伐之兵,小国有御备之策。”丕曰:“吴畏魏乎?”咨曰:“带甲百万,江汉为池,何畏之有?”丕曰:“东吴如大夫者几人?”咨曰:“聪明特达者八九十人;如臣之辈,车载斗量,不可胜数。”丕叹曰:“使于四方,不辱君命,卿可以当之矣。”于是即降诏,命太常卿邢贞赍册封孙权为吴王,加九锡。赵咨谢恩出城。

大夫刘晔谏曰:“今孙权惧蜀兵之势,故来请降。以臣愚见:蜀、吴交兵,乃天亡之也;今若遣上将提数万之兵,渡江袭之,蜀攻其外,魏攻其内,吴国之亡,不出旬日。吴亡则蜀孤矣。陛下何不早图之?”丕曰:“孙权既以礼服朕,朕若攻之,是沮天下欲降者之心;不若纳之为是。”刘晔又曰:“孙权虽有雄才,乃残汉骠骑将军、南昌侯之职。官轻则势微,尚有畏中原之心;若加以王位,则去陛下一阶耳。今陛下信其诈降,崇其位号以封殖之,是与虎添翼也。”丕曰:“不然。朕不助吴,亦不助蜀。待看吴、蜀交兵,若灭一国,止存一国,那时除之,有何难哉?朕意已决,卿勿复言。”遂命太常卿邢贞同赵咨捧执册锡,径至东吴。

却说孙权聚集百官,商议御蜀兵之策。忽报魏帝封主公为王,礼当远接,顾雍谏曰:“主公宜自称上将军、九州伯之位,不当受魏帝封爵。”权曰:“当日沛公受项羽之封,盖因时也;何故却之?”遂率百官出城迎接。邢贞自恃上国天使,入门不下车。张昭大怒,厉声曰:“礼无不敬,法无不肃,而君敢自尊大,岂以江南无方寸之刃耶?”邢贞慌忙下车,与孙权相见,并车入城。忽车后一人放声哭曰:“吾等不能奋身舍命,为主并魏吞蜀,乃令主公受人封爵,不亦辱乎!”众视之,乃徐盛也。邢贞闻之,叹曰:“江东将相如此,终非久在人下者也!”却说孙权受了封爵,众文武官僚拜贺已毕,命收拾美玉明珠等物,遣人赍进谢恩。早有细作报说蜀主引本国大兵,及蛮王沙摩柯番兵数万,又有洞溪汉将杜路、刘宁二枝兵,水陆并进,声势震天。水路军已出巫口,旱路军已到秭归。时孙权虽登王位,奈魏主不肯接应,乃问文武曰:“蜀兵势大,当复如何?”众皆默然。权叹曰:“周郎之后有鲁肃,鲁肃之后有吕蒙,今吕蒙已亡,无人与孤分忧也!”言未毕,忽班部中一少年将,奋然而出,伏地奏曰:“臣虽年幼,颇习兵书。愿乞数万之兵,以破蜀兵。”权视之,乃孙桓也。桓字叔武,其父名河,本姓俞氏,孙策爱之,赐姓孙,因此亦系吴王宗族。河生四子,桓居其长,弓马熟娴,常从吴王征讨,累立奇功,官授武卫都尉;时年二十五岁。权曰:“汝有何策胜之?”桓曰:“臣有大将二员:一名李异,一名谢旌,俱有万夫不当之勇。乞数万之众,往擒刘备。”权曰:“侄虽英勇,争奈年幼;必得一人相助,方可。”虎威将军朱然出曰:“臣愿与小将军同擒刘备。”权许之,遂点水陆军五万,封孙桓为左都督,朱然为右都督,即日起兵。哨马探得蜀兵已至宜都下寨,孙桓引二万五千军马,屯于宜都界口,前后分作三营,以拒蜀兵。却说蜀将吴班领先锋之印,自出川以来,所到之处,望风而降,兵不血刃,直到宜都;探知孙桓在彼下寨,飞奏先主。时先主已到秭归,闻奏怒曰:“量此小儿,安敢与朕抗耶!”关兴奏曰:“既孙权令此子为将,不劳陛下遣大将,臣愿往擒之。”先主曰:“朕正欲观汝壮气。”即命关兴前往。兴拜辞欲行,张苞出曰:“既关兴前去讨贼,臣愿同行。”先主曰:“二侄同行甚妙,但须谨慎,不可造次。”

二人拜辞先主,会合先锋,一同进兵,列成阵势。孙桓听知蜀兵大至,合寨多起。两阵对圆,桓领李异、谢旌立马于门旗之下,见蜀营中,拥出二员大将,皆银盔银铠,白马白旗:上首张苞挺丈八点钢矛,下首关兴横着大砍刀。苞大骂曰:“孙桓竖子!死在临时,尚敢抗拒天兵乎!”桓亦骂曰:“汝父已作无头之鬼;今汝又来讨死,好生不智!”张苞大怒,挺枪直取孙桓。桓背后谢旌,骤马来迎。两将战有三十余合,旌败走,苞乘胜赶来。李异见谢旌败了,慌忙拍马轮蘸金斧接战。张苞与战二十余合,不分胜负。吴军中裨将谭雄,见张苞英勇,李异不能胜,却放一冷箭,正射中张苞所骑之马。那马负痛奔回本阵,未到门旗边,扑地便倒,将张苞掀在地上。李异急向前轮起大斧,望张苞脑袋便砍。忽一道红光闪处,李异头早落地,原来关兴见张苞马回,正待接应,忽见张苞马倒,李异赶来,兴大喝一声,劈李异于马下,救了张苞。乘势掩杀,孙桓大败。各自鸣金收军。

次日,孙桓又引军来。张苞、关兴齐出。关兴立马于阵前,单搦孙桓交锋。桓大怒,拍马轮刀,与关兴战三十余合,气力不加,大败回阵。二小将追杀入营,吴班引着张南、冯习驱兵掩杀。张苞奋勇当先,杀入吴军,正遇谢旌,被苞一矛刺死。吴军四散奔走。蜀将得胜收兵,只不见了关兴。张苞大惊曰:“安国有失,吾不独生!”言讫,绰枪上马。寻不数里,只见关兴左手提刀,右手活挟一将。苞问曰:“此是何人?”兴笑答曰:“吾在乱军中,正遇仇人,故生擒来。”苞视之,乃昨日放冷箭的谭雄也。苞大喜,同回本营,斩首沥血,祭了死马。遂写表差人赴先主处报捷。

孙桓折了李异、谢旌、谭雄等许多将士,力穷势孤,不能抵敌,即差人回吴求救。蜀将张南、冯习谓吴班曰:“目今吴兵势败,正好乘虚劫寨。”班曰:“孙桓虽然折了许多将士,朱然水军现今结营江上,未曾损折。今日若去劫寨,倘水军上岸,断我归路,如之奈何?”南曰:“此事至易:可教关、张二将军,各引五千军伏于山谷中;如朱然来救,左右两军齐出夹攻,必然取胜。”班曰:“不如先使小卒诈作降兵,却将劫寨事告与朱然;然见火起,必来救应,却令伏兵击之,则大事济矣。”冯习等大喜,遂依计而行。

却说朱然听知孙桓损兵折将,正欲来救,忽伏路军引几个小卒上船投降。然问之,小卒曰:“我等是冯习帐下士卒,因赏罚不明,待来投降,就报机密。”然曰:“所报何事?”小卒曰:“今晚冯习乘虚要劫孙将军营寨,约定举火为号。”朱然听毕,即使人报知孙桓。报事人行至半途,被关兴杀了。朱然一面商议,欲引兵去救应孙桓。部将崔禹曰:“小卒之言,未可深信。倘有疏虞,水陆二军尽皆休矣。将军只宜稳守水寨,某愿替将军一行。”然从之,遂令崔禹引一万军前去。是夜,冯习、张南、吴班分兵三路,直杀入孙桓寨中,四面火起,吴兵大乱,寻路奔走。

且说崔禹正行之间,忽见火起,急催兵前进。刚才转过山来,忽山谷中鼓声大震:左边关兴,右边张苞,两路夹攻。崔禹大惊,方欲奔走,正遇张苞;交马只一合,被苞生擒而回。朱然听知危急,将船往下水退五六十里去了。孙桓引败军逃走,问部将曰:“前去何处城坚粮广?”部将曰:“此去正北彝陵城,可以屯兵。”桓引败军急望彝陵而走。方进得城,吴班等追至,将城四面围定。关兴、张苞等解崔禹到秭归来。先主大喜,传旨将崔禹斩却,大赏三军。自此威风震动,江南诸将无不胆寒。

却说孙桓令人求救于吴王,吴王大惊,即召文武商议曰:“今孙桓受困于彝陵,朱然大败于江中,蜀兵势大,如之奈何?”张昭奏曰:“今诸将虽多物故,然尚有十余人,何虑于刘备?可命韩当为正将,周泰为副将,潘璋为先锋,凌统为合后,甘宁为救应,起兵十万拒之。”权依所奏,即命诸将速行。此时甘宁已患痢疾,带病从征。

却说先主从巫峡建平起,直接彝陵界分,七十余里,连结四十余寨;见关兴、张苞屡立大功,叹曰:“昔日从朕诸将,皆老迈无用矣;复有二侄如此英雄,朕何虑孙权乎!”正言间,忽报韩当、周泰领兵来到。先主方欲遣将迎敌,近臣奏曰:“老将黄忠,引五六人投东吴去了。”先主笑曰:“黄汉升非反叛之人也;因朕失口误言老者无用,彼必不服老,故奋力去相持矣。”即召关兴、张苞曰:“黄汉升此去必然有失。贤侄休辞劳苦,可去相助。略有微功,便可令回,勿使有失。”二小将拜辞先主,引本部军来助黄忠。正是:老臣素矢忠君志,年少能成报国功。未知黄忠此去如何,且看下文分解。