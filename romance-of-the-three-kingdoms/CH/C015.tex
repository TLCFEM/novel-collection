\chapter{太史慈酣斗小霸王~孙伯符大战严白虎}

却说张飞拔剑要自刎,玄德向前抱住,夺剑掷地曰:“古人云:‘兄弟如手足,妻子如衣服。衣服破,尚可缝;手足断,安可续?’吾三人桃园结义,不求同生,但愿同死。今虽失了城池家小,安忍教兄弟中道而亡?况城池本非吾有;家眷虽被陷,吕布必不谋害,尚可设计救之。贤弟一时之误,何至遽欲捐生耶!”说罢大哭。关、张俱感泣。

且说袁术知吕布袭了徐州,星夜差人至吕布处,许以粮五万斛、马五百匹、金银一万两、彩缎一千匹,使夹攻刘备。布喜,令高顺领兵五万袭玄德之后。玄德闻得此信,乘阴雨撤兵,弃盱眙而走,思欲东取广陵。比及高顺军来,玄德已去。高顺与纪灵相见,就索所许之物。灵曰:“公且回军,容某见主公计之。”高顺乃别纪灵回军,见吕布具述纪灵语。布正在迟疑,忽有袁术书至。书意云:“高顺虽来,而刘备未除;且待捉了刘备,那时方以所许之物相送。”布怒骂袁术失信,欲起兵伐之。陈宫曰:“不可。术据寿春,兵多粮广,不可轻敌。不如请玄德还屯小沛,使为我羽翼。他日令玄德为先锋,那时先取袁术,后取袁绍,可纵横天下矣。”布听其言,令人赍书迎玄德回。却说玄德引兵东取广陵,被袁术劫寨,折兵大半。回来正遇吕布之使,呈上书札,玄德大喜。关、张曰:“吕布乃无义之人,不可信也。”玄德曰:“彼既以好情待我,奈何疑之!”遂来到徐州。布恐玄德疑惑,先令人送还家眷。甘、麋二夫人见玄德,具说吕布令兵把定宅门。禁诸人不得入;又常使侍妾送物,未尝有缺。玄德谓关、张曰:“我知吕布必不害我家眷也。”乃入城谢吕布。张飞恨吕布,不肯随往,先奉二嫂往小沛去了。玄德入见吕布拜谢。吕布曰:“我非欲夺城;因令弟张飞在此恃酒杀人,恐有失事,故来守之耳。”玄德曰:“备欲让兄久矣。”布假意仍让玄德。玄德力辞,还屯小沛住扎。关、张心中不忿。玄德曰:“屈身守分,以待天时,不可与命争也。”吕布令人送粮米缎匹。自此两家和好,不在话下。

却说袁术大宴将士于寿春。人报孙策征庐江太守陆康,得胜而回。术唤策至,策拜于堂下。问劳已毕,便令侍坐饮宴。原来孙策自父丧之后,退居江南,礼贤下士;后因陶谦与策母舅丹阳太守吴景不和,策乃移母并家属居于曲阿,自己却投袁术。术甚爱之,常叹曰:“使术有子如孙郎,死复何恨!”因使为怀义校尉,引兵攻泾县大帅祖郎得胜。术见策勇,复使攻陆康,今又得胜而回。

当日筵散,策归营寨。见术席间相待之礼甚傲,心中郁闷,乃步月于中庭。因思父孙坚如此英雄,我今沦落至此,不觉放声大哭。忽见一人自外而入,大笑曰:“伯符何故如此?尊父在日,多曾用我。君今有不决之事,何不问我,乃自哭耶!”策视之,乃丹阳故鄣人,姓朱,名治,字君理,孙坚旧从事官也。策收泪而延之坐曰:“策所哭者,恨不能继父之志耳。”治曰:“君何不告袁公路,借兵往江东,假名救吴景,实图大业,而乃久困于人之下乎?”正商议间,一人忽入曰:“公等所谋,吾已知之。吾手下有精壮百人,暂助伯符一马之力。”策视其人,乃袁术谋士,汝南细阳人,姓吕,名范,字子衡。策大喜,延坐共议。吕范曰:“只恐袁公路不肯借兵。”策曰:“吾有亡父留下传国玉玺,以为质当。”范曰:“公路款得此久矣!以此相质,必肯发兵。”三人计议已定。次日,策入见袁术,哭拜曰:“父仇不能报,今母舅吴景,又为扬州刺史刘繇所逼;策老母家小,皆在曲阿,必将被害。策敢借雄兵数千,渡江救难省亲。恐明公不信,有亡父遗下玉玺,权为质当。”术闻有玉玺,取而视之,大喜曰:“吾非要你玉玺,今且权留在此。我借兵三千、马五百匹与你。平定之后,可速回来。你职位卑微,难掌大权。我表你为折冲校尉、殄寇将军,克日领兵便行。”策拜谢,遂引军马,带领朱治、吕范、旧将程普、黄盖、韩当等,择日起兵。

行至历阳,见一军到。当先一人,姿质风流,仪容秀丽,见了孙策,下马便拜。策视其人,乃庐江舒城人,姓周,名瑜,字公瑾。原来孙坚讨董卓之时,移家舒城,瑜与孙策同年,交情甚密,因结为昆仲。策长瑜两月,瑜以兄事策。瑜叔周尚,为丹阳太守;今往省亲,到此与策相遇。策见瑜大喜,诉以衷情。瑜曰:“某愿施犬马之力,共图大事。”策喜曰:“吾得公瑾,大事谐矣!”便令与朱治、吕范等相见。瑜谓策曰:“吾兄欲济大事,亦知江东有二张乎?”策曰:“何为二张?”瑜曰:“一人乃彭城张昭,字子布;一人乃广陵张纮,字子纲。二人皆有经天纬地之才,因避乱隐居于此。吾兄何不聘之?”策喜,即便令人赍礼往聘,俱辞不至。策乃亲到其家,与语大悦,力聘之,二人许允。策遂拜张昭为长史,兼抚军中郎将;张纮为参谋正议校尉:商议攻击刘繇。

却说刘繇字正礼,东莱牟平人也,亦是汉室宗亲,太尉刘宠之侄,兖州刺史刘岱之弟;旧为扬州刺史,屯于寿春,被袁术赶过江屯,故来曲阿。当下闻孙策兵至,急聚众将商议。部将张英曰:“某领一军屯于牛渚,纵有百万之兵,亦不能近。”言未毕,帐下一人高叫曰:“某愿为前部先锋!”众视之,乃东莱黄县人太史慈也。慈自解了北海之围后,便来见刘繇,繇留于帐下。当日听得孙策来到,愿为前部先锋。繇曰:“你年尚轻,未可为大将,只在吾左右听命。”太史慈不喜而退。张英领兵至牛渚,积粮十万于邸阁。孙策引兵到,张英出迎,两军会于牛渚滩上。孙策出马,张英大骂,黄盖便出与张英战。不数合,忽然张英军中大乱,报说寨中有人放火。张英急回军。孙策引军前来,乘势掩杀。张英弃了牛渚,望深山而逃。原来那寨后放火的,只是两员健将:一人乃九江寿春人,姓蒋,名钦,字公奕;一人乃九江下蔡人,姓周,名泰,字幼平。二人皆遭世乱,聚人在洋子江中,劫掠为生;久闻孙策为江东豪杰,能招贤纳士,故特引其党三百余人,前来相投。策大喜,用为军前校尉。收得牛渚邸阁粮食、军器,并降卒四千余人,遂进兵神亭。却说张英败回见刘繇,繇怒欲斩之。谋士笮融、薛礼劝免,使屯兵零陵城拒敌。繇自领兵于神亭岭南下营,孙策于岭北下营。策问土人曰:“近山有汉光武庙否?”土人曰:“有庙在岭上。”策曰:“吾夜梦光武召我相见,当往祈之。”长史张昭曰:“不可。岭南乃刘繇寨,倘有伏兵,奈何?”策曰:“神人佑我,吾何惧焉!”遂披挂绰枪上马,引程普、黄盖、韩当、蒋钦、周泰等共十三骑,出寨上岭,到庙焚香。下马参拜已毕,策向前跪祝曰:“若孙策能于江东立业,复兴故父之基,即当重修庙宇,四时祭祀。”祝毕,出庙上马,回顾众将曰:“吾欲过岭,探看刘繇寨栅。”诸将皆以为不可。策不从,遂同上岭,南望村林。早有伏路小军飞报刘繇,繇曰:“此必是孙策诱敌之计,不可追之。”太史慈踊跃曰:“此时不捉孙策,更待何时!”遂不候刘繇将令,竟自披挂上马,绰枪出营,大叫曰:“有胆气者,都跟我来!”诸将不动。惟有一小将曰:“太史慈真猛将也!吾可助之!”拍马同行。众将皆笑。

却说孙策看了半晌,方始回马。正行过岭,只听得岭上叫:“孙策休走!”策回头视之,见两匹马飞下岭来。策将十三骑一齐摆开。策横枪立马于岭下待之。太史慈高叫曰:“那个是孙策?”策曰:“你是何人?”答曰:“我便是东莱太史慈也,特来捉孙策!”策笑曰:“只我便是。你两个一齐来并我一个,我不惧你!我若怕你,非孙信符也!”慈曰:“你便众人都来,我亦不怕!”纵马横枪,直取孙策。策挺枪来迎。两马相交,战五十合,不分胜负。程普等暗暗称奇。慈见孙策枪法无半点儿渗漏,乃佯输诈败,引孙策赶来。慈却不由旧路上岭,竟转过山背后。策赶来,大喝曰:“走的不算好汉!”慈心中自付:“这厮有十二从人,我只一个,便活捉了他,也吃众人夺去。再引一程,教这厮没寻处,方好下手。”于是且战且走。策那里肯舍,一直赶到平川之地。慈兜回马再战,又到五十合。策一枪搠去,慈闪过,挟住枪;慈也一枪搠去,策亦闪过,挟住枪。两个用力只一拖,都滚下马来。马不知走的那里去了。两个弃了枪,揪住厮打,战袍扯得粉碎。策手快,掣了太史慈背上的短戟,慈亦掣了策头上的兜鍪。策把戟来刺慈,慈把兜鍪遮架。忽然喊声后起,乃刘繇接应军到来,约有千余。策正慌急,程普等十二骑亦冲到。策与慈方才放手。慈于军中讨了一匹马,取了枪,上马复来。孙策的马却是程普收得,策亦取枪上马。刘繇一千余军,和程普等十二骑混战,逶迤杀到神亭岭下。喊声起处,周瑜领军来到。刘繇自引大军杀下岭来。时近黄昏,风雨暴至,两下各自收军。次日,孙策引军到刘繇营前,刘繇引军出迎。两阵圆处,孙策把枪挑太史慈的小戟于阵前,令军士大叫曰:“太史慈若不是走的快,已被刺死了!”太史慈亦将孙策兜鍪挑于阵前,也令军士大叫曰:“孙策头已在此!”两军呐喊,这边夸胜,那边道强。太史慈出马,要与孙策决个胜负,策遂欲出。程普曰:“不须主公劳力,某自擒之。”程普出到阵前,太史慈曰:“你非我之敌手,只教孙策出马来!”程普大怒,挺枪直取太史慈。两马相交,战到三十合,刘繇急鸣金收军。太史慈曰:“我正要捉拿贼将,何故收军?”刘繇曰:“人报周瑜领军袭取曲阿,有庐江松滋人陈武,字子烈,接应周瑜入去。吾家基业已失,不可久留。速往秣陵,会薛礼、笮融军马,急来接应。”太史慈跟着刘繇退军,孙策不赶,收住人马。长史张昭曰:“彼军被周瑜袭取曲阿,无恋战之心,今夜正好劫营。”孙策然之。当夜分军五路,长驱大进。刘繇军兵大败,众皆四纷五落。太史慈独力难当,引十数骑连夜投泾县去了。

却说孙策又得陈武为辅,其人身长七尺,面黄睛赤,形容古怪。策甚敬爱之,拜为校尉,使作先锋,攻薛札。武引十数骑突入阵去,斩首级五十余颗。薛札闭门不敢出。策正攻城,忽有人报刘繇会合笮融去取牛渚。孙策大怒,自提大军竟奔牛渚。刘繇,笮融二人出马迎敌。孙策曰:“吾今到此,你如何不降?”刘繇背后一人挺枪出马,乃部将于糜也,与策战不三合,被策生擒过去,拨马回阵。繇将樊能,见捉了于糜。挺枪来赶。那枪刚搠到策后心,策阵上军士大叫:“背后有人暗算!”策回头,怨见樊能马到,乃大喝一声,声如巨雷。樊能惊骇,倒翻身撞下马来,破头而死。策到门旗下,将于糜丢下,已被挟死。一霎时挟死一将,喝死一将:自此人皆呼孙策为“小霸王”。当日刘繇兵大败,人马大半降策。策斩首级万余。刘繇与笮融走豫章投刘表去了。孙策还兵复攻秣陵,亲到城壕边,招谕薛礼投降。城上暗放一冷箭,正中孙策左腿,翻身落马,众将急救起,还营拔箭,以金疮药傅之。策令军中诈称主将中箭身死。军中举哀。拔寨齐起。葬礼听知孙策已死,连夜起城内之军,与骁将张英、陈横杀出城来追之。忽然伏兵四起,孙策当先出马,高声大叫曰:“孙郎在此!”众军皆惊,尽弃枪习,拜于地下。策令休杀一人。张英拨马回走,被陈武一枪刺死。陈横被蒋钦一箭射死。薛礼死于乱军中。策入秣陵,安辑居民;移兵至泾县来捉太史慈。

却说太史慈招得精壮二千余人,并所部兵,正要来与刘繇报仇。孙策与周瑜商议活捉太史慈之计。瑜令三面攻县,只留东门放走;离城二十五里,三路各伏一军,太史慈到那里,人困马乏,必然被擒。原来太史慈所招军大半是山野之民,不谙纪律。泾县城头,苦不甚高。当夜孙策命陈武短衣持刀,首先爬上城放火。太史慈见城上火起,上马投东门走,背后孙策引军赶来。太史慈正走,后军赶至三十里,却不赶了。太史慈走了五十里,人困马乏,芦苇之中,喊声忽起。慈急待走,两下里绊马索齐来,将马绊翻了,生擒太史慈,解投大寨。策知解到太史慈,亲自出营喝散士卒,自释其缚,将自己锦袍衣之,请入寨中,谓曰:“我知子义真丈夫也。刘繇蠢辈,不能用为大将,以致此败。”慈见策待之甚厚,遂请降。

策执慈手笑曰:“神亭相战之时,若公获我,还相害否?”慈笑曰:“未可知也。”策大笑,请入帐,邀之上坐,设宴款待。慈曰:“刘君新破,士卒离心。某欲自往收拾余众,以助明公。不识能相信否?”策起谢曰:“此诚策所愿也。今与公约:明日日中,望公来还。”慈应诺而去。诸终曰:“太史慈此去必不来矣。”策曰:“子义乃信义之士,必不背我。”众皆未信。次日,立竿于营门以候日影。恰将日中,太史慈引一千余众到寨。孙策大喜。众皆服策之知人。于是孙策聚数万之众,下江东,安民恤众,投者无数。江东之民,皆呼策为“孙郎”。但闻孙郎兵至,皆丧胆而走。及策军到,并不许一人掳掠,鸡犬不惊,人民皆悦,赍牛酒到寨劳军。策以金帛答之,欢声遍野。其刘繇旧军,愿从军者听从,不愿为军者给赏归农。江南之民,无不仰颂。由是兵势大盛。策乃迎母叔诸弟俱归曲阿,使弟孙权与周泰守宣城。策领兵南取吴郡。

时有严白虎,自称东吴德王,据吴郡,遣部将守住乌程、嘉兴。当日白虎闻策兵至,令弟严舆出兵,会于枫桥。舆横刀立马于桥上。有人报入中军,策便欲出。张纮谏曰:“夫主将乃三军之所系命,不宜轻敌小寇。愿将军自重。”策谢曰:“先生之言如金石;但恐不亲冒矢石,则将士不用命耳。”随遣韩当出马。比及韩当到桥上时,蒋钦、陈武早驾小舟从河岸边杀过桥里。乱箭射倒岸上军,二人飞身上岸砍杀。严舆退走。韩当引军直杀到阊门下,贼退入城里去了。

策分兵水陆并进,围住吴城。一困三日,无人出战。策引众军到阊门外招谕。城上一员裨将,左手托定护梁,右手指着城下大骂。太史慈就马上拈弓取箭,顾军将曰:“看我射中这厮左手!”说声未绝,弓弦响处,果然射个正中,把那将的左手射透,反牢钉在护梁上。城上城下人见者,无不喝采。众人救了这人下城。白虎大惊曰:“彼军有如此人,安能敌乎!”遂商量求和。次日,使严舆出城,来见孙策。策请舆入帐饮酒。酒酣,问舆曰:“令兄意欲如何?”舆曰:“欲与将军平分江东。”策大怒曰:“鼠辈安敢与吾相等!”命斩严舆。舆拨剑起身,策飞剑砍之,应手而倒,割下首级,令人送入城中。白虎料敌不过,弃城而走。策进兵追袭,黄盖攻取嘉兴,太史慈攻取乌程,数州皆平。白虎奔余杭,于路劫掠,被土人凌操领乡人杀败,望会稽而走。凌操父子二人来接孙策,策使为从征校尉,遂同引兵渡江。严白虎聚寇,分布于西津渡口。程普与战,复大败之,连夜赶到会稽。会稽太守王朗,欲引兵救白虎。忽一人出曰:“不可。孙策用仁义之师,白虎乃暴虐之众,还宜擒白虎以献孙策。”朗视之,乃会稽余姚人,姓虞,名翻,字仲翔,现为郡吏。朗怒叱之,翻长叹而出。朗遂引兵会合白虎,同陈兵于山阴之野。两阵对圆,孙策出马,谓王朗曰:“吾兴仁义之兵,来安浙江,汝何故助贼?”朗骂曰:“汝童心不足!既得吴郡,而又强并吾界!今日特与严氏雪仇!”孙策大怒,正待交战,太史慈早出。王朗拍马舞刀,与慈战不数合,朗将周听,杀出助战;孙策阵中黄盖,飞马接住周听交锋。两下鼓声大震,互相鏖战。忽王朗阵后先乱,一彪军从背后抄来。朗大惊,急回马来迎:原来是周瑜与程普引军刺斜杀来,前后夹攻,王朗寡不敌众,与白虎、周听杀条血路,走入城中,拽起吊桥,坚闭城门。孙策大军乘势赶到城下。分布众军,四门攻打。

王朗在城中见孙策攻城甚急,欲再出兵决一死战。严白虎曰:“孙策兵势甚大,足下只宜深沟高垒,坚壁勿出。不消一月,彼军粮尽。自然退走。那时乘虚掩之,可不战而破也。”朗依其议,乃固守会稽城而不出。孙策一连攻了数日,不能成功,乃与众将计议。孙静曰:“王朗负固守城,难可卒拔。会稽钱粮,大半屯于查渎;其地离此数十里,莫若以兵先据其内:所谓攻其无备,出其不意也。”策大喜曰:“叔父妙用,足破贼人矣!”即下令于各门燃火,虚张旗号,设为疑兵,连夜撤围南去。周瑜进曰:“主公大兵一起,王朗必然出城来赶,可用奇兵胜之。”策曰:“吾今准备下了,取城只在今夜。”遂令军马起行。却说王朗闻报孙策军马退去,自引众人来敌楼上观望;见城下烟火并起,旌旗不杂,心下迟疑。周听曰:“孙策走矣,特设此计以疑我耳。可出兵袭之。”严白虎曰:“孙策此去,莫非要去查渎?我令部兵与周将军追之。”朗曰:“查渎是我屯粮之所,正须提防。汝引兵先行,吾随后接应。”白虎与周听领五千兵出城追赶。将近初更,离城二十余里,忽密林里一声鼓响,火把齐明。白虎大惊,便勒马回走,一将当先拦住,火光中视之,乃孙策也。周听舞刀来迎,被策一枪刺死。余众皆降。白虎杀条血路,望余杭而走。王朗听知前军已败,不敢入城,引部下奔遍海隅去了。孙策复回大军,乘势取了城池,安定人民。不隔一日,只见一人将着严白虎首级来孙策军前投献。策视其人,身长八尺,面方口阔。问其姓名,乃会稽余姚人,姓董,名袭,字元代。策喜,命为别部司马。自是东路皆平,令叔孙静守之,令朱治为吴郡太守,收军回江东。却说孙权与周泰守宣城,忽山贼窃发,四面杀至。时值更深,不及抵敌,泰抱权上马。数十贼众,用刀来砍。泰赤体步行,提刀杀贼,砍杀十余人。随后一贼跃马挺枪直取周泰,被泰扯住枪,拖下马来,夺了枪马,杀条血路。救出孙权。会贼远重。周泰身被十二枪,金疮发胀,命在须臾。策闻之大惊。帐下董袭曰:“某曾与海寇相持,身遭数枪,得会稽一个贤郡吏虞翻荐一医者,半月而愈。”策曰:“虞翻莫非虞仲翔乎?”袭曰:“然。”策曰:“此贤士也。我当用之。”乃令张昭与董袭同往聘请虞翻。翻至,策优礼相待,拜为攻曹,因言及求医之意。翻曰:“此人乃沛国谯郡人,姓华,名佗,字元化。真当世之神医也。当引之来见。”不一日引至。策见其人,童颜鹤发,飘然有出世之姿。乃待为上宾,请视周泰疮。佗曰:“此易事耳。”投之以药,一月而愈。策大喜,厚谢华佗。遂进兵杀除山贼。江南皆平。孙策分拨将士,守把各处隘口,一面写表申奏朝廷;一面结交曹操,一面使人致书与袁术取玉玺。却说袁术暗有称帝之心,乃回书推托不还;急聚长史杨大将,都督张勋、纪灵、桥蕤,上将雷薄、陈芬等三十余人商议,曰:“孙策借我军马起事,今日尽得江东地面;乃不思根本,而反来索玺,殊为无礼。当以何策图之?”长史杨大将曰:“孙策据长江之险,兵精粮广,未可图也。今当先伐刘备,以报前日无故相攻之恨,然后图取孙策未迟。某献一计,使备即日就擒。”正是:不去江东图虎豹,却来徐郡斗蛟龙。不知其计若何,且听下文分解。