\chapter{姜伯约归降孔明~武乡侯骂死王朗}

却说姜维献计于马遵曰:“诸葛亮必伏兵于郡后,赚我兵出城,乘虚袭我。某愿请精兵
三千,伏于要路。太守随后发兵出城,不可远去,止行三十里便回;但看火起为号,前后来
攻,可获大胜。如诸葛亮自来,必为某所擒矣。”遵用其计,付精兵与姜维去讫,然后自与
梁虔引兵出城等候;只留梁绪、尹赏守城。原来孔明果遣赵云引一军埋伏于山僻之中,只待
天水人马离城,便乘虚袭之。当日细作回报赵云,说天水太守马遵,起兵出城,只留文官守
城。赵云大喜,又令人报与张翼、高翔,教于要路截杀马遵。此二处兵亦是孔明预先埋伏。
却说赵云引五千兵,径投天水郡城下,高叫曰:“吾乃常山赵子龙也!汝知中计,早献城
池,免遭诛戮!”城上梁绪大笑曰:“汝中吾姜伯约之计,尚然不知耶?”云恰待攻城,忽
然喊声大震,四面火光冲天。当先一员少年将军,挺枪跃马而言曰:“汝见天水姜伯约
乎!”云挺枪直取姜维。战不数合,维精神倍长。云大惊,暗忖曰:“谁想此处有这般人
物!”正战时,两路军夹攻来,乃是马遵、梁虔引军杀回。赵云首尾不能相顾,冲开条路,
引败兵奔走,姜维赶来。亏得张翼、高翔两路军杀出,接应回去。

赵云归见孔明,说中了敌人之计。孔明惊问曰:“此是何人,识吾玄机?”有南安人告
曰:“此人姓姜名维,字伯约,天水冀人也;事母至孝,文武双全,智勇足备,真当世之英
杰也。”赵云又夸奖姜维枪法,与他人大不同。孔明曰:“吾今欲取天水,不想有此人。”
遂起大军前来。

却说姜维回见马遵曰:“赵云败去,孔明必然自来。彼料我军必在城中。今可将本部军
马,分为四枝:某引一军伏于城东,如彼兵到则截之。太守与梁虚、尹赏各引一军城外埋
伏。梁绪率百姓在城上守御。”分拨已定。

却说孔明因虑姜维,自为前部,望天水郡进发。将到城边,孔明传令曰:“凡攻城池,
以初到之日,激励三军,鼓噪直上。若迟延日久,锐气尽隳,急难破矣。”于是大军径到城
下。因见城上旗帜整齐,未敢轻攻。候至半夜,忽然四下火光冲天,喊声震地,正不知何处
兵来。只见城上亦鼓噪呐喊相应,蜀兵乱窜。孔明急上马,有关兴;张苞二将保护,杀出重
围。回头看时,正东上军马,一带火光,势若长蛇。孔明令关兴探视,回报曰:“此姜维兵
也。”孔明叹曰:“兵不在多,在人之调遣耳。此人真将才也!”收兵归寨,思之良久,乃
唤安定人问曰:“姜维之母,现在何处?”答曰:“维母今居冀县。”孔明唤魏延分付曰:
“汝可引一军,虚张声势,诈取冀县。若姜维到,可放入城。”又问:“此地何处紧要?”
安定人曰:“天水钱粮,皆在上邽;若打破上邽,则粮道自绝矣。”孔明大喜,教赵云引一
军去攻上邽。孔明离城三十里下寨。早有人报入天水郡,说蜀兵分为三路:一军守此郡,一
军取上邽,一军取冀城。姜维闻之,哀告马遵曰:“维母现在冀城,恐母有失。维乞一军往
救此城,兼保老母。”马遵从之,遂令姜维引三千军去保冀城;梁虔引三千军去保上邽。

却说姜维引兵至冀城,前面一彪军摆开,为首蜀将,乃是魏延。二将交锋数合,延诈败
奔走。维入城闭门,率兵守护,拜见老母,并不出战。赵云亦放过梁虎入上邽城去了。孔明
乃令人去南安郡,取夏侯楙至帐下。孔明曰:“汝惧死乎?”楙慌拜伏乞命。孔明曰:“目
今天水姜维现守冀城,使人持书来说:但得驸马在,我愿归降。吾今饶汝性命,汝肯招安姜
维否?”楙曰:“情愿招安。”孔明乃与衣服鞍马,不令人跟随,放之自去。楙得脱出寨,
欲寻路而走,奈不知路径。正行之间,逢数人奔走。楙问之,答曰:“我等是冀县百姓;今
被姜维献了城池,归降诸葛亮,蜀将魏延纵火劫财,我等因此弃家奔走,投上邽去也。”楙
又问曰:“今守天水城是谁?”土人曰:“天水城中乃马太守也。”楙闻之,纵马望天水而
行。又见百姓携男抱女远来,所说皆同。

楙至天水城下叫门,城上人认得是夏侯楙,慌忙开门迎接。马遵惊拜问之。楙细言姜维
之事;又将百姓所言说了。遵叹曰:“不想姜维反投蜀矣!”梁绪曰:“彼意欲救都督,故
以此言虚降。”楙曰:“今维已降,何为虚也?”正踌躇间,时已初更,蜀兵又来攻城。火
光中见姜维在城下挺枪勒马,大叫曰:“请夏侯都督答话!”夏侯楙与马遵等皆到城上,见
姜维耀武扬威大叫曰:“我为都督而降,都督何背前言?”楙曰:“汝受魏恩,何故降蜀?
有何前言耶?”维应曰:“汝写书教我降蜀,何出此言?汝要脱身,却将我陷了?我今降
蜀,加为上将,安有还魏之理?”言讫,驱兵打城,至晓方退。原来夜间妆姜维者,乃孔明
之计,令部卒形貌相似者,假扮姜维攻城,因火光之中,不辨真伪。

孔明却引兵来攻冀城。城中粮少,军食不敷。姜维在城上,见蜀军大车小辆,搬运粮
草,入魏延寨中去了。维引三千兵出城,径来劫粮。蜀兵尽弃了粮车,寻路而走。姜维夺得
粮车,欲要入城,忽然一彪军拦住,为首蜀将张翼也。二将交锋,战不数合,王平引一军又
到,两下夹攻。维力穷抵敌不住,夺路归城;城上早插蜀兵旗号:原来已被魏延袭了。维杀
条路奔天水城,手下尚有十余骑;又遇张苞杀了一阵,维止剩得匹马单枪,来到天水城下叫
门。城上军见是姜维,慌报马遵。遵曰:“此是姜维来赚我城门也。”令城上乱箭射下。姜
维回顾蜀兵至近,遂飞奔上邽城来。城上梁虔见了姜维,大骂曰:“反国之贼,安敢来赚我
城池!吾已知汝降蜀矣!”遂乱箭射下。姜维不能分说,仰天长叹,两眼泪流,拨马望长安
而走。行不数里,前至一派大树茂林之处,一声喊起,数千兵拥出:为首蜀将关兴,截住去
路。

维人困马乏,不能抵当,勒回马便走。忽然一辆小车从山坡中转出。其人头戴纶巾,身
披鹤氅,手摇羽扇,乃孔明也。孔明唤姜维曰:“伯约此时何尚不降?”维寻思良久,前有
孔明,后有关兴,又无去路,只得下马投降。孔明慌忙下车而迎,执维手曰:“吾自出茅庐
以来,遍求贤者,欲传授平生之学,恨未得其人。今遇伯约,吾愿足矣。”维大喜拜谢。

孔明遂同姜维回寨,升帐商议取天水、上邽之计。维曰:“天水城中尹赏、梁绪,与某
至厚;当写密书二封,射入城中,使其内乱,城可得矣。”孔明从之。姜维写了二封密书,
拴在箭上,纵马直至城下,射入城中。小校拾得,呈与马遵。遵大疑,与夏侯楙商议曰:
“梁绪、尹赏与姜维结连,欲为内应,都督宜早决之。”楙曰:“可杀二人。”尹赏知此消
息,乃谓梁绪曰:“不如纳城降蜀,以图进用。”是夜,夏侯楙数次使人请梁、尹二人说
话。二人料知事急,遂披挂上马,各执兵器,引本部军大开城门,放蜀兵入。夏侯楙、马遵
惊慌,引数百人出西门,弃城投羌胡城而去。梁绪、尹赏迎接孔明入城。安民已毕,孔明问
取上邽之计。梁绪曰:“此城乃某亲弟梁虚守之,愿招来降。”孔明大喜。绪当日到上都唤
梁虔出城来降孔明。孔明重加赏劳,就令梁绪为天水太守,尹赏为冀城令,梁虔为上邽令。
孔明分拨已毕,整兵进发。诸将问曰:“丞相何不去擒夏侯楙?”孔明曰:“吾放夏侯楙,
如放一鸭耳。今得伯约,得一凤也!”孔明自得三城之后,威声大震,远近州郡,望风归
降。孔明整顿军马,尽提汉中之兵,前出祁山,兵临渭水之西。细作报入洛阳。时魏主曹睿
太和元年,升殿设朝。近臣奏曰:“夏侯驸马已失三郡,逃窜羌中去了。今蜀兵已到祁山,
前军临渭水之西,乞早发兵破敌。”睿大惊,乃问群臣曰:“谁可为朕退蜀兵耶?”司徒王
朗出班奏曰:“臣观先帝每用大将军曹真,所到必克;今陛下何不拜为大都督,以退蜀
兵?”睿准奏,乃宣曹真曰:“先帝托孤与卿,今蜀兵入寇中原,卿安忍坐视乎?”真奏
曰:“臣才疏智浅,不称其职。”王朗曰:“将军乃社稷之臣,不可固辞。老臣虽驽钝,愿
随将军一往。”真又奏曰:“臣受大恩,安敢推辞?但乞一人为副将。”睿曰:“卿自举
之。”真乃保太原阳曲人,姓郭,名淮,字伯济,官封射亭侯,领雍州刺史。睿从之,遂拜
曹真为大都督,赐节钺;命郭淮为副都督,王朗为军师。朗时年已七十六岁矣。选拨东西二
京军马二十万与曹真。真命宗弟曹遵为先锋,又命荡寇将军朱赞为副先锋。当年十一月出
师,魏主曹睿亲自送出西门之外方回。曹真领大军来到长安,过渭河之西下寨。真与王朗、
郭淮共议退兵之策。朗曰:“来日可严整队伍,大展旌旗。老夫自出,只用一席话,管教诸
葛亮拱手而降,蜀兵不战自退。”真大喜,是夜传令:来日四更造饭,平明务要队伍整齐,
人马威仪,旌旗鼓角,各按次序。当时使人先下战书。次日,两军相迎,列成阵势于祁山之
前。蜀军见魏兵甚是雄壮,与夏侯楙大不相同。三军鼓角已罢,司徒王朗乘马而出。上首乃
都督曹真,下首乃副都督郭淮;两个先锋压住阵角。探子马出军前,大叫曰:“请对阵主将
答话!”只见蜀兵门旗开处,关兴、张苞分左右而出,立马于两边;次后一队队骁将分列;
门旗影下,中央一辆四轮车,孔明端坐车中,纶巾羽扇,素衣皂绦,飘然而出。孔明举目见
魏阵前三个麾盖,旗上大书姓名:中央白髯老者,乃军师、司徒王朗。孔明暗忖曰:“王朗
必下说词,吾当随机应之。”遂教推车出阵外,令护军小校传曰:“汉丞相与司徒会话。”
王朗纵马而出。孔明于车上拱手,朗在马上欠身答礼。朗曰:“久闻公之大名,今幸一会。
公既知天命、识时务,何故兴无名之兵?”孔明曰:“吾奉诏讨贼,何谓无名?”朗曰:
“天数有变,神器更易,而归有德之人,此自然之理也。曩自桓、灵以来,黄巾倡乱,天下
争横。降至初平、建安之岁,董卓造逆,傕、汜继虐;袁术僭号于寿春,袁绍称雄于邺土;
刘表占据荆州,吕布虎吞徐郡:盗贼蜂起,奸雄鹰扬,社稷有累卵之危,生灵有倒悬之急。
我太祖武皇帝,扫清六合席卷八荒;万姓倾心,四方仰德。非以权势取之,实天命所归也。
世祖文帝,神文圣武,以膺大统,应天合人,法尧禅舜,处中国以临万邦,岂非天心人意
乎?今公蕴大才、抱大器,自欲比于管、乐,何乃强欲逆天理、背人情而行事耶?岂不闻古
人曰:‘顺天者昌,逆天者亡。’今我大魏带甲百万,良将千员。谅腐草之萤光,怎及天心
之皓月?公可倒戈卸甲,以礼来降,不失封侯之位。国安民乐,岂不美哉!”

孔明在车上大笑曰:“吾以为汉朝大老元臣,必有高论,岂期出此鄙言!吾有一言,诸
军静听:昔日桓、灵之世,汉统陵替,宦官酿祸;国乱岁凶,四方扰攘。黄巾之后,董卓、
傕、汜等接踵而起,迁劫汉帝,残暴生灵。因庙堂之上,朽木为官,殿陛之间,禽兽食禄;
狼心狗行之辈,滚滚当道,奴颜婢膝之徒,纷纷秉政。以致社稷丘墟,苍生涂炭。吾素知汝
所行:世居东海之滨,初举孝廉入仕;理合匡君辅国,安汉兴刘;何期反助逆贼,同谋篡
位!罪恶深重,天地不容!天下之人,愿食汝肉!今幸天意不绝炎汉,昭烈皇帝继统西川。
吾今奉嗣君之旨,兴师讨贼。汝既为谄谀之臣,只可潜身缩首,苟图衣食;安敢在行伍之
前,妄称天数耶!皓首匹夫!苍髯老贼!汝即日将归于九泉之下,何面目见二十四帝乎!老
贼速退!可教反臣与吾共决胜负!”

王朗听罢,气满胸膛,大叫一声,撞死于马下。后人有诗赞孔明曰:“兵马出西秦,雄
才敌万人。轻摇三寸舌,骂死老奸臣。”孔明以扇指曹真曰:“吾不逼汝。汝可整顿军马,
来日决战。”言讫回车。于是两军皆退。曹真将王朗尸首,用棺木盛贮,送回长安去了。副
都督郭淮曰:“诸葛亮料吾军中治丧,今夜必来劫寨。可分兵四路:两路兵从山僻小路,乘
虚去劫蜀寨;两路兵伏于本寨外,左右击之。”曹真大喜曰:“此计与吾相合。”遂传令唤
曹遵、朱赞两个先锋分付曰:“汝二人各引一万军,抄出祁山之后。但见蜀兵望吾寨而来,
汝可进兵去劫蜀寨。如蜀兵不动,便撤兵回,不可轻进。”二人受计,引兵而去。真谓淮
曰:“我两个各引一枝军,伏于寨外,寨中虚堆柴草,只留数人。如蜀兵到,放火为号。”
诸将皆分左右,各自准备去了。却说孔明归帐,先唤赵云、魏延听令。孔明曰:“汝二人各
引本部军去劫魏寨。”魏延进曰:“曹真深明兵法,必料我乘丧劫寨。他岂不提防?”孔明
笑曰:“吾正欲曹真知吾去劫寨也。彼必伏兵在祁山之后,待我兵过去,却来袭我寨;吾故
令汝二人,引兵前去,过山脚后路,远下营寨,任魏兵来劫吾寨。汝看火起为号,分兵两
路:文长拒住山口;子龙引兵杀回,必遇魏兵,却放彼走回,汝乘势攻之,彼必自相掩杀。
可获全胜。”二将引兵受计而去。又唤关兴、张苞分付曰:“汝二人各引一军,伏于祁山要
路;放过魏兵,却从魏兵来路,杀奔魏寨而去。”二人引兵受计去了。又令马岱、王平、张
翼、张嶷四将,伏于寨外,四面迎击魏兵。孔明乃虚立寨栅,居中堆起柴草,以备火号;自
引诸将退于寨后,以观动静。

却说魏先锋曹遵、朱赞黄昏离寨,迤逦前进。二更左侧,遥望山前隐隐有军行动。曹遵
自思曰:“郭都督真神机妙算!”遂催兵急进。到蜀寨时,将及三更。曹遵先杀入寨,却是
空寨,并无一人。料知中计,急撤军回。寨中火起。朱赞兵到,自相掩杀,人马大乱。曹遵
与朱赞交马,方知自相践踏。急合兵时,忽四面喊声大震,王平、马岱、张嶷、张翼杀到。
曹、朱二人引心腹军百余骑,望大路奔走。忽然鼓角齐鸣,一彪军截住去路,为首大将乃常
山赵子龙也,大叫曰:“贼将那里去?早早受死!”曹、朱二人夺路而走。忽喊声又起,魏
延又引一彪军杀到。曹、朱二人大败,夺路奔回本寨。守寨军士,只道蜀兵来劫寨,慌忙放
起号火。左边曹真杀至,右边郭淮杀至,自相掩杀。背后三路蜀兵杀到:中央魏延,左边关
兴,右边张苞,大杀一阵。魏兵败走十余里,魏将死者极多。孔明全获大胜,方始收兵。曹
真、郭淮收拾败军回寨,商议曰:“今魏兵势孤,蜀兵势大,将何策以退之?”淮曰:“胜
负乃兵家常事,不足为忧。某有一计,使蜀兵首尾不能相顾,定然自走矣。”正是:可怜魏
将难成事,欲向西方索救兵。未知其计如何,且看下文分解。