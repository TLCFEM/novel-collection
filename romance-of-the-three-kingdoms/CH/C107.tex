\chapter{魏主政归司马氏~姜维兵败牛头山}

却说司马懿闻曹爽同弟曹羲、曹训、曹彦并心腹何晏,邓飏、丁谧、毕轨、李胜等及御
林军,随魏主曹芳,出城谒明帝墓,就去畋猎。懿大喜,即到省中,令司徒高柔,假以节钺
行大将军事,先据曹爽营;又令太仆王观行中领军事,据曹羲营。懿引旧官入后宫奏郭太
后,言爽背先帝托孤之恩,奸邪乱国,其罪当废。郭太后大惊曰:“天子在外,如之奈
何?”懿曰:“臣有奏天子之表,诛奸臣之计。太后勿忧。”太后惧怕,只得从之。懿急令
太尉蒋济、尚书令司马孚,一同写表,遣黄门赍出城外,径至帝前申奏。懿自引大军据武
库。早有人报知曹爽家。其妻刘氏急出厅前,唤守府官问曰:“今主公在外,仲达起兵何
意?”守门将潘举曰:“夫人勿惊,我去问来。”乃引弓弩手数十人,登门楼望之。正见司
马懿引兵过府前,举令人乱箭射下,懿不得过。偏将孙谦在后止之曰:“太傅为国家大事,
休得放箭。”连止三次,举方不射。司马昭护父司马懿而过,引兵出城屯于洛河,守住浮
桥。

且说曹爽手下司马鲁芝,见城中事变,来与参军辛敞商议曰:“今仲达如此变乱,将如
之何?”敞曰:“可引本部兵出城去见天子。”芝然其言。敞急入后堂。其姊辛宪英见之,
问曰:“汝有何事,慌速如此?”敞告曰:“天子在外,太傅闭了城门,必将谋逆。宪英
曰:“司马公未必谋逆,特欲杀曹将军耳。”敞惊曰:“此事未知如何?”宪英曰:“曹将
军非司马公之对手,必然败矣。”敞曰:“今鲁司马教我同去,未知可去否?”宪英曰:
“职守,人之大义也。凡人在难,犹或恤之;执鞭而弃其事,不祥莫大焉。”敞从其言,乃
与鲁芝引数十骑,斩关夺门而出。人报知司马懿。懿恐桓范亦走,急令人召之。范与其子商
议。其子曰:“车驾在外,不如南出。”范从其言,乃上马至平昌门,城门已闭,把门将乃
桓范旧吏司蕃也。范袖中取出一竹版曰:“太后有诏,可即开门。”司蕃曰:“请诏验
之。”范叱曰:“汝是吾故吏,何敢如此!”蕃只得开门放出。范出的城外,唤司蕃曰:
“太傅造反,汝可速随我去。”蕃大惊,追之不及。人报知司马懿。懿大惊曰:“智囊泄
矣!如之奈何?”蒋济曰:“驽马恋栈豆,必不能用也。”懿乃召许允、陈泰曰:“汝去见
曹爽,说太傅别无他事,只是削汝兄弟兵权而已。”许、陈二人去了。又召殿中校尉尹大目
至;令蒋济作书,与目持去见爽。懿分付曰:“汝与爽厚,可领此任。汝见爽,说吾与蒋济
指洛水为誓,只因兵权之事,别无他意。”尹大目依令而去。却说曹爽正飞鹰走犬之际,忽
报城内有变,太傅有表。爽大惊,几乎落马。黄门官捧表跪于天子之前。爽接表拆封,令近
臣读之。表略曰:“征西大都督、太傅臣司马懿,诚惶诚恐,顿首谨表:臣昔从辽东还,先
帝诏陛下与秦王及臣等,升御床,把臣臂,深以后事为念。今大将军曹爽,背弃顾命,败乱
国典;内则僭拟,外专威权;以黄门张当为都监,专共交关;看察至尊,候伺神器;离间二
宫,伤害骨肉;天下汹汹,人怀危惧:此非先帝诏陛下及嘱臣之本意也。臣虽朽迈,敢忘往
言?太尉臣济、尚书令臣孚等,皆以爽为有无君之心,兄弟不宜典兵宿卫。奏永宁宫,皇太
后令敕臣如奏施行。臣辄敕主者及黄门令,罢爽、羲、训吏兵,以侯就第,不得逗留,以稽
车驾;敢有稽留,便以军法从事。臣辄力疾将兵,屯于洛水浮桥,伺察非常。谨此上闻,伏
于圣听。”魏主曹芳听毕,乃唤曹爽曰:“太傅之言若此,卿如何裁处?”爽手足失措,回
顾二弟曰:“为之奈何?”羲曰:“劣弟亦曾谏兄,兄执迷不听,致有今日。司马懿谲诈无
比,孔明尚不能胜,况我兄弟乎?不如自缚见之,以免一死。”言未毕,参军辛敞、司马鲁
芝到。爽问之。二人告曰:“城中把得铁桶相似,太傅引兵屯于洛水浮桥,势将不可复归。
宜早定大计。”正言间,司农桓范骤马而至,谓爽曰:“太傅已变,将军何不请天子幸许
都,调外兵以讨司马懿耶?”爽曰:“吾等全家皆在城中,岂可投他处求援?”范曰:“匹
夫临难,尚欲望活!今主公身随天子,号令天下,谁敢不应?岂可自投死地乎?”爽闻言不
决,惟流涕而已。范又曰:“此去许都,不过中宿。城中粮草,足支数载。今主公别营兵
马,近在阙南,呼之即至。大司马之印,某将在此。主公可急行,迟则休矣!”爽曰:“多
官勿太催逼,待吾细细思之。”少顷,侍中许允、尚书陈泰至。二人告曰:“太傅只为将军
权重,不过要削去兵权,别无他意。将军可早归城中。”爽默然不语。又只见殿中校尉尹大
目到。目曰:“太傅指洛水为誓,并无他意。有蒋太尉书在此。将军可削去兵权,早归相
府。”爽信为良言。桓范又告曰:“事急矣,休听外言而就死地!”是夜,曹爽意不能决,
乃拔剑在手,嗟叹寻思;自黄昏直流泪到晓,终是狐疑不定。桓范入帐催之曰:“主公思虑
一昼夜,何尚不能决?”爽掷剑而叹曰:“我不起兵,情愿弃官,但为富家翁足矣!”范大
哭,出帐曰:“曹子丹以智谋自矜!今兄弟三人,真豚犊耳!”痛哭不已。

许允、陈泰令爽先纳印绶与司马懿。爽令将印送去,主簿杨综扯住印绶而哭曰:“主公
今日舍兵权自缚去降,不免东市受戮也!”爽曰:“太傅必不失信于我。”于是曹爽将印绶
与许、陈二人,先赍与司马懿。众军见无将印,尽皆四散。爽手下只有数骑官僚。到浮桥
时,懿传令,教曹爽兄弟三人,且回私宅;余皆发监,听候敕旨。爽等入城时,并无一人侍
从。桓范至浮桥边,懿在马上以鞭指之曰:“桓大夫何故如此?”范低头不语,入城而去。
于是司马懿请驾拔营入洛阳。曹爽兄弟三人回家之后,懿用大锁锁门,令居民八百人围守其
宅。曹爽心中忧闷。羲谓爽曰:“今家中乏粮,兄可作书与太傅借粮。如肯以粮借我,必无
相害之心。”爽乃作书令人持去。司马懿览毕,遂遣人送粮一百斛,运至曹爽府内。爽大喜
曰:“司马公本无害我之心也!”遂不以为忧。原来司马懿先将黄门张当捉下狱中问罪。当
曰:“非我一人,更有何晏、邓飏、李胜、毕轨,丁谧等五人,同谋篡逆。”懿取了张当供
词,却捉何晏等勘问明白:皆称三月间欲反。懿用长枷钉了。城门守将司蕃告称:“桓范矫
诏出城,口称太傅谋反。”懿曰:“诬人反情,抵罪反坐。”亦将桓范等皆下狱,然后押曹
爽兄弟三人并一干人犯,皆斩于市曹,灭其三族;其家产财物,尽抄入库。

时有曹爽从弟文叔之妻,乃夏侯令女也:早寡而无子,其父欲改嫁之,女截耳自誓。及
爽被诛,其父复将嫁之,女又断去其鼻。其家惊惶,谓之曰:“人生世间,如轻尘栖弱草,
何至自苦如此?且夫家又被司马氏诛戮已尽,守此欲谁为哉?”女泣曰:“吾闻仁者不以盛
衰改节,义者不以存亡易心。曹氏盛时,尚欲保终;况今灭亡,何忍弃之?此禽兽之行,吾
岂为乎!”懿闻而贤之,听使乞子以养,为曹氏后。后人有诗曰:“弱草微尘尽达观,夏侯
有女义如山。丈夫不及裙钗节,自顾须眉亦汗颜。”却说司马懿斩了曹爽,太尉蒋济曰:
“尚有鲁芝、辛敞斩关夺门而出,杨综夺印不与,皆不可纵。”懿曰:“彼各为其主,乃义
人也。”遂复各人旧职。辛敞叹曰:“吾若不问于姊,失大义矣!”后人有诗赞辛宪英曰:
“为臣食禄当思报,事主临危合尽忠。辛氏宪英曾劝弟,故令千载颂高风。”

司马懿饶了辛敞等,仍出榜晓谕:但有曹爽门下一应人等,尽皆免死;有官者照旧复
职。军民各守家业,内外安堵。何、邓二人死于非命,果应管辂之言。后人有诗赞管辂曰:
“传得圣贤真妙诀,平原管辂相通神。鬼幽鬼躁分何邓,未丧先知是死人。”却说魏主曹芳
封司马懿为丞相,加九锡。懿固辞不肯受。芳不准,令父子三人同领国事。懿忽然想起:
“曹爽全家虽诛,尚有夏侯玄守备雍州等处,系爽亲族,倘骤然作乱,如何提备?必当处
置。”即下诏遣使往雍州,取征西将军夏侯玄赴洛阳议事。玄叔夏侯霸听知大惊,便引本部
三千兵造反。有镇守雍州刺史郭淮,听知夏侯霸反,即率本部兵来,与夏侯霸交战。淮出马
大骂曰:“汝既是大魏皇族,天子又不曾亏汝,何故背反?”霸亦骂曰:“吾祖父于国家多
建勤劳,今司马懿何等匹夫,灭吾兄曹爽宗族,又来取我,早晚必思篡位。吾仗义讨贼,何
反之有?”淮大怒,挺枪骤马,直取夏侯霸。霸挥刀纵马来迎。战不十合,淮败走,霸随后
赶来。忽听的后军呐喊,霸急回马时,陈泰引兵杀来。郭淮复回,两路夹攻。霸大败而走,
折兵大半;寻思无计,遂投汉中来降后主。

有人报与姜维,维心不信,令人体访得实,方教入城。霸拜见毕,哭告前事。维曰:
“昔微子去周,成万古之名:公能匡扶汉室,无愧古人也。”遂设宴相待。维就席问曰:
“今司马懿父子掌握重权,有窥我国之志否?”霸曰:“老贼方图谋逆,未暇及外。但魏国
新有二人,正在妙龄之际,若使领兵马,实吴、蜀之大患也。”维问:“二人是谁?”霸告
曰:“一人现为秘书郎,乃颍川长社人,姓钟,名会,字士季,太傅钟繇之子,幼有胆智。
繇尝率二子见文帝,会时年七岁,其兄毓年八岁。毓见帝惶惧,汗流满面。帝问毓曰:卿何
以汗?毓对曰:战战惶惶,汗出如浆。帝问会曰:“卿何以不汗?会对曰:战战栗栗,汗不
敢出。帝独奇之。及稍长,喜读兵书,深明韬略;司马懿与蒋济皆奇其才。一人现为掾吏,
乃义阳人也,姓邓,名艾,字士载,幼年失父,素有大志,但见高山大泽,辄窥度指画,何
处可以屯兵,何处可以积粮,何处可以埋伏。人皆笑之,独司马懿奇其才,遂令参赞军机。
艾为人口吃,每奏事必称艾艾。懿戏谓曰:卿称艾艾,当有几艾?艾应声曰:凤兮凤兮,故
是一凤。其资性敏捷,大抵如此。此二人深可畏也。”维笑曰:“量此孺子,何足道哉!”

于是姜维引夏侯霸至成都,入见后主。维奏曰:“司马懿谋杀曹爽,又来赚夏侯霸,霸
因此投降。目今司马懿父子专权,曹芳懦弱,魏国将危。臣在汉中有年,兵精粮足;臣愿领
王师,即以霸为向导官,克服中原,重兴汉室:以报陛下之恩,以终丞相之志。”尚书令费
祎谏曰:“近者,蒋琬、董允皆相继而亡,内治无人。伯约只宜待时,不宜轻动。”维曰:
“不然。人生如白驹过隙,似此迁延岁月,何日恢复中原乎?”祎又曰:“孙子云:知彼知
己,百战百胜。我等皆不如丞相远甚,丞相尚不能恢复中原,何况我等?”维曰:“吾久居
陇上,深知羌人之心;今若结羌人为援,虽未能克复中原,自陇而西,可断而有也。”后主
曰:“卿既欲伐魏,可尽忠竭力,勿堕锐气,以负朕命。”于是姜维领敕辞朝,同夏侯霸径
到汉中,计议起兵。维曰:“可先遣使去羌人处通盟,然后出西平,近雍州。先筑二城于麴
山之下,令兵守之,以为掎角之势。我等尽发粮草于川口,依丞相旧制,次第进兵。”

是年秋八月,先差蜀将句安、李歆同引一万五千兵,往麴山前连筑二城:句安守东城,
李歆守西城。早有细作报与雍州刺史郭淮。淮一面申报洛阳,一面遣副将陈泰引兵五万,来
与蜀兵交战。句安、李歆各引一军出迎;因兵少不能抵敌,退入城中。泰令兵四面围住攻
打,又以兵断其汉中粮道。句安、李歆城中粮缺。郭淮自引兵亦到,看了地势,忻然而喜;
回到寨中,乃与陈泰计议曰:“此城山势高阜,必然水少,须出城取水;若断其上流,蜀兵
皆渴死矣。”遂令军士掘土堰断上流。城中果然无水。李歆引兵出城取水,雍州兵围困甚
急。歆死战不能出,只得退入城去。句安城中亦无水,乃会了李歆,引兵出城,并在一处;
大战良久,又败入城去。军士枯渴。安与歆曰:“姜都督之兵,至今未到,不知何故。”歆
曰:“我当舍命杀出求救。”遂引数十骑,开了城门,杀将出来。雍州兵四面围合,歆奋死
冲突,方才得脱;只落得独自一人,身带重伤,余皆没于乱军之中。是夜北风大起,阴云布
合,天降大雪,因此城内蜀兵分粮化雪而食。

却说李歆撞出重围,从西山小路行了两日,正迎着姜维人马。歆下马伏地告曰:“麴山
二城,皆被魏兵围困,绝了水道。幸得天降大雪,因此化雪度日。甚是危急。”维曰:“吾
非来迟;为聚羌兵未到,因此误了。”遂令人送李歆入川养病。维问夏侯霸曰:“羌兵未
到,魏兵围困麴山甚急,将军有何高见?”霸曰:“若等羌兵到,麴山二城皆陷矣。吾料雍
州兵,必尽来麴山攻打,雍州城定然空虚。将军可引兵径往牛头山,抄在雍州之后:郭淮、
陈泰必回救雍州,则麴山之围自解矣。”维大喜曰:“此计最善!”于是姜维引兵望牛头山
而去。

却说陈泰见李歆杀出城去了,乃谓郭淮曰:“李歆若告急于姜维,姜维料吾大兵皆在麴
山,必抄牛头山袭吾之后。将军可引一军去取洮水,断绝蜀兵粮道;吾分兵一半,径往牛头
山击之。彼若知粮道已绝,必然自走矣。”郭淮从之,遂引一军暗取洮水。陈泰引一军径往
牛头山来。

却说姜维兵至牛头山,忽听的前军发喊,报说魏兵截住去路。维慌忙自到军前视之。陈
泰大喝曰:“汝欲袭吾雍州!吾已等候多时了!”维大怒,挺枪纵马,直取陈泰。泰挥刀而
迎。战不三合,泰败走,维挥兵掩杀。雍州兵退回,占住山头。维收兵就牛头山下寨。维每
日令兵搦战,不分胜负。夏侯霸谓姜维曰:“此处不是久停之所。连日交战,不分胜负,乃
诱兵之计耳,必有异谋。不如暂退,再作良图。”正言间,忽报郭淮引一军取洮水,断了粮
道。维大惊,急令夏侯霸先退,维自断后。陈泰分兵五路赶来。维独拒五路总口,战住魏
兵。泰勒兵上山,矢石如雨。维急退到洮水之时,郭淮引兵杀来。维引兵往来冲突。魏兵阻
其去路,密如铁桶。维奋死杀出,折兵大半,飞奔上阳平关来。前面又一军杀到;为首一员
大将,纵马横刀而出。那人生得圆面大耳,方口厚唇,左目下生个黑瘤,瘤上生数十根黑
毛,乃司马懿长子骠骑将军司马师也。维大怒曰:“孺子焉敢阻吾归路!”拍马挺枪,直来
刺师。师挥刀相迎。只三合,杀败了司马师,维脱身径奔阳平关来。城上人开门放入姜维。
司马师也来抢关,两边伏弩齐发,一弩发十矢,乃武侯临终时所遗连弩之法也。正是:难支
此日三军败,独赖当年十矢传。未知司马师性命如何,且看下文分解。