\chapter{渡泸水再缚番王~识诈降三擒孟获}

却说孔明放了孟获,众将上帐问曰:“孟获乃南蛮渠魁,今幸被擒,南方便定;丞相何
故放之?”孔明笑曰:“吾擒此人,如囊中取物耳。直须降伏其心,自然平矣。”诸将闻
言,皆未肯信。当日孟获行至泸水,正遇手下败残的蛮兵,皆来寻探。众兵见了孟获,且惊
且喜,拜问曰:“大王如何能勾回来?”获曰:“蜀人监我在帐中,被我杀死十余人,乘夜
黑而走;正行间,逢着一哨马军,亦被我杀之,夺了此马:因此得脱。”众皆大喜,拥孟获
渡了泸水,下住寨栅,会集各洞酋长,陆续招聚原放回的蛮兵,约有十余万骑。此时董荼
那、阿会喃已在洞中。孟获使人去请,二人惧怕,只得也引洞兵来。获传令曰:“吾已知诸
葛亮之计矣,不可与战,战则中他诡计。彼川兵远来劳苦,况即日天炎,彼兵岂能久住?吾
等有此泸水之险,将船筏尽拘在南岸,一带皆筑土城,深沟高垒,看诸葛亮如何施谋!”众
酋长从其计,尽拘船筏于南岸,一带筑起土城:有依山傍崖之地,高竖敌楼;楼上多设弓弩
炮石,准备久处之计。粮草皆是各洞供运。孟获以为万全之策,坦然不忧。却说孔明提兵大
进,前军已至泸水,哨马飞报说:“泸水之内,并无船筏;又兼水势甚急,隔岸一带筑起土
城,皆有蛮兵守把。”时值五月,天气炎热,南方之地,分外炎酷,军马衣甲,皆穿不得。
孔明自至泸水边观毕,回到本寨,聚诸将至帐中,传令曰:“今孟获兵屯泸水之南,深沟高
垒,以拒我兵;吾既提兵至此,如何空回?汝等各各引兵,依山傍树,拣林木茂盛之处,与
我将息人马。”乃遣吕凯离泸水百里,拣阴凉之地,分作四个寨子;使王平、张嶷、张翼、
关索各守一寨,内外皆搭草棚,遮盖马匹,将士乘凉,以避暑气。参军蒋琬看了,入问孔明
曰:“某看吕凯所造之寨甚不好,正犯昔日先帝败于东吴时之地势矣,倘蛮兵偷渡泸水,前
来劫寨,若用火攻,如何解救?”孔明笑曰:“公勿多疑,吾自有妙算。”蒋琬等皆不晓其
意。忽报蜀中差马岱解暑药并粮米到。孔明令入。岱参拜毕,一面将米药分派四寨。孔明问
曰:“汝将带多少军来?”马岱曰:“有三千军。”孔明曰:“吾军累战疲困,欲用汝军,
未知肯向前否?”岱曰:“皆是朝廷军马,何分彼我?丞相要用,虽死不辞。”孔明曰:
“今孟获拒住泸水,无路可渡。吾欲先断其粮道,令彼军自乱。”岱曰:“如何断得?”孔
明曰:“离此一百五十里,泸水下流沙口,此处水慢,可以扎筏而渡。汝提本部三千军渡
水,直入蛮洞,先断其粮,然后会合董荼那、阿会喃两个洞主,便为内应。不可有误。”

马岱欣然去了,领兵前到沙口,驱兵渡水;因见水浅,大半不下筏,只裸衣而过,半渡
皆倒;急救傍岸,口鼻出血而死。马岱大惊,连夜回告孔明。孔明随唤向导土人问之。土人
曰:“目今炎天,毒聚泸水,日间甚热,毒气正发,有人渡水,必中其毒;或饮此水,其人
必死。若要渡时。须待夜静水冷,毒气不起,饱食渡之,方可无事。”孔明遂令土人引路,
又选精壮军五六百,随着马岱,来到泸水沙口,扎起木筏,半夜渡水,果然无事,岱领着二
千壮军,令土人引路,径取蛮洞运粮总路口夹山峪而来。那夹山峪,两下是山,中间一条
路,止容一人一马而过。马岱占了夹山峪,分拨军士,立起寨栅。洞蛮不知,正解粮到,被
岱前后截住,夺粮百余车,蛮人报入孟获大寨中。此时孟获在寨中,终日饮酒取乐,不理军
务,谓众酋长曰:“吾若与诸葛亮对敌,必中奸计。今靠此泸水之险,深沟高垒以待之;蜀
人受不过酷热,必然退走。那时吾与汝等随后击之,便可擒诸葛亮也。”言讫,呵呵大笑。
忽然班内一酋长曰:“沙口水浅,倘蜀兵透漏过来,深为利害;当分军守把。”获笑曰:
“汝是本处土人,如何不知?吾正要蜀兵来渡此水,渡则必死于水中矣。”酋长又曰:“倘
有土人说与夜渡之法,当复何如?”获曰:“不必多疑。吾境内之人,安肯助敌人耶?”正
言之间,忽报蜀兵不知多少,暗渡泸水,绝断了夹山粮道,打着“平北将军马岱”旗号。获
笑曰:“量此小辈,何足道哉!”即遣副将忙牙长,引三千兵投夹山峪来。

却说马岱望见蛮兵已到,遂将二千军摆在山前。两阵对圆,忙牙长出马,与马岱交锋,
只一合,被岱一刀,斩于马下。蛮兵大败走回,来见孟获,细言其事。获唤诸将问曰:“谁
敢去敌马岱?”言未毕,董荼那出曰:“某愿往。”孟获大喜,遂与三千兵而去。获又恐有
人再渡泸水,即遣阿会喃引三千兵,去守把沙口。却说董荼那引蛮兵到了夹山峪下寨,马岱
引兵来迎。部内军有认得是董荼那,说与马岱如此如此。岱纵马向前大骂曰:“无义背恩之
徒!吾丞相饶汝性命,今又背反,岂不自羞!”董荼那满面惭愧,无言可答,不战而退。马
岱掩杀一阵而回。董荼那回见孟获曰:“马岱英雄,抵敌不住。”获大怒曰:“吾知汝原受
诸葛亮之恩,今故不战而退,正是卖阵之计!”喝教推出斩了。众酋长再三哀告,方才免
死,叱武士将董荼那打了一百大棍,放归本寨。诸多酋长皆来告董荼那曰:“我等虽居蛮
方,未尝敢犯中国;中国亦不曾侵我。今因孟获势力相逼,不得已而造反。想孔明神机莫
测,曹操、孙权尚自惧之,何况我等蛮方乎?况我等皆受其活命之恩,无可为报。今欲舍一
死命,杀孟获去投孔明,以免洞中百姓涂炭之苦。”董荼那曰:“未知汝等心下若何?”内
有原蒙孔明放回的人,一齐同声应曰:“愿往!”于是董荼那手执钢刀,引百余人,直奔大
寨而来,时孟获大醉于帐中。董荼那引众人持刀而入,帐下有两将侍立。董荼那以刀指曰:
“汝等亦受诸葛丞相活命之恩,宜当报效。”二将曰:“不须将军下手,某当生擒孟获,去
献丞相。”于是一齐入帐,将孟获执缚已定,押到泸水边,驾船直过北岸,先使人报知孔
明。

却说孔明已有细作探知此事,于是密传号令,教各寨将士,整顿军器,方教为首酋长解
孟获入来,其余皆回本寨听候。董荼那先入中军见孔明,细说其事。孔明重加赏劳,用好言
抚慰,遣董荼那引众酋长去了,然后令刀斧手推孟获入。孔明笑曰:“汝前者有言:但再擒
得,便肯降服。今日如何?”获曰:“此非汝之能也;乃吾手下之人自相残害,以致如此。
如何肯服!”孔明曰:“吾今再放汝去,若何?”孟获曰:“吾虽蛮人,颇知兵法;若丞相
端的肯放吾回洞中,吾当率兵再决胜负。若丞相这番再擒得我,那时倾心吐胆归降,并不敢
改移也。”孔明曰:“这番生擒,如又不服,必无轻恕。”令左右去其绳索,仍前赐以酒
食,列坐于帐上。孔明曰:“吾自出茅庐,战无不胜,攻无不取。汝蛮邦之人,何为不
服?”获默然不答。孔明酒后,唤孟获同上马出寨,观看诸营寨栅所屯粮草,所积军器。孔
明指谓孟获曰:“汝不降吾,真愚人也。吾有如此之精兵猛将,粮草兵器,汝安能胜吾哉?
汝若早降,吾当奏闻天子,令汝不失王位,子子孙孙,永镇蛮邦。意下若何?”获曰:“某
虽肯降,怎奈洞中之人未肯心服。若丞相肯放回去,就当招安本部人马,同心合胆,方可归
顺。”孔明忻然,又与孟获回到大寨。饮酒至晚,获辞去;孔明亲自送至泸水边,以船送获
归寨。孟获来到本寨,先伏刀斧手于帐下,差心腹人到董荼那、阿会喃寨中,只推孔明有使
命至,将二人赚到大寨帐下,尽皆杀之,弃尸于涧。孟获随即遣亲信之人,守把隘口,自引
军出了夹山峪,要与马岱交战,却并不见一人;及问土人,皆言昨夜尽搬粮草,复渡泸水,
归大寨去了。获再回洞中,与亲弟孟优商议曰:“如今诸葛亮之虚实,吾已尽知,汝可去如
此如此。”孟优领了兄计,引百余蛮兵,搬载金珠、宝贝、象牙、犀角之类,渡了泸水,径
投孔明大寨而来;方才过了河时,前面鼓角齐鸣,一彪军摆开:为首大将乃马岱也。孟优大
惊。岱问了来情,令在外厢,差人来报孔明。孔明正在帐中与马谡、吕凯、蒋琬、费祎等共
议平蛮之事,忽帐下一人,报称孟获差弟孟优来进宝贝。孔明回顾马谡曰:“汝知其来意
否?”谡曰:“不敢明言。容某暗写于纸上,呈与丞相,看合钧意否?”孔明从之。马谡写
讫,呈与孔明。孔明看毕,抚掌大笑曰:“擒孟获之计,吾已差派下也。汝之所见,正与吾
同。”遂唤赵云入,向耳畔分付如此如此;又唤魏延入,亦低言分付;又唤王平、马忠、关
索入,亦密密地分付。

各人受了计策,皆依令而去,方召孟优入帐,优再拜于帐下曰:“家兄孟获,感丞相活
命之恩,无可奉献,辄具金珠宝贝若干,权为赏军之资。续后别有进贡天子礼物。”孔明
曰:“汝兄今在何处?”优曰:“为感丞相天恩,径往银坑山中收拾宝物去了,少时便回来
也。”孔明曰:“汝带多少人来?”优曰:“不敢多带。只是随行百余人,皆运货物者。”
孔明尽教入帐看时,皆是青眼黑面,黄发紫须,耳带金环,鬅头跣足,身长力大之士。孔明
就令随席而坐,教诸将劝酒,殷勤相待。

却说孟获在帐中专望回音,忽报有二人回了;唤入问之,具说:“诸葛亮受了礼物大
喜,将随行之人,皆唤入帐中,杀牛宰羊,设宴相待。二大王令某密报大王:今夜二更,里
应外合,以成大事。”孟获听知甚喜,即点起三万蛮兵,分为三队。获唤各洞酋长分付曰:
“各军尽带火具。今晚到了蜀寨时,放火为号。吾当自取中军,以擒诸葛亮。”诸多蛮将,
受了计策,黄昏左侧,各渡泸水而来。孟获带领心腹蛮将百余人,径投孔明大寨,于路并无
一军阻当。前至寨门,获率众将骤马而入,乃是空寨,并不见一人。获撞入中军,只见帐中
灯烛荧煌,孟优并番兵尽皆醉倒。原来孟优被孔明教马谡、吕凯二人管待,令乐人搬做杂
剧,殷勤劝酒,酒内下药,尽皆昏倒,浑如醉死之人。孟获入帐问之,内有醒者,但指口而
已。获知中计,急救了孟优等一干人;却待奔回中队,前面喊声大震,火光骤起,蛮兵各自
逃窜。一彪军杀到,乃是蜀将王平。获大惊,急奔左队时,火光冲天,一彪军杀到,为首蜀
将乃是魏延。获慌忙望右队而来,只见火光又起,又一彪军杀到,为首蜀将乃是赵云。三路
军夹攻将来,四下无路。孟获弃了军士,匹马望泸水面逃。正见泸水上数十个蛮兵,驾一小
舟,获慌令近岸。人马方才下船,一声号起,将孟获缚住。原来马岱受了计策,引本部兵扮
作蛮兵,撑船在此,诱擒孟获。

于是孔明招安蛮兵,降者无数。孔明一一抚慰,并不加害。就教救灭了余火。须臾,马
岱擒孟获至;赵云擒孟优至;魏延、马忠、王平、关索擒诸洞酋长至。孔明指孟获而笑曰:
“汝先令汝弟以礼诈降,如何瞒得过吾!今番又被我擒,汝可服否?”获曰:“此乃吾弟贪
口腹之故,误中汝毒,因此失了大事。吾若自来,弟以兵应之,必然成功。此乃天败,非吾
之不能也,如何肯服!”孔明曰:“今已三次,如何不服?”孟获低头无语。孔明笑曰:
“吾再放汝回去。”孟获曰:“丞相若肯放吾兄弟回去,收拾家下亲丁,和丞相大战一场。
那时擒得,方才死心塌地而降。”孔明曰:“再若擒住,必不轻恕。汝可小心在意,勤攻韬
略之书,再整亲信之士,早用良策,勿生后悔。”遂令武士去其绳索,放起孟获,并孟优及
各洞酋长,一齐都放。孟获等拜谢去了。此时蜀兵已渡泸水。孟获等过了泸水,只见岸口陈
兵列将,旗帜纷纷。获到营前,马岱高坐,以剑指之曰:“这番拿住,必无轻放!”孟获到
了自己寨时,赵云早已袭了此寨,布列兵马。云坐于大旗下,按剑而言曰:“丞相如此相
待,休忘大恩!”获喏喏连声而去。将出界口山坡,魏延引一千精兵,摆在坡上,勒马厉声
而言曰:“吾今已深入巢穴,夺汝险要;汝尚自愚迷,抗拒大军!这回拿住,碎尸万段,决
不轻饶!”孟获等抱头鼠窜,望本洞而去。后人有诗赞曰:“五月驱兵入不毛,月明泸水瘴
烟高。誓将雄略酬三顾,岂惮征蛮七纵劳。”

却说孔明渡了泸水,下寨已毕,大赏三军,聚众将于帐下曰:“孟获第二番擒来,吾令
遍观各营虚实,正欲令其来劫营也。吾知孟获颇晓兵法,吾以兵马粮草炫耀,实令孟获看吾
破绽,必用火攻。彼令其弟诈降,欲为内应耳。吾三番擒之而不杀,诚欲服其心,不欲灭其
类也。吾今明告汝等,勿得辞劳,可用心报国。”众将拜伏曰:“丞相智、仁、勇三者足
备,虽子牙、张良不能及也。”孔明曰:“吾今安敢望古人耶?皆赖汝等之力,共成功业
耳。”帐下诸将听得孔明之言,尽皆喜悦。却说孟获受了三擒之气,忿忿归到银坑洞中,即
差心腹人赍金珠宝贝,往八番九十三甸等处,并蛮方部落,借使牌刀獠丁军健数十万,克日
齐备,各队人马,云推雾拥,俱听孟获调用。伏路军探知其事,来报孔明,孔明笑曰:“吾
正欲令蛮兵皆至,见吾之能也。”遂上小车而行。正是:若非洞主威风猛,怎显军师手段
高!未知胜负如何,且看下文分解。