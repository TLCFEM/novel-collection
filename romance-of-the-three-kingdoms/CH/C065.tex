\chapter{马超大战葭萌关~刘备自领益州牧}

却说阎圃正劝张鲁勿助刘璋,只见马超挺身出曰:“超感主公之恩,无可上报,愿领一军攻取葭萌关,生擒刘备,务要刘璋割二十州奉还主公。”张鲁大喜,先遣黄权从小路而回,随即点兵二万与马超。此时庞德卧病不能行,留于汉中。张鲁令杨柏监军,超与弟马岱选日起程。

却说玄德军马在雒城,法正所差下书人回报说:“郑度劝刘璋尽烧野谷并各处仓廪,率巴西之民,避于涪水西,深沟高垒而不战。”玄德、孔明闻之,皆大惊曰:“若用此言,吾势危矣!”法正笑曰:“主公勿忧。此计虽毒,刘璋必不能用也。”不一日,人传刘璋不肯迁动百姓,不从郑度之言。玄德闻之,方始宽心。孔明曰:“可速进兵取绵竹。如得此处,成都易取矣。”遂遣黄忠、魏延领兵前进。费观听知玄德兵来,差李严出迎。严领三千兵也,各布阵完。黄忠出马,与李严战四五十合,不分胜败。孔明在阵中教鸣金收军。黄忠回阵,问曰:“正待要擒李严,军师何故收兵?”孔明曰:“吾已见李严武艺,不可力取。来日再战,汝可诈败,引入山峪,出奇兵以胜之。”黄忠领计。次日,李严再引兵来,黄忠又出战,不十合诈败,引兵便走。李严赶来,迤逦赶入出峪,猛然省悟。急待回来,前面魏延引兵摆开。孔明自在山头,唤曰:“公如不降,两下已伏强弩,欲与吾庞士元报仇矣。”李严慌下马卸甲投降。军士不曾伤害一人。孔明引李严见玄德。玄德待之甚厚。严曰:“费观虽是刘盖州亲戚,与某甚密,当往说之。”玄德即命李严回城招降费观。严入绵竹城,对费观赞玄德如此仁德;今若不降,必有大祸。观从其言,开门投降。玄德遂入绵竹,商议分兵取成都。

忽流星马急报,言孟达、霍峻守葭萌关,今被东川张鲁遣马超与杨柏、马岱领兵攻打甚急,救迟则关隘休矣。玄德大惊。孔明曰:“须是张、赵二将,方可与敌。”玄德曰:“子龙引兵在外未回。翼德已在此,可急遣之。”孔明曰:“主公且勿言,容亮激之。”却说张飞闻马超攻关,大叫而入曰:“辞了哥哥,便去战马超也!”孔明佯作不闻,对玄德曰:“今马超侵犯关隘,无人可敌;除非往荆州取关云长来,方可与敌。”张飞曰:“军师何故小觑吾!吾曾独拒曹操百万之兵,岂愁马超一匹夫乎!”孔明曰:“翼德拒水断桥,此因曹操不知虚实耳;若知虚实,将军岂得无事?今马超之勇,天下皆知,渭桥六战,杀得曹操割须弃袍,几乎丧命,非等闲之比。云长且未必可胜。”飞曰:“我只今便去;如胜不得马超,甘当军令!”孔明曰:“既尔肯写文书,便为先锋。请主公亲自去一遭,留亮守绵竹。待子龙来,却作商议。”魏延曰:“某亦愿往。”

孔明令魏延带五百哨马先行,张飞第二,玄德后队,望葭萌关进发。魏延哨马先到关下,正遇杨柏。魏延与杨柏交战,不十合,杨柏败走。魏延要夺张飞头功,乘势赶去。前面一军摆开,为首乃是马岱。魏延只道是马超,舞刀跃马迎之。与岱战不十合,岱败走。延赶去,被岱回身一箭,中了魏延左臂。延急回马走。马岱赶到关前,只见一将喊声如雷,从关上飞奔至面前。原来是张飞初到关上,听得关前厮杀,便来看时,正见魏延中箭,因骤马下关,救了魏延。飞喝马岱曰:“汝是何人?先通姓名,然后厮杀?”马岱曰:“吾乃西凉马岱是也。”张飞曰:“你原来不是马超,快回去!非吾对手!只令马超那厮自来,说道燕人张飞在此!”马岱大怒曰:“汝焉敢小觑我!”挺枪跃马,直取张飞。战不十合,马岱败走。张飞欲待追赶,关上一骑马到来,叫:“兄弟且休去!”飞回视之,原来是玄德到来。飞遂不赶,一同上关。玄德曰:“恐怕你性躁,故我随后赶来到此。既然胜了马岱,且歇一宵,来日战马超。”次日天明,关下鼓声大震,马超兵到。玄德在关上看时,门旗影里,马超纵骑持枪而出;狮盔兽带,银甲白袍:一来结束非凡,二者人才出众。玄德叹曰:“人言锦马超,名不虚传!”张飞便要下关。玄德急止之曰:“且休出战。先当避其锐气。”关下马超单搦张飞出马,关上张飞恨不得平吞马超,三五番皆被玄德当住。看看午后,玄德望见马超阵上人马皆倦,遂选五百骑,跟着张飞,冲下关来。马超见张飞军到,把枪望后一招,约退军有一箭之地。张飞军马一齐扎住;关上军马,陆续下来。张飞挺枪出马,大呼:“认得燕人张翼德么!”马超曰:“吾家屡世公侯,岂识村野匹夫!”张飞大怒。两马齐出,二枪并举。约战百余合,不分胜负。玄德观之,叹曰:“真虎将也!”恐张飞有失,急鸣金收军。两将各回。张飞回到阵中,略歇马片时,不用头盔,只裹包巾上马,又出阵前搦马超厮杀。超又出,两个再战。玄德恐张飞有失,自披挂下关,直至阵前;看张飞与马超又斗百余合,两个精神倍加。玄德教鸣金收军。二将分开,各回本阵。

是日天色已晚,玄德谓张飞曰:“马超英勇,不可轻敌,且退上关。来日再战。”张飞杀得性起,那里肯休?大叫曰:“誓死不回!”玄德曰:“今日天晚,不可战矣。”飞曰:“多点火把,安排夜战!”马超亦换了马,再出阵前,大叫曰:“张飞!敢夜战么?张飞性起,问玄德换了坐下马,抢出阵来,叫曰:“我捉你不得,誓不上关!”超曰:“我胜你不得,誓不回寨!”两军呐喊,点起千百火把,照耀如同白日。两将又向阵前鏖战。到二十余合,马超拨回马便走。张飞大叫曰:“走那里去!”原来马超见赢不得张飞,心生一计:诈败佯输,赚张飞赶来,暗掣铜锤在手,扭回身觑着张飞便打将来。张飞见马超走,心中也提防;比及铜锤打来时,张飞一闪,从耳朵边过去。张飞便勒回马走时,马超却又赶来。张飞带住马,拈弓搭箭,回射马超;超却闪过。二将各自回阵。玄德自于阵前叫曰:“吾以仁义待人。不施谲诈。马孟起,你收兵歇息,我不乘势赶你。”马超闻言,亲自断后,诸军渐退。玄德亦收军上关。次日,张飞又欲下关战马超。人报军师来到。玄德接着孔明。孔明曰:“亮闻孟起世之虎将,若与翼德死战,必有一伤;故令子龙、汉升守住绵竹,我星夜来此。可用条小计,令马超归降主公。”玄德曰:“吾见马超英勇,甚爱之。如何可得?”孔明曰:“亮闻东川张鲁,欲自立为汉宁王。手下谋士杨松,极贪贿赂。主公可差人从小路径投汉中,先用金银结好杨松,后进书与张鲁,云吾与刘璋争西川,是与汝报仇。不可听信离间之语。事定之后,保汝为汉宁王。令其撤回马超兵。待其来撤时,便可用计招降马超矣。”玄德大喜,即时修书,差孙乾赍金珠从小路径至汉中,先来见杨松,说知此事,送了金珠。松大喜,先引孙乾见张鲁,陈言方便。鲁曰:“玄德只是左将军,如何保得我为汉宁王?”杨松曰:“他是大汉皇叔,正合保奏。”张鲁大喜,便差人教马超罢兵。孙乾只在杨松家听回信。不一日,使者回报:“马超言:未成功,不可退兵。”张鲁又遣人去唤,又不肯回。一连三次不至。杨松曰:“此人素无信行,不肯罢兵,其意必反。”遂使人流言云:“马超意欲夺西川,自为蜀主,与父报仇,不肯臣于汉中。”张鲁闻之,问计于杨松。松曰:“一面差人去说与马超:汝既欲成功,与汝一月限,要依我三件事。若依得,便有赏;否则必诛:一要取西川,二要刘璋首级,三要退荆州兵。三件事不成,可献头来。一面教张卫点军守把关隘,防马超兵变。”鲁从之,差人到马超寨中,说这三件事。超大惊曰:“如何变得恁的!”乃与马岱商议:“不如罢兵。”杨松又流言曰:“马超回兵,必怀异心。”于是张卫分七路军,坚守隘口,不放马超兵入。超进退不得,无计可施。孔明谓玄德曰:“今马超正在进退两难之际,亮凭三寸不烂之舌,亲往超寨,说马超来降。”玄德曰:“先生乃吾之股肱心腹,倘有疏虞,如之奈何?”孔明坚意要去,玄德再三不肯放去。正踌躇间,忽报赵云有书荐西川一人来降。玄德召入问之。其人乃建宁俞元人也,姓李名恢,字德昂。玄德曰:“向日闻公苦谏刘璋,今何故归我?”恢曰:“吾闻良禽相木而栖,贤臣择主而事,前谏刘益州者,以尽人臣之心;既不能用,知必败矣。今将军仁德布于蜀中,知事必成,故来归耳。”玄德曰:“先生此来,必有益于刘备。”恢曰:“今闻马超在进退两难之际。恢昔在陇西,与彼有一面之交,愿往说马超归降,若何?”孔明曰:“正欲得一人替吾一往。愿闻公之说词。”李恢于孔明耳畔陈说如此如此。孔明大喜,即时遣行。

恢行至超寨,先使人通姓后。马超曰:“吾知李恢乃辩士,今必来说我。”先唤二十刀斧手伏于帐下,嘱曰:“令汝砍,即砍为肉酱!”须臾,李恢昂然而入。马超端坐帐中不动,叱李恢曰:“汝来为何?”恢曰:“特来作说客。”超曰:“吾匣中宝剑新磨。汝试言之,其言不通,便请试剑!”恢笑曰:“将军之祸不远矣!但恐新磨之剑,不能试吾之头,将欲自试也!”超曰:“吾有何祸?”恢曰:“吾闻越之西子,善毁者不能闭其美;齐之无盐,善美者不能掩其丑;日中则昃,月满则亏:此天下之常理也。今将军与曹操有杀父之仇,而陇西又有切齿之恨;前不能救刘璋而退荆州之兵,后不能制杨松而见张鲁之面;目下四海难容,一身无主;若复有渭桥之败,冀城之失,何面目见天下之人乎?”超顿首谢曰:“公言极善,但超无路可行。”恢曰:“公既听吾言,帐下何故伏刀斧手?”超大惭,尽叱退。恢曰:“刘皇叔礼贤下士,吾知其必成,故舍刘璋而归之。公之尊人,昔年曾与皇叔约共讨贼,公何不背暗投明,以图上报父仇,下立功名乎?”马超大喜,即唤杨柏入,一剑斩之,将首极共恢一同上关来降玄德。

玄德亲自接入,待以上宾之礼。超顿首谢曰:“今遇明主,如拨云雾而见青天!”时孙乾已回。玄德复命霍峻、孟达守关,便撤兵来取成都。赵云、黄忠接入绵竹。人报蜀将刘晙、马汉引军到。赵云曰:“某愿往擒此二人!”言讫,上马引军出。玄德在城上管待马超吃酒。未曾安席,子龙已斩二人之头,献于筵前。马超亦惊,倍加敬重。超曰:“不须主公军马厮杀,超自唤出刘璋来降。如不肯降,超自与弟马岱取成都,双手奉献。”玄德大喜。是日尽欢。

却说败兵回到益州,报刘璋。璋大惊,闭门不出。人报城北马超救兵到,刘璋方敢登城望之。见马超、马岱立于城下,大叫:“请刘季玉答话。”刘璋在城上问之。超在马上以鞭指曰:“吾本领张鲁兵来救益州,谁想张鲁听信杨松谗言,反欲害我。今已归降刘皇叔。公可纳士拜降,免致生灵受苦。如或执迷,吾先攻城矣!”刘璋惊得面如土色,气倒于城上。众官救醒。璋曰:“吾之不明,悔之何及!不若开门投降,以救满城百姓。”董和曰:“城中尚有兵三万余人;钱帛粮草,可支一年:奈何便降?”刘璋曰:“吾父子在蜀二十余年,无恩德以加百姓;攻战三年,血肉捐于草野,皆我罪也。我心何安?不如投降以安百姓。”众人闻之,皆堕泪。忽一人进曰:“主公之言,正合天意。”视之,乃巴西西充国人也,姓谯名周,字允南。此人素晓天文。璋问之,周曰:“某夜观乾象,见群星聚于蜀郡;其大星光如皓月,乃帝王之象也。况一载之前,小儿谣云:若要吃新饭,须待先主来。此乃预兆。不可逆天道。”黄权、刘巴闻言皆大怒,欲斩之。刘璋挡住。忽报:“蜀郡太守许靖,逾城出降矣。”刘璋大哭归府。

次日,人报刘皇叔遣幕宾简雍在城下唤门。璋令开门接入。雍坐车中,傲睨自若。忽一人掣剑大喝曰:“小辈得志,傍若无人!汝敢藐视吾蜀中人物耶!”雍慌下车迎之。此人乃广汉绵竹人也,姓秦名宓,字子敕。雍笑曰:“不识贤兄,幸勿见责。”遂同入见刘璋,具说玄德宽洪大度,并无相害之意。于是刘璋决计投降,厚待简雍。次日,亲赍印绶文籍,与简雍同车出城投降。玄德出寨迎接,握手流涕曰:“非吾不行仁义,奈势不得已也!”共入寨,交割印绶文籍,并马入城。

宏德入成都,百姓香花灯烛,迎门而接。玄德到公厅,升堂坐定。郡内诸官,皆拜于堂下!惟黄权、刘巴,闭门不出。众将忿怒,欲往杀之。玄德慌忙传令曰:“如有害此二人者,灭其三族!”玄德亲自登门,请二人出仕。二人感玄德恩礼,乃出。孔明请曰:“今西川平定,难容二主,可将刘璋送去荆州。”玄德曰:“吾方得蜀郡,未可令季玉远去。”孔明曰:“刘璋失基业者,皆因太弱耳。主公若以妇人之仁,临事不决,恐此土难以长久。”玄德从之,设一大宴,请刘璋收拾财物,佩领振威将军印绶,令将妻子良贱,尽赴南郡公安住歇,即日起行。玄德自领益州牧。其所降文武,尽皆重赏,定拟名爵:严颜为前将军,法正为蜀郡太守,董和为掌军中郎将,许靖为左将军长史,庞义为营中司马,刘巴为左将军,黄权为右将军。其余吴懿、费观、彭羕、卓膺、李严、吴兰、雷铜、李恢、张翼、秦宓、谯周、吕义,霍峻、邓芝、杨洪、周群、费祎、费诗、孟达,文武投降官员,共六十余人,并皆擢用。诸葛亮为军师,关云长为荡寇将军、汉寿亭侯,张飞为征虏将军、新亭侯,赵云为镇远将军,黄忠为征西将军,魏延为扬武将军,马超为平西将军。孙乾、简雍、糜竺、糜芳、刘封、吴班、关平、周仓、廖化、马良、马谡、蒋琬、伊籍,及旧日荆襄一班文武官员,尽皆升赏。遣使赍黄金五百斤、白银一千斤、钱五千万、蜀锦一千匹,赐与云长。其余官将,给赏有差。杀牛宰马,大饷士卒。开仓赈济百姓,军民大悦。

益州既定,玄德欲将成都有名田宅,分赐诸官。赵云谏曰:“益州人民,屡遭兵火,田宅皆空;今当归还百姓,令安居复业,民心方服;不宜夺之为私赏也。”玄德大喜,从其言。使诸葛军师定拟治国条例,刑法颇重。法正曰:“昔高祖约法三章,黎民皆感其德。愿军师宽刑省法。以慰民望。”孔明曰:“君知其一、未知其二:秦用法暴虐,万民皆怨,故高祖以宽仁得之。今刘璋暗弱,德政不举,威刑不肃;君臣之道,渐以陵替。宠之以位,位极则残;顺之以恩,恩竭则慢。所以致弊,实由于此。吾今威之以法,法行则知恩;限之以爵,爵加则知荣。恩荣并济,上下有节。为治之道,于斯著矣。”法正拜服。自此军民安堵。四十一州地面,分兵镇抚,并皆平定。法正为蜀郡太守,凡平日一餐之德,睚毗之怨,无不报复。或告孔明曰:“孝直太横,宜稍斥之。”孔明曰:“昔主公困守荆州,北畏曹操,东惮孙权,赖孝直为之辅翼,遂翻然翱翔,不可复制。今奈何禁止孝直,使不得少行其意耶?”因竟不问。法正闻之,亦自敛戢。

一日,玄德正与孔明闲叙,忽报云长遣关平来谢所赐金帛。玄德召入。平拜罢,呈上书信曰:“父亲知马超武艺过人,要入川来与之比试高低。教就禀伯父此事。”玄德大惊曰:“若云长入蜀,与孟起比试,势不两立。”孔明曰:“无妨。亮自作书回之。”玄德只恐云长性急,便教孔明写了书,发付关平星夜回荆州。平回至荆州,云长问曰:“我欲与马孟起比试,汝曾说否?”平答曰:“军师有书在此。”云长拆开视之。其书曰:“亮闻将军欲与孟起分别高下。以亮度之:孟起虽雄烈过人,亦乃黥布、彭越之徒耳;当与翼德并驱争先,犹未及美髯公之绝伦超群也。今公受任守荆州,不为不重;倘一入川,若荆州有失。罪莫大焉。惟冀明照。”云长看毕,自绰其髯笑曰:“孔明知我心也。”将书遍示宾客,遂无入川之意。

却说东吴孙权,知玄德并吞西川,将刘璋逐于公安,遂召张昭、顾雍商议曰:“当初刘备借我荆州时,说取了西川,便还荆州。今已得巴蜀四十一州,须用取索汉上诸郡。如其不还,即动干戈。”张昭曰:“吴中方宁,不可动兵。昭有一计,使刘备将荆州双手奉还主公。”正是:西蜀方开新日月,东吴又索旧山川。未知其计如何,且看下文分解。