\chapter{困司马汉将奇谋~废曹芳魏家果报}

蜀汉延熙十六年秋,将军姜维起兵二十万,令廖化、张翼为左右先锋,夏侯霸为参谋,张嶷为运粮使,大兵出阳平关伐魏。维与夏侯霸商议曰:“向取雍州,不克而还;今若再出,必又有准备。公有何高见?”霸曰:“陇上诸郡,只有南安钱粮最广;若先取之,足可为本。向者不克而还,盖因羌兵不至。今可先遣人会羌人于陇右,然后进兵出石营,从董亭直取南安。”维大喜曰:“公言甚妙!”遂遣郤正为使,赍金珠蜀锦入羌,结好羌王。羌王迷当,得了礼物,便起兵五万,令羌将俄何烧戈为大先锋,引兵南安来。

魏左将军郭淮闻报,飞奏洛阳。司马师问诸将曰:“谁敢去敌蜀兵?”辅国将军徐质曰:“某愿往。”师素知徐质英勇过人,心中大喜,即令徐质为先锋,令司马昭为大都督,领兵望陇西进发。军至董亭,正遇姜维,两军列成阵势。徐质使开出大斧,出马挑战。蜀阵中廖化出迎。战不数合,化拖刀败回。张翼纵马挺枪而迎,战不数合,又败入阵。徐质驱兵掩杀,蜀兵大败,退三十余里。司马昭亦收兵回,各自下寨。

姜维与夏侯霸商议曰:“徐质勇甚,当以何策擒之?”霸曰:“来日诈败,以埋伏之计胜之。”维曰:“司马昭乃仲达之子,岂不知兵法?若见地势掩映,必不肯追。吾见魏兵累次断吾粮道,今却用此计诱之,可斩徐质矣。”遂唤廖化分付如此如此,又唤张翼分付如此如此:二人领兵去了。一面令军士于路撒下铁蒺藜,寨外多排鹿角,示以久计。

徐质连日引兵搦战,蜀兵不出。哨马报司马昭说:“蜀兵在铁笼山后,用木牛流马搬运粮草,以为久计,只待羌兵策应。”昭唤徐质曰:“昔日所以胜蜀者,因断彼粮道也。今蜀兵在铁笼山后运粮,汝今夜引兵五千,断其粮道,蜀兵自退矣。”徐质领令,初更时分,引兵望铁笼山来,果见蜀兵二百余人,驱百余头木牛流马,装载粮草而行。魏兵一声喊起,徐质当先拦住。蜀兵尽弃粮草而走。质分兵一半,押送粮草回寨;自引兵一半追来。追不到十里,前面车仗横截去路。质令军士下马拆开车仗,只见两边忽然火起。质急勒马回走,后面山僻窄狭处,亦有车仗截路,火光迸起。质等冒烟突火,纵马而出。一声炮响,两路军杀来:左有廖化,右有张翼,大杀一阵,魏兵大败。

徐质奋死只身而走,人困马乏,正奔走间,前面一枝兵杀到,乃姜维也。质大惊无措,被维一枪刺倒座下马,徐质跌下马来,被众军乱刀砍死。质所分一半押粮兵,亦被夏侯霸所擒,尽降其众。霸将魏兵衣甲马匹,令蜀兵穿了,就令骑坐,打着魏军旗号,从小路径奔回魏寨来。魏军见本部兵回,开门放入,蜀兵就寨中杀起。司马昭大惊,慌忙上马走时,前面廖化杀来。昭不能前进,急退时,姜维引兵从小路杀到。昭四下无路,只得勒兵上铁笼山据守。原来此山只有一条路,四下皆险峻难上;其上惟有一泉,止够百人之饮,——此时昭手下有六千人,被姜维绝其路口,山上泉水不敷,人马枯渴。昭仰天长叹曰:“吾死于此地矣!”后人有诗曰:“妙算姜维不等闲,魏师受困铁笼间:庞涓始入马陵道,项羽初围九里山。”

主簿王韬曰:“昔日耿恭受困,拜井而得甘泉。将军何不效之?”昭从其言,遂上山顶泉边,再拜而祝曰:“昭奉诏来退蜀兵,若昭合死,令甘泉枯竭,昭自当刎颈,教部军尽降;如寿禄未终,愿苍天早赐甘泉,以活众命!”祝毕,泉水涌出,取之不竭,因此人马不死。

却说姜维在山下困住魏兵,谓众将曰:“昔日丞相在上方谷,不曾捉住司马懿,吾深为恨;今司马昭必被吾擒矣。”

却说郭淮听知司马昭困于铁笼山上,欲提兵来。陈泰曰:“姜维会合羌兵,欲先取南安。今羌兵已到,将军若撤兵去救,羌兵必乘虚袭我后也。可先令人诈降羌人,于中取事;若退了此兵,方可救铁笼之围。”郭淮从之,遂令陈泰引五千兵,径到羌王寨内,解甲而入,泣拜曰:“郭淮妄自尊大,常有杀泰之心,故来投降。郭淮军中虚实,某俱知之。只今夜愿引一军前去劫寨,便可成功。如兵到魏寨,自有内应。”迷当大喜,遂令俄何烧戈同陈泰来劫魏寨。俄何烧戈教泰降兵在后,令泰引羌兵为前部。是夜二更,竟到魏寨,寨门大开。陈泰一骑马先入。俄何烧戈骤马挺枪入寨之时,只叫得一声苦,连人带马,跌在陷坑里。陈泰兵从后面杀来,郭淮从左边杀来,羌兵大乱,自相践踏,死者无数,生者尽降。俄何烧戈自刎而死。郭淮、陈泰引兵直杀到羌人寨中,迷当大王急出帐上马时,被魏兵生擒活捉,来见郭淮。淮慌下马,亲去其缚,用好言抚慰曰:“朝廷素以公为忠义,今何故助蜀人也?”迷当惭愧伏罪。淮乃说迷当曰:“公今为前部,去解铁笼山之围,退了蜀兵,吾奏准天子,自有厚赐。”

迷当从之,遂引羌兵在前,魏兵在后,径奔铁笼山。时值三更,先令人报知姜维。维大喜,教请入相见。魏兵多半杂在羌人部内;行到蜀寨前,维令大兵皆在寨外屯扎,迷当引百余人到中军帐前。姜维、夏侯霸二人出迎。魏将不等迷当开言,就从背后杀将起来。维大惊,急上马而走。羌、魏之兵,一齐杀入。蜀兵四分五落,各自逃生。维手无器械,腰间止有一副弓箭,走得慌忙,箭皆落了,只有空壶。维望山中而走,背后郭淮引兵赶来;见维手无寸铁,乃骤马挺枪追之。看看至近,维虚拽弓弦,连响十余次。淮连躲数番,不见箭到,知维无箭,乃挂住钢枪,拈弓搭箭射之。维急闪过,顺手接了,就扣在弓弦上;待淮追近,望面门上尽力射去,淮应弦落马。维勒回马来杀郭淮,魏军骤至。维下手不及,只掣得淮枪而去。魏兵不敢追赶,急救淮归寨,拔出箭头,血流不止而死。司马昭下山引兵追赶,半途而回。夏侯霸随后逃至,与姜维一齐奔走。维折了许多人马,一路收扎不住,自回汉中。虽然兵败,却射死郭淮,杀死徐质,挫动魏国之威,将功补罪。却说司马昭犒劳羌兵,发遣回国去讫,班师还洛阳,与兄司马师专制朝权,群臣莫敢不服。魏主曹芳每见师入朝,战栗不已,如针刺背。一日,芳设朝,见师带剑上殿,慌忙下榻迎之。师笑曰:“岂有君迎臣之礼也,请陛下稳便。”须臾,群臣奏事,司马师俱自剖断,并不启奏魏主。少时朝退,师昂然下殿,乘车出内,前遮后拥,不下数千人马。

芳退入后殿,顾左右止有三人:乃太常夏侯玄,中书令李丰,光禄大夫张缉,缉乃张皇后之父,曹芳之皇丈也。芳叱退近侍,同三人至密室商议。芳执张缉之手而哭曰:“司马师视朕如小儿,觑百官如草芥,社稷早晚必归此人矣!”言讫大哭。李丰奏曰:“陛下勿忧。臣虽不才,愿以陛下之明诏,聚四方之英杰,以剿此贼。”夏侯玄奏曰:“臣叔夏侯霸降蜀,因惧司马兄弟谋害故耳;今若剿除此贼,臣叔必回也。臣乃国家旧戚,安敢坐视奸贼乱国,愿同奉诏讨之。”芳曰:“但恐不能耳。”三人哭奏曰:“臣等誓当同心灭贼,以报陛下!”芳脱下龙凤汗衫,咬破指尖,写了血诏,授与张缉,乃嘱曰:“朕祖武皇帝诛董承,盖为机事不密也。卿等须谨细,勿泄于外。”丰曰:“陛下何出此不利之言?臣等非董承之辈,司马师安比武祖也?陛下勿疑。”

三人辞出,至东华门左侧,正见司马师带剑而来,从者数百人,皆持兵器。三人立于道傍。师问曰:“汝三人退朝何迟?”李丰曰:“圣上在内廷观书,我三人侍读故耳。”师曰:“所看何书?”丰曰:“乃夏、商、周三代之书也。”师曰:“上见此书,问何故事?”丰曰:“天子所问伊尹扶商、周公摄政之事,我等皆奏曰:今司马大将军,即伊尹、周公也。”师冷笑曰:“汝等岂将吾比伊尹、周公!其心实指吾为王莽、董卓!”三人皆曰:“我等皆将军门下之人,安敢如此?”师大怒曰:“汝等乃口谀之人!适间与天子在密室中所哭何事?”三人曰:“实无此状。”师叱曰:“汝三人泪眼尚红,如何抵赖!”夏侯玄知事已泄,乃厉声大骂曰:“吾等所哭者,为汝威震其主,将谋篡逆耳!”师大怒,叱武士捉夏侯玄。玄揎拳裸袖,径击司马师,却被武士擒住。师令将各人搜检,于张缉身畔搜出一龙凤汗衫,上有血字。左右呈与司马师。师视之,乃密诏也。诏曰:“司马师弟兄,共持大权,将图篡逆。所行诏制,皆非朕意。各部官兵将士,可同仗忠义,讨灭贼臣,匡扶社稷。功成之日,重加爵赏。”司马师看毕,勃然大怒曰:“原来汝等正欲谋害吾兄弟!情理难容!”遂令将三人腰斩于市,灭其三族。三人骂不绝口。比临东市中,牙齿尽被打落,各人含糊数骂而死。

师直入后宫。魏主曹芳正与张皇后商议此事。皇后曰:“内廷耳目甚多,倘事泄露,必累妾矣!”正言间,忽见师入,皇后大惊。师按剑谓芳曰:“臣父立陛下为君,功德不在周公之下;臣事陛下,亦与伊尹何别乎?今反以恩为仇,以功为过,欲与二三小臣,谋害臣兄弟,何也?”芳曰:“朕无此心。”师袖中取出汗衫,掷之于地曰:“此谁人所作耶!”芳魂飞天外,魄散九霄,战栗而答曰:“此皆为他人所逼故也。朕岂敢兴此心?”师曰:“妄诬大臣造反,当加何罪?”芳跪告曰:“朕合有罪,望大将军恕之!”师曰:“陛下请起。国法未可废也。”乃指张皇后曰:“此是张缉之女,理当除之!”芳大哭求免,师不从,叱左右将张后捉出,至东华门内,用白练绞死。后人有诗曰:“当年伏后出宫门,跌足哀号别至尊。司马今朝依此例,天教还报在儿孙。”

次日,司马师大会群臣曰:“今主上荒淫无道,亵近娼优,听信谗言,闭塞贤路:其罪甚于汉之昌邑,不能主天下。吾谨按伊尹、霍光之法,别立新君,以保社稷,以安天下,如何?”众皆应曰:“大将军行伊、霍之事,所谓应天顺人,谁敢违命?”师遂同多官入永宁宫,奏闻太后。太后曰:“大将军欲立何人为君?”师曰:“臣观彭城王曹据,聪明仁孝,可以为天下之主。”太后曰:“彭城王乃老身之叔,今立为君,我何以当之?今有高贵乡公曹髦,乃文皇帝之孙;此人温恭克让,可以立之。卿等大臣,从长计议。”一人奏曰:“太后之言是也。便可立之。”众视之,乃司马师宗叔司马孚也。师遂遣使往元城召高贵乡公;请太后升太极殿,召芳责之曰:“汝荒淫无度,亵近娼优,不可承天下;当纳下玺绶,复齐王之爵,目下起程,非宣召不许入朝。”芳泣拜太后,纳了国宝,乘王车大哭而去。只有数员忠义之臣,含泪而送。后人有诗曰:“昔日曹瞒相汉时,欺他寡妇与孤儿。谁知四十余年后,寡妇孤儿亦被欺。”却说高贵乡公曹髦,字彦士,乃文帝之孙,东海定王霖之子也。当日,司马师以太后命宣至,文武官僚备銮驾于西掖门外拜迎。髦慌忙答礼。太尉王肃曰:“主上不当答礼。”髦曰:“吾亦人臣也,安得不答礼乎?”文武扶髦上辇入宫,髦辞曰:“太后诏命,不知为何,吾安敢乘辇而入?”遂步行至太极东堂。司马师迎着,髦先下拜,师急扶起。问候已毕,引见太后。后曰:“吾见汝年幼时,有帝王之相;汝今可为天下之主:务须恭俭节用,布德施仁,勿辱先帝也。”髦再三谦辞。师令文武请髦出太极殿,是日立为新君,改嘉平六年为正元元年,大赦天下,假大将军司马师黄钺,入朝不趋,奏事不名,带剑上殿。文武百官,各有封赐。

正元二年春正月,有细作飞报,说镇东将军毋丘俭、扬州刺史文钦,以废主为名,起兵前来。司马师大惊。正是:汉臣曾有勤王志,魏将还兴讨贼师。未知如何迎敌,且看下文分解。