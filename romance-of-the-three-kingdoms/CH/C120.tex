\chapter{荐杜预老将献新谋~降孙皓三分归一统}

却说吴主孙休,闻司马炎已篡魏,知其必将伐吴,忧虑成疾,卧床不起,乃召丞相濮阳兴入宫中,令太子孙【上雨下单】出拜。吴主把兴臂、手指【上雨下单】而卒。兴出,与群臣商议,欲立太子孙【上雨下单】为君。左典军万彧曰:“【上雨下单】幼不能专政,不若取乌程侯孙皓立之。”左将军张布亦曰:“皓才识明断,堪为帝王。”丞相濮阳兴不能决,入奏朱太后。太后曰:“吾寡妇人耳,安知社稷之事?卿等斟酌立之可也。”兴遂迎皓为君。

皓字元宗,大帝孙权太子孙和之子也。当年七月,即皇帝位,改元为元兴元年,封太子孙【上雨下单】为豫章王,追谥父和为文皇帝,尊母何氏为太后,加丁奉为右大司马。次年改为甘露元年。皓凶暴日甚,酷溺酒色,宠幸中常侍岑昏。濮阳兴、张布谏之,皓怒,斩二人,灭其三族。由是廷臣缄口,不敢再谏。又改宝鼎元年,以陆凯、万彧为左右丞相。时皓居武昌,扬州百姓溯流供给,甚苦之;又奢侈无度,公私匮乏。陆凯上疏谏曰:“今无灾而民命尽,无为而国财空,臣窃痛之。昔汉室既衰,三家鼎立;今曹、刘失道,皆为晋有:此目前之明验也。臣愚但为陛下惜国家耳。武昌土地险瘠,非王者之都。且童谣云:宁饮建业水,不食武昌鱼;宁还建业死,不止武昌居!此足明民心与天意也。今国无一年之蓄,有露根之渐;官吏为苛扰,莫之或恤。大帝时,后宫女不满百;景帝以来,乃有千数:此耗财之甚者也。又左右皆非其人,群党相挟,害忠隐贤,此皆蠹政病民者也。愿陛下省百役,罢苛扰,简出宫女,清选百官,则天悦民附而国安矣。”

疏奏,皓不悦。又大兴土木,作昭明宫,令文武各官入山采木;又召术士尚广,令筮蓍问取天下之事。尚对曰:“陛下筮得吉兆:庚子岁,青盖当入洛阳。”皓大喜,谓中书丞华覈曰:“先帝纳卿之言,分头命将,沿江一带,屯数百营,命老将丁奉总之。朕欲兼并汉土,以为蜀主复仇,当取何地为先?”覈谏曰:“今成都不守,社稷倾崩,司马炎必有吞吴之心。陛下宜修德以安吴民,乃为上计。若强动兵甲,正犹披麻救火,必致自焚也。愿陛下察之。”皓大怒曰:“朕欲乘时恢复旧业,汝出此不利之言!若不看汝旧臣之面,斩首号令!”叱武士推出殿门。华覈出朝叹曰:“可惜锦绣江山,不久属于他人矣!”遂隐居不出。于是皓令镇东将军陆抗部兵屯江口,以图襄阳。早有消息报入洛阳,近臣奏知晋主司马炎。晋主闻陆抗寇襄阳,与众官商议。贾充出班奏曰:“臣闻吴国孙皓,不修德政,专行无道。陛下可诏都督羊祜率兵拒之,俟其国中有变,乘势攻取,东吴反掌可得也。”炎大喜,即降诏遣使到襄阳,宣谕羊祜。祜奉诏,整点军马,预备迎敌。自是羊祜镇守襄阳,甚得军民之心。吴人有降而欲去者,皆听之。减戍逻之卒,用以垦田八百余顷。其初到时,军无百日之粮;及至末年,军中有十年之积。祜在军,尝着轻裘,系宽带,不披铠甲,帐前侍卫者不过十余人。一日,部将入帐禀祜曰:“哨马来报:吴兵皆懈怠。可乘其无备而袭之,必获大胜。”祜笑曰:“汝众人小觑陆抗耶?此人足智多谋,日前吴主命之攻拔西陵,斩了步阐及其将士数十人,吾救之无及。此人为将,我等只可自守;候其内有变,方可图取。若不审时势而轻进,此取败之道也。”众将服其论,只自守疆界而已。

一日,羊祜引诸将打猎,正值陆抗亦出猎。羊祜下令:“我军不许过界。”众将得令,止于晋地打围,不犯吴境。陆抗望见,叹曰:“羊将军有纪律,不可犯也。”日晚各退。祜归至军中,察问所得禽兽,被吴人先射伤者皆送还。吴人皆悦,来报陆抗。抗召来人入,问曰:“汝主帅能饮酒否?”来人答曰:“必得佳酿,则饮之。”抗笑曰:“吾有斗酒,藏之久矣。今付与汝持去,拜上都督:此酒陆某亲酿自饮者,特奉一勺,以表昨日出猎之情。”来人领诺,携酒而去。左右问抗曰:“将军以酒与彼,有何主意?”抗曰:“彼既施德于我,我岂得无以酬之?”众皆愕然。

却说来人回见羊祜,以抗所问并奉酒事,一一陈告。祜笑曰:“彼亦知吾能饮乎!”遂命开壶取饮。部将陈元曰:“其中恐有奸诈,都督且宜慢饮。”祜笑曰:“抗非毒人者也,不必疑虑。”竟倾壶饮之。自是使人通问,常相往来。一日,抗遣人候祜。祜问曰:“陆将军安否?”来人曰:“主帅卧病数日未出。”祜曰:“料彼之病,与我相同。吾已合成熟药在此,可送与服之。”来人持药回见抗。众将曰:“羊祜乃是吾敌也,此药必非良药。”抗曰:“岂有鸩人羊叔子哉!汝众人勿疑。”遂服之。次日病愈,众将皆拜贺。抗曰:“彼专以德,我专以暴,是彼将不战而服我也。今宜各保疆界而已,无求细利。”众将领命。忽报吴主遣使来到,抗接入问之。使曰:“天子传谕将军:作急进兵,勿使晋人先入。”抗曰:“汝先回,吾随有疏章上奏。”使人辞去,抗即草疏遣人赍到建业。近臣呈上,皓拆观其疏,疏中备言晋未可伐之状,且劝吴主修德慎罚,以安内为念,不当以黩武为事。吴主览毕,大怒曰:“朕闻抗在边境与敌人相通,今果然矣!”遂遣使罢其兵权,降为司马,却令左将军孙冀代领其军。群臣皆不敢谏。吴主皓自改元建衡,至凤凰元年,恣意妄为,穷兵屯戍,上下无不嗟怨。丞相万彧、将军留平、大司农楼玄三人见皓无道,直言苦谏,皆被所杀。前后十余年,杀忠臣四十余人。皓出入常带铁骑五万。群臣恐怖,莫敢奈何。却说羊祜闻陆抗罢兵,孙皓失德,见吴有可乘之机,乃作表遣人往洛阳请伐吴。其略曰:“夫期运虽天所授,而功业必因人而成。今江淮之险,不如剑阁;孙皓之暴,过于刘禅;吴人之困,甚于巴蜀,而大晋兵力,盛于往时:不于此际平一四海,而更阻兵相守,使天下困于征戍,经历盛衰,不可长久也。”司马炎观表,大喜,便令兴师。贾充、荀顗、冯紞三人,力言不可,炎因此不行。祜闻上不允其请,叹曰:“天下不如意事,十常八九。今天与不取,岂不大可惜哉!”至咸宁四年,羊祜入朝,奏辞归乡养病。炎间曰:“卿有何安邦之策,以教寡人?”祜曰:“孙皓暴虐已甚,于今可不战而克。若皓不幸而殁,更立贤君,则吴非陛下所能得也。”炎大悟曰:“卿今便提兵往伐,若何?”祜曰:“臣年老多病,不堪当此任。陛下另选智勇之士可也。”遂辞炎而归。

是年十一月,羊祜病危,司马炎车驾亲临其家问安。炎至卧榻前,祜下泪曰:“臣万死不能报陛下也!”炎亦泣曰:“朕深恨不能用卿伐吴之策。今日谁可继卿之志?”祜含泪而言曰:“臣死矣,不敢不尽愚诚:右将军杜预可任;劳伐吴,须当用之。”炎曰:“举善荐贤,乃美事也;卿何荐人于朝,即自焚奏稿,不令人知耶?”祜曰:“拜官公朝,谢恩私门,臣所不取也。”言讫而亡。炎大哭回宫,敕赠太傅、巨平侯。南州百姓闻羊祜死,罢市而哭。江南守边将士,亦皆哭泣。襄阳人思祜存日,常游于岘山,遂建庙立碑,四时祭之。往来人见其碑文者,无不流涕,故名为堕泪碑。后人有诗叹曰:“晓日登临感晋臣,古碑零落岘山春。松间残露频频滴,疑是当年堕泪人。”晋主以羊祜之言,拜杜预为镇南大将军都督荆州事。杜预为人,老成练达,好学不倦,最喜读左丘明《春秋传》,坐卧常自携,每出入必使人持《左传》于马前,时人谓之“《左传》癖”。及奉晋主之命,在襄阳抚民养兵,准备伐吴。

此时吴国丁奉、陆抗皆死,吴主皓每宴群臣,皆令沉醉;又置黄门郎十人为纠弹官。宴罢之后,各奏过失,有犯者或剥其面,或凿其眼。由是国人大惧。晋益州刺史王濬上疏请伐吴。其疏曰:“孙皓荒淫凶逆,宜速征伐。若一旦皓死,更立贤主,则强敌也;臣造船七年,日有朽败;臣年七十,死亡无日:三者一乖,则难图矣。愿陛下无失事机。”晋主览疏,遂与群臣议曰:“王公之论,与羊都督暗合。朕意决矣。”侍中王浑奏曰:“臣闻孙皓欲北上,军伍已皆整备,声势正盛,难与争锋。更迟一年以待其疲,方可成功。”晋主依其奏,乃降诏止兵莫动,退入后宫,与秘书丞张华围棋消遣。近臣奏边庭有表到。晋主开视之,乃杜预表也。表略云:“往者,羊祜不博谋于朝臣,而密与陛下计,故令朝臣多异同之议。凡事当以利害相校,度此举之利,十有八九,而其害止于无功耳。自秋以来,讨贼之形颇露;今若中止,孙皓恐怖,徙都武昌,完修江南诸城,迁其居民,城不可攻,野无所掠,则明年之计亦无及矣。”晋主览表才罢,张华突然而起,推却棋枰,敛手奏曰:“陛下圣武,国富民强;吴主淫虐,民忧国敝。今若讨之,可不劳而定。愿勿以为疑。”晋主曰:“卿言洞见利害,朕复何疑。”即出升殿,命镇南大将军杜预为大都督,引兵十万出江陵;镇东大将军琅琊王司马伷出涂中;安东大将军王浑出横江;建威将军王戎出武昌;平南将军胡奋出夏口:各引兵五万,皆听预调用。又遣龙骧将军王濬、广武将军唐彬,浮江东下:水陆兵二十余万,战船数万艘。又令冠军将军杨济出屯襄阳,节制诸路人马。

早有消息报入东吴。吴主皓大慌,急召丞相张悌、司徒何植、司空膝循,计议退兵之策。悌奏曰:“可令车骑将军伍延为都督,进兵江陵,迎敌杜预;骠骑将军孙歆进兵拒夏口等处军马。臣敢为军师,领左将军沈莹、右将军诸葛靓,引兵十万,出兵牛渚,接应诸路军马。”皓从之,遂令张悌引兵去了。皓退入后宫,不安忧色。幸臣中常侍岑昏问其故。皓曰:“晋兵大至,诸路已有兵迎之;争奈王濬率兵数万,战船齐备,顺流而下,其锋甚锐:朕因此忧也。”昏曰:“臣有一计,令王濬之舟,皆为齑粉矣。”皓大喜,遂问其计。岑昏奏曰:“江南多铁,可打连环索百余条,长数百丈,每环重二三十斤,于沿江紧要去处横截之。再造铁锥数万,长丈余,置于水中。若晋船乘风而来,逢锥则破,岂能渡江也?”皓大喜,传令拨匠工于江边连夜造成铁索、铁锥,设立停当。

却说晋都督杜预,兵出江陵,令牙将周旨:引水手八百人,乘小舟暗渡长江,夜袭乐乡,多立旌旗于山林之处,日则放炮擂鼓,夜则各处举火。旨领命,引众渡江,伏于巴山。次日,杜预领大军水陆并进。前哨报道:吴主遣伍延出陆路,陆景出水路,孙歆为先锋:三路来迎。”杜预引兵前进,孙歆船早到。两兵初交,杜预便退。歆引兵上岸,迤逦追时,不到二十里,一声炮响,四面晋兵大至。吴兵急回,杜预乘势掩杀,吴兵死者不计其数。孙歆奔到城边,周旨八百军混杂于中,就城上举火。歆大惊曰:“北来诸军乃飞渡江也?”急欲退时,被周旨大喝一声,斩于马下。陆景在船上,望见江南岸上一片火起,巴山上风飘出一面大旗,上书:“晋镇南大将军杜预”。陆景大惊,欲上岸逃命,被晋将张尚马到斩之。伍延见各军皆败,乃弃城走,被伏兵捉住,缚见杜预。预曰:“留之无用!”叱令武士斩之。遂得江陵。

于是沅、湘一带,直抵广州诸郡,守令皆望风赍印而降。预令人持节安抚,秋毫无犯。遂进兵攻武昌,武昌亦降,杜预军威大振,遂大会诸将,共议取建业之策。胡奋曰:“百年之寇,未可尽服。方今春水泛涨,难以久住。可俟来春,更为大举。”预曰:“昔乐毅济西一战而并强齐;今兵威大振,如破竹之势,数节之后,皆迎刃而解,无复有着手处也。”遂驰檄约会诸将,一齐进兵,攻取建业。

时龙骧将军王濬率水兵顺流而下。前哨报说:“吴人造铁索,沿江横截;又以铁锥置于水中为准备。”濬大笑,遂造大筏数十方,上缚草为人,披甲执杖,立于周围,顺水放下。吴兵见之,以为活人,望风先走。暗锥着筏,尽提而去。又于筏上作大炬,长十余丈,大十余围,以麻油灌之,但遇铁索,燃炬烧之,须臾皆断。两路从大江而来。所到之处,无不克胜。却说东吴丞相张悌,令左将军沈莹、右将军诸葛靓,来迎晋兵。莹谓靓曰:“上流诸军不作提防,吾料晋军必至此,宜尽力以敌之。若幸得胜,江南自安。今渡江与战,不幸而败,则大事去矣。”靓曰:“公言是也。”言未毕,人报晋兵顺流而下,势不可当。二人大惊,慌来见张悌商议。靓谓悌曰:“东吴危矣,何不遁去?”悌垂泣曰:“吴之将亡,贤愚共知;今若君臣皆降,无一人死于国难,不亦辱乎!”诸葛靓亦垂泣而去。张悌与沈莹挥兵抵敌,晋兵一齐围之。周旨首先杀入吴营。张悌独奋力搏战,死于乱军之中。沈莹被周旨所杀。吴兵四散败走。后人有诗赞张悌曰:“杜预巴山见大旗,江东张悌死忠时。已拚王气南中尽,不忍偷生负所知。”

却说晋兵克了牛渚,深入吴境。王濬遣人驰报捷音,晋主炎闻知大喜。贾充奏曰:“吾兵久劳于外,不服水土,必生疾病。宜召军还,再作后图。”张华曰:“今大兵已入其巢,吴人胆落,不出一月,孙皓必擒矣。若轻召还,前攻尽废,诚可惜也。”晋主未及应,贾充叱华曰:“汝不省天时地利,欲妄邀功绩,困弊士卒,虽斩汝不足以谢天下!”炎曰:“此是朕意,华但与朕同耳,何必争辩!”忽报杜预驰表到。晋主视表,亦言宜急进兵之意。晋主遂不复疑,竟下征进之命。

王濬等奉了晋主之命,水陆并进,风雷鼓动,吴人望旗而降。吴主皓闻之,大惊失色。诸臣告曰:“北兵日近,江南军民不战而降,将如之何?”皓曰:“何故不战?”众对曰:“今日之祸,皆岑昏之罪,请陛下诛之。臣等出城决一死战。”皓曰:“量一中贵,何能误国?”众大叫曰:“陛下岂不见蜀之黄皓乎!”遂不待吴主之命,一齐拥入宫中,碎割岑昏,生啖其肉。陶濬奏曰:“臣领战船皆小,愿得二万兵乘大船以战,自足破之。”皓从其言,遂拨御林诸军与陶濬上流迎敌。前将军张象,率水兵下江迎敌。二人部兵正行,不想西北风大起,吴兵旗帜,皆不能立,尽倒竖于舟中;兵卒不肯下船,四散奔走,只有张象数十军待敌。

却说晋将王濬,扬帆而行,过三山,舟师曰:“风波甚急,船不能行;且待风势少息行之。”濬大怒,拔剑叱之曰:“吾目下欲取石头城,何言住耶!”遂擂鼓大进。吴将张象引从军请降。濬曰:“若是真降,便为前部立功。”象回本船,直至石头城下,叫开城门,接入晋兵。孙皓闻晋兵已入城,欲自刎。中书今胡冲、光禄勋薛莹奏曰:“陛下何不效安乐公刘禅乎?”皓从之,亦舆榇自缚,率诸文武,诣王濬军前归降。濬释其缚,焚其榇,以王礼待之。唐人有诗叹曰:“西晋楼船下益州,金陵王气黯然收。千寻铁锁沉江底,一片降旗出石头。人世几回伤往事,山形依旧枕寒流。今逢四海为家日,故垒萧萧芦荻秋。”于是东吴四州,四十三郡,三百一十三县,户口五十二万三千,官吏三万二千,兵二十三万,男女老幼二百三十万,米谷二百八十万斛,舟船五千余艘,后官五千余人,皆归大晋。大事已定,出榜安民,尽封府库仓禀。

次日,陶濬兵不战自溃。琅琊王司马伷并王戎大兵皆至,见王濬成了大功,心中忻喜。次日,杜预亦至,大犒三军,开仓赈济吴民。于是吴民安堵。惟有建平太守吾彦,拒城不下;闻吴亡,乃降。王濬上表报捷。朝廷闻吴已平,君臣皆贺,上寿。晋主执杯流涕曰:“此羊太傅之功也,惜其不亲见之耳!”骠骑将军孙秀退朝,向南而哭曰:“昔讨逆壮年,以一校尉创立基业;今孙皓举江南而弃之!悠悠苍天,此何人哉!”

却说王濬班师,迁吴主皓赴洛阳面君。皓登殿稽首以见晋帝。帝赐坐曰:“朕设此座以待卿久矣。”皓对曰:“臣于南方,亦设此座以待陛下。”帝大笑。贾充问皓曰:“闻君在南方,每凿人眼目,剥人面皮,此何等刑耶?”皓曰:“人臣弑君及奸回不忠者,则加此刑耳。”充默然甚愧。帝封皓为归命侯,子孙封中郎,随降宰辅皆封列侯。丞相张悌阵亡,封其子孙。封王濬为辅国大将军。其余各加封赏。

自此三国归于晋帝司马炎,为一统之基矣。此所谓“天下大势,合久必分,分久必合”者也。后来后汉皇帝刘禅亡于晋泰始七年,魏主曹奂亡于太安元年,吴主孙皓亡于太康四年,皆善终。后人有古风一篇,以叙其事曰:

高祖提剑入咸阳,炎炎红日升扶桑。光武龙兴成大统,金乌飞上天中央。哀哉献帝绍海宇,红轮西坠咸池傍。何进无谋中贵乱,凉州董卓居朝堂。王允定计诛逆党,李傕郭汜兴刀枪。四方盗贼如蚁聚,六合奸雄皆鹰扬。孙坚孙策起江左,袁绍袁术兴河梁。刘焉父子据巴蜀,刘表军旅屯荆襄。张邈张鲁霸南郑,马腾韩遂守西凉。陶谦张绣公孙瓒,各逞雄才占一方。曹操专权居相府,牢笼英俊用文武。威挟天子令诸侯,总领貔貅镇中土。楼桑玄德本皇孙,义结关张愿扶主。东西奔走恨无家,将寡兵微作羁旅。南阳三顾情何深,卧龙一见分寰宇。先取荆州后取川,霸业王图在天府。呜呼三载逝升遐,白帝托孤堪痛楚。孔明六出祁山前,愿以只手将天补。何期历数到此终,长星半夜落山坞。姜维独凭气力高,九伐中原空劬劳。锺会邓艾分兵进,汉室江山尽属曹。丕睿芳髦才及奂,司马又将天下交。受禅台前云雾起,石头城下无波涛。陈留归命与安乐,王侯公爵从根苗。纷纷世事无穷尽,天数茫茫不可逃。鼎足三分已成梦,后人凭吊空牢骚。