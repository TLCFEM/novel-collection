\chapter{治风疾神医身死~传遗命奸雄数终}

却说汉中王闻关公父子遇害,哭倒于地;众文武急救,半晌方醒,扶入内殿。孔明劝
曰:“王上少忧。自古道死生有命;关公平日刚而自矜,故今日有此祸。王上且宜保养尊
体,徐图报仇。”玄德曰:“孤与关、张二弟桃园结义时,誓同生死。今云长已亡,孤岂能
独享富贵乎!”言未已,只见关兴号恸而来。玄德见了,大叫一声,又哭绝于地。众官救
醒。一日哭绝三五次,三日水浆不进,只是痛哭;泪湿衣襟,斑斑成血。孔明与众官再三劝
解。玄德曰:“孤与东吴,誓不同日月也!”孔明曰:“闻东吴将关公首级献与曹操,操以
王侯礼祭葬之。”玄德曰:“此何意也?”孔明曰:“此是东吴欲移祸于曹操,操知其谋,
故以厚礼葬关公,令王上归怨于吴也。”玄德曰:“吾今即提兵问罪于吴,以雪吾恨!”孔
明谏曰:“不可。方今吴欲令我伐魏,魏亦欲令我伐吴,各怀谲计,伺隙而乘。王上只宜按
兵不动,且与关公发丧。待吴、魏不和,乘时而伐之,可也。”众官又再三劝谏,玄德方才
进膳,传旨川中大小将士,尽皆挂孝。汉中王亲出南门招魂祭奠,号哭终日。

却说曹操在洛阳,自葬关公后,每夜合眼便见关公。操甚惊惧,问于众官。众官曰:
“洛阳行宫旧殿多妖,可造新殿居之。”操曰:“吾欲起一殿,名建始殿。恨无良工。”贾
诩曰:“洛阳良工有苏越者,最有巧思。”操召入,令画图像。苏越画成九间大殿,前后廊
庑楼阁,呈与操。操视之曰:“汝画甚合孤意,但恐无栋梁之材。”苏越曰:“此去离城三
十里,有一潭,名跃龙潭;前有一祠,名跃龙祠。祠傍有一株大梨树,高十余丈,堪作建始
殿之梁。”

操大喜,即令人工到彼砍伐。次日,回报此树锯解不开,斧砍不入,不能斩伐。操不
信,自领数百骑,直至跃龙祠前下马,仰观那树,亭亭如华盖,直侵云汉,并无曲节。操命
砍之,乡老数人前来谏曰:“此树已数百年矣,常有神人居其上,恐未可伐。”操大怒曰:
“吾平生游历,普天之下,四十余年,上至天子,下及庶人,无不惧孤;是何妖神,敢违孤
意!”言讫,拔所佩剑亲自砍之,铮然有声,血溅满身。操愕然大惊,掷剑上马,回至宫
内。是夜二更,操睡卧不安,坐于殿中,隐几而寐。忽见一人披发仗剑,身穿皂衣,直至面
前,指操喝曰:“吾乃梨树之神也。汝盖建始殿,意欲篡逆,却来伐吾神木!吾知汝数尽,
特来杀汝!”操大惊,急呼:“武士安在?”皂衣人仗剑砍操。操大叫一声,忽然惊觉,头
脑疼痛不可忍。急传旨遍求良医治疗,不能痊可。众官皆忧。

华歆入奏曰:“大王知有神医华伦否?”操曰:“即江东医周泰者乎?”歆曰:“是
也。”操曰:“虽闻其名,未知其术。”歆曰:“华佗字元化,沛国谯郡人也。其医术之
妙,世所罕有。但有患者,或用药,或用针,或用灸,随手而愈。若患五脏六腑之疾,药不
能效者,以麻肺汤饮之,令病者如醉死,却用尖刀剖开其腹,以药汤洗其脏腑,病人略无疼
痛。洗毕,然后以药线缝口,用药敷之;或一月,或二十日,即平复矣:其神妙如此!一
日,佗行于道上,闻一人呻吟之声。佗曰:此饮食不下之病。问之果然。佗令取蒜齑汁三升
饮之,吐蛇一条,长二三尺,饮食即下。广陵太守陈登,心中烦懑,面赤,不能饮食,求佗
医治。佗以药饮之,吐虫三升,皆赤头,首尾动摇。登问其故,佗曰:此因多食鱼腥,故有
此毒。今日虽可,三年之后,必将复发,不可救也。后陈登果三年而死。又有一人眉间生一
瘤,痒不可当,令佗视之。佗曰:内有飞物。人皆笑之。佗以刀割开,一黄雀飞去,病者即
愈。有一人被犬咬足指,随长肉二块,一痛一痒,俱不可忍。佗曰:痛者内有针十个,痒者
内有黑白棋子二枚。人皆不信。佗以刀割开,果应其言。此人真扁鹊,仓公之流也!现居金
城,离此不远,大王何不召之?”

操即差人星夜请华佗入内,令诊脉视疾。佗曰:“大王头脑疼痛,因患风而起。病根在
脑袋中,风涎不能出,枉服汤药,不可治疗。某有一法:先饮麻肺汤,然后用利斧砍开脑
袋,取出风涎,方可除根。”操大怒曰:“汝要杀孤耶!”佗曰:“大王曾闻关公中毒箭,
伤其右臂,某刮骨疗毒,关公略无惧色;今大王小可之疾,何多疑焉?”操曰:“臂痛可
刮,脑袋安可砍开?汝必与关公情熟,乘此机会,欲报仇耳!”呼左右拿下狱中,拷问其
情。贾诩谏曰:“似此良医,世罕其匹,未可废也。”操叱曰:“此人欲乘机害我,正与吉
平无异!”急令追拷。华佗在狱,有一狱卒,姓吴,人皆称为“吴押狱”。此人每日以酒食
供奉华佗。佗感其恩,乃告曰:“我今将死,恨有《青囊书》未传于世。感公厚意,无可为
报;我修一书,公可遣人送与我家,取《青囊书》来赠公,以继吾术。”吴押狱大喜曰:
“我若得此书,弃了此役,医治天下病人,以传先生之德。”佗即修书付吴押狱。吴押狱直
至金城,问佗之妻取了《青囊书》;回至狱中,付与华佗检看毕,佗即将书赠与吴押狱。吴
押狱持回家中藏之。旬日之后,华佗竟死于狱中。吴押狱买棺殡殓讫,脱了差役回家,欲取
《青囊书》看习,只见其妻正将书在那里焚烧。吴押狱大惊,连忙抢夺,全卷已被烧毁,只
剩得一两叶。吴押狱怒骂其妻。妻曰:“纵然学得与华佗一般神妙,只落得死于牢中,要他
何用!”吴押狱嗟叹而止。因此《青囊书》不曾传于世,所传者止阉鸡猪等小法,乃烧剩一
两叶中所载也。后人有诗叹曰:“华佗仙术比长桑,神识如窥垣一方。惆怅人亡书亦绝,后
人无复见青囊!”

却说曹操自杀华佗之后,病势愈重,又忧吴、蜀之事。正虑间,近臣忽奏东吴遣使上
书。操取书拆视之,略曰:“臣孙权久知天命已归王上,伏望早正大位,遣将剿灭刘备,扫
平两川,臣即率群下纳土归降矣。”操观毕大笑,出示群臣曰:“是儿欲使吾居炉火上
耶!”侍中陈群等奏曰:“汉室久已衰微,殿下功德巍巍,生灵仰望。今孙权称臣归命,此
天人之应,异气齐声。殿下宜应天顺人,早正大位。”操笑曰:“吾事汉多年,虽有功德及
民,然位至于王,名爵已极,何敢更有他望?苟天命在孤,孤为周文王矣。”司马懿曰:
“今孙权既称臣归附,王上可封官赐爵,令拒刘备。”操从之,表封孙权为骠骑将军、南昌
侯,领荆州牧。即日遣使赍诰敕赴东吴去讫。

操病势转加。忽一夜梦三马同槽而食,及晓,问贾诩曰:“孤向日曾梦三马同槽,疑是
马腾父子为祸;今腾已死,昨宵复梦三马同槽。主何吉凶?”诩曰:“禄马,吉兆也。禄马
归于曹,王上何必疑乎?”操因此不疑。后人有诗曰:“三马同槽事可疑,不知已植晋根
基。曹瞒空有奸雄略,岂识朝中司马师?”是夜,操卧寝室,至三更,觉头目昏眩,乃起,
伏几而卧。忽闻殿中声如裂帛,操惊视之,忽见伏皇后、董贵人、二皇子,并伏完、董承等
二十余人,浑身血污,立于愁云之内,隐隐闻索命之声。操急拔剑望空砍去,忽然一声响
亮,震塌殿宇西南一角。操惊倒于地,近侍救出,迁于别宫养病。次夜,又闻殿外男女哭声
不绝。至晓,操召群臣入曰:“孤在戎马之中,三十余年,未尝信怪异之事。今日为何如
此?”群臣奏曰:“大王当命道士设醮修禳。”操叹曰:“圣人云:获罪于天,无所祷也。
孤天命已尽,安可救乎?”遂不允设醮。

次日,觉气冲上焦,目不见物,急召夏侯惇商议。惇至殿门前,忽见伏皇后、董贵人、
二皇子、伏完、董承等,立在阴云之中。惇大惊昏倒,左右扶出,自此得病。操召曹洪、陈
群、贾诩、司马懿等,同至卧榻前,嘱以后事。曹洪等顿首曰:“大王善保玉体,不日定当
霍然。”操曰:“孤纵横天下三十余年,群雄皆灭,止有江东孙权,西蜀刘备,未曾剿除。
孤今病危,不能再与卿等相叙,特以家事相托。孤长子曹昂,刘氏所生,不幸早年殁于宛
城;今卞氏生四子:丕、彰、植、熊。孤平生所爱第三子植,为人虚华少诚实,嗜酒放纵,
因此不立。次子曹彰,勇而无谋;四子曹熊,多病难保。惟长子曹丕,笃厚恭谨,可继我
业。卿等宜辅佐之。”曹洪等涕泣领命而出。操令近侍取平日所藏名香,分赐诸侍妾,且嘱
曰:“吾死之后,汝等须勤习女工,多造丝履,卖之可以得钱自给。”又命诸妾多居于铜雀
台中,每日设祭,必令女伎奏乐上食。又遗命于彰德府讲武城外,设立疑冢七十二:“勿令
后人知吾葬处,恐为人所发掘故也。”嘱毕,长叹一声,泪如雨下。须臾,气绝而死。寿六
十六岁。时建安二十五年春正月也。后人有《邺中歌》一篇叹曹操云:“邺则邺城水漳水,
定有异人从此起:雄谋韵事与文心,君臣兄弟而父子;英雄未有俗胸中,出没岂随人眼底?
功首罪魁非两人,遗臭流芳本一身;文章有神霸有气,岂能苟尔化为群?横流筑台距太行,
气与理势相低昂;安有斯人不作逆,小不为霸大不王?霸王降作儿女鸣,无可奈何中不平;
向帐明知非有益,分香未可谓无情。呜呼!古人作事无巨细,寂寞豪华皆有意;书生轻议冢
中人,冢中笑尔书生气!”却说曹操身亡,文武百官尽皆举哀;一面遣人赴世子曹丕、鄢陵
侯曹彰、临淄侯曹植、萧怀侯曹熊处报丧。众官用金棺银椁将操入殓,星夜举灵榇赴邺郡
来。曹丕闻知父丧,放声痛哭,率大小官员出城十里,伏道迎榇入城,停于偏殿。官僚挂
孝,聚哭于殿上。忽一人挺身而出曰:“请世子息哀,且议大事。”众视之,乃中庶子司马
孚也。孚曰:“魏王既薨,天下震动;当早立嗣王,以安众心。何但哭泣耶?”群臣曰:
“世子宣嗣位,但未得天子诏命,岂可造次而行?”兵部尚书陈矫曰:“王薨于外,爱子私
立,彼此生变,则社稷危矣。”遂拔剑割下袍袖,厉声曰:“即今日便请世子嗣位。众官有
异议者,以此袍为例!”百官悚惧。

忽报华歆自许昌飞马而至,众皆大惊。须臾华歆入,众问其来意,歆曰:“今魏王薨
逝,天下震动,何不早请世子嗣位?”众官曰:“正因不及候诏命,方议欲以王后卞氏慈旨
立世子为王。”歆曰:“吾已于汉帝处索得诏命在此。”众皆踊跃称贺。歆于怀中取出诏命
开读。原来华歆谄事魏,故草此诏,威逼献帝降之;帝只得听从,故下诏即封曹丕为魏王、
丞相、冀州牧。丕即日登位,受大小官僚拜舞起居。

正宴会庆贺间,忽报鄢陵侯曹彰,自长安领十万大军来到。丕大惊,遂问群臣曰:“黄
须小弟;平日性刚,深通武艺。今提兵远来,必与孤争王位也。如之奈何?”忽阶下一人应
声出曰:“臣请往见鄢陵侯,以片言折之。”众皆曰:“非大夫莫能解此祸也。”正是:试
看曹氏丕彰事,几作袁家谭尚争。未知此人是谁,且看下文分解。