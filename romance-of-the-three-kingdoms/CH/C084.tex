\chapter{陆逊营烧七百里~孔明巧布八阵图}

却说韩当、周泰探知先主移营就凉,急来报知陆逊。逊大喜,遂引兵自来观看动静;只
见平地一屯,不满万余人,大半皆是老弱之众,大书“先锋吴班”旗号。周泰曰:“吾视此
等兵如儿戏耳。愿同韩将军分两路击之。如其不胜,甘当军令。”陆逊看了良久,以鞭指
曰:“前面山谷中。隐隐有杀气起;其下必有伏兵,故于平地设此弱兵,以诱我耳。诸公切
不可出。”众将听了,皆以为懦。

次日,吴班引兵到关前搦战,耀武扬威,辱骂不绝;多有解衣卸甲,赤身裸体,或睡或
坐。徐盛、丁奉入帐禀陆逊曰:“蜀兵欺我太甚!某等愿出击之!”逊笑曰:“公等但恃血
气之勇,未知孙、吴妙法,此彼诱敌之计也:三日后必见其诈矣。”徐盛曰:“三日后,彼
移营已定,安能击之乎?”逊曰:“吾正欲令彼移营也。”诸将哂笑而退。过三日后,会诸
将于关上观望,见吴班兵已退去。逊指曰:“杀气起矣。刘备必从山谷中出也。”言未毕,
只见蜀兵皆全装惯束,拥先主而过。吴兵见了,尽皆胆裂。逊曰:“吾之不听诸公击班者,
正为此也。今伏兵已出,旬日之内,必破蜀矣。”诸将皆曰:“破蜀当在初时,今连营五六
百里,相守经七八月,其诸要害,皆已固守,安能破乎?”逊曰:“诸公不知兵法。备乃世
之枭雄,更多智谋,其兵始集,法度精专;今守之久矣,不得我便,兵疲意阻,取之正在今
日。”诸将方才叹服。后人有诗赞曰:“虎帐谈兵按六韬,安排香饵钓鲸鳌。三分自是多英
俊,又显江南陆逊高。”却说陆逊已定了破蜀之策,遂修笺遣使奏闻孙权,言指日可以破蜀
之意。权览毕,大喜曰:“江东复有此异人,孤何忧哉!诸将皆上书言其懦,孤独不信,今
观其言,果非懦也。”于是大起吴兵来接应。却说先主于猇亭尽驱水军,顺流而下,沿江屯
扎水寨,深入吴境。黄权谏曰:“水军沿江而下,进则易,退则难。臣愿为前驱。陛下宜在
后阵,庶万无一失。”先主曰:“吴贼胆落,朕长驱大进,有何碍乎?”众官苦谏,先主不
从。遂分兵两路:命黄权督江北之兵,以防魏寇;先主自督江南诸军,夹江分立营寨,以图
进取。细作探知,连夜报知魏主,言蜀兵伐吴,树栅连营,纵横七百余里,分四十余屯,皆
傍山林下寨;今黄权督兵在江北岸,每日出哨百余里,不知何意。魏主闻之,仰面笑曰:
“刘备将败矣!”群臣请问其故。魏主曰:“刘玄德不晓兵法;岂有连营七百里,而可以拒
敌者乎?包原隰险阻屯兵者,此兵法之大忌也。玄德必败于东吴陆逊之手,旬日之内,消息
必至矣。”群臣犹未信,皆请拨兵备之。魏主曰:“陆逊若胜,必尽举吴兵去取西川;吴兵
远去,国中空虚,朕虚托以兵助战,令三路一齐进兵,东吴唾手可取也。”众皆拜服。魏主
下令,使曹仁督一军出濡须,曹休督一军出洞口,曹真督一军出南郡:“三路军马会合日
期,暗袭东吴。朕随后自来接应。”调遣已定。不说魏兵袭吴。且说马良至川,入见孔明,
呈上图本而言曰:“今移营夹江,横占七百里,下四十余屯,皆依溪傍涧,林木茂盛之处。
皇上令良将图本来与丞相观之。”孔明看讫,拍案叫苦曰:“是何人教主上如此下寨?可斩
此人!”马良曰:“皆主上自为,非他人之谋。”孔明叹曰:“汉朝气数休矣!”良问其
故。孔明曰:“包原隰险阻而结营,此兵家之大忌。倘彼用火攻,何以解救?又,岂有连营
七百里而可拒敌乎?祸不远矣!陆逊拒守不出,正为此也。汝当速去见天子,改屯诸营,不
可如此。”良曰:“倘今吴兵已胜,如之奈何?”孔明曰:“陆逊不敢来追,成都可保无
虞。”良曰:“逊何故不追?”孔明曰:“恐魏兵袭其后也。主上若有失,当投白帝城避
之。吾入川时,已伏下十万兵在鱼腹浦矣。”良大惊曰:“某于鱼腹浦往来数次,未尝见一
卒,丞相何作此诈语?”孔明曰:“后来必见,不劳多问。”马良求了表章,火速投御营
来。孔明自回成都,调拨军马救应。却说陆逊见蜀兵懈怠,不复提防,升帐聚大小将士听令
曰:“吾自受命以来,未尝出战。今观蜀兵,足知动静,故欲先取江南岸一营。谁敢去
取?”言未毕,韩当、周泰、凌统等应声而出曰:“某等愿往。”逊教皆退不用,独唤阶下
末将淳于丹曰:“吾与汝五千军,去取江南第四营:蜀将傅彤所守。今晚就要成功。吾自提
兵接应。”淳于丹引兵去了,又唤徐盛、丁奉曰:“汝等各领兵三千,屯于寨外五里,如淳
于丹败回,有兵赶来,当出救之,却不可追去。”二将自引军去了。

却说淳于丹于黄昏时分,领兵前进,到蜀寨时,已三更之后。丹令众军鼓噪而入。蜀营
内傅彤引军杀出,挺枪直取淳于丹;丹敌不住,拨马便回。忽然喊声大震,一彪军拦住去
路:为首大将赵融。丹夺路而走,折兵大半,正走之间,山后一彪蛮兵拦住:为首番将沙摩
柯。丹死战得脱,背后三路军赶来。比及离营五里,吴军徐盛、丁奉二人两下杀来,蜀兵退
去,救了淳于丹回营。丹带箭入见陆逊请罪。逊曰:“非汝之过也。吾欲试敌人之虚实耳。
破蜀之计,吾已定矣。”徐盛、丁奉曰:“蜀兵势大,难以破之,空自损兵折将耳。”逊笑
曰:“吾这条计,但瞒不过诸葛亮耳。天幸此人不在,使我成大功也。”遂集大小将士听
令:使朱然于水路进兵,来日午后东南风大作,用船装载茅草,依计而行;韩当引一军攻江
北岸,周泰引一军攻江南岸,每人手执茅草一把,内藏硫黄焰硝,各带火种,各执枪刀,一
齐而上,但到蜀营,顺风举火;蜀兵四十屯,只烧二十屯,每间一屯烧一屯。各军预带干
粮,不许暂退,昼夜追袭,只擒了刘备方止。众将听了军令,各受计而去。却说先主正在御
营寻思破吴之计,忽见帐前中军旗幡,无风自倒。乃问程畿曰:“此为何兆?”畿曰:“夜
今莫非吴兵来劫营?”先主曰:“昨夜杀尽,安敢再来?”畿曰:“倘是陆逊试敌,奈
何?”正言间,人报山上远远望见吴兵尽沿山望东去了。先主曰:“此是疑兵。”令众休
动,命关兴、张苞各引五百骑出巡。黄昏时分,关兴回奏曰:“江北营中火起。”先主急令
关兴往江北,张苞往江南,探看虚实:“倘吴兵到时,可急回报。”二将领命去了。

初更时分,东南风骤起。只见御营左屯火发。方欲救时,御营右屯又火起。风紧火急,
树木皆着,喊声大震。两屯军马齐出,奔离御营中,御营军自相践踏,死者不知其数。后面
吴兵杀到,又不知多少军马。先主急上马,奔冯习营时,习营中火光连天而起。江南、江
北,照耀如同白日。冯习慌上马引数十骑而走,正逢吴将徐盛军到,敌住厮杀。先主见了,
拨马投西便走。徐盛舍了冯习,引兵追来。先主正慌,前面又一军拦住,乃是吴将丁奉,两
下夹攻。先主大惊,四面无路。忽然喊声大震,一彪军杀入重围,乃是张苞,救了先主,引
御林军奔走。正行之间,前面一军又到,乃蜀将傅彤也,合兵一处而行。背后吴兵追至。先
主前到一山,名马鞍山。张苞、傅彤请先主上的山时,山下喊声又起:陆逊大队人马,将马
鞍山围住。张苞、傅彤死据山口。先主遥望遍野火光不绝,死尸重叠,塞江而下。次日,吴
兵又四下放火烧山,军士乱窜,先主惊慌。忽然火光中一将引数骑杀上山来,视之,乃关兴
也。兴伏地请曰:“四下火光逼近,不可久停。陛下速奔白帝城,再收军马可也。”先主
曰:“谁敢断后?”傅彤奏曰:“臣愿以死当之!”当日黄昏,关兴在前,张苞在中,留傅
彤断后,保着先主,杀下山来。吴兵见先主奔走,皆要争功,各引大军,遮天盖地,往西追
赶,先主令军士尽脱袍铠,塞道而焚,以断后军。正奔走间,喊声大震,吴将朱然引一军从
江岸边杀来,截住去路。先主叫曰:“朕死于此矣!”关兴、张苞纵马冲突,被乱箭射回,
各带重伤,不能杀出。背后喊声又起,陆逊引大军从山谷中杀来。

先主正慌急之间,此时天色已微明,只见前面喊声震天,朱然军纷纷落涧,滚滚投岩:
一彪军杀人,前来救驾。先主大喜,视之,乃常山赵子龙也。时赵云在川中江州,闻吴、蜀
交兵,遂引军出;忽见东南一带火光冲天,云心惊,远远探视,不想先主被困,云奋勇冲杀
而来。陆逊闻是赵云,急令军退。云正杀之间,忽遇朱然,便与交锋;不一合,一枪刺朱然
于马下,杀散吴兵,救出先主,望白帝城而走。先主曰:“朕虽得脱,诸将士将奈何?”云
曰:“敌军在后,不可久迟。陛下且入白帝城歇息,臣再引兵去救应诸将。”此时先主仅存
百余人入白帝城。后人有诗赞陆逊曰:“持矛举火破连营,玄德穷奔白帝城。一旦威名惊蜀
魏,吴王宁不敬书生。”

却说傅彤断后,被吴军八面围住。丁奉大叫曰:“川兵死者无数,降者极多,汝主刘备
已被擒获,今汝力穷势孤,何不早降!”傅彤叱曰:“吾乃汉将,安肯降吴狗乎!”挺枪纵
马,率蜀军奋力死战,不下百余合,往来冲突,不能得脱。彤长叹曰:“吾今休矣!”言
讫,口中吐血,死于吴军之中。后人赞傅彤诗曰:“彝陵吴蜀大交兵,陆逊施谋用火焚。至
死犹然骂吴狗,傅彤不愧汉将军。”

蜀祭酒程畿,匹马奔至江边,招呼水军赴敌,吴兵随后追来,水军四散奔逃。畿部将叫
曰:“吴兵至矣!程祭酒快走罢!”畿怒曰:“吾自从主上出军,未尝赴敌而逃!”言未
毕,吴兵骤至,四下无路,畿拔剑自刎。后人有诗赞曰:“慷慨蜀中程祭酒,身留一剑答君
王。临危不改平生志,博得声名万古香。”时吴班、张南久围彝陵城,忽冯习到,言蜀兵
败,遂引军来救先主,孙桓方才得脱。张、冯二将正行之间,前面吴兵杀来,背后孙桓从彝
陵城杀出,两下夹攻。张南、冯习奋力冲突,不能得脱,死于乱军之中。后人有诗赞曰:
“冯习忠无二,张南义少双。沙场甘战死,史册共流芳。”

吴班杀出重围,又遇吴兵追赶;幸得赵云接着,救回白帝城去了。时有蛮王沙摩柯,匹
马奔走,正逢周泰,战二十余合,被泰所杀。蜀将杜路,刘宁尽皆降吴。蜀营一应粮草器
仗,尺寸不存。蜀将川兵,降者无数。时孙夫人在吴,闻猇亭兵败,讹传先主死于军中,遂
驱车至江边,望西遥哭,投江而死。后人立庙江滨,号曰枭姬祠。尚论者作诗叹之曰:“先
主兵归白帝城,夫人闻难独捐生。至今江畔遗碑在,犹著千秋烈女名。”却说陆逊大获全
功,引得胜之兵,往西追袭。前离夔关不远,逊在马上看见前面临山傍江,一阵杀气,冲天
而起;遂勒马回顾众将曰:“前面必有埋伏,三军不可轻进。”即倒退十余里,于地势空阔
处,排成阵势,以御敌军;即差哨马前去探视。回报并无军屯在此,逊不信,下马登高望
之,杀气复起。逊再令人仔细探视,哨马回报,前面并无一人一骑。逊见日将西沉,杀气越
加,心中犹豫,令心腹人再往探看。回报江边止有乱石八九十堆,并无人马。逊大疑,令寻
土人问之。须臾,有数人到。逊问曰:“何人将乱石作堆?如何乱石堆中有杀气冲起?”土
人曰:“此处地名鱼腹浦。诸葛亮入川之时,驱兵到此,取石排成阵势于沙滩之上。自此常
常有气如云,从内而起。”陆逊听罢,上马引数十骑来看石阵,立马于山坡之上,但见四面
八方,皆有门有户。逊笑曰:“此乃惑人之术耳,有何益焉!”遂引数骑下山坡来,直入石
阵观看。部将曰:“日暮矣,请都督早回。”逊方欲出阵,忽然狂风大作,一霎时,飞沙走
石,遮天盖地。但见怪石嵯峨,槎枒似剑;横沙立土,重叠如山;江声浪涌,有如剑鼓之
声。逊大惊曰:“吾中诸葛之计也!”急欲回时,无路可出。正惊疑间,忽见一老人立于马
前,笑曰:“将军欲出此阵乎?”逊曰:“愿长者引出。”老人策杖徐徐而行,径出石阵,
并无所碍,送至山坡之上。逊问曰:“长者何人?”老人答曰:“老夫乃诸葛孔明之岳父黄
承彦也。昔小婿入川之时,于此布下石阵,名八阵图。反复八门,按遁甲休、生、伤、杜、
景、死、惊、开。每日每时,变化无端,可比十万精兵。临去之时,曾分付老夫道:后有东
吴大将迷于阵中,莫要引他出来。老夫适于山岩之上,见将军从死门而入,料想不识此阵,
必为所迷。老夫平生好善,不忍将军陷没于此,故特自生门引出也。”逊曰:“公曾学此阵
法否?”黄承彦曰:“变化无穷,不能学也。”逊慌忙下马拜谢而回。后杜工部有诗曰:
“功盖三分国,名成八阵图。江流石不转,遗恨失吞吴。”陆逊回寨,叹曰:“孔明真卧龙
也!吾不能及!”于是下令班师。左右曰:“刘备兵败势穷,困守一城,正好乘势击之;今
见石阵而退,何也?”逊曰:“吾非惧石阵而退;吾料魏主曹丕,其奸诈与父无异,今知吾
追赶蜀兵,必乘虚来袭。吾若深入西川,急难退矣。”遂令一将断后,逊率大军而回。退兵
未及二日,三处人来飞报:“魏兵曹仁出濡须,曹休出洞口,曹真出南郡:三路兵马数十
万,星夜至境,未知何意。”逊笑曰:“不出吾之所料。吾已令兵拒之矣。”正是:雄心方
欲吞西蜀,胜算还须御北朝。未知如何退兵,且看下文分解。