\chapter{张永年反难杨修~庞士元议取西蜀}

却说那进计于刘璋者,乃益州别驾,姓张,名松,字永年。其人生得额钁头尖,鼻僵齿
露,身短不满五尺,言语有若铜钟。刘璋问曰:“别驾有何高见,可解张鲁之危?”松曰:
“某闻许都曹操,扫荡中原,吕布、二袁皆为所灭,近又破马超,天下无敌矣。主公可备进
献之物,松亲往许都,说曹操兴兵取汉中,以图张鲁。则鲁拒敌不暇,何敢复窥蜀中耶?”
刘璋大喜,收拾金珠锦绮,为进献之物,遣张松为使。松乃暗画西川地理图本藏之,带从人
数骑,取路赴许都。早有人报入荆州。孔明便使人入许都打探消息。

却说张松到了许都馆驿中住定,每日去相府伺候,求见曹操。原来曹操自破马超回,傲
睨得志,每日饮宴,无事少出,国政皆在相府商议。张松候了三日,方得通姓名。左右近侍
先要贿赂,却才引入。操坐于堂上,松拜毕,操问曰:“汝主刘璋连年不进贡,何也?”松
曰:“为路途艰难,贼寇窃发,不能通进。”操叱曰:“吾扫清中原,有何盗贼?”松曰:
“南有孙权,北有张鲁,西有刘备,至少者亦带甲十余万,岂得为太平耶?”操先见张松人
物猥琐,五分不喜;又闻语言冲撞,遂拂袖而起,转入后堂。左右责松曰:“汝为使命,何
不知礼,一味冲撞?幸得丞相看汝远来之面,不见罪责。汝可急急回去!”松笑曰:“吾川
中无诌佞之人也。”忽然阶下一人大喝曰:“汝川中不会谄佞,吾中原岂有谄佞者乎?”

松观其人,单眉细眼,貌白神清。问其姓名,乃太尉杨彪之子杨修,字德祖,现为丞相
门下掌库主簿。此人博学能言,智识过人。松知修是个舌辩之士,有心难之。修亦自恃其
才,小觑天下之士。当时见张松言语讥讽,遂邀出外面书院中,分宾主而坐,谓松曰:“蜀
道崎岖,远来劳苦。”松曰:“奉主之命,虽赴汤蹈火,弗敢辞也。”修问:“蜀中风土何
如?”松曰:“蜀为西郡,古号益州。路有锦江之险,地连剑阁之雄。回还二百八程,纵横
三万余里。鸡鸣犬吠相闻,市井闾阎不断。田肥地茂,岁无水旱之忧;国富民丰,时有管弦
之乐。所产之物,阜如山积。天下莫可及也!”修又问曰:“蜀中人物如何?”松曰:“文
有相如之赋,武有伏波之才;医有仲景之能,卜有君平之隐。九流三教,出乎其类,拔乎其
萃者,不可胜记,岂能尽数!”修又问曰:“方今刘季玉手下,如公者还有几人?”松曰:
“文武全才,智勇足备,忠义慷慨之士,动以百数。如松不才之辈,车载斗量,不可胜
记。”修曰:“公近居何职?”松曰:“滥充别驾之任,甚不称职。敢问公为朝廷何官?”
修曰:“现为丞相府主簿。”松曰:“久闻公世代簪缨,何不立于庙堂,辅佐天子,乃区区
作相府门下一吏乎?”杨修闻言,满面羞惭,强颜而答曰:“某虽居下寮,丞相委以军政钱
粮之重,早晚多蒙丞相教诲,极有开发,故就此职耳。”松笑曰:“松闻曹丞相文不明孔、
孟之道,武不达孙、吴之机,专务强霸而居大位,安能有所教诲,以开发明公耶?”修曰:
“公居边隅,安知丞相大才乎?吾试令公观之。”呼左右于箧中取书一卷,以示张松。松观
其题曰《孟德新书》。从头至尾,看了一遍,共一十三篇,皆用兵之要法。松看毕,问曰:
“公以此为何书耶?”修曰:“此是丞相酌古准今,仿《孙子》十三篇而作。公欺丞相无
才,此堪以传后世否?”松大笑曰:“此书吾蜀中三尺小童,亦能暗诵,何为‘新书’?此
是战国时无名氏所作,曹丞相盗窃以为己能,止好瞒足下耳!”修曰:“丞相秘藏之书,虽
已成帙,未传于世。公言蜀中小儿暗诵如流,何相欺乎?”松曰:“公如不信,吾试诵
之。”遂将《孟德新书》,从头至尾,朗诵一遍,并无一字差错。修大惊曰:“公过目不
忘,真天下奇才也!”后人有诗赞曰:“古怪形容异,清高体貌疏。语倾三峡水,目视十行
书。胆量魁西蜀,文章贯太虚。百家并诸子,一览更无余。”

当下张松欲辞回。修曰:“公且暂居馆舍,容某再禀丞相,令公面君。”松谢而退。修
入见操曰:“适来丞相何慢张松乎?”操曰:“言语不逊,吾故慢之。”修曰:“丞相尚容
一祢衡,何不纳张松?”操曰:“祢衡文章,播于当今,吾故不忍杀之。松有何能?”修
曰:“且无论其口似悬河,辩才无碍。适修以丞相所撰《孟德新书》示之,彼观一遍,即能
暗诵,如此博闻强记,世所罕有。松言此书乃战国时无名氏所作,蜀中小儿,皆能熟记。”
操曰:“莫非古人与我暗合否?”令扯碎其书烧之。修曰:“此人可使面君,教见天朝气
象。”操曰:“来日我于西教场点军,汝可先引他来,使见我军容之盛,教他回去传说:吾
即日下了江南,便来收川。”修领命。

至次日,与张松同至西教场。操点虎卫雄兵五万,布于教场中。果然盔甲鲜明,衣袍灿
烂;金鼓震天,戈矛耀日;四方八面,各分队伍;旌旗扬彩,人马腾空。松斜目视之。良
久,操唤松指而示曰:“汝川中曾见此英雄人物否?”松曰:“吾蜀中不曾见此兵革,但以
仁义治人。”操变色视之。松全无惧意。杨修频以目视松。操谓松曰:“吾视天下鼠辈犹草
芥耳。大军到处,战无不胜,攻无不取,顺吾者生,逆吾者死。汝知之乎?”松曰:“丞相
驱兵到处,战必胜,攻必取,松亦素知。昔日濮阳攻吕布之时,宛城战张绣之日;赤壁遇周
郎,华容逢关羽;割须弃袍于潼关,夺船避箭于渭水:此皆无敌于天下也!”操大怒曰:
“竖儒怎敢揭吾短处!”喝令左右推出斩之。杨修谏曰:“松虽可斩,奈从蜀道而来入贡,
若斩之,恐失远人之意。”操怒气未息。荀彧亦谏。操方免其死,令乱棒打出。松归馆舍,
连夜出城,收拾回川。松自思曰:“吾本欲献西川州郡与曹操,谁想如此慢人!我来时于刘
璋之前,开了大口;今日怏怏空回。须被蜀中人所笑。吾闻荆州刘玄德仁义远播久矣,不如
径由那条路回。试看此人如何,我自有主见。”于是乘马引仆从望荆州界上而来,前至郢州
界口,忽见一队军马,约有五百余骑,为首一员大将,轻妆软扮,勒马前问曰:“来者莫非
张别驾乎?”松曰:“然也。”那将慌忙下马,声喏曰:“赵云等候多时。”松下马答礼
曰:“莫非常山赵子龙乎?”云曰:“然也,某奉主公刘玄德之命,为大夫远涉路途,鞍马
驱驰,特命赵云聊奉酒食。”言罢,军士跪奉酒食,云敬进之。松自思曰:“人言刘玄德宽
仁爱客,今果如此。”遂与赵云饮了数杯,上马同行。来到荆州界首,是日天晚,前到馆
驿,见驿门外百余人侍立,击鼓相接。一将于马前施礼曰:“奉兄长将令,为大夫远涉风
尘,令关某洒扫驿庭,以待歇宿。”松下马,与云长、赵云同入馆舍。讲礼叙坐。须臾,排
上酒筵,二人殷勤相劝。饮至更阑,方始罢席,宿了一宵。

次日早膳毕,上马行不到三五里,只见一簇人马到。乃是玄德引着伏龙、凤雏,亲自来
接。遥见张松,早先下马等候。松亦慌忙下马相见。玄德曰:“久闻大夫高名,如雷灌耳。
恨云山遥远,不得听教。今闻回都,专此相接。倘蒙不弃,到荒州暂歇片时,以叙渴仰之
思,实为万幸!”松大喜,遂上马并辔入城。至府堂上各各叙礼,分宾主依次而坐,设宴款
待。饮酒间,玄德只说闲话,并不提起西川之事。松以言挑之曰:“今皇叔守荆州,还有几
郡?”孔明答曰:“荆州乃暂借东吴的,每每使人取讨。今我主因是东吴女婿,故权且在此
安身。”松曰:“东吴据六郡八十一州,民强国富,犹且不知足耶?”庞统曰:“吾主汉朝
皇叔,反不能占据州郡;其他皆汉之蟊贼,却都恃强侵占地土;惟智者不平焉。”玄德曰:
“二公休言。吾有何德,敢多望乎?”松曰:“不然。明公乃汉室宗亲,仁义充塞乎四海。
休道占据州郡,便代正统而居帝位,亦非分外。”玄德拱手谢曰:“公言太过,备何敢
当!”

自此一连留张松饮宴三日,并不提起川中之事。松辞去,玄德于十里长亭设宴送行。玄
德举酒酌松曰:“甚荷大夫不外,留叙三日;今日相别,不知何时再得听教。”言罢,潸然
泪下。张松自思:“玄德如此宽仁爱士,安可舍之?不如说之,令取西川。”乃言曰:“松
亦思朝暮趋侍,恨未有便耳。松观荆州:东有孙权,常怀虎踞;北有曹操,每欲鲸吞。亦非
可久恋之地也。”玄德曰:“故知如此,但未有安迹之所。”松曰:“益州险塞,沃野千
里,民殷国富;智能之士,久慕皇叔之德。若起荆襄之众,长驱西指,霸业可成,汉室可兴
矣。”玄德曰:“备安敢当此?刘益州亦帝室宗亲,恩泽布蜀中久矣。他人岂可得而动摇
乎?”松曰:“某非卖主求荣;今遇明公,不敢不披沥肝胆:刘季玉虽有益州之地,禀性暗
弱,不能任贤用能;加之张鲁在北,时思侵犯;人心离散,思得明主。松此一行,专欲纳款
于操;何期逆贼恣逞奸雄,傲贤慢士,故特来见明公。明公先取西川为基,然后北图汉中,
收取中原,匡正天朝,名垂青史,功莫大焉。明公果有取西川之意,松愿施犬马之劳,以为
内应。未知钧意若何?”玄德曰:“深感君之厚意。奈刘季玉与备同宗,若攻之,恐天下人
唾骂。”松曰:“大丈夫处世,当努力建功立业,著鞭在先。今若不取,为他人所取,悔之
晚矣。”玄德曰:“备闻蜀道崎岖,千山万水,车不能方轨,马不能联辔;虽欲取之,用何
良策?”松于袖中取出一图,递与玄德曰:“深感明公盛德,敢献此图。但看此图,便知蜀
中道路矣。”玄德略展视之,上面尽写着地理行程,远近阔狭,山川险要,府库钱粮,一一
俱载明白。松曰:“明公可速图之。松有心腹契友二人:法正、孟达。此二人必能相助。如
二人到荆州时,可以心事共议。”玄德拱手谢曰:“青山不老,绿水长存。他日事成,必当
厚报。”松曰:“松遇明主,不得不尽情相告,岂敢望报乎?”说罢作别。孔明命云长等护
送数十里方回。张松回益州,先见友人法正。正字孝直,右扶风郿人也,贤士法真之子。松
见正,备说曹操轻贤傲士,只可同忧,不可同乐。吾已将益州许刘皇叔矣。专欲与兄共议。
法正曰:“吾料刘璋无能,已有心见刘皇叔久矣。此心相同,又何疑焉?”少顷,孟达至。
达字子庆,与法正同乡。达入,见正与松密语。达曰:“吾已知二公之意。将欲献益州
耶?”松曰:“是欲如此。兄试猜之,合献与谁?”达曰:“非刘玄德不可。”三人抚掌大
笑。法正谓松曰:“兄明日见刘璋,当若何?”松曰:“吾荐二公为使,可往荆州。”二人
应允。

次日,张松见刘璋。璋问:“干事若何?”松曰:“操乃汉贼,欲篡天下,不可为言。
彼已有取川之心。”璋曰:“似此如之奈何?”松曰;“松有一谋,使张鲁、曹操必不敢轻
犯西川。”璋曰:“何计?”松曰:“荆州刘皇叔,与主公同宗,仁慈宽厚,有长者风。赤
壁鏖兵之后,操闻之而胆裂,何况张鲁乎?”主公何不遣使结好,使为外援,可以拒曹操、
张鲁矣。”璋曰:“吾亦有此心久矣。谁可为使?”松曰:“非法正、孟达,不可往也。”
璋即召二人入,修书一封,令法正为使,先通情好;次遣孟达领精兵五千,迎玄德入川为
援。正商议间,一人自外突入,汗流满面,大叫曰:“主公若听张松之言,则四十一州郡,
已属他人矣!”松大惊;视其人,乃西阆中巴人,姓黄,名权,字公衡,现为刘璋府下主
簿。璋问曰:“玄德与我同宗,吾故结之为援;汝何出此言?”权曰:“某素知刘备宽以待
人,柔能克刚,英雄莫敌;远得人心,近得民望;兼有诸葛亮、庞统之智谋,关、张、赵
云、黄忠、魏延为羽翼。若召到蜀中,以部曲待之,刘备安肯伏低做小?若以客礼待之,又
一国不容二主。今听臣言,则西蜀有泰山之安;不听臣言,主公有累卵之危矣。张松昨从荆
州过,必与刘备同谋。可先斩张松,后绝刘备,则西川万幸也。”璋曰:“曹操、张鲁到
来,何以拒之?”权曰:“不如闭境绝塞,深沟高垒,以待时清。”璋曰:“贼兵犯界,有
烧眉之急;若待时清,则是慢计也。”遂不从其言,遣法正行。又一人阻曰:“不可!不
可!”璋视之,乃帐前从事官王累也。累顿首言曰:“主公今听张松之说,自取其祸。”璋
曰:“不然。吾结好刘玄德,实欲拒张鲁也。”累曰:“张鲁犯界,乃癣疥之疾;刘备入
川,乃心腹之大患。况刘备世之枭雄,先事曹操,便思谋害;后从孙权,便夺荆州。心术如
此,安可同处乎?”今若召来,西川休矣!”璋叱曰:“再休乱道!玄德是我同宗,他安肯
夺我基业?”便教扶二人出。遂命法正便行。

法正离益州,径取荆州,来见玄德。参拜已毕,呈上书信。玄德拆封视之。书曰:“族
弟刘璋,再拜致书于玄德宗兄将军麾下:久伏电天,蜀道崎岖,未及赍贡,甚切惶愧。璋闻
吉凶相救,患难相扶,朋友尚然,况宗族乎?今张鲁在北,旦夕兴兵,侵犯璋界,甚不自
安。专人谨奉尺书,上乞钧听。倘念同宗之情,全手足之义,即日兴师剿灭狂寇,永为唇
齿,自有重酬。书不尽言,耑候车骑。”玄德看毕大喜,设宴相待法正。酒过数巡,玄德屏
退左右,密谓正曰:“久仰孝直英名,张别驾多谈盛德。今获听教,甚慰平生。”法正谢
曰:“蜀中小吏,何足道哉!盖闻马逢伯乐而嘶,人遇知己而死。张别驾昔日之言,将军复
有意乎?”玄德曰:“备一身寄客,未尝不伤感而叹息。尝思鹪鹩尚存一枝,狡兔犹藏三
窟,何况人乎?蜀中丰余之地,非不欲取;奈刘季玉系备同宗,不忍相图。”法正曰:“益
州天府之国,非治乱之主,不可居也,今刘季玉不能用贤,此业不久必属他人。今日自付与
将军,不可错失。岂不闻逐兔先得之语乎?将军欲取,某当效死。”玄德拱手谢曰:“尚容
商议。”

当日席散,孔明亲送法正归馆舍。玄德独坐沉吟。庞统进曰:“事当决而不决者,愚人
也。主公高明,何多疑耶?”玄德问曰:“以公之意,当复何如?”统曰:“荆州东有孙
权,北有曹操,难以得志。益州户口百万,土广财富,可资大业。今幸张松、法正为内助,
此天赐也。何必疑哉?”玄德曰:“今与吾水火相敌者,曹操也。操以急,吾以宽;操以
暴,吾以仁;操以谲,吾以忠:每与操相反,事乃可成。若以小利而失信义于天下,吾不忍
也。”庞统笑曰:“主公之言,虽合天理,奈离乱之时,用兵争强,固非一道;若拘执常
理,寸步不可行矣,宜从权变。且兼弱攻昧、逆取顺守,汤、武之道也。若事定之后,报之
以义,封为大国,何负于信?今日不取,终被他人取耳。主公幸熟思焉。”玄德乃恍然曰:
“金石之言,当铭肺腑。”于是遂请孔明,同议起兵西行。孔明曰:“荆州重地,必须分兵
守之。”玄德曰:“吾与庞士元、黄忠、魏延前往西川;军师可与关云长、张翼德、赵子龙
守荆州。”孔明应允。于是孔明总守荆州;关公拒襄阳要路,当青泥隘口;张飞领四郡巡
江,赵云屯江陵,镇公安。玄德令黄忠为前部,魏延为后军,玄德自与刘封、关平在中军。
庞统为军师,马步兵五万,起程西行。临行时,忽廖化引一军来降。玄德便教廖化辅佐云长
以拒曹操。

是年冬月,引兵望西川进发。行不数程,孟达接着,拜见玄德,说刘益州令某领兵五千
远来迎接。玄德使人入益州,先报刘璋。璋便发书告报沿途州郡,供给钱粮。璋欲自出涪城
亲接玄德,即下令准备车乘帐幔,旌旗铠甲,务要鲜明。主簿黄权入谏曰:“主公此去,必
被刘备之害,某食禄多年,不忍主公中他人奸计。望三思之!”张松曰:“黄权此言,疏间
宗族之义,滋长寇盗之威,实无益于主公。”璋乃叱权曰:“吾意已决,汝何逆吾!”权叩
首流血,近前口衔璋衣而谏。璋大怒,扯衣而起。权不放,顿落门牙两个。璋喝左右,推出
黄权。权大哭而归。璋欲行,一人叫曰:“主公不纳黄公衡忠言,乃欲自就死地耶!”伏于
阶前而谏。璋视之,乃建宁俞元人也,姓李,名恢。叩首谏曰:“窃闻君有诤臣,父有诤
子。黄公衡忠义之言,必当听从。若容刘备入川,是犹迎虎于门也。”璋曰:“玄德是吾宗
兄,安肯害吾?再言者必斩!”叱左右推出李恢。张松曰:“今蜀中文官各顾妻子,不复为
主公效力;诸将恃功骄傲,各有外意。不得刘皇叔,则敌攻于外,民攻于内,必败之道
也。”璋曰:“公所谋,深于吾有益。”次日,上马出榆桥门。人报从事王累,自用绳索倒
吊于城门之上,一手执谏章,一手仗剑,口称如谏不从,自割断其绳索,撞死于此地。刘璋
教取所执谏章观之。其略曰:“益州从事臣王累,泣血恳告:窃闻良药苦口利于病,忠言逆
耳利于行。昔楚怀王不听屈原之言,会盟于武关,为秦所困。今主公轻离大郡,欲迎刘备于
涪城,恐有去路而无回路矣。倘能斩张松于市,绝刘备之约,则蜀中老幼幸甚,主公之基业
亦幸甚!”刘璋观毕,大怒曰:“吾与仁人相会,如亲芝兰,汝何数侮于吾耶!”王累大叫
一声,自割断其索,撞死于地,后人有诗叹曰:“倒挂城门捧谏章,拚将一死报刘璋。黄权
折齿终降备,矢节何如王累刚!”刘璋将三万人马往涪城来。后军装载资粮饯帛一千余辆,
来接玄德。却说玄德前军已到垫江。所到之处,一者是西川供给;二者是玄德号令严明,如
有妄取百姓一物者斩:于是所到之处,秋毫无犯。百姓扶老携幼,满路瞻观,焚香礼拜。玄
德皆用好言抚慰。却说法正密谓庞统曰:“近张松有密书到此,言于涪城相会刘璋,便可图
之。机会切不可失。”统曰:“此意且勿言。待二刘相见,乘便图之。若预走泄,于中有
变。”法正乃秘而不言。涪城离成都三百六十里。璋已到,使人迎接玄德。两军皆屯于涪江
之上。玄德入城,与刘璋相见,各叙兄弟之情。礼毕,挥泪诉告衷情。饮宴毕,各回寨中安
歇。

璋谓众官曰:“可笑黄权、王累等辈,不知宗兄之心,妄相猜疑。吾今日见之,真仁义
之人也。吾得他为外援,又何虑曹操、张鲁耶?非张松则失之矣。”乃脱所穿绿袍,并黄金
五百两,令人往成都赐与张松。时部下将佐刘璝、泠苞、张任、邓贤等一班文武官曰:“主
公且休欢喜。刘备柔中有刚,其心未可测,还宜防之。”璋笑曰:“汝等皆多虑。吾兄岂有
二心哉!”众皆嗟叹而退。

却说玄德归到寨中。庞统入见曰:“主公今日席上见刘季玉动静乎?”玄德吾:“季玉
真诚实人也。”统曰:“季玉虽善,其臣刘璝、张任等皆有不平之色,其间吉凶未可保也。
以统之计,莫若来日设宴,请季玉赴席;于壁衣中埋伏刀斧手一百人,主公掷杯为号,就筵
上杀之;一拥入成都,刀不出鞘,弓不上弦,可坐而定也。”玄德曰:“季玉是吾同宗,诚
心待吾;更兼吾初到蜀中,恩信未立;若行此事,上天不容,下民亦怨。公此谋,虽霸者亦
不为也。”统曰:“此非统之谋,是法孝直得张松密书,言事不宜迟,只在早晚当图之。”
言未已,法正入见,曰:“某等非为自己,乃顺天命也。”玄德曰:“刘季玉与吾同宗,不
忍取之。”正曰:“明公差矣。若不如此,张鲁与蜀有杀母之仇,必来攻取。明公远涉山
川,驱驰士马,既到此地,进则有功,退则无益。若执狐疑之心,迁延日久,大为失计。且
恐机谋一泄,反为他人所算。不若乘此天与人归之时,出其不意,早立基业,实为上策。”
庞统亦再三相劝。正是:人主几番存厚道,才臣一意进权谋。未知玄德心下如何,且看下文
分解。