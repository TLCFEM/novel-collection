\chapter{急兄仇张飞遇害~雪弟恨先主兴兵}

却说先主欲起兵东征,赵云谏曰:“国贼乃曹操,非孙权也。今曹丕篡汉,神人共怒。
陛下可早图关中,屯兵渭河上流,以讨凶逆,则关东义士,必裹粮策马以迎王师;若舍魏以
伐吴,兵势一交,岂能骤解。愿陛下察之。”先主曰:“孙权害了朕弟;又兼傅士仁、糜
芳、潘璋、马忠皆有切齿之仇:啖其肉而灭其族,方雪朕恨!卿何阻耶?”云曰:“汉贼之
仇,公也;兄弟之仇,私也。愿以天下为重。”先主答曰:“朕不为弟报仇,虽有万里江
山,何足为贵?”遂不听赵云之谏,下令起兵伐吴;且发使往五溪,借番兵五万,共相策
应;一面差使往阆中,迁张飞为车骑将军,领司隶校尉,封西乡侯,兼阆中牧。使命赍诏而
去。却说张飞在阆中,闻知关公被东吴所害,旦夕号泣,血湿衣襟。诸将以酒解劝,酒醉,
怒气愈加。帐上帐下,但有犯者即鞭挞之;多有鞭死者。每日望南切齿睁目怒恨,放声痛哭
不已。忽报使至,慌忙接入,开读诏旨。飞受爵望北拜毕,设酒款待来使。飞曰:“吾兄被
害,仇深似海;庙堂之臣,何不早奏兴兵?”使者曰:“多有劝先灭魏而后伐吴者。”飞怒
曰:“是何言也!昔我三人桃园结义,誓同生死;今不幸二兄半途而逝,吾安得独享富贵
耶!吾当面见天子,愿为前部先锋,挂孝伐吴,生擒逆贼,祭告二兄,以践前盟!”言讫,
就同使命望成都而来。却说先主每日自下教场操演军马,克日兴师,御驾亲征。于是公卿都
至丞相府中见孔明,曰:“今天子初临大位,亲统军伍,非所以重社稷也。丞相秉钧衡之
职,何不规谏?”孔明曰:“吾苦谏数次,只是不听。今日公等随我入教场谏去。”当下孔
明引百官来奏先主曰:“陛下初登宝位,若欲北讨汉贼,以伸大义于天下,方可亲统六师;
若只欲伐吴,命一上将统军伐之可也,何必亲劳圣驾?”先主见孔明苦谏,心中稍回。忽报
张飞到来,先主急召入。飞至演武厅拜伏于地,抱先主足而哭。先主亦哭。飞曰:“陛下今
日为君,早忘了桃园之誓!二兄之仇,如何不报?”先主曰:“多官谏阻,未敢轻举。”飞
曰:“他人岂知昔日之盟?若陛下不去,臣舍此躯与二兄报仇!若不能报时,臣宁死不见陛
下也!”先主曰:“朕与卿同往:卿提本部兵自阆州而出,朕统精兵会于江州,共伐东吴,
以雪此恨!”飞临行,先主嘱曰:“朕素知卿酒后暴怒,鞭挞健儿,而复令在左右:此取祸
之道也。今后务宜宽容,不可如前。”飞拜辞而去。次日,先主整兵要行。学士秦宓奏曰:
“陛下舍万乘之躯,而徇小义,古人所不取也。愿陛下思之。”先主曰:“云长与朕,犹一
体也。大义尚在,岂可忘耶?”宓伏地不起曰:“陛下不从臣言,诚恐有失。”先主大怒
曰:“朕欲兴兵,尔何出此不利之言!”叱武士推出斩之,宓面不改色,回顾先主而笑曰:
“臣死无恨,但可惜新创之业,又将颠覆耳!”众官皆为秦宓告免。先主曰:“暂且囚下,
待朕报仇回时发落。”孔明闻知,即上表救秦宓。其略曰:“臣亮等窃以吴贼逞奸诡之计,
致荆州有覆亡之祸;陨将星于斗牛,折天柱于楚地:此情哀痛,诚不可忘。但念迁汉鼎者,
罪由曹操;移刘祚者,过非孙权。窃谓魏贼若除,则吴自宾服。愿陛下纳秦宓金石之言,以
养士卒之力,别作良图,则社稷幸甚!天下幸甚!”先主看毕,掷表于地曰:“朕意已决,
无得再谏!”遂命丞相诸葛亮保太子守两川;骠骑将军马超并弟马岱,助镇北将军魏延守汉
中,以当魏兵;虎威将军赵云为后应,兼督粮草;黄权、程畿为参谋;马良、陈震掌理文
书;黄忠为前部先锋;冯习、张南为副将;傅彤、张翼为中军护尉;赵融、廖淳为合后。川
将数百员,并五溪番将等,共兵七十五万,择定章武元年七月丙寅日出师。却说张飞回到阆
中,下令军中;限三日内制办白旗白甲,三军挂孝伐吴。次日,帐下两员末将范疆、张达,
入帐告曰:“白旗白甲,一时无措,须宽限方可。飞大怒曰:“吾急欲报仇,恨不明日便到
逆贼之境,汝安敢违我将令!”叱武士缚于树上,各鞭背五十。鞭毕,以手指之曰:“来日
俱要完备!若违了限,即杀汝二人示众!”打得二人满口出血。回到营中商议,范疆曰:
“今日受了刑责,着我等如何办得?其人性暴如火,倘来日不完,你我皆被杀矣!”张达
曰:“比如他杀我,不如我杀他。”疆曰:“怎奈不得近前。”达曰:“我两个若不当死,
则他醉于床上;若是当死,则他不醉。”二人商议停当。

却说张飞在帐中,神思昏乱,动止恍惚,乃问部将曰:“吾今心惊肉颠,坐卧不安,此
何意也?”部将答曰:“此是君侯思念关公,以致如此。”飞令人将酒来,与部将同饮,不
觉大醉,卧于帐中。范、张二贼,探知消息,初更时分,各藏短刀,密入帐中,诈言欲禀机
密重事,直至床前。原来张飞每睡不合眼;当夜寝于帐中,二贼见他须竖目张,本不敢动
手。因闻鼻息如雷,方敢近前,以短刀刺入飞腹。飞大叫一声而亡。时年五十五岁。后人有
诗叹曰:“安喜曾闻鞭督邮,黄巾扫尽佐炎刘。虎牢关上声先震,长坂桥边水逆流。义释严
颜安蜀境,智欺张郃定中州。伐吴未克身先死,秋草长遗阆地愁。”却说二贼当夜割了张飞
首级,便引数十人连夜投东吴去了。次日,军中闻知,起兵追之不及。时有张飞部将吴班,
向自荆州来见先主,先主用为牙门将,使佐张飞守阆中。当下吴班先发表章,奏知天子;然
后令长子张苞具棺椁盛贮,令弟张绍守阆中,苞自来报先主。时先主已择期出师。大小官
僚,皆随孔明送十里方回。孔明回至成都,怏怏不乐,顾谓众官曰:“法孝直若在,必能制
主上东行也。”

却说先主是夜心惊肉颤,寝卧不安。出帐仰观天文,见西北一星,其大如斗,忽然坠
地。先主大疑,连夜令人求问孔明。孔明回奏曰:“合损一上将。三日之内,必有惊报。”
先主因此按兵不动。忽侍臣奏曰:“阆中张车骑部将吴班,差人赍表至。”先主顿足曰:
“噫!三弟休矣!”及至览表,果报张飞凶信。先主放声大哭,昏绝于地。众官救醒。

次日,人报一队军马骤风而至。先主出营观之。良久,见一员小将,白袍银铠,滚鞍下
马,伏地而哭,乃张苞也。苞曰:“范疆、张达杀了臣父,将首级投吴去了!”先主哀痛至
甚,饮食不进。群臣苦谏曰:“陛下方欲为二弟报仇,何可先自摧残龙体?”先主方才进
膳,遂谓张苞曰:“卿与吴班,敢引本部军作先锋,为卿父报仇否?”苞曰:“为国为父,
万死不辞!”先主正欲遣苞起兵,又报一彪军风拥而至。先主令侍臣探之。须臾,侍臣引一
小将军,白袍银铠,入营伏地而哭。先主视之,乃关兴也。先主见了关兴,想起关公,又放
声大哭。众官苦劝。先主曰:“朕想布衣时,与关、张结义,誓同生死;今朕为天子,正欲
与两弟同享富贵,不幸俱死于非命!见此二侄,能不断肠!”言讫又哭。众官曰:“二小将
军且退。容圣上将息龙体。”侍臣奏曰:“陛下年过六旬,不宜过于哀痛。”先主曰:“二
弟俱亡,朕安忍独生!”言讫,以头顿地而哭。

多官商议曰:“今天子如此烦恼,将何解劝?”马良曰:“主上亲统大兵伐吴,终日号
泣,于军不利。”陈震曰:“吾闻成都青城山之西,有一隐者,姓李,名意。世人传说此老
已三百余岁,能知人之生死吉凶,乃当世之神仙也。何不奏知天子,召此老来,问他吉凶,
胜如吾等之言。”遂入奏先主。先主从之,即遣陈震赍诏,往青城山宣召。震星夜到了青
城,令乡人引入出谷深处,遥望仙庄,清云隐隐,瑞气非凡。忽见一小童来迎曰:“来者莫
非陈孝起乎?”震大惊曰:“仙童如何知我姓字!”童子曰:“吾师昨者有言:今日必有皇
帝诏命至;使者必是陈孝起。”震曰:“真神仙也!人言信不诬矣!”遂与小童同入仙庄,
拜见李意,宣天子诏命。李意推老不行。震曰:“天子急欲见仙翁一面,幸勿吝鹤驾。”再
三敦请,李意方行。即至御营,入见先主。先主见李意鹤发童颜,碧眼方瞳,灼灼有光,身
如古柏之状,知是异人,优礼相待。李意曰:“老夫乃荒山村叟,无学无识。辱陛下宣召,
不知有何见谕?”先主曰:“朕与关、张二弟生死之交,三十余年矣。今二弟被害,亲统大
军报仇,未知休咎如何。久闻仙翁通晓玄机,望乞赐教。”李意曰:“此乃天数,非老夫所
知也。”先主再三求问,意乃索纸笔画兵马器械四十余张,画毕便一一扯碎。又画一大人仰
卧于地上,傍边一人掘土埋之,上写一大“白”字,遂稽首而去。先主不悦,谓群臣曰:
“此狂叟也!不足为信。”即以火焚之,便催军前进。

张苞入奏曰:“吴班军马已至。小臣乞为先锋。”先主壮其志,即取先锋印赐张苞。苞
方欲挂印,又一少年将奋然出曰:“留下印与我!”视之,乃关兴也。苞曰:“我已奉诏
矣。”兴曰:“汝有何能,敢当此任?”苞曰:“我自幼习学武艺,箭无虚发。”先主曰:
“朕正要观贤侄武艺,以定优劣。”苞令军士于百步之外,立一面旗,旗上画一红心。苞拈
弓取箭,连射三箭,皆中红心。众皆称善。关兴挽弓在手曰:“射中红心何足为奇?”正言
间,忽值头上一行雁过。兴指曰:“吾射这飞雁第三只。”一箭射去,那只雁应弦而落。文
武官僚,齐声喝采。苞大怒,飞身上马,手挺父所使丈八点钢矛,大叫曰:“你敢与我比试
武艺否?”兴亦上马,绰家传大砍刀纵马而出曰:“偏你能使矛!吾岂不能使刀!”

二将方欲交锋,先主喝曰:“二子休得无礼!”兴、苞二人慌忙下马,各弃兵器,拜伏
请罪。先主曰:“朕自涿郡与卿等之父结异姓之交,亲如骨肉;今汝二人亦是昆仲之分,正
当同心协力,共报父仇;奈何自相争竞,失其大义!父丧未远而犹如此,况日后乎?”二人
再拜伏罪。先主问曰:“卿二人谁年长?”苞曰:“臣长关兴一岁。”先主即命兴拜苞为
兄。二人就帐前折箭为誓,永相救护。先主下诏使吴班为先锋,令张苞、关兴护驾。水陆并
进,船骑双行,浩浩荡荡,杀奔吴国来。却说范疆、张达将张飞首级,投献吴侯,细告前
事。孙权听罢,收了二人,乃谓百官曰:“今刘玄德即了帝位,统精兵七十余万,御驾亲
征,其势甚大,如之奈何?”百官尽皆失色,面面相觑。诸葛瑾出曰:“某食君侯之禄久
矣,无可报效,愿舍残生,去见蜀主,以利害说之,使两国相和,共讨曹丕之罪。”权大
喜,即遣诸葛瑾为使,来说先主罢兵。正是:两国相争通使命,一言解难赖行人。未知诸葛
瑾此去如何,且看下文分解。