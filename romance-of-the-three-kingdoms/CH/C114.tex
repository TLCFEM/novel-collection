\chapter{曹髦驱车死南阙~姜维弃粮胜魏兵}

却说姜维传令退兵,廖化曰:“将在外,君命有所不受。今虽有诏,未可动也。”张翼曰:“蜀人为大将军连年动兵,皆有怨望;不如乘此得胜之时,收回人马,以安民心,再作良图。”维曰:“善。”遂令各军依法而退。命廖化、张翼断后,以防魏兵追袭。却说邓艾引兵追赶,只见前面蜀兵旗帜整齐,人马徐徐而退。艾叹曰:“姜维深得武侯之法也!”因此不敢追赶,勒军回祁山寨去了。且说姜维至成都,入见后主,问召回之故。后主曰:“朕为卿在边庭,久不还师,恐劳军士,故诏卿回朝,别无他意。”维曰:“臣已得祁山之寨,正欲收功,不期半途而废。此必中邓艾反间之计矣。”后主默然不语。姜维又奏曰:“臣誓讨贼,以报国恩。陛下休听小人之言,致生疑虑。”后主良久乃曰:“朕不疑卿;卿且回汉中,俟魏国有变,再伐之可也。”姜维叹息出朝,自投汉中去讫。

却说党均回到祁山寨中,报知此事。邓艾与司马望曰:“君臣不和,必有内变。”就令党均入洛阳,报知司马昭。昭大喜,便有图蜀之心,乃问中护军贾充曰:“吾今伐蜀,如何?”充曰:“未可伐也。天子方疑主公,若一旦轻出,内难必作矣。旧年黄龙两见于宁陵井中,群臣表贺,以为祥瑞;天子曰:‘非祥瑞也。龙者君象,乃上不在天,下不在田,屈于井中,是幽困之兆也。’遂作《潜龙诗》一首。诗中之意,明明道着主公。其诗曰:‘伤哉龙受困,不能跃深渊。上不飞天汉,下不见于田。蟠居于井底,鳅鳝舞其前。藏牙伏爪甲,嗟我亦同然!’”司马昭闻之大怒,谓贾充曰:“此人欲效曹芳也!若不早图,彼必害我。”充曰:“某愿为主公早晚图之。”时魏甘露五年夏四月,司马昭带剑上殿,髦起迎之。群臣皆奏曰:“大将军功德巍巍,合为晋公,加九锡。”髦低头不答。昭厉声曰:“吾父子兄弟三人有大功于魏,今为晋公,得毋不宜耶?”髦乃应曰:“敢不如命?”昭曰:“《潜龙》之诗,视吾等如鳅鳝,是何礼也?”髦不能答。昭冷笑下殿,众官凛然。髦归后宫,召侍中王沈、尚书王经、散骑常侍王业三人,入内计议。髦泣曰:“司马昭将怀篡逆,人所共知!朕不能坐受废辱,卿等可助朕讨之!”王经奏曰:“不可。昔鲁昭公不忍季氏,败走失国;今重权已归司马氏久矣,内外公卿,不顾顺逆之理,阿附奸贼,非一人也。且陛下宿卫寡弱,无用命之人。陛下若不隐忍,祸莫大焉。且宜缓图,不可造次。”髦曰:“是可忍也,孰不可忍也!朕意已决,便死何惧!”言讫,即入告太后。王沈、王业谓王经曰:“事已急矣。我等不可自取灭族之祸,当往司马公府下出首,以免一死。”经大怒曰:“主忧臣辱,主辱臣死,敢怀二心乎?”王沈、王业见经不从,径自往报司马昭去了。少顷,魏主曹髦出内,令护卫焦伯,聚集殿中宿卫苍头官僮三百余人,鼓噪而出。髦仗剑升辇,叱左右径出南阙。王经伏于辇前,大哭而谏曰:“今陛下领数百人伐昭,是驱羊而入虎口耳,空死无益。臣非惜命,实见事不可行也!”髦曰:“吾军已行,卿无阻当。”遂望云龙门而来。

只见贾充戎服乘马,左有成倅,右有成济,引数千铁甲禁兵,呐喊杀来。髦仗剑大喝曰:“吾乃天子也!汝等突入宫庭,欲弑君耶?”禁兵见了曹髦,皆不敢动。贾充呼成济曰:“司马公养你何用?正为今日之事也!”济乃绰戟在手,回顾充曰:“当杀耶?当缚耶?”充曰:“司马公有令;只要死的。”成济撚戟直奔辇前。髦大喝曰:“匹夫敢无礼乎!”言未讫,被成济一戟刺中前胸,撞出辇来;再一戟,刃从背上透出,死于辇傍。焦伯挺枪来迎,被成济一戟刺死。众皆逃走。王经随后赶来,大骂贾充曰:“逆贼安敢弑君耶!”充大怒,叱左右缚定,报知司马昭。昭入内,见髦已死,乃佯作大惊之状,以头撞辇而哭,令人报知各大臣。

时太傅司马孚入内,见髦尸,首枕其股而哭曰:“弑陛下者,臣之罪也!”遂将髦尸用棺椁盛贮,停于偏殿之西。昭入殿中,召群臣会议。群臣皆至,独有尚书仆射陈泰不至。昭令泰之舅尚书荀顗召之。泰大哭曰:“论者以泰比舅,今舅实不如泰也。”乃披麻带孝而入,哭拜于灵前。昭亦佯哭而问曰:“今日之事,何法处之?”泰曰:“独斩贾充,少可以谢天下耳。”昭沉吟良久,又问曰:“再思其次?”泰曰:“惟有进于此者,不知其次。”昭曰:“成济大逆不道,可剐之,灭其三族。”济大骂昭曰:“非我之罪,是贾充传汝之命!”昭令先割其舌。济至死叫屈不绝。弟成倅亦斩于市,尽灭三族。后人有诗叹曰:“司马当年命贾充,弑君南阙赭袍红。却将成济诛三族,只道军民尽耳聋。”

昭又使人收王经全家下狱。王经正在廷尉厅下,忽见缚其母至。经叩头大哭曰:“不孝子累及慈母矣!”母大笑曰:“人谁不死?正恐不得死所耳!以此弃命,何恨之有!”次日,王经全家皆押赴东市。王经母子含笑受刑。满城士庶,无不垂泪。后人有诗曰:“汉初夸伏剑,汉末见王经:真烈心无异,坚刚志更清。节如泰华重,命似鸿毛轻。母子声名在,应同天地倾。”太傅司马孚请以王礼葬曹髦,昭许之。贾充等劝司马昭受魏禅,即天子位。昭曰:昔文王三分天下有其二,以服事殷,故圣人称为至德。魏武帝不肯受禅于汉,犹吾之不肯受禅于魏也。”贾充等闻言,已知司马昭留意于子司马炎矣,遂不复劝进。是年六月,司马昭立常道乡公曹璜为帝,改元景元元年。璜改名曹免,字景明。乃武帝曹操之孙,燕王曹宇之子也。奂封昭为相国、晋公,赐钱十万、绢万匹。其文武多官,各有封赏。早有细作报入蜀中。姜维闻司马昭弑了曹髦,立了曹奂,喜曰:“吾今日伐魏,又有名矣。”遂发书入吴,令起兵问司马昭弑君之罪;一面奏准后主,起兵十五万,车乘数千辆,皆置板箱于上;令廖化、张翼为先锋:化取子午谷,翼取骆谷;维自取斜谷,皆要出祁山之前取齐。三路兵并起,杀奔祁山而来。时邓艾在祁山寨中,训练人马,闻报蜀兵三路杀到,乃聚诸将计议。参军王瓘曰:“吾有一计,不可明言,现写在此,谨呈将军台览。”艾接来展看毕,笑曰:“此计虽妙,只怕瞒不过姜维。”瓘曰:“某愿舍命前去。”艾曰:“公志若坚,必能成功。”遂拨五千兵与瓘。瓘连夜从斜谷迎来,正撞蜀兵前队哨马。瓘叫曰:“我是魏国降兵,可报与主帅。”

哨军报知姜维,维令拦住余兵,只教为首的将来见。瓘拜伏于地曰:“某乃王经之侄王瓘也。近见司马昭弑君,将叔父一门皆戮,某痛恨入骨。今幸将军兴师问罪,故特引本部兵五千来降。愿从调遣,剿除奸党,以报叔父之恨。”维大喜,谓瓘曰:“汝既诚心来降,吾岂不诚心相待?吾军中所患者,不过粮耳。今有粮车数千,现在川口,汝可运赴祁山。吾只今去取祁山寨也。”瓘心中大喜,以为中计,忻然领诺。姜维曰:“汝去运粮,不必用五千人,但引三千人去,留下二千人引路,以打祁山。”瓘恐维疑惑,乃引三千兵去了。维令傅佥引二千魏兵随征听用。忽报夏侯霸到。霸曰:“都督何故准信王瓘之言也?吾在魏,虽不知备细,未闻王瓘是王经之侄。其中多诈,请将军察之。”维大笑曰:“我已知王瓘之诈,故分其兵势,将计就计而行。”霸曰:“公试言之。”维曰:“司马昭奸雄比于曹操,既杀王经,灭其三族,安肯存亲侄于关外领兵?故知其诈也。仲权之见,与我暗合。”于是姜维不出斜谷,却令人于路暗伏,以防王瓘奸细。不旬日,果然伏兵捉得王瓘回报邓艾下书人来见。维问了情节,搜出私书,书中约于八月二十日,从小路运粮送归大寨,却教邓艾遣兵于墵山谷中接应。维将下书人杀了,却将书中之意,改作八月十五日,约邓艾自率大兵,于墵山谷中接应。一面令人扮作魏军往魏营下书;一面令人将现有粮车数百辆卸了粮米,装载干柴茅草引火之物,用青布罩之,令傅佥引二千原降魏兵,执打运粮旗号。维却与夏侯霸各引一军,去山谷中埋伏。令蒋舒出斜谷,廖化、张翼俱各进兵,来取祁山。却说邓艾得了王瓘书信,大喜,急写回书,今来人回报。至八月十五日,邓艾引五万精兵径往墵山谷中来,远远使人凭高眺探,只见无数粮车,接连不断,从山凹中而行。艾勒马望之,果然皆是魏兵。左右曰:“天已昏暮,可速接应王瓘出谷口。”艾曰:“前面山势掩映,倘有伏兵,急难退步;只可在此等候。”正言间,忽两骑马骤至,报曰:“王将军因将粮草过界,背后人马赶来,望早救应。”艾大惊,急催兵前进。

时值初更,月明如昼,只听得山后呐喊,艾只道王瓘在山后厮杀。径奔过山后时,忽树林后一彪军撞出,为首蜀将傅佥,纵马大叫曰:“邓艾匹夫!已中吾主将之计,何不早早下马受死!”艾大惊,勒回马便走。车上火尽着,那火便是号火。两势下蜀兵尽出,杀得魏兵七断八续,但闻四下山上只叫:“拿住邓艾的,赏千金,封万户侯!”?得邓艾弃甲丢盔,撇了坐下马,杂在步军之中,爬山越岭而逃。姜维、夏侯霸只望马上为首的径来擒捉,不想邓艾步行走脱。维领得胜兵去接王瓘粮车。却说王瓘密约邓艾,先期将粮草车仗,整备停当,专候举事。忽有心腹人报:“事已泄漏,邓将军大败,不知性命如何。”瓘大惊,令人哨探,回报三路兵围杀将来,背后又见尘头大起,四下无路。瓘叱左右令放火,尽烧粮草车辆。一霎时,火光突起,烈火烧空。灌大叫曰:“事已急矣!汝等宜死战!”乃提兵望西杀出。背后姜维三路追赶。维只道王瓘舍命撞回魏国,不想反杀入汉中而去。瓘因兵少,只恐追兵赶上,遂将栈道并各关隘尽皆烧毁。姜维恐汉中有失,遂不追邓艾,提兵连夜抄小路来追杀王瓘。瓘被四面蜀兵攻击,投黑龙江而死。余兵尽被姜维坑之。维虽然胜了邓艾,却折了许多粮车,又毁了栈道,乃引兵还汉中。邓艾引部下败兵,逃回祁山寨内,上表请罪,自贬其职。司马昭见艾数有大功,不忍贬之,复加厚赐。艾将原赐财物,尽分给被害将士之家。昭恐蜀兵又出,遂添兵五万,与艾守御。姜维连夜修了栈道,又议出师。正是:连修栈道兵连出,不伐中原死不休。未知胜负如何,且看下文分解。