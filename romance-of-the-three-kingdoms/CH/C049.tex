\chapter{七星坛诸葛祭风~三江口周瑜纵火}

却说周瑜立于山顶,观望良久,忽然望后而倒,口吐鲜血,不省人事。左右救回帐中。诸将皆来动问,尽皆愕然相顾曰:“江北百万之众,虎踞鲸吞。不争都督如此,倘曹兵一至,如之奈何?”慌忙差人申报吴侯,一面求医调治。

却说鲁肃见周瑜卧病,心中忧闷,来见孔明,言周瑜卒病之事。孔明曰:“公以为何如?”肃曰:“此乃曹操之福,江东之祸也。”孔明笑曰:“公瑾之病,亮亦能医。”肃曰:“诚如此,则国家万幸!”即请孔明同去看病。肃先入见周瑜。瑜以被蒙头而卧。肃曰:“都督病势若何?”周瑜曰:“心腹搅痛,时复昏迷。”肃曰:“曾服何药饵?”瑜曰:“心中呕逆,药不能下。”肃曰:“适来去望孔明,言能医都督之病。现在帐外,烦来医治,何如?”瑜命请入,教左右扶起,坐于床上。孔明曰:“连日不晤君颜,何期贵体不安!”瑜曰:“人有旦夕祸福,岂能自保?”孔明笑曰:“天有不测风云,人又岂能料乎?”瑜闻失色,乃作呻吟之声。孔明曰:“都督心中似觉烦积否?”瑜曰:“然,”孔明曰:“必须用凉药以解之。”瑜曰:“已服凉药,全然无效。”孔明曰:“须先理其气;气若顺,则呼吸之间,自然痊可。”瑜料孔明必知其意,乃以言挑之曰:“欲得顺气,当服何药?”孔明笑曰:“亮有一方,便教都督气顺。”瑜曰:“愿先生赐教。”孔明索纸笔,屏退左右,密书十六字曰:“欲破曹公,宜用火攻;万事俱备,只欠东风。”写毕,递与周瑜曰:“此都督病源也。”瑜见了大惊,暗思:“孔明真神人也!早已知我心事!只索以实情告之。”乃笑曰:“先生已知我病源,将用何药治之?事在危急,望即赐教。”孔明曰:“亮虽不才,曾遇异人,传授奇门遁甲天书,可以呼风唤雨。都督若要东南风时,可于南屏山建一台,名曰七星坛:高九尺,作三层,用一百二十人,手执旗幡围绕。亮于台上作法,借三日三夜东南大风,助都督用兵,何如?”瑜曰:“休道三日三夜,只一夜大风,大事可成矣。只是事在目前,不可迟缓。”孔明曰:“十一月二十日甲子祭风,至二十二日丙寅风息,如何?”瑜闻言大喜,矍然而起。便传令差五百精壮军士,往南屏山筑坛;拨一百二十人,执旗守坛,听候使令。

孔明辞别出帐,与鲁肃上马,来南屏山相度地势,令军士取东南方赤土筑坛。方圆二十四丈,每一层高三尺,共是九尺。下一层插二十八宿旗:东方七面青旗,按角、亢、氏、房、心、尾、箕,布苍龙之形;北方七面皂旗,按斗、牛、女、虚、危、室、壁,作玄武之势;西方七面白旗,按奎、娄、胃、昴、毕、觜、参,踞白虎之威;南方七面红旗,按井、鬼、柳、星、张、翼、轸,成朱雀之状。第二层周围黄旗六十四面,按六十四卦,分八位而立。上一层用四人,各人戴束发冠,穿皂罗袍,凤衣博带,朱履方裾。前左立一人,手执长竿,竿尖上用鸡羽为葆。以招风信;前右立一人,手执长竿,竿上系七星号带,以表风色;后左立一人,捧宝剑;后右立一人,捧香炉。坛下二十四人,各持旌旗、宝盖、大戟、长戈、黄钺、白旄、朱幡、皂纛,环绕四面。

孔明于十一月二十日甲子吉辰,沐浴斋戒,身披道衣,跣足散发,来到坛前。分付鲁肃曰:“子敬自往军中相助公瑾调兵。倘亮所祈无应,不可有怪。”鲁肃别去。孔明嘱付守坛将士:“不许擅离方位。不许交头接耳。不许失口乱言。不许失惊打怪。如违令者斩!”众皆领命。孔明缓步登坛,观瞻方位已定,焚香于炉,注水于盂,仰天暗祝。下坛入帐中少歇,令军士更替吃饭。孔明一日上坛三次,下坛三次。却并不见有东南风。且说周瑜请程普、鲁肃一班军官,在帐中伺候,只等东南风起,便调兵出;一面关报孙权接应。黄盖已自准备火船二十只,船头密布大钉;船内装载芦苇干柴,灌以鱼油,上铺硫黄、焰硝引火之物,各用青布油单遮盖;船头上插青龙牙旗,船尾各系走舸:在帐下听候,只等周瑜号令。甘宁、阚泽窝盘蔡和、蔡中在水寨中,每日饮酒,不放一卒登岸;周围尽是东吴军马,把得水泄不通:只等帐上号令下来。周瑜正在帐中坐议,探子来报:“吴侯船只离寨八十五里停泊,只等都督好音。”瑜即差鲁肃遍告各部下官兵将士:“俱各收拾船只、军器、帆橹等物。号令一出,时刻休违。倘有违误,即按军法。”众兵将得令,一个个磨拳擦掌,准备厮杀。

是日,看看近夜,天色清明,微风不动。瑜谓鲁肃曰:“孔明之言谬矣。隆冬之时,怎得东南风乎?”肃曰:“吾料孔明必不谬谈。”将近三更时分,忽听风声响,旗幡转动。瑜出帐看时,旗脚竟飘西北。霎时间东南风大起,瑜骇然曰:“此人有夺天地造化之法、鬼神不测之术!若留此人,乃东吴祸根也。及早杀却,免生他日之忧。”急唤帐前护军校尉丁奉、徐盛二将:“各带一百人。徐盛从江内去,丁奉从旱路去,都到南屏山七星坛前,休问长短,拿住诸葛亮便行斩首,将首级来请功。”二将领命。徐盛下船,一百刀斧手荡开棹桨;丁奉上马,一百弓弩手各跨征驹:往南屏山来。于路正迎着东南风起。后人有诗曰:“七星坛上卧龙登,一夜东风江水腾。不是孔明施妙计,周郎安得逞才能?”

丁奉马军先到,见坛上执旗将士,当风而立。丁奉下马提剑上坛,不见孔明,慌问守坛将士。答曰:“恰才下坛去了。”丁奉忙下坛寻时,徐盛船已到。二人聚于江边。小卒报曰:“昨晚一只快船停在前面滩口。适间却见孔明披发下船,那船望上水去了。”丁奉、徐盛便分水陆两路追袭。徐盛教拽起满帆,抢风而使。遥望前船不远,徐盛在船头上高声大叫:“军师休去!都督有请!”只见孔明立于船尾大笑曰:“上覆都督:好好用兵;诸葛亮暂回夏口,异日再容相见。”徐盛曰:“请暂少住,有紧话说。”孔明曰:“吾已料定都督不能容我,必来加害,预先教赵子龙来相接。将军不必追赶。”徐盛见前船无篷,只顾赶来。看看至近,赵云拈弓搭箭,立于船尾大叫曰:“吾乃常山赵子龙也!奉令特来接军师。你如何来追赶?本待一箭射死你来,显得两家失了和气。——教你知我手段!”言讫,箭到处,射断徐盛船上篷索。那篷堕落下水,其船便横。赵云却教自己船上拽起满帆,乘顺风而去。其船如飞,追之不及。岸上丁奉唤徐盛船近岸,言曰:“诸葛亮神机妙算,人不可及。更兼赵云有万夫不当之勇,汝知他当阳长坂时否?吾等只索回报便了。”于是二人回见周瑜,言孔明预先约赵云迎接去了。周瑜大惊曰:“此人如此多谋,使我晓夜不安矣!”鲁肃曰:“且待破曹之后,却再图之。”

瑜从其言,唤集诸将听令。先教甘宁:“带了蔡中并降卒沿南岸而走,只打北军旗号,直取乌林地面,正当曹操屯粮之所,深入军中,举火为号。只留下蔡和一人在帐下,我有用处。”第二唤太史慈分付:“你可领三千兵,直奔黄州地界,断曹操合淝接应之兵,就逼曹兵,放火为号;只看红旗,便是吴侯接应兵到。”这两队兵最远,先发。第三唤吕蒙领三千兵去乌林接应甘宁,焚烧曹操寨栅,第四唤凌统领三千兵,直截彝陵界首,只看乌林火起,以兵应之。第五唤董袭领三千兵,直取汉阳,从汉川杀奔曹操案中。看白旗接应。第六唤潘璋领三千兵,尽打白旗,往汉阳接应董袭。六队船只各自分路去了。却令黄盖安排火船,使小卒驰书约曹操,今夜来降。一面拨战船四只,随于黄盖船后接应。第一队领兵军官韩当,第二队领兵军官周泰,第三队领兵军官蒋钦,第四队领兵军官陈武:四队各引战船三百只,前面各摆列火船二十只。周瑜自与程普在大艨艟上督战,徐盛、丁奉为左右护卫,只留鲁肃共阚泽及众谋士守寨。程普见周瑜调军有法,甚相敬服。却说孙权差使命持兵符至,说已差陆逊为先锋,直抵蕲、黄地面进兵,吴侯自为后应。瑜又差人西山放火炮,南屏山举号旗。各各准备停当,只等黄昏举动。

话分两头。且说刘玄德在夏口专候孔明回来,忽见一队船到,乃是公子刘琦自来探听消息。玄德请上敌楼坐定,说:“东南风起多时,子龙去接孔明,至今不见到,吾心甚忧。”小校遥指樊口港上:“一帆风送扁舟来到,必军师也。”玄德与刘琦下楼迎接。须臾船到,孔明、子龙登岸。玄德大喜。问候毕,孔明曰:“且无暇告诉别事。前者所约军马战船,皆已办否?”玄德曰:“收拾久矣,只候军师调用。”

孔明便与玄德、刘琦升帐坐定,谓赵云曰:“子龙可带三千军马,渡江径取乌林小路,拣树木芦苇密处埋伏。今夜四更已后,曹操必然从那条路奔走。等他军马过,就半中间放起火来。虽然不杀他尽绝,也杀一半。”云曰:“乌林有两条路:一条通南郡,一条取荆州。不知向那条路来?”孔明曰:“南郡势迫,曹操不敢往;必来荆州,然后大军投许昌而去。”云领计去了。又唤张飞曰:“翼德可领三千兵渡江,截断彝陵这条路,去葫芦谷口埋伏。曹操不敢走南彝陵,必望北彝陵去。来日雨过,必然来埋锅造饭。只看烟起,便就山边放起火来。虽然不捉得曹操,翼德这场功料也不小。”飞领计去了。又唤糜竺、糜芳、刘封三人各驾船只,绕江剿擒败军,夺取器械。三人领计去了。孔明起身,谓公子刘琦曰:“武昌一望之地。最为紧要。公子便请回,率领所部之兵,陈于岸口。操一败必有逃来者,就而擒之,却不可轻离城郭。”刘琦便辞玄德、孔明去了。孔明谓玄德曰:“主公可于樊口屯兵,凭高而望,坐看今夜周郎成大功也。”

时云长在侧,孔明全然不睬。云长忍耐不住,乃高声曰:“关某自随兄长征战,许多年来,未尝落后。今日逢大敌,军师却不委用,此是何意?”孔明笑曰:“云长勿怪!某本欲烦足下把一个最紧要的隘口,怎奈有些违碍,不敢教去。”云长曰:“有何违碍?愿即见谕。”孔明曰:“昔日曹操待足下甚厚,足下当有以报之。今日操兵败,必走华容道;若令足下去时,必然放他过去。因此不敢教去。”云长曰:“军师好心多!当日曹操果是重待某,某已斩颜良,诛文丑,解白马之围,报过他了。今日撞见,岂肯放过!”孔明曰:“倘若放了时,却如何?”云长曰:“愿依军法!”孔明曰:“如此,立下文书。”云长便与了军令状。”云长曰:“若曹操不从那条路上来,如何?”孔明曰:“我亦与你军令状。云长大喜。孔明曰:“云长可于华容小路高山之处,堆积柴草,放起一把火烟,引曹操来。”云长曰:“曹操望见烟,知有埋伏,如何肯来?”孔明笑曰:“岂不闻兵法虚虚实实之论?操虽能用兵,只此可以瞒过他也。他见烟起,将谓虚张声势,必然投这条路来。将军休得容情。”云长领了将令,引关平、周仓并五百校刀手,投华容道埋伏去了。玄德曰:“吾弟义气深重,若曹操果然投华容道去时,只恐端的放了。”孔明曰:“亮夜观乾象,操贼未合身亡。留这人情,教云长做了,亦是美事。”玄德曰:“先生神算,世所罕及!”孔明遂与玄德往樊口,看周瑜用兵,留孙乾、简雍守城。却说曹操在大寨中,与众将商议,只等黄盖消息。当日东南风起甚紧。程昱入告曹操曰:“今日东南风起,宜预提防。”操笑曰:“冬至一阳生,来复之时,安得无东南风?何足为怪!”军士忽报江东一只小船来到,说有黄盖密书。操急唤入。其人呈上书。书中诉说:“周瑜关防得紧,因此无计脱身。今有鄱阳湖新运到粮,周瑜差盖巡哨,已有方便。好歹杀江东名将,献首来降。只在今晚二更,船上插青龙牙旗者,即粮船也。”操大喜,遂与众将来水寨中大船上,观望黄盖船到。

且说江东,天色向晚,周瑜唤出蔡和,令军士缚倒。和叫:“无罪!”瑜曰:“汝是何等人,敢来诈降!吾今缺少福物祭旗,愿借你首级。”和抵赖不过,大叫曰:“汝家阚泽、甘宁亦曾与谋!”瑜曰:“此乃吾之所使也。”蔡和悔之无及。瑜令捉至江边皂纛旗下,奠酒烧纸,一刀斩了蔡和,用血祭旗毕,便令开船。黄盖在第三只火船上,独披掩心,手提利刃,旗上大书“先锋黄盖”。盖乘一天顺风,望赤壁进发。是时东风大作,波浪汹涌。操在中军遥望隔江,看看月上,照耀江水,如万道金蛇,翻波戏浪。操迎风大笑,自以为得志。忽一军指说:“江南隐隐一簇帆幔,使风而来。”操凭高望之。报称:“皆插青龙牙旗。内中有大旗,上书先锋黄盖名字。”操笑曰:“公覆来降,此天助我也!”来船渐近。程昱观望良久,谓操曰:“来船必诈。且休教近寨。”操曰:“何以知之!”程昱曰:“粮在船中,船必稳重;今观来船,轻而且浮。更兼今夜东南风甚紧,倘有诈谋,何以当之?”操省悟,便问:“谁去止之?”文聘曰:“某在水上颇熟,愿请一往。”言毕,跳下小船,用手一指,十数只巡船,随文聘船出。聘立于船头,大叫:“丞相钧旨:南船且休近寨,就江心抛住。”众军齐喝:“快下了篷!”言未绝,弓弦响处,文聘被箭射中左臂,倒在船中。船上大乱,各自奔回。南船距操寨止隔二里水面。黄盖用刀一招,前船一齐发火。火趁风威,风助火势,船如箭发,烟焰涨天。二十只火船,撞入水寨,曹寨中船只一时尽着;又被铁环锁住,无处逃避。隔江炮响,四下火船齐到,但见三江面上,火逐风飞,一派通红,漫天彻地。

曹操回观岸上营寨,几处烟火。黄盖跳在小船上,背后数人驾舟,冒烟突火,来寻曹操。操见势急,方欲跳上岸,忽张辽驾一小脚船,扶操下得船时,那只大船,已自着了。张辽与十数人保护曹操,飞奔岸口。黄盖望见穿绛红袍者下船,料是曹操,乃催船速进,手提利刃,高声大叫:“曹贼休走!黄盖在此!”操叫苦连声。张辽拈弓搭箭,觑着黄盖较近,一箭射去。此时风声正大,黄盖在火光中,那里听得弓弦响?”正中肩窝,翻身落水。正是:火厄盛时遭水厄,棒疮愈后患金疮。未知黄盖性命如何,且看下文分解。