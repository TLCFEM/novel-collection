\chapter{关云长义释黄汉升~孙仲谋大战张文远}

却说孔明谓张飞曰:“前者子龙取桂阳郡时,责下军令状而去。今日翼德要取武陵,必须也责下军令状,方可领兵去。”张飞遂立军令状,欣然领三千军,星夜投武陵界上来。金旋听得张飞引兵到,乃集将校,整点精兵器械,出城迎敌。从事巩志谏曰:“刘玄德乃大汉皇叔,仁义布于天下;加之张翼德骁勇非常。不可迎敌,不如纳降为上。”金旋大怒曰:“汝欲与贼通连为内变耶?”喝令武士推出斩之。众官皆告曰:“先斩家人,于军不利。”金旋乃喝退巩志,自率兵出。离城二十里,正迎张飞。飞挺矛立马,大喝金旋。旋问部将:“谁敢出战?”众皆畏惧,莫敢向前。旋自骤马舞刀迎之。张飞大喝一声,浑如巨雷,金旋失色,不敢交锋,拨马便走。飞引众军随后掩杀。金旋走至城边,城上乱箭射下。旋惊视之,见巩志立于城上曰:“汝不顺天时,自取败亡,吾与百姓自降刘矣。”言未毕,一箭射中金旋面门,坠于马下,军士割头献张飞。巩志出城纳降,飞就令巩志赍印绶,往桂阳见玄德。玄德大喜,遂令巩志代金旋之职。

玄德亲至武陵安民毕,驰书报云长,言翼德、子龙各得一郡。云长乃回书上请曰:“闻长沙尚未取,如兄长不以弟为不才,教关某干这件功劳甚好。”玄德大喜,遂教张飞星夜去替云长守荆州,令云长来取长沙。

云长既至,入见玄德、孔明。孔明曰:“子龙取桂阳,翼德取武陵,都是三千军去。今长沙太守韩玄,固不足道。只是他有一员大将,乃南阳人,姓黄,名忠,字汉升;是刘表帐下中郎将,与刘表之侄刘磐共守长沙,后事韩玄;虽今年近六旬却有万夫不当之勇,不可轻敌。云长去,必须多带军马。”云长曰:“军师何故长别人锐气,灭自己威风?量一老卒,何足道哉!关某不须用三千军,只消本部下五百名校刀手,决定斩黄忠、韩玄之首,献来麾下。”玄德苦挡。云长不依,只领五百校刀手而去。孔明谓玄德曰:“云长轻敌黄忠,只恐有失。主公当往接应。”玄德从之,随后引兵望长沙进发。

却说长沙太守韩玄,平生性急,轻于杀戮,众皆恶之。是时听知云长军到,便唤老将黄忠商议。忠曰:“不须主公忧虑。凭某这口刀,这张弓,一千个来,一千个死!”原来黄忠能开二石力之弓,百发百中。言未毕,阶下一人应声而出曰:“不须老将军出战,只就某手中定活捉关某。”韩玄视之,乃管军校尉杨龄。韩玄大喜,遂令杨龄引军一千,飞奔出城。约行五十里,望见尘头起处,云长军马早到。杨龄挺枪出马,立于阵前骂战。云长大怒,更不打话,飞马舞刀,直取杨龄。龄挺枪来迎。不三合,云长手起刀落,砍杨龄于马下。追杀败兵,直至城下。韩玄闻之大惊,便教黄忠出马。玄自来城上观看。忠提刀纵马,引五百骑兵飞过吊桥。云长见一老将出马,知是黄忠,把五百校刀手一字摆开,横刀立马而问曰:“来将莫非黄忠否?”忠曰:“既知我名,焉敢犯我境!”云长曰:“特来取汝首级!”言罢,两马交锋。斗一百余合,不分胜负。韩玄恐黄忠有失,鸣金收军。黄忠收军入城。云长也退军,离城十里下寨,心中暗忖:“老将黄忠,名不虚传:斗一百合,全无破绽。来日必用拖刀计,背砍赢之。”

次日早饭毕,又来城下搦战。韩玄坐在城上,教黄忠出马。忠引数百骑杀过吊桥,再与云长交马。又斗五六十合,胜负不分,两军齐声喝采。鼓声正急时,云长拨马便走。黄忠赶来。云长方欲用刀砍去,忽听得脑后一声响;急回头看时,见黄忠被战马前失,掀在地下。云长急回马,双手举刀猛喝曰:“我且饶你性命!快换马来厮杀!”黄忠急提起马蹄,飞身上马,弃入城中。玄惊问之。忠曰:“此马久不上阵,故有此失。”玄曰:“汝箭百发百中,何不射之?”忠曰:“来日再战,必然诈败,诱到吊桥边射之。”玄以自己所乘一匹青马与黄忠。忠拜谢而退,寻思:“难得云长如此义气!他不忍杀害我,我又安忍射他?若不射,又恐违了将令。”是夜踌躇未定。

次日天晓,人报云长搦战。忠领兵出城。云长两日战黄忠不下,十分焦躁,抖擞威风,与忠交马。战不到三十余合,忠诈败,云长赶来。忠想昨日不杀之恩,不忍便射,带住刀,把弓虚拽弦响,云长急闪,却不见箭;云长又赶,忠又虚拽,云长急闪,又无箭;只道黄忠不会射,放心赶来。将近吊桥,黄忠在桥上搭箭开弓,弦响箭到,正射在云长盔缨根上。前面军齐声喊起。云长吃了一惊,带箭回寨,方知黄忠有百步穿杨之能,今日只射盔缨,正是报昨日不杀之恩也。云长领兵而退。黄忠回到城上来见韩玄,玄便喝左右捉下黄忠。忠叫曰:“无罪!”玄大怒曰:“我看了三日,汝敢欺我!汝前日不力战,必有私心;昨日马失,他不杀汝,必有关通;今日两番虚拽弓弦,第三箭却止射他盔缨,如何不是外通内连?若不斩汝,必为后患!”喝令刀斧手推下城门外斩之。众将欲告,玄曰:“但告免黄忠者,便是同情!”刚推到门外,恰欲举刀,忽然一将挥刀杀入,砍死刀手,救起黄忠,大叫曰:“黄汉升乃长沙之保障,今杀汉升,是杀长沙百姓也!韩玄残暴不仁,轻贤慢士,当众共殛之”愿随我者便来!”众视其人,面如重枣,目若朗星,乃义阳人魏延也。自襄阳赶刘玄德不着,来投韩玄;玄怪其傲慢少礼,不肯重用,故屈沉于此。当日救下黄忠,教百姓同杀韩玄,袒臂一呼,相从者数百余人。黄忠拦当不住。魏延直杀上城头,一刀砍韩玄为两段,提头上马,引百姓出城,投拜云长。云长大喜,遂入城。安抚已毕,请黄忠相见;忠托病不出。云长即使人去请玄德、孔明。

却说玄德自云长来取长沙,与孔明随后催促人马接应。正行间,青旗倒卷,一鸦自北南飞,连叫三声而去。玄德曰:“此应何祸福?”孔明就马上袖占一课,曰:“长沙郡已得,又主得大将。午时后定见分晓。”少顷。见一小校飞报前来,说:“关将军已得长沙郡,降将黄忠、魏延。耑等主公到彼。”玄德大喜,遂入长沙。云长接入厅上,具言黄忠之事。玄德乃亲往黄忠家相请,忠方出降,求葬韩玄尸首于长沙之东。后人有诗赞黄忠曰:“将军气概与天参,白发犹然困汉南。至死甘心无怨望,临降低首尚怀惭。宝刀灿雪彰神勇,铁骑临风忆战酗。千古高名应不泯,长随孤月照湘潭。”

玄德待黄忠甚厚。云长引魏延来见,孔明喝令刀斧手推下斩之。玄德惊问孔明曰:“魏延乃有功无罪之人,军师何故欲杀之?”孔明曰:“食其禄而杀其主,是不忠也;居其土而献其地,是不义也。吾观魏延脑后有反骨,久后必反,故先斩之,以绝祸根。”玄德曰:“若斩此人,恐降者人人自危。望军师恕之。”孔明指魏延曰:“吾今饶汝性命。汝可尽忠报主,勿生异心,若生异心,我好歹取汝首级。”魏延喏喏连声而退。黄忠荐刘表侄刘磐——现在攸县闲居,玄德取回,教掌长沙郡。四郡已平,玄德班师回荆州,改油江口为公安。自此钱粮广盛,贤士归之;将军马四散屯于隘口。

却说周瑜自回柴桑养病,令甘宁守巴陵郡,令凌统守汉阳郡,二处分布战船,听候调遣。程普引其余将士投合淝县来。原来孙权自从赤壁鏖兵之后,久在合淝,与曹兵交锋,大小十余战,未决胜负,不敢逼城下寨,离城五十里屯兵。闻程普兵到,孙权大喜,亲自出营劳军。人报鲁子敬先至,权乃下马立待之。肃慌忙滚鞍下马施礼。众将见权如此待肃,皆大惊异。权请肃上马,并辔而行,密谓曰:“孤下马相迎,足显公否?”肃曰:“未也。”权曰:“然则何如而后为显耶?”肃曰:“愿明公威德加于四海,总括九州,克成帝业,使肃名书竹帛,始为显矣。”权抚掌大笑。同至帐中,大设饮宴,犒劳鏖兵将士,商议破合淝之策。

忽报张辽差人来下战书。权拆书观毕,大怒曰:“张辽欺吾太甚!汝闻程普军来,故意使人搦战!来日吾不用新军赴敌,看我大战一场!”传令当夜五更,三军出寨,望合淝进发。辰时左右,军马行至半途,曹兵已到。两边布成阵势。孙权金盔金甲,披挂出马;左宋谦,右贾华,二将使方天画戟,两边护卫。三通鼓罢,曹军阵中,门旗两开,三员将全装惯带,立于阵前:中央张辽,左边李典,右边乐进。张辽纵马当先,专搦孙权决战。权绰枪欲自战,阵门中一将挺枪骤马早出,乃太史慈也。张辽挥刀来迎。两将战有七八十合,不分胜负。曹阵上李典谓乐进曰:“对面金盔者,孙权也。若捉得孙权,足可与八十三万大军报仇。”说犹未了,乐进一骑马,一口刀,从刺斜里径取孙权,如一道电光,飞至面前,手起刀落。宋谦、贾华急将画戟遮架。刀到处,两枝戟齐断,只将戟杆望马头上打。乐进回马,宋谦绰军士手中枪赶来。李典搭上箭,望宋谦心窝里便射,应弦落马。太史慈见背后有人堕马,弃却张辽,望本阵便回。张辽乘势掩杀过来,吴兵大乱,四散奔走。张辽望见孙权,骤马赶来。看看赶上,刺斜里撞出一军,为首大将,乃程普也;截杀一阵,救了孙权。张辽收军自回合淝。程普保孙权归大寨,败军陆续回营。孙权因见折了宋谦,放声大哭。长史张纮曰:“主公恃盛壮之气,轻视大敌,三军之众,莫不寒心。即使斩将搴旗,威振疆场,亦偏将之任,非主公所宜也。愿抑贲、育之勇,怀王霸之计。且今日宋谦死于锋镝之下,皆主公轻敌之故。今后切宜保重。”权曰:“是孤之过也。从今当改之。”少顷,太史慈入帐,言:“某手下有一人,姓戈,名定,与张辽手下养马后槽是弟兄,后槽被责怀怨,今晚使人报来,举火为号,刺杀张辽,以报宋谦之仇。某请引兵为外应。”权曰:“戈定何在?”太史慈曰:“已混入合淝城中去了。某愿乞五千兵去。”诸葛瑾曰:“张辽多谋,恐有准备,不可造次。”太史慈坚执要行。权因伤感宋谦之死,急要报仇,遂令太史慈引兵五千,去为外应。

却说戈定乃太史慈乡人;当日杂在军中,随入合淝城,寻见养马后槽,两个商议。戈定曰:“我已使人报太史慈将军去了,今夜必来接应。你如何用事?”后槽曰:“此间离中军较远,夜间急不能进,只就草堆上放起一把火,你去前面叫反,城中兵乱,就里刺杀张辽,余军自走也。”戈定曰:“此计大妙!”是夜张辽得胜回城,赏劳三军,传令不许解甲宿睡。左右曰:“今日全胜,吴兵远遁,将军何不卸甲安息?”辽曰:“非也。为将之道:勿以胜为喜,勿以败为忧。倘吴兵度我无备,乘虚攻击,何以应之?今夜防备,当比每夜更加谨慎。”说犹未了,后寨火起,一片声叫反,报者如麻。张辽出帐上马,唤亲从将校十数人,当道而立。左右曰:“喊声甚急,可往观之。”辽曰:“岂有一城皆反者?此是造反之人,故惊军士耳。如乱者先斩!”无移时,李典擒戈定并后槽至。辽询得其情,立斩于马前。只听得城门外鸣锣击鼓,喊声大震。辽曰:“此是吴兵外应,可就计破之。”便令人于城门内放起一把火,众皆叫反,大开城门,放下吊桥。太史慈见城门大开,只道内变,挺枪纵马先入。城上一声炮响,乱箭射下,太史慈急退,身中数箭。背后李典、乐进杀出,吴兵折其大半,乘势直赶到寨前。陆逊,董袭杀出,救了太史慈。曹兵自回。孙权见太史慈身带重伤,愈加伤感。张昭请权罢兵。权从之,遂收兵下船,回南徐润州。比及屯住军马,太史慈病重;权使张昭等问安,太史慈大叫曰:“大丈夫生于乱世,当带三尺剑立不世之功;今所志未遂,奈何死乎!”言讫而亡,年四十一岁。后人有诗赞曰:“矢志全忠孝,东莱太史慈:姓名昭远塞,弓马震雄师;北海酬恩日,神亭酣战时。临终言壮志,千古共嗟咨!”孙权闻慈死,伤悼不已,命厚葬于南徐北固山下,养其子太史亨于府中。却说玄德在荆州整顿军马,闻孙权合淝兵败,已回南徐,与孔明商议。孔明曰:“亮夜观星象,见西北有星坠地,必应折一皇族。”正言间,忽报公子刘琦病亡。玄德闻之,痛哭不已。孔明劝曰:“生死分定,主公勿忧,恐伤贵体。且理大事:可急差人到彼守御城池,并料理葬事。”玄德曰:“谁可去?”孔明曰:“非云长不可。”即时便教云长前去襄阳保守。玄德曰:“今日刘琦已死,东吴必来讨荆州,如何对答?”孔明曰:“若有人来,亮自有言对答。”过了半月,人报东吴鲁肃特来吊丧。正是:先将计策安排定,只等东吴使命来。未知孔明如何对答,且看下文分解。