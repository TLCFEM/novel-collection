\chapter{曹操大宴铜雀台~孔明三气周公瑾}

却说周瑜被诸葛亮预先埋伏关公、黄忠、魏延三枝军马,一击大败。黄盖、韩当急救下
船,折却水军无数。遥观玄德、孙夫人车马仆从,都停住于山顶之上,瑜如何不气?箭疮未
愈,因怒气冲激,疮口迸裂,昏绝于地。众将救醒,开船逃去。孔明教休追赶,自和玄德归
荆州庆喜,赏赐众将。

周瑜自回柴桑。蒋钦等一行人马自归南徐报孙权。权不胜忿怒,欲拜程普为都督,起兵
取荆州。周瑜又上书,请兴兵雪恨。张昭谏曰:“不可。曹操日夜思报赤壁之恨,因恐孙、
刘同心,故未敢兴兵。今主公若以一时之忿,自相吞并,操必乘虚来攻,国势危矣。”顾雍
曰:“许都岂无细作在此?若知孙、刘不睦,操必使人勾结刘备。备惧东吴,必投曹操。若
是,则江南何日得安?为今之计,莫若使人赴许都,表刘备为荆州牧。曹操知之,则惧而不
敢加兵于东南。且使刘备不恨于主公。然后使心腹用反间之计,令曹、刘相攻,吾乘隙而图
之,斯为得耳。”权曰:“元叹之言甚善。但谁可为使?”雍曰:“此间有一人,乃曹操敬
慕者,可以为使。”权问何人。雍曰:“华歆在此,何不遣之?”权大喜。即遣歆赍表赴许
都。歆领命起程,径到许都来见曹操。闻操会群臣于邺郡,庆赏铜雀台,歆乃赴邺郡候见。

操自赤壁败后,常思报仇;只疑孙、刘并力,因此不敢轻进,时建安十五年春,造铜雀
台成,操乃大会文武于邺郡,设宴庆贺。其台正临漳河,中央乃铜雀台,左边一座名玉龙
台,右边一座名金凤台,各高十丈,上横二桥相通,千门万户,金碧交辉。是日,曹操头戴
嵌宝金冠,身穿绿锦罗袍,玉带珠履,凭高而坐。文武侍立台下。

操欲观武官比试弓箭,乃使近侍将西川红锦战袍一领,挂于垂杨枝上,下设一箭垛,以
百步为界。分武官为两队:曹氏宗族俱穿红,其余将士俱穿绿:各带雕弓长箭,跨鞍勒马,
听候指挥。操传令曰:“有能射中箭垛红心者,即以锦袍赐之;如射不中,罚水一杯。”号
令方下,红袍队中,一个少年将军骤马而出,众视之,乃曹休也。休飞马往来,奔驰三次,
扣上箭,拽满弓,一箭射去,正中红心。金鼓齐鸣,众皆喝采。曹操于台上望见大喜,曰:
“此吾家千里驹也!”方欲使人取锦袍与曹休,只见绿袍队中,一骑飞出,叫曰:“丞相锦
袍,合让俺外姓先取,宗族中不宜搀越。”操视其人,乃文聘也。众官曰:“且看文仲业射
法。”文聘拈弓纵马一箭,亦中红心。众皆喝采,金鼓乱鸣。聘大呼曰:“快取袍来!”只
见红袍队中,又一将飞马而出,厉声曰:“文烈先射,汝何得争夺?看我与你两个解箭!”
拽满弓,一箭射去,也中红心。众人齐声喝采。视其人,乃曹洪也。洪方欲取袍,只见绿袍
队里又一将出,扬弓叫曰:“你三人射法,何足为奇!看我射来!”众视之,乃张郃也。郃飞
马翻身,背射一箭,也中红心。四枝箭齐齐的攒在红心里。众人都道:“好射法!”郃曰:
“锦袍须该是我的!”言未毕,红袍队中一将飞马而出,大叫曰:“汝翻身背射,何足称
异!看我夺射红心!”众视之,乃夏侯渊也,渊骤马至界口,纽回身一箭射去,正在四箭当
中,金鼓齐鸣。渊勒马按弓大叫曰:“此箭可夺得锦袍么?”只见绿袍队里,一将应声而
出,大叫:“且留下锦袍与我徐晃!”渊曰:“汝更有何射法,可夺我袍?”晃曰:“汝夺
射红心,不足为异。看我单取锦袍!”拈弓搭箭,遥望柳条射去,恰好射断柳条,锦袍坠
地。徐晃飞取锦袍,披于身上,骤马至台前声喏曰:“谢丞相袍!”曹操与众官无不称羡。
晃才勒马要回,猛然台边跃出一个绿袍将军,大呼曰:“你将锦袍那里去?早早留下与
我!”众视之,乃许褚也。晃曰:“袍已在此,汝何敢强夺!”褚更不回答,竟飞马来夺
袍。两马相近,徐晃便把弓打许褚。褚一手按住弓,把徐晃拖离鞍鞒。晃急弃了弓,翻身下
马,褚亦下马,两个揪住厮打。操急使人解开。那领锦袍已是扯得粉碎。操令二人都上台。
徐晃睁眉怒目,许褚切齿咬牙,各有相斗之意。操笑曰:“孤特视公等之勇耳。岂惜一锦袍
哉?”便教诸将尽都上台,各赐蜀锦一匹,诸将各各称谢。操命各依位次而坐。乐声竞奏,
水陆并陈。文官武将轮次把盏,献酬交错。操顾谓众文官曰:“武将既以骑射为乐,足显威
勇矣。公等皆饱学之士,登此高台,可不进佳章以纪一时之胜事乎?”众官皆躬身而言曰:
“愿从钧命。”时有王朗、钟繇、王粲、陈琳一班文官,进献诗章。诗中多有称颂曹操功德
巍巍、合当受命之意。曹操逐一览毕,笑曰:“诸公佳作,过誉甚矣。孤本愚陋,始举孝
廉。后值天下大乱,筑精舍于谯东五十里,欲春夏读书,秋冬射猎,以待天下清平,方出仕
耳。不意朝廷徵孤为典军校尉,遂更其意,专欲为国家讨贼立功,图死后得题墓道曰:‘汉
故征西将军曹侯之墓’,平生愿足矣。念自讨董卓,剿黄巾以来,除袁术、破吕布、灭袁
绍、定刘表,遂平天下。身为宰相,人臣之贵已极,又复何望哉?如国家无孤一人,正不知
几人称帝,几人称王。或见孤权重,妄相忖度,疑孤有异心,此大谬也。孤常念孔子称文王
之至德,此言耿耿在心。但欲孤委捐兵众,归就所封武平侯之国,实不可耳:诚恐一解兵
柄,为人所害;孤败则国家倾危;是以不得慕虚名而处实祸也。诸公必无知孤意者。”众皆
起拜曰:“虽伊尹、周公,不及丞相矣。”后人有诗曰:“周公恐惧流言日,王莽谦恭下士
时:假使当年身便死,一生真伪有谁知!”

曹操连饮数杯,不觉沉醉,唤左右捧过笔砚,亦欲作《铜雀台诗》。刚才下笔,忽报:
“东吴使华歆表奏刘备为荆州牧,孙权以妹嫁刘备,汉上九郡大半已属备矣。“操闻之,手
脚慌乱,投笔于地。程昱曰:“丞相在万军之中,矢石交攻之际,未尝动心;今闻刘备得了
荆州,何故如此失惊?”操曰:“刘备,人中之龙也,生平未尝得水。今得荆州,是困龙入
大海矣。孤安得不动心哉!”程昱曰:“丞相知华歆来意否?”操曰:“未知。”昱曰:
“孙权本忌刘备,欲以兵攻之;但恐丞相乘虚而击,故令华歆为使,表荐刘备,乃安备之
心,以塞丞相之望耳。”操点头曰:“是也。”昱曰:“某有一计,使孙、刘自相吞并,丞
相乘间图之,一鼓而二敌俱破。”操大喜,遂问其计。程昱曰:“东吴所倚者,周瑜也。丞
相今表奏周瑜为南郡太守,程普为江夏太守,留华歆在朝重用之;瑜必自与刘备为仇敌矣。
我乘其相并而图之,不亦善乎?”操曰:“仲德之言,正合孤意。”遂召华歆上台,重加赏
赐。当日筵散,操即引文武回许昌,表奏周瑜为总领南郡太守、程普为江夏太守。封华歆为
大理少卿,留在许都。

使命至东吴,周瑜、程普各受职讫。周瑜既领南郡,愈思报仇,遂上书吴侯,乞令鲁肃
去讨还荆州。孙权乃命肃曰:“汝昔保借荆州与刘备,今备迁延不还,等待何时?”肃曰:
“文书上明白写着,得了西川便还。”权叱曰:“只说取西川,到今又不动兵,不等老了
人!”肃曰:“某愿往言之。”遂乘船投荆州而来。却说玄德与孔明在荆州广聚粮草,调练
军马,远近之士多归之。忽报鲁肃到。玄德问孔明曰:“子敬此来何意?”孔明曰:“昨者
孙权表主公为荆州牧,此是惧曹操之计。操封周瑜为南郡太守,此欲令我两家自相吞并,他
好于中取事也。今鲁肃此来,又是周瑜既受太守之职,要来索荆州之意。”玄德曰:“何以
答之?”孔明曰:“若肃提起荆州之事,主公便放声大哭。哭到悲切之处,亮自出来解
劝。”

计会已定,接鲁肃入府,礼毕,叙坐。肃曰:“今日皇叔做了东吴女婿,便是鲁肃主
人,如何敢坐?”玄德笑曰:“子敬与我旧交,何必太谦?”肃乃就坐。茶罢,肃曰:“今
奉吴侯钧命,专为荆州一事而来。皇叔已借住多时,未蒙见还。今既两家结亲,当看亲情面
上,早早交付。”玄德闻言,掩面大哭。肃惊曰:“皇叔何故如此?”玄德哭声不绝。

孔明从屏后出曰:“亮听之久矣。子敬知吾主人哭的缘故么?”肃曰:“某实不知。”
孔明曰:“有何难见?当初我主人借荆州时,许下取得西川便还。仔细想来,益州刘璋是我
主人之弟,一般都是汉朝骨肉,若要兴兵去取他城池时,恐被外人唾骂;若要不取,还了荆
州,何处安身?若不还时,于尊舅面上又不好看。事实两难,因此泪出痛肠。”孔明说罢,
触动玄德衷肠,真个捶胸顿足,放声大哭。鲁肃劝曰:“皇叔且休烦恼,与孔明从长计
议。”孔明曰:“有烦子敬,回见吴侯,勿惜一言之劳,将此烦恼情节,恳告吴侯,再容几
时。”肃曰:“倘吴侯不从,如之奈何?”孔明曰:“吴侯既以亲妹聘嫁皇叔,安得不从
乎?望子敬善言回覆。”

鲁肃是个宽仁长者,见玄德如此哀痛,只得应允。玄德、孔明拜谢。宴毕,送鲁肃下
船。径到柴桑,见了周瑜,具言其事。周瑜顿足曰:“子敬又中诸葛亮之计也!当初刘备依
刘表时,常有吞并之意,何况西川刘璋乎?似此推调,未免累及老兄矣。吾有一计,使诸葛
亮不能出吾算中。子敬便当一行。”肃曰:“愿闻妙策。”瑜曰:“子敬不必去见吴侯,再
去荆州对刘备说:孙、刘两家,既结为亲,便是一家;若刘氏不忍去取西川,我东吴起兵去
敢,取得西川时,以作嫁资,却把荆州交还东吴。”肃曰:“西川迢递,取之非易。都督此
计,莫非不可?”瑜笑曰:“子敬真长者也。你道我真个去取西川与他?我只以此为名,实
欲去取荆州,且教他不做准备。东吴军马收川,路过荆州,就问他索要钱粮,刘备必然出城
劳军。那时乘势杀之,夺取荆州,雪吾之恨,解足下之祸。”

鲁肃大喜,便再往荆州来。玄德与孔明商议。孔明曰:“鲁肃必不曾见吴侯,只到柴桑
和周瑜商量了甚计策,来诱我耳。但说的话,主公只看我点头,便满口应承。”计会已定。
鲁肃入见。礼毕,曰:“吴侯甚是称赞皇叔盛德,遂与诸将商议,起兵替皇叔收川。取了西
川,却换荆州,以西川权当嫁资。但军马经过,却望应些钱粮。”孔明听了,忙点头曰:
“难得吴侯好心!”玄德拱手称谢曰:“此皆子敬善言之力。”孔明曰:“如雄师到日,即
当远接犒劳。”鲁肃暗喜,宴罢辞回。

玄德问孔明曰:“此是何意?”孔明大笑曰:“周瑜死日近矣!这等计策,小儿也瞒不
过!”玄德又问如何,孔明曰:“此乃假途灭虢之计也。虚名牧川,实取荆州。等主公出城
劳军,乘势拿下,杀入城来,攻其不备,出其不意也。”玄德曰:“如之奈何?”孔明曰:
“主公宽心,只顾准备窝弓以擒猛虎,安排香饵以钓鳌鱼。等周瑜到来,他便不死,也九分
无气。”便唤赵云听计:“如此如此,其余我自有摆布。”玄德大喜。后人有诗云:“周瑜
决策取荆州,诸葛先知第一筹。指望长江香饵稳,不知暗里钓鱼钩。”

却说鲁肃回见周瑜,说玄德、孔明欢喜一节,准备出城劳军。周瑜大笑曰:“原来今番
也中了吾计!”便教鲁肃禀报吴侯,并遣程普引军接应。周瑜此时箭疮已渐平愈,身躯无
事,使甘宁为先锋,自与徐盛、丁奉为第二,凌统、吕蒙为后队,水陆大兵五万,望荆州而
来。周瑜在船中,时复欢笑,以为孔明中计。前军至夏口,周瑜问:“荆州有人在前面接
否!”人报:“刘皇叔使糜竺来见都督。”瑜唤至,问劳军如何。糜竺曰:“主公皆准备安
排下了。”瑜曰:“皇叔何在?”竺曰:“在荆州城门外相等,与都督把盏。”瑜曰:“今
为汝家之事,出兵远征;劳军之礼,休得轻易。”糜竺领了言语先回。

战船密密排在江上,依次而进,看看至公安,并无一只军船,又无一人远接。周瑜催船
速行。离荆州十余里,只见江面上静荡荡的。哨探的回报:“荆州城上,插两面白旗,并不
见一人之影。”瑜心疑,教把船傍岸,亲自上岸乘马,带了甘宁、徐盛、丁奉一班军官,引
亲随精军三千人,径望荆州来。既至城下,并不见动静。瑜勒住马,令军士叫门。城上问是
谁人。吴军答曰:“是东吴周都督亲自在此。”言未毕,忽一声梆子响,城上军一齐都竖起
枪刀。敌楼上赵云出曰:“都督此行,端的为何?”瑜曰:“吾替汝主取西川,汝岂犹未知
耶?”云曰:“孔明军师已知都督假途灭虢之计,故留赵云在此。吾主公有言:孤与刘璋,
皆汉室宗亲,安忍背义而取西川?若汝东吴端的取蜀,吾当披发入山,不失信于天下也。”
周瑜闻之,勒马便回。只见一人打着令字旗,于马前报说:“探得四路军马,一齐杀到:关
某从江陵杀来,张飞从姊归杀来,黄忠从公安杀来,魏延从孱陵小路杀来,四路正不知多少
军马。喊声远近震动百余里,皆言要捉周瑜。”瑜马上大叫一声,箭疮复裂,坠于马下。正
是:一着棋高难对敌,几番算定总成空。未知性命如何,且看下文分解。