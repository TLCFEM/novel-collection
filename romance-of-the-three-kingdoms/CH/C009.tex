\chapter{除凶暴吕布助司徒~犯长安李傕听贾诩}

却说那撞倒董卓的人,正是李儒。当下李儒扶起董卓,至书院中坐定,卓曰:“汝为何来此?”儒曰:“儒适至府门,知太师怒入后园,寻问吕布。因急走来,正遇吕布奔走,云:‘太师杀我!’儒慌赶入园中劝解,不意误撞恩相。死罪!死罪!”卓曰:“叵耐逆贼!戏吾爱姬,誓必杀之!”儒曰:“恩相差矣。昔楚庄王绝缨之会,不究戏爱姬之蒋雄,后为秦兵所困,得其死力相救。今貂蝉不过一女子,而吕布乃太师心腹猛将也。太师若就此机会,以蝉赐布,布感大恩,必以死报太师。太师请自三思。”卓沈吟良久曰:“汝言亦是,我当思之。”儒谢而出。卓入后堂,唤貂蝉问曰:“汝何与吕布私通耶?”蝉泣曰:“妾在后园看花,吕布突至。妾方惊避,布曰:‘我乃太师之子,何必相避?’提戟赶妾至凤仪亭。妾见其心不良,恐为所逼,欲投荷池自尽,却被这厮抱住。正在生死之间,得太师来,救了性命。”董卓曰:“我今将汝赐与吕布,何如?”貂蝉大惊,哭曰:“妾身已事贵人,今忽欲下赐家奴,妾宁死不辱!”遂掣壁间宝剑欲自刎。卓慌夺剑拥抱曰:“吾戏汝!”貂蝉倒于卓怀,掩面大哭曰:“此必李儒之计也!儒与布交厚,故设此计;却不顾惜太师体面与贱妾性命。妾当生噬其肉!”卓曰:“吾安忍舍汝耶?”蝉曰:“虽蒙太师怜爱,但恐此处不宜久居,必被吕布所害。”卓曰:“吾明日和你归郿坞去,同受快乐,慎勿忧疑。”蝉方收泪拜谢。

次日,李儒入见曰:“今日良辰,可将貂蝉送与吕布。”卓曰:“布与我有父子之分,不便赐与。我只不究其罪。汝传我意,以好言慰之可也。”儒曰:“太师不可为妇人所惑。”卓变色曰:“汝之妻肯与吕布否?貂蝉之事,再勿多言;言则必斩!”李儒出,仰天叹曰:“吾等皆死于妇人之手矣!”后人读书至此。有诗叹之曰:“司徒妙算托红裙。不用干戈不用兵。三战虎牢徒费力,凯歌却奏凤仪亭。”

董卓即日下令还郿坞,百官俱拜送。貂蝉在车上,遥见吕布于稠人之内,眼望车中。貂蝉虚掩其面,如痛哭之状。车已去运,布缓辔于土冈之上,眼望车尘,叹惜痛恨。忽闻背后一人问曰:“温侯何不从太师去,乃在此遥望而发叹?”布视之,乃司徒王允也。相见毕,允曰:“老夫日来因染微恙,闭门不出,故久未得与将军一见。今日太师驾归郿坞,只得扶病出送,却喜得晤将军。请问将军,为何在此长叹?”布曰:“正为公女耳。”允佯惊曰:“许多时尚未与将军耶?”布曰:“老贼自宠幸久矣!”允佯大惊曰:“不信有此事!”布将前事一一告允。允仰面跌足,半晌不语;良久,乃言曰:“不意太师作此禽兽之行!”因挽布手曰:“且到寒舍商议。”布随允归。允延入密室,置酒款待。布又将凤仪亭相遇之事,细述一遍。允曰:“太师淫吾之女,夺将军之妻,诚为天下耻笑。非笑太师,笑允与将军耳!然允老迈无能之辈,不足为道;可惜将军盖世英雄,亦受此污辱也!”布怒气冲天,拍案大叫。允急曰:“老夫失语,将军息怒。”布曰:“誓当杀此老贼,以雪吾耻!”允急掩其口曰:“将军勿言,恐累及老夫。”布曰:“大丈夫生居天地间,岂能郁郁久居人下!”允曰:“以将军之才,诚非董太师所可限制。”布曰:“吾欲杀此老贼,奈是父子之情,恐惹后人议论。”允微笑曰:“将军自姓吕,太师自姓董。掷戟之时,岂有父子情耶?”布奋然曰:“非司徒言,布几自误!”允见其意已决,便说之曰:“将军若扶汉室,乃忠臣也,青史传名,流芳百世;将军若助董卓,乃反臣也,载之史笔,遗臭万年。”布避席下拜曰:“布意已决,司徒勿疑。”允曰:“但恐事或不成,反招大祸。”布拔带刀,刺臂出血为誓。允跪谢曰:“汉祀不斩,皆出将军之赐也。切勿泄漏!临期有计,自当相报。”布慨诺而去。允即请仆射士孙瑞、司隶校尉黄琬商议。瑞曰:“方今主上有疾新愈,可遣一能言之人,往郿坞请卓议事;一面以天子密诏付吕布,使伏甲兵于朝门之内,引卓入诛之:此上策也。”琬曰:“何人敢去?”瑞曰:“吕布同郡骑都尉李肃,以董卓不迁其官,甚是怀怨。若令此人去,卓必不疑。”允曰:“善。”请吕布共议。布曰:“昔日劝吾杀丁建阳,亦此人也。今若不去,吾先斩之。”使人密请肃至。布曰:“昔日公说布使杀丁建阳而投董卓;今卓上欺天子,下虐生灵,罪恶贯盈,人神共愤。公可传天子诏往郿坞,宣卓入朝,伏兵诛之,力扶汉室,共作忠臣。尊意若何?”肃曰:“我亦欲除此贼久矣,恨无同心者耳。今将军若此,是天赐也,肃岂敢有二心!”遂折箭为誓。允曰:“公若能干此事,何患不得显官。”

次日,李肃引十数骑,前到郿坞。人报天子有诏,卓教唤入。李肃入拜。卓曰:“天子有何诏?”肃曰:“天子病体新痊,欲会文武于未央殿,议将禅位于太师,故有此诏。”卓曰:“王允之意若何?”肃曰:“王司徒已命人筑受禅台,只等主公到来。”卓大喜曰:“吾夜梦一龙罩身,今日果得此喜信。时哉不可失!”便命心腹将李傕、郭汜、张济、樊稠四人领飞熊军三千守郿坞,自己即日排驾回京;顾谓李肃曰:“吾为帝,汝当为执金吾。”肃拜谢称臣。卓入辞其母。母时年九十余矣,问曰:“吾儿何往?”卓曰:“儿将往受汉禅,母亲早晚为太后也!”母曰:“吾近日肉颤心惊,恐非吉兆。”卓曰:“将为国母,岂不预有惊报!”遂辞母而行。临行,谓貂蝉曰:“吾为天子,当立汝为贵妃。”貂蝉已明知就里,假作欢喜拜谢。

卓出坞上车,前遮后拥,望长安来。行不到三十里,所乘之车,忽折一轮,卓下车乘马。又行不到十里,那马咆哮嘶喊,掣断辔头。卓问肃曰:“车折轮,马断辔,其兆若何?”肃曰:“乃太师应绍汉禅,弃旧换新,将乘玉辇金鞍之兆也。”卓喜而信其言。次日,正行间,忽然狂风骤起,昏雾蔽天。卓问肃曰:“此何祥也?”肃曰:“主公登龙位,必有红光紫雾,以壮天威耳。”卓又喜而不疑。既至城外,百官俱出迎接。只有李儒抱病在家,不能出迎。卓进至相府,吕布入贺。卓曰:“吾登九五,汝当总督天下兵马。”布拜谢,就帐前歇宿。是夜有十数小儿于郊外作歌,风吹歌声入帐。歌曰:“千里草,何青青!十日卜,不得生!”歌声悲切。卓问李肃曰:“童谣主何吉凶?”肃曰:“亦只是言刘氏灭、董氏兴之意。”

次日侵晨,董卓摆列仪从入朝,忽见一道人,青袍白巾,手执长竿,上缚布一丈,两头各书一“口”字。卓问肃曰:“此道人何意?”肃曰:“乃心恙之人也。”呼将士驱去。卓进朝,群臣各具朝服,迎谒于道。李肃手执宝剑扶车而行。到北掖门,军兵尽挡在门外,独有御车二十余人同入。董卓遥见王允等各执宝剑立于殿门,惊问肃曰:“持剑是何意?”肃不应,推车直入。王允大呼曰:“反贼至此,武士何在?”两旁转出百余人,持戟挺槊刺之。卓衷甲不入,伤臂坠车,大呼曰:“吾儿奉先何在?”吕布从车后厉声出曰:“有诏讨贼!”一鼓直刺咽喉,李肃早割头在手。吕布左手持戟,右手怀中取诏,大呼曰:“奉诏讨贼臣董卓,其余不问!”将吏皆呼万岁。后人有诗叹董卓曰:“霸业成时为帝王,不成且作富家郎。谁知天意无私曲,郿坞方成已灭亡。”

却说当下吕布大呼曰:“助卓为虐者,皆李儒也!谁可擒之?”李肃应声愿往。忽听朝门外发喊,人报李儒家奴已将李儒绑缚来献。王允命缚赴市曹斩之;又将董卓尸首,号令通衢。卓尸肥胖,看尸军士以火置其脐中为灯,膏流满地。百姓过者,莫不手掷其头,足践其尸。王允又命吕布同皇甫嵩、李肃领兵五万,至郿坞抄籍董卓家产、人口。

却说李傕、郭汜、张济、樊稠闻董卓已死,吕布将至,便引了飞熊军连夜奔凉州去了。吕布至郿坞,先取了貂蝉。皇甫嵩命将坞中所藏良家子女,尽行释放。但系董卓亲属,不分老幼,悉皆诛戮。卓母亦被杀。卓弟董旻、侄董璜皆斩首号令。收籍坞中所蓄,黄金数十万,白金数百万,绮罗、珠宝、器皿、粮食,不计其数。回报王允。允乃大犒军士,设宴于都堂,召集众官,酌酒称庆。

正饮宴间,忽人报曰:“董卓暴尸于市,忽有一人伏其尸而大哭。”允怒曰:“董卓伏诛,士民莫不称贺;此何人,独敢哭耶!”遂唤武士:“与吾擒来!”须臾擒至。众官见之,无不惊骇:原来那人不是别人,乃侍中蔡邕也,允叱曰:“董卓逆贼,今日伏诛,国之大幸。汝为汉臣,乃不为国庆,反为贼哭,何也?”邕伏罪曰:“邕虽不才,亦知大义,岂肯背国而向卓?只因一时知遇之感,不觉为之一哭,自知罪大。愿公见原:倘得黥首刖足,使续成汉史,以赎其辜,邕之幸也。”众官惜邕之才,皆力救之。太傅马日磾亦密谓允曰:“伯喈旷世逸才,若使续成汉史,诚为盛事。且其孝行素著,若遽杀之,恐失人望。”允曰:“昔孝武不杀司马迁,后使作史,遂致谤书流于后世。方今国运衰微,朝政错乱,不可令佞臣执笔于幼主左右,使吾等蒙其讪议也。”日磾无言而退,私谓众官曰:“王允其无后乎!善人,国之纪也;制作,国之典也。灭纪废典,岂能久乎?”当下王允不听马日磾之言,命将蔡邕下狱中缢死。一时士大夫闻者,尽为流涕。后人论蔡邕之哭董卓,固自不是;允之杀之,亦为已甚。有诗叹曰:“董卓专权肆不仁,侍中何自竟亡身?当时诸葛隆中卧,安肯轻身事乱臣。”且说李傕、郭汜、张济、樊稠逃居陕西,使人至长安上表求赦。王允曰:“卓之跋扈,皆此四人助之;今虽大赦天下,独不赦此四人。”使者回报李傕。傕曰:“求赦不得,各自逃生可也。”谋士贾诩曰:“诸君若弃军单行,则一亭长能缚君矣。不若诱集陕人并本部军马,杀入长安与董卓报仇。事济,奉朝廷以正天下;若其不胜,走亦未迟。”傕等然其说,遂流言于西凉州曰:“王允将欲洗荡此方之人矣!”众皆惊惶。乃复扬言曰:“徒死无益,能从我反乎?”众皆愿从。于是聚众十余万,分作四路,杀奔长安来。路逢董卓女婿中郎将牛辅,引军五千人,欲去与丈人报仇,李傕便与合兵,使为前驱。四人陆续进发。王允听知西凉兵来,与吕布商议。布曰:“司徒放心。量此鼠辈,何足数也!”遂引李肃将兵出敌。肃当先迎战,正与牛辅相遇,大杀一阵。牛辅抵敌不过,败阵而去。不想是夜二更,牛辅乘肃不备,竟来劫寨。肃军乱窜,败走三十余里,折军大半,来见吕布,布大怒曰:“汝何挫吾锐气!”遂斩李肃,悬头军门。次日吕布进兵与牛辅对敌。量牛辅如何敌得吕布,仍复大败而走。是夜牛辅唤心腹人胡赤儿商议曰:“吕布骁勇,万不能敌;不如瞒了李傕等四人,暗藏金珠,与亲随三五人弃军而去。”胡赤儿应允。是夜收拾金珠,弃营而走,随行者三四人。将渡一河,赤儿欲谋取金珠,竟杀死牛辅,将头来献吕布。布问起情由,从人出首:“胡赤儿谋杀牛辅,夺其金宝。”布怒,即将赤儿诛杀。领军前进,正迎着李傕军马。吕布不等他列阵,便挺戟跃马,麾军直冲过来。傕军不能抵当,退走五十余里,依山下寨,请郭汜、张济、樊稠共议,曰:“吕布虽勇,然而无谋,不足为虑。我引军守住谷口,每日诱他厮杀,郭将军可领军抄击其后,效彭越挠楚之法,鸣金进兵,擂鼓收兵。张、樊二公,却分兵两路,径取长安。彼首尾不能救应,必然大败。”众用其计。

却说吕布勒兵到山下,李傕引军搦战。布忿怒冲杀过去,傕退走上山。山上矢石如雨,布军不能进。忽报郭汜在阵后杀来,布急回战。只闻鼓声大震,汜军已退。布方欲收军,锣声响处,傕军又来。未及对敌,背后郭汜又领军杀到。及至吕布来时,却又擂鼓收军去了。激得吕布怒气填胸。一连如此几日,欲战不得,欲止不得。正在恼怒,忽然飞马报来,说张济、樊稠两路军马,竟犯长安,京城危急。布急领军回,背后李傕、郭汜杀来。布无心恋战,只顾奔走,折了好些人马。比及到长安城下。贼兵云屯雨集,围定城池,布军与战不利。军士畏吕布暴厉,多有降贼者,布心甚忧。

数日之后,董卓余党李蒙、王方在城中为贼内应,偷开城门,四路贼军一齐拥入。吕布左冲右突,拦挡不住,引数百骑往青琐门外,呼王允曰:“势急矣!请司徒上马,同出关去,别图良策。”允曰:“若蒙社稷之灵,得安国家,吾之愿也;若不获已,则允奉身以死。临难苟免,吾不为也。为我谢关东诸公,努力以国家为念!”吕布再三相劝,王允只是不肯去。不一时,各门火焰竟天,吕布只得弃却家小,引百余骑飞奔出关,投袁术去了。

李傕、郭汜纵兵大掠。太常卿种拂、太仆鲁馗、大鸿胪周奂、城门校尉崔烈、越骑校尉王颀皆死于国难。贼兵围绕内庭至急,侍臣请天子上宣平门止乱。李傕等望见黄盖,约住军士,口呼“万岁”。献帝倚楼问曰:“卿不候奏请,辄入长安,意欲何为?”李傕、郭汜仰面奏曰:“董太师乃陛下社稷之臣,无端被王允谋杀,臣等特来报仇,非敢造反。但见王允,臣便退兵。”王允时在帝侧,闻知此言,奏曰:“臣本为社稷计。事已至此,陛下不可惜臣,以误国家。臣请下见二贼。”帝徘徊不忍。允自宣平门楼上跳下楼去,大呼曰:“王允在此!”李傕、郭汜拔剑叱曰:“董太师何罪而见杀?”允曰:“董贼之罪,弥天亘地,不可胜言!受诛之日。长安士民,皆相庆贺,汝独不闻乎?”傕、汜曰:“太师有罪;我等何罪,不肯相赦?”王允大骂:“逆贼何必多言!我王允今日有死而已!”二贼手起,把王允杀于楼下。史官有诗赞曰:“王允运机筹,奸臣董卓休。心怀家国恨,眉锁庙堂忧。英气连霄汉,忠诚贯斗牛。至今魂与魄,犹绕凤凰楼。”

众贼杀了王允,一面又差人将王允宗族老幼,尽行杀害。士民无不下泪。当下李傕、郭汜寻思曰:“既到这里,不杀天子谋大事,更待何时?”便持剑大呼,杀入内来。正是:巨魁伏罪灾方息,从贼纵横祸又来。未知献帝性命如何,且听下文分解。