\chapter{曹丕乘乱纳甄氏~郭嘉遗计定辽东}

却说曹丕见二妇人啼哭,拔剑欲斩之。忽见红光满目,遂按剑而问曰:“汝何人也?”一妇人告曰:“妾乃袁将军之妻刘氏也。”丕曰:“此女何人?”刘氏曰:“此次男袁熙之妻甄氏也。因熙出镇幽州,甄氏不肯远行,故留于此。”丕拖此女近前,见披发垢而。不以衫袖拭其面而观之,见甄氏玉肌花貌,有倾国之色。遂对刘氏曰:“吾乃曹丞相之子也。愿保汝家。汝勿忧虑。”道按剑坐于堂上。

却说曹操统领众将入冀州城,将入城门,许攸纵马近前,以鞭指城门而呼操曰:“阿瞒,汝不得我,安得入此门?”操大笑。众将闻言,俱怀不平。操至绍府门下,问曰:“谁曾入此门来?”守将对曰:“世子在内。”操唤出责之。刘氏出拜曰:“非世子不能保全妾家,愿就甄氏为世子执箕帚。”操教唤出甄氏拜于前。操视之曰:“真吾儿妇也?”遂令曹不纳之。

操既定冀州,亲往袁绍墓下设祭,再拜而哭甚哀,顾谓众官曰:“昔日吾与本初共起兵时,本初问吾曰:‘若事不辑,方面何所可据?’吾问之曰:‘足下意欲若何?’本初曰:‘吾南据河,北阻燕代,兼沙漠之众,南向以争天下,庶可以济乎?’吾答曰:‘吾任天下之智力,以道御之,无所不可。’此言如昨,而今本初已丧,吾不能不为流涕也!”众皆叹息。操以金帛粮米赐绍妻刘氏。乃下令曰:“河北居民遭兵革之难,尽免今年租赋。”一面写表申朝;操自领冀州牧。

一日,许褚走马入东门,正迎许攸,饮唤褚曰:“汝等无我,安能出入此门乎?”褚怒曰:“吾等千主万死,身冒血战,夺得城池,汝安敢夸口!”攸骂曰:“汝等皆匹夫耳,何足道哉!”褚大怒,拔剑杀攸,提头来见曹操,说“许攸如此无礼,某杀之矣。”操曰:“子远与吾旧交,故相戏耳,何故杀之!”深责许褚,令厚葬许攸。乃令人遍访冀州贤士。冀民曰:“骑都尉崔琐,字季珪,清河东武城人也。数曾献计于袁绍,绍不从,因此托疾在家。”操即召琰为本州别驾从事,因谓曰:“昨按本州户籍,共计三十万众,可谓大州。”琰曰:“今天下分崩,九州幅裂,二袁兄弟相争,冀民暴骨原野,丞相不急存问风俗,救其涂炭,而先计校户籍,岂本州士女所望于明公哉?”操闻言,改容谢之,待为上宾。

操已定冀州,使人探袁谭消息。时谭引兵劫掠甘陵、安平、渤海、河间等处,闻袁尚败走中山,乃统军攻之。尚无心战斗,径奔幽州投袁熙。谭尽降其众,欲复图冀州。操使人召之,谭不至。操大怒,驰书绝其婚,自统大军征之,直抵平原。谭闻操自统军来,遣人求救于刘表。表请玄德商议。玄德曰:“今操已破冀州,兵势正盛,袁氏兄弟不久必为操擒,救之无益;况操常有窥荆襄之意,我只养兵自守,未可妄动。”表曰:“然则何以谢之?”玄德曰:“可作书与袁氏兄弟,以和解为名,婉词谢之。”表然其言,先遣人以书遗谭。书略曰:“君子违难,不适仇国。日前闻君屈膝降曹,则是忘先人之仇,弃手足之谊,而遗同盟之耻矣。若冀州不弟,当降心相从。待事定之后,使天下平其曲直,不亦高义耶?”又与袁尚书曰:“青州天性峭急,迷于曲直。君当先除曹操,以率先公之恨。事定之后,乃计曲直,不亦善乎?若迷而不返,则是韩卢、东郭自困于前,而遗田父之获也。”谭得表书,知表无发兵之意,又自料不能敌操,遂弃平原,走保南皮。

曹操追至南皮,时天气寒肃,河道尽冻,粮船不能行动。操令本处百姓敲冰拽船,百姓闻令而逃。操大怒,欲捕斩之。百姓闻得,乃亲往营中投首。操曰:“若不杀汝等,则吾号令不行;若杀汝等,吾又不忍:汝等快往山中藏避,休被我军士擒获。”百姓皆垂泪而去。

袁谭引兵出城,与曹军相敌。两阵对圆,操出马以鞭指谭而骂曰:“吾厚待汝,汝何生异心?”谭曰:“汝犯吾境界,夺吾城池,赖吾妻子,反说我有异心耶!”操大怒,使徐晃出马。谭使彭安接战。两马相交,不数合,晃斩彭安于马下。谭军败走,退入南皮。操遣军四面围住。谭着慌,使辛评见操约降。操曰:“袁谭小子,反覆无常,吾难准信。汝弟辛毗,吾已重用,汝亦留此可也。”评曰:“丞相差矣。某闻主贵臣荣,主忧臣辱。某久事袁氏,岂可背之!”操知其不可留,乃遣回。评回见谭,言操不准投降。谭叱曰:“汝弟现事曹操,汝怀二心耶?”评闻言,气满填胸,昏绝于地。谭令扶出,须臾而死。谭亦悔之。郭图谓谭曰:“来日尽驱百姓当先,以军继其后,与曹操决一死战。”谭从其言。

当夜尽驱南皮百姓,皆执刀枪听令。次日平明,大开四门,军在后,驱百姓在前,喊声大举,一齐拥出,直抵曹寨。两军混战,自辰至午,胜负未分,杀人遍地。操见未获全胜,弃马上山,亲自击鼓。将士见之,奋力向前,谭军大败。百姓被杀者无数。曹洪奋威突阵,正迎袁谭,举刀乱砍,谭竟被曹洪杀于阵中,郭图见阵大乱,急驰入城中。乐进望见,拈弓搭箭,射下城壕,人马俱陷。操引兵入南皮,安抚百姓。忽有一彪军来到,乃袁熙部将焦触、张南也。操自引军迎之。二将倒戈卸甲,特来投降。操封为列侯。又黑山贼张燕,引军十万来降,操封为平北将军。下令将袁谭首级号令,敢有哭者斩。头挂北门外。一人布冠衰衣,哭于头下。左右拿来见操。操问之,乃青州别驾王修也,因谏袁谭被逐,今知谭死,故来哭之。操曰:“汝知吾令否?”修曰:“知之。”操曰:“汝不怕死耶?”修曰:“我生受其辟命,亡而不哭,非义也。畏死忘义,何以立世乎!若得收葬谭尸,受戮无恨。”操曰:“河北义士,何其如此之多也!可惜袁氏不能用!若能用,则吾安敢正眼觑此地哉!”遂命收葬谭尸,礼修为上宾,以为司金中郎将。因问之曰:“今袁尚已投袁熙,取之当用何策?”修不答。操曰:“忠臣也。”问郭嘉,嘉曰:“可使袁氏降将焦触、张南等自攻之。”操用其言,随差焦触、张南、吕旷、吕翔、马延、张顗,各引本部兵,分三路进攻幽州;一面使李典、乐进会合张燕,打并州,攻高干。且说袁尚、袁熙知曹兵将至,料难迎敌,乃弃城引兵,星夜奔辽西投乌桓去了。幽州刺史乌桓触,聚幽州众官,歃血为盟,共议背袁向曹之事。乌桓触先言曰:“吾知曹丞相当世英雄,今往投降,有不遵令者斩。”依次歃血,循至别驾韩珩。珩乃掷剑于地,大呼曰:“吾受袁公父子厚恩,今主败亡,智不能救,勇不能死,于义缺矣!若北面而降操,吾不为也!”众皆失色。乌桓触曰:“夫兴大事,当立大义。事之济否,不待一人。韩珩既有志如此,听其自便。”推珩而出。乌桓触乃出城迎接三路军马,径来降操。操大喜,加为镇北将军。

忽探马来报:“乐进、李典、张燕攻打并州,高干守住壶关口,不能下。”操自勒兵前往。三将接着,说于拒关难击。操集众将共议破干之计。荀攸曰:“若破干,须用诈降计方可。”操然之。唤降将吕旷、吕翔,附耳低言如此如此。吕旷等引军数十,直抵关下,叫曰:“吾等原系袁氏旧将,不得已而降曹。曹操为人诡谲,薄待吾等;吾今还扶旧主。可疾开关相纳。”高干未信,只教二将自上关说话。二将卸甲弃马而入,谓干曰:“曹军新到,可乘其军心未定,今夜劫寨。某等愿当先。”于喜,从其言,是夜教二吕当先,引万余军前去。将至曹寨,背后喊声大震,伏兵四起。高干知是中计,急回壶关城,乐进、李典已夺了关、高于夺路走脱,往投单于。操领兵拒住关口,使人追袭高干。干到单于界,正迎北番左贤王。干下马拜伏于地,言曹操吞并疆土,今欲犯王子地面,万乞救援,同力克复,以保北方。左贤王曰:“吾与曹操无仇,岂有侵我土地?汝欲使我结怨于曹氏耶!”叱退高干。干寻思无路,只得去投刘表。行至上洛,被都尉王琰所杀,将头解送曹操。曹封琰为列侯。

并州既定,操商议西击乌桓。曹洪等曰:“袁熙、袁尚兵败将亡,势穷力尽,远投沙漠;我今引兵西击,倘刘备、刘表乘虚袭许都,我救应不及,为祸不浅矣:请回师勿进为上。”郭嘉曰:“诸公所言错矣。主公虽威震天下,沙漠之人恃其边远,必不设备;乘其无备,卒然击之,必可破也。且袁绍与乌桓有恩,而尚与熙兄弟犹存,不可不除。刘表坐谈之客耳,自知才不足以御刘备,重任之则恐不能制,轻任之则备不为用。虽虚国远征,公无忧也。”操曰:“奉孝之言极是。”遂率大小三军,车数千辆,望前进发。但见黄沙漠漠,狂风四起;道路崎岖,人马难行。操有回军之心,问于郭嘉。嘉此时不伏水土,卧病车上。操泣曰:“因我欲平沙漠,使公远涉艰辛,以至染病,吾心何安!”嘉曰:“某感丞相大恩,虽死不能报万一。”操曰:“吾见北地崎岖,意欲回军,若何?”嘉曰:“兵贵神速。今千里袭人,辎重多而难以趋利,不如轻兵兼道以出,掩其不备。但须得识径路者为引导耳。”

遂留郭嘉于易州养病,求向导官以引路。人荐袁绍旧将田畴深知此境,操召而问之。畴曰:“此道秋夏间有水,浅不通车马,深不载舟楫,最难行动。不如回军,从卢龙口越白檀之险,出空虚之地,前近柳城,掩其不备:蹋顿可一战而擒也。”操从其言,封田畴为靖北将军,作向导官,为前驱;张辽为次;操自押后:倍道轻骑而进。

田畴引张辽前至白狼山,正遇袁熙、袁尚会合蹋顿等数万骑前来。张辽飞报曹操。操自勒马登高望之,见蹋顿兵无队伍,参差不整。操谓张辽曰:“敌兵不整,便可击之。”乃以麾授辽。辽引许褚、于禁、徐晃分四路下山,奋力急攻,蹋顿大乱。辽拍马斩蹋顿于马下,余众皆降。袁熙、袁尚引数千骑投辽东去了。操收军入柳城,封田畴为柳亭侯,以守柳城。畴涕泣曰:“某负义逃窜之人耳,蒙厚恩全活,为幸多矣;岂可卖卢龙之寨以邀赏禄哉!死不敢受侯爵。”操义之,乃拜畴为议郎。操抚慰单于人等,收得骏马万匹,即日回兵。时天气寒且旱,二百里无水,军又乏粮,杀马为食,凿地三四十丈,方得水。操回至易州,重赏先曾谏者;因谓众将曰:“孤前者乘危远征,侥幸成功。虽得胜,天所佑也,不可以为法。诸君之谏,乃万安之计,是以相赏。后勿难言。”

操到易州时,郭嘉已死数日,停柩在公廨。操往祭之,大哭曰:“奉孝死,乃天丧吾也!”回顾众官曰:“诸君年齿,皆孤等辈,惟奉孝最少,吾欲托以后事。不期中年夭折,使吾心肠崩裂矣!”嘉之左右,将嘉临死所封之书呈上曰:“郭公临亡,亲笔书此,嘱曰:丞相若从书中所言,辽东事定矣。”操拆书视之,点头嗟叹。诸人皆不知其意。次日,夏侯惇引众人禀曰:“辽东太守公孙康,久不宾服。今袁熙、袁尚又往投之,必为后患。不如乘其未动,速往征之,辽东可得也。”操笑曰:“不烦诸公虎威。数日之后,公孙康自送二袁之首至矣。”诸将皆不肯信。却说袁熙、袁尚引数千骑奔辽东。辽东太守公孙康,本襄平人,武威将军公孙度之子也。当日知袁熙、袁尚来投,遂聚本部属官商议此事。公孙恭曰:“袁绍在日,常有吞辽东之心;今袁熙,袁尚兵败将亡,无处依栖,来此相投,是鸠夺鹊巢之意也。若容纳之,后必相图。不如赚入城中杀之,献头与曹公,曹公必重待我。”康曰:“只怕曹操引兵下辽东,又不如纳二袁使为我助。”恭曰:“可使人探听。如曹兵来攻,则留二袁;如其不动,则杀二袁,送与曹公。”康从之,使人去探消息。却说袁熙、袁尚至辽东,二人密议曰:“辽东军兵数万,足可与曹操争衡。今暂投之,后当杀公孙康而夺其地,养成气力而抗中原,可复河北也。”商议已定,乃入见公孙康。康留于馆驿,只推有病,不即相见。不一日,细作回报:“曹公兵屯易州,并无下辽东之意。”公孙康大喜,乃先伏刀斧手于壁衣中,使二袁入。相见礼毕,命坐。时天气严寒,尚见床榻上无茵褥,谓康曰:“愿铺坐席。”康瞋目言曰:“汝二人之头,将行万里!何席之有!尚大惊。康叱曰:“左右何不下手!”刀斧手拥出,就坐席上砍下二人之头,用木匣盛贮,使人送到易州,来见曹操。时操在易州,按兵不动。夏侯惇、张辽入禀曰:“如不下辽东,可回许都。恐刘表生心。”操曰:“待二袁首级至,即便回兵。”众皆暗笑。忽报辽东公孙康遣人送袁熙、袁尚首级至,众皆大惊。使者呈上书信。操大笑曰:“不出奉孝之料!”重赏来使,封公孙康为襄平侯、左将军。众官问曰:“何为不出奉孝之所料?”操遂出郭嘉书以示之。书略曰:“今闻袁熙、袁尚往投辽东,明公切不可加兵。公孙康久畏袁氏吞并,二袁往投必疑。若以兵击之,必并力迎敌,急不可下;若缓之,公孙康、袁氏必自相图,其势然也。”众皆踊跃称善。操引众官复设祭于郭嘉灵前。亡年三十八岁,从征十有一年,多立奇勋。后人有诗赞曰:“天生郭奉孝,豪杰冠群英:腹内藏经史,胸中隐甲兵;运谋如范蠡,决策似陈平。可惜身先丧,中原梁栋倾。”操领兵还冀州,使人先扶郭嘉灵柩于许都安葬。

程昱等请曰:“北方既定,今还许都,可早建下江南之策。”操笑曰:“吾有此志久矣。诸君所言,正合吾意。”是夜宿于冀州城东角楼上,凭栏仰观天文。时荀攸在侧,操指曰:“南方旺气灿然,恐未可图也。”攸曰:“以丞相天威,何所不服!正看间,忽见一道金光,从地而起。攸曰:“此必有宝于地下”。操下楼令人随光掘之。正是:星文方向南中指,金宝旋从北地生。不知所得何物,且听下文分解。