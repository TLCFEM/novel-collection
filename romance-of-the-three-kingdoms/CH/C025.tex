\chapter{屯土山关公约三事~救白马曹操解重围}

却说程昱献计曰:“云长有万人之敌,非智谋不能取之。今可即差刘备手下投降之兵,入下邳,见关公,只说是逃回的,伏于城中为内应;却引关公出战,诈败佯输,诱入他处,以精兵截其归路,然后说之可也。”操听其谋,即令徐州降兵数十,径投下邳来降关公。关公以为旧兵,留而不疑。

次日,夏侯惇为先锋,领兵五千来搦战。关公不出,惇即使人于城下辱骂。关公大怒,引三千人马出城,与夏侯惇交战。约战十馀合,惇拨回马走。关公赶来,惇且战且走。关公约赶二十里,恐下邳有失,提兵便回。只听得一声炮响,左有徐晃,右有许褚,两队军截住去路,关公夺路而走,两边伏兵排下硬弩百张,箭如飞蝗。关公不得过,勒兵再回,徐晃、许褚接住交战。关公奋力杀退二人,引军欲回下邳,夏侯惇又截住厮杀。公战至日晚,无路可归,只得到一座土山,引兵屯于山头,权且少歇。曹兵团团将土山围住。关公于山上遥望下邳城中火光冲天,却是那诈降兵卒偷开城门,曹操自提大军杀入城中,只教举火以惑关公之心。关公见下邳火起,心中惊惶,连夜几番冲下山来,皆被乱箭射回。

捱到天晓,再欲整顿下山冲突,忽见一人跑马上山来,视之乃张辽也。关公迎谓曰:“文远欲来相敌耶?”辽曰:“非也。想故人旧日之情,特来相见。”遂弃刀下马,与关公叙礼毕,坐于山顶。公曰:“文远莫非说关某乎?”辽曰:“不然。昔日蒙兄救弟,今日弟安得不救兄?”公曰:“然则文远将欲助我乎?”辽曰:“亦非也。”公曰:“既不助我,来此何干?”辽曰:“玄德不知存亡,翼德未知生死。昨夜曹公已破下邳,军民尽无伤害,差人护卫玄德家眷,不许惊忧。如此相待,弟特来报兄。”关公怒曰:“此言特说我也。吾今虽处绝地,视死如归。汝当速去,吾即下山迎战。”张辽大笑曰:“兄此言岂不为天下笑乎?”公曰:“吾仗忠义而死,安得为天下笑?”辽曰:“兄今即死,其罪有三。”公曰:“汝且说我那三罪?”辽曰:“当初刘使君与兄结义之时,誓同生死;今使君方败,而兄即战死,倘使君复出,欲求兄相助,而不可复得,岂不负当年之盟誓乎?其罪一也。刘使君以家眷付托于兄,兄今战死,二夫人无所依赖,负却使君依托之重。其罪二也。兄武艺超群,兼通经史,不思共使君匡扶汉室,徒欲赴汤蹈火,以成匹夫之勇,安得为义?其罪三也。兄有此三罪,弟不得不告。”

公沉吟曰:“汝说我有三罪,欲我如何?”辽曰:“今四面皆曹公之兵,兄若不降,则必死;徒死无益,不若且降曹公;却打听刘使君音信,如知何处,即往投之。一者可以保二夫人,二者不背桃园之约,三者可留有用之身:有此三便,兄宜详之。”公曰:“兄言三便,吾有三约。若丞相能从,我即当卸甲;如其不允,吾宁受三罪而死。”辽曰:“丞相宽洪大量,何所不容。愿闻三事。”公曰:“一者,吾与皇叔设誓,共扶汉室,吾今只降汉帝,不降曹操;二者,二嫂处请给皇叔俸禄养赡,一应上下人等,皆不许到门;三者,但知刘皇叔去向,不管千里万里,便当辞去:三者缺一,断不肯降。望文远急急回报。”张辽应诺,遂上马,回见曹操,先说降汉不降曹之事。操笑曰:“吾为汉相,汉即吾也。此可从之。”辽又言:“二夫人欲请皇叔俸给,并上下人等不许到门。”操曰:“吾于皇叔俸内,更加倍与之。至于严禁内外,乃是家法,又何疑焉!”辽又曰:“但知玄德信息,虽远必往。”操摇首曰:“然则吾养云长何用?此事却难从。”辽曰:“岂不闻豫让众人国士之论乎?刘玄德待云长不过恩厚耳。丞相更施厚恩以结其心,何忧云长之不服也?”操曰:“文远之言甚当,吾愿从此三事。”张辽再往山上回报关公。关公曰:“虽然如此,暂请丞相退军,容我入城见二嫂,告知其事,然后投降。”张辽再回,以此言报曹操。操即传令,退军三十里。荀彧曰:“不可,恐有诈。”操曰:“云长义士,必不失信。”遂引军退。关公引兵入下邳,见人民安妥不动,竟到府中。来见二嫂。甘、糜二夫人听得关公到来,急出迎之。公拜于阶下曰:“使二嫂受惊,某之罪也。”二夫人曰:“皇叔今在何处?”公曰:“不知去向。”二夫人曰:“二叔今将若何?”公曰:“关某出城死战,被困土山,张辽劝我投降,我以三事相约。曹操已皆允从,故特退兵,放我入城。我不曾得嫂嫂主意,未敢擅便。”二夫人问:“那三事?”关公将上项三事,备述一遍。甘夫人曰:“昨日曹军入城,我等皆以为必死;谁想毫发不动,一军不敢入门。叔叔既已领诺,何必问我二人?只恐日后曹操不容叔叔去寻皇叔。”公曰:“嫂嫂放心,关某自有主张。”二夫人曰:“叔叔自家裁处,凡事不必问俺女流。”

关公辞退,遂引数十骑来见曹操。操自出辕门相接。关公下马入拜,操慌忙答礼。关公曰:“败兵之将,深荷不杀之恩。”操曰:“素慕云长忠义,今日幸得相见,足慰平生之望。”关公曰:“文远代禀三事,蒙丞相应允,谅不食言。”操曰:“吾言既出,安敢失信。”关公曰:“关某若知皇叔所在,虽蹈水火、必往从之。此时恐不及拜辞,伏乞见原。”操曰:“玄德若在,必从公去;但恐乱军中亡矣。公且宽心,尚容缉听。”关公拜谢。操设宴相待。次日班师还许昌。关公收拾车仗,请二嫂上车,亲自护车而行。于路安歇馆驿,操欲乱其君臣之礼,使关公与二嫂共处一室。关公乃秉烛立于户外,自夜达旦,毫无倦色。操见公如此,愈加敬服。既到许昌,操拨一府与关公居住。关公分一宅为两院,内门拨老军十人把守,关公自居外宅。

操引关公朝见献帝,帝命为偏将军。公谢恩归宅。操次日设大宴,会众谋臣武士,以客礼待关公,延之上座;又备绫锦及金银器皿相送。关公都送与二嫂收贮。关公自到许昌,操待之甚厚:小宴三日,大宴五日;又送美女十人,使侍关公。关公尽送入内门,令伏侍二嫂。却又三日一次于内门外躬身施礼,动问二嫂安否。二夫人回问皇叔之事毕,曰“叔叔自便”,关公方敢退回。操闻之,又叹服关公不已。

一日,操见关公所穿绿锦战袍已旧,即度其身品,取异锦作战袍一领相赠。关公受之,穿于衣底,上仍用旧袍罩之。操笑曰:“云长何如此之俭乎?”公曰:“某非俭也。旧袍乃刘皇叔所赐,某穿之如见兄面,不敢以丞相之新赐而忘兄长之旧赐,故穿于上。”操叹曰:“真义士也!”然口虽称羡,心实不悦。一日,关公在府,忽报:“内院二夫人哭倒于地,不知为何,请将军速入。”关公乃整衣跪于内门外,问二嫂为何悲泣。甘夫人曰:“我夜梦皇叔身陷于土坑之内,觉来与糜夫人论之,想在九泉之下矣!是以相哭。”关公曰:“梦寐之事,不可凭信,此是嫂嫂想念之故。请勿忧愁。”

正说间,适曹操命使来请关公赴宴。公辞二嫂,往见操。操见公有泪容,问其故。公曰:“二嫂思兄痛哭,不由某心不悲。”操笑而宽解之,频以酒相劝。公醉,自绰其髯而言曰:“生不能报国家,而背其兄,徒为人也!”操问曰:“云长髯有数乎?”公曰:“约数百根。每秋月约退三五根。冬月多以皂纱囊裹之,恐其断也。”操以纱锦作囊,与关公护髯。次日,早朝见帝。帝见关公一纱锦囊垂于胸次,帝问之。关公奏曰:“臣髯颇长,丞相赐囊贮之。”帝令当殿披拂,过于其腹。帝曰:“真美髯公也!”因此人皆呼为“美髯公”。

忽一日,操请关公宴。临散,送公出府,见公马瘦,操曰:“公马因何而瘦?”关公曰:“贱躯颇重,马不能载,因此常瘦。”操令左右备一马来。须臾牵至。那马身如火炭,状甚雄伟。操指曰:“公识此马否?”公曰:“莫非吕布所骑赤兔马乎?”操曰:“然也。”遂并鞍辔送与关公。关公再拜称谢。操不悦曰:“吾累送美女金帛,公未尝下拜;今吾赠马,乃喜而再拜:何贱人而贵畜耶?”关公曰:“吾知此马日行千里,今幸得之,若知兄长下落,可一日而见面矣。”操愕然而悔。关公辞去。后人有诗叹曰:“威倾三国著英豪,一宅分居义气高。奸相枉将虚礼待,岂知关羽不降曹。”操问张辽曰:“吾待云长不薄,而彼常怀去心,何也?”辽曰:“容某探其情。”次日,往见关公。礼毕,辽曰:“我荐兄在丞相处,不曾落后?”公曰:“深感丞相厚意。只是吾身虽在此,心念皇叔,未尝去怀。”辽曰:“兄言差矣,处世不分轻重,非丈夫也。玄德待兄,未必过于丞相,兄何故只怀去志?”公曰:“吾固知曹公待吾甚厚。奈吾受刘皇叔厚恩,誓以共死,不可背之。吾终不留此。要必立效以报曹公,然后去耳。”辽曰:“倘玄德已弃世,公何所归乎?”公曰:“愿从于地下。”辽知公终不可留,乃告退,回见曹操,具以实告。操叹曰:“事主不忘其本,乃天下之义士也!”荀彧曰:“彼言立功方去,若不教彼立功,未必便去。”操然之。却说玄德在袁绍处,旦夕烦恼。绍曰:“玄德何故常忧?”玄德曰:“二弟不知音耗,妻小陷于曹贼;上不能报国,下不能保家:安得不忧?”绍曰:“吾欲进兵赴许都久矣。方今春暖,正好兴兵。”便商议破曹之策。田丰谏曰:“前操攻徐州,许都空虚,不及此时进兵;今徐州已破,操兵方锐,未可轻敌。不如以久持之,待其有隙而后可动也。”绍曰:“待我思之。”因问玄德曰:“田丰劝我固守,何如!”玄德曰:“曹操欺君之贼,明公若不讨之,恐失大义于天下。”绍曰:“玄德之言甚善。”遂欲兴兵。田丰又谏。绍怒曰:“汝等弄文轻武,使我失大义!”田丰顿首曰:“若不听臣良言,出师不利。”绍大怒,欲斩之。玄德力劝,乃囚于狱中,沮授见田丰下狱,乃会其宗族,尽散家财,与之诀曰:“吾随军而去,胜则威无不加,败则一身不保矣!”众皆下泪送之。

绍遣大将颜良作先锋,进攻白马。沮授谏曰:“颜良性狭,虽骁勇,不可独任。”绍曰:“吾之上将,非汝等可料。”大军进发至黎阳,东郡太守刘延告急许昌。曹操急议兴兵抵敌。关公闻知,遂入相府见操曰:“闻丞相起兵,某愿为前部。”操曰:“未敢烦将军。早晚有事,当来相请。”关公乃退。

操引兵十五万,分三队而行。于路又连接刘延告急文书,操先提五万军亲临白马,靠土山扎住。遥望山前平川旷野之地,颜良前部精兵十万,排成阵势。操骇然,回顾吕布旧将宋宪曰:“吾闻汝乃吕布部下猛将,今可与颜良一战。”宋宪领诺,绰枪上马,直出阵前。颜良横刀立马于门旗下;见宋宪马至,良大喝一声,纵马来迎。战不三合,手起刀落,斩宋宪于阵前。曹操大惊曰:“真勇将也!”魏续曰:“杀我同伴,愿去报仇!”操许之。续上马持矛,径出阵前,大骂颜良。良更不打话,交马一合,照头一刀,劈魏续于马下。操曰:“今谁敢当之?”徐晃应声而出,与颜良战二十合,败归本阵。诸将栗然。曹操收军,良亦引军退去。

操见连斩二将,心中忧闷。程昱曰:“某举一人可敌颜良。”操问是谁。昱曰:“非关公不可。”操曰:“吾恐他立了功便去。”昱曰:“刘备若在,必投袁绍。今若使云长破袁绍之兵,绍必疑刘备而杀之矣。备既死,云长又安往乎?”操大喜,遂差人去请关公。关公即入辞二嫂。二嫂曰:“叔今此去,可打听皇叔消息。”关公领诺而出,提青龙刀,上赤兔马,引从者数人,直至白马来见曹操。操叙说:“颜良连诛二将,勇不可当,特请云长商议。”关公曰:“容某观之。”操置酒相待。忽报颜良搦战。操引关公上土山观看。操与关公坐,诸将环立。曹操指山下颜良排的阵势,旗帜鲜明,枪刀森布,严整有威,乃谓关公曰:“河北人马,如此雄壮!”关公曰:“以吾观之,如土鸡瓦犬耳!”操又指曰:“麾盖之下,绣袍金甲,持刀立马者,乃颜良也。”关公举目一望,谓操曰:“吾观颜良,如插标卖首耳!”操曰:“未可轻视。”关公起身曰:“某虽不才,愿去万军中取其首级,来献丞相。”张辽曰:“军中无戏言,云长不可忽也。”关公奋然上马,倒提青龙刀,跑下山来,凤目圆睁,蚕眉直竖,直冲彼阵。河北军如波开浪裂,关公径奔颜良。颜良正在麾盖下,见关公冲来,方欲问时,关公赤兔马快,早已跑到面前;颜良措手不及,被云长手起一刀,刺于马下。忽地下马,割了颜良首级,拴于马项之下,飞身上马,提刀出阵,如入无人之境。河北兵将大惊,不战自乱。曹军乘势攻击,死者不可胜数;马匹器械,抢夺极多。关公纵马上山,众将尽皆称贺。公献首级于操前。操曰:“将军真神人也!”关公曰:“某何足道哉!吾弟张翼德于百万军中取上将之头,如探囊取物耳。”操大惊,回顾左右曰:“今后如遇张翼德,不可轻敌。”令写于衣袍襟底以记之。

却说颜良败军奔回,半路迎见袁绍,报说被赤面长须使大刀一勇将,匹马入阵,斩颜良而去,因此大败。绍惊问曰:“此人是谁?”沮授曰:“此必是刘玄德之弟关云长也。”绍大怒,指玄德曰:“汝弟斩吾爱将,汝必通谋,留尔何用!”唤刀斧手推出玄德斩之。正是:初见方为座上客,此日几同阶下囚。未知玄德性命如何,且听下文分解。