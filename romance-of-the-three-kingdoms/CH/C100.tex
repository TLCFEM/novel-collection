\chapter{汉兵劫寨破曹真~武侯斗阵辱仲达}

却说众将闻孔明不追魏兵,俱入帐告曰:“魏兵苦雨,不能屯扎,因此回去,正好乘势追之。丞相如何不追?”孔明曰:“司马懿善能用兵,今军退必有埋伏。吾若追之,正中其计。不如纵他远去,吾却分兵径出斜谷而取祁山,使魏人不提防也。”众将曰:“取长安之地,别有路途;丞相只取祁山,何也?”孔明曰:“祁山乃长安之首也:陇西诸郡,倘有兵来,必经由此地;更兼前临渭滨,后靠斜谷,左出右入。可以伏兵,乃用武之地。吾故欲先取此,得地利也。”众将皆拜服。孔明令魏延、张嶷、杜琼、陈式出箕谷;马岱、王平、张翼、马忠出斜谷:俱会于祁山。调拨已定,孔明自提大军,令关兴、廖化为先锋,随后进发。却说曹真、司马懿二人,在后监督人马,令一军入陈仓古道探视,回报说蜀兵不来。又行旬日,后面埋伏众将皆回,说蜀兵全无音耗。真曰:“连绵秋雨,栈道断绝,蜀人岂知吾等退军耶?”懿曰:“蜀兵随后出矣。”真曰:“何以知之?”懿曰:“连日晴明,蜀兵不赶,料吾有伏兵也,故纵我兵远去;待我兵过尽,他却夺祁山矣。”曹真不信。懿曰:“子丹如何不信?吾料孔明必从两谷而来。吾与子丹各守一谷口,十日为期。若无蜀兵来,我面涂红粉,身穿女衣,来营中伏罪。”真曰:“若有蜀兵来,我愿将天子所赐玉带一条、御马一匹与你。”即分兵两路:真引兵屯于祁山之西斜谷口;懿引军屯于祁山之东箕谷口。各下寨已毕。懿先引一枝兵伏于山谷中;其余军马,各于要路安营。懿更换衣装,杂在全军之内,遍观各营。忽到一营,有一偏将仰天而怨曰:“大雨淋了许多时,不肯回去;今又在这里顿住,强要赌赛,却不苦了官军!”懿闻言,归寨升帐,聚众将皆到帐下,挨出那将来。懿叱之曰:“朝廷养军千日,用在一时。汝安敢出怨言,以慢军心!”其人不招。懿叫出同伴之人对证,那将不能抵赖。懿曰:“吾非赌赛;欲胜蜀兵,令汝各人有功回朝,汝乃妄出怨言,自取罪戾!”喝令武士推出斩之。须臾,献首帐下。众将悚然。懿曰:“汝等诸将皆要尽心以防蜀兵。听吾中军炮响,四面皆进。”众将受令而退。

却说魏延、张嶷、陈式、杜琼四将,引二万兵,取箕谷而进。正行之间,忽报参谋邓芝到来。四将问其故,芝曰:“丞相有令:如出箕谷,提防魏兵埋伏,不可轻进。”陈式曰:“丞相用兵何多疑耶?吾料魏兵连遭大雨,衣甲皆毁,必然急归;安得又有埋伏?今吾兵倍道而进,可获大胜,如何又教休进?”芝曰:“丞相计无不中,谋无不成,汝安敢违令?”式笑曰:“丞相若果多谋,不致街亭之失!”魏延想起孔明向日不听其计,亦笑曰:“丞相若听吾言,径出子午谷,此时休说长安,连洛阳皆得矣!今执定要出祁山。有何益耶?既令进兵,今又教休进。何其号令不明!”式曰:“吾自有五千兵,径出箕谷,先到祁山下寨,看丞相羞也不羞!”芝再三阻当,式只不听,径自引五千兵出箕谷去了。邓芝只得飞报孔明。

却说陈式引兵行不数里,忽听的一声炮响,四面伏兵皆出。式急退时,魏兵塞满谷口,围得铁桶相似。式左冲右突,不能得脱。忽闻喊声大震,一彪军杀入,乃是魏延。救了陈式,回到谷中,五千兵只剩得四五百带伤人马。背后魏兵赶来,却得杜琼、张嶷引兵接应,魏兵方退。陈、魏二人方信孔明先见如神,懊悔不及。

且说邓芝回见孔明,言魏延、陈式如此无礼。孔明笑曰:“魏延素有反相,吾知彼常有不平之意;因怜其勇而用之。久后必生患害。”正言间,忽流星马报到,说陈式折了四千余人,止有四五百带伤人马,屯在谷中。孔明令邓芝再来箕谷抚慰陈式,防其生变;一面唤马岱、王平分付曰:“斜谷若有魏兵守把,汝二人引本部军越山岭,夜行昼伏,速出祁山之左,举火为号。”又唤马忠、张翼分付曰:“汝等亦从山僻小路,昼伏夜行,径出祁山之右,举火为号,与马岱、王平会合,共劫曹真营寨。吾自从谷中三面攻之,魏兵可破也。”四人领命分头引兵去了。孔明又唤关兴、廖化分付曰:如此如此。二人受了密计,引兵而去。孔明自领精兵倍道而行。正行间,又唤吴班、吴懿授与密计,亦引兵先行。

却说曹真心中不信蜀兵来,以此怠慢,纵令军士歇息;只等十日无事,要羞司马懿,不觉守了七日,忽有人报谷中有些小蜀兵出来。真令副将秦良引五千兵哨探,不许纵令蜀兵近界。秦良领命,引兵刚到谷口,哨见蜀兵退去。良急引兵赶来,行到五六十里,不见蜀兵,心下疑惑,教军士下马歇息。忽哨马报说:“前面有蜀兵埋伏。”良上马看时,只见山中尘土大起,急令军士提防。不一时,四壁厢喊声大震:前面吴班、吴懿引兵杀出,背后关兴、廖化引兵杀来。左右是山,皆无走路。山上蜀兵大叫:“下马投降者免死!”魏兵大半多降。秦良死战,被廖化一刀斩于马下。

孔明把降兵拘于后军,却将魏兵衣甲与蜀兵五千人穿了,扮作魏兵,令关兴、廖化、吴班、吴懿四将引着,径奔曹真寨来;先令报马入寨说:“只有些小蜀兵,尽赶去了。”真大喜。忽报司马都督差心腹人至。真唤入问之。其人告曰:“今都督用埋伏计,杀蜀兵四千余人。司马都督致意将军,教休将赌赛为念,务要用心提备。”真曰:“吾这里并无一个蜀兵。”遂打发来人回去。忽又报秦良引兵回来了。真自出帐迎之。比及到寨,人报前后两把火起。真急回寨后看时,关兴、廖化、吴班、吴懿四将,指麾蜀军,就营前杀将进来;马岱、王平从后面杀来;马忠、张翼亦引兵杀到。魏军措手不及,各自逃生。众将保曹真望东而走,背后蜀兵赶来。

曹真正奔走,忽然喊声大震,一彪军杀到。真胆战心惊,视之,乃司马懿也。懿大战一场,蜀兵方退。真得脱,羞惭无地。懿曰:“诸葛亮夺了祁山地势,吾等不可久居此处;宜去渭滨安营,再作良图。”真曰:“仲达何以知吾遭此大败也?”懿曰:“见来人报称子丹说并无一个蜀兵,吾料孔明暗来劫寨,因此知之,故相接应。今果中计。切莫言赌赛之事,只同心报国。”曹真甚是惶恐,气成疾病,卧床不起。兵屯渭滨,懿恐军心有乱,不敢教真引兵。

却说孔明大驱士马,复出祁山。劳军已毕,魏延、陈式、杜琼、张嶷入帐拜伏请罪。孔明曰:“是谁失陷了军来?”延曰:“陈式不听号令,潜入谷口,以此大败。”式曰:“此事魏延教我行来。”孔明曰:“他倒救你,你反攀他!将令已违,不必巧说!”即叱武士推出陈式斩之。须臾,悬首于帐前,以示诸将。此时孔明不杀魏延,欲留之以为后用也。

孔明既斩了陈式,正议进兵,忽有细作报说曹真卧病不起,现在营中治疗。孔明大喜,谓诸将曰:“若曹真病轻,必便回长安。今魏兵不退,必为病重,故留于军中,以安众人之心。吾写下一书,教秦良的降兵持与曹真,真若见之,必然死矣!”遂唤降兵至帐下,问曰:“汝等皆是魏军,父母妻子多在中原,不宜久居蜀中。今放汝等回家,若何?”众军泣泪拜谢。孔明曰:“曹子丹与吾有约;吾有一书,汝等带回,送与子丹,必有重赏。”魏军领了书,奔回本寨,将孔明书呈与曹真。真扶病而起,拆封视之。其书曰:“汉丞相、武乡侯诸葛亮,致书于大司马曹子丹之前:窃谓夫为将者,能去能就,能柔能刚;能进能退,能弱能强。不动如山岳,难测如阴阳;无穷如天地,充实如太仓;浩渺如四海,眩曜如三光。预知天文之旱涝,先识地理之平康;察阵势之期会,揣敌人之短长。嗟尔无学后辈,上逆穹苍;助篡国之反贼,称帝号于洛阳;走残兵于斜谷,遭霖雨于陈仓;水陆困乏,人马猖狂;抛盈郊之戈甲,弃满地之刀枪;都督心崩而胆裂,将军鼠窜而狼忙!无面见关中之父老,何颜入相府之厅堂!史官秉笔而记录,百姓众口而传扬:仲达闻阵而惕惕,子丹望风而遑遑!吾军兵强而马壮,大将虎奋以龙骧;扫秦川为平壤,荡魏国作丘荒!”曹真看毕,恨气填胸;至夜,死于军中。司马懿用兵车装载,差人送赴洛阳安葬。

魏主闻知曹真已死,即下诏催司马懿出战。懿提大军来与孔明交锋,隔日先下战书。孔明谓诸将曰:“曹真必死矣。”遂批回“来日交锋”,使者去了。孔明当夜教姜维受了密计:如此而行;又唤关兴分付:如此如此。

次日,孔明尽起祁山之兵前到谓滨:一边是河,一边是山,中央平川旷野,好片战场!两军相迎,以弓箭射住阵角。三通鼓罢,魏阵中门旗开处,司马懿出马,众将随后而出。只见孔明端坐于四轮车上,手摇羽扇。懿曰:“吾主上法尧禅舜,相传二帝,坐镇中原,容汝蜀、吴二国者,乃吾主宽慈仁厚,恐伤百姓也。汝乃南阳一耕夫,不识天数,强要相侵,理宜殄灭!如省心改过,宜即早回,各守疆界,以成鼎足之势,免致生灵涂炭,汝等皆得全生!”孔明笑曰:“吾受先帝托孤之重,安肯不倾心竭力以讨贼乎!汝曹氏不久为汉所灭。汝祖父皆为汉臣,世食汉禄,不思报效,反助篡逆,岂不自耻?”懿羞惭满面曰:“吾与汝决一雌雄!汝若能胜,吾誓不为大将!汝若败时,早归故里,吾并不加害。”

孔明曰:“汝欲斗将?斗兵?斗阵法?”懿曰:“先斗阵法?”孔明曰:“先布阵我看。懿入中军帐下,手执黄旗招飐,左右军动,排成一阵。复上马出阵,问曰:“汝识吾阵否?”孔明笑曰:“吾军中末将,亦能布之。此乃混元一气阵也。”懿曰:“汝布阵我看。”孔明入阵,把羽扇一摇,复出阵前,问曰:“汝识我阵否?”懿曰:“量此八卦阵,如何不识!”孔明曰:“识便识了,敢打我阵否?”懿曰:“既识之,如何不敢打!”孔明曰:“汝只管打来。”司马懿回到本阵中,唤戴陵、张虎、乐綝三将,分付曰:“今孔明所布之阵,按休、生、伤、杜、景、死、惊、开八门。汝三人可从正东生门打入,往西南休门杀出,复从正北开门杀入:此阵可破。汝等小心在意!”

于是戴陵在中,张虎在前,乐綝在后,各引三十骑,从生门打入。两军呐喊相助。三人杀入蜀阵,只见阵如连城,冲突不出。三人慌引骑转过阵脚,往西南冲去,却被蜀兵射住,冲突不出。阵中重重叠叠,都有门户,那里分东西南北?三将不能相顾,只管乱撞,但见愁云漠漠,惨雾蒙蒙。喊声起处,魏军一个个皆被缚了,送到中军。

孔明坐于帐中,左右将张虎、戴陵、乐綝并九十个军,皆缚在帐下。孔明笑曰:“吾纵然捉得汝等,何足为奇!吾放汝等回见司马懿,教他再读兵书,重观战策,那时来决雌雄,未为迟也。汝等性命既饶,当留下军器战马。”遂将众人衣服脱了,以墨涂面,步行出阵。司马懿见之大怒,回顾诸将曰:“如此挫败锐气,有何面目回见中原大臣耶!”即指挥三军,奋死掠阵,懿自拔剑在手,引百余骁将,催督冲杀。

两军恰才相会,忽然阵后鼓角齐鸣,喊声大震,一彪军从西南上杀来,乃关兴也。懿分后军当之,复催军向前厮杀。忽然魏兵大乱:原来姜维引一彪军悄地杀来,蜀兵三路夹攻。懿大惊,急忙退军。蜀兵周围杀到,懿引三军望南死命冲击。魏兵十伤六七。司马懿退在渭滨南岸下寨,坚守不出。

孔明收得胜之兵,回到祁山时,永安城李严遣都尉苟安解送粮米,至军中交割。苟安好酒,于路怠慢,违限十日。孔明大怒曰:“吾军中专以粮为大事,误了三日,便该处斩!汝今误了十日,有何理说?”喝令推出斩之。长史杨仪曰:“苟安乃李严用人,又兼钱粮多出于西川,若杀此人,后无人敢送粮也。”孔明乃叱武士去其缚,杖八十放之。苟安被责,心中怀恨,连夜引亲随五六骑,径奔魏寨投降。懿唤入,苟安拜告前事。懿曰:“虽然如此,孔明多谋,汝言难信。汝能为我干一件大功,吾那时奏准天子,保汝为上将。”安曰:“但有甚事,即当效力。”懿曰:“汝可回成都布散流言,说孔明有怨上之意,早晚欲称为帝,使汝主召回孔明:即是汝之功矣。”苟安允诺,径回成都,见了宦官,布散流言,说孔明自倚大功,早晚必将篡国。宦官闻知大惊,即入内奏帝,细言前事。后主惊讶曰:“似此如之奈何?宦官曰:“可诏还成都,削其兵权,免生叛逆。”后主下诏,宣孔明班师回朝。蒋琬出班奏曰:“丞相自出师以来,累建大功,何故宣回?”后主曰:“朕有机密事,必须与丞相面议。”即遣使赍诏星夜宣孔明回。

使命径到祁山大寨,孔明接入,受诏已毕,仰天叹曰:“主上年幼,必有佞臣在侧!吾正欲建功,何故取回?我如不回,是欺主矣。若奉命而退,日后再难得此机会也。”姜维问曰:“若大军退,司马懿乘势掩杀,当复如何?”孔明曰:“吾今退军,可分五路而退。今日先退此营,假如营内一千兵,却掘二千灶,明日掘三千灶,后日掘四千灶:每日退军,添灶而行。”杨仪曰:“昔孙膑擒庞滑,用添兵减灶之法而取胜;今丞相退兵,何故增灶?”孔明曰:“司马懿善能用兵,知吾兵退,必然追赶;心中疑吾有伏兵,定于旧营内数灶;见每日增灶,兵又不知退与不退,则疑而不敢追。吾徐徐而退,自无损兵之患。”遂传令退军。

却说司马懿料苟安行计停当,只待蜀兵退时,一齐掩杀。正踌躇间,忽报蜀寨空虚,人马皆去。懿因孔明多谋,不敢轻追,自引百余骑前来蜀营内踏看,教军士数灶,仍回本寨;次日,又教军士赶到那个营内,查点灶数。回报说:“这营内之灶,比前又增一分。”司马懿谓诸将曰:“吾料孔明多谋,今果添兵增灶,吾若追之,必中其计;不如且退,再作良图。”于是回军不追。孔明不折一人,望成都而去。次后,川口土人来报司马懿,说孔明退兵之时,未见添兵,只见增灶。懿仰天长叹曰:“孔明效虞诩之法,瞒过吾也!其谋略吾不如之!”遂引大军还洛阳。正是:棋逢敌手难相胜,将遇良才不敢骄。未知孔明退回成都,竟是如何,且看下文分解。