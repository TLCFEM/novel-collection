\chapter{蔡夫人隔屏听密语~刘皇叔跃马过檀溪}

却说曹操于金光处,掘出一铜雀,间荀攸曰:“此何兆也?”攸曰:“昔舜母梦玉雀入怀而生舜。今得铜雀,亦吉祥之兆也。”操大喜,遂命作高台以庆之。乃即日破土断木,烧瓦磨砖,筑铜雀台于漳河之上。约计一年而工毕。少子曹植进曰:“若建层台,必立三座:中间高者,名为铜雀;左边一座,名为玉龙;右边一座,名为金凤。更作两条飞桥,横空而上,乃为壮观。”操曰:“吾儿所言甚善。他日台成,足可娱吾者矣!”原来曹操有五子,惟植性敏慧,善文章,曹操平日最爱之。于是留曹植与曹丕在邺郡造台,使张燕守北寨。操将所得袁绍之兵,共五六十万,班师回许都。大封功臣;又表赠郭嘉为贞侯,养其子奕于府中。复聚众谋士商议,欲南征刘表。荀彧曰:“大军方北征而回,未可复动。且待半年,养精蓄锐,刘表、孙权可一鼓而下也。”操从之,遂分兵屯田,以候调用。

却说玄德自到荆州,刘表待之甚厚。一日,正相聚饮酒,忽报降将张武、陈孙在江夏掳掠人民,共谋造反。表惊曰:“二贼又反,为祸不小!”玄德曰:“不须兄长忧虑,备请往讨之。”表大喜,即点三万军,与玄德前去。玄德领命即行,不一日,来到江夏。张武、陈孙引兵来迎。玄德与关、张、赵云出马在门旗下,望见张武所骑之马,极其雄骏。玄德曰:“此必千里马也。”言未毕,赵云挺枪而出,径冲彼阵。张武纵马来迎,不三合,被赵云一枪刺落马下,随手扯住辔头,牵马回阵。陈孙见了,随赶来夺。张飞大喝一声,挺矛直出,将陈孙刺死。众皆溃散。玄德招安余党,平复江夏诸县,班师而回。表出郭迎接入城,设宴庆功。酒至半酣,表曰:“吾弟如此雄才,荆州有倚赖也。但忧南越不时来寇,张鲁、孙权皆足为虑。”玄德曰:“弟有三将,足可委用:使张飞巡南越之境;云长拒固子城,以镇张鲁;赵云拒三江,以当孙权。何足虑哉?”表喜,欲从其言。

蔡瑁告其姊蔡夫人曰:“刘备遣三将居外,而自居荆州,久必为患。”蔡夫人乃夜对刘表曰:“我闻荆州人多与刘备往来,不可不防之。今容其居住城中,无益,不若遣使他往。”表曰:“玄德仁人也。”蔡氏曰:“只恐他人不似汝心。”表沉吟不答。次日出城,见玄德所乘之马极骏,问之,知是张武之马,表称赞不已。玄德遂将此马送与刘表。表大喜,骑回城中。蒯越见而问之。表曰:“此玄德所送也。”越曰:“昔先兄蒯良,最善相马;越亦颇晓。此马眼下有泪槽,额边生白点,名为的卢,骑则妨主。张武为此马而亡。主公不可乘之。”表听其言。次日请玄德饮宴,因言曰:“昨承惠良马,深感厚意。但贤弟不时征进,可以用之。敬当送还。”玄德起谢。表又曰:“贤弟久居此间,恐废武事。襄阳属邑新野县,颇有钱粮。弟可引本部军马于本县屯扎,何如?”玄德领诺。次日,谢别刘表,引本部军马径往新野。

方出城门,只见一人在马前长揖曰:“公所骑马,不可乘也。”玄德视之,乃荆州幕宾伊籍,字机伯,山阳人也。玄德忙下马问之。籍曰:“昨闻蒯异度对刘荆州云:此马名的卢,乘则妨主。因此还公。公岂可复乘之?”玄德曰:“深感先生见爱。但凡人死生有命,岂马所能妨哉!”籍服其高见,自此常与玄德往来。玄德自到新野,军民皆喜,政治一新。建安十二年春,甘夫人生刘禅。是夜有白鹤一只,飞来县衙屋上,高鸣四十余声,望西飞去。临分娩时,异香满室。甘夫人尝夜梦仰吞北斗,因而怀孕,故乳名阿斗。此时曹操正统兵北征。玄德乃往荆州,说刘表曰:“今曹操悉兵北征,许昌空虚,若以荆襄之众,乘间袭之,大事可就也。”表曰:“吾坐据九郡足矣,岂可别图?”玄德默然。表邀入后堂饮酒。酒至半酣,表忽然长叹。玄德曰:“兄长何故长叹?”表曰:“吾有心事,未易明言。”玄德再欲问时,蔡夫人出立屏后。刘表乃垂头不语。须臾席散,玄德自归新野。至是年冬,闻曹操自柳城回,玄德甚叹表之不用其言。忽一日,刘表遣使至,请玄德赴荆州相会。玄德随使而往。刘表接着,叙礼毕,请入后堂饮宴;因谓玄德曰:“近闻曹操提兵回许都,势日强盛,必有吞并荆襄之心。昔日悔不听贤弟之言,失此好机会。”玄德曰:“今天下分裂,干戈日起,机会岂有尽乎?若能应之于后,未足为恨也。”表曰:“吾弟之言甚当。”相与对饮。酒酣,表忽潸然泪下。玄德问其故。表曰:“吾有心事,前者欲诉与贤弟,未得其便。”玄德曰:“兄长有何难决之事?倘有用弟之处,弟虽死不辞。”表曰:“前妻陈氏所生长子琦,为人虽贤,而柔懦不足立事;后妻蔡氏所生少子琼,颇聪明。吾欲废长立幼,恐碍于礼法;欲立长子,争奈蔡氏族中,皆掌军务,后必生乱:因此委决不下。”玄德曰:“自古废长立幼,取乱之道。若忧蔡氏权重,可徐徐削之,不可溺爱而立少子也。”表默然。

原来蔡夫人素疑玄德,凡遇玄德与表叙论,必来窃听。是时正在屏风后,闻玄德此言,心甚恨之。玄德自知语失,遂起身如厕。因见己身髀肉复生,亦不觉潸然流涕。少顷复入席。表见玄德有泪容,怪问之。玄德长叹曰:“备往常身不离鞍,髀肉皆散;分久不骑,髀里肉生。日月磋跎,老将至矣,而功业不建:是以悲耳!”表曰:“吾闻贤弟在许昌,与曹操青梅煮酒,共论英雄;贤弟尽举当世名士,操皆不许,而独曰天下英雄,惟使君与操耳,以曹操之权力,犹不敢居吾弟之先,何虑功业不建乎?”玄德乘着酒兴,失口答曰:“备若有基本,天下碌碌之辈,诚不足虑也。”表闻言默然。玄德自知语失,托醉而起,归馆舍安歇。后人有诗赞玄德曰:“曹公屈指从头数:天下英雄独使君。髀肉复生犹感叹,争教寰字不三分?”

却说刘表闻玄德语,口虽不言,心怀不足,别了玄德,退入内宅。蔡夫人曰:“适间我于屏后听得刘备之言,甚轻觑人,足见其有吞并荆州之意。今若不除,必为后患。”表不答,但摇头而已。蔡氏乃密召蔡瑁入,商议能事。瑁曰:“请先就馆舍杀之,然后告知主公。”蔡氏然其言。瑁出,便连夜点军。

却说玄德在馆舍中秉烛而坐,三更以后,方欲就寝。忽一人叩门而入,视之乃伊籍也:原来伊籍探知蔡瑁欲害玄德,特夤夜来报。当下伊籍将蔡瑁之谋,报知玄德,催促玄德速速起身。玄德曰:“未辞景升,如何便去?”籍曰:“公若辞,必遭蔡瑁之害矣。”玄德乃谢别伊籍,急唤从者,一齐上马,不待天明,星夜奔回新野。比及蔡瑁领军到馆舍时,玄德已去远矣。瑁悔恨无及,乃写诗一首于壁间,径入见表曰:“刘备有反叛之意,题反诗于壁上,不辞而去矣。”表不信,亲诣馆舍观之,果有诗四句。诗曰:“数年徒守困,空对旧山川。龙岂池中物,乘雷欲上天!”刘表见诗大怒,拔剑言曰:“誓杀此无义之徒!”行数步,猛省曰:“吾与玄德相处许多时,不曾见他作诗。此必外人离间之计也。”遂回步入馆舍,用剑尖削去此诗,弃剑上马。蔡瑁请曰:“军士已点齐,可就往新野擒刘备。”表曰:“未可造次,容徐图之。”蔡瑁见表持疑不决,乃暗与蔡夫人商议:即日大会众官于襄阳,就彼处谋之。次日,瑁禀表曰:“近年丰熟,合聚众官于襄阳,以示抚劝之意。请主公一行。”表曰:“吾近日气疾作,实不能行。可令二子为主待客。”瑁曰:“公子年幼,恐失于礼节。”表曰:“可往新野请玄德待客。”瑁暗喜正中其计,便差人请玄德赴襄阳。

却说玄德奔回新野,自知失言取祸,未对众人言之。忽使者至,请赴襄阳。孙乾曰:“昨见主公匆匆而回,意甚不乐。愚意度之,在荆州必有事故。今忽请赴会,不可轻往。”玄德方将前项事诉与诸人。云长曰:“兄自疑心语失。刘荆州并无嗔责之意。外人之言,未可轻信。襄阳离此不远,若不去,则荆州反生疑矣。”玄德曰:“云长之言是也。”张飞曰:“筵无好筵,会无好会,不如休去。”赵云曰:“某将马步军三百人同往,可保主公无事。”玄德曰:“如此甚好。”

遂与赵云即日赴襄阳。蔡瑁出郭迎接,意甚谦谨。随后刘琦、刘琮二子,引一班文武官僚出迎。玄德见二公子俱在,并不疑忌。是日请玄德于馆舍暂歇。赵云引三百军围绕保护。云披甲挂剑,行坐不离左右。刘琦告玄德曰:“父亲气疾作。不能行动,特请叔父待客,抚劝各处守收之官。”玄德曰:“吾本不敢当此;既有兄命,不敢不从。”次日,人报九郡四十二州官员,俱已到齐。蔡瑁预请蒯越计议曰:“刘备世之枭雄,久留于此,后必为害,可就今日除之。”越曰:“恐失士民之望。”瑁曰:“吾已密领刘荆州言语在此。”越曰:“既如此,可预作准备。”瑁曰:“东门岘山大路,已使吾弟蔡和引军守把;南门外已使蔡中守把;北门外已使蔡勋守把。止有西门不必守把:前有檀溪阻隔,虽有数万之众,不易过也。”越曰:“吾见赵云行坐不离玄德,恐难下手。”瑁曰:“吾伏五百军在城内准备。”越曰:“可使文聘、王威二人另设一席于外厅,以待武将。先请住赵云,然后可行事。”瑁从其言。

当日杀牛宰马,大张筵席。玄德乘的卢马至州衙,命牵入后园拴系。众官皆至堂中。玄德主席,二公子两边分坐,其余各依次而坐。赵云带剑立于玄德之侧。文聘、王威入请赵云赴席。云推辞不去。玄德令云就席,云勉强应命而出。蔡瑁在外收拾得铁桶相似,将玄德带来三百军,都遣归馆舍,只待半酣,号起下手。酒至三巡,伊籍起把盏,至玄德前,以目视玄德,低声谓曰:“请更衣,”玄德会意,即起如厕,伊籍把盏毕,疾入后园,接着玄德,附耳报曰:“蔡瑁设计害君,城外东、南、北三处,皆有军马守把。惟西门可走,公宜速逃!”玄德大惊,急解的卢马,开后园门牵出,飞身上马,不顾从者,匹马望西门而走。门吏问之,玄德不答,加鞭而出。门吏当之不住,飞报蔡瑁。瑁即上马,引五百军随后追赶。

却说玄德撞出西门,行无数里,前有大溪,拦住去路,那檀溪阔数丈,水通襄江,其波甚紧。玄德到溪边,见不可渡,勒马再回,遥望城西尘头大起,追兵将至。玄德曰:“今番死矣!”遂回马到溪边。回头看时,追兵已近。玄德着慌,纵马下溪。行不数步,马前蹄忽陷,浸湿衣袍。玄德乃加鞭大呼曰:“的卢,的卢!今日妨吾!言毕,那马忽从水中涌身而起,一跃三丈,飞上西岸。玄德如从云雾中起。后来苏学士有古风一篇,单咏跃马檀溪事。诗曰:“老去花残春日暮,宦游偶至檀溪路;停骖遥望独徘徊,眼前零落飘红絮。暗想咸阳火德衰,龙争虎斗交相持;襄阳会上王孙饮,坐中玄德身将危。逃生独出西门道,背后追兵复将到。一川烟水涨檀溪,急叱征骑往前跳。马蹄蹄碎青玻璃,天风响处金鞭挥。耳畔但闻千骑走,波中忽见双龙飞。西川独霸真英主,坐下龙驹两相遇。檀溪溪水自东流,龙驹英主今何处!临流三叹心欲酸,斜阳寂寂照空山;三分鼎足浑如梦,踪迹空留在世间。”玄德跃过溪西,顾望东岸。蔡瑁已引军赶到溪边,大叫:“使君何故逃席而去?”玄德曰:“吾与汝无仇,何故欲相害?”瑚曰:“吾并无此心。使君休听人言。”玄德见瑁手将拈弓取箭,乃急拨马望西南而去。瑁谓左右曰:“是何神助也?”方欲收军回城,只见西门内赵云引三百军赶来。正是:跃去龙驹能救主,追来虎将欲诛仇。未知蔡瑁性命如何,且听下文分解。