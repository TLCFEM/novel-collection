\chapter{三江口曹操折兵~群英会蒋干中计}

却说周瑜闻诸葛瑾之言,转恨孔明,存心欲谋杀之。次日,点齐军将,入辞孙权。权曰:“卿先行,孤即起兵继后。”瑜辞出,与程普、鲁肃领兵起行,便邀孔明同住。孔明欣然从之。一同登舟,驾起帆樯,迤逦望夏口而进。离三江口五六十里,船依次第歇定。周瑜在中央下寨,岸上依西山结营,周围屯住。孔明只在一叶小舟内安身。

周瑜分拨已定,使人请孔明议事。孔明至中军帐,叙礼毕,瑜曰:“昔曹操兵少,袁绍兵多,而操反胜绍者,因用许攸之谋,先断乌巢之粮也。今操兵八十三万,我兵只五六万,安能拒之?亦必须先断操之粮,然后可破。我已探知操军粮草,俱屯于聚铁山。先生久居汉上,熟知地理。敢烦先生与关、张、子龙辈——吾亦助兵千人——星夜往聚铁山断操粮道。彼此各为主人之事,幸勿推调。”孔明暗思:“此因说我不动,设计害我。我若推调,必为所笑。不如应之,别有计议。”乃欣然领诺。瑜大喜。孔明辞出。鲁肃密谓瑜曰:“公使孔明劫粮,是何意见?”瑜曰:“吾欲杀孔明,恐惹人笑,故借曹操之手杀之,以绝后患耳。”肃闻言,乃往见孔明,看他知也不知。只见孔明略无难色,整点军马要行。肃不忍,以言挑之曰:“先生此去可成功否?”孔明笑曰:“吾水战、步战、马战、车战,各尽其妙,何愁功绩不成,非比江东公与周郎辈止一能也。”肃曰:“吾与公瑾何谓一能?”孔明曰:“吾闻江南小儿谣言云:‘伏路把关饶子敬,临江水战有周郎。’公等于陆地但能伏路把关;周公瑾但堪水战,不能陆战耳。”

肃乃以此言告知周瑜。瑜怒曰:“何欺我不能陆战耶!不用他去!我自引一万马军,往聚铁山断操粮道:”肃又将此言告孔明。孔明笑曰:“公瑾令吾断粮者,实欲使曹操杀吾耳。吾故以片言戏之,公瑾便容纳不下。目今用人之际,只愿吴侯与刘使君同心,则功可成;如各相谋害,大事休矣。操贼多谋,他平生惯断人粮道,今如何不以重兵提备?公瑾若去,必为所擒。今只当先决水战,挫动北军锐气,别寻妙计破之。望子敬善言以告公瑾为幸。”鲁肃遂连夜回见周瑜,备述孔明之言。瑜摇首顿足曰:“此人见识胜吾十倍,今不除之,后必为我国之祸!”肃曰:“今用人之际,望以国家为重。且待破曹之后,图之未晚。”瑜然其说。

却说玄德分付刘琦守江夏,自领众将引兵往夏口。遥望江南岸旗幡隐隐,戈戟重重,料是东吴已动兵矣,乃尽移江夏之兵,至樊口屯扎。玄德聚众曰:“孔明一去东吴,杳无音信,不知事体如何。谁人可去探听虚实回报?”糜竺曰:“竺愿往。”玄德乃备羊酒礼物,令糜竺至东吴,以犒军为名,探听虚实。竺领命,驾小舟顺流而下,径至周瑜大寨前。军士入报周瑜,瑜召入。竺再拜,致玄德相敬之意,献上酒礼。瑜受讫,设宴款待糜竺。竺曰:“孔明在此已久,今愿与同回。”瑜曰:“孔明方与我同谋破曹,岂可便去?吾亦欲见刘豫州,共议良策;奈身统大军,不可暂离。若豫州肯枉驾来临,深慰所望。”竺应诺,拜辞而回。肃问瑜曰:“公欲见玄德,有何计议?”瑜曰:“玄德世之枭雄,不可不除。吾今乘机诱至杀之,实为国家除一后患。”鲁肃再三劝谏,瑜只不听,遂传密令:“如玄德至,先埋伏刀斧手五十人于壁衣中,看吾掷杯为号,便出下手。”却说糜竺回见玄德,具言周瑜欲请主公到彼面会,别有商议。玄德便教收拾快船一只,只今便行。云长谏曰:“周瑜多谋之士,又无孔明书信,恐其中有诈,不可轻去。”玄德曰:“我今结东吴以共破曹操,周郎欲见我,我若不往,非同盟之意。两相猜忌,事不谐矣。”云长曰:“兄长若坚意要去,弟愿同往。”张飞曰:“我也跟去。”玄德曰:“只云长随我去。翼德与子龙守寨。简雍固守鄂县。我去便回。”分付毕,即与云长乘小舟,并从者二十余人,飞棹赴江东。玄德观看江东艨艟战舰、旌旗甲兵,左右分布整齐,心中甚喜。军士飞报周瑜:“刘豫州来了。”瑜问:“带多少船只来?”军士答曰:“只有一只船,二十余从人。”瑜笑曰:“此人命合体矣!”乃命刀斧手先埋伏定,然后出寨迎接。玄德引云长等二十余人,直到中军帐,叙礼毕,瑜请玄德上坐。玄德曰:“将军名传天下,备不才,何烦将军重礼?”乃分宾主而坐。周瑜设宴相待。

且说孔明偶来江边,闻说玄德来此与都督相会,吃了一惊,急入中军帐窃看动静。只见周瑜面有杀气,两边壁衣中密排刀斧手。孔明大惊曰:“似此如之奈何?”回视玄德,谈笑自若;却见玄德背后一人,按剑而立,乃云长也。孔明喜曰:“吾主无危矣。”遂不复入,仍回身至江边等候。

周瑜与玄德饮宴,酒行数巡,瑜起身把盏,猛见云长按剑立于玄德背后,忙问何人。玄德曰:“吾弟关云长也。”瑜惊曰:“非向日斩颜良、文丑者乎?”玄德曰:“然也。”瑜大惊,汗流满背,便斟酒与云长把盏。少顷,鲁肃入。玄德曰:“孔明何在?烦子敬请来一会。”瑜曰:“且待破了曹操,与孔明相会未迟。”玄德不敢再言。云长以目视玄德。玄德会意,即起身辞瑜曰:“备暂告别。即日破敌收功之后,专当叩贺。”瑜亦不留,送出辕门。玄德别了周瑜,与云长等来至江边,只见孔明已在舟中。玄德大喜。孔明曰:“主公知今日之危乎?”玄德愕然曰:“不知也。”孔明曰:“若无云长,主公几为周郎所害矣。”玄德方才省悟,便请孔明同回樊口。孔明曰:“亮虽居虎口,安如泰山。今主公但收拾船只军马候用。以十一月二十甲子日后为期,可令子龙驾小舟来南岸边等候。切勿有误。”玄德问其意。孔明曰:“但看东南风起,亮必还矣。”玄德再欲问时,孔明催促玄德作速开船。言讫自回。玄德与云长及从人开船,行不数里,忽见上流头放下五六十只船来。船头上一员大将,横矛而立,乃张飞也。因恐玄德有失,云长独力难支,特来接应。于是三人一同回寨,不在话下。

却说周瑜送了玄德,回至寨中,鲁肃入问曰:“公既诱玄德至此,为何又不下手?”瑜曰:“关云长,世之虎将也,与玄德行坐相随,吾若下手,他必来害我。”肃愕然。忽报曹操遣使送书至。瑜唤入。使者呈上书看时,封面上判云:“汉大丞相付周都督开拆。”瑜大怒,更不开看,将书扯碎,掷于地下,喝斩来使。肃曰:“两国相争,不斩来使。瑜曰:“斩使以示威!”遂斩使者,将首级付从人持回。随令甘宁为先锋,韩当为左翼,蒋钦为右翼。瑜自部领诸将接应。来日四更造饭,五更开船,鸣鼓呐喊而进。

却说曹操知周瑜毁书斩使,大怒,便唤蔡瑁、张允等一班荆州降将为前部,操自为后军,催督战船,到三江口。早见东吴船只,蔽江而来。为首一员大将,坐在船头上大呼曰:“吾乃甘宁也!谁敢来与我决战?”蔡瑁令弟蔡壎前进。两船将近,甘宁拈弓搭箭,望蔡壎射来,应弦而倒。宁驱船大进,万弩齐发。曹军不能抵当。右边蒋钦,左边韩当,直冲入曹军队中。曹军大半是青、徐之兵,素不习水战,大江面上,战船一摆,早立脚不住。甘宁等三路战船,纵横水面。周瑜又催船助战。曹军中箭着炮者,不计其数,从巳时直杀到未时。周瑜虽得利,只恐寡不敌众,遂下令鸣金,收住船只。

曹军败回。操登旱寨,再整军士,唤蔡瑁、张允责之曰:“东吴兵少,反为所败,是汝等不用心耳!”蔡瑁曰:“荆州水军,久不操练;青、徐之军,又素不习水战。故尔致败。今当先立水寨,令青、徐军在中,荆州军在外,每日教习精熟,方可用之。”操曰:“汝既为水军都督,可以便宜从事,何必禀我!”于是张、蔡二人,自去训练水军。沿江一带分二十四座水门,以大船居于外为城郭,小船居于内,可通往来,至晚点上灯火,照得天心水面通红。旱寨三百余里,烟火不绝。

却说周瑜得胜回寨,犒赏三军,一面差人到吴侯处报捷。当夜瑜登高观望,只见西边火光接天。左右告曰:“此皆北军灯火之光也。”瑜亦心惊。次日,瑜欲亲往探看曹军水寨,乃命收拾楼船一只,带着鼓东,随行健将数员,各带强弓硬弩,一齐上船迤逦前进。至操寨边,瑜命下了矴石,楼船上鼓乐齐奏。瑜暗窥他水寨,大惊曰:“此深得水军之妙也!”问:“水军都督是谁?”左右曰:“蔡瑁、涨允。”瑜思曰:“二人久居江东,谙习水战,吾必设计先除此二人,然后可以破曹。”正窥看间,早有曹军飞报曹操,说:“周瑜偷看吾寨。”操命纵船擒捉。瑜见水寨中旗号动,急教收起矴石,两边四下一齐轮转橹棹,望江面上如飞而去。比及曹寨中船出时,周瑜的楼船已离了十数里远,追之不及,回报曹操。

操问众将曰:“昨日输了一阵,挫动锐气;今又被他深窥吾寨。吾当作何计破之?”言未毕,忽帐下一人出曰:“某自幼与周郎同窗交契,愿凭三寸不烂之舌,往江东说此人来降。”曹操大喜,视之,乃九江人,姓蒋,名干,字子翼,现为帐下幕宾。操问曰:“子翼与周公瑾相厚乎?”干曰:“丞相放心。干到江左,必要成功。”操问:“要将何物去?”干曰:“只消一童随往,二仆驾舟,其余不用。”操甚喜,置酒与蒋干送行。

干葛巾布袍,驾一只小舟,径到周瑜寨中,命传报:“故人蒋干相访。”周瑜正在帐中议事,闻干至,笑谓诸将曰:“说客至矣!”遂与众将附耳低言,如此如此。众皆应命而去。瑜整衣冠,引从者数百,皆锦衣花帽,前后簇拥而出。蒋干引一青衣小童,昂然而来。瑜拜迎之。干曰:“公瑾别来无恙!”瑜曰:“子翼良苦:远涉江湖,为曹氏作说客耶?”干愕然曰:“吾久别足下,特来叙旧,奈何疑我作说客也?”瑜笑曰:“吾虽不及师旷之聪,闻弦歌而知雅意。”干曰:“足下待故人如此,便请告退。”瑜笑而挽其臂曰:“吾但恐兄为曹氏作说客耳。既无此心,何速去也?”遂同入帐。

叙礼毕,坐定,即传令悉召江左英杰与子翼相见。须臾,文官武将,各穿锦衣;帐下偏裨将校,都披银铠:分两行而入。瑜都教相见毕,就列于两傍而坐。大张筵席,奏军中得胜之乐,轮换行酒。瑜告众官曰:“此吾同窗契友也。虽从江北到此,却不是曹家说客。公等勿疑。”遂解佩剑付太史慈曰:“公可佩我剑作监酒:今日宴饮,但叙朋友交情;如有提起曹操与东吴军旅之事者,即斩之!”太史慈应诺,按剑坐于席上。蒋干惊愕,不敢多言。周瑜曰:“吾自领军以来,滴酒不饮;今日见了故人,又无疑忌,当饮一醉。”说罢,大笑畅饮。座上觥筹交错。饮至半醋,瑜携干手,同步出帐外。左右军士,皆全装惯带,持戈执戟而立。瑜曰:“吾之军士,颇雄壮否?”干曰:“真熊虎之士也,”瑜又引干到帐后一望,粮草堆如山积。瑜曰:“吾之粮草,颇足备否?”干曰:“兵精粮足,名不虚传。”瑜佯醉大笑曰:“想周瑜与子翼同学业时,不曾望有今日。”干曰:“以吾兄高才,实不为过。”瑜执干手曰:“大丈夫处世,遇知己之主,外托君臣之义,内结骨肉之恩,言必行,计必从,祸福共之。假使苏秦、张仪、陆贾、郦生复出,口似悬河,舌如利刃,安能动我心哉!”言罢大笑。蒋干面如土色。

瑜复携干入帐,会诸将再饮;因指诸将曰:“此皆江东之英杰。今日此会,可名群英会。”饮至天晚,点上灯烛,瑜自起舞剑作歌。歌曰:“丈夫处世兮立功名;立功名兮慰平生。慰平生兮吾将醉;吾将醉兮发狂吟!”歇罢,满座欢笑。

至夜深,干辞曰:“不胜酒力矣。”瑜命撤席,诸将辞出。瑜曰:“久不与子翼同榻,今宵抵足而眠。”于是佯作大醉之状,携干入帐共寝。瑜和衣卧倒,呕吐狼藉。蒋干如何睡得着?伏枕听时,军中鼓打二更,起视残灯尚明。看周瑜时,鼻息如雷。干见帐内桌上,堆着一卷文书,乃起床偷视之,却都是往来书信。内有一封,上写“蔡瑁张允谨封。”干大惊,暗读之。书略曰:“某等降曹,非图仕禄,迫于势耳。今已赚北军困于寨中,但得其便,即将操贼之首,献于麾下。早晚人到,便有关报。幸勿见疑。先此敬覆。”干思曰:“原来蔡瑁、张允结连东吴!”遂将书暗藏于衣内。再欲检看他书时,床上周瑜翻身,干急灭灯就寝。瑜口内含糊曰:“子翼,我数日之内,教你看操贼之首!”干勉强应之。瑜又曰:“子翼,且住!……教你看操贼之首!……”及干问之,瑜又睡着。干伏于床上,将近四更,只听得有人入帐唤曰:“都督醒否?”周瑜梦中做忽觉之状,故问那人曰:“床上睡着何人?”答曰:“都督请子翼同寝,何故忘却?”瑜懊悔曰:“吾平日未尝饮醉;昨日醉后失事,不知可曾说甚言语?”那人曰:“江北有人到此。”瑜喝:“低声!”便唤:“子翼。”蒋干只妆睡着。瑜潜出帐。干窃听之,只闻有人在外曰:“张、蔡二都督道:急切不得下手,……”后面言语颇低,听不真实。少顷,瑜入帐,又唤:“子翼。”蒋干只是不应,蒙头假睡。瑜亦解衣就寝。

干寻思:“周瑜是个精细人,天明寻书不见,必然害我。”睡至五更,干起唤周瑜;瑜却睡着。干戴上巾帻,潜步出帐,唤了小童,径出辕门。军士问:“先生那里去?”干曰:“吾在此恐误都督事,权且告别。”军士亦不阻当。干下船,飞棹回见曹操。操问:“子翼干事若何?”干曰:“周瑜雅量高致,非言词所能动也。”操怒曰:“事又不济,反为所笑!”干曰:“虽不能说周瑜,却与丞相打听得一件事。乞退左右。”

干取出书信,将上项事逐一说与曹操。操大怒曰:“二贼如此无礼耶!”即便唤蔡瑁、张允到帐下。操曰:“我欲使汝二人进兵。”瑁曰:“军尚未曾练熟,不可轻进。”操怒曰:“军若练熟,吾首级献于周郎矣!”蔡、张二人不知其意,惊慌不能回答。操喝武士推出斩之。须臾,献头帐下,操方省悟曰:“吾中计矣!”后人有诗叹曰:“曹操奸雄不可当,一时诡计中周郎。蔡张卖主求生计,谁料今朝剑下亡!”众将见杀了张、蔡二人,入问其故。操虽心知中计,却不肯认错,乃谓众将曰:“二人怠慢军法,吾故斩之。”众皆嗟呀不已。

操于众将内选毛玠、于禁为水军都督,以代蔡、张二人之职。细作探知,报过江东。周瑜大喜曰:“吾所患者,此二人耳。今既剿除,吾无忧矣。”肃曰:“都督用兵如此,何愁曹贼不破乎!”瑜曰:“吾料诸将不知此计,独有诸葛亮识见胜我,想此谋亦不能瞒也。子敬试以言挑之,看他知也不知,便当回报。”正是:还将反间成功事,去试从旁冷眼人。未知肃去问孔明还是如何,且看下文分解。