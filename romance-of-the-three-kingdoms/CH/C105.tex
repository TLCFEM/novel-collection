\chapter{武侯预伏锦囊计~魏主拆取承露盘}

却说杨仪闻报前路有兵拦截,忙令人哨探。回报说魏延烧绝栈道,引兵拦路。仪大惊
曰:“丞相在日,料此人久后必反,谁想今日果然如此!今断吾归路,当复如何?”费祎
曰:“此人必先捏奏天子,诬吾等造反,故烧绝栈道,阻遏归路。吾等亦当表奏天子,陈魏
延反情,然后图之。”姜维曰:“此间有一小径,名槎山,虽崎岖险峻,可以抄出栈道之
后。”一面写表奏闻天子,一面将人马望槎山小道进发。

且说后主在成都,寝食不安,动止不宁;夜作一梦,梦见成都锦屏山崩倒;遂惊觉,坐
而待旦,聚集文武,入朝圆梦。谯周曰:“臣昨夜仰观天文,见一星,赤色,光芒有角,自
东北落于西南,主丞相有大凶之事。今陛下梦山崩,正应此兆。”后主愈加惊怖。忽报李福
到,后主急召入问之。福顿首泣奏丞相已亡;将丞相临终言语,细述一遍。后主闻言大哭
曰:“天丧我也!”哭倒于龙床之上。侍臣扶入后宫。吴太后闻之,亦放声大哭不已。多官
无不哀恸,百姓人人涕泣。后主连日伤感,不能设朝。忽报魏延表奏杨仪造反,群臣大骇,
入宫启奏后主,时吴太后亦在宫中。后主闻奏大惊,命近臣读魏延表。其略曰:“征西大将
军、南郑侯臣魏延,诚惶诚恐,顿首上言:杨仪自总兵权,率众造反,劫丞相灵柩,欲引敌
人入境。臣先烧绝栈道,以兵守御。谨此奏闻。”读毕,后主曰:“魏延乃勇将,足可拒杨
仪等众,何故烧绝栈道?”吴太后曰:“尝闻先帝有言:孔明识魏延脑后有反骨,每欲斩
之;因怜其勇,故姑留用。今彼奏杨仪等造反,未可轻信。杨仪乃文人,丞相委以长史之
任,必其人可用。今日若听此一面之词,杨仪等必投魏矣。此事当深虑远议,不可造次。”
众官正商议间,忽报:长史杨仪有紧急表到。近臣拆表读曰:“长史、绥军将军臣杨仪,诚
惶诚恐,顿首谨表:丞相临终,将大事委于臣,照依旧制,不敢变更,使魏延断后,姜维次
之。今魏延不遵丞相遗语,自提本部人马,先入汉中,放火烧断栈道,劫丞相灵车,谋为不
轨。变起仓卒,谨飞章奏闻。”太后听毕,问:“卿等所见若何?”蒋琬奏曰:“以臣愚
见:杨仪为人虽禀性过急,不能容物,至于筹度粮草,参赞军机,与丞相办事多时,今丞相
临终,委以大事,决非背反之人。魏延平日恃功务高,人皆下之;仪独不假借,延心怀恨;
今见仪总兵,心中不服,故烧栈道,断其归路,又诬奏而图陷害。臣愿将全家良贱,保杨仪
不反。实不敢保魏延。”董允亦奏曰:“魏延自恃功高,常有不平之心,口出怨言。向所以
不即反者,惧丞相耳。今丞相新亡,乘机为乱,势所必然。若杨仪,才干敏达,为丞相所任
用,必不背反。”后主曰:“若魏延果反,当用何策御之?”蒋琬曰:“丞相素疑此人,必
有遗计授与杨仪。若仪无恃,安能退入谷口乎?延必中计矣。陛下宽心。”不多时,魏延又
表至,告称杨仪背反。正览表之间,杨仪又表到,奏称魏延背反。二人接连具表,各陈是
非。忽报费祎到。后主召入,祎细奏魏延反情。后主曰:“若如此,且令董允假节释劝,用好
言抚慰。”允奉诏而去。

却说魏延烧断栈道,屯兵南谷,把住隘口,自以为得计;不想杨仪、姜维星夜引兵抄到
南谷之后。仪恐汉中有失,令先锋何平引三千兵先行。仪同姜维等引兵扶柩望汉中而来。

且说何平引兵径到南谷之后,擂鼓呐喊。哨马飞报魏延,说杨仪令先锋何平引兵自槎山
小路抄来搦战。延大怒,急披挂上马,提刀引兵来迎。两阵对圆,何平出马大骂曰:“反贼
魏延安在?”延亦骂曰:“汝助杨仪造反,何敢骂我!”平叱曰:“丞相新亡,骨肉未寒,
汝焉敢造反!”乃扬鞭指川兵曰:“汝等军士,皆是西川之人,川中多有父母妻子,兄弟亲
朋;丞相在日,不曾薄待汝等,今不可助反贼,宜各回家乡,听候赏赐。”众军闻言,大喊
一声,散去大半。延大怒,挥刀纵马,直取何平。平挺枪来迎。战不数合,平诈败而走,延
随后赶来。众军弓弩齐发,延拨马而回。见众军纷纷溃散,延转怒,拍马赶上,杀了数人,
却只止遏不住;只有马岱所领三百人不动,延谓岱曰:“公真心助我,事成之后,决不相
负。”遂与马岱追杀何平。平引兵飞奔而去。魏延收聚残军,与马岱商议曰:“我等投魏,
若何?”岱曰:“将军之言,不智甚也。大丈夫何不自图霸业,乃轻屈膝于人耶?吾观将军
智勇足备,两川之士,谁敢抵敌?吾誓同将军先取汉中,随后进攻西川。”

延大喜,遂同马岱引兵直取南郑。姜维在南郑城上,见魏延、马岱耀武扬威,风拥而
来。维急令拽起吊桥。延、岱二人大叫:“早降!”姜维令人请杨仪商议曰:“魏延勇猛,
更兼马岱相助,虽然军少,何计退之?”仪曰:“丞相临终,遗一锦囊,嘱曰:若魏延造
反,临阵对敌之时,方可开拆,便有斩魏延之计。今当取出一看。”遂出锦囊拆封看时,题
曰:“待与魏延对敌,马上方许拆开。”维大喜曰:“既丞相有戒约,长史可收执。吾先引
兵出城,列为阵势,公可便来。”姜维披挂上马,绰枪在手,引三千军,开了城门,一齐冲
出,鼓声大震,排成阵势。维挺枪立马于门旗之下,高声大骂曰:“反贼魏延!丞相不曾亏
你,今日如何背反?”延横刀勒马而言曰:“伯约,不干你事。只教杨仪来!”仪在门旗影
里,拆开锦囊视之,如此如此。仪大喜,轻骑而出,立马阵前,手指魏延而笑曰:“丞相在
日,知汝久后必反,教我提备,今果应其言。汝敢在马上连叫三声谁敢杀我,便是真大丈
夫,吾就献汉中城池与汝。”延大笑曰:“杨仪匹夫听着!若孔明在日,吾尚惧他三分;他
今已亡,天下谁敢敌我?休道连叫三声,便叫三万声,亦有何难!”遂提刀按辔,于马上大
叫曰:“谁敢杀我?”一声未毕,脑后一人厉声而应曰:“吾敢杀汝!”手起刀落,斩魏延
于马下。众皆骇然。斩魏延者,乃马岱也。原来孔明临终之时,授马岱以密计,只待魏延喊
叫时,便出其不意斩之;当日,杨仪读罢锦囊计策,已知伏下马岱在彼,故依计而行,果然
杀了魏延。后人有诗曰:“诸葛先机识魏延,已知日后反西川。锦囊遗计人难料,却见成功
在马前。”

却说董允未及到南郑,马岱已斩了魏延,与姜维合兵一处。杨仪具表星夜奏闻后主。后
主降旨曰:“既已名正其罪,仍念前功,赐棺椁葬之。”杨仪等扶孔明灵柩到成都,后主引
文武官僚,尽皆挂孝,出城二十里迎接。后主放声大哭。上至公卿大夫,下及山林百姓,男
女老幼,无不痛哭,哀声震地。后主命扶柩入城,停于丞相府中。其子诸葛瞻守孝居丧。

后主还朝,杨仪自缚请罪。后主令近臣去其缚曰:“若非卿能依丞相遗教,灵柩何日得
归,魏延如何得灭。大事保全,皆卿之力也。”遂加杨仪为中军师。马岱有讨逆之功,即以
魏延之爵爵之。仪呈上孔明遗表。后主览毕,大哭,降旨卜地安葬。费祎奏曰:“丞相临
终,命葬于定军山,不用墙垣砖石,亦不用一切祭物。”后主从之。择本年十月吉日,后主
自送灵柩至定军山安葬。后主降诏致祭,谥号忠武侯;令建庙于沔阳,四时享祭。后杜工部
有诗曰:“丞相祠堂何处寻,锦官城外柏森森。映阶碧草自春色,隔叶黄鹏空好音。三顾频
烦天下计,两朝开济老臣心。出师未捷身先死,长使英雄泪满襟!”又杜工部诗曰:“诸葛
大名垂宇宙,宗臣遗像肃清高。三分割据纡筹策,万古云霄一羽毛。伯仲之间见伊吕,指挥
若定失萧曹。运移汉祚终难复,志决身歼军务劳。”

却说后主回到成都,忽近臣奏曰:“边庭报来,东吴令全琮引兵数万,屯于巴丘界口,
未知何意。”后主惊曰:“丞相新亡,东吴负盟侵界,如之奈何?”蒋琬奏曰:“臣敢保王
平、张嶷引兵数万屯于永安,以防不测。陛下再命一人去东吴报丧,以探其动静。”后主
曰:“须得一舌辩之士为使。”一人应声而出曰:“微臣愿往。”众视之,乃南阳安众人,
姓宗,名预,字德艳,官任参军、右中郎将。后主大喜,即命宗预往东吴报丧,兼探虚实。
宗预领命,径到金陵,入见吴主孙权。礼毕,只见左右人皆着素衣。权作色而言曰:“吴、
蜀已为一家,卿主何故而增白帝之守也?”预曰:“臣以为东益巴丘之戍,西增白帝之守,
皆事势宜然,俱不足以相问也。”权笑曰:“卿不亚于邓芝。”乃谓宗预曰:“朕闻诸葛丞
相归天,每日流涕,令官僚尽皆挂孝。朕恐魏人乘丧取蜀,故增巴丘守兵万人,以为救援,
别无他意也。”预顿首拜谢。权曰:“朕既许以同盟,安有背义之理?”预曰:“天子因丞
相新亡,特命臣来报丧。”权遂取金鈚箭一枝折之,设誓曰:“朕若负前盟,子孙绝灭!”
又命使赍香帛奠仪,入川致祭。

宗预拜辞吴主,同吴使还成都,入见后主,奏曰:“吴主因丞相新亡,亦自流涕,令群
臣皆挂孝。其益兵巴丘者,恐魏人乘虚而入,别无异心。今折箭为誓,并不背盟。”后主大
喜,重赏宗预,厚待吴使去讫。遂依孔明遗言,加蒋琬为丞相、大将军,录尚书事;加费祎
为尚书令,同理丞相事;加吴懿为车骑将军,假节督汉中;姜维为辅汉将军、平襄侯,总督
诸处人马,同吴懿出屯汉中,以防魏兵。其余将校,各依旧职。杨仪自以为年宦先于蒋琬,
而位出琬下;且自恃功高,未有重赏,口出怨言,谓费祎曰:“昔日丞相初亡,吾若将全师
投魏,宁当寂寞如此耶!”费祎乃将此言具表密奏后主。后主大怒,命将杨仪下狱勘问,欲
斩之。蒋琬奏曰:“仪虽有罪,但日前随丞相多立功劳,未可斩也,当废为庶人。”后主从
之,遂贬杨仪赴汉嘉郡为民。仪羞惭自刎而死。

蜀汉建兴十三年,魏主曹睿青龙三年,吴主孙权嘉禾四年,三国各不兴兵,单说魏主封
司马懿为太尉,总督军马,安镇诸边。懿拜谢回洛阳去讫。魏主在许昌大兴土木,建盖宫
殿;又于洛阳造朝阳殿、太极殿,筑总章观,俱高十丈;又立崇华殿、青霄阁、凤凰楼、九
龙池,命博士马钧监造,极其华丽:雕梁画栋,碧瓦金砖,光辉耀日。选天下巧匠三万余
人,民夫三十余万,不分昼夜而造。民力疲困,怨声不绝。

睿又降旨起土木于芳林园,使公卿皆负土树木于其中。司徒董寻上表切谏曰。“伏自建
安以来,野战死亡,或门殚户尽;虽有存者,遗孤老弱。若今宫室狭小,欲广大之,犹宜随
时,不妨农务。况作无益之物乎?陛下既尊群臣,显以冠冕,被以文绣,载以华舆,所以异
于小人也。今又使负木担土,沾体涂足,毁国之光,以崇无益:甚无谓也。孔子云:君使臣
以礼,臣事君以忠。无忠无礼,国何以立?臣知言出必死;而自比于牛之一毛,生既无益,
死亦何损。秉笔流涕,心与世辞。臣有八子,臣死之后,累陛下矣。不胜战忄栗待命之
至!”睿览表怒曰:“董寻不怕死耶!”左右奏请斩之。睿曰:“此人素有忠义,今且废为
庶人。再有妄言者必斩!”时有太子舍人张茂,字彦材,亦上表切谏,睿命斩之。即日召马
钧问曰:“朕建高台峻阁,欲与神仙往来,以求长生不老之方。”钧奏曰:“汉朝二十四
帝,惟武帝享国最久,寿算极高,盖因服天上日精月华之气也:尝于长安宫中,建柏梁台;
台上立一铜人,手捧一盘,名曰承露盘,接三更北斗所降沆瀣之水,其名曰天浆,又曰甘
露。取此水用美玉为屑,调和服之,可以反老还童。”睿大喜曰:“汝今可引人夫星夜至长
安,拆取铜人,移置芳林园中”钧领命,引一万人至长安,令周围搭起木架,上柏梁台去。
不移时间,五千人连绳引索,旋环而上。那柏梁台高二十丈,铜柱圆十围。马钧教先拆铜
人。多人并力拆下铜人来,只见铜人眼中潸然泪下。众皆大惊。忽然台边一阵狂风起处,飞
砂走石,急若骤雨;一声响亮,就如天崩地裂:台倾柱倒,压死千余人。钧取铜人及金盘回
洛阳,入见魏主,献上铜人、承露盘。魏主问曰:“铜柱安在?”钧奏曰:“柱重百万斤,
不能运至。”睿令将铜柱打碎,运来洛阳,铸成两个铜人,号为翁仲,列于司马门外;又铸
铜龙凤两个:龙高四丈,凤高三丈余,立在殿前。又于上林苑中,种奇花异木,蓄养珍禽怪
兽。少傅杨阜上表谏曰:“臣闻尧尚茅茨,而万国安居;禹卑宫室,而天下乐业;及至殷、
周,或堂崇三尺,度以九筵耳。古之圣帝明王,未有极宫室之高丽,以凋敝百姓之财力者
也。桀作璇室、象廊,纣为倾宫、鹿台,以丧其社稷;楚灵以筑章华而身受其祸;秦始皇作
阿房而殃及其子,天下叛之,二世而灭。夫不度万民之力,以从耳目之欲,未有不亡者也。
陛下当以尧、舜、禹、汤、文、武为法则,以桀、纣、楚、秦为深诫。而乃自暇自逸,惟宫
台是饰,必有危亡之祸矣。君作元首,臣为股肱,存亡一体,得失同之。臣虽驽怯,敢忘诤
臣之义?言不切至,不足以感寤陛下。谨叩棺沐浴,伏俟重诛。”表上,睿不省,只催督马
钧建造高台,安置铜人、承露盘。又降旨广选天下美女,入芳林园中。众官纷纷上表谏诤,
睿俱不听。

却说曹睿之后毛氏,乃河内人也;先年睿为平原王时,最相恩爱;及即帝位,立为后;
后睿因宠郭夫人,毛后失宠。郭夫人美而慧,睿甚嬖之,每日取乐,月余不出宫闼。是岁春
三月,芳林园中百花争放,睿同郭夫人到园中赏玩饮酒。郭夫人曰:“何不请皇后同乐?”
壑曰;“若彼在,腾涓滴不能下咽也。”遂传谕宫娥,不许令毛后知道。毛后见睿月余不入
正宫,是日引十余宫人,来翠花楼上消遣,只听的乐声嘹亮,乃问曰:“何处奏乐?”一宫
官启曰:“乃圣上与郭夫人于御花园中赏花饮酒。”毛后闻之,心中烦恼,回宫安歇。次
日,毛皇后乘小车出宫游玩,正迎见睿于曲廊之间,乃笑曰:“陛下昨游北园,其乐不浅
也!”睿大怒,即命擒昨日侍奉诸人到,叱曰:“昨游北园,朕禁左右不许使毛后知道,何
得又宣露!”喝令宫官将诸侍奉人尽斩之。毛后大惊,回车至宫,睿即降诏赐毛皇后死,立
郭夫人为皇后。朝臣莫敢谏者。

忽一日,幽州刺史毋丘俭上表,报称辽东公孙渊造反,自号为燕王,改元绍汉元年,建
宫殿,立官职,兴兵入寇,摇动北方。睿大惊,即聚文武官僚,商议起兵退渊之策。正是:
才将土木劳中国,又见干戈起外方。未知何以御之,且看下文分解。