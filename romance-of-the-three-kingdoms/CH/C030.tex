\chapter{战官渡本初败绩~劫乌巢孟德烧粮}

却说袁绍兴兵,望官渡进发。夏侯惇发书告急。曹操起军七万,前往迎敌,留荀彧守许
都。绍兵临发,田丰从狱中上书谏曰:“今且宜静守以待天时,不可妄兴大兵,恐有不
利。”逢纪谮曰:“主公兴仁义之师,田丰何得出此不祥之语!”绍因怒,欲斩田丰。众官
告免。绍恨曰:“待吾破了曹操,明正其罪!”遂催军进发,旌旗遍野,刀剑如林。行至阳
武,下定寨栅。沮授曰:“我军虽众,而勇猛不及彼军;彼军虽精,而粮草不如我军。彼军
无粮,利在急战;我军有粮,宜且缓守。若能旷以日月,则彼军不战自败矣。”绍怒曰:
“田丰慢我军心,吾回日必斩之。汝安敢又如此!”叱左右:“将沮授锁禁军中,待我破曹
之后,与田丰一体治罪!”于是下令,将大军七十万,东西南北,周围安营,连络九十余
里。

细作探知虚实,报至官渡。曹军新到,闻之皆惧。曹操与众谋士商议。荀攸曰:“绍军
虽多,不足惧也。我军俱精锐之士,无不一以当十。但利在急战。若迁延日月,粮草不敷,
事可忧矣。”操曰:“所言正合吾意。”遂传令军将鼓噪而进。绍军来迎,两边排成阵势。
审配拨弩手一万,伏于两翼;弓箭手五千,伏于门旗内:约炮响齐发。三通鼓罢,袁绍金盔
金甲,锦袍玉带,立马阵前。左右排列着张郃、高览、韩猛、淳于琼等诸将。旌旗节钺,甚
是严整。曹阵上门旗开处,曹操出马。许诸、张辽、徐晃、李典等,各持兵器,前后拥卫。
曹操以鞭指袁绍曰:“吾于天子之前,保奏你为大将军,今何故谋反?”绍怒曰:“汝托名
汉相,实为汉贼!罪恶弥天,甚于莽、卓,乃反诬人造反耶!”操曰:“吾今奉诏讨汝!”
绍曰:“吾奉衣带诏讨贼!”操怒,使张辽出战。张邰跃马来迎。二将斗了四五十合,不分
胜负。曹操见了,暗暗称奇。许褚挥刀纵马,直出助战。高览挺枪接住。四员将捉对儿厮
杀。曹操令夏侯惇、曹洪,各引三千军,齐冲彼阵。审配见曹军来冲阵,便令放起号炮:两
下万弩并发,中军内弓箭手一齐拥出阵前乱射。曹军如何抵敌,望南急走。袁绍驱兵掩杀,
曹军大败,尽退至官渡。袁绍移军逼近官渡下寨。审配曰:“今可拨兵十万守官渡,就曹操
寨前筑起土山,令军人下视寨中放箭。操若弃此而去,吾得此隘口,许昌可破矣。”绍从
之,于各寨内选精壮军人,用铁锹土担,齐来曹操寨边,垒土成山。曹营内见袁军堆筑土
山,欲待出去冲突,被审配弓弩手当住咽喉要路,不能前进。十日之内,筑成土山五十余
座,上立高橹,分拨弓弩手于其上射箭。曹军大惧,皆顶着遮箭牌守御。土山上一声梆子响
处,箭下如雨。曹军皆蒙楯伏地,袁军呐喊而笑。

曹操见军慌乱,集众谋士问计。刘晔进曰:“可作发石车以破之。”操令晔进车式,连
夜造发石车数百乘,分布营墙内,正对着土山上云梯。候弓箭手射箭时,营内一齐拽动石
车,炮石飞空,往上乱打。人无躲处,弓箭手死者无数。袁军皆号其车为“霹雳车”。由是
袁军不敢登高射箭。审配又献一计:令军人用铁锹暗打地道,直透曹营内,号为“掘子
军”。曹兵望见袁军于山后掘土坑,报知曹操。操又问计于刘晔。晔曰:“此袁军不能攻明
而攻暗,发掘伏道,欲从地下透营而入耳。”操曰:“何以御之?”晔曰:“可绕营掘长
堑,则彼伏道无用也。”操连夜差军掘堑。袁军掘伏道到堑边,果不能入,空费军力。

却说曹操守官渡,自八月起,至九月终,军力渐乏,粮草不继。意欲弃官渡退回许昌,
迟疑未决,乃作书遣人赴许昌问荀彧。彧以书报之。书略曰:“承尊命,使决进退之疑。愚
以袁绍悉众聚于官渡,欲与明公决胜负,公以至弱当至强,若不能制,必为所乘:是天下之
大机也。绍军虽众,而不能用;以公之神武明哲,何向而不济!今军实虽少,未若楚、汉在
荥阳、成皋间也。公今画地而守,扼其喉而使不能进,情见势竭,必将有变。此用奇之时,
断不可失。惟明公裁察焉。”曹操得书大喜,令将士效力死守。

绍军约退三十余里,操遣将出营巡哨。有徐晃部将史涣获得袁军细作,解见徐晃。晃问
其军中虚实。答曰:“早晚大将韩猛运粮至军前接济,先令我等探路。”徐晃便将此事报知
曹操。荀攸曰:“韩猛匹夫之勇耳。若遣一人引轻骑数千,从半路击之,断其粮草,绍军自
乱。”操曰:“谁人可往?”攸曰:“即遣徐晃可也。”操遂差徐晃将带史涣并所部兵先
出,后使张辽、许褚引兵救应。当夜韩猛押粮车数千辆,解赴绍寨。正走之间,山谷内徐
晃、史涣引军截住去路。韩猛飞马来战,徐晃接住厮杀。史涣便杀散人夫,放火焚烧粮车。
韩猛抵当不住,拨回马走。徐晃催军烧尽辎重。袁绍军中,望见西北上火起,正惊疑间,败
军投来:“粮草被劫!”绍急遣张邰、高览去截大路,正遇徐晃烧粮而回,恰欲交锋,背后
张辽、许诸军到。两下夹攻,杀散袁军,四将合兵一处,回官渡寨中。曹操大喜,重加赏
劳。又分军于寨前结营,为掎角之势。

却说韩猛败军还营,绍大怒,欲斩韩猛,众官劝免。审配曰:“行军以粮食为重,不可
不用心提防。乌巢乃屯粮之处,必得重兵守之。”袁绍曰:“吾筹策已定。汝可回邺都监督
粮草,休教缺乏。”审配领命而去。袁绍遣大将淳于琼,部领督将眭元进、韩莒子、吕威
璜、赵睿等,引二万人马,守乌巢。那淳于琼性刚好酒,军士多畏之;既至乌巢,终日与诸
将聚饮。且说曹操军粮告竭,急发使往许昌教荀彧作速措办粮草,星夜解赴军前接济。使者
赍书而往,行不上三十里,被袁军捉住,缚见谋士许攸。那许攸字子远,少时曾与曹操为
友,此时却在袁绍处为谋士。当下搜得使者所赍曹操催粮书信,径来见绍曰:“曹操屯军官
渡,与我相持已久,许昌必空虚;若分一军星夜掩袭许昌,则许昌可拔,而操可擒也。今操
粮草已尽,正可乘此机会,两路击之。”绍曰:“曹操诡计极多,此书乃诱敌之计也。”攸
曰:“今若不取,后将反受其害。”正话间,忽有使者自邺郡来,呈上审配书。书中先说运
粮事;后言许攸在冀州时,尝滥受民间财物,且纵令子侄辈多科税,钱粮入己,今已收其子
侄下狱矣。绍见书大怒曰:“滥行匹夫!尚有面目于吾前献计耶!汝与曹操有旧,想今亦受
他财贿,为他作奸细,啜赚吾军耳!本当斩首,今权且寄头在项!可速退出,今后不许相
见!”许攸出,仰天叹曰:“忠言逆耳,竖子不足与谋!吾子侄已遭审配之害,吾何颜复见
冀州之人乎!”遂欲拔剑自刎。左右夺剑劝曰:“公何轻生至此?袁绍不绝直言,后必为曹
操所擒。公既与曹公有旧,何不弃暗投明?”只这两句言语,点醒许攸;于是许攸径投曹
操。后人有诗叹曰:“本初豪气盖中华,官渡相持枉叹嗟。若使许攸谋见用,山河争得属曹
家?”

却说许攸暗步出营,径投曹寨,伏路军人拿住。攸曰:“我是曹丞相故友,快与我通
报,说南阳许攸来见。”军士忙报入寨中。时操方解衣歇息,闻说许攸私奔到寨,大喜,不
及穿履,跣足出迎,遥见许攸,抚掌欢笑,携手共入,操先拜于地。攸慌扶起曰:“公乃汉
相,吾乃布衣,何谦恭如此?”操曰:“公乃操故友,岂敢以名爵相上下乎!”攸曰:“某
不能择主,屈身袁绍,言不听,计不从,今特弃之来见故人。愿赐收录。”操曰:“子远肯
来,吾事济矣!愿即教我以破绍之计:”攸曰:“吾曾教袁绍以轻骑乘虚袭许都,首尾相
攻。”操大惊曰:“若袁绍用子言,吾事败矣。”攸曰:“公今军粮尚有几何?”操曰:
“可支一年。”攸笑曰:“恐未必。”操曰:有半年耳。”攸拂袖而起,趋步出帐曰:“吾
以诚相投,而公见欺如是,岂吾所望哉!”操挽留曰:“子远勿嗔,尚容实诉:军中粮实可
支三月耳。”攸笑曰:“世人皆言孟德奸雄,今果然也。”操亦笑曰:“岂不闻兵不厌
诈!”遂附耳低言曰:“军中止有此月之粮。”攸大声曰:“休瞒我!粮已尽矣!”操愕然
曰:“何以知之?”攸乃出操与荀彧之书以示之曰:“此书何人所写?”操惊问曰:“何处
得之?”攸以获使之事相告。操执其手曰:“子远既念旧交而来,愿即有以教我。”攸曰:
“明公以孤军抗大敌,而不求急胜之方,此取死之道也。攸有一策,不过三日,使袁绍百万
之众,不战自破。明公还肯听否?”操喜曰:“愿闻良策。”攸曰:“袁绍军粮辎重,尽积
乌巢,今拨淳于琼守把,琼嗜酒无备。公可选精兵诈称袁将蒋奇领兵到彼护粮,乘间烧其粮
草辎重,则绍军不三日将自乱矣。”操大喜,重待许攸,留于塞中。次日,操自选马步军士
五千,准备往乌巢劫粮。张辽曰:“袁绍屯粮之所,安得无备?丞相未可轻往,恐许攸有
诈。”操曰:“不然,许攸此来,天败袁绍。今吾军粮不给,难以久持;若不用许攸之计,
是坐而待困也。彼若有诈,安肯留我寨中?且吾亦欲劫寨久矣。今劫粮之举,计在必行,君
请勿疑。”辽曰:“亦须防袁绍乘虚来袭。”操笑曰:“吾已筹之熟矣。”便教荀攸、贾
诩、曹洪同许攸守大寨,夏侯惇、夏侯渊领一军伏于左,曹仁、李典领一军伏于右,以备不
虞。教张辽、许褚在前,徐晃、于禁在后,操自引诸将居中:共五千人马,打着袁军旗号,
军士皆束草负薪,人衔枚,马勒口,黄昏时分,望乌巢进发。是夜星光满天。且说沮授被袁
绍拘禁在军中,是夜因见众星朗列,乃命监者引出中庭,仰观天象。忽见太白逆行,侵犯
牛、斗之分,大惊曰:“祸将至矣!”遂连夜求见袁绍。时绍已醉卧,听说沮授有密事启
报,唤入问之。授曰:“适观天象,见太白逆行于柳、鬼之间,流光射入牛、斗之分,恐有
贼兵劫掠之害。乌巢屯粮之所,不可不提备。宜速遣精兵猛将,于间道山路巡哨,免为曹操
所算。”绍怒叱曰:“汝乃得罪之人,何敢妄言惑众!”因叱监者曰:“吾令汝拘囚之,何
敢放出!”遂命斩监者,别唤人监押沮授。授出,掩泪叹曰:“我军亡在旦夕,我尸骸不知
落何处也!”后人有诗叹曰:“逆耳忠言反见仇,独夫袁绍少机谋。乌巢粮尽根基拔,犹欲
区区守冀州。”却说曹操领兵夜行,前过袁绍别寨,寨兵问是何处军马。操使人应曰:“蒋
奇奉命往乌巢护粮。”袁军见是自家旗号,遂不疑惑。凡过数处,皆诈称蒋奇之兵,并无阻
碍。及到乌巢,四更已尽。操教军士将束草周围举火,众将校鼓噪直入。时淳于琼方与众将
饮了酒,醉卧帐中;闻鼓噪之声,连忙跳起问:“何故喧闹?”言未已,早被挠钩拖翻。眭
元进、赵睿运粮方回,见屯上火起,急来救应。曹军飞报曹操,说:“贼兵在后,请分军拒
之。”操大喝曰:“诸将只顾奋力向前,待贼至背后,方可回战!”于是众军将无不争先掩
杀。一霎时,火焰四起,烟迷太空。眭、赵二将驱兵来救,操勒马回战。二将抵敌不住,皆
被曹军所杀,粮草尽行烧绝。淳于琼被擒见操,操命割去其耳鼻手指,缚于马上,放回绍营
以辱之。

却说袁绍在帐中,闻报正北上火光满天,知是乌巢有失,急出帐召文武各官,商议遣兵
往救。张郃曰:“某与高览同往救之。”郭图曰:“不可。曹军劫粮,曹操必然亲往;操既
自出,寨必空虚,可纵兵先击曹操之寨;操闻之,必速还:此孙膑围魏救赵之计也。”张邰
曰:“非也。曹操多谋,外出必为内备,以防不虞。今若攻操营而不拔,琼等见获,吾属皆
被擒矣。”郭图曰:“曹操只顾劫粮,岂留兵在寨耶!”再三请劫曹营。绍乃遣张郃、高览
引军五千,往官渡击曹营;遣蒋奇领兵一万,往救乌巢。且说曹操杀散淳于琼部率,尽夺其
衣甲旗帜,伪作淳于琼部下收军回寨,至山僻小路,正遇蒋奇军马。奇军问之,称是乌巢败
军奔回,奇遂不疑,驱马径过。张辽、许褚忽至,大喝:“蒋奇休走!”奇措手不及,被张
辽斩于马下,尽杀蒋奇之兵。又使人当先伪报云:“蒋奇已自杀散乌巢兵了”。袁绍因不复
遣人接应乌巢,只添兵往官渡。

却说张郃、高览攻打曹营,左边夏侯惇、右边曹仁,中路曹洪,一齐冲出:三下攻击,
袁军大败。比及接应军到,曹操又从背后杀来,四下围住掩杀。张邰、高览夺路走脱。袁绍
收得乌巢败残军马归寨,见淳于琼耳鼻皆无,手足尽落。绍问:“如何失了乌巢?”败军告
说:“淳于琼醉卧,因此不能抵敌。”绍怒,立斩之。郭图恐张邰、高览回寨证对是非,先
于袁绍前谮曰:“张邰、高览见主公兵败,心中必喜。”绍曰:“何出此言?”图曰:“二
人素有降曹之意,今遣击寨,故意不肯用力,以致损折士卒。”绍大怒,遂遣使急召二人归
寨问罪。郭图先使人报二人云:“主公将杀汝矣。”及绍使至,高览问曰:“主公唤我等为
何?”使者曰:“不知何故。”览遂拔剑斩来使。邰大惊。览曰:“袁绍听信谗言,必为曹
操所擒;吾等岂可坐而待死?不如去投曹操。”邰曰:“吾亦有此心久矣。”

于是二人领本部兵马,往曹操寨中投降。夏侯惇曰:“张、高二人来降,未知虚实。”
操曰:“吾以恩遇之,虽有异心,亦可变矣。”遂开营门命二人入。二人倒戈卸甲,拜伏于
地。操曰:“若使袁绍肯从二将军之言,不至有败。今二将军肯来相投,如微子去殷,韩信
归汉也。”遂封张邰为偏将军、都亭侯,高览为偏将军、东莱侯。二人大喜。

却说袁绍既去了许攸,又去了张邰、高览,又失了乌巢粮,军心皇皇。许攸又劝曹操作
速进兵;张邰、高览请为先锋;操从之。即令张邰、高览领兵往劫绍寨。当夜三更时分,出
军三路劫寨。混战到明,各自收兵,绍军折其大半。

荀攸献计曰:“今可扬言调拨人马,一路取酸枣,攻邺郡;一路取黎阳,断袁兵归路。
袁绍闻之,必然惊惶,分兵拒我;我乘其兵动时击之,绍可破也。”操用其计,使大小三
军,四远扬言。绍军闻此信,来寨中报说:“曹操分兵两路:一路取邺郡,一路取黎阳去
也。”绍大惊,急遣袁谭分兵五万救邺郡,辛明分兵五万救黎阳,连夜起行。

曹操探知袁绍兵动,便分大队军马,八路齐出,直冲绍营。袁军俱无斗志,四散奔走,
遂大溃。袁绍披甲不迭,单衣幅巾上马;幼子袁尚后随。张辽、许褚、徐晃、于禁四员将,
引军追赶袁绍。绍急渡河,尽弃图书车仗金帛,止引随行八百余骑而去。操军追之不及,尽
获遗下之物。所杀八万余人,血流盈沟,溺水死者不计其数。

操获全胜,将所得金宝缎匹,给赏军士。于图书中检出书信一束,皆许都及军中诸人与
绍暗通之书。左右曰:“可逐一点对姓名,收而杀之。”操曰:“当绍之强,孤亦不能自
保,况他人乎?”遂命尽焚之,更不再问。

却说袁绍兵败而奔,沮授因被囚禁,急走不脱,为曹军所获,擒见曹操。操素与授相
识。授见操,大呼曰:“授不降也!”操曰:“本初无谋,不用君言,君何尚执迷耶?吾若
早得足下,天下不足虑也。”因厚待之,留于军中。授乃于营中盗马,欲归袁氏。操怒,乃
杀之。授至死神色不变。操叹曰:“吾误杀忠义之士也!”命厚礼殡殓,为建坟安葬于黄河
渡口,题其墓曰:“忠烈沮君之墓。”后人有诗赞曰:“河北多名士,忠贞推沮君:凝眸知
阵法,仰面识天文;至死心如铁,临危气似云。曹公钦义烈,特与建孤坟。”操下令攻冀
州。正是:势弱只因多算胜,兵强却为寡谋亡。未知胜负如何,且看下文分解。