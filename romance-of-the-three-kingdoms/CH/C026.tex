\chapter{袁本初损兵折将~关云长挂印封金}

却说袁绍欲斩玄德。玄德从容进曰:“明公只听一面之词,而绝向日之情耶?备自徐州
失散,二弟云长未知存否;天下同貌者不少,岂赤面长须之人,即为关某也?明公何不察
之?”袁绍是个没主张的人,闻玄德之言,责沮授曰:“误听汝言,险杀好人。”遂仍请玄
德上帐坐,议报颜良之仇。帐下一人应声而进曰:“颜良与我如兄弟,今被曹贼所杀,我安
得不雪其恨?”玄德视其人,身长八尺,面如獬豸,乃河北名将文丑也。袁绍大喜曰:“非
汝不能报颜良之仇。吾与十万军兵,便渡黄河,追杀曹贼!”沮授曰:“不可。今宜留屯延
津,分兵官渡,乃为上策。若轻举渡河,设或有变,众皆不能还矣。”绍怒曰:“皆是汝等
迟缓军心,迁延日月,有妨大事!岂不闻兵贵神速乎?”沮授出,叹曰:“上盈其志,下务
其功;悠悠黄河,吾其济乎!”遂托疾不出议事。玄德曰:“备蒙大恩,无可报效,意欲与
文将军同行:一者报明公之德,二者就探云长的实信。”绍喜,唤文丑与玄德同领前部。文
丑曰:“刘玄德屡败之将,于军不利。既主公要他去时,某分三万军,教他为后部。”于是
文丑自领七万军先行,令玄德引三万军随后。

且说曹操见云长斩了颜良,倍加钦敬,表奏朝廷,封云长为汉寿亭侯,铸印送关公。忽
报袁绍又使大将文丑渡黄河,已据延津之上。操乃先使人移徙居民于西河,然后自领兵迎
之;传下将令:以后军为前军,以前军为后军;粮草先行,军兵在后。吕虔曰:“粮草在
先,军兵在后,何意也?”操曰:“粮草在后,多被剽掠,故令在前。”虔曰:“倘遇敌军
劫去,如之奈何?”操曰:“且待敌军到时,却又理会。”虚心疑未决。操令粮食辎重沿河
堑至延津。操在后军,听得前军发喊,急教人看时,报说:“河北大将文丑兵至,我军皆弃
粮草,四散奔走。后军又远,将如之何?”操以鞭指南阜曰:“此可暂避。”人马急奔土
阜。操令军士皆解衣卸甲少歇,尽放其马。文丑军掩至。众将曰:“贼至矣!可急收马匹,
退回白马!”荀攸急止之曰:“此正可以饵敌,何故反退?”操急以目视荀攸而笑。攸知其
意,不复言。文丑军既得粮草车仗,又来抢马。军士不依队伍,自相杂乱。曹操却令军将一
齐下土阜击之,文丑军大乱。曹兵围裹将来,文丑挺身独战,军士自相践踏。文丑止遏不
住,只得拨马回走。操在土阜上指曰:“文丑为河北名将、谁可擒之?”张辽、徐晃飞马齐
出,大叫:“文丑休走!”文丑回头见二将赶上,遂按住铁枪,拈弓搭箭,正射张辽。徐晃
大叫:“贼将休放箭!”张辽低头急躲,一箭射中头盔,将簪缨射去。辽奋力再赶,坐下战
马,又被文丑一箭射中面颊。那马跪倒前蹄,张辽落地。文丑回马复来,徐晃急轮大斧,截
住厮杀。只见文丑后面军马齐到,晃料敌不过,拨马而回。文丑沿河赶来。

忽见十余骑马,旗号翩翻,一将当头提刀飞马而来,乃关云长也,大喝:“贼将休
走!”与文丑交马,战不三合,文丑心怯,拨马绕河而走。关公马快,赶上文丑,脑后一
刀,将文丑斩下马来。曹操在土阜上,见关公砍了文丑,大驱人马掩杀。河北军大半落水,
粮草马匹仍被曹操夺回。

云长引数骑东冲西突。正杀之间,刘玄德领三万军随后到。前面哨马探知,报与玄德
云:“今番又是红面长髯的斩了文丑。”玄德慌忙骤马来看,隔河望见一簇人马,往来如
飞,旗上写着“汉寿亭侯关云长”七字。玄德暗谢天地曰:“原来吾弟果然在曹操处!”欲
待招呼相见,被曹兵大队拥来,只得收兵回去。袁绍接应至官渡,下定寨栅。郭图、审配入
见袁绍,说:“今番又是关某杀了文丑,刘备佯推不知。”袁绍大怒,骂曰:“大耳贼焉敢
如此!”少顷,玄德至,绍令推出斩之。玄德曰:“某有何罪?”绍曰:“你故使汝弟又坏
我一员大将,如何无罪?”玄德曰:“容伸一言而死:曹操素忌备,今知备在明公处,恐备
助公,故特使云长诛杀二将。公知必怒。此借公之手以杀刘备也。愿明公思之。”袁绍曰:
“玄德之言是也。汝等几使我受害贤之名。”喝退左右,请玄德上帐而坐。玄德谢曰:“荷
明公宽大之恩,无可补报,欲令一心腹人持密书去见云长,使知刘备消息,彼必星夜来到,
辅佐明公,共诛曹操,以报颜良、文丑之仇,若何?”袁绍大喜曰:“吾得云长,胜颜良、
文丑十倍也。”玄德修下书札,未有人送去。绍令退军武阳,连营数十里,按兵不动。操乃
使夏侯惇领兵守住官渡隘口,自己班师回许都,大宴众官,贺云长之功。因谓吕虔曰:“昔
日吾以粮草在前者,乃饵敌之计也。惟荀公达知吾心耳。”众皆叹服。正饮宴间,忽报:
“汝南有黄巾刘辟、龚都,甚是猖獗。曹洪累战不利,乞遣兵救之。”云长闻言,进曰:
“关某愿施犬马之劳,破汝南贼寇。”操曰:“云长建立大功,未曾重酬,岂可复劳征
进?”公曰:“关某久闲,必生疾病。愿再一行。”曹操壮之,点兵五万,使于禁、乐进为
副将,次日便行。荀彧密谓操曰:“云长常有归刘之心,倘知消息必去,不可频令出征。”
操曰:“今次收功,吾不复教临敌矣。”

且说云长领兵将近汝南,扎住营寨。当夜营外拿了两个细作人来。云长视之,内中认得
一人,乃孙乾也。关公叱退左右,问乾曰:“公自溃散之后,一向踪迹不闻,今何为在此
处?”乾曰:“某自逃难,飘泊汝南,幸得刘辟收留。今将军为何在曹操处?未识甘、糜二
夫人无恙否?”关公因将上项事细说一遍。乾曰:“近闻玄德公在袁绍处,欲往投之,未得
其便。今刘、龚二人归顺袁绍,相助攻曹。天幸得将军到此,因特令小军引路,教某为细
作,来报将军。来日二人当虚败一阵,公可速引二夫人投袁绍处,与玄德公相见。”关公
曰:“既兄在袁绍处,吾必星夜而往。但恨吾斩绍二将,恐今事变矣。”乾曰:“吾当先往
探彼虚实,再来报将军。”公曰:“吾见兄长一面,虽万死不辞。今回许昌,便辞曹操
也。”当夜密送孙乾去了。次日,关公引兵出,龚都披挂出阵。关公曰:“汝等何故背反朝
廷?”都曰:“汝乃背主之人,何反责我?”关公曰:“我何为背主?”都曰:“刘玄德在
袁本初处,汝却从曹操,何也?”关公更不打话,拍马舞刀向前。龚都便走,关公赶上。都
回身告关公曰:“故主之恩,不可忘也。公当速进,我让汝南。”关公会意,驱军掩杀。
刘、龚二人佯输诈败,四散去了。云长夺得州县,安民已定,班师回许昌。曹操出郭迎接,
赏劳军士。宴罢,云长回家,参拜二嫂于门外。甘夫人曰:“叔叔西番出军,可知皇叔音信
否?”公答曰:“未也”。关公退,二夫人于门内痛哭曰:“想皇叔休矣!二叔恐我妹妹烦
恼,故隐而不言。”正哭间,有一随行老军,听得哭声不绝,于门外告曰:“夫人休哭,主
人现在河北袁绍处。”夫人曰:“汝何由知之?”军曰:“跟关将军出征,有人在阵上说
来。”夫人急召云长责之曰:“皇叔未尝负汝,汝今受曹操之恩,顿忘旧日之义,不以实情
告我,何也?”关公顿首曰:“兄今委实在河北。未敢教嫂嫂知者,恐有泄漏也。事须缓
图,不可欲速。”甘夫人曰:“叔宜上紧。”公退,寻思去计,坐立不安。

原来于禁探知刘备在河北,报与曹操。操令张辽来探关公意。关公正闷坐,张辽入贺
曰:“闻兄在阵上知玄德音信,特来贺喜。”关公曰:“故主虽在,未得一见,何喜之
有!”辽曰:“兄与玄德交,比弟与兄交何如?”公曰:“我与兄,朋友之交也;我与玄
德,是朋友而兄弟、兄弟而主臣者也:岂可共论乎?”辽曰:“今玄德在河北,兄往从
否?”关公曰:“昔日之言,安肯背之!文远须为我致意丞相。”张辽将关公之言,回告曹
操,操曰:“吾自有计留之。”

且说关公正寻思间,忽报有故人相访。及请入,却不相识。关公问曰:“公何人也?”
答曰:“某乃袁绍部下南阳陈震也。”关公大惊,急退左右,问曰:“先生此来,必有所
为?”震出书一缄,递与关公。公视之,乃玄德书也。其略云:“备与足下,自桃园缔盟,
誓以同死。今何中道相违,割恩断义?君必欲取功名、图富贵,愿献备首级以成全功。书不
尽言,死待来命。”关公看书毕,大哭曰:“某非不欲寻兄,奈不知所在也。安肯图富贵而
背旧盟乎?”震曰:“玄德望公甚切,公既不背旧盟,宜速往见。”关公曰:“人生天地
间,无终始者,非君子也。吾来时明白,去时不可不明白。吾今作书,烦公先达知兄长,容
某辞却曹操,奉二嫂来相见。”震曰:“倘曹操不允。为之奈何?”公曰:“吾宁死,岂肯
久留于此!震曰:“公速作回书,免致刘使君悬望。”关公写书答云:“窃闻义不负心,忠
不顾死。羽自幼读书,粗知礼义,观羊角哀、左伯桃之事,未尝不三叹而流涕也。前守下
邳。内无积粟,外听援兵;欲即效死,奈有二嫂之重,未敢断首捐躯,致负所托;故尔暂且
羁身,冀图后会。近至汝南,方知兄信;即当面辞曹公,奉二嫂归。羽但怀异心,神人共
戮。披肝沥胆,笔楮难穷。瞻拜有期,伏惟照鉴。”陈震得书自回。

关公入内告知二嫂,随即至相府,拜辞曹操。操知来意,乃悬回避牌于门。关公怏怏而
回,命旧日跟随人役,收拾车马,早晚伺候;分付宅中,所有原赐之物,尽皆留下,分毫不
可带去。次日再往相府辞谢,门首又挂回避牌。关公一连去了数次,皆不得见。乃往张辽家
相探,欲言其事。辽亦托疾不出。关公思曰:“此曹丞相不容我去之意。我去志已决,岂可
复留!”即写书一封,辞谢曹操。书略曰:“羽少事皇叔,誓同生死;皇天后土,实闻斯
言。前者下邳失守,所请三事,已蒙恩诺。今探知故主现在袁绍军中,回思昔日之盟,岂容
违背?新恩虽厚,旧义难忘。兹特奉书告辞,伏惟照察。其有余恩未报,愿以俟之异日。”
写毕封固,差人去相府投递;一面将累次所受金银,一一封置库中,悬汉寿亭侯印于堂上,
请二夫人上车。关公上赤兔马,手提青龙刀,率领旧日跟随人役,护送车仗,径出北门。门
吏挡之。关公怒目横刀,大喝一声,门吏皆退避。关公既出门,谓从者曰:“汝等护送车仗
先行,但有追赶者,吾自当之,勿得惊动二位夫人。”从者推车,望官道进发。却说曹操正
论关公之事未定,左右报关公呈书。操即看毕,大惊曰:“云长去矣!”忽北门守将飞报:
“关公夺门而去,车仗鞍马二十余人,皆望北行。”又关公宅中人来报说:“关公尽封所赐
金银等物。美女十人,另居内室。其汉寿亭侯印悬于堂上。丞相所拨人役,皆不带去,只带
原跟从人,及随身行李,出北门去了。”众皆愕然。一将挺身出曰:“某愿将铁骑三千,去
生擒关某,献与丞相!”众视之,乃将军蔡阳也。正是:欲离万丈蛟龙穴,又遇三千狼虎
兵。蔡阳要赶关公,毕竟如何,且听下文分解。