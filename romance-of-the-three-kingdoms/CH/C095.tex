\chapter{马谡拒谏失街亭~武侯弹琴退仲达}

却说魏主曹睿令张郃为先锋,与司马懿一同征进;一面令辛毗、孙礼二人领兵五万,往
助曹真。二人奉诏而去。且说司马懿引二十万军,出关下寨,请先锋张郃至帐下曰:“诸葛
亮平生谨慎,未敢造次行事。若是吾用兵,先从子午谷径取长安,早得多时矣。他非无谋,
但怕有失,不肯弄险。今必出军斜谷,来取郿城。若取郿城,必分兵两路,一军取箕谷矣。吾
已发檄文,令子丹拒守郿城,若兵来不可出战;令孙礼、辛毗截住箕谷道口,若兵来则出奇
兵击之。”郃曰:“今将军当于何处进兵?”懿曰:“吾素知秦岭之西,有一条路,地名街
亭;傍有一城,名列柳城:此二处皆是汉中咽喉。诸葛亮欺子丹无备,定从此进。吾与汝径
取街亭,望阳平关不远矣。亮若知吾断其街亭要路,绝其粮道,则陇西一境,不能安守,必
然连夜奔回汉中去也。彼若回动,吾提兵于小路击之,可得全胜;若不归时,吾却将诸处小
路,尽皆垒断,俱以兵守之。一月无粮,蜀兵皆饿死,亮必被吾擒矣。”张郃大悟,拜伏于
地曰:“都督神算也!”懿曰:“虽然如此,诸葛亮不比孟达。将军为先锋,不可轻进。当
传与诸将:循山西路,远远哨探。如无伏兵,方可前进。若是怠忽,必中诸葛亮之计。”张
郃受计引军而行。

却说孔明在祁山寨中,忽报新城探细人来到。孔明急唤入问之,细作告曰:“司马懿倍
道而行,八日已到新城,孟达措手不及;又被申耽、申仪、李辅、邓贤为内应:孟达被乱军
所杀。今司马懿撤兵到长安,见了魏主,同张郃引兵出关,来拒我师也。”孔明大惊曰:
“孟达做事不密,死固当然。今司马懿出关,必取街亭,断吾咽喉之路。”便问:“谁敢引
兵去守街亭?”言未毕,参军马谡曰:“某愿往。”孔明曰:“街亭虽小,干系甚重:倘街
亭有失,吾大军皆休矣。汝虽深通谋略,此地奈无城郭,又无险阻,守之极难。”谡曰:
“某自幼熟读兵书,颇知兵法。岂一街亭不能守耶?”孔明曰:“司马懿非等闲之辈;更有
先锋张郃,乃魏之名将:恐汝不能敌之。”谡曰:“休道司马懿、张郃,便是曹睿亲来,有何
惧哉!若有差失,乞斩全家。”孔明曰:“军中无戏言。”谡曰:“愿立军令状。”孔明从
之,谡遂写了军令状呈上。孔明曰:“吾与汝二万五千精兵,再拨一员上将,相助你去。”
即唤王平分付曰:“吾素知汝平生谨慎,故特以此重任相托。汝可小心谨守此地:下寨必当
要道之处,使贼兵急切不能偷过。安营既毕,便画四至八道地理形状图本来我看。凡事商议
停当而行,不可轻易。如所守无危,则是取长安第一功也。戒之!戒之!”二人拜辞引兵而
去。孔明寻思,恐二人有失,又唤高翔曰:“街亭东北上有一城,名列柳城,乃山僻小路,
此可以屯兵扎寨。与汝一万兵,去此城屯扎。但街亭危,可引兵救之。”高翔引兵而去。孔
明又思:高翔非张郃对手,必得一员大将,屯兵于街亭之右,方可防之,遂唤魏延引本部兵
去街亭之后屯扎。延曰:“某为前部,理合当先破敌,何故置某于安闲之地?’孔明曰:
“前锋破敌,乃偏裨之事耳。今令汝接应街亭,当阳平关冲要道路,总守汉中咽喉:此乃大
任也,何为安闲乎?汝勿以等闲视之,失吾大事。切宜小心在意!”魏延大喜,引兵而去。
孔明恰才心安,乃唤赵云、邓芝分付曰:“今司马懿出兵,与旧日不同。汝二人各引一军出
箕谷,以为疑兵。如逢魏兵,或战、或不战,以惊其心。吾自统大军,由斜谷径取郿城;若
得郿城,长安可破矣。”二人受命而去。孔明令姜维作先锋,兵出斜谷。

却说马谡、王平二人兵到街亭,看了地势。马谡笑曰:“丞相何故多心也?量此山僻之
处,魏兵如何敢来!”王平曰:“虽然魏兵不敢来,可就此五路总口下寨;却令军士伐木为
栅,以图久计。”谡曰:“当道岂是下寨之地?此处侧边一山,四面皆不相连,且树木极
广,此乃天赐之险也:可就山上屯军。”平曰:“参军差矣。若屯兵当道,筑起城垣,贼兵
总有十万,不能偷过;今若弃此要路,屯兵于山上,倘魏兵骤至,四面围定,将何策保
之?”谡大笑曰:“汝真女子之见!兵法云:凭高视下,势如劈竹。若魏兵到来,吾教他片
甲不回!”平曰:“吾累随丞相经阵,每到之处,丞相尽意指教。今观此山,乃绝地也:若
魏兵断我汲水之道,军士不战自乱矣。”谡曰:“汝莫乱道!孙子云:置之死地而后生。若
魏兵绝我汲水之道,蜀兵岂不死战?以一可当百也。吾素读兵书,丞相诸事尚问于我,汝奈
何相阻耶!”平曰:“若参军欲在山上下寨,可分兵与我,自于山西下一小寨,为掎角之
势。倘魏兵至,可以相应。”马谡不从。忽然山中居民,成群结队,飞奔而来,报说魏兵已
到。王平欲辞去。马谡曰:“汝既不听吾令,与汝五千兵自去下寨。待吾破了魏兵,到丞相
面前须分不得功!”王平引兵离山十里下寨,画成图本,星夜差人去禀孔明,具说马谡自于
山上下寨。却说司马懿在城中,令次子司马昭去探前路:若街亭有兵守御,即当按兵不行。
司马昭奉令探了一遍,回见父曰:“街亭有兵守把。”懿叹曰:“诸葛亮真乃神人,吾不如
也!”昭笑曰:“父亲何故自堕志气耶?男料街亭易取。”懿问曰:“汝安敢出此大言?”
昭曰:“男亲自哨见,当道并无寨栅,军皆屯于山上,故知可破也。”懿大喜曰:“若兵果
在山上,乃天使吾成功矣!”遂更换衣服,引百余骑亲自来看。是夜天晴月朗,直至山下,
周围巡哨了一遍,方回。马谡在山上见之,大笑曰:“彼若有命,不来围山!”传令与诸
将:“倘兵来,只见山顶上红旗招动,即四面皆下。”

却说司马懿回到寨中,使人打听是何将引兵守街亭。回报曰:“乃马良之弟马谡也。”
懿笑曰:“徒有虚名,乃庸才耳!孔明用如此人物,如何不误事!”又问:“街亭左右别有
军否?”探马报曰:“离山十里有王平安营。”懿乃命张郃引一军,当住王平来路。又令申
耽、申仪引两路兵围山,先断了汲水道路;待蜀兵自乱,然后乘势击之。当夜调度已定。次
日天明,张郃引兵先往背后去了。司马懿大驱军马,一拥而进,把山四面围定。马谡在山上
看时,只见魏兵漫山遍野,旌旗队伍,甚是严整。蜀兵见之,尽皆丧胆,不敢下山。马谡将
红旗招动,军将你我相推,无一人敢动。谡大怒,自杀二将。众军惊惧,只得努力下山来冲
魏兵。魏兵端然不动。蜀兵又退上山去。马谡见事不谐,教军紧守寨门,只等外应。

却说王平见魏兵到,引军杀来,正遇张郃;战有数十余合,平力穷势孤,只得退去。魏
兵自辰时困至戌时,山上无水,军不得食,寨中大乱。嚷到半夜时分,山南蜀兵大开寨门,
下山降魏。马谡禁止不住。司马懿又令人于沿山放火,山上蜀兵愈乱。马谡料守不住,只得
驱残兵杀下山西逃奔。司马懿放条大路,让过马谡。背后张郃引兵追来。赶到三十余里,前
面鼓角齐鸣,一彪军出,放过马谡,拦住张郃;视之,乃魏延也。延挥刀纵马,直取张郃。郃
回军便走。延驱兵赶来,复夺街亭。赶到五十余里,一声喊起,两边伏兵齐出:左边司马
懿,右边司马昭,却抄在魏延背后,把延困在垓心。张郃复来,三路兵合在一处。魏延左冲
右突,不得脱身,折兵大半。正危急间,忽一彪军杀入,乃王平也。延大喜曰:“吾得生
矣!”二将合兵一处,大杀一阵,魏兵方退。二将慌忙奔回寨时,营中皆是魏兵旌旗。申
耽、申仪从营中杀出。王平、魏延径奔列柳城,来投高翔。此时高翔闻知街亭有失,尽起列
柳城之兵,前来救应,正遇延、平二人,诉说前事。高翔曰:“不如今晚去劫魏寨,再复街
亭。”当时三人在山坡下商议已定。待天色将晚,兵分三路。魏延引兵先进,径到街亭,不
见一人,心中大疑,未敢轻进,且伏在路口等候,忽见高翔兵到,二人共说魏兵不知在何
处。正没理会,又不见王平兵到。忽然一声炮响,火光冲天,鼓起震地:魏兵齐出,把魏
延、高翔围在垓心。二人往来冲突,不得脱身。忽听得山坡后喊声若雷,一彪军杀入,乃是
王平,救了高、魏二人,径奔列柳城来。比及奔到城下时,城边早有一军杀到,旗上大书
“魏都督郭淮”字样。原来郭淮与曹真商议,恐司马懿得了全功,乃分淮来取街亭;闻知司
马懿、张郃成了此功,遂引兵径袭列柳城。正遇三将,大杀一阵。蜀兵伤者极多。魏延恐阳
平关有失,慌与王平、高翔望阳平关来。

却说郭淮收了军马,乃谓左右曰:“吾虽不得街亭,却取了列柳城,亦是大功。”引兵
径到城下叫门,只见城上一声炮响,旗帜皆竖,当头一面大旗,上书“平西都督司马懿”。
懿撑起悬空板,倚定护心木栏干,大笑曰:“郭伯济来何迟也?”淮大惊曰:“仲达神机,
吾不及也!”遂入城。相见已毕,懿曰:“今街亭已失,诸葛亮必走。公可速与子丹星夜追
之。”郭淮从其言,出城而去。懿唤张郃曰:“子丹、伯济,恐吾全获大功,故来取此城
池。吾非独欲成功,乃侥幸而已。吾料魏延、王平、马谡、高翔等辈,必先去据阳平关。吾
若去取此关,诸葛亮必随后掩杀,中其计矣。兵法云:归师勿掩,穷寇莫追。汝可从小路抄
箕谷退兵。吾自引兵当斜谷之兵。若彼败走,不可相拒,只宜中途截住:蜀兵辎重,可尽得
也。”张郃受计,引兵一半去了。懿下令:“竟取斜谷,由西城而进。西城虽山僻小县,乃
蜀兵屯粮之所,又南安、天水、安定三郡总路。若得此城,三郡可复矣。”于是司马懿留申
耽、申仪守列柳城,自领大军望斜谷进发。

却说孔明自令马谡等守街亭去后,犹豫不定。忽报王平使人送图本至。孔明唤入,左右
呈上图本。孔明就文几上拆开视之,拍案大惊曰:“马谡无知,坑陷吾军矣!”左右问曰:
“丞相何故失惊?”孔明曰:“吾观此图本,失却要路,占山为寨。倘魏兵大至,四面围
合,断汲水道路,不须二日,军自乱矣。若街亭有失,吾等安归?”长史杨仪进曰:“某虽
不才,愿替马幼常回。”孔明将安营之法,一一分付与杨仪。正待要行,忽报马到来,说:
“街亭、列柳城,尽皆失了!”孔明跌足长叹曰:“大事去矣!此吾之过也!”急唤关兴、
张苞分付曰:“汝二人各引三千精兵,投武功山小路而行。如遇魏兵,不可大击,只鼓噪呐
喊,为疑兵惊之。彼当自走,亦不可追。待军退尽,便投阳平关去。”又令张冀先引军去修
理剑阁,以备归路。又密传号令,教大军暗暗收拾行装,以备起程。又令马岱、姜维断后,
先伏于山谷中,待诸军退尽,方始收兵。又差心腹人,分路报与天水、南安、安定三郡官吏
军民,皆入汉中。又遣心腹人到冀县搬取姜维老母,送入汉中。

孔明分拨已定,先引五千兵退去西城县搬运粮草。忽然十余次飞马报到,说:“司马懿
引大军十五万,望西城蜂拥而来!”时孔明身边别无大将,只有一班文官,所引五千兵,已
分一半先运粮草去了,只剩二千五百军在城中。众官听得这个消息,尽皆失色。孔明登城望
之,果然尘土冲天,魏兵分两路望西城县杀来。孔明传令,教“将旌旗尽皆隐匿;诸军各守
城铺,如有妄行出入,及高言大语者,斩之!大开四门,每一门用二十军士,扮作百姓,洒
扫街道。如魏兵到时,不可擅动,吾自有计。”孔明乃披鹤氅,戴纶巾,引二小童携琴一
张,于城上敌楼前,凭栏而坐,焚香操琴。

却说司马懿前军哨到城下,见了如此模样,皆不敢进,急报与司马懿。懿笑而不信,遂
止住三军,自飞马远远望之。果见孔明坐于城楼之上,笑容可掬,焚香操琴。左有一童子,
手捧宝剑;右有一童子,手执麈尾。城门内外,有二十余百姓,低头洒扫,傍若无人,懿看
毕大疑,便到中军,教后军作前军,前军作后军,望北山路而退。次子司马昭曰:“莫非诸
葛亮无军,故作此态?父亲何故便退兵?”懿曰:“亮平生谨慎,不曾弄险。今大开城门,
必有埋伏。我兵若进,中其计也。汝辈岂知?宜速退。”于是两路兵尽皆退去。孔明见魏军
远去,抚掌而笑。众官无不骇然,乃问孔明曰:“司马懿乃魏之名将,今统十五万精兵到
此,见了丞相,便速退去,何也?”孔明曰:“此人料吾生平谨慎,必不弄险;见如此模
样,疑有伏兵,所以退去。吾非行险,盖因不得已而用之。此人必引军投山北小路去也。吾
已令兴、苞二人在彼等候。”众皆惊服曰:“丞相之机,神鬼莫测。若某等之见,必弃城而
走矣。”孔明曰:“吾兵止有二千五百,若弃城而走,必不能远遁。得不为司马懿所擒
乎?”后人有诗赞曰:“瑶琴三尺胜雄师,诸葛西城退敌时。十五万人回马处,土人指点到
今疑。”言讫,拍手大笑,曰:“吾若为司马懿,必不便退也。”遂下令,教西城百姓,随
军入汉中;司马懿必将复来。于是孔明离西城望汉中而走。天水、安定、南安三郡官吏军
民,陆续而来。

却说司马懿望武功山小路而走。忽然山坡后喊杀连天,鼓声震地。懿回顾二子曰:“吾
若不走,必中诸葛亮之计矣。”只见大路上一军杀来,旗上大书“右护卫使虎冀将军张
苞”。魏兵皆弃甲抛戈而走。行不到一程,山谷中喊声震地,鼓角喧天,前面一杆大旗,上
书“左护卫使龙骧将军关兴”。山谷应声,不知蜀兵多少;更兼魏军心疑,不敢久停,只得
尽弃辎重而去。兴、苞二人皆遵将令,不敢追袭,多得军器粮草而归。司马懿见山谷中皆有
蜀兵,不敢出大路,遂回街亭。

此时曹真听知孔明退兵,急引兵追赶。山背后一声炮响,蜀兵漫山遍野而来:为首大
将,乃是姜维、马岱。真大惊,急退军时,先锋陈造已被马岱所斩。真引兵鼠窜而还。蜀兵
连夜皆奔回汉中。却说赵云、邓芝伏兵于箕谷道中。闻孔明传令回军,云谓芝曰:“魏军知
吾兵退,必然来追。吾先引一军伏于其后,公却引兵打吾旗号,徐徐而退。吾一步步自有护
送也。

却说郭淮提兵再回箕谷道中,唤先锋苏顒分付曰:“蜀将赵云,英勇无敌。汝可小心提
防,彼军若退,必有计也。”苏顒欣然曰:“都督若肯接应,某当生擒赵云。”遂引前部三
千兵,奔入箕谷。看看赶上蜀兵,只见山坡后闪出红旗白字,上书“赵云”。苏顒急收兵退
走。行不到数里,喊声大震,一彪军撞出:为首大将,挺枪跃马,大喝曰:“汝识赵子龙
否!”苏顒大惊曰:“如何这里又有赵云?”措手不及,被云一枪刺死于马下。余军溃散。
云迤逦前进,背后又一军到,乃郭淮部将万政也。云见魏兵追急,乃勒马挺枪,立于路口,
待来将交锋。蜀兵已去三十余里。万政认得是赵云,不敢前进,云等得天色黄昏,方才拨回
马缓缓而进。郭淮兵到,万政言赵云英勇如旧,因此不敢近前。淮传令教军急赶,政令数百
骑壮士赶来。行至一大林,忽听得背后大喝一声曰:“赵子龙在此!”惊得魏兵落马者百余
人,余者皆越岭而去。万政勉强来敌,被云一箭射中盔缨,惊跌于涧中。云以枪指之曰:
“吾饶汝性命回去!快教郭淮赶来!”万政脱命而回。云护送车仗人马,望汉中而去,沿途
并无遗失。曹真、郭淮复夺三郡,以为己功。却说司马懿分兵而进。此时蜀兵尽回汉中去
了,懿引一军复到西城,因问遗下居民及山僻隐者,皆言孔明止有二千五百军在城中,又无
武将,只有几个文官,别无埋伏。武功山小民告曰:“关兴、张苞,只各有三千军,转山呐
喊,鼓噪惊追,又无别军,并不敢厮杀。”懿悔之不及,仰天叹曰:“吾不如孔明也!”遂
安抚了诸处官民,引兵径还长安,朝见魏主。睿曰:“今日复得陇西诸郡,皆卿之功也。”
懿奏曰:“今蜀兵皆在汉中,未尽剿灭。臣乞大兵并力收川,以报陛下。”睿大喜,令懿即
便兴兵。忽班内一人出奏曰:“臣有一计,足可定蜀降吴。”正是:蜀中将相方归国,魏地
君臣又逞谋。未知献计者是谁,且看下文分解。