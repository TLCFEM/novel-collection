\chapter{废汉帝陈留践位~谋董贼孟德献刀}

且说董卓欲杀袁绍,李儒止之曰:“事未可定,不可妄杀。”袁绍手提宝剑,辞别百官
而出,悬节东门,奔冀州去了。卓谓太傅袁隗曰:“汝侄无礼,吾看汝面,姑恕之。废立之
事若何?”隗曰:“太尉所见是也。”卓曰:“敢有阻大议者,以军法从事!”群臣震恐,
皆云一听尊命。宴罢,卓问侍中周毖、校尉伍琼曰:“袁绍此去若何?”周毖曰:“袁绍忿
忿而去,若购之急,势必为变。且袁氏树恩四世,门生故吏遍于天下;倘收豪杰以聚徒众,
英雄因之而起,山东非公有也。不如赦之,拜为一郡守,则绍喜于免罪,必无患矣。”伍琼
曰:“袁绍好谋无断,不足为虑;诚不若加之一郡守,以收民心。”卓从之,即日差人拜绍
为渤海太守。

九月朔,请帝升嘉德殿,大会文武。卓拔剑在手,对众曰:“天子暗弱,不足以君天
下。今有策文一道,宜为宣读。”乃命李儒读策曰:“孝灵皇帝,早弃臣民;皇帝承嗣,海
内侧望。而帝天资轻佻,威仪不恪,居丧慢惰:否德既彰,有忝大位。皇太后教无母仪,统
政荒乱。永乐太后暴崩,众论惑焉。三纲之道,天地之纪,毋乃有阙?陈留王协,圣德伟
懋,规矩肃然;居丧哀戚,言不以邪;休声美誉,天下所闻,宜承洪业,为万世统。兹废皇
帝为弘农王,皇太后还政,请奉陈留王为皇帝,应天顺人,以慰生灵之望。”李儒读策毕,
卓叱左右扶帝下殿,解其玺绶,北面长跪,称臣听命。又呼太后去服候敕。帝后皆号哭,群
臣无不悲惨。

阶下一大臣,愤怒高叫曰:“贼臣董卓,敢为欺天之谋,吾当以颈血溅之!”挥手中象
简,直击董卓。卓大怒,喝武士拿下:乃尚书丁管也。卓命牵出斩之。管骂不绝口,至死神
色不变。后人有诗叹之曰:“董贼潜怀废立图,汉家宗社委丘墟。满朝臣宰皆囊括,惟有丁
公是丈夫。”

卓请陈留王登殿。群臣朝贺毕,卓命扶何太后并弘农王及帝妃唐氏永安宫闲住,封锁宫
门,禁群臣无得擅入。可怜少帝四月登基,至九月即被废。卓所立陈留王协,表字伯和,灵
帝中子,即献帝也;时年九岁。改元初平。董卓为相国,赞拜不名,入朝不趋,剑履上殿,
威福莫比。

李儒劝卓擢用名流,以收人望,因荐蔡邕之才。卓命徵之,邕不赴。卓怒,使人谓邕
曰:“如不来,当灭汝族。”邕惧,只得应命而至。卓见邕大喜,一月三迁其官,拜为侍
中,甚见亲厚。

却说少帝与何太后、唐妃困于永安宫中,衣服饮食,渐渐少缺;少帝泪不曾干。一日,
偶见双燕飞于庭中,遂吟诗一首。诗曰:“嫩草绿凝烟,袅袅双飞燕。洛水一条青,陌上人
称羡。远望碧云深,是吾旧宫殿。何人仗忠义,泄我心中怨!”董卓时常使人探听。是日获
得此诗,来呈董卓。卓曰:“怨望作诗,杀之有名矣。”遂命李儒带武士十人,入宫弑帝。
帝与后、妃正在楼上,宫女报李儒至,帝大惊。儒以鸩酒奉帝,帝问何故。儒曰:“春日融
和,董相国特上寿酒。”太后曰:“既云寿酒,汝可先饮。”儒怒曰:“汝不饮耶?”呼左
右持短刀白练于前曰:“寿酒不饮,可领此二物!”唐妃跪告曰:“妾身代帝饮酒,愿公存
母子性命。”儒叱曰:“汝何人,可代王死?”乃举酒与何太后曰:“汝可先饮?”后大骂
何进无谋,引贼入京,致有今日之祸。儒催逼帝,帝曰:“容我与太后作别。”乃大恸而作
歌,其歌曰:“天地易兮日月翻,弃万乘兮退守藩。为臣逼兮命不久,大势去兮空泪潸!”
唐妃亦作歌曰:“皇天将崩兮后土颓,身为帝姬兮命不随。生死异路兮从此毕,奈何茕速兮
心中悲!”歌罢,相抱而哭,李儒叱曰:“相国立等回报,汝等俄延,望谁救耶?”太后大
骂:“董贼逼我母子,皇天不佑!汝等助恶,必当灭族!”儒大怒,双手扯住太后,直撺下
楼;叱武士绞死唐妃;以鸩酒灌杀少帝。

还报董卓,卓命葬于城外。自此每夜入宫,奸淫宫女,夜宿龙床。尝引军出城,行到阳
城地方,时当二月,村民社赛,男女皆集。卓命军士围住,尽皆杀之,掠妇女财物,装载车
上,悬头千余颗于车下,连轸还都,扬言杀贼大胜而回;于城门外焚烧人头,以妇女财物分
散众军。越骑校尉伍孚,字德瑜,见卓残暴,愤恨不平,尝于朝服内披小铠,藏短刀,欲伺
便杀卓。一日,卓入朝,孚迎至阁下,拔刀直刺卓。卓气力大,两手抠住;吕布便入,揪倒
伍孚。卓问曰:“谁教汝反?”孚瞪目大喝曰:“汝非吾君,吾非汝臣,何反之有?汝罪恶
盈天,人人愿得而诛之!吾恨不车裂汝以谢天下!”卓大怒,命牵出剖剐之。孚至死骂不绝
口。后人有诗赞之曰:“汉末忠臣说伍孚,冲天豪气世间无。朝堂杀贼名犹在,万古堪称大
丈夫!”董卓自此出入常带甲士护卫。

时袁绍在渤海,闻知董卓弄权,乃差人赍密书来见王允。书略曰:“卓贼欺天废主,人
不忍言;而公恣其跋扈,如不听闻,岂报国效忠之臣哉?绍今集兵练卒,欲扫清王室,未敢
轻动。公若有心,当乘间图之。如有驱使,即当奉命。”王允得书,寻思无计。一日,于侍
班阁子内见旧臣俱在,允曰:“今日老夫贱降,晚间敢屈众位到舍小酌。”众官皆曰:“必
来祝寿。”当晚王允设宴后堂,公卿皆至。酒行数巡,王允忽然掩面大哭。众官惊问曰:
“司徒贵诞,何故发悲?”允曰:“今日并非贱降,因欲与众位一叙,恐董卓见疑,故托言
耳。董卓欺主弄权,社稷旦夕难保。想高皇诛秦灭楚,奄有天下;谁想传至今日,乃丧于董
卓之手:此吾所以哭也。”于是众官皆哭。坐中一人抚掌大笑曰:“满朝公卿,夜哭到明,
明哭到夜,还能哭死董卓否?”允视之,乃骁骑校尉曹操也。允怒曰:“汝祖宗亦食禄汉
朝,今不思报国而反笑耶?”操曰:“吾非笑别事,笑众位无一计杀董卓耳。操虽不才,愿
即断董卓头,悬之都门,以谢天下。”允避席问曰:“孟德有何高见?”操曰:“近日操屈

身以事卓者,实欲乘间图之耳。今卓颇信操,操因得时近卓。闻司徒有七宝刀一口,愿借与
操入相府刺杀之,虽死不恨!”允曰:“孟德果有是心,天下幸甚!”遂亲自酌酒奉操。操
沥酒设誓,允随取宝刀与之。操藏刀,饮酒毕,即起身辞别众官而去。众官又坐了一回,亦
俱散讫。

次日,曹操佩着宝刀,来至相府,问:“丞相何在?”从人云:“在小阁中。”操径
入。见董卓坐于床上,吕布侍立于侧。卓曰:“孟德来何迟?”操曰:“马羸行迟耳。”卓
顾谓布曰:“吾有西凉进来好马,奉先可亲去拣一骑赐与孟德。”布领令而出。操暗忖曰:
“此贼合死!”即欲拔刀刺之,惧卓力大,未敢轻动。卓胖大不耐久坐,遂倒身而卧,转面
向内。操又思曰:“此贼当休矣!”急掣宝刀在手,恰待要刺,不想董卓仰面看衣镜中,照
见曹操在背后拔刀,急回身问曰:“孟德何为?”时吕布已牵马至阁外。操惶遽,乃持刀跪
下曰:“操有宝刀一口,献上恩相。”卓接视之,见其刀长尺余,七宝嵌饰,极其锋利,果
宝刀也;遂递与吕布收了。操解鞘付布。卓引操出阁看马,操谢曰:“愿借试一骑。”卓就
教与鞍辔。操牵马出相府,加鞭望东南而去。

布对卓曰:“适来曹操似有行刺之状,及被喝破,故推献刀。”卓曰:“吾亦疑之。”
正说话间,适李儒至,卓以其事告之。儒曰:“操无妻小在京,只独居寓所。今差人往召,
如彼无疑而便来,则是献刀;如推托不来,则必是行刺,便可擒而问也。”卓然其说,即差
狱卒四人往唤操。去了良久,回报曰:“操不曾回寓,乘马飞出东门。门吏问之,操曰‘丞
相差我有紧急公事’,纵马而去矣。”儒曰:“操贼心虚逃窜,行刺无疑矣。”卓大怒曰:
“我如此重用,反欲害我!”儒曰:“此必有同谋者,待拿住曹操便可知矣。”卓遂令遍行
文书,画影图形,捉拿曹操:擒献者,赏千金,封万户侯;窝藏者同罪。

且说曹操逃出城外,飞奔谯郡。路经中牟县,为守关军士所获,擒见县令。操言:“我
是客商,覆姓皇甫。”县令熟视曹操,沉吟半晌,乃曰:“吾前在洛阳求官时,曾认得汝是
曹操,如何隐讳!且把来监下,明日解去京师请赏。”把关军士赐以酒食而去。至夜分,县
令唤亲随人暗地取出曹操,直至后院中审究;问曰:“我闻丞相待汝不薄,何故自取其
祸?”操曰:“燕雀安知鸿鹄志哉!汝既拿住我,便当解去请赏。何必多问!”县令屏退左
右,谓操曰:“汝休小觑我。我非俗吏,奈未遇其主耳。”操曰:“吾祖宗世食汉禄,若不
思报国,与禽兽何异?吾屈身事卓者,欲乘间图之,为国除害耳。今事不成,乃天意也!”
县令曰:“孟德此行,将欲何往?”操曰:“吾将归乡里,发矫诏,召天下诸侯兴兵共诛董
卓:吾之愿也。”县令闻言,乃亲释其缚,扶之上坐,再拜曰:“公真天下忠义之士也!”
曹操亦拜,问县令姓名。县令曰:“吾姓陈,名宫,字公台。老母妻子,皆在东郡。今感公
忠义,愿弃一官,从公而逃。”操甚喜。是夜陈宫收拾盘费,与曹操更衣易服,各背剑一
口,乘马投故乡来。

行了三日,至成皋地方,天色向晚。操以鞭指林深处谓宫曰:“此间有一人姓吕,名伯
奢,是吾父结义弟兄;就往问家中消息,觅一宿,如何?”宫曰:“最好。”二人至庄前下
马,入见伯奢。奢曰:“我闻朝廷遍行文书,捉汝甚急,汝父已避陈留去了。汝如何得至
此?”操告以前事,曰:“若非陈县令,已粉骨碎身矣。”伯奢拜陈宫曰:“小侄若非使
君,曹氏灭门矣。使君宽怀安坐,今晚便可下榻草舍。”说罢,即起身入内。良久乃出,谓
陈宫曰:“老夫家无好酒,容往西村沽一樽来相待。”言讫,匆匆上驴而去。

操与宫坐久,忽闻庄后有磨刀之声。操曰:“吕伯奢非吾至亲,此去可疑,当窃听
之。”二人潜步入草堂后,但闻人语曰:“缚而杀之,何如?”操曰:“是矣!今若不先下
手,必遭擒获。”遂与宫拔剑直入,不问男女,皆杀之,一连杀死八口。搜至厨下,却见缚
一猪欲杀。宫曰:“孟德心多,误杀好人矣!”急出庄上马而行。行不到二里,只见伯奢驴
鞍前鞒悬酒二瓶,手携果菜而来,叫曰:“贤侄与使君何故便去?”操曰:“被罪之人,不
敢久住。”伯奢曰:“吾已分付家人宰一猪相款,贤侄、使君何憎一宿?速请转骑。”操不
顾,策马便行。行不数步,忽拔剑复回,叫伯奢曰:“此来者何人?”伯奢回头看时,操挥
剑砍伯奢于驴下。宫大惊曰:“适才误耳,今何为也?”操曰:“伯奢到家,见杀死多人,
安肯干休?若率众来追,必遭其祸矣。”宫曰:“知而故杀,大不义也!”操曰:“宁教我
负天下人,休教天下人负我。”陈宫默然。

当夜,行数里,月明中敲开客店门投宿。喂饱了马,曹操先睡。陈宫寻思:“我将谓曹
操是好人,弃官跟他;原来是个狼心之徒!今日留之,必为后患。”便欲拔剑来杀曹操。正
是:设心狠毒非良士,操卓原来一路人。毕竟曹操性命如何,且听下文分解。