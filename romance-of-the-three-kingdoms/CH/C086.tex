\chapter{难张温秦宓逞天辩~破曹丕徐盛用火攻}

却说东吴陆逊,自退魏兵之后,吴王拜逊为辅国将军,江陵侯,领荆州牧,自此军权皆归于逊。张昭、顾雍启奏吴王,请自改元。权从之,遂改为黄武元年。忽报魏主遣使至,权召入。使命陈说:“蜀前使人求救于魏,魏一时不明,故发兵应之;今已大悔,欲起四路兵取川,东吴可来接应。若得蜀土,各分一半。”权闻言,不能决,乃问于张昭、顾雍等。昭曰:“陆伯言极有高见,可问之。”权即召陆逊至。逊奏曰:“曹丕坐镇中原,急不可图;今若不从,必为仇矣。臣料魏与吴皆无诸葛亮之敌手。今且勉强应允,整军预备,只探听四路如何。若四路兵胜,川中危急,诸葛亮首尾不能救,主上则发兵以应之,先取成都,深为上策;如四路兵败,别作商议。”权从之,乃谓魏使曰:“军需未办,择日便当起程。”使者拜辞而去。

权令人探得西番兵出西平关,见了马超,不战自退;南蛮孟获起兵攻四郡,皆被魏延用疑兵计杀退回洞去了;上庸孟达兵至半路,忽然染病不能行;曹真兵出阳平关,赵子龙拒住各处险道,果然“一将守关,万夫莫开”。曹真屯兵于斜谷道,不能取胜而回。孙权知了此信,乃谓文武曰:“陆伯言真神算也。孤苦妄动,又结怨于西蜀矣。”忽报西蜀遣邓芝到。张昭曰:“此又是诸葛亮退兵之计,遣邓芝为说客也。”权曰:“当何以答之?”昭曰:“先于殿前立一大鼎,贮油数百斤,下用炭烧。待其油沸,可选身长面大武士一千人,各执刀在手,从宫门前直摆至殿上,却唤芝入见。休等此人开言下说词,责以郦食其说齐故事,效此例烹之,看其人如何对答。”

权从其言,遂立油鼎,命武士立于左右,各执军器,召邓芝入。芝整衣冠而入。行至宫门前,只见两行武士,威风凛凛,各持钢刀、大斧、长戟、短剑,直列至殿上。芝晓其意,并无惧色,昂然而行。至殿前,又见鼎镬内热油正沸。左右武士以目视之,芝但微微而笑。近臣引至帘前,邓芝长揖不拜。权令卷起珠帘,大喝曰:“何不拜!”芝昂然而答曰:“上国天使,不拜小邦之主。”权大怒曰:“汝不自料,欲掉三寸之舌,效郦生说齐乎!可速入油鼎。”芝大笑曰:“人皆言东吴多贤,谁想惧一儒生!”权转怒曰:“孤何惧尔一匹夫耶?”芝曰:“既不惧邓伯苗,何愁来说汝等也?”权曰:“尔欲为诸葛亮作说客,来说孤绝魏向蜀,是否?”芝曰:“吾乃蜀中一儒生,特为吴国利害而来。乃设兵陈鼎,以拒一使,何其局量之不能容物耶!”权闻言惶愧,即叱退武士,命芝上殿,赐坐而问曰:“吴、魏之利害若何?愿先生教我。”芝曰:“大王欲与蜀和,还是欲与魏和?”权曰:“孤正欲与蜀主讲和;但恐蜀主年轻识浅,不能全始全终耳。”芝曰:“大王乃命世之英豪,诸葛亮亦一时之俊杰;蜀有山川之险,吴有三江之固:若二国连和,共为唇齿,进则可以兼吞天下,退则可以鼎足而立。今大王若委贽称臣于魏,魏必望大王朝觐,求太子以为内侍;如其不从,则兴兵来攻,蜀亦顺流而进取:如此则江南之地,不复为大王有矣。若大王以愚言为不然,愚将就死于大王之前,以绝说客之名也。”言讫,撩衣下殿,望油鼎中便跳。权急命止之,请入后殿,以上宾之礼相待。权曰:“先生之言,正合孤意。孤今欲与蜀主连和,先生肯为我介绍乎!”芝曰:“适欲烹小臣者,乃大王也;今欲使小臣者,亦大王也。大王犹自狐疑未定,安能取信于人?”权曰:“孤意已决,先生勿疑。”

于是吴王留住邓芝,集多官问曰:“孤掌江南八十一州,更有荆楚之地,反不如西蜀偏僻之处也。蜀有邓芝,不辱其主;吴并无一人入蜀,以达孤意。”忽一人出班奏曰:“臣愿为使。”众视之,乃吴郡吴人,姓张,名温,字惠恕,现为中郎将。权曰:“恐卿到蜀见诸葛亮,不能达孤之情。”温曰:“孔明亦人耳,臣何畏彼哉?”权大喜,重赏张温,使同邓芝入川通好。却说孔明自邓芝去后,奏后主曰:“邓芝此去,其事必成。吴地多贤,定有人来答礼。陛下当礼貌之,令彼回吴,以通盟好。吴若通和,魏必不敢加兵于蜀矣。吴、魏宁靖,臣当征南,平定蛮方,然后图魏。魏削则东吴亦不能久存,可以复一统之基业也。”后主然之。

忽报东吴遣张温与邓芝入川答礼。后主聚文武于丹墀,令邓芝、张温入。温自以为得志,昂然上殿,见后主施礼。后主赐锦墩,坐于殿左,设御宴待之。后主但敬礼而已。宴罢,百官送张温到馆舍。次日,孔明设宴相待。孔明谓张温曰:“先帝在日,与吴不睦,今已晏驾。当今主上,深慕吴王,欲捐旧忿,永结盟好,并力破魏。望大夫善言回奏。”张温领诺。酒至半酣,张温喜笑自若,颇有傲慢之意。

次日,后主将金帛赐与张温,设宴于城南邮亭之上,命众官相送。孔明殷勤劝酒。正饮酒间,忽一人乘醉而入,昂然长揖,入席就坐。温怪之,乃问孔明曰:“此何人也?”孔明答曰:“姓秦,名宓,字子勑,现为益州学士。”温笑曰:“名称学士,未知胸中曾学事否?”宓正色而言曰:“蜀中三尺小童,尚皆就学,何况于我?”温曰:“且说公何所学?”宓对曰:“上至天文,下至地理,三教九流,诸子百家,无所不通;古今兴废,圣贤经传,无所不览。”温笑曰:“公既出大言,请即以天为问:天有头乎?”宓曰:“有头。”温曰:“头在何方?”宓曰:“在西方。《诗》云:‘乃眷西顾。’以此推之,头在西方也。”温又问:“天有耳乎?”宓答曰:“天处高而听卑。《诗》云:‘鹤鸣九皋,声闻于天。’无耳何能听?”温又问:“天有足乎?”宓曰:“有足。《诗》云:‘天步艰难。’无足何能步?”温又问:“天有姓乎?”宓曰:“岂得无姓!”温曰:“何姓?”宓答曰:“姓刘。”温曰:“何以知之?”宓曰:“天子姓刘,以故知之。”温又问曰:“日生于东乎?”宓对曰:“虽生于东,而没于西。”此时秦宓语言清朗,答问如流,满座皆惊。张温无语,宓乃问曰:“先生东吴名士,既以天事下问,必能深明天之理。昔混沌既分,阴阳剖判;轻清者上浮而为天,重浊者下凝而为地;至共工氏战败,头触不周山,天柱折,地维缺:天倾西北,地陷东南。天既轻清而上浮,何以倾其西北乎?又未知轻清之外,还是何物?愿先生教我。”张温无言可对,乃避席而谢曰:“不意蜀中多出俊杰!恰闻讲论,使仆顿开茅塞。”孔明恐温羞愧,故以善言解之曰:“席间问难,皆戏谈耳。足下深知安邦定国之道,何在唇齿之戏哉!”温拜谢。孔明又令邓芝入吴答礼,就与张温同行。张、邓二人拜辞孔明,望东吴而来。却说吴王见张温入蜀未还,乃聚文武商议。忽近臣奏曰:“蜀遣邓芝同张温入国答礼。”权召入。张温拜于殿前,备称后主、孔明之德,愿求永结盟好,特遣邓尚书又来答礼。权大喜,乃设宴待之。权问邓芝曰:“若吴、蜀二国同心灭魏,得天下太平,二主分治,岂不乐乎?”芝答曰:“天无二日,民无二王。如灭魏之后,未识天命所归何人。但为君者,各修其德;为臣者,各尽其忠:则战争方息耳。”权大笑曰:“君之诚款,乃如是耶!”遂厚赠邓芝还蜀。自此吴、蜀通好。

却说魏国细作人探知此事,火速报入中原。魏主曹丕听知,大怒曰:“吴、蜀连和,必有图中原之意也。不若朕先伐之。”于是大集文武,商议起兵伐吴。此时大司马曹仁、太尉贾诩已亡。侍中辛毗出班奏曰:“中原之地,土阔民稀,而欲用兵,未见其利。今日之计,莫若养兵屯田十年,足食足兵,然后用之,则吴、蜀方可破也。”丕怒曰:“此迂儒之论也!今吴、蜀连和,早晚必来侵境,何暇等待十年!”即传旨起兵伐吴。司马懿奏曰:“吴有长江之险,非船莫渡。陛下必御驾亲征,可选大小战船,从蔡、颖而入淮,取寿春,至广陵,渡江口,径取南徐:此为上策。”丕从之。于是日夜并工,造龙舟十只,长二十余丈,可容二千余人,收拾战船三千余只。魏黄初五年秋八月,会聚大小将士,令曹真为前部,张辽、张郃、文聘、徐晃等为大将先行,许褚、吕虔为中军护卫,曹休为合后,刘晔、蒋济为参谋官。前后水陆军马三十余万,克日起兵。封司马懿为尚书仆射,留在许昌,凡国政大事,并皆听懿决断。不说魏兵起程。却说东吴细作探知此事,报入吴国。近臣慌奏吴王曰:“今魏王曹丕,亲自乘驾龙舟,提水陆大军三十余万,从蔡、颖出淮,必取广陵渡江,来下江南。甚为利害。”孙权大惊,即聚文武商议。顾雍曰:“今主上既与西蜀连和,可修书与诸葛孔明,令起兵出汉中,以分其势;一面遣一大将,屯兵南徐以拒之。”权曰:“非陆伯言不可当此大任。雍曰:“陆伯言镇守荆州,不可轻动。”权曰:“孤非不知,奈眼前无替力之人。”言未尽,一人从班部内应声而出曰:“臣虽不才,愿统一军以当魏兵。若曹丕亲渡大江,臣必主擒以献殿下;若不渡江,亦杀魏兵大半,今魏兵不敢正视东吴。”权视之,乃徐盛也。权大喜曰:“如得卿守江南一带,孤何忧哉!”遂封徐盛为安东将军,总镇都督建业、南徐军马。盛谢恩,领命而退;即传令教众官军多置器械,多设旌旗,以为守护江岸之计。忽一人挺身出曰:“今日大王以重任委托将军,欲破魏兵以擒曹丕,将军何不早发军马渡江,于淮南之地迎敌?直待曹丕兵至,恐无及矣。”盛视之,乃吴王侄孙韶也。韶字公礼,官授扬威将军,曾在广陵守御;年幼负气,极有胆勇。盛曰:“曹丕势大;更有名将为先锋,不可渡江迎敌。待彼船皆集于北岸,吾自有计破之。”韶曰:“吾手下自有三千军马,更兼深知广陵路势,吾愿自去江北,与曹丕决一死战。如不胜,甘当军令。”盛不从。韶坚执要去,盛只是不肯,韶再三要行。盛怒曰:“汝如此不听号令,吾安能制诸将乎?”叱武士推出斩之。刀斧手拥孙韶出辕门之外,立起皂旗。韶部将飞报孙权。权听知,急上马来救。武士恰待行刑,孙权早到,喝散刀斧手,救了孙韶。韶哭奏曰:“臣往年在广陵,深知地利;不就那里与曹丕厮杀,直待他下了长江,东吴指日休矣!”权径入营来。徐盛迎接入帐,奏曰:“大王命臣为都督,提兵拒魏;今扬威将军孙韶,不遵军法,违令当斩,大王何故赦之?”权曰:“韶倚血气之壮,误犯军法,万希宽恕。”盛曰:“法非臣所立,亦非大王所立,乃国家之典刑也。若以亲而免之,何以令众乎?”权曰:“韶犯法,本应任将军处治;奈此子虽本姓俞氏,然孤兄甚爱之,赐姓孙;于孤颇有劳绩。今若杀之,负兄义矣。”盛曰:“且看大王之面,寄下死罪。”权令孙韶拜谢。韶不肯拜,厉声而言曰:“据吾之见,只是引军去破曹丕!便死也不服你的见识!”徐盛变色。权叱退孙韶,谓徐盛曰:“便无此子,何损于兵?今后勿再用之。”言讫自回。是夜,人报徐盛说:“孙韶引本部三千精兵,潜地过江去了。”盛恐有失,于吴王面上不好看,乃唤丁奉授以密计,引三千兵渡江接应。却说魏主驾龙舟至广陵,前部曹真已领兵列于大江之岸。曹丕问曰:“江岸有多少兵?”真曰:“隔岸远望,并不见一人,亦无旌旗营寨。”丕曰:“此必诡计也。朕自往观其虚实。”于是大开江道,放龙舟直至大江,泊于江岸。船上建龙凤日月五色旌旗,仪銮簇拥,光耀射目。曹丕端坐舟中,遥望江南,不见一人,回顾刘晔、蒋济曰:“可渡江否?”晔曰:“兵法实实虚虚。彼见大军至,如何不作整备?陛下未可造次。且待三五日,看其动静,然后发先锋渡江以探之。”丕曰:“卿言正合朕意。”是日天晚,宿于江中。当夜月黑,军士皆执灯火,明耀天地,恰如白昼。遥望江南,并不见半点儿火光。丕问左右曰:“此何故也?”近臣奏曰:“想闻陛下天兵来到,故望风逃窜耳。”丕暗笑。及至天晓,大雾迷漫,对面不见。须臾风起,雾散云收,望见江南一带皆是连城:城楼上枪刀耀日,遍城尽插旌旗号带。顷刻数次人来报:“南徐沿江一带,直至石头城,一连数百里,城郭舟车,连绵不绝,一夜成就。”曹丕大惊。原来徐盛束缚芦苇为人,尽穿青衣,执旌旗,立于假城疑楼之上。魏兵见城上许多人马,如何不胆寒?丕叹曰:“魏虽有武士千群,无所用之。江南人物如此,未可图也!”

正惊讶间,忽然狂风大作,白浪滔天,江水溅湿龙袍,大船将覆。曹真慌令文聘撑小舟急来救驾。龙舟上人立站不住。文聘跳上龙舟,负丕下得小舟,奔入河港。忽流星马报道:“赵云引兵出阳平关,径取长安。”丕听得,大惊失色,便教回军。众军各自奔走。背后吴兵追至。丕传旨教尽弃御用之物而走。龙舟将次入淮,忽然鼓角齐鸣,喊声大震,刺斜里一彪军杀到:为首大将,乃孙韶也。魏兵不能抵当,折其大半,淹死者无数。诸将奋力救出魏主。魏主渡淮河,行不三十里,淮河中一带芦苇,预灌鱼油,尽皆火着;顺风而下,风势甚急,火焰漫空,绝住龙舟。丕大惊,急下小船傍岸时,龙舟上早已火着。丕慌忙上马。岸上一彪军杀来;为首一将,乃丁奉也。张辽急拍马来迎,被奉一箭射中其腰,却得徐晃救了,同保魏主而走,折军无数。背后孙韶、丁奉夺得马匹、车仗、船只、器械不计其数。魏兵大败而回。吴将徐盛全获大功,吴王重加赏赐。张辽回到许昌,箭疮迸裂而亡,曹丕厚葬之,不在话下。却说赵云引兵杀出阳平关之次,忽报丞相有文书到,说益州耆帅雍闿结连蛮王孟获,起十万蛮兵,侵掠四郡;因此宣云回军,令马超坚守阳平关,丞相欲自南征。赵云乃急收兵而回。此时孔明在成都整饬军马,亲自南征。正是:方见东吴敌北魏,又看西蜀战南蛮。未知胜负如何,且看下文分解。