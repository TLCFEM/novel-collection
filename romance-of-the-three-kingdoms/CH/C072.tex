\chapter{诸葛亮智取汉中~曹阿瞒兵退斜谷}

却说徐晃引军渡汉水,王平苦谏不听,渡过汉水扎营。黄忠、赵云告玄德曰:“某等各
引本部兵去迎曹兵。”玄德应允。二人引兵而行。忠谓云曰:“今徐晃恃勇而来,且休与
敌;待日暮兵疲,你我分兵两路击之可也。”云然之,各引一军据住寨栅。徐晃引兵从辰时
搦战,直至申时,蜀兵不动。晃尽教弓弩手向前,望蜀营射去。黄忠谓赵云曰:“徐晃令弓
弩射者,其军必将退也:可乘时击之。”言未已,忽报曹兵后队果然退动。于是蜀营鼓声大
震:黄忠领兵左出,赵云领兵右出。两下夹攻,徐晃大败,军士逼入汉水,死者无数。晃死
战得脱,回营责王平曰:“汝见吾军势将危,如何不救?”平曰:“我若来救,此寨亦不能
保。我曾谏公休去,公不肯所,以致此败。”晃大怒,欲杀王平。平当夜引本部军就营中放
起火来,曹兵大乱,徐晃弃营而走。王平渡汉水来投赵云,云引见玄德。王平尽言汉水地
理。玄德大喜曰:“孤得王子均,取汉中无疑矣。”遂命王平为偏将军,领向导使。却说徐
晃逃回见操,说:“王平反去降刘备矣!”操大怒,亲统大军来夺汉水寨栅。赵云恐孤军难
立,遂退于汉水之西。两军隔水相拒,玄德与孔明来观形势。孔明见汉水上流头,有一带土
山,可伏千余人;乃回到营中,唤赵云分付:“汝可引五百人,皆带鼓角,伏于土山之下;
或半夜,或黄昏,只听我营中炮响:炮响一番,擂鼓一番。只不要出战。”子龙受计去了。
孔明却在高山上暗窥。次日,曹兵到来搦战,蜀营中一人不出,弓弩亦都不发。曹兵自回。
当夜更深,孔明见曹营灯火方息,军士歇定,遂放号炮。子龙听得,令鼓角齐鸣。曹兵惊
慌,只疑劫寨。及至出营,不见一军。方才回营欲歇,号炮又响,鼓角又鸣,呐喊震地,山
谷应声。曹兵彻夜不安。一连三夜,如此惊疑,操心怯,拔寨退三十里,就空阔处扎营。孔
明笑曰:“曹操虽知兵法,不知诡计。”遂请玄德亲渡汉水,背水结营。玄德问计,孔明
曰:“可如此如此。”

曹操见玄德背水下寨,心中疑惑,使人来下战书。孔明批来日决战。次日,两军会于中
路五界山前,列成阵势。操出马立于门旗下,两行布列龙凤旌旗,擂鼓三通,唤玄德答话。
玄德引刘封、孟达并川中诸将而出。操扬鞭大骂曰:“刘备忘恩失义,反叛朝廷之贼!”玄
德曰:“吾乃大汉宗亲,奉诏讨贼。汝上弑母后,自立为王,僭用天子銮舆,非反而何?”
操怒,命徐晃出马来战,刘封出迎。交战之时,玄德先走入阵。封敌晃不住,拨马便走。操
下令:“捉得刘备,便为西川之主。”大军齐呐喊杀过阵来。蜀兵望汉水而逃,尽弃营寨;
马匹军器,丢满道上。曹军皆争取。操急鸣金收军。众将曰:“某等正待捉刘备,大王何故
收军?”操曰:“吾见蜀兵背汉水安营,其可疑一也;多弃马匹军器,其可疑二也。可急退
军,休取衣物。”遂下令曰:“妄取一物者立斩。火速退兵。”曹兵方回头时,孔明号旗举
起:玄德中军领兵便出,黄忠左边杀来,赵云右边杀来。曹兵大溃而逃,孔明连夜追赶。

操传令军回南郑,只见五路火起,原来魏延、张飞得严颜代守阆中,分兵杀来,先得了
南郑。操心惊,望阳平关而走。玄德大兵追至南郑褒州。安民已毕,玄德问孔明曰:“曹操
此来,何败之速也?”孔明曰:“操平生为人多疑,虽能用兵,疑则多败。吾以疑兵胜
之。”玄德曰:“今操退守阳平关,其势已孤,先生将何策以退之?”孔明曰?“亮已算定
了。”便差张飞、魏延分兵两路去截曹操粮道,令黄忠、赵云分兵两路去放火烧山。四路军
将,各引向导官军去了。

却说曹操退守阳平关,令军哨探。回报曰:“今蜀兵将远近小路,尽皆塞断;砍柴去
处,尽放火烧绝。不知兵在何处。”操正疑惑间,又报张飞、魏延分兵劫粮。操问曰:“谁
敢敌张飞?”许褚曰:“某愿往!”操令许褚引一千精兵,去阳平关路上护接粮草。解粮官
接着,喜曰:“若非将军到此,粮不得到阳平矣。”遂将车上的酒肉,献与许褚。褚痛饮,
不觉大醉,便乘酒兴,催粮车行。解粮官曰:“日已暮矣,前褒州之地,山势险恶,未可过
去。”褚曰:“吾有万夫之勇,岂惧他人哉!今夜乘着月色,正好使粮车行走。”许褚当
先,横刀纵马,引军前进。二更已后,往褒州路上而来。行至半路,忽山凹里鼓角震天,一
枝军当住。为首大将,乃张飞也,挺矛纵马,直取许褚。褚舞刀来迎,却因酒醉,敌不住张
飞;战不数合,被飞一矛刺中肩膀,翻身落马;军士急忙救起,退后便走。张飞尽夺粮草车
辆而回。却说众将保着许褚,回见曹操。操令医士疗治金疮,一面亲自提兵来与蜀兵决战。
玄德引军出迎。两阵对圆,玄德令刘封出马。操骂曰:“卖履小儿,常使假子拒敌!吾若唤
黄须儿来,汝假子为肉泥矣!”刘封大怒,挺枪骤马,径取曹操。操令徐晃来迎,封诈败而
走。操引兵追赶。蜀兵营中,四下炮响,鼓角齐鸣。操恐有伏兵,急教退军。曹兵自相践
踏,死者极多,奔回阳平关,方才歇定。蜀兵赶到城下:东门放火,西门呐喊;南门放火,
北门擂鼓。操大惧,弃关而走。蜀兵从后追袭。操正走之间,前面张飞引一枝兵截住,赵云
引一枝兵从背后杀来,黄忠又引兵从褒州杀来。操大败。诸将保护曹操,夺路而走。方逃至
斜谷界口,前面尘头忽起,一枝兵到。操曰:“此军若是伏兵,吾休矣!”及兵将近,乃操
次子曹彰也。彰字子文,少善骑射;膂力过人,能手格猛兽。操尝戒之曰:“汝不读书而好
弓马,此匹夫之勇,何足贵乎?”彰曰:“大丈夫当学卫青、霍去病,立功沙漠,长驱数十
万众,纵横天下;何能作博士耶?”操尝问诸子之志。彰曰:“好为将。”操问:“为将何
如?”彰曰:“披坚执锐,临难不顾,身先士卒;赏必行,罚必信。”操大笑。建安二十三
年,代郡乌桓反,操令彰引兵五万讨之;临行戒之曰:“居家为父子,受事为君臣。法不徇
情,尔宜深戒。”彰到代北,身先战阵,直杀至桑干,北方皆平;因闻操在阳平败阵,故来
助战。操见彰至,大喜曰:“我黄须儿来,破刘备必矣!”遂勒兵复回,于斜谷界口安营。
有人报玄德,言曹彰到。玄德问曰:“谁敢去战曹彰?”刘封曰:“某愿往。”孟达又说要
去。玄德曰:“汝二人同去,看谁成功。”各引兵五千来迎:“刘封在先,孟达在后,曹彰
出马与封交战,只三合,封大败而回。孟达引兵前进,方欲交锋,只见曹兵大乱。原来马
超、吴兰两军杀来,曹兵惊动。孟达引兵夹攻。马超士卒,蓄锐日久,到此耀武扬威,势不
可当。曹兵败走。曹彰正遇吴兰,两个交锋,不数合,曹彰一戟刺吴兰于马下。三军混战。
操收兵于斜谷界口扎住。操屯兵日久,欲要进兵,又被马超拒守;欲收兵回,又恐被蜀兵耻
笑,心中犹豫不决。适庖官进鸡汤。操见碗中有鸡肋,因而有感于怀。正沉吟间,夏侯惇入
帐,禀请夜间口号。操随口曰:“鸡肋!鸡肋!”惇传令众官,都称“鸡肋”。行军主簿杨
修,见传“鸡肋”二字,便教随行军士,各收拾行装,准备归程。有人报知夏侯惇。惇大
惊,遂请杨修至帐中问曰:“公何收拾行装?”修曰:“以今夜号令,便知魏王不日将退兵
归也:鸡肋者,食之无肉,弃之有味。今进不能胜,退恐人笑,在此无益,不如早归:来日
魏王必班师矣。故先收拾行装,免得临行慌乱。”夏侯惇曰:“公真知魏王肺腑也!”遂亦
收拾行装。于是寨中诸将,无不准备归计。当夜曹操心乱,不能稳睡,遂手提钢斧,绕寨私
行。只见夏侯惇寨内军士,各准备行装。操大惊,急回帐召惇问其故。惇曰:“主簿杨德祖
先知大王欲归之意。”操唤杨修问之,修以鸡肋之意对。操大怒曰:“汝怎敢造言乱我军
心!”喝刀斧手推出斩之,将首级号令于辕门外。原来杨修为人恃才放旷,数犯曹操之忌:
操尝造花园一所;造成,操往观之,不置褒贬,只取笔于门上书一“活”字而去。人皆不晓
其意。修曰:“门内添活字,乃阔字也。丞相嫌园门阔耳。”于是再筑墙围,改造停当,又
请操观之。操大喜,问曰:“谁知吾意?”左右曰:“杨修也。”操虽称美,心甚忌之。又
一日,塞北送酥一盒至。操自写“一合酥”三字于盒上,置之案头。修入见之,竟取匙与众
分食讫。操问其故,修答曰:“盒上明书一人一口酥,岂敢违丞相之命乎?”操虽喜笑,而
心恶之。操恐人暗中谋害己身,常分付左右:“吾梦中好杀人;凡吾睡着,汝等切勿近
前。”一日,昼寝帐中,落被于地,一近侍慌取覆盖。操跃起拔剑斩之,复上床睡;半晌而
起,佯惊问:“何人杀吾近侍?”众以实对。操痛哭,命厚葬之。人皆以为操果梦中杀人;
惟修知其意,临葬时指而叹曰:“丞相非在梦中,君乃在梦中耳!”操闻而愈恶之。操第三
子曹植,爱修之才,常邀修谈论,终夜不息。操与众商议,欲立植为世子,曹丕知之,密请
朝歌长吴质入内府商议;因恐有人知觉,乃用大簏藏吴质于中,只说是绢匹在内,载入府
中。修知其事,径来告操。操令人于丕府门伺察之。丕慌告吴质,质曰:“无忧也:明日用
大簏装绢再入以惑之。”丕如其言,以大簏载绢入。使者搜看簏中,果绢也,回报曹操。操
因疑修谮害曹丕,愈恶之。操欲试曹丕、曹植之才干。一日,令各出邺城门;却密使人分付
门吏,令勿放出。曹丕先至,门吏阻之,丕只得退回。植闻之,问于修。修曰:“君奉王命
而出,如有阻当者,竟斩之可也。”植然其言。及至门,门吏阻住。植叱曰:“吾奉王命,
谁敢阻当!”立斩之。于是曹操以植为能。后有人告操曰:“此乃杨修之所教也。”操大
怒,因此亦不喜植。修又尝为曹植作答教十余条,但操有问,植即依条答之。操每以军国之
事问植,植对答如流。操心中甚疑。后曹丕暗买植左右,偷答教来告操。操见了大怒曰:
“匹夫安敢欺我耶!”此时已有杀修之心;今乃借惑乱军心之罪杀之。修死年三十四岁。后
人有诗曰:“聪明杨德祖,世代继簪缨。笔下龙蛇走,胸中锦绣成。开谈惊四座,捷对冠群
英。身死因才误,非关欲退兵。”

曹操既杀杨修,佯怒夏侯惇,亦欲斩之。众官告免。操乃叱退夏侯惇,下令来日进兵。
次日,兵出斜谷界口,前面一军相迎,为首大将乃魏延也。操招魏延归降,延大骂。操令庞
德出战。二将正斗间,曹寨内火起。人报马超劫了中后二寨。操拔剑在手曰:“诸将退后者
斩!”众将努力向前,魏延诈败而走。操方麾军回战马超,自立马于高阜处,看两军争战。
忽一彪军撞至面前,大叫:“魏延在此!”拈弓搭箭,射中曹操。操翻身落马。延弃弓绰
刀,骤马上山坡来杀曹操。刺斜里闪出一将,大叫:“休伤吾主!”视之,乃庞德也。德奋
力向前,战退魏延,保操前行。马超已退。操带伤归寨:原来被魏延射中人中,折却门牙两
个,急令医士调治。方忆杨修之言,随将修尸收回厚葬,就令班师;却教庞德断后。操卧于
毡车之中,左右虎贲军护卫而行。忽报斜谷山上两边火起,伏兵赶来。曹兵人人惊恐。正
是:依稀昔日潼关厄,仿佛当年赤壁危。未知曹操性命如何,且看下文分解。