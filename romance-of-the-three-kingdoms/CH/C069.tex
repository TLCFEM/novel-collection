\chapter{卜周易管辂知机~讨汉贼五臣死节}

却说当日曹操见黑风中群尸皆起,惊倒于地。须臾风定,群尸皆不见。左右扶操回宫,惊而成疾。后人有诗赞左慈曰:“飞步凌云遍九州,独凭遁甲自遨游。等闲施设神仙术,点悟曹瞒不转头。”曹操染病,服药无愈。适太史丞许芝,自许昌来见操。操令芝卜易。芝曰:“大王曾闻神卜管辂否?”操曰:“颇闻其名,未知其术。汝可详言之。”芝曰:“管辂字公明,平原人也。容貌粗丑,好酒疏狂。其父曾为琅琊即丘长。辂自幼便喜仰视星辰,夜不肯寐,父母不能禁止。常云家鸡野鹄,尚自知时,何况为人在世乎?与邻儿共戏,辄画地为天文,分布日月星辰。及稍长,即深明《周易》,仰观风角,数学通神,兼善相术。琅琊太守单子春闻其名,召辂相见。时有坐客百余人,皆能言之士。辂谓子春曰:辂年少胆气未坚,先请美酒三升,饮而后言。子春奇之,遂与酒三升。饮毕,辂问子春:今欲与辂为对者,若府君四座之士耶?子春曰:吾自与卿旗鼓相当。于是与辂讲论易理。辂亹亹而谈,言言精奥。子春反覆辩难,辂对答如流。从晓至暮,酒食不行。子春及众宾客,无不叹服。于是天下号为神童。后有居民郭恩者,兄弟三人,皆得躄疾,请辂卜之。辂曰:卦中有君家本墓中女鬼,非君伯母即叔母也。昔饥荒之年,谋数升米之利,推之落井,以大石压破其头,孤魂痛苦,自诉于天,故君兄弟有此报。不可禳也。郭恩等涕泣伏罪。安平太守王基,知辂神卜,延辂至家。适信都令妻常患头风,其子又患心痛,因请辂卜之。辂曰:此堂之西角有二死尸:一男持矛,一男持弓箭。头在壁内,脚在壁外。持矛者主刺头,故头痛;持弓箭者主刺胸腹,故心痛。乃掘之。入地八尺,果有二棺。一棺中有矛,一棺中有角弓及箭,木俱已朽烂。辂令徙骸骨去城外十里埋之,妻与子遂无恙。馆陶令诸葛原,迁新兴太守,辂往送行。客言辂能覆射。诸葛原不信,暗取燕卵、蜂窠、蜘蛛三物,分置三盒之中,令辂卜之。卦成,各写四句于盒上。其一曰:含气须变,依乎宇堂;雌雄以形,羽翼舒张:此燕卵也。其二曰:家室倒悬,门户众多;藏精育毒,得秋乃化:此蜂窠也。其三曰:觳觫长足,吐丝成罗;寻网求食,利在昏夜:此蜘蛛也。满座惊骇。乡中有老妇失牛,求卜之。辂判曰:北溪之滨,七人宰烹;急往追寻,皮肉尚存。老妇果往寻之:七人于茅舍后煮食,皮肉犹存。妇告本郡太守刘?,捕七人罪之。因问老妇曰:汝何以知之?妇告以管辂之神卜。刘?不信,请辂至府,取印囊及山鸡毛藏于盒中,令卜之。辂卜其一曰:内方外圆,五色成文;含宝守信,出则有章:此印囊也。其二曰:岩岩有鸟,锦体朱衣;羽翼玄黄,鸣不失晨:此山鸡毛也。刘?大惊,遂待为上宾。一日,出郊闲行,见一少年耕于田中,辂立道傍,观之良久,问曰:“少年高姓、贵庚?答曰:姓赵,名颜,年十九岁矣。敢问先生为谁?辂曰:吾管辂也。吾见汝眉间有死气,三日内必死。汝貌美,可惜无寿。赵颜回家,急告其父。父闻之,赶上管辂,哭拜于地曰:请归救吾子!辂曰:“此乃天命也,安可禳乎?父告曰:老夫止有此子,望乞垂救!赵颜亦哭求。辂见其父子情切,乃谓赵颜曰:汝可备净酒一瓶,鹿脯一块,来日赍往南山之中,大树之下,看盘石上有二人弈棋:一人向南坐,穿白袍,其貌甚恶;一人向北坐,穿红袍,其貌甚美。汝可乘其弈兴浓时,将酒及鹿脯跑进之。待其饮食毕,汝乃哭拜求寿,必得益算矣。但切勿言是吾所教。老人留辂在家。次日,赵颜携酒脯杯盘入南山之中。约行五六里,果有二人于大松树下盘石上着棋,全然不顾。赵颜跪进酒脯。二人贪着棋,不觉饮酒已尽。赵颜哭拜于地而求寿,二人大惊。穿红袍者曰:此必管子之言也。吾二人既受其私,必须怜之。穿白袍者,乃于身边取出簿籍检看,谓赵颜曰:汝今年十九岁,当死。吾今于十字上添一九字,汝寿可至九十九。回见管辂,教再休泄漏天机;不然,必致天谴。穿红者出笔添讫,一阵香风过处,二人化作二白鹤,冲天而去。赵颜归问管辂。辂曰:穿红者,南斗也;穿白者,北斗也。颜曰:吾闻北斗九星,何止一人?辂曰:散而为九,合而为一也。北斗注死,南斗注生。今已添注寿算,子复何忧?父子拜谢。自此管辂恐泄天机,更不轻为人卜。此人现在平原,大王欲知休咎,何不召之?”

操大喜,即差人往平原召辂。辂至,参拜讫,操令卜之。辂答曰:“此幻术耳,何必为忧?”操心安,病乃渐可。操令卜天下之事。辂卜曰;“三八纵横,黄猪遇虎;定军之南,伤折一股。”又令卜传祚修短之数。辂卜曰:“狮子宫中,以安神位;王道鼎新,子孙极贵。”操问其详。辂曰:“茫茫天数,不可预知。待后自验。”操欲封辂为太史。辂曰:“命薄相穷,不称此职,不敢受也。”操问其故,答曰:“辂额无主骨,眼无守睛;鼻无梁柱,脚无天根;背无三甲,腹无三壬:只可泰山治鬼,不能治生人也。”操曰:“汝相吾若何?”辂曰:“位极人臣,又何必相?”再三问之,辂但笑而不答。操令辂遍相文武官僚。辂曰:“皆治世之臣也。”操问休咎,皆不肯尽言。后人有诗赞曰:“平原神卜管公明,能算南辰北斗星。八封幽微通鬼窍,六爻玄奥究天庭。预知相法应无寿,自觉心源极有灵。可惜当年奇异术,后人无复授遗经。”

操令卜东吴、西蜀二处。辂设卦云:“东吴主亡一大将,西蜀有兵犯界。”操不信。忽合淝报来:“东吴陆口守将鲁肃身故。”操大惊,便差人往汉中探听消息。不数日,飞报刘玄德遣张飞、马超兵屯下辨取关。操大怒,便欲自领大兵再入汉中,令管辂卜之。辂曰:“大王未可妄动,来春许都必有火灾。”操见辂言累验,故不敢轻动,留居邺郡。使曹洪领兵五万,往助夏侯渊、张郃同守东川;又差夏侯惇领兵三万,于许都来往巡警,以备不虞;又教长史王必总督御林军马。主簿司马懿曰;“王必嗜酒性宽,恐不堪任此职。”操曰:“王必是孤披荆棘厉艰难时相随之人,忠而且勤,心如铁石,最足相当。”遂委王必领御林军马屯于许都东华门外。

时有一人,姓耿,名纪,字季行,洛阳人也;旧为丞相府掾,后迁侍中少府,与司直韦晃甚厚;见曹操进封王爵,出入用天子车服,心甚不平。时建安二十三年春正月。耿纪与韦晃密议曰:“操贼奸恶日甚,将来必为篡逆之事。吾等为汉臣,岂可同恶相济?”韦晃曰:“吾有心腹人,姓金,名祎,乃汉相金日磾之后,素有讨操之心;更兼与王必甚厚。若得同谋,大事济矣。”耿纪曰:“他既与王必交厚,岂肯与我等同谋乎?”韦晃曰:“且往说之,看是如何。”于是二人同至金祎宅中。祎接入后堂,坐定。晃曰:“德伟与王长史甚厚,吾二人特来告求。”祎曰:“所求何事?”晃曰:“吾闻魏王早晚受禅,将登大宝,公与王长史必高迁。望不相弃,曲赐提携,感德非浅!”祎拂袖而起。适从者奉茶至,便将茶泼于地上。晃佯惊曰:“德伟故人,何薄情也?”祎曰:“吾与汝交厚,为汝等是汉朝臣宰之后;今不思报本,欲辅造反之人,吾有何面目与汝为友!”耿纪曰:“奈天数如此,不得不为耳!”祎大怒。

耿纪、韦晃见祎果有忠义之心,乃以实情相告曰:“吾等本欲讨贼,来求足下。前言特相试耳。”祎曰:“吾累世汉臣,安能从贼!公等欲扶汉室,有何高见?”晃曰:“虽有报国之心,未有讨贼之计。”祎曰:“吾欲里应外合,杀了王必,夺其兵权,扶助銮舆。更结刘皇叔为外援,操贼可灭矣。”二人闻之,抚掌称善。祎曰:“我有心腹二人,与操贼有杀父之仇,现居城外,可用为羽翼。”耿纪问是何人。祎曰:“太医吉平之子:长名吉邈,字文然;次名吉穆,字思然。操昔日为董承衣带诏事,曾杀其父;二子逃窜远乡,得免于难。今已潜归许都,若使相助讨贼,无有不从。”耿纪、韦晃大喜。金祎即使人密唤二吉。须臾,二人至。祎具言其事。二人感愤流泪,怨气冲天,誓杀国贼。金祎曰:“正月十五日夜间,城中大张灯火,庆赏元宵。耿少府、韦司直,你二人各领家僮,杀到王必营前;只看营中火起,分两路杀入;杀了王必,径跟我入内,请天子登五凤楼,召百官面谕讨贼。吉文然兄弟于城外杀入,放火为号,各要扬声,叫百姓诛杀国贼,截住城内救军;待天子降诏,招安已定,便进兵杀投邺郡擒曹操,即发使赍诏召刘皇叔。今日约定,至期二更举事。勿似董承自取其祸。”五人对天说誓,歃血为盟,各自归家,整顿军马器械,临期而行。且说耿纪、韦晃二人,各有家僮三四百,预备器械。吉邈兄弟,亦聚三百人口,只推围猎,安排已定。金祎先期来见王必,言:“方今海宇稍安,魏王威震天下;今值元宵令节,不可不放灯火以示太平气象。”王必然其言,告谕城内居民,尽张灯结彩,庆赏佳节。至正月十五夜,天色晴霁,星月交辉,六街三市,竞放花灯。真个金吾不禁,玉漏无催!王必与御林诸将在营中饮宴。二更以后,忽闻营中呐喊,人报营后火起。王必慌忙出帐看时,只见火光乱滚;又闻喊杀连天,知是营中有变,急上马出南门,正遇耿纪,一箭射中肩膊,几乎坠马,遂望西门而走。背后有军赶来。王必着忙,弃马步行。至金祎门首,慌叩其门。原来金祎一面使人于营中放火,一面亲领家僮随后助战,只留妇女在家。时家中闻王必叩门之声,只道金祎归来。祎妻从隔门便问曰:“王必那厮杀了么?”王必大惊,方悟金祎同谋,径投曹休家,报知金祎、耿纪等同谋反。休急披挂上马,引千余人在城中拒敌。城内四下火起,烧着五凤楼,帝避于深宫。曹氏心腹爪牙,死据宫门。城中但闻人叫:“杀尽曹贼,以扶汉室!”

原来夏侯惇奉曹操命,巡警许昌,领三万军,离城五里屯扎;是夜,遥望见城中火起,便领大军前来,围住许都,使一枝军入城接应曹休。直混杀至天明。耿纪、韦晃等无人相助。人报金祎、二吉皆被杀死。耿纪、韦晃夺路杀出城门,正遇夏侯惇大军围住,活捉去了。手下百余人皆被杀。夏侯惇入城,救灭遗火,尽收五人老小宗族,使人飞报曹操。操传令教将耿、韦二人,及五家宗族老小,皆斩于市,并将在朝大小百官,尽行拿解邺郡,听候发落。夏侯惇押耿、韦二人至市曹。耿纪厉声大叫曰:“曹阿瞒!吾生不能杀汝,死当作厉鬼以击贼!”刽子以刀搠其口,流血满地,大骂不绝而死。韦晃以面颊顿地曰:“可恨!可恨!”咬牙皆碎而死。后人有诗赞曰:“耿纪精忠韦晃贤,各持空手欲扶天。谁知汉祚相将尽,恨满心胸丧九泉。”夏侯惇尽杀五家老小宗族,将百官解赴邺郡。曹操于教场立红旗于左、白旗于右,下令曰:“耿纪、韦晃等造反,放火焚许都,汝等亦有出救火者,亦有闭门不出者。如曾救火者,可立于红旗下;如不曾救火者,可立于白旗下。”众官自思救火者必无罪,于是多奔红旗之下。三停内只有一停立于白旗下。操教尽拿立于红旗下者。众官各言无罪。操曰:“汝当时之心,非是救火,实欲助贼耳。”尽命牵出漳河边斩之,死者三百余员。其立于白旗下者,尽皆赏赐,仍令还许都。时王必已被箭疮发而死,操命厚葬之。令曹休总督御林军马,钟繇为相国,华歆为御史大夫。遂定侯爵六等十八级,关中侯爵十七级,皆金印紫绶;又置关内外侯十六级,银印龟纽墨绶;五大夫十五级,铜印环纽墨绶。定爵封官,朝廷又换一班人物。曹操方悟管辂火灾之说,遂重赏辂。辂不受。

却说曹洪领兵到汉中,令张郃、夏侯渊各据险要。曹洪亲自进兵拒敌。时张飞自与雷铜守把巴西。马超兵至下辨,令吴兰为先锋,领军哨出,正与曹洪军相遇。吴兰欲退,牙将任夔曰:“贼兵初至,若不先挫其锐气,何颜见孟起乎?”于是骤马挺枪搦曹洪战。洪自提刀跃马而出。交锋三合,斩夔于马下,乘势掩杀。吴兰大败,回见马超。超责之曰:“汝不得吾令,何故轻敌致败?”吴兰曰:“任夔不听吾言,故有此败?”马超曰:“可紧守隘口,勿与交锋。”一面申报成都,听候行止。曹洪见马超连日不出,恐有诈谋,引军退回南郑。张郃来见曹洪,问曰:“将军既已斩将,如何退兵?”洪曰:“吾见马超不出,恐有别谋。且我在邺都,闻神卜管辂有言:当于此地折一员大将。吾疑此言,故不敢轻进。”张郃大笑曰:“将军行兵半生,今奈何信卜者之言而惑其心哉!郃虽不才,愿以本部兵取巴西。若得巴西,蜀郡易耳。”洪曰:“巴西守将张飞,非比等闲,不可轻敌。”张郃曰:“人皆怕张飞,吾视之如小儿耳!此去必擒之!”洪曰:“倘有疏失,若何?”郃曰:“甘当军令。”洪勒了文状,张郃进兵。正是:自古骄兵多致败,从来轻敌少成功。未知胜负如何,且看下文分解。