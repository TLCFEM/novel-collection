\chapter{宴长江曹操赋诗~锁战船北军用武}

却说庞统闻言,吃了一惊,急回视其人,原来却是徐庶。统见是故人,心下方定。回顾左右无人,乃曰:“你若说破我计,可惜江南八十一州百姓,皆是你送了也!”庶笑曰:“此间八十三万人马,性命如何?”统曰:“元直真欲破我计耶?”庶曰:“吾感刘皇叔厚恩,未尝忘报。曹操送死吾母,吾已说过终身不设一谋,今安肯破兄良策?只是我亦随军在此,兵败之后,玉石不分,岂能免难?君当教我脱身之术,我即缄口远避矣。”统笑曰:“元直如此高见远识,谅此有何难哉!”庶曰:“愿先生赐教。”统去徐庶耳边略说数句。庶大喜,拜谢。庞统别却徐庶,下船自回江东。

且说徐庶当晚密使近人去各寨中暗布谣言。次日,寨中三三五五,交头接耳而说。早有探事人报知曹操,说:“军中传言西凉州韩遂、马腾谋反,杀奔许都来。”操大惊,急聚众谋士商议曰:“吾引兵南征,心中所忧者,韩遂、马腾耳。军中谣言,虽未辨虚实,然不可不防。”言未毕,徐庶进曰:“庶蒙丞相收录,恨无寸功报效。请得三千人马,星夜往散关把住隘口;如有紧急,再行告报。”操喜曰:“若得元直去,吾无忧矣!散关之上,亦有军兵,公统领之。目下拨三千马步军,命臧霸为先锋,星夜前去,不可稽迟。”徐庶辞了曹操,与臧霸便行。此便是庞统救徐庶之计。后人有诗曰:“曹操征南日日忧,马腾韩遂起戈矛。凤雏一语教徐庶,正似游鱼脱钓钩。”曹操自遣徐庶去后,心中稍安,遂上马先看沿江旱寨,次看水寨。乘大船一只于中央,上建帅字旗号,两傍皆列水寨,船上埋伏弓弩千张。操居于上。时建安十三年冬十一月十五日,天气晴明,平风静浪。操令:“置酒设乐于大船之上,吾今夕欲会诸将。”天色向晚,东山月上,皎皎如同白日。长江一带,如横素练。操坐大船之上,左右侍御者数百人,皆锦衣绣袄,荷戈执戟。文武众官,各依次而坐。操见南屏山色如画,东视柴桑之境,西观夏口之江,南望樊山,北觑乌林,四顾空阔,心中欢喜,谓众官曰:“吾自起义兵以来,与国家除凶去害,誓愿扫清四海,削平天下;所未得者江南也。今吾有百万雄师,更赖诸公用命,何患不成功耶!收服江南之后,天下无事,与诸公共享富贵,以乐太平。”文武皆起谢曰:“愿得早奏凯歌!我等终身皆赖丞相福荫。”操大喜,命左右行酒。饮至半夜,操酒酣,遥指南岸曰:“周瑜、鲁肃,不识天时!今幸有投降之人,为彼心腹之患,此天助吾也。”荀攸曰:“丞相勿言,恐有泄漏。”操大笑曰:“座上诸公,与近侍左右,皆吾心腹之人也,言之何碍!”又指夏口曰:“刘备、诸葛亮,汝不料蝼蚁之力,欲撼泰山,何其愚耶!”顾谓诸将曰:“吾今年五十四岁矣,如得江南,窃有所喜。昔日乔公与吾至契,吾知其二女皆有国色。后不料为孙策、周瑜所娶。吾今新构铜雀台于漳水之上,如得江南,当娶二乔,置之台上,以娱暮年,吾愿足矣!”言罢大笑。唐人杜牧之有诗曰:“折戟沉沙铁未消,自将磨洗认前朝。东风不与周郎便,铜雀春深锁二乔。”曹操正笑谈间,忽闻鸦声望南飞鸣而去。操问曰;“此鸦缘何夜鸣?”左右答曰:“鸦见月明,疑是天晓,故离树而鸣也。”操又大笑。时操已醉,乃取槊立于船头上,以酒奠于江中,满饮三爵,横槊谓诸将曰:“我持此槊,破黄巾、擒吕布、灭袁术、收袁绍,深入塞北,直抵辽东,纵横天下:颇不负大丈夫之志也。今对此景,甚有慷慨。吾当作歌,汝等和之。”歌曰:“对酒当歌,人生几何:譬如朝露,去日苦多。慨当以慷,忧思难忘;何以解忧,惟有杜康。青青子衿,悠悠我心;但为君故,沉吟至今。呦呦鹿鸣,食野之苹;我有嘉宾,鼓瑟吹笙。皎皎如月,何时可辍?忧从中来,不可断绝!越陌度阡,枉用相存;契阔谈宴,心念旧恩。月明星稀,乌鹊南飞;绕树三匝,无枝可依。山不厌高,水不厌深:周公吐哺,天下归心。”歌罢,众和之,共皆欢笑。忽座间一人进曰:“大军相当之际,将士用命之时,丞相何故出此不吉之言?”操视之,乃扬州刺史,沛国相人,姓刘,名馥,字元颖。馥起自合淝,创立州治,聚逃散之民,立学校,广屯田,兴治教,久事曹操,多立功绩。当下操横槊问曰:“吾言有何不吉?”馥曰:“月明星稀,乌鹊南飞;绕树三匝,无枝可依。此不吉之言也。”操大怒曰:“汝安敢败吾兴!”手起一槊,刺死刘馥。众皆惊骇。遂罢宴。次日,操酒醒,懊恨不已。馥子刘熙,告请父尸归葬。操泣曰:“吾昨因醉误伤汝父,悔之无及。可以三公厚礼葬之。”又拨军士护送灵柩,即日回葬。

次日,水军都督毛玠、于禁诣帐下,请曰:“大小船只,俱已配搭连锁停当。旌旗战具,一一齐备。请丞相调遣,克日进兵。”操至水军中央大战船上坐定,唤集诸将,各各听令。水旱二军,俱分五色旗号:水军中央黄旗毛玠、于禁,前军红旗张郃,后军皂旗吕虔,左军青旗文聘,右军白旗吕通;马步前军红旗徐晃,后军皂旗李典,左军青旗乐进,右军白旗夏侯渊。水陆路都接应使:夏侯惇、曹洪;护卫往来监战使:许褚、张辽。其余骁将,各依队伍。令毕,水军寨中发擂三通,各队伍战船,分门而出。是日西北风骤起,各船拽起风帆,冲波激浪,稳如平地。北军在船上,踊跃施勇,刺枪使刀。前后左右各军,旗幡不杂。又有小船五十余只,往来巡警催督。操立于将台之上,观看调练,心中大喜,以为必胜之法;教且收住帆幔,各依次序回寨。

操升帐谓众谋士曰:“若非天命助吾,安得凤雏妙计?铁索连舟,果然渡江如履平地。”程昱曰:“船皆连锁,固是平稳;但彼若用火攻,难以回避。不可不防。”操大笑曰:“程仲德虽有远虑,却还有见不到处。”荀攸曰:“仲德之言甚是。丞相何故笑之?”操曰:“凡用火攻,必藉风力。方今隆冬之际,但有西风北风,安有东风南风耶?吾居于西北之上,彼兵皆在南岸,彼若用火,是烧自己之兵也,吾何惧哉?若是十月小春之时,吾早已提备矣。”诸将皆拜伏曰:“丞相高见,众人不及。”操顾诸将曰:“青、徐、燕、代之众,不惯乘舟。今非此计,安能涉大江之险!”只见班部中二将挺身出曰:“小将虽幽、燕之人,也能乘舟。今愿借巡船二十只,直至江口,夺旗鼓而还,以显北军亦能乘舟也。”操视之,乃袁绍手下旧将焦触、张南也。操曰:“汝等皆生长北方,恐乘舟不便。江南之兵,往来水上,习练精熟,汝勿轻以性命为儿戏也。”焦触、张南大叫曰:“如其不胜,甘受军法!”操曰:“战船尽已连锁,惟有小舟。每舟可容二十人,只恐未便接战。”触曰:“若用大船,何足为奇?乞付小舟二十余只,某与张南各引一半,只今日直抵江南水寨,须要夺旗斩将而还。”操曰:“吾与汝二十只船,差拨精锐军五百人,皆长枪硬弩。到来日天明,将大寨船出到江面上,远为之势。更差文聘亦领三十只巡船接应汝回。”焦触、张南欣喜而退。

次日,四更造饭,五更结束已定,早听得水寨中擂鼓鸣金。船皆出寨,分布水面,长江一带,青红旗号交杂。焦触、张南领哨船二十只,穿寨而出,望江南进发。却说南岸隔夜听得鼓声喧震,遥望曹操调练水军,探事人报知周瑜。瑜往山顶观之,操军已收回。次日,忽又闻鼓声震天,军士急登高观望,见有小船冲波而来,飞报中军。周瑜问帐下:“谁敢先出?”韩当、周泰二人齐出曰:“某当权为先锋破敌。”瑜喜,传令各寨严加守御,不可轻动。韩当、周泰各引哨船五只,分左右而出。却说焦触、张南凭一勇之气,飞棹小船而来。韩当独披掩心,手执长枪,立于船头。焦触船先到,便命军士乱箭望韩当船上射来。当用牌遮隔。焦触捻长枪与韩当交锋。当手起一枪,刺死焦触。张南随后大叫赶来。隔斜里周泰船出。张南挺枪立于船头,两边弓矢乱射。周泰一臂挽牌,一手提刀,两船相离七八尺,泰即飞身一跃,直跃过张南船上,手起刀落,砍张南于水中,乱杀驾舟军士。众船飞棹急回。韩当、周泰催船追赶,到半江中,恰与文聘船相迎。两边便摆定船厮杀。却说周瑜引众将立于山顶,遥望江北水面艨艟战船,排合江上,旗帜号带,皆有次序。回看文聘与韩当、周泰相持,韩当、周泰奋力攻击,文聘抵敌不住,回船而走,韩、周二人,急催船追赶。周瑜恐二人深入重地,便将白旗招飐,令众鸣金。二人乃挥棹而回。周瑜于山顶看隔江战船,尽入水寨。瑜顾谓众将曰:“江北战船如芦苇之密,操又多谋,当用何计以破之?”众未及对,忽见曹军寨中,被风吹折中央黄旗,飘入江中。瑜大笑曰:“此不祥之兆也!”正观之际,忽狂风大作,江中波涛拍岸。一阵风过,刮起旗角于周瑜脸上拂过。瑜猛然想起一事在心,大叫一声,往后便倒,口吐鲜血。诸将急救起时,却早不省人事。正是:一时忽笑又忽叫,难使南军破北军。毕竟周瑜性命如何,且看下文分解。