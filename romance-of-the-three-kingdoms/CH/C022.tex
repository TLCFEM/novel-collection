\chapter{袁曹各起马步三军~关张共擒王刘二将}

却说陈登献计于玄德曰:“曹操所惧者袁绍。绍虎踞冀、青、幽、并诸郡,带甲百万,文官武将极多,今何不写书遣人到彼求救?”玄德曰:“绍向与我未通往来,今又新破其弟,安肯相助?”登曰:“此间有一人与袁绍三世通家,若得其一书致绍,绍必来相助。”玄德问何人。登曰:“此人乃公平日所折节敬礼者,何故忘之?”玄德猛省曰:“莫非郑康成先生乎?”登笑曰:“然也。”原来郑康成名玄,好学多才,尝受业于马融。融每当讲学,必设绛帐,前聚生徒,后陈声妓,侍女环列左右。玄听讲三年,目不邪视,融甚奇之。及学成而归。融叹曰:“得我学之秘者,惟郑玄一人耳!”玄家中侍婢俱通毛诗。一婢尝忤玄意,玄命长跪阶前。一婢戏谓之曰:“胡为乎泥中?”此婢应声曰:“薄言往愬,逢彼之怒。”其风雅如此。桓帝朝,玄官至尚书;后因十常侍之乱,弃官归田,居于徐州。玄德在涿郡时,已曾师事之;及为徐州牧,时时造庐请教,敬礼特甚。当下玄德想出此人,大喜,便同陈登亲至郑玄家中,求其作书。玄慨然依允,写书一封,付与玄德。玄德便差孙乾星夜赍往袁绍处投递。绍览毕,自忖曰:“玄德攻灭吾弟,本不当相助;但重以郑尚书之命,不得不往救之。”遂聚文武官,商议兴兵伐曹操。谋士田丰曰:“兵起连年,百姓疲弊,仓廪无积,不可复兴大军。宜先遣人献捷天子,若不得通,乃表称曹操隔我王路,然后提兵屯黎阳;更于河内增益舟楫,缮置军器,分遣精兵,屯扎边鄙。三年之中,大事可定也。”谋士审配曰:“不然。以明公之神武,抚河朔之强盛,兴兵讨曹贼,易如反掌,何必迁延日月?”谋士沮授曰:“制胜之策,不在强盛。曹操法令既行,士卒精练,比公孙瓒坐受困者不同。今弃献捷良策,而兴无名之兵,窃为明公不取。”谋士郭图曰:“非也。兵加曹操,岂曰无名?公正当及时早定大业。愿从郑尚书之言,与刘备共仗大义,剿灭曹贼,上合天意,下合民情,实为幸甚!”四人争论未定,绍躇踌不决。忽许攸、荀谌自外而入。绍曰:“二人多有见识,且看如何主张。”二人施礼毕,绍曰:“郑尚书有书来,令我起兵助刘备,攻曹操。起兵是乎?不起兵是乎?”二人齐声应曰:“明公以众克寡,以强攻弱,讨汉贼以扶王室:起兵是也。”绍曰:“二人所见,正合我心。”便商议兴兵。先令孙乾回授郑玄,并约玄德准备接应;一面令审配、逢纪为统军,田丰、荀谌、许攸为谋士,颜良、文丑为将军,起马军十五万,步兵十五万,共精兵三十万,望黎阳进发。分拨已定,郭图进曰:“以明公大义伐操,必须数操之恶,驰檄各郡,声罪致讨,然后名正言顺。”绍从之,遂令书记陈琳草檄。琳字孔璋,素有才名;灵帝时为主簿,因谏何进不听,复遭董卓之乱,避难冀州,绍用为记室。当下领命草檄,援笔立就。其文曰:

盖闻明主图危以制变,忠臣虑难以立权。是以有非常之人,然后有非常之事;有非常之事,然后立非常之功。夫非常者,固非常人所拟也。曩者,强秦弱主,赵高执柄,专制朝权,威福由己;时人迫胁,莫敢正言;终有望夷之败,祖宗焚灭,污辱至今,永为世鉴。及臻吕后季年,产禄专政,内兼二军,外统赵梁;擅断万机,决事省禁;下陵上替,海内寒心。于是绛侯朱虚兴兵奋怒,诛夷逆暴,尊立太宗,故能王道兴隆,光明显融:此则大臣立权之明表也。司空曹操:祖父中常侍腾,与左棺、徐璜并作妖孽,饕餮放横,伤化虐民;父嵩,乞匄携养,因赃假位,舆金辇璧,输货权门,窃盗鼎司,倾覆重器。操赘阉遗丑,本无懿德,[犭票]狡锋协,好乱乐祸。幕府董统鹰扬,扫除凶逆;续遇董卓,侵官暴国。于是提剑挥鼓,发命东夏,收罗英雄,弃瑕取用;故遂与操同谘合谋,授以裨师,谓其鹰犬之才,爪牙可任。至乃愚佻短略,轻进易退,伤夷折衄,数丧师徒;幕府辄复分兵命锐,修完补辑,表行东郡,领兖州刺史,被以虎文,奖蹙威柄,冀获秦师一克之报。而操遂承资跋扈,恣行凶忒,割剥元元,残贤害善。故九江太守边让,英才俊伟,天下知名;直言正色,论不阿谄;身首被枭悬之诛,妻孥受灰灭之咎。自是士林愤痛,民怨弥重;一夫奋臂,举州同声。故躬破于徐方,地夺于吕布;彷徨东裔,蹈据无所。幕府惟强干弱枝之义,且不登叛人之党,故复援旌擐甲,席卷起征,金鼓响振,布众奔沮;拯其死亡之患,复其方伯之位:则幕府无德于兖土之民,而有大造于操也。后会銮驾返旆,群虏寇攻。时冀州方有北鄙之警,匪遑离局;故使从事中郎徐勋,就发遣操,使缮修郊庙,翊卫幼主。操便放志:专行胁迁,当御省禁;卑侮王室,败法乱纪;坐领三台,专制朝政;爵赏由心,弄戮在口;所爱光五宗,所恶灭三族;群谈者受显诛,腹议者蒙隐戮;百僚钳口,道路以目;尚书记朝会,公卿充员品而已。故太尉杨彪,典历二司,享国极位。操因缘眦睚,被以非罪;榜楚参并,五毒备至;触情任忒,不顾宪纲。又议郎赵彦,忠谏直言,义有可纳,是以圣朝含听,改容加饰。操欲迷夺时明,杜绝言路,擅收立杀,不俟报国。又梁孝王,先帝母昆,坟陵尊显;桑梓松柏,犹宜肃恭。而操帅将吏士,亲临发掘,破棺裸尸,掠取金宝。至令圣朝流涕,士民伤怀!操又特置发丘中郎将、摸金校尉,所过隳突,无骸不露。身处三公之位,而行桀虏之态,污国害民,毒施人鬼!加其细致惨苛,科防互设;罾缴充蹊,坑阱塞路;举手挂网罗,动足触机陷:是以兖、豫有无聊之民,帝都有吁嗟之怨。历观载籍,无道之臣,贪残酷烈,于操为甚!幕府方诘外奸,未及整训;加绪含容,冀可弥缝。而操豺狼野心,潜包祸谋,乃欲摧挠栋梁,孤弱汉室,除灭忠正,专为袅雄。往者伐鼓北征公孙瓒,强寇桀逆,拒围一年。操因其未破,阴交书命,外助王师,内相掩袭。会其行人发露,瓒亦枭夷,故使锋芒挫缩,厥图不果。今乃屯据敷仓,阻河为固,欲以螳螂之斧,御隆车之隧。幕府奉汉威灵,折冲宇宙;长戟百万,胡骑千群;奋中黄育获之士,骋良弓劲弩之势;并州越太行,青州涉济漯;大军泛黄河而角其前,荆州下宛叶而掎其后:雷震虎步,若举炎火以焫飞蓬,覆沧海以沃[火票]炭,有何不灭者哉?又操军吏士,其可战者,皆出自幽冀,或故营部曲,咸怨旷思归,流涕北顾。其余兖豫之民,及吕布张杨之余众,覆亡迫胁,权时苟从;各被创夷,人为仇敌。若回旆方徂,登高冈而击鼓吹,扬素挥以启降路,必土崩瓦解,不俟血刃。方今汉室陵迟,纲维弛绝;圣朝无一介之辅,股肱无折冲之势。方畿之内,简练之臣,皆垂头□翼,莫所凭恃;虽有忠义之佐,胁于暴虐之臣,焉能展其节?又操持部曲精兵七百,围守宫阙,外托宿卫,内实拘执。惧其篡逆之萌,因斯而作。此乃忠臣肝脑涂地之秋,烈士立功之会,可不勖哉!操又矫命称制,遣使发兵。恐边远州郡,过听给与,违众旅叛,举以丧名,为天下笑,则明哲不取也。即日幽并青冀四州并进。书到荆州,便勒现兵,与建忠将军协同声势。州郡各整义兵,罗落境界,举武扬威,并匡社稷:则非常之功于是乎著。其得操首者,封五千户侯,赏钱五千万。部曲偏裨将校诸吏降者,勿有所问。广宜恩信,班扬符赏,布告天下,咸使知圣朝有拘迫之难。如律令!

绍览檄大喜,即命使将此檄遍行州郡,并于各处关津隘口张挂。檄文传至许都,时曹操方患头风,卧病在床。左右将此檄传进,操见之,毛骨悚然,出了一身冷汗,不觉头风顿愈,从床上一跃而起,顾谓曹洪曰:“此微何人所作?”洪曰:“闻是陈琳之笔。”操笑曰:“有文事者,必须以武略济之。陈琳文事虽佳,其如袁绍武略之不足何!”遂聚众谋士商议迎敌。孔融闻之,来见操曰:“袁绍势大,不可与战,只可与和。”荀彧曰:“袁绍无用之人,何必议和?”融曰:“袁绍士广民强。其部下如许攸、郭图、审配、逢纪皆智谋之士;田丰、沮授皆忠臣也;颜良、文丑勇冠三军;其余高览、张郃、淳于琼等俱世之名将。——何谓绍为无用之人乎?”彧笑曰:“绍兵多而不整。田丰刚而犯上,许攸贪而不智,审配专而无谋,逢纪果而无用:此数人者,势不相容,必生内变,颜良、文丑,匹夫之勇,一战可擒。其余碌碌等辈,纵有百万,何足道哉!”孔融默然。操大笑曰:“皆不出荀文若之料。”遂唤前军刘岱、后军王忠引军五万,打着丞相旗号,去徐州攻刘备。原来刘岱旧为兖州刺史;及操取兖州,岱降于操,操用为偏将,故今差他与王忠一同领兵。操却自引大军二十万,进黎阳,拒袁绍。程昱曰:“恐刘岱、王忠不称其使。”操曰:“吾亦知非刘备敌手,权且虚张声势。”分付:“不可轻进。待我破绍,再勒兵破备。”刘岱、王忠领兵去了。

曹操自引兵至黎阳。两军隔八十里,各自深沟高垒,相持不战。自八月守至十月。原来许攸不乐审配领兵,沮授又恨绍不用其谋,各不相和,不图进取。袁绍心怀疑惑,不思进兵,操乃唤吕布手下降将臧霸守把青、徐;于禁、李典屯兵河上;曹仁总督大军,屯于官渡,操自引一军,竟回许都。

且说刘岱、王忠引军五万,离徐州一百里下寨。中军虚打“曹丞相”旗号,未敢进兵,只打听河北消息。这里玄德也不知曹操虚实,未敢擅动,亦只探听河北。忽曹操差人催刘岱、王忠进战。二人在寨中商议。岱曰:“丞相催促攻城,你可先去。”王忠曰:“丞相先差你。”岱曰:“我是主将,如何先去?”忠曰:“我和你同引兵去。”岱曰:“我与你拈阄,拈着的便去。”王忠拈着“先”字,只得分一半军马,来攻徐州。

玄德听知军马到来,请陈登商议曰:“袁本初虽屯兵黎阳,奈谋臣不和,尚未进取。曹操不知在何处。闻黎阳军中,无操旗号,如何这里却反有他旗号?”登曰:“操诡计百出,必以河北为重,亲自监督,却故意不建旗号,乃于此处虚张旗号:吾意操必不在此。”玄德曰:“两弟谁可探听虚实?”张飞曰:“小弟愿往。”玄德曰:“汝为人躁暴,不可去。”飞曰:“便是有曹操也拿将来!”云长曰:“待弟往观其动静。”玄德曰:“云长若去,我却放心。”于是云长引三千人马出徐州来。

时值初冬,阴云布合,雪花乱飘,军马皆冒雪布阵。云长骤马提刀而出,大叫王忠打话。忠出曰:“丞相到此,缘何不降?”云长曰:“请丞相出阵,我自有话说。”忠曰:“丞相岂肯轻见你!”云长大怒,骤马向前。王忠挺枪来迎。两马相交,云长拨马便走。王忠赶来。转过山坡,云长回马,大叫一声,舞刀直取。王忠拦截不住,恰待骤马奔逃,云长左手倒提宝刀,右手揪住王忠勒甲绦,拖下鞍鞒,横担于马上,回本阵来。王忠军四散奔走。

云长押解王忠,回徐州见玄德。玄德问:“尔乃何人?现居何职?敢诈称曹丞相!”忠曰:“焉敢有诈。奉命教我虚张声势,以为疑兵。丞相实不在此。”玄德教付衣服酒食,且暂监下,待捉了刘岱,再作商议。云长曰:“某知兄有和解之意,故生擒将来。”玄德曰:“吾恐翼德躁暴,杀了王忠,故不教去。此等人杀之无益,留之可为解和之地。”张飞曰:“二哥捉了王忠,我去生擒刘岱来!”玄德曰:“刘岱昔为兖州刺史,虎牢关伐董卓时,也是一镇诸侯,今日为前军,不可轻敌。”飞曰:“量此辈何足道哉!我也似二哥生擒将来便了。”玄德曰:“只恐坏了他性命,误我大事。”飞曰:“如杀了,我偿他命!”玄德遂与军三千。飞引兵前进。

却说刘岱知王忠被擒,坚守不出。张飞每日在寨前叫骂,岱听知是张飞,越不敢出。飞守了数日,见岱不出,心生一计:传令今夜二更去劫寨;日间却在帐中饮酒诈醉,寻军士罪过,打了一顿,缚在营中,曰:“待我今夜出兵时,将来祭旗!”却暗使左右纵之去。军士得脱,偷走出营,径往刘岱营中来报劫寨之事。刘岱见降卒身受重伤,遂听其说,虚扎空寨,伏兵在外。是夜张飞却分兵三路,中间使三十余人,劫寨放火;却教两路军抄出他寨后,看火起为号,夹击之。三更时分,张飞自引精兵,先断刘岱后路;中路三十余人,抢入寨中放火。刘岱伏兵恰待杀入,张飞两路兵齐出。岱军自乱,正不知飞兵多少,各自溃散。刘岱引一队残军,夺路而走,正撞见张飞,狭路相逢,急难回避,交马只一合,早被张飞生擒过去。余众皆降。飞使人先报入徐州。玄德闻之,谓云长曰:“翼德自来粗莽,今亦用智,吾无忧矣!”乃亲自出郭迎之。飞曰:“哥哥道我躁暴,今日如何?玄德曰:“不用言语相激,如何肯使机谋!”飞大笑。

玄德见缚刘岱过来,慌下马解其缚曰:“小弟张飞误有冒渎,望乞恕罪。”遂迎入徐州,放出王忠,一同管待。玄德曰:“前因车胄欲害备,故不得不杀之。丞相错疑备反,遣二将军前来问罪。备受丞相大恩,正思报效,安敢反耶?二将军至许都,望善言为备分诉,备之幸也。”刘岱、王忠曰:“深荷使君不杀之恩,当于丞相处方便,以某两家老小保使君。”玄德称谢。次日尽还原领军马,送出郭外。

刘岱、王忠行不上十余里,一声鼓响,张飞拦路大喝曰:“我哥哥忒没分晓!捉住贼将如何又放了?”?得刘岱、王忠在马上发颤。张飞睁眼挺枪赶来,背后一人飞马大叫:“不得无礼!”视之,乃云长也。刘岱、王忠方才放心。云长曰:“既兄长放了,吾弟如何不遵法令?”飞曰:“今番放了,下次又来。”云长曰:“待他再来,杀之未迟。”刘岱、王忠连声告退曰:“便丞相诛我三族,也不来了。望将军宽恕。”飞曰:“便是曹操自来,也杀他片甲不回!今番权且寄下两颗头!”刘岱、王忠抱头鼠窜而去。云长、翼德回见玄德曰:“曹操必然复来。”孙乾谓玄德曰:“徐州受敌之地,不可久居;不若分兵屯小沛,守邳城,为掎角之势,以防曹操。”玄德用其言,令云长守下邳;甘、糜二夫人亦于下邳安置。甘夫人乃小沛人也,糜夫人乃糜竺之妹也。孙乾、简雍、糜竺、糜芳守徐州。玄德与张飞屯小沛。刘岱、王忠回见曹操,具言刘备不反之事。操怒骂:“辱国之徒,留你何用!”喝令左右推出斩之。正是:犬豕何堪共虎斗,鱼虾空自与龙争。不知二人性命如何,且听下文分解。