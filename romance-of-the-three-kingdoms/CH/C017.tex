\chapter{袁公路大起七军~曹孟德会合三将}

却说袁术在淮南,地广粮多,又有孙策所质玉玺,遂思僭称帝号;大会群下议曰:“昔汉高祖不过泗上一亭长,而有天下;今历年四百,气数已尽,海内鼎沸。吾家四世三公,百姓所归;吾效应天顺人,正位九五。尔众人以为何如?”主簿阁象曰:“不可。昔周后稷积德累功,至于文王,三分天下有其二,犹以服事殷。明公家世虽贵,未若有周之盛;汉室虽微,未若殷纣之暴也。此事决不可行。”术怒曰:“吾袁姓出于陈。陈乃大舜之后。以土承火,正应其运。又谶云:代汉者,当涂高也。吾字公路,正应其谶。又有传国玉玺。若不为君,背天道也。吾意已决,多言者斩!”遂建号仲氏,立台省等官,乘龙凤辇,祀南北郊,立冯方女为后,立子为东宫。因命使催取吕布之女为东宫妃,却闻布已将韩胤解赴许都,为曹操所斩,乃大怒;遂拜张勋为大将军,统领大军二十余万,分七路征徐州:第一路大将张勋居中,第二路上将桥蕤居左,第三路上将陈纪居右,第四路副将雷薄居左,第五路副将陈兰居右,第六路降将韩暹居左,第七路降将杨奉居右。各领部下健将,克日起行。命兖州刺史金尚为太尉,监运七路钱粮。尚不从,术杀之。以纪灵为七路都救应使。术自引军三万,使李丰、梁刚、乐就为催进使,接应七路之兵。

吕布使人探听得张勋一军从大路径取徐州,桥蕤一军取小沛,陈纪一军取沂都,雷薄一军取琅琊,陈兰一军取碣石,韩暹一军取下邳,杨奉一军取浚山:七路军马,日行五十里,于路劫掠将来。乃急召众谋士商议,陈宫与陈珪父子俱至。陈宫曰:“徐州之祸,乃陈珪父子所招,媚朝廷以求爵禄,今日移祸于将军。可斩二人之头献袁术,其军自退。”布听其言,即命擒下陈珪、陈登。陈登大笑曰:“何如是之懦也?吾观七路之兵,如七堆腐草,何足介意!”布曰:“汝若有计破敌、免汝死罪。”陈登曰:“将军若用老夫之言,徐州可保无虞。”布曰:“试言之。”登曰:“术兵虽众,皆乌合之师,素不亲信;我以正兵守之,出奇兵胜之,无不成功。更有一计,不止保安徐州,并可生擒袁术。”布曰:“计将安出?”登曰:“韩暹、杨奉乃汉旧臣,因惧曹操而走,无家可依,暂归袁术;术必轻之,彼亦不乐为术用。若凭尺书结为内应,更连刘备为外合,必擒袁术矣。”布曰:“汝须亲到韩暹、杨奉处下书。”陈登允诺。布乃发表上许都,并致书与豫州,然后令陈登引数骑,先于下邳道上候韩暹。退引兵至,下寨毕,登入见。暹问曰:“汝乃吕布之人,来此何干?”登笑曰:“某为大汉公卿,何谓吕布之人?若将军者,向为汉臣,今乃为叛贼之臣,使昔日关中保驾之功,化为乌有,窃为将军不取也。且袁术性最多疑,将军后必为其所害。今不早图,悔之无及!”暹叹曰:“吾欲归汉,恨无门耳。”登乃出布书。暹览书毕曰:“吾已知之。公先回。吾与杨将军反戈击之。但看火起为号,温侯以兵相应可也。”登辞暹,急回报吕布。

布乃分兵五路,高顺引一军进小沛,敌桥蕤;陈宫引一军进沂都,敌陈纪;张辽、臧霸引一军出琅琊,敌雷薄;宋宪、魏续引一军出碣石,敌陈兰;吕布自引一军出大道,敌张勋。各领军一万,余者守城。吕布出城三十里下寨。张勋军到,料敌吕布不过,且退二十里屯住,待四下兵接应。

是夜二更时分,韩暹、杨奉分兵到处放火,接应吕家军入寨。勋军大乱。吕布乘势掩杀,张勋败走。吕布赶到天明,正撞纪灵接应。两军相迎,恰待交锋,韩暹、杨奉两路杀来。纪灵大败而走,吕布引兵追杀。山背后一彪军到,门旗开处,只见一队军马,打龙凤日月旗幡,四斗五方旌帜,金瓜银斧,黄钺白旄,黄罗销金伞盖之下,袁术身披金甲,腕悬两刀,立于阵前,大骂:“吕布,背主家奴!”布怒,挺戟向前。术将李丰挺枪来迎;战不三合,被布刺伤其手,丰弃枪而走。吕布麾兵冲杀,术军大乱。吕布引军从后追赶,抢夺马匹衣甲无数。袁术引着败军,走不上数里,山背后一彪军出,截住去路。当先一将乃关云长也,大叫:“反贼!”还不受死!”袁术慌走,余众四散奔逃,被云长大杀了一阵。袁术收拾败军,奔回淮南去了。吕布得胜,邀请云长并杨奉、韩暹等一行人马到徐州,大排筵宴管待,军士都有犒赏。次日,云长辞归。布保韩暹为沂都牧、杨奉为琅琊牧,商议欲留二人在徐州。陈珪曰:“不可。韩、杨二人据山东,不出一年,则山东城敦皆属将军也。”布然之,遂送二将暂于沂都、琅琊二处屯扎,以候恩命。陈登私问父曰:“何不留二人在徐州,为杀吕布之根?”珪曰:“倘二人协助吕布,是反为虎添爪牙也。”登乃服父之高见。

却说袁术败回淮南,遣人往江东问孙策借兵报仇。策怒曰:“汝赖吾玉玺,僭称帝号,背反汉室,大逆不道!吾方欲加兵问罪,岂肯反助叛贼乎!”遂作书以绝之。使者赍书回见袁术。术看毕,怒曰:“黄口孺子,何敢乃尔!吾先伐之!”长史杨大将力谏方止。却说孙策自发书后,防袁术兵来,点军守住江口。忽曹操使至,拜策为会稽太守,令起兵征讨袁术。策乃商议。便欲起兵。长史张昭曰:“术虽新败,兵多粮足,未可轻敌。不如遗书曹操,劝他南征,吾为后应:两军相援,术军必败。万一有失,亦望操救援。”策从其言,遣使以此意达曹操。

却说曹操至许都,思幕典韦,立祀祭之;封其子典满为中郎,收养在府。忽报孙策遣使致书,操览书毕;又有人报袁术乏粮,劫掠陈留。欲乘虚攻之,遂兴兵南征。令曹仁守许都,其余皆从征:马步兵十七万,粮食辎重千余车。一面先发人会合孙策与刘备、吕布。兵至豫州界上,玄德早引兵来迎,操命请入营。相见毕,玄德献上首级二颗。操惊曰:“此是何人首级?”玄德曰:“此韩暹、杨奉之首级也。”操曰:“何以得之?”玄德曰:“吕布令二人权住沂都、琅琊两县。不意二人纵兵掠民,人人嗟怨。因此备乃说一宴,诈请议事:“饮酒间,掷盏为号,使关、张二弟杀之,尽降其众。今特来请罪。”操曰:“君为国家除害,正是大功,何言罪也?”遂厚劳玄德,合兵到徐州界。吕布出迎,操善言抚慰,封为左将军,许于还都之时,换给印绶。布大喜。操即分吕布一军在左,玄德一军在右,自统大军居中,令夏侯惇、于禁为先锋。

袁术知操兵至,令大将桥蕤引兵五万作先锋。两军会于寿春界口。桥蕤当先出马,与夏侯惇战不三合,被夏侯惇搠死。术军大败,奔走回城。忽报孙策发船攻江边西面,吕布引兵攻东面,刘备、关、张引兵攻南面,操自引兵十七万攻北面。术大惊,急聚众文武商议。杨大将曰:“寿春水旱连年,人皆缺食;今又动兵扰民,民既生怨,兵至难以拒敌。不如留军在寿春,不必与战;待彼兵粮尽,必然生变。陛下且统御林军渡淮,一者就熟,二者暂避其锐。”术用其言,留李丰、乐就、梁刚、陈纪四人分兵十万,坚守寿春;其余将卒并库藏金玉宝贝,尽数收拾过淮去了。

却说曹兵十七万,日费粮食浩大,诸郡又荒旱,接济不及。操催军速战,李丰等闭门不出。操军相拒月余,粮食将尽,致书于孙策,借得粮米十万斛,不敷支散。管粮官任峻部下仓官王垕人禀操曰:“兵多粮少,当如之何?”操曰:“可将小解散之,权且救一时之急。”垕曰:“兵士倘怨,如何?”操曰:“吾自有策。”垕依命,以小斛分散。操暗使人各寨探听,无不嗟怨,皆言丞相欺众。操乃密召王垕入曰:“吾欲问汝借一物,以压众心,汝必勿吝。”垕曰:“丞相欲用何物?”操曰:“欲借汝头以示众耳。”垕大惊曰:“某实无罪!”操曰:“吾亦知汝无罪,但不杀汝,军必变矣。汝死后,汝妻子吾自养之,汝勿虑也。”垕再欲言时,操早呼刀斧手推出门外,一刀斩讫,悬头高竿,出榜晓示曰:“王垕故行小斛,盗窃官粮,谨按军法。”于是众怨始解。

次日,操传令各营将领:“如三日内不并力破城,皆斩!”操亲自至城下,督诸军搬土运石,填壕塞堑。城上矢石如雨,有两员裨将畏避而回,操掣剑亲斩于城下,遂自下马接土填坑。于是大小将士无不向前,军威大振。城上抵敌不住,曹兵争先上城,斩关落锁,大队拥入。李丰、陈纪、乐就、梁刚都被生擒,操令皆斩于市。焚烧伪造宫室殿宇、一应犯禁之物;寿春城中,收掠一空。商议欲进兵渡淮,追赶袁术。荀彧谏曰:“年来荒旱,粮食艰难,若更进兵,劳军损民,未必有利。不若暂回许都,将来春麦熟,军粮足备,方可图之。”操踌躇未决。忽报马到,报说:“张绣依托刘表,复肆猖獗、南阳、江陵诸县复反;曹洪拒敌不住,连输数阵,今特来告急。”操乃驰书与孙策,令其跨江布阵,以为刘表疑兵,使不敢妄动;自己即日班师,别议征张绣之事。临行,令玄德仍屯兵小沛,与吕布结为兄弟,互相救助,再无相侵。吕布领兵自回徐州。操密谓玄德曰:“吾令汝屯兵小沛。是掘坑待虎之计也。公但与陈珪父子商议,勿致有失。某当为公外援。”话毕而别。却说曹操引军回许都,人报段煨杀了李傕,伍习杀了郭汜,将头来献。段煨并将李傕合族老小二百余口活解入许都。操令分于各门处斩,传首号令,人民称快。天子升殿,会集文武,作太平筵宴。封段煨为荡寇将军、伍习为殄虏将军,各引兵镇守长安。二人谢恩而去。操即奏张绣作乱,当兴兵伐之。天子乃亲排銮驾。送操出师。时建安三年夏四月也。

操留荀彧在许都,调遣兵将,自统大军进发。行军之次,见一路麦已熟;民因兵至,逃避在外,不敢刈麦。操使人远近遍谕村人父老,及各处守境官吏曰:“吾奉天子明诏,出兵讨逆,与民除害。方今麦熟之时,不得已而起兵,大小将校,凡过麦田,但有践踏者,并皆斩首。军法甚严,尔民勿得惊疑。”百姓闻谕,无不欢喜称颂,望尘遮道而拜。官军经过麦田,皆下马以手扶麦,递相传送而过,并不敢践踏。操乘马正行,忽田中惊起一鸠。那马眼生,窜入麦中,践坏了一大块麦田。操随呼行军主簿,拟议自己践麦之罪。主簿曰:“丞相岂可议罪?”操曰:“吾自制法,吾自犯之,何以服众?”即掣所佩之剑欲自刎。众急救住。郭嘉曰:“古者《春秋》之义:法不加于尊。丞相总统大军,岂可自戕?”操沉吟良久,乃曰:“既《春秋》有法不加于尊之义,吾姑免死。”乃以剑割自己之发,掷于地曰:“割发权代首。”使人以发传示三军曰:“丞相践麦,本当斩首号令,今割发以代。”于是三军悚然,无不懔遵军令。后人有诗论之曰:“十万貔貅十万心,一人号令众难禁。拔刀割发权为首,方见曹瞒诈术深。”

却说张绣知操引兵来,急发书报刘表,使为后应;一面与雷叙、张先二将领兵出城迎敌。两阵对圆,张绣出马,指操骂曰:“汝乃假仁义无廉耻之人,与禽兽何异!”操大怒,令许褚出马。绣令张先接战。只三合,许褚斩张先于马下,绣军大败。操引军赶至南阳城下。绣入城,闭门不出。操围城攻打,见城壕甚阔,水势又深,急难近城。乃令军士运土填壕;又用土布袋并柴薪草把相杂,于城边作梯凳;又立云梯窥望城中;操自骑马绕城观之,如此三日。传令教军士于西门角上,堆积柴薪,会集诸将,就那里上城。城中贾诩见如此光景,便谓张绣曰:“某已知曹操之意矣。今可将计就计而行。”正是:强中自有强中手,用诈还逢识诈人。不知其计若何,且听下文分解。