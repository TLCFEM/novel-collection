\chapter{用奇谋孔明借箭~献密计黄盖受刑}

却说鲁肃领了周瑜言语,径来舟中相探孔明。孔明接入小舟对坐。肃曰:“连日措办军
务,有失听教。”孔明曰:“便是亮亦未与都督贺喜。”肃曰:“何喜?”孔明曰:“公瑾
使先生来探亮知也不知,便是这件事可贺喜耳。”谈得鲁肃失色问曰:“先生何由知之?”
孔明曰:“这条计只好弄蒋干。曹操、虽被一时瞒过,必然便省悟,只是不肯认错耳。今
蔡、张两人既死,江东无患矣,如何不贺喜!吾闻曹操换毛玠、于禁为水军都督,则这两个
手里,好歹送了水军性命。”鲁肃听了,开口不得,把些言语支吾了半晌,别孔明而回。孔
明嘱曰:“望子敬在公瑾面前勿言亮先知此事。恐公瑾心怀妒忌,又要寻事害亮。”鲁肃应
诺而去,回见周瑜,把上项事只得实说了。瑜大惊曰:“此人决不可留!吾决意斩之!”肃
劝曰:“若杀孔明,却被曹操笑也。”瑜曰:“吾自有公道斩之,教他死而无怨。”肃曰:
“何以公道斩之?”瑜曰:“子敬休问,来日便见。”次日,聚众将于帐下,教请孔明议
事。孔明欣然而至。坐定,瑜问孔明曰:“即日将与曹军交战,水路交兵,当以何兵器为
先?”孔明曰:“大江之上,以弓箭为先。”瑜曰:“先生之言,甚合愚意。但今军中正缺
箭用,敢烦先生监造十万枝箭,以为应敌之具。此系公事,先生幸勿推却。”孔明曰:“都
督见委,自当效劳。敢问十万枝箭,何时要用?”瑜曰:“十日之内,可完办否?”孔明
曰:“操军即日将至,若候十日,必误大事。”瑜曰:“先生料几日可完办?”孔明曰:
“只消三日,便可拜纳十万枝箭。”瑜曰:“军中无戏言。”孔明曰:“怎敢戏都督!愿纳
军令状:三日不办,甘当重罚。”瑜大喜,唤军政司当面取了文书,置酒相待曰:“待军事
毕后,自有酬劳。”孔明曰:“今日已不及,来日造起。至第三日,可差五百小军到江边搬
箭。”饮了数杯,辞去。鲁肃曰:“此人莫非诈乎?”瑜曰:“他自送死,非我逼他。今明
白对众要了文书,他便两胁生翅,也飞不去。我只分付军匠人等,教他故意迟延,凡应用物
件,都不与齐备。如此,必然误了日期。那时定罪,有何理说?公今可去探他虚实,却来回
报。

肃领命来见孔明。孔明曰:“吾曾告子敬,休对公瑾说,他必要害我。不想子敬不肯为
我隐讳,今日果然又弄出事来。三日内如何造得十万箭?子敬只得救我!”肃曰:“公自取
其祸,我如何救得你?”孔明曰:“望子敬借我二十只船,每船要军士三十人,船上皆用青
布为幔,各束草千余个,分布两边。吾别有妙用。第三日包管有十万枝箭。只不可又教公瑾
得知,若彼知之,吾计败矣。”肃允诺,却不解其意,回报周瑜,果然不提起借船之事,只
言:“孔明并不用箭竹、翎毛、胶漆等物,自有道理。”瑜大疑曰:“且看他三日后如何回
覆我!”却说鲁肃私自拨轻快船二十只,各船三十余人,并布幔束草等物,尽皆齐备,候孔
明调用。第一日却不见孔明动静;第二日亦只不动。至第三日四更时分,孔明密请鲁肃到船
中。肃问曰:“公召我来何意?”孔明曰:“特请子敬同往取箭。”肃曰:“何处去取?”
孔明曰:“子敬休问,前去便见。”遂命将二十只船,用长索相连,径望北岸进发。是夜大
雾漫天,长江之中,雾气更甚,对面不相见。孔明促舟前进,果然是好大雾!前人有篇《大
雾垂江赋》曰:“大哉长江!西接岷、峨,南控三吴,北带九河。汇百川而入海,历万古以
扬波。至若龙伯、海若,江妃、水母,长鲸千丈,天蜈九首,鬼怪异类,咸集而有。盖夫鬼
神之所凭依,英雄之所战守也。时也阴阳既乱,昧爽不分。讶长空之一色,忽大雾之四屯。
虽舆薪而莫睹,惟金鼓之可闻。初若溟濛,才隐南山之豹;渐而充塞,欲迷北海之鲲。然后
上接高天,下垂厚地;渺乎苍茫,浩乎无际。鲸鲵出水而腾波,蛟龙潜渊而吐气。又如梅霖
收溽,春阴酿寒;溟溟漠漠,洁浩漫漫。东失柴桑之岸,南无夏口之山。战船千艘,俱沉沦
于岩壑;渔舟一叶,惊出没于波澜。甚则穹吴无光,朝阳失色;返白昼为昏黄,变丹山为水
碧。虽大禹之智,不能测其浅深;离娄之明,焉能辨乎咫尺?于是冯夷息浪,屏翳收功;鱼
鳖遁迹,鸟兽潜踪。隔断蓬莱之岛,暗围阊阖之宫。恍惚奔腾,如骤雨之将至;纷纭杂沓,
若寒云之欲同。乃能中隐毒蛇,因之而为瘴疠;内藏妖魅,凭之而为祸害。降疾厄于人间,
起风尘于塞外。小民遇之夭伤,大人观之感慨。盖将返元气于洪荒,混天地为大块。”

当夜五更时候,船已近曹操水寨。孔明教把船只头西尾东,一带摆开,就船上擂鼓呐
喊。鲁肃惊曰:“倘曹兵齐出,如之奈何?”孔明笑曰:“吾料曹操于重雾中必不敢出。吾
等只顾酌酒取乐,待雾散便回。

却说曹寨中,听得擂鼓呐喊,毛玠、于禁二人慌忙飞报曹操。操传令曰:“重雾迷江,
彼军忽至,必有埋伏,切不可轻动。可拨水军弓弩手乱箭射之。”又差人往旱寨内唤张辽、
徐晃各带弓弩军三千,火速到江边助射。比及号令到来,毛玠、于禁怕南军抢入水寨,已差
弓弩手在寨前放箭;少顷,旱寨内弓弩手亦到,约一万余人,尽皆向江中放箭:箭如雨发。
孔明教把船吊回,头东尾西,逼近水寨受箭,一面擂鼓呐喊。待至日高雾散,孔明令收船急
回。二十只船两边束草上,排满箭枝。孔明令各船上军士齐声叫曰:“谢丞相箭!”比及曹
军寨内报知曹操时,这里船轻水急,已放回二十余里,追之不及。曹操懊悔不已。却说孔明
回船谓鲁肃曰:“每船上箭约五六千矣。不费江东半分之力,已得十万余箭。明日即将来射
曹军,却不甚便!”肃曰:“先生真神人也!何以知今日如此大雾?”孔明曰:“为将而不
通天文,不识地利,不知奇门,不晓阴阳,不看阵图,不明兵势,是庸才也。亮于三日前已
算定今日有大雾,因此敢任三日之限。公瑾教我十日完办,工匠料物,都不应手,将这一件
风流罪过,明白要杀我。我命系于天,公瑾焉能害我哉!”鲁肃拜服。船到岸时,周瑜已差
五百军在江边等候搬箭。孔明教于船上取之,可得十余万枝,都搬入中军帐交纳。鲁肃人见
周瑜,备说孔明取箭之事。瑜大惊,慨然叹曰:“孔明神机妙算,吾不如也!”后人有诗赞
曰:“一天浓雾满长江,远近难分水渺茫。骤雨飞蝗来战舰,孔明今日伏周郎。”少顷,孔
明入寨见周瑜。瑜下帐迎之,称羡曰:“先生神算,使人敬服。”孔明曰:“诡谲小计,何
足为奇。”

瑜邀孔明入帐共饮。瑜曰:“昨吾主遣使来催督进军,瑜未有奇计,愿先生教我。”孔
明曰:“亮乃碌碌庸才,安有妙计?”瑜曰:“某昨观曹操水寨,极是严整有法,非等闲可
攻。思得一计,不知可否。先生幸为我一决之。”孔明曰:“都督且休言。各自写于手内,
看同也不同。”瑜大喜,教取笔砚来,先自暗写了,却送与孔明;孔明亦暗写了。两个移近
坐榻,各出掌中之字,互相观看,皆大笑。原来周瑜掌中字,乃一“火”字;孔明掌中,亦
一“火”字。瑜曰:“既我两人所见相同,更无疑矣。幸勿漏泄。”孔明曰:“两家公事,
岂有漏泄之理。吾料曹操虽两番经我这条计,然必不为备。今都督尽行之可也。”饮罢分
散,诸将皆不知其事。

却说曹操平白折了十五六万箭,心中气闷。荀攸进计曰:“江东有周瑜、诸葛亮二人用
计,急切难破。可差人去东吴诈降,为奸细内应,以通消息,方可图也。”操曰:“此言正
合吾意。汝料军中谁可行此计?”攸曰:“蔡瑁被诛,蔡氏宗族,皆在军中。瑁之族弟蔡
中、蔡和现为副将。丞相可以恩结之,差往诈降东吴,必不见疑。”操从之,当夜密唤二人
入帐嘱付曰:“汝二人可引些少军士,去东吴诈降。但有动静,使人密报,事成之后,重加
封赏。休怀二心!”二人曰:“吾等妻子俱在荆州,安敢怀二心,丞相勿疑。某二人必取周
瑜、诸葛亮之首,献于麾下。”操厚赏之。次日,二人带五百军士,驾船数只,顺风望着南
岸来。

且说周瑜正理会进兵之事,忽报江北有船来到江口,称是蔡瑁之弟蔡和、蔡中,特来投
降。瑜唤入。二人哭拜曰:“吾兄无罪,被操贼所杀。吾二人欲报兄仇,特来投降。望赐收
录,愿为前部。”瑜大喜,重赏二人,即命与甘宁引军为前部。二人拜谢,以为中计。瑜密
唤甘宁分付曰:“此二人不带家小,非真投降,乃曹操使来为奸细者。吾今欲将计就计,教
他通报消息。汝可殷勤相待,就里提防。至出兵之日,先要杀他两个祭旗。汝切须小心,不
可有误。”甘宁领命而去。

鲁肃入见周瑜曰:“蔡中、蔡和之降,多应是诈,不可收用。”瑜叱曰:“彼因曹操杀
其兄,欲报仇而来降,何诈之有!你若如此多疑,安能容天下之士乎!”肃默然而退,乃往
告孔明。孔明笑而不言。肃曰:“孔明何故哂笑?”孔明曰:“吾笑子敬不识公瑾用计耳。
大江隔远,细作极难往来。操使蔡中、蔡和诈降,刺探我军中事,公瑾将计就计,正要他通
报消息。兵不厌诈,公瑾之谋是也。”肃方才省悟。

却说周瑜夜坐帐中,忽见黄盖潜入中军来见周瑜。瑜问曰:“公覆夜至,必有良谋见
教?”盖曰:“彼众我寡,不宜久持,何不用火攻之?”瑜曰:“谁教公献此计?”盖曰:
“某出自己意,非他人之所教也。”瑜曰:“吾正欲如此,故留蔡中、蔡和诈降之人,以通
消息;但恨无一人为我行诈降计耳。”盖曰:“某愿行此计。”瑜曰:“不受些苦,彼如何
肯信?”盖曰:“某受孙氏厚恩,虽肝脑涂地,亦无怨悔。”瑜拜而谢之曰:“君若肯行此
苦肉计,则江东之万幸也。”盖曰:“某死亦无怨。”遂谢而出。次日,周瑜鸣鼓大会诸将
于帐下。孔明亦在座。周瑜曰:“操引百万之众,连络三百余里,非一日可破。今令诸将各
领三个月粮草,准备御敌。”言未讫,黄盖进曰:“莫说三个月,便支三十个月粮草,也不
济事!若是这个月破的,便破;若是这个月破不的,只可依张子布之言,弃甲倒戈,北面而
降之耳!”周瑜勃然变色,大怒曰:“吾奉主公之命,督兵破曹,敢有再言降者必斩。今两
军相敌之际,汝敢出此言,慢我军心,不斩汝首,难以服众!”喝左右将黄盖斩讫报来。黄
盖亦怒曰:“吾自随破虏将军,纵横东南,已历三世,那有你来?”瑜大怒,喝令速斩。甘
宁进前告曰:“公覆乃东吴旧臣,望宽恕之。”瑜喝曰:“汝何敢多言,乱吾法度!”先叱
左右将甘宁乱棒打出。众官皆跪告曰:“黄盖罪固当诛,但于军不利。望都督宽恕,权且记
罪。破曹之后,斩亦未迟。”瑜怒未息。众官苦苦告求。瑜曰:“若不看众官面皮,决须斩
首!今且免死!”命左右:“拖翻打一百脊杖,以正其罪!”众官又告免。瑜推翻案桌,叱
退众官,喝教行杖。将黄盖剥了衣服,拖翻在地,打了五十脊杖。众官又复苦苦求免。瑜跃
起指盖曰:“汝敢小觑我耶!且寄下五十棍!再有怠慢,二罪俱罚!”恨声不绝而入帐中。
众官扶起黄盖,打得皮开肉绽,鲜血进流,扶归本寨,昏绝几次。动问之人,无不下泪。鲁
肃也往看问了,来至孔明船中,谓孔明曰:“今日公瑾怒责公覆,我等皆是他部下,不敢犯
颜苦谏;先生是客,何故袖手旁观,不发一语?”孔明笑曰:“子敬欺我。”肃曰:“肃与
先生渡江以来,未尝一事相欺。今何出此言?”孔明曰:“子敬岂不知公瑾今日毒打黄公
覆,乃其计耶?如何要我劝他?”肃方悟。孔明曰:“不用苦肉计,何能瞒过曹操?今必令
黄公覆去诈降,却教蔡中、蔡和报知其事矣。子敬见公瑾时,切勿言亮先知其事,只说亮也
埋怨都督便了。”肃辞去,入帐见周瑜。瑜邀入帐后。肃曰:“今日何故痛责黄公覆?”瑜
曰:“诸将怨否?”肃曰:“多有心中不安者。”瑜曰:“孔明之意若何?”肃曰:“他也
埋怨都督忒情薄。”瑜笑曰:“今番须瞒过他也。”肃曰:“何谓也?”瑜曰:“今日痛打
黄盖,乃计也。吾欲令他诈降,先须用苦肉计瞒过曹操,就中用火攻之,可以取胜。”肃乃
暗思孔明之高见,却不敢明言。

且说黄盖卧于帐中,诸将皆来动问。盖不言语,但长吁而已。忽报参谋阚泽来问。盖令
请入卧内,叱退左右。阚泽曰:“将军莫非与都督有仇?”盖曰:“非也。”泽曰:“然则
公之受责,莫非苦肉计乎?”盖曰:“何以知之?”泽曰:“某观公瑾举动,已料着八九
分。”盖曰:“某受吴侯三世厚恩,无以为报,故献此计,以破曹操。吾虽受苦,亦无所
恨。吾遍观军中,无一人可为心腹者。惟公素有忠义之心,敢以心腹相告。”泽曰:“公之
告我,无非要我献诈降书耳。”盖曰:“实有此意。未知肯否?”阚泽欣然领诺。正是:勇
将轻身思报主,谋臣为国有同心。未知阚泽所言若何,且看下文分解。