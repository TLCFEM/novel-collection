\chapter{刘先主遗诏托孤儿~诸葛亮安居平五路}

却说章武二年夏六月,东吴陆逊大破蜀兵于猇亭彝陵之地;先主奔回白帝城,赵云引兵
据守。忽马良至,见大军已败,懊悔不及,将孔明之言,奏知先主。先主叹曰:“朕早听丞
相之言,不致今日之败!今有何面目复回成都见群臣乎!”遂传旨就白帝城住扎,将馆驿改
为永安宫。人报冯习、张南、傅彤,程畿、沙摩柯等皆殁于王事,先主伤感不已。又近臣奏
称:“黄权引江北之兵,降魏去了。陛下可将彼家属送有司问罪。”先主曰:“黄权被吴兵
隔断在江北岸,欲归无路,不得已而降魏:是朕负权,非权负朕也,何必罪其家属?”仍给
禄米以养之。却说黄权降魏,诸将引见曹丕,丕曰:“卿今降朕,欲追慕于陈、韩耶?”权
泣而奏曰:“臣受蜀帝之恩,殊遇甚厚,令臣督诸军于江北,被陆逊绝断。臣归蜀无路,降
吴不可,故来投陛下。败军之将,免死为幸,安敢追慕于古人耶!”丕大喜,遂拜黄权为镇
南将军。权坚辞不受。忽近臣奏曰:“有细作人自蜀中来,说蜀主将黄权家属尽皆诛戮。”
权曰:“臣与蜀主,推诚相信,知臣本心,必不肯杀臣之家小也。”丕然之。后人有诗责黄
权曰:“降吴不可却降曹,忠义安能事两朝?堪叹黄权惜一死,紫阳书法不轻饶。”

曹丕问贾诩曰:“朕欲一统天下,先取蜀乎?先取吴乎?”诩曰:“刘备雄才,更兼诸
葛亮善能治国;东吴孙权,能识虚实,陆逊现屯兵于险要,隔江泛湖,皆难卒谋。以臣观
之,诸将之中,皆无孙权、刘备敌手。虽以陛下天威临之,亦未见万全之势也。只可持守,
以待二国之变。”丕曰:“朕已遣三路大兵伐吴,安有不胜之理?”尚书刘晔曰:“近东吴
陆逊,新破蜀兵七十万,上下齐心,更有江湖之阻,不可卒制,陆逊多谋,必有准备。”丕
曰:“卿前劝朕伐吴,今又谏阻,何也?”晔曰:“时有不同也。昔东吴累败于蜀,其势顿
挫,故可击耳;今既获全胜,锐气百倍,未可攻也。”丕曰:“朕意已决,卿勿复言。”遂
引御林军亲往接应三路兵马。早有哨马报说东吴已有准备:令吕范引兵拒住曹休,诸葛瑾引
兵在南郡拒住曹真,朱桓引兵当住濡须以拒曹仁。刘晔曰:“既有准备,去恐无益。”丕不
从,引兵而去。

却说吴将朱桓,年方二十七岁,极有胆略,孙权甚爱之;时督军于濡须,闻曹仁引大军
去取羡溪,桓遂尽拨军守把羡溪去了,止留五千骑守城。忽报曹仁令大将常雕同诸葛虔、王
双、引五万精兵飞奔濡须城来。众军皆有惧色。桓按剑而言曰:“胜负在将,不在兵之多
寡。兵法云:客兵倍而主兵半者,主兵尚能胜于客兵。今曹仁千里跋涉,人马疲困。吾与汝
等共据高城,南临大江,北背山险,以逸待劳,以主制客:此乃百战百胜之势。虽曹丕自
来,尚不足忧,况仁等耶!”于是传令,教众军偃旗息鼓,只作无人守把之状。

且说魏将先锋常雕,领精兵来取濡须城,遥望城上并无军马。雕催军急进,离城不远,
一声炮响,旌旗齐竖。朱桓横刀飞马而出,直取常雕。战不三合,被桓一刀斩常雕于马下。
吴兵乘势冲杀一阵,魏兵大败,死者无数。朱桓大胜,得了无数旌旗军器战马。曹仁领兵随
后到来,却被吴兵从羡溪杀出。曹仁大败而退,回见魏主,细奏大败之事。丕大惊。正议之
间,忽探马报:“曹真、夏侯尚围了南郡,被陆逊伏兵于内,诸葛瑾伏兵于外,内外夹攻,
因此大败。”言未毕,忽探马又报:”曹休亦被吕范杀败。”丕听知三路兵败,乃喟然叹
曰:“朕不听贾诩、刘晔之言,果有此败!”时值夏天,大疫流行,马步军十死六七,遂引
军回洛阳。吴、魏自此不和。

却说先主在永安宫,染病不起,渐渐沉重,至章武三年夏四日,先主自知病入四肢,又
哭关、张二弟,其病愈深:两目昏花。厌见侍从之人,乃叱退左右,独卧于龙榻之上。忽然
阴风骤起,将灯吹摇,灭而复明,只见灯影之下,二人侍立。先主怒曰:“朕心绪不宁,教
汝等且退,何故又来!”叱之不退。先主起而视之,上首乃云长,下首乃翼德也。先主大惊
曰:“二弟原来尚在?”云长曰:“臣等非人,乃鬼也。上帝以臣二人平生不失信义,皆敕
命为神。哥哥与兄弟聚会不远矣。”先主扯定大哭。忽然惊觉,二弟不见。即唤从人问之,
时正三更。先主叹曰:“朕不久于人世矣!”遂遣使往成都,请丞相诸葛亮,尚书令李严
等,星夜来永安宫,听受遗命。孔明等与先主次子鲁王刘永、梁王刘理,来永安宫见帝,留
太子刘禅守成都。且说孔明到永安宫,见先主病危,慌忙拜伏于龙榻之下。先主传旨,请孔
明坐于龙榻之侧。抚其背曰:“朕自得丞相,幸成帝业;何期智识浅陋,不纳丞相之言,自
取其败。悔恨成疾,死在旦夕。嗣子孱弱,不得不以大事相托。”言讫,泪流满面。孔明亦
涕泣曰:“愿陛下善保龙体,以副下天之望!”先主以目遍视,只见马良之弟马谡在傍,先
主令且退。谡退出,先主谓孔明曰:“丞相观马谡之才何如?”孔明曰:“此人亦当世之英
才也。”先主曰:“不然。朕观此人,言过其实,不可大用。丞相宜深察之。”分付毕,传
旨召诸臣入殿,取纸笔写了遗诏,递与孔明而叹曰:“朕不读书,粗知大略。圣人云:鸟之
将死,其鸣也哀;人之将死,其言也善。朕本待与卿等同灭曹贼,共扶汉室;不幸中道而
别。烦丞相将诏付与太子禅,令勿以为常言。凡事更望丞相教之!”孔明等泣拜于地曰:
“愿陛下将息龙体!臣等尽施犬马之劳,以报陛下知遇之恩也。”先主命内侍扶起孔明,一
手掩泪,一手执其手,曰:“朕今死矣,有心腹之言相告!”孔明曰:“有何圣谕!”先主
泣曰:“君才十倍曹丕,必能安邦定国,终定大事。若嗣子可辅,则辅之;如其不才,君可
自为成都之主。”孔明听毕,汗流遍体,手足失措,泣拜于地曰:“臣安敢不竭股肱之力,
尽忠贞之节,继之以死乎!”言讫,叩头流血。先主又请孔明坐于榻上,唤鲁王刘永、梁王
刘理近前,分付曰:“尔等皆记朕言:朕亡之后,尔兄弟三人,皆以父事丞相,不可怠
慢。”言罢,遂命二王同拜孔明。二王拜毕,孔明曰:“臣虽肝脑涂地,安能报知遇之恩
也!”先主谓众官曰:“朕已托孤于丞相,令嗣子以父事之。卿等俱不可怠慢,以负朕
望。”又嘱赵云曰:“朕与卿于患难之中,相从到今,不想于此地分别。卿可想朕故交,早
晚看觑吾子,勿负朕言。”云泣拜曰:“臣敢不效犬马之劳!”先主又谓众官曰:“卿等众
官,朕不能一一分嘱,愿皆自爱。”言毕,驾崩,寿六十三岁。时章武三年夏四月二十四日
也。后杜工部有诗叹曰:“蜀主窥吴向三峡,崩年亦在永安宫。翠华想像空山外,玉殿虚无
野寺中。古庙杉松巢水鹤,岁时伏腊走村翁。武侯祠屋长邻近,一体君臣祭祀同。”

先主驾崩,文武官僚,无不哀痛。孔明率众官奉梓宫还成都。太子刘禅出城迎接灵柩,
安于正殿之内。举哀行礼毕,开读遗诏。诏曰:“朕初得疾,但下痢耳;后转生杂病,殆不
自济。朕闻人年五十,不称夭寿。今朕年六十有余,死复何恨?但以卿兄弟为念耳。勉之!
勉之!勿以恶小而为之,勿以善小而不为。惟贤惟德,可以服人;卿父德薄,不足效也。卿
与丞相从事,事之如父,勿怠!勿忘!卿兄弟更求闻达。至嘱!至嘱!”群臣读诏已毕。孔
明曰:“国不可一日无君,请立嗣君,以承汉统。”乃立太子禅即皇帝位,改元建兴。加诸
葛亮为武乡侯,领益州牧。葬先主于惠陵,谥曰昭烈皇帝。尊皇后吴氏为皇太后;谥甘夫人
为昭烈皇后,糜夫人亦追谥为皇后。升赏群臣,大赦天下。早有魏军探知此事,报入中原。
近臣奏知魏主。曹丕大喜曰:“刘备已亡,朕无忧矣。何不乘其国中无主,起兵伐之?”贾
诩谏曰:“刘备虽亡,必托孤于诸葛亮。亮感备知遇之恩,必倾心竭力,扶持嗣主。陛下不
可仓卒伐之。”正言间,忽一人从班部中奋然而出曰:“不乘此时进兵,更待何时?”众视
之,乃司马懿也。丕大喜,遂问计于懿。懿曰:“若只起中国之兵,急难取胜。须用五路大
兵,四面夹攻,令诸葛亮首尾不能救应,然后可图。”丕问何五路,懿曰:“可修书一封,
差使往辽东鲜卑国,见国王轲比能,赂以金帛,令起辽西羌兵十万,先从旱路取西平关:此
一路也。再修书遣使赍官诰赏赐,直入南蛮,见蛮王孟获,令起兵十万,攻打益州、永昌、
牂牁、越嶲四郡,以击西川之南:此二路也。再遣使入吴修好,许以割地,令孙权起兵十
万,攻两川峡口,径取涪城:此三路也。又可差使至降将孟达处,起上庸兵十万,西攻汉
中:此四路也。然后命大将军曹真为大都督,提兵十万,由京兆径出阳平关取西川;此五路
也。共大兵五十万,五路并进,诸葛亮便有吕望之才,安能当此乎?”丕大喜,随即密遣能
言官四员为使前去;又命曹真为大都督,领兵十万,径取阳平关。此时张辽等一班旧将,皆
封列侯、俱在冀、徐、青及合淝等处,据守关津隘口,故不复调用。却说蜀汉后主刘禅,自
即位以来,旧臣多有病亡者,不能细说。凡一应朝廷选法,钱粮、词讼等事,皆听诸葛丞相
裁处。时后主未立皇后,孔明与群臣上言曰:“故车骑将军张飞之女甚贤,年十七岁,可纳
为正宫皇后。”后主即纳之。

建兴元年秋八月,忽有边报说:“魏调五路大兵,来取西川;第一路,曹真为大都督,
起兵十万,取阳平关;第二路,乃反将孟达,起上庸兵十万,犯汉中;第三路,乃东吴孙
权,起精兵十万,取峡口入川;第四路,乃蛮王孟获,起蛮兵十万,犯益州四郡;第五路,
乃番王轲比能,起羌兵十万,犯西平关。此五路军马,甚是利害。”已先报知丞相,丞相不
知为何,数日不出视事。后主听罢大惊,即差近侍赍旨,宣召孔明入朝。使命去了半日,回
报:“丞相府下人言,丞相染病不出。”后主转慌;次日,又命黄门侍郎董允、谏议大夫杜
琼,去丞相卧榻前,告此大事。董、杜二人到丞相府前,皆不得入。杜琼曰:“先帝托孤于
丞相,今主上初登宝位,被曹丕五路兵犯境,军情至急,丞相何故推病不出?”良久,门吏
传丞相令,言:“病体稍可,明早出都堂议事。”董、杜二人叹息而回。次日,多官又来丞
相府前伺候。从早至晚,又不见出。多官惶惶,只得散去。杜琼入奏后主曰:“请陛下圣
驾,亲往丞相府问计。”后主即引多官入宫,启奏皇太后。太后大惊,曰:“丞相何故如
此?有负先帝委托之意也!我当自往。”董允奏曰:“娘娘未可轻往。臣料丞相必有高明之
见。且待主上先往。如果怠慢,请娘娘于太庙中,召丞相问之未迟。”太后依奏。

次日,后主车驾亲至相府。门吏见驾到,慌忙拜伏于地而迎。后主问曰:“丞相在何
处?”门吏曰:“不知在何处。只有丞相钧旨,教挡住百官,勿得辄入。”后主乃下车步
行,独进第三重门,见孔明独倚竹杖,在小池边观鱼。后主在后立久,乃徐徐而言曰:“丞
相安乐否?”孔明回顾,见是后主,慌忙弃杖,拜伏于地曰:“臣该万死!”后主扶起,问
曰:“今曹丕分兵五路,犯境甚急,相父缘何不肯出府视事?”孔明大笑,扶后主入内室坐
定,奏曰:“五路兵至,臣安得不知,臣非观鱼,有所思也。”后主曰:“如之奈何?”孔
明曰:“羌王轲比能,蛮王孟获,反将孟达,魏将曹真;此四路兵,臣已皆退去了也。止有
孙权这一路兵,臣已有退之之计,但须一能言之人为使。因未得其人,故熟思之。陛下何必
忧乎?”

后主听罢,又惊又喜,曰:“相父果有鬼神不测之机也!愿闻退兵之策。”孔明曰:
“先帝以陛下付托与臣,臣安敢旦夕怠慢。成都众官,皆不晓兵法之妙,贵在使人不测,岂
可泄漏于人?老臣先知西番国王轲比能,引兵犯西平关;臣料马超积祖西川人氏,素得羌人
之心,羌人以超为神威天将军,臣已先遣一人,星夜驰檄,令马超紧守西平关,伏四路奇
兵,每日交换,以兵拒之:此一路不必忧矣。又南蛮孟获,兵犯四郡,臣亦飞檄遣魏延领一
军左出右入,右出左入,为疑兵之计:蛮兵惟凭勇力,其心多疑,若见疑兵,必不敢进:此
一路又不足忧矣。又知孟达引兵出汉中;达与李严曾结生死之交;臣回成都时,留李严守永
安宫;臣已作一书、只做李严亲笔,令人送与孟达;达必然推病不出,以慢军心:此一路又
不足忧矣。又知曹真引兵犯阳平关;此地险峻,可以保守,臣已调赵云引一军守把关隘,并
不出战;曹真若见我军不出,不久自退矣。此四路兵俱不足忧。臣尚恐不能全保,又密调关
兴、张苞二将,各引兵三万,屯于紧要之处,为各路救应。此数处调遣之事,皆不曾经由成
都,故无人知觉。只有东吴这一路兵,未必便动:如见四路兵胜,川中危急,必来相攻;若
四路不济,安肯动乎?臣料孙权想曹丕三路侵吴之怨,必不肯从其言。虽然如此,须用一舌
辩之士,径往东吴,以利害说之,则先退东吴;其四路之兵,何足忧乎?但未得说吴之人,
臣故踌躇。何劳陛下圣驾来临?”后主曰:“太后亦欲来见相父。今朕闻相父之言,如梦初
觉。复何忧哉!”

孔明与后主共饮数杯,送后主出府。众官皆环立于门外,见后主面有喜色。后主别了孔
明,上御车回朝。众皆疑惑不定。孔明见众官中,一人仰天而笑,面亦有喜色。孔明视之,
乃义阳新野人,姓邓,名芝,字伯苗,现为户部尚书;汉司马邓禹之后。孔明暗令人留住邓
芝。多官皆散,孔明请芝到书院中,问芝曰:“今蜀、魏、吴鼎分三国,欲讨二国,一统中
兴,当先伐何国?”芝曰:“以愚意论之:魏虽汉贼,其势甚大,急难摇动,当徐徐缓图;
今主上初登宝位,民心未安,当与东吴连合,结为唇齿,一洗先帝旧怨,此乃长久之计也。
未审丞相钧意若何?”孔明大笑曰:“吾思之久矣,奈未得其人。今日方得也!”芝曰:
“丞相欲其人何为?”孔明曰:“吾欲使人往结东吴。公既能明此意,必能不辱君命。使吴
之任,非公不可。”芝曰:“愚才疏智浅,恐不堪当此任。”孔明曰:“吾来日奏知天子,
便请伯苗一行,切勿推辞。”芝应允而退。至次日,孔明奏准后主,差邓芝往说东吴。芝拜
辞,望东吴而来。正是:吴人方见干戈息,蜀使还将玉帛通。未知邓芝此去若何,且看下文
分解。