\chapter{定三分隆中决策~战长江孙氏报仇}

却说玄德访孔明两次不遇,欲再往访之。关公曰:“兄长两次亲往拜谒,其礼太过矣。
想诸葛亮有虚名而无实学,故避而不敢见。兄何惑于斯人之甚也!”玄德曰:“不然,昔齐
桓公欲见东郭野人,五反而方得一面。况吾欲见大贤耶?”张飞曰:“哥哥差矣。量此村
夫,何足为大贤;今番不须哥哥去;他如不来,我只用一条麻绳缚将来!”玄德叱曰:“汝
岂不闻周文王谒姜子牙之事乎?文王且如此敬贤,汝何太无礼!今番汝休去,我自与云长
去。”飞曰:“既两位哥哥都去,小弟如何落后!”玄德曰:“汝若同往,不可失礼。”飞
应诺。

于是三人乘马引从者往隆中。离草庐半里之外,玄德便下马步行,正遇诸葛均。玄德忙
施礼,问曰:“令兄在庄否?”均曰:“昨暮方归。将军今日可与相见。”言罢,飘然自
去。玄德曰:“今番侥幸得见先生矣!”张飞曰:“此人无礼!便引我等到庄也不妨,何故
竟自去了!”玄德曰:“彼各有事,岂可相强。”三人来到庄前叩门,童子开门出问。玄德
曰:“有劳仙童转报:刘备专来拜见先生。”童子曰:“今日先生虽在家,但今在草堂上昼
寝未醒。”玄德曰:“既如此,且休通报。”分付关、张二人,只在门首等着。玄德徐步而
入,见先生仰卧于草堂几席之上。玄德拱立阶下。半晌,先生未醒。关、张在外立久,不见
动静,入见玄德犹然侍立。张飞大怒,谓云长曰:“这先生如何傲慢!见我哥哥侍立阶下,
他竟高卧,推睡不起!等我去屋后放一把火,看他起不起!”云长再三劝住。玄德仍命二人
出门外等候。望堂上时,见先生翻身将起,忽又朝里壁睡着。童子欲报。玄德曰:“且勿惊
动。”又立了一个时辰,孔明才醒,口吟诗曰:“大梦谁先觉?平生我自知,草堂春睡足,
窗外日迟迟。”孔明吟罢,翻身问童子曰:“有俗客来否?”童子曰:“刘皇叔在此,立候
多时。”孔明乃起身曰:“何不早报!尚容更衣。”遂转入后堂。又半晌,方整衣冠出迎。

玄德见孔明身长八尺,面如冠玉,头戴纶巾,身披鹤氅,飘飘然有神仙之概。玄德下拜
曰:“汉室末胄、涿郡愚夫,久闻先生大名,如雷贯耳。昨两次晋谒,不得一见,已书贱名
于文几,未审得入览否?”孔明曰:“南阳野人,疏懒性成,屡蒙将军枉临,不胜愧赧。”
二人叙礼毕,分宾主而坐,童子献茶。茶罢,孔明曰:“昨观书意,足见将军忧民忧国之
心;但恨亮年幼才疏,有误下问。”玄德曰:“司马德操之言,徐元直之语,岂虚谈哉?望
先生不弃鄙贱,曲赐教诲。”孔明曰:“德操、元直,世之高士。亮乃一耕夫耳,安敢谈天
下事?二公谬举矣。将军奈何舍美玉而求顽石乎?”玄德曰:“大丈夫抱经世奇才,岂可空
老于林泉之下?愿先生以天下苍生为念,开备愚鲁而赐教。”孔明笑曰:“愿闻将军之
志。”玄德屏人促席而告曰:“汉室倾颓,奸臣窃命,备不量力,欲伸大义于天下,而智术
浅短,迄无所就。惟先生开其愚而拯其厄,实为万幸!”孔明曰:“自董卓造逆以来,天下
豪杰并起。曹操势不及袁绍,而竟能克绍者,非惟天时,抑亦人谋也。今操已拥百万之众,
挟天子以令诸侯,此诚不可与争锋。孙权据有江东,已历三世,国险而民附,此可用为援而
不可图也。荆州北据汉、沔,利尽南海,东连吴会,西通巴、蜀,此用武之地,非其主不能
守;是殆天所以资将军,将军岂有意乎?益州险塞,沃野千里,天府之国,高祖因之以成帝
业;今刘璋暗弱,民殷国富,而不知存恤,智能之士,思得明君。将军既帝室之胄,信义著
于四海,总揽英雄,思贤如渴,若跨有荆、益,保其岩阻,西和诸戎,南抚彝、越,外结孙
权,内修政理;待天下有变,则命一上将将荆州之兵以向宛、洛,将军身率益州之众以出秦
川,百姓有不箪食壶浆以迎将军者乎?诚如是,则大业可成,汉室可兴矣。此亮所以为将军
谋者也。惟将军图之。”言罢,命童子取出画一轴,挂于中堂,指谓玄德曰:“此西川五十
四州之图也。将军欲成霸业,北让曹操占天时,南让孙权占地利,将军可占人和。先取荆州
为家,后即取西川建基业,以成鼎足之势,然后可图中原也。”玄德闻言,避席拱手谢曰:
“先生之言,顿开茅塞,使备如拨云雾而睹青天。但荆州刘表、益州刘璋,皆汉室宗亲,备
安忍夺之?”孔明曰:“亮夜观天象,刘表不久人世;刘璋非立业之主:久后必归将军。”
玄德闻言,顿首拜谢。只这一席话,乃孔明未出茅庐,已知三分天下,真万古之人不及也!
后人有诗赞曰:“豫州当日叹孤穷,何幸南阳有卧龙!欲识他年分鼎处,先生笑指画图
中。”玄德拜请孔明曰:“备虽名微德薄,愿先生不弃鄙贱,出山相助。备当拱听明诲。”
孔明曰:“亮久乐耕锄,懒于应世,不能奉命。”玄德泣曰:“先生不出,如苍生何!”言
毕,泪沾袍袖,衣襟尽湿。孔明见其意甚诚,乃曰:“将军既不相弃,愿效犬马之劳。”玄
德大喜,遂命关、张入,拜献金麻礼物。孔明固辞不受。玄德曰:“此非聘大贤之礼,但表
刘备寸心耳。”孔明方受。于是玄德等在庄中共宿一宵。

次日,诸葛均回,孔明嘱付曰:“吾受刘皇叔三顾之恩,不容不出。汝可躬耕于此,勿
得荒芜田亩。待我功成之日,即当归隐。”后人有诗叹曰:“身未升腾思退步,功成应忆去
时言。只因先主丁宁后,星落秋风五丈原。”又有古风一篇曰:“高皇手提三尺雪,芒砀白
蛇夜流血;平秦灭楚入咸阳,二百年前几断绝。大哉光武兴洛阳,传至桓灵又崩裂;献帝迁
都幸许昌,纷纷四海生豪杰:曹操专权得天时,江东孙氏开鸿业;孤穷玄德走天下,独居新
野愁民厄。南阳卧龙有大志,腹内雄兵分正奇;只因徐庶临行语,茅庐三顾心相知。先生尔
时年三九,收拾琴书离陇亩;先取荆州后取川,大展经纶补天手;纵横舌上鼓风雷,谈笑胸
中换星斗;龙骧虎视安乾坤,万古千秋名不朽!”玄德等三人别了诸葛均,与孔明同归新
野。

玄德待孔明如师,食则同桌,寝则同榻,终日共论天下之事,孔明曰:“曹操于冀州作
玄武池以练水军,必有侵江南之意。可密令人过江探听虚实。”玄德从之,使人往江东探
听。

却说孙权自孙策死后,据住江东,承父兄基业,广纳贤士,开宾馆于吴会,命顾雍、张
纮延接四方宾客。连年以来,你我相荐。时有会稽阚泽,字德润;彭城严畯,字曼才;沛县
薛综,字敬文;汝阳程秉,字德枢;吴郡朱桓,字休穆;陆绩,字公纪;吴人张温,字惠
恕;乌伤骆统,字公绪;乌程吾粲,字孔休:此数人皆至江东,孙权敬礼甚厚。又得良将数
人:乃汝南吕蒙,字子明;吴郡陆逊,宇伯言;琅琊徐盛,字文向;东郡潘璋,字文珪;庐
江丁奉,字承渊。文武诸人,共相辅佐,由此江东称得人之盛。

建安七年,曹操破袁绍,遣使往江东,命孙权遣子入朝随驾。权犹豫未决。吴太夫人命
周瑜、张昭等面议。张昭曰:“操欲令我遣子入朝,是牵制诸侯之法也。然若不令去,恐其
兴兵下江东,势必危矣。”周瑜曰:“将军承父兄遗业,兼六郡之众,兵精粮足,将士用
命,有何逼迫而欲送质于人?质一入,不得不与曹氏连和;彼有命召,不得不往:如此,则
见制于人也。不如勿遣,徐观其变,别以良策御之。”吴太夫人曰:“公瑾之言是也。”权
遂从其言,谢使者,不遣子。自此曹操有下江南之意。但正值北方未宁,无暇南征。

建安八年十一月,孙权引兵伐黄祖,战于大江之中。祖军败绩。权部将凌操,轻舟当
先,杀入夏口,被黄祖部将甘宁一箭射死。凌操子凌统,时年方十五岁,奋力往夺父尸而
归。权见风色不利,收军还东吴。

却说孙权弟孙翊为丹阳太守,翊性刚好酒,醉后尝鞭挞士卒。丹阳督将妫览、郡丞戴员
二人,常有杀翊之心;乃与翊从人边洪结为心腹,共谋杀翊。时诸将县令,皆集丹阳,翊设
宴相待。翊妻徐氏美而慧,极善卜《易》,是日卜一卦,其象大凶,劝翊勿出会客。翊不
从,遂与众大会。至晚席散,边洪带刀跟出门外,即抽刀砍死孙翊。妫览、戴员乃归罪边
洪,斩之于市。二人乘势掳翊家资侍妾。妫览见徐氏美貌,乃谓之曰:“吾为汝夫报仇,汝
当从我;不从则死。”徐氏曰:“夫死未几,不忍便相从;可待至晦日,设祭除服,然后成
亲未迟。”览从之。徐氏乃密召孙翊心腹旧将孙高、傅婴二人入府,泣告曰:“先夫在日,
常言二公忠义。今妫、戴二贼,谋杀我夫,只归罪边洪,将我家资童婢尽皆分去。妫览又欲
强占妾身,妾已诈许之,以安其心。二将军可差人星夜报知吴侯,一面设密计以图二贼,雪
此仇辱,生死衔恩!”言毕再拜。孙高、傅婴皆泣曰:“我等平日感府君恩遇,今日所以不
即死难者,正欲为复仇计耳。夫人所命,敢不效力!”于是密遣心腹使者往报孙权。

至晦日,徐氏先召孙、傅二人,伏于密室韩幕之中,然后设祭于堂上。祭毕,即除去孝
服,沐浴薰香,浓妆艳裹,言笑自若。妫览闻之甚喜。至夜,徐氏遗婢妾请览入府,设席堂
中饮酒。饮既醉,徐氏乃邀览入密室。览喜,乘醉而入。徐氏大呼曰:“孙、傅二将军何
在!”二人即从帏幕中持刀跃出。妫览措手不及,被傅婴一刀砍倒在地,孙高再复一刀,登
时杀死。徐氏复传请戴员赴宴。员入府来,至堂中,亦被孙、傅二将所杀。一面使人诛戮二
贼家小及其余党。徐氏遂重穿孝服,将妫览、戴员首级,祭于孙翊灵前。不一日,孙权自领
军马至丹阳,见徐氏已杀妫、戴二贼,乃封孙高、傅婴为牙门将,令守丹阳,取徐氏归家养
老。江东人无不称徐氏之德。后人有诗赞曰:“才节双全世所无,奸回一旦受摧锄。庸臣从
贼忠臣死,不及东吴女丈夫。”

且说东吴各处山贼,尽皆平复。大江之中,有战船七千余只。孙权拜周瑜为大都督,总
统江东水陆军马。建安十二年,冬十月,权母吴太夫人病危,召周瑜、张昭二人至,谓曰:
“我本吴人,幼亡父母,与弟吴景徒居越中。后嫁与孙氏,生四子。长子策生时,吾梦月入
怀;后生次子权,又梦日入怀。卜者云:梦日月入怀者,其子大贵。不幸策早丧,今将江东
基业付权。望公等同心助之,吾死不朽矣!”又嘱权曰:“汝事子布、公瑾以师傅之礼,不
可怠慢。吾妹与我共嫁汝父,则亦汝之母也;吾死之后,事吾妹如事我。汝妹亦当恩养,择
佳婿以嫁之。”言讫遂终。孙权哀哭,具丧葬之礼,自不必说。

至来年春,孙权商议欲伐黄祖。张昭曰:“居丧未及期年,不可动兵。”周瑜曰:“报
仇雪恨,何待期年?”权犹豫未决。适平北都尉吕蒙入见,告权曰:“某把龙湫水口,忽有
黄祖部将甘宁来降。某细询之:宁字兴霸,巴郡临江人也;颇通书史,有气力,好游侠;尝
招合亡命,纵横于江湖之中;腰悬铜铃,人听铃声,尽皆避之。又尝以西川锦作帆幔,时人
皆称为锦帆贼。后悔前非,改行从善,引众投刘表。见表不能成事,即欲来投东吴,却被黄
祖留住在夏口。前东吴破祖时,祖得甘宁之力,救回夏口;乃待宁甚薄。都督苏飞屡荐宁于
祖。祖曰:宁乃劫江之贼,岂可重用!宁因此怀恨。苏飞知其意,乃置酒邀宁到家,谓之
曰:吾荐公数次,奈主公不能用。日月逾迈,人生几何,宜自远图。吾当保公为邾县长,自
作去就之计。宁因此得过夏口,欲投江东,恐江东恨其救黄祖杀凌操之事。某具言主公求贤
若渴,不记旧恨;况各为其主,又何恨焉?宁欣然引众渡江,来见主公。乞钧旨定夺。”孙
权大喜曰:“吾得兴霸,破黄祖必矣。”遂命吕蒙引甘宁入见。参拜已毕,权曰:“兴霸来
此,大获我心,岂有记恨之理?请无怀疑。愿教我以破黄祖之策。”宁曰:“今汉祚日危,
曹操终必篡窃。南荆之地操所必争也。刘表无远虑,其子又愚劣,不能承业传基,明公宜早
图之;若迟,则操先图之矣。今宜先取黄祖。祖今年老昏迈,务于货利;侵求吏民,人心皆
怨;战具不修,军无法律。明公若往攻之,其势必破。既破祖军,鼓行而西,据楚关而图
巴、蜀,霸业可定也。”孙权曰:“此金玉之论也!”遂命周瑜为大都督,总水陆军兵;吕
蒙为前部先锋;董袭与甘宁为副将;权自领大军十万,征讨黄祖。

细作探知,报至江夏。黄祖急聚众商议,令苏飞为大将,陈就、邓龙为先锋,尽起江夏
之兵迎敌。陈就、邓龙各引一队艨艟截住沔口,艨艟上各设强弓硬弩千余张,将大索系定艨
艟于水面上。东吴兵至,艨艟上鼓响,弓弩齐发,兵不敢进,约退数里水面。甘宁谓董袭
曰:“事已至此,不得不进。”乃选小船百余只,每船用精兵五十人:二十人撑船,三十人
各披衣甲,手执铜刀,不避矢石,直至艨艟傍边,砍断大索,艨艟遂横。甘宁飞上艨艟,将
邓龙砍死。陈就弃船而走。吕蒙见了,跳下小船,自举橹棹,直入船队,放火烧船。陈就急
待上岸,吕蒙舍命赶到跟前,当胸一刀砍翻。比及苏飞引军于岸上接应时,东吴诸将一齐上
岸,势不可当。祖军大败。苏飞落荒而走,正遇东吴大将潘璋,两马相交,战不数合,被璋
生擒过去,径至船中来见孙权。权命左右以槛车囚之,待活捉黄祖,一并诛戮。催动三军,
不分昼夜,攻打夏口。正是:只因不用锦帆贼,至令冲开大索船。未知黄祖胜负如何,且看
下文分解。