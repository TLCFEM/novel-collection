\chapter{曹仁大战东吴兵~孔明一气周公瑾}

却说孔明欲斩云长,玄德曰:“昔吾三人结义时,誓同生死。今云长虽犯法,不忍违却
前盟。望权记过,容将功赎罪。”孔明方才饶了。且说周瑜收军点将,各各叙功,申报吴
侯。所得降卒,尽行发付渡江,大犒三军,遂进兵攻取南郡。前队临江下寨,前后分五营。
周瑜居中。瑜正与众商议征进之策,忽报:“刘玄德使孙乾来与都督作贺。”瑜命请入。乾
施礼毕,言:“主公特命乾拜谢都督大德,有薄礼上献。”瑜问曰:“玄德在何处?”乾答
曰:“现移兵屯油江口。”瑜惊曰:“孔明亦在油江否?”乾曰;“孔明与主公同在油
江。”瑜曰:“足下先回,某亲来相谢也。”瑜收了礼物,发付孙乾先回。肃曰:“却才都
督为何失惊?”瑜曰:“刘备屯兵油江,必有取南郡之意。我等费了许多军马,用了许多钱
粮,目下南郡反手可得;彼等心怀不仁,要就现成,须放着周瑜不死!”肃曰:“当用何策
退之?”瑜曰:“吾自去和他说话。好便好;不好时,不等他取南郡,先结果了刘备!”肃
曰:“某愿同往。”于是瑜与鲁肃引三千轻骑,径投油江口来。先说孙乾回见玄德,言周瑜
将亲来相谢。玄德乃问孔明曰:“来意若何?”孔明笑曰:“那里为这些薄礼肯来相谢。止
为南郡而来。”玄德曰:“他若提兵来,何以待之?”孔明曰:“他来便可如此如此应
答。”遂于油江口摆开战船,岸上列着军马。人报:“周瑜、鲁肃引兵到来。”孔明使赵云
领数骑来接。瑜见军势雄壮,心甚不安。行至营门外,玄德、孔明迎入帐中。各叙礼毕,设
宴相待。玄德举酒致谢鏖兵之事。酒至数巡,瑜曰:“豫州移兵在此,莫非有取南郡之意
否?”玄德曰:“闻都督欲取南郡,故来相助。若都督不取,备必取之”。瑜笑曰:“吾东
吴久欲吞并汉江,今南郡已在掌中,如何不取?”玄德曰:“胜负不可预定。曹操临归,令
曹仁守南郡等处,必有奇计;更兼曹仁勇不可当:但恐都督不能取耳。”瑜曰:“吾若取不
得,那时任从公取。”玄德曰:“子敬、孔明在此为证,都督休悔。”鲁肃踌躇未对。瑜
曰:“大丈夫一言既出,何悔之有!”孔明曰:“都督此言,甚是公论。先让东吴去取;若
不下,主公取之,有何不可!”瑜与肃辞别玄德、孔明,上马而去。玄德问孔明曰:“却才
先生教备如此回答,虽一时说了,展转寻思,于理未然。我今孤穷一身,无置足之地,欲得
南郡,权且容身;若先教周瑜取了,城池已属东吴矣,却如何得住?”孔明大笑曰:“当初
亮劝主公取荆州,主公不听,今日却想耶?”玄德曰:“前为景升之地,故不忍取;今为曹
操之地,理合取之。”孔明曰:“不须主公忧虑。尽着周瑜去厮杀,早晚教主公在南郡城中
高坐。”玄德曰:“计将安出?”孔明曰:“只须如此如此。”玄德大喜,只在江口屯扎,
按兵不动。却说周瑜、鲁肃回寨。肃曰:“都督如何亦许玄德取南郡?”瑜曰:“吾弹指可
得南郡,落得虚做人情。”随问帐下将士:“谁敢先取南郡?”一人应声而出,乃蒋钦也。
瑜曰:“汝为先锋,徐盛、丁奉为副将,拨五千精锐军马,先渡江。吾随后引兵接应。”且
说曹仁在南郡,分付曹洪守彝陵,以为掎角之势。人报:“吴兵已渡汉江。”仁曰:“坚守
勿战为上。”骁将牛金奋然进曰:“兵临城下而不出战,是怯也。况吾兵新败,正当重振锐
气。某愿借精兵五百,决一死战。”仁从之,令牛金引五百军出战。丁奉纵马来迎。约战四
五合,奉诈败,牛金引军追赶入阵。奉指挥众军一裹围牛金于阵中。金左右冲突,不能得
出。曹仁在城上望见牛金困在垓心,遂披甲上马,引麾下壮士数百骑出城,奋力挥刀,杀入
吴阵。徐盛迎战,不能抵挡。曹仁杀到垓心,救出牛金。回顾尚有数十骑在阵,不能得出,
遂复翻身杀入,救出重围。正遇蒋钦拦路,曹仁与牛金奋力冲散。仁弟曹纯,亦引兵接应,
混杀一阵。吴军败走,曹仁得胜而回。蒋钦兵败,回见周瑜,瑜怒欲斩之,众将告免。瑜即
点兵,要亲与曹仁决战。甘宁曰:“都督未可造次。今曹仁令曹洪据守彝陵,为掎角之势;
某愿以精兵三千,径取彝陵,都督然后可取南郡。”瑜服其论,先教甘宁领三千兵攻打彝
陵,早有细作报知曹仁,仁与陈矫商议。矫曰:“彝陵有失,南郡亦不可守矣。宜速救
之。”仁遂令曹纯与牛金暗地引兵救曹洪。曹纯先使人报知曹洪,令洪出城诱敌。甘宁引兵
至彝陵,洪出与甘宁交锋。战有二十余合,洪败走。宁夺了彝陵。至黄昏时,曹纯、牛金兵
到,两下相合,围了彝陵。探马飞报周瑜,说甘宁困于彝陵城中,瑜大惊。程普曰:“可急
分兵救之。”瑜曰:“此地正当冲要之处,若分兵去救,倘曹仁引兵来袭,奈何?”吕蒙
曰:“甘兴霸乃江东大将,岂可不救?”瑜曰:“吾欲自往救之;但留何人在此,代当吾
任?”蒙曰:“留凌公绩当之。蒙为前驱,都督断后;不须十日,必奏凯歌。”瑜曰:“未
知凌公绩肯暂代吾任否?”凌统曰:“若十日为期,可当之;十日之外,不胜其任矣。”瑜
大喜,遂留兵万余,付与凌统;即日起大兵投彝陵来。蒙谓瑜曰:“彝陵南僻小路,取南郡
极便。可差五百军去砍倒树木,以断其路。彼军若败,必走此路;马不能行,必弃马而走,
吾可得其马也。”瑜从之,差军去讫。

大兵将至彝陵,瑜问:“谁可突围而入,以救甘宁?”周泰愿往,即时绰刀纵马,直杀
入曹军之中,径到城下。甘宁望见周泰至,自出城迎之。泰言:“都督自提兵至。”宁传令
教军士严装饱食,准备内应。却说曹洪、曹纯、牛金闻周瑜兵将至,先使人往南郡报知曹
仁,一面分兵拒敌。及吴兵至,曹兵迎之。比及交锋,甘宁、周泰分两路杀出,曹兵大乱,
吴兵四下掩杀。曹洪、曹纯、牛金果然投小路而走;却被乱柴塞道,马不能行,尽皆弃马而
走。吴兵得马五百余匹。周瑜驱兵星夜赶到南郡,正遇曹仁军来救彝陵。两军接着,混战一
场。天色已晚,各自收兵。

曹仁回城中,与众商议。曹洪曰:“目今失了彝陵,势已危急,何不拆丞相遗计观之,
以解此危?”曹仁曰:“汝言正合吾意。”遂拆书观之,大喜,便传令教五更造饭;平明,
大小军马,尽皆弃城;城上遍插旌旗,虚张声势。军分三门而出。却说周瑜救出甘宁,陈兵
于南郡城处。见曹兵分三门而出,瑜上将台观看。只见女墙边虚搠旌旗,无人守护;又见军
士腰下各束缚包裹。瑜暗忖曹仁必先准备走路,遂下将台号令,分布两军为左右翼;如前军
得胜,只顾向前追赶,直待鸣金,方许退步。命程普督后军,瑜亲自引军取城。对阵鼓声响
处,曹洪出马搦战,瑜自至门旗下,使韩当出马,与曹洪交锋;战到三十余合,洪败走。曹
仁自出接战,周泰纵马相迎;斗十余合,仁败走。阵势错乱。周瑜麾两翼军杀出,曹军大
败。瑜自引军马追至南郡城下,曹军皆不入城,望西北面走。韩当、周泰引前部尽力追赶。
瑜见城门大开,城上又无人,遂令众军抢城。数十骑当先而入。瑜在背后纵马加鞭,直入瓮
城。陈矫在敌楼上,望见周瑜亲自入城来,暗暗喝采道:“丞相妙策如神!”一声梆子响,
两边弓弩齐发,势如骤雨。争先入城的,都颠入陷坑内。周瑜急勒马回时,被一弩箭,正射
中左助,翻身落马。牛金从城中杀出,来捉周瑜;徐盛、丁奉二人舍命救去。城中曹兵突
出,吴兵自相践踏,落堑坑者无数。程普急收军时,曹仁、曹洪分兵两路杀回。吴兵大败。
幸得凌统引一军从刺斜里杀来,敌住曹兵。曹仁引得胜兵进城,程普收败军回寨。丁、徐二
将救得周瑜到帐中,唤行军医者用铁钳子拔出箭头,将金疮药敷掩疮口,疼不可当,饮食俱
废。医者曰:“此箭头上有毒,急切不能痊可。若怒气冲激,其疮复发。”程普令三军紧守
各寨,不许轻出,三日后,牛金引军来搦战,程普按兵不动。牛金骂至日暮方回,次日又来
骂战。程普恐瑜生气,不敢报知。第三日,牛金直至寨门外叫骂,声声只道要捉周瑜。程普
与众商议,欲暂且退兵,回见吴侯,却再理会。却说周瑜虽患疮痛,心中自有主张;已知曹
兵常来寨前叫骂,却不见众将来禀。一日,曹仁自引大军,擂鼓呐喊,前来搦战。程普拒住
不出。周瑜唤众将入帐问曰:“何处鼓噪呐喊?”众将曰:“军中教演士卒。”瑜怒曰:
“何欺我也!吾已知曹兵常来寨前辱骂。程德谋既同掌兵权,何故坐视?”遂命人请程普入
帐问之。普曰:“吾见公瑾病疮,医者言勿触怒,故曹兵搦战,不敢报知。”瑜曰:“公等
不战,主意若何?”普曰:“众将皆欲收兵暂回江东。待公箭疮平复,再作区处。”瑜听
罢,于床上奋然跃起曰:“大丈夫既食君禄,当死于战场,以马革裹尸还,幸也!岂可为我
一人,而废国家大事乎?”言讫,即披甲上马。诸军众将,无不骇然。遂引数百骑出营前。
望见曹兵已布成阵势,曹仁自立马于门旗下,扬鞭大骂曰:“周瑜孺子,料必横夭,再不敢
正觑我兵!”骂犹未绝,瑜从群骑内突然出曰:“曹仁匹夫!见周郎否!”曹军看见,尽皆
惊骇。曹仁回顾众将曰:“可大骂之!”众军厉声大骂。周瑜大怒,使潘璋出战。未及交
锋,周瑜忽大叫一声,口中喷血。坠于马下。曹兵冲来,众将向前抵住,混战一场,救起周
瑜,回到帐中。程普问曰:“都督贵体若何?”瑜密谓普曰:“此吾之计也。”普曰:“计
将安出?”瑜曰:“吾身本无甚痛楚;吾所以为此者,欲令曹兵知我病危,必然欺敌。可使
心腹军士去城中诈降,说吾已死。今夜曹仁必来劫寨。吾却于四下埋伏以应之,则曹仁可一
鼓而擒也。”程普曰:“此计大妙!”随就帐下举起哀声。众军大惊,尽传言都督箭疮大发
而死,各寨尽皆挂孝。却说曹仁在城中与众商议,言周瑜怒气冲发,金疮崩裂,以致口中喷
血,坠于马下,不久必亡。正论间,忽报:“吴寨内有十数个军士来降。中间亦有二人,原
是曹兵被掳过去的。”曹仁忙唤入问之。军士曰:“今日周瑜阵前金疮碎裂,归寨即死。今
众将皆已挂孝举哀。我等皆受程普之辱,故特归降,便报此事。”曹仁大喜,随即商议今晚
便去劫寨,夺周瑜之尸,斩其首级,送赴许都。陈矫曰:“此计速行,不可迟误。”

曹仁遂令牛金为先锋,自为中军,曹洪、曹纯为合后,只留陈矫领些少军士守城,其余
军兵尽起。初更后出城,径投周瑜大寨。来到寨门,不见一人,但见虚插旗枪而已。情知中
计,急忙退军。四下炮声齐发:东边韩当、蒋钦杀来,西边周泰、潘璋杀来,南边徐盛、丁
奉杀来,北边陈武、吕蒙杀来。曹兵大败,三路军皆被冲散,首尾不能相救。曹仁引十数骑
杀出重围,正遇曹洪,遂引败残军马一同奔走。杀到五更,离南郡不远,一声鼓响,凌统又
引一军拦住去路,截杀一阵。曹仁引军刺斜而走,又遇甘宁大杀一阵。曹仁不敢回南郡,径
投襄阳大路而行,吴军赶了一程,自回。

周瑜、程普收住众军,径到南郡城下,见旌旗布满,敌楼上一将叫曰:“都督少罪!吾
奉军师将令,已取城了。吾乃常山赵子龙也。”周瑜大怒,便命攻城。城上乱箭射下。瑜命
且回军商议,使甘宁引数千军马,径取荆州;凌统引数千军马,径取襄阳;然后却再取南郡
未迟。正分拨间,忽然探马急来报说:“诸葛亮自得了南郡,遂用兵符,星夜诈调荆州守城
军马来救,却教张飞袭了荆州。”又一探马飞来报说:“夏侯惇在襄阳,被诸葛亮差人赍兵
符,诈称曹仁求救,诱惇引兵出,却教云长袭取了襄阳。二处城池,全不费力,皆属刘玄德
矣。”周瑜曰:“诸葛亮怎得兵符?”程普曰:“他拿住陈矫,兵符自然尽属之矣。”周瑜
大叫一声,金疮迸裂。正是:几郡城池无我分,一场辛苦为谁忙!未知性命如何,且看下文
分解。