\chapter{公孙渊兵败死襄平~司马懿诈病赚曹爽}

却说公孙渊乃辽东公孙度之孙,公孙康之子也。建安十二年,曹操追袁尚,未到辽东,康斩尚首级献操,操封康为襄平侯;后康死,有二子:长曰晃,次曰渊,皆幼;康弟公孙恭继职。曹丕时封恭为车骑将军、襄平侯。太和二年,渊长大,文武兼备,性刚好斗,夺其叔公孙恭之位,曹睿封渊为扬烈将军、辽东太守。后孙权遣张弥、许晏赍金珠珍玉赴辽东,封渊为燕王。渊惧中原,乃斩张、许二人,送首与曹睿。睿封渊为大司马、乐浪公。渊心不足,与众商议,自号为燕王,改元绍汉元年。副将贾范谏曰:“中原待主公以上公之爵,不为卑贱;今若背反,实为不顺。更兼司马懿善能用兵,西蜀诸葛武侯且不能取胜,何况主公乎?”渊大怒,叱左右缚贾范,将斩之。参军伦直谏曰:“贾范之言是也。圣人云:国家将亡,必有妖孽。今国中屡见怪异之事:近有犬戴巾帻,身披红衣,上屋作人行;又城南乡民造饭,饭甑之中,忽有一小儿蒸死于内;襄平北市中,地忽陷一穴,涌出一块肉,周围数尺,头面眼耳口鼻都具,独无手足,刀箭不能伤,不知何物。卜者占之曰:有形不成,有口无声;国家亡灭,故现其形。有此三者,皆不祥之兆也。主公宜避凶就吉,不可轻举妄动。”渊勃然大怒,叱武士绑伦直并贾范同斩于市。令大将军卑衍为元帅,杨祚为先锋,起辽兵十五万,杀奔中原来。

边官报知魏主曹睿。睿大惊,乃召司马懿入朝计议。懿奏曰:“臣部下马步官军四万,足可破贼。”睿曰:“卿兵少路远,恐难收复。”懿曰:“兵不在多,在能设奇用智耳。臣托陛下洪福,必擒公孙渊以献陛下。”睿曰:“卿料公孙渊作何举动?”懿曰:“渊若弃城预走,是上计也;守辽东拒大军,是中计也;坐守襄平,是为下计,必被臣所擒矣。”睿曰:“此去往复几时?”懿曰:“四千里之地,往百日,攻百日,还百日,休息六十日,大约一年足矣。”睿曰:“倘吴、蜀入寇,如之奈何?”懿曰:“臣已定下守御之策,陛下勿忧。”睿大喜,即命司马懿兴师征讨公孙渊。

懿辞朝出城,令胡遵为先锋,引前部兵先到辽东下寨。哨马飞报公孙渊。渊令卑衍,杨祚分八万兵屯于辽隧,围堑二十余里,环绕鹿角,甚是严密。胡遵令人报知司马懿。懿笑曰:“贼不与我战,欲老我兵耳。我料贼众大半在此,其巢穴空虚,不若弃却此处,径奔襄平;贼必往救,却于中途击之,必获全功。”于是勒兵从小路向襄平进发。

却说卑衍与杨祚商议曰:“若魏兵来攻,休与交战。彼千里而来,粮草不继,难以持久,粮尽必退;待他退时,然后出奇兵击之,司马懿可擒也。昔司马懿与蜀兵相拒,坚守渭南,孔明竟卒于军中:今日正与此理相同。”二人正商议间,忽报:“魏兵往南去了。”卑衍大惊曰:“彼知吾襄平军少,去袭老营也。若襄平有失,我等守此处无益矣。”遂拔寨随后而起。早有探马飞报司马懿。懿笑曰:“中吾计矣!”乃令夏侯霸、夏侯威,各引一军伏于辽水之滨:“如辽兵到,两下齐出。”二人受计而往。早望见卑衍、杨祚引兵前来。一声炮响,两边鼓噪摇旗:左有夏侯霸、右有夏侯威,一齐杀出。卑、杨二人,无心恋战,夺路而走;奔至首山,正逢公孙渊兵到,合兵一处,回马再与魏兵交战。卑衍出马骂曰:“贼将休使诡计!汝敢出战否?”夏侯霸纵马挥刀来迎。战不数合,被夏侯霸一刀斩卑衍于马下,辽兵大乱。霸驱兵掩杀,公孙渊引败兵奔入襄平城去,闭门坚守不出。魏兵四面围合。

时值秋雨连绵,一月不止,平地水深三尺,运粮船自辽河口直至襄平城下。魏兵皆在水中,行坐不安。左都督裴景入帐告曰:“雨水不住,营中泥泞,军不可停,请移于前面山上。”懿怒曰:“捉公孙渊只在旦夕,安可移营?如有再言移营者斩!”裴景喏喏而退。少顷,右都督仇连又来告曰:“军土苦水,乞太尉移营高处。”懿大怒曰:“吾军令已发,汝何敢故违!”即命推出斩之,悬首于辕门外。于是军心震慑。

懿令南寨人马暂退二十里,纵城内军民出城樵采柴薪,牧放牛马。司马陈群问曰:“前太尉攻上庸之时,兵分八路,八日赶至城下,遂生擒孟达而成大功;今带甲四万,数千里而来,不令攻打城池,却使久居泥泞之中,又纵贼众樵牧。某实不知太尉是何主意?”懿笑曰:“公不知兵法耶?昔孟达粮多兵少,我粮少兵多,故不可不速战;出其不意,突然攻之,方可取胜。今辽兵多,我兵少,贼饥我饱,何必力攻?正当任彼自走,然后乘机击之。我今放开一条路,不绝彼之樵牧,是容彼自走也。”陈群拜服。

于是司马懿遣人赴洛阳催粮。魏主曹睿设朝,群臣皆奏曰:“近日秋雨连绵,一月不止,人马疲劳,可召回司马懿,权且罢兵。”睿曰:“司马太尉善能用兵,临危制变,多有良谋,捉公孙渊计日而待。卿等何必忧也?”遂不听群臣之谏,使人运粮解至司马懿军前。

懿在寨中,又过数日,雨止天晴。是夜,懿出帐外,仰观天文,忽见一星,其大如斗,流光数丈,自首山东北,坠于襄平东南。各营将士,无不惊骇。懿见之大喜,乃谓众将曰:“五日之后,星落处必斩公孙渊矣。来日可并力攻城。”众将得令,次日侵晨,引兵四面围合,筑土山,掘地道,立炮架,装云梯,日夜攻打不息,箭如急雨,射入城去。

公孙渊在城中粮尽,皆宰牛马为食。人人怨恨,各无守心,欲斩渊首,献城归降。渊闻之,甚是惊忧,慌令相国王建、御史大夫柳甫,往魏寨请降。二人自城上系下,来告司马懿曰:“请太尉退二十里,我君臣自来投降。”懿大怒曰:“公孙渊何不自来?殊为无理!”叱武士推出斩之,将首级付与从人。从人回报,公孙渊大惊,又遣侍中卫演来到魏营。司马懿升帐,聚众将立于两边。演膝行而进,跪于帐下,告曰:“愿太尉息雷霆之怒。克日先送世子公孙修为质当,然后君臣自缚来降。”懿曰:“军事大要有五:能战当战,不能战当守,不能守当走,不能走当降,不能降当死耳!何必送子为质当?”叱卫演回报公孙渊,演抱头鼠窜而去。

归告公孙渊,渊大惊,乃与子公孙修密议停当,选下一千人马,当夜二更时分,开了南门,往东南而走。渊见无人,心中暗喜。行不到十里,忽听得山上一声炮响,鼓角齐鸣:一枝兵拦住,中央乃司马懿也;左有司马师,右有司马昭,二人大叫曰:“反贼休走!”渊大惊,急拨马寻路欲走。早有胡遵兵到;左有夏侯霸、夏侯威,右有张虎、乐綝:四面围得铁桶相似。公孙渊父子,只得下马纳降。懿在马上顾诸将曰:“吾前夜丙寅日,见大星落于此处,今夜壬申日应矣。”众将称贺曰:“太尉真神机也!”懿传令斩之。公孙渊父子对面受戳。司马懿遂勒兵来取襄平。未及到城下时,胡遵早引兵入城。城中人民焚香拜迎,魏兵尽皆入城。懿坐于衙上,将公孙渊宗族,并同谋官僚人等,俱杀之,计首级七十余颗。出榜安民。人告懿曰:贾范、伦直苦谏渊不可反叛,俱被渊所杀。懿遂封其墓面荣其子孙。就将库内财物,赏劳三军,班师回洛阳。却说魏主在宫中,夜至三更,忽然一阵阴风,吹灭灯光,只见毛皇后引数十个宫人哭至座前索命。睿因此得病。病渐沉重,命侍中光禄大夫刘放、孙资,掌枢密院一切事务;又召文帝子燕王曹宇为大将军,佐太子曹芳摄政。宇为人恭俭温和,未肯当此大任,坚辞不受。睿召刘放、孙资问曰:“宗族之内,何人可任?”二人久得曹真之惠,乃保奏曰:“惟曹子丹之子曹爽可也。”睿从之。二人又奏曰:“欲用曹爽,当遣燕王归国。”睿然其言。二人遂请睿降诏,赍出谕燕王曰:“有天子手诏,命燕王归国,限即日就行;若无诏不许入朝。”燕王涕泣而去。遂封曹爽为大将军,总摄朝政。

睿病渐危,急令使持节诏司马懿还朝。懿受命,径到许昌,入见魏主。睿曰:“朕惟恐不得见卿;今日得见,死无恨矣。”懿顿首奏曰:“臣在途中,闻陛下圣体不安,恨不肋生两翼,飞至阙下。今日得睹龙颜,臣之幸也。”睿宣太子曹芳,大将军曹爽,侍中刘放、孙资等,皆至御榻之前。睿执司马懿之手曰:“昔刘玄德在白帝城病危,以幼子刘禅托孤于诸葛孔明,孔明因此竭尽忠诚,至死方休:偏邦尚然如此,何况大国乎?朕幼子曹芳,年才八岁,不堪掌理社稷。幸太尉及宗兄元勋旧臣,竭力相辅,无负朕心!”又唤芳曰:“仲达与朕一体,尔宜敬礼之。”遂命懿携芳近前。芳抱懿颈不放。睿曰:“太尉勿忘幼子今日相恋之情!”言讫,潸然泪下。懿顿首流涕。魏主昏沉,口不能言,只以手指太子,须臾而卒;在位十三年,寿三十六岁,时魏景初三年春正月下旬也。

当下司马懿、曹爽,扶太子曹芳即皇帝位。芳字兰卿,乃睿乞养之子,秘在宫中,人莫知其所由来。于是曹芳谥睿为明帝,葬于高平陵;尊郭皇后为皇太后;改元正始元年。司马懿与曹爽辅政。爽事懿甚谨,一应大事,必先启知。爽字昭伯,自幼出入宫中,明帝见爽谨慎,甚是爱敬。爽门下有客五百人,内有五人以浮华相尚:一是何晏,字平叔;一是邓飏,字玄茂,乃邓禹之后;一是李胜,字公昭;一是丁谧,字彦靖;一是毕轨,字昭先。又有大司农桓范字元则,颇有智谋,人多称为智囊。此数人皆爽所信任。

何晏告爽曰:“主公大权,不可委托他人,恐生后患。爽曰:“司马公与我同受先帝托孤之命,安忍背之?”晏曰:“昔日先公与仲达破蜀兵之时,累受此人之气,因而致死。主公如何不察也?”爽猛然省悟,遂与多官计议停当,入奏魏主曹芳曰:“司马懿功高德重,可加为太傅。”芳从之,自是兵权皆归于爽。爽命弟曹羲为中领军,曹训为武卫将军,曹彦为散骑常侍,各引三千御林军,任其出入禁宫。又用何晏、邓飏、丁谧为尚书,毕轨为司隶校尉,李胜为河南尹:此五人日夜与爽议事。于是曹爽门下宾客日盛。司马懿推病不出,二子亦皆退职闲居。爽每日与何晏等饮酒作乐:凡用衣服器皿,与朝廷无异;各处进贡玩好珍奇之物,先取上等者入己,然后进宫,佳人美女,充满府院。黄门张当,谄事曹爽,私选先帝侍妾七八人,送入府中;爽又选善歌舞良家子女三四十人,为家乐。又建重楼画阁,造金银器皿,用巧匠数百人,昼夜工作。却说何晏闻平原管辂明数术,请与论《易》。时邓飏在座,问辂曰:“君自谓善《易》而语不及《易》中词义,何也?”辂曰:“夫善《易》者,不言《易》也。”晏笑而赞之曰:“可谓要言不烦。”因谓辂曰:“试为我卜一卦:可至三公否?”又问:“连梦青蝇数十,来集鼻上,此是何兆?”辂曰:“元、恺辅舜,周公佐周,皆以和惠谦恭,享有多福。今君侯位尊势重,而怀德者鲜,畏威者众,殆非小心求福之道。且鼻者,山也;山高而不危,所以长守贵也。今青蝇臭恶而集焉。位峻者颠,可不惧乎?愿君侯裒多益寡,非礼勿履:然后三公可至,青蝇可驱也。”邓飏怒曰:“此老生之常谈耳!”辂曰:“老生者见不生,常谈者见不谈。”遂拂袖而去。二人大笑曰:“真狂士也!”辂到家,与舅言之。舅大惊曰:“何、邓二人,威权甚重,汝奈何犯之?”辂曰:“吾与死人语,何所畏耶!”舅问其故。辂曰:“邓飏行步,筋不束骨,脉不制肉,起立倾倚,若无手足:此为鬼躁之相。何晏视候,魂不守宅,血不华色,精爽烟浮,容若槁木:此为鬼幽之相。二人早晚必有杀身之祸,何足畏也!”其舅大骂辂为狂子而去。

却说曹爽尝与何晏、邓飏等畋猎。其弟曹羲谏曰:“兄威权太甚,而好出外游猎,倘为人所算,悔之无及。”爽叱曰:“兵权在吾手中,何惧之有!”司农桓范亦谏,不听。时魏主曹芳,改正始十年为嘉平元年。曹爽一向专权,不知仲达虚实,适魏主除李胜为荆州刺史,即令李胜往辞仲达,就探消息。胜径到太傅府中,早有门吏报入。司马懿谓二子曰:“此乃曹爽使来探吾病之虚实也。”乃去冠散发,上床拥被而坐,又令二婢扶策,方请李胜入府。胜至床前拜曰:“一向不见太傅,谁想如此病重。今天子命某为荆州刺吏,特来拜辞。”懿佯答曰:“并州近朔方,好为之备。”胜曰:“除荆州刺史,非并州也。”懿笑曰:“你方从并州来?”胜曰:“汉上荆州耳。懿大笑曰:“你从荆州来也!”胜曰:“太傅如何病得这等了?”左右曰:“太傅耳聋。”胜曰:“乞纸笔一用。”左右取纸笔与胜。胜写毕,呈上,懿看之,笑曰:“吾病的耳聋了。此去保重。”言讫,以手指口。侍婢进汤,懿将口就之,汤流满襟,乃作哽噎之声曰:“吾今衰老病笃,死在旦夕矣。二子不肖,望君教之。君若见大将军,千万看觑二子!”言讫,倒在床上,声嘶气喘。李胜拜辞仲达,回见曹爽,细言其事。爽大喜曰:“此老若死,吾无忧矣!”司马懿见李胜去了,遂起身谓二子曰:“李胜此去,回报消息,曹爽必不忌我矣。只待他出城畋猎之时,方可图之。”不一日,曹爽请魏主曹芳去谒高平陵,祭祀先帝。大小官僚,皆随驾出城。爽引三弟,并心腹人何晏等,及御林军护驾正行,司农桓范叩马谏曰:“主公总典禁兵,不宜兄弟皆出。倘城中有变,如之奈何?”爽以鞭指而叱之曰:“谁敢为变!再勿乱言!”当日,司马懿见爽出城,心中大喜,即起旧日手下破敌之人,并家将数十,引二子上马,径来谋杀曹爽。正是:闭户忽然有起色,驱兵自此逞雄风。未知曹爽性命如何,且看下文分解。