\chapter{陨大星汉丞相归天~见木像魏都督丧胆}

却说姜维见魏延踏灭了灯,心中忿怒,拔剑欲杀之。孔明止之曰:“此吾命当绝,非文长之过也。”维乃收剑。孔明吐血数口,卧倒床上,谓魏延曰:“此是司马懿料吾有病,故令人来探视虚实。汝可急出迎敌。”魏延领命,出帐上马,引兵杀出寨来。夏侯霸见了魏延,慌忙引军退走。延追赶二十余里方回。孔明令魏延自回本寨把守。

姜维入帐,直至孔明榻前问安。孔明曰:“吾本欲竭忠尽力,恢复中原,重兴汉室;奈天意如此,吾旦夕将死。吾平生所学,已著书二十四篇,计十万四千一百一十二字,内有八务、七戒、六恐、五惧之法。吾遍观诸将,无人可授,独汝可传我书。切勿轻忽!”维哭拜而受。孔明又曰:“吾有‘连弩’之法,不曾用得。其法矢长八寸,一弩可发十矢,皆画成图本。汝可依法造用。”维亦拜受。孔明又曰:“蜀中诸道,皆不必多忧;惟阴平之地,切须仔细。此地虽险峻,久必有失。”又唤马岱入帐,附耳低言,授以密计;嘱曰:“我死之后,汝可依计行之。”岱领计而出。少顷,杨仪入。孔明唤至榻前,授与一锦囊,密嘱曰:“我死,魏延必反;待其反时,汝与临阵,方开此囊。那时自有斩魏延之人也。”孔明一一调度已毕,便昏然而倒,至晚方苏,便连夜表奏后主。后主闻奏大惊,急命尚书李福,星夜至军中问安,兼询后事。李福领命,趱程赴五丈原,入见孔明,传后主之命,问安毕。孔明流涕曰:“吾不幸中道丧亡,虚废国家大事,得罪于天下。我死后,公等宜竭忠辅主。国家旧制,不可改易;吾所用之人,亦不可轻废。吾兵法皆授与姜维,他自能继吾之志,为国家出力。吾命已在旦夕,当即有遗表上奏天子也。”李福领了言语,匆匆辞去。孔明强支病体,令左右扶上小车,出寨遍观各营;自觉秋风吹面,彻骨生寒,乃长叹曰:“再不能临阵讨贼矣!悠悠苍天,曷此其极!”叹息良久。回到帐中,病转沉重,乃唤杨仪分付曰:“王平、廖化、张嶷、张翼、吴懿等,皆忠义之士,久经战阵,多负勤劳,堪可委用。我死之后,凡事俱依旧法而行。缓缓退兵,不可急骤。汝深通谋略,不必多嘱。姜伯约智勇足备,可以断后。”杨仪泣拜受命。孔明令取文房四宝,于卧榻上手书遗表,以达后主。表略曰:“伏闻生死有常,难逃定数;死之将至,愿尽愚忠:臣亮赋性愚拙,遭时艰难,分符拥节,专掌钧衡,兴师北伐,未获成功;何期病入膏肓,命垂旦夕,不及终事陛下,饮恨无穷!伏愿陛下:清心寡欲,约己爱民;达孝道于先皇,布仁恩于宇下;提拔幽隐,以进贤良;屏斥奸邪,以厚风俗。臣家成都有桑八百株,薄田十五顷,子弟衣食,自有余饶。至于臣在外任,别无调度,随身衣食,悉仰于官,不别治生,以长尺寸。臣死之日,不使内有余帛,外有赢财,以负陛下也。”孔明写毕,又嘱杨仪曰:“吾死之后,不可发丧。可作一大龛,将吾尸坐于龛中;以米七粒,放吾口内;脚下用明灯一盏;军中安静如常,切勿举哀:则将星不坠。吾阴魂更自起镇之。司马懿见将星不坠,必然惊疑。吾军可令后寨先行,然后一营一营缓缓而退。若司马懿来追,汝可布成阵势,回旗返鼓。等他来到,却将我先时所雕木像,安于车上,推出军前,令大小将士,分列左右。懿见之必惊走矣。”杨仪一一领诺。

是夜,孔明令人扶出,仰观北斗,遥指一星曰:“此吾之将星也。”众视之,见其色昏暗,摇摇欲坠。孔明以剑指之,口中念咒。咒毕急回帐时,不省人事。众将正慌乱间,忽尚书李福又至;见孔明昏绝,口不能言,乃大哭曰:“我误国家之大事也!”须臾,孔明复醒,开目遍视,见李福立于榻前。孔明曰:“吾已知公复来之意。福谢曰:“福奉天子命,问丞相百年后,谁可任大事者。适因匆遽,失于谘请,故复来耳。”孔明曰:“吾死之后,可任大事者:蒋公琰其宜也。”福曰:“公琰之后,谁可继之?”孔明曰:“费文伟可继之。”福又问:“文伟之后,谁当继者?”孔明不答。众将近前视之,已薨矣。时建兴十二年秋八月二十三日也,寿五十四岁。后杜工部有诗叹曰:“长星昨夜坠前营,讣报先生此日倾。虎帐不闻施号令,麟台惟显著勋名。空余门下三千客,辜负胸中十万兵。好看绿阴清昼里,于今无复雅歌声!”白乐天亦有诗曰:“先生晦迹卧山林,三顾那逢圣主寻。鱼到南阳方得水,龙飞天汉便为霖。托孤既尽殷勤礼,报国还倾忠义心。前后出师遗表在,令人一览泪沾襟。”初,蜀长水校尉廖立,自谓才名宜为孔明之副,尝以职位闲散,怏怏不平,怨谤无已。于是孔明废之为庶人,徒之汶山。及闻孔明亡,乃垂泣曰:“吾终为左衽矣!”李严闻之,亦大哭病死,盖严尝望孔明复收己,得自补前过;度孔明死后,人不能用之故也。后元微之有赞孔明诗曰:“拨乱扶危主,殷勤受托孤。英才过管乐,妙策胜孙吴。凛凛《出师表》,堂堂八阵图。如公全盛德,应叹古今无!”

是夜,天愁地惨,月色无光,孔明奄然归天。姜维、杨仪遵孔明遗命,不敢举哀,依法成殓,安置龛中,令心腹将卒三百人守护;随传密令,使魏延断后,各处营寨一一退去。

却说司马懿夜观天文,见一大星,赤色,光芒有角,自东北方流于西南方,坠于蜀营内,三投再起,隐隐有声。懿惊喜曰:“孔明死矣!”即传令起大兵追之。方出寨门,忽又疑虑曰:“孔明善会六丁六甲之法,今见我久不出战,故以此术诈死,诱我出耳。今若追之,必中其计。”遂复勒马回寨不出,只令夏侯霸暗引数十骑,往五丈原山僻哨探消息。

却说魏延在本寨中,夜作一梦,梦见头上忽生二角,醒来甚是疑异。次日,行军司马赵直至,延请入问曰:“久知足下深明《易》理,吾夜梦头生二角,不知主何吉凶?烦足下为我决之。”赵直想了半晌,答曰:“此大吉之兆:麒麟头上有角,苍龙头上有角,乃变化飞腾之象也。”延大喜曰:“如应公言,当有重谢!”直辞去,行不数里,正遇尚书费祎。祎问何来。直曰:“适至魏文长营中,文长梦头生角,令我决其吉凶。此本非吉兆,但恐直言见怪,因以麒麟苍龙解之。”祎曰:“足下何以知非吉兆?”直曰:“角之字形,乃刀下用也。今头上用刀,其凶甚矣!”祎曰:“君且勿泄漏。”直别去。费祎至魏延寨中,屏退左右,告曰:“昨夜三更,丞相已辞世矣。临终再三嘱付,令将军断后以当司马懿,缓缓而退,不可发丧。今兵符在此,便可起兵。”延曰:“何人代理丞相之大事?”祎曰:“丞相一应大事,尽托与杨仪;用兵密法,皆授与姜伯约。此兵符乃杨仪之令也。”延曰:“丞相虽亡,吾今现在。杨仪不过一长史,安能当此大任?他只宜扶柩入川安葬。我自率大兵攻司马懿,务要成功。岂可因丞相一人而废国家大事耶?”祎曰:“丞相遗令,教且暂退,不可有违。”延怒曰:“丞相当时若依我计,取长安久矣!吾今官任前将军、征西大将军、南郑侯,安肯与长史断后!“祎曰:“将军之言虽是,然不可轻动,令敌人耻笑。待吾往见杨仪,以利害说之,令彼将兵权让与将军,何如?”延依其言。

祎辞延出营,急到大寨见杨仪,具述魏延之语。仪曰:“丞相临终,曾密嘱我曰:魏延必有异志。今我以兵符往,实欲探其心耳。今果应丞相之言。吾自令伯约断后可也。”于是杨仪领兵扶柩先行,令姜维断后;依孔明遗令,徐徐而退。魏延在寨中,不见费祎来回覆,心中疑惑,乃令马岱引十数骑往探消息。回报曰:“后军乃姜维总督,前军大半退入谷中去了。”延大怒曰:“竖儒安敢欺我!我必杀之!”因顾谓岱曰:“公肯相助否?”岱曰:“某亦素恨杨仪,今愿助将军攻之。”延大喜,即拔寨引本部兵望南而行。

却说夏侯霸引军至五丈原看时,不见一人,急回报司马懿曰:“蜀兵已尽退矣。”懿跌足曰:“孔明真死矣!可速追之!”夏侯霸曰:“都督不可轻追。当令偏将先往。”懿曰:“此番须吾自行。”遂引兵同二子一齐杀奔五丈原来;呐喊摇旗,杀入蜀寨时,果无一人。懿顾二子曰:“汝急催兵赶来,吾先引军前进。”于是司马师、司马昭在后催军;懿自引军当先,追到山脚下,望见蜀兵不远,乃奋力追赶。忽然山后一声炮响,喊声大震,只见蜀兵俱回旗返鼓,树影中飘出中军大旗,上书一行大字曰:“汉丞相武乡侯诸葛亮”。懿大惊失色。定睛看时,只见中军数十员上将,拥出一辆四轮车来;车上端坐孔明:纶巾羽扇,鹤氅皂绦。懿大惊曰:“孔明尚在!吾轻入重地,堕其计矣!”急勒回马便走。背后姜维大叫:“贼将休走!你中了我丞相之计也!”魏兵魂飞魄散,弃甲丢盔,抛戈撇戟,各逃性命,自相践踏,死者无数。司马懿奔走了五十余里,背后两员魏将赶上,扯住马嚼环叫曰:“都督勿惊。”懿用手摸头曰:“我有头否?”二将曰:“都督休怕,蜀兵去远了。”懿喘息半晌,神色方定;睁目视之,乃夏侯霸、夏侯惠也;乃徐徐按辔,与二将寻小路奔归本寨,使众将引兵四散哨探。

过了两日,乡民奔告曰:“蜀兵退入谷中之时,哀声震地,军中扬起白旗:孔明果然死了,止留姜维引一千兵断后。前日车上之孔明,乃木人也。”懿叹曰:“吾能料其生,不能料其死也!”因此蜀中人谚曰:“死诸葛能走生仲达。”后人有诗叹曰:“长星半夜落天枢,奔走还疑亮未殂。关外至今人冷笑,头颅犹问有和无!”司马懿知孔明死信已确,乃复引兵追赶。行到赤岸坡,见蜀兵已去远,乃引还,顾谓众将曰:“孔明已死,我等皆高枕无忧矣!”遂班师回。一路上见孔明安营下寨之处,前后左右,整整有法,懿叹曰:“此天下奇才也!”于是引兵回长安,分调众将,各守隘口,懿自回洛阳面君去了。

却说杨仪、姜维排成阵势,缓缓退入栈阁道口,然后更衣发丧,扬幡举哀。蜀军皆撞跌而哭,至有哭死者。蜀兵前队正回到栈阁道口,忽见前面火光冲天,喊声震地,一彪军拦路。众将大惊,急报杨仪。正是:已见魏营诸将去,不知蜀地甚兵来。未知来者是何处军马,且看下文分解。