\chapter{邓士载偷度阴平~诸葛瞻战死绵竹}

却说辅国大将军董厥,闻魏兵十余路入境,乃引二万兵守住剑阁;当日望尘头大起,疑
是魏兵,急引军把住关口。董厥自临军前视之,乃姜维、廖化、张翼也。厥大喜,接入关
上,礼毕,哭诉后主黄皓之事。维曰:“公勿忧虑。若有维在,必不容魏来吞蜀也。且守剑
阁,徐图退敌之计。”厥曰:“此关虽然可守,争奈成都无人;倘为敌人所袭,大势瓦解
矣。”维曰:“成都山险地峻,非可易取,不必忧也。”正言间,忽报诸葛绪领兵杀至关
下,维大怒,急引五千兵杀下关来,直撞入魏阵中,左冲右突,杀得诸葛绪大败而走,退数
十里下寨,魏军死者无数。蜀兵抢了许多马匹器械,维收兵回关。

却说钟会离剑阁二十里下寨,诸葛绪自来伏罪。会怒曰:“吾令汝守把阴平桥头,以断
姜维归路,如何失了!今又不得吾令,擅自进兵,以致此败!”绪曰:“维诡计多端,诈取
雍州;绪恐雍州有失,引兵去救,维乘机走脱;绪因赶至关下,不想又为所败。”会大怒,
叱令斩之。监军卫瓘曰:“绪虽有罪,乃邓征西所督之人;不争将军杀之,恐伤和气。”会
曰:“吾奉天子明诏、晋公钧命,特来伐蜀。便是邓艾有罪,亦当斩之!”众皆力劝。会乃
将诸葛绪用槛车载赴洛阳,任晋公发落;随将绪所领之兵,收在部下调遣。

有人报与邓艾。艾大怒曰:“吾与汝官品一般,吾久镇边疆,于国多劳,汝安敢妄自尊
大耶!”子邓忠劝曰:“小不忍则乱大谋,父亲若与他不睦,必误国家大事。望且容忍
之。”艾从其言。然毕竟心中怀怒,乃引十数骑来见钟会。会闻艾至,便问左右:“艾引多
少军来?”左右答曰:“只有十数骑。”会乃令帐上帐下列武士数百人。

艾下马入见。会接入帐礼毕。艾见军容甚肃,心中不安,乃以言挑之曰:“将军得了汉
中,乃朝廷之大幸也,可定策早取剑阁。”会曰:“将军明见若何?”艾再三推称无能。会
固问之。艾答曰:“以愚意度之,可引一军从阴平小路出汉中德阳亭,用奇兵径取成都,姜
维必撤兵来救,将军乘虚就取剑阁,可获全功。”会大喜曰:“将军此计甚妙!可即引兵
去。吾在此专候捷音!”二人饮酒相别。会回本帐与诸将曰:“人皆谓邓艾有能。今日观
之,乃庸才耳!”众问其故。会曰:“阴平小路,皆高山峻岭,若蜀以百余人守其险要,断
其归路,则邓艾之兵皆饿死矣。吾只以正道而行,何愁蜀地不破乎!”遂置云梯炮架,只打
剑阁关。

却说邓艾出辕门上马,回顾从者曰:“钟会待吾若何?”从者曰:“观其辞色,甚不以
将军之言为然,但以口强应而已。”艾笑曰:“彼料我不能取成都,我偏欲取之!”回到本
寨,师纂、邓忠一班将士接问曰:“今日与钟镇西有何高论?”艾曰:“吾以实心告彼,彼
以庸才视我。彼今得汉中,以为莫大之功;若非吾屯沓中绊住姜维,彼安能成功耶!吾今若
取了成都,胜取汉中矣!”当夜下令,尽拔寨望阴平小路进兵,离剑阁七百里下寨,有人报
钟会说:“邓艾要去取成都了。”会笑艾不智。

却说邓艾一面修密书遣使驰报司马昭,一面聚诸将于帐下问曰:“吾今乘虚去取成都,
与汝等立功名于不朽,汝等肯从乎?”诸将应曰:“愿遵军令,万死不辞!”艾乃先令子邓
忠引五千精兵,不穿衣甲,各执斧凿器具,凡遇峻危之处,凿山开路,搭造桥阁,以便军
行。艾选兵三万,各带干粮绳索进发。约行百余里,选下三千兵,就彼扎寨;又行百余里,
又选三千兵下寨。是年十月自阴平进兵,至于巅崖峡谷之中,凡二十余日,行七百余里,皆
是无人之地。魏兵沿途下了数寨,只剩下二千人马。前至一岭,名摩天岭,马不堪行,艾步
行上岭,正见邓忠与开路壮士尽皆哭泣。艾问其故。忠告曰:“此岭西皆是峻壁巅崖,不能
开凿,虚废前劳,因此哭泣。”艾曰:“吾军到此,已行了七百余里,过此便是江油,岂可
复退?”乃唤诸军曰:“不入虎穴,焉得虎子?吾与汝等来到此地,若得成功,富贵共
之。”众皆应曰:“愿从将军之命。”艾令先将军器撺将下去。艾取毡自裹其身,先滚下
去。副将有毡衫者裹身滚下,无毡衫者各用绳索束腰,攀木挂树,鱼贯而进。邓艾、邓忠,
并二千军,及开山壮士,皆度了摩天岭。方才整顿衣甲器械而行,忽见道傍有一石碣,上
刻:“丞相诸葛武侯题”。其文云:“二火初兴,有人越此。二士争衡,不久自死。”艾观
讫大惊,慌忙对碣再拜曰:“武侯真神人也!艾不能以师事之,惜哉!”后人有诗曰:“阴
平峻岭与天齐,玄鹤徘徊尚怯飞。邓艾裹毡从此下,谁知诸葛有先机。”

却说邓艾暗度阴平,引兵行时,又见一个大空寨。左右告曰:“闻武侯在日,曾拨一千
兵守此险隘。今蜀主刘禅废之。”艾嗟呀不已,乃谓众人曰:“吾等有来路而无归路矣!前
江油城中,粮食足备:汝等前进可活,后退即死,须并力攻之。”众皆应曰:“愿死战!”
于是邓艾步行,引二千余人,星夜倍道来抢江油城。却说江油城守将马邈,闻东川已失,虽
为准备,只是提防大路;又仗着姜维全师守住剑阁关,遂将军情不以为重。当日操练人马回
家,与妻李氏拥炉饮酒。其妻问曰:“屡闻边情甚急,将军全无忧色,何也?”邈曰:“大
事自有姜伯约掌握,干我甚事?”其妻曰:“虽然如此,将军所守城池,不为不重。”邈
曰:“天子听信黄皓,溺于酒色,吾料祸不远矣。魏兵若到,降之为上,何必虑哉?”其妻
大怒,唾邈面曰:“汝为男子,先怀不忠不义之心,枉受国家爵禄,吾有何面目与汝相见
耶!”马邈羞惭无语。忽家人慌入报曰:“魏将邓艾不知从何而来,引二千余人,一拥而入
城矣!”邈大惊,慌出纳降,拜伏于公堂之下,泣告曰:“某有心归降久矣。今愿招城中居
民,及本部人马,尽降将军。”艾准其降。遂收江油军马于部下调遣,即用马邈为向导官。
忽报马邈夫人自缢身死。艾问其故,邈以实告。艾感其贤,令厚礼葬之,亲往致祭。魏人闻
者,无不嗟叹。后人有诗赞曰:“后主昏迷汉祚颠,天差邓艾取西川。可怜巴蜀多名将,不
及江油李氏贤。”

邓艾取了江油,遂接阴平小路诸军,皆到江油取齐,径来攻涪城。部将田续曰:“我军
涉险而来,甚是劳顿,且当休养数日,然后进兵。”艾大怒曰:“兵贵神速,汝敢乱我军心
耶!”喝令左右推出斩之。众将苦告方免。艾自驱兵至涪城。城内官吏军民疑从天降,尽皆
投降。

蜀人飞报入成都。后主闻知,慌召黄皓问之。皓奏曰:“此诈传耳。神人必不肯误陛下
也。”后主又宣师婆问时,却不知何处去了。此时远近告急表文,一似雪片,往来使者,联
络不绝。后主设朝计议,多官面面相觑,并无一言。郤正出班奏曰:“事已急矣!陛下可宣
武侯之子商议退兵之策。”原来武侯之子诸葛瞻,字思远。其母黄氏,即黄承彦之女也。母
貌甚陋,而有奇才:上通天文,下察地理;凡韬略遁甲诸书,无所不晓。武侯在南阳时,闻
其贤,求以为室。武侯之学,夫人多所赞助焉。及武侯死后,夫人寻逝,临终遗教,惟以忠
孝勉其子瞻。瞻自幼聪敏,尚后主女,为驸马都尉。后袭父武乡侯之爵。景耀四年,迁行军
护卫将军。时为黄皓用事,故托病不出。当下后主从郤正之言,即时连发三诏,召瞻至殿
下。后主泣诉曰:“邓艾兵已屯涪城,成都危矣。卿看先君之面,救朕之命!”瞻亦泣奏
曰:“臣父子蒙先帝厚恩、陛下殊遇,虽肝脑涂地,不能补报。愿陛下尽发成都之兵,与臣
领去决一死战。”后主即拨成都兵将七万与瞻。瞻辞了后主,整顿军马,聚集诸将问曰:
“谁敢为先锋?”言未讫,一少年将出曰:“父亲既掌大权,儿愿为先锋。”众视之,乃瞻
长子诸葛尚也。尚时年一十九岁。博览兵书。多习武艺。瞻大喜,遂命尚为先锋。是日,大
军离了成都,来迎魏兵。

却说邓艾得马邈献地理图一本,备写涪城至成都三百六十里山川道路,阔狭险峻,一一
分明。艾看毕,大惊曰:“若只守涪城,倘被蜀人据住前山,何能成功耶?如迁延日久,姜
维兵到,我军危矣。”速唤师纂并子邓忠,分付曰:“汝等可引一军,星夜径去绵竹,以拒
蜀兵。吾随后便至。切不可怠缓。若纵他先据了险要,决斩汝首!”

师、邓二人引兵将至锦竹,早遇蜀兵。两军各布成阵。师、邓二人勒马于门旗下,只见
蜀兵列成八阵。三鼕鼓罢,门旗两分,数十员将簇拥一辆四轮车,车上端坐一人:纶巾羽
扇,鹤氅方裾。车傍展开一面黄旗,上书:“汉丞相诸葛武侯”。?得师、邓二人汗流遍
身,回顾军士曰:“原来孔明尚在,我等休矣!”急勒兵回时,蜀兵掩杀将来,魏兵大败而
走。蜀兵掩杀二十余里,遇见邓艾援兵接应。两家各自收兵。艾升帐而坐,唤师纂、邓忠责
之曰:“汝二人不战而退,何也?”忠曰:“但见蜀阵中诸葛孔明领兵,因此奔还。”艾怒
曰:“纵使孔明更生,我何惧哉!汝等轻退,以致于败,宜速斩以正军法!”众皆苦劝,艾
方息怒。令人哨探,回说孔明之子诸葛瞻为大将,瞻之子诸葛尚为先锋。——车上坐者乃木
刻孔明遗像也。

艾闻之,谓师纂、邓忠曰:“成败之机,在此一举。汝二人再不取胜,必当斩首!”
师、邓二人又引一万兵来战。诸葛尚匹马单枪,抖擞精神,战退二人。诸葛瞻指挥两掖兵冲
出,直撞入魏阵中,左冲右突,往来杀有数十番,魏兵大败,死者不计其数。师纂、邓忠中
伤而逃。瞻驱士马随后掩杀二十余里,扎营相拒。师纂、邓忠回见邓艾,艾见二人俱伤,未
便加责,乃与众将商议曰:“蜀有诸葛瞻善继父志,两番杀吾万余人马,今若不速破,后必
为祸。”监军丘本曰:“何不作一书以诱之?”艾从其言,遂作书一封,遣使送人蜀寨。守
门将引至帐下,呈上其书。瞻拆封视之。书曰:“征西将军邓艾,致书于行军护卫将军诸葛
思远麾下:切观近代贤才,未有如公之尊父也。昔自出茅庐,一言已分三国,扫平荆、益,
遂成霸业,古今鲜有及者;后六出祁山,非其智力不足,乃天数耳。今后主昏弱,王气已
终,艾奉天子之命,以重兵伐蜀,已皆得其地矣。成都危在旦夕,公何不应天顺人,仗义来
归?艾当表公为琅琊王,以光耀祖宗,决不虚言。幸存照鉴。”瞻看毕,勃然大怒,扯碎其
书,叱武士立斩来使,令从者持首级回魏营见邓艾。艾大怒,即欲出战。丘本谏曰:“将军
不可轻出,当用奇兵胜之。”艾从其言,遂令天水太守王颀、陇西太守牵弘,伏两军于后,
艾自引兵而来。此时诸葛瞻正欲搦战,忽报邓艾自引兵到。瞻大怒,即引兵出,径杀入魏阵
中。邓艾败走,瞻随后掩杀将来。忽然两下伏兵杀出。蜀兵大败,退入绵竹。艾令围之。于
是魏兵一齐呐喊,将绵竹围的铁桶相似。诸葛瞻在城中,见事势已迫,乃令彭和赍书杀出,
往东吴求救。和至东吴,见了吴主孙休,呈上告急之书。吴主看罢,与群臣计议曰:“既蜀
中危急,孤岂可坐视不救。”即令考将丁奉为主帅,丁封、孙异为副将,率兵五万,前往救
蜀。丁奉领旨出师,分拨丁封、孙异引兵二万向沔中而进,自率兵三万向寿春而进:分兵三
路来援。

却说诸葛瞻见救兵不至,谓众将曰:“久守非良图。”遂留子尚与尚书张遵守城,瞻自
披挂上马,引三军大开三门杀出。邓艾见兵出,便撤兵退。瞻奋力追杀,忽然一声炮响,四
面兵合,把瞻困在垓心。瞻引兵左冲右突,杀死数百人。艾令众军放箭射之,蜀兵四散。瞻
中箭落马,乃大呼曰:“吾力竭矣,当以一死报国!”遂拔剑自刎而死。其子诸葛尚在城
上,见父死于军中,勃然大怒,遂披挂上马。张遵谏曰:“小将军勿得轻出。”尚叹曰:
“吾父子祖孙,荷国厚恩,今父既死于敌,我何用生为!”遂策马杀出,死于阵中。后人有
诗赞瞻、尚父子曰:“不是忠臣独少谋,苍天有意绝炎刘。当年诸葛留嘉胤,节义真堪继武
侯。”邓艾怜其忠,将父子合葬。乘虚攻打绵竹。张遵、黄崇、李球三人,各引一军杀出。
蜀兵寡,魏兵众,三人亦皆战死。艾因此得了绵竹。劳军已毕,遂来取成都。正是:试观后
主临危日,无异刘璋受逼时。未知成都如何守御,且看下文分解。