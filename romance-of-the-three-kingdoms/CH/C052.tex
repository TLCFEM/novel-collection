\chapter{诸葛亮智辞鲁肃~赵子龙计取桂阳}

却说周瑜见孔明袭了南郡,又闻他袭了荆襄,如何不气?气伤箭疮,半晌方苏,众将再三劝解。瑜曰:“若不杀诸葛村夫,怎息我心中怨气!程德谋可助我攻打南郡,定要夺还东吴。”正议间,鲁肃至。瑜谓之曰:“吾欲起兵与刘备、诸葛亮共决雌雄,复夺城池。子敬幸助我。”鲁肃曰:“不可。方今与曹操相持,尚未分成败;主公现攻合淝不下。不争自家互相吞并,倘曹兵乘虚而来,其势危矣。况刘玄德旧曾与曹操相厚,若逼得紧急,献了城池,一同攻打东吴,如之奈何?”瑜曰:“吾等用计策,损兵马,费钱粮,他去图现成,岂不可恨!”肃曰:“公瑾且耐。容某亲见玄德,将理来说他。若说不通,那时动兵未迟。”诸将曰:“子敬之言甚善。”

于是鲁肃引从者径投南郡来,到城下叫门。赵云出问,肃曰:“我要见刘玄德有话说。”云答曰:“吾主与军师在荆州城中。”肃遂不入南郡,径奔荆州。见旌旗整列,军容甚盛,肃暗羡曰:“孔明真非常人也!”军士报入城中,说鲁子敬要见。孔明令大开城门,接肃入衙。讲礼毕,分宾主而坐。茶罢,肃曰:“吾主吴侯,与都督公瑾,教某再三申意皇叔,前者,操引百万之众,名下江南,实欲来图皇叔;幸得东吴杀退曹兵,救了皇叔。所有荆州九郡,合当归于东吴。今皇叔用诡计,夺占荆襄,使江东空费钱粮军马,而皇叔安受其利,恐于理未顺。”孔明曰:“子敬乃高明之士,何故亦出此言?常言道:物必归主。荆襄九郡,非东吴之地,乃刘景升之基业。吾主固景升之弟也。景升虽亡,其子尚在;以叔辅侄,而取荆州,有何不可?”肃曰:“若果系公子刘琦占据,尚有可解;今公子在江夏,须不在这里!”孔明曰:“子敬欲见公子乎?”便命左右:“请公子出来。”只见两从者从屏风后扶出刘琦。琦谓肃曰:“病躯不能施礼,子敬勿罪。”鲁肃吃了一惊,默然无语,良久,言曰:“公子若不在,便如何?”孔明曰:“公子在一日,守一日;若不在,别有商议。”肃曰:“若公子不在,须将城池还我东吴。”孔明曰:“子敬之言是也。”遂设宴相待。

宴罢,肃辞出城,连夜归寨,具言前事。瑜曰:“刘琦正青春年少,如何便得他死?这荆州何日得还?”肃曰:“都督放心。只在鲁肃身上,务要讨荆襄还东吴。”瑜曰:“子敬有何高见?”肃曰:“吾观刘琦过于酒色,病入膏肓,现今面色羸瘦,气喘呕血,不过半年,其人必死。那时往取荆州,刘备须无得推故。”周瑜犹自忿气未消,忽孙权遣使至。瑜令请入。使曰:“主公围合淝,累战不捷。特令都督收回大军,且拨兵赴合淝相助。”周瑜只得班师回柴桑养病,令程普部领战船士卒,来合淝听孙权调用。

却说刘玄德自得荆州、南郡、襄阳,心中大喜,商议久远之计。忽见一人上厅献策,视之,乃伊籍也。玄德感其旧日之恩,十分相敬,坐而问之。籍曰:“要知荆州久远之计,何不求贤士以问之?”玄德曰:“贤士安在?”籍曰:“荆襄马氏,兄弟五人并有才名:幼者名谡,字幼常;其最贤者,眉间有白毛,名良,字季常。乡里为之谚曰:‘马氏五常,白眉最良。’公何不求此人而与之谋?”玄德遂命请之。马良至,玄德优礼相待,请问保守荆襄之策。良曰:“荆襄四面受敌之地,恐不可久守;可令公子刘琦于此养病,招谕旧人以守之,就表奏公子为荆州刺史,以安民心。然后南征武陵、长沙、桂阳、零陵四郡,积收钱粮,以为根本。此久远之计也。”玄德大喜,遂问:“四郡当先取何郡?”良曰:“湘江之西,零陵最近,可先取之;次取武陵。然后湘江之东取桂阳;长沙为后。”玄德遂用马良为从事,伊籍副之。请孔明商议送刘琦回襄阳,替云长回荆州。便调兵取零陵,差张飞为先锋,赵云合后,孔明;玄德为中军,人马一万五千;留云长守荆州、糜竺、刘封守江陵。却说零陵太守刘度,闻玄德军马到来,乃与其子刘贤商议。贤曰:“父亲放心。他虽有张飞、赵云之勇,我本州上将邢道荣,力敌万人,可以抵对。”刘度遂命刘贤与邢道荣引兵万余,离城三十里,依山靠水下寨。探马报说:“孔明自引一军到来。”道荣便引军出战。两阵对圆,道荣出马,手使开山大斧,厉声高叫:“反贼安敢侵我境界!”只见对阵中,一簇黄旗出。旗开处,推出一辆四轮车,车中端坐一人,头戴纶巾,身披鹤氅,手执羽扇,用扇招邢道荣曰:“吾乃南阳诸葛孔明也。曹操引百万之众,被吾聊施小计,杀得片甲不回。汝等岂堪与我对敌?我今来招安汝等,何不早降?”道荣大笑曰:“赤壁鏖兵,乃周郎之谋也,干汝何事,敢来诳语!”轮大斧竟奔孔明。孔明便回车,望阵中走,阵门复闭。道荣直冲杀过来,阵势急分两下而走。道荣遥望中央一簇黄旗,料是孔明,乃只望黄旗而赶。抹过山脚,黄旗扎住,忽地中央分开,不见四轮车,只见一将挺矛跃马,大喝一声,直取道荣,乃张翼德也。道荣轮大斧来迎,战不数合,气力不加,拨马便走。翼德随后赶来,喊声大震,两下伏兵齐出。道荣舍死冲过,前面一员大将,拦住去路,大叫:“认得常山赵子龙否!”道荣料敌不过,又无处奔走,只得下马请降。子龙缚来寨中见玄德、孔明。玄德喝教斩首。孔明急止之,问道荣曰:“汝若与我捉了刘贤,便准你投降。”道荣连声愿往。孔明曰:“你用何法捉他?”道荣曰:“军师若肯放某回去,某自有巧说。今晚军师调兵劫寨,某为内应,活捉刘贤,献与军师。刘贤既擒,刘度自降矣。”玄德不信其言。孔明曰:“邢将军非谬言也。”遂放道荣归。道荣得放回寨,将前事实诉刘贤。贤曰:“如之奈何?”道荣曰:“可将计就计。今夜将兵伏于寨外,寨中虚立旗幡,待孔明来劫寨,就而擒之。”刘贤依计。

当夜二更,果然有一彪军到寨口,每人各带草把,一齐放火。刘贤、道荣两下杀来,放火军便退。刘贤、道荣两军乘势追赶,赶了十余里,军皆不见。刘贤、道荣大惊,急回本寨,只见火光未灭,寨中突出一将,乃张翼德也。刘贤叫道荣:“不可入寨,却去劫孔明寨便了。”于是复回军。走不十里,赵云引一军刺斜里杀出,一枪刺道荣于马下。刘贤急拨马奔走,背后张飞赶来,活捉过马,绑缚见孔明。贤告曰:“邢道荣教某如此,实非本心也。”孔明令释其缚,与衣穿了,赐酒压惊,教人送入城说父投降;如其不降,打破城池,满门尽诛。刘贤回零陵见父刘度,备述孔明之德,劝父投降。度从之,遂于城上竖起降旗,大开城门,赍捧印绶出城,竟投玄德大寨纳降。孔明教刘度仍为郡守,其子刘贤赴荆州随军办事。零陵一郡居民,尽皆喜悦。

玄德入城安抚已毕,赏劳三军。乃问众将曰:“零陵已取了,桂阳郡何人敢取?”赵云应曰:“某愿往。”张飞奋然出曰:“飞亦愿往!”二人相争。孔明曰:“终是子龙先应,只教子龙去。”张飞不服,定要去取。孔明教拈阉,拈着的便去。又是子龙拈着。张飞怒曰:“我并不要人相帮,只独领三千军去,稳取城池。”赵云曰:“某也只领三千军去。如不得城,愿受军令。”孔明大喜,责了军令状,选三千精兵付赵云去。张飞不服,玄德喝退。赵云领了三千人马,径往桂阳进发。早有探马报知桂阳太守赵范。范急聚众商议。管军校尉陈应、鲍隆愿领兵出战。原来二人都是桂阳岭山乡猎户出身,陈应会使飞叉,鲍隆曾射杀双虎。二人自恃勇力,乃对赵范曰:“刘备若来,某二人愿为前部。”赵范曰:“我闻刘玄德乃大汉皇叔;更兼孔明多谋,关、张极勇;今领兵来的赵子龙,在当阳长坂百万军中,如入无人之境。我桂阳能有多少人马?不可迎敌,只可投降。”应曰:“某请出战。若擒不得赵云,那时任太守投降不迟。”赵范拗不过,只得应允。陈应领三千人马出城迎敌,早望见赵云领军来到。陈应列成阵势,飞马绰叉而出。赵云挺枪出马,责骂陈应曰:“吾主刘玄德,乃刘景升之弟,今辅公子刘琦同领荆州,特来抚民。汝何敢迎敌!”陈应骂曰:“我等只服曹丞相,岂顺刘备!”赵云大怒,挺枪骤马,直取陈应。应捻叉来迎,两马相交,战到四五合,陈应料敌不过,拨马便走。赵云追赶。陈应回顾赵云马来相近,用飞叉掷去,被赵云接住。回掷陈应。应急躲过,云马早到,将陈应活捉过马,掷于地下,喝军士绑缚回寨。败军四散奔走。云入寨叱陈应曰:“量汝安敢敌我!我今不杀汝,放汝回去;说与赵范,早来投降。”陈应谢罪,抱头鼠窜,回到城中,对赵范尽言其事。范曰:“我本欲降,汝强要战,以致如此。”遂叱退陈应,赍捧印绶,引十数骑出城投大寨纳降。云出寨迎接,待以宾礼,置酒共饮,纳了印绶,酒至数巡,范曰:“将军姓赵,某亦姓赵,五百年前,合是一家。将军乃真定人,某亦真定人,又是同乡。倘得不弃,结为兄弟,实为万幸。”云大喜,各叙年庚。云与范同年。云长范四个月,范遂拜云为兄。二人同乡,同年,又同姓,十分相得。至晚席散,范辞回城。次日,范请云入城安民。云教军士休动,只带五十骑随入城中。居民执香伏道而接。云安民已毕,赵范邀请入衙饮宴。酒至半酣,范复邀云入后堂深处,洗盏更酌。云饮微醉。范忽请出一妇人,与云把酒。子龙见妇人身穿缟素,有倾国倾城之色,乃问范曰:“此何人也?”范曰:“家嫂樊氏也。”子龙改容敬之。樊氏把盏毕,范令就坐。云辞谢。樊氏辞归后堂。云曰:“贤弟何必烦令嫂举杯耶?”范笑曰:“中间有个缘故,乞兄勿阻:先兄弃世已三载,家嫂寡居,终非了局,弟常劝其改嫁。嫂曰:‘若得三件事兼全之人,我方嫁之:第一要文武双全,名闻天下;第二要相貌堂堂,威仪出众;第三要与家兄同姓。’你道天下那得有这般凑巧的?今尊兄堂堂仪表,名震四海,又与家兄同姓,正合家嫂所言。若不嫌家嫂貌陋,愿陪嫁资,与将军为妻,结累世之亲,如何?”云闻言大怒而起,厉声曰:“吾既与汝结为兄弟,汝嫂即吾嫂也,岂可作此乱人伦之事乎!”赵范羞惭满面,答曰:“我好意相待,如何这般无礼!”遂目视左右,有相害之意。云已觉,一拳打倒赵范,径出府门,上马出城去了。

范急唤陈应、鲍隆商议。应曰:“这人发怒去了,只索与他厮杀。”范曰:“但恐赢他不得。”鲍隆曰:“我两个诈降在他军中,太守却引兵来搦战,我二人就阵上擒之。”陈应曰:“必须带些人马。”隆曰:“五百骑足矣。”当夜二人引五百军径奔赵云寨来投降。云已心知其诈,遂教唤入。二将到帐下,说:“赵范欲用美人计赚将军,只等将军醉了,扶入后堂谋杀,将头去曹丞相处献功:如此不仁。某二人见将军怒出,必连累于某,因此投降。”赵云佯喜,置酒与二人痛饮。二人大醉,云乃缚于帐中,擒其手下人问之,果是诈降。云唤五百军入,各赐酒食,传令曰:“要害我者,陈应、鲍隆也;不干众人之事。汝等听吾行计,皆有重赏。”众军拜谢。将降将陈、鲍二人当时斩了;却教五百军引路,云引一千军在后,连夜到桂阳城下叫门。城上听时,说陈、鲍二将军杀了赵云回军,请太守商议事务。城上将火照看,果是自家军马。赵范急忙出城。云喝左右捉下,遂入城,安抚百姓已定,飞报玄德。

玄德与孔明亲赴桂阳。云迎接入城,推赵范于阶下。孔明问之,范备言以嫂许嫁之事。孔明谓云曰:“此亦美事,公何如此?”云曰:“赵范既与某结为兄弟,今若娶其嫂,惹人唾骂,一也;其妇再嫁,使失大节,二也;赵范初降,其心难测,三也。主公新定江汉,枕席未安,云安敢以一妇人而废主公之大事?”玄德曰:“今日大事已定,与汝娶之,若何?”云吾:“天下女子不少,但恐名誉不立,何患无妻子乎?”玄德曰:“子龙真丈夫也!”遂释赵范,仍令为桂阳太守,重赏赵云。张飞大叫曰:“偏子龙干得功!偏我是无用之人!只拨三千军与我去取武陵郡,活捉太守金旋来献!”孔明大喜曰:“翼德要去不妨,但要依一件事。”正是:军师决胜多奇策,将士争先立战功。未知孔明说出那一件事来,且看下文分解。