\chapter{庞令明抬榇决死战~关云长放水淹七军}

却说曹操欲使于禁赴樊城救援,问众将谁敢作先锋。一人应声愿往。操视之,乃庞德
也。操大喜曰:“关某威震华夏,未逢对手;今遇令明,真劲敌也。”遂加于禁为征南将
军,加宠德为征西都先锋,大起七军,前往樊城。这七军,皆北方强壮之士。两员领军将
校:一名董衡,一名董超;当日引各头目参拜于禁。董衡曰:“今将军提七枝重兵,去解樊
城之厄,期在必胜,乃用庞德为先锋,岂不误事?”禁惊问其故。衡曰:“庞德原系马超手
下副将,不得已而降魏;今其故主在蜀,职居五虎上将;况其亲兄庞柔亦在西川为官,今使
他为先锋,是泼油救火也。将军何不启知魏王,别换一人去?”

禁闻此语,遂连夜入府启知曹操。操省悟,即唤庞德至阶下,令纳下先锋印。德大惊
曰:“某正欲与大王出力,何故不肯见用?”操曰:“孤本无猜疑;但今马超现在西川,汝
兄庞柔亦在西川,俱佐刘备。孤纵不疑,奈众口何?”庞德闻之,免冠顿首,流血满面而告
曰:“某自汉中投降大王,每感厚恩,虽肝脑涂地,不能补报;大王何疑于德也?德昔在故
乡时,与兄同居,嫂甚不贤,德乘醉杀之;兄恨德入骨髓,誓不相见,恩已断矣。故主马
超,有勇无谋,兵败地亡,孤身入川,今与德各事其主,旧义已绝。德感大王恩遇,安敢萌
异志?惟大王察之。”操乃扶起庞德,抚慰曰:“孤素知卿忠义,前言特以安众人之心耳。
卿可努力建功。卿不负孤,孤亦必不负卿也。”德拜谢回家,令匠人造一木榇。次日,请诸
友赴席,列榇于堂。众亲友见之,皆惊问曰:“将军出师,何用此不祥之物?”德举杯谓亲
友曰:“吾受魏王厚恩,誓以死报。今去樊城与关某决战,我若不能杀彼,必为彼所杀;即
不为彼所杀,我亦当自杀。故先备此榇,以示无空回之理。”众皆嗟叹。德唤其妻李氏与其
子庞会出,谓其妻曰:“吾今为先锋,义当效死疆场。我若死,汝好生看养吾儿;吾儿有异
相,长大必当与吾报仇也。”妻子痛哭送别,德令扶榇而行。临行,谓部将曰:“吾今去与
关某死战,我若被关某所杀,汝等即取吾尸置此榇中;我若杀了关某,吾亦即取其首,置此
榇内,回献魏王。”部将五百人皆曰:“将军如此忠勇,某等敢不竭力相助!”于是引军前
进。有人将此言报知曹操。操喜曰:“庞德忠勇如此,孤何忧焉!”贾诩曰:“庞德恃血气
之勇,欲与关某决死战,臣窃虑之。”操然其言,急令人传旨戒庞德曰:“关某智勇双全,
切不可轻敌。可取则取,不可取则宜谨守。”庞德闻命,谓众将曰:“大王何重视关某也?
吾料此去,当挫关某三十年之声价。”禁曰:“魏王之言,不可不从。”德奋然趱军前至樊
城,耀武扬威,鸣锣击鼓。

却说关公正坐帐中,忽探马飞报:“曹操差于禁为将,领七枝精壮兵到来。前部先锋庞
德,军前抬一木榇,口出不逊之言,誓欲与将军决一死战。兵离城止三十里矣。”关公闻
言,勃然变色,美髯飘动,大怒曰:“天下英雄,闻吾之名,无不畏服;庞德竖子,何敢藐
视吾耶!关平一面攻打樊城,吾自去斩此匹夫,以雪吾恨!”平曰:“父亲不可以泰山之
重,与顽石争高下。辱子愿代父去战庞德。”关公曰:“汝试一往,吾随后便来接应。”关
平出帐,提刀上马,领兵来迎庞德。两阵对圆,魏营一面皂旗上大书“南安庞德”四个白
字。庞德青袍银铠,钢刀白马,立于阵前;背后五百军兵紧随,步卒数人肩抬木榇而出。关
平大骂庞德:“背主之贼!”庞德问部卒曰:“此何人也?”或答曰:“此关公义子关平
也。”德叫曰:“吾奉魏王旨,来取汝父之首!汝乃疥癞小儿,吾不杀汝!快唤汝父来!”
平大怒,纵马舞刀,来取庞德。德横刀来迎。战三十合,不分胜负,两家各歇。早有人报知
关公。公大怒,令廖化去攻樊城,自己亲来迎敌庞德。关平接着,言与庞德交战,不分胜
负。关公随即横刀出马,大叫曰:“关云长在此,庞德何不早来受死!”鼓声响处,庞德出
马曰:“吾奉魏王旨,特来取汝首!恐汝不信,备榇在此。汝若怕死,早下马受降!”关公
大骂曰:“量汝一匹夫,亦何能为!可惜我青龙刀斩汝鼠贼!”纵马舞刀,来取庞德。德轮
刀来迎。二将战有百余合,精神倍长。两军各看得痴呆了。魏军恐庞德有失,急令鸣金收
军。关平恐父年老,亦急鸣金。二将各退。庞德归寨,对众曰:“人言关公英雄,今日方信
也。”正言间,于禁至。相见毕,禁曰:“闻将军战关公,百合之上,未得便宜,何不且退
军避之?”德奋然曰:“魏王命将军为大将,何太弱也?吾来日与关某共决一死,誓不退
避!”禁不敢阻而回。

却说关公回寨,谓关平曰:“庞德刀法惯熟,真吾敌手。”平曰:“俗云初生之犊不惧
虎,父亲纵然斩了此人,只是西羌一小卒耳;倘有疏虞,非所以重伯父之托也。”关公曰:
“吾不杀此人,何以雪恨?吾意已决,再勿多言!”次日,上马引兵前进。庞德亦引兵来
迎。两阵对圆,二将齐出,更不打话,出马交锋。斗至五十余合,庞德拨回马,拖刀而走。
关公随后追赶。关平恐有疏失,亦随后赶去。关公口中大骂:“庞贼!欲使拖刀计,吾岂惧
汝?”原来庞德虚作拖刀势,却把刀就鞍鞒挂住,偷拽雕弓,搭上箭,射将来。关平眼快,
见庞德拽弓,大叫:“贼将休放冷箭!”关公急睁眼看时,弓弦响处,箭早到来;躲闪不
及,正中左臂。关平马到,救父回营。庞德勒回马轮刀赶来,忽听得本营锣声大震。德恐后
军有失,急勒马回。原来于禁见庞德射中关公,恐他成了大功,灭己威风,故鸣金收军。庞
德回马,问:“何故鸣金?”于禁曰:“魏王有戒:关公智勇双全。他虽中箭,只恐有诈,
故鸣金收军。”德曰:“若不收军,吾已斩了此人也。”禁曰:“紧行无好步,当缓图
之。”庞德不知于禁之意,只懊悔不已。

却说关公回营,拔了箭头。幸得箭射不深,用金疮药敷之。关公痛恨庞德,谓众将曰:
“吾誓报此一箭之仇!”众将对曰:“将军且暂安息几日,然后与战未迟。”次日,人报庞
德引军搦战。关公就要出战。众将劝住。庞德令小军毁骂。关平把住隘口,分付众将休报知
关公。庞德搦战十余日,无人出迎,乃与于禁商议曰:“眼见关公箭疮举发,不能动止;不
若乘此机会,统七军一拥杀入寨中,可救樊城之围。”于禁恐庞德成功,只把魏王戒旨相
推,不肯动兵。庞德累欲动兵,于禁只不允,乃移七军转过山口,离樊城北十里,依山下
寨,禁自领兵截断大路,令庞德屯兵于谷后,使德不能进兵成功。

却说关平见关公箭疮已合,甚是喜悦。忽听得于禁移七军于樊城之北下寨,未知其谋,
即报知关公。公遂上马,引数骑上高阜处望之,见樊城城上旗号不整,军士慌乱;城北十里
山谷之内,屯着军马;又见襄江水势甚急,看了半响,唤向导官问曰:“樊城北十里山谷,
是何地名?”对曰:“罾口川也。”关公喜曰:“于禁必为我擒矣。”将士问曰:“将军何
以知之?”关公曰:“鱼入罾口,岂能久乎?”诸将未信。公回本寨。时值八月秋天,骤雨
数日。公令人预备船筏,收拾水具。关平问曰:“陆地相持,何用水具?”公曰:“非汝所
知也。于禁七军不屯于广易之地,而聚于罾口川险隘之处;方今秋雨连绵,襄江之水必然泛
涨;吾已差人堰住各处水口,待水发时,乘高就船,放水一淹,樊城罾口川之兵皆为鱼鳖
矣。”关平拜服。却说魏军屯于罾口川,连日大雨不止,督将成何来见于禁曰:“大军屯于
川口,地势甚低;虽有土山,离营稍远。即今秋雨连绵,军士艰辛。近有人报说荆州兵移于
高阜处,又于汉水口预备战筏;倘江水泛涨,我军危矣,宜早为计。”于禁叱曰:“匹夫惑
吾军心耶!再有多言者斩之!”成何羞惭而退,却来见庞德,说此事。德曰:“汝所见甚
当。于将军不肯移兵,吾明日自移军屯于他处。”

计议方定,是夜风雨大作。庞德坐于帐中,只听得万马争奔,征鼙震地。德大惊,急出
帐上马看时,四面八方,大水骤至;七军乱窜,随波逐浪者,不计其数。平地水深丈余,于
禁、庞德与诸将各登小山避水。比及平明,关公及众将皆摇旗鼓噪,乘大船而来。于禁见四
下无路,左右止有五六十人,料不能逃,口称愿降。关公令尽去衣甲,拘收入船,然后来擒
庞德。时庞德并二董及成何,与步卒五百人,皆无衣甲,立在堤上。见关公来,庞德全无惧
怯,奋然前来接战。关公将船四面围定,军士一齐放箭,射死魏兵大半。董衡、董超见势已
危,乃告庞德曰:“军士折伤大半,四下无路,不如投降。”庞德大怒曰:“吾受魏王厚
恩,岂肯屈节于人!”遂亲斩董衡、董超于前,厉声曰:“再说降者,以此二人为例!”于
是众皆奋力御敌。自平明战至日中,勇力倍增。关公催四面急攻,矢石如雨。德令军士用短
兵接战。德回顾成何曰:“吾闻勇将不怯死以苟免,壮士不毁节而求生。今日乃我死日也。
汝可努力死战。”成何依令向前,被关公一箭射落水中。众军皆降,止有庞德一人力战。正
遇荆州数十人,驾小船近堤来,德提刀飞身一跃,早上小船,立杀十余人,余皆弃船赴水逃
命。庞德一手提刀,一手使短棹,欲向樊城而走。只见上流头,一将撑大筏而至,将小船撞
翻,庞德落于水中。船上那将跳下水去,生擒庞德上船。众视之,擒庞德者,乃周仓也。仓
素知水性,又在荆州住了数年,愈加惯熟;更兼力大,因此擒了庞德。于禁所领七军,皆死
于水中。其会水者料无去路,亦皆投降。后人有诗曰:“夜半征鼙响震天,襄樊平地作深
渊。关公神算谁能及,华夏威名万古传。”

关公回到高阜去处,升帐而坐。群刀手押过于禁来。禁拜伏于地,乞哀请命。关公曰:
“汝怎敢抗吾?”禁曰:“上命差遣,身不由己。望君侯怜悯,誓以死报。”公绰髯笑曰:
“吾杀汝,犹杀狗彘耳,空污刀斧!”令人缚送荆州大牢内监候:“待吾回,别作区处。”
发落去讫。关公又令押过庞德。德睁眉怒目,立而不跪,关公曰:“汝兄现在汉中;汝故主
马超,亦在蜀中为大将。汝如何不早降?”德大怒曰:“吾宁死于刀下,岂降汝耶!”骂不
绝口。公大怒,喝令刀斧手推出斩之。德引颈受刑。关公怜而葬之。于是乘水势未退,复上
战船,引大小将校来攻樊城。却说樊城周围,白浪滔天,水势益甚,城垣渐渐浸塌,男女担
土搬砖,填塞不住。曹军众将,无不丧胆,慌忙来告曹仁曰:“今日之危,非力可救;可趁
敌军未至,乘舟夜走,虽然失城,尚可全身。”仁从其言。方欲备船出走,满宠谏曰:“不
可。山水骤至,岂能长存?不旬日即当自退。关公虽未攻城,已遣别将在郏下。其所以不敢
轻进者,虑吾军袭其后也。今若弃城而去,黄河以南,非国家之有矣。”愿将军固守此城,
以为保障。”仁拱手称谢曰:“非伯宁之教,几误大事。”乃骑白马上城,聚众将发誓曰:
“吾受魏王命,保守此城;但有言弃城而去者斩!”诸将皆曰:“某等愿以死据守!”仁大
喜,就城上设弓弩数百,军士昼夜防护,不敢懈怠。老幼居民,担土石填塞城垣。旬日之
内,水势渐退。

关公自擒魏将于禁等,威震天下,无不惊骇。忽次子关兴来寨内省亲。公就令兴赍诸官
立功文书去成都见汉中王,各求升迁。兴拜辞父亲,径投成都去讫。

却说关公分兵一半,直抵郏下。公自领兵四面攻打樊城。当日关公自到北门,立马扬
鞭,指而问曰:“汝等鼠辈,不早来降,更待何时?”正言间,曹仁在敌楼上,见关公身上
止披掩心甲,斜袒着绿袍,乃急招五百弓弩手,一齐放箭。公急勒马回时,右臂上中一弩
箭,翻身落马。正是:水里七军方丧胆,城中一箭忽伤身。未知关公性命如何,且看下文分
解。