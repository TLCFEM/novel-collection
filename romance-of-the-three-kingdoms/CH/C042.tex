\chapter{张翼德大闹长坂桥~刘豫州败走汉津口}

却说钟缙、钟绅二人拦住赵云厮杀。赵云挺枪便刺,钟缙当先挥大斧来迎。两马相交,战不三合。被云一枪刺落马下,夺路便走。背后钟绅持戟赶来,马尾相衔,那枝戟只在赵云后心内弄影。云急拨转马头,恰好两胸相拍。云左手持枪隔过画戟,右手拔出青釭宝剑砍去,带盔连脑,砍去一半,绅落马而死,余众奔散。赵云得脱,望长坂桥而走,只闻后面喊声大震,原来文聘引军赶来。赵云到得桥边,人困马乏。见张飞挺矛立马于桥上,云大呼曰:“翼德援我!”飞曰:“子龙速行,追兵我自当之。”

云纵马过桥,行二十余里,见玄德与众人憩于树下。云下马伏地而泣。玄德亦泣。云喘息而言曰:“赵云之罪,万死犹轻!糜夫人身带重伤,不肯上马,投井而死,云只得推土墙掩之。怀抱公子,身突重围;赖主公洪福,幸而得脱。适来公子尚在怀中啼哭,此一会不见动静,多是不能保也。”遂解视之,原来阿斗正睡着未醒。云喜曰:“幸得公子无恙!”双手递与玄德。玄德接过,掷之于地曰:“为汝这孺子,几损我一员大将!”赵云忙向地下抱起阿斗,泣拜曰:“云虽肝脑涂地,不能报也!”后人有诗曰:“曹操军中飞虎出,赵云怀内小龙眠。无由抚慰忠臣意,故把亲儿掷马前。”

却说文聘引军追赵云至长坂桥,只见张飞倒竖虎须,圆睁环眼,手绰蛇矛,立马桥上,又见桥东树林之后,尘头大起,疑有伏后,便勒住马,不敢近前。俄而曹仁、李典、夏侯惇、夏侯渊、乐进、张辽、张郃、许褚等都至。见飞怒目横矛,立马于桥上,又恐是诸葛孔明之计,都不敢近前。扎住阵脚,一字儿摆在桥西,使人飞报曹操。操闻知,急上马,从阵后来。张飞睁圆环眼,隐隐见后军青罗伞盖、旄钺旌旗来到,料得是曹操心疑,亲自来看。飞乃厉声大喝曰:“我乃燕人张翼德也!谁敢与我决一死战?”声如巨雷。曹军闻之,尽皆股栗。曹操急令去其伞盖,回顾左右曰:“我向曾闻云长言:翼德于百万军中,取上将之首,如探囊取物。今日相逢,不可轻敌。”言未已,张飞睁目又喝曰:“燕人张翼德在此!谁敢来决死战?”曹操见张飞如此气概,颇有退心。飞望见曹操后军阵脚移动,乃挺矛又喝曰:“战又不战,退又不退,却是何故!”喊声未绝,曹操身边夏侯杰惊得肝胆碎裂,倒撞于马下。操便回马而走。于是诸军众将一齐望西奔走。正是:黄口孺子,怎闻霹雳之声;病体樵夫,难听虎豹之吼。一时弃枪落盔者,不计其数,人如潮涌,马似山崩,自相践踏。后人有诗赞曰:“长坂桥头杀气生,横枪立马眼圆睁。一声好似轰雷震,独退曹家百万兵。”

却说曹操惧张飞之威,骤马望西而走,冠簪尽落,披发奔逃。张辽、许褚赶上,扯住辔环。曹操仓皇失措。张辽曰:“丞相休惊。料张飞一人,何足深惧!今急回军杀去,刘备可擒也。”曹操神色方才稍定,乃令张辽、许褚再至长坂桥探听消息。且说张飞见曹军一拥而退,不敢追赶;速唤回原随二十余骑,解去马尾树枝,令将桥梁拆断,然后回马来见玄德,具言断桥一事。玄德曰:“吾弟勇则勇矣,惜失于计较。”飞问其故。玄德曰:“曹操多谋。汝不合拆断桥梁,彼必追至矣。”飞曰:“他被我一喝,倒退数里,何敢再追?”玄德曰:“若不断桥,彼恐有埋伏,不敢进兵,今拆断了桥,彼料我无军而怯,必来追赶。彼有百万之众,虽涉江汉,可填而过,岂惧一桥之断耶?”于是即刻起身,从小路斜投汉津,望沔阳路而走。却说曹操使张辽、许褚探长坂桥消息,回报曰:“张飞已拆断桥梁而去矣。”操曰:“彼断桥而去,乃心怯也。”遂传令差一万军,速搭三座浮桥,只今夜就要过。李典曰:“此恐是诸葛亮之诈谋,不可轻进。”操曰:“张飞一勇之夫,岂有诈谋!”遂传下号令,火速进兵。

却说玄德行近汉津,忽见后面尘头大起,鼓声连天,喊声震地。玄德曰:“前有大江,后有追兵,如之奈何?”急命赵云准备抵敌。曹操下令军中曰:“今刘备釜中之鱼,阱中之虎;若不就此时擒捉,如放鱼入海,纵虎归山矣。众将可努力向前。”众将领命,一个个奋威追赶。忽山坡后鼓声响处,一队军马飞出,大叫曰:“我在此等候多时了!”当头那员大将,手执青龙刀,坐下赤兔马,原来是关云长,去江夏借得军马一万,探知当阳长坂大战,特地从此路截出。曹操一见云长,即勒住马回顾众将曰:“又中诸葛亮之计也!”传令大军速退。

云长追赶十数里,即回军保护玄德等到汉津,已有船只伺候,云长请玄德并甘夫人、阿斗至船中坐定。云长问曰:“二嫂嫂如何不见?”玄德诉说当阳之事。云长叹曰:“曩日猎于许田时,若从吾意,可无今日之患。”玄德曰:“我于此时亦投鼠忌器耳。”正说之间,忽见江南岸战鼓大鸣,舟船如蚁,顺风扬帆而来。玄德大惊。船来至近,只见一人白袍银铠,立于船头上大呼曰:“叔父别来无恙!”小侄得罪。”玄德视之,乃刘琦也。琦过船哭拜曰:“闻叔父困于曹操,小侄特来接应。”玄德大喜,遂合兵一处,放舟而行。在船中正诉情由,江西南上战船一字儿摆开,乘风唿哨而至,刘琦惊曰:“江夏之兵,小侄已尽起至此矣。今有战船拦路,非曹操之军,即江东之军也,如之奈何?”玄德出船头视之,见一人纶巾道服,坐在船头上,乃孔明也,背后立着孙乾。玄德慌请过船,问其何故却在此。孔明曰:“亮自至江夏,先令云长于汉津登陆地而接。我料曹操必来追赶,主公必不从江陵来,必斜取汉津矣;故特请公子先来接应,我竟往夏口,尽起军前来相助。”玄德大悦,合为一处,商议破曹之策。孔明曰:“夏口城险,颇有钱粮,可以久守。请主公且到夏口屯住。公子自回江夏,整顿战船,收拾军器,为掎角之势,可以抵当曹操。若共归江夏,则势反孤矣。”刘琦曰:“军师之言甚善。但愚意欲请叔父暂至江夏;整顿军马停当,再回夏口不迟。”玄德曰:“贤侄之言亦是。”遂留下云长,引五千军守夏口。玄德、孔明、刘琦共投江夏。

却说曹操见云长在旱路引军截出,疑有伏兵,不敢来追;又恐水路先被玄德夺了江陵,便星夜提兵赴江陵来。荆州治中邓义、别驾刘先,已备知襄阳之事,料不能抵敌曹操,遂引荆州军民出郭投降。曹操入城、安民已定,释韩嵩之囚,加为大鸿胪。其余众官,各有封赏。曹操与众将议曰:“今刘备已投江夏,恐结连东吴,是滋蔓也,当用何计破之?”荀攸曰:“我今大振兵威,遣使驰檄江东,请孙权会猎于江夏,共擒刘备,分荆州之地,永结盟好。孙权必惊疑而来降,则吾事济矣。”操从其计,一面发檄遣使赴东吴;一面计点马步水军共八十三万,诈称一百万,水陆并进,船骑双行,沿江而来,西连荆、峡、东接蕲、黄、赛栅联络三百余里。

话分两头。却说江东孙权,屯兵柴桑郡,闻曹操大军至襄阳,刘琮已降,今又星夜兼道取江陵,乃集众谋士商议御守之策。鲁肃曰:“荆州与国邻接,江山险固,士民殷富。吾若据而有之,此帝王之资也。今刘表新亡,刘备新败,肃请奉命往江夏吊丧,因说刘备使抚刘表众将,同心一意,共破曹操;备若喜而从命,则大事可定矣。”权喜从其言,即遣鲁肃赍礼往江夏吊丧。却说玄德至江夏,与孔明、刘琦共议良策。孔明曰:“曹操势大,急难抵敌,不如往投东吴孙权,以为应援。使南北相持,吾等于中取利,有何不可?”玄德曰:“江东人物极多,必有远谋,安肯相容耶?”孔明笑曰:“今操引百万之众,虎踞江汉,江东安得不使人来探听虚实?若有人到此,亮借一帆风,直至江东,凭三寸不烂之舌,说南北两军互相吞并。若南军胜,共诛曹操以取荆州之地;若北军胜,则我乘势以取江南可也。”玄德曰:“此论甚高。但如何得江东人到?”

正说间,人报江东孙权差鲁肃来吊丧,船已傍岸。孔明笑曰::大事济矣!”遂问刘琦曰:“往日孙策亡时,襄阳曾遣人去吊丧否?”琦曰:“江东与我家有杀父之仇,安得通庆吊之礼!”孔明曰:“然则鲁肃之来,非为吊丧,乃来探听军情也。”遂谓玄德曰:“鲁肃至,若问曹操动静,主公只推不知,再三问时,主公只说可问诸葛亮。”计会已定,使人迎接鲁肃。肃入城吊丧;收过礼物,刘琦请肃与玄德相见。礼毕,邀入后堂饮酒,肃曰:“久闻皇叔大名,无缘拜会;今幸得见。实为欣慰。近闻皇叔与曹操会战,必知彼虚实:敢问操军约有几何?”玄德曰:“备兵微将寡,一闻操至即走,竟不知彼虚实。”鲁肃曰:“闻皇叔用诸葛孔明之谋,两场火烧得曹操魂亡胆落,何言不知耶?”玄德曰:“徐非问孔明,便知其详。”肃曰:“孔明安在?愿求一见。”玄德教请孔明出来相见。

肃见孔明礼毕,问曰:“向慕先生才德,未得拜晤;今幸相遇,愿闻目今安危之事。”孔明曰:“曹操奸计,亮已尽知;但恨力未及,故且避之。”肃曰:“皇叔今将止于此乎?”孔明曰:“使君与苍梧太守吴臣有旧,将往投之。”肃曰:“吴臣粮少兵微,自不能保,焉能容人?”孔明曰:“吴臣处虽不足久居,今且暂依之,别有良图。”肃曰:“孙将军虎踞六郡,兵精粮足,又极敬贤礼士,江表英雄,多归附之。今为君计。莫若遣心腹往结东吴,以共图大事。”孔明曰:“刘使君与孙将军自来无旧,恐虚费词说。且别无心腹之人可使。”肃曰:“先生之兄,现为江东参谋,日望与先生相见。肃不才,愿与公同见孙将军,共议大事。”玄德曰:“孔明是吾之师,顷刻不可相离,安可去也?”肃坚请孔明同去。玄德佯不许。孔明曰:“事急矣,请奉命一行。玄德方才许诺。鲁肃遂别了玄德、刘琦,与孔明登舟,望柴桑郡来。正是:只因诸葛扁舟去,致使曹兵一旦休。不知孔明此去毕竟如何,且看下文分解。