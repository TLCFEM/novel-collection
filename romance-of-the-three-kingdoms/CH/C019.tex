\chapter{下邳城曹操鏖兵~白门楼吕布殒命}

却说高顺引张辽击关公寨,吕布自击张飞寨,关、张各出迎战,玄德引兵两路接应。吕
布分军从背后杀来,关、张两军皆溃,玄德引数十骑奔回沛城。吕布赶来,玄德急唤城上军
士放下吊桥。吕布随后也到。城上欲待放箭,又恐射了玄德。被吕布乘势杀入城门,把门将
士,抵敌不住,都四散奔避。吕布招军入城。玄德见势已急,到家不及,只得弃了妻小,穿
城而过,走出西门,匹马逃难,吕布赶到玄德家中,糜竺出迎,告布曰:“吾闻大丈夫不废
人之妻子。今与将军争天下者,曹公耳。玄德常念辕门射赖之恩,不敢背将军也。今不得已
而投曹公,惟将军怜之。”布曰:“吾与玄德旧交,岂忍害他妻子。”便令糜竺引玄德妻
小,去徐州安置。布自引军投山东兖州境上,留高顺、张辽守小沛。此时孙乾已逃出城外。
关、张二人亦各自收得些人马,往山中住扎。

且说玄德匹马逃难,正行间,背后一人赶至,视之乃孙乾也。玄德曰:“吾今两弟不知
存亡,妻小失散,为之奈何?”孙乾曰:“不若且投曹操,以图后计。”玄德依言,寻小路
投许都。途次绝粮,尝往村中求食。但到处,闻刘豫州,皆争进饮食。一日,到一家投宿,
其家一少年出拜,问其姓名,乃猎户刘安也。当下刘安闻豫州牧至,欲寻野味供食,一时不
能得,乃杀其妻以食之。玄值曰:“此何肉也?”安曰:“乃狼肉也。”玄德不疑,乃饱食
了一顿,天晚就宿。至晓将去,往后院取马,忽见一妇人杀于厨下,臂上肉已都割去。玄德
惊问,方知昨夜食者,乃其妻之肉也。玄德不胜伤感,洒泪上马。刘安告玄德曰:“本欲相
随使君,因老母在堂,未敢远行。”玄德称谢而别,取路出梁城。忽见尘头蔽日,一彪大军
来到。玄德知是曹操之军,同孙乾径至中军旗下,与曹操相见,具说失沛城、散二弟、陷妻
小之事。操亦为之下泪。又说刘安杀妻为食之事,操乃令孙乾以金百两往赐之。

军行至济北,夏侯渊等迎接入寨,备言兄夏侯惇损其一目,卧病未痊。操临卧处视之,
令先回许都调理。一面使人打探吕布现在何处。探马回报云:“吕布与陈宫、臧霸结连泰山
贼寇,共攻兖州诸郡。”操即令曹仁引三千兵打沛城;操亲提大军,与玄德来战吕布。前至
山东,路近萧关,正遇泰山寇孙观、吴敦、尹礼、昌豨领兵三万余拦住去路。操令许褚迎
战,四将一齐出马。许褚奋力死战,四将抵敌不住,各自败走。操乘势掩杀,追至萧关。探
马飞报吕布。

时布已回徐州,欲同陈登往救小沛,令陈珪守徐州。陈登临行,珪谓之曰:“昔曹公曾
言东方事尽付与汝。今布将败,可便图之。”登曰:“外面之事,儿自为之;倘布败回,父
亲便请糜竺一同守城,休放布入,儿自有脱身之计。”珪曰:“布妻小在此,心腹颇多,为
之奈何?”登曰:“儿亦有计了。”乃入见吕布曰:“徐州四面受敌,操必力攻,我当先思
退步:可将钱粮移于下邳,倘徐州被围,下邳有粮可救。主公盍早为计?”布曰:“元龙之
言甚善。吾当并妻小移去。”遂令宋宪、魏续保护妻小与钱粮移屯下邳;一面自引军与陈登
往救萧关。到半路,登曰:“容某先到关探曹操虚实,主公方可行。”布许之,登乃先到关
上。陈宫等接见。登曰:“温侯深怪公等不肯向前,要来责罚”。宫曰:“今曹兵势大,未
可轻敌。吾等紧守关隘,可劝主公深保沛城,乃为上策。”陈登唯唯。至晚,上关而望,见
曹兵直逼关下,乃乘夜连写三封书,拴在箭上,射下关去。次日辞了陈宫,飞马来见吕布
曰:“关上孙观等皆欲献关,某已留下陈宫守把,将军可于黄昏时杀去救应。”布曰:“非
公则此关休矣。”便教陈登飞骑先至关,约陈宫为内应,举火为号。登径往报宫曰:“曹兵
已抄小路到关内,恐徐州有失。公等宜急回。”宫遂引众弃关而走。登就关上放起火来。吕
布乘黑杀至,陈宫军和吕布军在黑暗里自相掩杀。曹兵望见号火,一齐杀到,乘势攻击。孙
观等各自四散逃避去了。吕布直杀到天明,方知是计;急与陈宫回徐州。到得城边叫门时,
城上乱箭射下。糜竺在敌楼上喝曰:“汝夺吾主城池,今当仍还吾主,汝不得复入此城
也。”布大怒曰:“陈珪何在?”竺曰:“吾已杀之矣”。布回顾宫曰:“陈登安在?”宫
曰:“将军尚执迷而问此佞贼乎?”布令遍寻军中,却只不见。宫劝布急投小沛,布从之。
行至半路,只见一彪军骤至,视之,乃高顺、张辽也。布问之,答曰:“陈登来报说主公被
围,令某等急来救解。”宫曰:“此又佞贼之计也。”布怒曰:“吾必杀此贼!”急驱马至
小沛。只见小沛城上尽插曹兵旗号。原来曹操已令曹仁袭了城池,引军守把。吕布于城下大
骂陈登。登在城上指布骂曰:“吾乃汉臣,安肯事汝反贼耶!”布大怒,正待攻城,忽听背
后喊声大起,一队人马来到,当先一将乃是张飞。高顺出马迎敌,不能取胜。布亲自接战。
正斗间,阵外喊声复起,曹操亲统大军冲杀前来。吕布料难抵敌,引军东走。曹兵随后追
赶。吕布走得人困马乏。忽又闪出一彪军拦住去路,为首一将,立马横刀,大喝:“吕布休
走!关云长在此!”吕布慌忙接战。背后张飞赶来。布无心恋战,与陈宫等杀开条路,径奔
下邳。侯成引兵接应去了。

关、张相见,各洒泪言失散之事。云长曰:“我在海州路上住扎,探得消息,故来至
此。”张飞曰:“弟在芒砀山住了这几时,今日幸得相遇。”两个叙话毕,一同引兵来见玄
德,哭拜于地。玄德悲喜交集,引二人见曹操,便随操入徐州。糜竺接见,具言家属无恙,
玄德甚喜。陈珪父子亦来参拜曹操。操设一大宴,犒劳诸将。操自居中,使陈珪居右、玄德
居左。其余将士,各依次坐。宴罢,操嘉陈珪父子之功,加封十县之禄,授登为伏波将军。
且说曹操得了徐州,心中大喜,商议起兵攻下邳。程昱曰:“布今止有下邳一城,若逼之太
急,必死战而投袁术矣。布与术合,其势难攻。今可使能事者守住淮南径路,内防吕布,外
当袁术。况今山东尚有臧霸、孙观之徒未曾归顺,防之亦不可忽也。”操曰:“吾自当山东
诸路。其淮南径路,请玄德当之。”玄德曰:“丞相将令,安敢有违。”次日,玄德留糜
竺、简雍在徐州,带孙乾、关、张引军住守淮南径路。曹操自引兵攻下邳。且说吕布在下
邳,自恃粮食足备,且有泗水之险,安心坐守,可保无虞。陈宫曰:“今操兵方来,可乘其
寨栅未定,以逸击劳,无不胜者。”布曰:“吾方屡败,不可轻出。待其来攻而后击之,皆
落泗水矣。”遂不听陈宫之言。过数日,曹兵下寨已定。操统众将至城下,大叫吕布答话,
布上城而立,操谓布曰:“闻奉先又欲结婚袁术,吾故领兵至此。夫术有反逆大罪,而公有
讨董卓之功,今何自弃其前功而从逆贼耶?倘城池一破,悔之晚矣!若早来降,共扶王室,
当不失封侯之位。”布曰:“丞相且退,尚容商议。”陈宫在布侧大骂曹操奸贼,一箭射中
其麾盖。操指宫恨曰:“吾誓杀汝!”遂引兵攻城。宫谓布曰:“曹操远来,势不能久。将
军可以步骑出屯于外,宫将余众闭守于内;操若攻将军,宫引兵击其背;若来攻城,将军为
救于后;不过旬日,操军食尽,可一鼓而破;此乃掎角之势也。”布曰:“公言极是。”遂
归府收拾戎装。时方冬寒,分付从人多带绵衣,布妻严氏闻之,出问曰:“君欲何往?”布
告以陈宫之谋。严氏曰:“君委全城,捐妻子,孤军远出,倘一旦有变,妾岂得为将军之妻
乎?”布踌躇未决,三日不出。宫入见曰:“操军四面围城,若不早出,必受其困。”布
曰:“吾思远出不如坚守。”宫曰:“近闻操军粮少,遣人往许都去取,早晚将至。将军可
引精兵往断其粮道。此计大妙。”布然其言,复入内对严氏说知此事。严氏泣曰:“将军若
出,陈宫、高顺安能坚守城池?倘有差失,悔无及矣!妾昔在长安,已为将军所弃,幸赖庞
舒私藏妾身,再得与将军相聚;孰知今又弃妾而去乎?将军前程万里,请勿以妾为念!”言
罢痛哭。布闻言愁闷不决,入告貂蝉。貂蝉曰:“将军与妾作主,勿轻身自出。”布曰:
“汝无忧虑。吾有画戟、赤兔马,谁敢近我!”乃出谓陈宫曰:“操军粮至者,诈也。操多
诡计,吾未敢动。”宫出,叹曰:“吾等死无葬身之地矣!”布于是终日不出,只同严氏、
貂蝉饮酒解闷。

谋士许汜、王楷入见布,进计曰:今袁术在淮南,声势大振。将军旧曾与彼约婚,今何
不仍求之?彼兵若至,内外夹攻,操不难破也。布从其计,即日修书,就着二人前去。许汜
曰:“须得一军引路冲出方好。”布令张辽、郝萌两个引兵一千,送出隘口。是夜二更,张
辽在前,郝萌在后,保着许汜、王楷杀出城去。抹过玄德寨,众将追赶不及,已出隘口。郝
萌将五百人,跟许汜、王楷而去。张辽引一半军回来,到隘口时,云长拦住。未及交锋,高
顺引兵出城救应,接入城中去了。且说许汜、王楷至寿春,拜见袁术,呈上书信。术曰:
“前者杀吾使命,赖我婚姻!今又来相问,何也?”汜曰:“此为曹操奸计所误,愿明上详
之。”术曰:“汝主不因曹兵困急,岂肯以女许我?”楷曰:“明上今不相救,恐唇亡齿
寒,亦非明上之福也。”术曰:“奉先反复无信,可先送女,然后发兵。”许汜、王楷只得
拜辞,和郝萌回来。到玄德寨边,汜曰:“日间不可过。夜半吾二人先行,郝将军断后。”
商量停当。夜过玄德寨,许汜、王楷先过去了。郝萌正行之次,张飞出寨拦路。郝萌交马只
一合,被张飞生擒过去,五百人马尽被杀散。张飞解郝萌来见玄德,玄德押往大寨见曹操。
郝萌备说求救许婚一事。操大怒,斩郝萌于军门,使人传谕各寨,小心防守:如有走透吕布
及彼军士者,依军法处治。各寨悚然。玄德回营,分付关、张曰:“我等正当淮南冲要之
处。二弟切宜小心在意,勿犯曹公军令。”飞曰:“捉了一员贼将,操不见有甚褒赏,却反
来?吓,何也?”玄德曰:“非也。曹操统领多军,不以军令,何能服人?弟勿犯之。”
关、张应诺而退。

却说许汜、王楷回见吕布,具言袁术先欲得妇,然后起兵救援。布曰:“如何送去?”
汜曰:“今郝萌被获,操必知我情,预作准备。若非将军亲自护送,谁能突出重围?”布
曰:“今日便送去,如何?”汜曰:“今日乃凶神值日,不可去。明日大利,宜用戌、亥
时。”布命张辽、高顺:“引三千军马,安排小车一辆;我亲送至二百里外,却使你两个送
去。”次夜二更时分,吕布将女以绵缠身,用甲包裹,负于背上,提戟上马。放开城门,布
当先出城,张辽、高顺跟着。将次到玄德寨前,一声鼓响,关、张二人拦住去路,大叫:休
走!”布无心恋战,只顾夺路而行。玄德自引一军杀来,两军混战。吕布虽勇,终是缚一女
在身上,只恐有伤,不敢冲突重围。后面徐晃、许褚皆杀来,众军皆大叫曰:“不要走了吕
布!”布见军来太急,只得仍退入城。玄德收军,徐晃等各归寨,端的不曾走透一个。吕布
回到城中,心中忧闷,只是饮酒。

却说曹操攻城,两月不下。忽报:“河内太守张杨出兵东市,欲救吕布;部将杨丑杀
之,欲将头献丞相,却被张杨心腹将眭固所杀,反投犬城去了。”操闻报,即遣史涣追斩眭
固。因聚众将曰:“张杨虽幸自灭,然北有袁绍之忧,东有表、绣之患,下邳久围不克,吾
欲舍布还都,暂且息战,何如?”荀攸急止曰:“不可。吕布屡败,锐气已堕,军以将为
主,将衰则军无战心。彼陈宫虽有谋而迟。今布之气未复,宫之谋未定,作速攻之,布可擒
也。”郭嘉曰:“某有一计,下邳城可立破,胜于二十万师。”荀彧曰:“莫非决沂、泗之
水乎?”嘉笑曰:“正是此意。”操大喜,即令军士决两河之水。曹兵皆居高原。坐视水淹
下邳。下邳一城,只剩得东门无水;其余各门,都被水淹。众军飞报吕布。布曰:“吾有赤
兔马,渡水如平地,又何惧哉!”乃日与妻妾痛饮美酒,因酒色过伤,形容销减;一日取镜
自照,惊曰:“吾被酒色伤矣!自今日始,当戒之。”遂下令城中,但有饮酒者皆斩。

却说侯成有马十五匹,被后槽人盗去,欲献与玄德。侯成知觉,追杀后槽人,将马夺
回;诸将与侯成作贺。侯成酿得五六斛酒,欲与诸将会饮,恐吕布见罪,乃先以酒五瓶诣布
府,禀曰:“托将军虎威,追得失马。众将皆来作贺。酿得些酒,未敢擅饮,特先奉上微
意。”布大怒曰:“吾方禁酒,汝却酿酒会饮,莫非同谋伐我乎!”命推出斩之。宋宪、魏
续等诸将俱入告饶。”布曰:“故犯吾令,理合斩首。今看众将面,且打一百!”众将又哀
告,打了五十背花,然后放归。众将无不丧气。宋宪、魏续至侯成家来探视,侯成泣曰:
“非公等则吾死矣!”宪曰:“布只恋妻子,视吾等如草芥。”续曰:“军围城下,水绕壕
边,吾等死无日矣!”宪曰:“布无仁无义,我等弃之而走,何如?”续曰:“非丈夫也。
不若擒布献曹公。”侯成曰:“我因追马受责,而布所倚恃者,赤兔马也。汝二人果能献门
擒布,吾当先盗马去见曹公。”三人商议定了。是夜侯成暗至马院,盗了那匹赤兔马,飞奔
东门来。魏续便开门放出,却佯作追赶之状。侯成到曹操寨,献上马匹,备言宋宪、魏续插
白旗为号,准备献门。曹操闻此信,便押榜数十张射入城去。其榜曰:“大将军曹,特奉明
诏,征伐吕布。如有抗拒大军者,破城之日,满门诛戮。上至将校,下至庶民,有能擒吕布
来献,或献其首级者,重加官赏。为此榜谕,各宜知悉。”次日平明,城外喊声震地。吕布
大惊,提戟上城,各门点视,责骂魏续走透侯成,失了战马,欲待治罪。城下曹兵望见城上
白旗,竭力攻城,布只得亲自抵敌。从平明直打到日中,曹兵稍退。布少憩门楼,不觉睡着
在椅上。宋宪赶退左右,先盗其画戟,便与魏续一齐动手,将吕布绳缠索绑,紧紧缚住。布
从睡梦中惊醒,急唤左右,却都被二人杀散,把白旗一招,曹兵齐至城下。魏续大叫:“已
生擒吕布矣!”夏侯渊尚未信。宋宪在城上掷下吕布画戟来,大开城门,曹兵一拥而入。高
顺、张辽在西门,水围难出,为曹兵所擒。陈宫奔至南门,为徐晃所获。

曹操入城,即传令退了所决之水,出榜安民;一面与玄德同坐白门楼上。关、张侍立于
侧,提过擒获一干人来。吕布虽然长大,却被绳索捆作一团,布叫曰:“缚太急,乞缓
之!”操曰:“缚虎不得不急。”布见侯成、魏续、宋宪皆立于侧,乃谓之曰:“我待诸将
不薄,汝等何忍背反?”宪曰:“听妻妾言,不听将计,何谓不薄?”布默然。须臾,众拥
高顺至。操问曰:“汝有何言?”顺不答。操怒命斩之。徐晃解陈宫至。操曰:“公台别来
无恙!”宫曰:“汝心术不正,吾故弃汝!”操曰:“吾心不正,公又奈何独事吕布?”宫
曰:“布虽无谋,不似你诡诈奸险。”操曰:“公自谓足智多谋,今竟何如?”宫顾吕布
曰:“恨此人不从吾言!若从吾言,未必被擒也。”操曰:“今日之事当如何?”宫大声
曰:“今日有死而已!”操曰:“公如是,奈公之老母妻子何?”宫曰:“吾闻以孝治天下
者,不害人之亲;施仁政于天下者,不绝人之祀。老母妻子之存亡,亦在于明公耳。吾身既
被擒,请即就戮,并无挂念。”操有留恋之意。宫径步下楼,左右牵之不住。操起身泣而送
之。宫并不回顾。操谓从者曰:“即送公台老母妻子回许都养老。怠慢者斩。”宫闻言,亦
不开口,伸颈就刑。众皆下泪。操以棺椁盛其尸,葬于许都。后人有诗叹之曰:“生死无二
志,丈夫何壮哉!不从金石论,空负栋梁材。辅主真堪敬,辞亲实可哀。白门身死日,谁肯
似公台!”

方操送宫下楼时,布告玄德曰:“公为坐上客,布为阶下囚,何不发一言而相宽乎?”
玄德点头。及操上楼来,布叫曰:“明公所患,不过于布;布今已服矣。公为大将,布副
之,天下不难定也。”操回顾玄德曰!“何如?”玄德答曰:“公不见丁建阳、董卓之事
乎?”布目视玄德曰:“是儿最无信者!”操令牵下楼缢之。布回顾玄德曰:“大耳儿!不
记辕门射戟时耶?”忽一人大叫曰:“吕布匹夫!死则死耳,何惧之有!”众视之,乃刀斧
手拥张辽至。操令将吕布缢死,然后枭首。后人有诗叹曰:“洪水滔滔淹下邳,当年吕布受
擒时:空余赤兔马千里,漫有方天戟一枝。缚虎望宽今太懦,养鹰休饱昔无疑。恋妻不纳陈
宫谏,枉骂无恩大耳儿。”又有诗论玄德曰:“伤人饿虎缚体宽,董卓丁原血未干。玄德既
知能啖父,争如留取害曹瞒?”却说武士拥张辽至。操指辽曰:“这人好生面善。”辽曰:
“濮阳城中曾相遇,如何忘却?”操笑曰:“你原来也记得!”辽曰:“只是可惜!”操
曰:“可惜甚的?”辽曰:“可惜当日火不大,不曾烧死你这国贼!”操大怒曰:“败将安
敢辱吾!”拔剑在手,亲自来杀张辽。辽全无惧色,引颈待杀。曹操背后一人攀住臂膊,一
人跪于面前,说道:“丞相且莫动手!”正是:乞哀吕布无人救,骂贼张辽反得生。毕竟救
张辽的是谁,且听下文分解。