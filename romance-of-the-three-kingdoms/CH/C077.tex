\chapter{玉泉山关公显圣~洛阳城曹操感神}

却说孙权求计于吕蒙。蒙曰:“吾料关某兵少,必不从大路而逃,麦成正北有险峻小路,必从此路而去。可令朱然引精兵五千,伏于麦城之北二十里;彼军至,不可与敌,只可随后掩杀。彼军定无战心,必奔临沮。却令潘璋引精兵五百,伏于临沮山僻小路,关某可擒矣。今遣将士各门攻打,只空北门,待其出走。”权闻计,令吕范再卜之。卦成,范告曰:“此卦主敌人投西北而走,今夜亥时必然就擒。”权大喜,遂令朱然、潘璋领两枝精兵,各依军令埋伏去讫。

且说关公在麦城,计点马步军兵,止剩三百余人;粮草又尽。是夜,城外吴兵招唤各军姓名,越城而去者甚多。救兵又不见到。心中无计,谓王甫曰:“吾悔昔日不用公言!今日危急,将复何如?”甫哭告曰:“今日之事,虽子牙复生,亦无计可施也。”赵累曰:“上庸救兵不至,乃刘封、孟达按兵不动之故。何不弃此孤城,奔入西川,再整兵来,以图恢复?”公曰:“吾亦欲如此。”遂上城观之。见北门外敌军不多,因问本城居民:“此去往北,地势若何?”答曰:“此去皆是山僻小路,可通西川。”公曰:“今夜可走此路。。王甫谏曰:“小路有埋伏,可走大路。”公曰:“虽有埋伏,吾何惧哉!”即下令马步官军:严整装束,准备出城。甫哭曰:“君侯于路,小心保重!某与部卒百余人,死据此城;城虽破,身不降也!专望君侯速来救援!”公亦与泣别。遂留周仓与王甫同守麦城,关公自与关平、赵累引残卒二百余人,突出北门。关公横刀前进,行至初更以后,约走二十余里,只见山凹处,金鼓齐鸣,喊声大震,一彪军到,为首大将朱然,骤马挺枪叫曰:“云长休走!趁早投降,免得一死!”公大怒,拍马轮刀来战。朱然便走,公乘势追杀。一棒鼓响,四下伏兵皆起。公不敢战,望临沮小路而走,朱然率兵掩杀。关公所随之兵,渐渐稀少。走不得四五里,前面喊声又震,火光大起,潘璋骤马舞刀杀来。公大怒,轮刀相迎,只三合,潘璋败走。公不敢恋战,急望山路而走。背后关平赶来,报说赵累已死于乱军中。关公不胜悲惶,遂令关平断后,公自在前开路,随行止剩得十余人。行至决石,两下是山,山边皆芦苇败草,树木丛杂。时已五更将尽。正走之间,一声喊起,两下伏兵尽出,长钩套索,一齐并举,先把关公坐下马绊倒。关公翻身落马,被潘璋部将马忠所获。关平知父被擒,火速来救;背后潘璋、朱然率兵齐至,把关平四下围住。平孤身独战,力尽亦被执。至天明,孙权闻关公父子已被擒获,大喜,聚众将于帐中。

少时,马忠簇拥关公至前。权曰:“孤久慕将军盛德,欲结秦晋之好,何相弃耶?公平昔自以为天下无敌,今日何由被吾所擒?将军今日还服孙权否?”关公厉声骂曰:“碧眼小儿,紫髯鼠辈!吾与刘皇叔桃园结义,誓扶汉室,岂与汝叛汉之贼为伍耶!我今误中奸计,有死而已,何必多言!”权回顾众官曰:“云长世之豪杰,孤深爱之。今欲以礼相待,劝使归降,何如?”主簿左咸曰:“不可。昔曹操得此人时,封侯赐爵,三日一小宴,五日一大宴,上马一提金,下马一提银:如此恩礼,毕竟留之不住,听其斩关杀将而去,致使今日反为所逼,几欲迁都以避其锋。今主公既已擒之,若不即除,恐贻后患。”孙权沉吟半晌,曰:“斯言是也。”遂命推出。于是关公父子皆遇害。时建安二十四年冬十二月也。关公亡年五十八岁。后人有诗叹曰:“汉末才无敌,云长独出群:神威能奋武,儒雅更知文。天日心如镜,《春秋》义薄云。昭然垂万古,不止冠三分。”又有诗曰:“人杰惟追古解良,士民争拜汉云长。桃园一日兄和弟,俎豆千秋帝与王。气挟风雷无匹敌,志垂日月有光芒。至今庙貌盈天下,古木寒鸦几夕阳。”

关公既殁,坐下赤兔马被马忠所获,献与孙权。权即赐马忠骑坐。其马数日不食草料而死。

却说王甫在麦城中,骨颤肉惊,乃问周仓曰:“昨夜梦见主公浑身血污,立于前;急问之,忽然惊觉。不知主何吉凶?”正说间,忽报吴兵在城下,将关公父子首级招安。王甫、周仓大惊,急登城视之,果关公父子首级也。王甫大叫一声,堕城而死。周仓自刎而亡。于是麦城亦属东吴。

却说关公一魂不散,荡荡悠悠,直至一处,乃荆门州当阳县一座山,名为玉泉山。山上有一老僧,法名普净,原是汜水关镇国寺中长老;后因云游天下,来到此处,见山明水秀,就此结草为庵,每日坐禅参道,身边只有一小行者,化饭度日。是夜月白风清,三更已后,普净正在庵中默坐,忽闻空中有人大呼曰:“还我头来!”普净仰面谛视,只见空中一人,骑赤兔马,提青龙刀,左有一白面将军、右有一黑脸虬髯之人相随,一齐按落云头,至玉泉山顶。普净认得是关公,遂以手中麈尾击其户曰:“云长安在?”关公英魂顿悟,即下马乘风落于庵前,叉手问曰:“吾师何人?愿求法号。”普净曰:“老僧普净,昔日汜水关前镇国寺中,曾与君侯相会,今日岂遂忘之耶?”公曰:“向蒙相救,铭感不忘。今某己遇祸而死,愿求清诲,指点迷途。”普净曰:“昔非今是,一切休论;后果前因,彼此不爽。今将军为吕蒙所害,大呼还我头来,然则颜良、文丑,五关六将等众人之头,又将向谁索耶?“于是关公恍然大悟,稽首皈依而去。后往往于玉泉山显圣护民,乡人感其德,就于山顶上建庙,四时致祭。后人题一联于其庙云:“赤面秉赤心、骑赤兔追风,驰驱时无忘赤帝,青灯观青史、仗青龙偃月,隐微处不愧青天。”

却说孙权既害了关公,遂尽收荆襄之地,赏稿三军,设宴大会诸将庆功;置吕蒙于上位,顾谓众将曰:“孤久不得荆州,今唾手而得,皆子明之功也。”蒙再三逊谢。权曰:“昔周郎雄略过人,破曹操于赤壁,不幸早夭,鲁子敬代之。子敬初见孤时,便及帝王大略,此一快也;曹操东下,诸人皆劝孤降,子敬独劝孤召公瑾逆而击之,此二快也;惟劝吾借荆州与刘备,是其一短。今子明设计定谋,立取荆州,胜子敬、周郎多矣!”于是亲酌酒赐吕蒙。吕蒙接酒欲饮,忽然掷杯于地,一手揪住孙权,厉声大骂曰:“碧眼小儿!紫髯鼠辈!还识我否?”众将大惊,急救时,蒙推倒孙权,大步前进,坐于孙权位上,两眉倒竖,双眼圆睁,大喝曰:“我自破黄巾以来,纵横天下三十余年,今被汝一旦以奸计图我,我生不能啖汝之肉,死当追吕贼之魂!我乃汉寿亭侯关云长也。”权大惊,慌忙率大小将士,皆下拜。只见吕蒙倒于地上,七窍流血而死。众将见之,无不恐惧。权将吕蒙尸首,具棺安葬,赠南郡太守、孱陵侯;命其子吕霸袭爵。孙权自此感关公之事,惊讶不已。

忽报张昭自建业而来。权召入问之。昭曰:“今主公损了关公父子,江东祸不远矣!此人与刘备桃园结义之时,誓同生死。今刘备已有两川之兵;更兼诸葛亮之谋,张、黄、马、赵之勇。备若知云长父子遇害,必起倾国之兵,奋力报仇,恐东吴难与敌也。”权闻之大惊,跌足曰:“孤失计较也!似此如之奈何?”昭曰:“主公勿忧。某有一计,令西蜀之兵不犯东吴,荆州如磐石之安。”权问何计。昭曰:“今曹操拥百万之众,虎视华夏,刘备急欲报仇,必与操约和。若二处连兵而来,东吴危矣。不如先遣人将关公首级,转送与曹操,明教刘备知是操之所使,必痛恨于操,西蜀之兵,不向吴而向魏矣。吾乃观其胜负,于中取事。此为上策。”

权从其言,随遣使者以木匣盛关公首级,星夜送与曹操。时操从摩陂班师回洛阳,闻东吴送关公首级至,喜曰:“云长已死,吾夜眠贴席矣。”阶下一人出曰:“此乃东吴移祸之计也。”操视之,乃主簿司马懿也。操问其故,懿曰:“昔刘、关、张三人桃园结义之时,誓同生死。今东吴害了关公,惧其复仇,故将首级献与大王,使刘备迁怒大王,不攻吴而攻魏,他却于中乘便而图事耳。”操曰:“仲达之言是也。孤以何策解之?”懿曰:“此事极易。大王可将关公首级,刻一香木之躯以配之,葬以大臣之礼;刘备知之,必深恨孙权,尽力南征。我却观其胜负!蜀胜则击吴,吴胜则击蜀。二处若得一处,那一处亦不久也。”操大喜,从其计,遂召吴使入。呈上木匣,操开匣视之,见关公面如平日。操笑曰:“云长公别来无恙!”言未讫,只见关公口开目动,须发皆张,操惊倒。众官急救,良久方醒,顾谓众官曰:“关将军真天神也!”吴使又将关公显圣附体、骂孙权追吕蒙之事告操。操愈加恐惧,遂设牲醴祭祀,刻沉香木为躯,以王侯之礼,葬于洛阳南门外,令大小官员送殡,操自拜祭,赠为荆王,差官守墓;即遣吴使回江东去讫。却说汉中王自东川回成都,法正奏曰:“王上先夫人去世;孙夫人又南归。未必再来。人伦之道,不可废也,必纳王妃,以襄内政。”汉中王从之,法正复奏曰:“吴懿有一妹,美而且贤。尝闻有相者,相此女后必大贵。先曾许刘焉之子刘瑁,瑁早夭。其女至今寡居,大王可纳之为妃。”汉中王曰:“刘瑁与我同宗,于理不可。”法正曰:“论其亲疏,何异晋文之与怀嬴乎?”汉中王乃依允,遂纳吴氏为王妃。后生二子:长刘永,字公寿;次刘理,字奉孝。

且说东西两川,民安国富,田禾大成。忽有人自荆州来,言东吴求婚于关公,关公力拒之。孔明曰:“荆州危矣!可使人替关公回。”正商议间,荆州捷报使命,络绎而至。不一日,关兴到,具言水淹七军之事。忽又报马到来,报说关公于江边多设墩台,提防甚密,万无一失。因此玄德放心。

忽一日,玄德自觉浑身肉颤,行坐不安;至夜,不能宁睡,起坐内室,秉烛看书,觉神思昏迷,伏几而卧;就室中起一阵冷风,灯灭复明,抬头见一人立于灯下。玄德问曰:“汝何人,夤度至吾内室?”其人不答。玄德疑怪,自起视之,乃是关公,于灯影下往来躲避。玄德曰:“贤弟别来无恙!夜深至此,必有大故。吾与汝情同骨肉,因何回避?”关公泣告曰:“愿兄起兵,以雪弟恨!”言讫,冷风骤起,关公不见。玄德忽然惊觉,乃是一梦。时正三鼓。玄德大疑,急出前殿,使人请孔明来。孔明入见,玄德细言梦警。孔明曰:“此乃王上心思关公,故有此梦。何必多疑?”玄德再三疑虑,孔明以善言解之。孔明辞出,至中门外,迎见许靖。靖曰:“某才赴军师府下报一机密,听知军师入宫,特来至此。”孔明曰:“有何机密?”靖曰:“某适闻外人传说,东吴吕蒙已袭荆州,关公已遇害!故特来密报军师。”孔明曰:“吾夜观天象,见将星落于荆楚之地,已知云长必然被祸,但恐王上忧虑,故未敢言。”

二人正说之间,忽然殿内转出一人,扯住孔明衣袖而言曰:“如此凶信,公何瞒我!”孔明视之,乃玄德也。孔明、许靖奏曰:“适来所言,皆传闻之事,未足深信。愿王上宽怀,勿生忧虑。”玄德曰:“孤与云长,誓同生死;彼若有失,孤岂能独生耶!”孔明、许靖正劝解之间,忽近侍奏曰:“马良、伊籍至。”玄德急召入问之。二人具说荆州已失,关公兵败求救,呈上表章。未及拆观,侍臣又奏荆州廖化至。玄德急召入。化哭拜于地,细奏刘封、孟达不发救兵之事。玄德大惊曰:“若如此,吾弟休矣!”孔明曰:“刘封、孟达如此无礼,罪不容诛!王上宽心,亮亲提一旅之师,去救荆襄之急。”玄德泣曰:“云长有失,孤断不独生!孤来日自提一军去救云长!”遂一面差人赴阆中报知翼德,一面差人会集人马。

未及天明,一连数次,报说关公夜走临沮,为吴将所获,义不屈节,父子归神。玄德听罢,大叫一声,昏绝于地。正是:为念当年同誓死,忍教今日独捐生!未知玄德性命如何,且看下文分解。