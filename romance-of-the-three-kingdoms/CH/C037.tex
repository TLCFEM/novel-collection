\chapter{司马徽再荐名士~刘玄德三顾草庐}

却说徐庶趱程赴许昌。曹操知徐庶已到,遂命荀彧、程昱等一班谋士往迎之。庶入相府拜见曹操。操曰:“公乃高明之士,何故屈身而事刘备乎?”庶曰:“某幼逃难,流落江湖,偶至新野,遂与玄德交厚,老母在此,幸蒙慈念,不胜愧感。”操曰:“公今至此,正可晨昏侍奉令堂,吾亦得听清诲矣。”庶拜谢而出。急往见其母,泣拜于堂下。母大惊曰:“汝何故至此?”庶曰:“近于新野事刘豫州;因得母书,故星夜至此。”徐母勃然大怒,拍案骂曰:“辱子飘荡江湖数年,吾以为汝学业有进,何其反不如初也!汝既读书,须知忠孝不能两全。岂不识曹操欺君罔上之贼?刘玄德仁义布于四海,况又汉室之胄,汝既事之,得其主矣,今凭一纸伪书,更不详察,遂弃明投暗,自取恶名,真愚夫也!吾有何面目与汝相见!汝玷辱祖宗,空生于天地间耳!”骂得徐庶拜伏于地,不敢仰视,母自转入屏风后去了。少顷,家人出报曰:“老夫人自缢于梁间。”徐庶慌入救时,母气已绝。后人有《徐母赞》曰:“贤哉徐母,流芳千古:守节无亏,于家有补;教子多方,处身自苦;气若丘山,义出肺腑;赞美豫州,毁触魏武;不畏鼎镬,不惧刀斧;唯恐后嗣,玷辱先祖。伏剑同流,断机堪伍;生得其名,死得其所:贤哉徐母,流芳千古!”徐虑见母已死,哭绝于地,良久方苏。曹操使人赍礼吊问,又亲往祭奠。徐庶葬母柩于许昌之南原,居丧守墓。凡曹操所赐,庶俱不受。

时操欲商议南征。荀彧谏曰:“天寒未可用兵;姑待春暖,方可长驱大进。”操从之,乃引漳河之水作一池,名玄武池,于内教练水军,准备南征。

却说玄德正安排礼物,欲往隆中谒诸葛亮,忽人报:“门外有一先生,峨冠博带,道貌非常,特来相探。”玄德曰:“此莫非即孔明否?”遂整衣出迎。视之,乃司马徽也。玄德大喜,请入后堂高坐,拜问曰:“备自别仙颜,因军务倥偬,有失拜访。今得光降,大慰仰慕之私。”徽曰:“闻徐元直在此,特来一会。”玄德曰:“近因曹操囚其母,似母遣人驰书,唤回许昌去矣。”徽曰:“此中曹操之计矣!吾素闻徐母最贤,虽为操所囚,必不肯驰书召其子;此书必诈也。元直不去,其母尚存;今若去,母必死矣!”玄德惊问其故,徽曰:“徐母高义,必羞见其子也。”玄德曰:“元直临行,荐南阳诸葛亮,其人若何?”徽笑曰:“元直欲去,自去便了,何又惹他出来呕心血也?”玄德曰:“先生何出此言?”徽曰:“孔明与博陵崔州平、颍川石广元、汝南孟公威与徐元直四人为密友。此四人务于精纯,惟孔明独观其大略。尝抱膝长吟,而指四人曰:“公等仕进可至刺史、郡守。众问孔明之志若何,孔明但笑而不答。每常自比管仲、乐毅,其才不可量也。”玄德曰:“何颍川之多贤乎!”徽曰:“昔有殷馗善观天文,尝谓群星聚于颍分,其地必多贤士。”时云长在侧曰:“某闻管仲、乐毅乃春秋、战国名人,功盖寰宇;孔明自比此二人,毋乃太过?”徽笑曰:“以吾观之,不当比此二人;我欲另以二人出之。”云长问:“那二人?”徽曰:“可比兴周八百年之姜子牙、旺汉四百年之张子房也。”众皆愕然。徽下阶相辞欲行,玄德留之不住。徽出门仰天大笑曰:“卧龙虽得其主,不得其时,惜哉!”言罢,飘然而去。玄德叹曰:“真隐居贤士也!”

次日,玄德同关、张并从人等来隆中。遥望山畔数人,荷锄耕于田间,而作歌曰:“苍天如圆盖,陆地似棋局;世人黑白分,往来争荣辱:荣者自安安,辱者定碌碌。南阳有隐居,高眠卧不足!”玄德闻歌,勒马唤农夫问曰:“此歌何人所作?”答曰:“乃卧龙先生所作也。”玄德曰:“卧龙先生住何处?”农夫曰:“自此山之南,一带高冈,乃卧龙冈也。冈前疏林内茅庐中,即诸葛先生高卧之地。”玄德谢之,策马前行。不数里,遥望卧龙冈,果然清景异常。后人有古风一篇,单道卧龙居处。诗曰:“襄阳城西二十里,一带高冈枕流水:高冈屈曲压云根,流水潺潺飞石髓;势若困龙石上蟠,形如单凤松阴里;柴门半掩闭茅庐,中有高人卧不起。修竹交加列翠屏,四时篱落野花馨;床头堆积皆黄卷,座上往来无白丁;叩户苍猿时献果,守门老鹤夜听经;囊里名琴藏古锦,壁间宝剑挂七星。庐中先生独幽雅,闲来亲自勤耕稼:专待春雷惊梦回,一声长啸安天下。”玄德来到庄前,下马亲叩柴门,一童出问。玄德曰:“汉左将军宜城亭侯领豫州牧皇叔刘备,特来拜见先生。”童子曰:“我记不得许多名字。”玄德曰:“你只说刘备来访。”童子曰:“先生今早少出。”玄德曰:“何处去了?”童子曰:“踪迹不定,不知何处去了。”玄德曰:“几时归?”童子曰:“归期亦不定,或三五日,或十数日。”玄德惆怅不已。张飞曰:”既不见,自归去罢了。”玄德曰:“且待片时。”云长曰:“不如且归,再使人来探听。”玄德从其言,嘱付童子:“如先生回,可言刘备拜访。”遂上马,行数里,勒马回观隆中景物,果然山不高而秀雅,水不深而澄清;地不广而平坦,林不大而茂盛;猿鹤相亲,松篁交翠。观之不已,忽见一人,容貌轩昂,丰姿俊爽,头戴逍遥巾,身穿皂布袍,杖藜从山僻小路而来。玄德曰:“此必卧龙先生也!”急下马向前施礼,问曰:“先生非卧龙否?”其人曰:“将军是谁?”玄德曰:“刘备也。”其人曰:“吾非孔明,乃孔明之友博陵崔州平也。”玄德曰:“久闻大名,幸得相遇。乞即席地权坐,请教一言。”二人对坐于林间石上,关、张侍立于侧。州平曰:“将军何故欲见孔明?”玄德曰:“方今天下大乱,四方云扰,欲见孔明,求安邦定国之策耳。”州平笑曰:“公以定乱为主,虽是仁心,但自古以来,治乱无常。自高祖斩蛇起义,诛无道秦,是由乱而入治也;至哀、平之世二百年,太平日久,王莽篡逆,又由治而入乱;光武中兴,重整基业,复由乱而入治;至今二百年,民安已久,故干戈又复四起:此正由治入乱之时,未可猝定也。将军欲使孔明斡旋天地,补缀乾坤,恐不易为,徒费心力耳。岂不闻顺天者逸,逆天者劳;数之所在,理不得而夺之;命之所在,人不得而强之乎?”玄德曰:“先生所言,诚为高见。但备身为汉胄,合当匡扶汉室,何敢委之数与命?”州平曰:“山野之夫,不足与论天下事,适承明问,故妄言之。”玄德曰:“蒙先生见教。但不知孔明往何处去了?”州平曰:“吾亦欲访之,正不知其何往。”玄德曰:“请先生同至敝县,若何?”州平曰:“愚性颇乐闲散,无意功名久矣;容他日再见。”言讫,长揖而去。玄德与关、张上马而行。张飞曰:“孔明又访不着,却遇此腐儒,闲谈许久!”玄德曰:“此亦隐者之言也。”

三人回至新野,过了数日,玄德使人探听孔明。回报曰:“卧龙先生已回矣。”玄德便教备马。张飞曰:“量一村夫,何必哥哥自去,可使人唤来便了。”玄德叱曰:“汝岂不闻孟子云:欲见贤而不以其道,犹欲其入而闭之门也。孔明当世大贤,岂可召乎!”遂上马再往访孔明。关、张亦乘马相随。时值隆冬,天气严寒,彤云密布。行无数里,忽然朔风凛凛,瑞雪霏霏:山如玉簇,林似银妆。张飞曰:“天寒地冻,尚不用兵,岂宜远见无益之人乎!不如回新野以避风雪。”玄德曰:“吾正欲使孔明知我殷勤之意。如弟辈怕冷,可先回去。”飞曰:“死且不怕,岂怕冷乎!但恐哥哥空劳神思。”玄德曰:“勿多言,只相随同去。”将近茅庐,忽闻路傍酒店中有人作歌。玄德立马听之。其歌曰:“壮士功名尚未成,呜呼久不遇阳春!君不见东海者叟辞荆榛,后车遂与文王亲;八百诸侯不期会,白鱼入舟涉孟津;牧野一战血流杵,鹰扬伟烈冠武臣。又不见高阳酒徒起草中,长楫芒砀隆准公;高谈王霸惊人耳,辍洗延坐钦英风;东下齐城七十二,天下无人能继踪。二人功迹尚如此,至今谁肯论英雄?”歇罢,又有一人击桌而歌。其歌曰:“吾皇提剑清寰海,创业垂基四百载;桓灵季业火德衰,奸臣贼子调鼎鼐。青蛇飞下御座傍,又见妖虹降玉堂;群盗四方如蚁聚,奸雄百辈皆鹰扬,吾侪长啸空拍手,闷来村店饮村酒;独善其身尽日安,何须千古名不朽!”

二人歌罢,抚掌大笑。玄德曰:“卧龙其在此间乎!”遂下马入店。见二人凭桌对饮:上首者白面长须,下首者清奇古貌。玄德揖而问曰:“二公谁是卧龙先生?”长须者曰:“公何人?欲寻卧龙何干?”玄德曰:“某乃刘备也。欲访先生,求济世安民之术。”长须者曰:“我等非卧龙,皆卧龙之友也:吾乃颍川石广元,此位是汝南孟公威。”玄德喜曰:“备久闻二公大名,幸得邂逅。今有随行马匹在此,敢请二公同往卧龙庄上一谈。”广元曰:“吾等皆山野慵懒之徒,不省治国安民之事,不劳下问。明公请自上马,寻访卧龙。”

玄德乃辞二人,上马投卧龙冈来。到庄前下马,扣门问童子曰:“先生今日在庄否?”童子曰:“现在堂上读书。”玄德大喜,遂跟童子而入。至中门,只见门上大书一联云:“淡泊以明志。宁静而致远。”玄德正看间,忽闻吟咏之声,乃立于门侧窥之,见草堂之上,一少年拥炉抱膝,歌曰:“凤翱翔于千仞兮,非梧不栖;士伏处于一方兮,非主不依。乐躬耕于陇亩兮,吾爱吾庐;聊寄傲于琴书兮,以待天时。”

玄德待其歌罢,上草堂施礼曰:“备久慕先生,无缘拜会。昨因徐元直称荐,敬至仙庄,不遇空回。今特冒风雪而来。得瞻道貌,实为万幸,”那少年慌忙答礼曰:“将军莫非刘豫州,欲见家兄否?”玄德惊讶曰:“先生又非卧龙耶?”少年曰:“某乃卧龙之弟诸葛均也。愚兄弟三人:长兄诸葛瑾,现在江东孙仲谋处为幕宾;孔明乃二家兄。”玄德曰:“卧龙今在家否?”均曰:“昨为崔州平相约,出外闲游去矣。”玄德曰:“何处闲游?”均曰:“或驾小舟游于江湖之中,或访僧道于山岭之上,或寻朋友于村落之间,或乐琴棋于洞府之内:往来莫测,不知去所。”玄德曰:“刘备直如此缘分浅薄,两番不遇大贤!”均曰:“少坐献茶。”张飞曰:“那先生既不在,请哥哥上马。”玄德曰:“我既到此间,如何无一语而回?”因问诸葛均曰:“闻令兄卧龙先生熟谙韬略,日看兵书,可得闻乎?”均曰:“不知。”张飞曰:“问他则甚!风雪甚紧,不如早归。”玄德叱止之。均曰:“家兄不在,不敢久留车骑;容日却来回礼。”玄德曰:“岂敢望先生枉驾。数日之后,备当再至。愿借纸笔作一书,留达令兄,以表刘备殷勤之意。”均遂进文房四宝。玄德呵开冻笔,拂展云笺,写书曰:“备久慕高名,两次晋谒,不遇空回,惆怅何似!窃念备汉朝苗裔,滥叨名爵,伏睹朝廷陵替,纲纪崩摧,群雄乱国,恶党欺君,备心胆俱裂。虽有匡济之诚,实乏经纶之策。仰望先生仁慈忠义,慨然展吕望之大才,施子房之鸿略,天下幸甚!社稷幸甚!先此布达,再容斋戒薰沐,特拜尊颜,面倾鄙悃。统希鉴原。”玄德写罢,递与诸葛均收了,拜辞出门。均送出,玄德再三殷勤致意而别。方上马欲行,忽见童子招手篱外,叫曰:“老先生来也。”玄德视之,见小桥之西,一人暖帽遮头,狐裘蔽体,骑着一驴,后随一青衣小童,携一葫芦酒,踏雪而来;转过小桥,口吟诗一首。诗曰:“一夜北风寒,万里彤云厚。长空雪乱飘,改尽江山旧。仰面观火虚,疑是玉龙斗。纷纷鳞甲飞,顷刻遍宇宙。骑驴过小桥,独叹梅花瘦!”玄德闻歌曰:“此真卧龙矣!”滚鞍下马,向前施礼曰:“先生冒寒不易!刘备等候久矣!”那人慌忙下驴答礼。

诸葛均在后曰:“此非卧龙家兄,乃家兄岳父黄承彦也。”玄德曰:“适间所吟之句,极其高妙。”承彦曰:“老夫在小婿家观《梁父吟》,记得这一篇;适过小桥,偶见篱落间梅花,故感而诵之。不期为尊客所闻。”玄德曰:“曾见令婿否?”承彦曰:“便是老夫也来看他。”玄德闻言,辞别承彦,上马而归。正值风雪又大,回望卧龙冈,悒怏不已。后人有诗单道玄德风雪访孔明。诗曰:“一天风雪访贤良,不遇空回意感伤。冻合溪桥山石滑,寒侵鞍马路途长。当头片片梨花落,扑面纷纷柳絮狂。回首停鞭遥望处,烂银堆满卧龙冈。”

玄德回新野之后,光阴荏苒,又早新春。乃令卜者揲蓍,选择吉期,斋戒三日,薰沐更衣,再往卧龙冈谒孔明。关、张闻之不悦,遂一齐入谏玄德。正是:高贤未服英雄志,屈节偏生杰士疑。未知其言若何,下文便晓。