\chapter{曹阿瞒许田打围~董国舅内阁受诏}

话说曹操举剑欲杀张辽,玄德攀住臂膊,云长跪于面前。玄德曰,“此等赤心之人,正
当留用。”云长曰:“关某素知文远忠义之士,愿以性命保之。”操掷剑笑曰:“我亦知文
远忠义,故戏之耳。”乃亲释其缚,解衣衣之,延之上坐,辽感其意,遂降。操拜辽为中郎
将,赐爵关内侯,使招安臧霸。霸闻吕布已死,张辽已降,遂亦引本部军投降。操厚赏之。
臧霸又招安孙观、吴敦、尹礼来降;独昌豨未肯归顺。操封臧霸为琅琊相。孙观等亦各加
官,令守青、徐沿海地面。将吕布妻女载回许都。大犒三军,拔寨班师。路过徐州,百姓焚
香遮道,请留刘使君为牧。操曰:“刘使君功大,且待面君封爵,回来未迟。”百姓叩谢。
操唤车骑将军车胄权领徐州。操军回许昌,封赏出征人员,留玄德在相府左近宅院歇定。

次日,献帝设朝,操表奏玄德军功,引玄德见帝。玄德具朝服拜于丹墀。帝宣上殿,问
曰:“卿祖何人?”玄德奏曰:“臣乃中山靖王之后,孝景皇帝阁下玄孙,刘雄之孙,刘弘
之子也。”帝教取宗族世谱检看,令宗正卿宣读曰:“孝景皇帝生十四子。第七子乃中山靖
王刘胜。胜生陆城亭侯刘贞。贞生沛侯刘昂。昂生漳侯刘禄。禄生沂水侯刘恋。恋生钦阳侯
刘英。英生安国侯刘建。建生广陵侯刘哀。哀生胶水侯刘宪。宪生祖邑侯刘舒。舒生祁阳侯
刘谊。谊生原泽侯刘必。必生颍川侯刘达。达生丰灵侯刘不疑。不疑生济川侯刘惠。惠生东
郡范令刘雄。雄生刘弘。弘不仕。刘备乃刘弘之子也。”帝排世谱,则玄德乃帝之叔也。帝
大喜,请入偏殿叙叔侄之礼。帝暗思:“曹操弄权,国事都不由朕主,今得此英雄之叔,朕
有助矣!”遂拜玄德为左将军、宜城亭侯。设宴款待毕,玄德谢恩出朝。自此人皆称为刘皇
叔。

曹操回府,荀彧等一班谋士入见曰:“天子认刘备为叔,恐无益于明公。”操曰:“彼
既认为皇叔,吾以天子之诏令之,彼愈不敢不服矣。况吾留彼在许都,名虽近君,实在吾掌
握之内,吾何惧哉?吾所虑者,太尉杨彪系袁术亲戚,倘与二袁为内应,为害不浅。当即除
之。”乃密使人诬告彪交通袁术,遂收彪下狱,命满宠按治之。时北海太守孔融在许都,因
谏操曰:“杨公四世清德,岂可因袁氏而罪之乎?”操曰:“此朝廷意也。”融曰:“使成
王杀召公,周公可得言不知耶?”操不得已,乃免彪官,放归田里。议郎赵彦愤操专横,上
疏劾操不奉帝旨、擅收大臣之罪。操大怒,即收赵彦杀之。于是百官无不悚惧。谋士程昱说
操曰:“今明公威名日盛,何不乘此时行王霸之事?”操曰:“朝廷股肱尚多,未可轻动。
吾当请天子田猎,以观动静。”于是拣选良马、名鹰、俊犬、弓矢俱备,先聚兵城外,操入
请天子田猎。帝曰:“田猎恐非正道。”操曰:“古之帝王,春搜夏苗,秋狝冬狩:四时出
郊,以示武于天下。今四海扰攘之时,正当借田猎以讲武。”帝不敢不从,随即上逍遥马,
带宝雕弓、金鈚箭,排銮驾出城。玄德与关、张各弯弓插箭,内穿掩心甲,手持兵器,引数
十骑随驾出许昌。曹操骑爪黄飞电马,引十万之众,与天子猎于许田。军士排开围场,周广
二百余里。操与天子并马而行,只争一马头。背后都是操之心腹将校。文武百官,远远侍
从,谁敢近前。当日献帝驰马到许田,刘玄德起居道傍。帝曰:“朕今欲看皇叔射猎。”玄
德领命上马,忽草中赶起一兔。玄德射之,一箭正中那兔。帝喝采。转过土坡,忽见荆棘中
赶出一只大鹿。帝连射三箭不中,顾谓操曰:“卿射之。”操就讨天子宝雕弓、金鈚箭,扣
满一射,正中鹿背,倒于草中。群臣将校,见了金鈚箭,只道天子射中,都踊跃向帝呼“万
岁”。曹操纵马直出,遮于天子之前以迎受之。众皆失色。玄德背后云长大怒,剔起卧蚕
眉,睁开丹凤眼,提刀拍马便出,要斩曹操。玄德见了,慌忙摇手送目。关公见兄如此,便
不敢动。玄德欠身向操称贺曰:“丞相神射,世所罕及!”操笑曰:“此天子洪福耳。”乃
回马向天子称贺,竟不献还宝雕弓,就自悬带。围场已罢,宴于许田。宴毕,驾回许都。众
人各自归歇。云长问玄德曰:“操贼欺君罔上,我欲杀之,为国除害,兄何止我?”玄德
曰:“投鼠忌器。操与帝相离只一马头,其心腹之人,周回拥侍;吾弟若逞一时之怒,轻有
举动,倘事不成,有伤天子,罪反坐我等矣。”云长曰:“今日不杀此贼,后必为祸。”玄
德曰:“且宜秘之,不可轻言。”却说献帝回宫,泣谓伏皇后曰:“朕自即位以来,奸雄并
起:先受董卓之殃,后遭傕、汜之乱。常人未受之苦,吾与汝当之。后得曹操,以为社稷之
臣;不意专国弄权,擅作威福。朕每见之,背若芒刺。今日在围场上,身迎呼贺,无礼已
极!早晚必有异谋,吾夫妇不知死所也!”伏皇后曰:“满朝公卿,俱食汉禄,竟无一人能
救国难乎?”言未毕,忽一人自外而入曰:“帝,后休忧。吾举一人,可除国害。”帝视
之,乃伏皇后之父伏完也。帝掩泪问曰:“皇丈亦知操贼之专横乎?”宪曰:“许田射鹿之
事,谁不见之?但满朝之中,非操宗族,则其门下。若非国戚,谁肯尽忠讨贼?老臣无权,
难行此事。车骑将军国舅董承可托也。”帝曰:“董国舅多赴国难,朕躬素知;可宜入内,
共议大事。”宪曰:“陛下左右皆操贼心腹,倘事泄,为祸不深。”帝曰:“然则奈何?”
完曰:“臣有一计:陛下可制衣一领,取玉带一条,密赐董承;却于带衬内缝一密诏以赐
之,令到家见诏,可以昼夜画策,神鬼不觉矣。”帝然之,伏完辞出。

帝乃自作一密诏,咬破指尖,以血写之,暗令伏皇后缝于玉带紫锦衬内,却自穿锦袍,
自系此带,令内史宣董承入。承见帝礼毕,帝曰:“朕夜来与后说霸河之苦,念国舅大功,
故特宣入慰劳。”承顿首谢。帝引承出殿,到太庙,转上功臣阁内。帝焚香礼毕,引承观画
像。中间画汉高祖容像。帝曰:“吾高祖皇帝起身何地?如何创业?”承大惊曰:“陛下戏
臣耳。圣祖之事,何为不知?高皇帝起自泗上亭长,提三尺剑,斩蛇起义,纵横四海,三载
亡秦,五年灭楚:遂有天下,立万世之基业。”帝曰:“祖宗如此英雄,子孙如此懦弱,岂
不可叹!”因指左右二辅之像曰:“此二人非留侯张良、酂侯萧何耶?”承曰:“然也。高
祖开基创业,实赖二人之力。”帝回顾左右较远,乃密谓承曰:“卿亦当如此二人立于朕
侧。”承曰:“臣无寸功,何以当此?”帝曰:“朕想卿西都救驾之功,未尝少忘,无可为
赐。”因指所着袍带曰:“卿当衣朕此袍,系朕此带,常如在朕左右也。”承顿首谢。帝解
袍带赐承,密语曰:“卿归可细观之,勿负朕意。”承会意,穿袍系带,辞帝下阁。

早有人报知曹操曰:“帝与董承登功臣阁说话。”操即入朝来看。董承出阁,才过宫
门,恰遇操来;急无躲避处,只得立于路侧施礼。操问曰:“国舅何来?”承曰:“适蒙天
子宣召,赐以锦袍玉带。”操问曰:“何故见赐?”承曰:“因念某旧日西都救驾之功,故
有此赐。”操曰:“解带我看。”承心知衣带中必有密诏,恐操看破,迟延不解。操叱左
右:“急解下来!”看了半晌,笑曰:“果然是条好玉带!再脱下锦袍来借看。”承心中畏
惧,不敢不从,遂脱袍献上。操亲自以手提起,对日影中细细详看。看毕,自己穿在身上,
系了玉带,回顾左右曰:“长短如何?”左右称美。操谓承曰:“国舅即以此袍带转赐与
吾,何如?”承告曰:“君恩所赐,不敢转赠;容某别制奉献。”操曰:“国舅受此衣带,
莫非其中有谋乎?”承惊曰:“某焉敢?丞相如要,便当留下。”操曰:“公受君赐,吾何
相夺?聊为戏耳。”遂脱袍带还承。

承辞操归家,至夜独坐书院中,将袍仔细反复看了,并无一物。承思曰:“天子赐我袍
带,命我细观,必非无意;今不见甚踪迹,何也?”随又取玉带检看,乃白玉玲珑,碾成小
龙穿花,背用紫锦为衬,缝缀端整,亦并无一物,承心疑,放于桌上,反复寻之。良久,倦
甚。正欲伏几而寝,忽然灯花落于带上,烧着背衬。承惊拭之,已烧破一处,微露素绢,隐
见血迹。急取刀拆开视之,乃天子手书血字密诏也。诏曰:“朕闻人伦之大,父子为先;尊
卑之殊,君臣为重。近日操贼弄权,欺压君父;结连党伍,败坏朝纲;敕赏封罚,不由朕
主。朕夙夜忧思,恐天下将危。卿乃国之大臣,朕之至戚,当念高帝创业之艰难,纠合忠义
两全之烈士,殄灭奸党,复安社稷,祖宗幸甚!破指洒血,书诏付卿,再四慎之,勿负朕
意!建安四年春三月诏。”

董承览毕,涕泪交流,一夜寝不能寐。晨起,复至书院中,将诏再三观看,无计可施。
乃放诏于几上,沈思灭操之计。忖量未定,隐几而卧。

忽侍郎王子服至。门吏知子服与董承交厚,不敢拦阻,竟入书院。见承伏几不醒,袖底
压着素绢,微露“朕”字。子服疑之,默取看毕,藏于袖中,呼承曰:“国舅好自在!亏你
如何睡得着!”承惊觉,不见诏书,魂不附体,手脚慌乱。子服曰:“汝欲杀曹公!吾当出
首。”承泣告曰:“若兄如此,汉室休矣!”子服曰:“吾戏耳。吾祖宗世食汉禄,岂无忠
心?愿助兄一臂之力,共诛国贼。”承曰:“兄有此心,国之大幸!”子服曰:“当于密室
同立义状,各舍三族,以报汉君。”承大喜,取白绢一幅,先书名画字。子服亦即书名画
字。书毕,子服曰:“将军吴子兰,与吾至厚,可与同谋。”承曰:“满朝大臣,惟有长水
校尉种辑、议郎吴硕是吾心腹,必能与我同事。”正商议间,家僮入报种辑、吴硕来探。承
曰:“此天助我也!”教子服暂避于屏后。承接二人入书院坐定,茶毕,辑曰:“许田射猎
之事,君亦怀恨乎?”承曰:“虽怀恨,无可奈何。”硕曰:“吾誓杀此贼,恨无助我者
耳!”辑曰:“为国除害,虽死无怨!”王子服从屏后出曰:“汝二人欲杀曹丞相!我当出
首,董国舅便是证见。”种辑怒曰:“忠臣不怕死!吾等死作汉鬼,强似你阿附国贼!”承
笑曰:“吾等正为此事,欲见二公。王侍郎之言乃戏耳。”便于袖中取出诏来与二人看。二
人读诏,挥泪不止。承遂请书名。子服曰:“二公在此少待,吾去请吴子兰来。”子服去不
多时,即同子兰至,与众相见,亦书名毕。承邀于后堂会饮。忽报西凉太守马腾相探。承
曰:“只推我病,不能接见。”门吏回报。腾大怒曰:“我夜来在东华门外,亲见他锦袍玉
带而出,何故推病耶!吾非无事而来,奈何拒我!”门吏入报,备言腾怒。承起曰:“诸公
少待,暂容承出。”随即出厅延接。礼毕坐定,腾曰:“腾入觐将还,故来相辞,何见拒
也?”承曰:“贱躯暴疾,有失迎候,罪甚!”腾曰:“面带春色,未见病容。”承无言可
答。腾拂袖便起,嗟叹下阶曰:“皆非救国之人也!”承感其言,挽留之,问曰:“公谓何
人非救国之人?”腾曰:“许田射猎之事,吾尚气满胸膛;公乃国之至戚,犹自殆于酒色,
而不思讨贼,安得为皇家救难扶灾之人乎!”承恐其诈,佯惊曰:“曹丞相乃国之大臣,朝
廷所倚赖,公何出此言?”腾大怒曰:“汝尚以曹贼为好人耶?”承曰:“耳目甚近,请公
低声。”腾曰:“贪生怕死之徒,不足以论大事!”说罢又欲起身。承知腾忠义,乃曰:
“公且息怒。某请公看一物。”遂邀腾入书院,取诏示之。腾读毕,毛发倒竖,咬齿嚼唇,
满口流血,谓承曰:“公若有举动,吾即统西凉兵为外应。”承请腾与诸公相见,取出义
状,教腾书名。腾乃取酒歃血为盟曰:“吾等誓死不负所约!”指坐上五人言曰:“若得十
人,大事谐矣。”承曰:“忠义之士,不可多得。若所与非人,则反相害矣。”腾教取《鸳
行鹭序簿》来检看。检到刘氏宗族,乃拍手言曰:“何不共此人商议?”众皆问何人。马腾
不慌不忙,说出那人来。正是:本因国舅承明诏,又见宗潢佐汉朝。毕竟马腾之言如何,且
听下文分解。