\chapter{马孟起兴兵雪恨~曹阿瞒割须弃袍}

却说献策之人,乃治书侍御史陈群,字长文。操问曰:“陈长文有何良策?”群曰:
“今刘备、孙权结为唇齿,若刘备欲取西川,丞相可命上将提兵,会合淝之众,径取江南,
则孙权必求救于刘备;备意在西川,必无心救权;权无救则力乏兵衰,江东之地,必为丞相
所得。若得江东,则荆州一鼓可平也;荆州既平,然后徐图西川:天下定矣。”操曰:“长
文之言,正合吾意。”即时起大兵三十万,径下江南;令合淝张辽,准备粮草,以为供给。

早有细作报知孙权。权聚众将商议。张昭曰:“可差人往鲁子敬处,教急发书到荆州,
使玄德同力拒曹。子敬有恩于玄德,其言必从;且玄德既为东吴之婿,亦义不容辞。若玄德
来相助。江南可无患矣。”权从其言,即遣人谕鲁肃,使求救于玄德。肃领命,随即修书使
人送玄德,玄德看了书中之意,留使者于馆舍,差人往南郡请孔明。孔明到荆州,玄德将鲁
肃书与孔明看毕,孔明曰:“也不消动江南之兵,也不必动荆州之兵,自使曹操不敢正觑东
南。”便回书与鲁肃,教高枕无忧,若但有北兵侵犯,皇叔自有退兵之策。使者去了。玄德
问曰:“今操起三十万大军,会合淝之众,一拥而来,先生有何妙计,可以退之?”孔明
曰:“操平生所虑者,乃西凉之兵也。今操杀马腾,其子马超现统西凉之众,必切齿操贼。
主公可作一书,往结马超,使超兴兵入关,则操又何暇下江南乎?”玄德大喜,即时作书,
遣一心腹人,径往西凉州投下。

却说马超在西凉州,夜感一梦:梦见身卧雪地,群虎来咬。惊惧而觉,心中疑惑,聚帐
下将佐,告说梦中之事。帐下一人应声曰:“此梦乃不祥之兆也。”众视其人,乃帐前心腹
校尉,姓庞,名德,字令明。超问:“令明所见若何?”德曰:“雪地遇虎,梦兆殊恶。莫
非老将军在许昌有事否?”言未毕,一人踉跄而入,哭拜于地曰:“叔父与弟皆死矣!”超
视之,乃马岱也。超惊问何为。岱曰:“叔父与侍郎黄奎同谋杀操,不幸事泄,皆被斩于
市,二弟亦遇害。惟岱扮作客商,星夜走脱。超闻言,哭倒于地。众将救起。超咬牙切齿,
痛恨操贼。忽报荆州刘皇叔遣人赍书至。超拆视之。书略曰:“伏念汉室不幸,操贼专权,
欺君罔上,黎民凋残。备昔与令先君同受密诏,誓诛此贼。今令先君被操所害,此将军不共
天地、不同日月之仇也。若能率西凉之兵,以攻操之右,备当举荆襄之众,以遏操之前:则
逆操可擒,奸党可灭,仇辱可报,汉室可兴矣。书不尽言,立待回音。”

马超看毕,即时挥涕回书,发使者先回,随后便起西凉军马,正欲进发,忽西凉太守韩
遂使人请马超往见。超至遂府,遂将出曹操书示之。内云:“若将马超擒赴许都,即封汝为
西凉侯。”超拜伏于地曰:“请叔父就缚俺兄弟二人,解赴许昌,免叔父戈戟之劳。”韩遂
扶起曰:“吾与汝父结为兄弟,安忍害汝?汝若兴兵,吾当相助。”马超拜谢。

韩遂便将操使者推出斩之,乃点手下八部军马,一同进发。那八部?乃侯选、程银、李
堪、张横、梁兴、成宜、马玩、杨秋也。八将随着韩遂,合马超手下庞德、马岱,共起二十
万大兵,杀奔长安来。

长安郡守钟繇,飞报曹操;一面引军拒敌,布阵于野。西凉州前部先锋马岱,引军一万
五千,浩浩荡荡,漫山遍野而来。钟繇出马答话。岱使宝刀一口,与繇交战。不一合,繇大
败奔走。岱提刀赶来。马超、韩遂引大军都到,围住长安。钟繇上城守护。长安乃西汉建都
之处,城郭坚固。壕堑险深,急切攻打不下。一连围了十日,不能攻破。庞德进计曰:“长
安城中土硬水碱,甚不堪食,更兼无柴。今围十日,军民饥荒。不如暂且收军,只须如此如
此,长安唾手可得。”马超曰:“此计大妙!”即时差“令”字旗传与各部,尽教退军,马
超亲自断后。各部军马渐渐退去。钟繇次日登城看时,军皆退了,只恐有计;令人哨探,果
然远去,方才放心。纵令军民出城打柴取水,大开城门,放人出入。至第五日,人报马超兵
又到,军民竞奔入城,钟繇仍复闭城坚守。

却说钟繇弟钟进,守把西门,约近三更,城门里一把火起。钟进急来救时,城边转过一
人,举刀纵马大喝曰:“庞德在此!”钟进措手不及,被庞德一刀斩于马下,杀散军校,斩
关断锁,放马超、韩遂军马入城。钟繇从东门弃城而走。马超、韩遂得了城池,赏劳三军。

钟繇退守潼关,飞报曹操。操知失了长安,不敢复议南征,遂唤曹洪、徐晃分付:“先
带一万人马,替钟繇紧守潼关。如十日内失了关隘,皆斩;十日外,不干汝二人之事。我统
大军随后便至。”二人领了将令,星夜便行。曹仁谏曰:“洪性躁,诚恐误事。”操曰:
“你与我押送粮草,便随后接应。”

却说曹洪、徐晃到潼关,替钟繇坚守关隘,并不出战。马超领军来关下,把曹操三代毁
骂。曹洪大怒,要提兵下关厮杀。徐晃谏曰:“此是马超要激将军厮杀,切不可与战。待丞
相大军来,必有主画。”马超军日夜轮流来骂。曹洪只要厮杀,徐晃苦苦挡住。至第九日,
在关上看时,西凉军都弃马在于关前草地上坐;多半困乏,就于地上睡卧。曹洪便教备马,
点起三千兵杀下关来。西凉兵弃马抛戈而走。洪迤逦追赶。时徐晃正在关上点视粮车,闻曹
洪下关厮杀,大惊,急引兵随后赶来,大叫曹洪回马。忽然背后喊声大震,马岱引军杀至。
曹洪、徐晃急回走时,一棒鼓响,山背后两军截出:左是马超、右是庞德,混杀一阵。曹洪
抵挡不住,折军大半,撞出重围,奔到关上。西凉兵随后赶来,洪等弃关而走。庞德直追过
潼关,撞见曹仁军马,救了曹洪等一军。马超接应庞德上关。

曹洪失了潼关。奔见曹操。操曰:“与你十日限,如何九日失了潼关?”洪曰:“西凉
军兵,百般辱骂,因见彼军懈怠,乘势赶去,不想中贼奸计。”操曰:“洪年幼躁暴,徐晃
你须晓事!”晃曰:“累谏不从。当日晃在关上点粮车,比及知道,小将军已下关了。晃恐
有失,连忙赶去,已中贼奸计矣。”操大怒,喝斩曹洪。众官告免。曹洪服罪而退。

操进兵直叩潼关。曹仁曰:“可先下定寨栅,然后打关未迟。”操令砍伐树木,起立排
栅,分作三寨:左寨曹仁,右寨夏侯渊,操自居中寨。次日,操引三寨大小将校,杀奔关隘
前去,正遇西凉军马。两边各布阵势。操出马于门旗下,看西凉之兵,人人勇健,个个英
雄。又见马超生得面如傅粉,唇若抹朱,腰细膀宽,声雄力猛,白袍银铠,手执长枪,立马
阵前;上首庞德,下首马岱。操暗暗称奇,自纵马谓超曰:“汝乃汉朝名将子孙,何故背反
耶?”超咬牙切齿,大骂:“操贼!歉君罔上,罪不容诛!害我父弟,不共戴天之仇!吾当
活捉生啖汝肉!”说罢,挺枪直杀过来。曹操背后于禁出迎。两马交战,斗得八九合,于禁
败走。张郃出迎,战二十合亦败走。李通出迎,超奋威交战,数合之中,一枪刺李通于马
下。超把枪望后一招,西凉兵一齐冲杀过来。操兵大败。西凉兵来得势猛,左右将佐,皆抵
当不住。马超、庞德、马岱引百余骑,直入中军来捉曹操。操在乱军中,只听得西凉军大
叫:“穿红袍的是曹操!”操就马上急脱下红袍。又听得大叫:“长髯者是曹操!”操惊
慌,掣所佩刀断其髯。军中有人将曹操割髯之事,告知马超,超遂令人叫拿:“短髯者是曹
操!”操闻知,即扯旗角包颈而逃。后人有诗曰:“潼关战败望风逃,孟德怆惶脱锦袍。剑
割髭髯应丧胆,马超声价盖天高。”

曹操正走之间,背后一骑赶来,回头视之,正是马超。操大惊。左右将校见超赶来,各
自逃命,只撤下曹操。超厉声大叫曰:“曹操休走!”操惊得马鞭坠地。看看赶上,马超从
后使枪搠来。操绕树而走,超一枪搠在树上;急拔下时,操已走远。超纵马赶来,山坡边转
过一将,大叫:“勿伤吾主!曹洪在此!”轮刀纵马,拦住马超。操得命走脱。洪与马超战
到四五十合,渐渐刀法散乱,气力不加。夏侯渊引数十骑随到。马超独自一人,恐被所算,
乃拨马而回,夏侯渊也不来赶。

曹操回寨,却得曹仁死据定了寨栅,因此不曾多折军马。操入帐叹曰:“吾若杀了曹
洪,今日必死于马超之手也!”遂唤曹洪,重加赏赐。收拾败军,坚守寨栅,深沟高垒,不
许出战。超每日引兵来寨前辱骂搦战。操传令教军士坚守,如乱动者斩。诸将曰:“西凉之
兵,尽使长枪,当选弓弩迎之。”操曰:“战与不战,皆在于我,非在贼也。贼虽有长枪,
安能便刺?诸公但坚壁观之,贼自退矣。”诸将皆私相议曰:“丞相自来征战,一身当先;
今败于马超,何如此之弱也?”

过了几日,细作报来:“马超又添二万生力兵来助战,乃是羌人部落。”操闻知大喜。
诸将曰:“马超添兵,丞相反喜。何也?”操曰:“待吾胜了,却对汝等说。”三日后又报
关上又添军马。操又大喜,就于帐中设宴作贺。诸将皆暗笑。操曰:“诸公笑我无破马超之
谋,公等有何良策?”徐晃进曰:“今丞相盛兵在此,贼亦全部现屯关上,此去河西,必无
准备;若得一军暗渡蒲阪津,先截贼归路,丞相径发河北击之,贼两不相应,势必危矣。”
操曰:“公明之言,正合吾意。”便教徐晃引精兵四千,和朱灵同去径袭河西,伏于山谷之
中,“待我渡河北同时击之。”、徐晃、朱灵领命、先引四千军暗暗去了。操下令,先教曹
洪于蒲阪津,安排船筏。留曹仁守寨,操自领兵渡渭河。早有细作报知马超。超曰:“今操
不攻潼关,而使人准备船筏,欲渡河北,必将遏吾之后也。吾当引一军循河拒住岸北。操兵
不得渡,不消二十日,河东粮尽,操兵必乱,却循河南而击之,操可擒矣。”韩遂曰:“不
必如此。岂不闻兵法有云:‘兵半渡可击,’待操兵渡至一半,汝却于南岸击之,操兵皆死
于河内矣。超曰:“叔父之言甚善。”即使人探听曹操几时渡河。却说曹操整兵已毕,分三
停军,前渡渭河,比及人马到河口时,日光初起。操先发精兵渡过北岸,开创营寨。操自引
亲随护卫军将百人,按剑坐于南岸,看军渡河。忽然人报:“后边白袍将军到了!”众皆认
得是马超。一拥下船。河边军争上船者,声喧不止。操犹坐而不动,按剑指约休闹。只听得
人喊马嘶,蜂拥而来,船上一将跃身上岸,呼曰:“贼至矣!请丞相下船!”操视之,乃许
褚也。操口内犹言:“贼至何妨?”回头视之,马超已离不得百余步,许褚拖操下船时,船
已离岸一丈有余,褚负操一跃上船。随行将士尽皆下水,扳住船边,争欲上船逃命。船小将
翻,褚掣刀乱砍,傍船手尽折,倒于水中。急将船望下水棹去。许褚立于梢上。忙用木篙撑
之。操伏在许褚脚边。马超赶到河岸,见船已流在半河,遂拈弓搭箭,喝令骁将绕河射之。
矢如雨急。褚恐伤曹操,以左手举马鞍遮之。马超箭不虚发,船上驾舟之人,应弦落水;船
中数十人皆被射倒。其船反撑不定,于急水中旋转。许褚独奋神威,将两腿夹舵摇撼,一手
使篙撑船,一手举鞍遮护曹操。时有渭南县令丁斐,在南山之上,见马超追操甚急,恐伤操
命,遂将寨内牛只马匹,尽驱于外,漫山遍野,皆是牛马。西凉兵见之。都回身争取牛马,
无心追赶,曹操因此得脱。方到北岸,便把船筏凿沉。诸将听得曹操在河中逃难,急来救
时,操已登岸。许褚身被重铠,箭皆嵌在甲上。众将保操至野寨中,皆拜于地而问安。操大
笑曰:“我今日几为小贼所困!”褚曰;“若非有人纵马放牛以诱贼,贼必努力渡河矣。”
操问曰:“诱贼者谁也?”有知者答曰:“渭南县令丁斐也。”少顷,斐入见。操谢曰:
“若非公之良谋,则吾被贼所擒矣。”遂命为典军校尉,斐曰:“贼虽暂去,明日必复来。
须以良策拒之。”操曰:“吾已准备了也。”遂唤诸将各分头循河筑起甬道,暂为寨脚,贼
若来时,陈兵于甬道外。内虚立旌旗,以为疑兵;更沿河掘下壕堑,虚土棚盖,河内以兵诱
之:“贼急来必陷,贼陷便可击矣。”却说马超回见韩遂,说:“几乎捉住曹操!有一将奋
勇负操下船去了,不知何人。”遂曰:“吾闻曹操选极精壮之人,为帐前侍卫,名曰虎卫
军,以骁将典韦、许褚领之。典韦已死,今救曹操者,必许褚也。此人勇力过人,人皆称为
虎痴;如遇之。不可轻敌。”超曰:“吾亦闻其名久矣。”遂曰:“今操渡河,将袭我后。
可速攻之。不可令他创立营寨。若立营寨,急难剿除。”超曰:“以侄愚意。还只拒住北
岸。使彼不得渡河,乃为上策。”遂曰:“贤侄守寨,吾引军循河战操,若何?”超曰:
“令庞德为先锋,跟叔父前去。”

于是韩遂与庞德将兵五万,直抵渭南。操令众将于甬道两旁诱之。庞德先引铁骑千余,
冲突而来。喊声起处,人马俱落于陷马坑内。庞德踊身一跳。跃出土坑,立于平地,立杀数
人,步行砍出重围。韩遂已被困在垓心,庞德步行救之。正遇着曹仁部将曹永,被庞德一刀
砍于马下,夺其马,杀开一条血路,救出韩遂,投东南而走。背后曹兵赶来,马超引军接
应,杀败曹兵,复救出大半军马。战至日暮方回。计点人马,折了将佐程银、张横,陷坑中
死者二百余人。超与韩遂商议:“若迁延日久,操于河北立了营寨,难以退敌;不若乘今夜
引轻骑去劫野营。”遂曰:“须分兵前后相救。”于是超自为前部,令庞德、马岱为后应,
当夜便行。

却说曹操收兵屯渭北,唤诸将曰:“贼欺我未立寨棚,必来劫野营。可四散伏兵,虚其
中军。号炮响时,伏兵尽起,一鼓可擒也。”众将依令,伏兵已毕。当夜,马超却先使成宜
引三十骑往前哨探,成宜见无人马,径入中军。操军见西凉兵到,遂放号炮。四面伏兵皆
出,只围得三十骑。成宜被夏侯渊所杀。马超却自从背后与庞德、马岱兵分三路蜂拥杀来。
正是:纵有伏兵能候敌,怎当键将共争先?未知胜负若何,且看下文分解。