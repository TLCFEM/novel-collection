\chapter{孔明挥泪斩马谡~周鲂断发赚曹休}

却说献计者,乃尚书孙资也。曹睿问曰:“卿有何妙计?”资奏曰:“昔太祖武皇帝收张鲁时,危而后济;常对群臣曰:南郑之地,真为天狱。中斜谷道为五百里石穴,非用武之地。今若尽起天下之兵伐蜀,则东吴又将入寇。不如以现在之兵,分命大将据守险要,养精蓄锐。不过数年,中国日盛,吴、蜀二国必自相残害:那时图之,岂非胜算?乞陛下裁之。”睿乃问司马懿曰:“此论若何?懿奏曰:“孙尚书所言极当。”睿从之,命懿分拨诸将守把险要,留郭淮、张郃守长安。大赏三军,驾回洛阳。却说孔明回到汉中,计点军士,只少赵云、邓芝,心中甚忧;乃令关兴、张苞,各引一军接应。二人正欲起身,忽报赵云、邓芝到来,并不曾折一人一骑;辎重等器,亦无遗失。孔明大喜,亲引诸将出迎。赵云慌忙下马伏地曰:“败军之将,何劳丞相远接?”孔明急扶起,执手而言曰:“是吾不识贤愚,以致如此!各处兵将败损,惟子龙不折一人一骑,何也?”邓芝告曰:“某引兵先行,子龙独自断后,斩将立功,敌人惊怕,因此军资什物,不曾遗弃。”孔明曰:“真将军也!”遂取金五十斤以赠赵云,又取绢一万匹赏云部卒。云辞曰:“三军无尺寸之功,某等俱各有罪;若反受赏,乃丞相赏罚不明也。且请寄库,候今冬赐与诸军未迟。”孔明叹曰:“先帝在日,常称子龙之德,今果如此!”乃倍加钦敬。

忽报马谡、王平、魏延、高翔至。孔明先唤王平入帐,责之曰:“吾令汝同马谡守街亭,汝何不谏之,致使失事?”平曰:“某再三相劝,要在当道筑土城,安营守把。参军大怒不从,某因此自引五千军离山十里下寨。魏兵骤至,把山四面围合,某引兵冲杀十余次,皆不能入。次日土崩瓦解,降者无数。某孤军难立,故投魏文长求救。半途又被魏兵困在山谷之中,某奋死杀出。比及归寨,早被魏兵占了。及投列柳城时,路逢高翔,遂分兵三路去劫魏寨,指望克复街亭。因见街亭并无伏路军,以此心疑。登高望之,只见魏延、高翔被魏兵围住,某即杀入重围,救出二将,就同参军并在一处。某恐失却阳平关,因此急来回守。非某之不谏也。丞相不信,可问各部将校。”孔明喝退,又唤马谡入帐。

谡自缚跪于帐前。孔明变色曰:“汝自幼饱读兵书,熟谙战法。吾累次丁宁告戒:街亭是吾根本。汝以全家之命,领此重任。汝若早听王平之言,岂有此祸?今败军折将,失地陷城,皆汝之过也!若不明正军律,何以服众?汝今犯法,休得怨吾。汝死之后,汝之家小,吾按月给与禄粮,汝不必挂心。”叱左右推出斩之。谡泣曰:“丞相视某如子,某以丞相为父。某之死罪,实已难逃;愿丞相思舜帝殛鲧用禹之义,某虽死亦无恨于九泉!”言讫大哭。孔明挥泪曰:“吾与汝义同兄弟,汝之子即吾之子也,不必多嘱。”左右推出马谡于辕门之外,将斩。参军蒋琬自成都至,见武士欲斩马谡,大惊,高叫:“留人!”入见孔明曰:“昔楚杀得臣而文公喜。今天下未定,而戮智谋之臣,岂不可惜乎?”孔明流涕而答曰:“昔孙武所以能制胜于天下者,用法明也。今四方分争,兵戈方始,若复废法,何以讨贼耶?合当斩之。”须臾,武士献马谡首级于阶下。孔明大哭不已。蒋琬问曰:“今幼常得罪,既正军法,丞相何故哭耶?”孔明曰:“吾非为马谡而哭。吾想先帝在白帝城临危之时,曾嘱吾曰:“马谡言过其实,不可大用。今果应此言。乃深恨己之不明,追思先帝之言,因此痛哭耳!”大小将士,无不流涕。马谡亡年三十九岁,时建兴六年夏五月也。后人有诗曰:“失守街亭罪不轻,堪嗟马谡枉谈兵。辕门斩首严军法,拭泪犹思先帝明。”

却说孔明斩了马谡,将首级遍示各营已毕,用线缝在尸上,具棺葬之,自修祭文享祀;将谡家小加意抚恤,按月给与禄米。于是孔明自作表文,令蒋琬申奏后主,请自贬丞相之职。琬回成都,入见后主,进上孔明表章。后主拆视之。表曰:“臣本庸才,叨窃非据,亲秉旄钺,以励三军。不能训章明法,临事而惧,至有街亭违命之阙,箕谷不戒之失。咎皆在臣,授任无方。臣明不知人,恤事多暗。《春秋》责帅,臣职是当。请自贬三等,以督厥咎。臣不胜惭愧,俯伏待命!”后主览毕曰:“胜负兵家常事,丞相何出此言?”侍中费祎奏曰:“臣闻治国者,必以奉法为重。法若不行,何以服人?丞相败绩,自行贬降,正其宜也。”后主从之,乃诏贬孔明为右将军,行丞相事,照旧总督军马,就命费祎赍诏到汉中。

孔明受诏贬降讫,祎恐孔明羞赧,乃贺曰:“蜀中之民,知丞相初拔四县,深以为喜。”孔明变色曰:“是何言也!得而复失,与不得同。公以此贺我,实足使我愧赧耳。”祎又曰:“近闻丞相得姜维,天子甚喜。”孔明怒曰:“兵败师还,不曾夺得寸土,此吾之大罪也。量得一姜维,于魏何损?”祎又曰:“丞相现统雄师数十万,可再伐魏乎?”孔明曰:“昔大军屯于祁山、箕谷之时,我兵多于贼兵,而不能破贼,反为贼所破:此病不在兵之多寡,在主将耳。今欲减兵省将,明罚思过,较变通之道于将来;如其不然,虽兵多何用?自今以后,诸人有远虑于国者,但勤攻吾之阙,责吾之短,则事可定,贼可灭,功可翘足而待矣。”费祎诸将皆服其论。费祎自回成都。

孔明在汉中,惜军爱民,励兵讲武,置造攻城渡水之器,聚积粮草,预备战筏,以为后图。细作探知,报入洛阳,魏主曹睿闻知,即召司马懿商议收川之策。懿曰:“蜀未可攻也。方今天道亢炎,蜀兵必不出;若我军深入其地,彼守其险要,急切难下。”睿曰:“倘蜀兵再来入寇,如之奈何?”懿曰:“臣已算定今番诸葛亮必效韩信暗度陈仓之计。臣举一人往陈仓道口,筑城守御,万无一失:此人身长九尺,猿臂善射,深有谋略。若诸葛亮入寇,此人足可当之。”睿大喜,问曰:“此何人也?”懿奏曰:“乃太原人,姓郝,名昭,字伯道,现为杂号将军,镇守河西。”睿从之,加郝昭为镇西将军,命守把陈仓道口,遣使持诏去讫。

忽报扬州司马大都督曹休上表,说东吴鄱阳太守周鲂,愿以郡来降,密遣人陈言七事,说东吴可破,乞早发兵取之。睿就御床上展开,与司马懿同观。懿奏曰:“此言极有理,吴当灭矣!臣愿引一军往助曹休。”忽班中一人进曰:“吴人之言,反覆不一,未可深信。周鲂智谋之士,必不肯降,此特诱兵之诡计也。”众视之,乃建威将军贾逵也。懿曰:“此言亦不可不听,机会亦不可错失。”魏主曰:“仲达可与贾逵同助曹休。”二人领命去讫。于是曹休引大军径取皖城;贾逵引前将军满宠、东莞太守胡质,径取阳城,直向东关;司马懿引本部军径取江陵。却说吴主孙权,在武昌东关,会多官商议曰:“今有鄱阳太守周鲂密表,奏称魏扬州都督曹休,有人寇之意。今鲂诈施诡计,暗陈七事,引诱魏兵深入重地,可设伏兵擒之。今魏兵分三路而来,诸卿有何高见?”顾雍进曰:“此大任非陆伯言不敢当也。”权大喜,乃召陆逊,封为辅国大将军、平北都元帅,统御林大兵,摄行王事:授以白旄黄钺,文武百官,皆听约束。权亲自与逊执鞭。逊领命谢恩毕,乃保二人为左右都督,分兵以迎三道。权问何人。逊曰:“奋威将军朱桓,绥南将军全琮,二人可为辅佐。”权从之,即命朱桓为左都督,全琮为右都督,于是陆逊总率江南八十一州并荆湖之众七十余万,令朱桓在左,全琮在右。逊自居中,三路进兵。朱桓献策曰:“曹休以亲见任,非智勇之将也。今听周鲂诱言,深入重地,元帅以兵击之,曹休必败。败后必走两条路:左乃夹石,右乃挂车。此二条路,皆山僻小径,最为险峻。某愿与全子璜各引一军,伏于山险,先以柴木大石塞断其路,曹休可擒矣。若擒了曹休,便长驱直进,唾手而得寿春,以窥许、洛,此万世一时也。”逊曰:“此非善策,吾自有妙用。”于是朱桓怀不平而退。逊令诸葛瑾等拒守江陵,以敌司马懿。诸路俱各调拨停当。却说曹休兵临皖城,周鲂来迎,径到曹休帐下。休问曰:“近得足下之书,所陈七事,深为有理,奏闻天子,故起大军三路进发。若得江东之地,足下之功不小。有人言足下多谋,诚恐所言不实。吾料足下必不欺我。”周鲂大哭,急掣从人所佩剑欲自刎。休急止之。鲂仗剑而言曰:“吾所陈七事,恨不能吐出心肝。今反生疑,必有吴人使反间之计也。若听其言,吾必死矣。吾之忠心,惟天可表!”言讫,又欲自刎。曹休大惊,慌忙抱住曰:“吾戏言耳,足下何故如此!”鲂乃用剑割发掷于地曰:“吾以忠心待公,公以吾为戏,吾割父母所遗之发,以表此心!”曹休乃深信之,设宴相待。席罢,周鲂辞去。忽报建威将军贾逵来见,休令入,问曰:“汝此来何为?”逵曰:“某料东吴之兵,必尽屯于皖城。都督不可轻进,待某两下夹攻,贼兵可破矣。”休怒曰:“汝欲夺吾功耶?”逵曰:“又闻周鲂截发为誓,此乃诈也,昔要离断臂,刺杀庆忌。未可深信。”休大怒曰:“吾正欲进兵,汝何出此言以慢军心!”叱左右推出斩之。众将告曰:“未及进兵,先斩大将,于军不利。且乞暂免。”休从之,将贾逵兵留在寨中调用,自引一军来取东关。时周鲂听知贾逵削去兵权,暗喜曰:“曹休若用贾逵之言,则东吴败矣!今天使我成功也!”即遣人密到皖城,报知陆逊。逊唤诸将听令曰:“前面石亭,虽是山路,足可埋伏。早先去占石亭阔处,布成阵势,以待魏军。”遂令徐盛为先锋,引兵前进。却说曹休命周鲂引兵而进,正行间,休问曰:“前至何处?”鲂曰:“前面石亭也,堪以屯兵。”休从之,遂率大军并车仗等器,尽赴石亭驻扎。次日,哨马报道:“前面吴兵不知多少,据住山口。”休大惊曰:“周鲂言无兵,为何有准备?”急寻鲂问之。人报周鲂引数十人,不知何处去了。休大悔曰:“吾中贼之计矣!虽然如此,亦不足惧!”遂令大将张普为先锋,引数千兵来与吴兵交战。两阵对圆,张普出马骂曰:“贼将早降!”徐盛出马相迎。战无数合,普抵敌不住,勒马收兵,回见曹休,言徐盛勇不可当。休曰:“吾当以奇兵胜之。”就令张普引二万军伏于石亭之南,又令薛乔引二万军伏于石亭之北。“明日吾自引一千兵搦战,却佯输诈败,诱到北山之前,放炮为号,三面夹攻,必获大胜。”二将受计,各引二万军到晚埋伏去了。却说陆逊唤朱桓、全琮分付曰:“汝二人各引三万军,从石亭山路抄到曹休寨后,放火为号;吾亲率大军从中路而进:可擒曹休也。”当日黄昏,二将受计引兵而进。二更时分,朱桓引一军正抄到魏寨后,迎着张普伏兵。普不知是吴兵,径来问时,被朱桓一刀斩于马下。魏兵便走。桓令后军放火。全琮引一军抄到魏寨后,正撞在薛乔阵里,就那里大杀一阵。薛乔败走,魏兵大损,奔回本寨。后面朱桓、全琮两路杀来。曹休寨中大乱,自相冲击。休慌上马,望夹石道奔走。徐盛引大队军马,从正路杀来。魏兵死者不可胜数,逃命者尽弃衣甲。曹休大惊,在夹石道中奋力奔走。忽见一彪军从小路冲出,为首大将,乃贾逵也。休惊慌少息,自愧曰:“吾不用公言,果遭此败!”逵曰:“都督可速出此道:若被吴兵以木石塞断,吾等皆危矣!”于是曹休骤马而行,贾逵断后。逵于林木盛茂处,及险峻小径,多设旌旗以为疑兵。及至徐盛赶到,见山坡下闪出旗角,疑有埋伏,不敢追赶,收兵而回。因此救了曹休。司马懿听知休败,亦引兵退去。

却说陆逊正望捷音,须臾,徐盛、朱桓、全琮皆到。所得车仗、牛马、驴骡、军资、器械,不计其数,降兵数万余人。逊大喜,即同太守周鲂并诸将班师还吴。吴主孙权,领文武官僚出武昌城迎接,以御盖覆逊而入。诸将尽皆升赏。权见周鲂无发,慰劳曰:“卿断发成此大事,功名当书于竹帛也。”即封周鲂为关内侯;大设筵会,劳军庆贺。陆逊奏曰:“今曹休大败,魏已丧胆;可修国书,遣使入川,教诸葛亮进兵攻之。”权从其言,遂遣使赍书入川去。正是:只因东国能施计,致令西川又动兵。未知孔明再来伐魏,胜负如何,且看下文分解。