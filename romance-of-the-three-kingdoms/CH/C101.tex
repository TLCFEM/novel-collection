\chapter{出陇上诸葛妆神~奔剑阁张中计}

却说孔明用减兵添灶之法,退兵到汉中;司马懿恐有埋伏,不敢追赶,亦收兵回长安去
了,因此蜀兵不曾折了一人。孔明大赏三军已毕,回到成都,入见后主,奏曰:“老臣出了
祁山,欲取长安,忽承陛下降诏召回,不知有何大事?”后主无言可对;良久,乃曰:“朕
久不见丞相之面,心甚思慕,故特诏回,一无他事。”孔明曰:“此非陛下本心,必有奸臣
谗谮,言臣有异志也。”后主闻言,默然无语。孔明曰:“老臣受先帝厚恩,誓以死报。今
若内有奸邪,臣安能讨贼乎?”后主曰:“朕因过听宦官之言,一时召回丞相。今日茅塞方
开,悔之不及矣!”孔明遂唤众宦官究问,方知是苟安流言;急令人捕之,已投魏国去了。
孔明将妄奏的宦官诛戮,余皆废出宫外;又深责蒋琬、费祎等不能觉察奸邪,规谏天子。二
人唯唯服罪。孔明拜辞后主,复到汉中,一面发檄令李严应付粮草,仍运赴军前;一面再议
出师。杨仪曰:“前数兴兵,军力罢敝,粮又不继;今不如分兵两班,以三个月为期:且如
二十万之兵,只领十万出祁山,住了三个月,却教这十万替回,循环相转。若此则兵力不
乏,然后徐徐而进,中原可图矣。”孔明曰:“此言正合我意。吾伐中原,非一朝一夕之
事,正当为此长久之计。”遂下令,分兵两班,限一百日为期,循环相转,违限者按军法处
治。建兴九年春二月,孔明复出师伐魏。时魏太和五年也。魏主曹睿知孔明又伐中原,急召
司马懿商议。懿曰:“今子丹已亡,臣愿竭一人之力,剿除寇贼,以报陛下。”睿大喜,设
宴待之。次日,人报蜀兵寇急。睿即命司马懿出师御敌,亲排銮驾送出城外。懿辞了魏主,
径到长安,大会诸路人马,计议破蜀兵之策。张郃曰:“吾愿引一军去守雍、郿,以拒蜀
兵。”懿曰:“吾前军不能独当孔明之众,而又分兵为前后,非胜算也。不如留兵守上邽,
余众悉往祁山。公肯为先锋否?”郃大喜曰:“吾素怀忠义,欲尽心报国,惜未遇知己;今
都督肯委重任,虽万死不辞!”于是司马懿令张郃为先锋,总督大军。又令郭淮守陇西诸
郡,其余众将各分道而进。

前军哨马报说:孔明率大军望祁山进发,前部先锋王平、张嶷,径出陈仓,过剑阁,由
散关望斜谷而来。司马懿谓张郃曰:“今孔明长驱大进,必将割陇西小麦,以资军粮。汝可
结营守祁山,吾与郭淮巡略天水诸郡,以防蜀兵割麦。”郃领诺,遂引四万兵守祁山。懿引
大军望陇西而去。

却说孔明兵至祁山,安营已毕,见渭滨有魏军提备,乃谓诸将曰:“此必是司马懿也。
即今营中乏粮,屡遣人催并李严运米应付,却只是不到。吾料陇上麦熟,可密引兵割之。”
于是留王平、张嶷、吴班、吴懿四将守祁山营,孔明自引姜维、魏延等诸将,前到卤城。卤
城太守素知孔明,慌忙开城出降。孔明抚慰毕,问曰:“此时何处麦熟?”太守告曰:“陇
上麦已熟。”孔明乃留张翼、马忠守卤城,自引诸将并三军望陇上而来。前军回报说:“司
马懿引兵在此。”孔明惊曰:“此人预知吾来割麦也!”即沐浴更衣,推过一般三辆四轮车
来,车上皆要一样妆饰。此车乃孔明在蜀中预先造下的。

当下令姜维引一千军护车,五百军擂鼓,伏在上邽之后;马岱在左,魏延在右,亦各引
一千军护车,五百军擂鼓。每一辆车,用二十四人,皂衣跣足,披发仗剑,手执七星皂旙,
在左右推车。三人各受计,引兵推车而去。孔明又令三万军皆执镰刀、驮绳,伺候割麦。却
选二十四个精壮之士,各穿皂衣,披发跣足,仗剑簇拥四轮车,为推车使者。令关兴结束做
天蓬模样,手执七星皂幡,步行于车前。孔明端坐于上,望魏营而来。哨探军见之大惊,不
知是人是鬼,火速报知司马懿。懿自出营视之,只见孔明簪冠鹤氅,手摇羽扇,端坐于四轮
车上;左右二十四人,披发仗剑;前面一人,手执皂幡,隐隐似天神一般。懿曰:“这个又
是孔明作怪也!”遂拨二千人马分付曰:“汝等疾去,连车带人,尽情都捉来!”魏兵领
命,一齐追赶。孔明见魏兵赶来,便教回车,遥望蜀营缓缓而行。魏兵皆骤马追赶,但见阴
风习习,冷雾漫漫。尽力赶了一程,追之不上。各人大惊,都勒住马言曰:“奇怪!我等急
急赶了三十里,只见在前,追之不上,如之奈何?”孔明见兵不来,又令推车过来,朝着魏
兵歇下。魏兵犹豫良久,又放马赶来。孔明复回车慢慢而行。魏兵又赶了二十里,只见在
前,不曾赶上,尽皆痴呆。孔明教回过车,朝着魏军,推车倒行。魏兵又欲追赶。后面司马
懿自引一军到,传令曰:“孔明善会八门遁甲,能驱六丁六甲之神。此乃六甲天书内缩地之
法也。众军不可追之。”众军方勒马回时,左势下战鼓大震,一彪军杀来。懿急令兵拒之,
只见蜀兵队里二十四人,披发仗剑,皂衣跣足,拥出一辆四轮车;车上端坐孔明,簪冠鹤
氅,手摇羽扇。懿大惊曰:“方才那个车上坐着孔明,赶了五十里,追之不上;如何这里又
有孔明?怪哉!怪哉!”言未毕,右势下战鼓又鸣,一彪军杀来,四轮车上亦坐着一个孔
明,左右亦有二十四人,皂衣跣足,披发仗剑,拥车而来。懿心中大疑,回顾诸将曰:“此
必神兵也!”众军心下大乱,不敢交战,各自奔走。正行之际,忽然鼓声大震,又一彪军杀
来:当先一辆四轮车,孔明端坐于上,左右前后推车使者,同前一般。魏兵无不骇然。

司马懿不知是人是鬼,又不知多少蜀兵,十分惊惧,急急引兵奔入上邽,闭门不出。此
时孔明早令三万精兵将陇上小麦割尽,运赴卤城打晒去了。司马懿在上邽城中,三日不敢出
城。后见蜀兵退去,方敢令军出哨;于路捉得一蜀兵,来见司马懿。懿问之,其人告曰:
“某乃割麦之人,因走失马匹,被捉前来。”懿曰:“前者是何神兵?答曰:“三路伏兵,
皆不是孔明,乃姜维、马岱、魏延也。每一路只有一千军护车,五百军擂鼓。只是先来诱阵
的车上乃孔明也。”懿仰天长叹曰:“孔明有神出鬼没之机!”忽报副都督郭淮入见。懿接
入,礼毕,淮曰:“吾闻蜀兵不多,现在卤城打麦,可以击之。”懿细言前事。淮笑曰:
“只瞒过一时,今已识破,何足道哉!吾引一军攻其后,公引一军攻其前,卤城可破,孔明
可擒类。”懿从之,遂分兵两路而来。

却说孔明引军在卤城打晒小麦,忽唤诸将听今曰:“今夜敌人必来攻城。吾料卤城东西
麦田之内,足可伏兵;谁敢为我一往?”姜维、魏延、马忠、马岱四将出曰:“某等愿
往。”孔明大喜,乃命姜维、魏延各引二千兵,伏在东南、西北两处;马岱、马忠各引二千
兵,伏在西南、东北两处:“只听炮响,四角一齐杀来。”四将受计,引兵去了。孔明自引
百余人,各带火炮出城,伏在麦田之内等候。

却说司马懿引兵径到卤城下,日已昏黑,乃谓诸将曰:“若白日进兵,城中必有准备;
今可乘夜晚攻之。此处城低壕浅,可便打破。”遂屯兵城外。一更时分,郭淮亦引兵到。两
下合兵,一声鼓响,把卤城围得铁桶相似。城上万弩齐发,矢石如雨,魏兵不敢前进。忽然
魏军中信炮连声,三军大惊,又不知何处兵来。淮令人去麦田搜时,四角上火光冲天,喊声
大震,四路蜀兵,一齐杀至;卤城四门大开,城内兵杀出:里应外合,大杀了一阵,魏兵死
者无数。司马懿引败兵奋死突出重围,占住了山头;郭淮亦引败兵奔到山后扎住。孔明入
城,令四将于四角下安营。

郭淮告司马懿曰:“今与蜀兵相持许久,无策可退;目下又被杀了一阵,折伤三千余
人;若不早图,日后难退矣。”懿曰:“当复如何?”淮曰:“可发檄文调雍、凉人马并力
剿杀。吾愿引军袭剑阁,截其归路,使彼粮草不通,三军慌乱:那时乘势击之,敌可灭
矣。”懿从之,即发檄文星夜往雍、凉调拨人马,不一日,大将孙礼引雍、凉诸郡人马到。
懿即令孙礼约会郭淮去袭剑阁。却说孔明在卤城相拒日久,不见魏兵出战,乃唤姜维、马岱
入城听令曰:“今魏兵守住山险,不与我战:一者料吾麦尽无粮;二者令兵去袭剑阁,断吾
粮道也。汝二人各引一万军先去守住险要,魏兵见有准备,自然退去。”二人引兵去了。

长史杨仪入帐告曰:“向者丞相令大兵一百日一换,今已限足,汉中兵已出川口,前路
公文已到,只待会兵交换:现存八万军,内四万该与换班。”孔明曰:“既有令,便教速
行。”众军闻知,各各收拾起程。忽报孙礼引雍、凉人马二十万来助战,去袭剑阁,司马懿
自引兵来攻卤城了。蜀兵无不惊骇。

杨仪入告孔明曰:“魏兵来得甚急,丞相可将换班军且留下退敌,待新来兵到,然后换
之。”孔明曰:“不可。吾用兵命将,以信为本;既有令在先,岂可失信?且蜀兵应去者,
皆准备归计,其父母妻子倚扉而望;吾今便有大难,决不留他。”即传令教应去之兵,当日
便行。众军闻之,皆大呼曰:“丞相如此施恩于众,我等愿且不回,各舍一命,大杀魏兵,
以报丞相!”孔明曰:“尔等该还家,岂可复留于此?”众军皆要出战,不愿回家。孔明
曰:“汝等既要与我出战,可出城安营,待魏兵到,莫待他息喘,便急攻之:此以逸待劳之
法也。”众兵领命,各执兵器,欢喜出城,列阵而待。却说西凉人马倍道而来,走的人马困
乏;方欲下营歇息,被蜀兵一拥而进,人人奋勇,将锐兵骁,雍、凉兵抵敌不住,望后便
退。蜀兵奋力追杀,杀得那雍、凉兵尸横遍野,血流成渠。孔明出城,收聚得胜之兵,入城
赏劳。忽报永安李严有书告急。孔明大惊,拆封视之。书云:“近闻东吴令人入洛阳,与魏
连和;魏令吴取蜀,幸吴尚未起兵。今严探知消息,伏望丞相,早作良图。”孔明览毕,甚
是惊疑,乃聚诸将曰:“若东吴兴兵寇蜀,吾须索速回也。”即传令,教祁山大寨人马,且
退回西川:“司马懿知吾屯军在此,必不敢追赶。”于是王平、张嶷、吴班、吴懿,分兵两
骆,徐徐退入西川去了。张郃见蜀兵退去,恐有计策,不敢来追,乃引兵往见司马懿曰:
“今蜀兵退去,不知何意?”懿曰:“孔明诡计极多,不可轻动。不如坚守,待他粮尽,自
然退去。”大将魏平出曰:“蜀兵拔祁山之营而退,正可乘势追之,都督按兵不动,畏蜀如
虎,奈天下笑何?”懿坚执不从。

却说孔明知祁山兵已回,遂令杨仪、马忠入帐,授以密计,令先引一万弓弩手,去剑阁
木门道,两下埋伏;若魏兵追到,听吾炮响,急滚下木石,先截其去路,两头一齐射之。二
人引兵去了。又唤魏延、关兴引兵断后,城上四面遍插旌旗,城内乱堆柴草,虚放烟火。大
兵尽望木门道而去。

魏营巡哨军来报司马懿曰:“蜀兵大队已退,但不知城中还有多少兵。”懿自往视之,
见城上插旗,城中烟起,笑曰:“此乃空城也。”令人探之,果是空城,懿大喜曰:“孔明
已退,谁敢追之?”先锋张郃曰:“吾愿往。”懿阻曰:“公性急躁,不可去。”郃曰:“都
督出关之时,命吾为先锋;今日正是立功之际,却不用吾,何也?”懿曰:“蜀兵退去,险
阻处必有埋伏,须十分仔细,方可追之。”郃曰:“吾已知得,不必挂虑。”懿曰:“公自
欲去,莫要追悔。”郃曰:“大丈夫舍身报国,虽万死无恨。”懿曰:“公既坚执要去,可
引五千兵先行;却教魏平引二万马步兵后行,以防埋伏。吾却引三千兵随后策应。”

张郃领命,引兵火速望前追赶。行到三十余里,忽然背后一声喊起,树林内闪出一彪
军,为首大将,横刀勒马大叫曰:“贼将引兵那里去!”郃回头视之,乃魏延也。郃大怒,回
马交锋。不十合,延诈败而走。郃又追赶三十余里,勒马回顾,全无伏兵,又策马前追。方
转过山坡,忽喊声大起,一彪军闪出,为首大将,乃关兴也,横刀勒马大叫曰:“张郃休
赶!有吾在此!”郃就拍马交锋。不十合,兴拨马便走。郃随后追之。赶到一密林内,郃心
疑,令人四下哨探,并无伏兵;于是放心又赶。不想魏延却抄在前面;郃又与战十余合,延
又败走。郃奋怒追来,又被关兴抄在前面,截住去路。郃大怒,拍马交锋,战有十合,蜀兵尽
弃衣甲什物等件,塞满道路,魏军皆下马争取。延、兴二将,轮流交战,张郃奋勇追赶。看
看天晚,赶到木门道口,魏延拨回马,高声大骂曰:“张郃逆贼!吾不与汝相拒,汝只顾赶
来,吾今与汝决一死战!”郃十分忿怒,挺枪骤马,直取魏延。延挥刀来迎。战不十合,延
大败,尽弃衣甲、头盔,匹马引败兵望木门道中而走。张郃杀得性起,又见魏延大败而逃,
乃骤马赶来。此时天色昏黑,一声炮响,山上火光冲天,大石乱柴滚将下来,阻截去路。郃
大惊曰:“我中计矣!”急回马时,背后已被木石塞满了归路,中间只有一段空地,两边皆
是峭壁,郃进退无路。忽一声梆子响,两下万弩齐发,将张郃并百余个部将,皆射死于木门道
中。后人有诗曰:“伏弩齐飞万点星,木门道上射雄兵。至今剑阁行人过,犹说军师旧日
名。”

却说张郃已死,随后魏兵追到,见塞了道路,已知张郃中计。众军勒回马急退。忽听得山
头上大叫曰:“诸葛丞相在此!”众军仰视,只见孔明立于火光之中,指众军而言曰:“吾
今日围猎,欲射一马,误中一獐。汝各人安心而去;上覆仲达:早晚必为吾所擒矣。”魏兵
回见司马懿,细告前事。懿悲伤不已,仰天叹曰:“张隽乂身死,吾之过也!”乃收兵回洛
阳。魏主闻张郃死,挥泪叹息,令人收其尸,厚葬之。

却说孔明入汉中,欲归成都见后主。都护李严妄奏后主曰:“臣已办备军粮,行将运赴
丞相军前,不知丞相何故忽然班师。”后主闻奏,即命尚书费祎入汉中见孔明,问班师之
故。祎至汉中,宣后主之意。孔明大惊曰:“李严发书告急,说东吴将兴兵寇川,因此回
师。”费祎曰:“李严奏称军粮已办,丞相无故回师,天子因此命某来问耳。”孔明大怒,
令人访察:乃是李严因军粮不济,怕丞相见罪,故发书取回,却又妄奏天子,遮饰己过。孔
明大怒曰:“匹夫为一己之故,废国家大事!”令人召至,欲斩之。费祎劝曰:“丞相念先
帝托孤之意,姑且宽恕。”孔明从之。费祎即具表启奏后主。后主览表,勃然大怒,叱武士
推李严出斩之。参军蒋琬出班奏曰:“李严乃先帝托孤之臣,乞望恩宽恕。”后主从之,即
谪为庶人,徙于梓潼郡闲住。孔明回到成都,用李严子李丰为长史;积草屯粮,讲阵论武,
整治军器,存恤将士:三年然后出征。两川人民军士,皆仰其恩德。光阴茬苒,不觉三年:
时建兴十二年春二月。孔明入朝奏曰:“臣今存恤军士,已经三年。粮草丰足,军器完备,
人马雄壮,可以伐魏。今番若不扫清奸党,恢复中原,誓不见陛下也!”后主曰:“方今已
成鼎足之势,吴、魏不曾入寇,相父何不安享太平?”孔明曰:“臣受先帝知遇之恩,梦寐
之间,未尝不设伐魏之策。竭力尽忠,为陛下克复中原,重兴汉室:臣之愿也。”言未毕,
班部中一人出曰:“丞相不可兴兵。”众视之,乃谯周也。正是:武侯尽瘁惟忧国,太史知
机又论天。未知谯周有何议论,且看下文分解。