\chapter{战猇亭先主得仇人~守江口书生拜大将}

却说章武二年春正月,武威后将军黄忠随先主伐吴;忽闻先主言老将无用,即提刀上
马,引亲随五六人,径到彝陵营中。吴班与张南、冯习接入,问曰:“老将军此来,有何事
故?”忠曰:“吾自长沙跟天子到今,多负勤劳。今虽七旬有余,尚食肉十斤,臂开二石之
弓,能乘千里之马,未足为老。昨日主上言吾等老迈无用,故来此与东吴交锋,看吾斩将,
老也不老!”正言间,忽报吴兵前部已到,哨马临营。忠奋然而起,出帐上马。冯习等劝
曰:“老将军且休轻进。”忠不听,纵马而去。吴班令冯习引兵助战。忠在吴军阵前,勒马
横刀,单搦先锋潘璋交战。璋引部将史迹出马。迹欺忠年老,挺枪出战;斗不三合,被忠一
刀斩于马下。潘璋大怒,挥关公使的青龙刀,来战黄忠。交马数合,不分胜负。忠奋力恶
战,璋料敌不过,拨马便走。忠乘势追杀,全胜而回。路逢关兴、张苞。兴曰:“我等奉圣
旨来助老将军;既已立了功,速请回营。”忠不听。次日,潘璋又来搦战。黄忠奋然上马。
兴、苞二人要助战,忠不从;吴班要助战,忠亦不从;只自引五千军出迎。战不数合,璋拖
刀便走。忠纵马追之,厉声大叫曰:“贼将休走!吾今为关公报仇!”追至三十余里,四面
喊声大震,伏兵齐出:右边周泰,左边韩当,前有潘璋,后有凌统,把黄忠困在垓心。忽然
狂风大起,忠急退时,山坡上马忠引一军出,一箭射中黄忠肩窝,险些儿落马。吴兵见忠中
箭,一齐来攻,忽后面喊声大起,两路军杀来,吴兵溃散,救出黄忠,乃关兴、张苞也。二
小将保送黄忠径到御前营中。忠年老血衰,箭疮痛裂,病甚沉重。先主御驾自来看视,抚其
背曰:“令老将军中伤,朕之过也!”忠曰:“臣乃一武夫耳,幸遇陛下。臣今年七十有
五,寿亦足矣。望陛下善保龙体,以图中原!”言讫,不省人事。是夜殒于御营。后人有诗
叹曰:“老将说黄忠,收川立大功。重披金锁甲,双挽铁胎弓。胆气惊河北,威名镇蜀中。
临亡头似雪,犹自显英雄。”

先主见黄忠气绝,哀伤不已,敕具棺椁,葬于成都。先主叹曰:“五虎大将,已亡三
人。朕尚不能复仇,深可痛哉!”乃引御林军直至猇亭,大会诸将,分军八路,水陆俱进。
水路令黄权领兵,先主自率大军于旱路进发。时章武二年二月中旬也。韩当、周泰听知先主
御驾来征,引兵出迎。两阵对圆,韩当、周泰出马,只见蜀营门旗开处,先主自出,黄罗销
金伞盖,左右白旌黄钺,金银旌节,前后围绕。当大叫曰:“陛下今为蜀主,何自轻出?倘
有疏虞,悔之何及!”先主遥指骂曰:“汝等吴狗,伤朕手足,誓不与立于天地之间!”当
回顾众将曰:“谁敢冲突蜀兵?”部将夏恂,挺枪出马。先主背后张苞挺丈八矛,纵马而
出,大喝一声,直取夏恂。恂见苞声若巨雷,心中惊惧;恰待要走,周泰弟周平见恂抵敌不
住,挥刀纵马而来。关兴见了,跃马提刀来迎。张苞大喝一声,一矛刺中夏恂,倒撞下马。
周平大惊,措手不及,被关兴一刀斩了。二小将便取韩当、周泰。韩、周二人,慌退入阵。
先主视之,叹曰:“虎父无犬子也!”用御鞭一指,蜀兵一齐掩杀过去,吴兵大败。那八路
兵,势如泉涌,杀的那吴军尸横遍野,血流成河。却说甘宁正在船中养病,听知蜀兵大至,
火急上马,正遇一彪蛮兵,人皆被发跣足,皆使弓弩长枪,搪牌刀斧;为首乃是番王沙摩
柯,生得面如噀血,碧眼突出,使一个铁蒺藜骨朵,腰带两张弓,威风抖擞。甘宁见其势
大,不敢交锋,拨马而走;被沙摩柯一箭射中头颅。宁带箭而走,到于富池口,坐于大树之
下而死。树上群鸦数百,围绕其尸。吴王闻之,哀痛不已,具礼厚葬,立庙祭祀。后人有诗
叹曰:“吴郡甘兴霸,长江锦幔舟。酬君重知已,报友化仇雠。劫寨将轻骑,驱兵饮巨瓯。
神鸦能显圣,香火永千秋。”

却说先主乘势追杀,遂得猇亭。吴兵四散逃走。先主收兵,只不见关兴。先主慌令张苞
等四面跟寻。原来关兴杀入吴阵,正遇仇人潘璋,骤马追之。璋大惊,奔入山谷内,不知所
往。兴寻思只在山里,往来寻觅不见。看看天晚,迷踪失路。幸得星月有光,追至山僻之
间,时已二更,到一庄上,下马叩门。一老者出问何人。兴曰:“吾是战将,迷路到此,求
一饭充饥。”老人引入,兴见堂内点着明烛,中堂绘画关公神像。兴大哭而拜。老人问曰:
“将军何故哭拜?”兴曰:“此吾父也。”老人闻言,即便下拜。兴曰:“何故供养吾
父?”老人答曰:“此间皆是尊神地方。在生之日,家家侍奉,何况今日为神乎?老夫只望
蜀兵早早报仇。今将军到此,百姓有福矣。”遂置酒食待之,卸鞍喂马。

三更已后,忽门外又一人击户。老人出而问之,乃吴将潘璋亦来投宿。恰入草堂,关兴
见了,按剑大喝曰:“歹贼休走!”璋回身便出。忽门外一人,面如重枣,丹凤眼,卧蚕
眉,飘三缕美髯,绿袍金铠,按剑而入。璋见是关公显圣,大叫一声,神魂惊散;欲待转
身,早被关兴手起剑落,斩于地上,取心沥血,就关公神像前祭祀。兴得了父亲的青龙偃月
刀,却将潘璋首级,擐于马项之下,辞了老人,就骑了潘璋的马,望本营而来。老人自将潘
璋之尸拖出烧化。

且说关兴行无数里,忽听得人言马嘶,一彪军来到;为首一将,乃潘璋部将马忠也。忠
见兴杀了主将潘璋,将首级擐于马项之下,青龙刀又被兴得了,勃然大怒,纵马来取关兴。
兴见马忠是害父仇人,气冲牛斗,举青龙刀望忠便砍。忠部下三百军并力上前,一声喊起,
将关兴围在垓心。兴力孤势危。忽见西北上一彪军杀来,乃是张苞。马忠见救兵到来,慌忙
引军自退。关兴、张苞一处赶来。赶不数里,前面糜芳、傅士仁引兵来寻马忠。两军相合,
混战一处。苞、兴二人兵少,慌忙撤退,回至猇亭,来见先主,献上首级,具言此事。先主
惊异,赏犒三军。却说马忠回见韩当、周泰,收聚败军,各分头守把。军士中伤者不计其
数。马忠引傅士仁、糜芳于江渚屯扎。当夜三更,军士皆哭声不止。糜芳暗听之,有一夥军
言曰:“我等皆是荆州之兵,被吕蒙诡计送了主公性命,今刘皇叔御驾亲征,东吴早晚休
矣。所恨者,糜芳、傅士仁也。我等何不杀此二贼,去蜀营投降?功劳不小。”又一夥军言
曰:“不要性急,等个空儿,便就下手。”

糜芳听毕,大惊,遂与傅士仁商议曰:“军心变动,我二人性命难保。今蜀主所恨者马
忠耳;何不杀了他,将首级去献蜀主,告称:我等不得已而降吴,今知御驾前来,特地诣营
请罪。”仁曰:“不可。去必有祸。”芳曰:“蜀主宽仁厚德:目今阿斗太子是我外甥,彼
但念我国戚之情,必不肯加害。”二人计较已定,先备了马。三更时分,入帐刺杀马忠,将
首级割了,二人带数十骑,径投猇亭而来。伏路军人先引见张南、冯习,具说其事。次日,
到御营中来见先主,献上马忠首级,哭告于前曰:“臣等实无反心;被吕蒙诡计,称言关公
已亡,赚开城门,臣等不得已而降。今闻圣驾前来,特杀此贼。以雪陛下之恨。伏乞陛下恕
臣等之罪。”先主大怒曰:“朕自离成都许多时,你两个如何不来请罪?今日势危,故来巧
言,欲全性命!朕若饶你,至九泉之下,有何面目见关公乎!”言讫,令关兴在御营中,设
关公灵位。先主亲捧马忠首级,诣前祭祀。又令关兴将糜芳、傅士仁剥去衣服,跪于灵前,
亲自用刀剐之,以祭关公。忽张苞上帐哭拜于前曰:“二伯父仇人皆已诛戮;臣父冤仇,何
日可报?”先主曰:“贤侄勿忧。朕当削平江南,杀尽吴狗,务擒二贼,与汝亲自醢之,以
祭汝父。“苞泣谢而退。

此时先主威声大震,江南之人尽皆胆裂,日夜号哭。韩当、周泰大惊,急奏吴王,具言
糜芳、傅士仁杀了马忠,去归蜀帝,亦被蜀帝杀了。孙权心怯,遂聚文武商议。步骘奏曰:
“蜀主所恨者,乃吕蒙、潘璋、马忠、糜芳、傅士仁也。今此数人皆亡,独有范疆、张达二
人,现在东吴。何不擒此二人,并张飞首级,遣使送还,交与荆州,送归夫人,上表求和,
再会前情,共图灭魏,则蜀兵自退矣。”权从其言,遂具沉香木匣,盛贮飞首,绑缚范疆、
张达,囚于槛车之内,令程秉为使,赍国书,望猇亭而来。

却说先主欲发兵前进。忽近臣奏曰:“东吴遣使送张车骑之首,并囚范疆、张达二贼
至。”先主两手加额曰:“此天之所赐,亦由三弟之灵也!“即令张苞设飞灵位。先主见张
飞首级在匣中面不改色,放声大哭。张苞自仗利刀,将范疆、张达万剐凌迟,祭父之灵。祭
毕,先主怒气不息,定要灭吴。马良奏曰:“仇人尽戳,其恨可雪矣。吴大夫程秉到此,欲
还荆州,送回夫人,永结盟好,共图灭魏,伏候圣旨。”先主怒曰:“朕切齿仇人,乃孙权
也。今若与之连和,是负二弟当日之盟矣。今先灭吴,次灭魏。”便欲斩来使,以绝吴情。
多官苦告方免。程秉抱头鼠窜,回奏吴主曰:“蜀不从讲和,誓欲先灭东吴,然后伐魏。众
臣苦谏不听,如之奈何?“

权大惊,举止失措。阚泽出班奏曰:“现有擎天之柱,如何不用耶?”权急问何人。泽
曰:“昔日东吴大事,全任周郎;后鲁子敬代之;子敬亡后,决于吕子明;今子明虽丧,现
有陆伯言在荆州。此人名虽儒生,实有雄才,大略,以臣论之,不在周郎之下;前破关公,
其谋皆出于伯言。主上若能用之,破蜀必矣。如或有失,臣愿与同罪。”权曰:“非德润之
言,孤几误大事。”张昭曰:“陆逊乃一书生耳,非刘备敌手;恐不可用。”顾雍亦曰:
“陆逊年幼望轻,恐诸公不服;若不服则生祸乱,必误大事。”来骘亦曰:“逊才堪治郡
耳;若托以大事,非其宜也。”阚泽大呼曰:“若不用陆伯言,则东吴休矣!臣愿以全家保
之!”权曰:“孤亦素知陆伯言乃奇才也!孤意已决,卿等勿言。”于是命召陆逊。逊本名
陆议,后改名逊,字伯言,乃吴郡吴人也;汉城门校尉陆纡之孙,九江都尉陆骏之子;身长
八尺,面如美玉;官领镇西将军。当下奉召而至,参拜毕,权曰:“今蜀兵临境,孤特命卿
总督军马,以破刘备。”逊曰:“江东文武,皆大王故旧之臣;臣年幼无才,安能制之?”
权曰:“阚德润以全家保卿,孤亦素知卿才。今拜卿为大都督,卿勿推辞。”逊曰:“倘文
武不服,何如?”权取所佩剑与之曰:“如有不听号令者,先斩后奏。”逊曰:“荷蒙重
托,敢不拜命;但乞大王于来日会聚众官,然后赐臣。”阚泽曰:“古之命将,必筑坛会
众,赐白旄黄钺、印绶兵符,然后威行令肃。今大王宜遵此礼,择日筑坛,拜伯言为大都
督,假节钺,则众人自无不服矣。”权从之,命人连夜筑坛完备,大会百官,请陆逊登坛,
拜为大都督、右护军镇西将军,进封娄候,赐以宝剑印绶,令掌六郡八十一州兼荆楚诸路军
马。吴王嘱之曰:“阃以内,孤主之;阃以外,将军制之。”

逊领命下坛,令徐盛、丁奉为护卫,即日出师;一面调诸路军马,水陆并进。文书到猇
亭,韩当、周泰大惊曰:“主上如何以一书生总兵耶?”比及逊至,众皆不服。逊升帐议
事,众人勉强参贺。逊曰:“主上命吾为大将,督军破蜀。军有常法,公等各宜遵守。违者
王法无亲,勿致后悔。”众皆默然。周泰曰:“目今安东将军孙桓,乃主上之侄,现困于彝
陵城中,内无粮草,外无救兵;请都督早施良策,救出孙桓,以安主上之心。”逊曰:“吾
素知孙安东深得军心,必能坚守,不必救之。待吾破蜀后,彼自出矣。”众皆暗笑而退。韩
当谓周泰曰:“命此孺子为将,东吴休矣!公见彼所行乎?”泰曰:“吾聊以言试之,早无
一计,安能破蜀也!”

次日,陆逊传下号令,教诸将各处关防,牢守隘口,不许轻敌。众皆笑其懦,不肯坚
守。次日,陆逊升帐唤诸将曰:“吾钦承王命,总督诸军,昨已三令五申,令汝等各处坚
守;俱不遵吾令,何也?”韩当曰:“吾自从孙将军平定江南,经数百战;其余诸将,或从
讨逆将军,或从当今大王,皆披坚执锐,出生入死之士。今主上命公为大都督,令退蜀兵,
宜早定计,调拨军马,分头征进,以图大事;乃只令坚守勿战,岂欲待天自杀贼耶?吾非贪
生怕死之人,奈何使吾等堕其锐气?”于是帐下诸将,皆应声而言曰:“韩将军之言是也。
吾等情愿决一死战!”陆逊听毕,掣剑在手,厉声曰:“仆虽一介书生,今蒙主上托以重任
者,以吾有尺寸可取,能忍辱负重故也。汝等只各守隘口,牢把险要,不许妄动,如违令者
皆斩!”众皆愤愤而退。却说先主自猇亭布列军马,直至川口,接连七百里,前后四十营
寨,昼则旌旗蔽日,夜则火光耀天。忽细作报说:“东吴用陆逊为大都督,总制军马。逊令
诸将各守险要不出。”先主问曰:“陆逊何如人也?’马良奏曰:“逊虽东吴一书生,然年
幼多才,深有谋略;前袭荆州,皆系此人之诡计。”先主大怒曰:“竖子诡计,损朕二弟,
今当擒之!”便传令进兵。马良谏曰:“陆逊之才,不亚周郎,未可轻敌。”先主曰:“朕
用兵老矣,岂反不如一黄口孺子耶!”遂亲领前军,攻打诸处关津隘口。韩当见先主兵来,
差人投知陆逊。逊恐韩当妄动,急飞马自来观看,正见韩当立马于山上;远望蜀兵漫山遍野
而来,军中隐隐有黄罗盖伞。韩当接着陆逊,并马而观。当指曰:“军中必有刘备,吾欲击
之。”逊曰:“刘备举兵东下,连胜十余阵,锐气正盛;今只乘高守险,不可轻出,出则不
利。但宜奖励将士,广布守御之策,以观其变。今彼驰骋于平原广野之间,正自得志;我坚
守不出,彼求战不得,必移屯于山林树木间。吾当以奇计胜之。”

韩当口虽应诺,心中只是不服,先主使前队搦战,辱骂百端。逊令塞耳休听,不许出
迎,亲自遍历诸关隘口,抚慰将士,皆令坚守。先主见吴军不出,心中焦躁。马良曰:“陆
逊深有谋略。今陛下远来攻战,自春历夏;彼之不出,欲待我军之变也。愿陛下察之。”先
主曰:“彼有何谋?但怯敌耳。向者数败,今安敢再出!”先锋冯习奏曰:“即今天气炎
热,军屯于赤火之中,取水深为不便。”先主遂命各营,皆移于山林茂盛之地,近溪傍涧;
待过夏到秋,并力进兵。冯习遂奉旨,将诸寨皆移于林木阴密之处。马良奏曰:“我军若
动,倘吴兵骤至,如之奈何?”先主曰:“朕令吴班引万余弱兵,近吴寨平地屯住;朕亲选
八千精兵,伏于山谷之中。若陆逊知朕移营,必乘势来击,却令吴班诈败;逊若追来,朕引
兵突出,断其归路,小子可擒矣。”文武皆贺曰:“陛下神机妙算,诸臣不及也!”马良
曰:“近闻诸葛丞相在东川点看各处隘口,恐魏兵入寇。陛下何不将各营移居之地,画成图
本,问于丞相?”先主曰:“朕亦颇知兵法,何必又问丞相?”良曰:“古云兼听则明,偏
听则蔽。望陛下察之。”先主曰:“卿可自去各营,画成四至八道图本,亲到东川去向丞
相。如有不便,可急来报知。”马良领命而去。于是先主移兵于林木阴密处避暑。早有细作
报知韩当、周泰。二人听得此事,大喜,来见陆逊曰:“目今蜀兵四十余营,皆移于山林密
处,依溪傍涧,就水歇凉。都督可乘虚击之。”正是:蜀主有谋能设伏,吴兵好勇定遭擒。
未知陆逊可听其言否,且看下文分解。