\chapter{诸葛亮舌战群儒~鲁子敬力排众议}

却说鲁肃、孔明辞了玄德、刘琦,登舟望柴桑郡来。二人在舟中共议、鲁肃谓孔明曰:
“先生见孙将军,切不可实言曹操兵多将广。”孔明曰:“不须子敬叮咛,亮自有对答之
语。”及船到岸,肃请孔明于馆驿中暂歇,先自往见孙权。权正聚文武于堂上议事,闻鲁肃
回,急召入问曰:“子敬往江夏,体探虚实若何?”肃曰:“已知其略,尚容徐禀。”权将
曹操檄文示肃曰:“操昨遣使赍文至此,孤先发遣来使,现今会众商议未定。”肃接檄文观
看。其略曰:“孤近承帝命,奉词伐罪。旄麾南指,刘琮束手;荆襄之民,望风归顺。今统
雄兵百万,上将千员,欲与将军会猎于江夏,共伐刘备,同分土地,永结盟好。幸勿观望,
速赐回音。”鲁肃看毕曰:“主公尊意若何?”权曰:“未有定论。”张昭曰:“曹操拥百
万之众,借天子之名,以征四方,拒之不顺。且主公大势可以拒操者,长江也。今操既得荆
州,长江之险,已与我共之矣,势不可敌。以愚之计,不如纳降,为万安之策。众谋士皆
曰:“子布之言,正合天意。”孙权沉吟不语。张昭又曰:“主公不必多疑。如降操,则东
吴民安,江南六郡可保矣。”孙权低头不语。

须臾,权起更衣,鲁肃随于权后。权知肃意,乃执肃手而言曰:“卿欲如何?”肃曰:
“恰才众人所言,深误将军。众人皆可降曹操,惟将军不可降曹操。”权曰:“何以言
之?”肃曰:“如肃等降操,当以肃还乡党,累官故不失州郡也;将军降操,欲安所归乎?
位不过封侯,车不过一乘,骑不过一匹,从不过数人,岂得南面称孤哉!众人之意,各自为
己,不可听也。将军宜早定大计。”权叹曰:“诸人议论,大失孤望。子敬开说大计,正与
吾见相同。此天以子敬赐我也!但操新得袁绍之众,近又得荆州之兵,恐势大难以抵敌。”
肃曰:“肃至江夏,引诸葛瑾之弟诸葛亮在此,主公可问之,便知虚实。”权曰:“卧龙先
生在此乎?”肃曰:“现在馆驿中安歇。”权曰:“今日天晚,且未相见。来日聚文武于帐
下,先教见我江东英俊,然后升堂议事。”肃领命而去。次日至馆驿中见孔明,又嘱曰:
“今见我主,切不可言曹操兵多。”孔明笑曰:“亮自见机而变,决不有误。”肃乃引孔明
至幕下。早见张昭、顾雍等一班文武二十余人,峨冠博带,整衣端坐。孔明逐一相见,各问
姓名。施礼已毕,坐于客位。张昭等见孔明丰神飘洒,器宇轩昂,料道此人必来游说。张昭
先以言挑之曰:“昭乃江东微末之士,久闻先生高卧隆中,自比管;乐。此语果有之乎?”
孔明曰:“此亮平生小可之比也。”昭曰:“近闻刘豫州三顾先生于草庐之中,幸得先生,
以为如鱼得水,思欲席卷荆襄。今一旦以属曹操,未审是何主见?”孔明自思张昭乃孙权手
下第一个谋士,若不先难倒他,如何说得孙权,遂答曰:“吾观取汉上之地,易如反掌。我
主刘豫州躬行仁义,不忍夺同宗之基业,故力辞之。刘琮孺子,听信佞言,暗自投降,致使
曹操得以猖獗。今我主屯兵江夏,别有良图,非等闲可知也。”昭曰:“若此,是先生言行
相违也。先生自比管、乐,管仲相桓公,霸诸侯,一国天下;乐毅扶持微弱之燕,下齐七十
余城:此二人者,真济世之才也。先生在草庐之中,但笑傲风月,抱膝危坐。今既从事刘豫
州,当为生灵兴利除害,剿灭乱贼。且刘豫州未得先生之前,尚且纵横寰宇,割据城池;今
得先生,人皆仰望。虽三尺童蒙,亦谓彪虎生翼,将见汉室复兴,曹氏即灭矣。朝廷旧臣,
山林隐士,无不拭目而待:以为拂高天之云翳,仰日月之光辉,拯民于水火之中,措天下于
衽席之上,在此时也。何先生自归豫州,曹兵一出,弃甲抛戈,望风而窜;上不能报刘表以
安庶民,下不能辅孤子而据疆土;乃弃新野,走樊城,败当阳,奔夏口,无容身之地:是豫
州既得先生之后,反不如其初也。管仲、乐毅,果如是乎?愚直之言,幸勿见怪!”孔明听
罢,哑然而笑曰:“鹏飞万里,其志岂群鸟能识哉?譬如人染沉疴,当先用糜粥以饮之,和
药以服之;待其腑脏调和,形体渐安,然后用肉食以补之,猛药以治之:则病根尽去,人得
全生也。若不待气脉和缓,便投以猛药厚味,欲求安保,诚为难矣。吾主刘豫州,向日军败
于汝南,寄迹刘表,兵不满千,将止关、张、赵云而已:此正如病势尪赢已极之时也,新野
山僻小县,人民稀少,粮食鲜薄,豫州不过暂借以容身,岂真将坐守于此耶?夫以甲兵不
完,城郭不固,军不经练,粮不继日,然而博望烧屯,白河用水,使夏侯惇,曹仁辈心惊胆
裂:窃谓管仲、乐毅之用兵,未必过此。至于刘琮降操,豫州实出不知;且又不忍乘乱夺同
宗之基业,此真大仁大义也。当阳之败,豫州见有数十万赴义之民,扶老携幼相随,不忍弃
之,日行十里,不思进取江陵,甘与同败,此亦大仁大义也。寡不敌众,胜负乃其常事。昔
高皇数败于项羽,而垓下一战成功,此非韩信之良谋乎?夫信久事高皇,未尝累胜。盖国家
大计,社稷安危,是有主谋。非比夸辩之徒,虚誉欺人:坐议立谈,无人可及;临机应变,
百无一能。诚为天下笑耳!”这一篇言语,说得张昭并无一言回答。

座上忽一人抗声问曰:“今曹公兵屯百万,将列千员,龙骧虎视,平吞江夏,公以为何
如?”孔明视之,乃虞翻也。孔明曰:“曹操收袁绍蚁聚之穷于夏口,区区求教于人,而犹
言不惧,此真大言欺人也!”孔明曰:“刘豫州以数千仁义之师,安能敌百万残暴之众?退
守夏口,所以待时也。今江东兵精粮足,且有长江之险,犹欲使其主屈膝降贼,不顾天下耻
笑。由此论之,刘豫州真不惧操贼者矣!”虞翻不能对。

座间又一人问曰:“孔明欲效仪、秦之舌,游说东吴耶?”孔明视之,乃步骘也。孔明
曰:“步子山以苏秦张仪为辩士,不知苏秦、张仪亦豪杰也。苏秦佩六国相印,张仪两次相
秦,皆有匡扶人国之谋,非比畏强凌弱,惧刀避剑之人也。君等闻曹操虚发诈伪之词,便畏
惧请降,敢笑苏秦、张仪乎?”步骘默然无语。忽一人问曰:“孔明以曹操何如人也?”孔
明视其人,乃薛综也。孔明答曰:“曹操乃汉贼也,又何必问?”综曰:“公言差矣。汉传
世至今,天数将终。今曹公已有天下三分之二,人皆归心。刘豫州不识天时,强欲与争,正
如以卵击石,安得不败乎?”孔明厉声曰:“薛敬文安得出此无父无君之言乎!夫人生天地
间,以忠孝为立身之本。公既为汉臣,则见有不臣之人,当誓共戮之:臣之道也。今曹操祖
宗叨食汉禄,不思报效,反怀篡逆之心,天下之所共愤;公乃以天数归之,真无父无君之人
也!不足与语!请勿复言!”薛综满面羞惭,不能对答。座上又一人应声问曰:“曹操虽挟
天子以令诸侯,犹是相国曹参之后。刘豫州虽云中山靖王苗裔,却无可稽考,眼见只是织席
贩屦之夫耳,何足与曹操抗衡哉!”孔明视之,乃陆绩也。孔明笑曰:“公非袁术座间怀桔
之陆郎乎?请安坐,听吾一言:曹操既为曹相国之后,则世为汉臣矣;今乃专权肆横,欺凌
君父,是不惟无君,亦且蔑祖,不惟汉室之乱臣,亦曹氏之贼子也。刘豫州堂堂帝胄,当今
皇帝,按谱赐爵,何云无可稽考?且高祖起身亭长,而终有天下;织席贩屦,又何足为辱
乎?公小儿之见,不足与高士共语!”陆绩语塞。

座上一人忽曰:“孔明所言,皆强词夺理,均非正论,不必再言。且请问孔明治何经
典?”孔明视之,乃严酸也。孔明曰:“寻章摘句,世之腐儒也,何能兴邦立事?且古耕莘
伊尹,钓渭子牙,张良、陈平之流。邓禹、耿弇之辈,皆有匡扶宇宙之才,未审其生平治何
经典。岂亦效书生,区区于笔砚之间,数黑论黄,舞文弄墨而已乎?”严峻低头丧气而不能
对。

忽又一人大声曰:“公好为大言,未必真有实学,恐适为儒者所笑耳。”孔明视其人,
乃汝南程德枢也。孔明答曰:“儒有君子小人之别。君子之儒,忠君爱国,守正恶邪,务使
泽及当时,名留后世。若夫小人之儒,惟务雕虫,专工翰墨,青春作赋,皓首穷经;笔下虽
有千言,胸中实无一策。且如杨雄以文章名世,而屈身事莽,不免投阁而死,此所谓小人之
儒也;虽日赋万言,亦何取哉!”程德枢不能对。众人见孔明对答如流,尽皆失色。时座上
张温、骆统二人,又欲问难。忽一人自外而入,厉声言曰:“孔明乃当世奇才,君等以唇舌
相难,非敬客之礼也。曹操大军临境,不思退敌之策,乃徒斗口耶!”众视其人,乃零陵
人,姓黄,名盖,字公覆,现为东吴粮官。当时黄盖谓孔明曰:“愚闻多言获利,不如默而
无言。何不将金石之论为我主言之,乃与众人辩论也?”孔明曰:“诸君不知世务,互相问
难,不容不答耳。”于是黄盖与鲁肃引孔明入。至中门,正遇诸葛瑾,孔明施礼。瑾曰:
“贤弟既到江东,如何不来见我?”孔明曰:“弟既事刘豫州,理宜先公后私。公事未毕,
不敢及私。望兄见谅。”瑾曰:“贤弟见过吴侯,却来叙话。”说罢自去。鲁肃曰:“适间
所嘱,不可有误。”孔明点头应诺。引至堂上,孙权降阶而迎,优礼相待。施礼毕,赐孔明
坐。众文武分两行而立。鲁肃立于孔明之侧,只看他讲话。孔明致玄德之意毕,偷眼看孙
权:碧眼紫髯,堂堂一表。孔明暗思:“此人相貌非常,只可激,不可说。等他问时,用言
激之便了。”献茶已毕,孙权曰:“多闻鲁子敬谈足下之才,今幸得相见,敢求教益。”孔
明曰:“不才无学,有辱明问。”权曰:“足下近在新野,佐刘豫州与曹操决战,必深知彼
军虚实。”孔明曰:“刘豫州兵微将寡,更兼新野城小无粮,安能与曹操相持。”权曰:
“曹兵共有多少?”孔明曰:“马步水军,约有一百余万。”权曰:“莫非诈乎?”孔明
曰:“非诈也。曹操就兖州已有青州军二十万;平了袁绍,又得五六十万;中原新招之兵三
四十万;今又得荆州之军二三十万:以此计之,不下一百五十万。亮以百万言之,恐惊江东
之士也。”鲁肃在旁,闻言失色,以目视孔明;孔明只做不见。权曰:“曹操部下战将,还
有多少?”孔明曰:“足智多谋之士,能征惯战之将,何止一二千人。”权曰:“今曹操平
了荆、楚,复有远图乎?”孔明曰:“即今沿江下寨,准备战船,不欲图江东,待取何
地?”权曰:“若彼有吞并之意,战与不战,请足下为我一决。”孔明曰:“亮有一言,但
恐将军不肯听从。”权曰:“愿闻高论。”孔明曰:“向者宇内大乱,故将军起江东,刘豫
州收众汉南,与曹操并争天下。今操芟除大难,略已平矣;近又新破荆州,威震海内;纵有
英雄,无用武之地:故豫州遁逃至此。愿将军量力而处之:若能以吴、越之众,与中国抗
衡,不如早与之绝;若其不能,何不从众谋士之论,按兵束甲,北面而事之?”权未及答。
孔明又曰:“将军外托服从之名,内怀疑贰之见,事急而不断,祸至无日矣!”权曰:“诚
如君言,刘豫州何不降操?”孔明曰:“昔田横,齐之壮士耳,犹守义不辱。况刘豫州王室
之胄,英才盖世,众士仰慕。事之不济,此乃天也。又安能屈处人下乎!”孙权听了孔明此
言,不觉勃然变色,拂衣而起,退入后堂。众皆哂笑而散,鲁肃责孔明曰:“先生何故出此
言?幸是吾主宽洪大度,不即面责。先生之言,藐视吾主甚矣。”孔明仰面笑曰:“何如此
不能容物耶!我自有破曹之计,彼不问我,我故不言。”肃曰:“果有良策,肃当请主公求
教。”孔明曰:“吾视曹操百万之众,如群蚁耳!但我一举手,则皆为齑粉矣!”肃闻言,
便入后堂见孙权。权怒气未息,顾谓肃曰:“孔明欺吾太甚!”肃曰:“臣亦以此责孔明,
孔明反笑主公不能容物。破曹之策,孔明不肯轻言,主公何不求之?”权回嗔作喜曰:“原
来孔明有良谋,故以言词激我。我一时浅见,几误大事。”便同鲁肃重复出堂,再请孔明叙
话。权见孔明,谢曰:“适来冒渎威严,幸勿见罪。”孔明亦谢曰:“亮言语冒犯,望乞恕
罪。”权邀孔明入后堂,置酒相待。

数巡之后,权曰:“曹操平生所恶者:吕布、刘表、袁绍、袁术、豫州与孤耳。今数雄
已灭,独豫州与孤尚存。孤不能以全吴之地,受制于人。吾计决矣。非刘豫州莫与当曹操
者;然豫州新败之后,安能抗此难乎?”孔明曰:“豫州虽新败,然关云长犹率精兵万人;
刘琦领江夏战士,亦不下万人。曹操之众,远来疲惫;近追豫州,轻骑一日夜行三百里,此
所谓强弩之末,势不能穿鲁缟者也。且北方之人,不习水战。荆州士民附操者,迫于势耳,
非本心也。今将军诚能与豫州协力同心,破曹军必矣。操军破,必北还,则荆、吴之势强,
而鼎足之形成矣。成败之机,在于今日。惟将军裁之。”权大悦曰:“先生之言,顿开茅
塞。吾意已决,更无他疑。即日商议起兵,共灭曹操!”遂令鲁肃将此意传谕文武官员,就
送孔明于馆驿安歇。张昭知孙权欲兴兵,遂与众议曰:“中了孔明之计也!”急入见权曰:
“昭等闻主公将兴兵与曹操争锋。主公自思比袁绍若何?曹操向日兵微将寡,尚能一鼓克袁
绍;何况今日拥百万之众南征,岂可轻敌?若听诸葛亮之言,妄动甲兵,此所谓负薪救火
也。”孙权只低头不语。顾雍曰:“刘备因为曹操所败,故欲借我江东之兵以拒之,主公奈
何为其所用乎;愿听子布之言。”孙权沉吟未决。张昭等出,鲁肃入见曰:“适张子布等,
又劝主公休动兵,力主降议,此皆全躯保妻子之臣,为自谋之计耳。原主公勿听也。”孙权
尚在沉吟。肃曰:“主公若迟疑,必为众人误矣。”权曰:“卿且暂退,容我三思。”肃乃
退出。时武将或有要战的,文官都是要降的,议论纷纷不一。且说孙权退入内宅,寝食不
安,犹豫不决。吴国太见权如此,问曰:“何事在心,寝食俱废?”权曰:“今曹操屯兵于
江汉,有下江南之意。问诸文武,或欲降者,或欲战者。欲待战来,恐寡不敌众;欲待降
来,又恐曹操不容:因此犹豫不决。”吴国太曰:“汝何不记吾姐临终之语乎?”孙权如醉
方醒,似梦初觉,想出这句话来。正是:追思国母临终语,引得周郎立战功。毕竟说着甚
的,且看下文分解。