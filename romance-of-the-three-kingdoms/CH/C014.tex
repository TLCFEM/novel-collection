\chapter{曹孟德移驾幸许都~吕奉先乘夜袭徐郡}

却说李乐引军诈称李傕、郭汜,来遍车驾,天子大惊。杨奉曰:“此李乐也。”遂令徐
晃出迎之。李乐亲自出战。两马相交,只一合,被徐晃一斧砍于马下,杀散余党,保护车驾
过箕关。太守张杨具粟帛迎驾于轵道。帝封张杨为大司马。杨辞帝屯兵野王去了。帝入洛
阳,见宫室烧尽,街市荒芜,满目皆是蒿草,宫院中只有颓墙坏壁。命杨奉且盖小宫居住。
百官朝贺,皆立于荆棘之中。诏改兴平为建安元年。是岁又大荒。洛阳居民,仅有数百家,
无可为食,尽出城去剥树皮、掘草根食之。尚书郎以下,皆自出城樵采,多有死于颓墙坏壁
之间者。汉末气运之衰,无甚于此。后人有诗叹之曰:“血流芒砀白蛇亡,赤帜纵横游四
方。秦鹿逐翻兴社稷,楚雅推倒立封疆。天子懦弱奸邪起,气色凋零盗贼狂。看到两京遭难
处,铁人无泪也怬惶!”太尉杨彪奏帝曰:“前蒙降诏,未曾发遣。今曹操在山东,兵强将
盛,可宣入朝,以辅王室。”帝曰:“朕前既降诏。卿何必再奏,今即差人前去便了。”彪
领旨,即差使命赴山东,宣召曹操。却说曹操在山东,闻知车驾已还洛阳,聚谋士商议,荀
彧进曰:“昔晋文公纳周襄王,而诸侯服从;汉高祖为义帝发丧,而天下归心。今天子蒙
尘,将军诚因此时首倡义兵,奉天子以从众望,不世之略也。若不早图,人将先我而为之
矣。”曹操大喜。正要收拾起兵,忽报有天使赍诏宣召。操接诏,克日兴师。却说帝在洛
阳,百事未备,城郭崩倒,欲修未能。人报李傕、郭汜领兵将到。帝大惊,问杨奉曰:“山
东之使未回,李、郭之兵又至,为之奈何?”杨奉、韩暹曰:“臣愿与贼决死战,以保陛
下!”董承曰:“城郭不坚,兵甲不多,战如不胜,当复如何?不若且奉驾往山东避之。”
帝从其言,即日起驾望山东进发。百官无马,皆随驾步行。出了洛阳,行无一箭之地,但见
尘头蔽日,金鼓喧天,无限人马来到。帝、后战慓不能言。忽见一骑飞来,乃前差往山东之
使命也,至车前拜启曰:“曹将军尽起山东之兵,应诏前来。闻李傕、郭汜犯洛阳,先差夏
侯惇为先锋,引上将十员,精兵五万,前来保驾。”帝心方安。

少顷,夏侯惇引许褚、典韦等,至驾前面君,俱以军礼见。帝慰谕方毕,忽报正东又有
一路军到。帝即命夏侯惇往探之,回妻曰:“乃曹操步军也。”须臾,曹洪、李典、乐进来
见驾。通名毕,洪奏曰:“臣兄知贼兵至近,恐夏侯惇孤力难为,故又差臣等倍道而来协
助。”帝曰:“曹将军真社稷臣也!”遂命护驾前行。探马来报:“李傕、郭汜领兵长驱而
来。”帝令夏侯惇分两路迎之。惇乃与曹洪分为两翼,马军先出,步军后随,尽力攻击。
傕、汜贼兵大败,斩首万余。于是请帝还洛阳故宫。夏侯惇屯兵于城外。

次日,曹操引大队人马到来。安营毕,入城见帝、拜于殿阶之下。帝赐平身,宣谕慰
劳。操曰:“臣向蒙国恩,刻思图报。今傕、汜二贼,罪恶贯盈;臣有精兵二十余万,以顺
讨逆,无不克捷。陛下善保龙体,以社稷为重。”帝乃封操领司隶校尉假节钺录尚书事。

却说李傕、郭汜知操远来,议欲速战。贾诩谏曰:“不可。操兵精将勇,不如降之,求
免本身之罪。”傕怒曰:“尔敢灭吾锐气!”拔剑欲斩诩。众将劝免。是夜,贾诩单马走回
乡里去了。次日,李傕军马来迎操兵。操先令许褚、曹仁、典韦领三百铁骑,于傕阵中冲突
三遭,方才布阵。阵圆处,李傕侄李暹、李别出马阵前,未及开言,许褚飞马过去,一刀先
斩李暹;李别吃了一惊,倒撞下马,褚亦斩之,双挽人头回阵。曹操抚许褚之背曰:“子真
吾之樊哙也!”随令夏侯惇领兵左出、曹仁领兵右出,操自领中军冲阵。鼓响一声,三军齐
进。贼兵抵敌不住,大败而走。操亲掣宝剑押阵,率众连夜追杀,剿戮极多,降者不计其
数。傕、汜望西逃命,忙忙似丧家之狗;自知无处容身,只得往山中落草去了。曹操回兵,
仍屯于洛阳城外。杨奉、韩暹两个商议:“今曹操成了大功,必掌重权,如何容得我等?”
乃入奏天子,只以追杀傕、汜为名,引本部军屯于大梁去了。

帝一日命人至操营,宣操入宫议事。操闻天使至,请入相见,只见那人眉清目秀,精神
充足。操暗想曰:“今东都大荒,官僚军民皆有饥色,此人何得独肥?”因问之曰:“公尊
颜充腴,以何调理而至此?”对曰:“某无他法,只食淡三十年矣。”操乃颔之;又问曰:
“君居何职?”对曰:“某举孝廉。原为袁绍、张杨从事。今闻天子还都,特来朝觐,官封
正议郎。济阴定陶人,姓董,名昭,字公仁。”曹操避席曰:“闻名久矣!幸得于此相
见。”遂置酒帐中相待,令与荀彧相会。忽人报曰:“一队军往东而去,不知何人。”操急
令人探之。董昭曰:“此乃李傕旧将杨奉,与白波帅韩暹,因明公来此,故引兵欲投大梁去
耳。”操曰:“莫非疑操乎?”昭曰:“此乃无谋之辈,明公何足虑也。”操又曰:“李、
郭二贼此去若何?”昭曰:“虎无爪,鸟无翼,不久当为明公所擒,无足介意。”

操见昭言语投机,便问以朝廷大事。昭曰:“明公兴义兵以除暴乱,入朝辅佐天子,此
五霸之功也。但诸将人殊意异,未必服从:今若留此,恐有不便。惟移驾幸许都为上策。然
朝廷播越,新还京师,远近仰望,以冀一朝之安;今复徒驾,不厌众心。夫行非常之事,乃
有非常之功,愿将军决计之。”操执昭手而笑曰:“此吾之本志也。但杨奉在大梁,大臣在
朝,不有他变否?”昭曰:“易也。以书与杨奉,先安其心。明告大臣,以京师无粮,欲车
驾幸许都,近鲁阳,转运粮食,庶无欠缺悬隔之忧。大臣闻之,当欣从也。”操大喜。昭谢
别,操执其手曰:“凡操有所图,惟公教之。”昭称谢而去。

操由是日与众谋士密议迁都之事。时侍中太史令王立私谓宗正刘艾曰:“吾仰观天文,
自去春太白犯镇星于斗牛,过天津,荧惑又逆行,与太白会于天关,金火交会,必有新天子
出。吾观大汉气数将终,晋魏之地,必有兴者。”又密奏献帝曰:“天命有去就,五行不常
盛。代火者土也。代汉而有天下者,当在魏。”操闻之,使人告立曰:“知公忠于朝廷,然
天道深远,幸勿多言。”操以是告彧。彧曰:“汉以火德王,而明公乃土命也。许都属土,
到彼必兴。火能生土,土能旺木:正合董昭、王立之言。他日必有兴者。”操意遂决。次
日,入见帝,奏曰:“东都荒废久矣,不可修葺;更兼转运粮食艰辛。许都地近鲁阳,城郭
宫室,钱粮民物,足可备用。臣敢请驾幸许都,惟陛下从之。”帝不敢不从;群臣皆惧操
势,亦莫敢有异议。遂择日起驾。操引军护行,百官皆从。

行不到数程,前至一高陵。忽然喊声大举,杨奉、韩暹领兵拦路。徐晃当先,大叫:
“曹操欲劫驾何住!”操出马视之,见徐晃威风凛凛,暗暗称奇;便令许褚出马与徐晃交
锋。刀斧相交,战五十余合,不分胜败。操即鸣金收军,召谋士议曰:“杨奉、韩暹诚不足
道;徐晃乃真良将也。吾不忍以力并之,当以计招之。”行军从事满宠曰:“主公勿虑。某
向与徐晃有一面之交,今晚扮作小卒,偷入其营,以言说之,管教他倾心来降。”操欣然遣
之。

是夜满宠扮作小卒,混入彼军队中,偷至徐晃帐前,只见晃秉烛被甲而坐。宠突至其
前,揖曰:“故人别来无恙乎!”徐晃惊起,熟视之曰:“子非山阳满伯宁耶!何以至
此?”宠曰:“某现为曹将军从事。今日于阵前得见故人,欲进一言,故特冒死而来。”晃
乃延之坐,问其来意。宠曰:“公之勇略,世所罕有,奈何屈身于杨、韩之徒?曹将军当世
英雄,其好贤礼士,天下所知也;今日阵前,见公之勇,十分敬爱,故不忍以健将决死战,
特遣宠来奉邀。公何不弃暗投明,共成大业?”晃沈吟良久,乃喟然叹曰:“吾固知奉、暹
非立业之人,奈从之久矣,不忍相舍。”宠曰:“岂不闻良禽择木而栖,贤臣择主而事。遇
可事之主,而交臂失之,非丈夫也。”晃起谢曰:“愿从公言。”宠曰:“何不就杀奉、暹
而去,以为进见之礼?”晃曰:“以臣弑主,大不义也。吾决不为。”宠曰:“公真义士
也!”晃遂引帐下数十骑,连夜同满宠来投曹操。早有人报知杨奉。奉大怒,自引千骑来
追,大叫:“徐晃反贼休走!”正追赶间,忽然一声炮响,山上山下,火把齐明,伏军四
出,曹操亲自引军当先,大喝:“我在此等候多时。休教走脱!”杨奉大惊,急待回军,早
被曹兵围住。恰好韩暹引兵来救,两军混战,杨奉走脱。曹操趁彼军乱,乘势攻击,两家军
士大半多降。杨奉、韩暹势孤,引败兵投袁术去了。

曹操收军回营,满宠引徐晃入见。操大喜,厚待之。于是迎銮驾到许都,盖造宫室殿
宇,立宗庙社稷、省台司院衙门,修城郭府库;封董承等十三人为列侯。赏功罚罪,并听曹
操处置。操自封为大将军武平侯,以荀彧为侍中尚书令,荀攸为军师,郭嘉为司马祭酒,刘
晔为司空仓曹掾,毛玠、任峻为典农中郎将,催督钱粮,程昱为东平相,范成、董昭为洛阳
令,满宠为许都令,夏侯惇、夏侯渊、曹仁、曹洪皆为将军,吕虔、李典、乐进、于禁、徐
晃皆为校尉,许褚、典韦皆为都尉;其余将士,各各封官。自此大权皆归于曹操:朝廷大
务,先禀曹操,然后方奏天子。

操既定大事,乃设宴后堂,聚众谋士共议曰:“刘备屯兵徐州,自领州事;近吕布以兵
败投之,备使居于小沛:若二人同心引兵来犯,乃心腹之患也。公等有何妙计可图之?”许
褚曰:“愿借精兵五万,斩刘备、吕布之头,献于丞相。”荀彧曰:“将军勇则勇矣,不知
用谋。今许都新定,未可造次用兵。彧有一计,名曰二虎竞食之计。今刘备虽领徐州,未得
诏命。明公可奏请诏命实授备为徐州牧,因密与一书,教杀吕布。事成则备无猛士为辅,亦
渐可图;事不成,则吕布必杀备矣:此乃二虎竞食之计也。”操从其言,即时奏请诏命,遣
使赍往徐州,封刘备为征东将军宜城亭侯领徐州牧;并附密书一封。却说刘玄德在徐州,闻
帝幸许都,正欲上表庆贺。忽报天使至,出郭迎接入郡,拜受恩命毕,设宴管待来使。使
曰:“君侯得此恩命,实曹将军于帝前保荐之力也。”玄德称谢。使者乃取出私书递与玄
德。玄德看罢,曰:“此事尚容计议。”席散,安歇来使于馆驿。玄德连夜与众商议此事。
张飞曰:“吕布本无义之人,杀之何碍!”玄德曰:“他势穷而来投我,我若杀之,亦是不
义。”张飞曰:“好人难做!”玄德不从。次日,吕布来贺,玄德教请入见。布曰:“闻公
受朝廷恩命,特来相贺。”玄德逊谢。只见张飞扯剑上厅,要杀吕布。玄德慌忙阻住。布大
惊曰:“翼德何故只要杀我?”张飞叫曰:“曹操道你是无义之人,教我哥哥杀你!”玄德
连声喝退。乃引吕布同入后堂,实告前因;就将曹操所送密书与吕布看。布看毕,泣曰:
“此乃曹贼欲令我二人不和耳!”玄德曰:“兄勿忧,刘备誓不为此不义之事。”吕布再三
拜谢。备留布饮酒,至晚方回。关、张曰:“兄长何故不杀吕布?”玄德曰:“此曹孟德恐
我与吕布同谋伐之,故用此计,使我两人自相吞并,彼却于中取利。奈何为所使乎?”关公
点头道是。张飞曰:“我只要杀此贼以绝后患!”玄德曰:“此非大丈夫之所为也。”

次日,玄德送使命回京,就拜表谢恩,并回书与曹操,只言容缓图之。使命回见曹操,
言玄德不杀吕布之事。操问荀彧曰:“此计不成,奈何?”或曰:“又有一计,名曰驱虎吞
狼之计。”操曰:“其计如何?”彧曰:“可暗令人往袁术处通问,报说刘备上密表,要略
南郡。术闻之,必怒而攻备;公乃明诏刘备讨袁术。两边相并,吕布必生异心:此驱虎吞狼
之计也。”操大喜,先发人往袁术处;次假天子诏,发人往徐州。

却说玄德在徐州,闻使命至,出郭迎接;开读诏书,却是要起兵讨袁术。玄德领命,送
使者先回。糜竺曰:“此又是曹操之计。”玄德曰:“虽是计,王命不可违也。”遂点军
马,克日起程,孙乾曰:“可先定守城之人。”玄德曰:“二弟之中,谁人可守?”关公
曰:“弟愿守此城。”玄德曰:“吾早晚欲与尔议事,岂可相离?”张飞曰:“小弟愿守此
城。”玄德曰:“你守不得此城:你一者酒后刚强,鞭挞士卒;二者作事轻易,不从人谏。
吾不放心。”张飞曰:“弟自今以后,不饮酒,不打军士,诸般听人劝谏便了。”糜竺曰:
“只恐口不应心。”飞怒曰:“吾跟哥哥多年,未尝失信,你如何轻料我!”玄德曰:“弟
言虽如此,吾终不放心。还请陈元龙辅之,早晚令其少饮酒,勿致失事。”陈登应诺。玄德
分付了当,乃统马步军三万,离徐州望南阳进发。却说袁术闻说刘备上表,欲吞其州县,乃
大怒曰:“汝乃织席编屦之去,今辄占据大郡,与诸侯同列;吾正欲伐汝,汝却反欲图我!
深为可恨!”乃使上将纪灵起兵十万,杀弃徐州。两军会于盱眙。玄德兵少,依山傍水下
寨。那纪灵乃山东人,使一口三尖刀,重五十斤。是日引兵出阵,大骂:“刘备村夫,安敢
侵吾境界!”玄德曰:“吾奉天子诏,以讨不臣。汝今敢来相拒,罪不容诛!”纪灵大怒,
拍马舞刀,直取玄德。关公大喝曰:“匹夫休得逞强!”出马与纪灵大战。一连三十合,不
分胜负。纪灵大叫少歇,关公便拨马回阵,立于阵前候之。纪灵却遣副将荀正出马。关公
曰:“只教纪灵来,与他决个雌雄!”荀正曰:“汝乃无名下将,非纪将军对手!”关公大
怒,直取荀正;交马一合,砍荀正于马下。玄德驱兵杀将过去,纪灵大败,退守淮阴河口,
不敢交战;只教军士来偷营劫寨,皆被徐州兵杀败。两军相拒,不在话下。

却说张飞自送玄德起身后,一应杂事,俱付陈元龙管理;军机大务,自家参酌,一日,
设宴请各官赴席。众人坐定,张飞开言曰:“我兄临去时,分付我少饮酒,恐致失事。众官
今日尽此一醉,明日都各戒酒,帮我守城。今日却都要满饮。”言罢,起身与众官把盏。酒
至曹豹面前,豹曰:“我从天戒,不饮酒。”飞曰:“厮杀汉如何不饮酒?我要你吃一
盏。”豹惧怕,只得饮了一杯。张飞把遍各官,自斟巨觥,连饮了几十杯,不觉大醉,却又
起身与众官把盏。酒至曹豹,豹曰:“某实不能饮矣。”飞曰:“你恰才吃了,如今为何推
却?”豹再三不饮。飞醉后使酒,便发怒曰:“你违我将令该打一百!”便喝军士拿下。陈
元龙曰:“玄德公临去时,分付你甚来?”飞曰:“你文官,只管文官事,休来管我!”曹
豹无奈,只得告求曰:“翼德公,看我女婿之面,且恕我罢。”飞曰:“你女婿是谁?”豹
曰:“吕布是也。”飞大怒曰:“我本不欲打你;你把吕布来唬我,我偏要打你!我打你,
便是打吕布!”诸人劝不住。将曹豹鞭至五十,众人苦苦告饶,方止。

席散,曹豹回去,深恨张飞,连夜差人赍书一封,径投小沛见吕布,备说张飞无礼;且
云:玄德已往淮南,今夜可乘飞醉,引兵来袭徐州,不可错此机会。吕布见书,便请陈宫来
议。宫曰:“小沛原非久居之地。今徐州既有可乘之隙,失此不取,悔之晚矣。”布从之,
随即披挂上马,领五百骑先行;使陈宫引大军继进,高顺亦随后进发。

小沛离徐州只四五十里,上马便到。吕布到城下时,恰才四更,月色澄清,城上更不知
觉。布到城门边叫曰:“刘使君有机密使人至。”城上有曹豹军报知曹豹,豹上城看之,便
令军士开门。吕布一声暗号。众军齐入,喊声大举。张飞正醉卧府中,左右急忙摇醒,报
说:“吕布赚开城门,杀将进来了!”张飞大怒,慌忙披挂,绰了丈八蛇矛;才出府门上得
马时,吕布军马已到,正与相迎。张飞此时酒犹未醒,不能力战。吕布素知飞勇,亦不敢相
逼。十八骑燕将,保着张飞,杀出东门,玄德家眷在府中,都不及顾了。

却说曹豹见张飞只十数人护从,又欺他醉,遂引百十人赶来。飞见豹,大怒,拍马来
迎。战了三合,曹豹败走,飞赶到河边,一枪正刺中曹豹后心,连人带马,死于河中。飞于
城外招呼士卒,出城者尽随飞投淮南而去。吕布入城安抚居民,令军士一百人守把玄德宅
门,诸人不许擅入。

却说张飞引数十骑,直到盱眙来见玄德,具说曹豹与吕布里应外合,夜袭徐州。众皆失
色。玄德叹曰:“得何足喜,失何足忧!”关公曰:“嫂嫂安在?”飞曰:“皆陷于城中
矣。”玄德默然无语。关公顿足埋怨曰:“你当初要守城时说甚来?兄长分付你甚来?今日
城池又失了,嫂嫂又陷了,如何是好!”张飞闻言,惶恐无地,掣剑欲自刎。正是:举杯畅
饮情何放,拔剑捐生悔已迟!不知性命如何,且听下文分解。