\chapter{美髯公智稳插翅虎~宋公明私放晁天王}

当时何观察与兄弟何清道:“这锭银子,是官司信赏的,非是我把来赚你,后
头再有重赏。兄弟,你且说这伙人如何在你便袋里?”只见何清去身边招文袋内摸
出一个经折儿来,指道:“这伙贼人都在上面。”何涛道:“你且说怎地写在上面?”
何清道:“不瞒哥哥说,兄弟前日为赌博输了,没一文盘缠,有个一般赌博的,引
兄弟去北门外十五里,地名安乐村,有个王家客店内,凑些碎赌。为是官司行下文
书来,着落本村,但凡开客店的,须要置立文簿,一面上用勘合印信;每夜有客商
来歇宿,须要问他:‘那里来?何处去?姓甚名谁?做甚买卖?’都要抄写在簿子上。
官司查照时,每月一次,去里正处报名。为是小二哥不识字,央我替他抄了半个月。
当日是六月初三日,有七个贩枣子的客人,推着七辆江州车儿来歇。我却认得一个
为头的客人,是郓城县东溪村晁保正。因何认得他?我比先曾跟一个赌汉去投奔他,
因此我认得。我写着文簿,问他道:‘客人高姓?’只见一个三髭须白净面皮的抢
将过来,答应道:‘我等姓李,从濠州来贩枣子,去东京卖。’我虽写了,有些疑
心。第二日,他自去了,店主带我去村里相赌,来到一处三叉路口,只见一个汉子
挑两个桶来。我不认得他。店主人自与他厮叫道:‘白大郎,那里去?’那人应道:
‘有担醋,将去村里财主家卖。’店主人和我说道:‘这人叫做白日鼠白胜,他是
个赌客。’我也只安在心里。后来听得沸沸扬扬地说道:‘黄泥冈上一伙贩枣子的
客人,把蒙汗药麻翻了人,劫了生辰纲去。’我猜不是晁保正,却是兀谁!如今只
捕了白胜,一问便知端的。这个经折儿,是我抄的副本。”

何涛听了大喜,随即引了兄弟何清,径到州衙里见了太守。府尹问道:“那公
事有些下落么?”何涛禀道:“略有些消息了。”府尹叫进后堂来说,仔细问了来
历。何清一一禀说了。当下便差八个做公的,一同何涛、何清,连夜来到安乐村,
叫了店主人做眼,径奔到白胜家里,却是三更时分。叫店主人赚开门来打火,只听
得白胜在床上做声。问他老婆时,却说道害热病,不曾得汗。从床上拖将起来,见
白胜面色红白,就把索子绑了,喝道:“黄泥冈上做得好事!”白胜那里肯认。把
那妇人捆了,也不肯招。众做公的绕屋寻赃,寻到床底下,见地面不平;众人掘开,
不到三尺深,众多公人发声喊,白胜面如土色,就地下取出一包金银,随即把白胜
头脸包了,带他老婆,扛抬赃物,都连夜赶回济州城里来。却好五更天明时分,把
白胜押到厅前,便将索子捆了。问他主情造意,白胜抵赖,死不肯招晁保正等七人。
连打三四顿,打的皮开肉绽,鲜血迸流。府尹喝道:“告的正主招了赃物,捕人已
知是郓城县东溪村晁保正了,你这厮如何赖得过!你快说那六人是谁,便不打你了。”
白胜又捱了一歇,打熬不过,只得招道:“为首的是晁保正。他自同六人来纠合白
胜,与他挑酒,其实不认得那六人。”知府道:“这个不难。只拿住晁保正,那六
人便有下落。”先取一面二十斤死枷,枷了白胜,他的老婆也锁了,押去女牢里监
收。

随即押一纸公文,就差何涛亲自带领二十个眼明手快的公人,径去郓城县投下,
着落本县,立等要捉晁保正,并不知姓名六个正贼。就带原解生辰纲的两个虞候,
作眼拿人。一同何观察领了一行人,去时不要大惊小怪,只恐怕走透了消息。星夜
来到郓城县,先把一行公人并两个虞候,都藏在客店里,只带一两个跟着,来下公
文,径奔郓城县衙门前来。当下巳牌时分,却值知县退了早衙,县前静悄悄地,何
涛走去县对门一个茶坊里坐下,吃茶相等。吃了一个泡茶,问茶博士道:“今日如
何县前恁地静?”茶博士说道:“知县相公早衙方散,一应公人和告状的,都去吃
饭了未来。”何涛又问道:“今日县里不知是那个押司直日?”茶博士指着道:“今
日直日的押司来也。”何涛看时,只见县里走出一个吏员来。看那人时,怎生模样,
但见:

眼如丹凤,眉似卧蚕。滴溜溜两耳悬珠,明皎皎双睛点漆。唇方口正,髭须地
阁轻盈;额阔顶平,皮肉天仓饱满。坐定时浑如虎相,走动时有若狼形。年及三旬,
有养济万人之度量;身躯六尺,怀扫除四海之心机。志气轩昂,胸襟秀丽。刀笔敢
欺萧相国,声名不让孟尝君。

那押司姓宋,名江,表字公明,排行第三,祖居郓城县宋家村人氏。为他面黑
身矮,人都唤他做黑宋江;又且于家大孝,为人仗义疏财,人皆称他做孝义黑三郎。
上有父亲在堂,母亲早丧,下有一个兄弟,唤做铁扇子宋清,自和他父亲宋太公在
村中务农,守些田园过活。这宋江自在郓城县做押司。他刀笔精通,吏道纯熟,更
兼爱习枪棒,学得武艺多般。平生只好结识江湖上好汉,但有人来投奔他的,若高
若低,无有不纳,便留在庄上馆谷,终日追陪,并无厌倦;若要起身,尽力资助,
端的是挥霍,视金似土。人问他求钱物,亦不推托,且好做方便,每每排难解纷,
只是周全人性命。如常散施棺材药饵,济人贫苦,人之急,扶人之困,以此山东、
河北闻名,都称他做及时雨;却把他比做天上下的及时雨一般,能救万物。曾有一
首《临江仙》赞宋江好处:

起自花村刀笔吏,英灵上应天星,疏财仗义更多能。事亲行孝敬,待士有声名。

济弱扶倾心慷慨,高名水月双清。及时甘雨四方称,山东呼保义,豪杰宋公明。

当时宋江带着一个伴当,走将出县前来。只见这何观察当街迎住,叫道:“押
司,此间请坐拜茶。”宋江见他似个公人打扮,慌忙答礼道:“尊兄何处?”何涛
道:“且请押司到茶坊里面吃茶说话。”宋公明道:“谨领。”两个入到茶坊里坐
定,伴当都叫去门前等候。宋江道:“不敢拜问尊兄高姓?”何涛答道:“小人是
济州府缉捕使臣何观察的便是。不敢动问押司高姓大名?”宋江道:“贱眼不识观
察,少罪。小吏姓宋名江的便是。”何涛倒地便拜,说道:“久闻大名,无缘不曾
拜识。”宋江道:“惶恐。观察请上坐。”何涛道:“小人安敢占上?”宋江道:
“观察是上司衙门的人,又是远来之客。”两个谦让了一回,宋江坐了主位,何涛
坐了客席。宋江便叫茶博士将两杯茶来。没多时,茶到。两个吃了茶。宋江道:“观
察到敝县,不知上司有何公务?”何涛道:“实不相瞒,来贵县有几个要紧的人。”
宋江道:“莫非贼情公事否?”何涛道:“有实封公文在此,敢烦押司作成。”宋
江道:“观察是上司差来捕盗的人,小吏怎敢怠慢?不知为甚么贼情紧事?”何涛
道:“押司是当案的人,便说也不妨:敝府管下黄泥冈上一伙贼人,共是八个,把
蒙汗药麻翻了北京大名府梁中书差遣送蔡太师的生辰纲军健一十五人,劫去了十一
担珍珠宝贝,计该十万贯正赃。今捕得从贼一名白胜,指说七个正贼,都在贵县。
这是太师府特差一个干办,在本府立等要这件公事,望押司早早维持。”宋江道:
“休说太师处着落,便是观察自赍公文来要,敢不捕送?只不知道白胜供指那七人
名字?”何涛道:“不瞒押司说:是贵县东溪村晁保正为首。更有六名从贼,不识
姓名,烦乞用心。”

宋江听罢,吃了一惊,肚里寻思道:“晁盖是我心腹弟兄。他如今犯了迷天大
罪,我不救他时,捕获将去,性命便休了!”心内自慌,却答应道:“晁盖这厮,
奸顽役户,本县内上下人,没一个不怪他。今番做出来了,好教他受!”何涛道:
“相烦押司便行此事。”宋江道:“不妨,这事容易,‘瓮中捉鳖,手到拿来。’
只是一件,这实封公文,须是观察自己当厅投下,本官看了,便好施行发落,差人
去捉,小吏如何敢私下擅开?这件公事,非是小可,不当轻泄于人。”何涛道:“押
司高见极明,相烦引进。”宋江道:“本官发放一早晨事务,倦怠了少歇。观察略
待一时,少刻坐厅时,小吏来请。”何涛道:“望押司千万作成。”宋江道:“理
之当然,休这等说话。小吏略到寒舍,分拨了些家务便到,观察少坐一坐。”何涛
道:“押司尊便,小弟只在此专等。”

宋江起身,出得阁儿,分付茶博士道:“那官人要再用茶,一发我还茶钱。”
离了茶坊,飞也似跑到下处。先分付伴当去叫直司在茶坊门前伺候:“若知县坐衙
时,便可去茶坊里安抚那公人道:‘押司稳便’,叫他略待一待。”却自槽上了
马,牵出后门外去,拿了鞭子,慌忙的跳上马,慢慢地离了县治。出得东门,打上
两鞭,那马拨喇喇的望东溪村撺将去,没半个时辰,早到晁盖庄上。庄客见了,入
去庄里报知。正是:
义重轻他不义财,奉天法网有时开。
剥民官府过于贼,应为知交放贼来。

且说晁盖正和吴用、公孙胜、刘唐在后园葡萄树下吃酒。此时三阮已得了钱财,
自回石碣村去了。晁盖见庄客报说宋押司在门前。晁盖问道:“有多少人随从着?”
庄客道:“只独自一个飞马而来,说快要见保正。”晁盖道:“必然有事。”慌忙
出来迎接。宋江道了一个喏,携了晁盖手,便投侧边小房里来。晁盖问道:“押司
如何来的慌速?”宋江道:“哥哥不知,兄弟是心腹弟兄,我舍着条性命来救你。
如今黄泥冈事发了!白胜已自拿在济州大牢里了,供出你等七人。济州府差一个何
缉捕,带着若干人,奉着太师府钧帖,并本州文书,来捉你等七人,道你为首。天
幸撞在我手里,我只推说知县睡着,且教何观察在县对门茶坊里等我。以此飞马而
来,报道哥哥。‘三十六计,走为上计’。若不快走时,更待甚么?我回去引他当
厅下了公文,知县不移时,便差人连夜下来,你们不可耽搁,倘有些疏失,如之奈
何!休怨小弟不来救你。”

晁盖听罢,吃了一惊道:“贤弟大恩难报!”宋江道:“哥哥,你休要多说,
只顾安排走路,不要缠障,我便回去也。”晁盖道:“七个人,三个是阮小二、阮
小五、阮小七,已得了财,自回石碣村去了;后面有三个在这里,贤弟且见他一面。”
宋江来到后园,晁盖指着道:“这三位,一个吴学究;一个公孙胜,蓟州来的;一
个刘唐,东潞州人。”宋江略讲一礼,回身便走,嘱付道:“哥哥保重,作急快走,
兄弟去也。”宋江出到庄前,上了马,打上两鞭,飞也似望县里来了。当时有个学
究,为此事作诗一首,也说得是。诗曰:
保正缘何养贼曹,押司纵贼罪难逃。
须知守法清名重,莫谓通情义气高。
爵固畏能害爵,猫如伴鼠岂成猫。
空持刀笔称文吏,羞说当年汉相萧。

且说晁盖与吴用、公孙胜、刘唐三人道:“你们认得那来相见的这个人么?”
吴用道:“却怎地慌慌忙忙便去了?正是谁人?”晁盖道:“你三位还不知哩!我们
不是他来时,性命只在咫尺休了!”三人大惊道:“莫不走了消息,这件事发了?”
晁盖道:“亏杀这个兄弟,担着血海也似干系,来报与我们。原来白胜已自捉在济
州大牢里了,供出我等七人。本州差个缉捕何观察,将带若干人,奉着太师钧帖来,
着落郓城县,立等要拿我们七个。亏了他稳住那公人在茶坊里俟候,他飞马先来报
知我们,如今回去下了公文,少刻便差人连夜到来,捕获我们,却是怎地好!”吴
用道:“若非此人来报,都打在网里。这大恩人姓甚名谁?”晁盖道:“他便是本
县押司呼保义宋江的便是。”吴用道:“只闻宋押司大名,小生却不曾得会。虽是
住居咫尺,无缘难得见面。”公孙胜、刘唐都道:“莫不是江湖上传说的及时雨宋
公明?”晁盖点头道:“正是此人。他和我心腹相交,结义弟兄,吴先生不曾得会,
四海之内,名不虚传,结义得这个兄弟,也不枉了。”

晁盖问吴用道:“我们事在危急,却是怎地解救?”吴学究道:“兄长不须商
议,‘三十六计,走为上计’。”晁盖道:“却才宋押司也教我们走为上计,却是
走那里去好?”吴用道:“我已寻思在肚里了。如今我们收拾五七担挑了,一径都
走奔石碣村三阮家里去。今急遣一人,先与他弟兄说知。”晁盖道:“三阮是个打
鱼人家,如何安得我等许多人?”吴用道:“兄长,你好不精细!石碣村那里一步
步近去,便是梁山泊。如今山寨里好生兴旺,官军捕盗,不敢正眼儿看他。若是赶
得紧,我们一发入了伙。”晁盖道:“这一论极是上策,只恐怕他们不肯收留我们。”
吴用道:“我等有的是金银,送献些与他,便入伙了。”正是:
无道之时多有盗,英雄进退两俱难。
只因秀士居山寨,买盗犹然似买官。

当时晁盖道:“既然恁地商量定了,事不宜迟。吴先生,你便和刘唐带了几个
庄客,挑担先去阮家安顿了,却来旱路上接我们。我和公孙先生两个打并了便来。”
吴用、刘唐把这生辰纲打劫得金珠宝贝,做五六担装了,叫五六个庄客,一发吃了
酒食。吴用袖了铜链,刘唐提了朴刀,监押着五七担,一行十数人,投石碣村来。
晁盖和公孙胜在庄上收拾。有些不肯去的庄客,赍发他些钱物,从他去投别主。有
愿去的,都在庄上并叠财物,打拴行李。正是:
须信钱财是毒蛇,钱财聚处即亡家。
人称义士犹难保,天鉴贪官漫自夸。

再说宋江飞马去到下处,连忙到茶坊里来,只见何观察正在门前望。宋江道:
“观察久等。却被村里有个亲戚,在下处说些家务,因此耽搁了些。”何涛道:“有
烦押司引进。”宋江道:“请观察到县里。”两个入得衙门来,正值知县时文彬在
厅上发落事务。宋江将着实封公文,引着何观察直至书案边,叫左右挂上回避牌,
宋江向前禀道:“奉济州府公文,为贼情紧急公务,特差缉捕使臣何观察到此下文
书。”知县接来拆开,就当厅看了,大惊,对宋江道:“这是太师府差干办来立等
要回话的勾当。这一干贼,便可差人去捉。”宋江道:“日间去,只怕走了消息,
只可差人就夜去捉。拿得晁保正来,那六人便有下落。”时知县道:“这东溪村晁
保正,闻名是个好汉,他如何肯做这等勾当?”随即叫唤尉司并两个都头,一个姓
朱,名仝,一个姓雷,名横。他两个,非是等闲人也。

当下朱仝、雷横,两个来到后堂,领了知县言语,和县尉上了马,径到尉司,
点起马步弓手并土兵一百余人,就同何观察并两个虞候,作眼拿人。当晚都带了绳
索军器,县尉骑着马,两个都头亦各乘马,各带了腰刀弓箭,手拿朴刀,前后马步
弓手簇拥着,出得东门,飞奔东溪村晁家来。

到得东溪村里,已是一更天气,都到一个观音庵取齐。朱仝道:“前面便是晁
家庄。晁盖家有前后两条路。若是一齐去打他前门,他望后门走了;一齐哄去打他
后门,他奔前门走了。我须知晁盖好生了得,又不知那六个是甚么人,必须也不是
善良君子。那厮们都是死命,倘或一齐杀出来,又有庄客协助,却如何抵敌他?只
好声东击西,等那厮们乱窜,便好下手。不若我和雷都头分做两路:我与你分一半
人,都是步行去,先望他后门埋伏了;等候唿哨响为号,你等向前门只顾打入来,
见一个捉一个,见两个捉一双。”雷横道:“也说的是。朱都头,你和县尉相公,
从前门打入来,我去截住后路。”朱仝道:“贤弟,你不省得。晁盖庄上有三条活
路,我闲常时都看在眼里了。我去那里,须认得他的路数,不用火把便见。你还不
知他出没的去处,倘若走漏了事情,不是耍处。”县尉道:“朱都头说得是,你带
一半人去。”朱仝道:“只消得三十来个够了。”朱仝领了十个弓手,二十个土兵,
先去了。县尉再上了马,雷横把马步弓手,都摆在前后,帮护着县尉。土兵等都在
马前,明晃晃照着三二十个火把,拿着叉、朴刀、留客住、钩镰刀,一齐都奔晁
家庄来。

到得庄前,兀自有半里多路,只见晁盖庄里一缕火起,从中堂烧将起来,涌得
黑烟遍地,红焰飞空。又走不到十数步,只见前后门四面八方,约有三四十把火发,
焰腾腾地一齐都着。前面雷横挺着朴刀,背后众土兵发着喊,一齐把庄门打开,都
扑入里面。看时,火光照得如同白日一般明亮,并不曾见有一个人,只听得后面发
着喊,叫将起来,叫前面捉人。原来朱仝有心要放晁盖,故意赚雷横去打前门。这
雷横亦有心要救晁盖,以此争先要来打后门;却被朱仝说开了,只得去打他前门。
故意这等大惊小怪,声东击西,要催逼晁盖走了。

朱仝那时到庄后时,兀自晁盖收拾未了。庄客看见,来报与晁盖说道:“官军
到了!事不宜迟!”晁盖叫庄客四下里只顾放火,他和公胜孙引了十数个去的庄客,
呐着喊,挺起朴刀,从后门杀将出来,大喝道:“当吾者死!避吾者生!”朱仝在
黑影里叫道:“保正休走!朱仝在这里等你多时。”晁盖那里顾他说,与同公孙胜,
舍命只顾杀出来。朱仝虚闪一闪,放开条路,让晁盖走了。晁盖却叫公孙胜引了庄
客先走,他独自押着后。朱仝使步弓手从后门扑入去,叫道:“前面赶捉贼人!”
雷横听的,转身便出庄门外,叫马步弓手分头去赶。雷横自在火光之下,东观西望
做寻人。朱仝撇了土兵,挺着刀,去赶晁盖。晁盖一面走,口里说道:“朱都头,
你只管追我做甚么?我须没歹处!”朱仝见后面没人,方才敢说道:“保正,你兀
自不见我好处:我怕雷横执迷,不会做人情,被我赚他打你前门,我在后面等你出
来放你。你见我闪开条路,让你过去。你不可投别处去,只除梁山泊可以安身。”
晁盖道:“深感救命之恩,异日必报!”有诗为证:
捕盗如何与盗通,官赃应与盗赃同。
莫疑官府能为盗,自有皇天不肯容。

朱仝正赶间,只听得背后雷横大叫道:“休教走了人!”朱仝分付晁盖道:“保
正,你休慌,只顾一面走,我自使转他去。”朱仝回头叫道:“有三个贼望东小路
去了,雷都头,你可急赶。”雷横领了人,便投东小路上,并土兵众人赶去。朱仝
一面和晁盖说着话,一面赶他,却如防送的相似。渐渐黑影里不见了晁盖。朱仝只
做失脚扑地,倒在地下。众土兵随后赶来,向前扶起,急救得。朱仝答道:“黑影
里不见路径,失脚走下野田里,滑倒了,闪挫了左腿。”县尉道:“走了正贼,怎
生奈何!”朱仝道:“非是小人不赶,其实月黑了,没做道理处。这些土兵,全无
几个有用的人,不敢向前。”县尉再叫土兵去赶,众土兵心里道:“两个都头,尚
兀自不济事,近他不得,我们有何用?”都去虚赶了一回,转来道:“黑地里正不
知那条路去了。”雷横也赶了一直回来,心内寻思道:“朱仝和晁盖最好,多敢是
放了他去,我没来由做甚么恶人。我也有心亦要放他,今已去了,只是不见了人情。
晁盖那人,也不是好惹的。”回来说道:“那里赶得上?这伙贼端的了得!”县尉
和两个都头回到庄前时,已是四更时分。何观察见众人四分五落,赶了一夜,不曾
拿得一个贼人,只叫苦道:“如何回得济州去见府尹!”县尉只得捉了几家邻舍去,
解将郓城县里来。

这时知县一夜不曾得睡,立等回报,听得道:“贼都走了,只拿得几个邻舍。”
知县把一干拿到的邻舍,当厅勘问。众邻舍告道:“小人等虽在晁保正邻近住居,
远者三二里田地,近者也隔着些村坊。他庄上如常有搠枪使棒的人来,如何知他做
这般的事?”知县逐一问了时,务要问他们一个下落。数内一个贴邻告道:“若要
知他端的,除非问他庄客。”知县道:“说他家庄客,也都跟着走了。”邻舍告道:
“也有不愿去的,还在这里。”知县听了,火速差人,就带了这个贴邻做眼,来东
溪村捉人。无两个时辰,早拿到两个庄客。当厅勘问时,那庄客初时抵赖,吃打不
过,只得招道:“先是六个人商议,小人只认得一个,是本乡中教学的先生,叫做
吴学究;一个叫做公孙胜,是全真先生;又有一个黑大汉,姓刘。更有那三个,小
人不认得,却是吴学究合将来的。听的说道:‘他姓阮,在石碣村住。他是打鱼的,
弟兄三个。’只此是实。”知县取了一纸招状,把两个庄客交割与何观察,回了一
道备细公文,申呈本府。宋江自周全那一干邻舍,保放回家听候。

且说这众人与何涛押解了两个庄客,连夜回到济州,正值府尹升厅。何涛引了
众人到厅前,禀说晁盖烧庄在逃一事,再把庄客口词说一遍。府尹道:“既是恁地
说时,再拿出白胜来!”问道:“那三个姓阮的,端的住在那里?”白胜抵赖不过,
只得供说:“三个姓阮的,一个叫做立地太岁阮小二,一个叫做短命二郎阮小五,
一个是活阎罗阮小七。都在石碣湖村里住。”知府道:“还有那三个姓甚么?”白
胜告道:“一个是智多星吴用,一个是入云龙公孙胜,一个叫做赤发鬼刘唐。”知
府听了,便道:“既有下落,且把白胜依原监了,收在牢里。”随即又唤何观察,
差去石碣村,缉捕这几个贼人。不是何涛去石碣村去,有分教:天罡地煞,来寻际
会风云;水浒山城,去聚纵横人马。

毕竟何观察怎生差去石碣村缉捕,且听下回分解。