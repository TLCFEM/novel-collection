\chapter{放冷箭燕青救主~劫法场石秀跳楼}

话说这卢俊义虽是了得,却不会水,被浪里白跳张顺排翻了船,倒撞下水去。
张顺却在水底下拦腰抱住,又钻过对岸来,抢了朴刀。张顺把卢俊义直奔岸边来,
早点起火把,有五六十人在那里等。接上岸来,团团围住,解了腰刀,尽脱下湿衣
服,便要将索绑缚。只见神行太保戴宗传令,高叫将来:“不得伤犯了卢员外贵体!”
随即差人将一包袱锦衣绣袄,与卢俊义穿着。八个小喽罗,抬过一乘轿来,扶卢员
外上轿便行。只见远远地早有二三十对红纱灯笼,照着一簇人马,动着鼓乐,前来
迎接。为头宋江、吴用、公孙胜,后面都是众头领,一齐下马。卢俊义慌忙下轿。
宋江先跪,后面众头领排排地都跪下。卢俊义亦跪下还礼道:“既被擒捉,愿求早
死!”宋江大笑,说道:“且请员外上轿。”众人一齐上马,动着鼓乐,迎上三关,
直到忠义堂前下马,请卢俊义到厅上,明晃晃地点着灯烛。宋江向前陪话道:“小
可久闻员外大名,如雷贯耳,今日幸得拜识,大慰平生。却才众兄弟甚是冒渎,万
乞恕罪。”吴用上前说道:“昨奉兄长之命,特令吴某亲诣门墙,以卖卦为由,赚
员外上山,共聚大义,一同替天行道。”宋江便请卢员外坐第一把交椅。卢俊义答
礼道:“不才无识无能,误犯虎威,万死尚轻,何故相戏?”宋江陪笑道:“怎敢
相戏。实慕员外威德,如饥如渴。万望不弃鄙处,为山寨之主,早晚共听严命。”
卢俊义回说:“宁就死亡,实难从命。”吴用道:“来日却又商议。”当时置备酒
食管待。卢俊义无计奈何,只得饮了几杯,小喽罗请去后堂歇了。

次日,宋江杀羊宰马,大排筵宴,请出卢员外来赴席,再三再四谦让,在中间
里坐了。酒至数巡,宋江起身把盏,陪话道:“夜来甚是冲撞,幸望宽恕。虽然山
寨窄小,不堪歇马,员外可看‘忠义’二字之面。宋江情愿让位,休得推却。”卢
俊义答道:“头领差矣!小可身无罪累,颇有些少家私。生为大宋人,死为大宋鬼,
宁死实难听从。”吴用并众头领一个个说,卢俊义越不肯落草。吴用道:“员外既
然不肯,难道逼勒?只留得员外身,留不得员外心。只是众弟兄难得员外到此,既
然不肯入伙,且请小寨略住数日,却送还宅。”卢俊义道:“小可在此不妨,只恐
家中老小,不知这般的消息。”吴用道:“这事容易,先教李固送了车仗回去,员
外迟去几日,却何妨?”吴用问道:“李都管,你的车仗货物都有么?”李固应道:
“一些儿不少。”宋江叫取两个大银,把与李固;两个小银,打发当直的;那十个
车脚,共与他白银十两。众人拜谢。卢俊义分付李固道:“我的苦,你都知了。你
回家中,说与娘子,不要忧心,我过三五日,便回也。”李固只要脱身,满口应说:
“但不妨事。”辞了,便下忠义堂去。吴用随即便起身,说道:“员外宽心少坐,
小生发送李都管下山,便来也。”

吴用只推发送李固,却先到金沙滩等候。少刻,李固和两个当直的,并车仗、
头口、人伴,都下山来。吴用将引五百小喽罗围在两边,坐在柳阴树下,便唤李固
近前说道:“你的主人,已和我们商议定了,今坐第二把交椅。此乃未曾上山时,
预先写下四句反诗,在家里壁上。我教你们知道:壁上二十八个字,每一句包着一
个字。‘芦花荡里一扁舟’,包个‘卢’字;‘俊杰那能此地游’,包个‘俊’字;
‘义士手提三尺剑’,包个‘义’字;‘反时须斩逆臣头’,包个‘反’字。这四
句诗,包藏‘卢俊义反’四字。今日上山,你们怎知?本待把你众人杀了,显得我
梁山泊行短。今日放你们星夜自回去,休想望你主人回来!”李固等只顾下拜。吴
用教把船送过渡口,一行人上路,奔回北京。正是:鳌鱼脱却金钩去,摆尾摇头更
不回。

话分两处。不说李固等归家,且说吴用回到忠义堂上,再入酒席,用巧言说诱
卢俊义。筵会直到二更方散。次日,山寨里再排筵会庆贺,卢俊义说道:“感承众
头领好意相留,只是小可度日如年,今日告辞。”宋江道:“小可不才,幸识员外,
来日宋江体己聊备小酌,对面论心一会,勿请推却。”又过了一日。明日宋江请,
后日吴用请,大后日公孙胜请。话休絮繁,三十余个上厅头领,每日轮一个做筵席。
光阴荏苒,日月如梭,早过一月有余。卢俊义寻思,又要告别。宋江道:“非是不
留员外,争奈急急要回;来日忠义堂上,安排薄酒送行。”

次日,宋江又体己送路,只见众头领都道:“俺哥哥敬员外十分,俺等众人当
敬员外十二分!偏我哥哥筵席便吃,‘砖儿何厚,瓦儿何薄!’”李逵在内大叫道:
“我舍着一条性命,直往北京请得你来,却不吃我弟兄们筵席,我和你眉尾相结,
性命相扑!”吴学究大笑道:“不曾见这般请客的,甚是粗卤,员外休怪。见他众
人薄意,再住几时。”不觉又过了四五日。卢俊义坚意要行,只见神机军师朱武,
将引一班头领直到忠义堂上,开话道:“我等虽是以次弟兄,也曾与哥哥出气力,
偏我们酒中藏着毒药?卢员外若是见怪,不肯吃我们的,我自不妨,只怕小兄弟们
做出事来,悔之晚矣。”吴用起身便道:“你们都不要烦恼,我与你央及员外,再
住几时,有何不可。常言道:‘将酒劝人,终无恶意。’”卢俊义抑众人不过,只
得又住了几日,前后却好三五十日。自离北京,是五月的话,不觉在梁山泊早过了
两个多月。但见金风淅淅,玉露泠泠,又早是中秋节近。卢俊义思想归期,对宋江
诉说。宋江见卢俊义思归苦切,便道:“这个容易,来日金沙滩送别。”卢俊义大
喜。有诗为证:
一别家山岁月赊,寸心无日不思家。
此身恨不生双翼,欲借天风过水涯。

次日,还把旧时衣裳刀棒,送还员外,一行众头领都送下山。宋江把一盘金银
相送。卢俊义推道:“非是卢某说口,金帛钱财,家中颇有,但得到北京盘缠足矣。
赐与之物,决不敢受。”宋江等众头领直送过金沙滩,作别自回,不在话下。

不说宋江回寨,只说卢俊义拽开脚步,星夜奔波。行了旬日,到得北京,日已
薄暮。赶不入城,就在店中歇了一夜。次日早晨,卢俊义离了村店,飞奔入城。尚
有一里多路,只见一人头巾破碎,衣裳蓝褛,看着卢俊义,纳头便拜。卢俊义抬眼
看时,却是浪子燕青。便问:“小乙,你怎地这般模样?”燕青道:“这里不是说
话处。”卢俊义转过土墙侧首,细问缘故。燕青说道:“自从主人去后,不过半月,
李固回来,对娘子说道:‘主人归顺了梁山泊宋江,坐了第二把交椅。’当时便去
官司首告了。他已和娘子做了一路,嗔怪燕青违拗,将我赶逐出门。将一应衣服尽
行夺了,赶出城外。更兼分付一应亲戚相识:但有人安着燕青在家歇的,他便舍半
个家私,和他打官司,因此无人敢着小乙。在城中安不得身,只得来城外求乞度日,
权在庵内安身。正要往梁山泊寻见主人,又不敢造次。若主人果自泊里来,可听小
乙言语,再回梁山泊去,别做个商议。若入城中,必中圈套。”卢俊义喝道:“我
的娘子不是这般人,你这厮休来放屁!”燕青又道:“主人脑后无眼,怎知就里?
主人平昔只顾打熬气力,不亲女色,娘子旧日和李固原有私情,今日推门相就,做
了夫妻。主人若去,必遭毒手!”卢俊义大怒,喝骂燕青道:“我家五代在北京住,
谁不识得?量李固有几颗头,敢做恁般勾当?莫不是你做出歹事来,今日倒来反说!
我到家中问出虚实,必不和你干休!”燕青痛哭,拜倒地下,拖住主人衣服。卢俊
义一脚踢倒燕青,大踏步便入城来。

奔到城内,径入家中,只见大小主管都吃一惊。李固慌忙前来迎接,请到堂上,
纳头便拜。卢俊义便问:“燕青安在?”李固答道:“主人且休问,端的一言难尽!
只怕发怒,待歇息定了却说。”贾氏从屏风后哭将出来,卢俊义说道:“娘子休哭,
且说燕小乙怎地来。”贾氏道:“丈夫且休问,慢慢地却说。”卢俊义心中疑虑,
定死要问燕青来历,李固便道:“主人且请换了衣服,吃了早膳,那时诉说不迟。”
一边安排饭食与卢员外吃。方才举箸,只听得前门后门喊声齐起,二三百个做公的
抢将入来。卢俊义惊得呆了,就被做公的绑了,一步一棍,直打到留守司来。

其时梁中书正坐公厅,左右两行,排列狼虎一般公人七八十个,把卢俊义拿到
当面,贾氏和李固也跪在侧边。厅上梁中书大喝道:“你这厮是北京本处百姓良民,
如何却去投降梁山泊落草,坐了第二把交椅?如今倒来里勾外连,要打北京!今被擒
来,有何理说?”卢俊义道:“小人一时愚蠢,被梁山泊吴用,假做卖卦先生来家,
口出讹言,煽惑良心,掇赚到梁山泊,软监了两个多月。今日幸得脱身归家,并无
歹意,望恩相明镜。”梁中书喝道:“如何说得过!你在梁山泊中,若不通情,如
何住了许多时?现放着你的妻子并李固告状出首,怎地是虚?”李固道:“主人既
到这里,招伏了罢。家中壁上现写下藏头反诗,便是老大的证见,不必多说。”贾
氏道:“不是我们要害你,只怕你连累我。常言道:‘一人造反,九族全诛。’”
卢俊义跪在厅下,叫起屈来。李固道:“主人不必叫屈,是真难灭,是假易除。早
早招了,免致吃苦。”贾氏道:“丈夫,虚事难入公门,实事难以抵对。你若做出
事来,送了我的性命。不奈有情皮肉,无情杖子。你便招了,也只吃得有数的官司。”
李固上下都使了钱,张孔目厅上禀说道:“这个顽皮赖骨,不打如何肯招?”梁中
书道:“说的是!”喝叫一声:“打!”左右公人把卢俊义捆翻在地,不由分说,
打的皮开肉绽,鲜血迸流,昏晕去了三四次。卢俊义打熬不过,仰天叹曰:“是我
命中合当横死,我今屈招了罢。”张孔目当下取了招状,讨一面一百斤死囚枷钉了,
押去大牢里监禁。府前府后看的人,都不忍见。当日推入牢门,吃了三十杀威棒,
押到庭心内,跪在面前。狱子炕上坐着那个两院押牢节级——带管刽子,把手指道:
“你认的我么?”卢俊义看了,不敢则声。那人是谁,有诗为证:
两院押牢称蔡福,堂堂仪表气凌云。
腰间紧系青鸾带,头上高悬垫角巾。
行刑问事人倾胆,使索施枷鬼断魂。
满郡夸称铁臂膊,杀人到处显精神。

这两院押狱兼充行刑刽子,姓蔡,名福,北京土居人氏。因为他手段高强,人
呼他为铁臂膊。旁边立着一个嫡亲兄弟,叫做蔡庆,亦有诗为证:
押狱丛中称蔡庆,眉浓眼大性刚强。
茜红衫上描,茶褐衣中绣木香。
曲曲领沿深染皂,飘飘博带浅涂黄。
金环灿烂头巾小,一朵花枝插鬓旁。
这个小押狱蔡庆,生来爱带一枝花,河北人顺口,都叫他做一枝花蔡庆。那人拄着
一条水火棍,立在哥哥侧边。蔡福道:“你且把这个死囚带在那一间牢里,我家去
走一遭便来。”蔡庆把卢俊义自带去了。

蔡福起身,出离牢门来,只见司前墙下转过一个人来,手里提个饭罐,面带忧
容。蔡福认的是浪子燕青。蔡福问道:“燕小乙哥,你做甚么?”燕青跪在地下,
擎着两行眼泪,告道:“节级哥哥,可怜见小人的主人卢员外吃屈官司,又无送饭
的钱财!小人城外叫化得这半罐子饭,权与主人充饥。节级哥哥,怎地做个方便。”
说罢,泪如雨下,拜倒在地。蔡福道:“我知此事,你自去送饭,把与他吃。”燕
青拜谢了,自进牢里去送饭。

蔡福转过州桥来,只见一个茶博士,叫住唱喏道:“节级,有个客人在小人茶
房内楼上,专等节级说话。”蔡福来到楼上看时,却是主管李固。各施礼罢,蔡福
道:“主管有何见教?”李固道:“奸不厮瞒,俏不厮欺,小人的事,都在节级肚
里。今夜晚间,只要光前绝后。无甚孝顺,五十两蒜条金在此,送与节级。厅上官
吏,小人自去打点。”蔡福笑道:“你不见正厅戒石上,刻着‘下民易虐,上苍难
欺’。你那瞒心昧己勾当,怕我不知!你又占了他家私,谋了他老婆,如今把五十
两金子与我,结果了他性命。日后提刑官下马,我吃不的这等官司。”李固道:“只
是节级嫌少,小人再添五十两。”蔡福道:“李固,你割猫儿尾,拌猫儿饭!北京
有名恁地一个卢员外,只值得这一百两金子?你若要我倒地他,不是我诈你,只把
五百两金子与我。”李固便道:“金子有在这里,便都送与节级,只要今夜晚些成
事。”蔡福收了金子,藏在身边,起身道:“明日早来扛尸。”李固拜谢,欢喜去
了。

蔡福回到家里,却才进门,只见一人揭起芦帘,随即入来。那人叫声:“蔡节
级相见。”蔡福看时,但见那一个人生得十分标致,且是打扮得整齐:

身穿鸦翅青团领,腰系羊脂玉闹妆,头带冠,足蹑珍珠履。

那人进得门,看着蔡福便拜。蔡福慌忙答礼,便问道:“官人高姓?有何见教?”
那人道:“可借里面说话。”蔡福便请入来一个商议阁里,分宾坐下。那人开话道:
“节级休要吃惊。在下便是沧州横海郡人氏,姓柴,名进,大周皇帝嫡派子孙,绰
号小旋风的便是。只因好义疏财,结识天下好汉,不幸犯罪,流落梁山泊。今奉宋
公明哥哥将令,差遣前来打听卢员外消息。谁知被赃官污吏、淫妇奸夫通情陷害,
监在死囚牢里,一命悬丝,尽在足下之手。不避生死,特来到宅告知:如是留得卢
员外性命在世,佛眼相看,不忘大德;但有半米儿差错,兵临城下,将至濠边,无
贤无愚,无老无幼,打破城池,尽皆斩首。久闻足下是个仗义全忠的好汉,无物相
送,今将一千两黄金薄礼在此。倘若要捉柴进,就此便请绳索,誓不皱眉。”

蔡福听罢,吓得一身冷汗,半晌答应不的。柴进起身道:“好汉做事,休要踌
躇,便请一决。”蔡福道:“且请壮士回步,小人自有措置。”柴进便拜道:“既
蒙语诺,当报大恩。”出门唤个从人,取出黄金,递与蔡福,唱个喏便走。外面从
人,乃是神行太保戴宗,又是一个不会走的。

蔡福得了这个消息,摆拨不下,思量半晌,回到牢中,把上项的事,却对兄弟
说了一遍。蔡庆道:“哥哥生平最会断决,量这些小事,有何难哉?常言道:‘杀
人须见血,救人须救彻。’既然有一千两金子在此,我和你替他上下使用。梁中书、
张孔目,都是好利之徒,接了贿赂,必然周全卢俊义性命。葫芦提配将出去,救得
救不得,自有他梁山泊好汉,俺们干的事便了也。”蔡福道:“兄弟这一论,正合
我意。你且把卢员外安顿好处,早晚把些好酒食将息他,传个消息与他。”蔡福、
蔡庆两个商议定了,暗地里把金子买上告下,关节已定。

次日,李固不见动静,前来蔡福家催并。蔡庆回说:“我们正要下手结果他,
中书相公不肯,已有人分付,要留他性命。你自去上面使用,嘱付下来,我这里何
难?”李固随即又央人去上面使用。中间过钱人去嘱托,梁中书道:“这是押牢节
级的勾当,难道教我下手?过一两日,教他自死。”两下里厮推,张孔目已得了金
子,只管把文案拖延了日期,蔡福就里又打关节,教及早发落。张孔目将了文案来
禀,梁中书道:“这事如何决断?”张孔目道:“小吏看来,卢俊义虽有原告,却
无实迹。虽是在梁山泊住了许多时,这个是扶同诖误,难问真犯。脊杖四十,刺配
三千里。不知相公意下如何?”梁中书道:“孔目见得极明,正与下官相合。”随
唤蔡福牢中取出卢俊义来,就当厅除了长枷,读了招状文案,决了四十脊杖。换一
具二十斤铁叶盘头枷,就厅前钉了,便差董超、薛霸管押前去,直配沙门岛。原来
这董超、薛霸,自从开封府做公人,押解林冲去沧州,路上害不得林冲,回来被高
太尉寻事刺配北京。梁中书因见他两个能干,就留在留守司勾当。今日又差他两个
监押卢俊义。

当下董超、薛霸领了公文,带了卢员外,离了州衙,把卢俊义监在使臣房里,
各自归家,收拾行李包裹,即便起程。诗曰:
不亲女色丈夫身,为甚离家忆内人?
谁料室中狮子吼,却能断送玉麒麟!

且说李固得知,只叫得苦,便叫人来请两个防送公人说话。董超、薛霸到得那
里酒店内,李固接着,请至阁儿里坐下,一面铺排酒食管待。三杯酒罢,李固开言
说道:“实不相瞒:卢员外是我仇家。如今配去沙门岛,路途遥远,他又没一文,
教你两个空费了盘缠。急待回来,也得三四个月。我没甚的相送,两锭大银,权为
压手。多只两程,少无数里,就僻静去处,结果了他性命,揭取脸上金印回来表证,
教我知道,每人再送五十两蒜条金与你。你们只动得一张文书;留守司房里,我自
理会。”董超、薛霸两两相觑,沉吟了半晌,见了两个大银,如何不起贪心。董超
道:“只怕行不得。”薛霸便道:“哥哥,这李官人也是个好男子,我们也把这件
事结识了他。若有急难之处,要他照管。”李固道:“我不是忘恩失义的人,慢慢
地报答你两个。”

董超、薛霸收了银子,相别归家,收拾包裹,连夜起身。卢俊义道:“小人今
日受刑,杖疮疼痛,容在明日上路。”薛霸骂道:“你便闭了鸟嘴!老爷自晦气,
撞着你这穷神!沙门岛往回六千里有余,费多少盘缠,你又没一文,教我们如何布
摆!”卢俊义诉道:“念小人负屈含冤,上下看觑则个。”董超骂道:“你这财主
们,闲常一毛不拔;今日天开眼,报应得快!你不要怨怅,我们相帮你走。”卢俊
义忍气吞声,只得走动。行出东门,董超、薛霸把衣包雨伞,都挂在卢员外枷头上。
卢员外一生财主,今做了囚人,无计奈何。那堪又值晚秋天气,纷纷黄叶坠,对对
塞鸿飞,忧闷之中,只听的横笛之声。正是:
谁家玉笛弄秋清,撩乱无端恼客情。
自是断肠听不得,非干吹出断肠声。

两个公人,一路上做好做恶,管押了行。看看天色傍晚,约行了十四五里,前
面一个村镇,寻觅客店安歇。当时小二哥引到后面房里,安放了包裹。薛霸说道:
“老爷们苦杀是个公人,那里倒来伏侍罪人?你若要饭吃,快去烧火!”卢俊义只
得带着枷来到厨下,问小二哥讨了个草柴,缚做一块,来灶前烧火。小二哥替他淘
米做饭,洗刷碗盏。卢俊义是财主出身,这般事却不会做。草柴火把又湿,又烧不
着,一齐灭了。甫能尽力一吹,被灰眯了眼睛。董超又喃喃讷讷地骂。做得饭熟,
两个都盛去了,卢俊义并不敢讨吃。两个自吃了一回,剩下些残汤冷饭,与卢俊义
吃了。薛霸又不住声骂了一回。吃了晚饭,又叫卢俊义去烧脚汤。等得汤滚,卢俊
义方敢去房里坐地。两个自洗了脚,掇一盆百煎滚汤,赚卢俊义洗脚。方才脱得草
鞋,被薛霸扯两条腿纳在滚汤里,大痛难禁。薛霸道:“老爷伏侍你,颠倒做嘴脸!”
两个公人自去炕上睡了,把一条铁索,将卢员外锁在房门背后,声唤到四更,两个
公人起来,叫小二哥做饭。自吃饱了,收拾包裹要行。卢俊义看脚时,都是潦浆泡,
点地不得。当日秋雨纷纷,路上又滑。卢俊义一步一,薛霸拿起水火棍,拦腰便
打,董超假意去劝,一路上埋冤叫苦。

离了村店,约行了十余里,到一座大林。卢俊义道:“小人其实捱不动了,可
怜见,权歇一歇!”两个公人带入林子来,正是东方渐明,未有人行。薛霸道:“我
两个起得早了,好生困倦,欲要就林子里睡一睡,只怕你走了。”卢俊义道:“小
人插翅也飞不去。”薛霸道:“莫要着你道儿,且等老爷缚一缚。”腰间解下麻索
来,兜住卢俊义肚皮,去那松树上只一勒,反拽过脚来,绑在树上。薛霸对董超道:
“大哥,你去林子外立着,若有人来撞着,咳嗽为号。”董超道:“兄弟,放手快
些个。”薛霸道:“你放心去看着外面。”说罢,拿起水火棍,看着卢员外道:“你
休怪我两个。你家主管李固,教我们路上结果你。便到沙门岛,也是死,不如及早
打发了你!阴司地府,不要怨我们。明年今日,是你周年。”

卢俊义听了,泪如雨下,低头受死。薛霸两只手拿起水火棍,望着卢员外脑门
上劈将下来。董超在外面,只听得一声扑地响,慌忙走入林子里来看时,卢员外依
旧缚在树上,薛霸倒仰卧树下,水火棍撇在一边。董超道:“却又作怪!莫不是他
使的力猛,倒吃一交?”仰着脸四下里看时,不见动静。薛霸口里出血,心窝里露
出三四寸长一枝小小箭杆。却待要叫,只见东北角树上坐着一个人。听的叫声:“着!”
撒手响处,董超脖项上早中了一箭,两脚蹬空,扑地也倒了。那人托地从树上跳将
下来,拔出解腕尖刀,割断绳索,劈碎盘头枷,就树边抱住卢员外,放声大哭。卢
俊义开眼看时,认得是浪子燕青,叫道:“小乙,莫不是魂魄和你相见么?”燕青
道:“小乙直从留守司前跟定这厮两个。见他把主人监在使臣房里,又见李固请去
说话,小乙疑猜这厮们要害主人,连夜直跟出城来。主人在村店里时,小乙伏侍在
外头,比及五更里起来,小乙先在这里等候。想这厮们必来这林子里下手。被我两
弩箭结果了他两个,主人见么?”这浪子燕青那把弩弓,三枝快箭,端的是百发百
中。怎见得弩箭好处:

弩桩劲裁乌木,山根对嵌红牙。拨手轻衬水晶,弦索半抽金线。背缠锦袋,弯
弯如秋月未圆;稳放雕翎,急急似流星飞迸。
卢俊义道:“虽是你强救了我性命,却射死这两个公人,这罪越添得重了,待走那
里去的是?”燕青道:“当初都是宋公明苦了主人,今日不上梁山泊时,别无去处。”
卢俊义道:“只是我杖疮发作,脚皮破损,点地不得。”燕青道:“事不宜迟,我
背着主人去。”便去公人身边,搜出银两,带着弩弓,插了腰刀,拿了水火棍,背
着卢俊义,一直望东边行走。不到十数里,早驮不动。见一个小小村店,入到里面,
寻房安下,买些酒肉,权且充饥,两个暂时安歇这里。

却说过往人看见林子里射死两个公人在彼,近处社长,报与里正得知,却来大
名府里首告。随即差官下来检验,却是留守司公人董超、薛霸。回复梁中书,着落
大名府缉捕观察,限了日期,要捉凶身。做公的人,都来看了。论这弩箭,眼见得
是浪子燕青的。事不宜迟,一二百做公的分头去到处贴了告示,说那两个模样,晓
谕远近村坊道店,市镇人家,挨捕捉拿。却说卢俊义正在村店房中将息杖疮,又走
不动,只得在那里且住。店小二听得有杀人公事,村坊里排头说来,画两个模样,
小二见了,连忙去报本处社长:“我店里有两个人,好生脚叉,不知是也不是。”
社长转报做公的去了。

却说燕青为无下饭,拿了弩子,去近边处寻几个虫蚁吃;却待回来,只听得满
村里发喊。燕青躲在树林里张时,看见一二百做公的,枪刀围定,把卢俊义缚在车
子上,推将过去。燕青要抢出去救时,又无军器,只叫得苦,寻思道:“若不去梁
山泊报与宋公明得知,叫他来救,却不是我误了主人性命?”

当时取路,行了半夜,肚里又饥,身边又没一文。走到一个土冈子上,丛丛杂
杂,有些树木,就林子里睡到天明,心中忧闷,只听得树枝上喜雀噪噪,寻思
道:“若是射得下来,村坊人家,讨些水,煮瀑得熟,也得充饥。”走出林子外,
抬头看时,那喜雀朝着燕青噪。燕青轻轻取出弩弓,暗暗问天买卦,望空祈祷,说
道:“燕青只有这一只箭了。若是救的主人性命,箭到处,灵雀坠空;若是主人命
运合休,箭到,灵雀飞去。”搭上箭,叫声:“如意子,不要误我!”弩子响处,
正中喜雀后尾,带了那枝箭,直飞下冈子去。燕青大踏步赶下冈子去,不见了喜雀。
正寻之间,只见两个人从前面走来。怎生打扮,但见:

前头的,带顶猪嘴头巾,脑后两个金裹银环,上穿香皂罗
衫,腰系销金膊。穿半膝软袜麻鞋,提一条齐眉棍棒。后面的,白范阳遮尘笠子,
茶褐攒线袖衫。腰系绯红缠袋,脚穿踢土皮鞋。背了衣包,提条短棒,跨口腰刀。
这两个来的人,正和燕青打个肩厮拍。燕青转回身,看了这两个,寻思道:“我正
没盘缠,何不两拳打倒两个,夺了包裹,却好上梁山泊。”揣了弩弓,抽身回来。
这两个低着头只顾走。燕青赶上,把后面带毡笠儿的后心一拳,扑地打倒。却待拽
拳再打那前面的,反被那汉子手起棒落,正中燕青左腿,打翻在地。后面那汉子爬
将起来,踏住燕青,掣出腰刀,劈面门便剁。燕青大叫道:“好汉,我死不妨,却
谁为主人报信!”那汉便不下刀,收住了手,提起燕青问道:“你这厮报甚么音信?”
燕青道:“你问我待怎地?”那前面的好汉把燕青手一拖,却露出手腕上花绣,慌
忙问道:“你不是卢员外家甚么浪子燕青?”燕青想道:“左右是死,索性说了,
教他捉去,和主人阴魂做一处!”便道:“我正是卢员外家浪子燕青。今要上梁山
泊报信,教宋公明救我主人则个。”二人见说,呵呵大笑,说道:“早是不杀了你!
原来正是燕小乙哥。你认得我两个么?”穿皂的不是别人,梁山泊头领病关索杨雄,
后面的便是拚命三郎石秀。杨雄道:“我两个今奉哥哥将令,差往北京,打听卢员
外消息。军师与戴院长亦随后下山,专候通报。”燕青听得是杨雄、石秀,把上件
事都对两个说了。杨雄道:“既是如此说时,我和燕青上山寨,报知哥哥,别做个
道理。你可自去北京,打听消息,便来回报。”石秀道:“最好。”便把包裹与燕
青背了,跟着杨雄连夜上梁山泊来。见了宋江,燕青把上项事备细说了一遍。宋江
大惊,便会众头领商议良策。

且说石秀只带自己随身衣服,来到北京城外,天色已晚,入不得城,就城外歇
了一宿。次日早饭罢,入得城来,但见人人嗟叹,个个伤情。石秀心疑。来到市心
里,只见人家闭户关门,石秀问市户人家时,只见一个老丈回言道:“客人,你不
知我这北京有个卢员外,等地财主,因被梁山泊贼人掳掠前去,逃得回来,倒吃了
一场屈官司,迭配去沙门岛。又不知怎地路上坏了两个公人,昨夜拿来,今日午时
三刻,解来这里市曹上斩他,客人可看一看。”

石秀听罢,走来市曹上看时,十字路口,是个酒楼,石秀便来酒楼上,临街占
个阁儿坐了。酒保前来问道:“客官,还是请人?只是独自酌杯?”石秀睁着怪眼
说道:“大碗酒,大块肉,只顾卖来,问甚么鸟!”酒保倒吃了一惊,打两角酒,
切一大盘牛肉将来。石秀大碗大块,吃了一回。坐不多时,只听得楼下街上热闹,
石秀便去楼窗外看时,只见家家闭户,铺铺关门。酒保上楼来道:“客官醉也!楼
下出公事,快算了酒钱,别处去回避。”石秀道:“我怕甚么鸟!你快走下去,莫
要讨老爷打!”酒保不敢做声,下楼去了。不多时,只见街上锣鼓喧天价来。但见:

两声破鼓响,一棒碎锣鸣。皂纛旗招展如云,柳叶枪交加似雪。犯由牌前引,
白混棍后随。押牢节级狰狞,仗刃公人猛勇。高头马上,监斩官胜似活阎罗;刀剑
林中,掌法吏犹如追命鬼。可怜十字街心里,要杀含冤负屈人!

石秀在楼窗外看时,十字路口,周回围住法场,十数对刀棒刽子,前排后拥,
把卢俊义绑押到楼前跪下。铁臂膊蔡福拿着法刀,一枝花蔡庆扶着枷梢,说道:“卢
员外,你自精细看,不是我弟兄两个救你不的,事做拙了。前面五圣堂里,我已安
排下你的坐位了,你可一魂去那里领受。”说罢,人丛里一声叫道:“午时三刻到
了!”一边开枷,蔡庆早拿住了头,蔡福早掣出法刀在手。当案孔目高声读罢犯由
牌,众人齐和一声。楼上石秀,只就那一声和里,掣着腰刀在手,应声大叫:“梁
山泊好汉全伙在此!”蔡福、蔡庆撇了卢员外,扯了绳索先走。石秀从楼上跳将下
来,手举钢刀,杀人似砍瓜切菜,走不迭的,杀翻十数个。一只手拖住卢俊义,投
南便走。

原来这石秀不认得北京的路,更兼卢员外惊得呆了,越走不动。梁中书听得报
来,大惊,便点帐前头目,引了人马,分头去把城四门关上;差前后做公的,合将
拢来。随你好汉英雄,怎出高城峻垒?正是:分开陆地无牙爪,飞上青天欠羽毛。

毕竟卢员外同石秀当下怎地脱身,且听下回分解。