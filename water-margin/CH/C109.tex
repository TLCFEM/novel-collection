\chapter{王庆渡江被捉~宋江剿寇成功}

话说当日宋江升帐,诸将拱立听调。放炮、鸣金鼓、升旗,随放静营炮,各营
哨头目,挨次至帐下,齐立肃静,听施号令。吹手点鼓,宣令官传令毕,营哨头目,
依次磕头,起站两边。巡视蓝旗手,跪听发放,凡呐喊不齐,行伍错乱,喧哗违令,
临阵退缩,拿来重处。又有旗牌官左右各二十员,宋先锋亲谕:“尔等下营督阵,
凡有军士遇敌不前,退缩不用命者,听你等拿来处治。”旗牌遵令,各下地方,鸣
金大吹,各归行伍,听令起行。宋江然后传令,遣调水陆诸将毕。吹手掌头号整队,
二号掣旗,三号各起行营向敌。敲金边,出五方旗,放大炮,掌号营,各各摆阵
出战。正是那:
震天鼙鼓摇山岳,映日旌旗避鬼神。
却说贼人王庆,调拨军兵抵敌,除水军将士闻人世崇等已差拨外,点差云安州伪兵
马都监刘以敬为正先锋,东川伪兵马都监上官义为副先锋,南丰伪统军李雄、毕先
为左哨,安德伪统军柳元、潘忠为右哨,伪统军大将段五为正合后,伪御营使丘翔
为副合后,伪枢密方翰为中军羽翼。王庆掌握中军,有许多伪尚书、御营金吾、卫
驾将军、校尉等项,及各人手下偏牙将佐,共数十员。李助为元帅。队伍军马,十
分齐整。王庆亲自监督。马带皮甲,人披铁铠,弓弩上弦,战鼓三通,诸军尽起。
行不过十里之外,尘土起处,早有宋军哨路来的渐近。鸾铃响处,约有三十余骑哨
马,都戴青将巾,各穿绿战袍,马上尽系着红缨,每边拴挂数十个铜铃,后插一把
雉尾,都是钏银细杆长枪,轻弓短箭。为头的战将是奉道君皇帝敕命,复还旧职,
虎骑将军没羽箭张清。头裹销金青巾帻,身穿挑绣绿战袍,腰系紫绒绦,足穿软香
皮,骑匹银鞍马。左边是敕封贞孝宜人的琼矢镞琼英,头带紫金嵌珠凤冠,身穿紫
罗挑绣战袍,腰系杂色彩绒绦,足穿朱绣小凤头鞋,坐匹银鬃骏马。那右边略下些,
捧旗的是敕授的义仆正排军叶清,直哨到李助军前,相离不远,只隔百十步,勒马
便回。前军先锋刘以敬、上官义骤马驱兵,便来冲击。张清拍马,拈出白梨花枪,
来战二将。琼英驰马,挺方天画戟来助战。

四将斗到十数合,张清、琼英隔开贼将兵器,拨马便回。刘以敬、上官义驱兵赶来,
左右高叫:“先锋不可追赶!此二人鞍后锦袋中都是石子,打人不曾放空!”刘以
敬、上官义听说,方才勒住得马,只见龙门山背后,鼓声振响,早转五百步兵来。
当先四个步将头领,乃是黑旋风李逵、混世魔王樊瑞、八臂那吒项充、飞天大圣李
衮,直奔前来。那五百步军,就在山坡下一字儿摆开,两边团牌,齐齐扎住。刘以
敬、上官义驱兵掩杀。李逵、樊瑞引步军分开两路,都倒提蛮牌,转过山坡便去。
那时王庆、李助大军已到,一齐冲击前来。李逵、樊瑞等都飞跑上山,度岭穿林,
都不见了。

李助传令,教就把军马在这个平原旷野之地列成阵势。只听得山后炮响,只见山南
一路军马飞涌出来,簇拥着三个将军:中间是矮脚虎王英,左是小尉迟孙新,右是
菜园子张青。总管马步军兵五千,杀向前来。王庆正欲遣将迎敌,又听得山后一声
炮响,山北一路军马飞涌出来,簇拥着三个女将:中间是一丈青扈三娘,左边是母
大虫顾大嫂,右边是母夜叉孙二娘。管领马步军兵五千,杀向前来,恰遇贼兵右哨
柳元、潘忠兵马,接住厮杀。王英等正遇贼兵左哨李雄、毕先军马,接住厮杀。两
边各斗到十余合,南边王英、孙新、张青勒转马,领兵望东便走;北边扈三娘、顾
大嫂、孙二娘,也接转马匹,率领军兵,望东便走。王庆看了笑道:“宋江手下,
都是这些鸟男女,我这里将士如何屡次输了?”遂驱大兵,追杀上来。

行不到五六里,忽听得一棒锣声响,却是适才去的李逵、樊瑞、项充、李衮,这四
个步军头领,从山左丛林里转向前来;又添了花和尚鲁智深、行者武松、没面目焦
挺、赤发鬼刘唐,四个步军将佐,并五百步兵,都执团牌短兵,直冲上来。贼将副
先锋上官义忙拨步军二千冲杀。李逵、鲁智深与贼兵略斗几合,却似抵敌不过的,
倒提团牌,分开两路,都飞奔入丛林中去了。贼兵赶来,那李逵等却是走得快,拈
指间,都四散奔走去了。李助见了,连忙对王庆道:“大王不宜追赶,这是诱敌之
计。我们且列阵迎敌。”

李助上将台列阵兀是未完,只听得山坡后轰天子母炮响,就山坡后涌出大队军将,
急先涌来,占住中央,里面列阵势。王庆令左右拢住战马,自上将台看时,只见正
南上这队人马,尽是红旗、红甲、红袍、朱缨、赤马,前面一把引军销金红旗。把
那红旗招展处,红旗中涌出一员大将,乃是霹雳火秦明,左手是圣水将军单廷圭,
右边是神火将军魏定国,三员大将,手掿兵器,都骑赤马,立于阵前。东壁一队人
马,尽是青旗、青甲、青袍、青缨、青马,前面一把引军销金青旗。招展处,青旗
中涌出一员大将,乃是大刀关胜,左手是丑郡马宣赞,右手是井木犴郝思文,三员
大将,手掿兵器,都骑青马,立于阵前。西壁一队人马,尽是白旗、白甲、白袍、
白缨、白马,前面一把引军销金白旗。招展处,白旗内涌出一员大将,乃是豹子头
林冲,左手是镇三山黄信,右手是病尉迟孙立,三员大将,手掿兵器,都骑白马,
立于阵前。后面一簇人马,都是皂旗、黑甲、黑袍、黑缨、黑马,前面一把引军销
金皂旗。招展处,皂旗中涌出一员大将,乃是双鞭将呼延灼,左手是百胜将韩滔,
右手是天目将彭玘,三员大将,手掿兵器,都骑黑马,立于阵前。东南方门旗影里,
一队军马,青旗红甲;前面一把引军绣旗招展,捧出一员大将,乃是双枪将董平,
左手是摩云金翅欧鹏,右手是火眼狻猊邓飞,三员大将,手掿兵器,都骑战马,立
于阵前。西南方门旗影里,一队军马,红旗白甲,前面一把引军绣旗招展处,捧出
一员大将,乃是急先锋索超,左手是锦毛虎燕顺,右手是铁笛仙马麟,三员大将,
手掿兵器,都骑战马,立于阵前。东北方门旗影里,一队军马,皂旗青甲,前面一
把引军绣旗招展处,捧出一员大将乃是九纹龙史进,左手是跳涧虎陈达,右手是白
花蛇杨春,三员大将,手掿兵器,都骑战马,立于阵前。西北方门旗影里,一队军
马,白旗黑甲;前面一把引军绣旗招展处,捧出一员大将,乃是青面兽杨志,左手
是锦豹子杨林,右手是小霸王周通,三员大将,手掿兵器,都骑战马,立于阵前。
八方摆布的铁桶相似。阵门里马军随马队,步军随步队,各持钢刀大斧,阔剑长枪,
旗齐整,队伍威严。八阵中央都是杏黄旗,间着六十四面长脚旗,上面金销六十
四卦,亦分四门。南门都是马军。正南上黄旗影里,捧出二员上将,上首是美髯公
朱仝,下手是插翅虎雷横,人马尽是黄旗、黄袍、铜甲、黄缨、黄马。中央阵,东
门是金眼彪施恩,西门是白面郎君郑天寿,南门是云里金刚宋万,北门是病大虫薛
永。那黄旗后,便是一丛炮架,立着那个炮手轰天雷凌振,引着副手二十余人,围
绕着炮架。架后都摆列捉将的挠钩套索,挠钩后又是一周遭杂彩旗,四面立着二
十八宿星辰。销金绣旗中间,立着一面堆绒绣就,真珠圈边,脚缀金铃,顶插雉尾,
鹅黄帅字旗。有一个守旗壮士,冠簪鱼尾,甲皱龙鳞,身长一丈,凛凛威风,便是
险道神郁保四。旗边设立两个护旗将士,都骑战马,一般结束,手执钢枪,一个是
毛头星孔明,一个是独火星孔亮。马前马后,排列二十四个执狼牙棍的铁甲军士。
后面两把领战绣旗,两边排列二十四枝方天画戟丛中,捧着两员骁将:左边是小温
侯吕方,右边是赛仁贵郭盛。两员将各持画戟,立马两边。画戟中间,一簇钢叉,
两员步军骁将,一般结束,一个是两头蛇解珍,一个是双尾蝎解宝,各执三股莲花
叉,守护中军。随后两匹锦鞍马上,左手是圣手书生萧让,右手是铁面孔目裴宣。
两个马后摆着紫衣持节的,并麻扎刀军士。那麻扎刀林中,立着两个行刑刽子:上
首是铁臂膊蔡福,下首是一枝花蔡庆。背阵两边,摆着金枪银枪手,两边有大将领
队。金枪队里,是金枪手徐宁;银枪队里,是小李广花荣。背后又是锦衣对对,花
帽双双,绯袍簇簇,锦袄攒攒。两壁厢碧幢翠幕,朱皂盖,黄钺白旄,青萍青电,
两行钺斧鞭挝中间,三把销金伞下,三匹锦鞍骏马上,坐着三个英雄:右边星冠鹤
氅,呼风唤雨的入云龙公孙胜;左边纶巾羽扇,文武双全的智多星吴用;正中间照
夜玉狮子金鞍马上,坐着那个有仁有义,退虏平寇的征西正先锋,山东及时雨呼保
义宋公明,全身结束,自仗锟宝剑,于阵中监战,掌握中军。马前左手,立着神
行太保戴宗,专管飞报军情,调兵遣将;右手立着浪子燕青,专一护持中军,能干
机密。马后大戟长戈,锦鞍骏马,整整齐齐,三十五员牙将,都骑战马,手执长枪,
全副弓箭。马后画角,全部鼓吹大乐。阵后又设两队游兵,伏于两侧,以为护持中
军羽翼:左是石将军石勇,同九尾龟陶宗旺,管领马步兵三千人;右是没遮拦穆弘,
引兄弟小遮拦穆春,管领马步兵三千,伏于两胁。那座阵排布得十分整密,正是:
军师多略帅恢弘,士涌貔貅马跨龙。
指挥要建平西绩,叱咤思成荡寇功。
那个草头天子王庆同李助在阵中将台上,定睛看了宋江兵马,拈指间,排成九宫八
卦阵势,军兵勇猛,将士英雄,军容整肃,刀枪锋利,惊得魂不附体,心胆俱落,
不住声道:“可知道兵将屡次亏输,原来那伙人如此利害!”

只听的宋军中,战鼓不绝声的发擂。王庆、李助下将台,骑上战马,左右有金吾护
驾等员役,马后有许多内侍簇拥着他。王庆传令旨,教前部先锋,出阵冲击。当下
东西对阵。是日干支属木。宋阵正西方门旗开处,豹子头林冲从门旗下飞马出阵,
两军一齐呐喊。林冲兜住马,横着丈八蛇矛,厉声高叫:“无知叛逆,谋反狂徒,
天兵到此,尚不投降!直待骨肉为泥,悔之何及!”贼阵中李助本是算命先生,甚
晓得相生相克之理,疾忙传令,教右哨柳元、潘忠,领红旗军去冲击。柳元、潘忠
遵令,领了红旗军,骤马抢来冲击。两阵迭声呐喊,战鼓齐鸣。林冲接住柳元厮杀,
四条臂膊纵横,八只马蹄撩乱。二将在征尘影里,杀气丛中,来来往往,左盘右旋,
斗经五十余合,胜败未分。那柳元是贼中勇猛之将,潘忠见柳元不能取胜,拍马提
刀,抢来助战。林冲力敌二将,大喝一声,奋神威,将柳元一矛戳于马下。林冲的
副将黄信、孙立,飞马冲出阵来。黄信挥丧门剑,望潘忠一剑砍去。只见一条血颡
光连肉,顿落金鍪在马边。

潘忠死于马下,手下军卒散乱,早冲动了阵脚,贼兵飞报入中军。王庆听的登时折
了二将,忙传令旨,急教退军。只听得宋军中一声炮响,兵马纷纷扰扰,白引黑,
黑引青,青引红,变作长蛇之阵,簸箕掌,栲栳圈,围裹将来。王庆、李助调将遣
兵,分头冲击,却似铜墙铁壁,急切不能冲得出来。官军与贼兵这场好杀,怎见得:

兵戈冲击,士马纵横。枪破刀,刀如劈脑而来,枪必钓鱼而应。刀如下发而起,
枪必绰地而迎。刀如倒拖而回,枪必裙拦而守。刀解枪,枪如刺心而来,刀用五花
以御。枪如点睛而来,刀用探马以格。筅破牌,牌或滚身以进,筅即风扫以当。牌
或从旁以追,筅必斜插以待。牌或摧挤以入,筅必退却以搠。牌解筅,筅若平胸,
牌用小坐之势以避。筅若簇拥,牌将碎剪之法以随。单刀披挂绞丝,佯输诈败。铁
叉上排下掩,侧进抵闪。袖箭于马上觑贼,钩镰于车前俟马。鞭、简、挝、,剑、
戟、矛、盾。那边破解无穷,这里转变莫测。须臾血流成河,顷刻尸如山积。
当下鏖战多时,贼兵大败,官军大胜。王庆叫且退入南丰大内,再作区处。只听得
后军炮响,哨马飞报将来说:“大王,后面又有宋军杀来!”那彪军,马上当先的
英雄大将,正是副先锋河北玉麒麟卢俊义,横着一条点钢枪;左边有使朴刀的好汉
病关索杨雄;右边有使朴刀的头领拚命三郎石秀,领着一万精兵,抖搜精神,将正
副合后贼兵杀散。杨雄砍翻段五,石秀搠死丘翔,并力冲杀进来。

王庆正在慌迫,又听得一声炮响,左有鲁智深、武松、李逵、焦挺、项充、李衮、
樊瑞、刘唐八个勇猛头领,引着一千步卒,抡动禅杖、戒刀、板斧、朴刀、丧门剑、
飞刀、标枪、团牌,杀死李雄、毕先,如割瓜切菜般直杀入来;右有张清、王英、
孙新、张青、琼英、扈三娘、顾大嫂、孙二娘,四对英雄夫妇,引着一千骑兵,舞
动梨花枪、鞭钢枪、方天画戟、日月双刀、钢枪、短刀,杀散左哨军兵,如摧枯拉
朽的直冲进来,杀得贼兵四分五裂,七断八续,雨零星散,乱窜奔逃。

卢俊义、杨雄、石秀杀入中军,正撞着方翰,被卢俊义一枪戳死,杀散中军羽翼军
兵,径来捉王庆,却遇了金剑先生李助。那李助有剑术,一把剑如掣电般舞将来。
卢俊义正在抵当不住,却得宋江中军兵到,右手下入云龙公孙胜,口中念念有词,
喝声道:“疾!”李助那口剑,托地离了手,落在地上。卢俊义骤马赶上,轻舒猿
臂,款扭狼腰,把李助只一拽,活挟过马来,教军士缚了。卢俊义拈枪拍马,再杀
入去寻捉王庆,好似皂雕追紫燕,猛虎啖羊羔。贼兵抛金弃鼓,撇戟丢枪,觅子寻
爷,呼兄唤弟,十余万贼兵,杀死大半。尸横遍野,流血成河。降者三万人,除那
逃走脱的,其余都是十死九活,七损八伤,颠翻在地,被人马践踏,骨肉如泥的,
不计其数。刘以敬、上官义两个猛将,都被焦挺砍翻战马,撞下马来,都被他杀死。
李雄被琼英飞石打下马来,一画戟搠死。毕先正在逃避,忽地里钻出活闪婆王定六,
一朴刀搠下马来,再向胸膛上一朴刀,结果了性命。其伪尚书、枢密、殿帅、金吾、
将军等项,都逃不脱,只不见了渠魁王庆。宋军大捷。

宋江教鸣金收集兵马,望南丰城来,教张清、琼英领五千马军,前去哨探;再差神
行太保戴宗先去打听孙安袭取南丰消息如何。戴宗遵令,作起神行法,赶过张清、
琼英,去了片晌,便来回报说:“孙安奉先锋将令,假扮西兵去赚城,被贼人知觉,
城门内掘下陷坑,开城东门,放军马进去。孙安手下梅玉、金祯、毕捷、潘迅、杨
芳、冯升、胡迈七个副将,争先抢入城去,并五百军士,连人和马,都攧入陷坑中。
两边伏兵齐发,都把长枪利戟,把梅玉等五百余人,尽行搠死。幸得孙安在后,乘
势奋勇杀进城门,教军士填了陷坑。孙安一骑当先,领兵杀入城中,贼兵不能抵当。
孙安夺了东门,后被贼人四面响应,把孙安兵马堵截在东门。小弟探知这消息,飞
来回复。半路遇了张将军及张宜人,说了此情,他两个催动人马疾驰去了。”
宋江闻报,催动大军,疾驰上前,将南丰城围住。那时张清、琼英进了东门,教孙
安据住东门,张清、琼英正与贼军鏖战,因此,宋江等将佐兵马,抢入东门,夺了
城池,杀散贼兵,四门竖起宋军旗号。城中许多伪文武多官范全等尽行杀死。那伪
妃段三娘听的军马进城,他素有膂力,也会骑马,遂拴缚结束,领了百余有膂力的
内侍,都执兵器,离王宫,出后苑,欲杀出西门,投云安军去,恰遇琼英领兵杀到
后苑来。段氏纵马,挺一口宝刀,抵死冲突。被琼英一石子飞来,正中段三娘面门,
鲜血迸流,撞下马来,攧个脚梢天,军士赶上,捉住绑缚了。那些内侍,都被宋兵
杀死。琼英领兵杀入后苑内宫,那些宫娥嫔女,闻得宋兵入城,或投环,或投井,
或刀刎,或撞阶,大半自尽,其余都被琼英教军士缚了,解到宋江帐前。宋江大喜,
将段氏一行人囚禁,待捉了王庆,一齐解京。再遣兵将,四面八方,去追王庆。
却说那王庆领着数百铁骑,撞透重围,逃奔到南丰城东,见城中有兵厮杀,惊得魂
不附体,后面大兵又到,望北奔走不迭。回顾左右,止有百余骑,其余的虽是平日
最亲信的,今日势败,都逃去了。王庆同了百余人,望云安奔走,在路对跟随近侍
说道:“寡人尚有云安、东川、安德三座城池,岂不是江东虽小,亦足以王?只恨
适才那些跟随逃散官员,平日受用了寡人大俸大禄,今日有事,都自去了。待寡人
兴兵来杀退宋兵,缉捕那些逃亡的,细细地醢他。”王庆同众人马不停蹄,人不歇
足,走到天明,幸的望见云安城池了。王庆在马上欣喜道:“城中将士,也是谨慎。
你看那旗齐整,兵器整密!”王庆一头说着,同众人奔近城来。随从人中,有识
字的说道:“大王不好了!怎么城上都是宋军旗号?”

王庆听了,定睛一看,果是东门城上,远远地闪出号旗,上有金销大字,乃是“御
西宋先锋麾下水军正将混江……”,下面尚有三个字,被风飘动旗脚,不甚分明。
王庆看了,惊的浑身麻木,半晌时动弹不得,真是宋兵从天而降。当有王庆手下一
个有智量近侍说道:“大王,事不宜迟!请大王速卸下袍服,急投东川去,恐城中
见了生变。”王庆道:“爱卿言之极当。”王庆随即卸下冲天转角金幞头,脱下日
月云肩蟒绣袍,解下金镶宝嵌碧玉带,脱下金显缝云根朝靴,换了巾帻、便服、软
皮靴。其余侍从,亦都脱卸外面衣服。急急如丧家之狗,忙忙如漏网之鱼,从小路
抄过云安城池,望东川投奔,走的人困马乏,腹中饥馁。百姓久被贼人伤残,又闻
得大兵厮杀,凡冲要通衢大路,都没一个人烟,静悄悄地鸡犬不闻,就要一滴水,
也没喝处,那讨酒食来?那时王庆手下亲幸跟随的,都是假登东,诈撒溺,又散去
了六七十人。

王庆带领三十余骑,走至晚,才到得云安属下开州地方,有一派江水阻路,这个江
叫做清江。其源出自达州万顷池,江水最是澄清,所以叫做清江。当下王庆道:“怎
得个船只渡过去?”后面一个近侍指道:“大王,兀那南涯疏芦落雁处,有一簇渔
船。”王庆看了,同众人走到江边。此时是孟冬时候,天气晴和,只见数十只渔船,
捕鱼的捕鱼,晒网的晒网。其中有几只船,放于中流,猜拳豁指头,大碗价吃酒。
王庆叹口气道:“这男女们恁般快乐!我今日反不如他了!这些都是我子民,却不
知寡人这般困乏。”近侍高叫道:“兀那渔人,撑拢几只船来,渡俺们过了江,多
与你渡钱。”只见两个渔人放下酒碗,摇着一只小渔艇,咿咿哑哑摇近岸来。船头
上渔人,向船旁拿根竹篙撑船拢岸,定睛把王庆从头上直看至脚下,便道:“快活,
又有吃酒东西了。上船,上船!”近侍扶王庆下马。

王庆看那渔人,身材长大,浓眉毛,大眼睛,红脸皮,铁丝般髭须,铜钟般声音。
那渔人一手执着竹篙,一手扶王庆上船,便把篙望岸上只一点,那船早离岸丈余。
那些随从贼人,在岸上忙乱起来,齐声叫道:“快撑拢船来!咱们也要过江的。”
那渔人睁眼喝道:“来了!忙到那里去?”便放下竹篙,将王庆劈胸扭住,双手向
下一按,扑通的按倒在板上。王庆待要挣扎,那船上摇橹的,放了橹,跳过来一
齐擒住。那边晒网船上人,见捉了王庆,都跳上岸,一拥上前,把那三十余个随从
贼人,一个个都擒住。

原来这撑船的,是混江龙李俊,那摇橹的,便是出洞蛟童威,那些渔人,多是水军。
李俊奉宋先锋将令,统驾水军船只,来敌贼人水军。李俊等与贼人水军大战于瞿塘
峡,杀其主帅水军都督闻人世崇,擒其副将胡俊,贼兵大败。李俊见胡俊状貌不凡,
遂义释胡俊。胡俊感恩,同李俊赚开云安水门,夺了城池,杀死伪留守施俊等。混
江龙李俊料着贼与大兵厮杀,若败溃下来,必要奔投巢穴。因此,教张横、张顺镇
守城池,自己与童威、童猛,带领水军,扮做渔船,在此巡探;又教阮氏三雄,也
扮做渔家,分投去滟堆、岷江、鱼复浦各路埋伏哨探。适才李俊望见王庆一骑当
先,后面又许多人簇拥着,料是贼中头目,却不知正是元凶。当下李俊审问从人,
知是王庆,拍手大笑,绑缚到云安城中。一面差人唤回三阮同二张守城,李俊同降
将胡俊,将王庆等一行人,解送到宋先锋军前来。于路探听得宋江已破南丰,李俊
等一径进城,将王庆解到帅府。宋江因众将捕缉王庆不着,正在纳闷,闻报不胜之
喜。当下李俊入府,参见了宋先锋,宋江称赞道:“贤弟这个功劳不小。”李俊引
降将胡俊,参见宋先锋。李俊道:“功劳都是这个人。”宋江问了胡俊姓名,及赚
取云安的事。

宋江抚赏慰劳毕,随即与众将计议,攻取东川、安德二处城池。只见新降将胡俊禀
道:“先锋不消费心。胡某有一言,管教两座城池,唾手可得!”宋江大喜,连忙
离坐,揖胡俊问计。胡俊躬着身,对宋江说出几句话来。有分教:一矢不加城克复,
三军镇静贼投降。
毕竟胡俊说出甚么话来,且听下回分解。