\chapter{花和尚倒拔垂杨柳~豹子头误入白虎堂}

话说那酸枣门外三二十个泼皮破落户中间,有两个为头的,一个叫做过街老鼠
张三,一个叫做青草蛇李四。这两个为头接将来,智深也却好去粪窖边,看见这伙
人都不走动,只立在窖边,齐道:“俺特来与和尚作庆。”智深道:“你们既是邻
舍街坊,都来廨宇里坐地。”张三、李四便拜在地上,不肯起来,只指望和尚来扶
他,便要动手。智深见了,心里早疑忌道:“这伙人不三不四,又不肯近前来,莫
不要攧洒家?那厮却是倒来捋虎须!俺且走向前去,教那厮看洒家手脚。”智深大踏
步近众人面前来,那张三、李四便道:“小人兄弟们特来参拜师父。”口里说,便
向前去,一个来抢左脚,一个来抢右脚。智深不等他占身,右脚早起,腾的把李四
先踢下粪窖里去;张三恰待走,智深左脚早起,两个泼皮都踢在粪窖里挣扎。后头
那二三十个破落户惊的目瞪口呆,都待要走。智深喝道:“一个走的,一个下去;
两个走的,两个下去。”众泼皮都不敢动弹。只见那张三、李四在粪窖里探起头来,
原来那座粪窖没底似深,两个一身臭屎,头发上蛆虫盘满,立在粪窖里叫道:“师
父饶恕我们。”智深喝道:“你那众泼皮,快扶那鸟上来,我便饶你众人。”众人
打一救,搀到葫芦架边,臭秽不可近前。智深呵呵大笑道:“兀那蠢物,你且去菜
园池子里洗了来,和你众人说话。”

两个泼皮洗了一回,众人脱件衣服,与他两个穿了。智深叫道:“都来廨宇里坐地说
话。”智深先居中坐了,指着众人道:“你那伙鸟人,休要瞒洒家:你等都是甚么鸟人?来
这里戏弄洒家!”那张三、李四并众火伴一齐跪下,说道:“小人祖居在这里,都只靠赌
博讨钱为生。这片菜园是俺们衣饭碗,大相国寺里几番使钱,要奈何我们不得。师父却是
那里来的长老,恁的了得!相国寺里不曾见有师父,今日我等情愿伏侍。”智深道:“洒家
是关西延安府老种经略相公帐前提辖官,只为杀的人多,因此情愿出家,五台山来到这里。
洒家俗姓鲁,法名智深。休说你这三二十个人直甚么,便是千军万马队中,俺敢直杀的入
去出来。”众泼皮喏喏连声,拜谢了去。智深自来廨宇里房内,收拾整顿歇卧。

次日,众泼皮商量凑些钱物,买了十瓶酒,牵了一个猪来请智深,都在廨宇安排了,
请鲁智深居中坐了,两边一带,坐定那二三十泼皮饮酒。智深道:“甚么道理叫你众人们
坏钞?”众人道:“我们有福,今日得师父在这里与我等众人做主。”智深大喜,吃到半
酣里,也有唱的,也有说的,也有拍手的,也有笑的。正在那里喧哄,只听得门外老鸦哇
哇的叫。众人有叩齿的,齐道:“赤口上天,白舌入地。”智深道:“你们做甚么鸟乱?”
众人道:“老鸦叫,怕有口舌。”智深道:“那里取这话?”那种地道人笑道:“墙角边
绿杨树上新添了一个老鸦巢,每日只聒到晚。”众人道:“把梯子去上面拆了那巢便了。”
有几个道:“我们便去。”智深也乘着酒兴,都到外面看时,果然绿杨树上一个老鸦巢。
众人道:“把梯子上去拆了,也得耳根清净。”李四便道:“我与你盘上去,不要梯子。”
智深相了一相,走到树前,把直裰脱了,用右手向下,把身倒缴着,却把左手拔住上截,
把腰只一趁,将那株绿杨树带根拔起。众泼皮见了,一齐拜倒在地,只叫:“师父非是凡
人,正是真罗汉身体,无千万斤气力,如何拔得起?”智深道:“打甚鸟紧?明日都看洒家
演武,使器械。”众泼皮当晚各自散了。

从明日为始,这二三十个破落户见智深匾匾的伏,每日将酒肉来请智深,看他演武使
拳。过了数日,智深寻思道:“每日吃他们酒食多矣,洒家今日也安排些还席。”叫道人
去城中买了几般果子,沽了两三担酒,杀翻一口猪,一腔羊。那时正是三月尽,天气正热。
智深道:“天色热。”叫道人绿槐树下铺了芦席,请那许多泼皮团团坐定。大碗斟酒,大
块切肉,叫众人吃得饱了,再取果子吃,酒又吃得正浓。众泼皮道:“这几日见师父演力,
不曾见师父使器械,怎得师父教我们看一看也好。”智深道:“说的是。”便去房内取出
浑铁禅杖,头尾长五尺,重六十二斤。众人看了,尽皆吃惊,都道:“两臂膊没水牛大小
气力,怎使得动?”智深接过来,飕飕的使动,浑身上下没半点儿参差。众人看了,一齐
喝采。

智深正使得活泛,只见墙外一个官人看见,喝采道:“端的使得好!”智深听得,收
住了手,看时,只见墙缺边立着一个官人,怎生打扮,但见:

头戴一顶青纱抓角儿头巾,脑后两个白玉圈连珠鬓环。身穿一领单绿罗团花战袍,腰
系一条双搭尾龟背银带。穿一对磕瓜头朝样皂靴,手中执一把折迭纸西川扇子。
那官人生的豹头环眼,燕颔虎须,八尺长短身材,三十四五年纪,口里道:“这个师父,
端的非凡,使的好器械!”众泼皮道:“这位教师喝采,必然是好。”智深问道:“那军
官是谁?”众人道:“这官人是八十万禁军枪棒教头林武师,名唤林冲。”智深道:“何
不就请来厮教。”那林教头便跳入墙来,两个就槐树下相见了,一同坐地。林教头便问道:
“师兄何处人氏?法讳唤做甚么?”智深道:“洒家是关西鲁达的便是。只为杀的人多,情
愿为僧,年幼时也曾到东京,认得令尊林提辖。”林冲大喜,就当结义智深为兄。智深道:
“教头今日缘何到此?”林冲答道:“恰才与拙荆一同来间壁岳庙里还香愿。林冲听得使
棒,看得入眼,着女使锦儿自和荆妇去庙里烧香,林冲就只此间相等,不想得遇师兄。”
智深道:“洒家初到这里,正没相识,得这几个大哥每日相伴;如今又得教头不弃,结为
弟兄,十分好了。”便叫道人再添酒来相待。恰才饮得三杯,只见女使锦儿慌慌急急,红
了脸,在墙缺边叫道:“官人休要坐地!娘子在庙中和人合口。”林冲连忙问道:“在那里?”
锦儿道:“正在五岳楼下来,撞见个奸诈不及的,把娘子拦住了不肯放。”林冲慌忙道:
“却再来望师兄,休怪,休怪。”

林冲别了智深,急跳过墙缺,和锦儿径奔岳庙里来,抢到五岳楼看时,见了数个人,
拿着弹弓、吹筒、粘竿,都立在栏干边;胡梯上一个年小的后生,独自背立着,把林冲的
娘子拦着道:“你且上楼去,和你说话。”林冲娘子红了脸道:“清平世界,是何道理把
良人调戏?”林冲赶到跟前,把那后生肩胛只一扳过来,喝道:“调戏良人妻子,当得何
罪?”恰待下拳打时,认的是本管高太尉螟蛉之子高衙内。原来高俅新发迹,不曾有亲儿,
无人帮助,因此过房这阿叔高三郎儿子在房内为子。本是叔伯弟兄,却与他做干儿子。因
此,高太尉爱惜他。那厮在东京倚势豪强,专一爱淫垢人家妻女。京师人惧怕他权势,谁
敢与他争口,叫他做花花太岁。有诗为证:
脸前花现丑难亲,心里花开爱妇人。
撞着年庚不顺利,方知太岁是凶神。
当时林冲扳将过来,却认得是本管高衙内,先自手软了。高衙内说道:“林冲,干你甚事!
你来多管!”原来高衙内不晓得他是林冲的娘子;若还晓的时,也没这场事。见林冲不动
手,他发这话。众多闲汉见闹,一齐拢来劝道:“教头休怪,衙内不认得,多有冲撞。”
林冲怒气未消,一双眼睁着瞅那高衙内。众闲汉劝了林冲,和哄高衙内出庙上马去了。

林冲将引妻小并使女锦儿,也转出廊下来,只见智深提着铁禅杖,引着那二三十个破
落户,大踏步抢入庙来。林冲见了,叫道:“师兄那里去?”智深道:“我来帮你厮打。”
林冲道:“原来是本官高太尉的衙内,不认得荆妇,时间无礼。林冲本待要痛打那厮一顿,
太尉面上须不好看。自古道:‘不怕官,只怕管。’林冲不合吃着他的请受,权且让他这
一次。”智深道:“你却怕他本官太尉,洒家怕他甚鸟?俺若撞见那撮鸟时,且教他吃洒家
三百禅杖了去。”林冲见智深醉了,便道:“师兄说得是。林冲一时被众人劝了,权且饶
他。”智深道:“但有事时,便来唤洒家与你去。”众泼皮见智深醉了,扶着道:“师父,
俺们且去,明日再得相会。”智深提着禅杖道:“阿嫂休怪,莫要笑话。阿哥,明日再会。”
智深相别,自和泼皮去了。林冲领了娘子并锦儿,取路回家,心中只是郁郁不乐。

且说这高衙内引了一班儿闲汉,自见了林冲娘子,又被他冲散了,心中好生着迷,怏
怏不乐,回到府中纳闷。过了三两日,众多闲汉都来伺候,见衙内心焦,没撩没乱,众人
散了。数内有一个帮闲的,唤作乾鸟头富安,理会得高衙内意思,独自一个到府中伺候。
见衙内在书房中闲坐,那富安走近前去道:“衙内近日面色清减,心中少乐,必然有件不
悦之事。”高衙内道:“你如何省得?”富安道:“小子一猜便着。”衙内道:“你猜我
心中甚事不乐。”富安道:“衙内是思想那双木的,这猜如何?”衙内笑道:“你猜得是,
只没个道理得他。”富安道:“有何难哉!衙内怕林冲是个好汉,不敢欺他,这个无伤。他
现在帐下听使唤,大请大受,怎敢恶了太尉?轻则便刺配了他,重则害了他性命。小闲寻思
有一计,使衙内能够得他。”高衙内听得,便道:“自见了许多好女娘,不知怎的只爱他,
心中着迷,郁郁不乐。你有甚见识能够他时,我自重重的赏你。”富安道:“门下知心腹
的陆虞候陆谦,他和林冲最好,明日衙内躲在陆虞候楼上深阁,摆下些酒食,却叫陆谦去
请林冲出来吃酒,教他直去樊楼上深阁里吃酒。小闲便去他家,对林冲娘子说道:‘你丈
夫教头和陆谦吃酒,一时重气,闷倒在楼上,叫娘子快去看哩!’赚得他来到楼上,妇人
家水性,见了衙内这般风流人物,再着些甜话儿调和他,不由他不肯。小闲这一计如何?”
高衙内喝采道:“好计!就今晚着人去唤陆虞候来分付了。”原来陆虞候家只在高太尉家隔
壁巷内。次日,商量了计策,陆虞候一时听允,也没奈何;只要小衙内欢喜,却顾不得朋
友交情。

且说林冲连日闷闷不已,懒上街去。巳牌时,听得门首有人叫道:“教头在家么?”
林冲出来看时,却是陆虞候,慌忙道:“陆兄何来?”陆谦道:“特来探望兄,何故连日
街前不见?”林冲道:“心里闷,不曾出去。”陆谦道:“我同兄长去吃三杯解闷。”林
冲道:“少坐拜茶。”两个吃了茶起身,陆虞候道:“阿嫂,我同兄长到家去吃三杯。”
林冲娘子赶到布帘下叫道:“大哥,少饮早归。”林冲与陆谦出得门来,街上闲走了一回。
陆虞候道:“兄长,我们休家去,只就樊楼内吃两杯。”当时两个上到樊楼内,占个阁儿,
唤酒保分付,叫取两瓶上色好酒,希奇果子按酒。两个叙说闲话,林冲叹了一口气,陆虞
候道:“兄长何故叹气?”林冲道:“贤弟不知,男子汉空有一身本事,不遇明主,屈沉
在小人之下,受这般腌的气!”陆虞候道:“如今禁军中虽有几个教头,谁人及得兄长
的本事?太尉又看承得好,却受谁的气?”林冲把前日高衙内的事告诉陆虞候一遍。陆虞候
道:“衙内必不认得嫂子,兄长休气,只顾饮酒。”林冲吃了八九杯酒,因要小遗,起身
道:“我去净手了来。”

林冲下得楼来,出酒店门,投东小巷内去净了手,回身转出巷口,只见女使锦儿叫道:
“官人寻得我苦,却在这里!”林冲慌忙问道:“做甚么?”锦儿道:“官人和陆虞候出
来,没半个时辰,只见一个汉子慌慌急急奔来家里,对娘子说道:‘我是陆虞候家邻舍。
你家教头和陆谦吃酒,只见教头一口气不来,便倒了,叫娘子且快来看视。’娘子听得,
连忙央间壁王婆看了家,和我跟那汉子去,直到太尉府前小巷内一家人家。上至楼上,只
见桌子上摆着些酒食,不见官人。恰待下楼,只见前日在岳庙里罗唣娘子的那后生出来道:
‘娘子少坐,你丈夫来也。’锦儿慌慌下得楼时,只听得娘子在楼上叫‘杀人’。因此我
一地里寻官人不见,正撞着卖药的张先生道:‘我在樊楼前过,见教头和一个人入去吃酒。’
因此特奔到这里。官人快去。”

林冲见说,吃了一惊,也不顾女使锦儿,三步做一步跑到陆虞候家,抢到胡梯上,却
关着楼门,只听得娘子叫道:“清平世界,如何把我良人妻子关在这里?”又听得高衙内
道:“娘子,可怜见救俺。便是铁石人,也告的回转。”林冲立在胡梯上叫道:“大嫂开
门。”那妇人听的是丈夫声音,只顾来开门,高衙内吃了一惊,斡开了楼窗,跳墙走了。
林冲上的楼上,寻不见高衙内,问娘子道:“不曾被这厮点污了?”娘子道:“不曾。”
林冲把陆虞候家打得粉碎。将娘子下楼,出得门外看时,邻舍两边都闭了门。女使锦儿接
着,三个人一处归家去了。

林冲拿了一把解腕尖刀,径奔到樊楼前,去寻陆虞候,也不见了。却回来他门前等了
一晚,不见回家,林冲自归。娘子劝道:“我又不曾被他骗了,你休得胡做。”林冲道:
“叵耐这陆谦畜生!我和你如兄若弟,你也来骗我!只怕不撞见高衙内,也照管着他头面。”
娘子苦劝,那里肯放他出门。陆虞候只躲在太尉府内,亦不敢回家。林冲一连等了三日,
并不见面。府前人见林冲面色不好,谁敢问他。

第四日饭时候,鲁智深径寻到林冲家相探,问道:“教头如何连日不见面?”林冲答
道:“小弟少冗,不曾探得师兄。既蒙到我寒家,本当草酌三杯,争奈一时不能周备。且
和师兄一同上街间玩一遭,市沽两盏如何?”智深道:“最好。”两个同上街来,吃了一
日酒,又约明日相会。自此每日与智深上街吃酒,把这件事都放慢了。正是:
丈夫心事有亲朋,谈笑酣歌散郁蒸。
只有女人愁闷处,深闺无语病难兴。

且说高衙内自从那日在陆虞候家楼上吃了那惊,跳墙脱走,不敢对太尉说知,因此在
府中卧病。陆虞候和富安两个来府里望衙内,见他容颜不好,精神憔悴,陆谦道:“衙内
何故如此精神少乐?”衙内道:“实不瞒你们说:我为林冲老婆,两次不能够得他,又吃
他那一惊,这病越添得重了。眼见的半年三个月性命难保。”二人道:“衙内且宽心,只
在小人两个身上,好歹要共那妇人完聚,只除他自缢死了便罢。”正说间,府里老都管也
来看衙内病证。只见:

不痒不痛,浑身上或寒或热;没撩没乱,满腹中又饱又饥。白昼忘餐,黄昏废寝。对
爷娘怎诉心中恨,见相识难遮脸上羞。
那陆虞候和富安见老都管来问病,两个商量道:“只除恁的……”等候老都管看病已了出
来,两个邀老都管僻净处说道:“若要衙内病好,只除教太尉得知,害了林冲性命,方能
够得他老婆和衙内在一处,这病便得好。若不如此,已定送了衙内性命。”老都管道:“这
个容易。老汉今晚便禀太尉得知。”两个道:“我们已有了计,只等你回话。”老都管至
晚来见太尉说道:“衙内不害别的证,却害林冲的老婆。”高俅道:“几时见了他的浑家?”
都管禀道:“便是前月二十八日在岳庙里见来,今经一月有余。”又把陆虞候设的计,备
细说了。高俅道:“如此因为他浑家,怎地害他?——我寻思起来,若为惜林冲一个人时,
须送了我孩儿性命,却怎生是好?”都管道:“陆虞候和富安有计较。”高俅道:“既是
如此,教唤二人来商议。”老都管随即唤陆谦、富安入到堂里,唱了喏。高俅问道:“我
这小衙内的事,你两个有甚计较?救得我孩儿好了时,我自抬举你二人。”陆虞候向前禀道:
“恩相在上,只除如此如此使得。”高俅见说了,喝采道:“好计!你两个明日便与我行。”
不在话下。

再说林冲每日和智深吃酒,把这件事不记心了。那一日,两个同行到阅武坊巷口,见
一条大汉,头戴一顶抓角儿头巾,穿一领旧战袍,手里拿着一口宝刀,插着个草标儿,立
在街上,口里自言自语说道:“不遇识者,屈沉了我这口宝刀。”林冲也不理会,只顾和
智深说着话走。那汉又跟在背后道:“好口宝刀,可惜不遇识者!”林冲只顾和智深走着,
说得入港,那汉又在背后说道:“偌大一个东京,没一个识得军器的。”林冲听的说,回
过头来,那汉飕的把那口刀掣将出来,明晃晃的夺人眼目。林冲合当有事,猛可地道:“将
来看。”那汉递将过来,林冲接在手内,同智深看了。但见:

清光夺目,冷气侵人。远看如玉沼春冰,近看似琼台瑞雪。花纹密布,如丰城狱内飞
来;紫气横空,似楚昭梦中收得。太阿巨阙应难比,莫邪干将亦等闲。

当时林冲看了,吃了一惊,失口道:“好刀!你要卖几钱?”那汉道:“索价三千贯,
实价二千贯。”林冲道:“值是值二千贯,只没个识主。你若一千贯肯时,我买你的。”
那汉道:“我急要些钱使,你若端的要时,饶你五百贯,实要一千五百贯。”林冲道:“只
是一千贯,我便买了。”那汉叹口气道:“金子做生铁卖了!罢,罢!一文也不要少了我的。”
林冲道:“跟我来家中取钱还你。”回身却与智深道:“师兄,且在茶房里少待,小弟便
来。”智深道:“洒家且回去,明日再相见。”

林冲别了智深,自引了卖刀的那汉,到家去取钱与他,就问那汉道:“你这口刀那里
得来?”那汉道:“小人祖上留下。因为家道消乏,没奈何,将出来卖了。”林冲道:“你
祖上是谁?”那汉道:“若说时,辱没杀人!”林冲再也不问。那汉得了银两,自去了。
林冲把这口刀翻来覆去看了一回,喝采道:“端的好把刀!高太尉府中有一口宝刀,胡乱不
肯教人看。我几番借看,也不肯将出来。今日我也买了这口好刀,慢慢和他比试。”林冲
当晚不落手看了一晚,夜间挂在壁上。未等天明,又去看那刀。

次日,巳牌时分,只听得门首有两个承局叫道:“林教头,太尉钧旨,道你买一口好
刀,就叫你将去比看,太尉在府里专等。”林冲听得说道:“又是甚么多口的报知了。”
两个承局催得林冲穿了衣服,拿了那口刀,随这两个承局来。林冲道:“我在府中不认的
你。”两个人说道:“小人新近参随。”却早来到府前,进得到厅前。林冲立住了脚,两
个又道:“太尉在里面后堂内坐地。”转入屏风至后堂,又不见太尉。林冲又住了脚,两
个又道:“太尉直在里面等你,叫引教头进来。”又过了两三重门,到一个去处,一周遭
都是绿栏杆。两个又引林冲到堂前,说道:“教头,你只在此少待,等我入去禀太尉。”
林冲拿着刀,立在檐前,两个人自入去了,一盏茶时,不见出来。林冲心疑,探头入帘看
时,只见檐前额上有四个青字,写道:“白虎节堂”。林冲猛省道:“这节堂是商议军机
大事处,如何敢无故辄入?”急待回身,只听的靴履响、脚步鸣,一个人从外面入来。林
冲看时,不是别人,却是本管高太尉。

林冲见了,执刀向前声喏。太尉喝道:“林冲,你又无呼唤,安敢辄入白虎节堂?你知
法度否?你手里拿着刀,莫非来刺杀下官?有人对我说,你两三日前,拿刀在府前伺候,必
有歹心。”林冲躬身禀道:“恩相,恰才蒙两个承局呼唤林冲,将刀来比看。”太尉喝道:
“承局在那里?”林冲道:“他两个已投堂里去了。”太尉道:“胡说!甚么承局,敢进我
府堂里去!左右与我拿下这厮!”说犹未了,傍边耳房里走出二十余人,把林冲横推倒拽,
恰似皂雕追紫燕,浑如猛虎啖羊羔。高太尉大怒道:“你既是禁军教头,法度也还不知道。
因何手执利刃,故入节堂,欲杀本官?”叫左右把林冲推下,不知性命如何。不因此等,
有分教:大闹中原,纵横海内。直教:农夫背上添心号,渔父舟中插认旗。

毕竟看林冲性命如何,且听下回分解。