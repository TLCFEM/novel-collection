\chapter{吴学究说三阮撞筹~公孙胜应七星聚义}

话说当时吴学究道:“我寻思起来,有三个人,义胆包身,武艺出众,敢赴汤
蹈火,同死同生。只除非得这三个人,方才完得这件事。”晁盖道:“这三个却是
甚么样人?姓甚名谁?何处居住?”吴用道:“这三个人是弟兄三个,在济州梁山泊
边石碣村住,日常只打鱼为生,亦曾在泊子里做私商勾当。本身姓阮,弟兄三人,
一个唤做立地太岁阮小二,一个唤做短命二郎阮小五,一个唤做活阎罗阮小七。这
三个是亲弟兄。小生旧日在那里住了数年,与他相交时,他虽是个不通文墨的人,
为见他与人结交真有义气,是个好男子,因此和他来往,今已好两年不曾相见。若
得此三人,大事必成。”晁盖道:“我也曾闻这阮家三弟兄的名字,只不曾相会。
石碣村离这里只有百十里以下路程,何不使人请他们来商议?”吴用道:“着人去
请,他们如何肯来?小生必须自去那里,凭三寸不烂之舌,说他们入伙。”晁盖大
喜道:“先生高见,几时可行?”吴用答道:“事不宜迟,只今夜三更便去,明日
晌午可到那里。”晁盖道:“最好。”

当时叫庄客且安排酒食来吃。吴用道:“北京到东京也曾行到,只不知生辰纲
从那条路来,再烦刘兄休辞生受,连夜去北京路上探听起程的日期,端的从那条路
上来。”刘唐道:“小弟只今夜也便去。”吴用道:“且住,他生辰是六月十五日,
如今却是五月初头,尚有四五十日,等小生先去说了三阮弟兄回来,那时却教刘兄
去。”晁盖道:“也是,刘兄弟只在我庄上等候。”

话休絮烦,当日吃了半晌酒食,至三更时分,吴用起来洗漱罢,吃了些早饭,
讨了些银两,藏在身边,穿上草鞋,晁盖、刘唐送出庄门,吴用连夜投石碣村来。
行到晌午时分,早来到那村中。但见:

青郁郁山峰叠翠,绿依依桑柘堆云。四边流水绕孤村,几处疏篁沿小径。茅檐
傍涧,古木成林。篱外高悬沽酒旆,柳阴闲缆钓鱼船。

吴学究自来认得,不用问人,来到石碣村中,径投阮小二家来。到得门前看时,
只见枯桩上缆着数只小渔船,疏篱外晒着一张破鱼网,倚山傍水,约有十数间草房。
吴用叫一声道:“二哥在家么?”只见一个人从里面走出来,生得如何,但见:

眍兜脸两眉竖起,略绰口四面连拳。胸前一带盖胆黄毛,背上两枝横生板肋。
臂膊有千百斤气力,眼晴射几万道寒光。休言村里一渔人,便是人间真太岁。

那阮小二走将出来,头戴一顶破头巾,身穿一领旧衣服,赤着双脚,出来见了
是吴用,慌忙声喏道:“教授何来?甚风吹得到此?”吴用答道:“有些小事,特
来相浼二郎。”阮小二道:“有何事,但说不妨。”吴用道:“小生自离了此间,
又早二年。如今在一个大财主家做门馆,他要办筵席,用着十数尾重十四五斤的金
色鲤鱼,因此特地来相投足下。”阮小二笑了一声,说道:“小人且和教授吃三杯,
却说。”吴用道:“小生的来意,也欲正要和二哥吃三杯。”阮小二道:“隔湖有
几处酒店,我们就在船里荡将过去。”吴用道:“最好。也要就与五郎说句话,不
知在家也不在?”阮小二道:“我们去寻他便了。”两个来到泊岸边,枯桩上缆的
小船解了一只,便扶着吴用下船去了。树根头拿了一把桦揪,只顾荡。早荡将开去,
望湖泊里来。正荡之间,只见阮小二把手一招,叫道:“七哥,曾见五郎么?”吴
用看时,只见芦苇丛中摇出一只船来。那汉生的如何,但见:

疙疸脸横生怪肉,玲珑眼突出双睛。腮边长短淡黄须,身上交加乌黑点。浑如
生铁打成,疑是顽铜铸就。世上降生真五道,村中唤作活阎罗。

那阮小七头戴一顶遮日黑箬笠,身上穿个棋子布背心,腰系着一条生布裙,把
那只船荡着,问道:“二哥,你寻五哥做甚么?”吴用叫一声:“七郎,小生特来
相央你们说话。”阮小七道:“教授恕罪,好几时不曾相见。”吴用道:“一同和
二哥去吃杯酒。”阮小七道:“小人也欲和教授吃杯酒,只是一向不曾见面。”两
只船厮跟着在湖泊里,不多时,划到个去处,团团都是水,高埠上有七八间草房,
阮小二叫道:“老娘,五哥在么?”那婆婆道:“说不得,鱼又不得打,连日去赌
钱,输得没了分文,却才讨了我头上钗儿,出镇上赌去了。”阮小二笑了一声,便
把船划开。阮小七便在背后船上说道:“哥哥,正不知怎地,赌钱只是输,却不晦
气!莫说哥哥不赢,我也输得赤条条地。”吴用暗想道:“中了我的计了。”两只
船厮并着,投石碣村镇上来。划了半个时辰,只见独木桥边一个汉子,把着两串铜
钱,下来解船。阮小二道:“五郎来了。”吴用看时,但见:

一双手浑如铁棒,两只眼有似铜铃。面上虽有些笑容,眉间却带着杀气。能生
横祸,善降非灾。拳打来,狮子心寒;脚踢处,蛇丧胆。何处觅行瘟使者,只此
是短命二郎。

那阮小五斜戴着一顶破头巾,鬓边插朵石榴花,披着一领旧布衫,露出胸前刺
着的青郁郁一个豹子来,里面匾扎起裤子,上面围着一条间道棋子布手巾。吴用叫
一声道:“五郎得采么?”阮小五道:“原来却是教授,好两年不曾见面,我在桥
上望你们半日了。”阮小二道:“我和教授直到你家寻你,老娘说道出镇上赌钱去
了,因此同来这里寻你。且来和教授去水阁上吃三杯。”阮小五慌忙去桥边解了小
船,跳在舱里,捉了桦楫,只一划,三只船厮并着划了一歇,早到那个水阁酒店前。
看时,但见:

前临湖泊,后映波心。数十株槐柳绿如烟,一两荡荷花红照水。凉亭上窗开碧
槛,水阁中风动朱帘。休言三醉岳阳楼,只此便是蓬岛客。

当下三只船撑到水亭下荷花荡中,三只船都缆了。扶吴学究上了岸,入酒店里
来,都到水阁内拣一副红油桌凳。阮小二便道:“先生休怪我三个弟兄粗俗,请教
授上坐。”吴用道:“却使不得。”阮小七道:“哥哥只顾坐主位,请教授坐客席,
我兄弟两个便先坐了。”吴用道:“七郎只是性快。”四个人坐定了,叫酒保打一
桶酒来。店小二把四只大盏子摆开,铺下四双箸,放了四盘菜蔬,打一桶酒,放在
桌子上。阮小二道:“有甚么下口?”小二哥道:“新宰得一头黄牛,花糕也似好
肥肉。”阮小二道:“大块切十斤来。”阮小五道:“教授休笑话,没甚孝顺。”
吴用道:“倒来相扰,多激恼你们。”阮小二道:“休恁地说!”催促小二哥只顾
筛酒,早把牛肉切做两盘,将来放在桌上。阮家三兄弟让吴用吃了几块,便吃不得
了,那三个狼虎食,吃了一回。

阮小五动问道:“教授到此贵干?”阮小二道:“教授如今在一个大财主家做
门馆教学,今来要对付十数尾金色鲤鱼,要重十四五斤的,特来寻我们。”阮小七
道:“若是每常要三五十尾也有,莫说十数个,再要多些,我弟兄们也包办得。如
今便要重十斤的也难得。”阮小五道:“教授远来,我们也对付十来个重五六斤的
相送。”吴用道:“小生多有银两在此,随算价钱,只是不用小的,须得十四五斤
重的便好。”阮小七道:“教授,却没讨处,便是五哥许五六斤的,也不能够,须
是等得几日才得。我的船里有一桶小活鱼,就把来吃酒。”阮小七便去船内取将一
桶小鱼上来,约有五七斤,自去灶上安排,盛做三盘,把来放在桌上。阮小七道:
“教授胡乱吃些个。”四个又吃了一回,看看天色渐晚,吴用寻思道:“这酒店里
须难说话,今夜必是他家权宿,到那里却又理会。”阮小二道:“今夜天色晚了,
请教授权在我家宿一宵,明日却再计较。”吴用道:“小生来这里走一遭,千难万
难,幸得你们弟兄今日做一处,眼见得这席酒不肯要小生还钱,今晚借二郎家歇一
夜,小生有些须银子在此,相烦就此店中沽一瓮酒,买些肉,村中寻一对鸡,夜间
同一醉如何?”阮小二道:“那里要教授坏钱,我们弟兄自去整理,不烦恼没对付
处。”吴用道:“径来要请你们三位。若还不依小生时,只此告退。”阮小七道:
“既是教授这般说时,且顺情吃了,却再理会。”吴用道:“还是七郎性直爽快!”
吴用取出一两银子,付与阮小七,就问主人家沽了一瓮酒,借个大瓮盛了,买了二
十斤生熟牛肉,一对大鸡。阮小二道:“我的酒钱,一发还你。”店主人道:“最
好,最好!”

四人离了酒店,再下了船,把酒肉都放在船舱里,解了缆索,径划将开去,一
直投阮小二家来。到得门前,上了岸,把船仍旧缆在桩上,取了酒肉,四人一齐都
到后面坐地,便叫点起灯来。原来阮家弟兄三个,只有阮小二有老小,阮小五、阮
小七都不曾婚娶,四个人都在阮小二家后面水亭上坐定。阮小七宰了鸡,叫阿嫂同
讨的小猴子在厨下安排。约有一更相次,酒肉都搬来摆在桌上。

吴用劝他弟兄们吃了几杯,又提起买鱼事来,说道:“你这里偌大一个去处,
却怎地没了这等大鱼?”阮小二道:“实不瞒教授说,这般大鱼,只除梁山泊里便
有,我这石碣湖中狭小,存不得这等大鱼。”吴用道:“这里和梁山泊一望不远,
相通一派之水,如何不去打些?”阮小二叹了一口气道:“休说!”吴用又问道:
“二哥如何叹气?”阮小五接了说道:“教授不知,在先这梁山泊是我弟兄们的衣
饭碗,如今绝不敢去。”吴用道:“偌大去处,终不成官司禁打鱼鲜。”阮小五道:
“甚么官司,敢来禁打鱼鲜!便是活阎王,也禁治不得!”吴用道:“既没官司禁
治,如何绝不敢去?”阮小五道:“原来教授不知来历,且和教授说知。”吴用道:
“小生却不理会得。”阮小七接着便道:“这个梁山泊去处,难说难言。如今泊子
里新有一伙强人占了,不容打鱼。”吴用道:“小生却不知,原来如今有强人,我
这里并不曾闻得说。”

阮小二道:“那伙强人,为头的是个落第举子,唤做白衣秀士王伦,第二个叫
做摸着天杜迁,第三个叫做云里金刚宋万。以下有个旱地忽律朱贵,现在李家道口
开酒店,专一探听事情,也不打紧。如今新来一个好汉,是东京禁军教头,甚么豹
子头林冲,十分好武艺。这几个贼男女聚集了五七百人,打家劫舍,抢掳来往客人。
我们有一年多不去那里打鱼,如今泊子里把住了,绝了我们的衣饭,因此一言难尽。”
吴用道:“小生实是不知有这段事,如何官司不来捉他们?”阮小五道:“如今那
官司一处处动弹,便害百姓;但一声下乡村来,倒先把好百姓家养的猪、羊、鸡、
鹅,尽都吃了,又要盘缠打发他。如今也好教这伙人奈何!那捕盗官司的人,那里
敢下乡村来!若是那上司官员差他们缉捕人来,都吓得尿屎齐流,怎敢正眼儿看他!”
阮小二道:“我虽然不打得大鱼,也省了若干科差。”吴用道:“恁地时,那厮们
倒快活!”阮小五道:“他们不怕天,不怕地,不怕官司,论秤分金银,异样穿绸
锦,成瓮吃酒,大块吃肉,如何不快活?我们弟兄三个空有一身本事,怎地学得他
们!”吴用听了,暗暗地欢喜道:“正好用计了。”阮小七说道:“人生一世,草
生一秋,我们只管打鱼营生,学得他们过一日也好!”

吴用道:“这等人学他做甚么?他做的勾当,不是笞杖五七十的罪犯,空自把
一身虎威都撇下。倘或被官司拿住了,也是自做的罪。”阮小二道:“如今该管官
司没甚分晓,一片糊涂,千万犯了迷天大罪的,倒都没事!我弟兄们不能快活,若
是但有肯带挈我们的,也去了罢。”阮小五道:“我也常常这般思量,我弟兄三个
的本事,又不是不如别人!谁是识我们的?”吴用道:“假如便有识你们的,你们
便如何肯去!”阮小七道:“若是有识我们的,水里水里去,火里火里去。若能够
受用得一日,便死了开眉展眼。”吴用暗暗喜道:“这三个都有意了,我且慢慢地
诱他。”吴用又劝他三个吃了两巡酒,正是:
只为奸邪屈有才,天教恶曜下凡来。
试看阮氏三兄弟,劫取生辰不义财。

吴用又说道:“你们三个敢上梁山泊捉这伙贼么?”阮小七道:“便捉的他们,
那里去请赏?也吃江湖上好汉们笑话!”吴用道:“小生短见:假如你们怨恨打鱼
不得,也去那里撞筹却不是好?”阮小二道:“先生,你不知,我弟兄们几遍商量
要去入伙,听得那白衣秀士王伦的手下人都说道他心地窄狭,安不得人。前番那个
东京林冲上山,怄尽他的气。王伦那厮,不肯胡乱着人,因此我弟兄们看了这般样,
一齐都心懒了。”阮小七道:“他们若似老兄这等慷慨,爱我弟兄们便好!”阮小
五道:“那王伦若得似教授这般情分时,我们也去了多时,不到今日!我弟兄三个,
便替他死也甘心!”吴用道:“量小生何足道哉!如今山东、河北多少英雄豪杰的
好汉!”阮小二道:“好汉们尽有,我弟兄自不曾遇着。”

吴用道:“只此间郓城县东溪村晁保正,你们曾认得他么?”阮小五道:“莫
不是叫做托塔天王的晁盖么?”吴用道:“正是此人。”阮小七道:“虽然与我们
只隔得百十里路程,缘分浅薄,闻名不曾相会。”吴用道:“这等一个仗义疏财的
好男子,如何不与他相见!”阮小二道:“我弟兄们无事也不曾到那里,因此不能
够与他相见。”吴用道:“小生这几年也只在晁保正庄上左近教些村学,如今打听
得他有一套富贵待取,特地来和你们商议,我等就那半路里拦住取了,如何?”阮
小五道:“这个却使不得。他既是仗义疏财的好男子,我们却去坏他的道路,须吃
江湖上好汉们知时笑话。”吴用道:“我只道你们弟兄心志不坚,原来真个惜客好
义。我对你们实说,果有协助之心,我教你们知此一事。我如今现在晁保正庄上住,
保正闻知你三个大名,特地教我来请你们说话。”阮小二道:“我弟兄三个,真真
实实地并没半点儿假!晁保正敢有件奢遮的私商买卖,有心要带挈我们,一定是烦
老兄来。若还端的有这事,我三个若舍不得性命相帮他时,残酒为誓:教我们都遭
横事,恶病临身,死于非命!”阮小五和阮小七把手拍着脖项道:“这腔热血,只
要卖与识货的!”

吴用道:“你们三位弟兄在这里,不是我坏心术来诱你们,这件事非同小可的
勾当!目今朝内蔡太师是六月十五日生辰,他的女婿是北京大名府梁中书,即目起
解十万贯金珠宝贝与他丈人庆生辰。今有一个好汉姓刘,名唐,特来报知。如今欲
要请你们去商议,聚几个好汉,向山凹僻静去处,取此一套富贵不义之财,大家图
个一世快活。因此特教小生只做买鱼来请你们三个计较,成此一事,不知你们心意
如何?”阮小五听了道:“罢!罢!”叫道:“七哥,我和你说甚么来!”阮小七
跳起来道:“一世的指望,今日还了愿心!正是搔着我痒处!我们几时去?”吴用道:
“请三位即便去来,明日起个五更,一齐都到晁天王庄上去。”阮家三弟兄大喜。
有诗为证:
学究知书岂爱财,阮郎渔乐亦悠哉!
只因不义金珠去,致使群雄聚义来。
当夜过了一宿,次早起来,吃了早饭,阮家三弟兄分付了家中,跟着吴学究,四个
人离了石碣村,拽开脚步,取路投东溪村来。行了一日,早望见晁家庄,只见远远
地绿槐树下晁盖和刘唐在那里等,望见吴用引着阮家三兄弟直到槐树前,两下都厮
见了。晁盖大喜道:“阮氏三雄名不虚传,且请到庄里说话。”六人俱从庄外入来,
到得后堂,分宾主坐定。吴用把前话说了,晁盖大喜,便叫庄客宰杀猪羊,安排烧
纸。阮家三弟兄见晁盖人物轩昂,语言洒落,三个说道:“我们最爱结识好汉,原
来只在此间。今日不得吴教授相引,如何得会?”三个弟兄好生欢喜。当晚且吃了
些饭,说了半夜话。

次日天晓,去后堂前面列了金钱、纸马、香花、灯烛,摆了夜来煮的猪羊、烧
纸。众人见晁盖如此志诚,尽皆欢喜,个个说誓道:“梁中书在北京害民,诈得钱
物,却把去东京与蔡太师庆生辰,此一等正是不义之财。我等六人中但有私意者,
天地诛灭,神明鉴察。”六人都说誓了,烧化纸钱。

六筹好汉,正在后堂散福饮酒,只见一个庄客报说:“门前有个先生要见保正
化斋粮。”晁盖道:“你好不晓事!见我管待客人在此吃酒,你便与他三五升米便
了,何须直来问我!”庄客道:“小人化米与他,他又不要,只要面见保正。”晁
盖道:“一定是嫌少!你便再与他三二斗米去。你说与他,保正今日在庄上请人吃
酒,没工夫相见。”庄客去了多时,只见又来说道:“那先生,与了他三斗米,又
不肯去。自称是一清道人,不为钱米而来,只要求见保正一面。”晁盖道:“你这
厮不会答应,便说今日委实没工夫,教他改日却来相见拜茶。”庄客道:“小人也
是这般说,那个先生说道:‘我不为钱米斋粮,闻知保正是个义士,特求一见。’”
晁盖道:“你也这般缠,全不替我分忧!他若再嫌少时,可与他三四斗去,何必又
来说!我若不和客人们饮时,便去厮见一面,打甚么紧!你去发付他罢,再休要来说!”

庄客去了没半个时,只听得庄门外热闹,又见一个庄客飞也似来报道:“那先
生发怒,把十来个庄客都打倒了。”晁盖听得,吃了一惊,慌忙起身道:“众位弟
兄少坐,晁盖自去看一看。”便从后堂出来,到庄门前看时,只见那个先生身长八
尺,道貌堂堂,生得古怪,正在庄门外绿槐树下打那众庄客。晁盖看那先生,但见:

头绾两枚松双丫髻,身穿一领巴山短褐袍,腰系杂色彩丝绦,背上松纹古铜
剑。白肉脚衬着多耳麻鞋,绵囊手拿着鳖壳扇子。八字眉,一双杏子眼;四方口,
一部落腮胡。
那先生一头打,一头口里说道:“不识好人。”晁盖见了,叫道:“先生息怒,你
来寻晁保正,无非是投斋化缘,他已与了你米,何故嗔怪如此?”那先生哈哈大笑
道:“贫道不为酒食钱米而来。我觑得十万贯如同等闲,特地来寻保正,有句话说。
叵耐村夫无理,毁骂贫道,因此性发。”晁盖道:“你可曾认得晁保正么?”那先
生道:“只闻其名,不曾会面。”晁盖道:“小子便是,先生有甚话说?”那先生
看了道:“保正休怪,贫道稽首。”晁盖道:“先生少请,到庄里拜茶如何?”那
先生道:“多感。”

两人入庄里来,吴用见那先生入来,自和刘唐、三阮一处躲过。且说晁盖请那
先生到后堂吃茶已罢,那先生道:“这里不是说话处。别有甚么去处可坐?”晁盖
见说,便邀那先生又到一处小小阁儿内,分宾坐定。晁盖道:“不敢拜问先生高姓?
贵乡何处?”那先生答道:“贫道复姓公孙,单讳一个胜字,道号一清先生。小道
是蓟州人氏,自幼乡中好习枪棒,学成武艺多般,人但呼为公孙胜大郎。为因学得
一家道术,亦能呼风唤雨,驾雾腾云,江湖上都称贫道做入云龙。贫道久闻郓城县
东溪村晁保正大名,无缘不曾拜识,今有十万贯金珠宝贝,专送与保正,作进见之
礼,未知义士肯纳受否?”晁盖大笑道:“先生所言,莫非北地生辰纲么?”那先
生大惊道:“保正何以知之?”晁盖道:“小子胡猜,未知合先生意否?”公孙胜
道:“此一套富贵,不可错过。古人有云:‘当取不取,过后莫悔。’晁保正心下
如何?”

正说之间,只见一个人从阁子外抢将入来,劈胸揪住公孙胜说道:“好呀!明
有王法,暗有神灵,你如何商量这等的勾当!我听得多时也!”吓得这公孙胜面如
土色。正是:机谋未就,争奈窗外人听;计策才施,又早萧墙祸起。

毕竟抢来揪住公孙胜的,却是何人,且听下回分解。