\chapter{徐宁教使钩镰枪~宋江大破连环马}

话说晁盖、宋江、吴用、公孙胜与众头领,就聚义厅上启请徐宁,教
使钩镰枪法。众人看徐宁时,果是一表好人物,六尺五六长身体,团团的
一个白脸,三牙细黑髭髯,十分腰围膀阔。曾有一篇《西江月》单道徐宁
模样:

臂健开弓有准,身轻上马如飞。弯弯两道卧蚕眉,凤翥鸾翔子弟。

战铠细穿柳叶,乌巾斜带花枝。常随宝驾侍丹墀,枪手徐宁无对。

当下徐宁选军已罢,便下聚义厅来,拿起一把钩镰枪,自使一回。众
人见了喝采。徐宁便教众军道:“但凡马上使这般军器,就腰胯里做步上
来,上中七路,三钩四拨,一搠一分,共使九个变法。若是步行使这钩镰
枪,亦最得用。先使八步四拨,荡开门户;十二步一变,十六步大转身。
分钩镰搠缴,二十四步,挪上攒下,钩东拨西;三十六步,浑身盖护,夺
硬斗强:此是钩镰枪正法。有诗诀为证:‘四拨三钩通七路,共分九变合
神机。二十四步挪前后,一十六翻大转围。’”徐宁将正法一路路敷演,
教众头领看。众军汉见了徐宁使钩镰枪,都喜欢。就当日为始,将选拣精
锐壮健之人,晓夜习学。又教步军藏林伏草,钩蹄拽腿,下面三路暗法。
不到半月之间,教成山寨五七百人,宋江并众头领看了大喜,准备破敌。

却说呼延灼自从折了彭、凌振,每日只把马军来水边搦战。山寨中
只教水军头领牢守各处滩头,水底钉了暗桩。呼延灼虽是在山西山北两路
出哨,决不能够到山寨边。梁山泊却叫凌振制造了诸般火炮,克日定时,
下山对敌;学使钩镰枪军士,已都学成。宋江道:“不才浅见,未知合众
位心意否?”吴用道:“愿闻其略。”宋江道:“明日并不用一骑马军,
众头领都是步战。孙吴兵法,却利于山林沮泽。今将步军下山,分作十队
诱敌,但见军马冲掩将来,都望芦苇荆棘林中乱走。却先把钩镰枪军士埋
伏在彼,每十个会使钩镰枪的,间着十个挠钩手,但见马到,一搅钩翻,
便把挠钩搭将入去捉了。平川窄路,也如此埋伏。此法如何?”吴学究道:
“正应如此藏兵捉将。”徐宁道:“钩镰枪并挠钩,正是此法。”

宋江当日,分拨十队步军人马:刘唐、杜迁引一队,穆弘、穆春引一
队,杨雄、陶宗旺引一队,朱仝、邓飞引一队,解珍、解宝引一队,邹渊、
邹润引一队,一丈青、王矮虎引一队,薛永、马麟引一队,燕顺、郑天寿
引一队,杨林、李云引一队。——这十队步军,先行下山诱引敌军。再差
李俊、张横、张顺、三阮、童威、童猛、孟康,九个水军头领,乘驾战船
接应。再叫花荣、秦明、李应、柴进、孙立、欧鹏,六个头领,乘马引军,
只在山边搦战;凌振、杜兴,专放号炮。却叫徐宁、汤隆,总行招引使钩
镰枪军士。中军宋江、吴用、公孙胜、戴宗、吕方、郭盛,总制军马,指
挥号令。其余头领俱各守寨。

宋江分拨已定,是夜三更,先载使钩镰枪军士过渡,四面去分头埋伏
已定。四更,却渡十队步军过去。凌振、杜兴,载过风火炮,架上高阜去
处,竖起炮架,搁上火炮。徐宁、汤隆,各执号带渡水。平明时分,宋江
守中军人马,隔水擂鼓呐喊摇旗。呼延灼正在中军帐内,听得探子报知,
传令便差先锋韩滔先来出哨。随即锁上连环甲马,呼延灼全身披挂,骑了
踢雪乌骓马,仗着双鞭,大驱车马,杀奔梁山泊来。隔水望见宋江引着许
多人马,呼延灼教摆开马军。先锋韩滔来与呼延灼商议道:“正南上一队
步军,不知多少的?”呼延灼道:“休问他多少,只顾把连环马冲将去!”
韩滔引着五百马军,飞哨出去。又见东南上一队军兵起来,却欲分兵去哨,
只见西南上又有起一队旗号,招呐喊。韩滔再引军回来,对呼延灼道:
“南边三队贼兵,都是梁山泊旗号。”呼延灼道:“这厮许多时不出来厮
杀,必有计策。”

说犹未了,只听得北边一声炮响。呼延灼骂道:“这炮必是凌振从贼,
教他施放。”众人平南一望,只见北边又拥起三队旗号,呼延灼对韩滔道:
“此必是贼人奸计。我和你把人马分为两路,我去杀北边人马,你去杀南
边人马。”正欲分兵之际,只见西边又是四队人马起来,呼延灼心慌;又
听的正北上连珠炮响,一带直接到土坡上。那一个母炮周回接着四十九个
子炮,名为“子母炮”,响处风威大作。呼延灼军兵,不战自乱,急和韩
滔各引马步军兵四下冲突。这十队步军,东赶东走,西赶西走,呼延灼看
了大怒,引兵望北冲将来。宋江军兵尽投芦苇中乱走,呼延灼大驱连环马,
卷地而来。那甲马一齐跑发,收勒不住,尽望败苇折芦之中,枯草荒林之
内跑了去。只听里面胡哨响处,钩镰枪一齐举手。先钩倒两边马脚,中间
的甲马,便自咆哮起来。那挠钩手军士,一齐搭住,芦苇中只顾缚人。呼
延灼见中了钩镰枪计,便勒马回南边去赶韩滔。背后风火炮当头打将下来,
这边那边,漫山遍野,都是步军追赶着。韩滔、呼延灼部领的连环甲马,
乱滚滚都入荒草芦苇之中,尽被捉了。

二人情知中了计策,纵马去四面跟寻马军,夺路奔走时,更兼那几条
路上,麻林般摆着梁山泊旗号,不敢投那几条路走,一直便望西北上来。
行不到五六里路,早拥出一队强人,当先两个好汉拦路:一个是没遮拦穆
弘,一个是小遮拦穆春,拈两条朴刀大喝道:“败将休走!”呼延灼忿怒,
舞起双鞭,纵马直取穆弘、穆春。略斗四五合,穆春便走。呼延灼只怕中
了计,不来追赶,望正北大路而走。山坡下又转出一队强人,当先两个好
汉拦路:一个是两头蛇解珍,一个是双尾蝎解宝。各挺钢叉,直奔前来。
呼延灼舞起双鞭,来战两个。斗不到五七合,解珍、解宝拔步便走。呼延
灼赶不过半里多路,两边钻出二十四把钩镰枪,着地卷将来。呼延灼无心
恋战,拨转马头,望东北上大路便走;又撞着王矮虎、一丈青夫妻二人,
截住去路。呼延灼见路径不平,四下兼有荆棘遮拦,拍马舞鞭,杀开条路,
直冲过去。王矮虎、一丈青赶了一直,赶不上,呼延灼自投东北上去了。
杀的大败亏输,雨零星乱。有诗为证:
十路军兵振地来,乌骓踢雪望风回。
连环尽被钩镰破,剩得双鞭出九垓。

话分两头。且说宋江鸣金收军回山,各请功赏。三千连环甲马,有停
半被钩镰枪拨倒,伤损了马蹄,剥去皮甲,把来做菜马食;二停多好马,
牵上山去喂养,作坐马。带甲军士,都被生擒上山。五千步军,被三面围
得紧急,有望中军躲的,都被钩镰枪拖翻捉了;望水边逃命的,尽被水军
头领围裹上船去,拽过滩头,拘捉上山。先前被拿去的马匹并捉去军士,
尽行复夺回寨。把呼延灼寨栅尽数拆来,水边泊内,搭盖小寨,再造两处
做眼酒店房屋等项,仍前着孙新、顾大嫂、石勇、时迁,两处开店。刘唐、
杜迁拿得韩滔,把来绑缚,解到山寨。宋江见了,亲解其缚,请上厅来,
以礼陪话,相待筵宴,令彭、凌振说他入伙。韩滔也是七十二煞之数,
自然意气相投,就梁山泊做了头领。宋江便教修书,使人往陈州搬取韩滔
老小,来山寨中完聚。宋江喜得破了连环马,又得了许多军马、衣甲、盔
刀,每日做筵席庆喜,仍旧调拨各路守把,提防官兵,不在话下。

却说呼延灼折了许多官军人马,不敢回京,独自一个骑着那匹踢雪乌
骓马,把衣甲拴在马上,于路逃难,却无盘缠;解下束腰金带,卖来盘缠,
在路寻思道:“不想今日闪得我如此,却是去投谁好?”猛然想起:“青
州慕容知府,旧与我有一面相识,何不去那里投奔他?却打慕容贵妃的关
节,那时再引军来报仇未迟。”

在路行了二日,当晚又饥又渴。见路旁一个村酒店,呼延灼下马,把
马拴在门前树上;入来店内,把鞭子放在桌上,坐下了,叫酒保取酒肉来
吃。酒保道:“小人这里只卖酒,要肉时,村里却才杀羊。若要,小人去
回买。”呼延灼把腰里料袋解下来,取出些金带倒换的碎银两,把与酒保
道:“你可回一脚羊肉,与我煮了,就对付草料,喂养我这匹马。今夜只
就你这里宿一宵,明日自投青州府里去。”酒保道:“官人,此间宿不妨,
只是没好床帐。”呼延灼道:“我是出军的人,但有歇处便罢。”酒保拿
了银子,自去买羊肉。呼延灼把马背上捎的衣甲取将下来,松了肚带,坐
在门前,等了半晌,只见酒保提一脚羊肉归来。呼延灼便叫煮了,回三斤
面来打饼,打两角酒来。酒保一面煮肉打饼,一面烧脚汤,与呼延灼洗了
脚,便把马牵放屋后小屋下。酒保一面切草煮料,呼延灼先讨热酒吃了一
回。少刻肉熟,呼延灼叫酒保,也与他些酒肉吃了,分付道:“我是朝廷
军官,为因收捕梁山泊失利,待往青州投慕容知府,你好生与我喂养这匹
马。——是今上御赐的,名为踢雪乌骓马。明日我重重赏你。”酒保道:
“感承相公。却有一件事教相公得知:离此间不远,有座山,唤做桃花山。
山上有一伙强人,为头的是打虎将李忠,第二个是小霸王周通,聚集着五
七百小喽罗,打家劫舍,时常来搅恼村坊。官司累次着仰捕盗官军来,收
捕他不得,相公夜间须用小心醒睡。”呼延灼说道:“我有万夫不当之勇,
便道那厮们全伙都来,也待怎生!只与我好生喂养这匹马。”吃了一回酒
肉饼子,酒保就店里打了一铺,安排呼延灼睡了。

一者呼延灼连日心闷,二乃又多了几杯酒,就和衣而卧。一觉直睡到
三更方醒,只听得屋后酒保在那里叫屈起来。呼延灼听得,连忙跳将起来,
提了双鞭,走去屋后问道:“你如何叫屈?”酒保道:“小人起来上草,
只见篱笆推翻,被人将相公的马偷将去了。远远地望见三四里火把尚明,
一定是那里去了。”呼延灼道:“那里正是何处?”酒保道:“眼见得那
条路上,正是桃花山小喽罗偷得去了。”呼延灼吃了一惊,便叫酒保引路,
就田塍上赶了二三里。火把看看不见,正不知投那里去了。呼延灼说道:
“若无了御赐的马,却怎的是好!”酒保道:“相公明日须去州里告了,
差官军来剿捕,方才能够这匹马。”

呼延灼闷闷不已,坐到天明,叫酒保挑了衣甲,径投青州。来到城里
时,天色已晚了,且在客店里歇了一夜。次日天晓,径到府堂阶下,参拜
了慕容知府。知府大惊,问道:“闻知将军收捕梁山泊草寇,如何却到此
间?”呼延灼只得把上项诉说了一遍。慕容知府听了道:“虽是将军折了
许多人马,此非慢功之罪,中了贼人奸计,亦无奈何。下官所辖地面,多
被草寇侵害。将军到此,可先扫清桃花山,夺取那匹御赐的马;却连那二
龙山、白虎山两处强人,一发剿捕了时,下官自当一力保奏,再教将军引
兵复仇如何?”呼延灼再拜道:“深谢恩相主监。若蒙如此,誓当效死报
德!”慕容知府教请呼延灼去客房里暂歇,一面更衣宿食。那挑甲酒保,
自叫他回去了。

一住三日,呼延灼急欲要这匹御赐马,又来禀复知府,便教点军。慕
容知府便点马步军二千,借与呼延灼,又与了一匹青鬃马。呼延灼谢了恩
相,披挂上马,带领军兵前来夺马,径往桃花山进发。

且说桃花山上打虎将李忠与小霸王周通,自得了这匹踢雪乌骓马,每
日在山上庆喜饮酒。当日有伏路小喽罗报道:“青州军马来也!”小霸王
周通起身道:“哥哥守寨,兄弟去退官军。”便点起一百小喽罗,绰枪上
马,下山来迎敌官军。

却说呼延灼引起二千兵马来到山前,摆开阵势,呼延灼当先出马,厉
声高叫:“强贼早来受缚!”小霸王周通将小喽罗一字摆开,便挺枪出马。
怎生打扮:
身着团花宫锦袄,手持走水绿沉枪。
声雄面阔须如戟,尽道周通赛霸王。
呼延灼见了周通,便纵马向前来战。周通也跃马来迎。二马相交,斗不到
六七合,周通气力不加,拨转马头,往山上便走。呼延灼赶了一直,怕有
计策,急下山来,扎住寨栅,等候再战。

却说周通回寨,见了李忠,诉说:“呼延灼武艺高强,遮拦不住,只
得且退上山;倘或他赶到寨前来,如之奈何!”李忠道:“我闻二龙山宝
珠寺花和尚鲁智深在彼,多有人伴;更兼有个甚么青面兽杨志,又新有个
行者武松,都有万夫不当之勇。不如写一封书,使小喽罗去那里求救。若
解得危难,拚得投托他大寨,月终纳他些进奉也好。”周通道:“小弟也
多知他那里豪杰,只恐那和尚记当初之事,不肯来救。”李忠笑道:“他
那时又打了你,又得了我们许多金银酒器,如何倒有见怪之心?他是个直
性的好人,使人到彼,必然亲引军来救应。”周通道:“哥哥也说得是。”
就写了一封书,差两个了事的小喽罗,从后山踅将下去,取路投二龙山来。
行了两日,早到山下,那里小喽罗问了备细来情。

且说宝珠寺里大殿上坐着三个头领:为首是花和尚鲁智深,第二是青
面兽杨志,第三是行者二郎武松。前面山门下坐着四个小头领:一个是金
眼彪施恩,原是孟州牢城施管营的儿子,为因武松杀了张都监一家人口,
官司着落他家追捉凶身,以此连夜挈家逃走在江湖上。后来父母俱亡,打
听得武松在二龙山,连夜投奔入伙。一个是操刀鬼曹正,原是同鲁智深、
杨志收夺宝珠寺,杀了邓龙,后来入伙。一个是菜园子张青,一个是母夜
叉孙二娘。这是夫妻两个,原是孟州道十字坡卖人肉馒头的;因鲁智深、
武松连连寄书招他,亦来投奔入伙。曹正听得说桃花山有书,先来问了详
细,直去殿上,禀复三个大头领知道。智深便道:“洒家当初离五台山时,
到一个桃花村投宿,好生打了那周通撮鸟一顿。李忠那厮,却来认得洒家,
却请去上山吃了一日酒,结识洒家为兄,留俺做个寨主。俺见这厮们悭吝,
被俺卷了若干金银酒器撒开他。如今来求救,且看他说甚么。放那小喽罗
上关来。”

曹正去不多时,把那小喽罗引到殿下,唱了喏,说道:“青州慕容知
府,近日收得个征进梁山泊失利的双鞭呼延灼。如今慕容知府,先教扫荡
俺这里桃花山、二龙山、白虎山几座山寨,却借军与他收捕梁山泊复仇。
俺的头领,今欲启请大头领将军,下山相救,明朝无事了时,情愿来纳进
奉。”杨志道:“俺们各守山寨,保护山头,本不去救应的是。洒家一者
怕坏了江湖上豪杰,二者恐那厮得了桃花山,便小觑了洒家这里。可留下
张青、孙二娘、施恩、曹正,看守寨栅,俺三个亲自走一遭。”随即点起
五百小喽罗,六十余骑军马,各带了衣甲军器,径往桃花山来。

却说李忠知二龙山消息,自引了三百小喽罗下山策应。呼延灼闻知,
急领所部军马,拦路列阵,舞鞭出马,来与李忠相持。怎见李忠模样:
头尖骨脸似蛇形,枪棒林中独擅名。
打虎将军心胆大,李忠祖是霸陵生。

原来李忠祖贯濠州定远人氏,家中祖传靠使枪棒为生,人见他身材壮
健,因此呼他做“打虎将”。当时下山来与呼延灼交战,李忠如何敌得呼
延灼过,斗了十合之上,见不是头,拨开军器便走。呼延灼见他本事低微,
纵马赶上山来。小霸王周通正在半山里看见,便飞下鹅卵石来,呼延灼慌
忙回马下山来。只见官军迭头呐喊,呼延灼便问道:“为何呐喊?”后军
答道:“远望见一彪军马飞奔而来。”呼延灼听了,便来后军队里看时,
见尘头起处,当头一个胖大和尚,骑一匹白马,那人是谁?正是:

自从落发寓禅林,万里曾将壮士寻。臂负千斤扛鼎力,天生一片杀人
心。

欺佛祖,喝观音,戒刀禅杖冷森森。不看经卷花和尚,酒肉沙门
鲁智深。

鲁智深在马上大喝道:“那个是梁山泊杀败的撮鸟,敢来俺这里唬吓
人!”呼延灼道:“先杀你这个秃驴,豁我心中怒气!”鲁智深抡动铁禅
杖,呼延灼舞起双鞭,二马相交,两边呐喊。斗四五十合,不分胜败。呼
延灼暗暗喝采道:“这个和尚,倒恁地了得!”两边鸣金,各自收军暂歇。

呼延灼少停,再纵马出阵,大叫:“贼和尚再出来,与你定个输赢,
见个胜败!”鲁智深却待正要出马,侧首恼犯了这个英雄,叫道:“大哥
少歇,看洒家去捉这厮!”那人舞刀出马。来战呼延灼的是谁?正是:

曾向京师为制使,花石纲累受艰难。虹霓气逼牛斗寒。刀能安宇宙,
弓可定尘寰。

虎体狼腰猿臂健,跨龙驹稳坐雕鞍。英雄声价满梁山。
人称青面兽,杨志是军班。

当下杨志出马,来与呼延灼交锋。两个斗到四十余合,不分胜败。呼
延灼见杨志手段高强,寻思道:“怎的那里走出这两个来?好生了得!不是
绿林中手段!”杨志也见呼延灼武艺高强,卖个破绽,拨回马,跑回本阵。
呼延灼也勒转马头,不来追赶。两边各自收军。鲁智深便和杨志商议道:
“俺们初到此处,不宜逼近下寨。且退二十里,明日却再来厮杀。”带领
小喽罗,自过附近山冈下寨去了。

却说呼延灼在帐中纳闷,心内想道:“指望到此势如劈竹,便拿了这
伙草寇,怎知却又逢着这般对手!我直如此命薄!”正没摆布处,只见慕
容知府使人来唤道:“叫将军且领兵回来,保守城中。今有白虎山强人孔
明、孔亮,引人马来青州借粮,怕府库有失,特令来请将军回城守备。”
呼延灼听了,就这机会,带领军马,连夜回青州去了。

次日,鲁智深与杨志、武松,又引了小喽罗摇旗呐喊,直到山下来看
时,一个军马也无了,倒吃了一惊。山上李忠、周通,引人下来,拜请三
位头领上到山寨里,杀牛宰马,筵席相待,一面使人下山,探听前路消息。

且说呼延灼引军回到城下,却见了一彪军马,正来到城边。为头的乃
是白虎山下孔太公的儿子毛头星孔明、独火星孔亮。两个因和本乡一个财
主争竞,把他一门良贱尽都杀了,聚集起五七百人,占住白虎山,打家劫
舍。因为青州城里有他的叔叔孔宾,被慕容知府捉下,监在牢里,孔明、
孔亮特地点起山寨小喽罗,来打青州,要救叔叔孔宾。正迎着呼延灼军马,
两边拥着,敌住厮杀,呼延灼便出马到阵前。慕容知府在城楼上观看,见
孔明当先,挺枪出马,直取呼延灼。两马相交,斗到二十余合,呼延灼要
在知府跟前显本事。又值孔明武艺不精,只办得架隔遮拦,斗到间深里,
被呼延灼就马上把孔明活捉了去,孔亮只得引了小喽罗便走。慕容知府在
敌楼上指着,叫呼延灼引军去赶,官兵一掩,活捉得百十余人。孔亮大败,
四散奔走,至晚寻个古庙安歇。

却说呼延灼活捉得孔明,解入城中,来见慕容知府。知府大喜,叫把
孔明大枷钉下牢里,和孔宾一处监收,一面赏劳三军,一面管待呼延灼,
备问桃花山消息。呼延灼道:“本待是‘瓮中捉鳖,手到拿来’,无端又
被一伙强人前来救应;数内一个和尚,一个青脸大汉,二次交锋,各无胜
败。这两个武艺不比寻常,不是绿林中手段,因此未曾拿得。”慕容知府
道:“这个和尚,便是延安府老种经略帐前军官提辖鲁达,今次落发为僧,
唤做花和尚鲁智深。这一个青脸大汉,亦是东京殿帅府制使官,唤做青面
兽杨志。再有一个行者,唤做武松,原是景阳冈打虎的武都头。这三个占
住二龙山,打家劫舍,累次拒敌官军,杀了三五个捕盗官,直至如今,未
曾捉得。”呼延灼道:“我见这厮们武艺精熟,原来却是杨制使和鲁提辖,
名不虚传!恩相放心,呼延灼已见他们本事了。只在早晚,一个个活捉了
解官。”知府大喜,设筵管待已了,且请客房内歇,不在话下。

却说孔亮引了败残人马,正行之间,猛可里树林中撞出一彪军马,当
先一筹好汉,怎生打扮,有《西江月》为证:

直裰冷披黑雾,戒箍光射秋霜。额前剪发拂眉长,脑后护头齐项。

顶骨数珠灿白,杂绒绦结微黄。钢刀两口迸寒光,行者武松形象。
孔亮见了是武松,慌忙滚鞍下马,便拜道:“壮士无恙?”武松连忙答应,
扶起问道:“闻知足下弟兄们占住白虎山聚义,几次要来拜望,一者不得
下山,二乃路途不顺,以此难得相见。今日何事到此?”孔亮把救叔叔孔
宾陷兄之事,告诉了一遍。武松道:“足下休慌。我有六七个弟兄,现在
二龙山聚义。今为桃花山李忠、周通,被青州官军攻击得紧,来我山寨求
救。鲁、杨二头领引了孩儿们先来与呼延灼交战。两个厮并了一日,呼延
灼夜间去了。山寨中留我弟兄三人筵宴,把这匹御赐马送与我们。今我部
领头队人马回山,他二位随后便到。我叫他去打青州。救你叔兄如何?”

孔亮拜谢武松,等了半晌,只见鲁智深、杨志两个并马都到。武松引
孔亮拜见二位,备说:“那时我与宋江在他庄上相会,多有相扰。今日俺
们可以义气为重,聚集三山人马,攻打青州,杀了慕容知府,擒获呼延灼,
各取府库钱粮,以供山寨之用,如何?”鲁智深道:“洒家也是这般思想。
便使人去桃花山报知,叫李忠、周通引孩儿们来,俺三处一同去打青州。”

杨志便道:“青州城池坚固,人马强壮,又有呼延灼那厮英勇。不是
俺自灭威风,若要攻打青州时,只除非依我一言,指日可得。”武松道:
“哥哥,愿闻其略。”那杨志言无数句,话不一席,有分教:青州百姓,
家家瓦裂烟飞;水浒英雄,个个磨拳擦掌。

毕竟杨志对武松说出怎地打青州,且听下回分解。