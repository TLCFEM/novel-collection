\chapter{杨志押送金银担~吴用智取生辰纲}

话说当时公孙胜正在阁儿里对晁盖说这北京生辰纲是不义之财,取之何碍。只
见一个人从外面抢将入来,揪住公孙胜道:“你好大胆!却才商议的事,我都知了
也。”那人却是智多星吴学究。晁盖笑道:“教授休慌,且请相见。”两个叙礼罢,
吴用道:“江湖上久闻人说入云龙公孙胜一清大名,不期今日此处得会!”晁盖道:
“这位秀才先生,便是智多星吴学究。”公孙胜道:“吾闻江湖上多人曾说加亮先
生大名,岂知缘法却在保正庄上得会。只是保正疏财仗义,以此天下豪杰,都投门
下。”晁盖道:“再有几个相识在里面,一发请进后堂深处相见。”

三个人入到里面,就与刘唐、三阮都相见了。正是:
金帛多藏祸有基,英雄聚会本无期。
一时豪侠欺黄屋,七宿光芒动紫薇。

众人道:“今日此一会,应非偶然,须请保正哥哥正面而坐。”晁盖道:“量
小子是个穷主人,怎敢占上!”吴用道:“保正哥哥年长,依着小生,且请坐了。”
晁盖只得坐了第一位,吴用坐了第二位,公孙胜坐了第三位,刘唐坐了第四位,阮
小二坐了第五位,阮小五坐第六位,阮小七坐第七位。却才聚义饮酒,重整杯盘,
再备酒肴,众人饮酌。吴用道:“保正梦见北斗七星坠在屋脊上,今日我等七人聚
义举事,岂不应天垂象!此一套富贵,唾手而取。前日所说央刘兄去探听路程从那
里来,今日天晚,来早便请登程。”公孙胜道:“这一事不须去了。贫道已打听,
知他来的路数了,只是黄泥冈大路上来。”晁盖道:“黄泥冈东十里路,地名安乐
村,有一个闲汉,叫做白日鼠白胜,也曾来投奔我,我曾赍助他盘缠。”吴用道:
“北斗上白光,莫不是应在这人?自有用他处。”刘唐道:“此处黄泥冈较远,何
处可以容身?”吴用道:“只这个白胜家便是我们安身处,亦还要用了白胜。”晁
盖道:“吴先生,我等还是软取,却是硬取?”吴用笑道:“我已安排定了圈套,
只看他来的光景,力则力取,智则智取。我有一条计策,不知中你们意否?如此,
如此……”晁盖听了大喜,攧着脚道:“好妙计!不枉了称你做智多星!果然赛过诸
葛亮!好计策!”吴用道:“休得再提,常言道:‘隔墙须有耳,窗外岂无人。’
只可你知我知。”晁盖便道:“阮家三兄且请回归,至期来小庄聚会;吴先生依旧
自去教学;公孙先生并刘唐,只在敝庄权住。”当日饮酒至晚,各自去客房里歇息。

次日五更起来,安排早饭吃了,晁盖取出三十两花银,送与阮家三兄弟道:“权
表薄意,切勿推却。”三阮那里肯受。吴用道:“朋友之意,不可相阻。”三阮方
才受了银两。一齐送出庄外来,吴用附耳低言道:“这般这般,至期不可有误。”
三阮相别了,自回石碣村去。晁盖留住公孙胜、刘唐在庄上,吴学究常来议事。正
是:
取非其有官皆盗,损彼盈余盗是公。
计就只须安稳待,笑他宝担去匆匆。

话休絮繁,却说北京大名府梁中书收买了十万贯庆贺生辰礼物完备,选日差人
起程,当下一日在后堂坐下,只见蔡夫人问道:“相公,生辰纲几时起程?”梁中
书道:“礼物都已完备,明后日便用起身。只是一件事,在此踌躇未决。”蔡夫人
道:“有甚事踌躇未决?”梁中书道:“上年费了十万贯收买金珠宝贝,送上东京
去,只因用人不着,半路被贼人劫将去了,至今无获。今年帐前眼见得又没个了事
的人送去,在此踌躇未决。”蔡夫人指着阶下道:“你常说这个人十分了得,何不
着他,委纸领状,送去走一遭,不致失误。”

梁中书看阶下那人时,却是青面兽杨志。梁中书大喜,随即唤杨志上厅说道:
“我正忘了你,你若与我送得生辰纲去,我自有抬举你处。”杨志叉手向前禀道:
“恩相差遣,不敢不依!只不知怎地打点?几时起身?”梁中书道:“着落大名府差
十辆太平车子,帐前拨十个厢禁军监押着车,每辆上各插一把黄旗,上写着:‘献
贺太师生辰纲’。每辆车子再使个军健跟着,三日内便要起身去。”杨志道:“非
是小人推托,其实去不得,乞钧旨别差英雄精细的人去。”梁中书道:“我有心要
抬举你,这献生辰纲的札子内,另修一封书在中间,太师跟前重重保你受道敕命回
来,如何倒生支调,推辞不去?”杨志道:“恩相在上,小人也曾听得上年已被贼
人劫去了,至今未获。今岁途中盗贼又多,此去东京,又无水路,都是旱路。经过
的是紫金山、二龙山、桃花山、伞盖山、黄泥冈、白沙坞、野云渡、赤松林,这几
处都是强人出没的去处。更兼单身客人亦不敢独自经过,他知道是金银宝物,如何
不来抢劫?枉结果了性命,以此去不得。”梁中书道:“恁地时,多着军校防护送
去便了。”杨志道:“恩相便差五百人去,也不济事。这厮们一声听得强人来时,
都是先走了的。”梁中书道:“你这般地说时,生辰纲不要送去了?”杨志又禀道:
“若依小人一件事,便敢送去。”梁中书道:“我既委在你身上,如何不依你说?”
杨志道:“若依小人说时,并不要车子,把礼物都装做十余条担子,只做客人的打
扮行货。也点十个壮健的厢禁军,却装做脚夫挑着。只消一个人和小人去,却打扮
做客人,悄悄连夜上东京交付,恁地时方好。”梁中书道:“你甚说的是。我写书
呈重重保你受道诰命回来。”杨志道:“深谢恩相抬举。”当日便叫杨志一面打拴
担脚,一面选拣军人。

次日,叫杨志来厅前伺候,梁中书出厅来问道:“杨志,你几时起身?”杨志
禀道:“告复恩相,只在明早准行,就委领状。”梁中书道:“夫人也有一担礼物,
另送与府中宝眷,也要你领。怕你不知头路,特地再教奶公谢都管,并两个虞候,
和你一同去。”杨志告道:“恩相,杨志去不得了。”梁中书说道:“礼物都已拴
缚完备,如何又去不得?”杨志禀道:“此十担礼物都在小人身上,和他众人,都
由杨志,要早行,便早行,要晚行,便晚行,要住,便住,要歇,便歇,亦依杨志
提调。如今又叫老都管并虞候和小人去,他是夫人行的人,又是太师府门下奶公,
倘或路上与小人别拗起来,杨志如何敢和他争执得?若误了大事时,杨志那其间如
何分说?”梁中书道:“这个也容易,我叫他三个都听你提调便了。”杨志答道:
“若是如此禀过,小人情愿便委领状。倘有疏失,甘当重罪。”梁中书大喜道:“我
也不枉了抬举你,真个有见识!”随即唤老谢都管并两个虞候出来,当厅分付道:
“杨志提辖情愿委了一纸领状,监押生辰纲,十一担金珠宝贝,赴京太师府交割,
这干系都在他身上。你三人和他做伴去,一路上早起,晚行,住歇,都要听他言语,
不可和他别拗。夫人处分付的勾当,你三人自理会,小心在意,早去早回,休教有
失。”老都管一一都应了。

当日杨志领了,次日早起五更,在府里把担仗都摆在厅前,老都管和两个虞候
又将一小担财帛,共十一担,拣了十一个壮健的厢禁军,都做脚夫打扮。杨志戴上
凉笠儿,穿着青纱衫子,系了缠带行履麻鞋,跨口腰刀,提条朴刀;老都管也打扮
做个客人模样;两个虞候假装做跟的伴当。各人都拿了条朴刀,又带几根藤条。梁
中书付与了札付书呈,一行人都吃得饱了,在厅上拜辞了梁中书。看那军人担仗起
程。杨志和谢都管、两个虞候监押着,一行共是十五人,离了梁府,出得北京城门,
取大路投东京进发。

此时正是五月半天气,虽是晴明得好,只是酷热难行。昔日吴七郡王有八句诗
道:
玉屏四下朱阑绕,簇簇游鱼戏萍藻。
簟铺八尺白虾须,头枕一枚红玛瑙。
六龙惧热不敢行,海水煎沸蓬莱岛。
公子犹嫌扇力微,行人正在红尘道。
这八句诗单题着炎天暑月,那公子王孙在凉亭上水阁中浸着浮瓜沉李,调冰雪藕避
暑,尚兀自嫌热;怎知客人为些微名薄利,又无枷锁拘缚,三伏内,只得在那途路
中行。今日杨志这一行人要取六月十五日生辰,只得在路途上行。自离了这北京五
七日,端的只是起五更,趁早凉便行,日中热时便歇。

五七日后,人家渐少,行路又稀,一站站都是山路。杨志却要辰牌起身,申时
便歇。那十一个厢禁军,担子又重,无有一个稍轻,天气热了行不得,见着林子,
便要去歇息,杨志赶着催促要行。如若停住,轻则痛骂,重则藤条便打,逼赶要行。
两个虞候虽只背些包裹行李,也气喘了行不上。杨志也嗔道:“你两个好不晓事!
这干系须是俺的,你们不替洒家打这夫子,却在背后也慢慢地挨,这路上不是耍处!”
那虞候道:“不是我两个要慢走,其实热了行不动,因此落后。前日只是趁早凉走,
如今怎地正热里要行,正是好歹不均匀。”杨志道:“你这般说话,却似放屁!前
日行的须是好地面,如今正是尴尬去处,若不日里赶过去,谁敢五更半夜走?”两
个虞候口里不道,肚中寻思:“这厮不直得便骂人。”杨志提了朴刀,拿着藤条,
自去赶那担子。

两个虞候坐在柳阴树下,等得老都管来,两个虞候告诉道:“杨家那厮,强杀
只是我相公门下一个提辖,直这般会做大老!”都管道:“须是相公当面分付道休
要和他别拗,因此我不做声,这两日也看他不得,权且耐他。”两个虞候道:“相
公也只是人情话儿,都管自做个主便了。”老都管又道:“且耐他一耐。”

当日行到申牌时分,寻得一个客店里歇了。那十个厢禁军雨汗通流,都叹气吹
嘘,对老都管说道:“我们不幸,做了军健,情知道被差出来,这般火似热的天气,
又挑着重担,这两日又不拣早凉行,动不动老大藤条打来,都是一般父母皮肉,我
们直恁地苦!”老都管道:“你们不要怨怅,巴到东京时,我自赏你。”众军汉道:
“若是似都管看待我们时,并不敢怨怅。”

又过了一夜,次日天色未明,众人起来,都要趁凉起身去。杨志跳起来喝道:
“那里去!且睡了,却理会。”众军汉道:“趁早不走,日里热时走不得,却打我
们。”杨志大骂道:“你们省得甚么?”拿了藤条要打,众军忍气吞声,只得睡了。
当日直到辰牌时分,慢慢地打火,吃了饭走,一路上赶打着,不许投凉处歇。那十
一个厢禁军口里喃喃讷讷地怨怅,两个虞候在老都管面前絮絮聒聒地搬口,老都管
听了,也不着意,心内自恼他。

话休絮繁,似此行了十四五日,那十四个人没一个不怨怅杨志。当日客店里辰
牌时分慢慢地打火,吃了早饭行,正是六月初四日时节,天气未及晌午,一轮红日
当天,没半点云彩,其日十分大热。古人有八句诗道:
祝融南来鞭火龙,火旗焰焰烧天红。
日轮当午凝不去,万国如在红炉中。
五岳翠干云彩灭,阳侯海底愁波竭。
何当一夕金风起,为我扫除天下热。
当日行的路,都是山僻崎岖小径,南山北岭,却监着那十一个军汉,约行了二十余
里路程。那军人们思量要去柳阴树下歇凉,被杨志拿着藤条打将来,喝道:“快走!
教你早歇!”众军人看那天时,四下里无半点云彩,其时那热不可当。但见:

热气蒸人,嚣尘扑面。万里乾坤如甑,一轮火伞当天。四野无云,风寂寂树焚
溪坼;千山灼焰,剥剥石裂灰飞。空中鸟雀命将休,倒攧入树林深处;水底鱼龙
鳞角脱,直钻入泥土窖中。直教石虎喘无休,便是铁人须汗落。
当时杨志催促一行人在山中僻路里行,看看日色当午,那石头上热了,脚疼走不得。
众军汉道:“这般天气热,兀的不晒杀人!”杨志喝着军汉道:“快走,赶过前面
冈子去,却再理会。”正行之间,前面迎着那土冈子。众人看这冈子时,但见:

顶上万株绿树,根头一派黄沙。嵯峨浑似老龙形,险峻但闻风雨响。山边茅草,
乱丝丝攒遍地刀枪;满地石头,碜可可睡两行虎豹。休道西川蜀道险,须知此是太
行山。

当时一行十五人奔上冈子来,歇下担仗,那十四人都去松阴树下睡倒了。杨志
说道:“苦也!这里是甚么去处,你们却在这里歇凉?起来快走!”众军汉道:“你
便剁做我七八段,其实去不得了!”杨志拿起藤条,劈头劈脑打去,打得这个起来,
那个睡倒,杨志无可奈何。

只见两个虞候和老都管气喘急急,也巴到冈子上松树下坐了喘气。看这杨志打
那军健,老都管见了说道:“提辖,端的热了走不得,休见他罪过。”杨志道:“都
管,你不知这里正是强人出没的去处,地名叫做黄泥冈。闲常太平时节,白日里兀
自出来劫人,休道是这般光景,谁敢在这里停脚!”两个虞候听杨志说了,便道:
“我见你说好几遍了,只管把这话来惊吓人!”老都管道:“权且教他们众人歇一
歇,略过日中行如何?”杨志道:“你也没分晓了!如何使得?这里下冈子去,兀自
有七八里没人家,甚么去处,敢在此歇凉!”老都管道:“我自坐一坐了走,你自
去赶他众人先走。”

杨志拿着藤条喝道:“一个不走的,吃俺二十棍。”众军汉一齐叫将起来,数
内一个分说道:“提辖,我们挑着百十斤担子,须不比你空手走的,你端的不把人
当人!便是留守相公自来监押时,也容我们说一句,你好不知疼痒,只顾逞辩!”
杨志骂道:“这畜生不怄死俺!只是打便了。”拿起藤条,劈脸便打去。老都管喝
道:“杨提辖,且住!你听我说:我在东京太师府里做奶公时,门下官军,见了无
千无万,都向着我喏喏连声。不是我口栈,量你是个遭死的军人,相公可怜抬举你
做个提辖,比得芥菜子大小的官职,直得恁地逞能!休说我是相公家都管,便是村
庄一个老的,也合依我劝一劝;只顾把他们打,是何看待?”杨志道:“都管,你
须是城市里人,生长在相府里,那里知道途路上千难万难。”老都管道:“四川、
两广也曾去来,不曾见你这般卖弄。”杨志道:“如今须不比太平时节。”都管道:
“你说这话,该剜口割舌,今日天下恁地不太平?”

杨志却待再要回言,只见对面松林里影着一个人,在那里舒头探脑价望,杨志
道:“俺说甚么?兀的不是歹人来了!”撇下藤条,拿了朴刀,赶入松林里来喝一
声道:“你这厮好大胆,怎敢看俺的行货!”正是:
说鬼便招鬼,说贼便招贼。
却是一家人,对面不能识。
杨志赶来看时,只见松林里一字儿摆着七辆江州车儿,七个人脱得赤条条的在那里
乘凉,一个鬓边老大一搭朱砂记,拿着一条朴刀,望杨志跟前来,七个人齐叫一声:
“呵也!”都跳起来。杨志喝道:“你等是甚么人?”那七人道:“你是甚么人?”
杨志又问道:“你等莫不是歹人?”那七人道:“你颠倒问,我等是小本经纪,那
里有钱与你?”杨志道:“你等小本经纪人,偏俺有大本钱!”那七人问道:“你
端的是甚么人?”杨志道:“你等且说那里来的人?”那七人道:“我等弟兄七人
是濠州人,贩枣子上东京去,路途打从这里经过,听得多人说这里黄泥冈上时常有
贼打劫客商。我等一面走,一头自说道:‘我七个只有些枣子,别无甚财赋。’只
顾过冈子来。上得冈子,当不过这热,权且在这林子里歇一歇,待晚凉了行。只听
得有人上冈子来,我们只怕是歹人,因此使这个兄弟出来看一看。”杨志道:“原
来如此,也是一般的客人。却才见你们窥望,惟恐是歹人,因此赶来看一看。”那
七个人道:“客官请几个枣子了去。”杨志道:“不必。”提了朴刀,再回担边来。
老都管道:“既是有贼,我们去休。”杨志说道:“俺只道是歹人,原来是几个贩
枣子的客人。”老都管道:“似你方才说时,他们都是没命的!”杨志道:“不必
相闹,只要没事便好。你们且歇了,等凉些走。”众军汉都笑了。杨志也把朴刀插
在地上,自去一边树下坐了歇凉。

没半碗饭时,只见远远地一个汉子挑着一副担桶,唱上冈子来,唱道:“赤日
炎炎似火烧,野田禾稻半枯焦。农夫心内如汤煮,公子王孙把扇摇。”那汉子口里
唱着,走上冈子来,松林里头歇下担桶,坐地乘凉。众军看见了,便问那汉子道:
“你桶里是甚么东西?”那汉子应道:“是白酒。”众军道:“挑往那里去?”那
汉子道:“挑出村里卖。”众军道:“多少钱一桶?”那汉子道:“五贯足钱。”
众军商量道:“我们又热又渴,何不买些吃,也解暑气。”正在那里凑钱,杨志见
了,喝道:“你们又做甚么?”众军道:“买碗酒吃。”杨志调过朴刀杆便打,骂
道:“你们不得洒家言语,胡乱便要买酒吃,好大胆!”众军道:“没事又来鸟乱!
我们自凑钱买酒吃,干你甚事?也来打人!”杨志道:“你这村鸟,理会的甚么!到
来只顾吃嘴!全不晓得路途上的勾当艰难,多少好汉,被蒙汗药麻翻了!”那挑酒
的汉子看着杨志冷笑道:“你这客官好不晓事!早是我不卖与你吃,却说出这般没
气力的话来!”

正在松树边闹动争说,只见对面松林里那伙贩枣子的客人都提着朴刀,走出来
问道:“你们做甚么闹?”那挑酒的汉子道:“我自挑这酒过冈子村里卖,热了,
在此歇凉,他众人要问我买些吃,我又不曾卖与他。这个客官道我酒里有甚么蒙汗
药,你道好笑么?说出这般话来!”那七个客人说道:“我只道有歹人出来,原来
是如此,说一声也不打紧。我们正想酒来解渴,既是他们疑心,且卖一桶与我们吃。”
那挑酒的道:“不卖!不卖!”这七个客人道:“你这鸟汉子也不晓事,我们须不
曾说你。你左右将到村里去卖,一般还你钱,便卖些与我们,打甚么不紧?看你不
道得舍施了茶汤,便又救了我们热渴。”那挑酒的汉子便道:“卖一桶与你,不争,
只是被他们说的不好,又没碗瓢舀吃。”那七人道:“你这汉子忒认真!便说了一
声,打甚么不紧?我们自有椰瓢在这里。”只见两个客人去车子前取出两个椰瓢来,
一个捧出一大捧枣子来,七个人立在桶边,开了桶盖,轮替换着舀那酒吃,把枣子
过口。无一时,一桶酒都吃尽了。七个客人道:“正不曾问得你多少价钱?”那汉
道:“我一了不说价,五贯足钱一桶,十贯一担。”七个客人道:“五贯便依你五
贯,只饶我们一瓢吃。”那汉道:“饶不的,做定的价钱。”一个客人把钱还他,
一个客人便去揭开桶盖,兜了一瓢,拿上便吃,那汉去夺时,这客人手拿半瓢酒,
望松林里便走,那汉赶将去。只见这边一个客人从松林里走将出来,手里拿一个瓢,
便来桶里舀了一瓢酒,那汉看见,抢来劈手夺住,望桶里一倾,便盖了桶盖,将瓢
望地下一丢,口里说道:“你这客人好不君子相!戴头识脸的,也这般罗唣!”

那对过众军汉见了,心内痒起来,都待要吃,数中一个看着老都管道:“老爷
爷与我们说一声,那卖枣子的客人买他一桶吃了,我们胡乱也买他这桶吃,润一润
喉也好。其实热渴了,没奈何。这里冈子上又没讨水吃处,老爷方便。”老都管见
众军所说,自心里也要吃得些,竟来对杨志说:“那贩枣子客人已买了他一桶酒吃,
只有这一桶,胡乱教他们买吃些避暑气,冈子上端的没处讨水吃。”杨志寻思道:
“俺在远远处望这厮们都买他的酒吃了,那桶里当面也见吃了半瓢,想是好的。打
了他们半日,胡乱容他买碗吃罢。”杨志道:“既然老都管说了,教这厮们买吃了,
便起身。”

众军健听了这话,凑了五贯足钱,来买酒吃。那卖酒的汉子道:“不卖了!不
卖了!这酒里有蒙汗药在里头!”众军陪着笑说道:“大哥直得便还言语!”那汉
道:“不卖了!休缠!”这贩枣子的客人劝道:“你这个鸟汉子,他也说得差了,
你也忒认真!连累我们也吃你说了几声。须不关他众人之事,胡乱卖与他众人吃些。”
那汉道:“没事讨别人疑心做甚么?”这贩枣子客人把那卖酒的汉子推开一边,只
顾将这桶酒提与众军去吃。那军汉开了桶盖,无甚舀吃,陪个小心,问客人借这椰
瓢用一用。众客人道:“就送这几个枣子与你们过酒。”众军谢道:“甚么道理。”
客人道:“休要相谢,都是一般客人,何争在这百十个枣子上。”众军谢了,先兜
两瓢,叫老都管吃一瓢,杨提辖吃一瓢,杨志那里肯吃。老都管自先吃了一瓢,两
个虞候各吃一瓢。众军汉一发上,那桶酒登时吃尽了。杨志见众人吃了无事,自本
不吃,一者天气甚热,二乃口渴难熬,拿起来只吃了一半,枣子分几个吃了。那卖
酒的汉子说道:“这桶酒被那客人饶一瓢吃了,少了你些酒,我今饶了你众人半贯
钱罢。”众军汉凑出钱来还他。那汉子收了钱,挑了空桶,依然唱着山歌,自下冈
子去了。

那七个贩枣子的客人,立在松树傍边,指着这一十五人说道:“倒也!倒也!”
只见这十五个人头重脚轻,一个个面面厮觑,都软倒了。那七个客人从松树林里推
出这七辆江州车儿,把车子上枣子丢在地上,将这十一担金珠宝贝都装在车子内,
遮盖好了,叫声:“聒噪!”一直望黄泥冈下推了去。正是:
诛求膏血庆生辰,不顾民生与死邻。
始信从来招劫盗,亏心必定有缘因。
杨志口里只是叫苦,软了身体,挣扎不起;十五人眼睁睁地看着那七个人都把这金
宝装了去,只是起不来、挣不动、说不的。我且问你,这七人端的是谁?不是别人,
原来正是晁盖、吴用、公孙胜、刘唐、三阮这七个。却才那个挑酒的汉子,便是白
日鼠白胜。却怎地用药?原来挑上冈子时,两桶都是好酒。七个人先吃了一桶,刘
唐揭起桶盖,又兜了半瓢吃,故意要他们看着,只是叫人死心搭地。次后吴用去松
林里取出药来,抖在瓢里,只做走来饶他酒吃,把瓢去兜时,药已搅在酒里,假意
兜半瓢吃,那白胜劈手夺来,倾在桶里,这个便是计策。那计较都是吴用主张,这
个唤做智取生辰纲。

原来杨志吃的酒少,便醒得快,爬将起来,兀自捉脚不住。看那十四个人时,
口角流涎,都动不得,正应俗语道:“饶你奸似鬼,吃了洗脚水。”杨志愤闷道:
“不争你把了生辰纲去,教俺如何回去见得梁中书?这纸领状须缴不得,就扯破了。
如今闪得俺有家难奔,有国难投,待走那里去?不如就这冈子上寻个死处。”撩衣
破步,望着黄泥冈下便跳。正是:断送落花三月雨,摧残杨柳九秋霜。

毕竟杨志在黄泥冈上寻死,性命如何,且听下回分解。