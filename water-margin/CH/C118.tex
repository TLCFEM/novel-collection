\chapter{卢俊义大战昱岭关~宋公明智取清溪洞}

话说当下关胜等四将,飞马引军,杀到乌龙岭上,正接着石宝军马。关胜在马
上大喝:“贼将安敢杀吾弟兄!”石宝见是关胜,无心恋战。便退上岭去,指挥白
钦,却来战关胜。两马相交,军器并举,两个斗不到十合,乌龙岭上急又鸣锣收军。
关胜不赶,岭上军兵自乱起来。原来石宝只顾在岭东厮杀,却不提防岭西已被童枢
密大驱人马,杀上岭来。宋军中大将王禀,便和南兵指挥景德厮杀。两个斗了十合
之上,王禀将景德斩于马下。自此吕方、郭盛首先奔上山来夺岭,未及到岭边,山
头上早飞下一块大石头,将郭盛和人连马打死在岭边。这面岭东关胜望见岭上大乱,
情知岭西有宋兵上岭了,急招众将,一齐都杀上去。两面夹攻,岭上混战。吕方却
好迎着白钦,两个交手厮杀。斗不到三合,白钦一枪搠来,吕方闪个过,白钦那条
枪从吕方肋下戳个空。吕方这枝戟,却被白钦拨个倒横。两将在马上,各施展不得,
都弃了手中军器,在马上你我厮相揪住。原来正遇着山岭峻处,那马如何立得脚
牢,二将使得力猛,不想连人和马都滚下岭去。这两将做一处攧死在那岭下。

这边关胜等众将步行,都杀上岭来,两面尽是宋兵,已杀到岭上。石宝看见两边全
无去路,恐吃捉了受辱,便用劈风刀自刎而死。宋江众将夺了乌龙岭关隘,关胜急
令人报知宋先锋。江里水寨中四个水军总管,见乌龙岭已失,睦州俱陷,都弃了船
只,逃过对江,被隔岸百姓生擒得成贵、谢福,解送献入睦州。走了翟源、乔正,
不知去向。宋兵大队,回到睦州。宋江得知,出城迎接。童枢密、刘都督入城屯驻,
安营已了,出榜招抚军民复业,南兵投降者勿知其数。宋江尽将仓廒粮米给散百姓,
各归本乡,复为良民。将水军总管成贵、谢福割腹取心,致祭兄弟阮小二、孟康,
并在乌龙岭亡过一应将佐,前后死魂,俱皆受享。再叫李俊等水军将佐,管领了许
多船只,把获到贼首伪官解送张招讨军前去了。宋江又见折了吕方、郭盛,惆怅不
已,按兵不动,等候卢先锋兵马,同取清溪。

且不说宋江在睦州屯驻,却说副先锋卢俊义,自从杭州分兵之后,统领三万人马,
本部下正偏将佐二十八员,引兵取山路望杭州进发,经过临安镇钱王故都,道近昱
岭关前。守关把隘,却是方腊手下一员大将,绰号小养由基庞万春,乃是江南方腊
国中第一个会射弓箭的。带领着两员副将:一个唤做雷炯,一个唤做计稷。这两个
副将,都蹬的七八百斤劲弩,各会使一枝蒺藜骨朵,手下有五千人马。三个守把住
昱岭关隘,听知宋兵分拨副先锋卢俊义引军到来,已都准备下了对敌器械,只待来
军相近。且说卢先锋军马将次近昱岭关前,当日先差史进、石秀、陈达、杨春、李
忠、薛永六员将校,带领三千步军,前去出哨。

当下史进等六将,都骑战马,其余都是步军,迤逦哨到关下,并不曾撞见一个军马。
史进在马上心疑,和众将商议。说言未了,早已来到关前。看时,见关上竖着一面
彩绣白旗,旗下立着那小养由基庞万春,看了史进等大笑,骂道:“你这伙草贼,
只好在梁山泊里住,勒宋朝招安诰命,如何敢来我这国土里装好汉!你也曾闻俺
小养由基的名字么?我听得你这厮伙里,有个甚么小李广花荣,着他出来,和我比
箭。先教你看我神箭。”说言未了,飕的一箭,正中史进,攧下马去。五将一齐急
急向前救得,上马便回。又见山顶上一声锣响,左右两边松树林里,一齐放箭。五
员将顾不得史进,各人逃命而走。转得过山嘴,对面两边山坡上,一边是雷炯,一
边是计稷,那弩箭如雨一般射将来,总是有十分英雄,也躲不得这般的箭矢。可怜
水浒六员将佐,都作南柯一梦。史进、石秀等六人,不曾透得一个出来,做一堆儿
都被射死在关下。

三千步卒,止剩得百余个小军,逃得回来,见卢先锋说知此事。卢先锋听了大惊,
如痴似醉,呆了半晌。神机军师朱武为陈达、杨春垂泪已毕,谏道:“先锋且勿烦
恼,有误大事,可以别商量一个计策,去夺关斩将,报此仇恨。”卢俊义道:“宋
公明兄长特分许多将校与我,今番不曾赢得一阵,首先倒折了六将,更兼三千军卒,
止有得百余人回来,似此怎生到歙州相见?”朱武答道:“古人有云:‘天时不如
地利,地利不如人和。’我等皆是中原、山东、河北人氏,不曾惯演水战,因此失
了地利。须获得本处乡民,指引路径,方才知得他此间山路曲折。”卢先锋道:“军
师言之极当,差谁去缉探路径好?”朱武道:“论我愚意,可差鼓上蚤时迁。他是
个飞檐走壁的人,好去山中寻路。”卢俊义随即教唤时迁,领了言语,捎带了干粮,
跨口腰刀,离寨去了。

且说时迁便望深山去处,只顾走寻路,去了半日,天色已晚,来到一个去处,远远
地望见一点灯光明朗。时迁道:“灯光处必有人家。”趁黑地里,摸到灯明之处看
时,却是个小小庵堂,里面透出灯光来。时迁来到庵前,便钻入去看时,见里面一
个老和尚,在那里坐地诵经。时迁便乃敲他房门,那老和尚唤一个小行者来开门。
时迁进到里面,便拜老和尚。

那老僧便道:“客官休拜。现今万马千军厮杀之地,你如何走得到这里?”时迁应
道:“实不敢瞒师父说,小人是梁山泊宋江的部下一个偏将时迁的便是。今来奉圣
旨剿收方腊,谁想夜来被昱岭关上守把贼将,乱箭射死了我六员首将,无计度关,
特差时迁前来寻路,探听有何小路过关。今从深山旷野寻到此间,万望师父指迷,
有何小径,私越过关,当以厚报。”那老僧道:“此间百姓,俱被方腊残害,无一
个不怨恨他。老僧亦靠此间当方百姓施主,赍粮养口。如今村里的人民都逃散了,
老僧没有去处,只得在此守死。今日幸得天兵到此,万民有福。将军来收此贼,与
民除害,老僧只是不敢多口,恐防贼人知得。今既是天兵处差来的头目,便多口也
不妨。我这里却无路过得关去,直到西山岭边,却有一条小路,可过关上。只怕近
日也被贼人筑断了,过去不得。”

时迁道:“师父,既然有这条小路,通得关上,只不知可到得贼寨里么?”老和尚
道:“这条私路,一径直到得庞万春寨背后,下岭去,便是过关的路了。只恐贼人
已把大石块筑断了,难得过去。”时迁道:“不妨!既有路径,不怕他筑断了,我
自有措置。既然如此,小人回去报知主将,却来酬谢。”老和尚道:“将军见外人
时,休说贫僧多口。”时迁道:“小人是个精细的人,不敢说出老师父来。”
当日辞了老和尚,径回到寨中,参见卢先锋,说知此事。卢俊义听了大喜,便请军
师,计议取关之策。朱武道:“若是有此路径,觑此昱岭关,唾手而得。再差一个
人和时迁同去,干此大事。”时迁道:“军师要干甚大事?”朱武道:“最要紧的
是放火放炮。你等身边,将带火炮、火刀、火石,直要去那寨背后放起号炮火来,
便是你干大事了。”时迁道:“既然只是要放火放炮,别无他事,不须再用别人同
去,只兄弟自往便是。再差一个同去,也跟我做不得飞檐走壁的事,倒误了时候。
假如我去那里行事,你这里如何到得关边?”朱武道:“这却容易,他那贼人的埋
伏,也只好使一遍。我如今不管他埋伏不埋伏,但是于路遇着树木稠密去处,便放
火烧将去,任他埋伏不妨。”时迁道:“军师高见极明。”当下收拾了火刀、火石
并引火煤筒,脊梁上用包袱背着大炮,来辞卢先锋便行。卢俊义叫时迁赍钱二十两,
粮米一石,送与老和尚,就着一个军校挑去。

当日午后,时迁引了这个军校挑米,再寻旧路来到庵里,见了老和尚,说道:“主
将先锋,多多拜复,些小薄礼相送。”便把银两米粮,都与了和尚。老僧收受了,
时迁分付小军自回寨去,却再来告复老和尚:“望烦指引路径,可着行者引小人去。”
那老和尚道:“将军少待,夜深可去,日间恐关上知觉。”当备晚饭待时迁。至夜,
却令行者引路:“送将军到于那边,便教行者即回,休教人知觉。”当时小行者领
着时迁,离了草庵,便望深山径里寻路。穿林透岭,揽葛攀藤,行过数里山径野坡。
月色微明,到一处山岭峻,石壁嵯峨,远远地望见开了个小路口。巅岩上尽把大
石堆迭砌断了,高高筑成墙壁。小行者道:“将军,关已望见,石迭墙壁那边便是。
过得那石壁,亦有大路。”时迁道:“小行者,你自回去,我已知路途了。”小行
者自回,时迁却把飞檐走壁、跳篱骗马的本事出来,这些石壁,拈指爬过去了。望
东去时,只见林木之间,半天价都红满了。却是卢先锋和朱武等拔寨都起,一路上
放火烧着,望关上来。先使三五百军人,于路上打并尸首,沿山巴岭,放火开路,
使其埋伏军兵,无处藏躲。昱岭关上小养由基庞万春闻知宋兵放火烧林开路,庞万
春道:“这是他进兵之法,使吾伏兵不能施展。我等只牢守此关,任汝何能得过?”
望见宋兵渐近关下,带了雷炯、计稷,都来关前守护。

却说时迁一步步摸到关上,爬在一株大树顶头,伏在枝叶稠密处,看那庞万春、雷
炯、计稷,都将弓箭踏弩,伏在关前伺候。看见宋兵时,一派价把火烧将来。中间
林冲、呼延灼立马在关下大骂:“贼将安敢抗拒天兵?”南兵庞万春等却待要放箭
射时,不提防时迁已在关上。那时迁悄悄地溜下树来,转到关后,见两堆柴草,时
迁便摸在里面,取出火刀、火石,发出火种,把火炮搁在柴堆上,先把些硫黄焰硝
去烧那边草堆,又来点着这边柴堆。却才方点着火炮,拿那火种带了,直爬上关屋
脊上去点着。那两边柴草堆里,一齐火起,火炮震天价响。关上众将,不杀自乱,
发起喊来,众军都只顾走,那里有心来迎敌。庞万春和两个副将急来关后救火时,
时迁就在屋脊上又放起火炮来。那火炮震得关屋也动,吓得南兵都弃了刀枪弓箭,
衣袍铠甲,尽望关后奔走。时迁在屋上大叫道:“已有一万宋兵先过关了,汝等急
早投降,免汝一死!”庞万春听了,惊得魂不附体,只管跌脚。雷炯、计稷惊得麻
木了,动弹不得。林冲、呼延灼首先上山,早赶到关顶,众将都要争先,一齐赶过
关去三十余里,追着南兵。孙立生擒得雷炯,魏定国活拿了计稷,单单只走了庞万
春。手下军兵,擒捉了大半。宋兵已到关上,屯驻人马。

卢先锋得了昱岭关,厚赏了时迁,将雷炯、计稷,就关上割腹取心,享祭史进、石
秀等六人,收拾尸骸,葬于关上。其余尸首,尽皆烧化。次日,与同诸将,披挂上
马,一面行文申复张招讨,飞报得了昱岭关,一面引军前进,迤逦追赶过关,直到
歙州城边下寨。

原来歙州守御,乃是皇叔大王方垕,是方腊的亲叔叔,与同两员大将,官封文职,
共守歙州。一个是尚书王寅,一个是侍郎高玉,统领十数员战将,屯军二万之众,
守住歙州城郭。原来王尚书是本州山里石匠出身,惯使一条钢枪,坐下有一骑好马,
名唤转山飞。那匹战马登山渡水,如行平地。那高侍郎也是本州士人,故家子孙,
会使一条鞭枪。因这两个颇通文墨,方腊加封做文职官爵,管领兵权之事。当有小
养由基庞万春败回到歙州,直至行宫,面奏皇叔,告道:“被土居人民,透漏诱引
宋兵私越小路过关。因此众军漫散,难以抵敌。”皇叔方垕听了大怒,喝骂庞万春
道:“这昱岭关是歙州第一处要紧的墙壁,今被宋兵已度关隘,早晚便到歙州,怎
与他迎敌?”王尚书奏道:“主上且息雷霆之怒。自古道:‘胜负兵家之常,非战
之罪。’今殿下权免庞将军本罪,取了军令必胜文状,着他引军,首先出战迎敌,
杀退宋兵。如或不胜,二罪俱并。”方垕然其言,拨与军五千,跟庞万春出城迎敌,
得胜回奏。

且说卢俊义度过昱岭关之后,催兵直赶到歙州城下,当日与诸将上前攻打歙州。城
门开处,庞万春引军出来交战。两军各列成阵势,庞万春出到阵前勒战。宋军队里
欧鹏出马,使根铁枪,便和庞万春交战。两个斗不过五合,庞万春败走,欧鹏要显
头功,纵马赶去。庞万春扭过身躯,背射一箭。欧鹏手段高强,绰箭在手。原来欧
鹏却不提防庞万春能放连珠箭,欧鹏绰了一箭,只顾放心去赶。弓弦响处,庞万春
又射第二只箭来,欧鹏早着,坠下马去。城上王尚书、高侍郎见射中了欧鹏落马,
庞万春得胜,引领城中军马,一发赶杀出来。宋军大败,退回三十里下寨,扎驻军
马安营。整点兵将时,乱军中又折了菜园子张青。孙二娘见丈夫死了,着令手下军
人,寻得尸首烧化,痛哭了一场。卢先锋看了,心中纳闷,思量不是良法,便和朱
武计议道:“今日进兵,又折了二将,似此如之奈何?”朱武道:“输赢胜负,兵
家常事。今日贼兵见我等退回军马,自逞其能,众贼计议,今晚乘势,必来劫寨。
我等可把军马众将,分调开去,四下埋伏。中军缚几只羊在彼,如此如此整顿。叫
呼延灼引一支军在左边埋伏,林冲引一支军在右边埋伏,单廷圭、魏定国引一支军
在背后埋伏。其余偏将,各于四散小路里埋伏。夜间贼兵来时,只看中军火起为号,
四下里各自捉人。”卢先锋都发放已了,各各自去守备。

且说南国王尚书、高侍郎两个颇有些谋略,便与庞万春等商议,上启皇叔方垕道:
“今日宋兵败回,退去三十余里屯驻,营寨空虚,军马必然疲倦,何不乘势去劫寨
栅,必获全胜。”方垕道:“你众官从长计议,可行便行。”高侍郎道:“我便和
庞将军引兵去劫寨,尚书与殿下紧守城池。”当夜二将披挂上马,引领军兵前进,
马摘銮铃,军士衔枚疾走,前到宋军寨栅。看见营门不开,南兵不敢擅进。初时听
得更点分明,向后更鼓便打得乱了。高侍郎勒住马道:“不可进去!”庞万春道:
“相公如何不进兵?”高侍郎答道:“听他营里更点不明,必然有计。”庞万春道:
“相公误矣!今日兵败胆寒,必然困倦。睡里打更,有甚分晓,因此不明。相公何
必见疑,只顾杀去!”高侍郎道:“也见得是。”当下催军劫寨,大刀阔斧,杀将
进去。

二将入得寨门,直到中军,并不见一个军将,却是柳树上缚着数只羊,羊蹄上拴着
鼓槌打鼓,因此更点不明。两将劫着空寨,心中自慌,急叫:“中计!”回身便走,
中军内却早火起,只见山头上炮响,又放起火来,四下里伏兵乱起,齐杀将拢来。
两将冲开寨门奔走,正迎着呼延灼,大喝:“贼将快下马受降,免汝一死!”高侍
郎心慌,只要脱身,无心恋战,被呼延灼赶进去,手起双鞭齐下,脑袋骨打碎了半
个天灵。庞万春死命撞透重围,得脱性命。正走之间,不提防汤隆伏在路边,被他
一钩镰枪拖倒马脚,活捉了解来。众将已都在山路里赶杀南兵,至天明,都赴寨里
来。卢先锋已先到中军坐下,随即下令,点本部将佐时,丁得孙在山路草中,被毒
蛇咬了脚,毒气入腹而死。将庞万春割腹剜心,祭献欧鹏并史进等,把首级解赴张
招讨军前去了。

次日,卢先锋与同诸将再进兵到歙州城下,见城门不关,城上并无旌旗,城楼上亦
无军士。单廷圭、魏定国两个要夺头功,引军便杀入城去。后面中军卢先锋赶到时,
只叫得苦,那二将已到城门里了。原来王尚书见折了劫寨人马,只诈做弃城而走,
城门里却掘下陷坑。二将是一夫之勇,却不提防,首先入来,不想连人和马,都陷
在坑里。那陷坑两边,却埋伏着长枪手、弓箭军士,一齐向前戳杀,两将死于坑中。
可怜圣水并神火,今日呜呼葬土坑。卢先锋又见折了二将,心中忿怒,急令差遣前
部军兵,各人兜土块入城,一面填塞陷坑,一面鏖战厮杀,杀倒南兵人马,俱填于
坑中。当下卢先锋当前,跃马杀入城中,正迎着皇叔方垕,交马只一合,卢俊义却
忿心头之火,展平生之威,只一朴刀,剁方垕于马下。城中军马开城西门,冲突而
走。宋兵众将,各各并力向前,剿捕南兵。

却说王尚书正走之间,撞着李云,截住厮杀。王尚书便挺枪向前,李云却是步斗。
那王尚书枪起马到,早把李云踏倒。石勇见冲翻了李云,便冲突向前,急来救时,
王尚书把条枪神出鬼没,石勇如何抵当得住?王尚书战了数合,得便处把石勇一枪,
结果了性命,当下身死。城里却早赶出孙立、黄信、邹渊、邹润四将,截住王尚书
厮杀。那王寅奋勇力敌四将,并无惧怯。不想又撞出林冲赶到,这个又是个会厮杀
的,那王寅便有三头六臂,也敌不过五将。众人齐上,乱戳杀王寅,可怜南国尚书
将,今日方知志莫伸。当下五将取了首级,飞马献与卢先锋。卢俊义已在歙州城内
行宫歇下,平复了百姓,出榜安民,将军马屯驻在城里。一面差人赍文报捷张招讨,
驰书转达宋先锋,知会进兵。

却说宋江等兵将在睦州屯驻,等候军齐,同攻贼洞。收得卢俊义书,报平复了歙州,
军将已到城中屯驻,专候进兵,同取贼巢。又见折了史进、石秀、陈达、杨春、李
忠、薛永、欧鹏、张青、丁得孙、单廷垕、魏定国、李云、石勇一十三人,许多将
佐,烦恼不已,痛哭哀伤。军师吴用劝道:“生死人皆分定,主将何必自伤玉体?
且请料理国家大事。”宋江道:“虽然如此,不由人不伤感!我想当初石碣天文所
载一百八人,谁知到此,渐渐雕零,损吾手足。”吴用劝了宋江烦恼,然后回书与
卢先锋,交约日期,起兵攻取清溪县。

且不说宋江回书与卢俊义,约日进兵,却说方腊在清溪帮源洞中大内设朝,与文武
百官计议宋江用兵之事。只听见西州败残军马回来,报说歙州已陷,皇叔、尚书、
侍郎俱已阵亡了。今宋兵作两路而来,攻取清溪。方腊见报大惊,当下聚集两班大
臣商议,方腊道:“汝等众卿,各受官爵,同占州郡城池,共享富贵。岂期今被宋
江军马席卷而来,州城俱陷,止有清溪大内。今闻宋兵两路而来,如何迎敌?”当
有左丞相娄敏中出班启奏道:“今次宋兵人马,已近神州,内苑宫廷,亦难保守。
奈缘兵微将寡,陛下若不御驾亲征,诚恐兵将不肯尽心向前。”方腊道:“卿言极
当!”随即传下圣旨:“命三省六部、御史台官、枢密院、都督府护驾,二营金吾、
龙虎,大小官僚,都跟随寡人御驾亲征,决此一战。”娄丞相又奏:“差何将帅,
可做前部先锋?”方腊道:“着殿前金吾上将军、内外诸军都招讨皇侄方杰为正先
锋,马步亲军都太尉、骠骑上将军杜微为副先锋,部领帮源洞大内护驾御林军一万
三千,战将三千余员前进。”原来这方杰是方腊的亲侄儿,是歙州皇叔方垕长孙,
闻知宋兵卢先锋杀了他公公,要来报仇,他愿为前部先锋。这方杰平生习学,惯使
一枝方天画戟,有万夫不当之勇。那杜微原是歙州市中铁匠,会打军器,亦是方腊
心腹之人,会使六口飞刀,只是步斗。方腊另行圣旨一道,差御林护驾都教师贺从
龙,拨与御林军一万,总督兵马,去敌歙州卢俊义军马。

不说方腊分调人马,两处迎敌,先说宋江大队军马起程,水陆并进,离了睦州,望
清溪县而来。水军头领李俊等引领水军船只,撑驾从溪滩里上去。且说吴用与宋江
在马上同行,并马商议道:“此行去取清溪帮源,诚恐贼首方腊知觉逃窜,深山旷
野,难以得获。若要生擒方腊,解赴京师,面见天子,必须里应外合,认得本人,
可以擒获。亦要知方腊去向下落,不致被其走失。”宋江道:“是若如此,须用诈
降,将计就计,方可得里应外合。前者柴进与燕青去做细作,至今不见些消耗,今
次着谁去好?须是会诈投降的。”吴用道:“若论愚意,只除非教水军头领李俊等,
就将船内粮米,去诈献投降,教他那里不疑。方腊那厮是山僻小人,见了许多粮米
船只,如何不收留了。”宋江道:“军师高见极明。”便唤戴宗,随即传令:从水
路直至李俊处,说知如此如此,“教你等众将行计”。李俊等领了计策,戴宗自回
中军。

李俊却叫阮小五、阮小七扮做艄公,童威、童猛扮做随行水手,乘驾六十只粮船,
船上都插着新换的献粮旗号,却从大溪里使将上去。将近清溪县,只见上水头早有
南国战船迎将来,敌军一齐放箭。李俊在船上叫道:“休要放箭,我有话说。俺等
都是投拜的人,特将粮米献纳大国,接济军士,万望收录。”对船上头目,看见李
俊等船上并无军器,因此就不放箭,使人过船来,问了备细,看了船内粮米,便去
报知娄丞相,禀说李俊献粮投降。娄敏中听了,叫唤投拜人上岸来。

李俊登岸,见娄丞相,拜罢,娄敏中问道:“汝是宋江手下甚人?有何职役?今番为
甚来献粮投拜?”李俊答道:“小人姓李名俊,原是浔阳江上好汉。就江州劫法场,
救了宋江性命。他如今受了朝廷招安,得做了先锋,便忘了我等前恩,累次窘辱小
人。现今宋江虽然占得大国州郡,手下弟兄,渐次折得没了。他犹自不知进退,威
逼小人等水军向前。因此受辱不过,特将他粮米船只,径自私来献纳,投拜大国。”
娄丞相见李俊说了这一席话,就便准信,便引李俊来大内朝见方腊,具说献粮投拜
一事。李俊见方腊再拜起居,奏说前事。方腊坦然不疑,且教李俊、阮小五、阮小
七、童威、童猛只在清溪管领水寨守船,“待寡人退了宋江军马还朝之时,别有赏
赐”。李俊拜谢了,出内自去搬运粮米上岸,进仓交收,不在话下。

再说宋江与吴用分调军马,差关胜、花荣、秦明、朱仝四员正将为前队,引军直进
清溪县界,正迎着南国皇侄方杰。两下军兵,各列阵势。南军阵上,方杰横戟出马,
杜微步行在后。那杜微横身挂甲,背藏飞刀五把,手中仗口七星宝剑,跟在后面。
两将出到阵前。宋江阵上秦明,首先出马,手舞狼牙大棍,直取方杰。那方杰年纪
后生,精神一撮,那枝戟使得精熟,和秦明连斗了三十余合,不分胜败。方杰见秦
明手段高强,也放出自己平生学识,不容半点空闲。两个正斗到分际,秦明也把出
本事来,不放方杰些空处。却不提防杜微那厮,在马后见方杰战秦明不下,从马后
闪将出来,掣起飞刀,望秦明脸上早飞将来。秦明急躲飞刀时,却被方杰一方天戟
耸下马去,死于非命。可怜霹雳火,灭地竟无声。方杰一戟戳死了秦明,却不敢追
过对阵,宋兵小将急把挠钩搭得尸首过来。宋军见说折了秦明,尽皆失色。宋江一
面叫备棺椁盛贮,一面再调军将出战。

且说这方杰得胜夸能,却在阵前高叫:“宋兵再有好汉,快出来厮杀!”宋江在中
军听得报来,急出到阵前,看见对阵方杰背后便是方腊御驾,直来到军前摆开。但
见:

金瓜密布,铁斧齐排。方天画戟成行,龙凤绣旗作队。旗旄旌节,一攒攒绿舞
红飞;玉镫雕鞍,一簇簇珠围翠绕。飞龙伞散青云紫雾,飞虎旗盘瑞霭祥烟。左侍
下一代文官,右侍下满排武将。虽是妄称天子位,也须伪列宰臣班。
南国阵中,只见九曲黄罗伞下,玉辔逍遥马上,坐着那个草头王子方腊。怎生打扮,
但见:

头戴一顶冲天转角明金幞头,身穿一领日月云肩九龙绣袍,腰系一条金镶宝嵌
玲珑玉带,足穿一对双金显缝云根朝靴。

那方腊骑着一匹银鬃白马,出到阵前,亲自监战。看见宋江亲在马上,便遣方杰出
战,要拿宋江。这边宋兵等众将亦准备迎敌,要擒方腊。南军方杰正要出阵,只听
得飞马报道:“御林都教师贺从龙总督军马去救歙州,被宋兵卢先锋活捉过阵去了。
军马俱已漫散,宋兵已杀到山后。”方腊听了大惊,急传圣旨,便教收军,且保大
内。当下方杰且委杜微押住阵脚,却待方腊御驾先行,方杰、杜微随后而退。方腊
御驾,回至清溪州界,只听得大内城中,喊起连天,火光遍满,兵马交加,却是李
俊、阮小五、阮小七、童威、童猛在清溪城里放起火来。方腊见了,大驱御林军马
来救城中,入城混战。宋江军马,见南兵退去,随后追杀。赶到清溪,见城中火起,
知有李俊等在彼行事,急令众将招起军马,分头杀将入去。此时卢先锋军马也过山
了,两下接应,却好凑着。四面宋兵,夹攻清溪大内。宋江等诸将,四面八方,杀
将入去,各各自去搜捉南军,打破了清溪城郭。方腊却得方杰引军保驾,防护送投
帮源洞中去了。

宋江等大队军马都入清溪县来。众将杀入方腊宫中,收拾违禁器仗、金银宝物,搜
检内里库藏,就殿上放起火来,把方腊内外宫殿尽皆烧毁,府库钱粮,搜索一空。
宋江会合卢俊义军马,屯驻在清溪县内,聚集众将,都来请功受赏。整点两处将佐
时,长汉郁保四、女将孙二娘,都被杜微飞刀伤死;邹渊、杜迁马军中踏杀;李立、
汤隆、蔡福,各带重伤,医治不痊,身死;阮小五先在清溪县,已被娄丞相杀死。
众将擒捉得南国伪官九十二员请功,赏赐已了,只不见娄丞相、杜微下落。一面且
出榜文,安抚了百姓,把那活捉伪官解赴张招讨军前,斩首示众。后有百姓报说,
娄丞相因杀了阮小五,见大兵打破清溪县,自缢松林而死。杜微那厮躲在他原养的
倡妓王娇娇家,被他社老献将出来。宋江赏了社老,却令人先取了娄丞相首级,叫
蔡庆将杜微剖腹剜心,滴血享祭秦明、阮小五、郁保四、孙二娘,并打清溪亡过众
将。宋江亲自拈香祭赛已了,次日与同卢俊义起军,直抵帮源洞口围住。

且说方腊只得方杰保驾,走到帮源洞口大内,屯驻人马,坚守洞口,不出迎敌。宋
江、卢俊义把军马周回围住了帮源洞,却无计可入。却说方腊在帮源洞,如坐针毡。
两军困住已经数日,方腊正忧闷间,忽见殿下锦衣绣袄一大臣,俯伏在金阶殿下启
奏:“我王,臣虽不才,深蒙主上圣恩宽大,无可补报。凭夙昔所学之兵法,仗平
日所韫之武功,六韬三略曾闻,七纵七擒曾习。愿借主上一枝军马,立退宋兵,中
兴国祚。未知圣意若何?”方腊见了大喜,便传敕令,尽点山洞内府兵马,教此将
引兵出洞去,与宋江相持。未知胜败如何,先见威风出众。不是方腊国中又出这个
人来引兵,有分教:金阶殿下人头滚,玉砌朝门热血喷。直使:扫清巢穴擒方腊,
竖立功勋显宋江。
毕竟方腊国中出来引兵的是甚人,且听下回分解。