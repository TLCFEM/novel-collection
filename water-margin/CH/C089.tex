\chapter{宋公明破阵成功~宿太尉颁恩降诏}

话说当下宋江梦中授得九天玄女之法,不忘一句,便请军师吴用计议定了,申
禀赵枢密。寨中合造雷车二十四部,都用画板铁叶钉成,下装油柴,上安火炮,连
更晓夜,催并完成。商议打阵,会集诸将人马,宋江传令,各各分派:便点按中央
戊己土黄袍军马,战辽国水星阵内,差大将一员双枪将董平,左右撞破皂旗军七门,
差副将七员:朱仝、史进、欧鹏、邓飞、燕顺、马麟、穆春;再点按西方庚辛金白
袍军马,战辽国木星阵内,差大将一员:豹子头林冲;左右撞破青旗军七门,差副
将七员:徐宁、穆弘、黄信、孙立、杨春、陈达、杨林;再点按南方丙丁火红袍军
马,战辽国金星阵内,差大将一员:霹雳火秦明;左右撞破白旗军七门,差副将七
员:刘唐、雷横、单廷圭、魏定国、周通、龚旺、丁得孙;再点按北方壬癸水黑袍
军马,战辽国火星阵内,差大将一员:双鞭呼延灼;左右撞破红旗军七门,差副将
七员:杨志、索超、韩滔、彭玘、孔明、邹渊、邹润;再点按东方甲乙木青袍军马,
战辽国土星主将阵内,差大将一员:大刀关胜;左右撞破中军黄旗主阵人马,差副
将八员:花荣、张清、李应、柴进、宣赞、郝思文、施恩、薛永;再差一枝绣旗花
袍军,打辽国太阳左军阵内,差大将七员:鲁智深、武松、杨雄、石秀、焦挺、汤
隆、蔡福;再差一枝素袍银甲军,打辽国太阴右军阵中,差大将七员:扈三娘、顾
大嫂、孙二娘、王英、孙新、张青、蔡庆;再差打中军一枝悍勇人马,直擒辽主,
差大将六员:卢俊义、燕青、吕方、郭盛、解珍、解宝;再遣护送雷车至中军,大
将五员:李逵、樊瑞、鲍旭、项充、李衮;其余水军头领,并应有人员,尽到阵前
协助破阵。阵前还立五方旗帜八面,分拨人员,仍排九宫八卦阵势。宋江传令已罢,
众将各各遵依。一面造雷车已了,装载法物,推到阵前。正是计就惊天地,谋成
破鬼神。

且说兀颜统军,连日见宋江不出交战,差遣压阵军马,直哨到宋江寨前。宋江连日
制造完备,选定日期,是晚起身,来与辽兵相接。一字儿摆开阵势,前面尽把强弓
硬弩,射住阵脚,只待天色傍晚。黄昏左侧,只见朔风凛凛,彤云密布,罩合天地,
未晚先黑。宋江教众军人等,断芦为笛,衔于口中,唿哨为号。当夜先分出四路兵
去,只留黄袍军摆在阵前。这分出四路军马,赶杀哨路番军,绕阵脚而走,杀投北
去。

初更左侧,宋江军中连珠炮响。呼延灼打开阵门,杀入后军,直取火星。关胜随即
杀入中军,直取土星主将。林冲引军杀入左军阵内,直取木星。秦明领军撞入右军
阵内,直取金星。董平便调军攻打头阵,直取水星。公孙胜在军中仗剑作法,踏罡
步斗,敕起五雷。是夜南风大作,吹得树梢垂地,走石飞沙。一齐点起二十四部雷
车,李逵、樊瑞、鲍旭、项充、李衮,将引五百牌手,悍勇军兵,护送雷车,推入
辽军阵内。一丈青扈三娘引兵便打入辽兵太阴阵中。花和尚鲁智深引兵便打入辽兵
太阳阵中。玉麒麟卢俊义引领一枝军马,随着雷车,直奔中军。你我自去寻队厮杀。
是夜雷车火起,空中霹雳交加,端的是杀得星移斗转,日月无光,鬼哭神号,人兵
撩乱。

且说兀颜统军正在中军遣将,只听得四下里喊声大振,四面厮杀。急上马时,雷车
已到中军,烈焰涨天,炮声震地,关胜一枝军马,早到帐前。兀颜统军急取方天画
戟,与关胜大战。怎禁没羽箭张清,取石子望空中乱打,打的四边牙将,中伤者多
逃命散走。李应、柴进、宣赞、郝思文,纵马横刀,乱杀军将。兀颜统军见身畔没
了羽翼,拨回马望北而走,关胜飞马紧追。正是饶君走上焰摩天,脚下腾云须赶上。
花荣在背后见兀颜统军输了,一骑马也追将来,急拈弓搭箭,望兀颜统军射将去。
那箭正中兀颜统军后心,听的铮地一声,火光迸散,正射在护心镜上。却待再射,
关胜赶上,提起青龙刀,当头便砍。那兀颜统军披着三重铠甲:贴里一层连环镔铁
铠,中间一重海兽皮甲,外面方是锁子黄金甲。关胜那一刀砍过,只透的两层。再
复一刀,兀颜统军就刀影里闪过,勒马挺方天戟来迎。两个又斗了三五合,花荣赶
上,觑兀颜统军面门,又放一箭。兀颜统军急躲,那枝箭带耳根穿住凤翅金冠。兀
颜统军急走,张清飞马赶上,拈起石子,望头脸上便打。石子飞去,打的兀颜统军
扑在马上,拖着画戟而走。关胜赶上,再复一刀。那青龙刀落处,把兀颜统军连腰
截骨带头砍着,攧下马去。花荣抢到,先换了那匹好马。张清赶来,再复一枪。可
怜兀颜统军,一世豪杰,一柄刀,一条枪,结果了性命。有诗为证:
李靖六花人亦识,孔明八卦世应知。
混天只想无人敌,也有神机打破时。

却说鲁智深引着武松等六员头领,众将呐声喊,杀入辽兵太阳阵内。那耶律得重急
待要走,被武松一戒刀,掠断马头,倒撞下马来,揪住头发,一刀取了首级,杀散
太阳阵势。鲁智深道:“俺们再去中军,拿了辽主,便是了事也!”

且说辽兵太阴阵中天寿公主听得四边喊起厮杀,慌忙整顿军器上马,引女兵伺候。
只见一丈青舞起双刀,纵马引着顾大嫂等六员头领,杀入帐来,正与天寿公主交锋。
两个斗无数合,一丈青放开双刀,抢入公主怀内,劈胸揪住。两个在马上扭做一团,
绞做一块。王矮虎赶上,活捉了天寿公主。顾大嫂、孙二娘在阵里杀散女兵。孙新、
张青、蔡庆在外面夹攻。可怜玉叶金枝女,却作归降被缚人。

且说卢俊义引兵杀到中军,解珍、解宝先把帅字旗砍翻,乱杀番兵番将。当有护驾
大臣与众多牙将,紧护辽国郎主銮驾,往北而走。阵内罗、月孛二皇侄,俱被刺
死于马下。计都皇侄,就马上活拿了。紫皇侄,不知去向。大兵重重围住,直杀
到四更方息,杀的辽兵二十余万,七损八伤。

将及天明,诸将都回。宋江鸣金收军下寨,传令教生擒活捉之众,各自献功。一丈
青献太阴星天寿公主;卢俊义献计都星皇侄耶律得华;朱仝献水星曲利出清;欧鹏、
邓飞、马麟献斗木獬萧大观;杨林、陈达献心月狐裴直;单廷圭、魏定国献胃土雉
高彪;韩滔、彭玘献柳土獐雷春、翼火蛇狄圣。诸将献首级,不计其数。宋江将生
擒八将,尽行解赴赵枢密中军收禁。所得马匹,就行俵拨各将骑坐。

且说辽国郎主慌速退入燕京,急传旨意,坚闭四门,紧守城池,不出对敌。宋江知
得辽主退回燕京,便教军马拔寨都起,直追至城下,团团围住。令人请赵枢密直至
后营,监临打城。宋江传令,教就燕京城外,团团竖起云梯炮石,扎下寨栅,准备
打城。

辽国郎主心慌,会集群臣商议,都道:“事在危急,莫若归降大宋,此为上计。”
辽主遂从众议。于是城上早竖起降旗,差人来宋营求告:“年年进牛马,岁岁献珠
珍,再不敢侵犯中国。”宋江引着来人直到后营,拜见赵枢密,通说投降一节。赵
枢密听了道:“此乃国家大事,须用取自上裁,我未敢擅便主张。你辽国有心投降,
可差的当大臣,亲赴东京朝见天子。圣旨准你辽国皈依表文,降诏赦罪,方敢退兵
罢战。”

来人领了这话,便入城回复郎主。当下国主聚集文武百官,商议此事,时有右丞相
太师褚坚出班奏曰:“目今本国兵微将寡,人马皆无,如何迎敌?论臣愚意,微臣
亲往宋先锋寨内,许以厚贿。一面令其住兵停战;一面收拾礼物,径往东京,投买
省院诸官,令其于天子之前,善言启奏,别作宛转。目今中国蔡京、童贯、高俅、
杨戬四个贼臣专权,童子皇帝听他四个主张。可把金帛贿赂,与此四人,买其请和,
必降诏赦,收兵罢战。”郎主准奏。

次日,丞相褚坚出城来,直到宋先锋寨中。宋江接至帐上,便问来意如何。褚坚先
说了国主投降一事,然后许宋先锋金帛玩好之物。宋江听了,说与丞相褚坚道:“俺
连日攻城,不愁打你这个城池不破,一发斩草除根,免了萌芽再发。看见你城上竖
起降旗,以此停兵罢战。两国交锋,自古国家有投降之理,准你投拜纳降,因此按
兵不动,容汝赴朝廷请罪献纳。汝今以贿赂相许,觑宋江为何等之人,再勿复言!”
褚坚惶恐。宋江又道:“容你修表朝京,取自上裁。俺等按兵不动,待汝速去快来,
汝勿迟滞!”

褚坚拜谢了宋先锋,作别出寨,上马回燕京来,奏知国主。众大臣商议已定,次日
辽国君臣,收拾玩好之物,金银宝贝,彩缯珍珠,装载上车,差丞相褚坚并同番官
一十五员,前往京师。鞍马三十余骑,修下请罪表章一道,离了燕京,到了宋江寨
内,参见了宋江。宋江引褚坚来见赵枢密,说知此事:“辽国今差丞相褚坚,亲往
京师朝见,告罪投降。”赵枢密留住褚坚,以礼相待。自来与宋先锋商议,亦动文
书,申达天子。就差柴进、萧让赍奏,就带行军公文,关会省院,一同相伴丞相褚
坚,前往东京。在路不止一日,早到京师,便将十车进奉金宝礼物,车仗人马,于
馆驿内安下。柴进、萧让赍捧行军公文,先去省院下了,禀说道:“即日兵马围困
燕京,旦夕可破。辽国郎主于城上竖起降旗,今遣丞相褚坚前来上表,请罪纳降,
告赦罢兵。未敢自专,来请圣旨。”省院官说道:“你且与他馆驿内权时安歇,待
俺这里从长计议。”

此时蔡京、童贯、高俅、杨戬并省院大小官僚,都是好利之徒。却说辽国丞相褚坚
并众人,先寻门路,见了太师蔡京等四个大臣。次后省院各官处都有贿赂,各各先
以门路馈送礼物诸官已了。次日早朝,百官朝贺拜舞已毕,枢密使童贯出班奏曰:
“有先锋使宋江杀退辽兵,直至燕京,围住城池攻击,旦夕可破。今有辽主早竖降
旗,情愿投降,遣使丞相褚坚,奉表称臣,纳降请罪,告赦讲和,求敕退兵罢战,
情愿年年进奉,不敢有违。伏乞圣鉴。”天子曰:“以此讲和,休兵罢战,汝等众
卿,如何计议?”旁有太师蔡京出班奏曰:“臣等众官,俱各计议:自古及今,四
夷未尝尽灭。臣等愚意,可存辽国,作北方之屏障。年年进纳岁币,于国有益。合
准投降请罪,休兵罢战,诏回军马,以护京师。臣等未敢擅便,乞陛下圣裁。”天
子准奏,传圣旨令辽国来使面君。当有殿头官传令,宣褚坚等一行来使,都到金殿
之下,扬尘拜舞,顿首山呼。侍臣呈上表章,就御案上展开。宣表学士高声读道:
辽国主臣耶律辉顿首顿首,百拜上言:臣生居朔漠,长在番邦,不通圣贤之经,罔
究纲常之礼。诈文伪武,左右多狼心狗行之徒。好赂贪财,前后悉鼠目獐头之辈。
小臣昏昧,屯众猖狂。侵犯疆封,以致天兵讨罪,妄驱士马,动劳王室兴师。量蝼
蚁安足撼泰山,想众水必然归大海。今特遣使臣褚坚冒干天威,纳土请罪。倘蒙圣
上怜悯蕞尔之微生,不废祖宗之遗业,赦其旧过,开以新图,退守戎狄之番邦,永
作天朝之屏障,老老幼幼,真获再生,子子孙孙,久远感戴。进纳岁币,誓不敢违!
臣等不胜战栗屏营之至!谨上表以闻。

宣和四年冬月

日辽国主臣耶律辉

表
徽宗天子御览表文已毕,阶下群臣称贺。天子命取御酒以赐来使,丞相褚坚等便取
金帛岁币进在朝前。天子命宝藏库收讫,仍另纳下每年岁币牛马等物。天子回赐缎
匹表里,光禄寺赐宴。敕令:“丞相褚坚等先回,待寡人差官自来降诏。”褚坚等
谢恩,拜辞出朝,且归馆驿。是日朝散,褚坚又令人再于各官门下,重打关节。蔡
京力许:“令丞相自回,都在我等四人身上。”褚坚谢了太师,自回辽国去了。
却说蔡太师次日引百官入朝,启奏降诏回下辽国。天子准奏,急敕翰林学士草诏一
道,就御前便差太尉宿元景赍擎丹诏,直往辽国开读。另敕赵枢密令宋先锋收兵罢
战,班师回京。将应有被擒之人,释放还国。原夺城池,仍旧给辽管领。府库器具,
交割辽邦归管。天子退朝,百官皆散。次日,省院诸官,都到宿太尉府,约日送行。
再说宿太尉领了诏敕,不敢久停,准备轿马从人,辞了天子,别了省院诸官,就同
柴进、萧让同上辽邦,出京师,望陈桥驿投边塞进发。在路行时,正值严冬之月,
彤云密布,瑞雪平铺,粉塑千林,银装万里。宿太尉一行人马,冒雪风,迤逦前
进。雪霁未消,渐临边塞。柴进、萧让先使哨马报知赵枢密,前去通报宋先锋。宋
江见哨马飞报,便携酒礼,引众出五十里伏道迎接。接着宿太尉,相见已毕,把了
接风酒,各官俱喜。请至寨中,设筵相待,同议朝廷之事。宿太尉言说省院等官,
蔡京、童贯、高俅、杨戬,俱各受了辽国贿赂,于天子前极力保奏此事,准其投降,
休兵罢战,诏回军马,守备京师。宋江听了叹道:“非是宋某怨望朝廷,功勋至此,
又成虚度。”宿太尉道:“先锋休忧!元景回朝,天子前必当重保。”赵枢密又道:
“放着下官为证,怎肯教虚费了将军大功!”宋江禀道:“某等一百八人,竭力报
国,并无异心,亦无希恩望赐之念。只得众弟兄同守劳苦,实为幸甚。若得枢相肯
做主张,深感厚德。”当日饮宴,众皆欢喜,至晚方散。随即差人一面报知辽国,
准备接诏。

次日,宋江拨十员大将护送宿太尉进辽国颁诏,都是锦袍金甲,戎装革带。那十员
上将:关胜、林冲、秦明、呼延灼、花荣、董平、李应、柴进、吕方、郭盛,引领
马步军三千,护持太尉,前遮后拥,摆布入城。燕京百姓,有数百年不见中国军容,
闻知太尉到来,尽皆欢喜,排门香花灯烛。辽主亲引百官文武,具服乘马,出南门
迎接诏旨,直至金銮殿上。十员大将,立于左右。宿太尉立于龙亭之左,国主同百
官跪于殿前。殿头官喝拜,国主同文武拜罢。辽国侍郎承恩请诏,就殿上开读。诏
曰:
大宋皇帝制曰:三皇立位,五帝禅宗,虽中华而有主,岂夷狄之无君?兹尔辽国,
不遵天命,数犯疆封,理合一鼓而灭。朕今览其情词,怜其哀切,悯汝孤,不忍
加诛,仍存其国。诏书至日,即将军前所擒之将,尽数释放还国。原夺一应城池,
仍旧给还本国管领。所供岁币,慎勿怠忽。於戏!敬事大国,祗畏天地,此藩翰之
职也。尔其钦哉!

宣和四年冬月
日

当时辽国侍郎开读诏旨已罢,郎主与百官再拜谢恩。行君臣礼毕,抬过诏书龙案,
郎主便与宿太尉相见。叙礼已毕,请入后殿,大设华筵,水陆俱备。番官进酒,戎
将传杯;歌舞满筵,胡笳聒耳;燕姬美女,各奏戎乐;羯鼓埙,胡旋慢舞。筵宴
已终,送宿太尉并众将于馆驿内安歇。是日跟去人员,都有赏劳。

次日,国主命丞相褚坚出城至寨,邀请赵枢密、宋先锋同入燕京赴宴。宋江便与军
师吴用计议不行,只请的赵枢密入城,相陪宿太尉饮宴。是日辽国郎主大张筵席,
管待朝使。葡萄酒熟倾银瓮,黄羊肉美满金盘。异果堆筵,奇花散彩。筵席将终,
只见国主金盘捧出玩好之物,上献宿太尉、赵枢密,直饮至更深方散。第三日,辽
主会集文武群臣,番戎鼓乐,送太尉、枢密出城还寨。再命丞相褚坚将牛羊马匹、
金银彩缎等项礼物,直至宋先锋军前寨内,大设广会,犒劳三军,重赏众将。
宋江传令,叫取天寿公主一干人口,放回本国。仍将夺过檀州、蓟州、霸州、幽州,
依旧给还辽国管领。一面先送宿太尉还京,次后收拾诸将军兵车仗人马,分拨人员,
先发中军军马,护送赵枢密起行。宋先锋寨内,自己设宴,一面赏劳水军头目已了,
着令乘驾船只从水路先回东京驻扎听调。

宋江再使人入城中,请出左右二丞相前赴军中说话。当下辽国郎主教左丞相幽西孛
瑾、右丞相太师褚坚,来至宋先锋行营,至于中军相见。宋江邀请上帐,分宾而坐。
宋江开话道:“俺武将兵临城下,将至壕边,奇功在迩,本不容汝投降。打破城池,
尽皆剿灭,正当其理。主帅听从,容汝申达朝廷。皇上怜悯,存恻隐之心,不肯尽
情追杀,准汝投降,纳表请罪。今王事已毕,吾待朝京。汝等勿以宋江等辈不能胜
尔,再生反复。年年进贡,不可有缺。吾今班师还国,汝宜谨慎自守,休得故犯!
天兵再至,决无轻恕!”二丞相叩首伏罪拜谢。宋江再用好言戒谕,二丞相恳谢而
去。

宋江却拨一队军兵,与女将一丈青等先行。随即唤令随军石匠,采石为碑,令萧让
作文,以记其事。金大坚镌石已毕,竖立在永清县东一十五里茅山之下,至今古迹
尚存。有诗为证:
每闻胡马度阴山,恨杀澶渊纵虏还。
谁造茅山功迹记,寇公泉下亦开颜。

宋江却将军马分作五起进发,克日起行。只见鲁智深忽到帐前,合掌作礼,对宋江
道:“小弟自从打死了镇关西,逃走到代州雁门县,赵员外送洒家上五台山,投礼
智真长老,落发为僧。不想醉后两番闹了禅门,师父送俺来东京大相国寺,投托智
清禅师,讨个执事僧做,相国寺里着洒家看守菜园。为救林冲,被高太尉要害,因
此落草。得遇哥哥,随从多时,已经数载,思念本师,一向不曾参礼。洒家常想师
父说,俺虽是杀人放火的性,久后却得正果真身。今日太平无事,兄弟权时告假数
日,欲往五台山参礼本师。就将平昔所得金帛之资,都做布施,再求问师父前程如
何。哥哥军马只顾前行,小弟随后便赶来也!”

宋江听罢愕然,默上心来,便道:“你既有这个活佛罗汉在彼,何不早说,与俺等
同去参礼,求问前程。”当时与众人商议,尽皆要去,惟有公孙胜道教不行。宋江
再与军师计议:“留下金大坚、皇甫端、萧让、乐和四个,委同副先锋卢俊义掌管
军马,陆续先行。俺们只带一千来人,随从众弟兄,跟着鲁智深同去参礼智真长老。”
宋江等众当时离了军前,收拾名香、彩帛、表里、金银,上五台山来。正是:暂弃
金戈甲马,来游方外丛林。雨花台畔,来访道德高僧;善法堂前,要见燃灯古佛。
直教:一语打开名利路,片言踢透死生关。
毕竟宋江与鲁智深怎地参禅,且听下回分解。