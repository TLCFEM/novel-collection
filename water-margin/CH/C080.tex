\chapter{张顺凿漏海鳅船~宋江三败高太尉}

话说高太尉在济州城中帅府坐地,唤过王焕等众节度商议:传令将各路军马,
拔寨收入城中;教现在节度使俱各全副披挂,伏于城内;各寨军士,尽数准备,摆
列于城中;城上俱各不竖旌旗,只于北门上立黄旗一面,上书“天诏”二字。高俅
与天使众官都在城上,只等宋江到来。

当日梁山泊中,先差没羽箭张清将带五百哨马,到济州城边周回转了一遭,望北去
了。须臾,神行太保戴宗步行来探了一遭。人报与高太尉,亲自临月城上,女墙边,
左右从者百余人,大张麾盖,前设香案。遥望北边宋江军马到来,前面金鼓,五方
旌旗,众头领簸箕掌,栲栳圈,雁翅一般,摆列将来。当先为首,宋江、卢俊义、
吴用、公孙胜,在马上欠身,与高太尉声喏。高太尉见了,使人在城上叫道:“如
今朝廷赦你们罪犯,特来招安,如何披甲前来?”宋江使戴宗至城下回复道:“我
等大小人员,未蒙恩泽,不知诏意如何,未敢去其介胄。望太尉周全。可尽唤在城
百姓耆老,一同听诏,那时承恩卸甲。”高太尉出令,教唤在城耆老百姓,尽都上
城听诏。无移时,纷纷滚滚,尽皆到了。宋江等在城下,看见城上百姓老幼摆满,
方才勒马向前。鸣鼓一通,众将下马。鸣鼓二通,众将步行到城边,背后小校,牵
着战马,离城一箭之地,齐齐地伺候着。鸣鼓三通,众将在城下拱手,听城上开读
诏书。那天使读道:
制曰:人之本心,本无二端;国之恒道,俱是一理。作善则
为良民,造恶则为逆党。朕闻梁山泊聚众已久,不蒙善化,未复良心。今差天使颁
降诏书,除宋江,卢俊义等大小人众所犯过恶,并与赦免。其为首者,诣京谢恩;
协随助者,各归乡闾。呜呼,速雨露,以就去邪归正之心;毋犯雷霆,当效革故
鼎新之意。故兹诏示,想宜悉知。

宣和

年

月

日

当时军师吴用正听读到“除宋江”三字,便目视花荣道:“将军听得么?”却才读
罢诏书,花荣大叫:“既不赦我哥哥,我等投降则甚?”搭上箭,拽满弓,望着那
个开诏使臣道:“看花荣神箭!”一箭射中面门,众人急救。城下众好汉一齐叫声:
“反!”乱箭望城上射来,高太尉回避不迭。四门突出军马来,宋江军中,一声鼓
响,一齐上马便走。城中官军追赶,约有五六里回来。只听得后军炮响,东有李逵,
引步军杀来;西有扈三娘,引马军杀来。两路军兵,一齐合到。官军只怕有埋伏,
急退时,宋江全伙却回身卷杀将来。三面夹攻,城中军马大乱,急急奔回,杀死者
多。宋江收军,不教追赶,自回梁山泊去了。

却说高太尉在济州写表,申奏朝廷说:“宋江贼寇,射死天使,不伏招安。”外写
密书,送与蔡太师、童枢密、杨太尉,烦为商议,教太师奏过天子,沿途接应粮草,
星夜发兵前来,并力剿捕群贼。

却说蔡太师收得高太尉密书,径自入朝,奏知天子。天子闻奏,龙颜不悦云:“此
寇数辱朝廷,累犯大逆。”随即降敕,教诸路各助军马,并听高太尉调遣。杨太尉
已知节次失利,再于御营司选拨二将,就于龙猛、虎翼、捧日、忠义四营内,各选
精兵五百,共计二千,跟随两个上将,去助高太尉杀贼。

这两员将军是谁?一个是八十万禁军都教头,官带左义卫亲军指挥使,护驾将军丘
岳;一个是八十万禁军副教头,官带右义卫亲军指挥使,车骑将军周昂。这两个将
军,累建奇功,名闻海外,深通武艺,威镇京师,又是高太尉心腹之人。当时杨太
尉点定二将,限目下起身,来辞蔡太师。蔡京分付道:“小心在意,早建大功,必
当重用!”二将辞谢了,去四营内,一个个选拣身长体健,腰细膀阔,山东、河北
能登山、惯赴水,那一等精锐军汉,拨与二将。这丘岳、周昂辞了众省院官,去辞
杨太尉,禀说:“明日出城。”杨太尉各赐与二将五匹好马,以为战阵之用。二将
谢了太尉,各自回营,收拾起身。次日,军兵拴束了行程,都在御营司前伺候。丘
岳、周昂二将分做四队:龙猛、虎翼二营一千军,有二千余骑军马,丘岳总领;捧
日、忠义二营一千军,也有二千余骑军马,周昂总领。又有一千步军,分与二将随
从。丘岳、周昂到辰牌时分,摆列出城。杨太尉亲自在城门上看军。且休说小校威
雄,亲随勇猛。去那两面绣旗下,一丛战马之中,簇拥着护驾将军丘岳。怎生打扮,
但见:
戴一顶缨撒火、锦兜鍪、双凤翅照天盔。披一副绿绒穿、红绵套、嵌连环锁子甲。
穿一领翠沿边、珠络缝、荔枝红、圈金绣戏狮袍。系一条衬金叶、玉玲珑、双獭尾、
红鞓钉盘螭带。着一双簇金线、海驴皮、胡桃纹、抹绿色云根靴。弯一张紫檀靶、
泥金梢、龙角面、虎筋弦宝雕弓。悬一壶紫竹杆、朱红扣、凤尾翎、狼牙金点钢箭。
挂一口七星装、沙鱼鞘、赛龙泉、欺巨阙霜锋剑。横一把撒朱缨、水磨杆、龙吞头、
偃月样三停刀。骑一匹快登山、能跳涧、背金鞍、摇玉勒胭脂马。

那丘岳坐在马上,昂昂奇伟,领着左队人马,东京百姓看了,无不喝采。随后便是
右队,捧日、忠义两营军马,端的整齐。去那两面绣旗下,一丛战马之中,簇拥着
车骑将军周昂。怎生打扮,但见:
戴一顶吞龙头、撒青缨、珠闪烁烂银盔。披一副损枪尖、坏箭头、衬香绵熟钢甲。
穿一领绣牡丹、飞双凤、圈金线绛红袍。系一条称狼腰、宜虎体、嵌七宝麒麟带。
着一双起三尖、
海兽皮、倒云根虎尾靴。弯一张雀画面、龙角靶、紫综绣六钧弓。攒一壶皂雕翎、
铁木杆、透唐猊凿子箭。使一柄欺袁达、赛石丙、劈开山金蘸斧。驶一匹负千斤、
高八尺、能冲阵火龙驹。悬一条简银杆、四方棱、赛金光劈楞简。
这周昂坐在马上,停停威猛。领着右队人马,来到城边,与丘岳下马,来拜辞杨太
尉,作别众官,离了东京,取路望济州进发。

且说高太尉在济州,和闻参谋商议,比及添拨得军马到来,先使人去近处山林,砍
伐木植大树;附近州县,拘刷造船匠人,就济州城外,搭起船场,打造战船;一面
出榜,招募敢勇水手军士。

济州城中客店内,歇着一个客人,姓叶名春,原是泗州人氏,善会造船。因来山东,
路经梁山泊过,被他那里小伙头目劫了本钱,流落在济州,不能够回乡。听得高太
尉要伐木造船,征进梁山泊,以图取胜,将纸画成船样,来见高太尉。拜罢,禀道:
“前者恩相以船征进,为何不能取胜?盖因船只皆是各处拘刷将来的,使风摇橹,
俱不得法;更兼船小底尖,难以用武。叶春今献一计,若要收伏此寇,必须先造大
船数百只。最大者名为大海鳅船。两边置二十四部水车,船中可容数百人,每车用
十二个人踏动。外用竹笆遮护,可避箭矢。船面上竖立弩楼,另造车摆布放于上。
如要进发,垛楼上一声梆子响,二十四部水车,一齐用力踏动,其船如飞,他将何
等船只可以拦当!若是遇着敌军,船面上伏弩齐发,他将何物可以遮护!其第二等船,
名为小海鳅船。两边只用十二部水车,船中可容百十人,前面后尾,都钉长钉,两
边亦立弩楼,仍设遮洋笆片。这船却行梁山泊小港,当住这厮私路伏兵。若依此计,
梁山之寇,指日唾手可平。”高太尉听说,看了图样,心中大喜。便叫取酒食衣服
赏了叶春,就着做监造战船都作头。连日晓夜催并,砍伐木植,限日定时,要到济
州交纳。各路府州县,均派合用造船物料。如若违限二日,笞四十,每一日加一等;
若违限五日外者,定依军令处斩。各处逼迫守令催督,百姓亡者数多,众民嗟怨。
有诗为证:
井蛙小见岂知天,可慨高俅听谲言。
毕竟鳅船难取胜,伤财劳众枉徒然。

且不说叶春监造海鳅等船,却说各处添拨水军人等,陆续都到济州。高太尉分拨各
寨节度使下听调,不在话下。只见门吏报道:“朝廷差遣丘岳、周昂二将到来。”
高太尉令众节度使出城迎接。二将到帅府,参见了,太尉亲赐酒食,抚慰已毕,一
面差人赏军,一面管待二将。二将便请太尉将令,引军出城搦战。高太尉道:“二
公且消停数日,待海鳅船完备,那时水陆并进,船骑双行,一鼓可平贼寇。”丘岳、
周昂禀道:“某等觑梁山泊草寇,如同儿戏,太尉放心,必然奏凯还京。”高俅道:
“二将若果应口,吾当奏知天子前,必当重用。”是日宴散,就帅府前上马,回归
本寨,且把军马屯驻听调。

不说高太尉催促造船征进,却说宋江与众头领自从济州城下叫反杀人,奔上梁山泊
来,却与吴用等商议道:“两次招安,都伤犯了天使,越增的罪恶重了,朝廷必然
又差军马来。”便差小喽罗下山去探事情如何,火急回报。不数日,只见小喽罗探
知备细,报上山来:“高俅近日招募一水军,叫叶春为作头,打造大小海鳅船数百
只,东京又新遣差两个御前指挥,俱到来助战。一个姓丘名岳,一个姓周名昂,二
将英勇;各路又添拨到许多人马,前来助战。”宋江便与吴用计议道:“似此大船,
飞游水面,如何破得?”吴用笑道:“有何惧哉!只消得几个水军头领便了。旱路
上交锋,自有猛将应敌。然虽如此,料这等大船,要造必在数旬间,方得成就。目
今尚有四五十日光景,先教一两个弟兄去那造船厂里,先薅恼他一遭,后却和他慢
慢地放对。”宋江道:“此言最好!可教鼓上蚤时迁、金毛犬段景住,这两个走一
遭。”吴用道:“再叫张青、孙新扮作拽树民夫,杂在人丛里入船厂去。叫顾大嫂、
孙二娘扮做送饭妇人,和一般的妇人杂将入去,却叫时迁、段景住相帮。再用张清
引军接应,方保万全。”前后唤到堂上,各各听令已了。众人欢喜无限,分投下山,
自去行事。

却说高太尉晓夜催促督造船只,朝暮捉拿民夫供役。那济州东路上一带都是船厂,
趱造大海鳅船百只,何止匠人数千,纷纷攘攘。那等蛮军,都拔出刀来,吓民夫,
无分星夜,要趱完备。是日,时迁、段景住先到了厂内,两个商量道:“眼见的孙、
张二夫妻,只是去船厂里放火,我和你也去那里,不显我和你高强。我们只伏在这
里左右,等他船厂里火发,我便却去城门边伺候,必然有救军出来,乘势闪将入去,
就城楼上放起火来,你便却去城西草料场里,也放起把火来,教他两下里救应不迭。
这场惊吓不小。”两个自暗暗地相约了,身边都藏了引火的药头,各自去寻个安身
之处。

却说张青、孙新两个来到济州城下,看见三五百人,拽木头入船厂里去。张、孙二
人杂在人丛里,也去拽木头投厂里去。厂门口约有二百来军汉,各带腰刀,手拿棍
棒,打着民夫,尽力拖拽入厂里面交纳。团团一遭,都是排栅;前后搭盖茅草厂屋,
有二三百间。张青、孙新入到里面看时,匠人数千,解板的在一处,钉船的在一处,
粘船的在一处。匠人民夫,乱滚滚往来,不记其数。这两个径投做饭的笆棚下去躲
避。孙二娘、顾大嫂两个穿了些腌腌衣服,各提着个饭罐,随着一般送饭的妇
人,打哄入去。看看天色渐晚,月色光明,众匠人大半尚兀自在那里挣趱未办的工
程。当时近有二更时分,孙新、张青在左边船厂里放火,孙二娘、顾大嫂在右边船
厂里放火。两下火起,草屋焰腾腾地价烧起来。船厂内民夫工匠,一齐发喊,拔翻
众栅,各自逃生。

高太尉正睡间,忽听得人报道:“船场里火起!”急忙起来,差拨官军出城救应。
丘岳、周昂二将各引本部军兵,出城救火。去不多时,城楼上一把火起。高太尉听
了,亲自上马,引军上城救火时,又见报道:“西草场内又一把火起,照耀浑如白
日。”丘、周二将引军去西草场中救护时,只听得鼓声振地,喊杀连天,原来没羽
箭张清引着五百骠骑马军,在那里埋伏,看见丘岳、周昂引军来救应,张清便直杀
将来,正迎着丘岳、周昂军马。张清大喝道:“梁山泊好汉全伙在此!”丘岳大怒,
拍马舞刀,直取张清。张清手掿长枪来迎,不过三合,拍马便走。丘岳要逞功劳,
随后赶来,大喝:“反贼休走!”张清按住长枪,轻轻去锦袋内偷取个石子在手,
扭回身躯,看丘岳来得较近,手起喝声道:“着!”一石子正中丘岳面门,翻身落
马。周昂见了,便和数个牙将,死命来救丘岳。周昂战住张清,众将救得丘岳上马
去了。张清与周昂战不到数合,回马便走。周昂不赶。张清又回来,却见王焕、徐
京、杨温、李从吉四路军到。张清手招引了五百骠骑军,竟回旧路去了。这里官军
恐有伏兵,不敢去赶,自收军兵回来,且只顾救火。三处火灭,天色已晓。
高太尉教看丘岳中伤如何。原来那一石子,正打着面门,唇口里打落了四个牙齿;
鼻子嘴唇,都打破了。高太尉着令医人治疗,见丘岳重伤,恨梁山泊深入骨髓。一
面使人唤叶春,分付教在意造船征进。船厂四围,都教节度使下了寨栅,早晚提备,
不在话下。

却说张青、孙新夫妻四人,俱各欢喜;时迁、段景住两个,都回旧路。六人已都有
部从人马,迎接回梁山泊去了。都到忠义堂,去说放火一事。宋江大喜,设宴特赏
六人。自此之后,不时间使人探视。

造船将完,看看冬到。其年天气甚暖,高太尉心中暗喜,以为天助。叶春造船,也
都完办,高太尉催趱水军,都要上船演习本事。大小海鳅等船陆续下水。城中帅府
招募到四山五岳水手人等,约有一万余人。先教一半去各船上学踏车,着一半学放
弩箭。不过二十余日,战船演习已都完足了。叶春请太尉看船,有诗为证:
自古兵机在速攻,锋摧师老岂成功。
高俅卤莽无通变,经岁劳民造战艟。

是日,高俅引领众多节度使、军官头目,都来看船。把海鳅船三百余只,分布水面。
选十数只船,遍插旌旗,筛锣击鼓,梆子响处,两边水车,一齐踏动,端的是风飞
电走。高太尉看了,心中大喜:似此如飞船只,此寇将何拦截,此战必胜。随取金
银缎匹,赏赐叶春;其余人匠,各给盘缠,疏放归家。次日,高俅令有司宰乌牛、
白马、猪、羊,果品,摆列金银钱纸,致祭水神。排列已了,众将请太尉行香。丘
岳疮口已完,恨入心髓,只要活捉张清报仇。当同周昂与众节度使,一齐都上马,
跟随高太尉到船边下马,随侍高俅致祭水神。焚香赞礼已毕,烧化楮帛,众将称贺
已了,高俅叫取京师原带来的歌儿舞女,都令上船作乐侍宴。一面教军健车船演习,
飞走水面,船上笙箫谩品,歌舞悠扬,游玩终夕不散。当夜就船中宿歇。次日,又
设席面饮酌,一连三日筵宴,不肯开船。忽有人报道:“梁山泊贼人写一首诗,贴
在济州城里土地庙前,有人揭得在此。”其诗写道:
帮闲得志一高俅,漫领三军水上游。
便有海鳅船万只,俱来泊内一齐休。

高太尉看了诗大怒,便要起军征剿:“若不杀尽贼寇,誓不回军!”闻参谋谏道:
“太尉暂息雷霆之怒。想此狂寇惧怕,特写恶言吓,不为大事。消停数日之间,
拨定了水陆军马,那时征进未迟。目今深冬,天气和暖,此天子洪福,元帅虎威也。”
高俅听罢甚喜,遂入城中,商议拨军遣将。旱路上便调周昂、王焕同领大军,随行
策应。却调项元镇、张开总领军马一万,直至梁山泊山前那条大路上守住厮杀。原
来梁山泊自古四面八方,茫茫荡荡,都是芦苇烟水。近来只有山前这条大路,却是
宋公明方才新筑的,旧不曾有。高太尉教调马军先进,截住这条路口。其余闻参谋、
丘岳、徐京、梅展、王文德、杨温、李从吉、长史王瑾、造船人叶春、随行牙将、
大小军校随从人等,都跟高太尉上船征进。闻参谋谏道:“主帅只可监督马军,陆
路进发,不可自登水路,亲临险地。”高太尉道:“无伤。前番二次,皆不得其人,
以致失陷了人马,折了许多船只。今番造得若干好船,我若不亲临监督,如何擒捉
此寇?今次正要与贼人决一死战,汝不必多言!”闻参谋再不敢开口,只得跟随高
太尉上船。高俅拨三十只大海鳅船,与先锋丘岳、徐京、梅展管领,拨五十只小海
鳅船开路,令杨温同长史王瑾、船匠叶春管领。头船上立两面大红绣旗,上书十四
个金字道:“搅海翻江冲巨浪,安邦定国灭洪妖。”中军船上,却是高太尉、闻参
谋,引着歌儿舞女,自守中军队伍。向那三五十只大海鳅船上,摆开碧油幢、帅字
旗、黄钺白旄、朱皂盖、中军器械。后面船上,便令王文德、李从吉压阵。此是
十一月中时。马军得令先行。水军先锋丘岳、徐京、梅展三个,在头船上,首先进
发,飞云卷雾,望梁山泊来。但见海鳅船:
前排箭洞,上列弩楼。冲波如蛟蜃之形,走水似鲲鲸之势。龙鳞密布,左右排二十
四部绞车;雁翅齐分,前后列一十八般军器。青布织成皂盖,紫竹制作遮洋。往来
冲击似飞梭,展转交锋欺快马。

宋江、吴用已知备细,预先布置已定,单等官军船只到来。当下三个先锋,催动船
只,把小海鳅分在两边,当住小港;大海鳅船望中进发。众军诸将,正如蟹眼鹤顶,
只望前面奔窜,迤逦来到梁山泊深处。只见远远地早有一簇船来,每只船上,只有
十四五人,身上都有衣甲,当中坐着一个头领。前面三只船上,插着三把白旗,旗
上写道:“梁山泊阮氏三雄。”中间阮小二,左边阮小五,右边阮小七。远远地望
见明晃晃都是戎装衣甲,却原来尽把金银箔纸糊成的。

三个先锋见了,便叫前船上将火炮、火枪、火箭,一齐打放。那三阮全然不惧,料
着船近,枪箭射得着时,发声喊,齐跳下水里去了。丘岳等夺得三只空船,又行不
过三里来水面,见三只快船抢风摇来。头只船上,只见十数个人,都把青黛黄丹土
朱泥粉,抹在身上,头上披着发,口中打着胡哨,飞也似来。两边两只船上,都只
五七个人,搽红画绿不等。中央是玉竿孟康,左边是出洞蛟童威,右边是翻江蜃
童猛。这里先锋丘岳,又叫打放火器,只见对面发声喊,都弃了船,一齐跳下水里
去了。又捉得三只空船。再行不得三里多路,又见水面上三只中等船来。每船上四
把橹,八个人摇动,十余个小喽罗,打着一面红旗,簇拥着一个头领坐在船头上,
旗上写:“水军头领混江龙李俊。”左边这只船上,坐着这个头领,手掿铁枪,打
着一面绿旗,上写道:“水军头领船火儿张横。”右边那只船上,立着那个好汉,
上面不穿衣服,下腿赤着双脚,腰间插着几个铁凿,手中挽个铜锤,打着一面皂旗,
银字上书:“头领浪里白跳张顺。”乘着船,高声说道:“承谢送船到泊。”三个
先锋听了,喝教:“放箭!”弓弩响时,对面三只船上众好汉,都翻筋斗跳下水里
去了。此是暮冬天气,官军船上招来的水手军士,那里敢下水去。

正犹豫间,只听得梁山泊顶上,号炮连珠价响,只见四分五落,芦苇丛中,钻出千
百只小船来,水面如飞蝗一般。每只船上,只三五个人,船舱中竟不知有何物。大
海鳅船要撞时,又撞不得。水车正要踏动时,前面水底下都填塞定了,车辐板竟踏
不动。弩楼上放箭时,小船上人,一个个自顶片板遮护。看看逼将拢来,一个把挠
钩搭住了舵,一个把板刀便砍那踏车的军士。早有五六十个爬上先锋船来。官军急
要退时,后面又塞定了,急切退不得。前船正混战间,后船又大叫起来。高太尉和
闻参谋在中军船上,听得大乱,急要上岸,只听得芦苇中金鼓大振,舱内军士一齐
喊道:“船底漏了。”滚滚走入水来。前船后船,尽皆都漏,看看沉下去。四下小
船,如蚂蚁相似,望大船边来。高太尉新船,缘何得漏?却原来是张顺引领一班儿
高手水军,都把锤凿在船底下凿透船底,四下里滚入水来。

高太尉爬去舵楼上,叫后船救应,只见一个人从水底下钻将起来,便跳上舵楼来,
口里说道:“太尉,我救你性命。”高俅看时,却不认得。那人近前,便一手揪往
高太尉巾帻,一手提住腰间束带,喝一声:“下去!”把高太尉扑通地丢下水里去。
堪嗟赫赫中军将,翻作淹淹水底人!只见旁边两只小船,飞来救应,拖起太尉上船
去。那个人便是浪里白跳张顺,水里拿人,浑如瓮中捉鳖,手到拈来。

前船丘岳见阵势大乱,急寻脱身之计,只见旁边水手丛中,走出一个水军来。丘岳
不曾提防,被他赶上,只一刀,把丘岳砍下船去。那个便是梁山泊锦豹子杨林。徐
京、梅展见杀了先锋丘岳,两节度奔来杀杨林。水军丛中,连抢出四个小头领来:
一个是白面郎君郑天寿,一个是病大虫薛永,一个是打虎将李忠,一个是操刀鬼曹
正,一发从后面杀来。徐京见不是头,便跳下水去逃命,不想水底下已有人在彼,
又吃拿了。薛永将梅展一枪,搠着腿股,跌下舱里去。原来八个头领,来投充水军,
尚兀自有三个在前船上:一个是青眼虎李云,一个是金钱豹子汤隆,一个是鬼脸儿
杜兴。众节度使便有三头六臂,到此也施展不得。

梁山泊宋江、卢俊义,已自各分水陆进攻。宋江掌水路,卢俊义掌旱路。休说水路
全胜,且说卢俊义引领诸将军马,从山前大路杀将出来,正与先锋周昂、王焕马头
相迎。周昂见了,当先出马,高声大骂:“反贼,认得俺么?”卢俊义大喝:“无
名小将,死在目前,尚且不知!”便挺枪跃马,直奔周昂,周昂也抡动大斧,纵马
来敌。两将就山前大路上交锋,斗不到二十余合,未见胜败。只听得后队马军,发
起喊来。原来梁山泊大队军马,都埋伏在山前两下大林丛中,一声喊起,四面杀将
出来。东南关胜、秦明,西北林冲、呼延灼,众多英雄,四路齐到。项元镇、张开
那里拦当得住,杀开条路,先逃性命走了。周昂、王焕不敢恋战,拖了枪斧,夺路
而走,逃入济州城中;扎住军马,打听消息。

再说宋江掌水路,捉了高太尉,急教戴宗传令,不可杀害军士。中军大海鳅船上闻
参谋等并歌儿舞女,一应部从,尽掳过船。鸣金收军,解投大寨。宋江、吴用、公
孙胜等,都在忠义堂上,见张顺水渌渌地解到高俅。宋江见了,慌忙下堂扶住,便
取过罗缎新鲜衣服,与高太尉从新换了,扶上堂来,请在正面而坐。宋江纳头便拜,
口称:“死罪!”高俅慌忙答礼。宋江叫吴用、公孙胜扶住,拜罢,就请上坐。再
叫燕青传令下去:“如若今后杀人者,定依军令,处以重刑!”号令下去,不多时,
只见纷纷解上人来:童威、童猛解上徐京;李俊、张横解上王文德;杨雄、石秀解
上杨温;三阮解上李从吉;郑天寿、薛永、李忠、曹正解上梅展;杨林解献丘岳首
级;李云、汤隆、杜兴,解献叶春、王瑾首级;解珍、解宝掳捉闻参谋并歌儿舞女
一应部从,解将到来。单单只走了四人:周昂、王焕、项元镇、张开。宋江都教换
了衣服,从新整顿,尽皆请到忠义堂上,列坐相待。但是活捉军士,尽数放回济州。
另教安排一只好船,安顿歌儿舞女一应部从,令他自行看守。有诗为证:
奉命高俅欠取裁,被人活捉上山来。
不知忠义为何物,翻宴梁山啸聚台。

当时宋江便教杀牛宰马,大设筵宴,一面分投赏军,一面大吹大擂,会集大小头领,
都来与高太尉相见。各施礼毕,宋江持盏擎杯,吴用、公孙胜执瓶捧案,卢俊义等
侍立相待。宋江开口道:“文面小吏,安敢叛逆圣朝,奈缘积累罪尤,逼得如此。
二次虽奉天恩,中间委曲奸弊,难以缕陈。万望太尉慈悯,救拔深陷之人,得瞻天
日,刻骨铭心,誓图死保。”

高俅见了众多好汉,一个个英雄猛烈,林冲、杨志怒目而视,有欲要发作之色,先
有了十分惧怯,便道:“宋公明,你等放心!高某回朝,必当重奏,请降宽恩大赦,
前来招安,重赏加官,大小义士,尽食天禄,以为良臣。”宋江听了大喜,拜谢太
尉。当日筵会,甚是整齐,大小头领,轮番把盏,殷勤相劝。高太尉大醉,酒后不
觉放荡,便道:“我自小学得一身相扑,天下无对。”卢俊义却也醉了,怪高太尉
自夸天下无对,便指着燕青道:“我这个小兄弟,也会相扑,三番上岱岳争交,天
下无对。”高俅便起身来,脱了衣裳,要与燕青厮扑。

众头领见宋江敬他是个天朝太尉,没奈何处,只得随顺听他说,不想要勒燕青相扑,
正要灭高俅的嘴,都起身来道:“好,好!且看相扑!”众人都哄下堂去。宋江亦
醉,主张不定。两个脱了衣裳,就厅阶上,宋江叫把软褥铺下。两个在剪绒毯上,
吐个门户。高俅抢将入来,燕青手到,把高俅扭得定,只一交,攧翻在地褥上,
做一块,半晌挣不起。这一扑,唤做守命扑。宋江、卢俊义慌忙扶起高俅,再穿了
衣服,都笑道:“太尉醉了,如何相扑得成功,切乞恕罪!”高俅惶恐无限,却再
入席,饮至夜深,扶入后堂歇了。

次日又排筵会,与高太尉压惊,高俅遂要辞回,与宋江等作别。宋江道:“某等淹
留大贵人在此,并无异心。惹有瞒昧,天地诛戮!”高俅道:“若是义士肯放高某
回京,便好全家于天子前保奏义士,定来招安,国家重用。若更翻变,天所不盖,
地所不载,死于枪箭之下!”宋江听罢,叩首拜谢。高俅又道:“义士恐不信高某
之言,可留下众将为当。”宋江道:“太尉乃大贵人之言,焉肯失信?何必拘留众
将。容日各备鞍马,俱送回营。”高太尉谢了:“既承如此相款,深感厚意。只此
告回。”宋江等众苦留。当日再排大宴,序旧论新,筵席直至更深方散。
第三日,高太尉定要下山,宋江等相留不住,再设筵宴送行,抬出金银彩缎之类,
约数千金,专送太尉,为折席之礼;众节度使以下,另有馈送。高太尉推却不得,
只得都受了。饮酒中间,宋江又提起招安一事。高俅道:“义士可叫一个精细之人,
跟随某去,我直引他面见天子,奏知你梁山泊衷曲之事,随即好降诏敕。”宋江一
心只要招安,便与吴用计议,教圣手书生萧让跟随太尉前去。吴用便道:“再教铁
叫子乐和作伴,两个同去。”高太尉道:“既然义士相托,便留闻参谋在此为信。”
宋江大喜。至第四日,宋江与吴用带二十余骑,送高太尉并众节度使下山,过金沙
滩二十里外饯别。拜辞了高太尉,自回山寨,专等招安消息。

却说高太尉等一行人马,望济州回来,先有人报知,济州先锋周昂、王焕、项元镇、
张开、太守张叔夜等出城迎接。高太尉进城,略住了数日,收拾军马,教众节度使
各自领兵回程暂歇,听候调用。高太尉自带了周昂并大小牙将头目,领了三军,同
萧让、乐和一行部从,离了济州,迤逦望东京进发。不因高太尉带领梁山泊两个人
来,有分教:风流出众,洞房深处遇君王;细作通神,相府园中寻俊杰。
毕竟高太尉回京,怎地保奏招安宋江等众,且听下回分解。