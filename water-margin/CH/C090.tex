\chapter{五台山宋江参禅~双林镇燕青遇故}

话说五台山这个智真长老,原来是故宋时一个当世的活佛,知得过去未来之
事。数载之前,已知鲁智深是个了身达命之人,只是俗缘未尽,要还杀生之债,因
此教他来尘世中走这一遭。本人宿根,还有道心,今日起这个念头,要来参禅投礼
本师。宋公明亦是素有善心,因此要同鲁智深来参智真长老。
当下宋江与众将只带随行人马,同鲁智深来到五台山下,就将人马屯扎下营,先使
人上山报知。宋江等众兄弟都脱去戎装,各穿随身衣服,步行上山。转到山门外,
只听寺内撞钟击鼓,众僧出来迎接,向前与宋江、鲁智深等施了礼。数内有认得鲁
智深的多,又见齐齐整整这许多头领跟着宋江,尽皆惊讶。堂头首座来禀宋江道:
“长老坐禅入定,不能相接,将军切勿见罪。”遂请宋江等先去知客寮内少坐。供
茶罢,侍者出来请道:“长老禅定方回,已在方丈专候。启请将军进来。”宋江等
一行百余人,直到方丈,来参智真长老。那长老慌忙降阶而接,邀至上堂。各施礼
罢,宋江看那和尚时,六旬之上,眉发尽白,骨格清奇,俨然有天台方广出山之相。
众人入进方丈之内,宋江便请智真长老上座,焚香礼拜。一行众将,都已拜罢,鲁
智深向前插香礼拜。智真长老道:“徒弟一去数年,杀人放火不易。”鲁智深默然
无言。宋江向前道:“久闻长老清德,争奈俗缘浅薄,无路拜见尊颜。今因奉诏破
辽到此,得以拜见堂头大和尚,平生万幸。智深兄弟,虽是杀人放火,忠心不害良
善,今引宋江等众兄弟来参大师。”智真长老道:“常有高僧到此,亦曾闲论世事。
久闻将军替天行道,忠义根心。吾弟子智深跟着将军,岂有差错!”宋江称谢不已。
鲁智深将出一包金银彩缎来,供献本师。智真长老道:“吾弟子,此物何处得来?
无义钱财,决不敢受。”智深禀道:“弟子累经功赏,积聚之物,弟子无用,特地
将来献纳本师,以充公用。”长老道:“众亦难消。与汝置经一藏,消灭罪恶,早
登善果。”鲁智深拜谢已了,宋江亦取金银彩缎上献智真长老,长老坚执不受。宋
江禀说:“我师不纳,可令库司办斋,供献本寺僧众。”当日就五台山寺中宿歇一
宵,长老设素斋相待,不在话下。
且说次日库司办斋完备,五台山寺中法堂上鸣钟击鼓,智真长老会集众僧于法堂
上,讲法参禅。须臾,合寺众僧都披袈裟坐具,到于法堂中坐下。宋江、鲁智深并
众头领,立于两边。引磬响处,两碗红纱灯笼,引长老上升法座。智真长老到法座
上,先拈信香祝赞道:“此一炷香,伏愿皇上圣寿齐天,万民乐业。再拈信香一炷,
愿今斋主,身心安乐,寿算延长。再拈信香一炷,愿今国安民泰,岁稔年和,三教
兴隆,四方宁静。”祝赞已罢,就法座而坐。两下众僧,打罢问讯,复皆侍立。宋
江向前拈香礼拜毕,合掌近前参禅道:“某有一语,敢问吾师:浮世光阴有限,苦
海无边,人身至微,生死最大。”智真长老便答偈曰:
六根束缚多年,四大牵缠已久。堪嗟石火光中,翻了几个筋斗。咦!阎浮世界诸众
生,泥沙堆里频哮吼。
长老说偈已毕,宋江礼拜侍立。众将都向前拈香礼拜,设誓道:“只愿弟兄同生同
死,世世相逢!”焚香已罢,众僧皆退,就请去云堂内赴斋。
众人斋罢,宋江与鲁智深跟随长老来到方丈内。至晚闲话间,宋江求问长老道:“弟
子与鲁智深本欲从师数日,指示愚迷,但以统领大军,不敢久恋。我师语录,实不
省悟。今者拜辞还京,某等众弟兄此去前程如何,万望吾师明彰点化。”智真长老
命取纸笔,写出四句偈语:
当风雁影翩,东阙不团圆。只眼功劳足,双林福寿全。
写毕,递与宋江道:“此是将军一生之事,可以秘藏,久而必应。”宋江看了,不
晓其意,又对长老道:“弟子愚蒙,不悟法语,乞吾师明白开解,以释忧疑。”智
真长老道:“此乃禅机隐语,汝宜自参,不可明说。”长老说罢,唤过智深近前道:
“吾弟子此去,与汝前程永别,正果将临也!与汝四句偈去,收取终身受用。”偈
曰:
逢夏而擒,遇腊而执。听潮而圆,见信而寂。
鲁智深拜受偈语,读了数遍,藏在身边,拜谢本师。又歇了一宵。次日,宋江、鲁
智深并吴用等众头领辞别长老下山,众人便出寺来,智真长老并众僧都送出山门外
作别。
不说长老众僧回寺,且说宋江等众将下到五台山下,引起军马,星火赶来。众将回
到军前,卢俊义、公孙胜等接着宋江众将,都相见了。宋江便对卢俊义等说五台山
众人参禅设誓一事,将出禅语,与卢俊义、公孙胜看了,皆不晓其意。萧让道:“禅
机法语,等闲如何省得?”众皆惊讶不已。
宋江传令,催趱军马起程,众将得令,催起三军人马,望东京进发。凡经过地方,
军士秋毫无犯。百姓扶老携幼,来看王师。见宋江等众将英雄,人人称奖,个个钦
服。宋江等在路行了数日,到一个去处,地名双林镇。当有镇上居民及近村几个农
夫,都走拢来观看。宋江等众兄弟雁行般排着,一对对并辔而行。正行之间,只见
前队里一个头领滚鞍下马,向左边看的人丛里,扯着一个人叫道:“兄长如何在这
里?”两个叙了礼,说着话。宋江的马渐渐近前,看时,却是浪子燕青和一个人说
话。燕青拱手道:“许兄,此位便是宋先锋。”宋江勒住马看那人时,生得:
目炯双瞳,眉分八字。七尺长短身材,三牙掩口髭须。戴一顶乌绉纱抹眉头巾,穿
一领皂沿边褐布道服。系一条杂彩吕公绦,着一双方头青布履。必非碌碌庸人,定
是山林逸士。
宋江见那人相貌古怪,丰神爽雅,忙下马来,躬身施礼道:“敢问高士大名?”那
人望宋江便拜道:“闻名久矣!今日得以拜见。”慌的宋江答拜不迭,连忙扶起道:
“小可宋江,何劳如此。”那人道:“小子姓许,名贯忠,祖贯大名府人氏,今移
居山野。昔日与燕将军交契,不想一别有十数个年头,不得相聚。后来小子在江湖
上,闻得小乙哥在将军麾下,小子欣慕不已。今闻将军破辽凯还,小子特来此处瞻
望,得见各位英雄,平生有幸。欲邀燕兄到敝庐略叙,不知将军肯放否?”燕青亦
禀道:“小弟与许兄久别,不意在此相遇。既蒙许兄雅意,小弟只得去一遭。哥哥
同众将先行,小弟随后赶来。”宋江猛省道:“兄弟燕青常道先生英雄肝胆,只恨
宋某命薄,无缘得遇。今承垂爱,敢邀同往请教。”许贯忠辞谢道:“将军慷慨忠
义,许某久欲相侍左右,因老母年过七旬,不敢远离。”宋江道:“恁地时,却不
敢相强。”又对燕青说道:“兄弟就回,免得我这里放心不下。况且到京,倘早晚
便要朝见。”燕青道:“小弟决不敢违哥哥将令。”又去禀知了卢俊义,两下辞别。
宋江上得马来,前行的众头领,已去了一箭之地,见宋江和贯忠说话,都勒马伺候。
当下宋江策马上前,同众将进发。
话分两头。且说燕青唤一个亲随军汉,拴缚了行囊,另备了一匹马,却把自己的骏
马,让与许贯忠乘坐。到前面酒店里,脱下戎装,穿了随身便服。两人各上了马,
军汉背着包裹,跟随在后,离了双林镇,望西北小路而行。过了些村舍林冈,前面
却是山僻曲折的路。两个说些旧日交情,胸中肝胆。出了山僻小路,转过一条大溪,
约行了三十余里,许贯忠用手指道:“兀那高峻的山中,方是小弟的敝庐在内。”
又行了十数里,才到山中。那山峰峦秀拔,溪涧澄清。燕青正看山景,不觉天色已
晚。但见:
落日带烟生碧雾,断霞映水散红光。
原来这座山叫做大,上古大禹圣人导河,曾到此处。书经上说道:“至于大”
这便是个证见。今属大名府地方。话休繁絮。且说许贯忠引了燕青转过几个山嘴,
来到一个山凹里,却有三四里方圆平旷的所在。树木丛中,闪着两三处草舍。内中
有几间向南傍溪的茅舍,门外竹篱围绕,柴扉半掩,修竹苍松,丹枫翠柏,森密前
后。许贯忠指着说道:“这个便是蜗居。”燕青看那竹篱内,一个黄发村童,穿一
领布衲袄,向地上收拾些晒干的松枝积于茅檐之下。听得马蹄响,立起身往外
看了,叫声奇怪:“这里那得有马经过!”仔细看时,后面马上,却是主人。慌忙
跑出门外,叉手立着,呆呆地看。原来临行备马时,许贯忠说不用銮铃,以此至近
方觉。二人下了马,走进竹篱。军人把马拴了。二人入得草堂,分宾主坐下。茶罢,
贯忠教随来的军人卸下鞍辔,把这两匹马牵到后面草房中,唤童子寻些草料喂养,
仍教军人前面耳房内歇息。燕青又去拜见了贯忠的老母。贯忠携着燕青,同到靠东
向西的草庐内。推开后窗,却临着一溪清水,两人就倚着窗槛坐地。
贯忠道:“敝庐窄陋,兄长休要笑话!”燕青答道:“山明水秀,令小弟应接不暇,
实是难得。”贯忠又问些征辽的事。多样时,童子点上灯来,闭了窗格,掇张桌子,
铺下五六碟菜蔬,又搬出一盘鸡,一盘鱼,及家中藏下的两样山果,旋了一壶热酒。
贯忠筛了一杯,递与燕青道:“特地邀兄到此,村醪野菜,岂堪待客?”燕青称谢
道:“相扰却是不当。”数杯酒后,窗外月光如昼。燕青推窗看时,又是一般情致:
云轻风静,月白溪清,水影山光,相映一室。燕青夸奖不已道:“昔日在大名府,
与兄长最为莫逆。自从兄长应武举后,便不得相见。却寻这个好去处,何等幽雅!
像劣弟恁地东征西逐,怎得一日清闲?”贯忠笑道:“宋公明及各位将军,英雄盖
世,上应罡星,今又威服强虏。像许某蜗伏荒山,那里有分毫及得兄等。俺又有几
分儿不合时宜处,每每见奸党专权,蒙蔽朝廷,因此无志进取,游荡江河,到几个
去处,俺也颇颇留心。”说罢大笑,洗盏更酌。燕青取白金二十两,送与贯忠道:
“些须薄礼,少尽鄙忱。”贯忠坚辞不受。燕青又劝贯忠道:“兄长恁般才略,同
小弟到京师觑方便,讨个出身。”贯忠叹口气说道:“今奸邪当道,妒贤嫉能,如
鬼如蜮的,都是峨冠博带;忠良正直的,尽被牢笼陷害。小弟的念头久灰。兄长到
功成名就之日,也宜寻个退步。自古道:‘雕鸟尽,良弓藏。’”燕青点头嗟叹。
两个说至半夜,方才歇息。
次早,洗漱罢,又早摆上饭来,请燕青吃了,便邀燕青去山前山后游玩。燕青登高
眺望,只见重峦迭障,四面皆山,惟有禽声上下,却无人迹往来。山中居住的人家,
颠倒数过,只有二十余家。燕青道:“这里赛过桃源。”燕青贪看山景,当日天晚,
又歇了一宵。
次日,燕青辞别贯忠道:“恐宋先锋悬念,就此拜别。”贯忠相送出门。贯忠道:
“兄长少待!”无移时,村童托一轴手卷儿出来,贯忠将来递与燕青道:“这是小
弟近来的几笔拙画。兄长到京师,细细的看,日后或者亦有用得着处。”燕青谢了,
教军人拴缚在行囊内。两个不忍分手,又同行了一二里。燕青道:“‘送君千里,
终须一别’,不必远劳,后图再会。”两人各悒怏分手。
燕青望许贯忠回去得远了,方才上马。便教军人也上了马,一齐上路。不则一日,
来到东京,恰好宋先锋屯驻军马于陈桥驿,听候圣旨,燕青入营参见,不题。
且说先是宿太尉并赵枢密中军人马入城,已将宋江等功劳奏闻天子。报说宋先锋等
诸将兵马,班师回军,已到关外。赵枢密前来启奏,说宋江等诸将边庭劳苦之事。
天子闻奏,大加称赞,就传圣旨,命黄门侍郎宣宋江等面君朝见,都教披挂入城。
宋江等众将遵奉圣旨,本身披挂,戎装革带,顶盔挂甲,身穿锦袄,悬带金银牌面,
从东华门而入,都至文德殿朝见天子,拜舞起居,山呼万岁。皇上看了宋江等众将
英雄,尽是锦袍金带,惟有吴用、公孙胜、鲁智深、武松,身着本身服色。天子圣
意大喜,乃曰:“寡人多知卿等征进劳苦,边塞用心,中伤者多,寡人甚为忧戚。”
宋江再拜奏道:“托圣上洪福齐天,臣等众将,虽有中伤,俱各无事。今逆虏投降,
边庭宁息,实陛下威德所致,臣等何劳之有?”再拜称谢。天子特命省院官计议封
爵。太师蔡京、枢密童贯商议奏道:“宋江等官爵,容臣等酌议奏闻。”天子准奏,
仍敕光禄寺大设御宴,钦赏宋江锦袍一领,金甲一副,名马一匹,卢俊义以下给赏
金帛,尽于内府关支。宋江与众将谢恩已罢,尽出宫禁,都到西华门外,上马回营
安歇,听候圣旨。不觉的过了数日,那蔡京、童贯等那里去议甚么封爵,只顾延挨。
且说宋江正在营中闲坐,与军师吴用议论些古今兴亡得失的事,只见戴宗、石秀各
穿微服,来禀道:“小弟辈在营中,兀坐无聊,今日和石秀兄弟闲走一回,特来禀
知兄长。”宋江道:“早些回营,候你们同饮几杯。”戴宗和石秀离了陈桥驿,望
北缓步行来。过了几个街坊市井,忽见路旁一个大石碑,碑上有“造字台”三字,
上面又有几行小字,因风雨剥落,不甚分明。戴宗仔细看了道:“却是苍颉造字之
处。”石秀笑道:“俺们用不着他。”两个笑着,望前又行。到一个去处,偌大一
块空地,地上都是瓦砾。正北上有个石牌坊,横着一片石板,上镌“博浪城”三字。
戴宗沉吟了一回,说道:“原来此处是汉留侯击始皇的所在。”戴宗啧啧称赞道:
“好个留侯!”石秀道:“只可惜这一椎不中!”两个嗟叹了一回,说着话,只顾
望北走去,离营却有二十余里。
石秀道:“俺两个鸟耍了这半日,寻那里吃碗酒回营去?”戴宗道:“兀那前面不
是个酒店?”两个进了酒店,拣个近窗明亮的座头坐地。戴宗敲着桌子叫道:“将
酒来!”酒保搬了五六碟菜蔬,摆在桌上,问道:“官人打多少酒?”石秀道:“先
打两角酒,下饭但是下得口的,只顾卖来。”无移时,酒保旋了两角酒,一盘牛肉,
一盘羊肉,一盘嫩鸡。两个正在那里吃酒闲话,只见一个汉子托着雨伞杆棒,背个
包裹,拽扎起皂衫,腰系着缠袋,腿绷护膝,八搭麻鞋,走得气急喘促,进了店门。
放下伞棒包裹,便向一个座头坐下,叫道:“快将些酒肉来!”过卖旋了一角酒,
摆下两三碟菜蔬。那汉道;“不必文诌了,有肉快切一盘来,俺吃了,要赶路进城
公干。”拿起酒,大口价吃。戴宗把眼瞅着,肚里寻思道:“这鸟是个公人,不知
甚么鸟事?”便向那汉拱手问道:“大哥,甚么事恁般要紧?”那汉一头吃酒吃肉,
一头夹七夹八的说出几句话来。有分教:宋公明再建奇功,汾沁地重归大宋。
毕竟那汉说出甚么话来,且听下回分解。