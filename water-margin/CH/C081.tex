\chapter{燕青月夜遇道君~戴宗定计出乐和}

话说梁山泊好汉,水战三败高俅,尽被擒捉上山。宋公明不肯杀害,尽数放还。
高太尉许多人马回京,就带萧让、乐和前往京师,听候招安一事,却留下参谋闻焕
章在梁山泊里。那高俅在梁山泊时,亲口说道:“我回到朝廷,亲引萧让等,面见
天子,便当力奏保举,火速差人前来招安。”因此上就叫乐和为伴,与萧让一同去
了,不在话下。

且说梁山泊众头目商议,宋江道:“我看高俅此去,未知真实。”吴用笑道:“我
观此人,生的蜂目蛇形,是个转面忘恩之人。他折了许多军马,废了朝廷许多钱粮,
回到京师,必然推病不出,朦胧奏过天子,权将军士歇息,萧让、乐和软监在府里。
若要等招安,空劳神力!”宋江道:“似此怎生奈何?招安犹可,又且陷了二人。”
吴用道:“哥哥再选两个乖觉的人,多将金宝前去京师,探听消息。就行钻刺关节,
把衷情达知今上,令高太尉藏匿不得。此为上计。”燕青便起身说道:“旧年闹了
东京,是小弟去李师师家入肩。不想这一场大闹,他家已自猜了八分。只有一件,
他却是天子心爱的人,官家那里疑他。他自必然奏说:‘梁山泊知得陛下在此私行,
故来惊吓。’已是遮过了。如今小弟多把些金珠去那里入肩,枕头上关节最快。小
弟可长可短,见机而作。”宋江道:“贤弟此去,须担干系!”戴宗便道:“小弟
帮他去走一遭。”神机军师朱武道:“兄长昔日打华州时,尝与宿太尉有恩。此人
是个好心的人。若得本官于天子前早晚题奏,亦是顺事。”

宋江想起九天玄女之言“遇宿重重喜”,莫非正应着此人身上?便请闻参谋来堂上
同坐。宋江道:“相公曾认得太尉宿元景么?”闻焕章道:“他是在下同窗朋友,
如今和圣上寸步不离。此人极是仁慈宽厚,待人接物,一团和气。”宋江道:“实
不瞒相公说,我等疑高太尉回京,必然不奏招安一节。宿太尉旧日在华州降香,曾
与宋江有一面之识。今要使人去他那里打个关节,求他添力,早晚于天子处题奏,
共成此事。”闻参谋答道:“将军既然如此,在下当修尺书奉去。”宋江大喜。随
即教取纸笔来,一面焚起好香,取出玄女课,望空祈祷,卜得个上上大吉之兆。随
即置酒,与戴宗、燕青送行。收拾金珠细软之物两大笼子,书信随身藏了,仍带了
开封府印信公文。两个扮作公人,辞了头领下山,渡过金沙滩,望东京进发。
戴宗托着雨伞,背着个包裹。燕青把水火棍挑着笼子,拽扎起皂衫,腰系着缠袋,
脚下都是腿绷护膝,八搭麻鞋。于路免不得饥餐渴饮,夜住晓行。不则一日,来到
东京,不由顺路入城,却转过万寿门来。两个到得城门边,把门军当住。燕青放下
笼子,打着乡谈说道:“你做甚么当我?”军汉道:“殿帅府有钧旨,梁山泊诸色
人等,恐有夹带入城,因此着仰各门,但有外乡客人出入,好生盘诘。”燕青笑道:
“你便是了事的公人,将着自家人,只管盘问。俺两个从小在开封府勾当,这门下
不知出入了几万遭,你颠倒只管盘问,梁山泊人,眼睁睁的都放他过去了。”便向
身边取出假公文,劈面丢将去道:“你看,这是开封府公文不是?”那监门官听得,
喝道:“既是开封府公文,只管问他怎地?放他入去!”燕青一把抓了公文,揣在
怀里,挑起笼子便走。戴宗也冷笑了一声。两个径奔开封府前来,寻个客店安歇了。
次日,燕青换领布衫穿了,将搭膊系了腰,换顶头巾,歪戴着,只妆做小闲模样。
笼内取了一帕子金珠,分付戴宗道:“哥哥,小弟今日去李师师家干事,倘有些撅
撒,哥哥自快回去。”分付戴宗了当,一直取路径奔李师师家来。到的门前看时,
依旧曲槛雕栏,绿窗朱户,比先时又修的好。燕青便揭起斑竹帘子,从侧首边转将
入来,早闻的异香馥郁。入到客位前,见周回吊挂名贤书画,阶檐下放着三二十盆
怪石苍松,坐榻尽是雕花香楠木,小床坐褥,尽铺锦绣。燕青微微地咳嗽一声,娅
出来见了,便传报李妈妈出来,看见是燕青,吃了一惊,便道:“你如何又来此
间?”燕青道:“请出娘子来,小人自有话说。”李妈妈道:“你前番连累我家,
坏了房子。你有话便说。”燕青道:“须是娘子出来,方才说的。”

李师师在窗子后听了多时,转将出来。燕青看时,别是一般风韵,但见:容貌似海
棠滋晓露,腰肢如杨柳袅东风,浑如阆苑琼姬,绝胜桂宫仙姊。当下李师师轻移莲
步,款蹙湘裙,走到客位里面。燕青起身,把那帕子放在桌上,先拜了李妈妈四拜,
后拜李行首两拜。李师师谦让道:“免礼。俺年纪幼小,难以受拜。”燕青拜罢,
起身道:“前者惊恐,小人等安身无处。”李师师道:“你休瞒我,你当初说道是
张闲,那两个是山东客人。临期闹了一场,不是我巧言奏过官家,别的人时,却不
满门遭祸!他留下词中两句,道是:‘六六雁行连八九,只等金鸡消息。’我那时
便自疑惑,正待要问,谁想驾到,后又闹了这场,不曾问的。今喜汝来,且释我心
中之疑。你不要隐瞒,实对我说知;若不明言,决无干休!”燕青道:“小人实诉
衷曲,花魁娘子休要吃惊。前番来的那个黑矮身材,为头坐的,正是呼保义宋江;
第二位坐的白俊面皮三牙髭须那个,便是柴世宗嫡派子孙,小旋风柴进;这公人打
扮,立在面前的,便是神行太保戴宗;门首和杨太尉厮打的,正是黑旋风李逵;小
人是北京大名府人氏,人都唤小人做浪子燕青。当初俺哥哥来东京求见娘子,教小
人诈作张闲,来宅上入肩。俺哥哥要见尊颜,非图买笑迎欢,只是久闻娘子遭际今
上,以此亲自特来告诉衷曲,指望将替天行道、保国安民之心,上达天听,早得招
安,免致生灵受苦。若蒙如此,则娘子是梁山泊数万人之恩主也!如今被奸臣当道,
谗佞专权,闭塞贤路,下情不能上达,因此上来寻这条门路,不想惊吓娘子。今俺
哥哥无可拜送,只有些少微物在此,万望笑留。”燕青便打开帕子,摊在桌上,都
是金珠宝贝器皿。那虔婆爱的是财,一见便喜,忙叫奶子收拾过了,便请燕青进里
面小阁儿内坐地,安排好细食茶果,殷勤相待。原来李师师家,皇帝不时间来,因
此上公子王孙,富豪子弟,谁敢来他家讨茶吃。

且说当时铺下盘馔酒果,李师师亲自相待。燕青道:“小人是个该死的人,如何敢
对花魁娘子坐地?”李师师道:“休恁地说!你这一班义士,久闻大名,只是奈缘
中间无有好人,与汝们众位作成,因此上屈沉水泊。”燕青道:“前番陈太尉来招
安,诏书上并无抚恤的言语,更兼抵换了御酒。第二番领诏招安,正是诏上要紧字
样,故意读破句读:‘除宋江,卢俊义等大小人众所犯过恶,并与赦免。’因此上,
又不曾归顺。童枢密引将军来,只两阵,杀的片甲不归。次后高太尉役天下民夫,
造船征进,只三阵,人马折其大半,高太尉被俺哥哥活捉上山,不肯杀害,重重管
待,送回京师,生擒人数,尽都放还。他在梁山泊说了大誓,如回到朝廷,奏过天
子,便来招安。因此带了梁山泊两个人来,一个是秀才萧让,一个是能唱乐和,眼
见的把这两人藏在家里,不肯令他出来;损兵折将,必然瞒着天子。”李师师道:
“他这等破耗钱粮,损折兵将,如何敢奏?这话我尽知了。且饮数杯,别作商议。”
燕青道:“小人天性不能饮酒。”李师师道:“路远风霜,到此开怀,也饮几杯。”
燕青被央不过,一杯两盏,只得陪侍。

原来这李师师是个风尘妓女,水性的人,见了燕青这表人物,能言快说,口舌利便,
倒有心看上他。酒席之间,用些话来嘲惹他;数杯酒后,一言半语,便来撩拨。燕
青是个百伶百俐的人,如何不省得?他却是好汉胸襟,怕误了哥哥大事,那里敢来
承惹?李师师道:“久闻的哥哥诸般乐艺,酒边闲听,愿闻也好。”燕青答道:“小
人颇学的些本事,怎敢在娘子跟前卖弄?”李师师道:“我便先吹一曲,教哥哥听!”
便唤娅取箫来,锦袋内掣出那管凤箫。李师师接来,口中轻轻吹动,端的是穿云
裂石之声。燕青听了,喝采不已。

李师师吹了一曲,递过箫来,与燕青道:“哥哥也吹一曲,与我听则个!”燕青却
要那婆娘欢喜,只得把出本事来,接过箫,便呜呜咽咽也吹一曲。李师师听了,不
住声喝采,说道:“哥哥原来恁地吹的好箫!”李师师取过阮来,拨个小小的曲儿,
教燕青听,果然是玉齐鸣,黄莺对啭,余韵悠扬。燕青拜谢道:“小人也唱个曲
儿,伏侍娘子。”顿开咽喉便唱,端的是声清韵美,字正腔真。唱罢又拜。李师师
执盏擎杯,亲与燕青回酒谢唱,口儿里悠悠放出些妖娆声嗽,来惹燕青;燕青紧紧
的低了头,唯喏而已。

数杯之后,李师师笑道:“闻知哥哥好身纹绣,愿求一观,如何?”燕青笑道:“小
人贱体,虽有些花绣,怎敢在娘子跟前揎衣裸体?”李师师说道:“锦体社家子弟,
那里去问揎衣裸体!”三回五次,定要讨看。燕青只的脱膊下来,李师师看了,十
分大喜,把尖尖玉手,便摸他身上。燕青慌忙穿了衣裳。李师师再与燕青把盏,又
把言语来调他。燕青恐怕他动手动脚,难以回避,心生一计,便动问道:“娘子今
年贵庚多少?”李师师答道:“师师今年二十有七。”燕青说道:“小人今年二十
有五,却小两年。娘子既然错爱,愿拜为姊姊!”燕青便起身,推金山,倒玉柱,
拜了八拜。这八拜是拜住那妇人一点邪心,中间里好干大事。若是第二个,在酒色
之中的,也把大事坏了。因此上单显燕青心如铁石,端的是好男子。当时燕青又请
李妈妈来,也拜了,拜做干娘。燕青辞回,李师师道:“小哥只在我家下,休去店
中宿。”燕青道:“既蒙错爱,小人回店中,取了些东西便来。”李师师道:“休
教我这里专望。”燕青道:“店中离此间不远,少刻便到。”

燕青暂别了李师师,径到客店中,把上件事和戴宗说了。戴宗道:“如此最好!
只恐兄弟心猿意马,拴缚不定。”燕青道:“大丈夫处世,若为酒色而忘其本,此
与禽兽何异?燕青但有此心,死于万剑之下!”戴宗笑道:“你我都是好汉,何必
说誓!”燕青道:“如何不说誓,兄长必然生疑!”戴宗道:“你当速去,善觑方
便,早干了事便回,休教我久等。宿太尉的书,也等你来下。”燕青收拾一包零碎
金珠细软之物,再回李师师家,将一半送与李妈妈,一半散与全家大小,无一个不
欢喜。便向客位侧边,收拾一间房,教燕青安歇,合家大小,都叫叔叔。也是缘法
凑巧,至夜,却好有人来报:“天子今晚到来。”燕青听的,便去拜告李师师道:
“姊姊做个方便,今夜教小弟得见圣颜,告的纸御笔赦书,赦了小弟罪犯,出自姊
姊之德!”李师师道:“今晚定教你见天子一面,你却把些本事,动达天颜,赦书
何愁没有?”

看看天晚,月色朦胧,花香馥郁,兰麝芬芳,只见道君皇帝引着一个小黄门,扮做
白衣秀士,从地道中径到李师师家后门来。到的阁子里坐下,便教前后关闭了门户,
明晃晃点起灯烛荧煌。李师师冠梳插带,整肃衣裳,前来接驾。拜舞起居寒温已了,
天子命:“去其整妆衣服,相待寡人。”李师师承旨,去其服色,迎驾入房。家间
已准备下诸般细果,异品肴馔,摆在面前。李师师举杯上劝天子,天子大喜,叫:
“爱卿近前,一处坐地。”李师师见天子龙颜大喜,向前奏道:“贱人有个姑舅兄
弟,从小流落外方,今日才归,要见圣上,未敢擅便,乞取我王圣鉴。”天子道:
“既然是你兄弟,便宣将来见寡人,有何妨?”奶子遂唤燕青直到房内,面见天子。
燕青纳头便拜。官家看了燕青一表人物,先自大喜。李师师叫燕青吹箫,伏侍圣上
饮酒,少刻又拨一回阮,然后叫燕青唱曲。燕青再拜奏道:“所记无非是淫词艳曲,
如何敢伏侍圣上?”官家道:“寡人私行妓馆,其意正要听艳曲消闷,卿当勿疑。”
燕青借过象板,再拜罢,对李师师道:“音韵差错,望姊姊见教。”燕青顿开喉咽,
手拿象板,唱《渔家傲》一曲,道是:
一别家山音信杳,百种相思,肠断何时了。燕子不来花又老,一春瘦的腰儿小。

薄
幸郎君何日到,想自当初,莫要
相逢好。好梦欲成还又觉,绿窗但觉莺啼晓。
燕青唱罢,真乃是新莺乍啭,清韵悠扬。天子甚喜,命教再唱。燕青拜倒在地,奏
道:“臣有一只《减字木兰花》,上达天听。”天子道:“好,寡人愿闻。”燕青
拜罢,遂唱《减字木兰花》一曲,道是:
听哀告,听哀告!贱躯流落谁知道,谁知道!极天罔地,罪恶难分颠倒。

有人提
出火坑中,肝胆常存忠孝,常存忠孝。有朝须把大恩人报!
燕青唱罢,天子失惊,便问:“卿何故有此曲?”燕青大哭,拜在地下。天子转疑,
便道:“卿且诉胸中之事,寡人与卿理会。”燕青奏道:“臣有迷天之罪,不敢上
奏!”天子曰:“赦卿无罪,但奏不妨!”燕青奏道:“臣自幼飘泊江湖,流落山
东,跟随客商,路经梁山泊过,致被劫掳上山,一住三年。今年方得脱身逃命,走
回京师,虽然见的姊姊,则是不敢上街行走。倘或有人认得,通与做公的,此时如
何分说?”李师师便奏道:“我兄弟心中,只有此苦,望陛下做主则个!”天子笑
道:“此事容易,你是李行首兄弟,谁敢拿你!”燕青以目送情与李师师。李师师
撒娇撒痴,奏天子道:“我只要陛下亲书一道赦书,赦免我兄弟,他才放心。”天
子云:“又无御宝在此,如何写的?”李师师又奏道:“陛下亲书御笔,便强似玉
宝天符。救济兄弟做的护身符时,也是贱人遭际圣时。”天子被逼不过,只得命取
纸笔,奶子随即捧过文房四宝。燕青磨的墨浓,李师师递过紫毫象管,天子拂开花
笺黄纸,横内大书一行。临写,又问燕青道:“寡人忘卿姓氏。”燕青道:“男女
唤做燕青。”天子便写御书道:

神霄玉府真主宣和羽士虚静道君皇帝,特赦燕青本身一应无罪,诸司不许拿问。
写罢,下面押个御书花字。燕青再拜叩头受命,李师师执盏擎杯谢恩。天子便问:
“汝在梁山泊,必知那里备细。”燕青奏道:“宋江这伙,旗上大书‘替天行道’,
堂设‘忠义’为名,不敢侵占州府,不肯扰害良民,单杀赃官污吏、谗佞之人,只
是早望招安,愿与国家出力。”天子乃曰:“寡人前者两番降诏,遣人招安,如何
抗拒,不伏归降?”燕青奏道:“头一番招安,诏书上并无抚恤招谕之言,更兼抵
换了御酒,尽是村醪,以此变了事情。第二番招安,故把诏书读破句读,要除宋江,
暗藏弊幸,因此又变了事情。童枢密引军到来,只两阵,杀得片甲不回。高太尉提
督军马,又役天下民夫,修造战船征进,不曾得梁山泊一根折箭;只三阵,杀的手
脚无措,军马折其三停,自己亦被活捉上山,许了招安,方才放回,又带了山上二
人在此,却留下闻参谋在彼质当。”天子听罢,便叹道:“寡人怎知此事!童贯回
京时奏说:‘军士不伏暑热,暂且收兵罢战。’高俅回京奏道:‘病患不能征进,
权且罢战回京。’”李师师奏道:“陛下虽然圣明,身居九重,却被奸臣闭塞贤路,
如之奈何?”天子嗟叹不已。约有更深,燕青拿了赦书,叩头安置,自去歇息。天
子与李师师上床同寝,当夜五更,自有内侍黄门接将去了。

燕青起来,推道清早干事,径来客店里,把说过的话,对戴宗一一说知。戴宗道:
“既然如此,多是幸事。我两个去下宿太尉的书。”燕青道:“饭罢便去。”两个
吃了些早饭,打挟了一笼子金珠细软之物,拿了书信,径投宿太尉府中来。街坊上
借问人时,说太尉在内里未归。燕青道:“这早晚正是退朝时分,如何未归?”街
坊人道:“宿太尉是今上心爱的近侍官员,早晚与天子寸步不离,归早归晚,难以
指定。”正说之间,有人报道:“这不是太尉来也!”燕青大喜,便对戴宗道:“哥
哥,你只在此衙门前伺候,我自去见太尉去。”燕青近前,看见一簇锦衣花帽从人,
捧着轿子。燕青就当街跪下,便道:“小人有书札上呈太尉。”宿太尉见了,叫道:
“跟将进来!”燕青随到厅前。太尉下了轿子,便投侧首书院里坐下。太尉叫燕青
入来,便问道:“你是那里来的干人?”燕青道:“小人从山东来,今有闻参谋书
札上呈。”太尉道:“那个闻参谋?”燕青便向怀中取出书,呈递上去。宿太尉看
了封皮,说道:“我道是那个闻参谋,原来是我幼年间同窗的闻焕章。”遂拆开书
来看时,写道:
侍生闻焕章沐手百拜,奉书太尉恩相钧座前:贱子自髫年时出入门墙,已三十载矣。
昨蒙高殿帅召至军前,参谋大事。奈缘劝谏不从,忠言不听,三番败绩,言之甚羞。
高太尉与贱子一同被掳,陷于缧绁。义士宋公明宽裕仁慈,不忍加害。今高殿帅带
领梁山萧让、乐和赴京,欲请招安,留贱子在此质当。万望恩相不惜齿牙,早晚于
天子前题奏,速降招安之典,俾令义士宋公明等早得释罪获恩,建功立业,国家幸
甚!天下幸甚!救取贱子,实领再生之赐。拂楮拳拳,幸垂照察。

宣和四年春正月

日

焕章再拜奉上
宿太尉看了书,大惊,便问道:“你是谁?”燕青答道:“男女是梁山泊浪子燕青。”
随即出来,取了笼子,径到书院里。燕青禀道:“太尉在华州降香时,多曾伏侍太
尉来,恩相缘何忘了?宋江哥哥有些微物相送,聊表我哥哥寸心。每日占卜课内,
只着求太尉提拔救济。宋江等满眼只望太尉来招安。若得恩相早晚于天子前题奏此
事,则梁山泊十万人之众,皆感大恩!哥哥责着限次,男女便回。”燕青拜辞了,
便出府来。宿太尉使人收了金珠宝物,已有在心。

且说燕青便和戴宗回店中商议:“这两件事都有些次第,只是萧让、乐和在高太尉
府中,怎生得出?”戴宗道:“我和你依旧扮作公人,去高太尉府前伺候。等他府
里有人出来,把些金银贿赂与他,赚得一个厮见。通了消息,便有商量。”当时两
个换了结束,带将金银,径投太平桥来,在衙门前窥望了一回。只见府里一个年纪
小的虞候摇摆将出来,燕青便向前与他施礼。那虞候道:“你是甚人?”燕青道:
“请干办到茶肆中说话。”两个到阁子内,与戴宗相见了,同坐吃茶。燕青道:“实
不瞒干办说,前者太尉从梁山泊带来那两个人,一个跟的叫做乐和,与我这哥哥是
亲眷,欲要见他一见,因此上相央干办。”虞候道:“你两个且休说,节堂深处的
勾当,谁理会的?”戴宗便向袖内取出一锭大银,放在桌子上,对虞候道:“足下
只引的乐和出来,相见一面,不要出衙门,便送这锭银子与足下。”那人见了财物,
一时利动人心,便道:“端的有这两个人在里面。太尉钧旨,只教养在后花园里歇
宿。我与你唤他出来,说了话,你休失信,把银子与我。”戴宗道:“这个自然。”
那人便起身分付道:“你两个只在此茶坊里等我。”那人急急入府去了。

戴宗、燕青两个在茶房中,等不到半个时辰,只见那小虞候慌慌出来说道:“先把
银子来,乐和已叫出在耳房里了。”戴宗与燕青附耳低言“如此如此”,就把银子
与他。虞候得了银子,便引燕青耳房里来见乐和。那虞候道:“你两个快说了话便
去!”燕青便与乐和道:“我同戴宗在这里定计,赚得你两个出去。”乐和道:“直
把我两个养在后花园中,墙垣又高,无计可出,折花梯子尽都藏过了,如何能够出
来?”燕青道:“靠墙有树么?”乐和道:“旁边一遭,都是大柳树。”燕青道:
“今夜晚间,只听咳嗽为号。我在外面,漾过两条索去,你就相近的柳树上,把索
子绞缚了。我两个在墙外,各把一条索子扯住,你两个就从索上盘将出来。四更为
期,不可失误。”那虞候便道:“你两个只管说甚的?快去罢!”乐和自入去了,
暗暗通报了萧让。燕青急急去与戴宗说知,当日至夜伺候着。

且说燕青、戴宗两个,就街上买了两条粗索,藏在身边,先去高太尉府后看了落脚
处。原来离府后是条河,河边却有两只空船缆着,离岸不远。两个便就空船里伏了,
看看听得更鼓已打四更,两个便上岸来,绕着墙后咳嗽,只听的墙里应声咳嗽,两
边都已会意,燕青便把索来漾将过去。约莫里面拴缚牢了,两个在外面对绞定,紧
紧地拽住索头。只见乐和先盘出来,随后便是萧让。两个都溜将下来,却把索子丢
入墙内去了。却去敲开客店门,房中取了行李,就店中打火,做了早饭吃,算了房
宿钱。四个来到城门边,等门开时,一涌出来,望梁山泊回报消息。不是这四个回
来,有分教:宿太尉单奏此事,梁山泊全受招安。
毕竟宿太尉怎生奏请圣旨,且听下回分解。