\chapter{锦豹子小径逢戴宗~病关索长街遇石秀}

话说当时李逵挺着朴刀来斗李云,两个就官路旁边斗了五七合,不分胜败。朱
富便把朴刀去中间隔开,叫道:“且不要斗,都听我说。”二人都住了手。朱富道:
“师父听说,小弟多蒙错爱,指教枪棒,非不感恩。只是我哥哥朱贵现在梁山泊做
了头领,今奉及时雨宋公明将令,着他来照管李大哥。不争被你拿了解官,教我哥
哥如何回去见得宋公明?因此做下这场手段。却才李大哥乘势要坏师父,却是小弟
不肯容他下手,只杀了这些土兵。我们本待去得远了,猜道师父回去不得,必来赶
我。小弟又想师父日常恩念,特地在此相等。师父,你是个精细的人,有甚不省得?
如今杀害了许多人性命,又走了黑旋风,你怎生回去见得知县?你若回去时,定吃
官司,又无人来相救,不如今日和我们一同上山,投奔宋公明,入了伙。未知尊意
若何?”李云寻思了半晌,便道:“贤弟,只怕他那里不肯收留我。”朱富笑道:
“师父,你如何不知山东及时雨大名,专一招贤纳士,结识天下好汉?”李云听了,
叹口气道:“闪得我有家难奔,有国难投,只喜得我又无妻小,不怕吃官司拿了,
只得随你们去休。”李逵便笑道:“我哥哥,你何不早说?”便和李云剪拂了。这
李云不曾娶老小,亦无家当,当下三人合作一处,来赶车子,半路上朱贵接见了大
喜。

四筹好汉跟了车仗便行,于路无话。看看相近梁山泊路上,又迎着马麟、郑天
寿,都相见了,说道:“晁、宋二头领又差我两个下山来探听你消息。今既见了,
我两个先去回报。”当下二人先上山来报知。次日,四筹好汉带了朱富家眷,都至
梁山泊大寨聚义厅来。朱贵向前,先引李云拜见晁、宋二头领,相见众好汉,说道:
“此人是沂水县都头,姓李,名云,绰号青眼虎。”次后朱贵引朱富参拜众位说道:
“这是舍弟朱富,绰号笑面虎。”都相见了。李逵拜了宋江,给还了两把板斧,诉
说取娘至沂岭,被虎吃了,因此杀了四虎。又说假李逵剪径被杀一事,众人大笑。
晁、宋二人笑道:“被你杀了四个猛虎,今日山寨里又添得两个活虎,正宜作庆。”
众多好汉大喜,便教杀羊宰马,做筵席庆贺两个新到头领。晁盖便叫去左边白胜上
首坐定。

吴用道:“近来山寨十分兴旺,感得四方豪杰望风而来,皆是晁、宋二兄之德,
亦众弟兄之福也。然是如此,还请朱贵仍复掌管山东酒店,替回石勇、侯健。朱富
老小,另拨一所房舍住居。目今山寨事业大了,非同旧日,可再设三处酒馆,专一
探听吉凶事情,往来义士上山。如若朝廷调遣官兵捕盗,可以报知如何进兵,好做
准备。西山地面广阔,可令童威、童猛弟兄带领十数个火伴那里开店,令李立带十
数个火家去山南边那里开店,令石勇也带十来个伴当去北山那里开店。仍复都要设
立水亭号箭,接应船只,但有缓急军情,飞捷报来。山前设置三座大关,专令杜迁
总行守把,但有一应委差,不许调遣,早晚不得擅离。又令陶宗旺把总监工,掘港
汊,修水路,开河道,整理宛子城垣,修筑山前大路。他原是庄户出身,修理久惯。
令蒋敬掌管库藏仓廒,支出纳入,积万累千,书算帐目。令萧让设置寨中寨外,山
上山下,三关把隘,许多行移关防文约,大小头领号数。烦令金大坚刊造雕刻,一
应兵符、印信、牌面等项。令侯健管造衣袍铠甲五方旗号等件。令李云监造梁山泊
一应房舍、厅堂。令马麟监管修造大小战船。令宋万、白胜去金沙滩下寨。令王矮
虎、郑天寿去鸭嘴滩下寨。令穆春、朱富管收山寨钱粮。吕方、郭盛于聚义厅两边
耳房安歇。令宋清专管筵宴。”都分拨已定,筵席了三日,不在话下。梁山泊自此
无事,每日只是操练人马,教演武艺。水寨里头领都教习驾船,赴水,船上厮杀,
亦不在话下。

忽一日,宋江与晁盖、吴学究并众人闲话道:“我等弟兄众位今日都共聚大义,
只有公孙一清不见回还。我想他回蓟州探母参师,期约百日便回,今经日久,不知
信息,莫非昧信不来。可烦戴宗兄弟与我去走一遭,探听他虚实下落,如何不来。”
戴宗愿往。宋江大喜,说道:“只有贤弟去得快,旬日便知信息。”当日戴宗别了
众人,次早打扮做承局,下山去了。正是:

虽为走卒,不占军班。一生常作异乡人,两腿欠他行路债。监司出入,皂花藤
杖挂宣牌;帅府行军,黄色绢旗书令字。家居千里,日不移时;紧急军情,时不过
刻。早向山东餐黍米,晚来魏府吃鹅梨。

且说戴宗自离了梁山泊,取路望蓟州来。把四个甲马拴在腿上,作起神行法来,
于路只吃些素茶素食。在路行了三日,来到沂水县界,只闻人说道:“前日走了黑
旋风,伤了好多人,连累了都头李云不知去向,至今无获处。”戴宗听了冷笑。当
日正行之次,只见远远地转过一个人来,手里提着一根浑铁笔管枪。那人看见戴宗
走得快,便立住了脚,叫一声:“神行太保!”戴宗听得,回过脸来,定睛看时,
见山坡下小径边,立着一个大汉,生得头圆耳大,鼻直口方,眉秀目疏,腰细膀阔。
戴宗连忙回转身来问道:“壮士素不曾拜识,如何呼唤贱名?”那汉慌忙答道:“足
下果是神行太保!”撇了枪,便拜倒在地。戴宗连忙扶住答礼,问道:“足下高姓
大名?”那汉道:“小弟姓杨,名林,祖贯彰德府人氏,多在绿林丛中安身,江湖
上都叫小弟做锦豹子杨林。数月之前,路上酒肆里遇见公孙胜先生,同在店中吃酒
相会,备说梁山泊晁、宋二公招贤纳士,如此义气,写下一封书,教小弟自来投大
寨入伙,只是不敢轻易擅进。公孙先生又说:‘李家道口旧有朱贵开酒店在彼,招
引上山入伙的人。山寨中亦有一个招贤飞报头领,唤做神行太保戴院长,日行八百
里路。’今见兄长行步非常,因此唤一声看,不想果是仁兄,正是天幸,无心得遇。”
戴宗道:“小可特为公孙胜先生回蓟州去,杳无音信,今奉晁、宋二公将令,差遣
来蓟州探听消息,寻取公孙胜还寨,不期却遇足下。”杨林道:“小弟虽是彰德府
人,这蓟州管下地方州郡都走遍了。倘若不弃,就随侍兄长同去走一遭。”戴宗道:
“若得足下作伴,实是万幸。寻得公孙先生见了,一同回梁山泊去未迟。”

杨林见说了,大喜,就邀住戴宗,结拜为兄。戴宗收了甲马,两个缓缓而行,
到晚就投村店歇了。杨林置酒请戴宗,戴宗道:“我使神行法,不敢食荤。”两个
只买些素馔相待。过了一夜,次日早起,打火吃了早饭,收拾动身。杨林便问道:
“兄长使神行法走路,小弟如何走得上?只怕同行不得!”戴宗笑道:“我的神行
法也带得人同走。我把两个甲马拴在你腿上,作起法来,也和我一般走得快,要行
便行,要住便住。不然,你如何赶得我走?”杨林道:“只恐小弟是凡胎浊骨,比
不得兄长神体。”戴宗道:“不妨,我这法,诸人都带得。作用了时,和我一般行;
只是我自吃素,并无妨碍。”当时取两个甲马,替杨林缚在腿上,戴宗也只缚了两
个,作用了神行法,吹口气在上面,两个轻轻地走了去,要紧要慢,都随着戴宗行。
两个于路闲说些江湖上的事,虽只见缓缓而行,正不知走了多少路。

两个行到巳牌时分,前面来到一个去处,四围都是高山,中间一条驿路。杨林
却自认得,便对戴宗说道:“哥哥,此间地名,唤做饮马川,前面兀那高山里常常
有大伙在内,近日不知如何。因为山势秀丽,水绕峰环,以此唤做饮马川。”两个
正来到山边过,只听得忽地一声锣响,战鼓乱鸣,走出一二百小喽罗,拦住去路,
当先拥着两筹好汉,各挺一条朴刀,大喝道:“行人须住脚。你两个是甚么鸟人?
那里去的?会事的快把买路钱来,饶你两个性命!”杨林笑道:“哥哥,你看我结
果那呆鸟!”拈着笔管枪抢将入去。那两个好汉见他来得凶,走近前来看了,上首
的那个便叫道:“且不要动手,兀的不是杨林哥哥么!”杨林见了,却才认得。上
首那个大汉提着军器向前剪拂了,便唤下首这个长汉都来施礼罢。杨林请过戴宗说
道:“兄长且来和这两个弟兄相见。”戴宗问道:“这两个壮士是谁?如何认得贤
弟?”杨林便道:“这个认得小弟的好汉,他原是盖天军襄阳府人氏,姓邓,名飞。
为他双睛红赤,江湖上人都唤他做火眼狻猊。能使一条铁链,人皆近他不得。多曾
合伙,一别五年,不曾见面,谁想今日却在这里相遇着!”邓飞便问道:“杨林哥
哥,这位兄长是谁,必不是等闲人也。”杨林道:“我这仁兄,是梁山泊好汉中神
行太保戴宗的便是。”邓飞听了道:“莫不是江州的戴院长,能行八百里路程的?”
戴宗答道:“小可便是。”那两个头领慌忙剪拂道:“平日只听得说大名,不想今
日在此拜识尊颜!”戴宗看那邓飞时,生得如何,有诗为证:
原是襄阳闲扑汉,江湖飘荡不思归。
多人肉双睛赤,火眼狻猊是邓飞。

当下二位壮士施礼罢,戴宗又问道:“这位好汉高姓大名?”邓飞道:“我这
兄弟,姓孟,名康,祖贯是真定州人氏,善造大小船只。原因押送花石纲,要造大
船,嗔怪这提调官催并责罚他,把本官一时杀了,弃家逃走在江湖上绿林中安身,
已得年久。因他长大白净,人都见他一身好肉体,起他一个绰号,叫他做玉幡竿孟
康。”戴宗见说,大喜。看那孟康怎生模样,有诗为证:
能攀强弩冲头阵,善造艨艟越大江。
真州妙手楼船匠,白玉幡竿是孟康。

当时戴宗见了二人,心中甚喜,四筹好汉说话间,杨林问道:“二位兄弟在此
聚义几时了?”邓飞道:“不瞒兄长说,也有一年多了。只半载前在这直西地面上
遇着一个哥哥,姓裴,名宣,祖贯是京兆府人氏,原是本府六案孔目出身,极好刀
笔;为人忠直聪明,分毫不肯苟且,本处人都称他铁面孔目。亦会拈枪使棒,舞剑
抡刀,智勇足备。为因朝廷除将一员贪滥知府到来,把他寻事刺配沙门岛,从我这
里经过,被我们杀了防送公人,救了他在此安身,聚集得三二百人。这裴宣极使得
好双剑,让他年长,现在山寨中为主。烦请二位义士同往小寨,相会片时。”便叫
小喽罗牵过马来,请戴宗、杨林都上了马,四骑马望山寨来。行不多时,早到寨前,
下了马,裴宣已有人报知,连忙出寨,降阶而接。戴宗、杨林看裴宣时,果然好表
人物,生得面白肥胖,四平八稳,心中暗喜。有诗为证:
问事时巧智心灵,落笔处神号鬼哭。
心平恕毫发无私,称裴宣铁面孔目。

当下裴宣邀请二位义士到聚义厅上,俱各讲礼罢,谦让戴宗正面坐了,次是裴
宣、杨林、邓飞、孟康,五筹好汉,宾主相待,坐定筵宴,当日大吹大擂饮酒。看
官听说,这也都是地煞星之数,时节到来,天幸自然义聚相逢,有诗为证:
豪杰遭逢信有因,连环钩锁共相寻。
汉廷将相由屠钓,莫怪梁山错用心。

当下众人饮酒中间,戴宗在筵上说起晁、宋二头领招贤纳士,结识天下四方豪
杰,待人接物,一团和气,仗义疏财,许多好处。众头领同心协力,八百里梁山泊
如此雄壮,中间宛子城、蓼儿洼,四下里都是茫茫烟水,更有许多兵马,何愁官兵
来到。只管把言语说他三个。裴宣回道:“小弟寨中也有三百来人马,财赋亦有十
余辆车子,粮食草料不算,倘若仁兄不弃微贱时,引荐于大寨入伙,愿听号令效力。
未知尊意若何?”戴宗大喜道:“晁、宋二公待人接物,并无异心;更得诸公相助,
如锦上添花。若果有此心,可便收拾下行李,待小可和杨林去蓟州见了公孙胜先生
回来,那时一同扮做官军,星夜前往。”众人大喜。酒至半酣,移去后山断金亭上,
看那饮马川景致吃酒,端的好个饮马川。但见:

一望茫茫野水,周回隐隐青山。几多老树映残霞,数片彩云飘远岫。荒田寂寞,
应无稚子看牛;古渡凄凉,那得奚人饮马。只好强人安寨栅,偏宜好汉展旌旗。

戴宗看了这饮马川一派山景,喝采道:“好山好水,真乃秀丽,你等二位如何
来得到此?”邓飞道:“原是几个不成材小厮们在这里屯扎,后被我两个来夺了这
个去处。”众皆大笑。五筹好汉吃得大醉。裴宣起身舞剑助酒,戴宗称赞不已。至
晚,各自回寨内安歇。次日,戴宗定要和杨林下山,三位好汉苦留不住,相送到山
下作别,自回寨里收拾行装,整理动身,不在话下。

且说戴宗和杨林离了饮马川山寨,在路晓行夜住,早来到蓟州城外,投个客店
安歇了。杨林便道:“哥哥,我想公孙胜先生是个出家人,必是山间林下村落中住,
不在城里。”戴宗道:“说得是。”当时二人先去城外,到处询问公孙胜先生下落
消息,并无一个人晓得他。住了一日,次早起来,又去远远村坊街市访问人时,亦
无一个认得。两个又回店中歇了。第三日,戴宗道:“敢怕城中有人认得他。”当
日和杨林却入蓟州城里来寻他。两个寻问老成人时,都道:“不认得,敢不是城中
人。只怕是外县名山大刹居住。”

杨林正行到一个大街,只见远远地一派鼓乐,迎将一个人来。戴宗、杨林立在
街上看时,前面两个小牢子,一个驮着许多礼物花红,一个捧着若干缎子彩缯之物;
后面青罗伞下,罩着一个押狱刽子。那人生得好表人物,露出蓝靛般一身花绣,两
眉入鬓,凤眼朝天,淡黄面皮,细细有几根髭髯。那人祖贯是河南人氏,姓杨,名
雄,因跟一个叔伯哥哥来蓟州做知府,一向流落在此。续后一个新任知府,却认得
他,因此就参他做两院押狱,兼充市曹行刑刽子。因为他一身好武艺,面貌微黄,
以此人都称他做病关索杨雄。有一首《临江仙》词,单道着杨雄好处:

两臂雕青镌嫩玉,头巾环眼嵌玲珑。鬓边爱插翠芙蓉。背心书刽字,衫串染猩
红。

问事厅前逞手段,行刑刀利如风。微黄面色细眉浓。人称病关索,好汉是
杨雄。

当时杨雄在中间走着,背后一个小牢子擎着鬼头靶法刀。原来才去市心里决刑
了回来,众相识与他挂红贺喜,送回家去,正从戴宗、杨林面前迎将过来。一簇人
在路口拦住了把盏,只见侧首小路里又撞出七八个军汉来,为头的一个,叫做踢杀
羊张保。这汉是蓟州守御城池的军,带着这几个,都是城里城外时常讨闲钱使的破
落户汉子,官司累次奈何他不改,为见杨雄原是外乡人来蓟州,却有人惧怕他,因
此不怯气。当日正见他赏赐得许多缎匹,带了这几个没头神,吃得半醉,却好赶来
要惹他。又见众人拦住他在路口把盏,那张保拨开众人,钻过面前叫道:“节级拜
揖。”杨雄道:“大哥来吃酒。”张保道:“我不要吃酒,我特来问你借百十贯钱
使用。”杨雄道:“虽是我认得大哥,不曾钱财相交,如何问我借钱?”张保道:
“你今日诈得百姓许多财物,如何不借我些?”杨雄应道:“这都是别人与我做好
看的,怎么是诈得百姓的?你来放刁,我与你军卫有司,各无统属。”张保不应,
便叫众人向前一哄,先把花红缎子都抢了去。杨雄叫道:“这厮们无礼。”却待向
前打那抢物事的人,被张保劈胸带住,背后又是两个来拖住了手,那几个都动起手
来,小牢子们各自回避了。

杨雄被张保并两个军汉逼住了,施展不得,只得忍气,解拆不开。正闹中间,
只见一条大汉挑着一担柴来,看见众人逼住杨雄,动弹不得。那大汉看了,路见不
平,便放下柴担,分开众人,前来劝道:“你们因甚打这节级?”那张保睁起眼来
喝道:“你这打脊饿不死冻不杀的乞丐,敢来多管!”那大汉大怒,焦躁起来,将
张保劈头只一提,一交颠翻在地。那几个帮闲的见了,却待要来动手,早被那大汉
一拳一个,都打的东倒西歪。杨雄方才脱得身,把出本事来施展动,一对拳头穿梭
相似,那几个破落户都打翻在地。张保见不是头,爬将起来,一直走了。杨雄忿怒,
大踏步赶将去。张保跟着抢包袱的走,杨雄在后面追着,赶转小巷去了。那大汉兀
自不歇手,在路口寻人厮打。戴宗、杨林看了,暗暗地喝采道:“端的是好汉,此
乃‘路见不平,拔刀相助’,真壮士也!”正是:
匣里龙泉争欲出,只因世有不平人。
旁观能辨非和是,相助安知疏与亲。

当时戴宗、杨林便向前邀住劝道:“好汉看我二人薄面,且罢休了。”两个把
他扶劝到一个巷内。杨林替他挑了柴担,戴宗挽住那汉手,邀入酒店里来。杨林放
下柴担,同到阁儿里面。那大汉叉手道:“感蒙二位大哥解救了小人之祸。”戴宗
道:“我弟兄两个也是外乡人,因见壮士仗义之事,只恐一时拳手太重,误伤人命,
特地做这个出场,请壮士酌三杯,到此相会结义则个。”那大汉道:“多得二位仁
兄解拆小人这场,却又蒙赐酒相待,实是不当。”杨林便道:“‘四海之内,皆兄
弟也’,有何伤乎?且请坐。”戴宗相让,那汉那里肯僭上?戴宗、杨林一带坐了,
那汉坐于对席。叫过酒保,杨林身边取出一两银子来,把与酒保道:“不必来问,
但有下饭,只顾买来与我们吃了,一发总算。”酒保接了银子去,一面铺下菜蔬、
果品、按酒之类。

三人饮过数杯,戴宗问道:“壮士高姓大名?贵乡何处?”那汉答道:“小人
姓石,名秀,祖贯是金陵建康府人氏。自小学得些枪棒在身,一生执意,路见不平,
但要去相助,人都呼小弟作‘拚命三郎’。因随叔父来外乡贩羊马卖,不想叔父半
途亡故,消折了本钱,还乡不得,流落在此蓟州卖柴度日。既蒙拜识,当以实告。”
戴宗道:“小可两个因来此间干事,得遇壮士。如此豪杰流落在此卖柴,怎能够发
迹?不若挺身江湖上去,做个下半世快乐也好。”石秀道:“小人只会使些枪棒,
别无甚本事,如何能够发达快乐?”戴宗道:“这般时节认不得真,一者朝廷不明,
二乃奸臣闭塞。小可一个薄识,因一口气去投奔了梁山泊宋公明入伙,如今论秤分
金银,换套穿衣服,只等朝廷招安了,早晚都做个官人。”石秀叹口气道:“小人
便要去,也无门路可进。”戴宗道:“壮士若肯去时,小可当以相荐。”石秀道:
“小人不敢拜问二位官人贵姓?”戴宗道:“小可姓戴名宗,兄弟姓杨名林。”石
秀道:“江湖上听的说个江州神行太保,莫非正是足下?”戴宗道:“小可便是。”
叫杨林身边包袱内取一锭十两银子,送与石秀做本钱。石秀不敢受,再三谦让,方
才收了,才知道他是梁山泊神行太保。正欲诉说些心腹之话,投托入伙,只听得外
面有人寻问入来。三个看时,却是杨雄带领着二十余人,都是做公的,赶入酒店里
来。戴宗、杨林见人多,吃了一惊,乘闹哄里,两个慌忙走了。

石秀起身迎住道:“节级那里去来?”杨雄便道:“大哥,何处不寻你,却在
这里饮酒?我一时被那厮封住了手,施展不得,多蒙足下气力,救了我这场便宜。
一时间只顾赶了那厮去,夺他包袱,却撇了足下。这伙兄弟听得我厮打,都来相助,
依还夺得抢去的花红缎匹回来,只寻足下不见。却才有人说道:‘两个客人,劝他
去酒店里吃酒。’因此才知得,特地寻将来。”石秀道:“却才是两个外乡客人,
邀在这里酌三杯,说些闲话,不知节级呼唤。”杨雄大喜,便问道:“足下高姓大
名?贵乡何处?因何在此?”石秀答道:“小人姓石,名秀,祖贯是金陵建康府人
氏。平生性直,路见不平,便要去舍命相护,以此都唤小人做‘拚命三郎’。因随
叔父来此地贩卖羊马,不期叔父半途亡故,消折了本钱,流落在此蓟州卖柴度日。”
杨雄看石秀时,好个壮士,生得上下相等。有首《西江月》词,单道着石秀好处。
但见:

身似山中猛虎,性如火上浇油。心雄胆大有机谋,到处逢人搭救。

全仗一
条杆棒,只凭两个拳头。掀天声价满皇
州,拚命三郎石秀。

当下杨雄又问石秀道:“却才和足下一处饮酒的客人何处去了?”石秀道:“他
两个见节级带人进来,只道相闹,以此去了。”杨雄道:“恁地时,先唤酒保取两
瓮酒来,大碗叫众人一家三碗,吃了去,明日却得来相会。”众人都吃了酒,自去
散了。杨雄便道:“石秀三郎,你休见外。想你此间必无亲眷,我今日就结义你做
个弟兄如何?”石秀见说大喜,便说道:“不敢动问节级贵庚?”杨雄道:“我今
年二十九岁。”石秀道:“小弟今年二十八岁,就请节级坐,受小弟拜为哥哥。”
石秀拜了四拜。杨雄大喜,便叫酒保安排饮馔酒果来,“我和兄弟今日吃个尽醉方
休。”

正饮酒之间,只见杨雄的丈人潘公,带领了五七个人,直寻到酒店里来。杨雄
见了,起身道:“泰山来做甚么?”潘公道:“我听得你和人厮打,特地寻将来。”
杨雄道:“多谢这个兄弟救护了我,打得张保那厮见影也害怕。我如今就认义了石
家兄弟做我兄弟。”潘公叫:“好,好,且叫这几个弟兄吃碗酒了去。”杨雄便叫
酒保讨酒来,每人三碗吃了去,便叫潘公中间坐了,杨雄对席上首,石秀下首。三
人坐下,酒保自来斟酒。潘公见了石秀这等英雄长大,心中甚喜,便说道:“我女
婿得你做个兄弟相帮,也不枉了公门中出入,谁敢欺负他!”又问道:“叔叔原曾
做甚买卖道路?”石秀道:“先父原是操刀屠户。”潘公道:“叔叔曾省得杀牲口
的勾当么?”石秀笑道:“自小吃屠家饭,如何不省得宰杀牲口?”潘公道:“老
汉原是屠户出身,只因年老做不得了,止有这个女婿,他又自一身入官府差遣,因
此撇下这行衣饭。”三人酒至半酣,计算酒钱,石秀将这担柴也都准折了。三人取
路回来,杨雄入得门,便叫:“大嫂,快来与这叔叔相见。”只见布帘里面应道:
“大哥,你有甚叔叔?”杨雄道:“你且休问,先出来相见。”布帘起处,走出那
个妇人来。原来那妇人是七月七日生的,因此小字唤做巧云,先嫁了一个吏员,是
蓟州人,唤做王押司,两年前身故了,方才晚嫁得杨雄,未及一年夫妻。石秀见那
妇人出来,慌忙向前施礼道:“嫂嫂请坐。”石秀便拜,那妇人道:“奴家年轻,
如何敢受礼?”杨雄道:“这个是我今日新认义的兄弟,你是嫂嫂,可受半礼。”
当下石秀推金山,倒玉柱,拜了四拜。那妇人还了两礼,请入来里面坐地,收拾一
间空房,教叔叔安歇。话休絮烦。次日,杨雄自出去应当官府,分付家中道:“安
排石秀衣服巾帻。”客店内有些行李包裹,都教去取来杨雄家里安放了。

却说戴宗、杨林自酒店里看见那伙做公的入来寻访石秀,闹哄里两个自走了,
回到城外客店中歇了。次日,又去寻问公孙胜两日,绝无人认得,又不知他下落住
处,两个商量了且回去。当日收拾了行李,便起身离了蓟州,自投饮马川来,和裴
宣、邓飞、孟康一行人马,扮作官军,星夜望梁山泊来。戴宗要见他功劳,又纠合
得许多人马上山,山上自做庆贺筵席,不在话下。

再说有杨雄的丈人潘公,自和石秀商量,要开屠宰作坊。潘公道:“我家后门
头是一条断路小巷,又有一间空房在后面,那里井水又便,可做作坊,就教叔叔做
房在里面,又好照管。”石秀见了也喜,“端的便益。”潘公再寻了个旧时识熟副
手,“只央叔叔掌管帐目。”石秀应承了,叫了副手,便把大青大绿妆点起肉案子、
水盆、砧头,打磨了许多刀杖,整顿了肉案,打并了作坊、猪圈,起上十数个肥猪,
选个吉日,开张肉铺。众邻舍亲戚都来挂红贺喜,吃了一两日酒,杨雄一家,得石
秀开了店,都欢喜。自此无话。一向潘公、石秀,自做买卖。

不觉光阴迅速,又早过了两个月有余。时值秋残冬到,石秀里里外外,身上都
换了新衣穿着。石秀一日早起五更,出外县买猪,三日了方回家来,只见铺店不开。
却到家里看时,肉店砧头也都收过了,刀杖家火亦藏过了。石秀是个精细的人,看
在肚里便省得了,自心中忖道:“常言:‘人无千日好,花无百日红。’哥哥自出
外去当官,不管家事,必然嫂嫂见我做了这些衣裳,一定背后有说话。又见我两日
不回,必有人搬口弄舌,想是疑心,不做买卖。我休等他言语出来,我自先辞了回
乡去休。自古道:‘那得长远心的人?’”石秀已把猪赶在圈里,却去房中换了脚
手,收拾了包裹行李,细细写了一本清帐,从后面入来。潘公已安排下些素酒食,
请石秀坐定吃酒。潘公道:“叔叔远出劳心,自赶猪来辛苦。”石秀道:“丈丈,
礼当。且收过了这本明白帐目。若上面有半点私心,天地诛灭。”潘公道:“叔叔
何故出此言?并不曾有个甚事。”石秀道:“小人离乡五七年了,今欲要回家去走
一遭,特地交还帐目。今晚辞了哥哥,明早便行。”潘公听了,大笑起来道:“叔
叔差矣。你且住,听老汉说。”那老子言无数句,话不一席。有分教:报恩壮士提
三尺,破戒沙门丧九泉。

毕竟潘公说出甚言语来,且听下回分解。