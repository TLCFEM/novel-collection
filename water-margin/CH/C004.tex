\chapter{赵员外重修文殊院~鲁智深大闹五台山}

话说当下鲁提辖扭过身来看时,拖扯的不是别人,却是渭州酒楼上救了的金老。
那老儿直拖鲁达到僻静处,说道:“恩人,你好大胆!现今明明地张挂榜文,出一
千贯赏钱捉你,你缘何却去看榜?若不是老汉遇见时,却不被做公的拿了。榜上现
写着你年甲、貌相、贯址。”鲁达道:“洒家不瞒你说,因为你上,就那日回到状
元桥下,正迎着郑屠那厮,被洒家三拳打死了,因此上在逃。一到处撞了四五十日,
不想来到这里。你缘何不回东京去,也来到这里?”金老道:“恩人在上:自从得
恩人救了,老汉寻得一辆车子,本欲要回东京去,又怕这厮赶来,亦无恩人在彼搭
救,因此不上东京去。随路望北来,撞见一个京师古邻,来这里做买卖,就带老汉
父子两口儿到这里。亏杀了他,就与老汉女儿做媒,结交此间一个大财主赵员外,
养做外宅,衣食丰足,皆出于恩人。我女儿常常对他孤老说提辖大恩。那个员外也
爱刺枪使棒,常说道:‘怎地得恩人相会一面也好。’想念如何能够得见。且请恩
人到家过几日,却再商议。”

鲁提辖便和金老行不得半里,到门首,只见老儿揭起帘子,叫道:“我儿,大
恩人在此。”那女孩儿浓妆艳饰,从里面出来,请鲁达居中坐了,插烛也似拜了六
拜,说道:“若非恩人垂救,怎能够有今日。”鲁达看那女子时,另是一般丰韵,
比前不同。但见:

金钗斜插,掩映乌云;翠袖巧裁,轻笼瑞雪。樱桃口浅晕微红,春笋手半舒嫩
玉。纤腰袅娜,绿罗裙微露金莲;素体轻盈,红绣袄偏宜玉体。脸堆三月娇花,
眉扫初春嫩柳。香肌扑簌瑶台月,翠鬓笼松楚岫云。

那女子拜罢,便请鲁提辖道:“恩人上楼去请坐。”鲁达道:“不须生受,洒
家便要去。”金老便道:“恩人既到这里,如何肯放教你便去?”老儿接了杆棒包
裹,请到楼上坐定。老儿分付道:“我儿陪侍恩人坐坐,我去安排饭来。”鲁达道:
“不消多事,随分便好。”老儿道:“提辖恩念,杀身难报,量些粗食薄味,何足
挂齿。”女子留住鲁达在楼上坐地,金老下来,叫了家中新讨的小厮,分付那个丫
嬛一面烧着火。老儿和这小厮上街来,买了些鲜鱼、嫩鸡、酿鹅、肥鮓、时新果
子之类归来。一面开酒,收拾菜蔬,都早摆了,搬上楼来。春台上放下三个盏子,
三双箸,铺下菜蔬、果子、嗄饭等物,丫嬛将银酒壶烫上酒来。女父二人,轮番把
盏。金老倒地便拜。鲁提辖道:“老人家如何恁地下礼,折杀俺也。”金老说道:
“恩人听禀:前日老汉初到这里,写个红纸牌儿,旦夕一炷香,父女两个兀自拜哩。
今日恩人亲身到此,如何不拜?”鲁达道:“却也难得你这片心。”

三人慢慢地饮酒。将及天晚,只听得楼下打将起来。鲁提辖开窗看时,只见楼
下三二十人,各执白木棍棒,口里都叫拿将下来。人丛里一个人,骑在马上,口里
大喝道:“休教走了这贼!”鲁达见不是头,拿起凳子,从楼上打将下来。金老连
忙摇手叫道:“都不要动手。”那老儿抢下楼去,直至那骑马的官人身边,说了几
句言语,那官人笑将起来,便喝散了那二三十人,各自去了。那官人下马,入到里
面,老儿请下鲁提辖来,那官人扑翻身便拜道:“闻名不如见面,见面胜似闻名,
义士提辖受礼。”鲁达便问那金老道:“这官人是谁?素不相识,缘何便拜洒家?”
老儿道:“这个便是我儿的官人赵员外。却才只道老汉引甚么郎君子弟在楼上吃酒,
因此引庄客来厮打。老汉说知,方才喝散了。”鲁达道:“原来如此。怪员外不得。”
赵员外再请鲁提辖上楼坐定。金老重整杯盘,再备酒食相待。赵员外让鲁达上首坐
地,鲁达道:“洒家怎敢!”员外道:“聊表相敬之礼,小子多闻提辖如此豪杰,
今日天赐相见,实为万幸。”鲁达道:“洒家是个粗卤汉子,又犯了该死的罪过。
若蒙员外不弃贫贱,结为相识,但有用洒家处,便与你去。”赵员外大喜,动问打
死郑屠一事,说些闲话,较量些枪法。吃了半夜酒,各自歇了。

次日天明,赵员外道:“此处恐不稳便,可请提辖到敝庄住几时。”鲁达问道:
“贵庄在何处?”员外道:“离此间十里多路,地名七宝村便是。”鲁达道:“最
好。”员外先使人去庄上叫牵两匹马来。未及晌午,马已到来,员外便请鲁提辖上
马,叫庄客担了行李,鲁达相辞了金老父女二人,和赵员外上了马。两个并马行程,
于路说些闲话,投七宝村来。不多时,早到庄前下马,赵员外携住鲁达的手,直至
草堂上,分宾而坐;一面叫杀羊置酒相待。晚间收拾客房安歇,次日又备酒食管待。
鲁达道:“员外错爱,洒家如何报答。”赵员外便道:“‘四海之内,皆兄弟也。’
如何言报答之事。”

话休絮烦。鲁达自此之后,在这赵员外庄上住了五七日。忽一日,两个正在书
院里闲坐说话,只见金老急急奔来庄上,径到书院里,见了赵员外并鲁提辖。见没
人,便对鲁达道:“恩人,不是老汉心多,为是恩人前日老汉请在楼上吃酒,员外
误听人报,引领庄客来闹了街坊,后却散了,人都有些疑心,说开去。昨日有三四
个做公的来,邻舍街坊打听得紧,只怕要来村里缉捕恩人。倘或有些疏失,如之奈
何?”鲁达道:“恁地时,洒家自去便了。”赵员外道:“若是留提辖在此,诚恐
有些山高水低,教提辖怨怅;若不留提辖来,许多面皮都不好看。赵某却有个道理,
教提辖万无一失,足可安身避难,只怕提辖不肯。”鲁达道:“洒家是个该死的人,
但得一处安身便了,做甚么不肯?”赵员外道:“若如此,最好。离此间三十余里
有座山,唤做五台山,山上有一个文殊院,原是文殊菩萨道场。寺里有五七百僧人,
为头智真长老,是我弟兄。我祖上曾舍钱在寺里,是本寺的施主檀越。我曾许下剃
度一僧在寺里,已买下一道五花度牒在此,只不曾有个心腹之人,了这条愿心。如
是提辖肯时,一应费用,都是赵某备办,委实肯落发做和尚么?”鲁达寻思:“如
今便要去时,那里投奔人,不如就了这条路罢。”便道:“既蒙员外做主,洒家情
愿做了和尚,专靠员外照管。”当时说定了,连夜收拾衣服盘缠,缎匹礼物,排担
了。

次日早起来,叫庄客挑了,两个取路望五台山来。辰牌已后,早到那山下。鲁
提辖看那五台山时,果然好座大山!但见:

云遮峰顶,日转山腰。嵯峨仿佛接天关,崒嵂参差侵汉表。岩前花木舞春风,
暗吐清香;洞口藤萝披宿雨,倒悬嫩线。飞云瀑布,银河影浸月光寒;峭壁苍
松,铁角铃摇龙尾动。山根雄峙三千界,峦势高擎几万年。
赵员外与鲁提辖两乘轿子,抬上山来,一面使庄客前去通报。到得寺前,早有
寺中都寺、监寺,出来迎接。两个下了轿子,去山门外亭子上坐定。寺内智真长老
得知,引着首座、侍者,出山门外来迎接。赵员外和鲁达向前施礼,真长老打了问
讯,说道:“施主远出不易。”赵员外答道:“有些小事,特来上刹相浼。”真长老便
道:“且请员外方丈吃茶。”赵员外前行,鲁达跟在背后,看那文殊寺,果然是好座
大刹!但见:

山门侵翠岭,佛殿接青云。钟楼与月窟相连,经阁共峰峦对立。香积厨通一泓
泉水,众僧寮纳四面烟霞。老僧方丈斗牛边,禅客经堂云雾里。白面猿时时献
果,将怪石敲响木鱼;黄斑鹿日日衔花,向宝殿供养金佛。七层宝塔接丹霄,
千古圣僧来大刹。
当时真长老请赵员外并鲁达到方丈。长老邀员外向客席而坐,鲁达便去下首,
坐在禅椅上。员外叫鲁达附耳低言:“你来这里出家,如何便对长老坐地?”鲁达道:
“洒家不省得。”起身立在员外肩下。面前首座、维那、侍者、监寺、都寺、知客、
书记,依次排立东西两班。庄客把轿子安顿了,一齐搬将盒子入方丈来,摆在面前。
长老道:“何故又将礼物来?寺中多有相渎檀越处。”赵员外道:“些小薄礼,何
足称谢!”道人、行童收拾去了。赵员外起身道:“一事启堂头大和尚:赵某旧有
一条愿心,许剃一僧在上刹,度牒词簿都已有了,到今不曾剃得。今有这个表弟姓
鲁,是关西军汉出身,因见尘世艰辛,情愿弃俗出家。万望长老收录,慈悲慈悲,
看赵某薄面,披剃为僧。一应所用,弟子自当准备,烦望长老玉成,幸甚!”长老
见说,答道:“这个事缘是光辉老僧山门,容易容易,且请拜茶。”只见行童托出
茶来。
玉蕊金芽真绝品,僧家制造甚工夫。
免毫盏内香云白,蟹眼汤中细浪铺。
战退睡魔离枕席,增添清气入肌肤。
仙茶自合桃源种,不许移根傍帝都。

真长老与赵员外众人茶罢,收了盏托。真长老便唤首座、维那,商议剃度这人;
分付监寺、都寺,安排斋食。只见首座与众僧自去商议道:“这个人不似出家的模
样,一双眼却恁凶险。”众僧道:“知客,你去邀请客人坐地,我们与长老计较。”
知客出来,请赵员外、鲁达到客馆里坐地。首座众僧禀长老说道:“却才这个要出家
的人,形容丑恶,貌相凶顽,不可剃度他,恐久后累及山门。”长老道:“他是赵员
外檀越的兄弟,如何撇得他的面皮?你等众人且休疑心,待我看一看。”焚起一炷信
香,长老上禅椅,盘膝而坐,口诵咒语,入定去了。一炷香过,却好回来,对众僧
说道:“只顾剃度他。此人上应天星,心地刚直。虽然时下凶顽,命中驳杂,久后却
得清净,正果非凡,汝等皆不及他。可记吾言,勿得推阻。”首座道:“长老只是护
短,我等只得从他。不谏不是,谏他不从,便了。”

长老叫备斋食,请赵员外等方丈会斋。斋罢,监寺打了单帐。赵员外取出银两,
教人买办物料;一面在寺里做僧鞋、僧衣、僧帽、袈裟、拜具。一两日都已完备。
长老选了吉日良时,教鸣钟击鼓,就法堂内会集大众,整整齐齐,五六百僧人,尽
披袈裟,都到法座下合掌作礼,分作两班。赵员外取出银锭、表礼、信香,向法座
前礼拜了。表白宣疏已罢,行童引鲁达到法座下。维那教鲁达除了巾帻,把头发分
做九路绾了,扌周揲起来。净发人先把一周遭都剃了,却待剃髭须,鲁达道:“留了
这些儿还洒家也好。”众僧忍笑不住。真长老在法座上道:“大众听偈。”念道:
“寸草不留,六根清净,与汝剃除,免得争竞。”长老念罢偈言,喝一声:“咄!
尽皆剃去!”净发人只一刀,尽皆剃了。首座呈将度牒上法座前,请长老赐法名。
长老拿着空头度牒,而说偈曰:“灵光一点,价值千金,佛法广大,赐名智深。”

长老赐名已罢,把度牒转将下来,书记僧填写了度牒,付与鲁智深收受。长老
又赐法衣袈裟,教智深穿了。监寺引上法座前,长老用手与他摩顶受记道:“一要
皈依佛性,二要归奉正法,三要归敬师友,此是三归。五戒者:一不要杀生,二不
要偷盗,三不要邪淫,四不要贪酒,五不要妄语。”智深不晓得禅宗答应能否两字,
却便道:“洒家记得。”众僧都笑。受记已罢,赵员外请众僧到云堂里坐下,焚香
设斋供献。大小职事僧人,各有上贺礼物。都寺引鲁智深参拜了众师兄师弟,又引
去僧堂背后丛林里选佛场坐地。当夜无事。

次日赵员外要回,告辞长老,留连不住,早斋已罢,并众僧都送出山门。赵员
外合掌道:“长老在上,众师父在此,凡事慈悲。小弟智深,乃是愚卤直人,早晚
礼数不到,言语冒渎,误犯清规,万望觑赵某薄面,恕免恕免。”长老道:“员外
放心,老僧自慢慢地教他念经诵咒,办道参禅。”员外道:“日后自得报答。”人
丛里唤智深到松树下,低低分付道:“贤弟,你从今日难比往常,凡事自宜省戒,
切不可托大。倘有不然,难以相见,保重保重。早晚衣服,我自使人送来。”智深
道:“不索哥哥说,洒家都依了。”当时赵员外相辞长老,再别了众人上轿;引了
庄客,拕了一乘空轿,取了盒子,下山回家去了。当下长老自引了众僧回寺。

话说鲁智深回到丛林选佛场中禅床上,扑倒头便睡,上下肩两个禅和子推他起
来,说道:“使不得。既要出家,如何不学坐禅?”智深道:“洒家自睡,干你甚
事?”禅和子道:“善哉!”智深裸袖道:“团鱼洒家也吃,甚么‘鳝哉’?”禅
和子道:“却是苦也!”智深便道:“团鱼大腹,又肥甜了,好吃,那得‘苦也’。”
上下肩禅和子都不睬他,由他自睡了。次日要去对长老说知智深如此无礼,首座劝
道:“长老说道他后来正果非凡,我等皆不及他,只是护短,你们且没奈何,休与
他一般见识。”禅和子自去了。智深见没人说他,每到晚便放翻身体,横罗十字,
倒在禅床上睡,夜间鼻如雷响;要起来净手,大惊小怪,只在佛殿后撒尿撒屎,遍
地都是。侍者禀长老说:“智深好生无礼,全没些个出家人体面,丛林中如何安着
得此等之人?”长老喝道:“胡说!且看檀越之面,后来必改。”自此无人敢说。

鲁智深在五台山寺中,不觉搅了四五个月。时遇初冬天气,智深久静思动。当
日晴明得好,智深穿了皂布直裰,系了鸦青绦,换了僧鞋,大踏步走出山门来。信
步行到半山亭子上,坐在鹅项懒凳上,寻思道:“干鸟么!俺往常好酒好肉,每日
不离口,如今教洒家做了和尚,饿得干瘪了。赵员外这几日又不使人送些东西来与
洒家吃,口中淡出鸟来。这早晚怎地得些酒来吃也好。”正想酒哩,只见远远地一
个汉子,挑着一付担桶,唱上山来,上面盖着桶盖。那汉子手里拿着一个旋子,唱
着上来,唱道:“九里山前作战场,牧童拾得旧刀枪。顺风吹动乌江水,好似虞姬
别霸王。”

鲁智深观见那汉子挑担桶上来,坐在亭子上,看这汉子,也来亭子上,歇下担
桶。智深道:“兀那汉子,你那桶里,甚么东西?”那汉子道:“好酒!”智深道:
“多少钱一桶?”那汉子道:“和尚,你真个也是作耍?”智深道:“洒家和你耍
甚么?”那汉子道:“我这酒挑上去,只卖与寺内火工道人、直厅、轿夫、老郎们
做生活的吃。本寺长老已有法旨:但卖与和尚们吃了,我们都被长老责罚,追了本
钱,赶出屋去。我们现关着本寺的本钱,现住着本寺的屋宇,如何敢卖与你吃?”
智深道:“真个不卖?”那汉子道:“杀了我也不卖!”智深道:“洒家也不杀你,
只要问你买酒吃。”那汉子见不是头,挑了担桶便走。智深赶下亭子来,双手拿住
匾担,只一脚,交裆踢着,那汉子双手掩着,做一堆蹲在地下,半日起不得。智深
把那两桶酒都提在亭子上,地下拾起旋子,开了桶盖,只顾舀冷酒吃。无移时,两
大桶酒吃了一桶。智深道:“汉子,明日来寺里讨钱。”那汉子方才疼止,又怕寺
里长老得知,坏了衣饭,忍气吞声,那里敢讨钱?把酒分做两半桶挑了,拿了旋子,
飞也似下山去了。

只说鲁智深在亭子上坐了半日,酒却上来。下得亭子,松树根边又坐了半歇,
酒越涌上来。智深把皂直裰褪膊下来,把两只袖子缠在腰里,露出脊背上花绣来,
扇着两个膀子上山来。但见:

头重脚轻,眼红面赤;前合后仰,东倒西歪。踉踉跄跄上山来,似当风之鹤;
摆摆摇摇回寺去,如出水之蛇。指定天宫,叫骂天蓬元帅;踏开地府,要拿催命判
官。裸形赤体醉魔君,放火杀人花和尚。
鲁达看看来到山门下,两个门子远远地望见,拿着竹篦来到山门下,拦住鲁智
深便喝道:“你是佛家弟子,如何噇得烂醉了上山来?你须不瞎,也见库局里贴的晓
示:但凡和尚破戒吃酒,决打四十竹篦,赶出寺去,如门子纵容醉的僧人入寺,也
吃十下。你快下山去,饶你几下竹篦。”鲁智深一者初做和尚,二来旧性未改,睁起
双眼骂道:“直娘贼!你两个要打洒家,俺便和你厮打。”门子见势头不好,一个飞
也似入来报监寺,一个虚拖竹篦拦他。智深用手隔过,揸开五指,去那门子脸上只
一掌,打得踉踉跄跄;却待挣扎,智深再复一拳,打倒在山门下,只是叫苦。智深
道:“洒家饶你这厮。”踉踉跄跄攧入寺里来。
监寺听得门子报说,叫起老郎、火工、直厅、轿夫,三二十人,各执白木棍棒,
从西廊下抢出来,却好迎着智深。智深望见,大吼了一声,却似嘴边起个霹雳,大
踏步抢入来。众人初时不知他是军官出身,次后见他行得凶了,慌忙都退入藏殿里
去,便把亮槅关上。智深抢入阶来,一拳一脚,打开亮槅,三二十人都赶得没路,
夺条棒,从藏殿里打将出来。

监寺慌忙报知长老,长老听得,急引了三五个侍者直来廊下,喝道:“智深不
得无礼!”智深虽然酒醉,却认得是长老,撇了棒,向前来打个问讯,指着廊下对
长老道:“智深吃了两碗酒,又不曾撩拨他们,他众人又引人来打洒家。”长老道:
“你看我面,快去睡了,明日却说。”鲁智深道:“俺不看长老面,洒家直打死你
那几个秃驴!”长老叫侍者扶智深到禅床上,扑地便倒了,齁齁地睡了。众多职事
僧人围定长老告诉道:“向日徒弟们曾谏长老来,今日如何?本寺那里容得这个野
猫,乱了清规!”长老道:“虽是如今眼下有些罗唣,后来却成得正果,无奈何,
且看赵员外檀越之面,容恕他这一番。我自明日叫去埋怨他便了。”众僧冷笑道:
“好个没分晓的长老!”各自散去歇息。

次日,早斋罢,长老使侍者到僧堂里坐禅处唤智深时,尚兀自未起。待他起来,
穿了直裰,赤着脚,一道烟走出僧堂来。侍者吃了一惊,赶出外来寻时,却走在佛
殿后撒屎。侍者忍笑不住,等他净了手,说道:“长老请你说话。”智深跟着侍者
到方丈,长老道:“智深虽是个武夫出身,今来赵员外檀越剃度了你,我与你摩顶
受记,教你‘一不可杀生,二不可偷盗,三不可邪淫,四不可贪酒,五不可妄语。’
此五戒乃僧家常理。出家人第一不可贪酒,你如何夜来吃得大醉?打了门子,伤坏
了藏殿上朱红槅子,又把火工道人都打走了,口出喊声,如何这般所为?”智深跪
下道:“今番不敢了。”长老道:“既然出家,如何先破了酒戒,又乱了清规?我
不看你施主赵员外面,定赶你出寺!再后休犯!”智深起来合掌道:“不敢,不敢。”
长老留在方丈里,安排早饭与他吃,又用好言语劝他,取一领细布直裰,一双僧鞋,
与了智深,教回僧堂去了。昔有一名贤,走笔作一篇口号,单说那酒。端的做得好!
道是:
从来过恶皆归酒,我有一言为世剖。
地水火风合成人,面曲米水和醇酎。
酒在瓶中寂不波,人未酣时若无口。
谁说孩提即醉翁,未闻食糯颠如狗。
如何三杯放手倾,遂令四大不自有!
几人涓滴不能尝,几人一饮三百斗。
亦有醒眼是狂徒,亦有酕醄神不谬。
酒中贤圣得人传,人负邦家因酒覆。
解嘲破惑有常言,“酒不醉人人醉酒。”

【又有版本:
昔大唐有一名贤,姓张名旭,作一篇《醉歌行》,单说那酒。端的做得好!
道是:
金瓯潋滟倾欢伯,双手擎来两眸白。延颈长舒似玉虹,咽吞犹恨江湖窄。
昔年侍宴玉皇前,敌饮都无两三客。蟠桃烂熟堆珊瑚,琼液浓斟浮虎珀。
流霞畅饮数百杯,肌肤润泽腮微赤。天地闻知酒量洪,劝令受赐三千石。
飞仙劝我不记数,酩酊神清爽筋骨。东君命我赋新诗,笑指三山咏标格。
信笔挥成五百言,不觉尊前堕巾帻。宴罢昏迷不记归,乘惊误入云光宅。
仙童扶下紫云来,不辨东西与南北。一饮千锺百首诗,草书乱散纵横刘。
】

但凡饮酒,不可尽欢,常言:“酒能成事,酒能败事。”便是小胆的吃了,也胡乱
做了大胆,何况性高的人?

再说这鲁智深自从吃酒醉闹了这一场,一连三四个月,不敢出寺门去。忽一日,
天气暴暖,是二月间天气,离了僧房,信步踱出山门外立地,看着五台山,喝采一
回。猛听得山下叮叮当当的响声,顺风吹上山来。智深再回僧堂里取了些银两,揣
在怀里,一步步走下山去。出得那“五台福地”的牌楼来,看时,原来却是一个市
井,约有五七百人家。智深看那市镇上时,也有卖肉的,也有卖菜的,也有酒店面
店。智深寻思道:“干呆么!俺早知有这个去处,不夺他那桶酒吃,也自下来买些
吃。这几日熬得清水流,且过去看,有甚东西买些吃?”听得那响处,却是打铁的
在那里打铁,间壁一家门上,写着“父子客店”。智深走到铁匠铺门前看时,见三
个人打铁。智深便道:“兀那待诏,有好钢铁么?”那打铁的看见鲁智深腮边新剃,
暴长短须,戗戗地好渗濑人,先有五分怕他。那待诏住了手道:“师父请坐,要打
甚么生活?”智深道:“洒家要打条禅杖,一口戒刀,不知有上等好铁么?”待诏
道:“小人这里正有些好铁,不知师父要打多少重的禅杖、戒刀,但凭分付。”智
深道:“洒家只要打一条一百斤重的。”待诏笑道:“重了。师父,小人打怕不打
了,只恐师父如何使得动?便是关王刀,也只有八十一斤。”智深焦躁道:“俺便
不及关王,他也只是个人。”那待诏道:“小人据常说,只可打条四五十斤的,也
十分重了。”智深道:“便依你说,比关王刀,也打八十一斤的。”待诏道:“师
父,肥了不好看,又不中使。依着小人,好生打一条六十二斤的水磨禅杖与师父,
使不动时,休怪小人。戒刀已说了,不用分付,小人自用十分好铁打造在此。”智
深道:“两件家生,要几两银子?”待诏道:“不讨价,实要五两银子。”智深道:
“俺便依你五两银子;你若打得好时,再有赏你。”那待诏接了银两道:“小人便
打在此。”智深道:“俺有些碎银子在这里,和你买碗酒吃。”待诏道:“师父稳
便,小人赶趁些生活,不及相陪。”

智深离了铁匠人家,行不到三二十步,见一个酒望子,挑出在房檐上。智深掀
起帘子,入到里面坐下,敲着桌子叫道:“将酒来!”卖酒的主人家说道:“师父
少罪,小人住的房屋,也是寺里的,本钱也是寺里的。长老已有法旨:但是小人们
卖酒与寺里僧人吃了,便要追了小人们本钱,又赶出屋,因此,只得休怪。”智深
道:“胡乱卖些与洒家吃,俺须不说是你家便了。”店主人道:“胡乱不得,师父
别处去吃,休怪,休怪。”智深只得起身,便道:“洒家别处吃得,却来和你说话。”
出得店门,行了几步,又望见一家酒旗儿,直挑出在门前。智深一直走进去,坐下
叫道:“主人家,快把酒来卖与俺吃。”店主人道:“师父,你好不晓事,长老已
有法旨,你须也知,却来坏我们衣饭。”智深不肯动身,三回五次,那里肯卖。智
深情知不肯,起身又走。连走了三五家,都不肯卖。智深寻思一计,若不生个道理,
如何能够酒吃?远远地杏花深处,市梢尽头,一家挑出个草帚儿来。智深走到那里
看时,却是个傍村小酒店。但见:
傍村酒肆已多年,斜插桑麻古道边。
白板凳铺宾客坐,须篱笆用棘荆编。
破瓮榨成黄米酒,柴门挑出布青帘。
更有一般堪笑处,牛屎泥墙尽酒仙。
智深走入店里来,靠窗坐下,便叫道:“主人家,过往僧人买碗酒吃。”庄家
看了一看道:“和尚,你那里来?”智深道:“俺是行脚僧人,游方到此经过,要买
碗酒吃。”庄家道:“和尚,若是五台山寺里的师父,我却不敢卖与你吃。”智深
道:“洒家不是,你快将酒卖来。”庄家看见鲁智深这般模样,声音各别,便道:
“你要打多少酒?”智深道:“休问多少,大碗只顾筛来。”约莫也吃了十来碗,智
深问道:“有甚肉,把一盘来吃。”庄家道:“早来有些牛肉,都卖没了。”智深猛
闻得一阵肉香,走出空地上看时,只见墙边沙锅里煮着一只狗在那里。智深道:“你
家现有狗肉,如何不卖与俺吃?”庄家道:“我怕你是出家人,不吃狗肉,因此不
来问你。”智深道:“洒家的银子有在这里。”便将银子递与庄家道:“你且卖半
只与俺。”那庄家连忙取半只熟狗肉,捣些蒜泥,将来放在智深面前。智深大喜,
用手扯那狗肉,蘸着蒜泥吃,一连又吃了十来碗酒。吃得口滑,只顾要吃,那里肯
住。庄家倒都呆了,叫道:“和尚,只恁地罢!”智深睁起眼道:“洒家又不白吃
你的,管俺怎地?”庄家道:“再要多少?”智深道:“再打一桶来。”庄家只得
又舀一桶来。智深无移时,又吃了这桶酒,剩下一脚狗腿,把来揣在怀里,临出门
又道:“多的银子,明日又来吃。”吓得庄家目瞪口呆,罔知所措。看见他早望五
台山上去了。

智深走到半山亭子上,坐了一回,酒却涌上来,跳起身,口里道:“俺好些时
不曾拽拳使脚,觉道身体都困倦了,洒家且使几路看。”下得亭子,把两只袖子掿
在手里,上下左右,使了一回。使得力发,只一膀子,扇在亭子柱上,只听得刮剌
剌一声响亮,把亭子柱打折了,坍了亭子半边。门子听得半山里响,高处看时,只
见鲁智深一步一攧,抢上山来。两个门子叫道:“苦也!这畜生今番又醉得不小,
可便把山门关上,把拴拴了。”只在门缝里张时,见智深抢到山门下,见关了门,
把拳头擂鼓也似敲门,两个门子那里敢开。智深敲了一回,扭过身来,看了左边的
金刚,喝一声道:“你这个鸟大汉,不替俺敲门,却拿着拳头吓洒家,俺须不怕你。”
跳上台基,把栅剌子只一拔,却似葱般拔开了;拿起一根折木头,去那金刚腿上
便打,簌簌地泥和颜色都脱下来。门子张见道:“苦也!”只得报知长老。智深等
了一会,调转身来,看着右边金刚,喝一声道:“你这厮张开大口,也来笑洒家。”
便跳过右边台基上,把那金刚脚上打了两下,只听得一声震天价响,那尊金刚从台
基上倒撞下来,智深提着折木头大笑。

两个门子去报长老,长老道:“休要惹他,你们自去。”只见这首座、监寺、
都寺并一应职事僧人,都到方丈禀说:“这野猫今日醉得不好,把半山亭子,山门
下金刚,都打坏了,如何是好?”长老道:“自古天子尚且避醉汉,何况老僧乎?
若是打坏了金刚,请他的施主赵员外自来塑新的;倒了亭子,也要他修盖。这个且
由他。”众僧道:“金刚乃是山门之主,如何把来换过?”长老道:“休说坏了金
刚,便是打坏了殿上三世佛,也没奈何,只可回避他。你们见前日的行凶么?”众
僧出得方丈,都道:“好个囫囵竹的长老!门子,你且休开,只在里面听。”智深
在外面大叫道:“直娘的秃驴们,不放洒家入寺时,山门外讨把火来,烧了这个鸟
寺!”众僧听得叫,只得叫门子拽了大拴,由那畜生入来;若不开时,真个做出来。
门子只得捻脚捻手,把拴拽了,飞也似闪入房里躲了,众僧也各自回避。

只说那鲁智深双手把山门尽力一推,扑地攧将入来,吃了一交。扒将起来,把
头摸一摸,直奔僧堂来。到得选佛场中,禅和子正打坐间,看见智深揭起帘子,钻
将入来,都吃一惊,尽低了头。智深到得禅床边,喉咙里咯咯地响,看着地下便吐。
众僧都闻不得那臭,个个道:“善哉!”齐掩了口鼻。智深吐了一回,扒上禅床,
解下绦,把直裰带子都必必剥剥扯断了,脱下那脚狗腿来。智深道:“好好,正肚
饥哩!”扯来便吃。众僧看见,便把袖子遮了脸,上下肩两个禅和子远远地躲开。
智深见他躲开,便扯一块狗肉,看着上首的道:“你也到口。”上首的那和尚,把
两只袖子死掩了脸。智深道:“你不吃。”把肉望下首的禅和子嘴边塞将去,那和
尚躲不迭,却待下禅床,智深把他劈耳朵揪住,将肉便塞。对床四五个禅和子跳过
来劝时,智深撇了狗肉,提起拳头,去那光脑袋上必必剥剥只顾凿。满堂僧众大喊
起来,都去柜中取了衣钵要走。此乱唤做卷堂大散。首座那里禁约得住?

智深一味地打将出来,大半禅客都躲出廊下来。监寺、都寺不与长老说知,叫
起一班职事僧人,点起老郎、火工道人、直厅、轿夫,约有一二百人,都执杖叉棍
棒,尽使手巾盘头,一齐打入僧堂来。智深见了,大吼一声,别无器械,抢入僧堂
里,佛面前推翻供桌,两条桌脚,从堂里打将出来。但见:

心头火起,口角雷鸣。奋八九尺猛兽身躯,吐三千丈凌云志气。按不住杀人怪
胆,圆睁起卷海双睛。直截横冲,似中箭投崖虎豹;前奔后涌,如着枪跳涧豺狼。
直饶揭帝也难当,便是金刚须拱手。

当时鲁智深抡两条桌脚,打将出来,众多僧行见他来得凶了,都拖了棒,退到
廊下。智深两条桌脚,着地卷将来,众僧早两下合拢来。智深大怒,指东打西,指
南打北,只饶了两头的。当时智深直打到法堂下,只见长老喝道:“智深不得无礼,
众僧也休动手。”两边众人,被打伤了数十个,见长老来,各自退去。智深见众人
退散,撇了桌脚,叫道:“长老,与洒家做主。”此时酒已七八分醒了。长老道:
“智深,你连累杀老僧。前番醉了一次,搅扰了一场,我教你兄赵员外得知,他写
书来,与众僧陪话。今番你又如此大醉无礼,乱了清规,打坍了亭子,又打坏了金
刚。这个且由他。你搅得众僧卷堂而走,这个罪业非小,我这里五台山文殊菩萨道
场,千百年清净香火去处,如何容得你这个秽污?你且随我来方丈里过几日,我安
排你一个去处。”智深随长老到方丈去。长老一面叫职事僧人留住众禅客,再回僧
堂,自去坐禅;打伤了的和尚,自去将息。长老领智深到方丈,歇了一夜。

次日,真长老与首座商议:“收拾了些银两赍发他,教他别处去,可先说与赵
员外知道。”长老随即修书一封,使两个直厅道人,径到赵员外庄上,说知就里,
立等回报。赵员外看了来书,好生不然。回书来拜复长老说道:“坏了的金刚、亭
子,赵某随即备价来修。智深任从长老发遣。”长老得了回书,便叫侍者取领皂布
直裰,一双僧鞋,十两白银,房中唤过智深。长老道:“智深,你前番一次大醉,
闹了僧堂,便是误犯。今次又大醉,打坏了金刚,坍了亭子,卷堂闹了选佛场,你
这罪业非轻;又把众禅客打伤了。我这里出家,是个清净去处,你这等做,甚是不
好。看你赵檀越面皮,与你这封书,投一个去处安身。我这里决然安你不得了。我
夜来看了,赠汝四句偈言,终身受用。”智深道:“师父教弟子那里去安身立命?
愿听俺师四句偈言。”真长老指着鲁智深,说出这几句言语,去这个去处。有分教:
这人笑挥禅杖,战天下英雄好汉;怒掣戒刀,砍世上逆子谗臣。直教:名驰塞北三
千里,果证江南第一州。

毕竟真长老与智深说出甚言语来,且听下回分解。