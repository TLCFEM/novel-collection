\chapter{赤发鬼醉卧灵官殿晁天王认义东溪村~}

话说当时雷横来到灵官殿上,见了这条大汉,睡在供桌上,众土兵向前,把条
索子绑了,捉离灵官殿来,天色却早,是五更时分。雷横道:“我们且押这厮去晁
保正庄上讨些点心吃了,却解去县里取问。”一行众人却都奔这保正庄上来。

原来那东溪村保正姓晁,名盖,祖是本县本乡富户,平生仗义疏财,专爱结识
天下好汉,但有人来投奔他的,不论好歹,便留在庄上住;若要去时,又将银两赍
助他起身。最爱刺枪使棒,亦自身强力壮,不娶妻室,终日只是打熬筋骨。郓城县
管下东门外有两个村坊,一个东溪村,一个西溪村,只隔着一条大溪。当初这西溪
村常常有鬼,白日迷人下水,在溪里,无可奈何。忽一日,有个僧人经过,村中人
备细说知此事,僧人指个去处,教用青石凿个宝塔,放于所在,镇住溪边。其时西
溪村的鬼,都赶过东溪村来。那时晁盖得知了,大怒,从这里走将过去,把青石宝
塔独自夺了过来东溪村放下,因此人皆称他做托塔天王。晁盖独霸在那村坊,江湖
都闻他名字。

却早雷横并土兵押着那汉来到庄前敲门,庄里庄客闻知,报与保正。此时晁盖
未起,听得报是雷都头到来,慌忙叫开门。庄客开得庄门,众土兵先把那汉子吊在
门房里。雷横自引了十数个为头的人到草堂上坐下。晁盖起来接待,动问道:“都
头有甚公干到这里?”雷横答道:“奉知县相公钧旨:着我与朱仝两个引了部下土
兵,分投下乡村各处巡捕贼盗。因走得力乏,欲得少歇,径投贵庄暂息,有惊保正
安寝。”晁盖道:“这个何妨!”一面叫庄客安排酒食管待,先把汤来吃。晁盖动
问道:“敝村曾拿得个把小贼么?”雷横道:“却才前面灵官殿上有个大汉睡着在
那里,我看那厮不是良善君子,以定是醉了,就便睡着。我们把索子缚绑了,本待
便解去县里见官,一者忒早些,二者也要教保正知道,恐日后父母官问时,保正也
好答应。现今吊在贵庄门房里。”晁盖听了,记在心,称谢道:“多亏都头见报。”
少刻庄客捧出盘馔酒食,晁盖喝道:“此间不好说话,不如去后厅轩下少坐。”便
叫庄客里面点起灯烛,请都头到里面酌杯。晁盖坐了主位,雷横坐了客席。两个坐
定,庄客铺下果品、按酒、菜蔬、盘馔。庄客一面筛酒,晁盖又叫买酒与土兵众人
吃,庄客请众人都引去廊下客位里管待,大盘酒肉只管叫众人吃。晁盖一头相待雷
横吃酒,一面自肚里寻思:“村中有甚小贼吃他拿了?我且自去看是谁。”相陪吃
了五七杯酒,便叫家里一个主管出来:“陪奉都头坐一坐,我去净了手便来。”

那主管陪侍着雷横吃酒,晁盖却去里面拿了个灯笼,径来门楼下看时,土兵都
去吃酒,没一个在外面。晁盖便问看门的庄客:“都头拿的贼吊在那里?”庄客道:
“在门房里关着。”晁盖去推开门,打一看时,只见高高吊起那汉子在里面,露出
一身黑肉,下面抓扎起两条黑毛腿,赤着一双脚。晁盖把灯照那人脸时,紫黑
阔脸,鬓边一搭朱砂记,上面生一片黑黄毛。晁盖便问道:“汉子,你是那里人?
我村中不曾见有你。”那汉道:“小人是远乡客人,来这里投奔一个人,却把我来
拿做贼,我须有分辨处。”晁盖道:“你来我这村中投奔谁?”那汉道:“我来这
村中投奔一个好汉。”晁盖道:“这好汉叫做甚么?”那汉道:“他唤做晁保正。”
晁盖道:“你却寻他有甚勾当?”那汉道:“他是天下闻名的义士好汉。如今我有
一套富贵要与他说知,因此而来。”晁盖道:“你且住,只我便是晁保正,却要我
救你,你只认我做娘舅之亲。少刻,我送雷都头那人出来时,你便叫我做阿舅,我
便认你做外甥,只说四五岁离了这里,今番来寻阿舅,因此不认得。”那汉道:“若
得如此救护,深感厚恩,义士提携则个!”正是:
黑甜一枕古祠中,被获高悬草舍东。
百万赃私天不佑,解围晁盖有奇功。

当时晁盖提了灯笼,自出房来,仍旧把门拽上,急入后厅来见雷横,说道:“甚
是慢客。”雷横道:“多多相扰,理甚不当。”两个又吃了数杯酒,只见窗子外射
入天光来,雷横道:“东方动了,小人告退,好去县中画卯。”晁盖道:“都头官
身,不敢久留。若再到敝村公干,千万来走一遭。”雷横道:“却得再来拜望,不
须保正分付。请保正免送。”晁盖道:“却罢,也送到庄门口。”

两个同走出来,那伙土兵众人都得了酒食,吃得饱了,各自拿了枪棒,便去门
房里解了那汉,背剪缚着带出门外。晁盖见了,说道:“好条大汉!”雷横道:“这
厮便是灵官庙里捉的贼,……”说犹未了,只见那汉叫一声:“阿舅,救我则个!”
晁盖假意看他一看,喝问道:“兀的这厮不是王小三么?”那汉道:“我便是,阿
舅救我。”众人吃了一惊。雷横便问晁盖道:“这人是谁?如何却认得保正?”晁
盖道:“原来是我外甥王小三。这厮如何在庙里歇?乃是家姐的孩儿,从小在这里
过活,四五岁时随家姐夫和家姐上南京去住,一去了十数年。这厮十四五岁又来走
了一遭,跟个本京客人来这里贩卖,向后再不曾见面。多听得人说这厮不成器,如
何却在这里?小可本也认他不得,为他鬓边有这一塔朱砂记,因此影影认得。”晁
盖喝道:“小三,你如何不径来见我?却去村中做贼!”那汉叫道:“阿舅,我不
曾做贼。”晁盖喝道:“你既不做贼,如何拿你在这里?”夺过土兵手里棍棒,劈
头劈脸便打。雷横并众人劝道:“且不要打,听他说。”那汉道:“阿舅息怒,且
听我说:自从十四五岁时来走了这遭,如今不是十年了?昨夜路上多吃了一杯酒,
不敢来见阿舅,权去庙里睡得醒了,却来寻阿舅;不想被他们不问事由,将我拿了,
却不曾做贼。”晁盖拿起棍来又要打,口里骂道:“畜生!你却不径来见我,且在
路上贪这口黄汤,我家中没有与你吃,辱没杀人!”雷横劝道:“保正息怒,你
令甥本不曾做贼。我们见他偌大一条大汉在庙里睡得跷蹊,亦且面生,又不认得,
因此设疑,捉了他来这里。若早知是保正的令甥,定不拿他。”唤土兵快解了绑缚
的索子,放还保正,众土兵登时放了那汉。雷横道:“保正休怪,早知是令甥,不
致如此,甚是得罪,小人们回去。”晁盖道:“都头且住,请入小庄,再有话说。”

雷横放了那汉,一齐再入草堂里来。晁盖取出十两花银送与雷横,说道:“都
头休嫌轻微,望赐笑留。”雷横道:“不当如此。”晁盖道:“若是不肯收受时,
便是怪小人。”雷横道:“既是保正厚意,权且收受,改日却得报答。”晁盖叫那
汉拜谢了雷横,晁盖又取些银两赏了众土兵,再送出庄门外。雷横相别了,引着土
兵自去。

晁盖却同那汉到后轩下,取几件衣裳与他换了,取顶头巾与他戴了,便问那汉
姓甚名谁,何处人氏。那汉道:“小人姓刘,名唐,祖贯东潞州人氏,因这鬓边有
这塔朱砂记,人都唤小人做赤发鬼,特地送一套富贵来与保正哥哥。昨夜晚了,因
醉倒庙里,不想被这厮们捉住,绑缚了来,正是‘有缘千里来相会,无缘对面不相
逢’。今日幸得在此,哥哥坐定,受刘唐四拜。”拜罢,晁盖道:“你且说送一套
富贵与我,现在何处?”

刘唐道:“小人自幼飘荡江湖,多走途路,专好结识好汉,往往多闻哥哥大名,
不期有缘得遇。曾见山东、河北做私商的,多曾来投奔哥哥,因此刘唐敢说这话。
这里别无外人,方可倾心吐胆对哥哥说。”晁盖道:“这里都是我心腹人,但说不
妨。”刘唐道:“小弟打听得北京大名府梁中书收买十万贯金珠、宝贝、玩器等物,
送上东京,与他丈人蔡太师庆生辰。去年也曾送十万贯金珠宝贝,来到半路里,不
知被谁人打劫了,至今也无捉处;今年又收买十万贯金珠宝贝,早晚安排起程,要
赶这六月十五日生辰。小弟想此一套是不义之财,取之何碍!便可商议个道理去半
路上取了,天理知之,也不为罪。闻知哥哥大名,是个真男子,武艺过人。小弟不
才,颇也学得本事,休道三五个汉子,便是一二千军马队中,拿条枪,也不惧他。
倘蒙哥哥不弃时,献此一套富贵,不知哥哥心内如何?”晁盖道:“壮哉!且再计
较。你既来这里,想你吃了些艰辛,且去客房里将息少歇,待我从长商议,来日说
话。”晁盖叫庄客引刘唐廊下客房里歇息,庄客引到房中,也自去干事了。

且说刘唐在房里寻思道:“我着甚来由,苦恼这遭!多亏晁盖完成,解脱了这
件事。只叵耐雷横那厮平白骗了晁保正十两银子,又吊我一夜。想那厮去未远,我
不如拿了条棒赶上去,齐打翻了那厮们,却夺回那银子,送还晁盖,也出一口恶气。
此计大妙。”刘唐便出房门,去枪架上拿了一条朴刀,便出庄门,大踏步投南赶来。
此时天色已明,但见:

北斗初横,东方欲白。天涯曙色才分,海角残星渐落。金鸡三唱,唤佳人傅粉
施朱;宝马频嘶,催行客争名竞利。几缕丹霞横碧汉,一轮红日上扶桑。
这赤发鬼刘唐挺着朴刀,赶了五六里路,却早望见雷横引着土兵,慢慢地行将去。
刘唐赶上来,大喝一声:“兀那都头不要走!”

雷横吃了一惊,回过头来,见是刘唐拈着朴刀赶来。雷横慌忙去土兵手里夺条
朴刀拿着,喝道:“你那厮赶将来做甚么?”刘唐道:“你晓事的,留下那十两银
子还了我,我便饶了你!”雷横道:“是你阿舅送我的,干你甚事?我若不看你阿
舅面上,直结果了你这厮性命,地问我取银子?”刘唐道:“我须不是贼,你却
把我吊了一夜,又骗我阿舅十两银子。是会的将来还我,佛眼相看;你若不还我,
叫你目前流血!”雷横大怒,指着刘唐大骂道:“辱门败户的谎贼,怎敢无礼!”
刘唐道:“你那作害百姓的腌泼才,怎敢骂我!”雷横又骂道:“贼头贼脸贼骨
头,必然要连累晁盖!你这等贼心贼肝,我行须使不得!”刘唐大怒道:“我来和
你见个输赢。”拈着朴刀,直奔雷横。雷横见刘唐赶上来,呵呵大笑,挺手中朴刀
来迎。两个就大路上厮并,但见:

一来一往,似凤翻身;一撞一冲,如鹰展翅。一个照搠,尽依良法;一个遮拦,
自有悟头。这个丁字脚,抢将入来;那个四换头,奔将进去。两句道:虽然不上凌
烟阁,只此堪描入画图。

当时雷横和刘唐就路上斗了五十余合,不分胜败。众土兵见雷横赢刘唐不得,
却待都要一齐上并他。只见侧首篱门开处,一个人掣两条铜链,叫道:“你们两个
好汉且不要斗,我看了多时,权且歇一歇,我有话说。”便把铜链就中一隔,两个
都收住了朴刀,跳出圈子外来,立住了脚。看那人时,似秀才打扮,戴一顶桶子样
抹眉梁头巾,穿一领皂沿边麻布宽衫,腰系一条茶褐銮带,下面丝鞋净袜,生得眉
清目秀,面白须长。这人乃是智多星吴用,表字学究,道号加亮先生,祖贯本乡人
氏。曾有一首《临江仙》赞吴用的好处:

万卷经书曾读过,平生机巧心灵,六韬三略究来精。胸中藏战将,腹内隐雄兵。

谋略敢欺诸葛亮,陈平岂敌才能。略施小计鬼神惊。字称吴学究,人号智多星。

当时吴用手提铜链,指着刘唐叫道:“那汉且住,你因甚和都头争执?”刘唐
光着眼看吴用道:“不干你秀才事!”雷横便道:“教授不知,这厮夜来赤条条地
睡在灵官庙里,被我们拿了这厮,带到晁保正庄上,原来却是保正的外甥,看他母
舅面上放了他。晁天王请我们吃了酒,送些礼物与我,这厮瞒了他阿舅,直赶到这
里问我取,你道这厮大胆么?”吴用寻思道:“晁盖我都是自幼结交,但有些事,
便和我相议计较。他的亲眷相识,我都知道,不曾见有这个外甥。亦且年甲也不相
登,必有些跷蹊。我且劝开了这场闹,却再问他。”吴用便道:“大汉休执迷,你
的母舅与我至交,又和这都头亦过得好,他便送些人情与这都头,你却来讨了,也
须坏了你母舅面皮。且看小生面,我自与你母舅说。”刘唐道:“秀才,你不省得。
这个不是我阿舅甘心与他,他诈取了我阿舅的银两;若是不还我,誓不回去。”雷
横道:“只除是保正自来取,便还他,却不还你。”刘唐道:“你屈冤人做贼,诈
了银子,怎地不还?”雷横道:“不是你的银子,不还,不还!”刘唐道:“你不
还,只除问得我手里朴刀肯便罢。”吴用又劝:“你两个斗了半日,又没输赢,只
管斗到几时是了?”刘唐道:“他不还我银子,直和他拚个你死我活便罢。”雷横
大怒道:“我若怕你,添个土兵来并你,也不算好汉,我自好歹搠翻你便罢!”刘
唐大怒,拍着胸前叫道:“不怕!不怕!”便赶上来。这边雷横便指手划脚也赶拢
来。两个又要厮并,这吴用横身在里面劝,那里劝得住。刘唐拈着朴刀,只待钻将
过来。雷横口里千贼万贼骂,挺起朴刀,正待要斗,只见众土兵指道:“保正来了。”

刘唐回身看时,只见晁盖披着衣裳,前襟摊开,从大路上赶来,大喝道:“畜
生不得无礼!”那吴用大笑道:“须是保正自来,方才劝得这场闹。”晁盖赶得气
喘,问道:“你怎的赶来这里斗朴刀?”雷横道:“你的令甥拿着朴刀赶来问我取
银子,小人道:‘不还你,我自送还保正,非干你事。’他和小人斗了五十合,教
授解劝在此。”晁盖道:“这畜生,小人并不知道,都头看小人之面请回,自当改
日登门陪话。”雷横道:“小人也知那厮胡为,不与他一般见识,又劳保正远出。”
作别自去,不在话下。

且说吴用对晁盖说道:“不是保正自来,几乎做出一场大事。这个令甥端的非
凡,是好武艺。小生在篱笆里看了。这个有名惯使朴刀的雷都头,也敌不过,只办
得架隔遮拦。若再斗几合,雷横必然有失性命,因此小人慌忙出来间隔了。这个令
甥从何而来?往常时庄上不曾见有。”晁盖道:“却待正要求请先生到敝庄商议句
话,正欲使人来,只是不见了他,枪架上朴刀又没寻处,只见牧童报说,一个大汉
拿条朴刀望南一直赶去,我慌忙随后追得来,早是得教授谏劝住了。请尊步同到敝
庄,有句话计较计较。”那吴用还至书斋,挂了铜链在书房里,分付主人家道:“学
生来时,说道先生今日有干,权放一日假。”有诗为证:
文才不下武才高,铜链犹能劝朴刀。
只爱雄谈偕义士,岂甘枯坐伴儿曹。
放他众鸟笼中出,许尔群蛙野外跳。
自是先生多好动,学生欢喜主人焦。
吴用拽上书斋门,将锁锁了,同晁盖、刘唐到晁家庄上,晁盖径邀入后堂深处,分
宾而坐。

吴用问道:“保正,此人是谁?”晁盖道:“江湖上好汉,此人姓刘,名唐,
是东潞州人氏。因此有一套富贵,特来投奔我。夜来他醉卧在灵官庙里,却被雷横
捉了,拿到我庄上,我因认他做外甥,方得脱身。他说:‘有北京大名府梁中书收
买十万贯金珠宝贝,送上东京,与他丈人蔡太师庆生辰,早晚从这里经过,此等不
义之财,取之何碍!’他来的意,正应我一梦。我昨夜梦见北斗七星,直坠在我屋
脊上,斗柄上另有一颗小星,化道白光去了。我想星照本家,安得不利?今早正要
求请教授商议,此一件事若何?”吴用笑道:“小生见刘兄赶得来跷蹊,也猜个七
八分了。此一事却好,只是一件,人多做不得,人少又做不得。宅上空有许多庄客,
一个也用不得。如今只有保正、刘兄、小生三人,这件事如何团弄?便是保正与刘
兄十分了得,也担负不下。这段事须得七八个好汉方可,多也无用。”晁盖道:“莫
非要应梦之星数?”吴用便道:“兄长这一梦也非同小可,莫非北地上再有扶助的
人来?”吴用寻思了半晌,眉头一纵,计上心来,说道:“有了!有了!”晁盖道:
“先生既有心腹好汉,可以便去请来,成就这件事。”吴用不慌不忙,叠两个指头,
说出这句话来,有分教:东溪庄上,聚义汉翻作强人;石碣村中,打鱼船权为战舰。
正是:指挥说地谈天口,来诱翻江搅海人。

毕竟智多星吴用说出甚么人来,且听下回分解。