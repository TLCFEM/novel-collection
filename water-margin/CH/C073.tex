\chapter{黑旋风乔捉鬼~梁山泊双献头}

话说当下李逵从客店里抢将出来,手掿双斧,要奔城边劈门,被燕青抱住腰胯,
只一交,攧个脚捎天。燕青拖将起来,望小路便走,李逵只得随他。为何李逵怕燕
青?原来燕青小厮扑天下第一,因此宋公明着令燕青相守李逵。李逵若不随他,燕
青小厮扑,手到一交。李逵多曾着他手脚,以此怕他,只得随顺。燕青和李逵不敢
从大路上走,恐有军马追来,难以抵敌,只得大宽转奔陈留县路来。李逵再穿上衣
裳,把大斧藏在衣襟底下,又因没了头巾,却把焦黄发分开,绾做两个丫髻。行到
天明,燕青身边有钱,村店中买些酒肉吃了,拽开脚步趱行。次日天晓,东京城中
好场热闹,高太尉引军出城,追赶不上自回。李师师只推不知,杨太尉也自归家将
息,抄点城中被伤人数,计有四五百人,推倒跌损者,不计其数。高太尉会同枢密
院童贯,都到太师府商议,启奏早早调兵剿捕。

且说李逵和燕青两个,在路行到一个去处,地名唤做四柳村。不觉天晚,两个
便投一个大庄院来,敲开门,直进到草厅上。庄主狄太公出来迎接,看见李逵绾着
两个丫髻,却不见穿道袍,面貌生得又丑,正不知是甚么人。太公随口问燕青道:
“这位是那里来的师父?”燕青笑道:“这师父是个跷蹊人,你们都不省得他。胡
乱趁些晚饭吃,借宿一夜,明日早行。”李逵只不做声。太公听得这话,倒地便拜
李逵,说道:“师父,可救弟子则个。”李逵道:“你要我救你甚事,实对我说。”
那太公道:“我家一百余口,夫妻两个,嫡亲止有一个女儿,年二十余岁,半年之
前,着了一个邪祟,只在房中,茶饭并不出来讨吃。若还有人去叫他,砖石乱打出
来,家中人都被他打伤了,累累请将法官来,也捉他不得。”李逵道:“太公,我
是蓟州罗真人的徒弟,会得腾云驾雾,专能捉鬼,你若舍得东西,我与你今夜捉鬼。
如今先要一猪一羊,祭祀神将。”太公道:“猪羊我家尽有,酒自不必得说。”李
逵道:“你拣得膘肥的宰了,烂煮将来,好酒更要几瓶,便可安排,今夜三更与你
捉鬼。”太公道:“师父如要书符纸札,老汉家中也有。”李逵道:“我的法只是
一样,都没什么鸟符,身到房里,便揪出鬼来。”燕青忍笑不住。

老儿只道他是好话,安排了半夜,猪羊都煮得熟了,摆在厅上。李逵叫讨十个
大碗,滚热酒十瓶,做一巡筛,明晃晃点着两枝蜡烛,焰腾腾烧着一炉好香。李逵
掇条凳子,坐在当中,并不念甚言语。腰间拔出大斧,砍开猪羊,大块价扯将下来
吃。又叫燕青道:“小乙哥,你也来吃些。”燕青冷笑,那里肯来吃。李逵吃得饱
了,饮过五六碗好酒,看得太公呆了。李逵便叫众庄客:“你们都来散福。”拈指
间散了残肉。李逵道:“快舀桶汤来,与我们洗手洗脚。”无移时,洗了手脚,问
太公讨茶吃了。又问燕青道:“你曾吃饭也不曾?”燕青道:“吃得饱了。”李逵
对太公道:“酒又醉,肉又饱,明日要走路程,老爷们去睡。”太公道:“却是苦
也!这鬼几时捉得?”李逵道:“你真个要我捉鬼,着人引我到你女儿房里去。”
太公道:“便是神道如今在房中,砖石乱打出来,谁人敢去?”

李逵拔两把板斧在手,叫人将火把远远照着。李逵大踏步直抢到房边,只见房
内隐隐的有灯。李逵把眼看时,见一个后生搂着一个妇人在那里说话。李逵一脚踢
开了房门,斧到处,只见砍得火光爆散,霹雳交加。定睛打一看时,原来把灯盏砍
翻了。那后生却待要走,被李逵大喝一声,斧起处,早把后生砍翻。这婆娘便钻入
床底下躲了。李逵把那汉子先一斧砍下头来,提在床上,把斧敲着床边喝道:“婆
娘,你快出来。若不钻出来时,和床都剁的粉碎。”婆娘连声叫道:“你饶我性命,
我出来。”却才钻出头来,被李逵揪住头发,直拖到死尸边,问道:“我杀的那厮
是谁?”婆娘道:“是我奸夫王小二。”李逵又问道:“砖头饭食,那里得来?”
婆娘道:“这是我把金银头面与他,三二更从墙上运将入来。”李逵道:“这等腌
婆娘,要你何用!”揪到床边,一斧砍下头来,把两个人头拴做一处,再提婆娘
尸首和汉子身尸相并,李逵道:“吃得饱,正没消食处。”就解下上半截衣裳,拿
起双斧,看着两个死尸,一上一下,恰似发擂的乱剁了一阵。

李逵笑道:“眼见这两个不得活了。”插起大斧,提着人头,大叫出厅前来:
“两个鬼我都捉了。”撇下人头,满庄里人都吃一惊,都来看时,认得这个是太公
的女儿;那个人头,无人认得。数内一个庄客相了一回,认出道:“有些像东村头
会粘雀儿的王小二。”李逵道:“这个庄客倒眼乖!”太公道:“师父怎生得知?”
李逵道:“你女儿躲在床底下,被我揪出来问时,说道:‘他是奸夫王小二,吃的
饮食,都是他运来。’问了备细,方才下手。”太公哭道:“师父,留得我女儿也
罢。”李逵骂道:“打脊老牛,女儿偷了汉子,兀自要留他!你恁地哭时,倒要赖
我不谢。我明日却和你说话。”燕青寻了个房,和李逵自去歇息。太公却引人点着
灯烛,入房里去看时,照见两个没头尸首,剁做十来段,丢在地下。太公、太婆烦
恼啼哭,便叫人扛出后面,去烧化了。李逵睡到天明,跳将起来,对太公道:“昨
夜与你捉了鬼,你如何不谢?”太公只得收拾酒食相待,李逵、燕青吃了便行。狄
太公自理家事,不在话下。

且说李逵和燕青离了四柳村,依前上路,此时草枯地阔,木落山空,于路无话。
两个因大宽转梁山泊北,到寨尚有七八十里,巴不到山,离荆门镇不远。当日天晚,
两个奔到一个大庄院敲门,燕青道:“俺们寻客店中歇去。”李逵道:“这大户人
家,却不强似客店多少!”说犹未了,庄客出来,对说道:“我主太公正烦恼哩!
你两个别处去歇。”李逵直走入去,燕青拖扯不住,直到草厅上。李逵口里叫道:
“过往客人借宿一宵,打甚鸟紧?便道太公烦恼,我正要和烦恼的说话。”里面太
公张时,看见李逵生得凶恶,暗地教人出来接纳,请去厅外侧首,有间耳房,叫他
两个安歇,造些饭食,与他两个吃,着他里面去睡。多样时,搬出饭来,两个吃了,
就便歇息。

李逵当夜没些酒,在土炕子上翻来覆去睡不着,只听得太公、太婆在里面哽哽
咽咽的哭,李逵心焦,那双眼怎地得合。巴到天明,跳将起来,便向厅前问道:“你
家甚么人,哭这一夜,搅得老爷睡不着。”太公听了,只得出来答道:“我家有个
女儿,年方一十八岁,被人强夺了去,以此烦恼。”李逵道:“又来作怪!夺你女
儿的是谁?”太公道:“我与你说他姓名,惊得你屁滚尿流!他是梁山泊头领宋江,
有一百单八个好汉,不算小军。”李逵道:“我且问你:他是几个来?”太公道:
“两日前,他和一个小后生各骑着一匹马来。”李逵便叫燕青:“小乙哥,你来听
这老儿说的话,俺哥哥原来口是心非,不是好人了也。”燕青道:“大哥莫要造次,
定没这事!”李逵道:“他在东京兀自去李师师家去,到这里怕不做出来!”李逵
便对太公说道:“你庄里有饭,讨些我们吃。我实对你说,则我便是梁山泊黑旋风
李逵,这个便是浪子燕青。既是宋江夺了你的女儿,我去讨来还你。”太公拜谢了。

李逵、燕青径望梁山泊来,直到忠义堂上。宋江见了李逵、燕青回来,便问道:
“兄弟,你两个那里来?错了许多路,如今方到。”李逵那里答应,睁圆怪眼,拔
出大斧,先砍倒了杏黄旗,把“替天行道”四个字扯做粉碎,众人都吃一惊。宋江
喝道:“黑厮又做甚么?”李逵拿了双斧,抢上堂来,径奔宋江。诗曰:
梁山泊里无奸佞,忠义堂前有诤臣。
留得李逵双斧在,世间直气尚能伸。
当有关胜、林冲、秦明、呼延灼、董平五虎将,慌忙拦住,夺了大斧,揪下堂来。
宋江大怒,喝道:“这厮又来作怪!你且说我的过失。”李逵气做一团,那里说得
出。燕青向前道:“哥哥听禀一路上备细:他在东京城外客店里跳将出来,拿着双
斧,要去劈门,被我一交攧翻,拖将起来。说与他:‘哥哥已自去了,独自一个风
甚么?’恰才信小弟说,不敢从大路走。他又没了头巾,把头发绾做两个丫髻。正
来到四柳村狄太公庄上,他去做法官捉鬼,正拿了他女儿并奸夫两个,都剁做肉酱。
后来却从大路西边上山,他定要大宽转,将近荆门镇,当日天晚了,便去刘太公庄
上投宿。只听得太公两口儿一夜啼哭,他睡不着,巴得天明,起去问他。刘太公说
道:‘两日前梁山泊宋江和一个年纪小的后生,骑着两匹马到庄上来,老儿听得说
是替天行道的人,因此叫这十八岁的女儿出来把酒,吃到半夜,两个把他女儿夺了
去。’李逵大哥听了这话,便道是实,我再三解说道:‘俺哥哥不是这般的人,多
有依草附木,假名托姓的在外头胡做。’李大哥道:‘我见他在东京时,兀自恋着
唱的李师师不肯放,不是他是谁?’因此来发作。”

宋江听罢,便道:“这般屈事,怎地得知?如何不说?”李逵道:“我闲常把
你做好汉,你原来却是畜生!你做得这等好事!”宋江喝道:“你且听我说!我和三
二千军马回来,两匹马落路时,须瞒不得众人。若还抢得一个妇人,必然只在寨里!
你却去我房里搜看。”李逵道:“哥哥,你说甚么鸟闲话!山寨里都是你手下的人,
护你的多,那里不藏过了?我当初敬你是个不贪色欲的好汉,你原来是酒色之徒:
杀了阎婆惜,便是小样;去东京养李师师,便是大样。你不要赖,早早把女儿送还
老刘,倒有个商量。你若不把女儿还他时,我早做早杀了你,晚做晚杀了你。”宋
江道:“你且不要闹嚷,那刘太公不死,庄客都在,俺们同去面对。若还对翻了,
就那里舒着脖子,受你板斧;如若对不翻,你这厮没上下,当得何罪?”李逵道:
“我若还拿你不着,便输这颗头与你!”宋江道:“最好,你众兄弟都是证见。”
便叫铁面孔目裴宣写了赌赛军令状二纸,两个各书了字,宋江的把与李逵收了,李
逵的把与宋江收了。李逵又道:“这后生不是别人,只是柴进。”柴进道:“我便
同去。”李逵道:“不怕你不来。若到那里对翻了之时,不怕你柴大官人,是米大
官人,也吃我几斧。”柴进道:“这个不妨,你先去那里等。我们前去时,又怕有
跷蹊。”李逵道:“正是。”便唤了燕青:“俺两个依前先去,他若不来,便是心
虚,回来罢休不得。”正是:
至上无过任评论,其次纳谏以为恩。
最下自差偏自是,令人敢怒不敢言。

燕青与李逵再到刘太公庄上,太公接见,问道:“好汉,所事如何?”李逵道:
“如今我那宋江,他自来教你认他,你和太婆并庄客都仔细认他。若还是时,只管
实说,不要怕他,我自替你做主。”只见庄客报道:“有十数骑马来到庄上了。”
李逵道:“正是了。”侧边屯住了人马,只教宋江、柴进入来。宋江、柴进径到草
厅上坐下。

李逵提着板斧立在侧边,只等老儿叫声“是”,李逵便要下手。那刘太公近前
来拜了宋江。李逵问老儿道:“这个是夺你女儿的不是?”那老儿睁开羸眼,打
起老精神,定睛看了道:“不是。”宋江对李逵道:“你却如何?”李逵道:“你
两个先着眼瞅他,这老儿惧怕你,便不敢说是。”宋江道:“你叫满庄人都来认我。”
李逵随即叫到众庄客人等认时,齐声叫道:“不是。”宋江道:“刘太公,我便是
梁山泊宋江,这位兄弟,便是柴进。你的女儿,都是吃假名托姓的骗将去了。你若
打听得出来,报上山寨,我与你做主。”宋江对李逵道:“这里不和你说话,你回
来寨里,自有辩理。”宋江、柴进自与一行人马,先回大寨里去。

燕青道:“李大哥,怎地好?”李逵道:“只是我性紧上,错做了事。既然输
了这颗头,我自一刀割将下来,你把去献与哥哥便了。”燕青道:“你没来由寻死
做甚么?我叫你一个法则,唤做负荆请罪。”李逵道:“怎地是负荆?”燕青道:
“自把衣服脱了,将麻绳绑缚了,脊梁上背着一把荆杖,拜伏在忠义堂前,告道:
‘由哥哥打多少。’他自然不忍下手。这个唤做负荆请罪。”李逵道:“好却好,
只是有些惶恐,不如割了头去干净。”燕青道:“山寨里都是你兄弟,何人笑你?”
李逵没奈何,只得同燕青回寨来,负荆请罪。

却说宋江、柴进先归到忠义堂上,和众兄弟们正说李逵的事,只见黑旋风脱得
赤条条地,背上负着一把荆杖,跪在堂前,低着头,口里不做一声。宋江笑道:“你
那黑厮,怎地负荆?只这等饶了你不成?”李逵道:“兄弟的不是了!哥哥拣大棍打
几十罢!”宋江道:“我和你赌砍头,你如何却来负荆?”李逵道:“哥哥既是不
肯饶我,把刀来割这颗头去,也是了当。”众人都替李逵陪话。宋江道:“若要我
饶他,只教他捉得那两个假宋江,讨得刘太公女儿来还他,这等方才饶你。”李逵
听了,跳将起来,说道:“我去瓮中捉鳖,手到拿来。”宋江道:“他是两个好汉,
又有两副鞍马,你只独自一个,如何近傍得他?再叫燕青和你同去。”燕青道:“哥
哥差遣,小弟愿往。”便去房中取了弩子,绰了齐眉棍,随着李逵,再到刘太公庄
上。

燕青细问他来情,刘太公说道:“日平西时来,三更里去了,不知所在,又不
敢跟去。那为头的生的矮小,黑瘦面皮。第二个夹壮身材,短须大眼。”二人问了
备细,便叫:“太公放心,好歹要救女儿还你。我哥哥宋公明的将令,务要我两个
寻将来,不敢违误。”便叫煮下干肉,做下蒸饼,各把料袋装了,拴在身边,离了
刘太公庄上。先去正北上寻,但见荒僻无人烟去处。走了一两日,绝不见些消耗。
却去正东上,又寻了两日,直到凌州高唐界内,又无消息。李逵心焦面热,却回来
望西边寻去。又寻了两日,绝无些动静。

当晚,两个且向山边一个古庙中供床上宿歇,李逵那里睡得着,爬起来坐地。
只听得庙外有人走的响,李逵跳将起来,开了庙门看时,只见一条汉子,提着把朴
刀,转过庙后山脚下上去,李逵在背后跟去。燕青听得,拿了弩弓,提了杆棍,随
后跟来,叫道:“李大哥,不要赶,我自有道理。”是夜月色朦胧,燕青递杆棍与
了李逵,远远望见那汉低着头只顾走。燕青赶近,搭上箭,弩弦稳放,叫声:“如
意子,不要误我。”只一箭,正中那汉的右腿,扑地倒了。李逵赶上,劈衣领揪住,
直拿到古庙中,喝问道:“你把刘太公的女儿抢的那里去了?”那汉告道:“好汉,
小人不知此事,不曾抢甚么刘太公女儿。小人只是这里剪径,做些小买卖,那里敢
大弄,抢夺人家子女!”李逵把那汉捆做一块,提起斧来喝道:“你若不实说,砍
你做二十段。”那汉叫道:“且放小人起来商议。”燕青道:“汉子,我且与你拔
了这箭。”放将起来问道:“刘太公女儿,端的是甚么人抢了去?只是你这里剪径
的,你岂可不知些风声?”那汉道:“小人胡猜,未知真实。离此间西北上约有十
五里,有一座山,唤做牛头山,山上旧有一个道院。近来新被两个强人:一个姓王,
名江,一个姓董,名海,这两个都是绿林中草贼,先把道士道童都杀了,随从只有
五七个伴当,占住了道院,专一下来打劫。但到处只称是宋江,多敢是这两个抢了
去。”燕青道:“这话有些来历。汉子,你休怕我!我便是梁山泊浪子燕青,他便
是黑旋风李逵。我与你调理箭疮,你便引我两个到那里去。”那人道:“小人愿往。”

燕青去寻朴刀还了他,又与他扎缚了疮口,趁着月色微明,燕青、李逵扶着他
走过十五里来路,到那山看时,苦不甚高,果似牛头之状。三个上得山来,天尚未
明,来到山头看时,团团一遭土墙,里面约有二十来间房子。李逵道:“我与你先
跳入墙去。”燕青道:“且等天明却理会。”李逵那里忍耐得,腾地跳将过去了。
只听得里面有人喝声,门开处,早有人出来,便挺朴刀来奔李逵。燕青生怕撅撒了
事,拄着杆棒,也跳过墙来。那中箭的汉子一道烟走了。燕青见这出来的好汉正斗
李逵,潜身暗行,一棒正中那好汉脸颊骨上,倒入李逵怀里来,被李逵后心只一斧,
砍翻在地,里面绝不见一个人出来。燕青道:“这厮必有后路走了。我与你去截住
后门,你却把着前门,不要胡乱入去。”且说燕青来到后门墙外,伏在黑暗处,只
见后门开处,早有一条汉子拿了钥匙,来开后面墙门。燕青转将过去,那汉见了,
绕房檐便走出前门来。燕青大叫:“前门截住!”李逵抢将过来,只一斧,劈胸膛
砍倒,便把两颗头都割下来,拴做一处。李逵性起,砍将入去,泥神也似都推倒了。
那几个伴当躲在灶前,被李逵赶去,一斧一个,都杀了。来到房中看时,果然见那
个女儿在床上呜呜的啼哭。看那女子,云鬓花颜,其实美丽。有诗为证:
弓鞋窄窄起春罗,香沁酥胸玉一窝。
丽质难禁风雨骤,不胜幽恨蹙秋波。
燕青问道:“你莫不是刘太公女儿么?”那女子答道:“奴家在十数日之前,被这
两个贼掳在这里,每夜轮一个将奴家奸宿。奴家昼夜泪雨成行,要寻死处,被他监
看得紧。今日得将军搭救,便是重生父母,再养爹娘。”燕青道:“他有两匹马,
在那里放着?”女子道:“只在东边房内。”燕青备上鞍子,牵出门外,便来收拾
房中积攒下的黄白之资,约有三五千两。燕青便叫那女子上了马,将金银包了,和
人头抓了,拴在一匹马上。李逵缚了个草把,将窗下残灯,把草房四边点着烧起。
他两个开了墙门,步送女子下山,直到刘太公庄上。爹娘见了女子,十分欢喜,烦
恼都没了,尽来拜谢两位头领。燕青道:“你不要谢我两个,你来寨里拜谢俺哥哥
宋公明。”两个酒食都不肯吃,一家骑了一匹马,飞奔山上来。

回到寨中,红日衔山之际,都到三关之上。两个牵着马,驼着金银,提了人头,
径到忠义堂上,拜见宋江。燕青将前事细细说了一遍。宋江大喜,叫把人头埋了,
金银收入库中,马放去战马群内喂养。次日,设筵宴与燕青、李逵作贺。刘太公也
收拾金银上山,来到忠义堂上,拜谢宋江。宋江那里肯受,与了酒饭,教送下山回
庄去了,不在话下。梁山泊自是无话,不觉时光迅速:

看看鹅黄着柳,渐渐鸭绿生波。桃腮乱簇红英,杏脸微开绛蕊。山前花,山后
树,俱发萌芽;州上苹,水中芦,都回生意。谷雨初晴,可是丽人天气;禁烟才过,
正当三月韶华。

宋江正坐,只见关下解一伙人到来,说道:“拿到一伙牛子,有七八个车箱,
又有几束哨棒。”宋江看时,这伙人都是彪形大汉,跪在堂前告道:“小人等几个
直从凤翔府来,今上泰安州烧香。目今三月二十八日天齐圣帝降诞之辰,我们都去
台上使棒,一连三日,何止有千百对在那里。今年有个扑手好汉,是太原府人氏,
姓任,名原,身长一丈,自号擎天柱,口出大言,说道:‘相扑世间无对手,争交
天下我为魁。’闻他两年曾在庙上争交,不曾有对手,白白地拿了若干利物,今年
又贴招儿,单搦天下人相扑。小人等因这个人来,一者烧香,二乃为看任原本事,
三来也要偷学他几路好棒,伏望大王慈悲则个。”宋江听了,便叫小校:“快送这
伙人下山去,分毫不得侵犯。今后遇有往来烧香的人,休要惊吓他,任从过往。”
那伙人得了性命,拜谢下山去了。只见燕青起身禀复宋江,说无数句,话不一席,
有分教:惊动了泰安州,大闹了祥符县。正是:东岳庙中双虎斗,嘉宁殿上二龙争。

毕竟燕青说出甚么话来,且听下回分解。