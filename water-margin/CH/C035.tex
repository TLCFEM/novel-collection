\chapter{石将军村店寄书~小李广梁山射雁}

当下秦明和黄信两个到栅门外看时,望见两路来的军马,却好都到。一路是宋
江、花荣,一路是燕顺、王矮虎,各带一百五十余人。黄信便叫寨兵放下吊桥,大
开寨门,迎接两路人马都到镇上。宋江早传下号令:休要害一个百姓,休伤一个寨
兵;叫先打入南寨,把刘高一家老小尽都杀了。王矮虎自先夺了那个妇人。小喽罗
尽把应有家私、金银、财物、宝货之资,都装上车子;再有马匹牛羊,尽数牵了。
花荣自到家中,将应有的财物等项,装载上车,搬取妻小、妹子;内有清风镇上人
数,都发还了。众多好汉收拾已了,一行人马离了清风镇,都回到山寨里来。

车辆人马,都到山寨,郑天寿迎接向聚义厅上相会。黄信与众好汉讲礼罢,坐
于花荣肩下。宋江叫把花荣老小安顿一所歇处;将刘高财物分赏与众小喽罗。王矮
虎拿得那妇人,将去藏在自己房内。燕顺便问道:“刘高的妻,今在何处?”王矮
虎答道:“今番须与小弟做个押寨夫人。”燕顺道:“与却与你,且唤他出来,我
有一句话说。”宋江便道:“我正要问他。”王矮虎便唤到厅前,那婆娘哭着告饶。
宋江喝道:“你这泼妇!我好意救你下山,念你是个命官的恭人,你如何反将冤报?
今日擒来,有何理说?”燕顺跳起身来便道:“这等淫妇,问他则甚?”拔出腰刀,
一刀挥为两段。王矮虎见砍了这妇人,心中大怒,夺过一把朴刀,便要和燕顺交并,
宋江等起身来劝住。宋江便道:“燕顺杀了这妇人也是。兄弟,你看我这等一力救
了他下山,教他夫妻团圆完聚,尚兀自转过脸来,叫丈夫害我。贤弟,你留在身边,
久后有损无益。宋江日后别娶一个好的,教贤弟满意。”燕顺道:“兄弟便是这等
寻思,不杀了,要他无用,久后必被他害了。”王矮虎被众人劝了,默默无言。燕
顺喝叫小喽罗打扫过尸首血迹,且排筵席庆贺。

次日,宋江和黄信主婚,燕顺、王矮虎、郑天寿做媒说合,要花荣把妹子嫁与
秦明,一应礼物,都是宋江和燕顺出备。吃了三五日筵席。自成亲之后,又过了五
七日,小喽罗探得事情,上山来报道:“打听得青州慕容知府申将文书,去中书省
奏说,反了花荣、秦明、黄信,要起大军来征剿,扫荡清风山。”众好汉听罢,商
量道:“此间小寨,不是久恋之地。倘或大军到来,四面围住,如何迎敌?”宋江
道:“小可有一计,不知中得诸位心否?”当下众好汉都道:“愿闻良策。”宋江
道:“自这南方有个去处,地名唤做梁山泊,方圆八百余里,中间宛子城、蓼儿洼,
晁天王聚集着三五千军马,把住着水泊,官兵捕盗,不敢正眼觑他。我等何不收拾
起人马,去那里入伙?”秦明道:“既然有这个去处,却是十分好。只是没人引进,
他如何肯便纳我们?”宋江大笑,却把这打劫生辰纲金银一事,直说到:“刘唐寄
书,将金子谢我,因此上杀了阎婆惜,逃去在江湖上。”秦明听了大喜道:“恁地,
兄长正是他那里大恩人。事不宜迟,可以收拾起快去。”

只就当日商量定了,便打并起十数辆车子,把老小并金银财物、衣服、行李等
件,都装载车子上,共有三二百匹好马。小喽罗们有不愿去的,赍发他些银两,任
从他下山去投别主;有愿去的,编入队里,就和秦明带来的军汉,通有三五百人。
宋江教分作三起下山,只做去收捕梁山泊的官军。山上都收拾的停当,装上车子,
放起火来,把山寨烧作光地,分为三队下山。宋江便与花荣引着四五十人,三五十
骑马,簇拥着五七辆车子,老小队仗先行;秦明、黄信引领八九十匹马,和这应用
车子,作第二起;后面便是燕顺、王矮虎、郑天寿三个,引着四五十匹马。一二百
人离了清风山,取路投梁山泊来。于路中见了这许多军马,旗号上又明明写着收捕
草寇官军,因此无人敢来阻当。在路行五七日,离得青州远了。

且说宋江、花荣两个骑马在前头,背后车辆载着老小,与后面人马只隔着二十
来里远近。前面到一个去处,地名唤对影山,两边两座高山,一般形势,中间却是
一条大阔驿路。两个在马上正行之间,只听得前山里锣鸣鼓响。花荣便道:“前面
必有强人。”把枪带住,取弓箭来整顿得端正,再插放飞鱼袋内,一面叫骑马的军
士,催趱后面两起军马上来,且把车辆人马扎住了。宋江和花荣两个引了二十余骑
军马,向前探路。

至前面半里多路,早见一簇人马,约有一百余人,前面簇拥着一个年少的壮士。
怎生打扮,但见:

头上三叉冠,金圈玉钿;身上百花袍,织锦团花。甲披千道火龙鳞,带束一条
红玛瑙。骑一匹胭脂抹就如龙马,使一条朱红画杆方天戟。背后小校,尽是红衣红
甲。
那个壮士,横戟立马,在山坡前大叫道:“今日我和你比试,分个胜败,见个输赢。”
只见对过山冈子背后早拥出一队人马来,也有百十余人,前面也拥着一个穿白年少
的壮士。怎生模样,但见:

头上三叉冠,顶一团瑞雪;身上镔铁甲,披千点寒霜。素罗袍光射太阳,银花
带色欺明月。坐下骑一匹征宛玉兽,手中抡一枝寒戟银绞。背后小校,都是白衣白
甲。
这个壮士,手中也使一枝方天画戟。这边都是素白旗号,那壁都是绛红旗号。只见
两边红白旗摇,震地花腔鼓擂。那两个壮士更不打话,各挺手中画戟,纵坐下马,
两个就中间大阔路上交锋,比试胜败。花荣和宋江见了,勒住马看时,果然是一对
好厮杀。但见:

旗仗盘旋,战衣飘。绛霞影里,卷几片拂地飞云;白雪
光中,滚数团燎原烈火。故园冬暮,山茶和梅蕊争辉;上苑春浓,李粉共桃脂斗彩。
这个按南方丙丁火,似焰摩天上走丹炉;那个按西方庚辛金,如泰华峰头翻玉井。
宋无忌忿怒,骑火骡子奔走霜林;冯夷神生嗔,跨玉狻猊纵横花界。

两个壮士各使方天画戟,斗到三十余合,不分胜败。花荣和宋江两个在马上看
了喝采。花荣一步步趱马向前看时,只见那两个壮士斗到深涧里。这两枝戟上,一
枝是金钱豹子尾,一枝是金钱五色,却搅做一团,上面绒绦结住了,那里分拆得
开。花荣在马上看见了,便把马带住,左手去飞鱼袋内取弓,右手向走兽壶中拔箭,
搭上箭,曳满弓,觑着豹尾绒绦较亲处,“飕”的一箭,恰好正把绒绦射断。只见
两枝画戟分开做两下,那二百余人一齐喝声采。

那两个壮士便不斗,都纵马跑来,直到宋江、花荣马前,就马上欠身声喏,都
道:“愿求神箭将军大名。”花荣在马上答道:“我这个义兄,乃是郓城县押司、
山东及时雨宋公明,我便是清风镇知寨小李广花荣。”那两个壮士听罢,扎住了戟,
便下马推金山,倒玉柱,都拜道:“闻名久矣。”宋江、花荣慌忙下马,扶起那两
位壮士道:“且请问二位壮士高姓大名?”那个穿红的说道:“小人姓吕,名方,
祖贯潭州人氏,平昔爱学吕布为人,因此习学这枝方天画戟,人都唤小人做小温侯
吕方。因贩生药到山东,消折了本钱,不能够还乡,权且占住这对影山打家劫舍。
近日走这个壮士来,要夺吕方的山寨,和他各分一山,他又不肯,因此每日下山厮
杀。不想原来缘法注定,今日得遇尊颜。”宋江又问这穿白的壮士高姓,那人答道:
“小人姓郭,名盛,祖贯西川嘉陵人氏,因贩水银货卖,黄河里遭风翻了船,回乡
不得。原在嘉陵学得本处兵马张提辖的方天戟,向后使得精熟,人都称小人做赛仁
贵郭盛。江湖上听得说对影山有个使戟的占住了山头,打家劫舍,因此一径来比并
戟法。连连战了十数日,不分胜败。不期今日得遇二公,天与之幸。”

宋江把上件事都告诉了,便道:“既幸相遇,就与二位劝和如何?”两个壮士
大喜,都依允了。诗曰:
铜链劝刀犹易事,箭锋劝戟更希奇。
须知豪杰同心处,利断坚金不用疑。
后队人马已都到了,一个个都引着相见了。吕方先请上山,杀牛宰马筵会。次日,
却是郭盛置酒设席筵宴。宋江就说他两个撞筹入伙,辏队上梁山泊去,投奔晁盖聚
义。那两个欢天喜地,都依允了。便将两山人马点起,收拾了财物,待要起身,宋
江便道:“且住!非是如此去。假如我这里有三五百人马投梁山泊去,他那里亦有
探细的人,在四下里探听,倘或只道我们真是来收捕他,不是耍处!等我和燕顺先
去报知了,你们随后却来,还作三起而行。”花荣、秦明道:“兄长高见,正是如
此计较,陆续进程。兄长先行半日,我等催督人马,随后起身来。”

且不说对影山人马陆续登程,只说宋江和燕顺各骑了马,带领随行十数人,先
投梁山泊来。在路上行了两日,当日行到晌午时分,正走之间,只见官道旁边一个
大酒店。宋江看了道:“孩儿们走得困乏,都叫买些酒吃了过去。”当时宋江和燕
顺下了马,入酒店里来;叫孩儿们松了马肚带,都入酒店里坐。

宋江和燕顺先入店里来看时,只有三副大座头,小座头不多几副。只见一副大
座头上,先有一个在那里占了。宋江看那人时,怎生打扮,但见:

裹一顶猪嘴头巾,脑后两个太原府金不换纽丝铜上穿一领皂袖衫,腰系一条
白膊。下面腿护膝,八答麻鞋。桌子边倚着短棒,横头上放着个衣包。

那人生得八尺来长,淡黄骨查脸,一双鲜眼,没根髭髯。宋江便叫酒保过来说
道:“我的伴当人多,我两个借你里面坐一坐,你叫那个客人移换那副大座头与我
伴当们坐地吃些酒。”酒保应道:“小人理会得。”宋江与燕顺里面坐了,先叫酒
保:“打酒来,大碗先与伴当一人三碗,有肉便买些来与他众人吃,却来我这里斟
酒。”酒保又见伴当们都立满在垆边,酒保却去看着那个公人模样的客人道:“有
劳上下,挪借这副大座头与里面两个官人的伴当坐一坐。”那汉嗔怪呼他做上下,
便焦躁道:“也有个先来后到。甚么官人的伴当,要换座头!老爷不换!”燕顺听
了,对宋江道:“你看他无礼么!”宋江道:“由他便了,你也和他一般见识!”
却把燕顺按住了。

只见那汉转头看了宋江、燕顺冷笑。酒保又陪小心道:“上下,周全小人的买
卖,换一换有何妨。”那汉大怒,拍着桌子道:“你这鸟男女,好不识人,欺负老
爷独自一个,要换座头。便是赵官家,老爷也鸟不换。高则声,大脖子拳不认得
你。”酒保道:“小人又不曾说甚么!”那汉喝道:“量你这厮敢说甚么!”燕顺
听了,那里忍耐得住,便说道:“兀那汉子!你也鸟强,不换便罢,没可得鸟吓他。”
那汉便跳起来,绰了短棒在手里,便应道:“我自骂他,要你多管!老爷天下只让
得两个人,其余的都把来做脚底下的泥。”燕顺焦躁,便提起板凳,却待要打将去。

宋江因见那人出语不俗,横身在里面劝解:“且都不要闹。我且请问你:你天
下只让的那两个人?”那汉道:“我说与你,惊得你呆了。”宋江道:“愿闻那两
个好汉大名。”那汉道:“一个是沧州横海郡柴世宗的孙子,唤做小旋风柴进柴大
官人。”宋江暗暗地点头,又问道:“那一个是谁?”那汉道:“这一个又奢遮,
是郓城县押司山东及时雨呼保义宋公明。”宋江看了燕顺暗笑,燕顺早把板凳放下
了。那汉又道:“老爷只除了这两个,便是大宋皇帝,也不怕他。”宋江道:“你
且住,我问你:你既说起这两个人,我却都认得。你在那里与他两个厮会?”那汉
道:“你既认得,我不说谎,三年前在柴大官人庄上住了四个月有余,只不曾见得
宋公明。”宋江道:“你便要认黑三郎么?”那汉道:“我如今正要去寻他。”宋
江问道:“谁教你寻他?”那汉道:“他的亲兄弟铁扇子宋清教我寄家书去寻他。”

宋江听了大喜,向前拖住道:“‘有缘千里来相会,无缘对面不相逢’,只我
便是黑三郎宋江。”那汉相了一面,便拜道:“天幸使令小弟得遇哥哥,争些儿错
过,空去孔太公那里走一遭。”宋江便把那汉拖入里面问道:“家中近日没甚事?”
那汉道:“哥哥听禀:小人姓石,名勇,原是大名府人氏,日常只靠放赌为生。本
乡起小人一个异名,唤做石将军。为因赌博上一拳打死了个人,逃走在柴大官人庄
上。多听得往来江湖上人说哥哥大名,因此特去郓城县投奔哥哥,却又听得说道为
事出外,因见四郎,听得小人说起柴大官人来,却说哥哥在白虎山孔太公庄上。因
小弟要拜识哥哥,四郎特写这封家书,与小人寄来孔太公庄上。如寻见哥哥时,可
叫兄长作急回来。”宋江见说,心中疑惑,便问道:“你到我庄上住了几日?曾见
我父亲么?”石勇道:“小人在彼只住的一夜,便来了,不曾得见太公。”宋江把
上梁山泊一节都对石勇说了。石勇道:“小人自离了柴大官人庄上,江湖中只闻得
哥哥大名,疏财仗义,济困扶危。如今哥哥既去那里入伙,是必携带。”宋江道:
“这不必你说,何争你一个人!且来和燕顺厮见。”叫酒保且来这里斟酒三杯。酒
罢,石勇便去包裹内取出家书,慌忙递与宋江。

宋江接来看时,封皮逆封着,又没“平安”二字。宋江心内越是疑惑,连忙扯
开封皮,从头读至一半,后面写道:

父亲于今年正月初头因病身故,现今停丧在家,专等哥哥来家迁葬。千万,千
万,切不可误!宋清泣血奉书。
宋江读罢,叫声苦,不知高低,自把胸脯捶将起来,自骂道:“不孝逆子!做下非
为,老父身亡,不能尽人子之道,畜生何异!”自把头去壁上磕撞,大哭起来。燕
顺、石勇拘住。宋江哭得昏迷,半晌方才苏醒。燕顺、石勇两个劝道:“哥哥且省
烦恼。”宋江便分付燕顺道:“不是我寡情薄意,其实只有这个老父记挂,今已没
了,只得星夜赶归去,教兄弟们自上山则个。”燕顺劝道:“哥哥,太公既已没了,
便到家时,也不得见了。世上人无有不死的父母,且请宽心,引我们弟兄去了。那
时小弟却陪侍哥哥归去奔丧,未为晚矣。自古道:‘蛇无头而不行。’若无仁兄去
时,他那里如何肯收留我们?”宋江道:“若等我送你们上山去时,误了我多少日
期,却是使不得。我只写一封备细书札,都说在内,就带了石勇一发入伙,等他们
一处上山。我如今不知便罢;既是天教我知了,正是度日如年,烧眉之急。我马也
不要,从人也不带一个,连夜自赶回家。”燕顺、石勇那里留得住。

宋江问酒保借笔砚,讨了一幅纸,一头哭着,一面写书,再三叮咛在上面。写
了,封皮不粘,交与燕顺收了。讨石勇的八答麻鞋穿上,取了些银两,藏放在身边,
跨了一口腰刀,就拿了石勇的短棒,酒食都不肯唇,便出门要走。燕顺道:“哥
哥也等秦总管、花知寨都来,相见一面了,去也未迟。”宋江道:“我不等了,我
的书去,并无阻滞。石家贤弟,自说备细。可为我上复众兄弟们,可怜见宋江奔丧
之急,休怪则个。”宋江恨不得一步跨到家中,飞也似独自一个去了。

且说燕顺同石勇只就那店里吃了些酒食、点心,还了酒钱,却教石勇骑了宋江
的马,带了从人,只离酒店三五里路,寻个大客店歇了等候。次日辰牌时分,全伙
都到。燕顺、石勇接着,备细说宋江哥哥奔丧去了。众人都埋怨燕顺道:“你如何
不留他一留?”石勇分说道:“他闻得父亲没了,恨不得自也寻死,如何肯停脚,
巴不得飞到家里。写了一封备细书札在此,教我们只顾去,他那里看了书,并无阻
滞。”花荣与秦明看了书,与众人商议道:“事在途中,进退两难:回又不得,散
了又不成。只顾且去,还把书来封了,都到山上看,那里不容,却别作道理。”

九个好汉并作一伙,带了三五百人马,渐近梁山泊,来寻大路上山。一行人马
正在芦苇中过,只见水面上锣鼓振响。众人看时,漫山遍野,都是杂彩旗,水泊
中棹出两只快船来。当先一只船上,摆着三五十个小喽罗,船头上中间坐着一个头
领,乃是豹子头林冲。背后那只哨船上,也是三五十个小喽罗,船头上也坐着一个
头领,乃是赤发鬼刘唐。前面林冲在船上喝问道:“汝等是甚么人?那里的官军?敢
来收捕我们?教你人人皆死,个个不留,你也须知俺梁山泊的大名!”花荣、秦明
等都下马,立在岸边答应道:“我等众人非是官军,有山东及时雨宋公明哥哥书札
在此,特来相投大寨入伙。”林冲听了道:“既有宋公明兄长的书札,且请过前面,
到朱贵酒店里,先请书来看了,却来相请厮会。”船上把青旗只一招,芦苇里棹出
一只小船,内有三个渔人,一个看船,两个上岸来说道:“你们众位将军都跟我来。”
水面上见两只哨船,一只船上把白旗招动,铜锣响处,两只哨船,一齐去了。

一行众人看了,都惊呆了,说道:“端的此处,官军谁敢侵傍?我等山寨如何
及得?”众人跟着两个渔人,从大宽转直到旱地忽律朱贵酒店里。朱贵见说了,迎
接众人,都相见了。便叫放翻两头黄牛,散了分例酒食,讨书札看了。先向水亭上
放一枝响箭,射过对岸芦苇中,早摇过一只快船来。朱贵便唤小喽罗分付罢,叫把
书先赍上山去报知,一面店里杀宰猪羊,管待九个好汉,把军马屯住在四散歇了。

第二日辰牌时分,只见军师吴学究自来朱贵酒店里迎接众人,一个个都相见了。
叙礼罢,动问备细,早有二三十只大白棹船来接。吴用、朱贵邀请九位好汉下船,
老小车辆,人马行李,亦各自都搬在各船上,前望金沙滩来。上得岸,松树径里,
众多好汉随着晁头领,全副鼓乐来接。晁盖为头,与九个好汉相见了,迎上关来。
各自乘马坐轿,直到聚义厅上,一对对讲礼罢。左边一带交椅上,却是晁盖、吴用、
公孙胜、林冲、刘唐、阮小二、阮小五、阮小七、杜迁、宋万、朱贵、白胜;那时
白日鼠白胜,数月之前,已从济州大牢里越狱逃走,到梁山上入伙,皆是吴学究使
人去用度,救得白胜脱身。右边一带交椅上,却是花荣、秦明、黄信、燕顺、王英、
郑天寿、吕方、郭盛、石勇。列两行坐下,中间焚起一炉香来,各设了誓。当日大
吹大擂,杀牛宰马筵宴。一面叫新到火伴厅下参拜了,自和小头目管待筵席。收拾
了后山房舍,教搬老小家眷都安顿了。秦明、花荣在席上称赞宋公明许多好处,清
风山报冤相杀一事,众头领听了大喜。后说吕方、郭盛两个比试戟法,花荣一箭射
断绒绦,分开画戟。晁盖听罢,意思不信,口里含糊应道:“直如此射得亲切,改
日却看比箭。”

当日酒至半酣,食供数品,众头领都道:“且去山前闲玩一回,再来赴席。”
当下众头领相谦相让,下阶闲步乐情,观看山景。行至寨前第三关上,只听得空中
数行宾鸿嘹亮。花荣寻思道:“晁盖却才意思不信我射断绒绦,何不今日就此施逞
些手段,教他们众人看,日后敬伏我。”把眼一观,随行人伴数内却有带弓箭的,
花荣便问他讨过一张弓来,在手看时,却是一张泥金鹊画细弓,正中花荣意,急取
过一枝好箭,便对晁盖道:“恰才兄长见说花荣射断绒绦,众头领似有不信之意,
远远的有一行雁来,花荣未敢夸口,这枝箭要射雁行内第三只雁的头上。射不中时,
众头领休笑。”花荣搭上箭,曳满弓,觑得亲切,望空中只一箭射去。但见:

鹊画弓弯满月,雕翎箭迸飞星。挽手既强,离弦甚疾。雁排空如张皮鹄,人发
矢似展胶竿。影落云中,声在草内。天汉雁行惊折断,英雄雁序喜相联。

当下花荣一箭,果然正中雁行内第三只,直坠落山坡下。急叫军士取来看时,
那枝箭正穿在雁头上。晁盖和众头领看了,尽皆骇然,都称花荣做神臂将军。吴学
究称赞道:“休言将军比小李广,便是养由基也不及神手,真乃是山寨有幸!”自
此梁山泊无一个不钦敬花荣。

众头领再回厅上筵会,到晚各自歇息。次日,山寨中再备筵席,议定坐次。本
是秦明才及花荣,因为花荣是秦明大舅,众人推让花荣在林冲肩下,坐了第五位,
秦明坐第六位,刘唐坐第七位,黄信坐第八位,三阮之下,便是燕顺、王矮虎、吕
方、郭盛、郑天寿、石勇、杜迁、宋万、朱贵、白胜,一行共是二十一个头领。坐
定,庆贺筵宴已毕,山寨中添造大船、屋宇、车辆、什物,打造枪刀、军器、铠甲、
头盔,整顿旌旗、袍袄、弓弩、箭矢,准备抵敌官军,不在话下。

却说宋江自离了村店,连夜赶归。当日申牌时候,奔到本乡村口张社长酒店里
暂歇一歇。那张社长却和宋江家来往得好。张社长见了宋江容颜不乐,眼泪暗流,
张社长动问道:“押司有年半来不到家中,今日且喜归来,如何尊颜有些烦恼,心
中为甚不乐?且喜官事已遇赦了,必是减罪了。”宋江答道:“老叔自说得是。家
中官事且靠后,只有一个生身老父殁了,如何不烦恼?”张社长大笑道:“押司真
个,也是作耍?令尊太公却才在我这里吃酒了回去,只有半个时辰来去,如何却说
这话?”宋江道:“老叔休要取笑小侄。”便取出家书,教张社长看了。“兄弟宋
清明明写道父亲于今年正月初头殁了,专等我归来奔丧。”张社长看罢,说道:“呸,
那里这般事!只午时前后和东村王太公在我这里吃酒了去,我如何肯说谎?”宋江
听了,心中疑影,没做道理处。寻思了半晌,只等天晚,别了社长,便奔归家。

入得庄门看时,没些动静。庄客见了宋江,都来参拜,宋江便问道:“我父亲
和四郎有么?”庄客道:“太公每日望得押司眼穿,今得归来,却是欢喜。方才和
东村里王社长在村口张社长店里吃酒了回来,睡在里面房内。”宋江听了大惊,撇
了短棒,径入草堂上来,只见宋清迎着哥哥便拜。宋江见了兄弟不戴孝,心中十分
大怒,便指着宋清骂道:“你这忤逆畜生,是何道理!父亲见今在堂,如何却写书
来戏弄我?教我两三遍自寻死处,一哭一个昏迷。你做这等不孝之子!”

宋清却待分说,只见屏风背后转出宋太公来叫道:“我儿不要焦躁,这个不干
你兄弟之事。是我每日思量,要见你一面,因此教四郎只写道我殁了,你便归得快。
我又听得人说,白虎山地面多有强人,又怕你一时被人撺掇,落草去了,做个不忠
不孝的人,为此急急寄书去,唤你归家;又得柴大官人那里来的石勇,寄书去与你。
这件事尽都是我主意,不干四郎之事,你休埋怨他。我恰才在张社长店里回来,听
得是你归来了。”

宋江听罢,纳头便拜太公,忧喜相伴。宋江又问父亲道:“不知近日官司如何?
已经赦宥,必然减罪。适间张社长也这般说了。”宋太公道:“你兄弟宋清未回之
先,多有朱仝、雷横的气力,向后只动了一个海捕文书,再也不曾来勾扰。我如今
为何唤你归来,近闻朝廷册立皇太子,已降下一道赦书,应有民间犯了大罪,尽减
一等科断,俱已行开各处施行。便是发露到官,也只该个徒流之罪,不到得害了性
命。且由他,却又别作道理。”宋江又问道:“朱、雷二都头曾来庄上么?”宋清
说道:“我前日听得说来,这两个都差出去了。朱仝差往东京去,雷横不知差到那
里去了。如今县里却是新添两个姓赵的勾摄公事。”宋太公道:“我儿远路风尘,
且去房里将息几时。”合家欢喜,不在话下。

天色看看将晚,玉兔东生,约有一更时分,庄上人都睡了,只听得前后门发喊
起来,看时,四下里都是火把,团团围住宋家庄,一片声叫道:“不要走了宋江!”
太公听了,连声叫苦。不因此起,有分教:大江岸上,聚集好汉英雄;闹市丛中,
来显忠肝义胆。

毕竟宋公明在庄上怎地脱身,且听下回分解。