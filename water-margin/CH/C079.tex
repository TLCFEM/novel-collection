\chapter{刘唐放火烧战船~宋江两败高太尉}

话说当下高太尉望见水路军士,情知不济,正欲回军,只听得四边炮响,急收
聚众将,夺路而走。原来梁山泊只把号炮四下里施放,却无伏兵,只吓得高太尉心
惊胆战,鼠窜狼奔,连夜收军回济州。计点步军,折陷不多;水军折其大半,战船
没一只回来;刘梦龙逃难得回;军士会水的,逃得性命,不会水的,都淹死在水中。
高太尉军威折挫,锐气摧残,且向城中屯驻军马,等候牛邦喜拘刷船到。再差人赍
公文去催,不论是何船只,堪中的尽数拘拿,解赴济州,整顿征进。
却说水浒寨中,宋江先和董平上山,拔了箭矢,唤神医安道全用药调治。安道全使
金疮药敷住疮口,在寨中养病。吴用收住众头领上山,水军头领张横解党世雄到忠
义堂上请功。宋江教且押去后寨软监着,将夺到的船只,尽数都收入水寨,分派与
各头领去了。
再说高太尉在济州城中会集诸将,商议收剿梁山之策,数内上党节度使徐京禀道:
“徐某幼年游历江湖,使枪卖药之时,曾与一人交游。那人深通韬略,善晓兵机,
有孙吴之才调,诸葛之智谋,姓闻名焕章,现在东京城外安仁村教学。若得此人来
为参谋,可以敌吴用之诡计。”高太尉听说,便差首将一员,赍带缎匹鞍马,星夜
回东京,礼请这教村学秀才闻焕章来,为军前参谋。便要早赴济州,一同参赞军务。
那员首将回京去,不得三五日,城外报来,宋江军马,直到城边搦战。高太尉听了
大怒,随即点就本部军兵,出城迎敌,就令各寨节度使同出交锋。
却说宋江军马见高太尉提兵至近,急忙退十五里外平川旷野之地。高太尉引军赶去,
宋江兵马已向山坡边摆成阵势,红旗队里,捧出一员猛将,号旗上写得分明,乃是
双鞭呼延灼。兜住马,横着枪,立在阵前。高太尉看见道:“这厮便是统领连环马
时背反朝廷的。”便差云中节度使韩存保出马迎敌。这韩存保善使一枝方天画戟。
两个在阵前,更不打话,一个使戟去搠,一个用枪来迎。两个战到五十余合,呼延
灼卖个破绽,闪出去,拍着马,望山坡下便走。韩存保紧要干功,跑着马赶来。八
个马蹄翻盏撒钹相似,约赶过五七里无人之处,看看赶上,呼延灼勒回马,带转枪,
舞起双鞭来迎。两个又斗十数合之上,用双鞭分开画戟,回马又走。
韩存保寻思:这厮枪又近不得我,鞭又赢不得我,我不就这里赶上,活拿这贼,更
待何时?抢将近来,赶转一个山嘴,有两条路,竟不知呼延灼何处去了。韩存保勒
马上坡来望时,只见呼延灼绕着一条溪走。存保大叫:“泼贼,你走那里去!快下
马来受降,饶你命!”呼延灼不走,大骂存保。韩存保却大宽转来抄呼延灼后路。
两个却好在溪边相迎着。一边是山,一边是溪,只中间一条路,两匹马盘旋不得。
呼延灼道:“你不降我,更待何时!”韩存保道:“你是我手里败将,倒要我降你?”
呼延灼道:“我漏你到这里,正要活捉你。你性命只在顷刻!”韩存保道:“我正
来活捉你!”
两个旧气又起。韩存保挺着长戟,望呼延灼前心两胁软肚上,雨点般搠将来。呼延
灼用枪左拨右逼,风般搠入来。两个又斗了三十来合。正斗到浓深处,韩存保一
戟,望呼延灼软胁搠来,呼延灼一枪,望韩存保前心刺去。两个各把身躯一闪,两
般军器,都从胁下搠来。呼延灼挟住韩存保戟杆,韩存保扭住呼延灼枪杆;两个都
在马上,你扯我拽,挟住腰胯,用力相争。韩存保的马,后蹄先塌下溪里去了,呼
延灼连人和马,也拽下溪里去了。两个在水中扭做一块。那两匹马溅起水来,一人
一身水。呼延灼弃了手里的枪,挟住他的戟杆,急去掣鞭时,韩存保也撇了他的枪
杆,双手按住呼延灼两条臂。你揪我扯,两个都滚下水去。那两匹马迸星也似跑上
岸来,望山边去了。两个在溪水中都滚没了军器,头上戴的盔没了,身上衣甲飘零,
两个只把空拳来在水中厮打,一递一拳,正在水深里,又拖上浅水里来。正解拆不
开,岸上一彪军马赶到,为头的是没羽箭张清。众人下手,活捉了韩存保。差人急
去寻那走了的两匹战马,只见那马却听得马嘶人喊,也跑回来寻队,因此收住。又
去溪中捞起军器,还呼延灼,带湿上马,却把韩存保背剪缚在马上,一齐都奔峪口。
只见前面一彪军马,来寻韩存保,两家却好当住。为头两员节度使:一个是梅展,
一个是张开。因见水渌渌地马上缚着韩存保,梅展大怒,舞三尖两刃刀,直取张清。
交马不到三合,张清便走,梅展赶来,张清轻舒猿臂,款扭狼腰,只一石子飞来,
正打中梅展额角,鲜血迸流,撇了手中刀,双手掩面。张清急便回马,却被张开搭
上箭,拽满弓,一箭射来。张清把马头一提,正射中马眼,那马便倒。张清跳在一
边,拈着枪便来步战。那张清原来只有飞石打将的本事,枪法上却慢。张开先救了
梅展,次后来战张清。马上这条枪,神出鬼没,张清只办得架隔,遮拦不住,拖了
枪,便走入马军队里躲闪。张开枪马到处,杀得五六十马军,四分五落,再夺得韩
存保。却待回来,只见喊声大举,峪口两彪军到:一队是霹雳火秦明,一队是大刀
关胜,两个猛将杀来。张开只保得梅展走了,众军两路杀入来,又夺了韩存保。张
清抢了一匹马,呼延灼使尽气力,只好随众厮杀,一齐掩击到官军队前,乘势冲动,
退回济州。梁山泊军马也不追赶,只将韩存保连夜解上山寨来。
宋江等坐在忠义堂上,见缚到韩存保来,喝退军士,亲解其索,请坐厅上,殷勤相
待。韩存保感激无地,就请出党世雄相见,一同管待。宋江道:“二位将军,切勿
相疑,宋江等并无异心,只被滥官污吏,逼得如此。若蒙朝廷赦罪招安,情愿与国
家出力。”韩存保道:“前者陈太尉赍到招安诏敕来山,如何不乘机会去邪归正?”
宋江答道:“便是朝廷诏书,写得不明,更兼用村醪倒换御酒,因此弟兄众人,心
皆不伏。那两个张干办、李虞候,擅作威福,耻辱众将。”韩存保道:“只因中间
无好人维持,误了国家大事。”宋江设筵管待已了,次日,具备鞍马,送出谷口。
这两个在路上说宋江许多好处,回到济州城外,却好晚了。次早入城,来见高太尉,
说宋江把二将放回之事。高俅大怒道:“这是贼人诡计,慢我军心。你这二人,有
何面目见吾!左右与我推出,斩讫报来!”王焕等众官都跪下告道:“非干此二人
之事,乃是宋江、吴用之计。若斩此二人,反被贼人耻笑。”高太尉被众人苦告,
饶了两个性命,削去本身职事,发回东京泰乙宫听罪。这两个解回京师。
原来这韩存保是韩忠彦的侄儿。忠彦乃是国老太师,朝廷官员,都有出他门下。有
个门馆教授,姓郑,名居忠,原是韩忠彦抬举的人,现任御史大夫。韩存保把上件
事告诉他;居忠上轿,带了存保来见尚书余深,同议此事。余深道:“须是禀得太
师,方可面奏。”二人来见蔡京说:“宋江本无异心,只望朝廷招安。”蔡京道:
“前者毁诏谤上,如此无礼,不可招安,只可剿捕!”二人禀说:“前番招安,惜
为去人不布朝廷德意,用心抚恤;不用嘉言,专说利害,以此不能成事。”蔡京方
允。约至次日早朝,道君天子升殿,蔡京奏准再降诏敕,令人招安。天子曰:“现
今高太尉使人来请安仁村闻焕章为参谋,早赴军前委用,就差此人伴使前去。如肯
来降,悉免本罪;如仍不伏,就着高俅定限,日下剿捕尽绝还京。”蔡太师写成草
诏,一面取闻焕章赴省筵宴。原来这闻焕章是有名文士,朝廷大臣多有知识的,俱
备酒食迎接。席终各散,一边收拾起行。有诗为证:
年来教授隐安仁,忽召军前捧纶。
权贵满朝多旧识,可无一个荐贤人。
且不说闻焕章同天使出京,却说高太尉在济州心中烦恼。门吏报道:“牛邦喜到来。”
高太尉便教唤进,拜罢,问道:“船只如何?”邦喜禀道:“于路拘刷得大小船一
千五百余只,都到闸下。”太尉大喜。赏了牛邦喜,便传号令,教把船都放入阔港,
每三只一排钉住,上用板铺,船尾用铁环锁定;尽数发步军上船,其余马军,近水
护送船只。比及编排得军士上船,训练得熟,已得半月之久,梁山泊尽都知了。吴
用唤刘唐受计,掌管水路建功。众多水军头领,各各准备小船,船头上排排钉住铁
叶,船舱里装载芦苇干柴,柴中灌着硫黄焰硝引火之物,屯住在小港内。却教炮手
凌振,于四望高山上,放炮为号;又于水边树木丛杂之处,都缚旌旗于树上,每一
处设金鼓火炮,虚屯人马,假设营垒,请公孙胜作法祭风。旱地上分三队军马接应。
吴用指画已了。
却说高太尉在济州催起军马,水路统军却是牛邦喜,又同刘梦龙并党世英这三个掌
管。高太尉披挂了,发三通擂鼓,水港里船开,旱路上马发,船行似箭,马去如飞,
杀奔梁山泊来。先说水路里船只,连篙不断,金鼓齐鸣,迤杀入梁山泊深处,并
不见一只船。看看渐近金沙滩,只见荷花荡里,两只打鱼船,每只船上只有两个人,
拍手大笑。头船上刘梦龙便叫放箭乱射,渔人都跳下水底去了。刘梦龙急催动战船,
渐近金沙滩头。一带阴阴的都是细柳,柳树上拴着两头黄牛,绿莎草上睡着三四个
牧童,远远地又有一个牧童,倒骑着一头黄牛,口中呜呜咽咽吹着一管笛子来。
刘梦龙便教先锋悍勇的首先登岸。那几个牧童跳起来,呵呵大笑,尽穿入柳阴深处
去了。前阵五七百人抢上岸去。那柳阴树中,一声炮响,两边战鼓齐鸣:左边就冲
出一队红甲军,为头是霹雳火秦明;右边冲出一队黑甲军,为头是双鞭呼延灼。各
带五百军马,截出水边。刘梦龙急招呼军士下船时,已折了大半军校。牛邦喜听得
前军喊起,便教后船且退。只听得山顶上连珠炮响,芦苇中飕飕有声,却是公孙胜
披发仗剑,踏罡布斗,在山顶上祭风。初时穿林透树,次后走石飞砂,须臾白浪掀
天,顷刻黑云覆地,红日无光,狂风大作。刘梦龙急教棹船回时,只见芦苇丛中,
藕花深处,小港狭汊,都棹出小船来,钻入大船队里。鼓声响处,一齐点着火把,
霎时间,大火竟起,烈焰飞天,四分五落,都穿在大船内。前后官船,一齐烧着。
怎见得火起,但见:
黑烟迷绿水,红焰起清波。风威卷荷叶满天飞,火势燎芦林连梗断。神号鬼哭,昏
昏日色无光;岳撼山崩,浩浩波声若怒。舰航尽倒,舵橹皆休。船尾旌旗不见青红
交杂,楼头剑戟难排霜雪争叉。僵尸与鱼鳖同浮,热血共波涛并沸。千条火焰连天
起,万道烟霞贴水飞。
当时刘梦龙见满港火飞,战船都烧着了,只得弃了头盔衣甲,跳下水去,又不敢傍
岸,拣港深水阔处,赴将开去逃命。芦林里面一个人,独驾着小船,直迎将来,刘
梦龙便钻入水底下去了。却好有一个人拦腰抱住,拖上船来。撑船的是出洞蛟童威,
拦腰抱的是混江龙李俊。却说牛邦喜见四下官船队里火着,也弃了戎装披挂,却待
下水,船梢上钻起一个人来,拿着铙钩,劈头搭住,倒拖下水里去。那人是船火儿
张横。这梁山泊内杀得尸横水面,血溅波心,焦头烂额者,不计其数。只有党世英
摇着小船,正走之间,芦林两边,弩箭弓矢齐发,射死水中。众多军卒,会水的逃
得性命回去;不会水的,尽皆淹死;生擒活捉者,都解投大寨。李俊捉得刘梦龙,
张横捉得牛邦喜,欲待解上山寨,惟恐宋江又放了。两个好汉自商量,把这二人,
就路边结果了性命,割下首级,送上山来。
再说高太尉引领军马在水边策应,只听得连珠炮响,鼓声不绝,料道是水面上厮杀,
骤着马前来,靠山临水探望。只见纷纷军士,都从水里逃命,爬上岸来。高俅认得
是自家军校,问其缘故,说被放火烧尽船只,俱各不知所在。高太尉听了,心内越
慌。但望见喊声不断,黑烟满空,急引军回旧路时,山前鼓声响处,冲出一队马军
拦路,当先急先锋索超抡起开山大斧,骤马抢近前来。高太尉身边节度使王焕,挺
枪便出,与索超交战。斗不到五合,索超拨回马便走。高太尉引军追赶,转过山嘴,
早不见了索超。正走间,背后豹子头林冲,引军赶来,又杀一阵。再走不过六七里,
又是青面兽杨志,引军赶来,又杀一阵。又奔不到八九里,背后美髯公朱仝赶上来,
又杀一阵。这是吴用使的追赶之计:不去前面拦截,只在背后赶杀,败军无心恋战,
只顾奔走,救护不得后军。因此高太尉被赶得慌,飞奔济州,比及入得城时,已自
三更。又听得城外寨中火起,喊声不绝,原来被石秀、杨雄埋伏下五百步军,放了
三五把火,潜地去了。惊得高太尉魂不附体,连使人探视,回报去了,方才放心。
整点军马,折其大半。
高俅正在纳闷间,远探报道:“天使到来。”高俅遂引军马,并节度使出城迎接,
见了天使,就说降诏招安一事。都与闻焕章参谋使相见了,同进城中帅府商议。高
太尉先讨抄白备照观看。待不招安来,又连折了两阵,拘刷得许多船只,又被尽行
烧毁;待要招安来,恰又羞回京师;心下踌躇,数日主张不定。不想济州有一个老
吏,姓王名瑾,那人平生克毒,人尽呼为剜心王,却是济州府拨在帅府供给的吏。
因见了诏书抄白,更打听得高太尉心内迟疑不决,遂来帅府,呈献利便事件,禀说:
“贵人不必沉吟,小吏看见诏上已有活路:这个写草诏的翰林待诏,必与贵人好,
先开下一个后门了。”
高太尉见说大惊,便问道:“你怎见得先开下后门?”王瑾禀道:“诏书上最要紧
是中间一行。道是:‘除宋江、卢俊义等大小人众,所犯过恶,并与赦免。’此一
句是囫囵话。如今开读时,却分作两句读,将‘除宋江’另做一句,‘卢俊义等大
小人众,所犯过恶,并与赦免’另做一句;赚他漏到城里,捉下为头宋江一个,把
来杀了,却将他手下众人,尽数拆散,分调开去。自古道:‘蛇无头而不行,鸟无
翅而不飞。’但没了宋江,其余的做得甚用?此论不知恩相贵意若何?”高俅大喜,
随即升王瑾为帅府长史,便请闻参谋说知此事。闻焕章谏道:“堂堂天使,只可以
正理相待,不可行诡诈于人。倘或宋江以下有智谋之人识破,翻变起来,深为未便。”
高太尉道:“非也!自古兵书有云:‘兵行诡道。’岂可用得正大?”闻参谋道:
“然虽兵行诡道,这一事是天子圣旨,乃以取信天下。自古王言如纶如,因此号
为玉音,不可移改。今若如此,后有知者,难以此为准信。”高太尉道:“且顾眼
下,却又理会。”遂不听闻焕章之言。先遣一人往梁山泊报知,令宋江等全伙,前
来济州城下,听天子诏敕,赦免罪犯。
却说宋江又赢了高太尉这一阵。烧了的船,令小校搬运做柴,不曾烧的,拘收入水
寨。但是活捉的军将,尽数陆续放回济州。当日宋江与大小头领正在忠义堂上商议,
小校报道:“济州府差人上山来报道:‘朝廷特遣天使,颁降诏书,赦罪招安,加
官赐爵,特来报喜。’”宋江听罢,喜从天降,笑逐颜开,便叫请那报事人到堂上
问时,那人说道:“朝廷降诏,特来招安。高太尉差小人前来,报请大小头领,都
要到济州城下行礼,开读诏书。并无异议,勿请疑惑。”宋江叫请军师商议定了,
且取银两缎匹,赏赐来人,先发付回济州去了。
宋江传下号令,大小头领,尽教收拾去听开读诏书。卢俊义道:“兄长且未可性急,
诚恐这是高太尉的见识,兄长不宜便去。”宋江道:“你们若如此疑心时,如何能
够归正?还是好歹去走一遭。”吴用笑道:“高俅那厮,被我们杀得胆寒心碎,便
有十分的计策,也施展不得。放着众兄弟一班好汉,不要疑心,只顾跟随宋公明哥
哥下山。我这里先差黑旋风李逵引着樊瑞、鲍旭、项充、李衮,将带步军一千,埋
伏在济州东路;再差一丈青扈三娘,引着顾大嫂、孙二娘、王矮虎、孙新、张青,
将带马军一千,埋伏在济州西路。若听得连珠炮响,杀奔北门来取齐。”吴用分调
已定,众头领都下山,只留水军头领看守寨栅。只因高太尉要用诈术,诱引这伙英
雄下山,不听闻参谋谏劝,谁想只就济州城下,翻为九里山前。正是:只因一纸君
王诏,惹起全班壮士心。
毕竟众好汉怎地大闹济州,且听下回分解。