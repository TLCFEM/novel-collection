\chapter{鲁智深浙江坐化~宋公明衣锦还乡}

话说当下方腊殿前启奏愿领兵出洞征战的,正是东床驸马主爵都尉柯引。方腊
见奏,不胜之喜。柯驸马当下同领南兵,带了云璧奉尉,披挂上马出师。方腊将自
己金甲锦袍,赐与附马,又选一骑好马,叫他出战。那柯驸马与同皇侄方杰,引领
洞中护御军兵一万人马,驾前上将二十余员,出到帮源洞口,列成阵势。
却说宋江军马困住洞口,已教将佐分调守护。宋江在阵中,因见手下弟兄,三停内
折了二停,方腊又未曾拿得,南兵又不出战,眉头不展,面带忧容。只听得前军报
来说:“洞中有军马出来交战。”宋江、卢俊义见报,急令诸将上马,引军出战,
摆开阵势,看南军阵里,当先是柯驸马出战。宋江军中,谁不认得是柴进?宋江便
令花荣出马迎敌。花荣得令,便横枪跃马,出到阵前,高声喝问:“你那厮是甚人,
敢助反贼,与吾天兵敌对?我若拿住你时,碎尸万段,骨肉为泥!好好下马受降,免
汝一命!”柯驸马答道:“我乃山东柯引,谁不闻我大名?量你这厮们,是梁山泊
一伙强徒草寇,何足道哉!偏俺不如你们手段?我直把你们杀尽,克复城池,是吾之
愿!”宋江与卢俊义在马上听了,寻思柴进口里说的话,知他心里的事。他把“柴”
字改作“柯”字,“柴”即是“柯”也。“进”字改作“引”字,“引”即是“进”
也。吴用道:“且看花荣与他迎敌。”当下花荣挺枪跃马,来战柯引。两马相交,
二般军器并举。两将斗到间深里,绞做一团,扭做一块。柴进低低道:“兄长可且
诈败,来日议事。”花荣听了,略战三合,拨回马便走。柯引喝道:“败将,吾不
赶你!别有了得的,叫他出来,和俺交战!”花荣跑马回阵,对宋江、卢俊义说知
就里。吴用道:“再叫关胜出战交锋。”当时关胜舞起青龙偃月刀,飞马出战,大
喝道:“山东小将,敢与吾敌?”那柯驸马挺枪,便来迎敌。两个交锋,全无惧怯。
二将斗不到五合,关胜也诈败佯输,走回本阵。柯驸马不赶,只在阵前大喝:“宋
兵敢有强将出来,与吾对敌?”宋江再叫朱仝出阵,与柴进交锋。往来厮杀,只瞒
众军。两个斗不过五七合,朱仝诈败而走。柴进赶来虚搠一枪,朱仝弃马跑归本阵,
南军先抢得这匹好马。柯驸马招动南军,抢杀过来,宋江急令诸将引军退去十里下
寨。柯驸马引军追赶了一程,收兵退回洞中。
已自有人先去报知方腊,说道:“柯驸马如此英雄,战退宋兵,连胜三将。宋江等
又折一阵,杀退十里。”方腊大喜,叫排下御宴,等待驸马卸了戎装披挂,请入后
宫赐坐,亲捧金杯,满劝柯附马道:“不想驸马有此文武双全!寡人只道贤婿只是
文才秀士,若早知有此等英雄豪杰,不致折许多州郡。烦望驸马大展奇才,立诛贼
将,重兴基业,与寡人共享太平无穷之富贵。”柯引奏道:“主上放心!为臣子当
以尽心报效,同兴国祚。明日谨请圣上登山,看柯引厮杀,立斩宋江等辈。”方腊
见奏,心中大喜,当夜宴至更深,各还宫中去了。次早,方腊设朝,叫洞中敲牛宰
马,令三军都饱食已了,各自披挂上马,出到帮源洞口,摇旗发喊,擂鼓搦战。方
腊却领引内侍近臣,登帮源洞山顶,看柯驸马厮杀。
且说宋江当日传令,分付诸将:“今日厮杀,非比他时,正在要紧之际。汝等军将,
各各用心,擒获贼首方腊,休得杀害。你众军士,只看南军阵上柴进回马引领,就
便杀入洞中,并力追捉方腊,不可违误!”三军诸将得令,各自摩拳擦掌,掣剑拔
枪,都要掳掠洞中金帛,尽要活捉方腊,建功请赏。当时宋江诸将,都到洞前,把
军马摆开,列成阵势。只见南兵阵上,柯驸马立在门旗之下,正待要出战,只见皇
侄方杰立马横戟道:“都尉且押手停骑,看方某先斩宋兵一将,然后都尉出马,用
兵对敌。”宋兵望见燕青跟在柴进后头,众将皆喜道:“今日计必成矣!”各人自
行准备。
且说皇侄方杰争先纵马搦战,宋江阵上,关胜出马,舞起青龙刀,来与方杰对敌。
两将交马,一往一来,一翻一复,战不过十数合,宋江又遣花荣出阵,共战方杰。
方杰见二将来夹攻,全无惧怯,力敌二将。又战数合,虽然难见输赢,也只办得遮
拦躲避。宋江队里,再差李应、朱仝骤马出阵,并力追杀。方杰见四将来夹攻,方
才拨回马头,望本阵中便走。柯驸马却在门旗下截住,把手一招,宋将关胜、花荣、
朱仝、李应四将赶过来。柯驸马便挺起手中铁枪奔来,直取方杰。方杰见头势不好,
急下马逃命时,措手不及,早被柴进一枪戳着。背后云奉尉燕青赶上一刀,杀了方
杰。南军众将惊得呆了,各自逃生,柯驸马大叫:“我非柯引,吾乃柴进,宋先锋
部下正将小旋风的便是。随行云奉尉,即是浪子燕青。今者已知得洞中内外备细,
若有人活捉得方腊的,高官任做,细马拣骑。三军投降者,俱免血刃,抗拒者全家
斩首!”回身引领四将,招起大军,杀入洞中。方腊领着内侍近臣,在帮源洞顶上,
看见杀了方杰,三军溃乱,情知事急,一脚踢翻了金交椅,便望深山中奔走。
宋江领起大队军马,分开五路,杀入洞来,争捉方腊,不想已被方腊逃去,止拿得
侍从人员。燕青抢入洞中,叫了数个心腹伴当,去那库里掳了两担金珠细软出来,
就内宫禁苑放起火来。柴进杀入东宫时,那金芝公主自缢身死。柴进见了,就连宫
苑烧化,以下细人,放其各自逃生。众军将都入正宫,杀尽嫔妃彩女、亲军侍御、
皇亲国戚,都掳掠了方腊内宫金帛。宋江大纵军将,入宫搜寻方腊。
却说阮小七杀入内苑深宫里面,搜出一箱,却是方腊伪造的平天冠、衮龙袍、碧玉
带、白玉、无忧履。阮小七看见上面都是珍珠异宝,龙凤锦文,心里想道:“这
是方腊穿的,我便着一着,也不打紧。”便把衮龙袍穿了,系上碧玉带,着了无忧
履,戴起平天冠,却把白玉插放怀里,跳上马,手执鞭,跑出宫前。三军众将,
只道是方腊,一齐闹动,抢将拢来看时,却是阮小七,众皆大笑。这阮小七也只把
做好嬉,骑着马东走西走,看那众将多军抢掳。
正在那里闹动,早有童枢密带来的大将王禀、赵谭入洞助战。听得三军闹嚷,只说
拿得方腊,径来争功。却见是阮小七穿了御衣服,戴着平天冠,在那里嬉笑。王禀、
赵谭骂道:“你这厮莫非要学方腊,做这等样子!”阮小七大怒,指着王禀、赵谭
道:“你这两个,直得甚鸟!若不是俺哥哥宋公明时,你这两个驴马头,早被方腊
已都砍下了!今日我等众将弟兄成了功劳,你们颠倒来欺负!朝廷不知备细,只道是
两员大将来协助成功。”王禀、赵谭大怒,便要和阮小七火并。当时阮小七夺了小
校枪,便奔上来戳王禀。呼延灼看见,急飞马来隔开,已自有军校报知宋江。飞马
到来,见阮小七穿着御衣服。宋江、吴用喝下马来,剥下违禁衣服,丢去一边。宋
江陪话解劝。王禀、赵谭二人虽被宋江并众将劝和了,只是记恨于心。
当日帮源洞中,杀的尸横遍野,流血成渠,按《宋鉴》所载,斩杀方腊蛮兵二万余
级。当下宋江传令,教四下举火,监临烧毁宫殿。龙楼凤阁,内苑深宫,珠轩翠屋,
尽皆焚化。有诗为证:
黄屋朱轩半入云,涂膏衅血自。
若还天意容奢侈,琼室阿房可不焚。
当时宋江等众将监看烧毁已了,引军都来洞口屯驻,下了寨栅,计点生擒人数,只
有贼首方腊未曾获得。传下将令,教军将沿山搜捉。告示乡民:但有人拿得方腊者,
奏闻朝廷,高官任做;知而首者,随即给赏。
却说方腊从帮源洞山顶落路而走,便望深山旷野,透岭穿林,脱了赭黄袍,丢去金
花幞头,脱下朝靴,穿上草履麻鞋,爬山奔走,要逃性命。连夜退过五座山头,走
到一处山凹边,见一个草,嵌在山凹里。方腊肚中饥饿,却待正要去茅内寻讨
些饭吃,只见松树背后转出一个胖大和尚来,一禅杖打翻,便取条绳索绑了。那和
尚不是别人,是花和尚鲁智深。拿了方腊,带到草中,取了些饭吃,正解出山来,
却好迎着搜山的军健,一同绑住捉来见宋先锋。
宋江见拿得方腊,大喜,便问道:“吾师,你却如何正等得这贼首着?”鲁智深道:
“洒家自从在乌龙岭上万松林里厮杀,追赶夏侯成入深山里去,被洒家杀了贪战贼
兵,直赶入乱山深处。迷踪失径,迤随路寻去,正到旷野琳琅山内,忽遇一个老
僧,引领洒家到此处茅中,嘱付道:‘柴米菜蔬都有,只在此间等候。但见个长
大汉从松林深处来,你便捉住。’夜来望见山前火起,小僧看了一夜,又不知此间
山径路数是何处。今早正见这贼爬过山来,因此,俺一禅杖打翻,就捉来绑,不想
正是方腊!”
宋江又问道:“那一个老僧,今在何处?”鲁智深道:“那个老僧,自引小僧到茅
里,分付了柴米出来,竟不知投何处去了。”宋江道:“那和尚眼见得是圣僧罗
汉,如此显灵,令吾师成此大功,回京奏闻朝廷,可以还俗为官,在京师图个荫子
封妻,光耀祖宗,报答父母劬劳之恩。”鲁智深答道:“洒家心已成灰,不愿为官,
只图寻个净了去处,安身立命足矣!”宋江道:“吾师既不肯还俗,便到京师去住
持一个名山大刹,为一僧首,也光显宗风,亦报答得父母。”智深听了,摇首叫道:
“都不要,要多也无用。只得个囫囵尸首,便是强了。”宋江听罢,默上心来,各
不喜欢。点本部下将佐,俱已数足,教将方腊陷车盛了,解上东京,面见天子,催
起三军,带领诸将,离了帮源洞清溪县,都回睦州。
却说张招讨会集刘都督、童枢密,从、耿二参谋,都在睦州聚齐,合兵一处,屯驻
军马。见说宋江获了大功,拿住方腊,解来睦州,众官都来庆贺。宋江等诸将参拜
已了,张招讨道:“已知将军边塞劳苦,损折弟兄。今已全功,实为万幸。”宋江
再拜泣涕道:“当初小将等一百八人,破辽还京,都不曾损了一个。谁想首先去了
公孙胜,京师已留下数人。克复扬州,渡大江,怎知十停去七!今日宋江虽存,有
何面目再见山东父老,故乡亲戚?”张招讨道:“先锋休如此说。自古道:‘贫富
贵贱,宿生所载;寿夭短长,人生分定。’常言道:‘有福人送无福人。’何以损
折将佐为耻!今日功成名显,朝廷知道,必当重用。封官赐爵,光显门闾,衣锦还
乡,谁不称羡!闲事不须挂意,只顾收拾回军。”
宋江拜谢了总兵等官,自来号令诸将。张招讨已传下军令,教把生擒到贼徒伪官等
众,除留方腊另行解赴东京,其余从贼都就睦州市曹斩首施行。所有未收去处——
衢、婺等县贼役赃官,得知方腊已被擒获,一半逃散,一半自行投首。张招讨尽皆
准首,复为良民。就行出榜,去各处招抚,以安百姓。其余随从贼徒,不伤人者,
亦准其自首投降,复为乡民,拨还产业田园。克复州县已了,各调守御官军,护境
安民,不在话下。再说张招讨众官,都在睦州设太平宴,庆贺众将官僚,赏劳三军
将校,传令教先锋头目,收拾朝京。军令传下,各各准备行装,陆续登程。
且说先锋使宋江思念亡过众将,洒然泪下,不想患病在杭州张横、穆弘等六人,朱
富、穆春看视,共是八人在彼。后亦各患病身死,止留得杨林、穆春到来,随军征
进。想起诸将劳苦,今日太平,当以超度,便就睦州宫观净处,扬起长,修设超
度九幽拔罪好事,做三百六十分罗天大醮,追荐前亡后化列位偏正将佐已了。次日,
椎牛宰马,致备牲醴,与同军师吴用等众将,俱到乌龙神庙里,焚帛享祭乌龙大王,
谢祈龙君护佑之恩。回至寨中,所有部下正偏将佐阵亡之人,收得尸骸者,俱令各
自安葬已了。宋江与卢俊义收拾军马将校人员,随张招讨回杭州,听候圣旨,班师
回京。众多将佐功劳,俱各造册,上了文簿,进呈御前。先写表章,申奏天子。三
军齐备,陆续起程。宋江看了部下正偏将佐,止剩得三十六员回军。那三十六人是:
呼保义宋江

玉麒麟卢俊义

智多星吴用
大刀关胜

豹子头林冲

双鞭呼延灼
小李广花荣

小旋风柴进

扑天雕李应
美髯公朱仝

花和尚鲁智深

行者武松
神行太保戴宗

黑旋风李逵

病关索杨雄
混江龙李俊

活阎罗阮小七

浪子燕青
神机军师朱武

镇三山黄信

病尉迟孙立
混世魔王樊瑞

轰天雷凌振

铁面孔目裴宣
神算子蒋敬

鬼脸儿杜兴

铁扇子宋清
独角龙邹润

一枝花蔡庆

锦豹子杨林
小遮拦穆春

出洞蛟童威

翻江蜃童猛
鼓上蚤时迁

小尉迟孙新

母大虫顾大嫂
当下宋江与同诸将引兵马离了睦州,前往杭州进发。正是收军锣响千山震,得胜旗
开十里红。于路无话,已回到杭州。因张招讨军马在城,宋先锋且屯兵在六和塔驻
扎,诸将都在六和寺安歇。先锋使宋江、卢俊义早晚入城听令。
且说鲁智深自与武松在寺中一处歇马听候,看见城外江山秀丽,景物非常,心中欢
喜。是夜月白风清,水天共碧,二人正在僧房里,睡至半夜,忽听得江上潮声雷响。
鲁智深是关西汉子,不曾省得浙江潮信,只道是战鼓响,贼人生发,跳将起来,摸
了禅杖,大喝着,便抢出来。众僧吃了一惊,都来问道:“师父何为如此?赶出何
处去?”鲁智深道:“洒家听得战鼓响,待要出去厮杀。”众僧都笑将起来道:“师
父错听了!不是战鼓响,乃是钱塘江潮信响。”鲁智深见说,吃了一惊,问道:“师
父,怎地唤做潮信响?”寺内众僧,推开窗,指着那潮头,叫鲁智深看,说道:“这
潮信日夜两番来,并不违时刻。今朝是八月十五日,合当三更子时潮来。因不失信,
谓之潮信。”鲁智深看了,从此心中忽然大悟,拍掌笑道:“俺师父智真长老,曾
嘱付与洒家四句偈言,道是‘逢夏而擒’,俺在万松林里厮杀,活捉了个夏侯成;
‘遇腊而执’,俺生擒方腊;今日正应了‘听潮而圆,见信而寂’,俺想既逢潮信,
合当圆寂。众和尚,俺家问你,如何唤做圆寂?”寺内众僧答道:“你是出家人,
还不省得佛门中圆寂便是死?”鲁智深笑道:“既然死乃唤做圆寂,洒家今日必当
圆寂。烦与俺烧桶汤来,洒家沐浴。”寺内众僧,都只道他说耍,又见他这般性格,
不敢不依他,只得唤道人烧汤来,与鲁智深洗浴。换了一身御赐的僧衣,便叫部下
军校:“去报宋公明先锋哥哥,来看洒家。”又问寺内众僧处讨纸笔,写了一篇颂
子,去法堂上捉把禅椅,当中坐了,焚起一炉好香,放了那张纸在禅床上,自叠起
两只脚,左脚搭在右脚,自然天性腾空。比及宋公明见报,急引众头领来看时,鲁
智深已自坐在禅椅上不动了。颂曰:

平生不修善果,只爱杀人放火。忽地顿开金绳,这里扯断玉锁。咦!钱塘江上
潮信来,今日方知我是我。

宋江与卢俊义看了偈语,嗟叹不已。众多头领都来看视鲁智深,焚香拜礼。城
内张招讨并童枢密等众官,亦来拈香拜礼。宋江自取出金帛,散众僧,做个三昼
夜功果,合个朱红龛子盛了,直去请径山住持大惠禅师,来与鲁智深下火;五山十
刹禅师,都来诵经;迎出龛子,去六和塔后烧化。那径山大惠禅师手执火把,直来
龛子前,指着鲁智深,道几句法语,是:

鲁智深,鲁智深,起身自绿林。两只放火眼,一片杀人心。忽地随潮归去,果
然无处跟寻。咄!解使满空飞白玉,能令大地作黄金。
大惠禅师下了火已了,众僧诵经忏悔,焚化龛子,在六和塔山后,收取骨殖,葬入
塔院。所有鲁智深随身多余衣钵,及朝廷赏赐金银并各官布施,尽都纳入六和寺里,
常住公用。浑铁禅杖并皂布直裰,亦留于寺中供养。
当下宋江看视武松,虽然不死,已成废人。武松对宋江说道:“小弟今已残疾,不
愿赴京朝觐。尽将身边金银赏赐,都纳此六和寺中,陪堂公用,已作清闲道人,十
分好了。哥哥造册,休写小弟进京。”宋江见说:“任从你心。”武松自此,只在
六和寺中出家,后至八十善终,这是后话。
再说先锋宋江,每日去城中听令,待张招讨中军人马前进,已将军兵入城屯扎。半
月中间,朝廷天使到来,奉圣旨令先锋宋江等班师回京。张招讨、童枢密、都督刘
光世,从、耿二参谋,大将王禀、赵谭,中军人马,陆续先回京师去了。宋江等随
即收拾军马回京。比及起程,不想林冲染患风病瘫了,杨雄发背疮而死,时迁又感
搅肠痧而死。宋江见了,感伤不已。丹徒县又申将文书来报,说杨志已死,葬于本
县山园。林冲风瘫,又不能痊,就留在六和寺中,教武松看视,后半载而亡。
再说宋江与同诸将离了杭州,望京师进发,只见浪子燕青私自来劝主人卢俊义道:
“小乙自幼随侍主人,蒙恩感德,一言难尽。今既大事已毕,欲同主人纳还原受官
诰,私去隐迹埋名,寻个僻净去处,以终天年。未知主人意下若何?”卢俊义道:
“自从梁山泊归顺宋朝已来,俺弟兄们身经百战,勤劳不易,边塞苦楚,弟兄损折,
幸存我一家二人性命。正要衣锦还乡,图个封妻荫子,你如何却寻这等没结果?”
燕青笑道:“主人差矣!小乙此去,正有结果,只恐主人此去无结果耳。”若燕青,
可谓知进退存亡之机矣。有诗为证:
略地攻城志已酬,陈辞欲伴赤松游。
时人苦把功名恋,只怕功名不到头。
卢俊义道:“燕青,我不曾存半点异心,朝廷如何负我?”燕青道:“主人岂不闻
韩信立下十大功劳,只落得未央宫里斩首。彭越醢为肉酱,英布弓弦药酒?主公,
你可寻思,祸到临头难走!”卢俊义道:“我闻韩信三齐擅自称王,教陈造反;
彭越杀身亡家,大梁不朝高祖;英布九江受任,要谋汉帝江山。以此汉高帝诈游云
梦,令吕后斩之。我虽不曾受这般重爵,亦不曾有此等罪过。”燕青道:“既然主
公不听小乙之言,只怕悔之晚矣!小乙本待去辞宋先锋,他是个义重的人,必不肯
放,只此辞别主公。”卢俊义道:“你辞我,待要那里去?”燕青道:“也只在主
公前后。”卢俊义笑道:“原来也只恁地。看你到那里?”燕青纳头拜了八拜,当
夜收拾了一担金珠宝贝挑着,竟不知投何处去了。次日早晨,军人收拾字纸一张,
来报复宋先锋。宋江看那一张字纸时,上面写道是:

辱弟燕青百拜恳告先锋主将麾下:自蒙收录,多感厚恩。效死干功,补报难尽。
今自思命薄身微,不堪国家任用,情愿退居山野,为一闲人。本待拜辞,恐主将义
气深重,不肯轻放,连夜潜去。今留口号四句拜辞,望乞主帅恕罪:
雁序分飞自可惊,纳还官诰不求荣。
身边自有君王赦,洒脱风尘过此生。
宋江看了燕青的书,并四句口号,心中郁悒不乐。当时尽收拾损折将佐的官诰牌面,
送回京师,缴纳还官。
宋兵人马,迤前进,比及行至苏州城外,只见混江龙李俊诈中风疾,倒在床上。
手下军人来报宋先锋。宋江见报,亲自领医人来看治,李俊道:“哥哥休误了回军
的程限,朝廷见责,亦恐张招讨先回日久。哥哥怜悯李俊时,可以丢下童威、童猛
看视兄弟。待病体痊可,随后赶来朝觐。哥哥军马,请自赴京。”宋江见说,心虽
不然,倒不疑虑,只得引军前进。又被张招讨行文催趱,宋江只得留下李俊、童威、
童猛三人,自同诸将上马赴京去了。
且说李俊三人竟来寻见费保四个,不负前约,七人都在榆柳庄上商议定了,尽将家
私打造船只,从太仓港乘驾出海,自投化外国去了,后来为暹罗国之主。童威、费
保等都做了化外官职,自取其乐,另霸海滨,这是李俊的后话。诗曰:
知机君子事,明哲迈夷伦。
重结义中义,更全身外身。
浔水舟无系,榆庄柳又新。
谁知天海阔,别有一家人。
再说宋江等诸将一行军马,在路无话,复过常州、润州相战去处,宋江无不伤感。
军马渡江,十存二三。过扬州,进淮安,望京师不远了。宋江传令,叫众将各各准
备朝觐。三军人马,九月二十后,回到东京。张招讨中军人马,先进城去。宋江等
军马,只就城外屯住,扎营于旧时陈桥驿,听候圣旨。此时有先前留下伏侍李俊等
小校,从苏州来,报说李俊原非患病,只是不愿朝京为官,今与童威、童猛不知何
处去了。宋江又复嗟叹,叫裴宣写录现在朝京大小正偏将佐数目,共计二十七员,
并殁于王事者,俱录其名数,写成谢恩表章,仍令正偏将佐俱各准备幞头公服,伺
候朝见天子。三日之后,上皇设朝,近臣奏闻天子,教宣宋江等面君朝见。
此日东方渐明,宋江、卢俊义等二十七员将佐,奉旨即忙上马入城。东京百姓看了
时,此是第三番朝见。想这宋江等初受招安时,却奉圣旨,都穿御赐的红绿锦袄子,
悬挂金银牌面,入城朝见。破辽兵之后,回京师时,天子宣命,都是披袍挂甲,戎
装入朝朝见。今番太平回朝,天子特命文扮,却是幞头公服,入城朝觐。东京百姓
看了,只剩得这几个回来,众皆嗟叹不已。宋江等二十七人,来到正阳门下,齐齐
下马入朝。待御史引至丹墀玉阶之下,宋江、卢俊义为首,上前八拜,退后八拜,
进中八拜,三八二十四拜,扬尘舞蹈,山呼万岁。君臣礼足,徽宗天子看见宋江等
只剩得这些人员,心中嗟念。上皇命都宣上殿,宋江、卢俊义引领众将,都上金阶,
齐跪在珠帘之下。上皇命赐众将平身,左右近臣,早把珠帘卷起。天子乃曰:“朕
知卿等众将收剿江南,多负劳苦。卿等弟兄,损折大半,朕闻不胜伤悼。”宋江垂
泪不止,仍自再拜奏曰:“以臣卤钝薄才,肝脑涂地,亦不能报国家大恩。昔日念
臣共聚义兵一百八人,登五台发愿,谁想今日十损其八。谨录人数,未敢擅便具奏,
伏望天慈,俯赐圣鉴。”上皇曰:“卿等部下,殁于王事者,朕命各坟加封,不没
其功。”宋江再拜,进上表文一通。表曰:
平南都总管正先锋使臣宋江等谨上表:伏念臣江等愚拙庸才,孤陋俗吏,往犯无涯
之罪,幸蒙莫大之恩。高天厚地岂能酬,粉骨碎身何足报!股肱竭力,离水泊以除
邪;兄弟同心,登五台而发愿。全忠秉义,护国保民。幽州城鏖战辽兵,清溪洞力
擒方腊。虽则微功上达,奈缘良将下沉。臣江日夕忧怀,旦暮悲怆。伏望天恩俯赐
圣鉴,使已殁者皆蒙恩泽,在生者得庇洪休。臣江乞归田野,愿作农民,实陛下仁
育之赐。臣江等不胜战悚之至!谨录存殁人数,随表上进以闻。
阵亡正偏将佐五十九员:

正将一十四员:
秦明

徐宁

董平

张清

刘唐
史进

索超

张顺

阮小二

阮小五
雷横

石秀

解珍

解宝

偏将四十五员:
宋万

焦挺

陶宗旺

韩滔

彭
郑天寿

曹正

王定六

宣赞

孔亮
施恩

郝思文

邓飞

周通

龚旺
鲍旭

段景住

侯健

孟康

王英
扈三娘

项充

李衮

燕顺

马麟
单廷

魏定国

吕方

郭盛

欧鹏
陈达

杨春

郁保四

李忠

薛永
李云

石勇

杜迁

丁得孙

邹渊
李立

汤隆

蔡福

张青

孙二娘
于路病故正偏将佐一十员:

正将五员:

林冲

杨志

张横

穆弘

杨雄

偏将五员:

孔明

朱贵

朱富

白胜

时迁
杭州六和寺坐化正将一员:

鲁智深
折臂不愿恩赐,六和寺出家正将一员:

武松
旧在京回还蓟州出家正将一员:

公孙胜
不愿恩赐,于路上去正偏将四员:

正将二员:

燕青

李俊

偏将二员:

童威

童猛
旧留在京师,并取回医士,现在京偏将五员:

安道全

皇甫端

金大坚

萧让

乐和
现在朝觐正偏将佐二十七员:

正将一十二员:
宋江

卢俊义

吴用

关胜

呼延灼
花荣

柴进

李应

朱仝

戴宗
李逵

阮小七

偏将一十五员:
朱武

黄信

孙立

樊瑞

凌振
裴宣

蒋敬

杜兴

宋清

邹润
蔡庆

杨林

穆春

孙新

顾大嫂
宣和五年九月

日,先锋使臣宋江

副先锋臣卢俊义等谨上表。
上皇览表,嗟叹不已。乃曰:“卿等一百八人,上应星曜,今止有二十七人见存,
又辞去了四个,真乃十去其八矣!”随降圣旨,将这已殁于王事者,正将偏将,各
授名爵。正将封为忠武郎,偏将封为义节郎。如有子孙者,就令赴京,照名承袭官
爵;如无子孙者,敕赐立庙,所在享祭。惟有张顺显灵有功,敕封金华将军。僧人
鲁智深擒获贼寇有功,善终坐化于大刹,加赠义烈照暨禅师。武松对敌有功,伤残
折臂,现于六和寺出家,封清忠祖师,赐钱十万贯,以终天年。已故女将二人:扈
三娘加赠花阳郡夫人,孙二娘加赠旌德郡君。现在朝觐,除先锋使另封外,正将十
员,各授武节将军,诸州统制;偏将十五员,各授武奕郎,诸路都统领;管军管民,
省院听调。女将一员顾大嫂,封授东源县君。
先锋使宋江加授武德大夫,楚州安抚使兼兵马都总

管。
副先锋卢俊义加授武功大夫,庐州安抚使兼兵马副总

管。
军师吴用授武胜军承宣使。
关胜授大名府正兵马总管。
呼延灼授御营兵马指挥使。
花荣授应天府兵马都统制。
柴进授横海军沧州都统制。
李应授中山府郓州都统制。
朱仝授保定府都统制。
戴宗授兖州府都统制。
李逵授镇江润州都统制。
阮小七授盖天军都统制。
上皇敕命,各各正偏将佐,封官授职,谢恩听命,给付赏赐。偏将一十五员,各赐
金银三百两,彩缎五表里;正将一十员,各赐金银五百两,彩缎八表里。先锋使宋
江、卢俊义,各赐金银一千两,锦缎十表里,御花袍一套,名马一匹。宋江等谢恩
毕,又奏睦州乌龙大王,二次显灵,护国保民,救护军将,以致全胜。上皇准奏,
圣敕加封忠靖灵德普孚惠龙王。御笔改睦州为严州,歙州为徽州,因是方腊造反
之地,各带反文字体。清溪县改为淳安县,帮源洞凿开为山岛。敕委本州官库内支
钱,起建乌龙大王庙,御赐牌额,至今古迹尚存。江南但是方腊残破去处,被害人
民,普免差徭三年。
当日宋江等各各谢恩已了,天子命设太平筵宴,庆贺功臣。文武百官,九卿四相,
同登御宴。是日,贺宴已毕,众将谢恩。宋江又奏:“臣部下自梁山泊受招安,军
卒亡过大半,尚有愿还家者,乞陛下圣恩优恤。”天子准奏,降敕:“如愿为军者,
赐钱一百贯,绢十匹,于龙猛、虎威二营收操,月支俸粮养赡;如不愿者,赐钱二
百贯,绢十匹,各令回乡,为民当差。”宋江又奏:“臣生居郓城县,获罪以来,
自不敢还乡,乞圣上宽恩给假,回乡拜扫,省视亲族,却还楚州之任。未敢擅便,
乞请圣旨。”上皇闻奏大喜,再赐钱十万贯,作还乡之资。宋江谢恩已罢,辞驾出
朝。次日,中书省作太平筵宴,管待众将。第三日,枢密院又设宴庆贺太平。其张
招讨、刘都督、童枢密,从、耿二参谋,王、赵二大将,朝廷自升重爵,不在此本
话内。太乙院题本奏请圣旨,将方腊于东京市曹上凌迟处死,剐了三日示众。有诗
为证:
宋江重赏升官日,方腊当刑受剐时。
善恶到头终有报,只争来早与来迟!
再说宋江奏请了圣旨,给假回乡省亲。部下军将,愿为军者报名,送发龙猛、虎威
二营收操,关给赏赐,马军守备;愿为民者,关请银两,各各还乡,为民当差。部
下偏将,亦各请受恩赐,听除管军管民,护境为官,关领诰命,各人赴任,与国安
民。
宋江分派已了,与众暂别,自引兄弟宋清,带领随行军健一二百人,挑担御物、行
李、衣装、赏赐,离了东京,望山东进发。宋江、宋清在马上,衣锦还乡,离了京
师,回归故里。于路无话,自来到山东郓城县宋家村。乡中故旧父老亲戚,都来迎
接宋江,回到庄上。不期宋太公已死,灵柩尚存。宋江、宋清痛哭伤感,不胜哀戚。
家眷庄客,都来拜见宋江。庄院田产,家私什物,宋太公存日,整置得齐备,亦如
旧时。宋江在庄上修设好事,请僧命道,修建功果,荐拔亡过父母宗亲。州县官僚,
探望不绝。择日选时,亲扶太公灵柩,高原安葬。是日,本州官员,亲邻父老,宾
朋眷属,尽来送葬已了,不在话下。宋江思念玄女娘娘愿心未酬,将钱五万贯,命
工匠人等,重建九天玄女娘娘庙宇,两廊山门,装饰圣像,彩画两廊,俱已完备。
不觉在乡日久,诚恐上皇见责,选日除了孝服,又做了几日道场,次后设一大会,
请当村乡尊父老,饮宴酌杯,以叙阔别之情。次日,亲戚亦皆置筵庆贺,不在话下。
宋江将庄院交割与次弟宋清,虽受官爵,只在乡中务农,奉祀宗亲香火。将多余钱
帛,散惠下民。
宋江在乡中住了数月,辞别乡老故旧,再回东京,与众弟兄相见。众人有搬取老小
家眷回京住的,有往任所去的,亦有夫主兄弟殁于王事的,朝廷已自颁降恩赐金帛,
令归乡里,优恤其家。宋江自到东京,发遣回乡,都已完足。朝前听命,辞别省院
诸官,收拾赴任。只见神行太保戴宗来探宋江,坐间说出一席话来,有分教:宋公
明生为郓城县英雄,死作蓼儿洼土地。正是:凛凛清风生庙宇,堂堂遗像在凌烟。

毕竟戴宗对宋江说出甚话来,且听下回分解。