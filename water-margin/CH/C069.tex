\chapter{东平府误陷九纹龙~宋公明义释双枪将}

话说宋江不负晁盖遗言,要把主位让与卢员外,众人不伏。宋江又道:“目今
山寨钱粮缺少,梁山泊东,有两个州府,却有钱粮:一处是东平府,一处是东昌府。
我们自来不曾搅扰他那里百姓,若去问他借粮,公然不肯。今写下两个阄儿,我和
卢员外各拈一处,如先打破城子的,便做梁山泊主,如何?”吴用道:“也好。听
从天命。”卢俊义道:“休如此说。只是哥哥为梁山泊主,某听从差遣。”此时不
由卢俊义。当下便唤铁面孔目裴宣,写下两个阄儿。焚香对天祈祷已罢,各拈一个。
宋江拈着东平府,卢俊义拈着东昌府,众皆无语。

当日设筵,饮酒中间,宋江传令,调拨人马。宋江部下:林冲、花荣、刘唐、
史进、徐宁、燕顺、吕方、郭盛、韩滔、彭玘、孔明、孔亮、解珍、解宝、王矮虎、
一丈青、张青、孙二娘、孙新、顾大嫂、石勇、郁保四、王定六、段景住,大小头
领二十五员,马步军兵一万;水军头领三员:阮小二、阮小五、阮小七,领水军驾
船接应。

卢俊义部下:吴用、公孙胜、关胜、呼延灼、朱仝、雷横、索超、杨志、单廷
圭、魏定国、宣赞、郝思文、燕青、杨林、欧鹏、凌振、马麟、邓飞、施恩、樊瑞、
项充、李衮、时迁、白胜,大小头领二十五员,马步军兵一万;水军头领三员:李
俊、童威、童猛,引水手驾船接应。其余头领并中伤者,看守寨栅。

分俵已定,宋江与众头领去打东平府,卢俊义与众头领去打东昌府。众多头领
各自下山。此是三月初一日的话。日暖风和,草青沙软,正好厮杀。

却说宋江领兵前到东平府,离城只有四十里路,地名安山镇,扎驻军马。宋江
道:“东平府太守程万里和一个兵马都监,乃是河东上党郡人氏。此人姓董,名平,
善使双枪,人皆称为双枪将,有万夫不当之勇。虽然去打他城子,也和他通些礼数。
差两个人,赍一封战书,去那里下。若肯归降,免致动兵;若不听从,那时大行杀
戮,使人无怨。谁敢与我先去下书?”只见部下走过一人,身长一丈,腰阔数围。
那人是谁,有诗为证:
不好资财惟好义,貌似金刚离古寺。
身长唤做险道神,此是青州郁保四。
郁保四道:“小人认得董平,情愿赍书去下。”又见部下转过一人,瘦小身材,叫
道:“我帮他去。”那人是谁:
蚱蜢头尖光眼目,鹭鸶瘦腿全无肉。
路遥行走疾如飞,扬子江边王定六。
这两个便道:“我们不曾与山寨中出得些气力,今日情愿去走一遭。”宋江大喜,
随即写了战书,与郁保四、王定六两个去下。书上只说借粮一事。且说东平府程太
守,闻知宋江起军马到了安山镇驻扎,便请本州兵马都监双枪将董平商议军情重事。
正坐间,门人报道:“宋江差人下战书。”程太守教唤至,郁保四、王定六当府厮
见了,将书呈上。程万里看罢来书,对董都监说道:“要借本府钱粮,此事如何?”
董平听了大怒,叫推出去,即便斩首。程太守说道:“不可。自古‘两国相战,不
斩来使。’于礼不当。只将二人各打二十讯棍,发回原寨,看他如何。”董平怒气
未息,喝把郁保四、王定六一索捆翻,打得皮开肉绽,推出城去。两个回到大寨,
哭告宋江说:“董平那厮无礼,好生眇视大寨!”

宋江见打了两个,怒气填胸,便要平吞州郡;先叫郁保四、王定六上车回山将
息。只见九纹龙史进起身说道:“小弟旧在东平府时,与院子里一个娼妓有交,唤
做李瑞兰,往来情熟。我如今多将些金银,潜地入城,借他家里安歇。约时定日,
哥哥可打城池。只等董平出来交战,我便爬去更鼓楼上,放起火来,里应外合,可
成大事。”宋江道:“最好。”史进随即收拾金银,安在包袱里,身边藏了暗器,
拜辞起身。宋江道:“兄弟善觑方便,我且顿兵不动。”

且说史进转入城中,径到西瓦子李瑞兰家。大伯见是史进,吃了一惊,接入里
面,叫女儿出去厮见。李瑞兰生的甚是标格出尘。有诗为证:
万种风流不可当,梨花带雨玉生香。
翠禽啼醒罗浮梦,疑是梅花靓晓妆。

李瑞兰引去楼上坐了,遂问史进道:“一向如何不见你头影?听的你在梁山泊
做了大王,官司出榜捉你,这两日街上乱哄哄地说,宋江要来打城借粮,你如何却
到这里?”史进道:“我实不瞒你说:我如今在梁山泊做了头领,不曾有功,如今
哥哥要来打城借粮,我把你家备细说了。如今我特地来做细作,有一包金银,相送
与你,切不可走漏了消息。明日事完,一发带你一家上山快活。”李瑞兰葫芦提应
承,收了金银,且安排些酒肉相待,却来和大娘商量道:“他往常做客时,是个好
人,在我家出入不妨;如今他做了歹人,倘或事发,不是耍处。”大伯说道:“梁
山泊宋江这伙好汉,不是好惹的;但打城池,无有不破。若还出了言语,他们有日
打破城子入来,和我们不干罢!”虔婆便骂道:“老蠢物,你省得甚么人事?自古
道:‘蜂刺入怀,解衣去赶。’天下通例:自首者即免本罪。你快去东平府里首告,
拿了他去,省得日后负累不好。”李公道:“他把许多金银与我家,不与他担些干
系,买我们做甚么?”虔婆骂道:“老畜生,你这般说,却似放屁!我这行院人家,
坑陷了千千万万的人,岂争他一个!你若不去首告,我亲自去衙前叫屈,和你也说
在里面。”李公道:“你不要性发,且叫女儿款住他,休得‘打草惊蛇’,吃他走
了。待我去报与做公的,先来拿了,却去首告。”且说史进见这李瑞兰上楼来,觉
得面色红白不定,史进便问道:“你家莫不有甚事,这般失惊打怪?”李瑞兰道:
“却才上胡梯,踏了个空,争些儿跌了一交,因此心慌撩乱。”史进虽是英勇,又
吃他瞒过了,更不猜疑。有诗为证:
可叹青楼伎俩多,粉头毕竟护虔婆。
早知暗里施奸计,错用黄金买笑歌。
当下李瑞兰相叙间阔之情,争不过一个时辰,只听得胡梯边脚步响,有人奔上来;
窗外呐声喊,数十个做公的抢到楼上,史进措手不及,正如鹰拿野雀,弹打斑鸠,
把史进似抱头狮子绑将下楼来,径解到东平府里厅上。程太守看了,大骂道:“你
这厮胆包身体,怎敢独自个来做细作!若不是李瑞兰父亲首告,误了我一府良民!快
招你的情由!宋江教你来怎地?”史进只不言语。董平便道:“这等贼骨头,不打
如何肯招!”程太守喝道:“与我加力打这厮!”两边走过狱卒牢子,先将冷水来
喷腿上,两腿各打一百大棍。史进由他拷打,不招实情。董平道:“且把这厮长枷
木,送在死囚牢里,等拿了宋江,一并解京施行。”

却说宋江自从史进去了,备细写书与吴用知道。吴用看了宋公明来书,说史进
去娼妓李瑞兰家做细作,大惊,急与卢俊义说知,连夜来见宋江,问道:“谁叫史
进去来?”宋江道:“他自愿去。说这李行首,是他旧日的表子,好生情重,因此
前去。”吴用道:“兄长欠些主张,若吴某在此决不教去。常言道:娼妓之家,讳
‘者扯丐漏走’五个字。得便熟闲,迎新送旧,陷了多少才人。更兼水性无定,总
有恩情,也难出虔婆之手。此人今去,必然吃亏!”宋江便问吴用请计。吴用便叫
顾大嫂:“劳烦你去走一遭,可扮做贫婆,潜入城中,只做求乞的。若有些动静,
火急便回。若是史进陷在牢中,你可去告狱卒,只说:‘有旧情恩念,我要与他送
一口饭。’捵入牢中,暗与史进说知:‘我们月尽夜黄昏前后,必来打城。你可就
水火之处,安排脱身之计。’月尽夜,你就城中放火为号,此间进兵,方好成事。
兄长可先打汶上县,百姓必然都奔东平府。却叫顾大嫂杂在数内,乘势入城,便无
人知觉。”吴用设计已罢,上马便回东昌府去了。宋江点起解珍、解宝,引五百余
人,攻打汶上县,果然百姓扶老携幼,鼠窜狼奔,都奔东平府来。

却说顾大嫂头髻蓬松,衣服蓝缕,杂在众人里面,捵入城来,绕街求乞。到于
衙前,打听得果然史进陷在牢中,方知吴用智料如神。次日,提着饭罐,只在司狱
司前,往来伺候。见一个年老公人从牢里出来,顾大嫂看着便拜,泪如雨下。那年
老公人问道:“你这贫婆哭做甚么?”顾大嫂道:“牢中监的史大郎,是我旧的主
人。自从离了,又早十年。只说道在江湖上做买卖,不知为甚事陷在牢里?眼见得
无人送饭,老身叫化得这一口儿饭,特要与他充饥。哥哥,怎生可怜见,引进则个,
强如造七层宝塔!”那公人道:“他是梁山泊强人,犯着该死的罪,谁敢带你入去?”
顾大嫂道:“便是一刀一剐,自教他瞑目而受;只可怜见,引老身入去,送这口儿
饭,也显得旧日之情。”说罢又哭。那老公人寻思道:“若是个男子汉,难带他入
去,一个妇人家有甚利害?”当时引顾大嫂直入牢中来,看见史进项带沉枷,腰缠
铁索。史进见了顾大嫂,吃了一惊,则声不得。顾大嫂一头假啼哭,一头喂饭。别
的节级,便来喝道:“这是该死的歹人!‘狱不通风’,谁放你来送饭?即忙出去,
饶你两棍!”顾大嫂见这牢内人多,难说备细,只说得:“月尽夜打城,叫你牢中
自挣扎。”史进再要问时,顾大嫂被小节级打出牢门。史进只记得“月尽夜”。

原来那个三月,却是大尽。到二十九,史进在牢中,见两个节级说话,问道:
“今朝是几时?”那个小节级却错记了,回说道:“今日是月尽夜,晚些买帖孤魂
纸来烧。”史进得了这话,巴不得晚。一个小节级吃的半醉,带史进到水火坑边,
史进哄小节级道:“背后的是谁?”赚得他回头,挣脱了枷,只一枷梢,把那小节
级面上正着一下,打倒在地;就拾砖头,敲开了木,睁着鹘眼,抢到亭心里。几
个公人都酒醉了,被史进迎头打着,死的死了,走的走了。拔开牢门,只等外面救
应。又把牢中应有罪人,尽数放了,总有五六十人,就在牢内发起喊来,一齐走了。
有人报知太守,程万里惊得面如土色,连忙便请兵马都监商量。董平道:“城中必
有细作,且差多人围困了这贼。我却乘此机会,领军出城,去捉宋江。相公便紧守
城池,差数十公人围定牢门,休教走了。”董平上马,点军去了。程太守便点起一
应节级、虞候、押番,各执枪棒,去大牢前呐喊。史进在牢里,不敢轻出。外厢的
人,又不敢进去。顾大嫂只叫得苦。

却说都监董平,点起兵马,四更上马,杀奔宋江寨来。伏路小军报知宋江,宋
江道:“此必是顾大嫂在城中又吃亏了。他既杀来,准备迎敌。”号令一下,诸军
都起。当时天色方明,却好接着董平军马。两个摆开阵势,董平出马,真乃英雄盖
世,谋勇过人。有诗为证:
两面旗牌耀日明,锼银铁铠似霜凝。
水磨凤翅头盔白,锦绣麒麟战袄青。
一对白龙争上下,两条银蟒递飞腾。
河东英勇风流将,能使双枪是董平。

原来董平心灵机巧,三教九流,无所不通,品竹调弦,无有不会,山东、河北
皆号他为风流双枪将。宋江在阵前看了董平这表人品,一见便喜;又见他箭壶中插
一面小旗,上写一联道:“英雄双枪将,风流万户侯。”宋江遣韩滔出马迎敌。韩
滔手执铁搠,直取董平,董平那对铁枪,神出鬼没,人不可当。宋江再叫金枪手徐
宁,仗钩镰枪前去,替回韩滔。徐宁飞马便出,接住董平厮杀。两个在战场上斗到
五十余合,不分胜败。交战良久,宋江恐怕徐宁有失,便叫鸣金收军。徐宁勒马回
来,董平手举双枪,直追杀入阵来。宋江鞭梢一展,四下军兵,一齐围住。宋江勒
马上高阜处看望,只见董平围在阵内。他若投东,宋江便把号旗望东指,军马向东
来围他;他若投西,号旗便往西指,军马便向西来围他。董平在阵中横冲直撞,两
枝枪直杀到申牌已后,冲开条路,杀出去了。宋江不赶。董平因见交战不胜,当晚
收军回城去了。宋江连夜起兵,直抵城下,团团调兵围住。顾大嫂在城中,未敢放
火,史进又不得出来,两下拒住。

原来程太守有个女儿,十分颜色。董平无妻,累累使人去求为亲,程万里不允。
因此,日常间有些言和意不和。董平当晚领军入城,其日使个就里的人,乘势来问
这头亲事。程太守回说:“我是文官,他是武官,相赘为婿,正当其理。只是如今
贼寇临城,事在危急,若还便许,被人耻笑。待得退了贼兵,保护城池无事,那时
议亲,亦未为晚。”那人把这话回复董平,董平虽是口里应道:“说得是。”只是
心中踌躇,不十分欢喜,恐怕他日后不肯。

这里宋江连夜攻打得紧,太守催请出战。董平大怒,披挂上马,带领三军,出
城交战。宋江亲在阵前门旗下喝道:“量你这个寡将,怎敢当吾?岂不闻古人曾有
言:‘大厦将倾,非一木可支。’你看我手下雄兵十万,猛将千员,替天行道,济
困扶危,早来就降,免汝一死!”董平大怒,回道:“文面小吏,该死狂徒,怎敢
乱言!”说罢,手举双枪,直奔宋江。左有林冲,右有花荣,两将齐出,各使军器,
来战董平。约斗数合,两将便走。宋江军马佯败,四散而奔。董平要逞功劳,拍马
赶来。宋江等却好退到寿春县界,宋江前面走,董平后面追。离城有十数里,前至
一个村镇,两边都是草屋,中间一条驿路。董平不知是计,只顾纵马赶来。宋江因
见董平了得,隔夜已使王矮虎、一丈青、张青、孙二娘四个,带一百余人,先在草
屋两边埋伏;却拴数条绊马索在路上,又用薄土遮盖,只等来时,鸣锣为号,绊马
索齐起,准备捉这董平。董平正赶之间,来到那里,只听得背后孔明、孔亮大叫:
“勿伤吾主!”却好到草屋前,一声锣响,两边门扇齐开,拽起绳索。那马却待回
头,背后绊马索齐起,将马绊倒,董平落马。左边撞出一丈青、王矮虎;右边走出
张青、孙二娘。一齐都上,把董平捉了。头盔、衣甲、双枪、只马,尽数夺了。两
个女头领,将董平捉住,用麻绳背剪绑了。两个女将,各执钢刀,监押董平来见宋
江。

却说宋江过了草屋,勒住马,立在绿杨树下,迎见这两个女头领解着董平,宋
江随即喝退两个女将:“我教你去相请董将军,谁教你们绑缚他来!”二女将喏喏
而退。宋江慌忙下马,自来解其绳索,便脱护甲锦袍与董平穿着,纳头便拜。董平
慌忙答礼。宋江道:“倘蒙将军不弃微贱,就为山寨之主。”董平答道:“小将被
擒之人,万死犹轻!若得容恕安身,实为万幸。”宋江道:“敝寨地连水泊,素无
扰害。今为缺少粮食,特来东平府借粮,别无他意。”董平道:“程万里那厮,原
是童贯门下门馆先生,得此美任,安得不害百姓?若是兄长肯容董平今去赚开城门,
杀入城中,共取钱粮,以为报效。”宋江大喜,便令一行人,将过盔、甲、枪、马,
还了董平,披挂上马。董平在前,宋江军马在后,卷起旗,都在东平城下。

董平军马在前,大叫:“城上快开城门。”把门军士将火把照时,认得是董都
监,随即大开城门,放下吊桥。董平拍马先入,砍断铁锁;背后宋江等长驱人马,
杀入城来。都到东平府里,急传将令,不许杀害百姓、放火烧人房屋。董平径奔私
衙,杀了程太守一家人口,夺了这女儿。宋江先叫开放大牢,救出史进,便开府库,
尽数取了金银财帛,大开仓廒,装载粮米上车。先使人护送上梁山泊金沙滩,交割
与三阮头领接递上山。史进自引人去西瓦子李瑞兰家,把虔婆老幼一门大小,碎尸
万段。宋江将太守家私俵散居民,仍给沿街告示,晓谕百姓:害民州官,已自杀戮;
汝等良民,各安生理。告示已罢,收拾回军。

大小将校再到安山镇。只见白日鼠白胜飞奔前来,报说东昌府交战之事。宋江
听罢,神眉踢竖,怪眼圆睁,大叫:“众多兄弟,不要回山,且跟我来!”正是:
重驱水泊英雄将,再夺东昌锦绣城。

毕竟宋江复引军马投何处来,且听下回分解。