\chapter{颜统军阵列混天象~宋公明梦授玄女法}

话说当时宋江在高阜处,看了辽兵势大,慌忙回马来到本阵,且教将军马退回
永清县山口屯扎。便就帐中与卢俊义、吴用、公孙胜等商议道:“今日虽是赢了他
一阵,损了他两个先锋,我上高阜处观望辽兵,其势浩大,漫天遍地而来,此乃是
大队番军人马。来日必用与他大战交锋,恐寡不敌众,如之奈何?”吴用道:“古
之善用兵者,能使寡敌众。昔晋谢玄五万人马,战退苻坚百万雄兵,先锋何为惧哉!
可传令与三军众将,来日务要旗严整,弓弩上弦,刀剑出鞘,深栽鹿角,警守营
寨,濠堑齐备,军器并施,整顿云梯炮石之类,预先伺候。还只摆九宫八卦阵势。
如若他来打阵,依次而起,纵他有百万之众,安敢冲突。”宋江道:“军师言之甚
妙。”随即传令已毕,诸将三军,尽皆听令。五更造饭,平明拔寨都起,前抵昌平
县界,即将军马摆开阵势,扎下营寨。前面摆列马军,还是虎军大将:秦明在前,
呼延灼在后,关胜居左,林冲居右,东南索超,东北徐宁,西南董平,西北杨志。
宋江守领中军,其余众将,各依旧职。后面步军,另做一阵在后,卢俊义、鲁智深、
武松三个为主。数万之中,都是能征惯战之将,个个磨拳擦掌,准备厮杀。阵势已
定,专候番军。

不多时,遥望辽兵远远而来。前面六队番军人马,每队各有五百,左设三队,右设
三队,循环往来,其势不定。此六队游兵,又号哨路,又号压阵。次后大队盖地来
时,前军尽是皂纛旗,一代有七座旗门,每门有千匹马,各有一员大将。怎生打扮?
头顶黑盔,身披玄甲,上穿皂袍,坐骑乌马。手中一般军器,正按北方斗、牛、女、
虚、危、室、壁。七门之内,总设一员把总上将,按上界北方玄武水星。怎生打扮?
头披青丝细发,黄抹额紧束金箍,身穿秃袖皂袍,乌油甲密铺银铠。足跨一匹乌骓
千里马,手擎一口黑柄三尖刀。乃是番将曲利出清,引三千披发黑甲人马,按北辰
五星君。皂旗下军兵,不计其数。正是冻云截断东方日,黑气平吞北海风。

左军尽是青龙旗,一代也有七座旗门,每门有千匹马,各有一员大将。怎生打扮?
头戴四缝盔,身披柳叶甲,上穿翠色袍,下坐青鬃马。手拿一般军器,正按东方角、
亢、氐、房、心、尾、箕。七门之内,总设一员把总大将,按上界东方苍龙木星。
怎生打扮?头戴狮子盔,身披狻猊铠,堆翠绣青袍,缕金碧玉带。手中月斧金丝杆,
身坐龙驹玉块青。乃是番将只儿拂郎,引三千青色宝幡人马,按东震九气星君。青
旗下左右围绕军兵,不计其数。正似翠色点开黄道路,青霞截断紫云根。

右军尽是白虎旗,一代也有七座旗门,每门有千匹马,各有一员大将。怎生打扮?
头戴水磨盔,身披烂银铠,上穿素罗袍,坐骑雪白马。各拿伏手军器,正按西方奎、
娄、胃、昴、毕、觜、参。七门之内,总设一员把总大将,按上界西方咸池金星。
怎生打扮?头顶兜鍪凤翅盔,身披花银双钩甲,腰间玉带迸寒光,称体素袍飞雪练。
骑一匹照夜玉狻猊马,使一枝纯钢银枣槊。乃是番将乌利可安,引三千白缨素旗人
马,按西兑七气星君。白旗下前后护御军兵,不计其数。正似征驼卷尽阴山雪,番
将斜披玉井冰。

后军尽是绯红旗,一代亦有七座旗门,每门有千匹马,各有一员大将。怎生打扮?
头戴箱朱红漆笠,身披猩猩血染征袍。桃红锁甲现鱼鳞,冲阵龙驹名赤兔。各掿
伏手军器,正按南方井、鬼、柳、星、张、翼、轸。七门之内,总设一员把总大将,
按上界南方朱雀火星。怎生打扮?头顶着绛冠,朱缨粲烂。身穿绯红袍,茜色光辉。
甲披一片红霞,靴刺数条花缝。腰间宝带红鞓,臂挂硬弓长箭。手持八尺火龙刀,
坐骑一匹胭脂马。乃是番将洞仙文荣,引三千红罗宝人马,按南离三星君。红
旗下朱缨绛衣军兵,不计其数。正似离宫走却六丁神,霹雳震开三昧火。

阵前左有一队五千猛兵,人马尽是金缕弁冠,镀金铜甲,绯袍朱缨,火焰红旗,绛
鞍赤马,簇拥着一员大将。头戴簇芙蓉如意缕金冠,身披结连环兽面锁子黄金甲,
猩红烈火绣花袍,碧玉嵌金七宝带。使两口日月双刀,骑一匹五明赤马。乃是辽国
御弟大王耶律得重,正按上界太阳星君。正似金乌拥出扶桑国,火伞初离东海洋。
阵前右设一队五千女兵,人马尽是银花弁冠,银钩锁甲,素袍素缨,白旗白马,银
杆刀枪,簇拥着一员女将。金凤钗对插青丝,红抹额乱铺珠翠,云肩巧衬锦裙,绣
袄深笼银甲。小小花靴金镫稳,翩翩翠袖玉鞭轻。使一口七星宝剑,骑一匹银鬃白
马。乃是辽国天寿公主答里孛,按上界太阴星君。正似玉兔团团离海角,冰轮皎皎
照瑶台。

两队阵中,团团一遭,尽是黄旗簇簇,军将尽骑黄马,都披金甲。衬甲袍起一片黄
云,绣包巾散半天黄雾。黄军队中,有军马大将四员,各领兵三千,分于四角。每
角上一员大将,团团守护。东南一员大将,青袍金甲,手持宝枪,坐骑粉青马,立
于阵前,按上界罗星君,乃是辽国皇侄耶律得荣。西南一员大将,紫袍银甲,使
一口宝刀,坐骑海骝马,立于阵前,按上界计都星君,乃是辽国皇侄耶律得华。东
北一员大将,绿袍银甲,手执方天画戟,坐骑五明黄马,立于阵前,按上界紫星
君,乃是辽国皇侄耶律得忠。西北一员大将,白袍铜甲,手仗七星宝剑,坐骑踢云
乌骓马,立于阵前,按上界月孛星君,乃是辽国皇侄耶律得信。

黄军阵内,簇拥着一员上将,左有执青旗,右有持白钺,前有擎朱,后有张皂盖。
周回旗号,按二十四气,六十四卦,南辰北斗,飞龙飞虎,飞熊飞豹,明分阴阳左
右,暗合璇玑玉衡乾坤混沌之象。那员上将,使一枝朱红画杆方天戟。怎生打扮?
头戴七宝紫金冠,身穿龟背黄金甲,西川红锦绣花袍,蓝田美玉玲珑带。左悬金画
铁胎弓,右带凤翎子箭。足穿鹰嘴云根靴,坐骑铁脊银鬃马。锦雕鞍稳踏金镫,
紫丝缰牢绊山鞒。腰间挂剑驱番将,手内挥鞭统大军。这簇军马光辉,四边浑如金
色,按上界中宫土星一天君,乃是辽国都统军大元帅兀颜光。

黄旗之后,中军是凤辇龙车。前后左右,七重剑戟枪刀围绕。九重之内,又有三十
六对黄巾力士,推捧车驾。前有九骑金鞍骏马驾辕,后有八对锦衣卫士随阵。辇上
中间,坐着辽国郎主:头戴冲天唐巾,身穿九龙黄袍,腰系蓝田玉带,足穿朱履朝
靴。左右两个大臣:左丞相幽西孛瑾,右丞相太师褚坚。各带貂蝉冠,火裙朱服,
紫绶金章,象简玉带。龙床两边,金童玉女,执简捧圭。龙车前后左右两边,簇拥
护驾天兵。辽国郎主,自按上界北极紫微大帝,总领镇星。左右二丞相,按上界左
辅、右弼星君。正是一天星斗离乾位,万象森罗降世间。有诗为证:
宿曜随宜列八方,更将土德镇中央。
胡人从不关天象,何事纷纷渎上苍?

那辽国番军摆列天阵已定,正如鸡卵之形,似覆盆之状,旗排四角,枪摆八方,循
环无定,进退有则。宋江看见,便教强弓硬弩,射住阵脚,就中军竖起云梯将台,
引吴用、朱武上台观望。宋江看了,惊讶不已。朱武看了,认的是天阵,便对宋江、
吴用道:“此乃是太乙混天象阵也!”宋江问道:“如何攻击?”朱武道:“此天
阵变化无穷,机关莫测,不可造次攻打。”宋江道:“若不打得开阵势,如何得他
军退?”吴用道:“急切不知他阵内虚实,如何便去打得?”

正商议间,兀颜统军在中军传令:“今日属金,可差亢金龙张起、牛金牛薛雄、娄
金狗阿里义、鬼金羊王景四将,跟随太白金星大将乌利可安,离阵攻打宋兵。”宋
江众将在阵前,望见对阵右军七门,或开或闭,军中雷响,阵势团团,那引军旗在
阵内自东转北,北转西,西投南。朱武见了,在马上道:“此乃是天盘左旋之象。
今日属金,天盘左动,必有兵来。”说犹未了,五炮齐响,早是对阵踊出军来。中
是金星,四下是四宿,引动五队军马,卷杀过来,势如山倒,力不可当。宋江军马,
措手不及,望后急退。大队压住阵脚,辽兵两面夹攻,宋江大败,急忙退兵,回到
本寨,辽兵也不来追赶。点视军中头领,孔亮伤刀,李云中箭,朱富着炮,石勇着
枪,中伤军卒,不计其数。随即发付上车,去后寨令安道全医治。宋江教前军下了
铁蒺藜,深栽鹿角,坚守寨门。

宋江在中军纳闷,与卢俊义等商议:“今日折了一阵,如之奈何?再若不出交战,
必来攻打。”卢俊义道:“来日着两路军马,撞住他那压阵军兵。再调两路军马,
撞那厮正北七门。却教步军从中间打将入去,且看里面虚实如何。”宋江道:“也
是。”次日便依卢俊义之言,收拾起寨,前至阵前准备,大开寨门,引兵前进。遥
望辽兵不远,六队压阵辽兵远探将来。宋江便差关胜在左,呼延灼在右,引本部军
马,撞退压阵辽兵。大队前进,与辽兵相接,宋江再差花荣、秦明、董平、杨志在
左,林冲、徐宁、索超、朱仝在右,两队军兵来撞皂旗七门。果然撞开皂旗阵势,
杀散皂旗人马,正北七座旗门,队伍不整。宋江阵中,却转过李逵、樊瑞、鲍旭、
项充、李衮五百牌手向前,背后鲁智深、武松、杨雄、石秀、解珍、解宝,将带应
有步军头目,撞杀入去。混天阵内,只听四面炮响,东西两军,正面黄旗军撞杀将
来。宋江军马,抵当不住,转身便走。后面架隔不定,大败奔走,退回原寨。急点
军时,折其大半。杜迁、宋万,又带重伤。于内不见了黑旋风李逵。原来李逵杀的
性起,只顾砍入他阵里去,被他挠钩搭住,活捉去了。宋江在寨中听的,心中纳闷。
传令教先送杜迁、宋万去后寨,令安道全调治。带伤马匹,叫牵去与皇甫端料理。
宋江又与吴用等商议:“今日又折了李逵,输了这一阵,似此怎生奈何?”吴用道:
“前日我这里活捉的他那个小将军,是兀颜统军的孩儿,正好与他打换。”宋江道:
“这番换了,后来倘若折将,何以解救?”吴用道:“兄长何故执迷,且顾眼下。”
说犹未了,小校来报,有辽将遣使到来打话。宋江唤入中军,那番官来与宋江厮见,
说道:“俺奉元帅将令,今日拿得你的一个头目,到俺总兵面前,不肯杀害,好生
与他酒肉,管待在那里。统军要送来与你,换他孩儿小将军还他。如是将军肯时,
便送那个头目来还。”宋江道:“既是恁地,俺明日取小将军来到阵前,两相交换。”
番官领了宋江言语,上马去了。宋江再与吴用商议道:“我等无计破他阵势,不若
取将小将军来,就这里解和这阵,两边各自罢战。”吴用道:“且将军马暂歇,别
生良策,再来破敌,未为晚矣。”到晓,差人星夜去取兀颜小将军来,也差个人直
往兀颜统军处,说知就里。

且说兀颜统军正在帐中坐地,小军来报,宋先锋使人来打话。统军传令,教唤入来。
到帐前,见了兀颜统军,说道:“俺的宋先锋拜意统军麾下,今送小将军回来,换
俺这个头目。即今天气严寒,军士劳苦,两边权且罢战,待来春别作商议,俱免人
马冻伤。请统军将令。”兀颜统军听了大喝道:“无智辱子,被汝生擒,纵使得活,
有何面目见咱?不用相换,便拿下替俺斩了。若要罢战权歇,教你宋江束手来降,
免汝一死。若不如此,吾引大兵一到,寸草不留!”大喝一声:“退去!”使者飞
马回寨,将这话诉与宋江。宋江慌速,只怕救不得李逵,拔寨便起,带了兀颜小将
军直抵前军,隔阵大叫:“可放过俺的头目来,我还你小将军。不罢战不妨,自与
你对阵厮杀。”只见辽兵阵中,无移时,把李逵一骑马送出阵前来。这里也牵一匹
马,送兀颜小将军出阵去。两家如此,一言为定。两边一齐同收同放,李将军回寨,
小将军也骑马过去了。当日两边,都不厮杀。宋江退兵回寨,且与李逵贺喜。

宋江在帐中与诸将相议道:“辽兵势大,无计可破,使我忧煎,度日如年,怎生奈
何?”呼延灼道:“我等来日,可分十队军马,两路去当压阵军兵,八路一齐撞击,
决此一战。”宋江道:“全靠你等众弟兄同心力,来日必行。”吴用道:“两番
撞击不动,不如守等他来交战。”宋江道:“等他来,也不是良法。只是众弟兄当
以力敌,岂有连败之理!”当日传令,次早拔寨起军,分作十队,飞抢前去。两路
先截住后背压阵军兵,八路军马更不打话,呐喊摇旗,撞入混天阵去。听的里面雷
声高举,四七二十八门,一齐分开,变作一字长蛇之阵,便杀出来。宋江军马,措
手不及,急令回军,大败而走,旗枪不整,金鼓偏斜,速退回来。到得本寨,于路
损折军马数多。宋江传令,教军将紧守山口寨栅,深掘濠堑,牢栽鹿角,坚闭不出,
且过冬寒。

却说副枢密赵安抚累次申达文书赴京,奏请索取衣袄等件。因此朝廷特差御前八十
万禁军枪棒教头,正受郑州团练使,姓王,双名文斌。此人文武双全,满朝钦敬,
将带京师一万余人,起差民夫车辆,押运衣袄五十万领,前赴宋先锋军前交割,就
行催并军将,向前交战,早奏凯歌。王文斌领了圣旨文书,将带随行军器,拴束衣
甲鞍马,催人夫军马,起运车仗,出东京,望陈桥驿进发。监押着一二百辆车子,
上插黄旗,书“御赐衣袄”,迤逦前进。经过去处,自有官司供给口粮。在路非则
一日,来到边庭,参见了赵枢密,呈上中书省公文。赵安抚看了大喜道:“将军来
的正好,目今宋先锋被辽国兀颜统军,把兵马摆成混天阵势,连输了数阵。头目人
等中伤者多,现今发在此间将养,令安道全医治。宋先锋扎寨在永清县地方,并不
敢出战,好生纳闷。”王文斌禀道:“朝廷因此就差某来,催并军士向前,早要取
胜。今日既然累败,王某回京师,见省院官,难以回奏。文斌不才,自幼颇读兵书,
略晓些阵法,就到军前,略施小策,愿决一阵,与宋先锋分忧。未知枢相钧命若何?”
赵枢密大喜,置酒宴赏,就军中犒劳押车人夫,就教王文斌转运衣袄,解付宋江军
前给散。赵安抚先使人报知宋先锋去了。

且说宋江在中军帐中纳闷,闻知赵枢密使人来,转报东京差教头郑州团练使王文斌,
押送衣袄五十万领,就来军前,催并进兵。宋江差人接至寨中下马,请入帐内,把
酒接风。数杯酒后,询问缘由。宋江道:“宋某自蒙朝廷差遣到边,上托天子洪福,
得了四个大郡。今到幽州,不想被番邦兀颜统军设此混天象阵,兵屯二十万,整整
齐齐,按周天星象,请启郎主御驾亲征。宋江连败数阵,无计可施,屯驻不敢轻动。
今幸得将军降临,愿赐指教。”王文斌道:“量这个混天阵,何足为奇!王某不才,
同到军前一观,别有主见。”宋江大喜,先令裴宣,且将衣袄给散军将,众人穿罢,
望南谢恩。当日中军置酒,殷勤管待,就行赏劳三军。

来日结束,五军都起。王文斌取过带来的头盔衣甲,全副披挂上马,都到阵前。对
阵辽兵望见宋兵出战,报入中军。金鼓齐鸣,喊声大举,六队战马哨出阵来。宋江
分兵杀退。王文斌上将台亲自看一回,下云梯来说道:“这个阵势,也只如常,不
见有甚惊人之处。”不想王文斌自己不识,且图诈人要誉,便叫前军擂鼓搦战。对
阵番军,也挝鼓鸣金。宋江立马大喝道:“不要狐朋狗党,敢出来挑战么?”说犹
未了,黑旗队里,第四座门内,飞出一将。那番官披头散发,黄罗抹额,衬着金箍
乌油铠甲,秃袖皂袍,骑匹乌骓马,挺三尖刀,直临阵前,背后牙将,不记其数。
引军皂旗上书银字“大将曲利出清”,跃马阵前搦战。王文斌寻思道:“我不就这
里显扬本事,再于何处施逞?”便挺枪跃马出阵,与番官更不打话,骤马相交。王
文斌挺枪便搠,番将舞刀来迎。斗不到二十余合,番将回身便走。王文斌见了,便
骤马飞枪,直赶将去。原来番将不输,特地要卖个破绽,漏他来赶。番将抡起刀,
觑着王文斌较亲,翻身背砍一刀,把王文斌连肩和胸脯,砍做两段,死于马下。宋
江见了,急叫收军。那辽兵撞掩过来,又折了一阵,慌慌忙忙,收拾还寨。众多军
将,看见立马斩了王文斌,面面厮觑,俱各骇然。宋江回到寨中,动纸文书,申复
赵枢密说:“王文斌自愿出战身死,发付带来人伴回京。”赵枢密听知此事,展转
忧闷,甚是烦恼,只得写了申呈奏本,关会省院打发来的人伴回京去了。有诗为证:
赵括徒能读父书,文斌殒命又何愚。
平时夸口千人有,临阵成功一个无。

且说宋江自在寨中纳闷,百般寻思,无计可施,怎生破的辽兵,寝食俱废,梦寐不
安。是夜严冬,天气甚冷,宋江闭上帐房,秉烛沉吟闷坐。时已二鼓,神思困倦,
和衣隐几而卧。觉道寨中狂风忽起,冷气侵人。宋江起身,见一青衣女童,向前打
个稽首。宋江便问:“童子自何而来?”童子答曰:“小童奉娘娘法旨,有请将军,
便烦移步。”宋江道:“娘娘现在何处?”童子指道:“离此间不远。”宋江遂随
童子出的帐房,但见上下天光一色,金碧交加,香风细细,瑞霭飘飘,有如二三月
间天气。行不过三二里多路,见座大林,青松茂盛,翠柏森然,紫桂亭亭,石栏隐
隐,两边都是茂林修竹,垂柳夭桃,曲折阑干。转过石桥,朱红棂星门一座。仰观
四面,萧墙粉壁,画栋雕梁,金钉朱户,碧瓦重檐,四边帘卷虾须,正面窗横龟背。
女童引宋江从左廊下而进,到东向一个阁子前。推开朱户,教宋江里面少坐。举目
望时,四面云窗寂静,霞彩满阶,天花缤纷,异香缭绕。

童子进去,复又出来传旨道:“娘娘有请,星主便行。”宋江坐未暖席,即时起身。
又见外面两个仙女入来,头戴芙蓉碧玉冠,身穿金缕绛绡衣,与宋江施礼。宋江不
敢仰视。那两个仙女道:“将军何故作谦?娘娘更衣便出,请将军议论国家大事,
便请同行。”宋江唯然而行,听的殿上金钟声响,玉磬音鸣。青衣迎请宋江上殿。
二仙女前进,引宋江自东阶而上,行至珠帘之前。宋江只听的帘内玎隐隐,玉
锵锵。青衣请宋江入帘内,跪在香案之前。举目观望殿上,祥云霭霭,紫雾腾腾,
正面九龙床上,坐着九天玄女娘娘。头戴九龙飞凤冠,身穿七宝龙凤绛绡衣,腰系
山河日月裙,足穿云霞珍珠履,手执无瑕白玉圭。两边侍从女仙,约有三二十个。
玄女娘娘与宋江曰:“吾传天书与汝,不觉又早数年矣!汝能忠义坚守,未尝少怠。
今宋天子令汝破辽,胜负如何?”宋江俯伏在地,拜奏曰:“臣自得蒙娘娘赐与天
书,未尝轻慢泄漏于人。今奉天子敕命破辽,不期被兀颜统军设此混天象阵,累败
数次。臣无计可施,正在危急之际。”玄女娘娘曰:“汝知混天象阵法否?”宋江
再拜奏道:“臣乃下土愚人,不晓其法,望乞娘娘赐教。”玄女娘娘曰:“此阵之
法,聚阳象也。只此攻打,永不能破。若欲要破,须取相生相克之理。且如前面皂
旗军马内设水星,按上界北方五辰星。你宋兵中,可选大将七员,黄旗黄甲,黄
衣黄马,撞破辽兵皂旗七门。续后命猛将一员,身披黄袍,直取水星,此乃土克水
之义也。却以白袍军马,选将八员,打透他左边青旗军阵,此乃金克木之义也。却
以红袍军马,选将八员,打透他右边白旗军阵,此乃火克金之义也。却以皂旗军马,
选将八员,打透他后军红旗军阵,此乃水克火之义也。却命一枝青旗军马,选将九
员,直取中央黄旗军阵主将,此乃木克土之义也。再选两枝军马,命一枝绣旗花袍
军马,扮作罗,独破辽兵太阳军阵。命一枝素旗银甲军马,扮作计都,直破辽兵
太阴军阵。再造二十四部雷车,按二十四气,上放火石火炮,直推入辽兵中军。令
公孙胜布起风雷天罡正法,径奔入辽主驾前。可行此计,足取全胜。日间不可行兵,
须是夜黑可进。汝当亲自领兵,掌握中军,催动人马,一鼓成功。吾之所言,汝当
秘受。保国安民,勿生退悔。天凡有限,从此永别。他日琼楼金阙,别当重会。汝
宜速还,不可久留。”特命青衣献茶,宋江吃罢,令青衣即送星主还寨。

宋江再拜,恳谢娘娘,出离殿庭。青衣前引宋江下殿,从西阶而出,转过棂星红门,
再登旧路。才过石桥松径,青衣用手指道:“辽兵在那里,汝当破之!”宋江回顾,
青衣用手一推,猛然惊觉,就帐中做了一梦。

静听军中更鼓,已打四更,宋江便叫请军师圆梦。吴用来到中军帐内,宋江道:“军
师有计破混天阵否?”吴学究道:“未有良策可施。”宋江道:“我已梦玄女娘娘
传与秘诀,寻思定了,特请军师商议,可以会集诸将,分拨行事。”正是:动达天
机施妙策,摆开星斗破迷关。
毕竟宋江怎生打阵,且听下回分解。