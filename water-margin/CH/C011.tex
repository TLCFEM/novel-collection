\chapter{朱贵水亭施号箭~林冲雪夜上梁山}

话说豹子头林冲当夜醉倒在雪里地上,挣扎不起,被众庄客向前绑缚了,解送
来一个庄院。只见一个庄客从院里出来,说道:“大官人未起,众人且把这厮高吊
起在门楼底下。”看天色晓来,林冲酒醒,打一看时,果然好个大庄院。林冲大叫
道:“甚么人敢吊我在这里?”那庄客听得叫,手拿着白木棍,从门里走出来,喝
道:“你这厮还自好口!”那个被烧了髭须的老庄客道:“休要问他,只顾打!等
大官人起来,问明送官。”庄客一齐上,林冲被打,挣扎不得,只叫道:“不要打
我,我自有说处。”只见一个庄客来叫道:“大官人来了。”林冲看时,只见个官
人,背叉着手,行将出来,至廊下问道:“你们在此打甚么人?”众庄客答道:“昨
夜捉得个偷米贼人。”那官人向前来看时,认得是林冲,慌忙喝退庄客,亲自解下,
问道:“教头缘何被吊在这里?”众庄客看见,一齐走了。

林冲看时,不是别人,却是小旋风柴进,连忙叫道:“大官人救我!”柴进道:
“教头为何到此,被村夫耻辱?”林冲道:“一言难尽!”两个且到里面坐下,把
这火烧草料场一事,备细告诉。柴进听罢道:“兄长如此命蹇!今日天假其便,但
请放心,这里是小弟的东庄,且住几时,却再商量。”叫庄客取一笼衣裳出来,叫
林冲彻里至外都换了,请去暖阁里坐地,安排酒食杯盘管待。自此林冲只在柴进东
庄上住了五七日,不在话下。

却说沧州牢城营里管营首告:林冲杀死差拨、陆虞候、富安等三人,放火沿烧
大军草料场。州尹大惊,随即押了公文帖,仰缉捕人员将带做公的,沿乡,历邑,
道店,村坊,四处张挂,出三千贯信赏钱,捉拿正犯林冲。看看挨捕甚紧,各处村
坊讲动了。

且说林冲在柴大官人东庄上,听得个信息紧急,俟候柴进回庄,林冲便说道:
“非是大官人不留小人,只因官司追捕甚紧,排家搜捉;倘或寻到大官人庄上,犹
恐负累大官人不好。既蒙大官人仗义疏财,求借林冲些小盘缠,投奔他处栖身,异
日不死,当效犬马之报。”柴进道:“既是兄长要行,小人有个去处,作书一封与
兄长前去。”正是:
豪杰蹉跎运未通,行藏随处被牢笼。
不因柴进修书荐,焉得驰名水浒中。
林冲道:“若得大官人如此周济,教小人安身立命。只不知投何处去?”柴进道:
“是山东济州管下一个水乡,地名梁山泊,方圆八百余里,中间是宛子城、蓼儿洼。
如今有三个好汉,在那里扎寨。为头的唤做白衣秀士王伦,第二个唤做摸着天杜迁,
第三个唤做云里金刚宋万。那三个好汉,聚集着七八百小喽罗,打家劫舍;多有做
下迷天大罪的人,都投奔那里躲灾避难,他都收留在彼。三位好汉,亦与我交厚,
尝寄书缄来。我今修一封书与兄长,去投那里入伙如何?”林冲道:“若得如此顾
盼最好!”柴进道:“只是沧州道口现今官司张挂榜文,又差两个军官在那里搜检,
把住道口。兄长必用从那里经过。”柴进低头一想道:“再有个计策,送兄长过去。”
林冲道:“若蒙周全,死而不忘!”

柴进当日先叫庄客背了包裹出关去等。柴进却备了三二十匹马,带了弓箭旗枪,
驾了鹰雕,牵着猎狗,一行人马都打扮了,却把林冲杂在里面,一齐上马,都投关
外。却说把关军官坐在关上,看见是柴大官人,却都认得。原来这军官未袭职时,
曾到柴进庄上,因此识熟。军官起身道:“大官人又去快活!”柴进下马问道:“二
位官人缘何在此?”军官道:“沧州太尹行移文书,画影图形,捉拿犯人林冲,特
差某等在此守把。但有过往客商,一一盘问,才放出关。”柴进笑道:“我这一伙
人内中间夹带着林冲,你缘何不认得?”军官也笑道:“大官人是识法度的,不到
得肯夹带了出去?请尊便上马。”柴进又笑道:“只恁地相托得过,拿得野味回来
相送。”作别了,一齐上马出关去了。行得十四五里,却见先去的庄客在那里等候。
柴进叫林冲下了马,脱去打猎的衣服,却穿上庄客带来的自己衣裳,系了腰刀,戴
上红缨毡笠,背上包裹,提了衮刀,相辞柴进,拜别了便行。只说那柴进一行人上
马,自去打猎,到晚方回,依旧过关送些野味与军官,回庄上去了,不在话下。

且说林冲与柴大官人别后,上路行了十数日,时遇暮冬天气,彤云密布,朔风
紧起,又见纷纷扬扬,下着满天大雪。行不到二十余里,只见满地如银。昔金完颜
亮有篇词,名《百字令》,单题着大雪,壮那胸中杀气:

天丁震怒,掀翻银海,散乱珠箔。六出奇花飞滚滚,平填了山中丘壑。皓虎颠
狂,素麟猖獗,掣断珍珠索。玉龙酣战,鳞甲满天飘落。谁念万里关山,征夫僵立,
缟带旗脚。色映戈矛,光摇剑戟,杀气横戎幕。貔虎豪雄,偏裨英勇,共与谈兵
略。须拚一醉,看取碧空寥廓。

话说林冲踏着雪只顾走,看看天色冷得紧切,渐渐晚了。远远望见枕溪靠湖一
个酒店,被雪漫漫地压着。但见:

银迷草舍,玉映茅檐。数十株老树杈,三五处小窗关闭。疏荆篱落,浑如腻
粉轻铺;黄土绕墙,却似铅华布就。千团柳絮飘帘幕,万片鹅毛舞酒旗。

林冲看见,奔入那酒店里来,揭开芦帘,拂身入去,倒侧身看时,都是座头。
拣一处坐下,倚了衮刀,解放包裹,抬了毡笠,把腰刀也挂了。只见一个酒保来问
道:“客官打多少酒?”林冲道:“先取两角酒来。”酒保将个桶儿打两角酒,将
来放在桌上。林冲又问道:“有甚么下酒?”酒保道:“有生熟牛肉、肥鹅、嫩鸡。”
林冲道:“先切二斤熟牛肉来。”酒保去不多时,将来铺下一大盘牛肉,数盘菜蔬,
放个大碗,一面筛酒。林冲吃了三四碗酒,只见店里一个人背叉着手,走出来门前
看雪。那人问酒保道:“甚么人吃酒?”林冲看那人时,头戴深檐暖帽,身穿貂鼠
皮袄,脚着一双獐皮窄靴;身材长大,貌相魁宏;双拳骨脸,三叉黄须,只把头
来摸着看雪。

林冲叫酒保只顾筛酒。林冲说道:“酒保,你也来吃碗酒。”酒保吃了一碗。
林冲问道:“此间去梁山泊还有多少路?”酒保答道:“此间要去梁山泊,虽只数
里,却是水路,全无旱路。若要去时,须用船去,方才渡得到那里。”林冲道:“你
可与我觅只船儿。”酒保道:“这般大雪,天色又晚了,那里去寻船只?”林冲道:
“我多与你些钱,央你觅只船来,渡我过去。”酒保道:“却是没讨处。”林冲寻
思道:“这般却怎的好?”又吃了几碗酒,闷上心来,蓦然想起:“我先在京师做
教头,每日六街三市游玩吃酒,谁想今日被高俅这贼坑陷了我这一场,文了面,直
断送到这里,闪得我有家难奔,有国难投,受此寂寞!”因感伤怀抱,问酒保借笔
砚来,乘着一时酒兴,向那白粉壁上写下八句道:“仗义是林冲,为人最朴忠。江
湖驰誉望,京国显英雄。身世悲浮梗,功名类转蓬。他年若得志,威镇泰山东。”
撇下笔,再取酒来。

正饮之间,只见那个穿皮袄的汉子走向前来,把林冲劈腰揪住,说道:“你好
大胆!你在沧州做下迷天大罪,却在这里!现今官司出三千贯信赏钱捉你,却是要怎
地?”林冲道:“你道我是谁?”那汉道:“你不是豹子头林冲?”林冲道:“我
自姓张。”那汉笑道:“你莫胡说,现今壁上写下名字,你脸上文着金印,如何要
赖得过?”林冲道:“你真个要拿我!”那汉笑道:“我却拿你做甚么?你跟我进
来,到里面和你说话。”那汉放了手,林冲跟着,到后面一个水亭上,叫酒保点起
灯来,和林冲施礼,对面坐下。那汉问道:“却才见兄长只顾问梁山泊路头,要寻
船去,那里是强人山寨,你待要去做甚么?”林冲道:“实不相瞒:如今官司追捕
小人紧急,无安身处,特投这山寨里好汉入伙,因此要去。”那汉道:“虽然如此,
必有个人荐兄长来入伙。”林冲道:“沧州横海郡故友举荐将来。”那汉道:“莫
非小旋风柴进么?”林冲道:“足下何以知之?”那汉道:“柴大官人与山寨中大
王头领交厚,常有书信往来。”原来王伦当初不得第之时,与杜迁投奔柴进,多得
柴进留在庄子上,住了几时。临起身,又赍发盘缠银两,因此有恩。

林冲听了,便拜道:“有眼不识泰山,愿求大名。”那汉慌忙答礼,说道:“小
人是王头领手下耳目,姓朱,名贵,原是沂州沂水县人氏,江湖上但叫小弟做旱地
忽律。山寨里教小弟在此间开酒店为名,专一探听往来客商经过。但有财帛者,便
去山寨里报知。但是孤单客人到此,无财帛的,放他过去;有财帛的,来到这里,
轻则蒙汗药麻翻,重则登时结果,将精肉片为子,肥肉煎油点灯。却才见兄长只
顾问梁山泊路头,因此不敢下手。次后见写出大名来,曾有东京来的人,传说兄长
的豪杰,不期今日得会。既有柴大官人书缄相荐,亦是兄长名震寰海,王头领必当
重用。”随即安排鱼肉、盘馔、酒肴,到来相待。两个在水亭上,吃了半夜酒。林
冲道:“如何能够船来渡过去?”朱贵道:“这里自有船只,兄长放心。且暂宿一
宵,五更却请起来同往。”当时两个各自去歇息。

睡到五更时分,朱贵自来叫林冲起来,洗漱罢,再取三五杯酒相待,吃了些肉
食之类。此时天尚未明,朱贵把水亭上窗子开了,取出一张鹊画弓,搭上那一枝响
箭,觑着对港败芦折苇里面射将去。林冲道:“此是何意?”朱贵道:“此是山寨
里的号箭,少顷便有船来。”没多时,只见对过芦苇泊里三五个小喽罗,摇着一只
快船过来,径到水亭下。朱贵当时引了林冲,取了刀仗行李下船。小喽罗把船摇开,
望泊子里去奔金沙滩来。林冲看时,见那八百里梁山水泊,果然是个陷人去处!但
见:

山排巨浪,水接遥天。乱芦攒万队刀枪,怪树列千层剑戟。濠边鹿角,俱将骸
骨攒成;寨内碗瓢,尽使骷髅做就。剥下人皮蒙战鼓,截来头发做缰绳。阻当官军,
有无限断头港陌;遮拦盗贼,是许多绝径林峦。鹅卵石迭迭如山,苦竹枪森森似雨。
断金亭上愁云起,聚义厅前杀气生。

当时小喽罗把船摇到金沙滩岸边,朱贵同林冲上了岸,小喽罗背了包裹,拿了
刀杖,两个好汉上山寨来。那几个小喽罗,自把船摇到小港里去了。林冲看岸上时,
两边都是合抱的大树,半山里一座断金亭子。再转将过来,见座大关,关前摆着枪、
刀、剑、戟、弓、弩、戈、矛,四边都是擂木炮石。小喽罗先去报知。二人进得关
来,两边夹道遍摆着队伍旗号。又过了两座关隘,方才到寨门口。林冲看见四面高
山,三关雄壮,团团围定。中间里镜面也似一片平地,可方三五百丈。靠着山口,
才是正门,两边都是耳房。

朱贵引着林冲来到聚义厅上,中间交椅上坐着一个好汉,正是白衣秀士王伦。
左边交椅上坐着摸着天杜迁,右边交椅坐着云里金刚宋万。朱贵、林冲向前声喏了。
林冲立在朱贵侧边,朱贵便道:“这位是东京八十万禁军教头,姓林,名冲,绰号
豹子头。因被高太尉陷害,刺配沧州,那里又被火烧了大军草料场,争奈杀死三人,
逃走在柴大官人家,好生相敬。因此,特写书来举荐入伙。”

林冲怀中取书递上,王伦接来拆开看了,便请林冲来坐第四位交椅,朱贵坐了
第五位。一面叫小喽罗取酒来,把了三巡,动问柴大官人近日无恙。林冲答道:“每
日只在郊外猎较乐情。”王伦动问了一回,蓦然寻思道:“我却是个不及第的秀才。
因鸟气,合着杜迁来这里落草;续后宋万来,聚集这许多人马伴当。我又没十分本
事,杜迁、宋万武艺也只平常。如今不争添了这个人,他是京师禁军教头,必然好
武艺。倘若被他识破我们手段,他须占强,我们如何迎敌?不若只是一怪,推却事
故,发付他下山去便了,免致后患。只是柴进面上却不好看,忘了日前之恩,如今
也顾他不得。”正是:
未同豪气岂相求,纵遇英雄不肯留。
秀士自来多嫉妒,豹头空叹觅封侯。

当下王伦叫小喽罗一面安排酒食,整理筵宴,请林冲赴席,众好汉一同吃酒。
将次席终,王伦叫小喽罗把一个盘子,托出五十两白银,两匹丝来。王伦起身说
道:“柴大官人举荐将教头来敝寨入伙,争奈小寨粮食缺少,屋宇不整,人力寡薄,
恐日后误了足下,亦不好看。略有些薄礼,望乞笑留。寻个大寨安身歇马,切勿见
怪。”林冲道:“三位头领容复:小人‘千里投名,万里投主’,凭托柴大官人面
皮,径投大寨入伙。林冲虽然不才,望赐收录。当以一死向前,并无谄佞,实为平
生之幸。不为银两赍发而来,乞头领照察。”王伦道:“我这里是个小去处,如何
安着得你?休怪,休怪。”朱贵见了,便谏道:“哥哥在上,莫怪小弟多言。山寨
中粮食虽少,近村远镇,可以去借。山场水泊木植广有,便要盖千间房屋,却也无
妨。这位是柴大官人力举荐来的人,如何教他别处去?抑且柴大官人自来与山上有
恩,日后得知不纳此人,须不好看。这位又是有本事的人,他必然来出气力。”杜
迁道:“山寨中那争他一个!哥哥若不收留,柴大官人知道时见怪,显的我们忘恩
背义。日前多曾亏了他,今日荐个人来,便恁推却,发付他去!”宋万也劝道:“柴
大官人面上,可容他在这里做个头领也好;不然,见得我们无义气,使江湖上好汉
见笑。”

王伦道:“兄弟们不知,他在沧州虽是犯了迷天大罪,今日上山,却不知心腹。
倘或来看虚实,如之奈何?”林冲道:“小人一身犯了死罪,因此来投入伙,何故
相疑?”王伦道:“既然如此,你若真心入伙,把一个‘投名状’来。”林冲便道:
“小人颇识几字,乞纸笔来便写。”朱贵笑道:“教头你错了。但凡好汉们入伙,
须要纳投名状,是教你下山去杀得一个人,将头献纳,他便无疑心,这个便谓之投
名状。”林冲道:“这事也不难,林冲便下山去等,只怕没人过。”王伦道:“与
你三日限。若三日内有投名状来,便容你入伙;若三日内没时,只得休怪。”林冲
应承了,自回房中宿歇,闷闷不已。正是:
愁怀郁郁若难开,可恨王伦忒弄乖。
明日早寻山路去,不知那个送头来。
当夜席散,朱贵相别下山,自去守店。

林冲到晚,取了刀仗行李,小喽罗引去客房内歇了一夜。次日早起来,吃些茶
饭,带了腰刀,提了朴刀,叫一个小喽罗领路下山,把船渡过去,僻静小路上等候
客人过往。从朝至暮,等了一日,并无一个孤单客人经过。林冲闷闷不已,和小喽
罗再过渡来,回到山寨中。王伦问道:“投名状何在?”林冲答道:“今日并无一
个过往,以此不曾取得。”王伦道:“你明日若无投名状时,也难在这里了。”林
冲再不敢答应,心内自己不乐,来到房中,讨些饭吃了,又歇了一夜。

次日清早起来,和小喽罗吃了早饭,拿了朴刀,又下山来。小喽罗道:“俺们
今日投南山路去等。”两个来到林子里潜伏等候,并不见一个客人过往。伏到午牌
时候,一伙客人约有三百余人,结踪而过,林冲又不敢动手,看他过去。又等了一
歇,看看天色晚来,又不见一个客人过。林冲对小喽罗道:“我恁地晦气!等了两
日,不见一个孤单客人过往,如何是好?”小喽罗道:“哥哥且宽心,明日还有一
日限,我和哥哥去东山路上等候。”当晚依旧上山。王伦说道:“今日投名状如何?”
林冲不敢答应,只叹了一口气。王伦笑道:“想是今日又没了。我说与你三日限,
今已两日了。若明日再无,不必相见了,便请挪步下山,投别处去。”

林冲回到房中,端的是心内好闷,有《临江仙》词一篇云:

闷似蛟龙离海岛,愁如虎困荒田,悲秋宋玉泪涟涟。江淹
初去笔,项羽恨无船。

高祖荥阳遭困厄,昭关伍相忧煎,曹公赤壁火连天。李
陵台上望,苏武陷居延。

当晚林冲仰天长叹道:“不想我今日被高俅那贼陷害,流落到此,天地也不容
我,直如此命蹇时乖!”过了一夜,次日天明起来,讨些饭食吃了,打拴那包裹,
撇在房中,跨了腰刀,提了朴刀,又和小喽罗下山过渡,投东山路上来。林冲道:
“我今日若还取不得投名状时,只得去别处安身立命。”两个来到山下东路林子里
潜伏等候,看看日头中了,又没一个人来。

时遇残雪初晴,日色明朗,林冲提着朴刀对小喽罗道:“眼见得又不济事了。
不如趁早,天色未晚,取了行李,只得往别处去寻个所在。”小校用手指道:“好
了!兀的不是一个人来?”林冲看时,叫声:“惭愧!”只见那个人远远在山坡下
望见行来。待他来得较近,林冲把朴刀捍剪了一下,蓦地跳将出来。那汉子见了林
冲,叫声:“阿也!”撇了担子,转身便走。林冲赶将去,那里赶得上,那汉子闪
过山坡去了。林冲道:“你看,我命苦么!来了三日,甫能等得一个人来,又吃他
走了。”小校道:“虽然不杀得人,这一担财帛,可以抵当。”林冲道:“你先挑
了上山去,我再等一等。”小喽罗先把担儿挑出林去。只见山坡下转出一个大汉来,
林冲见了,说道:“天赐其便。”只见那人挺着朴刀,大叫如雷,喝道:“泼贼,
杀不尽的强徒,将俺行李那里去!洒家正要捉你这厮们,倒来拔虎须。”飞也似踊
跃而来。林冲见他来得势猛,也使步迎他。不是这个人来斗林冲,有分教:梁山泊
内,添几个弄风白额大虫;水浒寨中,辏几只跳涧金睛猛兽。

毕竟来与林冲斗的,正是甚人,且听下回分解。