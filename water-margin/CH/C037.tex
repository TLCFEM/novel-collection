\chapter{没遮拦追赶及时雨~船火儿大闹浔阳江}

话说当下宋江不合将五两银子赍发了那个教师,只见这揭阳镇上众人丛中钻过
这条大汉,睁着眼喝道:“这厮那里学得这些鸟枪棒,来俺这揭阳镇上逞强,我已
分付了众人休睬他,你这厮如何卖弄有钱,把银子赏他,灭俺揭阳镇上的威风!”
宋江应道:“我自赏他银两,却干你甚事?”那大汉揪住宋江喝道:“你这贼配军
敢回我话?”宋江道:“做甚么不敢回你话?”那大汉提起双拳,劈脸打来,宋江
躲个过。那大汉又赶入一步来,宋江却待要和他放对,只见那个使枪棒的教头从人
背后赶将来,一只手揪住那大汉头巾,一只手提住腰胯,望那大汉肋骨上只一兜,
踉跄一交,颠翻在地。那大汉却待挣扎起来,又被这教头只一脚踢翻了。两个公人
劝住教头,那大汉从地下爬将起来,看了宋江和教头说道:“使得使不得,叫你两
个不要慌。”一直望南去了。

宋江且请问:“教头高姓?何处人氏?”教头答道:“小人祖贯河南洛阳人氏,
姓薛,名永,祖父是老种经略相公帐前军官,为因恶了同僚,不得升用,子孙靠使
枪棒卖药度日,江湖上但呼小人病大虫薛永。不敢拜问恩官高姓大名?”宋江道:
“小可姓宋,名江,祖贯郓城县人氏。”薛永道:“莫非山东及时雨宋公明么?”
宋江道:“小可便是。”薛永听罢便拜,宋江连忙扶住道:“少叙三杯如何?”薛
永道:“好!正要拜识尊颜,小人无门得遇兄长。”慌忙收拾起枪棒和药囊,同宋
江便往邻近酒肆内去吃酒。只见酒家说道:“酒肉自有,只是不敢卖与你们吃。”
宋江问道:“缘何不卖与我们吃?”酒家道:“却才和你们厮打的大汉,已使人分
付了:若是卖与你们吃时,把我这店子都打得粉碎。我这里却是不敢恶他。这人是
此间揭阳镇上一霸,谁敢不听他说?”宋江道:“既然恁地,我们去休,那厮必然
要来寻闹。”薛永道:“小人也去店里算了房钱还他,一两日间,也来江州相会。
兄长先行。”宋江又取一二十两银子与了薛永,辞别了自去。

宋江只得自和两个公人也离了酒店,又自去一处吃酒,那店家说道:“小郎已
自都分付了,我们如何敢卖与你们吃?你枉走,甘自费力,不济事。”宋江和两个
公人都则声不得。连连走了几家,都是一般话说。三个来到市梢尽头,见了几家打
火小客店,正待要去投宿,却被他那里不肯相容。宋江问时,都道:“他已着小郎
连连分付去了,不许安着你们三个。”当下宋江见不是话头,三个便拽开脚步,望
大路上走着,看见一轮红日低坠,天色昏暗。但见:

暮烟迷远岫,寒雾锁长空。群星拱皓月争辉,绿水共青山斗碧。疏林古寺,数
声钟韵悠扬;小浦渔舟,几点残灯明灭。枝上子规啼夜月,园中粉蝶宿花丛。

宋江和两个公人见天色晚了,心里越慌。三个商量道:“没来由看使枪棒,恶
了这厮!如今闪得前不巴村,后不着店,却是投那里去宿是好?”只见远远地小路
上望见隔林深处射出灯光来。宋江见了道:“兀那里灯光明处,必有人家,遮莫怎
地陪个小心,借宿一夜,明日早行。”公人看了道:“这灯光处又不在正路上。”
宋江道:“没奈何。虽然不在正路上,明日多行三二里,却打甚么不紧!”三个人
当时落路来,行不到二里多路,林子背后闪出一座大庄院来。

宋江和两个公人来到庄院前敲门,庄客听得,出来开门道:“你是甚人?黄昏
半夜来敲门打户!”宋江陪着小心答道:“小人是个犯罪配送江州的人,今日错过
了宿头,无处安歇,欲求贵庄借宿一宵,来早依例拜纳房金。”庄客道:“既是恁
地,你且在这里少待,等我入去报知庄主太公,可容即歇。”庄客入去通报了,复
翻身出来说道:“太公相请。”宋江和两个公人到里面草堂上参见了庄主太公,太
公分付,教庄客领去门房里安歇,就与他们些晚饭吃。庄客听了,引去门首草房下,
点起一碗灯,教三个歇定了,取三分饭食、羹汤、菜蔬,教他三个吃了。庄客收了
碗碟,自入里面去。两个公人道:“押司,这里又无外人,一发除了行枷,快活睡
一夜,明日早行。”宋江道:“说得是。”当时去了行枷,和两个公人去房外净手,
看见星光满天,又见打麦场边屋后,是一条村僻小路,宋江看在眼里。三个净了手,
入进房里,关上门去睡。宋江和两个公人说道:“也难得这个庄主太公留俺们歇这
一夜。”正说间,听得庄里有人点火把来打麦场上,一到处照看。宋江在门缝里张
时,见是太公引着三个庄客,把火一到处照看。宋江对公人道:“这太公和我父亲
一般,件件都要自来照管。这早晚也未曾去睡,一地里亲自点看。”

正说之间,只听得外面有人叫开庄门,庄客连忙来开了门,放入五七个人来,
为头的手里拿着朴刀,背后的都拿着稻叉棍棒。火把光下,宋江张看时,“那个提
朴刀的,正是在揭阳镇上要打我们的那汉”。宋江又听得那太公问道:“小郎,你
那里去来?和甚人厮打?日晚了,拖枪拽棒?”那大汉道:“阿爹不知,哥哥在家里
么?”太公道:“你哥哥吃得醉了,去睡在后面亭子上。”那汉道:“我自去叫他
起来,我和他赶人。”太公道:“你又和谁合口,叫起哥哥来时,他却不肯干休。
你且对我说这缘故。”那汉道:“阿爹,你不知,今日镇上一个使枪棒卖药的汉子,
叵耐那厮不先来见我弟兄两个,便去镇上撇科卖药,教使枪棒,被我都分付了镇上
的人,分文不要与他赏钱,不知那里走一个囚徒来,那厮做好汉出尖,把五两银子
赏他,灭俺揭阳镇上威风。我正要打那厮,堪恨那卖药的脑揪翻我,打了一顿,又
踢了我一脚,至今腰里还疼。我已教人四下里分付了酒店客店,不许着这厮们吃酒
安歇,先教那厮三个今夜没存身处。随后吃我叫了赌房里一伙人,赶将去客店里,
拿得那卖药的来,尽气力打了一顿,如今把来吊在都头家里。明日送去江边,捆做
一块,抛在江里,出那口鸟气。却只赶这两个公人押的囚徒不着,前面又没客店,
竟不知投那里去宿了。我如今叫起哥哥来,分投赶去,捉拿这厮。”太公道:“我
儿休恁地短命相。他自有银子赏那卖药的,却干你甚事!你去打他做甚么?可知道着
他打了,也不曾伤重。快依我口便罢,休教哥哥得知。你吃人打了,他肯干罢?又
是去害人性命!你依我说,且去房里睡了。半夜三更,莫去敲门打户,激恼村坊。
你也积些阴德。”那汉不顾太公说,拿着朴刀,径入庄内去了。太公随后也赶入去。

宋江听罢,对公人说道:“这般不巧的事,怎生是好?却又撞在他家投宿,我
们只宜走了好。倘或这厮得知,必然吃他害了性命。便是太公不说,庄客如何敢瞒?”
两个公人都道:“说的是,事不宜迟,及早快走。”宋江道:“我们休从大路出去,
掇开屋后一堵壁子出去罢。”两个公人挑了包裹,宋江自提了行枷,便从房里挖开
屋后一堵壁子,三个人便趁星月之下,望林木深处小路上只顾走。正是慌不择路,
走了一个更次,望见前面满目芦花,一派大江,滔滔浪滚,正来到浔阳江边。有诗
为证:
撞入天罗地网来,宋江时蹇实堪哀。
才离黑煞凶神难,又遇丧门白虎灾。
只听得背后喊叫,火把乱明,吹风胡哨赶将来,宋江只叫得苦道:“上苍救一救则
个!”三人躲在芦苇丛中,望后面时,那火把渐近,三人心里越慌,脚高步低,在
芦苇里撞。前面一看,不到天尽头,早到地尽处。定目一观,看见大江拦截,侧边
又是一条阔港。宋江仰天叹道:“早知如此的苦,权且在梁山泊也罢。谁想直断送
在这里!”

宋江正在危急之际,只见芦苇丛中悄悄地忽然摇出一只船来。宋江见了,便叫:
“梢公,且把船来救我们三个,俺与你几两银子。”那梢公在船上问道:“你三个
是甚么人?却走在这里来?”宋江道:“背后有强人打劫我们,一昧地撞在这里。
你快把船来渡我们,我多与你些银两。”那梢公听得多与银两,把船便放拢来,三
个连忙跳上船去,一个公人便把包裹丢下舱里,一个公人便将水火棍开了船。那
梢公一头搭上橹,一面听着包裹落舱,有些好响声,心里暗喜欢。把橹一摇,那只
小船早荡在江心里去。

岸上那伙赶来的人,早赶到滩头,有十数个火把,为头两个大汉,各挺着一条
朴刀,随后有二十余人,各执枪棒,口里叫道:“你那梢公,快摇船拢来!”宋江
和两个公人做一块儿伏在船舱里,说道:“梢公,却是不要拢船,我们自多与你些
银子相谢。”那梢公点头,只不应岸上的人,把船望上水咿咿哑哑的摇将去。那岸
上这伙人大喝道:“你那梢公,不摇拢船来,教你都死!”那梢公冷笑几声,也不
应。岸上那伙人又叫道:“你是那个梢公?直恁大胆!不摇拢来!”那梢公冷笑应道:
“老爷叫做张梢公,你不要咬我鸟。”岸上火把丛中那个长汉说道:“原来是张大
哥,你见我弟兄两个么?”那梢公应道:“我又不瞎,做甚么不见你?”那长汉道:
“你既见我时,且摇拢来和你说话。”那梢公道:“有话明朝来说,趁船的要去得
紧。”那长汉道:“我弟兄两个正要捉这趁船的三个人。”那梢公道:“趁船的三
个都是我家亲眷,衣食父母,请他归去吃碗板刀面子来。”那长汉道:“你且摇拢
来和你商量。”那梢公又道:“我的衣饭,倒摇拢来把与你,倒乐意!”那长汉道:
“张大哥,不是这般说,我弟兄只要捉这囚徒,你且拢来。”那梢公一头摇橹,一
面说道:“我自好几日接得这个主顾,却是不摇拢来,倒吃你接了去!你两个只得
休怪,改日相见。”宋江不晓得梢公话里藏阄,在船舱里悄悄的和两个公人说:“也
难得这个梢公救了我们三个性命。又与他分说,不要忘了他恩德。却不是幸得这只
船来渡了我们。”

却说那梢公摇开船去,离得江岸远了,三个人在舱里望岸上时,火把也自去芦
苇中明亮。宋江道:“惭愧!正是‘好人相逢,恶人远离’。且得脱了这场灾难。”
只见那梢公摇着橹,口里唱起湖州歌来。唱道:
老爷生长在江边,不怕官司不怕天。
昨夜华光来趁我,临行夺下一金砖。
宋江和两个公人听了这首歌,都酥软了。宋江又想道:“他是唱耍。”三个正在那
里议论未了,只见那梢公放下橹,说道:“你这个撮鸟,两个公人,平日最会诈害
做私商的人,今日却撞在老爷手里!你三个却是要吃板刀面?却是要吃馄饨?”宋江
道:“家长休要取笑!怎地唤做板刀面?怎地是馄饨?”那梢公睁着眼道:“老爷和
你耍甚鸟!若还要吃板刀面时,俺有一把泼风也似快刀在这板底下,我不消三刀
五刀,我只一刀一个,都剁你三个人下水去;你若要吃馄饨时,你三个快脱了衣裳,
都赤条条地跳下江里自死。”宋江听罢,扯定两个公人说道:“却是苦也!正是‘福
无双至,祸不单行’。”那梢公喝道:“你三个好好商量,快回我话。”宋江答道:
“梢公不知,我们也是没奈何,犯下了罪,迭配江州的人,你如何可怜见饶了我三
个!”那梢公喝道:“你说甚么闲话!饶你三个!我半个也不饶你。老爷唤做有名的
狗脸张爷爷,来也不认得爹,去也不认得娘。你便都闭了鸟嘴,快下水里去!”宋
江又求告道:“我们都把包裹内金银、财帛、衣服等项,尽数与你,只饶了我三人
性命。”那梢公便去板底下摸出那把明晃晃板刀来,大喝道:“你三个要怎地?”
宋江仰天叹道:“为因我不敬天地,不孝父母,犯下罪责,连累了你两个。”那两
个公人也扯着宋江道:“押司,罢,罢!我们三个一处死休。”那梢公又喝道:“你
三个好好快脱了衣裳,跳下江去。跳便跳,不跳时,老爷便剁下水里去。”

宋江和那两个公人抱做一块,恰待要跳水,只见江面上咿咿哑哑橹声响,宋江
探头看时,一只快船飞也似从上水头摇将下来。船上有三个人,一条大汉手里横着
托叉,立在船头上;梢头两个后生,摇着两把快橹,星光之下,早到面前。那船头
上横叉的大汉便喝道:“前面是甚么梢公,敢在当港行事?船里货物,见者有分。”
这船梢公回头看了,慌忙应道:“原来却是李大哥,我只道是谁来。大哥又去做买
卖,只是不曾带挈兄弟。”大汉道:“张家兄弟,你在这里又弄这一手!船里甚么
行货?有些油水么?”梢公答道:“教你得知好笑。我这几日没道路,又赌输了,
没一文,正在沙滩上闷坐,岸上一伙人赶着三头行货来我船里。却是鸟两个公人,
解一个黑矮囚徒,正不知是那里人。他说道,迭配江州来的,却又项上不带行枷。
赶来的岸上一伙人,却是镇上穆家哥儿两个,定要讨他,我见有些油水吃,我不还
他。”船上那大汉道:“咄!莫不是我哥哥宋公明?”宋江听得声音厮熟,便舱里
叫道:“船上好汉是谁?救宋江则个!”那大汉失惊道:“真个是我哥哥,早不做
出来。”宋江钻出船上来看时,星光明亮,那立在船头上的大汉,不是别人,正是:

家住浔阳江浦上,最称豪杰英雄。眉浓眼大面皮红,髭须垂铁线,语话若铜钟。
凛凛身躯长八尺,能挥利剑霜锋,冲波跃浪立奇功。庐州生李俊,绰号混江龙。
那船头上立的大汉,正是混江龙李俊。背后船梢上两个摇橹的,一个是出洞蛟童威,
一个是翻江蜃童猛。

这李俊听得是宋公明,便跳过船来,口里叫苦道:“哥哥惊恐。若是小弟来得
迟了些个,误了仁兄性命。今日天使李俊在家坐立不安,棹船出来江里,赶些私盐,
不想又遇着哥哥在此受难!”那梢公呆了半晌,做声不得,方才问道:“李大哥,
这黑汉便是山东及时雨宋公明么?”李俊道:“可知是哩!”那梢公便拜道:“我
那爷,你何不早通个大名!省得着我做出歹事来,争些儿伤了仁兄。”宋江问李俊
道:“这个好汉是谁?高姓何名?”李俊道:“哥哥不知,这个好汉却是小弟结义
的兄弟,原是小孤山下人氏,姓张,名横,绰号船火儿,专在此浔阳江做这件稳善
的道路。”宋江和两个公人都笑起来。

当时两只船并着摇奔滩边来,缆了船,舱里扶宋江并两个公人上岸。李俊又与
张横说道:“兄弟,我常和你说,天下义士,只除非山东及时雨郓城宋押司,今日
你可仔细认看。”张横敲开火石,点起灯来,照着宋江,扑翻身,又在沙滩上拜道:
“望哥哥恕兄弟罪过!”宋江看那张横时,但见:

七尺身躯三角眼,黄髯赤发红睛,浔阳江上有声名。冲波如水怪,跃浪似飞鲸,
恶水狂风都不惧,蛟龙见处魂惊。天差列宿害生灵。小孤山下住,船火号张横。

张横拜罢问道:“义士哥哥为何事配来此间?”李俊便把宋江犯罪的事说了,
今来迭配江州。张横听了说道:“好教哥哥得知,小弟一母所生的亲弟兄两个,长
的便是小弟,我有个兄弟,却又了得。浑身雪练也似一身白肉,没得四五十里水面,
水底下伏得七日七夜,水里行一似一根白条,更兼一身好武艺。因此人起他一个异
名,唤做浪里白跳张顺。当初我弟兄两个,只在扬子江边做一件依本分的道路。”
宋江道:“愿闻则个。”张横道:“我弟兄两个,但赌输了时,我便先驾一只船渡
在江边净处做私渡。有那一等客人贪省贯百钱的,又要快,便来下我船。等船里都
坐满了,却教兄弟张顺也扮做单身客人,背着一个大包,也来趁船。我把船摇到半
江里,歇了橹,抛了钉,插一把板刀,却讨船钱,本合五百足钱一个人,我便定要
他三贯。却先问兄弟讨起,教他假意不肯还我,我便把他来起手,一手揪住他头,
一手提定腰胯,扑通地撺下江里,排头儿定要三贯,一个个都惊得呆了,把出来不
迭。都敛得足了,却送他到僻净处上岸。我那兄弟自从水底下走过对岸,等没了人,
却与兄弟分钱去赌。那时我两个只靠这件道路过日。”宋江道:“可知江边多有主
顾来寻你私渡!”李俊等都笑起来。张横又道:“如今我弟兄两个都改了业,我便
只在这浔阳江里做些私商。兄弟张顺,他却如今自在江州做卖鱼牙子。如今哥哥去
时,小弟寄一封书去;只是不识字,写不得。”李俊道:“我们去村里央个门馆先
生来写。”留下童威、童猛看船。三个人跟了李俊,张横提了灯,投村里来。

走不过半里路,看见火把还在岸上明亮。张横说道:“他弟兄两个还未归去。”
李俊道:“你说兀谁弟兄两个?”张横道:“便是镇上那穆家哥儿两个。”李俊道:
“一发叫他两个来拜见哥哥。”宋江连忙说道:“使不得,他两个赶着要捉我。”
李俊道:“仁兄放心,他弟兄不知是哥哥。他亦是我们一路人。”李俊用手一招,
胡哨了一声,只见火把人伴都飞奔将来。看见李俊、张横都恭奉着宋江做一处说话,
那弟兄二人大惊道:“二位大哥如何与这三人厮熟?”李俊大笑道:“你道他是兀
谁?”那二人道:“便是不认得。只见他在镇上出银两赏那使枪棒的,灭俺镇上威
风,正待要捉他。”李俊道:“他便是我日常和你们说的山东及时雨郓城宋押司公
明哥哥,你两个还不快拜。”那弟兄两个撇了朴刀,扑翻身便拜道:“闻名久矣,
不期今日方得相会。却才甚是冒渎,犯伤了哥哥,望乞怜悯恕罪。”宋江扶起二位
道:“壮士,愿求大名。”李俊便道:“这弟兄两个富户,是此间人:姓穆,名弘,
绰号没遮拦;兄弟穆春,唤做小遮拦。是揭阳镇上一霸。我这里有三霸,哥哥不知,
一发说与哥哥知道。揭阳岭上岭下,便是小弟和李立一霸;揭阳镇上,是他弟兄两
个一霸;浔阳江边做私商的,却是张横、张顺两个一霸。以此谓之三霸。”宋江答
道:“我们如何省得?既然都是自家弟兄情分,望乞放还了薛永。”穆弘笑道:“便
是使枪棒的那厮?哥哥放心,随即便教兄弟穆春去取来还哥哥。我们且请仁兄到敝
庄伏礼请罪。”李俊说道:“最好,最好!便到你庄上去。”穆弘叫庄客着两个去
看了船只,就请童威、童猛一同都到庄上去相会。一面又着人去庄上报知,置办酒
食,杀羊宰猪,整理筵宴。

一行众人等了童威、童猛,一同取路投庄上来,却好五更天气。都到庄里,请
出穆太公来相见了,就草堂上分宾主坐下。宋江看那穆弘时,端的好表人物。但见:

面似银盆身似玉,头圆眼细眉单,威风凛凛逼人寒。灵官
离斗府,佑圣下天关。武艺高强心胆大,阵前不肯空还,攻城野战夺旗。穆弘真
壮士,人号没遮拦。

宋江与穆太公对坐。说话未久,天色明朗,穆春已取到病大虫薛永进来,一处
相会了。穆弘安排筵席,管待宋江等众位饮宴,至晚都留在庄上歇宿。次日,宋江
要行,穆弘那里肯放,把众人都留庄上,陪侍宋江去镇上闲玩,观看揭阳市村景致。
又住了三日,宋江怕违了限次,坚意要行,穆弘并众人苦留不住,当日做个送路筵
席。次日早起来,宋江作别穆太公并众位好汉,临行分付薛永,且在穆弘处住几时,
却来江州,再得相会。穆弘道:“哥哥但请放心,我这里自看顾他。”取出一盘金
银,送与宋江,又赍发两个公人些银两。临动身,张横在穆弘庄上央人修了一封家
书,央宋江付与张顺,当时宋江收放包裹内了。一行人都送到浔阳江边。穆弘叫只
船来,取过先头行李下船。众人都在江边,安排行枷,取酒食上船饯行,当下众人
洒泪而别。李俊、张横、穆弘、穆春、薛永、童威、童猛一行人,各自回家,不在
话下。

只说宋江自和两个公人下船投江州来。这梢公非比前番,拽起一帆风篷,早送
到江州上岸。宋江依前带上行枷,两个公人取出文书,挑了行李,直至江州府前来,
正值府尹升厅。原来那江州知府,姓蔡,双名得章,是当朝蔡太师蔡京的第九个儿
子,因此江州人叫他做蔡九知府。那人为官贪滥,作事骄奢。为这江州是个钱粮浩
大的去处,抑且人广物盈,因此太师特地教他来做个知府。

当时两个公人当厅下了公文,押宋江投厅下。蔡九知府看见宋江一表非俗,便
问道:“你为何枷上没了本州的封皮?”两个公人告道:“于路上春雨淋漓,却被
水湿坏了。”知府道:“快写个帖来,便送下城外牢城营里去,本府自差公人押解
下去。”这两个公人就送宋江到牢城营内交割。当时江州府公人赍了文帖,监押宋
江并同公人,出州衙,前来酒店里买酒吃。宋江取三两来银子,与了江州府公人,
当讨了收管,将宋江押送单身房里听候。那公人先去对管营差拨处替宋江说了方便,
交割,讨了收管,自回江州府去了。这两个公人也交还了宋江包裹行李,千酬万谢,
相辞了入城来。两个自说道:“我们虽是吃了惊恐,却赚得许多银两。”自到州衙
府里伺候,讨了回文,两个取路往济州去了。

话里只说宋江又自央浼人情,差拨到单身房里,送了十两银子与他,管营处又
自加倍送十两并人事,营里管事的人,并使唤的军健人等,都送些银两与他们买茶
吃,因此无一个不欢喜宋江。少刻引到点视厅前,除了行枷,参见管营,为得了贿
赂,在厅上说道:“这个新配到犯人宋江听着:先朝太祖武德皇帝圣旨事例,但凡
新入流配的人,须先吃一百杀威棒,左右与我捉去背起来。”宋江告道:“小人于
路感冒风寒时症,至今未曾痊可。”管营道:“这汉端的似有病的,不见他面黄肌
瘦,有些病症。且与他权寄下这顿棒。此人既是县吏出身,着他本营抄事房做个抄
事。”就时立了文案,便教发去抄事。宋江谢了,去单身房取了行李,到抄事房安
顿了。众囚徒见宋江有面目,都买酒来与他庆贺。次日,宋江置备酒食,与众人回
礼。不时间,又请差拨牌头递杯,管营处常常送礼物与他。宋江身边有的是金银财
帛,自落的结识他们。住了半月之间,满营里没一个不欢喜他。自古道:“世情看
冷暖,人面逐高低。”

宋江一日与差拨在抄事房吃酒,那差拨说与宋江道:“贤兄,我前日和你说的
那个节级常例人情,如何多日不使人送去与他?今已一旬之上了。他明日下来时,
须不好看。”宋江道:“这个不妨。那人要钱,不与他。若是差拨哥哥但要时,只
顾问宋江取不妨。那节级要时,一文也没。等他下来,宋江自有话说。”差拨道:
“押司,那人好生利害,更兼手脚了得。倘或有些言语高低,吃了他些羞辱,却道
我不与你通知。”宋江道:“兄长由他,但请放心,小可自有措置。敢是送些与他,
也不见得。他有个不敢要我的,也不见得。”正恁的说未了,只见牌头来报道:“节
级下在这里了,正在厅上大发作,骂道:‘新到配军,如何不送常例钱来与我!’”
差拨道:“我说是么,那人自来,连我们都怪。”宋江笑道:“差拨哥哥休罪,不
及陪侍,改日再得作杯。小可且去和他说话。”差拨也起身道:“我们不要见他。”
宋江别了差拨,离了抄事房,自来点视厅上,见这节级。

不是宋江来和这人厮见,有分教:江州城里,翻为虎窟狼窝;十字街头,变作
尸山血海。直教:撞破天罗归水浒,掀开地网上梁山。

毕竟宋江来与这个节级怎么相见,且听下回分解。