\chapter{解珍解宝双越狱~孙立孙新大劫牢}

话说当时吴学究对宋公明说道:“今日有个机会,却是石勇面上来投入伙的人,
又与栾廷玉那厮最好,亦是杨林、邓飞的至爱相识。他知道哥哥打祝家庄不利,特
献这条计策来入伙,以为进身之报,随后便至。五日之内,可行此计,却是好么?”
宋江听了,大喜道:“妙哉!”方才笑逐颜开。说话的,却是甚么计策?下来便见。
看官牢记这段话头。原来和宋公明初打祝家庄时,一同事发。却难这边说一句,那
边说一回,因此权记下这两打祝家庄的话头,却先说那一回来投入伙的人乘机会的
话,下来接着关目。

原来山东海边有个州郡,唤做登州。登州城外有一座山,山上多有豺狼虎豹,
出来伤人,因此登州知府拘集猎户,当厅委了杖限文书,捉捕登州山上大虫。又仰
山前山后里正之家,也要捕虎文状,限外不行解官,痛责枷号不恕。且说登州山下
有一家猎户,兄弟两个,哥哥唤做解珍,兄弟唤做解宝。弟兄两个,都使浑铁点钢
叉,有一身惊人的武艺。当州里的猎户们,都让他第一。那解珍一个绰号唤做两头
蛇,这解宝绰号叫做双尾蝎。二人父母俱亡,不曾婚娶。那哥哥七尺以上身材,紫
棠色面皮,腰细膀阔;这个兄弟解宝,更是利害,也有七尺以上身材,面圆身黑,
两只腿上刺着两个飞天夜叉,有时性起,恨不得腾天倒地,拔树摇山。有一篇《西
江月》,单道他弟兄的好处:

世本登州猎户,生来骁勇英豪。穿山越岭健如猱,麋鹿见
时惊倒。

手执莲花铁,腰悬蒲叶尖刀。豹皮裙子虎筋绦,解氏二难年少。

那弟兄两个当官受了甘限文书,回到家中,整顿窝弓药箭,弩子叉,穿了豹
皮裤,虎皮套体,拿了铁叉,两个径奔登州山上,下了窝弓,去树上等了一日,不
济事了,收拾窝弓下去。次日,又带了干粮,再上山伺候,看看天晚,弟兄两个再
把窝弓下了,爬上树去,直等到五更,又没动静。两个移了窝弓,却来西山边下了,
坐到天明,又等不着。两个心焦,说道:“限三日内要纳大虫,迟时须用受责,却
是怎地好!”

两个到第三日夜,伏至四更时分,不觉身体困倦。两个背厮靠着且睡,未曾合
眼,忽听得窝弓发响。两个跳将起来,拿了钢叉,四下里看时,只见一个大虫中了
药箭,在那地上滚。两个拈着钢叉向前来,那大虫见了人来,带着箭便走。两个追
将向前去,不到半山里时,药力透来,那大虫当不住,吼了一声,骨滚将下山
去了。解宝道:“好了!我认得这山,是毛太公庄后园里,我和你下去他家取讨大
虫。”

当时弟兄两个提了钢叉,径下山来,投毛太公庄上敲门。此时方才天明,两个
敲开庄门入去,庄客报与太公知道。多时,毛太公出来,解珍、解宝放下钢叉,声
了喏,说道:“伯伯,多时不见,今日特来拜扰。”毛太公道:“贤侄如何来得这
等早?有甚话说?”解珍道:“无事不敢惊动伯伯睡寝。如今小侄因为官司委了甘
限文书,要捕获大虫,一连等了三日,今早五更,射得一个,不想从后山滚下,在
伯伯园里,望烦借一路,取大虫则个。”毛太公道:“不妨,既是落在我园里,二
位且少坐,敢是肚饥了,吃些早饭去取。”叫庄客且去安排早膳来相待。当时劝二
位吃了酒饭,解珍、解宝起身谢道:“感承伯伯厚意,望烦引去,取大虫还小侄。”
毛太公道:“既是在我庄后,却怕怎地?且坐吃茶,却去取未迟。”解珍、解宝不
敢相违,只得又坐下,庄客拿茶来,叫二位吃了。毛太公道:“如今我和贤侄去取
大虫。”解珍、解宝道:“深谢伯伯。”

毛太公引了二人,入到庄后,叫庄客把钥匙来开门,百般开不开。毛太公道:
“这园多时不曾有人来开,敢是锁锈了,因此开不得,去取铁锤来打开了罢。”
庄客便将铁锤来,敲开了锁,众人都入园里去看时,遍山边去看,寻不见。毛太公
道:“贤侄,你两个莫不错看了,认不仔细?敢不曾落在我园里?”解珍道:“怎
地得我两个错看了?是这里生长的人,如何不认得?”毛太公道:“你自寻便了,
有时自抬去。”解宝道:“哥哥,你且来看,这里一带草,滚得平平地都倒了;又
有血路在上头,如何说不在这里?必是伯伯家庄客抬过了。”毛太公道:“你休这
等说,我家庄上的人,如何得知有大虫在园里?便又抬得过?你也须看见方才当面敲
开锁来,和你两个一同入园里来寻。你如何这般说话!”解珍道:“伯伯,你须还
我这个大虫去解官。”毛太公道:“你这两个好无道理!我好意请你吃酒饭,你颠
倒赖我大虫。”解宝道:“有甚么赖处!你家也现当里正,官府中也委了甘限文书,
却没本事去捉,倒来就我现成,你倒将去请功,教我兄弟两个吃限棒。”毛太公道:
“你吃限棒,干我甚事!”解珍、解宝睁起眼来,便道:“你敢教我搜一搜么?”
毛太公道:“我家比你家,各有内外。你看这两个教化头倒来无礼。”解宝抢近厅
前寻不见,心中火起,便在厅前打将起来;解珍也就厅前攀折栏杆,打将入去。毛
太公叫道:“解珍、解宝白昼抢劫!”那两个打碎了厅前椅桌,见庄上都有准备,
两个便拔步出门,指着庄上骂道:“你赖我大虫,和你官司里去理会。”
解氏深机捕获,毛家巧计牢笼。
当日因争一虎,后来引起双龙。

那两个正骂之间,只见两三匹马投庄上来,引着一伙伴当。解珍认得是毛太公
儿子毛仲义,接着说道:“你家庄上庄客捉过了我大虫,你爹不讨还我,颠倒要打
我弟兄两个。”毛仲义道:“这厮村人不省事,我父亲必是被他们瞒过了。你两个
不要发怒,随我到家里,讨还你便了。”解珍、解宝谢了毛仲义,叫开庄门,教他
两个进去。待得解珍、解宝入得门来,便叫关上庄门,喝一声:“下手!”两廊下
走出二三十个庄客,并恰才马后带来的,都是做公的。那兄弟两个措手不及,众人
一发上,把解珍、解宝绑了。毛仲义道:“我家昨夜自射得一个大虫,如何来白赖
我的?乘势抢掳我家财,打碎家中什物,当得何罪?解上本州,也与本州除了一害。”
原来毛仲义五更时,先把大虫解上州里去了,却带了若干做公的来捉解珍、解宝。
不想他这两个不识局面,正中了他的计策,分说不得。毛太公教把他两个使的钢叉
并一包赃物,扛抬了许多打碎的家伙什物,将解珍、解宝剥得赤条条地,背剪绑了,
解上州里来。本州有个六案孔目,姓王,名正,却是毛太公的女婿,已自先去知府
面前禀说了。才把解珍、解宝押到厅前,不由分说,捆翻便打,定要他两个招做混
赖大虫,各执钢叉,因而抢掳财物。解珍、解宝吃拷不过,只得依他招了。知府教
取两面二十五斤的重枷来枷了,钉下大牢里去。毛太公、毛仲义自回庄上商议道:
“这两个男女,却放他不得,不如一发结果了他,免致后患。”当时子父二人自来
州里,分付孔目王正:“与我一发斩草除根,萌芽不发,我这里自行与知府的打关
节。”

却说解珍、解宝押到死囚牢里,引至亭心上来,见这个节级。为头的那人,姓
包,名吉,已自得了毛太公银两,并听信王孔目之言,教对付他两个性命,便来亭
心里坐下。小牢子对他两个说道:“快过来,跪在亭子前。”包节级喝道:“你两
个便是甚么两头蛇、双尾蝎,是你么?”解珍道:“虽然别人叫小人们这等混名,
实不曾陷害良善。”包节级喝道:“你这两个畜生,今番我手里教你两头蛇做一头
蛇,双尾蝎做单尾蝎,且与我押入大牢里去。”

那一个小牢子把他两个带在牢里来,见没人,那小节级便道:“你两个认得我
么?我是你哥哥的妻舅。”解珍道:“我只亲弟兄两个,别无那个哥哥。”那小牢
子道:“你两个须是孙提辖的兄弟。”解珍道:“孙提辖是我姑舅哥哥,我却不曾
与你相会。足下莫非是乐和舅?”那小节级道:“正是,我姓乐,名和,祖贯茅州
人氏。先祖挈家到此,将姐姐嫁与孙提辖为妻。我自在此州里勾当,做小牢子。人
见我唱得好,都叫我做铁叫子乐和。姐夫见我好武艺,教我学了几路枪法在身。”
怎见得,有诗为证:
玲珑心地衣冠整,俊俏肝肠语话清。
能唱人称铁叫子,乐和聪慧自天生。

原来这乐和是一个聪明伶俐的人,诸般乐品,尽皆晓得,学着便会。作事见头
知尾。说起枪棒武艺,如糖似蜜价爱。为见解珍、解宝是个好汉,有心要救他,只
是单丝不成线,孤掌岂能鸣,只报得他一个信。乐和说道:“好教你两个得知:如
今包节级得受了毛太公钱财,必然要害你两个性命,你两个却是怎生好?”解珍道:
“你不说起孙提辖则休,你既说起他来,只央你寄一个信。”乐和道:“你却教我
寄信与谁?”解珍道:“我有个姐姐,是我爷面上的,却与孙提辖兄弟为妻,现在
东门外十里牌住。他是我姑娘的女儿,叫做母大虫顾大嫂,开张酒店,家里又杀牛
开赌。我那姐姐有三二十人近他不得,姐夫孙新这等本事,也输与他。只有那个姐
姐,和我弟兄两个最好。孙新、孙立的姑娘,却是我母亲,以此他两个又是我姑舅
哥哥。央烦的你暗暗地寄个信与他,把我的事说知,姐姐必然自来救我。”

乐和听罢,分付说:“贤亲,你两个且宽心着。”先去藏些烧饼肉食,来牢里
开了门,把与解珍、解宝吃了。推了事故,锁了牢门,教别个小节级看守了门,一
径奔到东门外,望十里牌来。早望见一个酒店,门前悬挂着牛羊等肉,后面屋下一
簇人在那里赌博。乐和见酒店里一个妇人坐在柜上,但见:

眉粗眼大,胖面肥腰。插一头异样钗露两个时兴钏镯。有时怒起,提井栏便
打老公头;忽地心焦,拿石锥敲翻庄客腿。生来不会拈针线,弄棒持枪当女工。
乐和入进店内,看着顾大嫂,唱个喏道:“此间姓孙么?”顾大嫂慌忙答道:“便
是。足下却要沽酒,却要买肉?如要赌钱,后面请坐。”乐和道:“小人便是孙提
辖妻弟乐和的便是。”顾大嫂笑道:“原来却是乐和舅,可知尊颜和姆姆一般模样。
且请里面拜茶。”乐和跟进里面客位里坐下,顾大嫂便动问道:“闻知得舅舅在州
里勾当,家下穷忙少闲,不曾相会。今日甚风吹得到此?”乐和答道:“小人无事,
也不敢来相恼。今日厅上偶然发下两个罪人进来,虽不曾相会,多闻他的大名。一
个是两头蛇解珍,一个是双尾蝎解宝。”顾大嫂道:“这两个是我的兄弟,不知因
甚罪犯下在牢里?”乐和道:“他两个因射得一个大虫,被本乡一个财主毛太公赖
了,又把他两个强扭做贼,抢掳家财,解入州里来。他又上上下下都使了钱物,早
晚间要教包节级牢里做翻他两个,结果了性命。小人路见不平,独力难救。只想一
者沾亲,二乃义气为重,特地与他通个消息。他说道:‘只除是姐姐便救得他。’
若不早早用心着力,难以救拔。”

顾大嫂听罢,一片声叫起苦来,便叫火家快去寻得二哥家来说话。有几个火家
去不多时,寻得孙新归来,与乐和相见。怎见得孙新的好处,有诗为证:
军班才俊子,眉目有神威。
身在蓬莱寓,家从琼海移。
自藏鸿鹄志,恰配虎狼妻。
鞭举龙双见,枪来蟒独飞。
年似孙郎少,人称小尉迟。

原来这孙新祖是琼州人氏,军官子孙,因调来登州驻扎,弟兄就此为家。孙新
生得身长力壮,全学得他哥哥的本事,使得几路好鞭枪,因此多人把他弟兄两个比
尉迟恭,叫他做小尉迟。顾大嫂把上件事对孙新说了,孙新道:“既然如此,叫舅
舅先回去。他两个已下在牢里,全望舅舅看觑则个。我夫妻商量个长便道理,却径
来相投。”乐和道:“但有用着小人处,尽可出力向前。”顾大嫂置酒相待已了,
将出一包碎银,付与乐和:“望烦舅舅将去牢里,散与众人并小牢子们,好生周全
他两个弟兄。”乐和谢了,收了银两,自回牢里来替他使用,不在话下。

且说顾大嫂和孙新商议道:“你有甚么道理,救我两个兄弟?”孙新道:“毛
太公那厮,有钱有势,他防你两个兄弟出来,须不肯干休,定要做翻了他两个,似
此必然死在他手。若不去劫牢,别样也救他不得。”顾大嫂道:“我和你今夜便去。”
孙新笑道:“你好粗卤。我和你也要算个长便,劫了牢,也要个去向。若不得我那
哥哥,和这两个人时,行不得这件事。”顾大嫂道:“这两个是谁?”孙新道:“便
是那叔侄两个最好赌的邹渊、邹润,如今现在登云山台峪里,聚众打劫。他和我最
好,若得他两个相帮助,此事便成。”顾大嫂道:“登云山离这里不远,你可连夜
去请他叔侄两个来商议。”孙新道:“我如今便去。你可收拾了酒食肴馔,我去定
请得来。”顾大嫂分付火家,宰了一口猪,铺下数盘果品按酒,排下桌子。

天色黄昏时候,只见孙新引了两筹好汉归来。那个为头的姓邹,名渊,原是莱
州人氏,自小最好赌钱,闲汉出身,为人忠良慷慨,更兼一身好武艺,性气高强,
不肯容人,江湖上唤他绰号出林龙。第二个好汉,名唤邹润,是他侄儿,年纪与叔
叔仿佛,二人争差不多,身材长大,天生一等异相,脑后一个肉瘤,以此人都唤他
做独角龙。那邹润往常但和人争闹,性起来,一头撞去,忽然一日,一头撞折了涧
边一株松树,看的人都惊呆了。有《西江月》一首,单道他叔侄的好处:

厮打场中为首,呼卢队里称雄。天生忠直气如虹,武艺惊人出众。

结寨登
云台上,英名播满山东。翻江搅海似双龙,岂作池中玩弄?

当时顾大嫂见了,请入后面屋下坐地,却把上件事告诉与他,次后商量劫牢一
节。邹渊道:“我那里虽有八九十人,只有二十来个心腹的。明日干了这件事,便
是这里安身不得了。我却有个去处,我也有心要去多时,只不知你夫妇二人肯去
么?”顾大嫂道:“遮莫甚么去处,都随你去,只要救了我两个兄弟。”邹渊道:
“如今梁山泊十分兴旺,宋公明大肯招贤纳士。他手下现有我的三个相识在彼:一
个是锦豹子杨林,一个是火眼狻猊邓飞,一个是石将军石勇,都在那里入伙了多时。
我们救了你两个兄弟,都一发上梁山泊投奔入伙去如何?”顾大嫂道:“最好,有
一个不去的,我便乱枪戳死他。”邹润道:“还有一件,我们倘或得了人,诚恐登
州有些军马追来,如之奈何?”孙新道:“我的亲哥哥现做本州军马提辖,如今登
州只有他一个了得。几番草寇临城,都是他杀散了,到处闻名。我明日自去请他来,
要他依允便了。”邹渊道:“只怕他不肯落草。”孙新说道:“我自有良法。”

当夜吃了半夜酒,歇到天明,留下两个好汉在家里,却使一个火家带领了一两
个人,推一辆车子,“快走城中营里,请我哥哥孙提辖并嫂嫂乐大娘子,说道:‘家
中大嫂害病沉重,便烦来家看觑。’”顾大嫂分付火家道:“只说我病重临危,有
几句紧要的话,须是便来,只有几番相见嘱付。”火家推车儿去了。孙新专在门前
伺候,等接哥哥。饭罢时分,远远望见车儿来了,载着乐大娘子,背后孙提辖骑着
马,十数个军汉跟着,望十里牌来。孙新入去报与顾大嫂得知,说:“哥嫂来了。”
顾大嫂分付道:“只依我如此行。”孙新出来,接见哥嫂,且请嫂嫂下了车儿,同
到房里,看视弟媳妇病症。

孙提辖下了马,入门来,端的好条大汉,淡黄面皮,落腮胡须,八尺以上身材,
姓孙,名立,绰号病尉迟,射得硬弓,骑得劣马,使一管长枪,腕上悬一条虎眼竹
节钢鞭,海边人见了,望风而降。有诗为证:
胡须黑雾飘,性格流星急。
鞭枪最熟惯,弓箭常温习。
阔脸似妆金,双睛如点漆。
军中显姓名,病尉迟孙立。

当下病尉迟孙立下马来,进得门便问道:“兄弟,婶子害甚么病?”孙新答道:
“他害得症候病得跷蹊,请哥哥到里面说话。”孙立便入来。孙新分付火家,着这
伙跟马的军士去对门店里吃酒,便教火家牵过马,请孙立入到里面来坐下。良久,
孙新道:“请哥哥、嫂嫂去房里看病。”孙立同乐大娘子入进房里,见没有病人,
孙立问道:“婶子病在那里房内?”只见外面走入顾大嫂来,邹渊、邹润跟在背后。
孙立道:“婶子,你正是害甚么病?”顾大嫂道:“伯伯拜了。我害些救兄弟的病。”
孙立道:“却又作怪,救甚么兄弟?”顾大嫂道:“伯伯,你不要推聋妆哑。你在
城中,岂不知道他两个是我兄弟,偏不是你的兄弟?”孙立道:“我并不知因由。
是那两个兄弟?”

顾大嫂道:“伯伯在上,今日事急,只得直言拜禀:这解珍、解宝被登云山下
毛太公与同王孔目设计陷害,早晚要谋他两个性命。我如今和这两个好汉商量已定,
要去城中劫牢,救出他两个兄弟,都投梁山泊入伙去,恐怕明日事发,先负累伯伯,
因此我只推患病,请伯伯、姆姆到此说个长便。若是伯伯不肯去时,我们自去上梁
山泊去了。如今朝廷有甚分晓,走了的倒没事,见在的便吃官司。常言道:‘近火
先焦。’伯伯便替我们吃官司坐牢,那时又没人送饭来救你。伯伯尊意如何?”孙
立道:“我却是登州的军官,怎地敢做这等事!”顾大嫂道:“既是伯伯不肯,我
们今日先和伯伯并个你死我活。”顾大嫂身边便掣出两把刀来,邹渊、邹润各拔出
短刀在手。孙立叫道:“婶子且住!休要急速行,我从长计较,慢慢地商量。”乐
大娘子惊得半晌做声不得。顾大嫂又道:“既是伯伯不肯去时,即便先送姆姆前行,
我们自去下手。”孙立道:“虽要如此行时,也待我归家去收拾包裹行李,看个虚
实,方可行事。”顾大嫂道:“伯伯,你的乐阿舅透风与我们了。一就去劫牢,一
就去取行李不迟。”孙立叹了一口气,说道:“你众人既是如此行了,我怎地推却
得开?不成日后倒要替你们吃官司?罢,罢,罢!都做一处商议了行。”先叫邹渊去
登云山寨里收拾起财物人马,带了那二十个心腹的人,来店里取齐,邹渊去了。又
使孙新入城里来,问乐和讨信,就约会了,暗通消息解珍、解宝得知。

次日,登云山寨里邹渊收拾金银已了,自和那起人到来相助。孙新家里也有七
八个知心腹的火家,并孙立带来的十数个军汉,共有四十余人。孙新宰了两口猪,
一腔羊,众人尽吃了一饱。顾大嫂贴肉藏了尖刀,扮做个送饭的妇人先去。孙新跟
着孙立,邹渊领了邹润,各带了火家,分作两路入去。正是:
捉虎翻成纵虎灾,虎官虎吏枉安排。
全凭铁叫通关节,始得牢城铁瓮开。

且说登州府牢里包节级得了毛太公钱物,只要陷害解珍、解宝的性命。当日乐
和拿着水火棍,正立在牢门里狮子口边,只听得拽铃子响,乐和道:“甚么人?”
顾大嫂应道:“送饭的妇人。”乐和已自瞧科了,便来开门,放顾大嫂入来,再关
了门。将过廊下去,包节级正在亭心里,看见便喝道:“这妇人是甚么人?敢进牢
里来送饭?自古狱不通风。”乐和道:“这是解珍、解宝的姐姐,自来送饭。”包
节级喝道:“休要教他入去,你们自与他送进去便了。”乐和讨了饭,却来开了牢
门,把与他两个。解珍、解宝问道:“舅舅夜来所言的事如何?”乐和道:“你姐
姐入来了,只等前后相应。”乐和便把匣床与他两个开了。只听的小牢子入来报道:
“孙提辖敲门,要走入来。”包节级道:“他自是营官,来我牢里有何事干?休要
开门!”顾大嫂一踅,踅下亭心边去。外面又叫道:“孙提辖焦躁了打门。”包节
级忿怒,便下亭心来,顾大嫂大叫一声:“我的兄弟在那里?”身边便掣出两把明
晃晃尖刀来。包节级见不是头,望亭心外便走。解珍、解宝提起枷,从牢眼里钻将
出来,正迎着包节级。包节级措手不及,被解宝一枷梢打重,把脑盖擗得粉碎。当
时顾大嫂手起,早戳翻了三五个小牢子,一齐发喊,从牢里打将出来。孙立、孙新
把两个当住了,见四个从牢里出来,一发望州衙前便走。邹渊、邹润早从州衙里提
出王孔目头来。街市上人大喊起,先奔出城去。孙提辖骑着马,弯着弓,搭着箭,
压在后面。街上人家都关上门,不敢出来,州里做公的人,认得是孙提辖,谁敢向
前拦当。众人簇拥着孙立,奔出城门去,一直望十里牌来,扶搀乐大娘子上了车儿。
顾大嫂上了马,帮着便行。解珍、解宝对众人道:“叵耐毛太公老贼冤家,如何不
报了去?”孙立道:“说得是。”便令兄弟孙新与舅舅乐和先护持车儿前行着,“我
们随后赶来。”孙新、乐和簇拥着车儿先行去了。

孙立引着解珍、解宝、邹渊、邹润,并火家伴当,一径奔毛太公庄上来,正值
毛仲义与太公在庄上庆寿饮酒,却不提备。一伙好汉呐声喊,杀将入去,就把毛太
公、毛仲义,并一门老小,尽皆杀了,不留一个。去卧房里搜检得十数包金银财宝,
后院里牵得七八匹好马,把四匹捎带驮载,解珍、解宝拣几件好的衣服穿了,将庄
院一把火,齐放起烧了。各人上马,带了一行人,赶不到三十里路,早赶上车仗人
马,一处上路行程。于路庄户人家,又夺得三五匹好马,一行星夜奔上梁山泊去。
有《西江月》为证:

忠义立身之本,奸邪坏国之端。狼心狗幸滥居官,致使英雄扼腕。

夺虎机
谋可恶,劫牢计策堪观。登州城廓痛悲酸,顷刻横尸遍满。
不一二日,来到石勇酒店里,那邹渊与他相见了,问起杨林、邓飞二人。石勇答言,
说起宋公明去打祝家庄,二人都跟去,两次失利,听得报来说,杨林、邓飞俱被陷
在那里,不知如何。备闻祝家庄三子豪杰,又有教师铁棒栾廷玉相助,因此二次打
不破那庄。孙立听罢,大笑道:“我等众人来投大寨入伙,正没半分功劳,献此一
条计策打破祝家庄,为进身之报如何?”石勇大喜道:“愿闻良策。”孙立道:“栾
廷玉那厮,和我是一个师父教的武艺。我学的枪刀,他也知道;他学的武艺,我也
尽知。我们今日只做登州对调来郓州守把,经过来此相望,他必然出来迎接。我们
进身入去,里应外合,必成大事。此计如何?”正与石勇说计未了,只见小校报道:
“吴学究下山来,前往祝家庄救应去。”石勇听得,便叫小校快去报知军师,请来
这里相见。说犹未了,已有军马来到店前,乃是吕方、郭盛,并阮氏三雄,随后军
师吴用带领五百人马到来。石勇接入店内,引着这一行人都相见了,备说投托入伙,
献计一节。吴用听了大喜,说道:“既然众位好汉肯作成山寨,且休上山,便烦请
往祝家庄行此一事,成全这段功劳如何?”孙立等众人皆喜,一齐都依允了。吴用
道:“小生今去,也如此见阵,我人马前行,众位好汉随后一发便来。”

吴学究商议已了,先来宋江寨中,见宋公明眉头不展,面带忧容,吴用置酒与
宋江解闷,备说起石勇、杨林、邓飞三个的一起相识,是登州兵马提辖病尉迟孙立,
和这祝家庄教师栾廷玉是一个师父教的。今来共有八人,投托大寨入伙,特献这条
计策,以为进身之报。今已计较定了,里应外合,如此行事,随后便来参见兄长。
宋江听说罢,大喜,把愁闷都撇在九霄云外,忙叫寨内置酒,安排筵席等来相待。

却说孙立教自己的伴当人等,跟着车仗人马,投一处歇下,只带了解珍、解宝、
邹渊、邹润、孙新、顾大嫂、乐和,共是八人,来参宋江,都讲礼已毕,宋江置酒
设席管待,不在话下。吴学究暗传号令与众人,教第三日如此行,第五日如此行。
分付已了,孙立等众人领了计策,一行人自来和车仗人马投祝家庄进身行事。

再说吴学究道:“启动戴院长到山寨里走一遭,快与我取将这四个头领来,我
自有用他处。”不是教戴宗连夜来取这四个人来,有分教:水泊重添新羽翼,山庄
无复旧衣冠。

毕竟吴学究取那四个人来,且听下回分解。