\chapter{急先锋东郭争功~青面兽北京斗武}

话说当时周谨、杨志两个勒马,在于旗下,正欲出战交锋,只见兵马都监闻达
喝道:“且住!”自上厅来禀复梁中书道:“复恩相:论这两个比试武艺,虽然未
见本事高低,枪刀本是无情之物,只宜杀贼剿寇,今日军中自家比试,恐有伤损,
轻则残疾,重则致命,此乃于军不利。可将两根枪去了枪头,各用毡片包裹,地下
蘸了石灰,再各上马,都与皂衫穿着。但是枪杆厮搠,如白点多者,当输。”梁中
书道:“言之极当。”随即传令下去。

两个领了言语,向这演武厅后去了枪尖,都用毡片包了,缚成骨朵,身上各换
了皂衫,各用枪去石灰桶里蘸了石灰,再各上马,出到阵前。那周谨跃马挺枪,直
取杨志;这杨志也拍战马,拈手中枪,来战周谨。两个在阵前,来来往往,番番覆
覆,搅做一团,扭做一块,鞍上人斗人,坐下马斗马,两个斗了四五十合。看周谨
时,恰似打翻了豆腐的,斑斑点点,约有三五十处;看杨志时,只有左肩牌下一点
白。梁中书大喜,叫唤周谨上厅,看了迹道:“前官参你做个军中副牌,量你这般
武艺,如何南征北讨?怎生做得正请受的副牌?”教杨志替此人职役。管军兵马都
监李成上厅禀复梁中书道:“周谨枪法生疏,弓马熟闲,不争把他来逐了职事,恐
怕慢了军心。再教周谨与杨志比箭如何?”梁中书道:“言之极当。”再传下将令
来,叫杨志与周谨比箭。

两个得了将令,都扎了枪,各关了弓箭。杨志就弓袋内取出那张弓来,扣得端
正;擎了弓,跳上马,跑到厅前,立在马上,欠身禀复道:“恩相,弓箭发处,事
不容情,恐有伤损,乞请钧旨。”梁中书道:“武夫比试,何虑伤残?但有本事,
射死勿论。”杨志得令,回到阵前。李成传下言语,叫两个比箭好汉,各关与一面
遮箭牌,防护身体。两个各领遮箭防牌,绾在臂上。杨志说道:“你先射我三箭,
后却还你三箭。”周谨听了,恨不得把杨志一箭射个透明。杨志终是个军官出身,
识破了他手段,全不把他为事。怎见得两个比箭:

这个曾向山中射虎,那个惯从风里穿杨。彀满处,兔狐丧命;箭发时,雕鹗魂
伤。较艺术,当场比并;施手段,对众揄扬。一个磨秋解,实难抵当;一个闪身解,
不可提防。顷刻内要观胜负,霎时间便见存亡。

当时将台上早把青旗麾动,杨志拍马望南边去,周谨纵马赶来,将缰绳搭在马
鞍鞒上,左手拿着弓,右手搭上箭,拽得满满地望杨志后心飕地一箭。杨志听得背
后弓弦响,霍地一闪,去镫里藏身,那枝箭早射个空。周谨见一箭射不着,却早慌
了,再去壶中急取第二枝箭来,搭上弓弦,觑的杨志较亲,望后心再射一箭。杨志
听得第二枝箭来,却不去镫里藏身,那枝箭风也似来,杨志那时也取弓在手,用弓
梢只一拨,那枝箭滴溜溜拨下草地里去了。周谨见第二枝箭又射不着,心里越慌。
杨志的马早跑到教场尽头,霍地把马一兜,那马便转身望正厅上走回来;周谨也把
马只一勒,那马也跑回,就势里赶将来去。那绿茸茸芳草地上,八个马蹄翻盏撒钹
相似,勃喇喇地风团儿也似般走。周谨再取第三枝箭,搭在弓弦上,扣得满满地,
尽平生气力,眼睁睁地看着杨志后心窝上只一箭射将来。杨志听得弓弦响,扭回身,
就鞍上把那枝箭只一绰,绰在手里,便纵马入演武厅前,撇下周谨的箭。

梁中书见了大喜,传下号令,却叫杨志也射周谨三箭。将台上又把青旗麾动,
周谨撇了弓箭,拿了防牌在手,拍马望南而走。杨志在马上把腰只一纵,略将脚一
拍,那马泼喇喇的便赶。杨志先把弓虚扯一扯,周谨在马上听得脑后弓弦响,扭转
身来,便把防牌来迎,却早接个空。周谨寻思道:“那厮只会使枪,不会射箭。等
他第二枝箭再虚诈时,我便喝住了他,便算我赢了。”周谨的马早到教场南尽头,
那马便转望演武厅来。杨志的马见周谨马跑转来,那马也便回身。杨志早去壶中掣
出一枝箭来,搭弓在弦上,心里想道:“射中他后心窝,必至伤了他性命。他和我
又没冤仇,洒家只射他不致命处便了。”左手如托太山,右手如抱婴孩,弓开如满
月,箭去似流星,说时迟,那时快,一箭正中周谨左肩。周谨措手不及,翻身落马。
那匹空马直跑过演武厅背后去了。众军卒自去救那周谨去了。梁中书见了大喜,叫
军政司便呈文案来,教杨志截替了周谨职役。

杨志喜气洋洋,下了马,便向厅前来拜谢恩相,充其职役。正是:
得罪幽燕作配兵,当场比试死相争。
能将一箭穿杨手,夺得牌军半职荣。
不想阶下左边转上一个人来叫道:“休要谢职,我和你两个比试!”杨志看那人时,
身材七尺以上长短,面圆耳大,唇阔口方,腮边一部落腮胡须,威风凛凛,相貌堂
堂,直到梁中书面前声了喏,禀道:“周谨患病未痊,精神不在,因此误输与杨志。
小将不才,愿与杨志比试武艺,如若小将折半点便宜与杨志,休教截替周谨,便教
杨志替了小将职役,虽死而不怨。”梁中书看时,不是别人,却是大名府留守司正
牌军索超。为是他性急,撮盐入火,为国家面上,只要争气,当先厮杀,以此人都
叫他做急先锋。李成听得,便下将台来,直到厅前禀复道:“相公,这杨志既是殿
司制使,必然好武艺,须知周谨不是对手;正好与索正牌比试武艺,便见优劣。”
梁中书听了,心中想道:“我指望一力要抬举杨志,众将不伏;一发等他赢了索超,
他们也死而无怨,却无话说。”

梁中书随即唤杨志上厅问道:“你与索超比试武艺如何?”杨志禀道:“恩相
将令,安敢有违。”梁中书道:“既然如此,你去厅后换了装束,好生披挂,教甲
仗库随行官吏取应用军器给与,就叫牵我的战马借与杨志骑,小心在意,休觑得等
闲。”杨志谢了,自去结束。

却说李成分付索超道:“你却难比别人,周谨是你徒弟,先自输了。你若有些
疏失,吃他把大名府军官都看得轻了。我有一匹惯曾上阵的战马,并一副披挂,都
借与你,小心在意,休教折了锐气。”索超谢了,也自去结束。

梁中书起身,走出阶前来,从人移转银交椅,直到月台栏干边放下,梁中书坐
定,左右祗候两行,唤打伞的撑开那把银葫芦顶茶褐罗三檐凉伞来,盖定在梁中书
背后。将台上传下将令,早把红旗招动。两边金鼓齐鸣,发一通擂,去那教场中两
阵内,各放了个炮。炮响处,索超跑马入阵内,藏在门旗下;杨志也从阵里跑马入
军中,直到门旗背后。将台上又把黄旗招动,又发了一通擂,两军齐呐一声喊;教
场中谁敢做声,静荡荡的;再一声锣响,扯起净平白旗,两下众官没一个敢走动胡
言说话,静静地立着。

将台上又把青旗招动,只见第三通战鼓响处,去那左边阵内门旗下看看分开,
鸾铃响处,正牌军索超出马直到阵前,兜住马,拿军器在手,果是英雄豪杰。但见:
头带一顶熟钢狮子盔,脑后斗大来一颗红缨;身披一副铁叶攒成铠甲,腰系一条镀
金兽面束带,前后两面青铜护心镜;上笼着一领绯红团花袍,上面垂两条绿绒缕颔
带;下穿一双斜皮气跨靴;左带一张弓,右悬一壶箭;手里横着一柄金蘸斧;坐下
李都监那匹惯战能征雪白马。看那马时,又是一匹好马。但见:

色按庚辛,仿佛南山白额虎;毛堆腻粉,如同北海玉麒麟。冲得阵,跳得溪,
喜战鼓,性如君子;负得重,走得远,惯嘶风,必是龙媒。胜如伍相梨花马,赛过
秦王白玉驹。
左阵上急先锋索超兜住马,着金蘸斧,立马在阵前。

右边阵内门旗下看看分开,鸾铃响处,杨志提手中枪出马,直至阵前,勒住马,
横着枪在手,果是勇猛。但见头戴一顶铺霜耀日镔铁盔,上撒着一把青缨;身穿一
副钩嵌梅花榆叶甲,系一条红绒打就勒甲绦,前后兽面掩心;上笼着一领白罗生色
花袍,垂着条紫绒飞带;脚登一双黄皮衬底靴;一张皮靶弓,数根凿子箭;手中挺
着浑铁点钢枪;骑的是梁中书那匹火块赤千里嘶风马。看那马时,又是匹无敌的好
马。但见:

分火焰,尾摆朝霞。浑身乱扫胭脂,两耳对攒红叶。侵晨临紫塞,马蹄迸四
点寒星;日暮转沙堤,就地滚一团火块。休言南极神驹,真乃寿亭赤兔。
右阵上青面兽杨志拈手中枪,勒坐下马,立于阵前,两边军将暗暗地喝采,虽不知
武艺如何,先见威风出众。

正南上旗牌官拿着销金令字旗,骤马而来,喝道:“奉相公钧旨,教你两个俱
各用心,如有亏误处,定行责罚;若是赢时,多有重赏。”二人得令,纵马出阵,
都到教场中心,两马相交,二般兵器并举。索超忿怒,抡手中大斧,拍马来战杨志;
杨志逞威,拈手中神枪,来迎索超。两个在教场中间,将台前面,二将相交,各赌
平生本事。一来一往,一去一回,四条臂膊纵横,八只马蹄撩乱。但见:

征旗蔽日,杀气遮天。一个金蘸斧直奔顶门,一个浑铁枪不离心坎。这个是扶
持社稷沙门,托塔李天王;那个是整顿江山掌金阙,天蓬大元帅。一个枪尖上吐
一条火焰,一个斧刃中迸几道寒光。那个是七国中袁达重生,这个是三分内张飞出
世。一个是巨灵神忿怒,挥大斧劈碎山根;一个如华光藏生嗔,仗金枪搠开地府。
这个圆彪彪睁开双眼,查查斜砍斧头来;那个必剥剥咬碎牙关,火焰焰摇得枪杆
断。各人窥破绽,
那放半些闲。
两个斗到五十余合,不分胜败。月台上梁中书看得呆了,两边众军官看了,喝采不
迭,阵面上军士们递相厮觑道:“我们做了许多年军,也曾出了几遭征,何曾见这
等一对好汉厮杀!”李成、闻达在将台上,不住声叫道:“好斗!”闻达心上只恐
两个内伤了一个,慌忙招呼旗牌官,拿着令字旗,与他分了。将台上忽的一声锣响,
杨志和索超斗到是处,各自要争功,那里肯回马。旗牌官飞来叫道:“两个好汉歇
了,相公有令。”杨志、索超方才收了手中军器,勒坐下马,各跑回本阵来,立马
在旗下,看那梁中书,只等将令。

李成、闻达下将台来,直到月台下,禀复梁中书道:“相公,据这两个武艺,
一般皆可重用。”梁中书大喜,传下将令,唤杨志、索超。牌旗中传令,唤两个到
厅前,都下了马,小校接了二人的军器,两个都上厅来,躬身听令。梁中书叫取两
锭白银,两副表里,来赏赐二人,就叫军政司将两个都升做管军提辖使,便叫贴了
文案,从今日便参了他两个。索超、杨志都拜谢了梁中书,将着赏赐下厅来,解了
枪刀弓箭,卸了头盔衣甲,换了衣裳。索超也自去了披挂,换了锦祆,都上厅来,
再拜谢了众军官。梁中书叫索超、杨志两个也见了礼,入班做了提辖。众军卒便打
着得胜鼓,把着那金鼓旗先散。

梁中书和大小军官,都在演武厅上筵宴。看看红日沉西,筵席已罢,梁中书上
了马,众官员都送归府。马头前摆着这两个新参的提辖,上下肩都骑着马,头上亦
都带着红花,迎入东郭门来。两边街道扶老携幼,都看了欢喜。梁中书在马上问道:
“你那百姓,欢喜为何?”众老人都跪了禀道:“老汉等生在北京,长在大名府,
不曾见今日这等两个好汉将军比试。今日教场中看了这般敌手,如何不欢喜?”梁
中书在马上听了大喜,回到府中,众官各自散了。索超自有一班弟兄请去作庆饮酒;
杨志新来,未有相识,自去梁府宿歇,早晚殷勤听候使唤,都不在话下。

且把这闲话丢过,只说正话。自东郭演武之后,梁中书十分爱惜杨志,早晚与
他并不相离,月中又有一分请受,自渐渐地有人来结识他。那索超见了杨志手段高
强,心中也自钦伏。不觉光阴迅速,又早春尽夏来,时逢端午,蕤宾节至,梁中书
与蔡夫人在后堂家宴,庆贺端阳。但见:

盆栽绿艾,瓶插红榴。水晶帘卷虾须,锦绣屏开孔雀。菖蒲切玉,佳人笑捧紫
霞杯;角黍堆银,美女高擎青玉案。食烹异品,果献时新。葵扇风中,奏一派声清
韵美;荷衣香里,出百般舞态娇姿。

当日梁中书正在后堂与蔡夫人家宴,庆赏端阳,酒至数杯,食供两套,只见蔡
夫人道:“相公自从出身,今日为一统帅,掌握国家重任,这功名富贵从何而来?”
梁中书道:“世杰自幼读书,颇知经史,人非草木,岂不知泰山之恩,提携之力,
感激不尽!”蔡夫人道:“丈夫既知我父亲恩德,如何忘了他生辰?”梁中书道:
“下官如何不记得,泰山是六月十五日生辰,已使人将十万贯收买金珠宝贝,送上
京师庆寿。一月之前,干人都关领去了。现今九分齐备,数日之间,也待打点停当,
差人起程。只是一件,在此踌躇。上年收买了许多玩器并金珠宝贝,使人送去,不
到半路,尽被贼人劫了,枉费了这一遭财物,至今严捕贼人不获。今年叫谁人去好?”
蔡夫人道:“帐前现有许多军校,你选择知心腹的人去便了。”梁中书道:“尚有
四五十日,早晚催并礼物完足,那时选择去人未迟。夫人不必挂心,世杰自有理会。”
当日家宴,午牌至二更方散,自此不在话下。

不说梁中书收买礼物玩器,选人上京去庆贺蔡太师生辰,且说山东济州郓城县
新到任一个知县,姓时,名文彬,此人为官清正,作事廉明,每怀恻隐之心,常有
仁慈之念。争田夺地,辨曲直而后施行;闲殴相争,分轻重方才决断。闲暇时抚琴
会客,忙迫里飞笔判词。名为县之宰官,实乃民之父母。

当日知县时文彬升厅公座,左右两边排着公吏人等。知县随即叫唤尉司捕盗官
员并两个巡捕都头。本县尉司管下有两个都头:一个唤做步兵都头,一个唤做马兵
都头。这马兵都头,管着二十匹坐马弓手,二十个土兵;那步兵都头管着二十个使
枪的头目,二十个土兵。这马兵都头姓朱名仝,身长八尺四五,有一部虎须髯,长
一尺五寸,面如重枣,目若朗星,似关云长模样,满县人都称他做美髯公。原是本
处富户,只因他仗义疏财,结识江湖上好汉,学得一身好武艺。怎见的朱仝气象,
但见:

义胆忠肝豪杰,胸中武艺精通,超群出众果英雄。弯弓能射虎,提剑可诛龙。
一表堂堂神鬼怕,形容凛凛威风。面如重枣色通红,云长重出世,人号美髯公。

那步兵都头姓雷名横,身长七尺五寸,紫棠色面皮,有一部扇圈胡须,为他膂
力过人,跳二三丈阔涧,满县人都称他做插翅虎。原是本县打铁匠人出身,后来开
张碓房,杀牛放赌,虽然仗义,只有些心地匾窄,也学得一身好武艺。怎见得雷横
的气象,但见:

天上罡星临世上,就中一个偏能,都头好汉是雷横。拽拳神臂健,飞脚电光生。
江海英雄推武勇,跳墙过涧身轻,豪雄谁敢与相争!山东插翅虎,寰海尽闻名。

那朱仝、雷横两个,专管擒拿贼盗。当日知县呼唤两个上厅来,声了喏,取台
旨。知县道:“我自到任以来,闻知本府济州管下所属水乡梁山泊贼盗聚众打劫,
拒敌官军。亦恐各处乡村盗贼猖狂,小人甚多,今唤你等两个,休辞辛苦,与我将
带本管土兵人等,一个出西门,一个出东门,分投巡捕。若有贼人,随即剿获申解,
不可扰动乡民。体知东溪村山上有株大红叶树,别处皆无,你们众人采几片来县里
呈纳,方表你们曾巡到那里。若无红叶,便是汝等虚妄,定行责罚不恕。”两个都
头领了台旨,各自回归,点了本管土兵,分投自去巡察。

不说朱仝引人出西门自去巡捕,只说雷横当晚引了二十个土兵出东门,绕村巡
察,遍地里走了一遭,回来到东溪村山上,众人采了那红叶,就下村来。行不到三
二里,早到灵官庙前,见殿门不关,雷横道:“这殿里又没有庙祝,殿门不关,莫
不有歹人在里面么?我们直入去看一看。”众人拿着火,一齐照将入来,只见供桌
上赤条条地睡着一个大汉。天道又热,那汉子把些破衣裳团做一块作枕头,枕在项
下,的沉睡着了在供桌上。雷横看了道:“好怪,好怪!知县相公忒神明,原
来这东溪村真个有贼!”大喝一声,那汉却待要挣扎,被二十个土兵一齐向前,把
那汉子一条索绑了,押出庙门,投一个保正庄上来。不是投那个去处,有分教:东
溪村里,聚三四筹好汉英雄;郓城县中,寻十万贯金珠宝贝。正是:天上罡星来聚
会,人间地煞得相逢。

毕竟雷横拿住那汉,投解甚处来,且听下回分解。