\chapter{托塔天王梦中显圣~浪里白跳水上报冤}

话说宋江军中,因这一场大雪,吴用定出这条计策,就这雪中捉了索超。其余
军马,都逃入城去,报说索超被擒。梁中书听得这个消息,不由他不慌,传令教众
将只是坚守,不许出战。意欲杀了卢俊义、石秀,犹恐激恼了宋江,朝廷急无兵马
救应,其祸愈速;只得教监守着二人,再行申报京师,听凭蔡太师处分。

且说宋江到寨,中军帐上坐下,早有伏兵解索超到麾下。宋江见了大喜,喝退
军健,亲解其缚,请入帐中,致酒相待,用好言抚慰道:“你看我众兄弟们,一大
半都是朝廷军官,盖为朝廷不明,纵容滥官当道,污吏专权,酷害良民,都情愿协
助宋江,替天行道。若是将军不弃,同以忠义为主。”杨志向前另叙一礼,又细劝
了一番。索超本是天罡星之数,自然凑合,降了宋江。当夜帐中置酒作贺。

次日,商议打城,一连打了数日,不得城破。宋江好生忧闷。当夜帐中伏枕而
卧,忽然阴风飒飒,寒气逼人。宋江抬头看时,只见天王晁盖欲进不进,叫声:“兄
弟,你不回去,更待何时?”立在面前。宋江吃了一惊,急起身问道:“哥哥从何
而来?屈死冤仇,不曾报得,中心日夜不安。前者一向不曾致祭,以此显灵,必有
见责。”晁盖道:“非为此也。兄弟靠后,阳气逼人,我不敢近前。今特来报你,
贤弟有百日血光之灾,则除江南地灵星可治。你可早早收兵,此为上计。”宋江却
欲再问明白,赶向前去说道:“哥哥阴魂到此,望说真实。”被晁盖一推,撒然觉
来,却是南柯一梦。便叫小校请军师圆梦。吴用来到中军帐上,宋江说其异事。吴
用道:“既是晁天王显圣,不可不依。目今天寒地冻,军马难以久住,权且回山守
待。冬尽春初,雪消冰解,那时再来打城,亦未为晚。”宋江道:“军师之言甚当。
只是卢员外和石秀兄弟陷在缧绁,度日如年,只望我等弟兄来救。不争我们回去,
诚恐这厮们害他性命。此事进退两难。”计议未定。

次日,只见宋江觉道神思疲倦,身体酸疼,头如斧劈,身似笼蒸,一卧不起。
众头领都到面前看视。宋江道:“我只觉背上好生热疼。”众人看时,只见鏊子一
般红肿起来。吴用道:“此疾非痈即疽。吾看方书,绿豆粉可以护心,毒气不能侵
犯,便买此物,安排与哥哥吃。”一面使人寻药医治,亦不能好。只见浪里白跳张
顺说道:“小弟旧在浔阳江时,因母得患背疾,百药不能得治,后请得建康府安道
全,手到病除。向后小弟但得些银两,便着人送去与他。今见兄长如此病症,此去
东途路远,急速不能便到。为哥哥的事,只得星夜前去,拜请他来。”吴用道:“兄
长梦晁天王所言:‘百日之灾,则除江南地灵星可治。’莫非正应此人?”宋江道:
“兄弟,你若有这个人,快与我去,休辞生受,只以义气为重,星夜去请此人,救
我一命。”吴用叫取蒜条金一百两与医人,再将三二十两碎银作为盘缠,分付与张
顺:“只今便行,好歹定要和他同来,切勿有误。我今拔寨回山,和他山寨里相会。
兄弟可作急快来。”张顺别了众人,背上包裹,望前便去。

且说军师吴用传令诸将:“权且收军,罢战回山。”车子上载了宋江,连夜起
发。北京城内,曾经了伏兵之计,只猜他引诱,不敢来追。次日,梁中书见报,说
道:“此去未知何意。”李成、闻达道:“吴用那厮诡计极多,只可坚守,不宜追
赶。”

话分两头。且说张顺要救宋江,连夜趱行。时值冬尽,无雨即雪,路上好生艰
难。更兼慌张,不曾带得雨具,行了十多日,早近扬子江边。是日北风大作,冻云
低垂,飞飞扬扬,下一天大雪。张顺冒着风雪,要过大江,舍命而行。虽是景物凄
凉,江内别是几般清致,有《西江月》为证:

嘹唳冻云孤雁,盘旋枯木寒鸦。空中雪下似梨花,片片飘琼乱洒。

玉压桥
边酒旆,银铺渡口鱼。前村隐隐两三家,江上晚来堪画。

那张顺独自一个奔至扬子江边,看那渡船时,并无一只,只叫得苦。绕着这江
边走,只见败苇折芦里面,有些烟起。张顺叫道:“艄公,快把渡船来载我!”只
见芦苇里簌簌地响,走出一个人来,头戴箬笠,身披蓑衣,问道:“客人要那里去?”
张顺道:“我要渡江,去建康府干事至紧,多与你些船钱,渡我则个。”那艄公道:
“载你不妨,只是今日晚了,便过江去,也没歇处。你只在我船里歇了,到四更风
静月明时,我便渡你过去,多出些船钱与我。”张顺道:“也说的是。”便与艄公
钻入芦苇里来,见滩边缆着一只小船,见蓬底下一个瘦后生在那里向火。艄公扶张
顺下船,走入舱里,把身上湿衣服都脱下来,叫那小后生就火上烘焙。张顺自打开
衣包,取出绵被,和身上卷倒在舱里,叫艄公道:“这里有酒卖么?买些来吃也好。”
艄公道:“酒却没买处,要饭便吃一碗。”张顺吃了一碗饭,放倒头便睡。一来连
日辛苦,二来十分托大,到初更左侧,不觉睡着。

那瘦后生向着炭火,烘着上盖的衲袄,看见张顺睡着了,便叫艄公道:“大哥,
你见么?”艄公盘将来,去头边只一捏,觉道是金帛之物,把手摇道:“你去把船
放开,去江心里下手不迟。”那后生推开蓬,跳上岸,解了缆索,上船把竹篙点开,
搭上橹,咿咿哑哑地摇出江心里来。艄公在船舱里取缆船索,轻轻地把张顺捆缚做
一块,便去船梢板底下,取出板刀来。张顺却好觉来,双手被缚,挣挫不得。艄
公手拿大刀,按在他身上。张顺道:“好汉,你饶我性命,都把金子与你。”艄公
道:“金子也要,你的性命也要。”张顺连声叫道:“你只教我囫囵死,冤魂便不
来缠你。”艄公放下板刀,把张顺扑通的丢下水去。

那艄公便去打开包来看时,见了许多金银,便没心分与那瘦后生,叫道:“五
哥,和你说话。”那人钻入舱里来,被艄公一手揪住,一刀落时,砍的伶仃,推下
水去。艄公打并了船中血迹,自摇船去了。

却说张顺是在水底下伏得三五夜的人,一时被推下去,就江底下咬断索子,赴
水过南岸时,见树林中隐隐有灯光。张顺爬上岸,水渌渌地转入林子里看时,却是
一个村酒店,半夜里起来酒,破壁缝透出灯光。张顺叫开门时,见个老丈,纳头
便拜。老儿道:“你莫不是江中被人劫了,跳水逃命的么?”张顺道:“实不相瞒
老丈:小人来建康干事。晚了,隔江觅船,不想撞着两个歹人,把小子应有衣服金
银,尽都劫了,撺入江中。小人却会赴水,逃得性命,公公救度则个。”老丈见说,
领张顺入后屋下,把个衲头与他,替下湿衣服来烘,烫些热酒与他吃。老丈道:“汉
子,你姓甚么?山东人来这里干何事?”张顺道:“小人姓张。建康府安太医是我
弟兄,特来探望他。”老丈道:“你从山东来,曾经梁山泊过?”张顺道:“正从
那里经过。”老丈道:“他山上宋头领,不劫来往客人,又不杀害人性命,只是替
天行道。”张顺道:“宋头领专以忠义为主,不害良民,只怪滥官污吏。”老丈道:
“老汉听得说:宋江这伙端的仁义,只是救贫济老,那里似我这里草贼?若得他来
这里,百姓都快活,不吃这伙滥污官吏薅恼!”张顺听罢道:“公公不要吃惊,小
人便是浪里白跳张顺。因为俺哥哥宋公明,害发背疮,教我将一百两黄金,来请安
道全。谁想托大,在船中睡着,被这两个贼男女缚了双手,撺下江里;被我咬断绳
索,到得这里。”老丈道:“你既是那里好汉,我教儿子出来,和你相见。”不多
时,后面走出一个后生来,看着张顺便拜道:“小人久闻哥哥大名,只是无缘,不
曾拜识。小人姓王,排行第六;因为走跳得快,人都唤小人做活闪婆王定六。平生
只好赴水使棒,多曾投师,不得传受,权在江边卖酒度日。却才哥哥被两个劫了的,
小人都认得:一个是截江鬼张旺;那一个瘦后生,却是华亭县人,唤做油里鳅孙五。
这两个男女,时常在这江里劫人。哥哥放心,在此住几日,等这厮来吃酒,我与哥
哥报仇。”张顺道:“感承兄弟好意。我为兄长宋公明,恨不得一日奔回寨里。只
等天明,便入城去,请了安太医,回来相会。”王定六把自己衣裳,都与张顺换了。
连忙置酒相待,不在话下。次日,天晴雪消,把十数两银子与张顺,且教入建康府
来。

张顺进得城中,径到槐桥下,看见安道全正在门前货药。张顺进得门,看着安
道全,纳头便拜。有首诗单题安道全好处:
肘后良方有百篇,金针玉刃得师传。
重生扁鹊应难比,万里传名安道全。

这安道全祖传内科外科,尽皆医得,以此远方驰名。当时看了张顺,便问道:
“兄弟多年不见,甚风吹得到此?”张顺随至里面,把这闹江州,跟宋江上山的事,
一一告诉了。后说宋江见患背疮,特地来请神医;扬子江中,险些儿送了性命,因
此空手而来,都实诉了。安道全道:“若论宋公明,天下义士,去走一遭最好;只
是拙妇亡过,家中别无亲人,离远不得,以此难出。”张顺苦苦求告:“若是兄长
推却不去,张顺也难回山。”安道全道:“再作商议。”张顺百般哀告,安道全方
才应允。原来这安道全却和建康府一个烟花娼妓,唤做李巧奴,时常往来。这李巧
奴生的十分美丽,安道全以此眷顾他,有诗为证:
蕙质温柔更老成,玉壶明月逼人清。
步摇宝髻寻春去,露湿凌波带月行。
丹脸笑回花萼丽,朱弦歌罢彩云停。
愿教心地常相忆,莫学章台赠柳情。

当晚就带张顺同去他家,安排酒吃。李巧奴拜张顺为叔叔。三杯五盏,酒至半
酣,安道全对巧奴说道:“我今晚就你这里宿歇,明日早,和这兄弟去山东地面走
一遭,多则是一个月,少是二十余日,便回来望你。”那李巧奴道:“我却不要你
去。你若不依我口,再也休上我门!”安道全道:“我药囊都已收拾了,只要动身,
明日便去。你且宽心,我便去也,又不耽搁。”李巧奴撒娇撒痴,便倒在安道全怀
里,说道:“你若还不依我,去了,我只咒得你肉片片儿飞!”张顺听了这话,恨
不得一口水吞吃了这婆娘。看看天色晚了,安道全大醉倒了,搀去巧奴房里,睡在
床上。巧奴却来发付张顺道:“你自归去,我家又没睡处。”张顺道:“只待哥哥
酒醒同去。”以此发遣他不动,只得安他在门首小房里歇。

张顺心中忧煎,那里睡得着。初更时分,有人敲门。张顺在壁缝里张时,只见
一个人闪将入来,便与虔婆说话。那婆子问道:“你许多时不来,却在那里?今晚
太医醉倒在房里,却怎生奈何?”那人道:“我有十两金子送与姐姐打些钗环,老
娘怎地做个方便,教他和我厮会则个。”虔婆道:“你只在我房里,我叫女儿来。”
张顺在灯影下张时,却见是截江鬼张旺。原来这厮,但是江中寻得些财,便来他家
使。张顺见了,按不住火起。再细听时,只见虔婆安排酒食在房里,叫巧奴相伴张
旺。张顺本待要抢入去,却又怕弄坏了事,走了这贼。约莫三更时候,厨下两个使
唤的也醉了;虔婆东倒西歪,却在灯前打醉眼子。张顺悄悄开了房门,踅到厨下,
见一把厨刀,明晃晃放在灶上,看这虔婆,倒在侧首板凳上。张顺走将入来,拿起
厨刀,先杀了虔婆。要杀使唤的时,原来厨刀不甚快,砍了一个人,刀口早卷了。
那两个正待要叫,却好一把劈柴斧正在手边,绰起来,一斧一个,砍杀了。房中婆
娘听得,慌忙开门,正迎着张顺,手起斧落,劈胸膛砍翻在地。张旺灯影下见砍翻
婆娘,推开后窗,跳墙走了。张顺懊恼无极,随即割下衣襟,蘸血去粉墙上写道:
“杀人者安道全也!”连写数十处。

捱到五更将明,只听得安道全在房中酒醒,便叫巧奴。张顺道:“哥哥,不要
则声,我教你看两个人。”安道全起来,看见四个死尸,吓得浑身麻木,颤做一团。
张顺道:“哥哥,你见壁上写的么?”安道全道:“你苦了我也!”张顺道:“只
有两条路,从你行。若是声张起来,我自走了,哥哥却用去偿命;若还你要没事,
家中取了药囊,连夜径上梁山泊,救我哥哥。这两件随你行。”安道全道:“兄弟,
忒这般短命见识!”有诗为证:
红粉无情只爱钱,临行何事更流连。
冤魂不赴阳台梦,笑煞痴心安道全。
到天明,张顺卷了盘缠,同安道全回家,敲开门,取了药囊,出城来,径到王定六
酒店里。王定六接着说道:“昨日张旺从这里过,可惜不遇见哥哥。”张顺道:“我
自要干大事,那里且报小仇。”说言未了,王定六报道:“张旺那厮来也。”张顺
道:“且不要惊他,看他投那里去。”只见张旺去滩头看船。王定六叫道:“张大
哥,你留船来,载我两个亲眷过去。”张旺道:“要趁船快来!”王定六报与张顺。
张顺道:“安兄,你可借衣服与小弟穿;小弟衣裳,却换与兄长穿了,才去趁船。”
安道全道:“此是何意?”张顺道:“自有主张,兄长莫问。”安道全脱下衣服,
与张顺换穿了。张顺戴上头巾,遮尘暖笠影身。王定六背了药囊,走到船边,张旺
拢船傍岸,三个人上船。

张顺爬入后梢,揭起板看时,板刀尚在,张顺拿了,再入船舱里。张旺把船
摇开,咿哑之声,直到江心里面。张顺脱去上盖,叫一声:“艄公快来!你看船舱
里漏进水来!”张旺不知是计,把头钻入舱里来,被张顺地揪住,喝一声:“强
贼,认得前日雪天趁船的客人么?”张旺看了,则声不得。张顺喝道:“你这厮谋
了我一百两黄金,又要害我性命!你那个瘦后生那里去了?”张旺道:“好汉,小
人得了财,无心分与他,恐他争论,被我杀死,撺入江里去了。”张顺道:“你认
得我么?”张旺道:“不识得好汉,只求饶了小人一命。”张顺喝道:“我生在浔
阳江边,长在小孤山下,作卖鱼牙子,谁不认得?只因闹了江州,上梁山泊随从宋
公明纵横天下,谁不惧我?你这厮漏我下船,缚住双手,撺下江心,不是我会识水
时,却不送了性命!今日冤仇相见,饶你不得!”就势只一拖,提在船舱中,把手
脚四马攒蹄,捆缚做一块,看着那扬子大江,直撺下去:“也免了你一刀!”张旺
性命,眼见得黄昏做鬼。

王定六看了,十分叹息。张顺就船内搜出前日金子并零碎银两,都收拾包裹里,
三人棹船到岸。张顺对王定六道:“贤弟恩义,生死难忘。你若不弃,便可同父亲
收拾起酒店,赶上梁山泊来,一同归顺大义,未知你心下如何?”王定六道:“哥
哥所言,正合小弟之心。”说罢分别。张顺和安道全就北岸上路。王定六作辞二人,
复上小船,自回家去,收拾行李赶来。

且说张顺与同安道全上得北岸,背了药囊,移身便走。那安道全是个文墨的人,
不会走路,行不得三十余里,早走不动。张顺请入村店,买酒相待。正吃之间,只
见外面一个客人走到面前,叫声:“兄弟,如何这般迟误!”张顺看时,却是神行
太保戴宗,扮做客人赶来。张顺慌忙教与安道全相见了,便问宋公明哥哥消息。戴
宗道:“如今宋哥哥神思昏迷,水米不吃,看看待死。”张顺闻言,泪如雨下。安
道全问道:“皮肉血色如何?”戴宗答道:“肌肤憔悴,终夜叫唤,疼痛不止,性
命早晚难保。”安道全道:“若是皮肉身体,得知疼痛,便可医治;只怕误了日期。”
戴宗道:“这个容易。”取两个甲马,拴在安道全腿上。戴宗自背了药囊,分付张
顺:“你自慢来,我同太医前去。”两个离了村店,作起神行法,先去了。

且说这张顺在本处村店里,一连安歇了两三日,只见王定六背了包裹,同父亲
果然过来。张顺接见,心中大喜,说道:“我专在此等你。”王定六问道:“安太
医何在?”张顺道:“神行太保戴宗接来迎着,已和他先行去了。”王定六却和张
顺并父亲一同起身,投梁山泊来。

且说戴宗引着安道全,作起神行法,连夜赶到梁山泊。寨中大小头领接着,拥
到宋江卧榻内,就床上看时,口内一丝两气。安道全先诊了脉息,说道:“众头领
休慌,脉体无事。身躯虽见沉重,大体不妨。不是安某说口,只十日之间,便要复
旧。”众人见说,一齐便拜。安道全先把艾焙引出毒气,然后用药。外使敷贴之饵,
内用长托之剂。五日之间,渐渐皮肤红白,肉体滋润,饮食渐进。不过十日,虽然
疮口未完,饮食复旧。只见张顺引着王定六父子二人,拜见宋江并众头领,诉说江
中被劫,水上报冤之事。众皆称叹:“险不误了兄长之患!”

宋江才得病好,便与吴用商量,要打北京,救取卢员外、石秀。安道全谏道:
“将军疮口未完,不可轻动,动则急难痊可。”吴用道:“不劳兄长挂心,只顾自
己将息,调理体中元阳真气。吴用虽然不才,只就目今春秋时候,定要打破北京城
池,救取卢员外、石秀二人性命,擒拿淫妇奸夫,不知兄长意下如何?”宋江道:
“若得军师如此扶持,宋江虽死瞑目!”吴用便就忠义堂上传令。有分教:北京城
内,变成火窟枪林;大名府中,翻作尸山血海。正是:谈笑鬼神皆丧胆,指挥豪杰
尽倾心。

毕竟军师吴用说出甚么计来,且听下回分解。