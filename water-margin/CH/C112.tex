\chapter{卢俊义分兵宣州道~宋公明大战毗陵郡}

话说元帅邢政和关胜交马,战不到十四五合,被关胜手起一刀,砍于马下。呼
延灼见砍了邢政,大驱人马,卷杀将去,六个统制官望南而走。吕枢密见本部军兵
大败亏输,弃了丹徒县,领了伤残军马,望常州府而走。宋兵十员大将,夺了县治,
报捷与宋先锋知道,部领大队军兵,前进丹徒县驻扎。赏劳三军,飞报张招讨,移
兵镇守润州。次日,中军从、耿二参谋赍送赏赐到丹徒县,宋江祗受,给赐众将。
宋江请卢俊义计议调兵征进,宋江道:“目今宣、湖二州,亦是贼寇方腊占据。我
今与你分兵拨将,作两路征剿,写下两个阄子,对天拈取。若拈得所征地方,便引
兵去。”当下宋江阄得常、苏二处,卢俊义阄得宣、湖二处,宋江便叫铁面孔目裴
宣把众将均分。除杨志患病不能征进,寄留丹徒外,其余将校拨开两路。宋先锋分
领将佐攻打常、苏二处,正偏将共计四十二人,正将一十三员,偏将二十九员:
正将先锋使呼保义宋江

军师智多星吴用
扑天雕李应


大刀关胜
小李广花荣


霹雳火秦明
金枪手徐宁


美髯公朱仝
花和尚鲁智深

行者武松
九纹龙史进


黑旋风李逵
神行太保戴宗
偏将镇三山黄信


病尉迟孙立
井木犴郝思文

丑郡马宣赞
百胜将韩滔


天目将彭
混世魔王樊瑞

铁笛仙马麟
锦毛虎燕顺


八臂那吒项充
飞天大圣李衮

丧门神鲍旭
矮脚虎王英


一丈青扈三娘
锦豹子杨林


金眼彪施恩
鬼脸儿杜兴


毛头星孔明
独火星孔亮


轰天雷凌振
铁臂膊蔡福


一枝花蔡庆
金毛犬段景住

通臂猿侯健
神算子蒋敬


神医安道全
险道神郁保四

铁扇子宋清
铁面孔目裴宣
大小正偏将佐四十二员,随行精兵三万人马,宋先锋总领。
副先锋卢俊义亦分将佐攻打宣、湖二处,正偏将佐共四十七员,正将一十五员,偏
将三十二员,朱武偏将之首,受军师之职。
正将副先锋玉麒麟卢俊义

军师神机朱武
小旋风柴进


豹子头林冲
双枪将董平


双鞭呼延灼
急先锋索超


没遮拦穆弘
病关索杨雄


插翅虎雷横
两头蛇解珍


双尾蝎解宝
没羽箭张清


赤发鬼刘唐
浪子燕青
偏将圣水将单廷

神火将魏定国
小温侯吕方


赛仁贵郭盛
摩云金翅欧鹏

火眼狻猊邓飞
打虎将李忠


小霸王周通
跳涧虎陈达


白花蛇杨春
病大虫薛永


摸着天杜迁
小遮拦穆春


出林龙邹渊
独角龙邹润


催命判官李立
青眼虎李云


石将军石勇
旱地忽律朱贵

笑面虎朱富
小尉迟孙新


母大虫顾大嫂
菜园子张青


母夜叉孙二娘
白面郎君郑天寿

金钱豹子汤隆
操刀鬼曹正


白日鼠白胜
花项虎龚旺


中箭虎丁得孙
活闪婆王定六

鼓上蚤时迁
大小正偏将佐四十七员,随征精兵三万人马,卢俊义管领。
看官牢记话头,卢先锋攻打宣、湖二州,共是四十七人;宋公明攻打常、苏二处,
共是四十二人。计有水军首领,自是一伙。为因童威、童猛差去焦山,寻见了石秀、
阮小七,回报道:“石秀、阮小七来到江边,杀了一家老小,夺得一只快船,前到
焦山寺内。寺主知道是梁山泊好汉,留在寺中宿食。后知张顺干了功劳,打听得焦
山下船,取茆港,好去攻伐江阴、太仓沿海州县,使人申将文书来,索请水军头领,
并要战具船只。”宋江即差李俊等八员,拨与水军五千,跟随石秀、阮小七等,共
取水路,计正偏将一十员。那十员,正将七员,偏将三员:
拚命三郎石秀

混江龙李俊
船火儿张横


浪里白跳张顺
立地太岁阮小二

短命二郎阮小五
活阎罗阮小七

出洞蛟童威
翻江蜃童猛


玉竿孟康
大小正偏将佐一十员,水军精兵五千,战船一百只。
看官听说,宋江自丹徒分兵,共是九十九人,已自不满百数。大战船都拨与水军头
领攻打江阴、太仓,小战船却俱入丹徒,都在里港,随军攻打常州。
话说吕师囊引了六个统制官,退保常州毗陵郡。这常州原有守城统制官钱振鹏,手
下两员副将:一个是晋陵县上濠人氏,姓金名节;一个是钱振鹏心腹之人许定。钱
振鹏原是清溪县都头出身,协助方腊,累得城池,升做常州制置使。听得吕枢密失
利,折了润州,一路退回常州。随即引金节、许定,开门迎接,请入州治,管待已
了,商议迎战之策。钱振鹏道:“枢相放心。钱某不才,愿施犬马之劳,直杀的宋
江那厮们大败过江,恢复润州,方遂吾愿!”吕枢密抚慰道:“若得制置如此用心,
何虑国家不安?成功之后,吕某当极力保奏,高迁重爵。”当日筵宴,不在话下。
且说宋先锋领起分定人马,攻打常、苏二州,拨马军长驱大进,望毗陵郡来。为头
正将一员关胜,部领十员将佐。那十人:秦明、徐宁、黄信、孙立、郝思文、宣赞、
韩滔、彭、马麟、燕顺;正偏将佐共计十一员,引马军三千,直取常州城下,摇
旗擂鼓搦战。吕枢密看了道:“谁敢去退敌军?”钱振鹏备了战马道:“钱某当以
效力向前。”吕枢密随即拨六个统制官相助。六个是谁:应明、张近仁、赵毅、沈
、高可立、范畴。七员将带领五千人马,开了城门,放下吊桥。钱振鹏使口泼风
刀,骑一匹卷毛赤兔马,当先出城。
关胜见了,把军马暂退一步,让钱振鹏列成阵势,六个统制官分在两下。对阵关胜
当先立马横刀,厉声高叫:“反贼听着!汝等助一匹夫谋反,损害生灵,人神共怒!
今日天兵临境,尚不知死,敢来与我拒敌!我等不把你这贼徒诛尽杀绝,誓不回兵!”
钱振鹏听了大怒,骂道:“量你等一伙,是梁山泊草寇,不知天时,却不思图王霸
业,倒去降无道昏君,要来和俺大国相并。我今直杀的你片甲不回才罢!”关胜大
怒,舞起青龙偃月刀,直冲将来。钱振鹏使动泼风刀,迎杀将去。两员将厮杀,斗
了三十合之上,钱振鹏渐渐力怯,抵当不住。南军门旗下,两个统制官看见钱振鹏
力怯,挺两条枪,一齐出马,前去夹攻关胜,上首赵毅,下首范畴。宋军门旗下,
恼犯了两员偏将,一个舞动丧门剑,一个使起虎眼鞭,抢出马来,乃是镇三山黄信、
病尉迟孙立。六员将,三对儿在阵前厮杀。吕枢密急使许定、金节出城助战。两将
得令,各持兵器,都上马直到阵前,见赵毅战黄信,范畴战孙立,却也都是对手。
斗到间深里,赵毅、范畴渐折便宜。许定、金节各使一口大刀出阵。宋军阵中韩滔、
彭二将,双出来迎。金节战住韩滔,许定战住彭,四将又斗,五队儿在阵前厮
杀。
原来金节素有归降大宋之心,故意要本队阵乱,略斗数合,拨回马望本阵先走,韩
滔乘势追将去。南军阵上高可立,看见金节被韩滔追赶得紧急,取雕弓,搭上硬箭,
满满地拽开,飕的一箭,把韩滔面颊上射着,倒撞下马来。这里秦明急把马一拍,
抡起狼牙棍前来救时,早被那里张近仁抢出来,咽喉上复一枪,结果了性命。彭
和韩滔是一正一副的兄弟,见他身死,急要报仇,撇了许定,直奔阵上,去寻高可
立。许定赶来,却得秦明占住厮杀。高可立看见彭赶来,挺枪便迎。不提防张近
仁从肋窝里撞将出来,把彭一枪搠下马去。关胜见损了二将,心中忿怒,恨不得
杀进常州,使转神威,把钱振鹏一刀也剁于马下。待要抢他那骑赤兔卷毛马,不提
防自己坐下赤兔马,一脚前失,倒把关胜掀下马来,南阵上高可立、张近仁两骑马
便来抢关胜,却得徐宁引宣赞、郝思文二将齐出,救得关胜回归本阵。吕枢密大驱
人马,卷杀出城,关胜众将失利,望北退走,南兵追赶二十余里。
此日关胜折了些人马,引军回见宋江,诉说折了韩滔、彭。宋江大哭道:“谁想
渡江已来,损折我五个兄弟。莫非皇天有怒,不容宋江收捕方腊,以致损兵折将?”
吴用劝道:“主帅差矣!输赢胜败,兵家常事,不足为怪。此是两个将军禄绝之日,
以致如此。请先锋免忧,且理大事。”只见帐前转过李逵便说道:“着几个认得杀
俺兄弟的人,引我去杀那贼徒,替我两个哥哥报仇!”宋江传令,教来日打起一面
白旗,“我亲自引众将,直至城边,与贼交锋,决个胜负”。次日,宋公明领起大
队人马,水陆并进,船骑相迎,拔寨都起。黑旋风李逵引着鲍旭、项充、李衮,带
领五百悍勇步军,先来出哨,直到常州城下。
吕枢密见折了钱振鹏,心下甚忧,连发了三道飞报文书,去苏州三大王方貌处求救,
一面写表申奏朝廷。又听得报道:“城下有五百步军打城,认旗上写道为首的是黑
旋风李逵。”吕枢密道:“这厮是梁山泊第一个凶徒,惯杀人的好汉,谁敢与我先
去拿他?”帐前转过两个得胜获功的统制官高可立、张近仁。吕枢密道:“你两个
若拿得这个贼人,我当一力保奏,加官重赏。”张、高二统制,各绰了枪上马,带
领一千马步兵,出城迎敌。黑旋风李逵见了,便把五百步军一字儿摆开,手两把
板斧,立在阵前。丧门神鲍旭仗着一口大阔板刀,随于侧首;项充、李衮两个,各
人手挽着蛮牌,右手拿着铁标,四个人各披前后掩心铁甲,列于阵前。高、张二统
制正是得胜狸猫强似虎,及时鸦鹊便欺雕,统着一千军马,靠城排开。
宋军内有几个探子,却认得高可立、张近仁两个,是杀韩滔、彭的,便指与黑旋
风道:“这两个领军的,便是杀俺韩、彭二将军的!”李逵听了这说,也不打话,
拿起两把板斧,直抢过对阵去。鲍旭见李逵杀过对阵,急呼项充、李衮舞起蛮牌,
便去策应。四个齐发一声喊,滚过对阵。高可立、张近仁吃了一惊,措手不及,急
待回马,那两个蛮牌,早滚到马颔下。高可立、张近仁在马上把枪望下搠时,项充、
李衮把牌迎住。李逵斧起,早砍翻高可立马脚,高可立下马来。项充叫道“留下
活的”时,李逵是个好杀人的汉子,那里忍耐得住,早一斧砍下头来。鲍旭从马上
揪下张近仁,一刀也割了头。四个在阵里乱杀。黑旋风把高可立的头缚在腰里,抡
起两把板斧,不问天地,横身在里面砍杀,杀得一千马步军,退入城去,也杀了三
四百人,直赶到吊桥边。李逵和鲍旭两个,便要杀入城去,项充、李衮死当回来。
城上擂木炮石,早打下来。四个回到阵前,五百军兵依原一字摆开,那里敢轻动?
本是也要来混战,怕黑旋风不分皂白,见的便砍,因此不敢近前。
尘头起处,宋先锋军马已到,李逵、鲍旭各献首级,众将认的是高可立、张近仁的
头,都吃了一惊道:“如何获得仇人首级?”两个说:“杀了许多人众,本待要捉
活的来,一时手痒,忍耐不住,就便杀了。”宋江道:“既有仇人首级,可于白旗
下,望空祭祀韩、彭二将。”宋江又哭了一场,放倒白旗,赏了李逵、鲍旭、项充、
李衮四人,便进兵到常州城下。
且说吕枢密在城中心慌,便与金节、许定,并四个统制官商议退宋江之策。诸将见
李逵等杀了这一阵,众人都胆颤心寒,不敢出战。问了数声,如箭穿雁嘴,钩搭鱼
腮,默默无言,无人敢应。吕枢密心内纳闷,教人上城看时,宋江军马,三面围住
常州,尽在城下擂鼓摇旗,呐喊搦战。吕枢密叫众将且各上城守护。众将退去,吕
枢密自在后堂寻思,无计可施,唤集亲随左右心腹人商量,自欲弃城逃走,不在话
下。
且说守将金节回到自己家中,与其妻秦玉兰说道:“如今宋先锋围住城池,三面攻
击。我等城中粮食缺少,不经久困。倘或打破城池,我等那时皆为刀下之鬼。”秦
玉兰答道:“你素有忠孝之心,归降之意,更兼原是宋朝旧官,朝廷不曾有甚负汝,
不若去邪归正,擒捉吕师囊献与宋先锋,便是进身之计。”金节道:“他手下现有
四个统制官,各有军马。许定这厮,又与我不睦,与吕师囊又是心腹之人。我恐事
未必谐,反惹其祸。”其妻道:“你只密密地夤夜修一封书缄,拴在箭上,射出城
去,和宋先锋达知,里应外合取城。你来日出战,诈败佯输,引诱入城,便是你的
功劳。”金节道:“贤妻此言极当,依汝行之。”史官诗曰:
弃暗投明免祸机,毗陵重见负羁妻。
妇人尚且存忠义,何事男儿识见迷。
次日,宋江领兵攻城得紧,吕枢密聚众商议,金节答道:“常州城池高广,只宜守,
不可敌。众将且坚守,等待苏州救兵来到,方可会合出战。”吕枢密道:“此言极
是。”分拨众将:应明、赵毅守把东门,沈、范畴守把北门,金节守把西门,许
定守把南门。调拨已定,各自领兵坚守。当晚金节写了私书,拴在箭上,待夜深人
静,在城上望着西门外探路军人射将下去。那军校拾得箭矢,慌忙报入寨里来。守
西寨正将花和尚鲁智深同行者武松两个见了,随即使偏将杜兴赍了,飞报东北门大
寨里来。宋江、吴用点着明烛,在帐里议事。杜兴呈上金节的私书,宋江看了大喜,
便传令教三寨中知会。
次日,三寨内头领三面攻城。吕枢密在战楼上,正观见宋江阵里轰天雷凌振,扎起
炮架,却放了一个风火炮,直飞起去,正打在敌楼角上,骨碌碌一声响,平塌了半
边。吕枢密急走,救得性命下城来,催督四门守将,出城搦战。擂了三通战鼓,大
开城门,放下吊桥,北门沈、范畴引军出战。宋军中大刀关胜,坐下钱振鹏的卷
毛赤兔马,出于阵前,与范畴交战。两个正待相持,西门金节又引出一彪军来搦战。
宋江阵上病尉迟孙立出马。两个交战,斗不到三合,金节诈败,拨转马头便走。孙
立当先,燕顺、马麟为次,鲁智深、武松、孔明、孔亮、施恩、杜兴,一发进兵。
金节便退入城,孙立已赶入城门边,占住西门。城中闹起,知道大宋军马,已从西
门进城了。那时百姓都被方腊残害不过,怨气冲天,听得宋军入城,尽出来助战。
城上早竖起宋先锋旗号。范畴、沈见了城中事变,急待奔入城去保全老小时,左
边冲出王矮虎、一丈青,早把范畴捉了。右边冲出宣赞、郝思文两个,一齐向前,
把沈一枪刺下马去,众军活捉了。宋江、吴用大驱人马入城,四下里搜捉南兵,
尽行诛杀。吕枢密引了许定,自投南门而走,死命夺路,众军追赶不上,自回常州
听令,论功升赏。赵毅躲在百姓人家,被百姓捉来献出。应明乱军中杀死,获得首
级。宋江来到州治,便出榜安抚,百姓扶老携幼,诣州拜谢。宋江抚慰百姓,复为
良民。众将各来请功。
金节赴州治拜见宋江,宋江亲自下阶迎接金节,上厅请坐。金节感激无限,复为宋
朝良臣,此皆其妻赞成之功,不在话下。宋江叫把范畴、沈、赵毅三个,陷车盛
了,写道申状,就叫金节亲自解赴润州张招讨中军帐前。金节领了公文,监押三将,
前赴润州交割。比及去时,宋江已自先叫神行太保戴宗,赍飞报文书,保举金节到
中军了。张招讨见宋江申复金节如此忠义,后金节到润州,张招讨大喜,赏赐金节
金银缎匹、鞍马酒礼。有副都督刘光世,就留了金节,升做行军都统,留于军前听
用。后来金节跟随刘光世大破金兀四太子,多立功劳,直做到亲军指挥使,至中
山阵亡,这是金节的结果。有诗为证:
从邪廊庙生堪愧,殉义沙场骨也香。
他日中山忠义鬼,何如方腊阵中亡。
当日张招讨、刘都督赏了金节,把三个贼人碎尸万段,枭首示众。随即使人来常州,
犒劳宋先锋军马。
且说宋江在常州屯驻军马,使戴宗去宣州、湖州卢先锋处,飞报调兵消息,一面又
有探马报来说,吕枢密逃回在无锡县,又会合苏州救兵,正欲前来迎敌。宋江闻知,
便调马军步军,正偏将佐十员头领,拨与军兵一万,望南迎敌。那十员将佐:关胜、
秦明、朱仝、李应、鲁智深、武松、李逵、鲍旭、项充、李衮。当下关胜等领起前
部军兵人马,与同众将,辞了宋先锋,离城去了。
且说戴宗探听宣、湖二州进兵的消息,与同柴进回见宋江,报说副先锋卢俊义得了
宣州,特使柴大官人到来报捷。宋江甚喜。柴进到州治,参拜已了,宋江把了接风
酒,同入后堂坐下,动问卢先锋破宣州备细缘由。柴进将出申达文书,与宋江看了,
备说打宣州一事:“方腊部下镇守宣州经略使家余庆,手下统制官六员,都是歙州、
睦州人氏。那六人?李韶、韩明、杜敬臣、鲁安、潘、程胜祖。当日家余庆分调
六个统制,做三路出城对阵,卢先锋也分三路军兵迎敌。中间是呼延灼和李韶交战,
董平共韩明相持。战到十合,韩明被董平两枪刺死,李韶遁去,中路军马大败。左
军是林冲和杜敬臣交战,索超与鲁安相持。林冲蛇矛刺死杜敬臣,索超斧劈死鲁安。
右军是张清和潘交战,穆弘共程胜祖相持。张清一石子打下潘,打虎将李忠赶
出去杀了。程胜祖弃马逃回。此日连胜四将,贼兵退入城去。卢先锋急驱众将夺城,
赶到门边,不提防贼兵城上,飞下一片磨扇来,打死俺一个偏将。城上箭如雨点一
般射下来,那箭矢都有毒药,射中俺两个偏将,比及到寨,俱各身死。卢先锋因见
折了三将,连夜攻城。守东门贼将不紧,因此得了宣州,乱军中杀死了李韶,家余
庆领了些败残军兵,望湖州去了。智深困于阵上,不知去向。磨扇打死了白面郎君
郑天寿;两个中药箭的是操刀鬼曹正、活闪婆王定六。”

宋江听得又折了三个兄弟,大哭一声,蓦然倒地,未知五脏如何,先见四肢不
举。正是:花开又被风吹落,月皎那堪云雾遮。
毕竟宋江昏晕倒了性命如何,且听下回分解。