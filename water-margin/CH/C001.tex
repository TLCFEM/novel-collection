\chapter{张天师祈禳瘟疫~洪太尉误走妖魔}

话说大宋仁宗天子在位,嘉祐三年三月三日五更三点,天子驾坐紫宸殿,受百
官朝贺。但见:

祥云迷凤阁,瑞气罩龙楼。含烟御柳拂旌旗,带露宫花迎剑戟。天香影里,玉
簪朱履聚丹墀;仙乐声中,绣袄锦衣扶御驾。珍珠帘卷,黄金殿上现金轝,凤羽扇
开,白玉阶前停宝辇。隐隐净鞭三下响,层层文武两班齐。
当有殿头官喝道:“有事出班早奏,无事卷帘退朝。”只见班部丛中,宰相赵
哲、参政文彦博出班奏曰:“目今京师瘟疫盛行,伤损军民甚多。伏望陛下释罪宽
恩,省刑薄税,祈禳天灾,救济万民。”天子听奏,急敕翰林院随即草诏,一面降
赦天下罪囚,应有民间税赋,悉皆赦免;一面命在京宫观寺院,修设好事禳灾。不
料其年瘟疫转盛,仁宗天子闻知,龙体不安,复会百官计议。向那班部中,有一大
臣,越班启奏。天子看时,乃是参知政事范仲淹,拜罢起居,奏曰:“目今天灾盛
行,军民涂炭,日夕不能聊生。以臣愚意,要禳此灾,可宣嗣汉天师星夜临朝,就
京师禁院,修设三千六百分罗天大醮,奏闻上帝,可以禳保民间瘟疫。”仁宗天子
准奏,急令翰林学士草诏一道,天子御笔亲书,并降御香一炷,钦差内外提点殿前
太尉洪信为天使,前往江西信州龙虎山,宣请嗣汉天师张真人星夜来朝,祈禳瘟疫。
就金殿上焚起御香,亲将丹诏付与洪太尉,即便登程前去。

洪信领了圣敕,辞别天子,背了诏书,盛了御香,带了数十人,上了铺马,一
行部队,离了东京,取路径投信州贵溪县来。但见:

遥山叠翠,远水澄清。奇花绽锦绣铺林,嫩柳舞金丝拂地。风和日暖,时过野
店山村;路直沙平,夜宿邮亭驿馆。罗衣荡漾红尘内,骏马驰驱紫陌中。

且说太尉洪信赍擎御诏,一行人从,上了路途,不止一日,来到江西信州。大
小官员,出郭迎接。随即差人报知龙虎山上清宫住持道众,准备接诏。次日,众位
官同送太尉到于龙虎山下,只见上清宫许多道众,鸣钟击鼓,香花灯烛,幢幡宝盖,
一派仙乐,都下山来迎接丹诏,直至上清宫前下马。太尉看那宫殿时,端的是好座
上清宫!但见:

青松屈曲,翠柏阴森。门悬敕额金书,户列灵符玉篆。虚皇坛畔,依稀垂柳名
花;炼药炉边,掩映苍松老桧。左壁厢天丁力士,参随着太乙真君;右势下玉女金
童,簇捧定紫微大帝。披发仗剑,北方真武踏龟蛇;趿履顶冠,南极老人伏龙虎。
前排二十八宿星君,后列三十二帝天子。阶砌下流水潺。墙院后好山环绕。鹤生
丹顶,龟长绿毛。树梢头献果苍猿,莎草内衔芝白鹿。三清殿上,击金钟道士步虚;
四圣堂前,敲玉罄真人礼斗。献香台砌,彩霞光射碧琉璃;召将瑶坛,赤日影摇红
玛瑙。早来门外祥云现,疑是天师送老君。

当下上自住持真人,下及道童侍从,前迎后引,接至三清殿上,请将诏书居中
供养着。洪太尉便问监宫真人道:“天师今在何处?”住持真人向前禀道:“好教
太尉得知:这代祖师,号曰虚靖天师,性好清高,倦于迎送,自向龙虎山顶,结一
茅庵,修真养性。因此不住本宫。”太尉道:“目今天子宣诏,如何得见?”真人
答道:“容禀:诏敕权供在殿上,贫道等亦不敢开读。且请太尉到方丈献茶,再烦
计议。”当时将丹诏供养在三清殿上,与众官都到方丈。太尉居中坐下,执事人等
献茶,就进斋供,水陆俱备。斋罢,太尉再问真人道:“既然天师在山顶庵中,何
不着人请将下来相见,开宣丹诏。”真人禀道:“这代祖师,虽在山顶,其实道行
非常,能驾雾兴云,踪迹不定。贫道等如常亦难得见,怎生教人请得下来?”太尉
道:“似此如何得见?目今京师瘟疫盛行,今上天子特遣下官赍捧御书丹诏,亲奉
龙香,来请天师,要做三千六百分罗天大醮,以禳天灾,救济万民。似此怎生奈何?”
真人禀道:“天子要救万民,只除是太尉办一点志诚心,斋戒沐浴,更换布衣,休
带从人,自背诏书,焚烧御香,步行上山礼拜,叩请天师,方许得见。如若心不志
诚,空走一遭,亦难得见。”太尉听说,便道:“俺从京师食素到此,如何心不志
诚。既然恁地,依着你说,明日绝早上山。”

当晚各自权歇。次日五更时分,众道士起来,备下香汤,请太尉起来沐浴,换
了一身新鲜布衣,脚下穿上麻鞋草履,吃了素斋,取过丹诏,用黄罗包袱背在脊梁
上,手里提着银手炉,降降地烧着御香,许多道众人等,送到后山,指与路径。真
人又禀道:“太尉要救万民,休生退悔之心,只顾志诚上去。”

太尉别了众人,口诵天尊宝号,纵步上山来。将至半山,望见大顶直侵霄汉,
果然好座大山!正是:

根盘地角,顶接天心。远观磨断乱云痕,近看平吞明月魄。高低不等谓之山,
侧石通道谓之岫,孤岭崎岖谓之路,上面平极谓之顶。头圆下壮谓之峦,藏虎藏豹
谓之穴,隐风隐云谓之岩,高人隐居谓之洞。有境有界谓之府,樵人出没谓之径,
能通车马谓之道,流水有声谓之涧,古渡源头谓之溪,岩崖滴水谓之泉。左壁为掩,
右壁为映。出的是云,纳的是雾。锥尖像小,崎峻似峭,悬空似险,削如平。千
峰竞秀,万壑争流,瀑布斜飞,藤萝倒挂。虎啸时风生谷口,猿啼时月坠山腰。恰
似青黛染成千块玉,碧纱笼罩万堆烟。

这洪太尉独自一个行了一回,盘坡转径,揽葛攀藤。约莫走过了数个山头,三
二里多路,看看脚酸腿软,正走不动,口里不说,肚里踌躇,心中想道:“我是朝
廷贵官,在京师时,重裀而卧,列鼎而食,尚兀自倦怠,何曾穿草鞋,走这般山路!
知他天师在那里,却教下官受这般苦!”又行不到三五十步,掇着肩气喘,只见山
凹里起一阵风。风过处,向那松树背后,奔雷也似吼了一声,扑地跳出一个吊睛白
额锦毛大虫来,洪太尉吃了一惊,叫声:“阿呀!”扑地望后便倒。偷眼看那大虫
时,但见:
毛披一带黄金色,爪露银钩十八只。
睛如闪电尾如鞭,口似血盆牙似戟。
伸腰展臂势狰狞,摆尾摇头声霹雳。
山中狐兔尽潜藏,涧下獐皆敛迹。

那大虫望着洪太尉,左盘右旋,咆哮了一回,托地望后山坡下跳了去。洪太尉
倒在树根底下,唬的三十六个牙齿捉对儿厮打,那心头一似十五个吊桶,七上八落
的响,浑身却如重风麻木,两腿一似斗败公鸡,口里连声叫苦。

大虫去了一盏茶时,方才爬将起来,再收拾地上香炉,还把龙香烧着,再上山
来,务要寻见天师。又行过三五十步,口里叹了数口气,怨道:“皇帝御限差俺来
这里,教我受这场惊恐。”说犹未了,只觉得那里又一阵风,吹得毒气直冲将来,
太尉定睛看时,山边竹藤里簌簌地响,抢出一条吊桶大小雪花也似蛇来。太尉见了,
又吃一惊,撇了手炉,叫一声:“我今番死也!”往后便倒在盘陀石边。微闪开眼
来看那蛇时,但见:

昂首惊飙起,掣目电光生。动荡则折峡倒冈,呼吸则吹云吐雾。鳞甲乱分千片
玉,尾梢斜卷一堆银。
那条大蛇,径抢到盘陀石边,朝着洪太尉盘做一堆,两只眼迸出金光,张开巨
口,吐出舌头,喷那毒气在洪太尉脸上,惊得太尉三魂荡荡,七魄悠悠。那蛇看了
洪太尉一回,望山下一溜,却早不见了。太尉方才爬得起来,说道:“惭愧!惊杀下
官!”看身上时,寒栗子比餶飿儿大小,口里骂那道士:“叵耐无礼,戏弄下官,
教俺受这般惊恐!若山上寻不见天师,下去和他别有话说。”再拿了银提炉,整顿身
上诏敕,并衣服巾帻,却待再要上山去。正欲移步,只听得松树背后隐隐地笛声吹
响,渐渐近来。太尉定睛看时,只见那一个道童,倒骑着一头黄牛,横吹着一管铁
笛,转出山凹来。太尉看那道童时:

头绾两枚丫髻,身穿一领青衣,腰间绦结草来编,脚下芒鞋麻间隔。明眸皓齿,
飘飘并不染尘埃;绿鬓朱颜,耿耿全然无俗态。
昔日吕洞宾有首牧童诗道得好:
草铺横野六七里,笛弄晚风三四声。
归来饱饭黄昏后,不脱蓑衣卧月明。

但见那个道童笑吟吟地骑着黄牛,横吹着那管铁笛,正过山来。洪太尉见了,
便唤那个道童:“你从那里来?认得我么?”道童不睬,只顾吹笛。太尉连问数声,
道童呵呵大笑,拿着铁笛,指着洪太尉说道:“你来此间,莫非要见天师么?”太
尉大惊,便道:“你是牧童,如何得知?”道童笑道:“我早间在草庵中伏侍天师,
听得天师说道:‘今上皇帝差个洪太尉赍擎丹诏御香,到来山中,宣我往东京做三
千六百分罗天大醮,祈禳天下瘟疫,我如今乘鹤驾云去也。’这早晚想是去了,不
在庵中。你休上去,山内毒虫猛兽极多,恐伤害了你性命。”太尉再问道:“你不
要说谎。”道童笑了一声,也不回应,又吹着铁笛,转过山坡去了。太尉寻思道:
“这小的如何尽知此事?想是天师分付他,已定是了。”欲待再上山去;方才惊
的苦,争些儿送了性命,不如下山去罢。

太尉拿着提炉,再寻旧路,奔下山来。众道士接着,请至方丈坐下。真人便问
太尉道:“曾见天师么?”太尉说道:“我是朝中贵官,如何教俺走得山路,吃了
这般辛苦,争些儿送了性命。为头上至半山里,跳出一只吊睛白额大虫,惊得下官
魂魄都没了;又行不过一个山嘴,竹藤里抢出一条雪花大蛇来,盘做一堆,拦住去
路。若不是俺福分大,如何得性命回京?尽是你这道众戏弄下官。”真人复道:“贫
道等怎敢轻慢大臣?这是祖师试探太尉之心。本山虽有蛇虎,并不伤人。”太尉又
道:“我正走不动,方欲再上山坡,只见松树旁边转出一个道童,骑着一头黄牛,
吹着管铁笛,正过山来,我便问他:‘那里来?识得俺么?’他道:‘已都知了。’
说天师分付,早晨乘鹤驾云,往东京去了,下官因此回来。”真人道:“太尉可惜
错过,这个牧童,正是天师。”太尉道:“他既是天师,如何这等猥獕?”真人答
道:“这代天师,非同小可。虽然年幼,其实道行非常。他是额外之人,四方显化,
极是灵验,世人皆称为道通祖师。”洪太尉道:“我直如此有眼不识真师,当面错
过!”真人道:“太尉且请放心。既然祖师法旨道是去了,比及太尉回京之日,这
场醮事,祖师已都完了。”太尉见说,方才放心。真人一面教安排筵宴,管待太尉,
请将丹诏收藏于御书匣内,留在上清宫中,龙香就三清殿上烧了。当日方丈内大排
斋供,设宴饮酌,至晚席罢,止宿到晓。

次日早膳以后,真人、道众并提点、执事人等,请太尉游山。太尉大喜。许多
人从跟随着,步行出方丈,前面两个道童引路。行至宫前宫后,看玩许多景致。三
清殿上,富贵不可尽言。左廊下九天殿、紫微殿、北极殿;右廊下太乙殿、三官殿、
驱邪殿。诸宫看遍,行到右廊后一所去处。洪太尉看时,另外一所殿宇:一遭都是
捣椒红泥墙;正面两扇朱红格子,门上使着膊大锁锁着,交叉上面贴着十数道封
皮,封皮上又是重重叠叠使着朱印;檐前一面朱红漆金字牌额,左书四个金字,写
道:“伏魔之殿”。太尉指着门道:“此殿是甚么去处?”真人答道:“此乃是前
代老祖天师锁镇魔王之殿。”太尉又问道:“如何上面重重叠叠贴着许多封皮?”
真人答道:“此是老祖大唐洞玄国师封锁魔王在此。但是经传一代天师,亲手便添
一道封皮,使其子子孙孙,不得妄开。走了魔君,非常利害。今经八九代祖师,誓
不敢开。锁用铜汁灌铸,谁知里面的事?小道自来住持本宫三十余年,也只听闻。”

洪太尉听了,心中惊怪,想道:“我且试看魔王一看。”便对真人说道:“你
且开门来,我看魔王甚么模样。”真人告道:“太尉,此殿决不敢开!先祖天师叮
咛告戒:今后诸人不许擅开。”太尉笑道:“胡说!你等要妄生怪事,煽惑良民,
故意安排这等去处,假称锁镇魔王,显耀你们道术。我读一鉴之书,何曾见锁魔之
法!神鬼之道,处隔幽冥,我不信有魔王在内。快与我打开,我看魔王如何!”真
人三回五次禀说:“此殿开不得,恐惹利害,有伤于人。”太尉大怒,指着道众说
道:“你等不开与我看,回到朝廷,先奏你们众道士阻当宣诏,违别圣旨,不令我
见天师的罪犯;后奏你等私设此殿,假称锁镇魔王,煽惑军民百姓。把你都追了度
牒,刺配远恶军州受苦。”

真人等惧怕太尉权势,只得唤几个火工道人来,先把封皮揭了,将铁锤打开大
锁,众人把门推开,看里面时,黑洞洞地,但见:

昏昏默默,杳杳冥冥,数百年不见太阳光,亿万载难瞻明月影。不分南北,怎
辨东西。黑烟霭霭扑人寒,冷气阴阴侵体颤。人迹不到之处,妖精往来之乡,闪开
双目有如盲,伸出两手不见掌。常如三十夜,却似五更时。

众人一齐都到殿内,黑暗暗不见一物。太尉教从人取十数个火把点着,将来打
一照时,四边并无一物,只中央一个石碑,约高五六尺,下面石龟趺坐,大半陷在
泥里。照那碑碣上时,前面都是龙章凤篆,天书符籙,人皆不识。照那碑后时,却
有四个真字大书,凿着“遇洪而开”。却不是一来天罡星合当出世,二来宋朝必显
忠良,三来凑巧遇着洪信,岂不是天数?洪太尉看了这四个字,大喜,便对真人说
道:“你等阻当我,却怎地数百年前已注定我姓字在此?遇洪而开,分明是教我开
看,却何妨。我想这个魔王,都只在石碑底下。汝等从人,与我多唤几个火工人等,
将锄头铁锹来掘开。”

真人慌忙谏道:“太尉不可掘动,恐有利害,伤犯于人,不当稳便。”太尉大
怒,喝道:“你等道众,省得甚么?碑上分明凿着遇我教开,你如何阻当?快与我唤
人来开。”真人又三回五次禀道:“恐有不好。”太尉那里肯听,只得聚集众人,
先把石碑放倒,一齐并力掘那石龟,半日方才掘得起。又掘下去,约有三四尺深,
见一片大青石板,可方丈围。洪太尉叫再掘起来,真人又苦禀道:“不可掘动。”
太尉那里肯听,众人只得把石板一齐扛起,看时,石板底下,却是一个万丈深浅地
穴。只见穴内刮喇喇一声响亮。那响非同小可,恰似:

天摧地塌,岳撼山崩。钱塘江上,潮头浪拥出海门来;泰华山头,巨灵神一劈
山峰碎。共工奋怒,去盔撞倒了不周山;力士施威,飞锤击碎了始皇辇。一风撼折
千竿竹,十万军中半夜雷。
那一声响亮过处,只见一道黑气,从穴里滚将起来,掀塌了半个殿角。那道黑气,
直冲到半天里空中,散作百十道金光,望四面八方去了。众人吃了一惊,发声喊,
都走了,撇下锄头铁锹,尽从殿内奔将出来,推倒翻无数。惊得洪太尉目睁口呆,
罔知所措,面色如土,奔到廊下,只见真人向前叫苦不迭。

太尉问道:“走了的却是甚么妖魔?”那真人言不过数句,话不过一席,说出
这个缘由。有分教:一朝皇帝,夜眠不稳,昼食忘餐。直使:宛子城中藏虎豹,蓼
儿洼内聚神蛟。

毕竟龙虎山真人说出甚么言语来,且听下回分解。