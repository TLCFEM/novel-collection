\chapter{横海郡柴进留宾~景阳冈武松打虎}

话说宋江因躲一杯酒,去净手了,转出廊下来,了火锨柄,引得那汉焦燥,
跳将起来,就欲要打宋江。柴进赶将出来,偶叫起宋押司,因此露出姓名来。那大
汉听得是宋江,跪在地下,那里肯起,说道:“小人‘有眼不识泰山’!一时冒渎
兄长,望乞恕罪。”宋江扶起那汉,问道:“足下是谁?高姓大名?”柴进指着道:
“这人是清河县人氏,姓武,名松,排行第二,今在此间一年矣。”宋江道:“江
湖上多闻说武二郎名字,不期今日却在这里相会,多幸,多幸!”柴进道:“偶然
豪杰相聚,实是难得。就请同做一席说话。”

宋江大喜,携住武松的手,一同到后堂席上,便唤宋清与武松相见。柴进便邀
武松坐地。宋江连忙让他一同在上面坐。武松那里肯坐,谦了半晌,武松坐了第三
位。柴进教再整杯盘来,劝三人痛饮。宋江在灯下看那武松时,果然是一条好汉。
但见:

身躯凛凛,相貌堂堂。一双眼光射寒星,两弯眉浑如刷漆。胸脯横阔,有万夫
难敌之威风;语话轩昂,吐千丈凌云之志气。心雄胆大,似撼天狮子下云端;骨健
筋强,如摇地貔貅临座上。如同天上降魔主,真是人间太岁神。

当下宋江在灯下看了武松这表人物,心中甚喜,便问武松道:“二郎因何在此?”
武松答道:“小弟在清河县,因酒后醉了,与本处机密相争,一时间怒起,只一拳,
打得那厮昏沉。小弟只道他死了,因此一径地逃来,投奔大官人处,躲灾避难,今
已一年有余。后来打听得那厮却不曾死,救得活了。今欲正要回乡去寻哥哥,不想
染患疟疾,不能够动身回去。却才正发寒冷,在那廊下向火,被兄长了锨柄,吃
了那一惊,惊出一身冷汗,觉得这病好了。”宋江听了大喜。当夜饮至三更。酒罢,
宋江就留武松在西轩下做一处安歇。次日起来,柴进安排席面,杀羊宰猪,管待宋
江,不在话下。过了数日,宋江将出些银两来与武松做衣裳。柴进知道,那里肯要
他坏钱?自取出一箱缎匹绸绢,门下自有针工,便教做三人的称体衣裳。

说话的,柴进因何不喜武松?原来武松初来投奔柴进时,也一般接纳管待;次
后在庄上,但吃醉了酒,性气刚,庄客有些顾管不到处,他便要下拳打他们,因此
满庄里庄客,没一个道他好。众人只是嫌他,都去柴进面前,告诉他许多不是处。
柴进虽然不赶他,只是相待得他慢了。却得宋江每日带挈他一处,饮酒相陪,武松
的前病都不发了。

相伴宋江住了十数日,武松思乡,要回清河县看望哥哥。柴进、宋江两个都留
他再住几时,武松道:“小弟的哥哥多时不通信息,因此要去望他。”宋江道:“实
是二郎要去,不敢苦留。如若得闲时,再来相会几时。”武松相谢了宋江。柴进取
出些金银,送与武松,武松谢道:“实是多多相扰了大官人。”武松缚了包裹,拴
了哨棒,要行。柴进又治酒食送路。武松穿了一领新纳红绸祆,戴着个白范阳毡笠
儿,背上包裹,提了杆棒,相辞了便行。宋江道:“贤弟少等一等。”回到自己房
内,取了些银两,赶出到庄门前来,说道:“我送兄弟一程。”宋江和兄弟宋清两
个送武松,待他辞了柴大官人,宋江也道:“大官人,暂别了便来。”

三个离了柴进东庄,行了五七里路,武松作别道:“尊兄远了,请回。柴大官
人必然专望。”宋江道:“何妨再送几步。”路上说些闲话,不觉又过了三二里。
武松挽住宋江说道:“尊兄不必远送。常言道:‘送君千里,终须一别。’”宋江
指着道:“容我再行几步。兀那官道上有个小酒店,我们吃三钟了作别。”

三个来到酒店里,宋江上首坐了,武松倚了哨棒,下席坐了,宋清横头坐定。
便叫酒保打酒来,且买些盘馔、果品、菜蔬之类,都搬来摆在桌子上。三人饮了几
杯,看看红日平西,武松便道:“天色将晚,哥哥不弃武二时,就此受武二四拜,
拜为义兄。”宋江大喜。武松纳头拜了四拜,宋江叫宋清身边取出一锭十两银子,
送与武松。武松那里肯受,说道:“哥哥,客中自用盘费。”宋江道:“贤弟不必
多虑。你若推却,我便不认你做兄弟。”武松只得拜受了,收放缠袋里。宋江取些
碎银子,还了酒钱。武松拿了哨棒,三个出酒店前来作别。武松堕泪,拜辞了自去。
宋江和宋清立在酒店门前,望武松不见了,方才转身回来。行不到五里路头,只见
柴大官人骑着马,背后牵着两匹空马来接。宋江望见了大喜,一同上马回庄上来。
下了马,请入后堂饮酒。宋江弟兄两个,自此只在柴大官人庄上。

话分两头。只说武松自与宋江分别之后,当晚投客店歇了。次日早,起来打火,
吃了饭,还了房钱,拴束包裹,提了哨棒,便走上路,寻思道:“江湖上只闻说及
时雨宋公明,果然不虚。结识得这般弟兄,也不枉了!”

武松在路上行了几日,来到阳谷县地面。此去离县治还远。当日晌午时分,走
得肚中饥渴,望见前面有一个酒店,挑着一面招旗在门前,上头写着五个字道:“三
碗不过冈”。

武松入到里面坐下,把哨棒倚了,叫道:“主人家,快把酒来吃。”只见店主
人把三只碗,一双箸,一碟热菜,放在武松面前,满满筛一碗酒来。武松拿起碗,
一饮而尽,叫道:“这酒好生有气力!主人家,有饱肚的买些吃酒。”酒家道:“只
有熟牛肉。”武松道:“好的,切二三斤来吃酒。”店家去里面切出二斤熟牛肉,
做一大盘子,将来放在武松面前,随即再筛一碗酒。武松吃了道:“好酒!”又筛
下一碗。恰好吃了三碗酒,再也不来筛。武松敲着桌子叫道:“主人家,怎的不来
筛酒?”酒家道:“客官要肉便添来。”武松道:“我也要酒,也再切些肉来。”
酒家道:“肉便切来添与客官吃,酒却不添了。”武松道:“却又作怪!”便问主
人家道:“你如何不肯卖酒与我吃?”酒家道:“客官,你须见我门前招旗上面明
明写道:‘三碗不过冈’。”武松道:“怎地唤做‘三碗不过冈’?”

酒家道:“俺家的酒,虽是村酒,却比老酒的滋味。但凡客人来我店中,吃了
三碗的,便醉了,过不得前面的山冈去,因此唤做‘三碗不过冈’。若是过往客人
到此,只吃三碗,更不再问。”武松笑道:“原来恁地。我却吃了三碗,如何不醉?”
酒家道:“我这酒叫做透瓶香,又唤做出门倒。初入口时,醇好吃,少刻时便倒。”
武松道:“休要胡说!没地不还你钱,再筛三碗来我吃!”酒家见武松全然不动,
又筛三碗。武松吃道:“端的好酒!主人家,我吃一碗,还你一碗钱,只顾筛来。”
酒家道:“客官休只管要饮,这酒端的要醉倒人,没药医。”武松道:“休得胡鸟
说!便是你使蒙汗药在里面,我也有鼻子。”店家被他发话不过,一连又筛了三碗。
武松道:“肉便再把二斤来吃。”酒家又切了二斤熟牛肉,再筛了三碗酒。武松吃
得口滑,只顾要吃,去身边取出些碎银子,叫道:“主人家,你且来看我银子,还
你酒肉钱够么?”酒家看了道:“有余。还有些贴钱与你。”武松道:“不要你贴
钱。只将酒来筛。”酒家道:“客官,你要吃酒时,还有五六碗酒哩!只怕你吃不
的了。”武松道:“就有五六碗多时,你尽数筛将来。”酒家道:“你这条长汉,
倘或醉倒了时,怎扶的你住?”武松答道:“要你扶的,不算好汉。”酒家那里肯
将酒来筛。武松焦燥道:“我又不白吃你的!休要引老爷性发,通教你屋里粉碎!把
你这鸟店子倒翻转来!”酒家道:“这厮醉了,休惹他。”再筛了六碗酒,与武松
吃了。前后共吃了十五碗,绰了哨棒,立起身来道:“我却又不曾醉!”走出门前
来笑道:“却不说‘三碗不过冈’!”手提哨棒便走。

酒家赶出来叫道:“客官那里去!”武松立住了,问道:“叫我做甚么?我又
不少你酒钱,唤我怎地?”酒家叫道:“我是好意。你且回来我家,看抄白官司榜
文。”武松道:“甚么榜文?”酒家道:“如今前面景阳冈上有只吊睛白额大虫,
晚了出来伤人,坏了三二十条大汉性命。官司如今杖限猎户擒捉发落。冈子路口,
多有榜文:可教往来客人,结伙成队,于巳、午、未三个时辰过冈,其余寅、卯、
申、酉、戌、亥六个时辰,不许过冈。更兼单身客人,务要等伴结伙而过。这早晚
正是未末申初时分,我见你走都不问人,枉送了自家性命。不如就我此间歇了,等
明日慢慢凑的三二十人,一齐好过冈子。”武松听了,笑道:“我是清河县人氏,
这条景阳冈上,少也走过了一二十遭,几时见说有大虫?你休说这般鸟话来吓我。
便有大虫,我也不怕!”酒家道:“我是好意救你,你不信时,进来看官司榜文。”
武松道:“你鸟子声!便真个有虎,老爷也不怕!你留我在家里歇,莫不半夜三更,
要谋我财,害我性命,却把鸟大虫唬吓我。”酒家道:“你看么!我是一片好心,
反做恶意,倒落得你恁地!你不信我时,请尊便自行!”正是:
前车倒了千千辆,后车过了亦如然。
分明指与平川路,却把忠言当恶言。
那酒店里主人摇着头,自进店里去了。这武松提了哨棒,大着步,自过景阳冈来。
约行了四五里路,来到冈子下,见一大树,刮去了皮,一片白,上写两行字。武松
也颇识几字,抬头看时,上面写道:

近因景阳冈大虫伤人,但有过往客商,可于巳、午、未三个时辰,结伙成队过
冈,勿请自误。
武松看了,笑道:“这是酒家诡诈,惊吓那等客人,便去那厮家里宿歇。我却怕甚
么鸟!”横拖着哨棒,便上冈子来。

那时已有申牌时分,这轮红日,厌厌地相傍下山。武松乘着酒兴,只管走上冈
子来。走不到半里多路,见一个败落的山神庙。行到庙前,见这庙门上贴着一张印
信榜文。武松住了脚读时,上面写道:

阳谷县示:为景阳冈上,新有一只大虫,伤害人命。现今杖限各乡里正并猎户
人等行捕,未获。如有过往客商人等,可于巳、午、未三个时辰,结伴过冈;其余
时分及单身客人,不许过冈,恐被伤害性命。各宜知悉。

武松读了印信榜文,方知端的有虎。欲待转身再回酒店里来,寻思道:“我回
去时,须吃他耻笑,不是好汉,难以转去。”存想了一回,说道:“怕甚么鸟!且
只顾上去看怎地!”

武松正走,看看酒涌上来,便把毡笠儿背在脊梁上,将哨棒绾在肋下,一步步
上那冈子来。回头看这日色时,渐渐地坠下去了。此时正是十月间天气,日短夜长,
容易得晚。武松自言自说道:“那得甚么大虫?人自怕了,不敢上山。”武松走了
一直,酒力发作,焦热起来。一只手提着哨棒,一只手把胸膛前袒开,踉踉跄跄,
直奔过乱树林来。见一块光挞挞大青石,把那哨棒倚在一边,放翻身体,却待要睡,
只见发起一阵狂风来。古人有四句诗单道那风:
无形无影透人怀,四季能吹万物开。
就树撮将黄叶去,入山推出白云来。

原来但凡世上云生从龙,风生从虎。那一阵风过处,只听得乱树背后扑地一声
响,跳出一只吊睛白额大虫来。武松见了,叫声:“阿呀!”从青石上翻将下来,
便拿那条哨棒在手里,闪在青石边。

那个大虫又饥又渴,把两只爪在地下略按一按,和身望上一扑,从半空里撺将
下来。武松被那一惊,酒都做冷汗出了。说时迟,那时快,武松见大虫扑来,只一
闪,闪在大虫背后。那大虫背后看人最难,便把前爪搭在地下,把腰胯一掀,掀将
起来。武松只一躲,躲在一边。大虫见掀他不着,吼一声,却似半天里起个霹雳,
振得那山冈也动,把这铁棒也似虎尾,倒竖起来只一剪。武松却又闪在一边。原来
那大虫拿人,只是一扑,一掀,一剪;三般提不着时,气性先自没了一半。那大虫
又剪不着,再吼了一声,一兜兜将回来。武松见那大虫复翻身回来,双手抡起哨棒,
尽平生气力只一棒,从半空劈将下来。只听得一声响,簌簌地将那树连枝带叶劈脸
打将下来。定睛看时,一棒劈不着大虫。原来打急了,正打在枯树上,把那条哨棒
折做两截,只拿得一半在手里。

那大虫咆哮,性发起来,翻身又只一扑,扑将来。武松又只一跳,却退了十步
远。那大虫恰好把两只前爪搭在武松面前。武松将半截棒丢在一边,两只手就势把
大虫顶花皮地揪住,一按按将下来。那只大虫急要挣扎,被武松尽气力纳定,
那里肯放半点儿松宽?武松把只脚望大虫面门上、眼睛里,只顾乱踢。那大虫咆哮
起来,把身底下爬起两堆黄泥,做了一个土坑。武松把那大虫嘴直按下黄泥坑里去,
那大虫吃武松奈何得没了些气力。武松把左手紧紧地揪住顶花皮,偷出右手来,提
起铁锤般大小拳头,尽平生之力,只顾打。打到五七十拳,那大虫眼里、口里、鼻
子里、耳朵里,都迸出鲜血来。那武松尽平昔神威,仗胸中武艺,半歇儿把大虫打
做一堆,却似挡着一个锦皮袋。有一篇古风单道景阳冈武松打虎:
景阳冈头风正狂,万里阴云霾日光。
触目晚霞挂林薮,侵人冷雾弥穹苍。
忽闻一声霹雳响,山腰飞出兽中王。
昂头踊跃逞牙爪,麋鹿之属皆奔忙。
清河壮士酒未醒,冈头独坐忙相迎。
上下寻人虎饥渴,一掀一扑何狰狞!
虎来扑人似山倒,人往迎虎如岩倾。
臂腕落时坠飞炮,爪牙爬处成泥坑。
拳头脚尖如雨点,淋漓两手猩红染。
腥风血雨满松林,散乱毛须坠山奄。
近看千钧势有余,远观八面威风敛。
身横野草锦斑销,紧闭双睛光不闪。

当下景阳冈上那只猛虎,被武松没顿饭之间,一顿拳脚,打得那大虫动弹不得,
使得口里兀自气喘。武松放了手,来松树边寻那打折的棒橛,拿在手里;只怕大虫
不死,把棒橛又打了一回。那大虫气都没了,武松再寻思道:“我就地拖得这死大
虫下冈子去。”就血泊里双手来提时,那里提得动,原来使尽了气力,手脚都苏软
了。武松再来青石坐了半歇,寻思道:“天色看看黑了,倘或又跳出一只大虫来时,
却怎地斗得他过?且挣扎下冈子去,明早却来理会。”就石头边寻了毡笠儿,转过
乱树林边,一步步捱下冈子来。

走不到半里多路,只见枯草丛中,钻出两只大虫来。武松道:“阿呀!我今番
罢了!”只见那两个大虫,于黑影里直立起来。武松定睛看时,却是两个人,把虎
皮缝做衣裳,紧紧拼在身上。那两个人手里各拿着一条五股叉,见了武松,吃一惊
道:“你那人吃了律心、豹子肝、狮子腿,胆倒包着身躯,如何敢独自一个,昏
黑将夜,又没器械,走过冈子来!不知你是人是鬼?”武松道:“你两个是甚么人?”
那个人道:“我们是本处猎户。”武松道:“你们上岭来做甚么?”两个猎户失惊
道:“你兀自不知哩!如今景阳冈上,有一只极大的大虫,夜夜出来伤人。只我们
猎户,也折了七八个;过往客人,不记其数,都被这畜生吃了。本县知县着落当乡
里正和我们猎户人等捕捉。那业畜势大难近,谁敢向前!我们为他,正不知吃了多
少限棒,只捉他不得!今夜又该我们两个捕猎,和十数个乡夫在此,上上下下,放
了窝弓药箭等他。正在这里埋伏,却见你大剌剌地从冈子上走将下来,我两个吃了
一惊。你却正是甚人?曾见大虫么?”武松道:“我是清河县人氏,姓武,排行第
二。却才冈子上乱树林边,正撞见那大虫,被我一顿拳脚打死了。”两个猎户听得
痴呆了,说道:“怕没这话?”武松道:“你不信时,只看我身上兀自有血迹。”
两个道:“怎地打来?”武松把那打大虫的本事,再说了一遍。两个猎户听了,又
惊又喜,叫拢那十个乡夫来。

只见这十个乡夫,都拿着钢叉、踏弩、刀、枪,随即拢来。武松问道:“他们
众人,如何不随着你两个上山?”猎户道:“便是那畜生利害,他们如何敢上来?”
一伙十数个人,都在面前。两个猎户把武松打杀大虫的事,说向众人,众人都不肯
信。武松道:“你众人不信时,我和你去看便了。”众人身边都有火刀、火石、随
即发出火来,点起五七个火把。众人都跟着武松,一同再上冈子来,看见那大虫做
一堆儿死在那里。众人见了大喜,先叫一个去报知本县里正并该管上户。这里五七
个乡夫,自把大虫缚了,抬下冈子来。

到得岭下,早有七八十人,都哄将来,先把死大虫抬在前面,将一乘兜轿,抬
了武松,径投本处一个上户家来。那户里正,都在庄前迎接,把这大虫扛到草厅上。
却有本乡上户,本乡猎户,三二十人,都来相探武松。众人问道:“壮士高姓大名?
贵乡何处?”武松道:“小人是此间邻郡清河县人氏,姓武,名松,排行第二。因
从沧州回乡来,昨晚在冈子那边酒店吃得大醉了,上冈子来,正撞见这畜生。”把
那打虎的身分、拳脚,细说了一遍。众上户道:“真乃英雄好汉!”众猎户先把野
味将来与武松把杯。武松因打大虫困乏了,要睡,大户便叫庄客打并客房,且教武
松歇息。

到天明,上户先使人去县里报知,一面合具虎床,安排端正,迎送县里去。天
明,武松起来洗漱罢,众多上户牵一腔羊,挑一担酒,都在厅前伺候。武松穿了衣
裳,整顿巾帻,出到前面,与众人相见。众上户把盏说道:“被这个畜生,正不知
害了多少人性命,连累猎户,吃了几顿限棒。今日幸得壮士来到,除了这个大害。
第一,乡中人民有福;第二,客侣通行:实出壮士之赐!”武松谢道:“非小子之
能,托赖众长上福荫。”众人都来作贺。吃了一早晨酒食,抬出大虫,放在虎床上。
众乡村上户,都把缎匹花红,来挂与武松。武松有些行李包裹,寄在庄上。一齐都
出庄门前来。早有阳谷县知县相公,使人来接武松。都相见了,叫四个庄客,将乘
凉轿,来抬了武松。把那大虫扛在前面,挂着花红缎匹,迎到阳谷县里来。

那阳谷县人民,听得说一个壮士打死了景阳冈上大虫,迎喝将来,尽皆出来看,
哄动了那个县治。武松在轿上看时,只见亚肩迭背,闹闹穰穰,屯街塞巷,都来看
迎大虫。到县前衙门口,知县已在厅上专等。武松下了轿,扛着大虫,都到厅前,
放在甬道上。知县看了武松这般模样,又见了这个老大锦毛大虫,心中自忖道:“不
是这个汉,怎地打的这个猛虎!”便唤武松上厅来。武松去厅前声了喏,知县问道:
“你那打虎的壮士,你却说怎生打了这个大虫?”武松就厅前,将打虎的本事,说
了一遍。厅上厅下众多人等都惊的呆了,知县就厅上赐了几杯酒,将出上户凑的赏
赐钱一千贯,给与武松。武松禀道:“小人托赖相公的福荫,偶然侥幸,打死了这
个大虫,非小人之能,如何敢受赏赐?小人闻知这众猎户,因这个大虫,受了相公
责罚,何不就把这一千贯给散与众人去用?”知县道:“既是如此,任从壮士。”
武松就把这赏钱,在厅上散与众人猎户。知县见他忠厚仁德,有心要抬举他,便道:
“虽你原是清河县人氏,与我这阳谷县只在咫尺。我今日就参你在本县做个都头如
何?”武松跪谢道:“若蒙恩相抬举,小人终身受赐。”知县随即唤押司立了文案,
当日便参武松做了步兵都头。众上户都来与武松作贺庆喜,连连吃了三五日酒。武
松自心中想道:“我本要回清河县去看望哥哥,谁想倒来做了阳谷县都头。”自此
上官见爱,乡里闻名。

又过了三二日,那一日,武松走出县前来闲玩,只听得背后一个人叫声:“武
都头,你今日发迹了,如何不看觑我则个?”武松回过头来看了,叫声:“阿呀!
你如何却在这里?”不是武松见了这个人,有分教:阳谷县里,尸横血染。直教:
钢刀响处人头滚,宝剑挥时热血流。

毕竟叫唤武都头的正是甚人,且听下回分解。