\chapter{入云龙斗法破高廉~黑旋风探穴救柴进}

话说当下罗真人道:“弟子,你往日学的法术,却与高廉的一般。吾今传授与
汝五雷天罡正法,依此而行,可救宋江,保国安民,替天行道。休被人欲所缚,误
了大事,专精从前学道之心。你的老母,我自使人早晚看视,勿得忧念。汝应上界
天闲星,以此容汝去助宋公明。吾有八个字,汝当记取,休得临期有误。”罗真人
说那八个字,道是:“逢幽而止,遇汴而还。”公孙胜拜授了诀法,便和戴宗、李
逵三个,拜辞了罗真人,别了众道伴下山。归到家中,收拾了道衣,宝剑二口,并
铁冠如意等物了当,拜辞了老母,离山上路。行过了三四十里路程,戴宗道:“小
可先去报知哥哥,先生和李逵大路上来,却得再来相接。”公孙胜道:“正好。贤
弟先往报知,吾亦趱行来也。”戴宗分付李逵道:“于路小心伏侍先生。但有些差
池,教你受苦。”李逵道:“他和罗真人一般的法术,我如何敢轻慢了他?”戴宗
拴上甲马,作起神行法来,预先去了。

却说公孙胜和李逵两个,离了二仙山九宫县,取大路而行,到晚寻店安歇。李
逵惧怕罗真人法术,十分小心伏侍公孙胜,那里敢使性。两个行了三日,来到一个
去处,地名唤做武冈镇。只见街市人烟辏集,公孙胜道:“这两日于路走的困倦,
买碗素酒素面吃了行。”李逵道:“也好。”却见驿道旁边一个小酒店,两个人来
店里坐下。公孙胜坐了上首,李逵解了腰包,下首坐了。叫过卖一面打酒,就安排
些素馔来,与二人吃。公孙胜道:“你这里有甚素点心卖?”过卖道:“我店里只
卖酒肉,没有素点心,市口人家有枣糕卖。”李逵道:“我去买些来。”便去包内
取了铜钱,径投市镇上来,买了一包枣糕。欲待回来,只听得路旁侧首有人喝采道:
“好气力!”李逵看时,一伙人围定一个大汉,把铁瓜锤在那里使,众人看了喝采
他。

李逵看那大汉时,七尺以上身材,面皮有麻,鼻子上一条大路。李逵看那铁锤
时,约有三十来斤。那汉使的发了,一瓜锤正打在压街石上,把那石头打做粉碎,
众人喝采。李逵忍不住,便把枣糕揣在怀中,便来拿那铁锤。那汉喝道:“你是甚
么鸟人?敢来拿我的锤!”李逵道:“你使的甚么鸟好,教众人喝采!看了倒污眼!
你看老爷使一回,教众人看。”那汉道:“我借与你,你若使不动时,且吃我一顿
脖子拳了去。”李逵接过瓜锤,如弄弹丸一般。使了一回,轻轻放下,面又不红,
心头不跳,口内不喘。那汉看了,倒身下拜,说道:“愿求哥哥大名。”李逵道:
“你家在那里住?”那汉道:“只在前面便是。”引了李逵到一个所在,见一把锁
锁着门。那汉把钥匙开了门,请李逵到里面坐地。李逵看他屋里都是铁砧、铁锤、
火炉、钳、凿家火,寻思道:“这人必是个打铁匠人,山寨里正用得着,何不叫他
也去入伙?”

李逵又道:“汉子,你通个姓名,教我知道。”那汉道:“小人姓汤,名隆。
父亲原是延安府知寨官,因为打铁上,遭际老种经略相公帐前叙用。近年父亲在任
亡故,小人贪赌,流落在江湖上,因此权在此间打铁度日。入骨好使枪棒。为是自
家浑身有麻点,人都叫小人做‘金钱豹子’。敢问哥哥高姓大名?”李逵道:“我
便是梁山泊好汉黑旋风李逵。”汤隆听了,再拜道:“多闻哥哥威名,谁想今日偶
然得遇。”李逵道:“你在这里,几时得发迹?不如跟我上梁山泊入伙,叫你也做
个头领。”汤隆道:“若得哥哥不弃,肯带携兄弟时,愿随鞭镫。”就拜李逵为兄。
李逵认汤隆为弟。汤隆道:“我又无家人伴当,同哥哥去市镇上吃三杯淡酒,表结
拜之意。今晚歇一夜,明日早行。”李逵道:“我有个师父在前面酒店里,等我买
枣糕去吃了便行,搁不得,只可如今便行。”汤隆道:“如何这般要紧?”李逵
道:“你不知宋公明哥哥,现今在高唐州界首厮杀,只等我这师父到来救应。”汤
隆道:“这个师父是谁?”李逵道:“你且休问,快收拾了去。”汤隆急急拴了包
裹、盘缠、银两,戴上毡笠儿,跨了口腰刀,提条朴刀,弃了家中破房旧屋,粗重
家火,跟了李逵,直到酒店里来见公孙胜。

公孙胜埋怨道:“你如何去了许多时?再来迟些,我依前回去了。”李逵不敢
做声回话。引过汤隆拜了公孙胜,备说结义一事。公孙胜见说他是打铁出身,心中
也喜。李逵取出枣糕,叫过卖将去整理。三个一同饮了几杯酒,吃了枣糕,算还了
酒钱。李逵、汤隆各背上包裹,与公孙胜离了武冈镇,迤望高唐州来。三个于路,
三停中走了两停多路,那日早,却好迎着戴宗来接。公孙胜见了大喜,连忙问道:
“近日相战如何?”戴宗道:“高廉那厮,近日箭疮平复,每日领兵来搦战。哥哥
坚守,不敢出敌,只等先生到来。”公孙胜道:“这个容易。”李逵引着汤隆拜见
戴宗,说了备细,四人一处奔高唐州来。离寨五里远,早有吕方、郭盛,引一百余
骑军马迎接着。四人都上了马,一同到寨,宋江、吴用等出寨迎接。各施礼罢,摆
了接风酒,叙问间阔之情,请入中军帐内,众头领亦来作庆。李逵引过汤隆来参见
宋江、吴用,并众头领等。讲礼已罢,寨中且做庆贺筵席。

次日中军帐上,宋江、吴用、公孙胜商议破高廉一事,公孙胜道:“主将传令,
且着拔寨都起,看敌军如何,贫道自有区处。”当日宋江传令各寨,一齐引军起身,
直抵高唐州城壕,下寨已定。次早五更造饭,军人都披挂衣甲。宋公明、吴学究、
公孙胜,三骑马直到军前,摇旗擂鼓,呐喊筛锣,杀到城下来。

再说知府高廉在城中箭疮已痊,隔夜小军来报知宋江军马又到,早晨都披挂了
衣甲,便开了城门,放下吊桥,将引三百神兵并大小将校,出城迎敌。两军渐近,
旗鼓相望,各摆开阵势。两阵里花腔鼍鼓擂,杂彩绣旗摇。宋江阵门开处,分十骑
马来,雁翅般摆开在两边。左手下五将:花荣、秦明、朱仝、欧鹏、吕方;右手下
五将:是林冲、孙立、邓飞、马麟、郭盛;中间三骑马上,为头是主将宋公明,怎
生打扮:

头顶茜红巾,腰系狮蛮带。锦征袍大鹏贴背,水银盔彩凤飞檐。抹绿靴斜踏宝
镫,黄金甲光动龙鳞。描金随定紫丝鞭,锦鞍稳称桃花马。
左边那骑马上,坐着的便是梁山泊掌握兵权军师吴学究,怎生打扮:

五明扇齐攒白羽,九纶巾巧簇乌纱。素罗袍香皂沿边,碧玉环丝绦束定。凫舄
稳踏葵花镫,银鞍不离紫丝缰。两条铜链腰间挂,一骑青骢出战场。
右边那骑马上,坐着的便是梁山泊掌握行兵布阵副军师公孙胜,怎生打扮:

星冠耀日,神剑飞霜。九霞衣服绣春云,六甲风雷藏宝诀。腰间系杂色短须绦,
背上悬松文古定剑。穿一双云头点翠早朝靴,骑一匹分鬃昂首黄花马。名标蕊笈玄
功著,身列仙班道行高。
三个总军主将,三骑马出到阵前。看对阵金鼓齐鸣,门旗开处,也有二三十个军官,
簇拥着高唐州知府高廉出在阵前,立马于门旗下。怎生结束,但见:

束发冠珍珠嵌就,绛红袍锦绣攒成。连环铠甲耀黄金,双翅银盔飞彩凤。足穿
云缝吊墩靴,腰系狮蛮金带。手内剑横三尺水,阵前马跨一条龙。

那知府高廉出到阵前,厉声高叫,喝骂道:“你那水洼草贼,既有心要来厮杀,
定要分个胜败,见个输赢,走的不是好汉!”宋江听罢,问一声:“谁人出马立斩
此贼?”小李广花荣挺枪跃马,直至垓心。高廉见了,喝问道:“谁与我直取此贼
去?”那统制官队里转出一员上将,唤做薛元辉,使两口双刀,骑一匹劣马,飞出
垓心,来战花荣。两个在阵前斗了数合,花荣拨回马,望本阵便走。薛元辉不知是
计,纵马舞刀,尽力来赶。花荣略带住了马,拈弓取箭,扭转身躯,只一箭,把薛
元辉头重脚轻,射下马去。两军齐呐声喊。高廉在马上见了大怒,急去马鞍鞒前,
取下那面聚兽铜牌,把剑去击。那里敲得三下,只见神兵队里卷起一阵黄砂来,罩
的天昏地暗,日色无光。喊声起处,豺狼虎豹,怪兽毒虫,就这黄砂内卷将出来。
众军恰待都走,公孙胜在马上,早掣出那一把松文古定剑来,指着敌军,口中念念
有词,喝声道:“疾!”只见一道金光射去,那伙怪兽毒虫,都就黄砂中乱纷纷坠
于阵前。众军人看时,却都是白纸剪的虎豹走兽,黄砂尽皆荡散不起。宋江看了,
鞭梢一指,大小三军,一齐掩杀过去。但见人亡马倒,旗鼓交横。高廉急把神兵退
走入城。宋江军马赶到城下,城上急拽起吊桥,闭上城门,擂木炮石,如雨般打将
下来。宋江叫且鸣金,收聚军马下寨,整点人数,各获大胜。回帐称谢公孙先生神
功道德,随即赏劳三军。

次日,分兵四面围城,尽力攻打,公孙胜对宋江、吴用道:“昨夜虽是杀败敌
军大半,眼见得那三百神兵退入城中去了。今日攻击得紧,那厮夜间必来偷营劫寨。
今晚可收军一处,至夜深,分去四面埋伏。这里虚扎寨栅,教众将只听霹雳响,看
寨中火起,一齐进兵。”传令已了。当日攻城至未牌时分,都收四面军兵还寨,却
在营中大吹大擂饮酒。看看天色渐晚,众头领暗暗分拨开去,四面埋伏已定。

却说宋江、吴用、公孙胜、花荣、秦明、吕方、郭盛上土坡等候。是夜,高廉
果然点起三百神兵,背上各带铁葫芦,于内藏着硫黄焰硝,烟火药料;各人俱执钩
刃、铁扫帚,口内都衔芦哨。二更前后,大开城门,放下吊桥,高廉当先,驱领神
兵前进,背后却带三十余骑,奔杀前来。离寨渐近,高廉在马上作起妖法,却早黑
气冲天,狂风大作,飞砂走石,播土扬尘。三百神兵各取火种,去那葫芦口上点着,
一声芦哨齐响,黑气中间,火光罩身,大刀阔斧,滚入寨里来。高埠处,公孙胜仗
剑作法,就空寨中平地上刮剌剌起个霹雳。三百神兵急待退步,只见那空寨中火起,
光焰乱飞,上下通红,无路可出。四面伏兵齐赶,围定寨栅,黑处遍见。三百神兵,
不曾走得一个,都被杀在寨里。高廉急引了三十余骑,奔走回城。背后一枝军马追
赶将来,乃是豹子头林冲。看看赶上,急叫得放下吊桥,高廉只带得八九骑入城,
其余尽被林冲和人连马生擒活捉了去。高廉进到城中,尽点百姓上城守护。高廉军
马神兵,被宋江、林冲杀个尽绝。

次日,宋江又引军马四面围城甚急。高廉寻思:“我数年学得术法,不想今日
被他破了,似此如之奈何?只得使人去邻近州府求救。”急急修书二封,教去东昌、
寇州,二处离此不远,“这两个知府,都是我哥哥抬举的人。”教星夜起兵来接应。
差了两个帐前统制官,赍擎书信,放开西门,杀将出来,投西夺路去了。众将却待
去追赶,吴用传令:“且放他出去,可以将计就计。”宋江问道:“军师如何作用?”
吴学究道:“城中兵微将寡,所以他去求救。我这里可使两枝人马,诈作救应军兵,
于路混战。高廉必然开门助战,乘势一面取城,把高廉引入小路,必然擒获。”宋
江听了大喜。令戴宗回梁山泊另取两枝军马,分作两路而来。

且说高廉每夜在城中空阔处,堆积柴草,竟天价放火为号,城上只望救兵到来。
过了数日,守城军兵望见宋江阵中不战自乱,急忙报知。高廉听了,连忙披挂上城
瞻望,只见两路人马战尘蔽日,喊杀连天,冲奔前来,四面围城军马,四散奔走。
高廉知是两路救军到了,尽点在城军马,大开城门,分头掩杀出去。

且说高廉撞到宋江阵前,看见宋江引着花荣、秦明,三骑马望小路而走。高廉
引了人马,急去追赶,忽听得山坡后连珠炮响,心中疑惑,便收转人马回来。两边
锣响,左手下吕方,右手下郭盛,各引五百人马冲将出来。高廉急夺路走时,部下
军马折其大半,奔走脱得垓心时,望见城上已都是梁山泊旗号。举眼再看,无一处
是救应军马,只得引着些败卒残兵,投山僻小路而走。行不到十里之外,山背后撞
出一彪人马,当先拥出病尉迟孙立,拦住去路,厉声高叫:“我等你多时,好好下
马受缚!”高廉引军便回,背后早有一彪人马,截住去路,当先马上却是美髯公朱
仝。两头夹攻将来,四面截了去路,高廉便弃了坐下马便走上山。四下里部军一齐
赶上山去,高廉慌忙口中念念有词,喝声道:“起!”驾一片黑云,冉冉腾空,直
上山顶。只见山坡边转出公孙胜来,见了,便把剑在马上望空作用,口中也念念有
词,喝声道:“疾!”将剑望上一指,只见高廉从云中倒撞下来。侧首抢过插翅虎
雷横,一朴刀把高廉挥做两段。可怜五马诸侯贵,化作南柯梦里人。有诗为证:
上临之以天鉴,下察之以地。
明有王法相继,暗有鬼神相随。
行凶毕竟逢凶,恃势还归失势。
劝君自警平生,可叹可惊可畏。

且说雷横提了首级,都下山来,先使人去飞报主帅。宋江已知杀了高廉,收军
进高唐州城内,先传下将令,休得伤害百姓。一面出榜安民,秋毫无犯,且去大牢
中救出柴大官人来。那时当牢节级、押狱禁子,已都走了,止有三五十个罪囚,尽
数开了枷锁释放。数中只不见柴大官人一个,宋江心中忧闷。寻到一处监房内,却
监着柴皇城一家老小;又一座牢内,监着沧州提捉到柴进一家老小,同监在彼,为
是连日厮杀,未曾取问发落,只是没寻柴大官人处。

吴学究教唤集高唐州押狱禁子跟问时,数内有一个禀道:“小人是当牢节级蔺
仁。前日蒙知府高廉所委,专一牢固监守柴进,不得有失。又分付道:‘但有凶吉,
你可便下手。’三日之前,知府高廉要取柴进出来施刑。小人为见本人是个好男子,
不忍下手,只推道:‘本人病至八分,不必下手。’后又催并得紧,小人回称‘柴
进已死’。因是连日厮杀,知府不闲,小人却恐他差人下来看视,必见罪责,昨日
引柴进去后面枯井边,开了枷锁,推放里面躲避,如今不知存亡。”

宋江听了,慌忙着蔺仁引入。直到后牢枯井边望时,见里面黑洞洞地,不知多
少深浅。上面叫时,那得人应,把索子放下去探时,约有八九丈深。宋江道:“柴
大官人眼见得多是没了。”宋江垂泪。吴学究道:“主帅且休烦恼。谁人敢下去探
看一遭,便见有无。”说犹未了,转过黑旋风李逵来,大叫道:“等我下去。”宋
江道:“正好。当初也是你送了他,今日正宜报本。”李逵笑道:“我下去不怕,
你们莫割断了绳索。”吴学究道:“你却也忒奸猾。”且取一个大篾箩,把索子络
了,接长索头,扎起一个架子,把索挂在上面。李逵脱得赤条条的,手拿两把板斧,
坐在箩里,却放下井里去,索上缚两个铜铃。渐渐放到底下,李逵却从箩里爬将出
来,去井底下摸时,摸着一堆,却是骸骨。李逵道:“爷娘,甚鸟东西在这里!”
又去这边摸时,底下湿漉漉的,没下脚处。李逵把双斧拔放箩里,两手去摸底下,
四边却宽。一摸摸着一个人,做一堆儿蹲在水坑里。李逵叫一声:“柴大官人!”
那里见动,把手去摸时,只觉口内微微声唤。李逵道:“谢天地,恁地时,还有救
哩!”随即爬在箩里,摇动铜铃,众人扯将上来。李逵说下面的事,宋江道:“你
可再下去,先把柴大官人放在箩里,先发上来,却再放箩下来取你。”李逵道:“哥
哥不知我去蓟州,着了两道儿,今番休撞第三遍。”宋江笑道:“我如何肯弄你?
你快下去。”李逵只得再坐箩里,又下井去。

到得底下,李逵爬将出箩去,却把柴大官人抱在箩里,摇动索上铜铃。上面听
得,早扯起来。到上面,众人看了大喜。宋江见柴进头破额裂,两腿皮肉打烂,眼
目略开又闭。宋江心中甚是凄惨,叫请医生调治。李逵却在井底下发喊大叫。宋江
听得,急叫把箩放将下去,取他上来。李逵到得上面,发作道:“你们也不是好人,
便不把箩放下来救我!”宋江道:“我们只顾看顾柴大官人,因此忘了你,休怪。”
宋江就令众人把柴进扛扶上车睡了。先把两家老小,并夺转许多家财,共有二十余
辆车子,叫李逵、雷横,先护送上梁山泊去。却把高廉一家老小良贱三四十口,处
斩于市。赏谢了蔺仁,再把府库财帛,仓廒粮米,并高廉所有家私,尽数装载上山。
大小将校离了高唐州,得胜回梁山泊。所过州县,秋毫无犯。在路已经数日,回到
大寨,柴进扶病起来,称谢晁、宋二公并众头领。晁盖教请柴大官人就山顶宋公明
歇处,另建一所房子,与柴进并家眷安歇。晁盖、宋江等众皆大喜。自高唐州回来,
又添得柴进、汤隆两个头领,且作庆贺筵席,不在话下。

再说东昌、寇州两处,已知高唐州杀了高廉,失陷了城池,只得写表差人申奏
朝廷。又有高唐州逃难官员,都到京师说知真实。高太尉听了,知道杀死他兄弟高
廉。次日五更,在待漏院中,专等景阳钟响。百官各具公服,直临丹墀,伺候朝见。
当日五更三点,道君皇帝升殿。净鞭三下响,文武两班齐。天子驾坐,殿头官喝道:
“有事出班启奏,无事卷帘退朝。”高太尉出班奏曰:“今有济州梁山泊贼首晁盖、
宋江,累造大恶。打劫城池,抢掳仓廒,聚集凶徒恶党,现在济州杀害官军,闹了
江州无为军,今又将高唐州官民杀戮一空,仓廒库藏,尽被掳去。此是心腹大患,
若不早行诛剿,他日养成贼势,难以制伏。伏乞圣断。”天子闻奏大惊,随即降下
圣旨,就委高太尉选将调兵,前去剿捕,务要扫清水泊,杀绝种类。高太尉又奏道:
“量此草寇,不必兴举大兵。臣保一人,可去收复。”天子道:“卿若举用,必无
差错,即令起行,飞捷报功,加官赐赏,高迁任用。”高太尉奏道:“此人乃开国
之初,河东名将呼延赞嫡派子孙,单名唤个灼字,使两条铜鞭,有万夫不当之勇。
现受汝宁郡都统制,手下多有精兵勇将。臣举保此人,可以征剿梁山泊。可授兵马
指挥使,领马步精锐军士,克日扫清山寨,班师还朝。”天子准奏,降下圣旨:“着
枢密院即便差人赍敕前往汝宁州,星夜宣取。”当日朝罢,高太尉就于帅府着枢密
院拨一员军官,赍擎圣旨,前去宣取。当日起行,限时定日,要呼延灼赴京听命。

却说呼延灼在汝宁州统军司坐衙,听得门人报道:“有圣旨特来宣取将军赴京,
有委用的事。”呼延灼与本州官员出郭迎接到统军司。开读已罢,设宴管待使臣,
火急收拾了头盔衣甲,鞍马器械,带引三四十从人,一同使命,离了汝宁州,星夜
赴京。于路无话,早到京师城内殿司府前下马,来见高太尉。当日高俅正在殿帅府
坐衙,门吏报道:“汝宁州宣到呼延灼,现在门外。”高太尉大喜,叫唤进来参见
了。看那呼延灼一表非俗,正是:

开国功臣后裔,先朝良将玄孙。家传鞭法最通神,英武熟经战阵。

仗剑能
探虎穴,弯弓解射雕群。将军出世定乾坤,呼延灼威名大振。
当下高太尉问慰已毕,与了赏赐。次日早朝,引见道君皇帝。徽宗天子看了呼延灼
一表非俗,喜动天颜,就赐踢雪乌骓一匹。那马浑身墨锭似黑,四蹄雪练价白,因
此名为踢雪乌骓。那马日行千里。圣旨赐与呼延灼骑坐。呼延灼就谢恩已罢,随高
太尉再到殿帅府,商议起军,剿捕梁山泊一事。呼延灼道:“禀明恩相:小人觑探
梁山泊兵多将广,武艺高强,不可轻敌小觑。乞保二将为先锋,同提军马到彼,必
获大功。”高太尉听罢大喜,问道:“将军所保谁人,可为前部先锋?”不争呼延
灼举保此二将,有分教:宛子城重添良将,梁山泊大破官军。且教功名未上凌烟阁,
姓字先标聚义厅。

毕竟呼延灼对高太尉保出谁来,且听下回分解。