\chapter{吴学究双掌连环计~宋公明三打祝家庄}

话说当时军师吴用启烦戴宗道:“贤弟可与我回山寨去取铁面孔目裴宣、圣手
书生萧让、通臂猿侯健、玉臂匠金大坚。可教此四人带了如此行头,连夜下山来,
我自有用他处。”戴宗去了。

只见寨外军士来报,西村扈家庄上扈成牵牛担酒,特来求见。宋江叫请入来。
扈成来到中军帐前,再拜恳告道:“小妹一时粗卤,年幼不省人事,误犯威颜,今
者被擒,望乞将军宽恕。奈缘小妹原许祝家庄上,前者不合奋一时之勇,陷于缧绁。
如蒙将军饶放,但用之物,当依命拜奉。”宋江道:“且请坐说话。祝家庄那厮,
好生无礼,平白欺负俺山寨,因此行兵报仇,须与你扈家无冤。只是令妹引人捉了
我王矮虎,因此还礼,拿了令妹。你把王矮虎放回还我,我便把令妹还你。”扈成
答道:“不期已被祝家庄拿了这个好汉去。”吴学究便道:“我这王矮虎,今在何
处?”扈成道:“如今拘锁在祝家庄上,小人怎敢去取?”宋江道:“你不去取得
王矮虎来还我,如何能够得你令妹回去?”吴学究道:“兄长休如此说,只依小生
一言:今后早晚祝家庄上,但有些响亮,你的庄上,切不可令人来救护。倘或祝家
庄上有人投奔你处,你可就缚在彼。若是捉下得人时,那时送还令妹到贵庄。只是
如今不在本寨,前日已使人送在山寨,奉养在宋太公处。你且放心回去,我这里自
有个道理。”扈成道:“今番断然不敢去救应他,若是他庄上果有人来投我时,定
缚来奉献将军麾下。”宋江道:“你若是如此,便强似送我金帛。”扈成拜谢了去。

且说孙立却把旗号上改唤作“登州兵马提辖孙立”,领了一行人马,都来到祝
家庄后门前。庄上墙里望见是登州旗号,报入庄里去。栾廷玉听得是登州孙提辖到
来相望,说与祝氏三杰道:“这孙提辖是我弟兄,自幼与他同师学艺,今日不知如
何到此?”带了二十余人马,开了庄门,放下吊桥,出来迎接。孙立一行人都下了
马,众人讲礼已罢,栾廷玉问道:“贤弟在登州守把,如何到此?”孙立答道:“总
兵府行下文书,对调我来此间郓州守把城池,提防梁山泊强寇,便道经过,闻知仁
兄在此祝家庄,特来相探。本待从前门来,因见村口庄前俱屯下许多军马,不好冲
突,特地寻觅村里,从小路问到庄后,入来拜望仁兄。”栾廷玉道:“便是这几时
连日与梁山泊强寇厮杀,已拿得他几个头领在庄里了,只要捉了宋江贼首,一并解
官。天幸今得贤弟来此间镇守,正如锦上添花,旱苗得雨。”孙立笑道:“小弟不
才,且看相助捉拿这厮们,成全兄长之功。”栾廷玉大喜,当下都引一行人进庄里
来,再拽起了吊桥,关上了庄门。孙立一行人安顿车仗人马,更换衣裳,都在前厅
来相见。祝朝奉与祝龙、祝虎、祝彪三杰,都相见了,一家儿都在厅前相接。

栾廷玉引孙立等上到厅上相见,讲礼已罢,便对祝朝奉说道:“我这个贤弟孙
立,绰号病尉迟,任登州兵马提辖。今奉总兵府对调他来,镇守此间郓州。”祝朝
奉道:“老夫亦是治下。”孙立道:“卑小之职,何足道哉!早晚也要望朝奉提携
指教。”祝氏三杰相请众位尊坐。孙立动问道:“连日相杀,征阵劳神。”祝龙答
道:“也未见胜败。众位尊兄,鞍马劳神不易。”孙立便叫顾大嫂引了乐大娘子叔
伯姆两个去后堂见拜宅眷,唤过孙新、解珍、解宝参见了,说道:这三个是我兄弟。”
指着乐和便道:“这位是此间郓州差来取的公吏。”指着邹渊、邹润道:“这两个
是登州送来的军官。”祝朝奉并三子虽是聪明,却见他又有老小,并许多行李车仗
人马,又是栾廷玉教师的兄弟,那里有疑心,只顾杀牛宰马,做筵席管待众人,且
饮酒食。

过了一两日,到第三日,庄兵报道:“宋江又调军马杀奔庄上来了。”祝彪道:
“我自去上马拿此贼。”便出庄门,放下吊桥,引一百余骑马军杀将出来。早迎见
一彪军马,约有五百来人,当先拥出那个头领,弯弓插箭,拍马抡枪,乃是小李广
花荣。祝彪见了,跃马挺枪,向前来斗,花荣也纵马来战祝彪。两个在独龙冈前,
约斗了十数合,不分胜败。花荣卖个破绽,拨回马便走,引他赶来。祝彪正待要纵
马追去,背后有认得的说道:“将军休要去赶。恐防暗器,此人深好弓箭。”祝彪
听罢,便勒转马来不赶,领回人马投庄上来,拽起吊桥,看花荣时,也引军马回去
了。祝彪直到厅前下马,进后堂来饮酒。孙立动问道:“小将军今日拿得甚贼?”
祝彪道:“这厮们伙里有个甚么小李广花荣,枪法好生了得。斗了五十余合,那厮
走了,我却待要赶去追他,军人们道,那厮好弓箭,因此各自收兵回来。”孙立道:
“来日看小弟不才,拿他几个。”当日筵席上叫乐和唱曲,众人皆喜。

至晚席散,又歇了一夜,到第四日午牌,忽有庄兵报道:“宋江军马又来在庄
前了。”堂下祝龙、祝虎、祝彪三子都披挂了,出到庄前门外,远远地望见,早听
得鸣锣擂鼓,呐喊摇旗,对面早摆下阵势。这里祝朝奉坐在庄门上,左边栾廷玉,
右边孙提辖,祝家三杰,并孙立带来的许多人伴,都摆在两边。早见宋江阵上豹子
头林冲高声叫骂,祝龙焦躁,喝叫放下吊桥,绰枪上马,引一二百人马,大喊一声,
直奔林冲阵上。庄门下擂起鼓来,两边各把弓弩射住阵脚。林冲挺起丈八蛇矛,和
祝龙交战,连斗到三十余合,不分胜败。两边鸣锣,各回了马。祝虎大怒,提刀上
马,跑到阵前,高声大叫宋江决战。说言未了,宋江阵上早有一将出马,乃是没遮
拦穆弘来战祝虎。两个斗了三十余合,又没胜败。祝彪见了大怒,便绰枪飞身上马,
引二百余骑,奔到阵前。宋江队里病关索杨雄,一骑马,一条枪,飞抢出来战祝彪。

孙立看见两队儿在阵前厮杀,心中忍耐不住,便唤孙新:“取我的鞭枪来,就
将我的衣甲、头盔、袍袄把来披挂了。”牵过自己马来,——这骑马号乌骓马,鞴
上鞍子,扣了三条肚带,腕上悬了虎眼钢鞭,绰枪上马。祝家庄上,一声锣响,孙
立出马在阵前。宋江阵上林冲、穆弘、杨雄都勒住马,立于阵前。孙立早跑马出来,
说道:“看小可捉这厮们!”孙立把马兜住,喝问道:“你那贼兵阵上有好厮杀的,
出来与我决战。”宋江阵内鸾铃响处,一骑马跑将出来,众人看时,乃是拚命三郎
石秀来战孙立。两马相交,双枪并举。两个斗到五十合,孙立卖个破绽,让石秀枪
搠入来,虚闪一个过,把石秀轻轻的从马上捉过来,直挟到庄前撇下,喝道:“把
来缚了。”祝家三子把宋江军马一搅,都赶散了。三子收军回到门楼下,见了孙立,
众皆拱手钦伏。孙立便问道:“共是捉得几个贼人?”祝朝奉道:“起初先捉得一
个时迁,次后拿得一个细作杨林,又捉得一个黄信;扈家庄一丈青捉得一个王矮虎;
阵上拿得两个:秦明、邓飞;今番将军又捉得这个石秀,这厮正是烧了我店屋的。
共是七个了。”孙立道:“一个也不要坏他,快做七辆囚车装了,与些酒饭,将养
身体,休教饿损了他,不好看。他日拿了宋江,一并解上东京去,教天下传名,说
这个祝家庄三杰。”祝朝奉谢道:“多幸得提辖相助,想是这梁山泊当灭也。”邀
请孙立到后堂筵宴,石秀自把囚车装了。看官听说,石秀的武艺不低似孙立,要赚
祝家庄人,故意教孙立捉了,使他庄上人一发信他。孙立又暗暗地使邹渊、邹润、
乐和去后房里把门户都看了出入的路数。杨林、邓飞见了邹渊、邹润,心中暗喜。
乐和张看得没人,便透个消息与众人知了。顾大嫂与乐大娘子在里面已看了房户出
入的门径。

至第五日,孙立等众人都在庄上闲行,当日辰牌时候,早饭已后,只见庄兵报
道:“今日宋江分兵做四路,来打本庄。”孙立道:“分十路待怎地?你手下人且
不要慌,早作准备便了。先安排些挠钩套索,须要活捉,拿死的也不算。”庄上人
都披挂了,祝朝奉亲自率引着一班儿上门楼来看时,见正东上一彪人马,当先一个
头领,乃是豹子头林冲,背后便是李俊、阮小二,约有五百以上人马在此。正西上
又有五百来人马,当先一个头领,乃是小李广花荣,随背后是张横、张顺。正南门
楼上望时,也有五百来人马,当先三个头领,乃是没遮拦穆弘、病关索杨雄、黑旋
风李逵。四面都是兵马,战鼓齐鸣,喊声大举。栾廷玉听了道:“今日这厮们厮杀,
不可轻敌。我引了一队人马出后门,杀这正西北上的人马。”祝龙道:“我出前门,
杀这正东上的人马。”祝虎道:“我也出后门,杀那西南上的人马。”祝彪道:“我
自出前门,捉宋江,是要紧的贼首。”祝朝奉大喜,都赏了酒。各人上马,尽带了
三百余骑奔出庄门,其余的都守庄院门楼前呐喊。此时邹渊、邹润已藏了大斧,只
守在监门左侧。解珍、解宝藏了暗器,不离后门。孙新、乐和已守定前门左右。顾
大嫂先拨军兵保护乐大娘子,却自拿了两把双刀在堂前踅,只听风声,便乃下手。

且说祝家庄上擂了三通战鼓,放了一个炮,把前后门都开,放下吊桥,一齐杀
将出来。四路军兵出了门,四下里分投去厮杀。临后孙立带了十数个军兵,立在吊
桥上。门里孙新便把原带来的旗号插起在门楼上,乐和便提着枪,直唱将出来。邹
渊、邹润听得乐和唱,便唿哨了几声,抡动大斧,早把守监门的庄兵砍翻了数十个,
便开了陷车,放出七只大虫来,各各寻了器械,一声喊起。顾大嫂掣出两把刀,直
奔入房里,把应有妇人,一刀一个,尽都杀了。祝朝奉见头势不好了,却待要投井
时,早被石秀一刀剁翻,割了首级。那十数个好汉,分投来杀庄兵。后门头解珍、
解宝便去马草堆里放起把火,黑焰冲天而起。

四路人马见庄上火起,并力向前。祝虎见庄里火起,先奔回来。孙立守在吊桥
上,大喝一声:“你那厮那里去?”拦住吊桥。祝虎省口,便拨转马头,再奔宋江
阵上来。这里吕方、郭盛两戟齐举,早把祝虎和人连马搠翻在地,众军乱上,剁做
肉泥。前军四散奔走。孙立、孙新迎接宋公明入庄。

且说东路祝龙斗林冲不住,飞马望庄后而来。到得吊桥边,见后门头解珍、解
宝把庄客的尸首一个个撺将下来火焰里,祝龙急回马,望北而走。猛然撞着黑旋风,
踊身便到,抡动双斧,早砍翻马脚。祝龙措手不及,倒撞下来,被李逵只一斧,把
头劈翻在地。祝彪见庄兵走来报知,不敢回,直望扈家庄投奔,被扈成叫庄客捉了,
绑缚下,正解将来见宋江。恰好遇着李逵,只一斧,砍翻祝彪头来,庄客都四散走
了。李逵再抡起双斧,便看着扈成砍来。扈成见局面不好,投马落荒而走,弃家逃
命,投延安府去了。后来中兴内也做了个军官武将。

且说李逵正杀得手顺,直抢入扈家庄里,把扈太公一门老幼,尽数杀了,不留
一个。叫小喽罗牵了有的马匹,把庄里一应有的财赋,捎搭有四五十驮,将庄院门
一把火烧了,却回来献纳。

再说宋江已在祝家庄上正厅坐下,众头领都来献功,生擒得四五百人,夺得好
马五百余匹,活捉牛羊不计其数。宋江见了,大喜道:“只可惜杀了栾廷玉那个好
汉。”正嗟叹间,闻人报道:“黑旋风烧了扈家庄,砍得头来献纳。”宋江便道:
“前日扈成已来投降,谁教他杀了此人?如何烧了他庄院?”只见黑旋风一身血污,
腰里插着两把板斧,直到宋江面前,唱个大喏,说道:“祝龙是兄弟杀了,祝彪也
是兄弟砍了,扈成那厮走了,扈太公一家,都杀得干干净净,兄弟特来请功。”宋
江喝道:“祝龙曾有人见你杀了,别的怎地是你杀了?”黑旋风道:“我砍得手顺,
望扈家庄赶去,正撞见一丈青的哥哥,解那祝彪出来,被我一斧砍了,只可惜走了
扈成那厮。他家庄上,被我杀得一个也没了。”宋江喝道:“你这厮,谁叫你去来?
你也须知扈成前日牵牛担酒,前来投降了,如何不听得我的言语,擅自去杀他一家,
故违了我的将令?”李逵道:“你便忘记了,我须不忘记。那厮前日教那个鸟婆娘
赶着哥哥要杀,你今却又做人情。你又不曾和他妹子成亲,便又思量阿舅、丈人。”
宋江喝道:“你这铁牛,休得胡说!我如何肯要这妇人?我自有个处置。你这黑厮,
拿得活的有几个?”李逵答道:“谁鸟耐烦,见着活的便砍了。”宋江道:“你这
厮违了我的军令,本合斩首,且把杀祝龙、祝彪的功劳折过了,下次违令,定行不
饶。”黑旋风笑道:“虽然没了功劳,也吃我杀得快活。”

只见军师吴学究引着一行人马,都到庄上来与宋江把盏贺喜。宋江与吴用商议
道,要把这祝家庄村坊洗荡了。石秀禀说起:“这钟离老人仁德之人,指路之力,
救济大忠,也有此等善心良民在内,亦不可屈坏了这等好人。”宋江听罢,叫石秀
去寻那老人来。石秀去不多时,引着那个钟离老人来到庄上,拜见宋江、吴学究。
宋江取一包金帛赏与老人,永为乡民:“不是你这个老人面上有恩,把你这个村坊,
尽数洗荡了,不留一家。因为你一家为善,以此饶了你这一境村坊人民。”那钟离
老人只是下拜。宋江又道:“我连日在此搅扰你们百姓,今日打破祝家庄,与你村
中除害,所有各家赐粮米一石,以表人心。”就着钟离老人为头给散,一面把祝家
庄多余粮米,尽数装载上车;金银财赋,犒赏三军众将;其余牛羊骡马等物,将去
山中支用。打破祝家庄,得粮五十万石。宋江大喜。大小头领,将军马收拾起身,
又得若干新到头领,孙立、孙新、解珍、解宝、邹渊、邹润、乐和、顾大嫂,并救
出七个好汉。孙立等将自己马也捎带了自己的财赋,同老小乐大娘子,跟随了大队
军马上山。当有村坊乡民,扶老挈幼,香花灯烛,于路拜谢。宋江等众将一齐上马,
将军兵分作三队摆开,前队鞭敲金镫,后军齐唱凯歌,正是:

盗可盗,非常盗;强可强,真能强。只因灭恶除凶,聊作打家劫舍。地方恨土
豪欺压,乡村喜义士济施。众虎有情,为救偷鸡钓狗;独龙无助,难留飞虎扑雕。
谨具上万资粮,填平水泊;更赔许多人畜,踏破梁山。

话分两头,且说扑天雕李应恰才将息得箭疮平复,闭门在庄上不出,暗地使人
常常去探听祝家庄消息,已知被宋江打破了,惊喜相半。只见庄客入来报说,有本
州知府带领三五十部汉到庄,便问祝家庄事情。李应慌忙叫杜兴开了庄门,放下吊
桥,迎接入庄。李应把条白绢搭膊络着手,出来迎迓,邀请进庄里前厅。知府下了
马,来到厅上,居中坐了,侧首坐着孔目,下面一个押番,几个虞候,阶下尽是许
多节级、牢子。李应拜罢,立在厅前,知府问道:“祝家庄被杀一事如何?”李应
答道:“小人因被祝彪射了一箭,有伤左臂,一向闭门,不敢出去,不知其实。”
知府道:“胡说!祝家庄现有状子,告你结连梁山泊强寇,引诱他军马,打破了庄,
前日又受他鞍马、羊酒、彩缎、金银,你如何赖得过?”李应告道:“小人是知法
度的人,如何敢受他的东西?”知府道:“难信你说,且提去府里,你自与他对理
明白。”喝教狱卒牢子捉了,带他州里去,与祝家分辩。两下押番虞候,把李应缚
了,众人簇拥知府上了马。知府又问道:“那个是杜主管杜兴?”杜兴道:“小人
便是。”知府道:“状上也有你名,一同带去,也与他锁了。”一行人都出庄门。
当时拿了李应、杜兴,离了李家庄,脚不停地解来。行不过三十余里,只见林子边
撞出宋江、林冲、花荣、杨雄、石秀一班人马,拦住去路。林冲大喝道:“梁山泊
好汉,合伙在此!”那知府人等不敢抵敌,撇了李应、杜兴,逃命去了。宋江喝叫
赶上,众人赶了一程,回来说道:“我们若赶上时,也把这个鸟知府杀了,但自不
知去向。”便与李应、杜兴解了缚索,开了锁,便牵两匹马过来,与他两个骑了。
宋江便道:“且请大官人上梁山泊躲几时,如何?”李应道:“却是使不得。知府
是你们杀了,不干我事。”宋江笑道:“官司里怎肯与你如此分辩?我们去了,必
然要负累了你。既然大官人不肯落草,且在山寨消停几日,打听得没事了时,再下
山来不迟。”

当下不由李应、杜兴不行,大队军马中间,如何回得来?一行三军人马,迤
回到梁山泊了。寨里头领晁盖等众人擂鼓吹笛,下山来迎接,把了接风酒,都上到
大寨里聚义厅上,扇圈也似坐下,请上李应与众头领都相见了。两个讲礼已罢,李
应禀宋江道:“小可两个已送将军到大寨了,既与众头领亦都相见了,在此趋侍不
妨,只不知家中老小如何?可教小人下山则个。”吴学究笑道:“大官人差矣!宝眷
已都取到山寨了。贵庄一把火已都烧做白地,大官人却回到那里去?”李应不信,
早见车仗人马,队队上山来。李应看时,却见是自家的庄客,并老小人等。李应连
忙来问时,妻子说道:“你被知府捉了来,随后又有两个巡检,引着四个都头,带
领三百来土兵,到来抄扎家私,把我们好好地教上车子,将家里一应箱笼、牛羊、
马匹、驴骡等项,都拿了去,又把庄院放起火来都烧了。”李应听罢,只叫得苦。
晁盖、宋江都下厅伏罪道:“我等兄弟们端的久闻大官人好处,因此行出这条计来,
万望大官人情恕。”李应见了如此言语,只得随顺了。宋江道:“且请宅眷后厅耳
房中安歇。”李应又见厅前厅后这许多头领亦有家眷老小在彼,便与妻子道:“只
得依允他过。”宋江等当时请至厅前叙说闲话,众皆大喜。宋江便取笑道:“大官
人,你看我叫过两个巡检并那知府过来相见。”那扮知府的是萧让,扮巡检的两个
是戴宗、杨林,扮孔目的是裴宣,扮虞候的是金大坚、侯健。又叫唤那四个都头,
却是李俊、张顺、马麟、白胜。李应都看了,目睁口呆,言语不得。宋江喝叫小头
目快杀牛宰马,与大官人陪话,庆贺新上山的十二位头领,乃是李应、孙立、孙新、
解珍、解宝、邹渊、邹润、杜兴、乐和、时迁,女头领扈三娘、顾大嫂,同乐大娘
子、李应宅眷另做一席,在后堂饮酒。大小三军,自有犒赏。正厅上大吹大擂,众
多好汉,饮酒至晚方散。新到头领,俱各拨房安顿。

次日,又作席面会请众头领作主张。宋江唤王矮虎来说道:“我当初在清风山
时,许下你一头亲事,悬悬挂在心中,不曾完得此愿。今日我父亲有个女儿,招你
为婿。”宋江自去请出宋太公来,引着一丈青扈三娘到筵前。宋江亲自与他陪话,
说道:“我这兄弟王英虽有武艺,不及贤妹,是我当初曾许下他一头亲事,一向未
曾成得,今日贤妹你认义我父亲了,众头领都是媒人,今朝是个良辰吉日,贤妹与
王英结为夫妇。”一丈青见宋江义气深重,推却不得,两口儿只得拜谢了。晁盖等
众人皆喜,都称颂宋公明真乃有德有义之士。当日尽皆筵宴饮酒庆贺。正饮宴间,
只见山下有人来报道:“朱贵头领酒店里,有个郓城县人在那里,要来见头领。”
晁盖、宋江听得报了,大喜道:“既是这恩人上山来入伙,足遂平生之愿。”正是:
恩仇不辨非豪杰,黑白分明是丈夫。

毕竟来的是郓城县甚么人,且听下回分解。