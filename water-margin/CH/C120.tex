\chapter{宋公明神聚蓼儿洼~徽宗帝梦游梁山泊}

话说宋江衣锦还乡,还至东京,与众弟兄相会,令其各人收拾行装,前往任所。
当有神行太保戴宗来探宋江,二人坐间闲话。只见戴宗起身道:“小弟已蒙圣恩,
除授兖州都统制。今情愿纳下官诰,要去泰安州岳庙里,陪堂求闲,过了此生,实
为万幸。”宋江道:“贤弟何故行此念头?”戴宗道:“是弟夜梦崔府君勾唤,因
此发了这片善心。”宋江道:“贤弟生身既为神行太保,他日必作岳府灵聪。”自
此相别之后,戴宗纳还了官诰,去到泰安州岳庙里,陪堂出家,每日殷勤奉祀圣帝
香火,虔诚无忽。后数月,一夕无恙,请众道伴相辞作别,大笑而终。后来在岳庙
里累次显灵,州人庙祝,随塑戴宗神像于庙里,胎骨是他真身。
又有阮小七受了诰命,辞别宋江,已往盖天军做都统制职事。未及数月,被大将王
禀、赵谭怀挟帮源洞辱骂旧恨,累累于童枢密前诉说阮小七的过失,曾穿着方腊的
赭黄袍、龙衣玉带,虽是一时戏耍,终久怀心不良,亦且盖天军地僻人蛮,必致造
反。童贯把此事达知蔡京,奏过天子,请降了圣旨,行移公文到彼处,追夺阮小七
本身的官诰,复为庶民。阮小七见了,心中也自欢喜,带了老母,回还梁山泊石碣
村,依旧打鱼为生,奉养老母,以终天年,后来寿至六十而亡。

且说小旋风柴进在京师,见戴宗纳还官诰,求闲去了;又见说朝廷追夺了阮小
七官诰,不合戴了方腊的平天冠、龙衣玉带,意在学他造反,罚为庶民,寻思:“我
亦曾在方腊处做驸马,倘或日后奸臣们知得,于天子前谗佞,见责起来,追了诰命,
岂不受辱?不如自识时务,免受玷辱。”推称风疾病患,不时举发,难以任用,情
愿纳还官诰,求闲为农。辞别众官,再回沧州横海郡为民,自在过活。忽然一日,
无疾而终。
李应受中山府都统制,赴任半年,闻知柴进求闲去了,自思也推称风瘫,不能为官,
申达省院,缴纳官诰,复还故乡独龙冈村中过活。后与杜兴一处作富豪,俱得善终。
关胜在北京大名府总管兵马,甚得军心,众皆钦伏。一日,操练军马回来,因大醉,
失脚落马,得病身亡。
呼延灼受御营指挥使,每日随驾操备。后领大军,破大金兀四太子,出军杀至淮
西,阵亡。只有朱仝在保定府管军有功,后随刘光世破了大金,直做到太平军节度
使。
花荣带同妻小妹子,前赴应天府到任。吴用自来单身,只带了随行安童,去武胜军
到任。李逵亦是独自带了两个仆从,自来润州到任。话说为何只说这三个到任,别
的都说了绝后结果?为这七员正将,都不厮见着,先说了结果。后这五员正将,宋
江、卢俊义、花荣、吴用、李逵还有厮会处,以此未说绝了,结果下来便见。
再说宋江、卢俊义在京师,都分派了诸将赏赐,各各令其赴任去讫。殁于王事者,
止将家眷人口,关给与恩赏钱帛金银,仍各送回故乡,听从其便。再有现在朝京偏
将一十五员,除兄弟宋清还乡为农外,杜兴已自跟随李应还乡去了;黄信仍任青州;
孙立带同兄弟孙新、顾大嫂,并妻小,自依旧登州任用;邹润不愿为官,回登云山
去了;蔡庆跟随关胜,仍回北京为民;裴宣自与杨林商议了,自回饮马川,受职求
闲去了;蒋敬思念故乡,愿回潭州为民;朱武自来投授樊瑞道法,两个做了全真先
生,云游江湖,去投公孙胜出家,以终天年;穆春自回揭阳镇乡中,复为良民;凌
振炮手非凡,仍受火药局御营任用。旧在京师偏将五员:安道全钦取回京,就于太
医院做了金紫医官;皇甫端原受御马监大使;金大坚已在内府御宝监为官;萧让在
蔡太师府中受职,作门馆先生;乐和在驸马王都尉府中尽老清闲,终身快乐,不在
话下。
且说宋江自与卢俊义分别之后,各自前去赴任。卢俊义亦无家眷,带了数个随行伴
当,自望庐州去了。宋江谢恩辞朝,别了省院诸官,带同几个家人仆从,前往楚州
赴任。自此相别,都各分散去了,亦不在话下。
且说宋朝原来自太宗传太祖帝位之时,说了誓愿,以致朝代奸佞不清。至今徽宗天
子,至圣至明,不期致被奸臣当道,谗佞专权,屈害忠良,深可悯念。当此之时,
却是蔡京、童贯、高俅、杨四个贼臣,变乱天下,坏国,坏家,坏民。当有殿帅府
太尉高俅、杨,因见天子重礼厚赐宋江等这伙将校,心内好生不然。两个自来商
议道:“这宋江、卢俊义皆是我等仇人,今日倒吃他做了有功之臣,受朝廷这等恩
赐,却教他上马管军,下马管民。我等省院官僚,如何不惹人耻笑?自古道:‘恨
小非君子,无毒不丈夫!’”杨道:“我有一计,先对付了卢俊义,便是绝了宋
江一只臂膊。这人十分英勇,若先对付了宋江,他若得知,必变了事,倒惹出一场
不好。”高俅道:“愿闻你的妙计如何?”杨道:“排出几个庐州军汉,来省院
首告卢安抚招军买马,积草屯粮,意在造反,便与他申呈去太师府启奏,和这蔡太
师都瞒了。等太师奏过天子,请旨定夺,却令人赚他来京师。待上皇赐御食与他,
于内下了些水银,却坠了那人腰肾,做用不得,便成不得大事。再差天使却赐御酒
与宋江吃,酒里也与他下了慢药,只消半月之间,以定没救。”高俅道:“此计大
妙!”有诗堪笑:
自古权奸害善良,不容忠义立家邦。
皇天若肯明昭报,男作俳优女作倡。

两个贼臣计议定了,着心腹人出来寻觅两个庐州土人,写与他状子,叫他去枢
密院首告卢安抚在庐州即日招军买马,积草屯粮,意欲造反,使人常往楚州,结连
安抚宋江,通情起义。枢密院却是童贯,亦与宋江等有仇,当即收了原告状子,径
呈来太师府启奏。蔡京见了申文,便会官计议。此时高俅、杨俱各在彼,四个奸
臣,定了计策,引领原告人,入内启奏天子。上皇曰:“朕想宋江、卢俊义征讨四
方虏寇,掌握十万兵权,尚且不生歹心。今已去邪归正,焉肯背反?寡人不曾亏负
他,如何敢叛逆朝廷?其中有诈,未审虚的,难以准信。”当有高俅、杨在旁奏
道:“圣上道理虽然,人心难忖。想必是卢俊义嫌官卑职小,不满其心,复怀反意,
不幸被人知觉。”上皇曰:“可唤来寡人亲问,自取实招。”蔡京、童贯又奏道:
“卢俊义是一猛兽,未保其心。倘若惊动了他,必致走透,深为未便,今后难以收
捕。只可赚来京师,陛下亲赐御膳御酒,将圣言抚谕之,窥其虚实动静。若无,不
必究问,亦显陛下不负功臣之念。”上皇准奏,随即降下圣旨,差一使命径往庐州,
宣取卢俊义还朝,有委用的事。天使奉命来到庐州,大小官员,出郭迎接,直至州
衙,开读已罢。
话休絮烦。卢俊义听了圣旨,宣取回朝,便同使命离了庐州,一齐上了铺马来京。
于路无话,早至东京皇城司前歇了。次日,早到东华门外,伺候早朝。时有太师蔡
京、枢密院童贯、太尉高俅、杨,引卢俊义于偏殿,朝见上皇。拜舞已罢,天子
道:“寡人欲见卿一面。”又问:“庐州可容身否?”卢俊义再拜奏道:“托赖圣
上洪福齐天,彼处军民,亦皆安泰。”上皇又问了些闲话,俄延至午,尚膳厨官奏
道:“进呈御膳在此,未敢擅便,乞取圣旨。”此时高俅、杨已把水银暗地着放
在里面,供呈在御案上。天子当面将膳赐与卢俊义,卢俊义拜受而食。上皇抚谕道:
“卿去庐州,务要尽心,安养军士,勿生非意。”卢俊义顿首谢恩,出朝回还庐州,
全然不知四个贼臣设计相害。高俅、杨相谓曰:“此后大事定矣!”
再说卢俊义是夜便回庐州来,觉道腰肾疼痛,动举不得,不能乘马,坐船回来。行
至泗州淮河,天数将尽,自然生出事来。其夜因醉,要立在船头上消遣,不想水银
坠下腰胯并骨髓里去,册立不牢,亦且酒后失脚,落于淮河深处而死。可怜河北玉
麒麟,屈作水中冤抑鬼。从人打捞起尸首,具棺椁殡于泗州高原深处。本州官员动
文书申复省院,不在话下。
且说蔡京、童贯、高俅、杨四个贼臣,计较定了,将赍泗州申达文书,早朝奏闻
天子说:“泗州申复卢安抚行至淮河,因酒醉堕水而死。臣等省院,不敢不奏。今
卢俊义已死,只恐宋江心内设疑,别生他事。乞陛下圣鉴,可差天使,赍御酒往楚
州赏赐,以安其心。”上皇沉吟良久,欲道不准,未知其心;意欲准行,诚恐有弊。
上皇无奈,终被奸臣谗佞所惑,片口张舌,花言巧语,缓里取事,无不纳受。遂降
御酒二樽,差天使一人,赍往楚州,限目下便行。眼见得这使臣亦是高俅、杨二
贼手下心腹之辈,天数只注宋公明合当命尽,不期被这奸臣们将御酒内放了慢药在
里面,却教天使赍擎了,径往楚州来。
且说宋公明自从到楚州为安抚,兼管总领兵马。到任之后,惜军爱民,百姓敬之如
父母,军校仰之若神明,讼庭肃然,六事俱备,人心既服,军民钦敬。宋江公事之
暇,时常出郭游玩。原来楚州南门外,有个去处,地名唤做蓼儿洼。其山四面都是
水港,中有高山一座。其山秀丽,松柏森然,甚有风水。虽然是个小去处,其内山
峰环绕,龙虎踞盘,曲折峰峦,陂阶台砌。四围港汊,前后湖荡,俨然是梁山泊水
浒寨一般。宋江看了,心中甚喜,自己想道:“我若死于此处,堪为阴宅。但若身
闲,常去游玩,乐情消遣。”
话休絮烦。自此宋江到任以来,将及半载,时是宣和六年首夏初旬,忽听得朝廷降
赐御酒到来,与众出郭迎接。入到公廨,开读圣旨已罢,天使捧过御酒,教宋安抚
饮毕。宋江亦将御酒回劝天使,天使推称自来不会饮酒。御酒宴罢,天使回京。宋
江备礼,馈送天使,天使不受而去。
宋江自饮御酒之后,觉道肚腹疼痛,心中疑虑,想被下药在酒里。却自急令从人打
听那来使时,于路馆驿,却又饮酒。宋江已知中了奸计,必是贼臣们下了药酒,乃
叹曰:“我自幼学儒,长而通吏,不幸失身于罪人,并不曾行半点异心之事。今日
天子轻听谗佞,赐我药酒,得罪何辜。我死不争,只有李逵现在润州都统制,他若
闻知朝廷行此奸弊,必然再去哨聚山林,把我等一世清名忠义之事坏了。只除是如
此行方可。”连夜使人往润州唤取李逵星夜到楚州,别有商议。
且说李逵自到润州为都统制,只是心中闷倦,与众终日饮酒,只爱贪杯。听得宋江
差人到来有请,李逵道:“哥哥取我,必有话说。”便同干人下了船,直到楚州,
径入州治,拜见宋江罢。宋江道:“兄弟,自从分散之后,日夜只是想念众人。吴
用军师武胜军又远,花知寨在应天府,又不知消耗。只有兄弟在润州镇江较近,特
请你来商量一件大事。”李逵道:“哥哥,甚么大事?”宋江道:“你且饮酒!”
宋江请进后厅,现成杯盘,随即管待李逵,吃了半晌酒食。
将至半酣,宋江便道:“贤弟不知,我听得朝廷差人赍药酒来,赐与我吃。如死,
却是怎的好?”李逵大叫一声:“哥哥,反了罢!”宋江道:“兄弟,军马尽都没
了,兄弟们又各分散,如何反得成?”李逵道:“我镇江有三千军马,哥哥这里楚
州军马,尽点起来,并这百姓,都尽数起去,并气力招军买马杀将去!只是再上梁
山泊倒快活!强似在这奸臣们手下受气!”宋江道:“兄弟且慢着,再有计较。”
原来那接风酒内,已下了慢药。
当夜李逵饮酒了,次日,具舟相送。李逵道:“哥哥几时起义兵,我那里也起军来
接应。”宋江道:“兄弟,你休怪我!前日朝廷差天使,赐药酒与我服了,死在旦
夕。我为人一世,只主张‘忠义’二字,不肯半点欺心。今日朝廷赐死无辜,宁可
朝廷负我,我忠心不负朝廷。我死之后,恐怕你造反,坏了我梁山泊替天行道忠义
之名。因此,请将你来,相见一面。昨日酒中,已与了你慢药服了,回至润州必死。
你死之后,可来此处楚州南门外,有个蓼儿洼,风景尽与梁山泊无异,和你阴魂相
聚。我死之后,尸首定葬于此处,我已看定了也!”言讫,堕泪如雨。
李逵见说,亦垂泪道:“罢,罢,罢!生时伏侍哥哥,死了也只是哥哥部下一个小
鬼!”言讫泪下,便觉道身体有些沉重。当时洒泪,拜别了宋江下船。回到润州,
果然药发身死。李逵临死之时,嘱咐从人:“我死了,可千万将我灵柩去楚州南门
外蓼儿洼和哥哥一处埋葬。”嘱罢而死。从人置备棺椁盛贮,不负其言,扶柩而往。
再说宋江自从与李逵别后,心中伤感,思念吴用、花荣,不得会面。是夜药发临危,
嘱咐从人亲随之辈:“可依我言,将我灵柩安葬此间南门外蓼儿洼高原深处,必报
你众人之德。乞依我嘱!”言讫而逝。宋江从人置备棺椁,依礼殡葬。楚州官吏听
从其言,不负遗嘱,当与亲随人从、本州吏胥老幼,扶宋公明灵柩,葬于蓼儿洼。
数日之后,李逵灵柩,亦从润州到来,葬于宋江墓侧,不在话下。且说宋清在家患
病,闻知家人回来,报说哥哥宋江已故在楚州,病在郓城,不能前来津送。后又闻
说葬于本州南门外蓼儿洼,只令得家人到来祭祀,看视坟茔,修筑完备,回复宋清,
不在话下。
却说武胜军承宣使军师吴用,自到任之后,常常心中不乐,每每思念宋公明相爱之
心。忽一日,心情恍惚,寝寐不安。至夜,梦见宋江、李逵二人,扯住衣服,说道:
“军师,我等以忠义为主,替天行道,于心不曾负了天子。今朝廷赐饮药酒,我死
无辜。身亡之后,现已葬于楚州南门外蓼儿洼深处。军师若想旧日之交情,可到坟
茔亲来看视一遭。”吴用要问备细,撒然觉来,乃是南柯一梦。吴用泪如雨下,坐
而待旦。得了此梦,寝食不安。次日,便收拾行李,径往楚州来。不带从人,独自
奔来。前至楚州,果然宋江已死,只闻彼处人民无不嗟叹。吴用安排祭仪,直至南
门外蓼儿洼,寻到坟茔,置祭宋公明、李逵,就于墓前,以手掴其坟冢,哭道:“仁
兄英灵不昧,乞为昭鉴。吴用是一村中学究,始随晁盖,后遇仁兄,救护一命,坐
享荣华。到今数十余载,皆赖兄之德。今日既为国家而死,托梦显灵与我,兄弟无
以报答,愿得将此良梦,与仁兄同会于九泉之下。”言罢痛哭。正欲自缢,只见花
荣从船上飞奔到于墓前,见了吴用,各吃一惊。吴学究便问道:“贤弟在应天府为
官,缘何得知宋兄已丧?”花荣道:“兄弟自从分散到任之后,无日身心得安,常
想念众兄之情。因夜得一异梦,梦见宋公明哥哥和李逵前来,扯住小弟,诉说‘朝
廷赐饮药酒鸩死,现葬于楚州南门外蓼儿洼高原之上。兄弟如不弃旧,可到坟前,
看望一遭。’因此,小弟掷了家间,不避驱驰,星夜到此。”吴用道:“我得异梦,
亦是如此,与贤弟无异,因此而来。今得贤弟到此最好,吴某心中想念宋公明恩义
难舍,交情难报,正欲就此处自缢而死,魂魄与仁兄同聚一处。身后之事,托与贤
弟。”花荣道:“军师既有此心,小弟便当随从,亦与仁兄同归一处。”似此真乃
死生契合者也。有诗为证:
红蓼洼中托梦长,花荣吴用各悲伤。
一腔义血元同有,岂忍田横独丧亡?
吴用道:“我指望贤弟看见我死之后,葬我于此,你如何也行此事?”花荣道:“小
弟寻思宋兄长仁义难舍,恩念难忘。我等在梁山泊时,已是大罪之人,幸然不死。
感得天子赦罪招安,北讨南征,建立功勋。今已姓扬名显,天下皆闻。朝廷既已生
疑,必然来寻风流罪过。倘若被他奸谋所施,误受刑戮,那时悔之无及。如今随仁
兄同死于黄泉,也留得个清名于世,尸必归坟矣!”吴用道:“贤弟,你听我说,
我已单身,又无家眷,死却何妨?你今现有幼子娇妻,使其何依?”花荣道:“此
事不妨,自有囊箧足以糊口。妻室之家,亦自有人料理。”两个大哭一场,双双悬
于树上,自缢而死。船上从人久等,不见本官出来,都到坟前看时,只见吴用、花
荣自缢身死。慌忙报与本州官僚,置备棺椁,葬于蓼儿洼宋江墓侧,宛然东西四。
楚州百姓,感念宋江仁德,忠义两全,建立祠堂,四时享祭,里人祈祷,无不感应。
且不说宋江在蓼儿洼累累显灵,所求立应。却说道君皇帝,在东京内院,自从赐御
酒与宋江之后,圣意累累设疑,又不知宋江消息,常只挂念于怀。每日被高俅、杨
议论奢华受用所惑,只要闭塞贤路,谋害忠良。忽然一日,上皇在内宫闲玩,猛
然思想起李师师,就从地道中和两个小黄门,径来到他后园中,拽动铃索。李师师
慌忙迎接圣驾,到于卧房内坐定。上皇便叫前后关闭了门户。李师师盛妆向前起居
已罢,天子道:“寡人近感微疾,现令神医安道全看治,有数十日不曾来与爱卿相
会,思慕之甚!今一见卿,朕怀不胜悦乐!”李师师奏道:“深蒙陛下眷爱之心,
贱人愧感莫尽!”房内铺设酒肴,与上皇饮酌取乐。才饮过数杯,只见上皇神思困
倦。点的灯烛荧煌,忽然就房里起一阵冷风,上皇见个穿黄衫的立在面前。上皇惊
起问道:“你是甚人,直来到这里?”那穿黄衫的人奏道:“臣乃是梁山泊宋江部
下神行太保戴宗。”上皇道:“你缘何到此?”戴宗奏道:“臣兄宋江,只在左右,
启请陛下车驾同行。”上皇曰:“轻屈寡人车驾何往?”戴宗道:“自有清秀好去
处,请陛下游玩。”上皇听罢此语,便起身随戴宗出得后院来,见马车足备,戴宗
请上皇乘马而行。但见如云似雾,耳闻风雨之声,到一个去处。但见:

漫漫烟水,隐隐云山。不观日月光明,只见水天一色。红瑟瑟满目蓼花,绿依
依一洲芦叶。双双鸿雁,哀鸣在沙渚矶头;对对,倦宿在败荷汀畔。霜枫簇簇,
似离人点染泪波;风柳疏疏,如怨妇蹙颦眉黛。淡月寒星长夜景,凉风冷露九秋天。
当下上皇在马上观之不足,问戴宗道:“此是何处,要寡人到此?”戴宗指着山上
关路道:“请陛下行去,到彼便知。”上皇纵马登山,行过三重关道,至第三座关
前,见有上百人,俯伏在地,尽是披袍挂铠,戎装革带,金盔金甲之将。上皇大惊,
连问道:“卿等皆是何人?”只见为头一个,凤翅金盔,锦袍金甲,向前奏道:“臣
乃梁山泊宋江是也。”上皇曰:“寡人已教卿在楚州为安抚使,却缘何在此?”宋
江奏道:“臣等谨请陛下到忠义堂上,容臣细诉衷曲枉死之冤。”上皇到忠义堂前
下马,上堂坐定,看堂下时,烟雾中拜伏着许多人。上皇犹豫不定。只见为首的宋
江上阶,跪膝向前,垂泪启奏。上皇道:“卿何故泪下?”宋江奏道:“臣等虽曾
抗拒天兵,素秉忠义,并无分毫异心。自从奉陛下敕命招安之后,先退辽兵,次平
三寇,弟兄手足,十损其八。臣蒙陛下命守楚州,到任已来,与军民水米无交,天
地共知。今陛下赐臣药酒,与臣服吃,臣死无憾。但恐李逵怀恨,辄起异心。臣特
令人去润州唤李逵到来,亲与药酒鸩死。吴用、花荣,亦为忠义而来,在臣冢上,
俱皆自缢而亡。臣等四人,同葬于楚州南门外蓼儿洼。里人怜悯,建立祠堂于墓前。
今臣等阴魂不散,俱聚于此,伸告陛下,诉平生衷曲,始终无异。乞陛下圣鉴。”
上皇听了,大惊曰:“寡人亲差天使,亲赐黄封御酒,不知是何人换了药酒赐卿?”
宋江奏道:“陛下可问来使,便知奸弊所出。”上皇看见三关寨栅雄壮,惨然问曰:
“此是何所,卿等聚会于此?”宋江奏曰:“此是臣等旧日聚义梁山泊也。”上皇
又曰:“卿等已死,当往受生,何故相聚于此?”宋江奏道:“天帝哀怜臣等忠义,
蒙玉帝符牒敕命,封为梁山泊都土地。众将已会于此,有屈难伸,特令戴宗屈万乘
之主,亲临水泊,恳告平日衷曲。”上皇曰:“卿等何不诣九重深院,显告寡人?”
宋江奏道:“臣乃幽阴魂魄,怎得到凤阙龙楼?今者陛下出离宫禁,屈邀至此。”
上皇曰:“寡人可以观玩否?”宋江等再拜谢恩。上皇下堂,回首观看堂上牌额,
大书“忠义堂”三字,上皇点头下阶。忽见宋江背后转过李逵,手双斧,厉声高
叫道:“皇帝,皇帝!你怎地听信四个贼臣挑拨,屈坏了我们性命?今日既见,正好
报仇!”黑旋风说罢,抡起双斧,径奔上皇。天子吃这一惊,撒然觉来,乃是南柯
一梦,浑身冷汗。闪开双眼,见灯烛荧煌,李师师犹然未寝。上皇问曰:“寡人恰
在何处去来?”李师师奏道:“陛下适间伏枕而卧。”上皇却把梦中神异之事,对
李师师一一说知。李师师又奏曰:“凡人正直者,必然为神。莫非宋江端的已死,
是他故显神灵,托梦与陛下?”上皇曰:“寡人来日,必当举问此事。若是如果死
了,必须与他建立庙宇,敕封烈侯。”李师师奏曰:“若圣上果然加封,显陛下不
负功臣之德。”上皇当夜嗟叹不已。
次日临朝,传圣旨,会群臣于偏殿。当有蔡京、童贯、高俅、杨等,只虑恐圣上
问宋江之事,已出宫去了。只有宿太尉等几位大臣,在彼侍侧,上皇便问宿元景曰:
“卿知楚州安抚宋江消息否?”宿太尉奏道:“臣虽一向不知宋安抚消息,臣昨夜
得一异梦,甚是奇怪。”上皇曰:“卿得异梦,可奏与寡人知道。”宿太尉奏曰:
“臣梦见宋江,亲到私宅,戎装带,顶盔明甲,见臣诉说陛下以药酒见赐而亡。
楚人怜其忠义,葬在楚州南门外蓼儿洼内,建立祠堂,四时享祭。”上皇听罢,便
颠头道:“此诚异事,与朕梦一般。”又分付宿元景道:“卿可差心腹之人,往楚
州体察此事,有无急来回报。”宿太尉道:“是。”便领了圣旨,自出宫禁。归到
私宅,便差心腹之人,前去楚州探听宋江消息,不在话下。
次日,上皇驾坐文德殿,见高俅、杨在侧,圣旨问道:“汝等省院,近日知楚州
宋江消息否?”二人不敢启奏,各言不知。上皇辗转心疑,龙体不乐。
且说宿太尉干人,已到楚州打探回来,备说宋江蒙赐饮药酒而死。已丧之后,楚
人感其忠义,今葬于楚州蓼儿洼高山之上。更有吴用、花荣、李逵三人,一处埋葬。
百姓哀怜,盖造祠堂于墓前,春秋祭赛,虔诚奉祀,士庶祈祷,极有灵验。宿太尉
听了,慌忙引领干人入内,备将此事,回奏天子。上皇见说,不胜伤感。次日早朝,
天子大怒,当百官前,责骂高俅、杨:“败国奸臣,坏寡人天下!”二人俯伏在
地,叩头谢罪。蔡京、童贯亦向前奏道:“人之生死,皆由注定。省院未有来文,
不敢妄奏。昨夜楚州才有申文到院,臣等正欲启奏。”上皇终被四贼曲为掩饰,不
加其罪,当即喝退高俅、杨,便教追要原赍御酒使臣。不期天使自离楚州回还,
已死于路。
宿太尉次日见上皇于偏殿,再以宋江忠义显灵之事,奏闻天子。上皇准宣宋江亲弟
宋清,承袭宋江名爵。不期宋清已感风疾在身,不能为官,上表辞谢,只愿郓城为
农。上皇怜其孝道,赐钱十万贯,田三千亩,以赡其家。待有子嗣,朝廷录用。后
来宋清生一子宋安平,应过科举,官至秘书学士,这是后话。
再说上皇具宿太尉所奏,亲书圣旨,敕封宋江为忠烈义济灵应侯,仍敕赐钱于梁山
泊,起盖庙宇,大建祠堂,妆塑宋江等殁于王事诸多将佐神像。敕赐殿宇牌额,御
笔亲书“靖忠之庙”。济州奉敕,于梁山泊起造庙宇。但见:

金钉朱户,玉柱银门。画栋雕梁,朱檐碧瓦。绿栏干低绕轩窗,绣帘幕高悬宝
槛。五间大殿,中悬敕额金书;两庑长廊,彩画出朝入相。绿槐影里,星门高接
青云;翠柳阴中,靖忠庙直侵霄汉。黄金殿上,塑宋公明等三十六员天罡正将;两
廊之内,列朱武为头七十二座地煞将军。门前侍从狰狞,部下神兵勇猛。纸炉巧匠
砌楼台,四季焚烧楮帛。桅竿高竖挂长,二社乡人祭赛。庶民恭礼正神,祀典
朝参忠烈帝。万年香火享无穷,千载功勋表史记。
又有绝句一首,诗曰:
天罡尽已归天界,地煞还应入地中。
千古为神皆庙食,万年青史播英雄。
后来宋公明累累显灵,百姓四时享祭不绝。梁山泊内祈风得风,祷雨得雨。楚州蓼
儿洼亦显灵验。彼处人民,重建大殿,添设两廊,奏请赐额。妆塑神像三十六员于
正殿,两廊仍塑七十二将。年年享祭,万民顶礼,至今古迹尚存。史官有唐律二首
哀挽,诗曰:
莫把行藏怨老天,韩彭赤族已堪怜。
一心报国摧锋日,百战擒辽破腊年。
煞曜罡星今已矣,谗臣贼子尚依然!
早知鸩毒埋黄壤,学取鸱夷范蠡船。
又诗:
生当鼎食死封侯,男子生平志已酬。
铁马夜嘶山月晓,玄猿秋啸暮云稠。
不须出处求真迹,却喜忠良作话头。
千古蓼洼埋玉地,落花啼鸟总关愁。