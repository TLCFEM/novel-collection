\chapter{花和尚单打二龙山~青面兽双夺宝珠寺}

话说杨志当时在黄泥冈上,被取了生辰纲去,如何回转去见得梁中书,欲要就
冈子上自寻死路。却待望黄泥冈下跃身一跳,猛可醒悟,曳住了脚,寻思道:“爹
娘生下洒家,堂堂一表,凛凛一躯,自小学成十八般武艺在身,终不成只这般休了!
比及今日寻个死处,不如日后等他拿得着时,却再理会。”回身再看那十四个人时,
只是眼睁睁地看着杨志,没个挣扎得起。杨志指着骂道:“都是你这厮们不听我言
语,因此做将出来,连累了洒家。”树根头拿了朴刀,挂了腰刀,周围看时,别无
物件,杨志叹了口气,一直下冈子去了。

那十四个人,直到二更,方才得醒,一个个爬将起来,口里只叫得连珠箭的苦。
老都管道:“你们众人不听杨提辖的好言语,今日送了我也!”众人道:“老爷,
今日事已做出来了,且通个商量。”老都管道:“你们有甚见识?”众人道:“是
我们不是了。古人有言:‘火烧到身,各自去扫;蜂入怀,随即解衣。’若还杨
提辖在这里,我们都说不过;如今他自去的不知方向,我们回去见梁中书相公,何
不都推在他身上?只说道:‘他一路上,凌辱打骂众人,逼迫得我们都动不得。他
和强人做一路,把蒙汗药将俺们麻翻了,缚了手脚,将金宝都掳去了。’”老都管
道:“这话也说的是。我们等天明,先去本处官司首告。留下两个虞候,随衙听候,
捉拿贼人。我等众人,连夜赶回北京,报与本官知道,教动文书,申复太师得知,
着落济州府,追获这伙强人便了。”次日天晓,老都管自和一行人来济州府该管官
吏首告,不在话下。

且说杨志提着朴刀,闷闷不已,离黄泥冈,望南行了半日,看看又走了半夜,
去林子里歇了,寻思道:“盘缠又没了,举眼无个相识,却是怎地好?”渐渐天色
明亮,只得趁早凉了行。又走了二十余里,正是:
面皮青毒逞雄豪,白送金珠十一挑。
今日为何行急急,不知若个打藤条。
当时杨志走得辛苦,到一酒店门前。杨志道:“若不得些酒吃,怎地打熬得过?”
便入那酒店去,向这桑木桌凳座头上坐了,身边倚了朴刀。只见灶边一个妇人问道:
“客官莫不要打火?”杨志道:“先取两角酒来吃,借些米来做饭,有肉安排些个,
少停一发算钱还你。”只见那妇人先叫一个后生来面前筛酒,一面做饭,一边炒肉,
都把来杨志吃了。杨志起身,绰了朴刀,便出店门。那妇人道:“你的酒肉饭钱都
不曾有!”杨志道:“待俺回来还你,权赊咱一赊。”说了便走。

那筛酒的后生赶将出来,揪住杨志,被杨志一拳打翻了。那妇人叫起屈来。杨
志只顾走,只听得背后一个人赶来,叫道:“你那厮走那里去!”杨志回头看时,
那人大脱着膊,拖着杆棒,抢奔将来。杨志道:“这厮却不是晦气,倒来寻洒家!”
立脚住了不走。看后面时,那筛酒后生也拿条叉,随后赶来,又引着三两个庄客,
各拿杆棒,飞也似都奔将来。杨志道:“结果了这厮一个,那厮们都不敢追来。”
便挺了手中朴刀来斗这汉。这汉也抡转手中杆棒,抢来相迎。两个斗了三二十合,
这汉怎地敌的杨志,只办得架隔遮拦,上下躲闪。那后来的后生并庄客,却待一发
上,只见这汉托地跳出圈子外来叫道:“且都不要动手!兀那使朴刀的大汉,你可
通个姓名。”那杨志拍着胸道:“洒家行不更名,坐不改姓,青面兽杨志的便是!”
这汉道:“莫不是东京殿司杨制使么?”杨志道:“你怎地知道洒家是杨制使?”
这汉撇了枪棒,便拜道:“小人有眼不识泰山。”杨志便扶这人起来,问道:“足
下是谁?”这汉道:“小人原是开封府人氏,乃是八十万禁军都教头林冲的徒弟,
姓曹,名正,祖代屠户出身。小人杀的好牲口,挑剐骨,开剥推只此被人唤做
操刀鬼。为因本处一个财主,将五千贯钱,教小人来此山东做客,不想折了本,回
乡不得,在此入赘在这个庄农人家。却才灶边妇人,便是小人的浑家。这个拿叉
的,便是小人的妻舅。却才小人和制使交手,见制使手段和小人师父林教师一般,
因此抵敌不住。”杨志道:“原来你却是林教师的徒弟。你的师父,被高太尉陷害,
落草去了,如今现在梁山泊。”曹正道:“小人也听得人这般说将来,未知真实。
且请制使到家少歇。”

杨志便同曹正再回到酒店里来。曹正请杨志里面坐下,叫老婆和妻舅都来拜了
杨志,一面再置酒食相待。饮酒中间,曹正动问道:“制使缘何到此?”杨志把做
制使失陷花石纲,并如今又失陷了梁中书的生辰纲一事,从头备细告诉了。曹正道:
“既然如此,制使且在小人家里住几时,再有商议。”杨志道:“如此却是深感你
的厚意。只恐官司追捕将来,不敢久住。”曹正道:“制使这般说时,要投那里去?”
杨志道:“洒家欲投梁山泊,去寻你师父林教头。俺先前在那里经过时,正撞着他
下山来,与洒家交手。王伦见了俺两个本事一般,因此都留在山寨里相会,以此认
得你师父林冲。王伦当初苦苦相留,俺却不曾落草,如今脸上又添了金印,却去投
奔他时,好没志气。因此踌躇未决,进退两难。”曹正道:“制使见的是。小人也
听的人传说,王伦那厮,心地偏窄,安不得人,说我师父林教头上山时,受尽他的
气。不若小人此间离不远,却是青州地面,有座山,唤做二龙山,山上有座寺,唤
做宝珠寺。那座山生来却好,裹着这座寺,只有一条路上的去。如今寺里住持还了
俗,养了头发,余者和尚都随顺了。说道他聚集的四五百人,打家劫舍。为头那人,
唤做金眼虎邓龙。制使若有心落草时,到去那里入伙,足可安身。”杨志道:“既
有这个去处,何不去夺来安身立命?”

当下就曹正家里住了一宿,借了些盘缠,拿了朴刀,相别曹正,曳开脚步,投
二龙山来。行了一日,看看渐晚,却早望见一座高山。杨志道:“俺去林子里且歇
一夜,明日却上山去。”转入林子里来,吃了一惊。只见一个胖大和尚,脱的赤条
条的,背上刺着花绣,坐在松树根头乘凉。那和尚见了杨志,就树根头绰了禅杖,
跳将起来,大喝道:“兀那撮鸟,你是那里来的?”正是:
平将珠宝担落空,却问宝珠寺讨帐。
要投入寺里强人,先引出寺外和尚。
杨志听了道:“原来也是关西和尚。俺和他是乡中,问他一声。”杨志叫道:“你
是那里来的僧人?”那和尚也不回说,抡起手中禅杖,只顾打来。杨志道:“怎奈
这秃厮无礼,且把他来出口气!”挺起手中朴刀,来奔那和尚。两个就林子里,一
来一往,一上一下,两个放对。但见:

两条龙竞宝,一对虎争。禅杖起如虎尾龙筋,朴刀飞似龙虎爪。,
忽喇喇,天崩地塌,阵云中黑气盘旋;恶狠狠,雄赳赳,雷吼风呼,杀气内金光闪
烁。两条龙竞宝,吓得那身长力壮仗霜锋周处眼无光;一对虎争,惊的这胆大心
粗施雪刃卞庄魂魄丧。两条龙竞宝,眼珠放彩,尾摆得水母殿台摇;一对虎争,
野兽奔驰,声震的山神毛发竖。
当时杨志和那和尚斗到四五十合,不分胜败。那和尚卖个破绽,托地跳出圈子外来,
喝一声:“且歇!”两个都住了手。杨志暗暗地喝采道:“那里来的这个和尚,真
个好本事,手段高!俺却刚刚地只敌的他住!”那僧人叫道:“兀那青面汉子,你
是甚么人?”杨志道:“洒家是东京制使杨志的便是。”那和尚道:“你不是在东
京卖刀杀了破落户牛二的?”杨志道:“你不见俺脸上金印?”那和尚笑道:“却
原来在这里相见。”杨志道:“不敢问师兄却是谁?缘何知道洒家卖刀?”那和尚
道:“洒家不是别人,俺是延安府老种经略相公帐前军官鲁提辖的便是。为因三拳
打死了镇关西,却去五台山净发为僧。人见洒家背上有花绣,都叫俺做花和尚鲁智
深。”杨志笑道:“原来是自家乡里,俺在江湖上多闻师兄大名。听得说道,师兄
在大相国寺里挂搭,如今何故来在这里?”

鲁智深道:“一言难尽。洒家在大相国寺管菜园,遇着那豹子头林冲,被高太
尉要陷害他性命。俺却路见不平,直送他到沧州,救了他一命。不想那两个防送公
人回来,对高俅那厮说道:‘正要在野猪林里结果林冲,却被大相国寺鲁智深救了。
那和尚直送到沧州,因此害他不得。’这直娘贼恨杀洒家,分付寺里长老不许俺挂
搭;又差人来捉洒家,却得一伙泼皮通报,不是着了那厮的手。吃俺一把火烧了那
菜园里廨宇,逃走在江湖上,东又不着,西又不着。来到孟州十字坡过,险些儿被
个酒店妇人害了性命,把洒家着蒙汗药麻翻了。得他的丈夫归来得早,见了洒家这
般模样,又看了俺的禅杖、戒刀吃惊,连忙把解药救俺醒来。因问起洒家名字,留
住俺过了几日,结义洒家做了弟兄。那人夫妻两个,亦是江湖上好汉有名的,都叫
他做菜园子张青,其妻母夜叉孙二娘,甚是好义气。住了四五日,打听的这里二龙
山宝珠寺可以安身,洒家特地来奔那邓龙入伙,叵耐那厮不肯安着洒家在这山上。
和俺厮并,又敌洒家不过,只把这山下三座关,牢牢地拴住。又没别路上去,那撮
鸟由你叫骂,只是不下来厮杀,气得洒家正苦在这里没个委结,不想却是大哥来。”

杨志大喜。两个就林子里剪拂了,就地坐了一夜。杨志诉说了卖刀杀死牛二的
事,并解生辰纲失陷一节,都备细说了。又说曹正指点来此一事,便道:“既是闭
了关隘,俺们休在这里,如何得他下来?不若且去曹正家商议。”

两个厮赶着行离了那林子,来到曹正酒店里。杨志引鲁智深与他相见了,曹正
慌忙置酒相待,商量要打二龙山一事。曹正道:“若是端的闭了关时,休说道你二
位,便有一万军马,也上去不得。似此只可智取,不可力求。”鲁智深道:“叵耐
那撮鸟,初投他时,只在关外相见。因不留俺,厮并起来,那厮小肚上,被俺一脚
点翻了。却待要结果了他性命,被他那里人多,救了上山去,闭了这鸟关,由你自
在下面骂,只是不肯下来厮杀。”杨志道:“既然好去处,俺和你如何不用心去打!”
鲁智深道:“便是没做个道理上去,奈何不得他!”

曹正道:“小人有条计策,不知中二位意也不中?”杨志道:“愿闻良策则个。”
曹正道:“制使也休这般打扮,只照依小人这里近村庄家穿着。小人把这位师父禅
杖、戒刀都拿了,却叫小人的妻弟,带六个火家,直送到那山下,把一条索子,绑
了师父,小人自会做活结头。却去山下叫道:‘我们近村开酒店庄家,这和尚来我
店中吃酒,吃得大醉了,不肯还钱,口里说道,去报人来打你山寨,因此我们听的;
乘他醉了,把他绑缚在这里,献与大王。’那厮必然放我们上山去。到得他山寨里
面,见邓龙时,把索子曳脱了活结头,小人便递过禅杖与师父。你两个好汉一发上,
那厮走往那里去!若结果了他时,以下的人,不敢不伏。此计若何?”鲁智深、杨
志齐道:“妙哉!妙哉!”有诗为证:
乳虎称龙亦枉然,二龙山许二龙蟠。
人逢忠义情偏洽,事到颠危策愈全。

当晚众人吃了酒食,又安排了些路上干粮。次日五更起来,众人都吃得饱了。
鲁智深的行李包裹,都寄放在曹正家。当日杨志、鲁智深、曹正,带了小舅并五七
个庄家,取路投二龙山来。晌午后,直到林子里,脱了衣裳,把鲁智深用活结头使
索子绑了,教两个庄家,牢牢地牵着索头。杨志戴了遮日头凉笠儿,身穿破布衫,
手里倒提着朴刀。曹正拿着他的禅杖,众人都提着棍棒,在前后簇拥着。到得山下,
看那关时,都摆着强弩硬弓,灰瓶炮石。小喽罗在关上,看见绑得这个和尚来,飞
也似报上山去。多样时,只见两个小头目上关来问道:“你等何处人?来我这里做
甚么?那里捉得这个和尚来?”曹正答道:“小人等是这山下近村庄家,开着一个
小酒店。这个胖和尚,不时来我店中吃酒。吃得大醉,不肯还钱,口里说道:‘要
去梁山泊叫千百个人来,打此二龙山,和你这近村坊,都洗荡了!’因此小人只得
又将好酒请他,灌得醉了,一条索子绑缚这厮,来献与大王,表我等村邻孝顺之心,
免的村中后患。”

两个小头目听了这话,欢天喜地,说道:“好了!众人在此少待一时。”两个
小头目就上山来报知邓龙,说拿得那胖和尚来。邓龙听了大喜,叫:“解上山来,
且取这厮的心肝,来做下酒,消我这点冤仇之恨!”小喽罗得令,来把关隘门开了,
便叫送上来。

杨志、曹正,紧押鲁智深解上山来,看那三座关时,端的险峻:两下里山环绕
将来,包住这座寺;山峰生得雄壮,中间只一条路上关来;三重关上,摆着擂木炮
石,硬弩强弓,苦竹枪密密地攒着。过得三处关闸,来到宝珠寺前看时,三座殿门,
一段镜面也似平地,周遭都是木栅为城。寺前山门下立着七八个小喽罗,看见缚的
鲁智深来,都指手骂道:“你这秃驴,伤了大王,今日也吃拿了!慢慢的碎割了这
厮!”鲁智深只不做声。押到佛殿看时,殿上都把佛来抬去了;中间放着一把虎皮
交椅;众多小喽罗,拿着枪棒,立在两边。

少刻,只见两个小喽罗扶出邓龙来,坐在交椅上。曹正、杨志紧紧地帮着鲁智
深到阶下。邓龙道:“你那厮秃驴!前日点翻了我,伤了小腹,至今青肿未消,今
日也有见我的时节。”鲁智深睁圆怪眼,大喝一声:“撮鸟休走!”两个庄家把索
头只一曳,曳脱了活结头,散开索子,鲁智深就曹正手里接过禅杖,云飞抡动,杨
志撇了凉笠儿,提起手中朴刀,曹正又抡起杆棒,众庄家一齐发作,并力向前。邓
龙急待挣扎时,早被鲁智深一禅杖,当头打着,把脑盖劈作两半个,和交椅都打碎
了。手下的小喽罗,早被杨志搠翻了四五个。曹正叫道:“都来投降!若不从者,
便行扫除处死!”寺前寺后,五六百小喽罗并几个小头目,惊吓的呆了,只得都来
归降投伏。随即叫把邓龙等尸首,扛抬去后山烧化了。一面去点仓廒,整顿房舍,
再去看那寺后有多少物件,且把酒肉安排些来吃。鲁智深并杨志做了山寨之主,置
酒设宴庆贺。小喽罗们尽皆投伏了,仍设小头目管领。曹正别了二位好汉,领了庄
家,自回家去了,不在话下。正是:
古刹雄奇隐翠微,翻为贼寨假慈悲。
天生神力花和尚,弄棒磨刀作住持。
又有诗一首并及杨志:
有智能深助智深,绿林豪客主丛林。
降龙伏虎真同志,兽面谁知有佛心。

不说鲁智深、杨志自在二龙山落草,却说那押生辰纲老都管并这几个厢禁军,
晓行夜住,赶回北京,到的梁中书府,直至厅前,齐齐都拜翻在地下告罪。梁中书
道:“你们路上辛苦,多亏了你众人。”又问:“杨提辖何在?”众人告道:“不
可说!这人是个大胆忘恩的贼!自离了此间五七日后,行到黄泥冈时,天气大热,都
在林子里歇凉。不想杨志和七个贼人通同,假装做贩枣子客商。杨志约会与他做一
路,先推七辆江州车儿,在这黄泥冈上松林里等候,却叫一个汉子,挑一担酒来冈
子上歇下。小的众人不合买他酒吃,被那厮把蒙汗药都麻翻了,又将索子捆缚众人。
杨志和那七个贼人,却把生辰纲财宝并行李,尽装载车上将了去。现今去本管济州
府呈告了,留两个虞候在那里随衙听候,捉拿贼人。小人等众人,星夜赶回来告知
恩相。”

梁中书听了大惊,骂道:“这贼配军!你是犯罪的囚徒,我一力抬举你成人,
怎敢做这等不仁忘恩的事!我若拿住他时,碎尸万段!”随即便唤书吏,写了文书,
当时差人星夜来济州投下;又写一封家书,着人也连夜上东京,报与太师知道。

且不说差人去济州下公文,只说着人上东京来到太师府报知。见了太师,呈上
书札。蔡太师看了,大惊道:“这班贼人,甚是胆大!去年将我女婿送来的礼物,
打劫了去,至今未获;今年又来无礼,如何干罢!”随即押了一纸公文,着一个府
干,亲自赍了,星夜望济州来,着落府尹,立等捉拿这伙贼人,便要回报。

且说济州府尹自从受了北京大名府留守司梁中书札付,每日理论不下。正忧闷
间,只见门吏报道:“东京太师府里,差府干现到厅前,有紧急公文,要见相公。”
府尹听得,大惊道:“多管是生辰纲的事!”慌忙升厅,来与府干相见了,说道:
“这件事,下官已受了梁府虞候的状子,已经差缉捕的人,跟捉贼人,未见踪迹。
前日留守司又差人行札付到来,又经着仰尉司并缉捕观察,杖限跟捉,未曾得获。
若有些动静消息,下官亲到相府回话。”府干道:“小人是太师府里心腹人。今奉
太师钧旨,特差来这里要这一干人。临行时,太师亲自分付,教小人到本府,只就
州衙里宿歇,立等相公,要拿这七个贩枣子的,并卖酒一人,在逃军官杨志,各贼
正身。限在十日捉拿完备,差人解赴东京。若十日不获得这件公事时,怕不先来请
相公去沙门岛走一遭。小人也难回太师府里去,性命亦不知如何。相公不信,请看
太师府里行来的钧帖。”

府尹看罢大惊,随即便唤缉捕人等。只见阶下一人声喏,立在帘前,太守道:
“你是甚人?”那人禀道:“小人是三都缉捕使臣何涛。”太守道:“前日黄泥冈
上打劫了去的生辰纲,是你该管么?”何涛答道:“禀复相公:何涛自从领了这件
公事,昼夜无眠,差下本管眼明手快的公人,去黄泥冈上往来缉捕;虽是累经杖责,
到今未见踪迹。非是何涛怠慢官府,实出于无奈。”府尹喝道:“胡说!‘上不紧
则下慢’。我自进士出身,历任到这一郡诸侯,非同容易!今日东京太师府,差一
干办,来到这里,领太师台旨:限十日内,须要捕获各贼正身,完备解京。若还违
了限次,我非止罢官,必陷我投沙门岛走一遭。你是个缉捕使臣,倒不用心,以致
祸及于我。先把你这厮迭配远恶军州,雁飞不到去处!”便唤过文笔匠来,去何涛
脸上刺下“迭配……州”字样,空着甚处州名,发落道:“何涛,你若获不得贼人,
重罪决不饶恕!”正是:
脸皮打稿太乖张,自要平安人受殃。
贱面可无烦作计,本心也合细商量。

却说何涛领了台旨,下厅前来到使臣房里,会集许多做公的,都到机密房中,
商议公事。众做公的都面面相觑,如箭穿雁嘴,钩搭鱼腮,尽无言语。何涛道:“你
们闲常时,都在这房里赚钱使用;如今有此一事难捉,都不做声。你众人也可怜我
脸上刺的字样。”众人道:“上复观察:小人们人非草木,岂不省的?只是这一伙
做客商的,必是他州外府深山旷野强人遇着,一时劫了他的财宝,自去山寨里快活,
如何拿的着?便是知道,也只看得他一看。”何涛听了,当初只有五分烦恼,见说
了这话,又添了五分烦恼,自离了使臣房里,上马回到家中,把马牵去后槽上拴了,
独自一个,闷闷不已。正是:
双眉重上三锁,满腹填平万斛愁。
网里漏鱼何处觅?瓮中捉鳖向谁求?

只见老婆问道:“丈夫,你如何今日这般嘴脸?”何涛道:“你不知,前日太
守委我一纸批文,为因黄泥冈上一伙贼人,打劫了梁中书与丈人蔡太师庆生辰的金
珠宝贝,计十一担,正不知是甚么样人打劫了去。我自从领了这道钧批,到今未曾
得获。今日正去转限,不想太师府又差干办来,立等要拿这一伙贼人解京。太守问
我贼人消息,我回复道:‘未见次第,不曾获得。’府尹将我脸上刺下‘迭配……
州’字样,只不曾填甚去处,在后知我性命如何!”老婆道:“似此怎地好?却是
如何得了!”

正说之间,只见兄弟何清来望哥哥,何涛道:“你来做甚么?不去赌钱,却来
怎地?”何涛的妻子乖觉,连忙招手说道:“阿叔,你且来厨下,和你说话。”何
清当时跟了嫂嫂进到厨下坐了。嫂嫂安排些酒肉菜蔬,烫几杯酒,请何清吃。何清
问嫂嫂道:“哥哥忒杀欺负人!我不中,也是你一个亲兄弟!你便奢遮杀,只做得个
缉捕观察,便叫我一处吃盏酒,有甚么辱没了你!”阿嫂道:“阿叔,你不知道,
你哥哥心里自过活不得哩!”何清道:“他每日起了大钱大物,那里去了?有的是
钱和米,有甚么过活不得处?”阿嫂道:“你不知,为这黄泥冈上,前日一伙贩枣
子的客人,打劫了北京梁中书庆贺蔡太师的生辰纲去。如今济州府尹奉着太师钧旨:
限十日内,定要捉拿各贼解京;若还捉不着正身时,便要刺配远恶军州去。你不见
你哥哥先吃府尹刺了脸上‘迭配……州’字样,只不曾填甚么去处,早晚捉不着时,
实是受苦!他如何有心和你吃酒?我却才安排些酒食与你吃。他闷了几时了,你却怪
他不得。”何清道:“我也诽诽地听得人说道:‘有贼打劫了生辰纲去。’正在那
里地面上?”阿嫂道:“只听的说道黄泥冈上。”何清道:“却是甚么样人劫了?”
阿嫂道:“叔叔,你又不醉,我方才说了,是七个贩枣子的客人打劫了去。”何清
呵呵的大笑道:“原来恁地。知道是贩枣子的客人了,却闷怎地?何不差精细的人
去捉。”阿嫂道:“你倒说得好,便是没捉处。”何清笑道:“嫂嫂,倒要你忧。
哥哥放着常来的一班儿好酒肉弟兄,闲常不睬的是亲兄弟,今日才有事,便叫没捉
处。若是教兄弟得知,赚得几贯钱使,量这伙小贼,有甚难处!”阿嫂道:“阿叔,
你倒敢知得些风路?”何清笑道:“直等哥哥临危之际,兄弟却来有个道理救他。”
说了,便起身要去。阿嫂留住再吃两杯。

那妇人听了这话说得跷蹊,慌忙来对丈夫备细说了。何涛连忙叫请兄弟到面前。
何涛陪着笑脸说道:“兄弟,你既知此贼去向,如何不救我?”何清道:“我不知
甚么来历,我自和嫂子说耍。兄弟如何救的哥哥?”何涛道:“好兄弟,休得要看
冷暖。只想我日常的好处,休记我闲时的歹处,救我这条性命!”何清道:“哥哥,
你管下许多眼明手快的公人,也有三二百个,何不与哥哥出些大气?量兄弟一个,
怎救的哥哥!”何涛道:“兄弟休说他们,你的话眼里有些门路,休要把与别人做
好汉。你且说与我些去向,我自有补报你处。正教我怎地心宽!”何清道:“有甚
么去向,兄弟不省的!”何涛道:“你不要怄我,只看同胞共母之面。”何清道:
“不要慌。且待到至急处,兄弟自来出些气力,拿这伙小贼。”

阿嫂便道:“阿叔,胡乱救你哥哥,也是弟兄情分。如今被太师府钧帖,立等
要这一干人,天来大事,你却说小贼!”何清道:“嫂嫂,你须知我只为赌钱上,
吃哥哥多少言语。但是打骂,不曾和他争涉。闲常有酒有食,只和别人快活,今日
兄弟也有用处。”何涛见他话眼有些来历,慌忙取一个十两银子,放在桌上,说道:
“兄弟,权将这锭银收了。日后捕得贼人时,金银缎匹赏赐,我一力包办。”何清
笑道:“哥哥正是‘急来抱佛脚,闲时不烧香’。我若要你银子时,便是兄弟勒
你。你且把去收了,不要将来赚我。你若如此,我便不说。既是你两口儿我行陪话,
我说与你,不要把银子出来惊我。”何涛道:“银两都是官司信赏出的,如何没三
五百贯钱?兄弟,你休推却。我且问你:这伙贼却在那里有些来历?”何清拍着大
腿道:“这伙贼,我都捉在便袋里了。”何涛大惊道:“兄弟,你如何说这伙贼在
你便袋里?”何清道:“哥哥,你莫管我,自都有在这里便了。你只把银子收了去,
不要将来赚我,只要常情便了。我却说与你知道。”何清不慌不忙,叠着两个指头
说出来。有分教:郓城县里,引出个仗义英雄;梁山泊中,聚一伙擎天好汉。

毕竟何清对何涛说出甚人来,且听下回分解。