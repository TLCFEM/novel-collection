\chapter{还道村受三卷天书~宋公明遇九天玄女}

话说当下宋江在筵上对众好汉道:“小可宋江自蒙救护上山,到此连日饮宴,
甚是快乐,不知老父在家,正是何如。即目江州申奏京师,必然行移济州,着落郓
城县追捉家属,比捕正犯,恐老父存亡不保。宋江想念,欲往家中搬取老父上山,
以绝挂念,不知众弟兄还肯容否?”晁盖道:“贤弟,这件是人伦中大事,不成我
和你受用快乐,倒教家中老父吃苦,如何不依贤弟?只是众兄弟们连日辛苦,寨中
人马未定,再停两日,点起山寨人马,一径去取了来。”宋江道:“仁兄,再过几
日不妨,只恐江州行文到济州追捉家属,以此事不宜迟。今也不须点多人去,只宋
江潜地自去,和兄弟宋清搬取老父连夜上山来,那时乡中神不知,鬼不觉。若还多
带了人伴去,必然惊吓乡里,反招不便。”晁盖道:“贤弟路中倘有疏失,无人可
救。”宋江道:“若为父亲,死而无怨。”当日苦留不住,宋江坚执要行,便取个
毡笠带了,提条短棒,腰带利刃,便下山去。众头领送过金沙滩自回。

且说宋江过了渡,到朱贵酒店里上岸,出大路投郓城县来。路上少不得饥餐渴
饮,夜住晓行。一日奔宋家村,晚了,到不得,且投客店歇了。次日趱行到宋家村
时,却早,且在林子里伏了,等待到晚,却投庄上来敲后门。庄里听得,只见宋清
出来开门,见了哥哥,吃那一惊,慌忙道:“哥哥,你回家来怎地?”宋江道:“我
特来家取父亲和你。”宋清道:“哥哥,你在江州做了的事,如今这里都知道了。
本县差下这两个赵都头,每日来勾取,管定了我们,不得转动,只等江州文书到来,
便要捉我们父子二人,下在牢里监禁,听候拿你。日里夜间,一二百土兵巡绰。你
不宜迟,快去梁山泊请下众头领来,救父亲并兄弟。”

宋江听了,惊得一身冷汗,不敢进门,转身便走,奔梁山泊路上来。是夜月色
朦胧,路不分明,宋江只顾拣僻静小路去处走。约莫也走了一个更次,只听得背后
有人发喊起来。宋江回头听时,只隔一二里路,看见一簇火把照亮,只听得叫道:
“宋江休走!”宋江一头走,一面肚里寻思:“不听晁盖之言,果有今日之祸,皇
天可怜,垂救宋江则个。”远远望见一个去处,只顾走。少间风扫薄云,现出那轮
明月,宋江方才认得仔细,叫声苦,不知高低。看了那个去处,有名唤做还道村。
原来团团都是高山峻岭,山下一遭涧水,中间单单只一条路。入来这村,左来右去
走,只是这条路,更没第二条路。宋江认的这个村口,欲待回身,却被背后赶来的
人,已把住了路口,火把照耀如同白日。宋江只得奔入村里来,寻路躲避。抹过一
座林子,早看见一所古庙。但见:

墙垣颓损,殿宇倾斜。两廊画壁长苍苔,满地花砖生碧草。门前小鬼,折臂膊
不显狰狞;殿上判官,无幞头不成礼数。供床上蜘蛛结网,香炉内蝼蚁营窠。狐狸
常睡纸炉中,蝙蝠不离神帐里。
宋江只得推开庙门,乘着月光,入进庙里来,寻个躲避处。前殿后殿,相了一回,
安不得身,心里越慌。只听得外面有人道:“都管只走在这庙里!”宋江听得时,
是赵能声音,急没躲处,见这殿上一所神厨,宋江揭起帐幔,望里面探身便钻入神
厨里,安了短棒,做一堆儿伏在厨内,气也不敢喘。只听的外面拿着火把,照将入
来。

宋江在神厨里偷眼看时,赵能、赵得引着四五十人,拿着火把,各到处照,看
看照上殿来。宋江道:“我今番走了死路,望阴灵庇护则个,神明庇佑。”一个个
都走过了,没人看着神厨里。宋江道:“却不是天幸!”只见赵得将火把来神厨内
照一照,宋江道:“我这番端的受缚。”赵得一只手将朴刀杆挑起神帐,上下把火
只一照,火烟冲将起来,冲下一片黑尘来,正落在赵得眼里,眯了眼,便将火把丢
在地下,一脚踏灭了,走出殿门外来,对土兵们道:“这厮不在庙里,别又无路,
却走向那里去了?”众土兵道:“多应这厮走入村中树林里去了,这里不怕他走脱。
这个村唤做还道村,只有这条路出入,里面虽有高山林木,却无路上的去。都头只
把住村口,他便会插翅飞上天去,也走不脱了。待天明,村里去细细搜捉。”赵得
道:“也是。”引了土兵下殿去了。

宋江道:“却不是神明护佑!若还得了性命,必当重修庙宇,再建祠堂,阴灵
保佑则个。”说犹未了,只听的有几个土兵在于庙门前叫道:“都头,在这里了。”
赵能、赵得和众人一伙抢入来。宋江道:“却不又是晦气,这遭必被擒捉。”赵能
到庙前问道:“在那里?”土兵道:“都头,你来看庙门上两个尘手迹,以定是却
才推开庙门,闪在里面去了。”赵能道:“说的是,再仔细搜一搜看。”

这伙人再入庙里来搜看,宋江道:“我命运这般蹇拙,今番必是休了!”那伙
人去殿前殿后搜遍,只不曾翻过砖来。众人又搜了一回,火把看看照上殿来。赵能
道:“多是只在神厨里。却才兄弟看不仔细,我自照一照看。”一个土兵拿着火把,
赵能一手揭起帐幔,五七个人伸头来看。不看万事俱休,才看一看,只见神厨里卷
起一阵恶风,将那火把都吹灭了,黑腾腾罩了庙宇,对面不见。赵能道:“却又作
怪!平地里卷起这阵恶风来,想是神明在里面,定嗔怪我们只管来照,因此起这阵
恶风显应。我们且去罢,只守住村口,待天明再来寻。”赵得道:“只是神厨里不
曾看得仔细,再把枪去搠一搠。”赵能道:“也是。”

两个却待向前,只听的殿后又卷起一阵怪风,吹的飞沙走石,滚将下来,摇的
那殿宇吸吸地动,罩下一阵黑云,布合了上下,冷气侵人,毛发竖起。赵能情知不
好,叫了赵得道:“兄弟快走,神明不乐。”众人一哄都奔下殿来,望庙门外跑走,
有几个攧翻了的,也有闪腿的,爬得起来,奔命走出庙门。只听得庙里有人叫:
“饶恕我们!”赵能再入来看时,两三个土兵跌倒在龙墀里,被树根钩住了衣服,
死也挣不脱,手里丢了朴刀,扯着衣裳叫饶。宋江在神厨里听了,忍不住笑。

赵能把土兵衣服解脱了,领出庙门去。有几个在前面的土兵说道:“我说这神
道最灵,你们只管在里面缠障,引的小鬼发作起来。我们只去守住了村口等他,须
不吃他飞了去。”赵能、赵得道:“说得是。只消村口四下里守定。”众人都望村
口去了。

只说宋江在神厨里口称惭愧道:“虽不被这厮们拿了,却怎能够出村口去?”
正在厨内寻思,百般无计,只听的后面廊下有人出来。宋江道:“却又是苦也!早
是不钻出去。”只见两个青衣童子,径到厨边举口道:“小童奉娘娘法旨,请星主
说话。”宋江那里敢做声答应。外面童子又道:“娘娘有请,星主可行。”宋江也
不敢答应。外面童子又道:“宋星主休得迟疑,娘娘久等。”宋江听的莺声燕语,
不是男子之音,便从神柜底下钻将出来,看时,却是两个青衣女童侍立在床边。宋
江吃了一惊,却是两个泥神。只听的外面又说道:“宋星主,娘娘有请。”宋江分
开帐幔,钻将出来,只见是两个青衣螺髻女童,齐齐躬身,各打个稽首。宋江看那
女童时,但见:

朱颜绿发,皓齿明眸。飘飘不染尘埃,耿耿天仙风韵。螺蛳髻山峰堆拥,凤头
鞋莲瓣轻盈。领抹深青,一色织成银缕;带飞真紫,双环结就金霞。依稀阆苑董双
成,仿佛蓬莱花鸟使。
当下宋江问道:“二位仙童自何而来?”青衣道:“奉娘娘法旨,有请星主赴宫。”
宋江道:“仙童差矣!我自姓宋,名江,不是甚么星主。”青衣道:“如何差了?请
星主便行,娘娘久等。”宋江道:“甚么娘娘?亦不曾拜识,如何敢去?”青衣道:
“星主到彼便知,不必询问。”宋江道:“娘娘在何处?”青衣道:“只在后面宫
中。”青衣前引便行,宋江随后跟下殿来,转过后殿侧首一座子墙角门,青衣道:
“宋星主从此间进来。”宋江跟入角门来看时,星月满天,香风拂拂,四下里都是
茂林修竹。宋江寻思道:“原来这庙后又有这个去处。早知如此,却不来这里躲避,
不受那许多惊恐。”

宋江行时,觉道香坞两行夹种着大松树,都是合抱不交的,中间平坦一条龟背
大街。宋江看了,暗暗寻思道:“我倒不想古庙后有这般好路径。”跟着青衣,行
不过一里来路,听得潺潺的涧水响。看前面时,一座青石桥,两边都是朱栏杆,岸
上栽种奇花、异草、苍松、茂竹、翠柳、夭桃、桥下翻银滚雪般的水,流从石洞里
去。过的桥基看时,两行奇树,中间一座大朱红棂星门。宋江入的棂星门看时,抬
头见一所宫殿。但见:

金钉朱户,碧瓦雕檐。飞龙盘柱戏明珠,双凤帏屏明晓日。红泥墙壁,纷纷御
柳间宫花;翠霭楼台,淡淡祥光笼瑞影。窗横龟背,香风冉冉透黄纱;帘卷虾须,
皓月团团悬紫绮。若非天上神仙府,定是人间帝主家。
宋江见了,寻思道:“我生居郓城县,不曾听的说有这个去处。”心中惊恐,不敢
动脚。青衣催促请星主行。一引,引入门内,有个龙墀,两廊下尽是朱红亭柱,都
挂着绣帘,正中一所大殿,殿上灯烛荧煌。青衣从龙墀内一步步引到月台上,听得
殿上阶前又有几个青衣道:“娘娘有请星主进来。”宋江到大殿上,不觉肌肤战栗,
毛发倒竖,下面都是龙凤砖阶。青衣入帘内奏道:“请至宋星主在阶前。”宋江到
帘前御阶之下,躬身再拜,俯伏在地,口称:“臣乃下浊庶民,不识圣上,伏望天
慈,俯赐怜悯。”御帘内传旨,教请星主坐。宋江那里敢抬头。教四个青衣扶上锦
墩坐,宋江只得勉强坐下。殿上喝声卷帘,数个青衣早把珠帘卷起,搭在金钩上。
娘娘问道:“星主别来无恙?”宋江起身再拜道:“臣乃庶民,不敢面觑圣容。”
娘娘道:“星主既然至此,不必多礼。”宋江恰才敢抬头舒眼,看见殿上金碧交辉,
点着龙灯凤烛;两边都是青衣女童,持笏捧圭,执旌擎扇侍从;正中七宝九龙床上,
坐着那个娘娘。宋江看时,但见:

头绾九龙飞凤髻,身穿金缕绛绡衣。蓝田玉带曳长裙,白玉圭璋擎彩袖。脸如
莲萼,天然眉目映云环;唇似樱桃,自在规模端雪体。正大仙容描不就,威严形象
画难成。

那娘娘口中说道:“请星主到此。”命童子献酒。两下青衣女童,执着奇花宝
瓶,捧酒过来,斟在玉杯内。一个为首的女童,执玉杯递酒,来劝宋江。宋江起身,
不敢推辞,接过玉杯,朝娘娘跪饮了一杯。宋江觉道这酒馨香馥郁,如醍醐灌顶,
甘露洒心。又是一个青衣,捧过一盘仙枣,上劝宋江。宋江战战兢兢,怕失了体面,
尖着指头,拿了一枚,就而食之,怀核在手。青衣又斟过一杯酒来劝宋江,宋江又
一饮而尽。娘娘法旨,教再劝一杯,青衣再斟一杯酒过来劝宋江,宋江又饮了。仙
女托过仙枣,又食了两枚。共饮过三杯仙酒,三枚仙枣。宋江便觉道春色微醺,又
怕酒后醉失体面,再拜道:“臣不胜酒量,望乞娘娘免赐。”殿上法旨道:“既是
星主不能饮酒,可止。教取那三卷天书赐与星主。”青衣去屏风背后,玉盘中托出
黄罗袱子,包着三卷天书,度与宋江。宋江看时,可长五寸,阔三寸,厚三寸,不
敢开看,再拜祗受,藏于袖中。娘娘法旨道:“宋星主,传汝三卷天书,汝可替天
行道为主,全忠仗义为臣,辅国安民,去邪归正。吾有四句天言,汝当记取,终身
佩受,勿忘勿泄。”宋江再拜,愿受天言。娘娘法旨道:
遇宿重重喜,逢高不是凶。
外夷及内寇,几处见奇功。
宋江听毕,再拜谨受。娘娘法旨道:“玉帝因为星主魔心未断,道行未完,暂罚下
方,不久重登紫府,切不可分毫懈怠!若是他日罪下酆都,吾亦不能救汝。此三卷
之书,可以善观熟视,只可与天机星同观,其他皆不可见。功成之后,便可焚之,
勿留在世。所嘱之言,汝当记取。目今天凡相隔,难以久留,汝当速回。”便令童
子急送星主回去,“他日琼楼金阙,再当重会”。

宋江便谢了娘娘,跟随青衣女童下得殿庭来,出得棂星门,送至石桥边,青衣
道:“恰才星主受惊,不是娘娘护佑,已被擒拿。天明时,自然脱离了此难。星主
看石桥下水里二龙相戏。”宋江凭栏看时,果见二龙戏水。二青衣望下一推,宋江
大叫一声,却撞在神厨内,觉来乃是南柯一梦。

宋江爬将起来看时,月影正午,料是三更时分。宋江把袖子里摸时,手内枣核
三个,袖里帕子包着天书。摸将出来看时,果是三卷天书,又只觉口里酒香。宋江
想道:“这一梦真乃奇异,似梦非梦。若把做梦来,如何有这天书在袖子里,口中
又酒香,枣核在手里,说与我的言语,都记得,不曾忘了一句?不把做梦来,我自
分明在神厨里,一交攧将入来。有甚难见处?想是此间神圣最灵,显化如此。只是
不知是何神明?”揭起帐幔看时,九龙椅上坐着一个妙面娘娘,正和梦中一般。宋
江寻思道:“这娘娘呼我做星主,想我前生非等闲人也。这三卷天书,必然有用。
分付我的四句天言,不曾忘了。青衣女童道:‘天明时自然脱离此村之厄。’如今
天色渐明,我却出去。”

便探手去厨里摸了短棒,把衣服拂拭了,一步步走下殿来,便从左廊下转出庙
前,仰面看时,旧牌额上刻着四个金字道:“玄女之庙。”宋江以手加额称谢道:
“惭愧!原来是九天玄女娘娘传受与我三卷天书,又救了我的性命。如若能够再见
天日之面,必当来此重修庙宇,再建殿庭。伏望圣慈俯垂护佑。”称谢已毕,只得
望着村口悄悄出来。

离庙未远,只听得前面远远地喊声连天。宋江寻思道:“又不济了。立住了脚,
且未可出去。我若到他面前,定吃他拿了。不如且在这里路旁树背后躲一躲。”却
才闪得入树背后去,只见数个土兵急急走得喘做一堆,把刀枪拄着,一步步攧将入
来,口里声声都只叫道:“神圣救命则个。”宋江在树背后看了,寻思道:“却又
作怪。他们把着村口,等我出来拿我,却又怎地抢入来?”再看时,赵能也抢入来,
口里叫道:“我们都是死也!”宋江道:“那厮如何恁地慌?”却见背后一条大汉
追将入来。那大汉上半截不着一丝,露出鬼怪般肉,手里拿着两把夹钢板斧,口里
喝道:“含鸟休走!”远观不睹,近看分明,正是黑旋风李逵。宋江想道:“莫非
是梦里么?”不敢走出去。

那赵能正走到庙前,被松树根只一绊,一交攧在地下。李逵赶上,就势一脚踏
住脊背,手起大斧,却待要砍,背后又是两筹好汉赶上来,把毡笠儿掀在脊梁上,
各挺一条朴刀,上首的是欧鹏,下首的是陶宗旺。李逵见他两个赶来,恐怕争功,
坏了义气,就手把赵能一斧,砍做两半,连胸脯都砍开了,跳将起来,把土兵赶杀,
四散走了。宋江兀自不敢便走出来。

背后只见又赶上三筹好汉,也杀将来。前面赤发鬼刘唐,第二石将军石勇,第
三催命判官李立。这六筹好汉说道:“这厮们都杀散了,只寻不见哥哥,却怎生是
好?”石勇叫道:“兀那松树背后一个人立在那里!”宋江方才敢挺身出来,说道:
“感谢众兄弟们又来救我性命,将何以报大恩?”六筹好汉见了宋江,大喜道:“哥
哥有了!快去报与晁头领得知。”石勇、李立分头去了。

宋江问刘唐道:“你们如何得知,来这里救我?”刘唐答道:“哥哥前脚下得
山来,晁头领与吴军师放心不下,便叫戴院长随即下来,探听哥哥下落。晁头领又
自己放心不下,再着我等众人前来接应,只恐哥哥有些疏失,半路里撞见戴宗道:
‘两个贼驴追赶捕捉哥哥。’晁头领大怒,分付戴宗去山寨,只教留下吴军师、公
孙胜、阮家三兄弟、吕方、郭盛、朱贵、白胜看守寨栅,其余兄弟,都叫来此间寻
觅哥哥,听得人说道:‘赶宋江入还道村去了。’村口守把的这厮们,尽数杀了,
不留一个,只有这几个奔进村里来。随即李大哥追来,我等都赶入来,不想哥哥在
这里。”说犹未了,石勇引将晁盖、花荣、秦明、黄信、薛永、蒋敬、马麟到来,
李立引将李俊、穆弘、张横、张顺、穆春、侯健、萧让、金大坚,一行众多好汉都
相见了。

宋江作谢众位头领。晁盖道:“我叫贤弟不须亲自下山,不听愚兄之言,险些
儿又做出来。”宋江道:“小可兄弟,只为父亲这一事,悬肠挂肚,坐卧不安,不
由宋江不来取。”晁盖道:“好教贤弟欢喜,令尊并令弟家眷,我先叫戴宗引杜迁、
宋万、王矮虎、郑天寿、童威、童猛送去,已到山寨中了。”宋江听得,大喜,拜
谢晁盖道:“得仁兄如此施恩,宋江死亦无怨!”

晁盖、宋江俱各欢喜,与众头领各各上马,离了还道村口,宋江在马上以手加
额,望空顶礼,称谢神明庇佑之力,容日专当拜还心愿。有古风一篇,单道宋江忠
义得天之助:
昏朝气运将颠覆,四海英雄起微族。
流光垂象在山东,天罡上应三十六。
瑞气盘旋绕郓城,此乡生降宋公明。
幼年涉猎诸经史,长来为吏惜人情。
仁义礼智信皆备,兼受九天玄女经。
豪杰交游满天下,逢凶化吉天生成。
他年直上梁山泊,替天行道动天兵。

且说一行人马离了还道村,径回梁山泊来。吴学究领了守山头领,直到金沙滩,
都来迎接,前到得大寨聚义厅上,众好汉都相见了。宋江急问道:“老父何在?”
晁盖便叫请宋太公出来,不多时,铁扇子宋清策着一乘山轿,抬着宋太公到来,众
人扶策下轿上厅来。宋江见了,喜从天降,笑逐颜开。宋江再拜道:“老父惊恐,
宋江做了不孝之子,负累了父亲吃惊受怕。”宋太公道:“叵耐赵能那厮弟兄两个,
每日拨人来守定了我们,只待江州公文到来,便要捉取我父子二人,解送官司。听
得你在庄后敲门,此时已有八九个土兵在前面草厅上,续后不见了,不知怎地赶出
去了。到三更时候,又有二百余人把庄门开了,将我搭扶上轿,抬了,教你兄弟四
郎收拾了箱笼,放火烧了庄院。那时不由我问个缘由,径来到这里。”宋江道:“今
日父子团圆相见,皆赖众兄弟之力也。”叫兄弟宋清拜谢了众头领,晁盖众人都来
参拜宋太公已毕,一面杀牛宰马,且做庆喜筵席,作贺宋公明父子团圆。当日尽醉
方散,次日又排筵席贺喜,大小头领尽皆欢喜。

第三日,晁盖又体己备个筵席,庆贺宋江父子完聚,忽然感动公孙胜一个念头:
思忆老母在蓟州,离家日久,未知如何。众人饮酒之时,只见公孙胜起身对众头领
说道:“感蒙众位豪杰相带贫道许多时,恩同骨肉。只是小道自从跟着晁头领到山,
逐日宴乐,一向不曾还乡看视老母。亦恐我真人本师悬望,欲待回乡省视一遭,暂
别众头领三五个月,再回来相见,以满小道之愿,免致老母挂念悬望。”晁盖道:
“向日已闻先生所言,令堂在北方无人侍奉,今既如此说时,难以阻当,只是不忍
分别。虽然要行,再待来日相送。”公孙胜谢了。当日尽醉方散,各自归房安歇。
次日早,就关下排了筵席,与公孙胜饯行。

且说公孙胜依旧做云游道士打扮了,腰裹腰包、肚包,背上雌雄宝剑,肩胛上
挂着棕笠,手中拿把鳖壳扇,便下山来。众头领接住,就关下筵席,各各把盏送别。
饯行已遍,晁盖道:“一清先生,此去难留,却不可失信。本是不容先生去,只是
老尊堂在上,不敢阻当。百日之外,专望鹤驾降临,切不可爽约。”公孙胜道:“重
蒙列位头领看待许久,小道岂敢失信!回家参过本师真人,安顿了老母,便回山寨。”
宋江道:“先生何不将带几个人去,一发就搬取老尊堂上山,早晚也得侍奉。”公
孙胜道:“老母平生只爱清幽,吃不得惊,因此不敢取来。家中自有田产山庄,
老母自能料理。小道只去省视一遭,便来再得聚义。”宋江道:“既然如此,专听
尊命。只望早早降临为幸!”晁盖取出一盘黄白之资相送,公孙胜道:“不消许多,
但只够盘缠足矣。”晁盖定教收了一半,打拴在腰包里,打个稽首,别了众人,过
金沙滩便行,望蓟州去了。

众头领席散,却待上山,只见黑旋风李逵就关下放声大哭起来。宋江连忙问道:
“兄弟,你如何烦恼?”李逵哭道:“干鸟气么!这个也去取爷,那个也去望娘,
偏铁牛是土掘坑里钻出来的!”晁盖便问道:“你如今待要怎地?”李逵道:“我
只有一个老娘在家里。我的哥哥,又在别人家做长工,如何养得我娘快乐?我要去
取他来这里快乐几时也好。”晁盖道:“兄弟说的是。我差几个人同你去,取了上
山来,也是十分好事。”宋江便道:“使不得。李家兄弟生性不好,回乡去必然有
失。若是教人和他去,亦是不好。况且他性如烈火,到路上必有冲撞。他又在江州
杀了许多人,那个不认得他是黑旋风?这几时,官司如何不行移文书到那里了,必
然原籍追捕。你又形貌凶恶,倘有疏失,路程遥远,如何得知?你且过几时,打听
得平静了去取未迟。”李逵焦躁,叫道:“哥哥,你也是个不平心的人。你的爷,
便要取上山来快活,我的娘,由他在村里受苦。兀的不是气破了铁牛的肚子!”宋
江道:“兄弟,你不要焦躁。既是要去取娘,只依我三件事,便放你去。”李逵道:
“你且说那三件事?”宋江点两个指头,说出这三件事来。有分教:李逵施为撼地
摇天手,来斗巴山跳涧虫。

毕竟宋江对李逵说出那三件事来,且听下回分解。