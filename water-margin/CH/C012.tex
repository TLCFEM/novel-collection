\chapter{梁山泊林冲落草~汴京城杨志卖刀}

话说林冲打一看时,只见那汉子头戴一顶范阳毡笠,上撒着一托红缨;穿一领
白缎子征衫,系一条纵线绦;下面青白间道行缠,抓着裤子口,獐皮袜,带毛牛膀
靴;跨口腰刀,提条朴刀;生得七尺五六身材;面皮上老大一搭青记,腮边微露些
少赤须;把毡笠子掀在脊梁上,坦开胸脯,带着抓角儿软头巾,挺手中朴刀,高声
喝道:“你那泼贼,将俺行李财帛那里去了?”林冲正没好气,那里答应,睁圆怪
眼,倒竖虎须,挺著朴刀,抢将来斗那个大汉。此时残雪初晴,薄云方散,溪边踏
一片寒冰,岸畔涌两条杀气,一往一来,斗到三十来合,不分胜败。

两个又斗了十数合,正斗到分际,只见山高处叫道:“两位好汉不要斗了!”
林冲听得,蓦地跳出圈子外来。两个收住手中朴刀,看那山顶上时,却是白衣秀士
王伦和杜迁、宋万,并许多小喽罗,走下山来,将船渡过了河,说道:“两位好汉,
端的好两口朴刀,神出鬼没!这个是俺的兄弟豹子头林冲。青面汉,你却是谁?愿通
姓名。”那汉道:“洒家是三代将门之后,五侯杨令公之孙,姓杨,名志,流落在
此关西。年纪小时,曾应过武举,做到殿司制使官,道君因盖万岁山,差一般十个
制使去太湖边搬运花石纲,赴京交纳。不想洒家时乖运蹇,押着那花石纲,来到黄
河里,遭风打翻了船,失陷了花石纲,不能回京赴任,逃去他处避难。如今赦了俺
们罪犯,洒家今来收的一担儿钱物,待回东京去枢密院使用,再理会本身的勾当,
打从这里经过,顾倩庄家挑那担儿,不想被你们夺了。可把来还洒家如何?”王伦
道:“你莫是绰号唤做青面兽的?”杨志道:“洒家便是。”王伦道:“既然是杨
制使,就请到山寨吃三杯水酒,纳还行李如何?”杨志道:“好汉既然认得洒家,
便还了俺行李,更强似请吃酒。”王伦道:“制使,小可数年前到东京应举时,便
闻制使大名。今日幸得相见,如何教你空去!且请到山寨少叙片时,并无他意。”

杨志听说了,只得跟了王伦一行人等过了河,上山寨来。就叫朱贵同上山寨相
会,都来到寨中聚义厅上。左边一带四把交椅,却是王伦、杜迁、宋万、朱贵;右
边一带两把交椅,上首杨志,下首林冲,都坐定了。王伦叫杀羊置酒,安排筵宴,
管待杨志,不在话下。

话休絮烦,酒至数杯,王伦心里想道:“若留林冲,实形容得我们不济,不如
我做个人情,并留了杨志,与他作敌。”因指着林冲对杨志道:“这个兄弟,他是
东京八十万禁军教头,唤做豹子头林冲,因这高太尉那厮安不得好人,把他寻事刺
配沧州,那里又犯了事,如今也新到这里。却才制使要上东京勾当,不是王伦纠合
制使,小可兀自弃文就武,来此落草,制使又是有罪的人,虽经赦宥,难复前职。
亦且高俅那厮现掌军权,他如何肯容你?不如只就小寨歇马,大秤分金银,大碗吃
酒肉,同做好汉,不知制使心下主意若何?”杨志答道:“重蒙众头领如此带携,
只是洒家有个亲眷,现在东京居住。前者官事连累了他,不曾酬谢得。今日欲要投
那里走一遭,望众头领还了洒家行李;如不肯还,杨志空手也去了。”王伦笑道:
“既是制使不肯在此,如何敢勒逼入伙?且请宽心住一宵,明日早行。”杨志大喜,
当日饮酒到一更方歇,各自去歇息了。

次日早起来,又置酒与杨志送行。吃了早饭,众头领叫一个小喽罗,把昨夜担
儿挑了,一齐都送下山来,到路口与杨志作别。叫小喽罗渡河,送出大路。众人相
别了,自回山寨。王伦自此方才肯教林冲坐第四位,朱贵坐第五位。从此五个好汉
在梁山泊打家劫舍,不在话下。

只说杨志出了大路,寻个庄家挑了担子,发付小喽罗自回山寨。杨志取路,不
数日,来到东京。入得城来,寻个客店安歇下;庄客交还担儿,与了些银两,自回
去了。杨志到店中放下行李,解了腰刀、朴刀,叫店小二将些碎银子买些酒肉吃了。
过数日,央人来枢密院打点,理会本等的勾当,将出那担儿内金银财物,买上告下,
再要补殿司府制使职役。把许多东西都使尽了,方才得申文书,引去见殿帅高太尉。
来到厅前,那高俅把从前历事文书都看了,大怒道:“既是你等十个制使去运花石
纲,九个回到京师交纳了,偏你这厮把花石纲失陷了;又不来首告,倒又在逃,许
多时捉拿不着。今日再要勾当,虽经赦宥所犯罪名,难以委用。”把文书一笔都批
倒了,将杨志赶出殿帅府来。

杨志闷闷不已,回到客店中,思量:“王伦劝俺,也见得是。只为洒家清白姓
字,不肯将父母遗体来玷污了。指望把一身本事,边庭上一枪一刀,博个封妻荫子,
也与祖宗争口气;不想又吃这一闪。高太尉,你忒毒害,恁地刻薄!”心中烦恼了
一回。在客店里又住几日,盘缠都使尽了。正是:
花石纲原没纪纲,奸邪到底困忠良。
早知廊庙当权重,不若山林聚义长。

杨志寻思道:“却是恁地好?只有祖上留下这口宝刀,从来跟着洒家,如今事
急无措,只得拿去街上货卖得千百贯钱钞,好做盘缠,投往他处安身。”当日将了
宝刀,插了草标儿,上市去卖,走到马行街内,立了两个时辰,并无一个人问。将
立到晌午时分,转来到天汉州桥热闹处去卖。杨志立未久,只见两边的人都跑入河
下巷内去躲。杨志看时,只见都乱撺,口里说道:“快躲了!大虫来也!”杨志道:
“好作怪!这等一片锦城池,却那得大虫来!”当下立住脚看时,只见远远地黑凛
凛一大汉,吃得半醉,一步一攧撞将来。杨志看那人时,形貌生得粗陋。但见:

面目依稀似鬼,身持仿佛如人。杈怪树,变为形骸;臭秽枯桩,化作腌
魍魉。浑身遍体,都生渗渗濑濑沙鱼皮;夹脑连头,尽长拳拳弯弯卷螺发。胸前
一片紧顽皮,额上三条强拗皱。
原来这人是京师有名的破落户泼皮,叫做没毛大虫牛二,专在街上撒泼、行凶、撞
闹,连为几头官司,开封府也治他不下,以此满城人见那厮来都躲了。

却说牛二抢到杨志面前,就手里把那口宝刀扯将出来,问道:“汉子,你这刀
要卖几钱?”杨志道:“祖上留下宝刀,要卖三千贯。”牛二喝道:“甚么鸟刀,
要卖许多钱!我三十文买一把,也切得肉,切得豆腐。你的鸟刀有甚好处,叫做宝
刀!”杨志道:“洒家的须不是店上卖的白铁刀,这是宝刀。”牛二道:“怎的唤
做宝刀?”杨志道:“第一件,砍铜剁铁,刀口不卷;第二件,吹毛得过;第三件,
杀人刀上没血。”牛二道:“你敢剁铜钱么?”杨志道:“你便将来剁与你看。”

牛二便去州桥下香椒铺里讨了二十文当三钱,一垛儿将来放在州桥栏干上,叫
杨志道:“汉子,你若剁得开时,我还你三千贯。”那时看的人,虽然不敢近前,
向远远地围住了望。杨志道:“这个直得甚么?”把衣袖卷起,拿刀在手,看的较
准,只一刀,把铜钱剁做两半,众人都喝采。牛二道:“喝甚么鸟采!你且说第二
件是甚么?”杨志道:“吹毛得过:若把几根头发,望刀口上只一吹,齐齐都断。”
牛二道:“我不信。”自把头上拔下一把头发,递与杨志,“你且吹我看。”杨志
左手接过头发,照着刀口上尽气力一吹,那头发都做两段,纷纷飘下地来,众人喝
采,看的人越多了。牛二又问:“第三件是甚么?”杨志道:“杀人刀上没血。”
牛二道:“怎么杀人刀上没血?”杨志道:“把人一刀砍了,并无血痕,只是个快。”
牛二道:“我不信,你把刀来剁一个人我看。”杨志道:“禁城之中,如何敢杀人?
你不信时,取一只狗来杀与你看。”牛二道:“你说杀人,不曾说杀狗!”杨志道:
“你不买便罢,只管缠人做甚么?”牛二道:“你将来我看。”杨志道:“你只顾
没了当,洒家又不是你撩拨的!”牛二道:“你敢杀我?”杨志道:“和你往日无
冤,昔日无仇,一物不成两物,现在没来由杀你做甚么?”

牛二紧揪住杨志说道:“我偏要买你这口刀。”杨志道:“你要买,将钱来。”
牛二道:“我没钱。”杨志道:“你没钱,揪住洒家怎地?”牛二道:“我要你这
口刀。”杨志道:“我不与你。”牛二道:“你好男子,剁我一刀。”杨志大怒,
把牛二推了一交。牛二爬将起来,钻入杨志怀里。杨志叫道:“街坊邻舍,都是证
见:杨志无盘缠,自卖这口刀,这个泼皮强夺洒家的刀,又把俺打。”街坊人都怕
这牛二,谁敢向前来劝。牛二喝道:“你说我打你,便打杀直甚么?”口里说,一
面挥起右手一拳打来,杨志霍地躲过,拿着刀抢入来,一时性起,望牛二嗓根上搠
个着,扑地倒了。杨志赶入去,把牛二胸脯上又连搠了两刀,血流满地,死在地上。

杨志叫道:“洒家杀死这个泼皮,怎肯连累你们!泼皮既已死了,你们都来同
洒家去官府里出首。”坊隅众人慌忙拢来,随同杨志径投开封府出首,正值府尹坐
衙,杨志拿着刀和地方邻舍众人都上厅来,一齐跪下,把刀放在面前。杨志告道:
“小人原是殿司制使,为因失陷花石纲,削去本身职役,无有盘缠,将这口刀在街
货卖。不期被个泼皮破落户牛二强夺小人的刀,又用拳打小人;因此一时性起,将
那人杀死,众邻舍都是证见。”众人亦替杨志告说,分诉了一回。府尹道:“既是
自行前来出首,免了这厮入门的款打。”且叫取一面长枷枷了。差两员相官带了仵
作行人,监押杨志并众邻舍一干人犯,都来天汉州桥边登场检验了,迭成文案,众
邻舍都出了供状,保放,随衙听候,当厅发落,将杨志于死囚牢里监守。但见:

推临狱内,拥入牢门。黄须节级,麻绳准备吊绷揪;黑面押牢,木匣安排牢锁
镣。杀威棒,狱卒断时腰痛;撒子角,囚人
见了心惊。休言死去见阎王,只此便如真地狱。

且说杨志押到死囚牢里,众多押牢禁子、节级,见说杨志杀死没毛大虫牛二,
都可怜他是个好男子,不来问他取钱,又好生看觑他。天汉州桥下众人,为是杨志
除了街上害人之物,都敛些盘缠,凑些银两,来与他送饭,上下又替他使用。推司
也觑他是个身首的好汉,又与东京街上除了一害,牛二家又没苦主,把款状都改得
轻了。三推六问,却招做一时斗殴杀伤,误伤人命。待了六十日限满,当厅推司禀
过府尹,将杨志带出厅前,除了长枷,断了二十脊杖,唤个文墨匠人刺了两行金印,
迭配北京大名府留守司充军。那口宝刀没官入库。

当厅押了文牒,差两个防送公人,免不得是张龙、赵虎,把七斤半铁叶子盘头
护身枷钉了。分付两个公人,便教监押上路。天汉州桥那几个大户科敛些银两钱物,
等候杨志到来,请他两个公人一同到酒店里吃了些酒食,把出银两,赍发两位防送
公人,说道:“念杨志是个好汉,与民除害,今去北京,路途中望乞二位上下照觑,
好生看他一看。”张龙、赵虎道:“我两个也知他是好汉,亦不必你众位分付,但
请放心。”杨志谢了众人,其余多的银两,尽送与杨志做盘缠,众人各自散了。

话里只说杨志同两个公人来到原下的客店里,算还了房钱,取了原寄的衣服行
李。安排些酒食,请了两位公人。寻医士赎了几个棒疮的膏药,贴了棒疮,便同两
个公人上路。三个望北京进发,五里单牌,十里双牌,逢州过县,买些酒肉,不时
间请张龙、赵虎同吃。三个在路,夜宿旅馆,晓行驿道,不数日来到北京,入得城
中,寻个客店安下。

原来北京大名府留守司,上马管军,下马管民,最有权势。那留守唤作梁中书,
讳世杰,他是东京当朝太师蔡京的女婿。当日是二月初九日,留守升厅,两个公人
解杨志到留守司厅前,呈上开封府公文,梁中书看了。原在东京时,也曾认得杨志,
当下一见了,备问情由。杨志便把高太尉不容复职,使尽钱财,将宝刀货卖,因而
杀死牛二的实情通前一一告禀了。梁中书听得大喜,当厅就开了枷,留在厅前听用。
押了批回与两个公人,自回东京了,不在话下。

只说杨志自在梁中书府中早晚殷勤听候使唤,梁中书见他勤谨,有心要抬举他,
欲要迁他做个军中副牌,月支一分请受,只恐众人不伏;因此传下号令,教军政司
告示大小诸将人员,来日都要出东郭门教场中去演武试艺。当晚梁中书唤杨志到厅
前,梁中书道:“我有心要抬举你做个军中副牌,月支一分请受,只不知你武艺如
何?”杨志禀道:“小人应过武举出身,曾做殿司府制使职役。这十八般武艺,自
小习学。今日蒙恩相抬举,如拨云见日一般,杨志若得寸进,当效衔环背鞍之报。”
梁中书大喜,赐与一副衣甲。

当夜无事。次日天晓,时当二月中旬,正值风和日暖。梁中书早饭已罢,带领
杨志上马,前遮后拥,往东郭门来,上得教场中,大小军卒,并许多官员接见。就
演武厅前下马,到厅上,正面撒着一把浑银交椅,坐下。左右两边,齐臻臻地排着
两行官员,指挥使、团练使、正制使、统领使、牙将、校尉、正牌军、副牌军,前
后周围,恶狠狠地列着百员将校。正将台上立着两个都监:一个唤做李天王李成,
一个唤做闻大刀闻达,二人皆有万夫不当之勇,统领着许多军马,一齐都来朝着梁
中书呼三声喏。却早将台上竖起一面黄旗来,将台两边左右列着三五十对金鼓手,
一齐发起擂来,品了三通画角,发了三通擂鼓,教场里面谁敢高声。又见将台上竖
起一面净平旗来,前后五军,一齐整肃;将台上把一面引军红旗麾动,只见鼓声响
处,五百军列成两阵,军士各执器械在手;将台上又把白旗招动,两阵马军齐齐地
都立在面前,各把马勒住。

梁中书传下令来,叫唤副牌军周谨向前听令。右阵里周谨听得呼唤,跃马到厅
前,跳下马,插了枪,暴雷也似声个大喏。梁中书道:“着副牌军施逞本身武艺。”
周谨得了将令,绰枪上马,在演武厅前,左盘右旋,右盘左旋,将手中枪使了几路,
众人喝采。梁中书道:“叫东京对拨来的军健杨志。”杨志转过厅前,唱个大喏。
梁中书道:“杨志,我知你原是东京殿司府制使军官,犯罪配来此间,即目盗贼猖
狂,国家用人之际,你敢与周谨比试武艺高低?如若赢得,便迁你充其职役。”杨
志道:“若蒙恩相差遣,安敢有违钧旨。”梁中书叫取一匹战马来,教甲仗库随行
官吏应付军器,教杨志披挂上马,与周谨比试。杨志去厅后把取来衣甲穿了,拴束
罢,带了头盔、弓、箭、腰刀,手拿长枪上马,从厅后跑将出来。梁中书看了道:
“着杨志与周谨先比枪。”周谨怒道:“这个贼配军敢来与我交枪!”谁知恼犯了
这个好汉,来与周谨斗武。不因这番比试,有分教:杨志在万马丛中闻姓字,千军
队里夺头功。

毕竟杨志与周谨比试,引出甚么人来,且听下回分解。