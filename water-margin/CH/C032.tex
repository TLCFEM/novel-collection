\chapter{武行者醉打孔亮~锦毛虎义释宋江}

当时两个斗了十数合,那先生被武行者卖个破绽,让那先生两口剑斫将入来,
被武行者转过身来,看得亲切,只一戒刀,那先生的头,滚落在一边,尸首倒在石
上。武行者大叫:“庵里婆娘出来,我不杀你,只问你个缘故。”只见庵里走出那
个妇人来,倒地便拜。武行者道:“你休拜我。你且说,这里是甚么去处?那先生
却是你的甚么人?”那妇人哭着道:“奴是这岭下张太公家女儿。这庵是奴家祖上
坟庵。这先生不知是那里人,来我家里投宿,言说善习阴阳,能识风水。我家爹娘,
不合留他在庄上,因请他来这里坟上观看地理,被他说诱,又留他住了几日。那厮
一日见了奴家,便不肯去了。住了三两个月,把奴家爹娘哥嫂都害了性命,却把奴
家强骗在此坟庵里住。这个道童,也是别处掳掠来的。这岭唤做蜈蚣岭。这先生见
这条岭好风水,以此他便自号飞天蜈蚣王道人。”武行者道:“你还有亲眷么?”
那妇人道:“亲戚自有几家,都是庄农之人,谁敢和他争论?”武行者道:“这厮
有些财帛么?”妇人道:“他也积蓄得一二百两金银。”武行者道:“有时,你快
去收拾。我便要放火烧庵也。”那妇人问道:“师父,你要酒肉吃么?”武行者道:
“有时,将来请我。”那妇人道:“请师父进庵里去吃。”武行者道:“怕别有人
暗算我么?”那妇人道:“奴有几颗头,敢赚得师父?”武行者随那妇人入到庵里,
见小窗边桌子上,摆着酒肉。武行者讨大碗,吃了一回。那妇人收拾得金银财帛已
了,武行者便就里面放起火来。那妇人捧着一包金银,献与武行者,乞性命。武行
者道:“我不要你的,你自将去养身。快走!快走!”那妇人拜谢了,自下岭去。
武行者把那两个尸首,都撺在火里烧了,插了戒刀,连夜自过岭来,迤逦取路,望
着青州地面来。

又行了十数日,但遇村坊道店,市镇乡城,果然都有榜文张挂在彼处,捕获武
松。到处虽有榜文,武松已自做了行者,于路却没人盘诘他。时遇十一月间,天色
好生严寒。当日武行者一路上买酒买肉吃,只是敌不过寒威。上得一条土冈,早望
见前面有一座高山,生得十分险峻。武行者下土冈子来,走得三五里路,早见一个
酒店。门前一道清溪,屋后都是颠石乱山。看那酒店时,却是个村落小酒肆。但见:

门迎溪涧,山映茅茨。疏篱畔梅开玉蕊,小窗前松偃苍龙。乌皮桌椅,尽列着
瓦钵磁瓯;黄土墙垣,都画着酒仙诗客。一条青旆舞寒风,两句诗词招过客。端的
是走骠骑闻香须住马,使风帆知味也停舟。
武行者过得那土冈子来,径奔入那村酒店里坐下,便叫道:“店主人家,先打两角
酒来。肉便买些来吃。”店主人应道:“实不瞒师父说:酒却有些茅柴白酒,肉却
都卖没了。”武行者道:“且把酒来挡寒。”店主人便去打两角酒,大碗价筛来,
教武行者吃,将一碟熟菜,与他过口。片时间,吃尽了两角酒,又叫再打两角酒来,
店主人又打了两角酒,大碗筛来。武行者只顾吃。比及过冈子时,先有三五分酒了;
一发吃过这四角酒,又被朔风一吹,酒却涌上。武松却大呼小叫道:“主人家,你
真个没东西卖?你便自家吃的肉食,也回些与我吃了,一发还你银子。”店主人笑
道:“也不曾见这个出家人,酒和肉只顾要吃,却那里去取?师父,你也只好罢休。”
武行者道:“我又不白吃你的,如何不卖与我?”店主人道:“我和你说过,只有
这些白酒,那得别的东西卖?”正在店里论口,只见外面走入一条大汉,引着三四
个人入店里来。武行者看那大汉时,但见:

顶上头巾鱼尾赤,身上战袍鸭头绿。脚穿一对踢土靴,腰系数尺红膊。面圆
耳大,唇阔口方。长七尺以上身材,有二十四五年纪。相貌堂堂强壮士,未侵女色
少年郎。

那条大汉引着众人入进店里,主人笑容可掬迎接道:“大郎请坐。”那汉道:
“我分付你的,安排也未?”店主人答道:“鸡与肉,都已煮熟了,只等大郎来。”
那汉道:“我那青花瓮酒在那里?”店主人道:“有在这里。”那汉引了众人,便
向武行者对席上头坐了;那同来的三四人,却坐在肩下。店主人却捧出一樽青花瓮
酒来,开了泥头,倾在一个大白盆里。武行者偷眼看时,却是一瓮窨下的好酒,被
风吹过酒的香味来。武行者闻了那酒香味,喉咙痒将起来,恨不得钻过来抢吃。只
见店主人又去厨下,把盘子托出一对熟鸡、一大盘精肉来,放在那汉面前,便摆了
菜蔬,用杓子舀酒去烫。武行者看了自己面前,只是一碟儿熟菜,不由的不气。正
是眼饱肚中饥,武行者酒又发作,恨不得一拳打碎了那桌子,大叫道:“主人家,
你来!你这厮好欺负客人!”店主人连忙来问道:“师父,休要焦躁。要酒便好说。”
武行者睁着双眼喝道:“你这厮好不晓道理!这青花瓮酒和鸡肉之类,如何不卖与
我?我也一般还你银子。”店主人道:“青花瓮酒和鸡肉,都是那大郎家里自将来
的,只借我店里坐地吃酒。”武行者心中要吃,那里听他分说,一片声喝道:“放
屁!放屁!”店主人道:“也不曾见你这个出家人,恁地蛮法!”武行者喝道:“怎
地是老爷蛮法?我白吃你的!”那店主人道:“我倒不曾见出家人自称老爷。”武
行者听了,跳起身来,叉开五指望店主人脸上只一掌,把那店主人打个踉跄,直撞
过那边去。

那对席的大汉,见了大怒。看那店主人时,打得半边脸都肿了,半日挣扎不起。
那大汉跳起身来,指定武松道:“你这个鸟头陀,好不依本分!却怎地便动手动脚!
却不道是:‘出家人勿起嗔心’!”武行者道:“我自打他,干你甚事!”那大汉
怒道:“我好意劝你,你这鸟头陀,敢把言语伤我!”武行者听得大怒,便把桌子
推开,走出来喝道:“你那厮说谁!”那大汉笑道:“你这鸟头陀,要和我厮打,
正是来太岁头上动土!”那大汉便点手叫道:“你这贼行者,出来和你说话!”武
行者喝道:“你道我怕你,不敢打你!”一抢抢到门边,那大汉便闪出门外去。武
行者赶到门外,那大汉见武松长壮,那里敢轻敌,便做个门户等着他。武行者抢入
去,接住那汉手。那大汉却待用力跌武松,怎禁得他千百斤神力,就手一扯,扯入
怀来,只一拨,拨将去,恰似放翻小孩子的一般,那里做得半分手脚。那三四个村
汉看了,手颤脚麻,那里敢上前来。武行者踏住那大汉,提起拳头来,只打实落处。
打了二三十拳,就地下提起来,望门外溪里只一丢。那三四个村汉叫声苦,不知高
低,都下溪里来救起那大汉,自搀扶着投南去了。这店主人吃了这一掌,打得麻了,
动弹不得,自入屋后去躲避了。

武行者道:“好呀!你们都去了,老爷却吃酒肉!”把个碗去白盆内舀那酒来,
只顾吃。桌子上那对鸡,一盘子肉,都未曾吃动。武行者且不用箸,双手扯来任意
吃。没半个时辰,把这酒肉和鸡都吃个八分。武行者醉饱了,把直裰袖结在背上,
便出店门,沿溪而走。却被那北风卷将起来,武行者捉脚不住,一路上抢将来。离
那酒店,走不得四五里路,旁边土墙里,走出一只黄狗,看着武松叫。武行者看时,
一只大黄狗赶着吠。武行者大醉,正要寻事,恨那只狗赶着他只管吠,便将左手鞘
里掣出一口戒刀来,大踏步赶。那只黄狗绕着溪岸叫。武行者一刀斫将去,却斫个
空,使得力猛,头重脚轻,翻筋斗倒撞下溪里去,却起不来。冬月天道,溪水正涸,
虽是只有一二尺深浅的水,却寒冷的当不得。爬起来,淋淋的一身水,却见那口戒
刀,浸在溪里。武行者便低头去捞那刀时,扑地又落下去了,只在那溪水里滚。

岸上侧首墙边,转出一伙人来,当先一个大汉,头戴毡笠子,身穿鹅黄丝衲
袄,手里拿着一条哨棒,背后十数个人跟着,都拿木杷白棍。数内一个指道:“这
溪里的贼行者,便是打了小哥哥的。如今小哥哥寻不见大哥哥,自引了二三十个庄
客,径奔酒店里捉他去了。他却来到这里。”说犹未了,只见远远地那个吃打的汉
子,换了一身衣服,手里提着一条朴刀,背后引着三二十个庄客,都是有名的汉子。
怎见的,正是叫做:

长王三,矮李四。急三千,慢八百。笆上粪,屎里蛆。米中虫,饭内屁。鸟上
刺,沙小生。木伴哥,牛筋等。

这一二十个尽是为头的庄客,余者皆是村中捣子,都拖枪拽棒,跟着那个大汉,
吹风胡哨来寻武松。赶到墙边,见了,指着武松,对那穿鹅黄袄子的大汉道:“这
个贼头陀,正是打兄弟的。”那个大汉道:“且捉这厮,去庄里细细拷打。”那汉
喝声:“下手!”三四十人一发上。可怜武松醉了,挣扎不得,急要爬起来,被众
人一齐下手,横拖倒拽,捉上溪来。转过侧首墙边一所大庄院,两下都是高墙粉壁,
垂柳乔松,围绕着墙院。众人把武松推抢入去,剥了衣裳,夺了戒刀、包裹,揪过
来绑在大柳树上,教取一束藤条来,细细的打那厮。

却才打得三五下,只见庄里走出一个人来问道:“你兄弟两个,又打甚么人?”
只见这两个大汉叉手道:“师父听禀:兄弟今日和邻庄三四个相识,去前面小路店
里吃三杯酒,叵耐这个贼行者倒来寻闹,把兄弟痛打了一顿,又将来撺在水里,头
脸都磕破了,险些冻死,却得相识救了回来。归家换了衣服,带了人,再去寻他。
那厮把我酒肉都吃了,却大醉倒在门前溪里;因此捉拿在这里,细细的拷打。看起
这贼头陀来,也不是出家人,脸上现刺着两个金印,这贼却把头发披下来遮了,必
是个避罪在逃的囚徒。问出那厮根原,解送官司理论。”这个吃打伤的大汉道:“问
他做甚么!这秃贼打得我一身伤损,不着一两个月,将息不起。不如把这秃贼一顿
打死了,一把火烧了罢,才与我消得这口恨气。”说罢,拿起藤条,恰待又打,只
见出来的那人说道:“贤弟,且休打,待我看他一看,这人也像是一个好汉。”

此时武行者心中已自酒醒了,理会得,只把眼来闭了,由他打,只不做声。那
个人先去背上看了杖疮,便道:“作怪,这模样想是决断不多时的疤痕。”转过面
前看了,便将手把武松头发揪起来,定睛看了,叫道:“这个不是我兄弟武二郎!”
武行者方才闪开双眼,看了那人道:“你不是我哥哥!”那人喝叫:“快与我解下
来,这是我的兄弟。”那穿鹅黄袄子的并吃打的尽皆吃惊,连忙问道:“这个行者,
如何却是师父的兄弟?”那人便道:“他便是我时常和你们说的那景阳冈上打虎的
武松。我也不知他如今怎地做了行者。”那弟兄两个听了,慌忙解下武松来,便讨
几件干衣服,与他穿了,便扶入草堂里来。武松便要下拜,那个人惊喜相半,扶住
武松道:“兄弟酒还未醒,且坐一坐说话。”武松见了那人,欢喜上来,酒早醒了
五分。讨些汤水洗漱了,吃些醒酒之物,便来拜了那人,相叙旧话。

那人不是别人,正是郓城县人氏,姓宋,名江,表字公明。武行者道:“只想
哥哥在柴大官人庄上,却如何来在这里?兄弟莫不是和哥哥梦中相会么?”宋江道:
“我自从和你在柴大官人庄上分别之后,我却在那里住得半年。不知家中如何,恐
父亲烦恼,先发付兄弟宋清归去。后却收拾得家中书信说道:‘官司一事,全得朱、
雷二都头气力,已自家中无事,只要缉捕正身。因此已动了个海捕文书,各处追获。’
这事已自慢了。却有这里孔太公,屡次使人去庄上问信。后见宋清回家,说道宋江
在柴大官人庄上。因此,特地使人直来柴大官人庄上,取我在这里。此间便是白虎
山。这庄便是孔太公庄上。恰才和兄弟相打的,便是孔太公小儿子,因他性急,好
与人厮闹,到处叫他做独火星孔亮。这个穿鹅黄袄子的,便是孔太公大儿子,人都
叫他做毛头星孔明。因他两个好习枪棒,却是我点拨他些个,以此叫我做师父。我
在此间住半年了。我如今正欲要上清风寨走一遭,这两日方欲起身。我在柴大官人
庄上时,只听得人传说道兄弟在景阳冈上打了大虫,又听知你在阳谷县做了都头,
又闻斗杀了西门庆。向后不知你配到何处去。兄弟如何做了行者?”

武松答道:“小弟自从柴大官人庄上别了哥哥,去到得景阳冈上打了大虫,送
去阳谷县,知县就抬举我做了都头。后因嫂嫂不仁,与西门庆通奸,药死了我先兄
武大。被武松把两个都杀了,自首告到本县,转发东平府。后得陈府尹一力救济,
断配孟州。”至十字坡,怎生遇见张青、孙二娘;到孟州,怎地会施恩,怎地打了
蒋门神,如何杀了张都监一十五口,又逃在张青家;“母夜叉孙二娘教我做了头陀
行者的缘故;过蜈蚣岭试刀,杀了王道人;至村店吃酒,醉打了孔兄。”把自家的
事,从头备细告诉了宋江一遍。孔明、孔亮两个听了大惊,扑翻身便拜。武松慌忙
答礼道:“却才甚是冲撞,休怪,休怪!”孔明、孔亮道:“我弟兄两个‘有眼不
识泰山’,万望恕罪!”武行者道:“既然二位相觑武松时,却是与我烘焙度牒、
书信,并行李衣服,不可失落了那两口戒刀,这串数珠。”孔明道:“这个不须足
下挂心,小弟已自着人收拾去了,整顿端正拜还。”武行者拜谢了。宋江请出孔太
公,都相见了。孔太公置酒设席管待,不在话下。

当晚宋江邀武松同榻,叙说一年有余的事,宋江心内喜悦。武松次日天明起来,
都洗漱罢,出到中堂相会,吃早饭。孔明自在那里相陪。孔亮捱着痛疼,也来管待。
孔太公便叫杀羊宰猪,安排筵宴。是日,村中有几家街坊亲戚,都来相探。又有几
个门下人,亦来谒见。宋江心中大喜。当日筵宴散了,宋江问武松道:“二哥,今
欲往何处安身?”武松道:“昨夜已对哥哥说了:菜园子张青写书与我,着兄弟投
二龙山宝珠寺花和尚鲁智深那里入伙。他也随后便上山来。”宋江道:“也好。我
不瞒你说:我家近日有书来,说道清风寨知寨小李广花荣,他知道我杀了阎婆惜,
每每寄书来与我,千万教我去寨里住几时。此间又离清风寨不远,我这两日正待要
起身去;因见天气阴晴不定,未曾起程。早晚要去那里走一遭,不若和你同往如何?”
武松道:“哥哥,怕不是好情分,带携兄弟投那里去住几时!只是武松做下的罪犯
至重,遇赦不宥,因此发心,只是投二龙山落草避难。亦且我又做了头陀,难以和
哥哥同往。路上被人设疑,倘或有些决撒了,须连累了哥哥。便是哥哥与兄弟同死
同生,也须累及了花荣山寨不好。只是由兄弟投二龙山去了罢。天可怜见,异日不
死,受了招安,那时却来寻访哥哥未迟。”宋江道:“兄弟既有此心归顺朝廷,皇
天必佑。若如此行,不敢苦劝,你只相陪我住几日了去。”

自此,两个在孔太公庄上,一住过了十日之上,宋江与武松要行,孔太公父子
那里肯放?又留住了三五日,宋江坚执要行,孔太公只得安排筵席送行。管待一日
了,次日,将出新做的一套行者衣服,皂布直裰,并带来的度牒、书信、界箍、数
珠、戒刀、金银之类,交还武松;又各送银五十两,权为路费。宋江推却不受,孔
太公父子那里肯,只顾将来拴缚在包裹里。宋江整顿了衣服器械,武松依前穿了行
者的衣裳,带上铁界箍,挂了人顶骨数珠,跨了两口戒刀,收拾了包裹,拴在腰里。
宋江提了朴刀,悬口腰刀,带上毡笠子,辞别了孔太公。孔明、孔亮叫庄客背了行
李,弟兄二人直送了二十余里路,拜辞了宋江、武行者两个。宋江自把包裹背了,
说道:“不须庄客远送,我自和武兄弟去。”孔明、孔亮相别,自和庄客归家,不
在话下。

只说宋江和武松两个,在路上行着,于路说些闲话,走到晚,歇了一宵。次日
早起,打伙又行。两个吃罢饭,又走了四五十里,却来到一市镇上,地名唤做瑞龙
镇,却是个三岔路口。宋江借问那里人道:“小人们欲投二龙山、清风镇上,不知
从那条路去?”那镇上人答道:“这两处不是一条路去了:这里要投二龙山去,只
是投西落路;若要投清风镇去,须用投东落路,过了清风山便是。”宋江听了备细,
便道:“兄弟,我和你今日分手,就这里吃三杯相别。”词寄《浣溪沙》,单题别
意:

握手临期话别难,山林景物正阑珊,壮怀寂寞客囊殚。
旅次愁来魂欲断,邮亭宿处铗空弹,独怜长夜苦漫漫。
武行者道:“我送哥哥一程,方却回来。”宋江道:“不须如此。自古道:‘送君
千里,终有一别。’兄弟,你只顾自己前程万里,早早的到了彼处。入伙之后,少
戒酒性。如得朝廷招安,你便可撺掇鲁智深、杨志投降了。日后但是去边上,一刀
一枪,博得个封妻荫子,久后青史上留一个好名,也不枉了为人一世。我自百无一
能,虽有忠心,不能得进步。兄弟,你如此英雄,决定做得大事业,可以记心。听
愚兄之言,图个日后相见。”武行者听了,酒店上饮了数杯,还了酒钱。二人出得
店来,行到市镇梢头,三岔路口,武行者下了四拜。宋江洒泪,不忍分别,又分付
武松道:“兄弟,休忘了我的言语,少戒酒性。保重,保重!”武行者自投西去了。
看官牢记话头,武行者自来二龙山投鲁智深、杨志入伙了,不在话下。

且说宋江自别了武松,转身望东,投清风山路上来,于路只忆武行者。又自行
了几日,却早远远的望见清风山。看那山时,但见:

八面嵯峨,四围险峻。古怪乔松盘鹤盖,杈老树挂藤萝。瀑布飞流,寒气逼
人毛发冷;绿阴散下,清光射目梦魂惊。涧水时听,樵人斧响;峰峦特起,山鸟声
哀。麋鹿成群,穿荆棘往来跳跃;狐狸结队,寻野食前后呼号。若非佛祖修行处,
定是强人打劫场。
宋江看见前面那座高山,生得古怪,树木稠密,心中欢喜,观之不足,贪走了几程,
不曾问的宿头。看看天色晚了,宋江心内惊慌,肚里寻思道:“若是夏月天道,胡
乱在林子里歇一夜;却恨又是仲冬天气,风霜正冽,夜间寒冷,难以打熬。倘或走
出一个毒虫虎豹来时,如何抵当?却不害了性命!”只顾望东小路里撞将去。约莫
走了也是一更时分,心里越慌,看不见地下,了一条绊脚索。树林里铜铃响,走
出十四五个伏路小喽罗来,发声喊,把宋江捉翻,一条麻索缚了,夺了朴刀、包裹,
吹起火把,将宋江解上山来。宋江只得叫苦。却早押到山寨里。

宋江在火光下看时,四下里都是木栅,当中一座草厅,厅上放着三把虎皮交椅,
后面有百十间草房。小喽罗把宋江捆做粽子相似,将来绑在将军柱上,有几个在厅
上的小喽罗说道:“大王方才睡,且不要去报。等大王酒醒时,却请起来,剖这牛
子心肝,做醒酒汤,我们大家吃块新鲜肉。”宋江被绑在将军柱上,心里寻思道:
“我的造物只如此偃蹇!只为杀了一个烟花妇人,变出得如此之苦。谁想这把骨头,
却断送在这里!”只见小喽罗点起灯烛荧煌。宋江已自冻得身体麻木了,动弹不得,
只把眼来四下里张望,低了头叹气。

约有二三更天气,只见厅背后走出三五个小喽罗来叫道:“大王起来了。”便
去把厅上灯烛剔得明亮。宋江偷眼看时,只见那个出来的大王,头上绾着鹅梨角儿,
一条红绢帕裹着,身上披着一领枣红丝衲袄,便来坐在当中虎皮交椅上。看那大
王时,生得如何,但见:
赤发黄须双眼圆,臂长腰阔气冲天。
江湖称作锦毛虎,好汉原来却姓燕。

那个好汉,祖贯山东莱州人氏,姓燕,名顺,绰号锦毛虎。原是贩羊马客人出
身,因为消折了本钱,流落在绿林丛内打劫。那燕顺酒醒起来,坐在中间交椅上,
问道:“孩儿们那里拿得这个牛子?”小喽罗答道:“孩儿们正在后山伏路,只听
得树林里铜铃响。原来这个牛子,独自个背些包裹,撞了绳索,一交绊翻,因此拿
得来,献与大王做醒酒汤。”燕顺道:“正好!快去与我请得二位大王来同吃。”
小喽罗去不多时,只见厅侧两边走上两个好汉来:左边一个,五短身材,一双光眼。
怎生打扮,但见:
天青衲袄锦绣补,形貌峥嵘性粗卤。
贪财好色最强梁,放火杀人王矮虎。

这个好汉,祖贯两淮人氏,姓王,名英,为他五短身材,江湖上叫他做矮脚虎。
原是车家出身,为因半路里见财起意,就势劫了客人,事发到官,越狱走了,上清
风山,和燕顺占住此山,打家劫舍。右边这个,生的白净面皮,三牙掩口髭须,瘦
长膀阔,清秀模样,也裹着顶绛红头巾。怎地结束,但见:
衲袄销金油绿,狼腰紧系征裙。
山寨红巾好汉,江湖白面郎君。

这个好汉,祖贯浙西苏州人氏,姓郑,双名天寿,为他生得白净俊俏,人都号
他做白面郎君。原是打银为生,因他自小好习枪棒,流落在江湖上,因来清风山过,
撞着王矮虎,和他斗了五六十合,不分胜败。因此燕顺见他好手段,留在山上,坐
了第三把交椅。

当下三个头领坐下,王矮虎便道:“孩儿们,正好做醒酒汤。快动手,取下这
牛子心肝来,造三分醒酒酸辣汤来。”只见一个小喽罗掇一大铜盆水来,放在宋江
面前;又一个小喽罗卷起袖子,手中明晃晃拿着一把剜心尖刀。那个掇水的小喽罗,
便把双手泼起水来,浇那宋江心窝里。原来但凡人心,都是热血裹着,把这冷水泼
散了热血,取出心肝来时,便脆了好吃。那小喽罗把水直泼到宋江脸上,宋江叹口
气道:“可惜宋江死在这里!”

燕顺亲耳听得“宋江”两字,便喝住小喽罗道:“且不要泼水。”燕顺问道:
“他那厮说甚么‘宋江’?”小喽罗答道:“这厮口里说道:‘可惜宋江死在这里。’”
燕顺便起身来问道:“兀那汉子,你认得宋江?”宋江道:“只我便是宋江。”燕
顺走近跟前,又问道:“你是那里的宋江?”宋江答道:“我是济州郓城县做押司
的宋江。”燕顺道:“你莫不是山东及时雨宋公明,杀了阎婆惜,逃出在江湖上的
宋江么?”宋江道:“你怎得知?我正是宋三郎。”

燕顺听罢,吃了一惊,便夺过小喽罗手内尖刀,把麻索都割断了,便把自身上
披的枣红丝衲袄脱下来,裹在宋江身上,抱在中间虎皮交椅上,唤起王矮虎、郑
天寿,快下来。三人纳头便拜。宋江滚下来答礼,问道:“三位壮士,何故不杀小
人,反行重礼?此意如何?”亦拜在地。那三个好汉一齐跪下。燕顺道:“小弟只
要把尖刀剜了自己的眼睛,原来不识好人。一时间见不到处,少问个缘由,争些儿
坏了义士。若非天幸,使令仁兄自说出大名来,我等如何得知仔细!小弟在江湖上
绿林丛中,走了十数年,闻得贤兄仗义疏财,济困扶危的大名,只恨缘分浅薄,不
能拜识尊颜。今日天使相会,真乃称心满意。”宋江答道:“量宋江有何德能,教
足下如此挂心错爱。”燕顺道:“仁兄礼贤下士,结纳豪杰,名闻寰海,谁不钦敬!
梁山泊近来如此兴旺,四海皆闻。曾有人说道,尽出仁兄之赐。不知仁兄独自何来?
今却到此?”宋江把救晁盖一节,杀阎婆惜一节,却投柴进同孔太公许多时,并今
次要往清风寨寻小李广花荣,这几件事,一一备细说了。三个头领大喜,随即取套
衣服,与宋江穿了。一面叫杀羊宰马,连夜筵席,当夜直吃到五更,叫小喽罗伏侍
宋江歇了。次日辰牌起来,诉说路上许多事务,又说武松如此英雄了得。三个头领
跌脚懊恨道:“我们无缘,若得他来这里,十分是好,却恨他投那里去了。”

话休絮繁。宋江自到清风山,住了五七日,每日好酒好食管待,不在话下。

时当腊月初旬,山东人年例,腊日上坟。只见小喽罗山下报上来说道:“大路
上有一乘轿子,七八个人跟着,挑着两个盒子,去坟头化纸。”王矮虎是个好色之
徒,见报了,想此轿子,必是个妇人,点起三五十小喽罗,便要下山。宋江、燕顺
那里拦当得住。绰了枪刀,敲一棒铜锣,下山去了。宋江、燕顺、郑天寿三人,自
在寨中饮酒。

那王矮虎去了约有三两个时辰,远探小喽罗报将来,说道:“王头领直赶到半
路里,七八个军汉都走了,拿得轿子里抬着的一个妇人。只有一个银香盒,别无物
件财物。”燕顺问道:“那妇人如今抬到那里?”小喽罗道:“王头领已自抬在山
后房中去了。”燕顺大笑。宋江道:“原来王英兄弟,要贪女色,不是好汉的勾当。”
燕顺道:“这个兄弟,诸般都肯向前,只是有这些毛病。”宋江道:“二位和我同
去劝他。”

燕顺、郑天寿便引了宋江,直来到后山王矮虎房中,推开房门,只见王矮虎正
搂住那妇人求欢。见了三位入来,慌忙推开那妇人,请三位坐。宋江看那妇人时,
但见:

身穿缟素,腰系孝裙。不施脂粉,自然体态妖娆;懒染铅华,生定天姿秀丽。
云含春黛,恰如西子颦眉;雨滴秋波,浑似骊姬垂涕。

宋江看见那妇人,便问道:“娘子,你是谁家宅眷?这般时节,出来闲走,有
甚么要紧?”那妇人含羞向前,深深地道了三个万福,便答道:“侍儿是清风寨知
寨的浑家。为因母亲弃世,今得小祥,特来坟前化纸。那里敢无事出来闲走?告大
王垂救性命!”宋江听罢,吃了一惊,肚里寻思道:“我正来投奔花知寨,莫不是
花荣之妻?我如何不救?”宋江问道:“你丈夫花知寨,如何不同你出来上坟?”
那妇人道:“告大王:侍儿不是花知寨的浑家。”宋江道:“你恰才说是清风寨知
寨的恭人。”那妇人道:“大王不知:这清风寨如今有两个知寨,一文一武。武官
便是知寨花荣;文官便是侍儿的丈夫,知寨刘高。”

宋江寻思道:“他丈夫既是和花荣同僚,我不救时,明日到那里,须不好看。”
宋江便对王矮虎说道:“小人有句话说,不知你肯依么?”王英道:“哥哥有话,
但说不妨。”宋江道:“但凡好汉犯了‘溜骨髓’三个字的,好生惹人耻笑,我看
这娘子说来,是个朝廷命官的恭人。怎生看在下薄面,并江湖上‘大义’两字,放
他下山回去,教他夫妻完聚如何?”王英道:“哥哥听禀:王英自来没个押寨夫人
做伴;况兼如今世上,都是那大头巾弄得歹了,哥哥管他则甚?胡乱容小弟这些个。”
宋江便跪一跪道:“贤弟若要押寨夫人时,日后宋江拣一个停当好的,在下纳财进
礼,娶一个伏侍贤弟。只是这个娘子,是小人友人同僚正官之妻,怎地做个人情,
放了他则个。”燕顺、郑天寿一齐扶住宋江道:“哥哥且请起来,这个容易。”宋
江又谢道:“恁的时,重承不阻。”

燕顺见宋江坚意要救这妇人,因此不顾王矮虎肯与不肯,喝令轿夫抬了去。那
妇人听了这话,插烛也似拜谢宋江,一口一声叫道:“谢大王!”宋江道:“恭人
你休谢我:我不是山寨里大王,我自是郓城县客人。”那妇人拜谢了下山,两个轿
夫也得了性命,抬着那妇人下山来,飞也似走,只恨爷娘少生了两只脚。这王矮虎
又羞又闷,只不做声,被宋江拖出前厅劝道:“兄弟,你不要焦躁。宋江日后好歹
要与兄弟完娶一个,教你欢喜便了。小人并不失信。”燕顺、郑天寿都笑起来。王
矮虎一时被宋江以礼义缚了,虽不满意,敢怒而不敢言,只得陪笑。自同宋江在山
寨中吃筵席,不在话下。

且说清风寨军人,一时间被掳了恭人去,只得回来,到寨里报与刘知寨,说道:
“恭人被清风山强人掳去了。”刘高听了大怒,喝骂去的军人不了事,如何撇了恭
人,大棍打那去的军汉。众人分说道:“我们只有五七个,他那里三四十人,如何
与他敌得!”刘高喝道:“胡说!你们若不去夺得恭人回来时,我都把你们下在牢
里问罪。”那几个军人吃逼不过,没奈何,只得央浼本寨内军健七八十人,各执枪
棒,用意来夺。不想来到半路,正撞见两个轿夫,抬得恭人飞也似来了。

众军汉接见恭人问道:“怎地能够下山?”那妇人道:“那厮捉我到山寨里,
见我说道是刘知寨的夫人,唬得那厮慌忙拜我,便叫轿夫送我下山来。”众军汉道:
“恭人可怜见我们,只对相公说:我们打夺得恭人回来,权救我众人这顿打。”那
妇人道:“我自有道理说便了。”众军汉拜谢了,簇拥着轿子便行。众人见轿夫走
得快,便说道:“你两个闲常在镇上抬轿时,只是鹅行鸭步,如今却怎地这等走的
快?”那两个轿夫应道:“本是走不动,却被背后老大栗暴打将来。”众人笑道:
“你莫不见鬼,背后那得人?”轿夫方才敢回头,看了道:“哎也!是我走的慌了,
脚后跟直打着脑杓子。”众人都笑。簇着轿子,回到寨中。刘知寨见了大喜,便问
恭人道:“你得谁人救了你回来?”那妇人道:“便是那厮们掳我去,不从奸骗,
正要杀我;见我说是知寨的恭人,不敢下手,慌忙拜我,却得这许多人来抢夺得我
回来。”刘高听了这话,便叫取十瓶酒,一口猪,赏了众人,不在话下。

且说宋江自救了那妇人下山,又在山寨中住了五七日,思量要来投奔花知寨,
当时作别要下山。三个头领,苦留不住,做了送路筵席饯行,各送些金宝与宋江,
打缚在包裹里。当日宋江早起来,洗漱罢,吃了早饭,拴束了行李,作别了三位头
领下山。那三个好汉,将了酒果肴馔,直送到山下二十余里官道旁边,把酒分别。
三人不舍,叮嘱道:“哥哥去清风寨回来,是必再到山寨相会几时。”宋江背上包
裹,提了朴刀,说道:“再得相见。”唱个大喏,分手去了。若是说话的同时生,
并肩长,拦腰抱住,把臂拖回。宋公明只因要来投奔花知寨,险些儿死无葬身之地。
正是:遭逢坎坷皆天数,际会风云岂偶然。

毕竟宋江来寻花知寨,撞着甚人,且听下回分解。