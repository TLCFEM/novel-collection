\chapter{宋公明兵渡黄河~卢俊义赚城黑夜}

话说戴宗、石秀见那汉像个公人打扮,又见他慌慌张张。戴宗问道:“端的是
甚么公干?”那汉放下箸,抹抹嘴,对戴宗道:“河北田虎作乱,你也知道么?”
戴宗道:“俺们也知一二。”那汉道:“田虎那厮,侵州夺县,官兵不能抵敌。近
日打破盖州,早晚便要攻打卫州。城中百姓,日夜惊恐,城外居民,四散的逃窜。
因此本府差俺到省院,投告急公文的。”说罢,便起身,背了包裹,托着伞棒,急
急算还酒钱,出门叹口气道:“真个是官差不自由,俺们的老小,都在城中。皇天,
只愿早早发救兵便好!”拽开步,望京城赶去了。
戴宗、石秀得了这个消息,也算还酒钱,离了酒店,回到营中,见宋先锋报知此事。
宋江与吴用商议道:“我等诸将,闲居在此,甚是不宜。不若奏闻天子,我等情愿
起兵前去征进。”吴用道:“此事须得宿太尉保奏方可。”当时会集诸将商议,尽
皆欢喜。次日,宋江穿了公服,引十数骑入城,直至太尉府前下马。正值太尉在府,
令人传报。太尉知道,忙教请进。宋江到堂上再拜起居。宿太尉道:“将军何事光
降?”宋江道:“上告恩相,宋某听得河北田虎造反,占据州郡,擅改年号,侵至
盖州,早晚来打卫州。宋江等人马久闲,某等情愿部领兵马,前去征剿,尽忠报国。
望恩相保奏则个。”宿太尉听了大喜道:“将军等如此忠义,肯替国家出力,宿某
当一力保奏。”宋江谢道:“宋某等屡蒙太尉厚恩,虽铭心镂骨,不能补报。”宿
太尉又令置酒相待。至晚,宋江回营,与众头领说知。
却说宿太尉次日早朝入内,见天子在披香殿。省院官正奏:“河北田虎造反,占据
五府五十六县,改年建号,自霸称王。目今打破陵川,怀州震邻,申文告急。”天
子大惊,向百官文武问道:“卿等谁与寡人出力,剿灭此寇?”只见班部丛中闪出
宿太尉,执简当胸,俯伏启奏道:“臣闻田虎斩木揭竿之势,今已燎原,非猛将雄
兵,难以剿灭。今有破辽得胜宋先锋,屯兵城外,乞陛下降敕,遣这枝军马前去征
剿,必成大功。”天子大喜,即令省院官奉旨出城,宣取宋江、卢俊义直到披香殿
下,朝见天子。拜舞已毕,玉音道:“朕知卿等英雄忠义,今敕卿等征讨河北,卿
等勿辞劳苦。早奏凯歌而回,朕当优擢。”宋江、卢俊义叩头奏道:“臣等蒙圣恩
委任,敢不鞠躬尽瘁,死而后已!”天子龙颜欣悦,降敕封宋江为平北正先锋,卢
俊义为副先锋。各赐御酒、金带、锦袍、金甲、彩缎,其余正偏将佐,各赐缎匹银
两。待奏荡平,论功升赏,加封官爵。三军头目,给赐银两,都就于内府关支。限
定日期,出师起行。宋江、卢俊义再拜谢恩,领旨辞朝,上马回营,升帐而坐。当
时会集诸将,尽教收拾鞍马衣甲,准备起身,征讨田虎。
次日,于内府关到赏赐缎匹银两,分诸将,给散三军头目。宋江与吴用计议,着
令水军头领,整顿战船先进,自汴河入黄河,至原武县界,等候大军到来,接济渡
河。传令与马军头领,整顿马匹,水陆并进,船骑同行,准备出师。
且说河北田虎这厮,是威胜州沁源县一个猎户,有膂力,熟武艺,专一交结恶少。
本处万山环列,易于哨聚。又值水旱频仍,民穷财尽,人心思乱。田虎乘机纠集亡
命,捏造妖言,煽惑愚民。初时掳掠些财物,后来侵州夺县,官兵不敢当其锋。说
话的,田虎不过一个猎户,为何就这般猖獗?看官听着:却因那时文官要钱,武将
怕死,各州县虽有官兵防御,都是老弱虚冒。或一名吃两三名的兵饷,或势要人家
闲着的伴当,出了十数两顶首,也买一名充当,落得关支些粮饷使用。到得点名操
练,却去雇人答应。上下相蒙,牢不可破。国家费尽金钱,竟无一毫实用。到那临
阵时节,却不知厮杀,横的竖的,一见前面尘起炮响,只恨爷娘少生两只脚。当时
也有几个军官,引了些兵马,前去追剿田虎,那里敢上前,只是尾其后,东奔西逐,
虚张声势,甚至杀良冒功。百姓愈加怨恨,反去从贼,以避官兵。所以被他占去了
五州五十六县。那五州:一是威胜,即今时沁州;二是汾阳,即今时汾州;三是昭
德,即今时潞安;四是晋宁,即今时平阳;五是盖州,即今时泽州。那五十六县,
都是这五州管下的属县。田虎就汾阳起造宫殿,伪设文武官僚,内相外将,独霸一
方,称为晋王。兵精将猛,山川险峻。目今分兵两路,前来侵犯。
再说宋江选日出师,相辞了省院诸官,当有宿太尉亲来送行,赵安抚遵旨,至营前
赏劳三军。宋江、卢俊义谢了宿太尉、赵枢密,兵分三队而进,令五虎八骠骑为前
部。

五虎将五员:
大刀关胜

豹子头林冲
霹雳火秦明

双鞭将呼延灼
双枪将董平
八骠骑八员:
小李广花荣

金枪手徐宁
青面兽杨志

急先锋索超
没羽箭张清

美髯公朱仝
九纹龙史进

没遮拦穆弘
令十六彪将为后队。

小彪将十六员:
镇三山黄信

病尉迟孙立
丑郡马宣赞

井木犴郝思文
百胜将韩滔

天目将彭
圣水将军单廷

神火将魏定国
摩云金翅欧鹏

火眼狻猊邓飞
锦毛虎燕顺

铁笛仙马麟
跳涧虎陈达

白花蛇杨春
锦豹子杨林

小霸王周通
宋江、卢俊义、吴用、公孙胜及其余将佐,马步头领,统领中军。当日三声号炮,
金鼓乐器齐鸣,离了陈桥驿,望东北进发。
宋江号令严明,行伍整肃,所过地方,秋毫无犯,是不必说。兵至原武县界,县官
出郊迎接,前部哨报水军头领船只,已在河滨等候渡河。宋江传令李俊等领水兵六
百,分为两哨,分哨左右。再拘聚些当地船只,装载马匹车仗。宋江等大兵次第渡
过黄河北岸,便令李俊等统领战船,前至卫州卫河齐取。
宋江兵马前部,行至卫州屯扎。当有卫州官员,置筵设席,等接宋先锋到来,请进
城中管待,诉说:“田虎贼兵浩大,不可轻敌。泽州是田虎手下伪枢密钮文忠镇守,
差部下张翔、王吉,领兵一万,来攻本州所属辉县;沈安、秦升,领兵一万,来攻
怀州属县武涉。求先锋速行解救则个!”宋江听罢,回营与吴用商议,发兵前去救
应。吴用道:“陵川乃盖州之要地,不若竟领兵去打陵川,则两县之围自解。”当
下卢俊义道:“小弟不才,愿领兵去取陵川。”宋江大喜,拨卢俊义马军一万,步
兵五百。马军头领乃是花荣、秦明、董平、索超、黄信、孙立、杨志、史进、、朱
仝、穆弘。步军头领乃是李逵、鲍旭、项充、李衮、鲁智深、武松、刘唐、杨雄、
石秀。
次日,卢俊义领兵去了。宋江在帐中,再与吴用计议进兵良策。吴用道:“贼兵久
骄,卢先锋此去,必然成功。只有一件,三晋山川险峻,须得两个头领做细作,先
去打探山川形势,方可进兵。”道犹未了,只见帐前走过燕青禀道:“军师不消费
心,山川形势,已有在此。”当下燕青取出一轴手卷,展放桌上。宋江与吴用从头
仔细观看,却是三晋山川城池关隘之图。凡何处可以屯扎,何处可以埋伏,何处可
以厮杀,细细的都写在上面。吴用惊问道:“此图何处得来?”燕青对宋江道:“前
日破辽班师,回至双林镇,所遇那个姓许双名贯忠的,他邀小弟到家,临别时,将
此图相赠。他说是几笔丑画,弟回到营中闲坐,偶取来展看,才知是三晋之图。”
宋江道:“你前日回来,正值收拾朝见,忙忙地不曾问得备细。我看此人,也是个
好汉,你平日也常对我说他的好处,他如今何所作为?”燕青道:“贯忠博学多才,
也好武艺,有肝胆,其余小伎,琴弈丹青,件件都省的。”因他不愿出仕,山居幽
僻,及相叙的言语,备细说了一遍。吴用道:“诚天下有心人也。”宋江、吴用嗟
叹称赞不已。
且说卢俊义领了兵马,先令黄信、孙立,领三千兵去陵川城东五里外埋伏,史进、
杨志领三千军去陵川城西五里外埋伏。“今夜五鼓,衔枚摘铃,悄地各去。明日我
等进兵,敌人若无准备,我兵已得城池,只看南门旗号,众头领领了军马,徐徐进
城。倘敌人有准备,放炮为号,两路一齐杀出接应”。四将领计去了。卢俊义次早
五更造饭,平明,军马直逼陵川城下。兵分三队,一带儿摆开,摇旗擂鼓搦战。
守城军慌的飞去报知守将董澄及偏将沈骥、耿恭。那董澄是钮文忠部下先锋,身长
九尺,膂力过人,使一口三十斤重泼风刀。当下听的报宋朝调遣梁山泊兵马,已到
城下扎营,要来打城。董澄急升帐,整点军马,出城迎敌。耿恭谏道:“某闻宋江
这伙英雄,不可轻敌,只宜坚守。差人去盖州求取救兵到来,内外夹攻,方能取胜。”
董澄大怒道:“叵耐那厮小觑俺这里,怎敢就来攻城!彼远来必疲,待俺出去,教
他片甲不回!”耿恭苦谏不听。董澄道:“既如此,留下一千军马与你城中守护。
你去城楼坐着,看俺杀那厮。”急披挂提刀,同沈骥领兵出城迎敌。
城门开处,放下吊桥,二三千兵马,拥过吊桥。宋军阵里,用强弓硬弩,射住阵脚。
只听得鼙鼓冬冬,陵川阵中捧出一员将来。怎生打扮:
戴一顶点金束发浑铁盔,顶上撒斗来大小红缨。披一副摆连环锁子铁甲,穿一领绣
云霞团花战袍,着一双斜皮嵌线云跟靴,系一条红钉就迭胜带。一张弓,一壶箭。
骑一匹银色卷毛马,手使一口泼风刀。
董澄立马横刀,大叫道:“水泊草寇,到此送死!”朱仝纵马喝道:“天兵到此,
早早下马受缚,免污刀斧!”两军呐喊。朱仝、董澄抢到垓心,两马相交,两器并
举。二将斗不过十余合,朱仝拨马望东便走,董澄赶来。东队里花荣挺枪接住厮杀,
斗到三十余合,不分胜败。吊桥边沈骥见董澄不能取胜,抡起出白点钢枪,拍马向
前助战。花荣见两个夹攻,拨马望东便走。董澄、沈骥紧紧赶来,花荣回马再战。
耿恭在城头上,看见董澄、沈骥赶去,恐怕有失,正欲鸣锣收兵,宋军队里,忽冲
出一彪军来,李逵、鲁智深、鲍旭、项充等十数个头领,飞也似抢过吊桥来,北兵
怎当得这样凶猛,不能拦当。耿恭急叫闭门,说时迟,那时快,鲁智深、李逵早已
抢入城来。守门军一齐向前,被智深大叫一声,一禅杖打翻了两个。李逵抡斧,劈
倒五六个。鲍旭等一拥而入,夺了城门,杀散军士。耿恭见头势不好,急滚下来,
望北要走,被步军赶上活捉了。
董澄、沈骥正斗花荣,听的吊桥边喊起,急回马赶去。花荣不去追赶,就了事环带
住钢枪,拈弓取箭,觑定董澄,望董澄后心,飕的一箭,董澄两脚蹬空,扑通的倒
撞下马来。卢俊义等招动军马,掩杀过来。沈骥被董平一枪戳死,陵川兵马,杀死
大半,其余的四散逃窜去了。众将领兵,一齐进城。黑旋风李逵兀是火剌剌的只顾
砍杀,卢俊义连叫:“兄弟,不要杀害百姓。”李逵方肯住手。
卢俊义教军士快于南门竖立认军旗号,好教两路伏兵知道,再分拨军士各门把守。
少顷,黄信、孙立、史进、杨志,两路伏兵,一齐都到。花荣献董澄首级,董平献
沈骥首级,鲍旭等活捉得耿恭并部下几个头目解来。卢先锋都教解了绑缚,扶耿恭
于客位,分宾主而坐。耿恭拜谢道:“被擒之将,反蒙厚礼相待。”俊义扶起道:
“将军不出城迎敌,良有深意,岂董澄辈可比。宋先锋招贤纳士,将军若肯归顺天
朝,宋先锋必行保奏重用。”耿恭叩领谢道:“既蒙不杀之恩,愿为麾下小卒。”
卢俊义大喜,再用好言抚慰了这几个头目,一面出榜安民,一面备办酒食,犒劳军
士,置酒管待耿恭及众将。
卢俊义问耿恭盖州城中兵将多寡,耿恭道:“盖州有钮枢密重兵镇守,阳城、沈水,
俱在盖州之西;惟高平县去此只六十里远近,城池傍着韩王山,守将张礼、赵能,
部下有二万军马。”卢先锋听罢,举杯向耿恭道:“将军满饮此杯,只今夜卢某便
要将军去干一件功劳,万勿推却。”耿恭道:“蒙先锋如此厚恩,耿恭敢不尽心!”
俊义喜道:“将军既肯去,卢某拨几个兄弟并将军部下头目,依着卢某如此如此,
即刻就烦起身。”又唤过那新降的六七个头目,各赏酒食银两,功成另行重赏。当
下酒罢,卢俊义传令李逵、鲍旭等七个步兵头领,并一百名步兵,穿换了陵川军卒
的衣甲旗号;又令史进、杨志,领五百马军,衔枚摘铃,远远地随在耿恭兵后;却
令花荣等众将,在城镇守,自己领三千兵,随后接应。
分拨已定,耿恭等领计出城,日色已晚,行至高平城南门外,已是黄昏时候。星光
之下,望城上旗帜森密,听城中更鼓严明。耿恭到城下高叫道:“我是陵川守将耿
恭,只为董、沈二将,不肯听我说话,开门轻敌,以此失陷。我急领了这百余人,
开北门从小路潜走至此,快放我进城则个!”守城军士把火照认了,急去报知张礼、
赵能。那张礼、赵能亲上城楼,军士打着数把火炬,前后照耀。
张礼向下对耿恭道:“虽是自家人马,也要看个明白。”望下仔细辨认,真个是陵
川耿恭,领着百余军卒,号衣旗帜,无半点差错。城上军人多有认得头目的,便指
道:“这个是孙如虎。”又道:“这个是李擒龙。”张礼笑道:“放他进来!”只
见城门开处,放下吊桥,又令三四十个军士,把住吊桥两边,方才放耿恭进城。后
面这那军人,一拥抢进道:“快进去!快进去!后面追赶来了。”也不顾甚么耿将军。
把门军士喝道:“这是甚么去处?这般乱窜!”正在那里争让,只见韩王山嘴边火
起,飞出一彪军马来,二将当先,大喊:“贼将休走!”那耿恭的军卒内,已浑着
李逵、鲍旭、项充、李衮、刘唐、杨雄、石秀这七个大虫在内。当时各掣出兵器,
发声喊,百余人一齐发作,抢进城来。城中措手不及,那里关得城门迭。城门内外
军士,早被他们砍翻数十个,夺了城门。
张礼叫苦不迭,急挺枪下城来寻耿恭,正撞着石秀。斗了三五合,张礼无心恋战,
拖枪便走,被李逵赶上,察的一斧,剁为两段。再说韩王山嘴边那彪军,飞到城
边,一拥而入,正是史进、杨志,分投赶杀北兵。赵能被乱兵所杀。高平军士,杀
死大半。把张礼老小,尽行诛戮。城中百姓,在睡梦里惊醒,号哭振天。须臾,卢
先锋领兵也到了,下令守把各门,教十数个军士分头高叫,不得杀害百姓。天明,
出榜安民,赏赐军士,差人飞报宋先锋知道。
为何卢俊义攻破两座城池,恁般容易?恁般神速?却因田虎部下纵横,久无敌手,轻
视官军,却不知宋江等众将如此英雄。卢俊义得了这个窍,出其不意,连破二城,
所以吴用说:“卢先锋此去一定成功。”
话休絮烦。且说宋江军马屯扎卫州城外。宋先锋正在帐中议事,忽报卢先锋差人飞
报捷音,并乞宋先锋再议进兵之策。宋江大喜,对吴用道:“卢先锋一日连克二城,
贼已丧胆。”正说间,又有两路哨军报道:“辉县、武涉两处围城兵马,闻陵川失
守,都解围去了。”宋江对吴用道:“军师神算,古今罕有!”欲拔寨西行,与卢
先锋合兵一处,计议进兵。吴用道:“卫州左孟门,右太行,南滨大河,西压上党,
地当冲要。倘贼人知大兵西去,从昭德提兵南下,我兵东西不能相顾,将如之何?”
宋江道:“军师之言最当!”便令关胜、呼延灼、公孙胜,领五千军马,镇守卫州,
再令水军头领李俊、二张、三阮、二童,统领水军船只,泊聚卫河,与城内相为犄
角。分拨已定,诸将领命去了。
宋江众将,统领大兵,即日拔寨起行。于路无话。来到高平,卢俊义等出城迎接。
宋江道:“兄弟们连克二城,功劳不小,功绩簿上,都一一纪录。”卢俊义领新降
将耿恭参见。宋江道:“将军弃邪归正,与宋某等同替国家出力,朝廷自当重用。”
耿恭拜谢侍立。宋江以人马众多,不便入城,就于城外扎寨。即日与吴用、卢俊义
商议,如今当去打那个州郡。吴用道:“盖州山高涧深,道路险阻,今已克了两个
属县,其势已孤。当先取盖州,以分敌势,然后分兵两路夹剿,威胜可破也。”宋
江道:“先生之言,正合我意。”传令柴进同李应去守陵川,替回花荣等六将前来
听用,史进同穆弘守高平。柴进等四人遵令去了。当下有没羽箭张清禀道:“小将
两日感冒风寒,欲于高平暂住,调摄痊可,赴营听用。”宋江便教神医安道全,同
张清往高平疗治。
次日,花荣等已到。宋江令花荣、秦明、索超、孙立,领兵五千为先锋;董平、杨
志、朱仝、史进、穆弘、韩滔、彭,领兵一万为左翼;黄信、林冲、宣赞、郝思
文、欧鹏、邓飞,领兵一万为右翼;徐宁、燕顺、马麟、陈达、杨春、杨林、周通、
李忠为后队;宋江、卢俊义等其余将佐,统领大兵为中军。这五路雄兵,杀奔盖州
来,却似龙离大海,虎出深林。正是:人人要建封侯绩,个个思成荡寇功。
毕竟宋江兵马如何攻打盖州,且听下回分解。