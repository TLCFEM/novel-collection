\chapter{插翅虎枷打白秀英~美髯公误失小衙内}

话说宋江主张一丈青与王英配为夫妇,众人都称赞宋公明仁德,当日又设席庆
贺。正饮宴间,只见朱贵酒店里使人上山来报道:“林子前大路上一伙客人经过,
小喽罗出去拦截,数内一个称是郓城县都头雷横,朱头领邀请住了。现在店里饮分
例酒食,先使小校报知。”晁盖、宋江听了大喜,随即同军师吴用三个下山迎接。
朱贵早把船送至金沙滩上岸。宋江见了,慌忙下拜道:“久别尊颜,常切思想,今
日缘何经过贱处?”雷横连忙答礼道:“小弟蒙本县差遣,往东昌府公干回来,经
过路口,小喽罗拦讨买路钱,小弟提起贱名,因此朱兄坚意留住。”宋江道:“天
与之幸!”请到大寨,教众头领都相见了,置酒管待。一连住了五日,每日与宋江
闲话。晁盖动问朱仝消息,雷横答道:“朱仝现今参做本县当牢节级,新任知县好
生欢喜。”宋江宛曲把话来说雷横上山入伙,雷横推辞老母年高,不能相从,“待
小弟送母终年之后,却来相投。”雷横当下拜辞了下山,宋江等再三苦留不住。众
头领各以金帛相赠,宋江、晁盖自不必说。雷横得了一大包金银下山,众头领都送
至路口作别,把船渡过大路,自回郓城县去了,不在话下。

且说晁盖、宋江回至大寨聚义厅上,起请军师吴学究定议山寨职事。吴用已与
宋公明商议已定,次日会合众头领听号令。先拨外面守店头领。宋江道:“孙新、
顾大嫂原是开酒店之家,着令夫妇二人替回童威、童猛别用。”再令时迁去帮助石
勇,乐和去帮助朱贵,郑天寿去帮助李立,东南西北四座店内卖酒卖肉,招接四方
入伙好汉。每店内设两个头领。一丈青、王矮虎后山下寨,监督马匹。金沙滩小寨,
童威、童猛弟兄两个守把。鸭嘴滩小寨,邹渊、邹润叔侄两个守把。山前大路,黄
信、燕顺部领马军下寨守护。解珍、解宝守把山前第一关。杜迁、宋万守把宛子城
第二关。刘唐、穆弘守把大寨口第三关。阮家三雄守把山南水寨。孟康仍前监造战
船。李应、杜兴、蒋敬总管山寨钱粮金帛。陶宗旺、薛永监筑梁山泊内城垣雁台。
侯健专管监造衣袍、铠甲、旌旗、战袄。朱富、宋清提调筵宴。穆春、李云监造屋
宇寨栅。萧让、金大坚掌管一应宾客书信公文。裴宣专管军政司赏功罚罪。其余吕
方、郭盛、孙立、欧鹏、马麟、邓飞、杨林、白胜分调大寨八面安歇。晁盖、宋江、
吴用居于山顶寨内。花荣、秦明居于山左寨内。林冲、戴宗居于山右寨内。李俊、
李逵居于山前。张横、张顺居于山后。杨雄、石秀守护聚义厅两侧。一班头领,分
拨已定,每日轮流一位头领做筵席庆贺,山寨体统,甚是齐整。有诗为证:
巍巍高寨水中央,列职分头任所长。
只为朝廷无驾驭,遂令草泽有鹰扬。

再说雷横离了梁山泊,背了包裹,提了朴刀,取路回到郓城县,到家参见老母,
更换些衣服,赍了回文,径投县里来拜见了知县,回了话,销缴公文批帖,且自归
家暂歇。依旧每日县中书画卯酉,听候差使。因一日行到县衙东首,只听得背后有
人叫道:“都头,几时回来?”雷横回过脸来看时,却是本县一个帮闲的李小二。
雷横答道:“我却才前日来家。”李小二道:“都头出去了许多时,不知此处近日
有个东京新来打踅的行院,色艺双绝,叫做白秀英。那妮子来参都头,却值公差出
外不在,如今现在勾栏里说唱诸般品调,每日有那一般打散,或是戏舞,或是吹弹,
或是歌唱,赚得那人山人海价看。都头如何不去睃一睃?端的是好个粉头!”雷横
听了,又遇心闲,便和那李小二径到勾栏里来看,只见门首挂着许多金字帐额,旗
杆吊着等身靠背。入到里面,便去青龙头上第一位坐了。看戏台上,却做笑乐院本。
那李小二人丛里撇了雷横,自出外面赶碗头脑去了。院本下来,只见一个老儿,裹
着磕脑儿头巾,穿着一领茶褐罗衫,系一条皂绦,拿把扇子,上来开呵道:“老汉
是东京人氏,白玉乔的便是。如今年迈,只凭女儿秀英歌舞吹弹,普天下伏侍看官。”
锣声响处,那白秀英早上戏台,参拜四方,拈起锣棒,如撒豆般点动,拍下一声界
方,念了四句七言诗,便说道:“今日秀英招牌上明写着这场话本,是一段风流蕴
藉的格范,唤做豫章城双渐赶苏卿。”说了,开话又唱,唱了又说,合棚价众人喝
采不绝。雷横坐在上面看那妇人时,果然是色艺双绝。但见:

罗衣叠雪,宝髻堆云。樱桃口,杏脸桃腮;杨柳腰,兰心蕙性。歌喉宛转,声
如枝上莺啼;舞态蹁跹,影似花间凤转。腔依古调,音出天然,高低紧慢按宫商,
轻重疾徐依格范。笛吹紫竹篇篇锦,板拍红牙字字新。

那白秀英唱到务头,这白玉乔按喝道:“虽无买马博金艺,要动聪明鉴事人。
看官喝采道是去过了,我儿且回一回,下来便是衬交鼓儿的院本。”白秀英拿起盘
子,指着道:“财门上起,利地上住,吉地上过,旺地上行,手到面前,休教空过。”
白玉乔道:“我儿且走一遭,看官都待赏你。”白秀英托着盘子,先到雷横面前,
雷横便去身边袋里摸时,不想并无一文。雷横道:“今日忘了,不曾带得些出来,
明日一发赏你。”白秀英笑道:“‘头醋不酽彻底薄’,官人坐当其位,可出个标
首。”雷横通红了面皮道:“我一时不曾带得出来,非是我舍不得。”白秀英道:
“官人既是来听唱,如何不记得带钱出来?”雷横道:“我赏你三五两银子,也不
打紧,却恨今日忘记带来。”白秀英道:“官人今日见一文也无,提甚三五两银子,
正是教俺‘望梅止渴,画饼充饥’。”白玉乔叫道:“我儿,你自没眼,不看城里
人,村里人,只顾问他讨甚么?且过去自问晓事的恩官,告个标首。”雷横道:“我
怎地不是晓事的?”白玉乔道:“你若省得这子弟门庭时,狗头上生角。”众人齐
和起来。雷横大怒,便骂道:“这忤奴,怎敢辱我?”白玉乔道:“便骂你这三家
村使牛的,打甚么紧?”有认得的喝道:“使不得,这个是本县雷都头。”白玉乔
道:“只怕是驴筋头。”雷横那里忍耐得住,从坐椅上直跳下戏台来,揪住白玉乔,
一拳一脚,便打得唇绽齿落。众人见打得凶,都来解拆开了,又劝雷横自回去了。
勾栏里人,一哄尽散了。

原来这白秀英却和那新任知县旧在东京两个来往,今日特地在郓城县开勾栏。
那娼妓见父亲被雷横打了,又带重伤,叫一乘轿子,径到知县衙内,诉告雷横殴打
父亲,搅散勾栏,意在欺骗奴家。知县听了,大怒道:“快写状来。”这个唤做“枕
边灵”。便教白玉乔写了状子,验了伤痕,指定证见。本处县里有人都和雷横好的,
替他去知县处打关节,怎当那婆娘守定在衙内,撒娇撒痴,不由知县不行。立等知
县差人把雷横捉拿到官,当厅责打,取了招状,将具枷来枷了,押出去号令示众。
那婆娘要逞好手,又去知县行说了,定要把雷横号令在勾栏门首。第二日,那婆娘
再去做场,知县却教把雷横号令在勾栏门首。这一班禁子人等,都是和雷横一般的
公人,如何肯扒他?这婆娘寻思一会,既是出名奈何了他,只是一怪,走出勾栏
门,去茶坊里坐下,叫禁子过去发话道:“你们都和他有首尾,却放他自在,知县
相公教你们扒他,你倒做人情。少刻我对知县说了,看道奈何得你们也不?”禁
子道:“娘子不必发怒,我们自去扒他便了。”白秀英道:“恁地时,我自将钱
赏你。”禁子们只得来对雷横说道:“兄长,没奈何,且胡乱一。”把雷横
扒在街上。

人闹里,却好雷横的母亲正来送饭,看见儿子吃他扒在那里,便哭起来,骂
那禁子们道:“你众人也和我儿一般在衙门里出入的人,钱财直这般好使!谁保的
常没事?”禁子答道:“我那老娘听我说,我们却也要容情,怎禁被原告人监定在
这里要,我们也没做道理处。不时,便要去和知县说,苦害我们,因此上做不的
面皮。”那婆婆道:“几曾见原告人自监着被告号令的道理。”禁子们又低低道:
“老娘,他和知县来往得好,一句话便送了我们,因此两难。”那婆婆一面自去解
索,一头口里骂道:“这个贼贱人直恁的倚势!我且解了这索子,看他如今怎的!”
白秀英却在茶坊里听得,走将过来,便道:“你那老婢子,却才道甚么?”那婆婆
那里有好气,便指着骂道:“你这贱母狗,做甚么倒骂我!”白秀英听得,柳眉倒
竖,星眼圆睁,大骂道:“老咬虫,吃贫婆,贱人,怎敢骂我?”婆婆道:“我骂
你待怎的?你须不是郓城县知县!”白秀英大怒,抢向前只一掌,把那婆婆打个踉
跄。那婆婆却待挣扎,白秀英再赶入去,老大耳光子,只顾打。这雷横是个大孝的
人,见了母亲吃打,一时怒从心发,扯起枷来,望着白秀英脑盖上打将下来。那一
枷梢打个正着,劈开了脑盖,扑地倒了。众人看时,那白秀英打得脑浆迸流,眼珠
突出,动弹不得,情知死了。

众人见打死了白秀英,就押带了雷横,一发来县里首告,见知县备诉前事。知
县随即差人押雷横下来,会集相官,拘唤里正、邻佑人等,对尸检验已了,都押回
县来。雷横一面都招承了,并无难意。他娘自保领回家听候。把雷横枷了,下在牢
里。当牢节级却是美髯公朱仝,见发下雷横来,也没做奈何处,只得安排些酒食管
待,教小牢子打扫一间净房,安顿了雷横。少间,他娘来牢里送饭,哭着哀告朱仝
道:“老身年纪六旬之上,眼睁睁地只看着这个孩儿,望烦节级哥哥看日常间弟兄
面上,可怜见我这个孩儿,看觑看觑。”朱仝道:“老娘自请放心归去,今后饭食
不必来送,小人自管待他。倘有方便处,可以救之。”雷横娘道:“哥哥救得孩儿,
却是重生父母。若孩儿有些好歹,老身性命也便休了。”朱仝道:“小人专记在心,
老娘不必挂念。”那婆婆拜谢去了。朱仝寻思了一日,没做道理救他处。朱仝自央
人去知县处打关节,上下替他使用人情。那知县虽然爱朱仝,只是恨这雷横打死了
他表子白秀英,也容不得他说了。又怎奈白玉乔那厮催并,叠成文案,要知县断教
雷横偿命。因在牢里六十日,限满断结,解上济州,主案押司抱了文卷先行,却教
朱仝解送雷横。

朱仝引了十数个小牢子,监押雷横,离了郓城县,约行了十数里地,见个酒店,
朱仝道:“我等众人就此吃两碗酒去。”众人都到店里吃酒,朱仝独自带过雷横,
只做水火,来后面僻净处开了枷,放了雷横,分付道:“贤弟自回,快去家里取了
老母,星夜去别处逃难,这里我自替你吃官司。”雷横道:“小弟走了自不妨,必
须要连累了哥哥。”朱仝道:“兄弟,你不知,知县怪你打死了他表子,把这文案
却做死了,解到州里,必是要你偿命。我放了你,我须不该死罪。况兼我又无父母
挂念,家私尽可赔偿。你顾前程万里自去。”雷横拜谢了,便从后门小路奔回家里,
收拾了细软包裹,引了老母,星夜自投梁山泊入伙去了,不在话下。

却说朱仝拿着空枷撺在草里,却出来对众小牢子说道:“吃雷横走了,却是怎
地好?”众人道:“我们快赶去他家里捉。”朱仝故意延迟了半晌,料着雷横去得
远了,却引众人来县里出首。朱仝告道:“小人自不小心,路上被雷横走了,在逃
无获,情愿甘罪无辞。”知县本爱朱仝,有心将就出脱他,被白玉乔要赴上司陈告
朱仝故意脱放雷横,知县只得把朱仝所犯情由申将济州去。朱仝家中,自着人去上
州里使钱透了,却解朱仝到济州来,当厅审录明白,断了二十脊杖,刺配沧州牢城。
朱仝只得带上行枷,两个防送公人领了文案,押送朱仝上路。家间自有人送衣服盘
缠,先赍发了两个公人。当下离了郓城县,迤望沧州横海郡来,于路无话。到得
沧州,入进城中,投州衙里来,正值知府升厅,两个公人押朱仝在厅阶下,呈上公
文。知府看了,见朱仝一表非俗,貌如重枣,美髯过腹,知府先有八分欢喜,便教
这个犯人休发下牢城营里,只留在本府听候使唤。当下除了行枷,便与了回文,两
个公人相辞了自回。

只说朱仝自在府中,每日只在厅前伺候呼唤。那沧州府里押番、虞候、门子、
承局、节级、牢子,都送了些人情;又见朱仝和气,因此上都欢喜他。忽一日,本
官知府正在厅上坐堂,朱仝在阶侍立,知府唤朱仝上厅,问道:“你缘何放了雷横,
自遭配在这里?”朱仝禀道:“小人怎敢故放了雷横,只是一时间不小心,被他走
了。”知府道:“你如何得此重罪?”朱仝道:“被原告人执定,要小人如此招做
故放,以此问得重了。”知府道:“雷横如何打死了那娼妓?”朱仝却把雷横上项
的事,备细说了一遍。知府道:“你敢见他孝道,为义气上放了他?”朱仝道:“小
人怎敢欺公罔上?”

正问之间,只见屏风背后转出一个小衙内来,方年四岁,生得端严美貌,乃是
知府亲子,知府爱惜如金似玉。那小衙内见了朱仝,径走过来,便要他抱,朱仝只
得抱起小衙内在怀里。那小衙内双手扯住朱仝长髯,说道:“我只要这胡子抱。”
知府道:“孩儿快放了手,休要罗唣。”小衙内又道:“我只要这胡子抱,和我去
耍。”朱仝禀道:“小人抱衙内去府前闲走,耍一回了来。”知府道:“孩儿既是
要你抱,你和他去耍一回了来。”朱仝抱了小衙内,出府衙前来,买些细糖果子与
他吃,转了一遭,再抱入府里来。知府看见,问衙内道:“孩儿那里去来?”小衙
内道:“这胡子和我街上看耍,又买糖和果子请我吃。”知府说道:“你那里得钱
买物事与孩儿吃?”朱仝禀道:“微表小人孝顺之心,何足挂齿!”知府教取酒来
与朱仝吃。府里侍婢捧着银瓶果合筛酒,连与朱仝吃了三大赏钟。知府道:“早晚
孩儿要你耍时,你可自行去抱他耍去。”朱仝道:“恩相台旨,怎敢有违?”自此
为始,每日来和小衙内上街闲耍。朱仝囊箧又有,只要本官见喜,小衙内面上尽自
倍费。

时过半月之后,便是七月十五日盂兰盆大斋之日,年例各处点放河灯,修设好
事。当日天晚,堂里侍婢奶子叫道:“朱都头,小衙内今夜要去看河灯,夫人分付,
你可抱他去看一看。”朱仝道:“小人抱去。”那小衙内穿一领绿纱衫儿,头上角
儿拴两条珠子头须,从里面走出来。朱仝在肩头上,转出府衙内前来,望地藏寺
里去看点放河灯。那时恰才是初更时分,但见:

钟声杳霭,幡影招摇。炉中焚百和名香,盘内贮诸般素食。僧持金杵,诵真言
荐拔幽魂;人列银钱,挂孝服超升滞魄。合堂功德,画阴司八难三涂;绕寺庄严,
列地狱四生六道。杨柳枝头分净水,莲花池内放明灯。
当时朱仝肩背着小衙内,绕寺看了一遭,却来水陆堂放生池边看放河灯,那小衙内
爬在栏杆上,看了笑耍。只见背后有人拽朱仝袖子道:“哥哥借一步说话。”朱仝
回头看时,却是雷横,吃了一惊,便道:“小衙内且下来,坐在这里。我去买糖来
与你吃,切不要走动。”小衙内道:“你快来,我要去桥上看河灯。”朱仝道:“我
便来也。”转身却与雷横说话。

朱仝道:“贤弟因何到此?”雷横扯朱仝到净处拜道:“自从哥哥救了性命,
和老母无处归着,只得上梁山泊,投奔了宋公明入伙。小弟说哥哥恩德,宋公明亦
然思想哥哥旧日放他的恩念,晁天王和众头领,皆感激不浅,因此特地教吴军师同
兄弟前来相探。”朱仝道:“吴先生现在何处?”背后转过吴学究道:“吴用在此。”
言罢便拜。朱仝慌忙答礼道:“多时不见,先生一向安乐。”吴学究道:“山寨里
头领多多致意,今番教吴用和雷都头特来相请足下上山,同聚大义。到此多日了,
不敢相见,今夜伺候得着,请仁兄便挪尊步,同赴山寨,以满晁、宋二公之意。”
朱仝听罢,半晌答应不得,便道:“先生差矣!这话休题,恐被外人听了不好。雷
横兄弟,他自犯了该死的罪,我因义气放了他,出头不得,上山入伙,我亦为他配
在这里。天可怜见,一年半载,挣扎还乡,复为良民。我却如何肯做这等的事?你
二位便可请回,休在此间惹口面不好。”雷横道:“哥哥在此,无非只是在人之下,
伏侍他人,非大丈夫男子汉的勾当。不是小弟裹合上山,端的晁、宋二公仰望哥哥
久矣,休得迟延自误。”朱仝道:“兄弟,你是甚么言语?你不想我为你母老家寒
上放了你去,今日你倒来陷我为不义!”吴学究道:“既然都头不肯去时,我们自
告退,相辞了去休。”朱仝道:“说我贱名,上复众位头领。”一同到桥边。

朱仝回来,不见了小衙内,叫起苦来,两头没路去寻。雷横扯住朱仝道:“哥
哥休寻,多管是我带来的两个伴当,听得哥哥不肯去,因此倒抱了小衙内去了,我
们一同去寻。”朱仝道:“兄弟,不是耍处。这个小衙内,是知府相公的性命,分
付在我身上。”雷横道:“哥哥且跟我来。”朱仝帮住雷横、吴用三个离了地藏寺,
径出城外。朱仝心慌,便问道:“你的伴当,抱小衙内在那里?”雷横道:“哥哥
且走,到我下处,包还你小衙内。”朱仝道:“迟了时,恐知府相公见怪。”吴用
道:“我那带来的两个伴当,是个没分晓的,以定直抱到我们的下处去了。”朱仝
道:“你那伴当姓甚名谁?”雷横答道:“我也不认得,只听闻叫做黑旋风李逵。”
朱仝失惊道:“莫不是江州杀人的李逵么?”吴用道:“便是此人。”朱仝跌脚叫
苦,慌忙便赶。离城约走到二十里,只见李逵在前面叫道:“我在这里。”朱仝抢
近前来问道:“小衙内放在那里?”李逵唱个喏道:“拜揖节级哥哥,小衙内有在
这里。”朱仝道:“你好好的抱出小衙内还我。”李逵指着头上道:“小衙内头须
儿却在我头上。”朱仝看了,又问小衙内正在何处。李逵道:“被我拿些麻药,抹
在口里,直出城来,如今睡在林子里,你自请去看。”朱仝乘着月色明朗,径抢
入林子里寻时,只见小衙内倒在地上。朱仝便把手去扶时,只见头劈做两半个,已
死在那里。

当时朱仝心下大怒,奔出林子来,早不见了三个人。四下里望时,只见黑旋风
远远地拍着双斧叫道:“来,来,来!和你斗二三十合。”朱仝性起,奋不顾身,
拽扎起布衫,大踏步赶将来。李逵回身便走,背后朱仝赶来。这李逵却是穿山度岭
惯走的人,朱仝如何赶得上,先自喘做一块。李逵却在前面,又叫:“来,来,来,
和你并个你死我活。”朱仝恨不得一口气吞了他,只是赶他不上。赶来赶去,天色
渐明。李逵在前面急赶急走,慢赶慢行,不赶不走。看看赶入一个大庄院里去了。
朱仝看了道:“那厮既有下落,我和他干休不得。”

朱仝直赶入庄院内厅前去,见里面两边都插着许多军器,朱仝道:“想必也是
个官宦之家。”立住了脚,高声叫道:“庄里有人么?”只见屏风背后转出一个人
来。那人是谁?正是:

累代金枝玉叶,先朝凤子龙孙。丹书铁券护家门,万里招贤名振。

待客一
团和气,挥金满面阳春。能文会武孟尝君,小旋风聪明柴进。
出来的正是小旋风柴进,问道:“兀的是谁?”朱仝见那人人物轩昂,资质秀丽,
慌忙施礼,答道:“小人是郓城县当牢节级朱仝,犯罪刺配到此。昨晚因和知府的
小衙内出来看放河灯,被黑旋风杀了小衙内,现今走在贵庄,望烦添力捉拿送官。”
柴进道:“既是美髯公,且请坐。”朱仝道:“小人不敢拜问官人高姓?”柴进答
道:“小可姓柴名进,小旋风便是。“朱仝道:“久闻大名。”连忙下拜,又道:
“不期今日得识尊颜!”柴进说道:“美髯公,亦久闻名,且请后堂说话。”朱仝
随着柴进直到里面。朱仝道:“黑旋风那厮,如何却敢径入贵庄躲避?”柴进道:
“容复:小可平生专爱结识江湖上好汉。为是家间祖上有陈桥让位之功,先朝曾敕
赐丹书铁券,但有做下不是的人,停藏在家,无人敢搜。近间有个爱友,和足下亦
是旧交,目今在那梁山泊内做头领,名唤及时雨宋公明,写一封密书,令吴学究、
雷横、黑旋风俱在敝庄安歇,礼请足下上山,同聚大义。因见足下推阻不从,故意
教李逵杀害了小衙内,先绝了足下归路,只得上山坐把交椅。吴先生、雷兄,如何
不出来陪话?”

只见吴用、雷横从侧首阁子里出来,望着朱仝便拜,说道:“兄长,望乞恕罪,
皆是宋公明哥哥将令,分付如此。若到山寨,自有分晓。”朱仝道:“是则是你们
弟兄好情意,只是忒毒些个!”柴进一力相劝,朱仝道:“我去则去,只教我见黑
旋风面罢!”柴进道:“李大哥,你快出来陪话。”李逵也从侧首出来,唱个大喏。
朱仝见了,心头一把无明业火,高三千丈,按纳不下,起身抢近前来,要和李逵性
命相搏。柴进、雷横、吴用三个苦死劝住。朱仝道:“若要我上山时,依得我一件
事,我便去。”吴用道:“休说一件事,遮莫几十件,也都依你。愿闻那一件事。”
不争朱仝说出这件事来,有分教:大闹高唐州,惹动梁山泊。直教:昭贤国戚遭刑
法,好客皇亲丧土坑。

毕竟朱仝说出甚么事来,且听下回分解。