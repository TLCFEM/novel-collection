\chapter{九纹龙剪径赤松林~鲁智深火烧瓦罐寺}

话说鲁智深走过数个山坡,见一座大松林,一条山路。随着那山路行去,走不
得半里,抬头看时,却见一所败落寺院,被风吹得铃铎响。看那山门时,上有一面
旧朱红牌额,内有四个金字,都昏了,写着“瓦罐之寺”。又行不得四五十步,过
座石桥,再看时,一座古寺,已有年代。入得山门里,仔细看来,虽是大刹,好生
崩损。但见:

钟楼倒塌,殿宇崩摧。山门尽长苍苔,经阁都生碧藓。释迦佛芦芽穿膝,浑如
在雪岭之时;观世音荆棘缠身,却似守香山之日。诸天坏损,怀中鸟雀营巢;帝释
欹斜,口内蜘蛛结网。没头罗汉,这法身也受灾殃;折臂金刚,有神通如何施展。
香积厨中藏兔穴,龙华台上印狐踪。

鲁智深入得寺来,便投知客寮去。只见知客寮门前大门也没了,四围壁落全无。
智深寻思道:“这个大寺,如何败落的恁地?”直入方丈前看时,只见满地都是燕
子粪,门上一把锁锁着,锁上尽是蜘蛛网。智深把禅杖就地下搠着,叫道:“过往
僧人来投斋。”叫了半日,没一个答应。回到香积厨下看时,锅也没了,灶头都塌
损。智深把包裹解下,放在监斋使者面前,提了禅杖,到处寻去。寻到厨房后面一
间小屋,见几个老和尚坐地,一个个面黄肌瘦。智深喝一声道:“你们这和尚,好
没道理!由洒家叫唤,没一个应。”那和尚摇手道:“不要高声。”智深道:“俺
是过往僧人,讨顿饭吃,有甚利害。”老和尚道:“我们三日不曾有饭落肚,那里
讨饭与你吃?”智深道:“俺是五台山来的僧人,粥也胡乱请洒家吃半碗。”老和
尚道:“你是活佛去处来的僧,我们合当斋你,争奈我寺中僧众走散,并无一粒斋
粮。老僧等端的饿了三日。”智深道:“胡说,这等一个大去处,不信没斋粮。”

老和尚道:“我这里是个非细去处。只因是十方常住,被一个云游和尚,引着
一个道人,来此住持,把常住有的没的都毁坏了。他两个无所不为,把众僧赶出去
了。我几个老的走不动,只得在这里过,因此没饭吃。”智深道:“胡说!量他一
个和尚,一个道人,做得甚事,却不去官府告他?”老和尚道:“师父,你不知这
里衙门又远,便是官军,也禁不的他。这和尚、道人好生了得,都是杀人放火的人,
如今向方丈后面一个去处安身。”智深道:“这两个唤做甚么?”老和尚道:“那
和尚姓崔,法号道成,绰号生铁佛;道人姓丘,排行小乙,绰号飞天夜叉。这两个
那里似个出家人,只是绿林中强贼一般,把这出家影占身体。”智深正问间,猛闻
得一阵香来。智深提了禅杖,踅过后面打一看时,见一个土灶,盖着一个草盖,气
腾腾透将起来。智深揭起看时,煮着一锅粟米粥。智深骂道:“你这几个老和尚没
道理!只说三日没吃饭,如今现煮一锅粥,出家人何故说谎?”那几个老和尚被智
深寻出粥来,只叫得苦,把碗碟、钵头、杓子、水桶,都抢过了。智深肚饥,没奈
何,见了粥要吃,没做道理处,只见灶边破漆春台,只有些灰尘在上面。智深见了,
人急智生,便把禅杖倚了,就灶边拾把草,把春台揩抹了灰尘;双手把锅掇起来,
把粥望春台只一倾。那几个老和尚都来抢粥吃,被智深一推一交,倒的倒了,走的
走了。智深却把手来捧那粥吃。才吃几口,那老和尚道:“我等端的三日没饭吃,
却才去那里抄化得这些粟米,胡乱熬些粥吃,你又吃我们的。”智深吃五七口,听
得了这话,便撇了不吃。

只听的外面有人嘲歌。智深洗了手,提了禅杖,出来看时,破壁子里望见一个
道人,头带皂巾,身穿布衫,腰系杂色绦,脚穿麻鞋,挑着一担儿,一头是个竹篮
儿,里面露些鱼尾,并荷叶托着些肉;一头担着一瓶酒,也是荷叶盖着。口里嘲歌
着唱道:“你在东时我在西,你无男子我无妻。我无妻时犹闲可,你无夫时好孤。”
那几个老和尚赶出来,摇着手,悄悄地指与智深道:“这个道人便是飞天夜叉丘小
乙。”智深见指说了,便提着禅杖,随后跟去。那道人不知智深在后面跟来,只顾
走入方丈后墙里去。智深随即跟到里面,看时,见绿槐树下放着一条桌子,铺着些
盘馔,三个盏子,三双箸子,当中坐着一个胖和尚,生的眉如漆刷,脸似墨装,
的一身横肉,胸脯下露出黑肚皮来。边厢坐着一个年幼妇人。那道人把竹篮放下,
也来坐地。智深走到面前,那和尚吃了一惊,跳起身来,便道:“请师兄坐,同吃
一盏。”智深提着禅杖道:“你这两个如何把寺来废了?”那和尚便道:“师兄请
坐,听小僧说。”智深睁着眼道:“你说!你说!”那和尚道:“在先敝寺十分好
个去处,田庄又广,僧众极多,只被廊下那几个老和尚吃酒撒泼,将钱养女,长老
禁约他们不得,又把长老排告了出去。因此把寺来都废了,僧众尽皆走散,田土已
都卖了。小僧却和这个道人,新来住持此间,正欲要整理山门,修盖殿宇。”智深
道:“这妇人是谁,却在这里吃酒?”那和尚道:“师兄容禀:这个娘子,他是前
村王有金的女儿。在先他的父亲是本寺檀越,如今消乏了家私,近日好生狼狈,家
间人口都没了,丈夫又患病,因来敝寺借米。小僧看施主檀越面,取酒相待,别无
他意,师兄休听那几个老畜生说。”智深听了他这篇话,又见他如此小心,便道:
“叵耐几个老僧戏弄洒家。”提了禅杖,再回香积厨来。这几个老僧方才吃些粥,
正在那里。看见智深嗔忿的出来,指着老和尚道:“原来是你这几个坏了常住,犹
自在俺面前说谎。”老和尚们一齐都道:“师兄休听他说,现今养着一个妇女在那
里。他恰才见你有戒刀、禅杖,他无器械,不敢与你相争。你若不信时,再去走遭,
看他和你怎地。师兄,你自寻思:他们吃酒吃肉,我们粥也没的吃,恰才还只怕师
兄吃了。”智深道:“也说得是。”倒提了禅杖,再往方丈后来,见那角门却早关
了。

智深大怒,只一脚踢开了,抢入里面,看时,只见那生铁佛崔道成仗着一条朴
刀,从里面赶到槐树下来抢智深。智深见了,大吼一声,轮起手中禅杖,来斗崔道
成。两个斗了十四五合,那崔道成斗智深不过,只有架隔遮拦,掣仗躲闪,抵当不
住,却待要走。这丘道人见他当不住,却从背后拿了条朴刀,大踏步搠将来,智深
正斗间,忽听的背后脚步响,却又不敢回头看他。不时见一个人影来,知道有暗算
的人,叫一声:“着!”那崔道成心慌,只道着他禅杖,托地跳出圈子外去。智深
恰才回身,正好三个摘脚儿厮见。崔道成和丘道人两个又并了十合之上。智深一来
肚里无食,二来走了许多路途,三者当不的他两个生力,只得卖个破绽,拖了禅杖
便走。两个拈着朴刀,直杀出山门外来。智深又斗了十合,掣了禅杖便走。两个赶
到石桥下,坐在栏杆上,再不来赶。

智深走得远了,喘息方定,寻思道:“洒家的包裹放在监斋使者面前,只顾走
来,不曾拿得,路上又没一分盘缠,又是饥饿,如何是好?待要回去,又敌他不过。
他两个并我一个,枉送了性命。”信步望前面去,行一步,懒一步。走了几里,见
前面一个大林,都是赤松树。但见:

虬枝错落,盘数千条赤脚老龙;怪影参差,立几万道红鳞巨蟒。远观却似判官
须,近看宛如魔鬼发。谁将鲜血洒林梢,疑是朱砂铺树顶。
鲁智深看了道:“好座猛恶林子。”观看之间,只见树影里一个人探头探脑,望了
一望,吐了一口唾,闪入去了。智深道:“俺猜这个撮鸟是个剪径的强人,正在此
间等买卖。见洒家是个和尚,他道不利市,吐一口唾,走入去了。那厮却不是鸟晦
气,撞了洒家!洒家又一肚皮鸟气,正没处发落,且剥小厮衣裳当酒吃。”提了禅
杖,径抢到松林边,喝一声:“兀那林子里的撮鸟快出来!”那汉子在林子听的,
大笑道:“我晦气,他倒来惹我!”就从林子里拿着朴刀,背翻身跳出来,喝一声:
“秃驴,你是当死,不是我来寻你。”智深道:“教你认的洒家。”抡起禅杖抢那
汉。那汉拈着朴刀来斗和尚,恰待向前,肚里寻思道:“这和尚声音好熟。”便道:
“兀那和尚,你的声音好熟,你姓甚?”智深道:“俺且和你斗三百合,却说姓名。”
那汉大怒,仗手中朴刀来迎禅杖。两个斗到十数合,那汉暗暗的喝采道:“好个莽
和尚。”又斗了四五合,那汉叫道:“少歇,我有话说。”

两个都跳出圈子外来,那汉便问道:“你端的姓甚名谁?声音好熟。”智深说
姓名毕,那汉撇了朴刀,翻身便剪拂,说道:“认得史进么?”智深笑道:“原来
是史大郎。”两个再剪拂了,同到林子里坐定。智深问道:“史大郎,自渭州别后,
你一向在何处?”史进答道:“自那日酒楼前与哥哥分手,次日听得哥哥打死了郑
屠,逃走去了。有缉捕的访知史进和哥哥赍发那唱的金老,因此小弟亦便离了渭州,
寻师父王进,直到延州,又寻不着。回到北京,住了几时,盘缠使尽,以此来在这
里寻些盘缠,不想得遇哥哥。缘何做了和尚?”智深把前面过的话,从头说了一遍。
史进道:“哥哥既是肚饥,小弟有干肉烧饼在此。”便取出来教智深吃。史进又道:
“哥哥既有包裹在寺内,我和你讨去。若还不肯时,一发结果了那厮。”智深道:
“是。”当下和史进吃得饱了,各拿了器械,再回瓦罐寺来。

到寺前,看见那崔道成、丘小乙两个兀自在桥上坐地。智深大喝一声道:“你
这厮们,来,来!今番和你斗个你死我活!”那和尚笑道:“你是我手里败将,如
何再来敢厮并?”智深大怒,抡起铁禅杖,奔过桥来。那生铁佛生嗔,仗着朴刀,
杀下桥去。智深一者得了史进,肚里胆壮;二乃吃得饱了,那精神气力,越使得出
来。两个斗到八九合,崔道成渐渐力怯,只办得走路;那飞天夜叉丘道人见和尚输
了,便仗着朴刀来协助。这边史进见了,便从树林子里跳将出来,大喝一声:“都
不要走!”掀起笠儿,挺着朴刀,来战丘小乙。四个人两对厮杀。智深与崔道成正
斗到间深里,智深得便处喝一声:“着!”只一禅杖,把生铁佛打下桥去。那道人
见倒了和尚,无心恋战,卖个破绽便走。史进喝道:“那里去?”赶上望后心一朴
刀,扑地一声响,道人倒在一边。史进踏入去,掉转朴刀,望下面只顾肐肢肐察的
搠。智深赶下桥去,把崔道成背后一禅杖。可怜两个强徒,化作南柯一梦。正是:
从前作过事,无幸一齐来。

智深、史进把这丘小乙、崔道成两个尸首都缚了,撺在涧里。两个再打入寺里
来,香积厨下那几个老和尚,因见智深输了去,怕崔道成、丘小乙来杀他,已自都
吊死了。智深、史进直走入方丈后角门内看时,那个掳来的妇人投井而死。直寻到
里面八九间小屋,打将入去,并无一人。只见包裹已拿在彼,未曾打开。鲁智深见
有了包裹,依原背了。再寻到里面,只见床上三四包衣服,史进打开,都是衣裳,
包了些金银,拣好的包了一包袱,背在身上。寻到厨房,见有酒有肉,两个都吃饱
了。灶前缚了两个火把,拨开火炉,火上点着,焰腾腾的先烧着后面小屋,烧到门
前;再缚几个火把,直来佛殿下后檐,点着烧起来。凑巧风紧,刮刮杂杂地火起,
竟天价烧起来。智深与史进看着,等了一回,四下火都着了。二人道:“梁园虽好,
不是久恋之家,俺二人只好撒开。”

二人厮赶着,行了一夜。天色微明,两个远远地望见一簇人家,看来是个村镇。
两个投那村镇上来,独木桥边,一个小小酒店。但见:

柴门半掩,布幕低垂。酸醨酒瓮土床边,墨画神仙尘壁上,村童量酒,想非涤
器之相如;丑妇当垆,不是当时之卓氏。墙间大字,村中学究醉时题;架上蓑衣,
野外渔郎乘兴当。
智深、史进来到村中酒店内,一面吃酒,一面叫酒保买些肉来,借些米来,打火做
饭。两个吃酒,诉说路上许多事务。吃了酒饭,智深便问史进道:“你今投那里去?”
史进道:“我如今只得再回少华山去,投奔朱武等三人,入了伙,且过几时,却再
理会。”智深见说了道:“兄弟也是。”便打开包裹,取些金银,与了史进。二人
拴了包裹,拿了器械,还了酒钱。二人出得店门,离了村镇,又行不过五七里,到
一个三岔路口。智深道:“兄弟须要分手,洒家投东京去,你休相送。你打华州,
须从这条路去,他日却得相会。若有个便人,可通个信息来往。”史进拜辞了智深,
各自分了路,史进去了。

只说智深自往东京,在路又行了八九日,早望见东京。入得城来,但见:

千门万户,纷纷朱翠交辉;三市六街,济济衣冠聚集。凤阁列九重金玉,龙楼
显一派玻璃。花街柳陌,众多娇艳名姬;楚馆秦楼,无限风流歌妓。豪门富户呼卢
会,公子王孙买笑来。
智深看见东京热闹,市井喧哗,来到城中,陪个小心问人道:“大相国寺在何处?”
街坊人答道:“前面州桥便是。”智深提了禅杖便走,早来到寺前。入得山门看时,
端的好一座大刹!但见:

山门高耸,梵宇清幽。当头敕额字分明,两下金刚形猛烈。五间大殿,龙鳞瓦
砌碧成行;四壁僧房,龟背磨砖花嵌缝。钟楼森立,经阁巍峨。幡竿高峻接青云,
宝塔依稀侵碧汉。木鱼横挂,云板高悬。佛前灯烛荧煌,炉内香烟缭绕。幢幡不断,
观音殿接祖师堂;宝盖相连,水陆会通罗汉院。时时护法诸天降,岁岁降魔尊者来。
智深进得寺来,东西廊下看时,径投知客寮内去,道人撞见,报与知客。无移时,
知客僧出来,见了智深生得凶猛,提着铁禅杖,跨着戒刀,背着个大包裹,先有五
分惧他。知客问道:“师兄何方来?”智深放下包裹禅杖,打个问讯,知客回了问
讯。智深说道:“小徒五台山来,本师真长老有书在此,着小僧来投上刹清大师长
老处,讨个职事僧做。”知客道:“既是真大师长老有书札,合当同到方丈里去。”
知客引了智深直到方丈,解开包裹,取出书来,拿在手里。知客道:“师兄,你如
何不知体面,即目长老出来,你可解了戒刀,取出那七条坐具信香来礼拜长老使得。”
智深道:“你却何不早说!”随即解了戒刀,包裹内取出片香一炷,坐具七条,半
晌没做道理处。知客又与他披了袈裟,教他先铺坐具。少刻,只见智清禅师出来,
知客向前禀道:“这僧人从五台山来,有真禅师书在此。”清长老道:“师兄多时
不曾有法帖来。”知客叫智深道:“师兄,快来礼拜长老。”只见智深先把那炷香
插在炉内,拜了三拜,将书呈上。清长老接书拆开看时,中间备细说着鲁智深出家
缘由,并今下山投托上刹之故,“万望慈悲收录,做个职事人员,切不可推故。此
僧久后必当证果。”清长老读罢来书,便道:“远来僧人且去僧堂中暂歇,吃些斋
饭。”智深谢了,收拾起坐具七条,提了包裹,拿了禅杖、戒刀,跟着行童去了。

清长老唤集两班许多职事僧人,尽到方丈,乃言:“汝等众僧在此,你看我师
兄智真禅师好没分晓。这个来的僧人,原来是经略府军官,为因打死了人,落发为
僧。二次在彼闹了僧堂,因此难着他。你那里安他不的,却推来与我。待要不收留
他,师兄如此千万嘱付,不可推故;待要着他在这里,倘或乱了清规,如何使得?”
知客道:“便是弟子们看那僧人,全不似出家人模样,本寺如何安着得他?”都寺
便道:“弟子寻思起来,只有酸枣门外退居廨宇后那片菜园,时常被营内军健们并
门外那二十来个破落户侵害,纵放羊马,好生罗唣。一个老和尚在那里住持,那里
敢管他?何不教智深去那里住持,倒敢管的下。”清长老道:“都寺说的是。”教
侍者去僧堂内客房里等他吃罢饭,便唤将他来。

侍者去不多时,引着智深到方丈里。清长老道:“你既是我师兄真大师荐将来
我这寺中挂搭,做个职事人员,我这敝寺有个大菜园,在酸枣门外岳庙间壁,你可
去那里住持管领。每日教种地人纳十担菜蔬,余者都属你用度。”智深便道:“本
师真长老着小僧投大刹,讨个职事僧做,却不教俺做个都寺、监寺,如何教洒家去
管菜园?”首座便道:“师兄,你不省得,你新来挂搭,又不曾有功劳,如何便做
得都寺?这管菜园也是个大职事人员了。”智深道:“洒家不管菜园,俺只要做都
寺、监寺。”知客又道:“你听我说与你:僧门中职事人员,各有头项。且如小僧
做个知客,只理会管待往来客官僧众。至如维那、侍者、书记、首座,这都是清职,
不容易得做。都寺、监寺、提点、院主,这个都是掌管常住财物。你才到的方丈,
怎便得上等职事?还有那管藏的,唤做藏主;管殿的,唤做殿主;管阁的,唤做阁
主;管化缘的,唤做化主;管浴堂的,唤做浴主。这个都是主事人员,中等职事。
还有那管塔的塔头,管饭的饭头,管茶的茶头,管东厕的净头,与这管菜园的菜头。
这个都是头事人员,末等职事。假如师兄你管了一年菜园好,便升你做个塔头;又
管了一年好,升你做个浴主;又一年好,才做监寺。”智深道:“既然如此,也有
出身时,洒家明日便去。”清长老见智深肯去,就留在方丈里歇了。当日议定了职
事,随即写了榜文,先使人去菜园里退居廨宇内,挂起库司榜文,明日交割。当夜
各自散了。次早,清长老升法座,押了法帖,委智深管菜园。智深到座前,领了法
帖,辞了长老,背上包裹,跨了戒刀,提了禅杖,和两个送入院的和尚,直来酸枣
门外廨宇里来住持。诗曰:
萍踪浪迹入东京,行尽山林数十程。
古刹今番经劫火,中原从此动刀兵。
相国寺中重挂搭,种蔬园内且经营。
自古白云无去住,几多变化任纵横。

且说菜园左近有二三十个赌博不成才破落户泼皮,泛常在园内偷盗菜蔬,靠着
养身,因来偷菜,看见廨宇门上新挂一道库司榜文,上说:“大相国寺仰委管菜园
僧人鲁智深前来住持,自明日为始掌管,并不许闲杂人等入园搅扰。”那几个泼皮
看了,便去与众破落户商议道:“大相国寺里差一个和尚,甚么鲁智深,来管菜园。
我们趁他新来,寻一场闹,一顿打下头来,教那厮伏我们。”数中一个道:“我有
一个道理。他又不曾认的我,我们如何便去寻的闹?等他来时,诱他去粪窖边,只
做恭贺他,双手抢住脚,翻筋斗,攧那厮下粪窖去,只是小耍他。”众泼皮道:“好,
好!”商量已定,且看他来。

却说鲁智深来到廨宇退居内房中,安顿了包裹行李,倚了禅杖,挂了戒刀。那
数个种地道人,都来参拜了,但有一应锁钥,尽行交割。那两个和尚,同旧住持老
和尚相别了,尽回寺去。且说智深出到菜园地上,东观西望,看那园圃。只见这二
三十个泼皮,拿着些果盒、酒礼,都嘻嘻的笑道:“闻知和尚新来住持,我们邻居
街坊都来作庆。”智深不知是计,直走到粪窖边来。那伙泼皮一齐向前,一个来抢
左脚,一个便抢右脚,指望来攧智深。只教:智深脚尖起处,山前猛虎心惊;拳头
落时,海内蛟龙丧胆。正是:方圆一片闲园圃,目下排成小战场。

那伙泼皮怎的来攧智深,且听下回分解。