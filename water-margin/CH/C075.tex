\chapter{活阎罗倒船偷御酒~黑旋风扯诏骂钦差}

话说陈宗善领了诏书,回到府中,收拾起身,多有人来作贺:“太尉此行,一
为国家干事,二为百姓分忧,军民除患。梁山泊以忠义为主,只待朝廷招安,太尉
可着些甜言美语,加意抚恤。”正话间,只见太师府干人来请,说道:“太师相邀
太尉说话。”陈宗善上轿,直到新宋门大街太师府前下轿,干人直引进节堂内书院
中,见了太师,侧边坐下。茶汤已罢,蔡太师问道:“听得天子差你去梁山泊招安,
特请你来说知:到那里不要失了朝廷纲纪,乱了国家法度。你曾闻《论语》有云:
‘行己有耻,使于四方,不辱君命,可谓使矣。’”陈太尉道:“宗善尽知,承太
师指教。”蔡京又道:“我叫这个干人跟随你去。他多省得法度,怕你见不到处,
就与你提拨。”陈太尉道:“深谢恩相厚意。”辞了太师,引着干人离了相府,上
轿回家。方才歇定,门吏来报,高殿帅下马。陈太尉慌忙出来迎接,请到厅上坐定,
叙问寒温已毕,高太尉道:“今日朝廷商量招安宋江一事,若是高俅在内,必然阻
住。此贼累辱朝廷,罪恶滔天,今更赦宥罪犯,引入京城,必成后患。欲待回奏,
玉音已出,且看大意如何。若还此贼仍昧良心,怠慢圣旨,太尉早早回京,不才奏
过天子,整点大军,亲身到彼,剪草除根,是吾之愿。太慰此去,下官手下有个虞
候,能言快语,问一答十,好与太尉提拨事情。”陈太尉谢道:“感蒙殿帅忧心。”
高俅起身,陈太尉送至府前,上马去了。

次日,蔡太师府张干办、高殿帅府李虞候,二人都到了。陈太尉拴束马匹,整
点人数,将十瓶御酒,装在龙凤担内挑了,前插黄旗。陈太尉上马,亲随五六人,
张干办、李虞候都乘马匹,丹诏背在前面,引一行人出新宋门。以下官员,亦有送
路的,都回去了。迤逦来到济州。太守张叔夜接着,请到府中设筵相待,动问招安
一节,陈太尉都说了备细。张叔夜道:“论某愚意,招安一事最好;只是一件,太
尉到那里,须是陪些和气,用甜言美语,抚恤他众人,好共歹,只要成全大事。他
数内有几个性如烈火的汉子,倘或一言半语冲撞了他,便坏了大事。”张干办、李
虞候道:“放着我两个跟着太尉,定不致差迟。太守,你只管教小心和气,须坏了
朝廷纲纪,小辈人常压着,不得一半;若放他头起,便做模样。”张叔夜道:“这
两个是甚么人?”陈太尉道:“这一个人是蔡太师府内干办,这一个是高太尉府里
虞候。”张叔夜道:“只好教这两位干办不去罢!”陈太尉道:“他是蔡府、高府
心腹人,不带他去,必然疑心。”张叔夜道:“下官这话,只是要好,恐怕劳而无
功。”张干办道:“放着我两个,万丈水无涓滴漏。”张叔夜再不敢言语。一面安
排筵宴管待,送至馆驿内安歇。次日,济州先使人去梁山泊报知。

却说宋江每日在忠义堂上聚众相会,商议军情,早有细作人报知此事,未见真
实,心中甚喜。当日小喽罗领着济州报信的直到忠义堂上,说道:“朝廷今差一个
太尉陈宗善,赍到十瓶御酒,赦罪招安丹诏一道,已到济州城内,这里整备迎接。”
宋江大喜,遂取酒食,并彩缎二匹、花银十两,打发报信人先回。宋江与众人道:
“我们受了招安,得为国家臣子,不枉吃了许多时磨难,今日方成正果!”吴用笑
道:“论吴某的意,这番必然招安不成;纵使招安,也看得俺们如草芥。等这厮引
将大军来到,教他着些毒手,杀得他人亡马倒,梦里也怕,那时方受招安,才有些
气度。”宋江道:“你们若如此说时,须坏了‘忠义’二字。”林冲道:“朝廷中
贵官来时,有多少装么,中间未必是好事。”关胜便道:“诏书上必然写着些吓
的言语,来惊我们。”徐宁又道:“来的人必然是高太尉门下。”宋江道:“你们
都休要疑心,且只顾安排接诏。”先令宋清、曹正准备筵席,委柴进都管提调,务
要十分齐整。铺设下太尉幕次,列五色绢缎,堂上堂下,搭彩悬花。先使裴宣、萧
让、吕方、郭盛预前下山,离二十里伏道迎接。水军头领准备大船傍岸。吴用传令:
“你们尽依我行,不如此,行不得。”

且说萧让引着三个随行,带引五六人,并无寸铁,将着酒果,在二十里外迎接。
陈太尉当日在途中,张干办、李虞候不乘马匹,在马前步行,背后从人,何止二三
百,济州的军官约有十数骑,前面摆列导引人马。龙凤担内挑着御酒,骑马的背着
诏匣。济州牢子,前后也有五六十人,都要去梁山泊内,指望觅个小富贵。萧让、
裴宣、吕方、郭盛在半路上接着,都俯伏道旁迎接。那张干办便问道:“你那宋江
大似谁?皇帝诏敕到来,如何不亲自来接?甚是欺君!你这伙本是该死的人,怎受得
朝廷招安?请太尉回去!”萧让、裴宣、吕方、郭盛俯伏在地,请罪道:“自来朝
廷不曾有诏到寨,未见真实。宋江与大小头领都在金沙滩迎接,万望太尉暂息雷霆
之怒,只要与国家成全好事,恕免则个。”李虞候便道:“不成全好事,也不愁你
这伙贼飞上天去了。”有诗为证:
贝锦生谗自古然,小人凡事不宜先。
九天恩雨今宣布,可惜招安未十全。

当时吕方、郭盛道:“是何言语?只如此轻看人!”萧让、裴宣只得恳请他。
捧去酒果,又不肯吃。众人相随来到水边,梁山泊已摆着三只战船在彼,一只装载
马匹,一只装裴宣等一干人,一只请太尉下船,并随从一应人等,先把诏书御酒放
在船头上。那只船正是活阎罗阮小七监督。当日阮小七坐在船梢上,分拨二十余个
军健棹船,一家带一口腰刀。陈太尉初下船时,昂昂然,旁若无人,坐在中间。阮
小七招呼众人,把船棹动,两边水手齐唱起歌来。李虞候便骂道:“村驴,贵人在
此,全无忌惮!”那水手那里睬他,只顾唱歌。李虞候拿起藤条,来打两边水手,
众人并无惧色。有几个为头的回话道:“我们自唱歌,干你甚事?”李虞候道:“杀
不尽的反贼,怎敢回我话?”便把藤条去打,两边水手都跳在水里去了。阮小七在
艄上说道:“直这般打我水手下水里去了,这船如何得去?”只见上流头两只快船
下来接。原来阮小七预先积下两舱水,见后头来船相近,阮小七便去拔了禊子,叫
一声:“船漏了!”水早滚上舱里来,急叫救时,船里有一尺多水。那两只船帮将
拢来,众人急救陈太尉过船去。各人且把船只顾摇开,那里来顾御酒诏书。

两只快船先行去了。阮小七叫上水手来,舀了舱里水,把展布都拭抹了,却叫
水手道:“你且掇一瓶御酒过来,我先尝一尝滋味。”一个水手便去担中取一瓶酒
出来,解了封头,递与阮小七。阮小七接过来,闻得喷鼻馨香。阮小七道:“只怕
有毒,我且做个不着,先尝些个。”也无碗瓢,和瓶便呷,一饮而尽。阮小七吃了
一瓶道:“有些滋味。”一瓶那里济事,再取一瓶来,又一饮而尽。吃得口滑,一
连吃了四瓶。阮小七道:“怎地好?”水手道:“船梢头有一桶白酒在那里。”阮
小七道:“与我取舀水的瓢来,我都教你们到口。”将那六瓶御酒,都分与水手众
人吃了,却装上十瓶村醪水白酒,还把原封头缚了,再放在龙凤担内,飞也似摇着
船来,赶到金沙滩,却好上岸。

宋江等都在那里迎接,香花灯烛,鸣金擂鼓,并山寨里鼓乐,一齐都响。将御
酒摆在桌子上,每一桌令四个人抬;诏书也在一个桌子上抬着。陈太尉上岸,宋江
等接着,纳头便拜。宋江道:“文面小吏,罪恶迷天,曲辱贵人到此,接待不及,
望乞恕罪。”李虞候道:“太尉是朝廷大贵人大臣,来招安你们,非同小可!如何
把这等漏船,差那不晓事的村贼乘驾,险些儿误了大贵人性命!”宋江道:“我这
里有的是好船,怎敢把漏船来载贵人?”张干办道:“太尉衣襟上兀自湿了,你如
何要赖!”宋江背后五虎将紧随定,不离左右,又有八骠骑将簇拥前后,见这李虞
候、张干办在宋江前面指手划脚,你来我去,都有心要杀这厮,只是碍着宋江一个,
不敢下手。

当日宋江请太尉上轿,开读诏书,四五次才请得上轿。牵过两匹马来,与张干
办、李虞候骑。这两个男女,不知身已多大,装煞臭幺。宋江央及得上马行了,令
众人大吹大擂,迎上三关来。宋江等一百余个头领,都跟在后面,直迎至忠义堂前,
一齐下马,请太尉上堂,正面放着御酒诏匣,陈太尉、张干办、李虞候立在左边,
萧让、裴宣立在右边。宋江叫点众头领时,一百七人,于内单只不见了李逵。此时
是四月间天气,都穿夹罗战袄,跪在堂上,拱听开读。陈太尉于诏书匣内取出诏书,
度与萧让。裴宣赞礼,众将拜罢,萧让展开诏书,高声读道:

制曰:文能安邦,武能定国。五帝凭礼乐而有疆封,三皇用杀伐而定天下。事
从顺逆,人有贤愚。朕承祖宗之大业,开日月之光辉,普天率土,罔不臣伏。近为
尔宋江等啸聚山林,劫掳郡邑,本欲用彰天讨,诚恐劳我生民。今差太尉陈宗善前
来招安,诏书到日,即将应有钱粮、军器、马匹、船只,目下纳官,拆毁巢穴,率
领赴京,原免本罪。倘或仍昧良心,违戾诏制,天兵一至,龆龀不留。故兹诏示,
想宜知悉。宣和三年孟夏四月

日诏示
萧让却才读罢,宋江已下皆有怒色。只见黑旋风李逵从梁上跳将下来,就萧让手里
夺过诏书,扯的粉碎,便来揪住陈太尉,拽拳便打。此时宋江、卢俊义大横身抱住,
那里肯放他下手。恰才解拆得开,李虞候喝道:“这厮是甚么人,敢如此大胆!”
李逵正没寻人打处,劈头揪住李虞候便打,喝道:“写来的诏书,是谁说的话?”
张干办道:“这是皇帝圣旨。”李逵道:“你那皇帝,正不知我这里众好汉,来招
安老爷们,倒要做大!你的皇帝姓宋,我的哥哥也姓宋,你做得皇帝,偏我哥哥做
不得皇帝?你莫要来恼犯着黑爹爹,好歹把你那写诏的官员,尽都杀了!”众人都
来解劝,把黑旋风推下堂去。

宋江道:“太尉且宽心,休想有半星儿差池。且取御酒,教众人恩。”随即
取过一副嵌宝金花钟,令裴宣取一瓶御酒,倾在银酒海内,看时,却是村醪白酒;
再将九瓶都打开,倾在酒海内,却是一般的淡薄村醪。众人见了,尽都骇然,一个
个都走下堂去了。鲁智深提着铁禅杖,高声叫骂:“入娘撮鸟!忒煞是欺负人!把水
酒做御酒来哄俺们吃!”赤发鬼刘唐也挺着朴刀杀上来,行者武松掣出双戒刀,没
遮拦穆弘、九纹龙史进,一齐发作。六个水军头领都骂下关去了。宋江见不是话,
横身在里面拦当,急传将令,叫轿马护送太尉下山,休教伤犯。此时四下大小头领,
一大半闹将起来。宋江、卢俊义只得亲身上马,将太尉并开诏一干人数护送下三关,
再拜伏罪:“非宋江等无心归降,实是草诏的官员不知我梁山泊的弯曲。若以数句
善言抚恤,我等尽忠报国,万死无怨。太尉若回到朝廷,善言则个。”急急送过渡
口,这一干人吓得屁滚尿流,飞奔济州去了。

却说宋江回到忠义堂上,再聚众头领筵席。宋江道:“虽是朝廷诏旨不明,你
们众人也忒性躁。”吴用道:“哥哥,你休执迷!招安须自有日,如何怪得众兄弟
们发怒?朝廷忒不将人为念!如今闲话都打迭起,兄长且传将令:马军拴束马匹,步
军安排军器,水军整顿船只,早晚必有大军前来征讨。一两阵杀得他人亡马倒,片
甲不回,梦着也怕,那时却再商量。”众人道:“军师言之极当。”是日散席,各
归本帐。

且说陈太尉回到济州,把梁山泊开诏一事,诉与张叔夜。张叔夜道:“敢是你
们多说甚言语来?”陈太尉道:“我几曾敢发一言!”张叔夜道:“既是如此,枉
费了心力,坏了事情,太尉急急回京,奏知圣上,事不宜迟。”

陈太尉、张干办、李虞候一行人从,星夜回京来,见了蔡太师,备说梁山泊贼
寇扯诏毁谤一节。蔡京听了大怒道:“这伙草寇,安敢如此无礼!堂堂宋朝,如何
教你这伙横行!”陈太尉哭道:“若不是太师福荫,小官粉骨碎身在梁山泊!今日
死里逃生,再见恩相!”太师随即叫请童枢密、高、杨二太尉,都来相府,商议军
情重事。无片时,都请到太师府白虎堂内。众官坐下,蔡太师教唤过张干办、李虞
候,备说梁山泊扯诏毁谤一事。杨太尉道:“这伙贼徒如何主张招安他?当初是那
一个官奏来?”高太尉道:“那日我若在朝内,必然阻住,如何肯行此事!”童枢
密道:“鼠窃狗偷之徒,何足虑哉!区区不才,亲引一支军马,克时定日,扫清水
泊而回。”众官道:“来日奏闻。”当下都散。

次日早朝,众官三呼万岁,君臣礼毕,蔡太师出班,将此事上奏天子。天子大
怒,问道:“当日谁奏寡人,主张招安?”侍臣给事中奏道:“此日是御史大夫崔
靖所言。”天子教拿崔靖送大理寺问罪。天子又问蔡京道:“此贼为害多时,差何
人可以收剿?”蔡太师奏道:“非以重兵,不能收伏。以臣愚意,必得枢密院官亲
率大军,前去剿扫,可以刻日取胜。”天子教宣枢密使童贯问道:“卿肯领兵收捕
梁山泊草寇么?”童贯跪下奏曰:“古人有云:‘孝当竭力,忠则尽命。’臣愿效
犬马之劳,以除心腹之患。”高俅、杨戬亦皆保举。天子随即降下圣旨,赐与金印
兵符,拜东厅枢密使童贯为大元帅,任从各处选调军马,前去剿捕梁山泊贼寇,择
日出师起行。正是:登坛攘臂称元帅,败阵攒眉似小儿。

毕竟童枢密怎地出师,且听下回分解。