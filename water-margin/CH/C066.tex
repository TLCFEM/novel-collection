\chapter{时迁火烧翠云楼~吴用智取大名府}

话说吴用对宋江道:“今日幸喜得兄长无事,又得安太医在寨中看视贵疾,此
是梁山泊万千之幸。比及兄长卧病之时,小生累累使人去大名探听消息:梁中书昼
夜忧惊,只恐俺军马临城。又使人直往北京城里城外市井去处,遍贴无头告示,晓
谕居民,勿得疑虑:冤各有头,债各有主,大军到郡,自有对头。因此,梁中书越
怀鬼胎。东京蔡太师见说降了关胜,天子之前更不敢提;只是主张招安,大家无事。
因此累累寄书与梁中书,教道且留卢俊义、石秀二人性命,好做手脚。”宋江见说,
便要催趱军马下山去打北京。吴用道:“即今冬尽春初,早晚元宵节近,北京年例,
大张灯火。我欲乘此机会,先令城中埋伏,外面驱兵大进,里应外合,可以破之。”
宋江道:“此计大妙!便请军师发落。”吴用道:“为头最要紧的,是城中放火为
号。你众弟兄中,谁敢与我先去城中放火?”只见阶下走过一人道:“小弟愿往。”
众人看时,却是鼓上蚤时迁。时迁道:“小弟幼年间曾到北京。城内有座楼,唤做
翠云楼;楼上楼下,大小有百十个阁子。眼见得元宵之夜,必然喧哄。乘空潜地入
城,正月十五日夜,盘去翠云楼上放起火来为号,军师可自调人马劫牢,此为上计。”
吴用道:“我心正待如此。你明日天晓先下山去,只在元宵夜一更时候,楼上放起
火来,便是你的功劳。”时迁应允,得令去了。

吴用次日却调解珍、解宝,扮做猎户,去北京城内官员府里,献纳野味。正月
十五日夜间,只看火起为号,便去留守司前,截住报事官兵。两个听令去了。再调
杜迁、宋万,扮做粜米客人,推辆车子,去城中宿歇。元宵夜只看号火起时,却来
先夺东门。“此是你两个功劳。”两个听令去了。再调孔明、孔亮,扮做仆者,去
北京城内闹市里房檐下宿歇,只看楼前火起,便去往来接应。两个听令去了。再调
李应、史进,扮做客人,去北京东门外安歇,只看城中号火起时,先斩把门军士,
夺下东门,好做出路。两个听令去了。再调鲁智深、武松,扮做行脚僧行,去北京
城外庵院挂搭,只看城中号火起时,便去南门外截住大军,冲击去路。两个听令去
了。再调邹渊、邹润,扮做卖灯客人,直往北京城中,寻客店安歇,只看楼中火起,
便去司狱司前策应。两个听令去了。再调刘唐、杨雄,扮作公人,直去北京州衙前
宿歇,只看号火起时,便去截住一应报事人员,令他首尾不能救应。两个听令去了。
再调公孙胜先生,扮做云游道士,却教凌振扮做道童跟着,将带风火、轰天等炮数
百个,直去北京城内净处守待,只看号火起时施放。两个听令去了。再调张顺,跟
随燕青,从水门里入城,径奔卢员外家,单捉淫妇奸夫。再调王矮虎、孙新、张青、
扈三娘、顾大嫂、孙二娘,扮做三对村里夫妻,入城看灯,寻至卢俊义家中放火。
再调柴进,带同乐和,扮做军官,直去蔡节级家中,要保救二人性命。调拨已定,
众头领俱各听令去了。各各遵依军令,不可有误。

此是正月初头,不说梁山泊好汉依次各各下山进发,且说北京梁中书唤过李成、
闻达、王太守等一干官员,商议放灯一事。梁中书道:“年例北京大张灯火,庆贺
元宵,与民同乐,全似东京体例;如今被梁山泊贼人两次侵境,只恐放灯因而惹祸,
下官意欲住歇放灯,你众官心下如何计议?”闻达便道:“想此贼人,潜地退去,
没头告示乱贴,此是计穷,必无主意,相公何必多虑。若还今年不放灯时,这厮们
细作探知,必然被他耻笑。可以传下钧旨,晓示居民:比上年多设花灯,添扮社火,
市心中添搭两座鳌山,照依东京体例,通宵不禁,十三至十七,放灯五夜。教府尹
点视居民,勿令缺少,相公亲自行春,务要与民同乐。闻某亲领一彪军马出城,去
飞虎峪驻扎,以防贼人奸计。再着李都监亲引铁骑马军,绕城巡逻,勿令居民惊忧。”
梁中书见说大喜。众官商议已定,随即出榜,晓谕居民。

这北京大名府是河北头一个大郡,冲要去处。却有诸路买卖,云屯雾集;只听
放灯,都来赶趁。在城坊隅巷陌,该管厢官每日点视,只得装扮社火;豪富之家,
各自去赛花灯。远者三二百里去买,近者也过百十里之外。便有客商,年年将灯到
城货卖。家家门前扎起灯栅,都要赛挂好灯,巧样烟火;户内缚起山棚,摆放五色
屏风炮灯,四边都挂名人书画,并奇异古董玩器之物;在城大街小巷,家家都要点
灯。大名府留守司州桥边,搭起一座鳌山,上面盘红黄纸龙两条,每片鳞甲上点灯
一盏,口喷净水。去州桥河内周围上下,点灯不计其数。铜佛寺前扎起一座鳌山,
上面盘青龙一条,周回也有千百盏花灯。翠云楼前也扎起一座鳌山,上面盘着一条
白龙,四面灯火,不计其数。原来这座酒楼,名贯河北,号为第一。上有三檐滴水,
雕梁绣柱,极是造得好。楼上楼下,有百十处阁子,终朝鼓乐喧天,每日笙歌聒耳。
城中各处宫观寺院、佛殿法堂中,各设灯火,庆赏丰年。三瓦两舍,更不必说。

那梁山泊探细人得了这个消息,报上山来,吴用得知大喜,去对宋江说知备细。
宋江便要亲自领兵去打北京,安道全谏道:“将军疮口未完,切不可轻动。稍若怒
气相侵,实难痊可。”吴用道:“小生替哥哥走一遭。”随即与铁面孔目裴宣,点
拨八路军马:第一队,双鞭呼延灼引领韩滔、彭为前部,镇三山黄信在后策应,
都是马军。前者呼延灼阵上打了的是假的,故意要赚关胜,故设此计。第二队,豹
子头林冲引领马麟、邓飞为前部,小李广花荣在后策应,都是马军。第三队,大刀
关胜引领宣赞、郝思文为前部,病尉迟孙立在后策应,都是马军。第四队,霹雳火
秦明引领欧鹏、燕顺为前部,青面兽杨志在后策应,都是马军。第五队,却调步军
头领没遮拦穆弘,将引杜兴、郑天寿。第六队,步军头领黑旋风李逵,将引李立、
曹正。第七队,步军头领插翅虎雷横,将引施恩、穆春。第八队,步军头领混世魔
王樊瑞,将引项充、李衮。——“这八路马步军兵,各自取路,即今便要起行,毋
得时刻有误。正月十五日二更为期,都要到北京城下。马军步军,一齐进发。”那
八路人马依令下山,其余头领,尽跟宋江保守山寨。

且说时迁是个飞檐走壁的人,不从正路入城,夜间越墙而过,城中客店内,却
不着单身客人,他自白日在街上闲走,到晚来,东岳庙内神座底下安身。正月十三
日,却在城中往来观看居民百姓搭缚灯棚,悬挂灯火。正看之间,只见解珍、解宝
挑着野味,在城中往来观看;又撞见杜迁、宋万两个,从瓦子里走将出来。时迁当
日先去翠云楼上打一个踅,只见孔明披着头发,身穿羊皮破衣,右手拄一条杖子,
左手拿个碗,腌腌,在那里求乞。见了时迁,打抹他去背后说话。时迁道:“哥
哥,你这般一个汉子,红红白白面皮,不像叫化的,北京做公的多,倘或被他看破,
须误了大事,哥哥可以躲闪回避。”说不了,又见个丐者从墙边来,看时,却是孔
亮。时迁道:“哥哥,你又露出雪也似白面来,亦不像忍饥受饿的人。这般模样,
必然决撒。”却才道罢,背后两个人劈角儿揪住,喝道:“你们做得好事!”回头
看时,却是杨雄、刘唐。时迁道:“你惊杀我也!”杨雄道:“都跟我来。”带去
僻静处埋冤道:“你三个好没分晓,却怎地在那里说话!倒是我两个看见,倘若被
他眼明手快的公人看破,却不误了哥哥大事?我两个都已见了,弟兄们不必再上街
去。”孔明道:“邹渊、邹润自在街上卖灯,鲁智深、武松已在城外庵里。再不必
多说,只顾临期各自行事。”五个说了,都出到一个寺前,正撞见一个先生,从寺
里出来。众人抬头看时,却是入云龙公孙胜,背后凌振扮做道童跟着。七个人都点
头会意,各自去了。

看看相近上元,梁中书先令大刀闻达将引军马出城,去飞虎峪驻扎,以防贼寇。
十四日,却令李天王李成亲引铁骑马军五百,全副披挂,绕城巡视。次日,正是正
月十五日上元佳节,好生晴明。黄昏月上,六街三市,各处坊隅巷陌,点放花灯,
大街小巷,都有社火。有诗为证:
北京三五风光好,膏雨初晴春意早。
银花火树不夜城,陆地拥出蓬莱岛。
烛龙衔照夜光寒,人民歌舞欣时安。
五凤羽扶双贝阙,六鳌背驾三神山。
红妆女立朱帘下,白面郎骑紫骝马。
笙箫嘹亮入青云,月光清射鸳鸯瓦。
翠云楼高侵碧天,嬉游来往多婵娟。
灯球灿烂若锦绣,王孙公子真神仙。
游人尚未绝,高楼顷刻生云烟。
是夜节级蔡福分付,教兄弟蔡庆看守着大牢:“我自回家看看便来。”方才进得家
门,只见两个人闪将入来:前面那个军官打扮,后面仆者模样。灯光之下看时,蔡
福认得是小旋风柴进,后面的已自是铁叫子乐和。蔡节级只认得柴进,便请入里面
去,现成杯盘,随即管待。柴进道:“不必赐酒。在下到此,有件紧事相央:卢员
外、石秀全得足下相觑,称谢难尽。今晚小子就欲大牢里赶此元宵热闹,看望一遭,
望你相烦引进,休得推却。”蔡福是个公人,早猜了八分。欲待不依,诚恐打破城
池,都不见了好处,又陷了老小一家人口性命。只得担着血海的干系,便取些旧衣
裳,教他两个换了,也扮做公人,换了巾帻,带柴进、乐和径奔牢中去了。

初更左右,王矮虎、一丈青、孙新、顾大嫂、张青、孙二娘,三对儿村里夫妇,
乔乔画画,装扮做乡村人,挨在人丛里,便入东门去了。公孙胜带同凌振,挑着荆
篓,去城隍庙里廊下坐地。这城隍庙,只在州衙侧边。邹渊、邹润,挑着灯,在城
中闲走。杜迁、宋万,各推一辆车子,径到梁中书衙前,闪在人闹处。原来梁中书
衙,只在东门里大街住。刘唐、杨雄,各提着水火棍,身边都自有暗器,来州桥上
两边坐定。燕青领了张顺,自从水门里入城,静处埋伏。都不在话下。

不移时,楼上鼓打二更。却说时迁挟着一个篮儿,里面都是硫黄、焰硝放火的
药头,篮儿上插几朵闹鹅儿,踅入翠云楼后。走上楼去,只见阁子内,吹笙箫,动
鼓板,掀云闹社,子弟们闹闹穰穰,都在楼上打哄赏灯。时迁上到楼上,只做买闹
鹅儿的,各处阁子里去看。撞见解珍、解宝拖着钢叉,叉上挂着兔儿,在阁子前踅。
时迁便道:“更次到了,怎生不见外面动弹?”解珍道:“我两个方才在楼前,见
探马过去,多管兵马到了,你只顾去行事。”言犹未了,只见楼前都发起喊来,说
道:“梁山泊军马到了西门外。”解珍分付时迁:“你自快去,我自去留守司前接
应。”奔到留守司前,只见败残军马,一齐奔入城来,说道:“闻大刀吃劫了寨也!
梁山泊贼寇,引军都到城下。”李成正在城上巡逻,听见说了,飞马来到留守司前,
教点军兵,分付闭上城门,守护本州。

却说王太守亲引随从百余人,长枷铁锁,在街镇压。听得报说这话,慌忙到留
守司前。

却说梁中书正在衙前醉了闲坐,初听报说,尚自不甚慌;次后没半个更次,流
星探马,接连报来,吓得魂不附体,慌忙快叫备马。说言未了,只见翠云楼上,烈
焰冲天,火光夺月,十分浩大。梁中书见了,急上得马,却待要去看时,只见两条
大汉,推两辆车子,放在当路,便去取碗挂的灯来,望车子上点着,随即火起。梁
中书要出东门时,两条大汉口称:“李应、史进在此!”手拈朴刀,大踏步杀来。
把门官军,吓得走了,手边的伤了十数个。杜迁、宋万却好接着出来,四个合做一
处,把住东门。梁中书见不是头势,带领随行伴当,飞奔南门。南门传说道:“一
个胖大和尚,抡动铁禅杖;一个虎面行者,掣出双戒刀,发喊杀入城来。”梁中书
回马,再到留守司前,只见解珍、解宝手拈钢叉,在那里东撞西撞;急待回州衙,
不敢近前。王太守却好过来,刘唐、杨雄两条水火棍齐下,打得脑浆迸流,眼珠突
出,死于街前。虞候押番,各逃残生去了。梁中书急急回马奔西门,只听得城隍庙
里,火炮齐响,轰天震地。邹渊、邹润手拿竹竿,只顾就房檐下放起火来。南瓦子
前,王矮虎、一丈青杀将来。孙新、顾大嫂身边掣出暗器,就那里协助。铜佛寺前,
张青、孙二娘入去,爬上鳌山,放起火来。此时北京城内百姓黎民,一个个鼠撺狼
奔,一家家神号鬼哭,四下里十数处火光亘天,四方不辨。

却说梁中书奔到西门,接着李成军马,急到南门城上,勒住马,在鼓楼上看时,
只见城下兵马摆满,旗号上写道:“大将呼延灼。”火焰光中,抖擞精神,施逞骁
勇;左有韩滔,右有彭,黄信在后,催动人马,雁翅一般横杀将来,随到门下。
梁中书出不得城去,和李成躲在北门城下,望见火光明亮,军马不知其数,却是豹
子头林冲,跃马横枪,左有马麟,右有邓飞,花荣在后,催动人马,飞奔将来。再
转东门,一连火把丛中,只见没遮拦穆弘,左有杜兴,右有郑天寿,三筹步军好汉
当先,手拈朴刀,引领一千余人,杀入城来。梁中书径奔南门,舍命夺路而走。吊
桥边火把齐明,只见黑旋风李逵,左有李立,右有曹正。李逵浑身脱剥,咬定牙根,
手双斧,从城濠里飞杀过来。李立、曹正,一齐俱到。李成当先,杀开条血路,
奔出城来,护着梁中书便走。只见左手下杀声震响,火把丛中,军马无数,却是大
刀关胜,拍动赤兔马,手舞青龙刀,径抢梁中书。李成手举双刀,前来迎敌。那时
李成无心恋战,拨马便走。左有宣赞,右有郝思文,两肋里撞来。孙立在后,催动
人马,并力杀来。正斗间,背后赶上小李广花荣,拈弓搭箭,射中李成副将,翻身
落马。李成见了,飞马奔走,未及半箭之地,只见右手下锣鼓乱鸣,火光夺目,却
是霹雳火秦明跃马舞棍,引着燕顺、欧鹏,背后杨志,又杀将来。李成且战且走,
折军大半,护着梁中书,冲路走脱。

话分两头,却说城中之事。杜迁、宋万,去杀梁中书老小一门良贱。刘唐、杨
雄,去杀王太守一家老小。孔明、孔亮,已从司狱司后墙爬将入去。邹渊、邹润,
却在司狱司前接住往来之人。大牢里柴进、乐和,看见号火起了,便对蔡福、蔡庆
道:“你弟兄两个,见也不见?更待几时?”蔡庆在门边看时,邹渊、邹润早撞开
牢门,大叫道:“梁山泊好汉全伙在此!好好送出卢员外、石秀哥哥来!”蔡庆慌
忙报蔡福时,孔明、孔亮早从牢屋上跳将下来。不由他弟兄两个肯与不肯,柴进身
边取出器械,便去开枷,放了卢俊义、石秀。柴进说与蔡福:“你快跟我去家中保
护老小!”一齐都出牢门来。邹渊、邹润接着,合做一处。蔡福、蔡庆,跟随柴进,
来家中保全老小。

卢俊义将引石秀、孔明、孔亮、邹渊、邹润五个弟兄,径奔家中,来捉李固、
贾氏。却说李固听得梁山泊好汉引军马入城,又见四下里火起,正在家中有些眼跳,
便和贾氏商量,收拾了一包金珠细软背了,便出门奔走。只听得排门一带都倒,正
不知多少人抢将入来。李固和贾氏慌忙回身,便望里面开了后门,踅过墙边,径投
河下,来寻自家躲避处。只见岸上张顺大叫:“那婆娘走那里去!”李固心慌,便
跳下船中去躲。却待攒入舱里,又见一个人伸出手来,劈儿揪住,喝道:“李固,
你认得我么?”李固听得是燕青的声音,慌忙叫道:“小乙哥,我不曾和你有甚冤
仇,你休得揪我上岸!”岸上张顺早把那婆娘挟在肋下,拖到船边。燕青拿了李固,
都望东门来了。

再说卢俊义奔到家中,不见了李固和那婆娘,且叫众人把应有家私金银财宝,
都搬来装在车子上,往梁山泊给散。却说柴进和蔡福到家中收拾家资老小,同上山
寨。蔡福道:“大官人,可救一城百姓,休教残害。”柴进见说,便去寻军师吴用。
比及柴进寻着吴用,急传下号令去,教休杀害良民时,城中将及损伤一半。但见:

烟迷城市,火燎楼台。红光影里碎琉璃,黑焰丛中烧翡翠。娱人傀儡,顾不得
面是背非;照夜山棚,谁管取前明后暗。斑毛老子,猖狂燎尽白髭须;绿发儿郎,
奔走不收华盖伞。踏竹马的暗中刀枪,舞鲍老的难免刃槊。如花仕女,人丛中金坠
玉崩;玩景佳人,片时间星飞云散。可惜千年歌舞地,翻成一
片战争场。
当时天色大明,吴用、柴进在城内鸣金收军。众头领却接着卢员外并石秀,都到留
守司相见,备说牢中多亏了蔡福、蔡庆弟兄两个看觑,已逃得残生。燕青、张顺早
把这李固、贾氏解来。卢俊义见了,且教燕青监下,自行看管,听候发落,不在话
下。

再说李成保护梁中书出城逃难,又撞着闻达领着败残军马回来,合兵一处,投
南便走。正走之间,前军发起喊来,却是混世魔王樊瑞,左有项充,右有李衮,三
筹步军好汉,舞动飞刀飞枪,直杀将来。背后又是插翅虎雷横,将引施恩、穆春,
各引一千步军,前来截住退路。正是:狱囚遇赦重回禁,病客逢医又上床。

毕竟梁中书一行人马怎地计结,且听下回分解。