\chapter{公孙胜芒砀山降魔~晁天王曾头市中箭}

话说公孙胜对宋江、吴用献出那个阵图:“便是汉末三分,诸葛孔明摆石为阵
的法:四面八方,分八八六十四队,中间大将居之;其象四头八尾,左旋右转,按
天地风云之机,龙虎鸟蛇之状。待他下山冲入阵来,两军齐开,如若伺候他入阵,
只看七星号带起处,把阵变为长蛇之势。贫道作起道法,教这三人在阵中前后无路,
左右无门。却于坎地上掘一陷坎,直逼此三人到于那里。两边埋伏下挠钩手,准备
捉将。”宋江听了大喜,便传将令,叫大小将校依令而行。再用八员猛将守阵,那
八员:呼延灼、朱仝、花荣、徐宁、穆弘、孙立、史进、黄信。却叫柴进、吕方、
郭盛权摄中军;宋江、吴用、公孙胜带领陈达磨旗;叫朱武指引五个军士,在近山
高坡上看对阵报事。

是日巳牌时分,众军近山摆开阵势,摇旗擂鼓搦战。只见芒砀山上有三二十面
锣声震地价响,三个头领一齐来到山下,便将三千余人摆开;左右两边,项充、李
衮;中间马上,拥出那个为头的好汉,姓樊,名瑞,祖贯濮州人氏,幼年作全真先
生,江湖上学得一身好武艺。马上惯使一个流星锤,神出鬼没,斩将搴旗,人不敢
近,绰号混世魔王。怎见得樊瑞英雄,有《西江月》为证:

头散青丝细发,身穿绒绣皂袍,连环铁甲晃寒霄,惯使铜锤神妙。

好似北
方真武,世间伏怪除妖,云游江海把名标,混世魔王绰号。

那个混世魔王樊瑞骑一匹黑马,立于阵前。上首是项充,下首是李衮。那樊瑞
虽会使神术妖法,却不识阵势。看了宋江军马,四面八方,摆成阵势,心中暗喜道:
“你若摆阵,中我计了!”分付项充、李衮道:“若见风起,你两个便引五百滚刀
手杀入阵去。”项充、李衮得令,各执定蛮牌,挺着标枪飞剑,只等樊瑞作用。只
看樊瑞立于马上,左手挽定流星铜锤,右手伏着混世魔王宝剑,口中念念有词,喝
声道:“疾!”只见狂风四起,飞沙走石,天昏地暗,日月无光。项充、李衮呐声
喊,带了五百滚刀手,杀将过来。宋江军马见杀将过来,便分开做两下。项充、李
衮,一搅入阵,两下里强弓硬弩,射住来人,只带得四五十人入去,其余的都回本
阵去了。宋江在高坡上望见项充、李衮已入阵里了,便叫陈达把七星号旗只一招,
那座阵势,纷纷滚滚,变作长蛇之阵。项充、李衮正在阵里东赶西走,左盘右转,
寻路不见。高坡上朱武把小旗在那里指引:他两个投东,朱武便望东指;若是投西,
便望西指。原来公孙胜在高埠处看了,已先拔出那松文古定剑来,口中念动咒语,
喝声道:“疾!”将那风尽随着项充、李衮脚跟边乱卷。两个在阵中,只见天昏地
暗,日色无光,四边并不见一个军马,一望都是黑气。后面跟的都不见了。项充、
李衮心慌起来,只要夺路回阵,百般地没寻归路处。正走之间,忽然地雷大振一声,
两个在阵叫苦不迭,一齐了双脚,翻筋斗颠下陷马坑里去。两边都是挠钩手,早
把两个搭将起来,便把麻绳绑缚了,解上山坡请功。宋江把鞭梢一指,三军一齐掩
杀过去,樊瑞引人马奔走上山,走不迭的,折其大半。

宋江收军,众头领都在帐前坐下,军健早解项充、李衮到于麾下。宋江见了,
忙叫解了绳索,亲自把盏,说道:“二位壮士,其实休怪,临敌之际,不如此不得。
小可宋江,久闻三位壮士大名,欲来礼请上山,同聚大义;盖因不得其便,因此错
过。倘若不弃,同归山寨,不胜万幸。”两个听了,拜伏在地道:“已闻及时雨大
名,只是小弟等无缘,不曾拜识。原来兄长果有大义!我等两个不识好人,要与天
地相拗,今日既被擒获,万死尚轻,反以礼待。若蒙不杀,誓当效死,报答大恩!
樊瑞那人,无我两个,如何行得?义士头领若肯放我们一个回去,就说樊瑞来投拜,
不知头领尊意如何?”宋江便道:“壮士,不必留一人在此为当,便请二位同回贵
寨。宋江来日专候佳音。”两个拜谢道:“真乃大丈夫!若是樊瑞不从投降,我等
擒来,奉献头领麾下。”宋江听说大喜,请入中军,待了酒食,换了两套新衣,取
两匹好马,呼小喽罗拿了枪牌,送二人下山回寨。两个于路,在马上感恩不尽。来
到芒砀山下,小喽罗见了大惊,接上山寨。樊瑞问两个来意如何。项充、李衮道:
“我等逆天之人,合该万死!”樊瑞道:“兄弟,如何说这话?”两个便把宋江如
此义气,说了一遍。樊瑞道:“既然宋公明如此大贤,义气最重,我等不可逆天,
来早都下山投拜。”两个道:“我们也为如此而来。”当夜把寨内收拾已了,次日
天晓,三个一齐下山,直到宋江寨前,拜伏在地。宋江扶起三人,请入帐中坐定。
三个见了宋江,没半点相疑之意,彼此倾心吐胆,诉说平生之事。三人拜请众头领,
都到芒砀山寨中,杀牛宰马,管待宋公明等众多头领,一面赏劳三军。饮宴已罢,
樊瑞就拜公孙胜为师。宋江立主教公孙胜传授五雷天心正法与樊瑞,樊瑞大喜。数
日之间,牵牛拽马,卷了山寨钱粮,驮了行李,收聚人马,烧毁了寨栅,跟宋江等
班师回梁山泊,于路无话。

宋江同众好汉军马,已到梁山泊边,却欲过渡,只见芦苇岸边大路上,一个大
汉望着宋江便拜。宋江慌忙下马扶住,问道:“足下姓甚名谁?何处人氏?”那汉
答道:“小人姓段,双名景住;人见小弟赤发黄须,都呼小人为金毛犬。祖贯是涿
州人氏,平生只靠去北边地面盗马。今春去到枪竿岭北边,盗得一匹好马,雪练也
似价白,浑身并无一根杂毛,头至尾,长一丈,蹄至脊,高八尺。那马又高又大,
一日能行千里,北方有名,唤做‘照夜玉狮子马’,乃是大金王子骑坐的,放在枪
竿岭下,被小人盗得来。江湖上只闻及时雨大名,无路可见,欲将此马前来进献与
头领,权表我进身之意。不期来到凌州西南上曾头市过,被那曾家五虎夺了去。小
人称说是梁山泊宋公明的,不想那厮多有污秽的言语,小人不敢尽说。逃走得脱,
特来告知。”宋江看这人时,虽是骨瘦形粗,却甚生得奇怪。怎见得,有诗为证:
焦黄头发髭须卷,捷足不辞千里远。
但能盗马不看家,如何唤做金毛犬?
宋江见了段景住一表非俗,心中暗喜,便道:“既然如此,且同到山寨里商议。”
带了段景住,一同都下船,到金沙滩上岸。晁天王并众头领接到聚义厅上,宋江教
樊瑞、项充、李衮和众头领相见,段景住一同都参拜了;打起聒厅鼓来,且做庆贺
筵席。宋江见山寨连添了许多人马,四方豪杰,望风而来;因此叫李云、陶宗旺监
工,添造房屋,并四边寨栅。段景住又说起那匹马的好处,宋江叫神行太保戴宗,
去曾头市探听那匹马的下落。

戴宗去了四五日,回来对众头领说道:“这个曾头市上,共有三千余家,内有
一家,唤做曾家府。这老子原是大金国人,名为曾长者;生下五个孩儿,号为曾家
五虎:大的儿子,唤做曾涂,第二个唤做曾密,第三个唤做曾索,第四个唤做曾魁,
第五个唤做曾升。又有一个教师史文恭,一个副教师苏定。去那曾头市上,聚集着
五七千人马,扎下寨栅,造下五十余辆陷车,发愿说,他与我们势不两立,定要捉
尽俺山寨中头领,做个对头。那匹千里玉狮子马,现今与教师史文恭骑坐。更有一
般堪恨那厮之处,杜撰几句言语,教市上小儿们都唱道:‘摇动铁铃,神鬼尽皆
惊。铁车并铁锁,上下有尖钉。扫荡梁山清水泊,剿除晁盖上东京!生擒及时雨,
活捉智多星!曾家生五虎,天下尽闻名!’”晁盖听罢,心中大怒道:“这畜生怎
敢如此无礼!我须亲自走一遭,不捉的此辈,誓不回山!”宋江道:“哥哥是山寨
之主,不可轻动,小弟愿往。”晁盖道:“不是我要夺你的功劳,你下山多遍了,
厮杀劳困,我今替你走一遭,下次有事,却是贤弟去。”宋江苦谏不听,晁盖忿怒,
便点起五千人马,请启二十个头领相助下山;其余都和宋公明保守山寨。

晁盖点那二十个头领:林冲、呼延灼、徐宁、穆弘、刘唐、张横、阮小二、阮
小五、阮小七、杨雄、石秀、孙立、黄信、杜迁、宋万、燕顺、邓飞、欧鹏、杨林、
白胜,共是二十个头领,部领三军人马下山,征进曾头市。宋江与吴用、公孙胜众
头领,就山下金沙滩饯行。饮酒之间,忽起一阵狂风,正把晁盖新制的认军旗,半
腰吹折。众人见了,尽皆失色。吴学究谏道:“此乃不祥之兆,兄长改日出军。”
宋江劝道:“哥哥方才出军,风吹折认旗,于军不利;不若停待几时,却去和那厮
理会。”晁盖道:“天地风云,何足为怪?趁此春暖之时,不去拿他,直待养成那
厮气势,却去进兵,那时迟了。你且休阻我,遮莫怎地要去走一遭!”宋江那里别
拗得住,晁盖引兵渡水去了。宋江悒怏不已。回到山寨,再叫戴宗下山,去探听消
息。

且说晁盖领着五千人马,二十个头领,来到曾头市相近,对面下了寨栅。次日,
先引众头领,上马去看曾头市。众多好汉立马看时,果然这曾头市是个险隘去处。
但见:

周回一遭野水,四围三面高冈,堑边河港似蛇盘,濠下柳林如雨密。凭高远望,
绿阴浓不见人家;附近潜窥,青影乱深藏寨栅。村中壮汉,出来的勇似金刚;田野
小儿,生下地便如鬼子。果然是铁壁铜墙,端的尽人强马壮。

晁盖与众头领正看之间,只见柳林中飞出一彪人马来,约有七八百人,当先一
个好汉,戴熟铜盔,披连环甲,使一条点钢枪,骑着匹冲阵马,乃是曾家第四子曾
魁,高声喝道:“你等是梁山泊反国草寇,我正要来拿你解官请赏,原来天赐其便!
还不下马受缚,更待何时!”晁盖大怒,回头一观,早有一将出马,去战曾魁。那
人是梁山初结义的好汉豹子头林冲。两个交马,斗了二十余合,不分胜败。曾魁斗
到二十合之后,料道斗林冲不过,掣枪回马,便往柳林中走,林冲勒住马不赶。晁
盖领转军马回寨,商议打曾头市之策。林冲道:“来日直去市口搦战,就看虚实如
何,再作商议。”次日平明,引领五千人马,向曾头市口平川旷野之地,列成阵势,
擂鼓呐喊。曾头市上炮声响处,大队人马出来,一字儿摆着七个好汉:中间便是都
教师史文恭,上首副教师苏定,下首便是曾家长子曾涂,左边曾密、曾魁,右军曾
升、曾索,都是全身披挂。教师史文恭弯弓插箭,坐下那匹却是千里玉狮子马,手
里使一枝方天画戟。三通鼓罢,只见曾家阵里推出数辆陷车,放在阵前,曾涂指着
对阵骂道:“反国草贼,见俺陷车么?我曾家府里杀你死的,不算好汉!我一个个直
要捉你活的,装载陷车里,解上东京,碎尸万段。你们趁早纳降,再有商议。”晁
盖听了大怒,挺枪出马,直奔曾涂。众将怕晁盖有失,一发掩杀过去,两军混战。
曾家军马,一步步退入村里。林冲、呼延灼紧护定晁盖,东西赶杀。林冲见路途不
好,急退回来收兵。看得两边各皆折了些人马。晁盖回到寨中,心中甚忧。众将劝
道:“哥哥且宽心,休得愁闷,有伤贵体。往常宋公明哥哥出军,亦曾失利,好歹
得胜回寨,今日混战,各折了些军马,又不曾输了与他,何须忧闷?”晁盖只是郁
郁不乐。在寨内一连三日,每日搦战,曾头市上并不曾见一个。

第四日,忽有两个和尚直到晁盖寨里来投拜,军人引到中军帐前,两个和尚跪
下告道:“小僧是曾头市上东边法华寺里监寺僧人,今被曾家五虎不时常来本寺作
践罗唣,索要金银财帛,无所不为。小僧已知他的备细出没去处,特地前来拜请头
领入去劫寨,剿除了他时,当坊有幸。”晁盖见说大喜,便请两个和尚坐了,置酒
相待。林冲谏道:“哥哥休得听信,其中莫非有诈。”和尚道:“小僧是个出家人,
怎敢妄语?久闻梁山泊行仁义之道,所过之处,并不扰民,因此特来拜投,如何故
来掇赚将军?况兼曾家未必赢得头领大军,何故相疑?”晁盖道:“兄弟休生疑心,
误了大事。今晚我自去走一遭。”林冲道:“哥哥休去,我等分一半人马去劫寨,
哥哥在外面接应。”晁盖道:“我不自去,谁肯向前?你可留一半军马在外接应。”
林冲道:“哥哥带谁入去?”晁盖道:“点十个头领,分二千五百人马入去。”十
个头领是:刘唐、阮小二、呼延灼、阮小五、欧鹏、阮小七、燕顺、杜迁、宋万、
白胜。当晚造饭吃了,马摘鸾铃,军士衔枚,黑夜疾走,悄悄地跟了两个和尚,直
奔法华寺内,看时,是一个古寺。晁盖下马,入到寺内,见没僧众,问那两个和尚
道:“怎地这个大寺院,没一个僧众?”和尚道:“便是曾家畜生薅恼,不得已各
自归俗去了;只有长老并几个侍者,自在塔院里居住。头领暂且屯住了人马,等更
深些,小僧直引到那厮寨里。”晁盖道:“他的寨在那里?”和尚道:“他有四个
寨栅,只是北寨里,便是曾家弟兄屯军之处。若只打得那个寨子时,别的都不打紧。
这三个寨便罢了。”晁盖道:“那个时分可去?”和尚道:“如今只是二更天气,
且待三更时分,他无准备。”初时听得曾头市上,整整齐齐打更鼓响。又听了半个
更次,绝不闻更点之声。和尚道:“军人想是已睡了,如今可去。”和尚当先引路。
晁盖带同诸将上马,领兵离了法华寺,跟着和尚。

行不到五里多路,黑影处不见了两个僧人,前军不敢行动。看四边路杂难行,
又不见有人家。军士却慌起来,报与晁盖知道。呼延灼便叫急回旧路。走不到百十
步,只见四下里金鼓齐鸣,喊声震地,一望都是火把。晁盖众将引军夺路而走,才
转得两个弯,撞出一彪军马,当头乱箭射将来,不期一箭,正中晁盖脸上,倒撞下
马来;却得呼延灼、燕顺两骑马,死并将去,背后刘唐、白胜,救得晁盖上马,杀
出村中来。村口林冲等,引军接应,刚才敌得住。两军混战,直杀到天明,各自归
寨。林冲回来点军时,三阮、宋万、杜迁,水里逃得性命;带入去二千五百人马,
止剩得一千二三百人,跟着欧鹏,都回到帐中。众头领且来看晁盖时,那枝箭正射
在面颊上;急拔得箭出,血晕倒了。看那箭时,上有史文恭字,林冲叫取金枪药敷
贴上,原来却是一枝药箭。晁盖中了箭毒,已自言语不得。林冲叫扶上车子,便差
三阮、杜迁、宋万先送回山寨。其余十五个头领,在寨中商议:“今番晁天王哥哥
下山来,不想遭这一场,正应了风折认旗之兆;我等只可收兵回去,这曾头市急切
不能取得。”呼延灼道:“须等宋公明哥哥将令来,方可回军。”当日众头领闷闷
不已,众军亦无恋战之心,人人都有还山之意。

当晚二更时分,天色微明,十五个头领,都在寨中纳闷,正是蛇无头而不行,
鸟无翅而不飞,嗟咨叹惜,进退无措。忽听的伏路小校,慌急来报:“前面四五路
军马杀来,火把不计其数。”林冲听了,一齐上马。三面山字火把齐明,照见如同
白日,四下里呐喊到寨前。林冲领了众头领,不去抵敌,拔寨都起,回马便走。曾
家军马,背后卷杀将来,两军且战且走。走过了五六十里,方才得脱。计点人兵,
又折了五七百人。大败亏输,急取旧路,望梁山泊回来。退到半路,正迎着戴宗传
下军令,教众头领引军且回山寨,别作良策。众将得令,引军回到水浒寨上山,都
来看视晁头领时,已自水米不能入口,饮食不进,浑身虚肿。宋江等守定在床前啼
哭,亲手敷贴药饵,灌下汤散。众头领都守在帐前看视。

当日夜至三更,晁盖身体沉重,转头看着宋江嘱付道:“贤弟保重。若那个捉
得射死我的,便教他做梁山泊主!”言罢,便瞑目而死。宋江见晁盖死了,比似丧
考妣一般,哭得发昏。众头领扶策宋江出来主事。吴用、公孙胜劝道:“哥哥且省
烦恼,生死人之分定,何故痛伤?且请理会大事。”宋江哭罢,便教把香汤沐浴了
尸首,装殓衣服巾帻,停在聚义厅上。众头领都来举哀祭祀。一面合造内棺外椁,
选了吉时,盛放在正厅上,建起灵帏,中间设个神主,上写道:“梁山泊主天王晁
公神主”。山寨中头领,自宋公明以下,都带重孝;小头目并众小喽罗,亦带孝头
巾。把那枝誓箭,就供养在灵前。寨内扬起长,请附近寺院僧众上山做功德,追
荐晁天王。宋江每日领众举哀,无心管理山寨事务。林冲与公孙胜、吴用,并众头
领商议,立宋公明为梁山泊主,诸人拱听号令。

次日清晨,香花灯烛,林冲为首,与众等请出宋公明在聚义厅上坐定。吴用、
林冲开话道:“哥哥听禀:‘国一日不可无君,家一日不可无主。’晁头领是归天
去了,山寨中事业,岂可无主?四海之内,皆闻哥哥大名,来日吉日良辰,请哥哥
为山寨之主,诸人拱听号令。”宋江道:“晁天王临死时嘱付:‘如有人捉得史文
恭者,便立为梁山泊主。’此话众头领皆知。今骨肉未寒,岂可忘了?又不曾报得
仇,雪得恨,如何便居得此位?”吴学究又劝道:“晁天王虽是如此说,今日又未
曾捉得那人,山寨中岂可一日无主?若哥哥不坐时,谁人敢当此位?寨中人马如何管
领?然虽遗言如此,哥哥权且尊临此位,坐一坐,待日后别有计较。”宋江道:“军
师言之极当。今日小可权当此位,待日后报仇雪恨已了,拿住史文恭的,不拘何人,
须当此位。”黑旋风李逵在侧边叫道:“哥哥休说做梁山泊主,便做了大宋皇帝,
却不好!”宋江喝道:“这黑厮又来胡说!再休如此乱言,先割了你这厮舌头!”
李逵道:“我又不教哥哥做社长,请哥哥做皇帝,倒要割了我舌头!”吴学究道:
“这厮不识尊卑的人,兄长不要和他一般见识。且请哥哥主张大事。”

宋江焚香已罢,权居主位,坐了第一把椅子。上首军师吴用,下首公孙胜;左
一带林冲为头,右一带呼延灼居长。众人参拜了,两边坐下。宋江乃言道:“小可
今日权居此位,全赖众兄弟扶助,同心合意,共为股肱,一同替天行道。如今山寨,
人马数多,非比往日,可请众兄弟分做六寨驻扎。聚义厅今改为忠义堂。前后左右
立四个旱寨,后山两个小寨,前山三座关隘,山下一个水寨,两滩两个小寨,今日
各请弟兄分投去管。忠义堂上,是我权居尊位。第二位军师吴学究,第三位法师公
孙胜,第四位花荣,第五位秦明,第六位吕方,第七位郭盛;左军寨内:第一位林
冲,第二位刘唐,第三位史进,第四位杨雄,第五位石秀,第六位杜迁,第七位宋
万;右军寨内:第一位呼延灼,第二位朱仝,第三位戴宗,第四位穆弘,第五位李
逵,第六位欧鹏,第七位穆春;前军寨内:第一位李应,第二位徐宁,第三位鲁智
深,第四位武松,第五位杨志,第六位马麟,第七位施恩;后军寨内:第一位柴进,
第二位孙立,第三位黄信,第四位韩滔,第五位彭,第六位邓飞,第七位薛永;
水军寨内:第一位李俊,第二位阮小二,第三位阮小五,第四位阮小七,第五位张
横,第六位张顺,第七位童威,第八位童猛。——六寨计四十三员头领。山前第一
关,令雷横、樊瑞守把;第二关,令解珍、解宝守把;第三关,令项充、李衮守把。
金沙滩小寨内,令燕顺、郑天寿、孔明、孔亮四个守把;鸭嘴滩小寨内,令李忠、
周通、邹渊、邹润四个守把。山后两个小寨:左一个旱寨内,令王矮虎、一丈青、
曹正;右一个旱寨内,令朱武、陈达、杨春六人守把。忠义堂内,左一带房中,掌
文卷,萧让;掌赏罚,裴宣;掌印信,金大坚;掌算钱粮,蒋敬。右一带房中,管
炮,凌振;管造船,孟康;管造衣甲,侯健;管筑城垣,陶宗旺。忠义堂后两厢房
中管事人员:监造房屋,李云;铁匠总管,汤隆;监造酒醋,朱富;监备筵宴,宋
清;掌管什物,杜兴、白胜。山下四路作眼酒店,原拨定朱贵、乐和、时迁、李立、
孙新、顾大嫂、张青、孙二娘,已自定数。管北地收买马匹,杨林、石勇、段景住。
分拨已定,各自遵守,毋得违犯。”梁山泊水浒寨内,大小头领,自从宋公明为寨
主,尽皆欢喜,拱听约束。一日,宋江聚众商议,欲要与晁盖报仇,兴兵去打曾头
市。军师吴用谏道:“哥哥,庶民居丧,尚且不可轻动,哥哥兴师,且待百日之后,
方可举兵。”宋江依吴学究之言,守住山寨,每日修设好事,只做功果,追荐晁盖。

一日,请到一僧,法名大圆,乃是北京大名府在城龙华寺僧人,只为游方来到
济宁,经过梁山泊,就请在寨内做道场。因吃斋之次,闲话间,宋江问起北京风土
人物,那大圆和尚说道:“头领如何不闻河北玉麒麟之名?”宋江、吴用听了,猛
然省起,说道:“你看我们未老,却恁地忘事!北京城里是有个卢大员外,双名俊
义,绰号玉麒麟,是河北三绝,祖居北京人氏,一身好武艺,棍棒天下无对。梁山
泊寨中若得此人时,何怕官军缉捕,岂愁兵马来临?”吴用笑道:“哥哥何故自丧
志气?若要此人上山,有何难哉!”宋江答道:“他是北京大名府第一等长者,如
何能够得他来落草?”吴学究道:“吴用也在心多时了,不想一向忘却。小生略施
小计,便教本人上山。”宋江便道:“人称足下为智多星,端的名不虚传!敢问军
师用甚计策,赚得本人上山?”吴用不慌不忙,叠两个指头,说出这段计来。有分
教:卢俊义撇却锦簇珠围,来试龙潭虎穴。正是:只为一人归水浒,致令百姓受兵
戈。

毕竟吴学究怎地赚卢俊义上山,且听下回分解。