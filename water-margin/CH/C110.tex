\chapter{燕青秋林渡射雁~宋江东京城献俘}

话说当下宋江问降将胡俊有何计策去取东川、安德两处城池。胡俊道:“东川
城中守将,是小将的兄弟胡显。小将蒙李将军不杀之恩,愿往东川招兄弟胡显来降。
剩下安德孤城,亦将不战而自降矣。”宋江大喜,仍令李俊同去。一面调遣将士,
提兵分头去招抚所属未复州县;一面差戴宗赍表,申奏朝廷,请旨定夺;并领文申
呈陈安抚,及上宿太尉书札。宋江令将士到王庆宫中,搜掳了金珠细软,珍宝玉帛,
将违禁的龙楼凤阁,翠屋珠轩,及违禁器仗衣服,尽行烧毁;又差人到云安,教张
横等将违禁行宫器仗等项,亦皆烧毁。
却说戴宗先将申文到荆南,报呈陈安抚,陈安抚也写了表文,一同上达。戴宗到东
京,将书札投递宿太尉,并送礼物。宿太尉将表进呈御览。徽宗皇帝龙颜大喜,即
时降下圣旨,行到淮西,将反贼王庆,解赴东京,候旨处决,其余擒下伪妃、伪官
等众从贼,都就淮西市曹处斩,枭示施行。淮西百姓遭王庆暴虐,准留兵饷若干,
计户给散,以赡穷民。其阵亡有功降将,俱从厚赠荫。淮西各州县所缺正佐官员,
速推补赴任交代。各州官多有先行被贼胁从,以后归正者,都着陈分别事情轻重,
便宜处分。其征讨有功正偏将佐,俱俟还京之日,论功升赏。敕命一下,戴宗先来
报知。那陈安抚等,已都到南丰城中了。那时胡俊已是招降了兄弟胡显,将东川军
民版籍、户口,及钱粮册籍,前来献纳听罪。那安德州贼人,望风归降。云安、东
川、安德三处,农不离其田业,贾不离其肆宅,皆李俊之功。王庆占据的八郡八十
六州县,都收复了。
自戴宗从东京回到南丰十余日,天使捧诏书,驰驿到来。陈安抚与各官接了圣旨,
一一奉行。次早,天使还京。陈令监中取出段氏、李助,及一行叛逆从贼,判了
斩字,推出南丰市曹处斩,将首级各门枭示讫。段三娘从小不循闺训,自家择配,
做下迷天大罪,如今身首异处,又连累了若干眷属,其父段太公先死于房山寨。
话不絮繁。却说陈安抚、宋先锋标录李俊、胡俊、琼英、孙安功次,出榜去各处招
抚,以安百姓。八十六州县,复见天日,复为良民,其余随从贼徒不伤人者,拨还
产业,复为乡民。西京守将乔道清、马灵,已有新官到任,次第都到南丰。各州县
正佐贰官,陆续都到。李俊、二张、三阮、二童,已将州务交代,尽到南丰相叙。
陈安抚众官及宋江以下一百单八个头领,及河北降将,都在南丰设太平宴,庆贺众
将官僚,赏劳三军将佐。宋江教公孙胜、乔道清主持醮事,打了七日七夜醮事,超
度阵亡军将,及淮西屈死冤魂。
醮事方完,忽报孙安患暴疾,卒于营中。宋江悲悼不已,以礼殡殓,葬于龙门山侧。
乔道清因孙安死了,十分痛哭,对宋江说道:“孙安与贫道同乡,又与贫道最厚,
他为父报仇,因而犯罪,陷身于贼,蒙先锋收录他,指望日后有个结果,不意他中
道而死。贫道得蒙先锋收录,亦是他来指迷。今日他死,贫道何以为情。乔某蒙二
位先锋厚恩,铭心镂骨,终难补报。愿乞骸骨归田野,以延残喘。”马灵见乔道清
要去,也来拜辞宋江:“恳求先锋允放马某与乔法师同往。”宋江听说,惨然不乐,
因二人坚意要去,十分挽留不住,宋江只得允放,乃置酒饯别。公孙胜在旁,只不
做声。乔道清、马灵拜辞了宋江、公孙胜,又去拜辞了陈安抚。二人飘然去了。后
来乔道清、马灵都到罗真人处,从师学道,以终天年。
陈安抚招抚赈济淮西诸郡军民已毕。那淮西乃淮渎之西,因此,宋人叫宛州、南丰
等处是淮西。陈安抚传令,教先锋头目,收拾朝京。军令传下,宋江一面先发中军
军马,护送陈安抚、侯参谋、罗武谕起行,一面着令水军头领,乘驾船只,从水路
先回东京,驻扎听调。宋江教萧让撰文,金大坚镌石勒碑以记其事,立石于南丰城
东龙门山下,至今古迹尚存。降将胡俊、胡显置酒饯别宋先锋。后来宋江入朝,将
胡俊、胡显反邪归正,招降二将之功,奏过天子,特授胡俊、胡显为东川水军团练
之职,此是后话。
当下宋江将兵马分作五起进发,克日起行,军士除留下各州县镇守外,其间亦有乞
归田里者。现今兵马共十余万,离了南丰,取路望东京来。军有纪律,所过地方,
秋毫无犯。百姓香花灯烛价拜送。于路行了数日,到一个去处,地名秋林渡。那秋
林渡在宛州属下内乡县秋林山之南。那山泉石佳丽,宋江在马上遥看山景,仰观天
上,见空中数行塞雁,不依次序,高低乱飞,都有惊鸣之意。宋江见了,心疑作怪。
又听的前军喝采,使人去问缘由,飞马回报,原来是浪子燕青,初学弓箭,向空中
射雁,箭箭不空。却才须臾之间,射下十数只鸿雁,因此诸将惊讶不已。宋江教唤
燕青来。只见燕青弯弓插箭,即飞马而来,背后马上捎带死雁数只,来见宋江,下
马离鞍,立在一边。
宋公明问道:“恰才你射雁来?”燕青答道:“小弟初学弓箭,见空中一群雁过,
偶然射之,不想箭箭皆中。”宋江道:“为军的人,学射弓箭,是本等的事。射的
亲是你能处。我想宾鸿避寒,离了天山,衔芦过关,趁江南地暖,求食稻粱,初春
方回。此宾鸿仁义之禽,或数十,或三五十只,递相谦让,尊者在前,卑者在后,
次序而飞,不越群伴,遇晚宿歇,亦有当更之报。且雄失其雌,雌失其雄,至死不
配。此禽仁义礼智信,五常俱备:空中遥见死雁,尽有哀鸣之意,失伴孤雁,并无
侵犯,此为仁也;一失雌雄,死而不配,此为义也;依次而飞,不越前后,此为礼
也;预避鹰雕,衔芦过关,此为智也;秋南春北,不越而来,此为信也。此禽五常
足备之物,岂忍害之。天上一群鸿雁相呼而过,正如我等弟兄一般。你却射了那数
只,比俺兄弟中失了几个,众人心内如何?兄弟今后不可害此礼义之禽。”燕青默
默无语,悔罪不及。宋江有感于心,在马上口占诗一首:
山岭崎岖水渺茫,横空雁阵两三行。
忽然失却双飞伴,月冷风清也断肠。
宋江吟诗罢,不觉自己心中凄惨,睹物伤情。当晚屯兵于秋林渡口。宋江在帐中,
因复感叹燕青射雁之事,心中纳闷,叫取过纸笔,作词一首:

楚天空阔,雁离群万里,恍然惊散。自顾影欲下寒塘,正草枯沙净,水平天远。
写不成书,只寄的相思一点。暮日空濠,晓烟古堑,诉不尽许多哀怨。拣尽芦花无
处宿,叹何时玉关重见。嘹呖忧愁呜咽,恨江渚难留恋。请观他春昼归来,画梁双
燕。
宋江写毕,递与吴用、公孙胜看。词中之意,甚有悲哀忧戚之思,宋江心中,郁郁
不乐。当夜,吴用等设酒备肴,尽醉方休。次日天明,俱各上马,望南而行。路上
行程,正值暮冬,景物凄凉。宋江于路,此心终有所感。不则一日,回到京师,屯
驻军马于陈桥驿,听候圣旨。
且说先是陈安抚并侯参谋中军人马入城,已将宋江等功劳,奏闻天子,报说宋先锋
等诸将兵马,班师回京,已到关外。陈安抚前来启奏,说宋江等诸将征战劳苦之事,
天子闻奏,大加称赞。陈、侯蒙、罗各封升官爵,钦赏银两缎匹,传下圣旨,
命黄门侍郎宣宋江等面君朝见,都教披挂入城。有诗为证:
去时三十六,回来十八双。
纵横千万里,谈笑却还乡。

且说宋江等众将一百八人,遵奉圣旨,本身披挂。戎装革带,顶盔挂甲,身穿
锦袄,悬带金银牌面,从东华门而入,都至文德殿朝见天子,拜舞起居,山呼万岁。
皇上看了宋江等众将英雄,尽是锦袍金带,惟有吴用、公孙胜、鲁智深、武松身着
本身服色,天子圣意大喜,乃曰:“寡人多知卿等征进劳苦,剿寇用心,中伤者多,
寡人甚为忧戚。”宋江再拜奏道:“托圣上洪福齐天,臣等众将虽有金伤,俱各无
事,今元凶授首,淮西平定,实陛下威德所致,臣等何劳之有。”再拜称谢奏道:
“臣等奉旨,将王庆献俘阙下,候旨定夺。”天子降旨:“着法司会官,将王庆凌
迟处决。”宋江将萧嘉穗用奇计克复城池,保全生灵,有功不伐,超然高举。天子
称奖道:“皆卿等忠诚感动!”命省院官访取萧嘉穗赴京擢用。宋江叩头称谢。那
些省院官,那个肯替朝廷出力,访问贤良。此是后话。
是日,天子特命省院等官计议封爵。太师蔡京、枢密童贯商议奏道:“目今天下尚
未静平,不可升迁。且加宋江为保义郎,带御器械,正受皇城使;副先锋卢俊义加
为宣武郎,带御器械,行营团练使;吴用等三十四员,加封为正将军;朱武等七十
二员,加封为偏将军;支给金银,赏赐三军人等。”天子准奏,仍敕与省院众官,
加封爵禄,与宋江等支给赏赐,宋江等就于文德殿顿首谢恩。天子命光禄寺大设御
宴,钦赏宋江锦袍一领,金甲一副,名马一匹;卢俊义以下,赏赐有差,尽于内府
关支。宋江与众将谢恩已罢,尽出宫禁,都到西华门外,上马回营。一行众将,出
的城来,直至行营安歇,听候朝廷委用。
当日法司奉旨会官,写了犯由牌,打开囚车,取出王庆,判了“剐”字,拥到市曹。
看的人压肩迭背,也有唾骂的,也有嗟叹的。那王庆的父王砉及前妻丈人等诸亲眷
属,已于王庆初反时收捕,诛夷殆尽。今日只有王庆一个,簇拥在刀剑林中。两声
破鼓响,一棒碎锣鸣,枪刀排白雪,皂纛展乌云。刽子手叫起恶杀都来,恰好午时
三刻,将王庆押到十字路头,读罢犯由,如法凌迟处死。看的人都道:
此是恶人榜样,到底骈首戕身。
若非犯着十恶,如何受此极刑?
当下监斩官将王庆处决了当,枭首施行,不在话下。
再说宋江众人,受恩回营。次日,只见公孙胜直至行营中军帐内,与宋江等众人,
打了稽首,便禀宋江道:“向日本师罗真人嘱咐小道,令送兄长还京之后,便回山
中。今日兄长功成名遂,贫道就今拜别仁兄,辞别众位,便归山中,从师学道,侍
养老母,以终天年。”宋江见公孙胜说起前言,不敢翻悔,潸然泪下,便对公孙胜
道:“我想昔日弟兄相聚,如花始开;今日弟兄分别,如花零落。吾虽不敢负汝前
言,心中岂忍分别?”公孙胜道:“若是小道半途撇了仁兄,便是寡情薄意。今来
仁兄功成名遂,只得曲允。”宋江再四挽留不住,便乃设一筵宴,令众弟兄相别,
筵上举杯,众皆叹息,人人洒泪,各以金帛相赆。公孙胜推却不受,众兄弟只顾打
拴在包裹。次日,众皆相别。公孙胜穿上麻鞋,背上包裹,打个稽首,望北登程去
了。宋江连日思忆,泪如雨下,郁郁不乐。
时下又值正旦节相近,诸官准备朝贺。蔡太师恐宋江人等都来朝贺,天子见之,必
当重用,随即奏闻天子,降下圣旨,使人当住,只教宋江、卢俊义两个有职人员,
随班朝贺,其余出征官员,俱系白身,恐有惊御,尽皆免礼。是日正旦,百官朝贺,
宋江、卢俊义俱各公服,都在待漏院伺候早朝,随班行礼。是日驾坐紫宸殿受朝,
宋江、卢俊义随班拜罢,于两班侍下,不能上殿。仰观殿上,玉簪珠履,紫绶金章,
往来称觞献寿,自天明直至午牌,方始得沾谢恩御酒。百官朝散,天子驾起。宋江、
卢俊义出内,卸了公服幞头,上马回营,面有愁颜赧色。吴用等接着。众将见宋江
面带忧容,心闷不乐,都来贺节。百余人拜罢,立于两边,宋江低首不语。
吴用问道:“兄长今日朝贺天子回来,何以愁闷?”宋江叹口气道:“想我生来八
字浅薄,命运蹇滞。破辽平寇,东征西讨,受了许多劳苦,今日连累众兄弟无功,
因此愁闷。”吴用答道:“兄长既知造化未通,何故不乐?万事分定,不必多忧。”
黑旋风李逵道:“哥哥,好没寻思!当初在梁山泊里,不受一个的气,却今日也要
招安,明日也要招安,讨得招安了,却惹烦恼。放着兄弟们都在这里,再上梁山泊
去,却不快活!”宋江大喝道:“这黑禽兽又来无礼!如今做了国家臣子,都是朝
廷良臣。你这厮不省得道理,反心尚兀自未除!”李逵又应道:“哥哥不听我说,
明朝有的气受哩!”众人都笑,且捧酒与宋江添寿。是日只饮到二更,各自散了。
次日引十数骑马入城,到宿太尉、赵枢密,并省院各官处贺节,往来城中,观看者
甚众。就里有人对蔡京说知此事。次日,奏过天子,传旨教省院出榜禁约,于各城
门上张挂:“但凡一应出征官员将军头目,许于城外下营屯扎,听候调遣。非奉上
司明文呼唤,不许擅自入城。如违,定依军令拟罪施行。”差人赍榜,径来陈桥门
外张挂榜文。有人看了,径来报知宋江。宋江转添愁闷,众将得知,亦皆焦躁,尽
有反心,只碍宋江一个。
且说水军头领特地来请军师吴用商议事务。吴用去到船中,见了李俊、张横、张顺、
阮家三昆仲,俱对军师说道:“朝廷失信,奸臣弄权,闭塞贤路。俺哥哥破了大辽,
剿灭田虎,如今又平了王庆,止得个皇城使做,又未曾升赏我等众人。如今倒出榜
文,来禁约我等,不许入城。我想那伙奸臣,渐渐的待要拆散我们弟兄,各调开去。
今请军师自做个主张,若和哥哥商量,断然不肯。就这里杀将起来,把东京劫掠一
空,再回梁山泊去,只是落草倒好。”吴用道:“宋公明兄长断然不肯。你众人枉
费了力,箭头不发,努折箭杆。自古蛇无头而不行,我如何敢自主张?这话须是哥
哥肯时,方才行得;他若不肯做主张,你们要反,也反不出去!”六个水军头领见
吴用不敢主张,都做声不得。
吴用回至中军寨中,来与宋江闲话,计较军情,便道:“仁兄往常千自由,百自在,
众多弟兄亦皆快活。自从受了招安,与国家出力,为国家臣子,不想倒受拘束,不
能任用,兄弟们都有怨心。”宋江听罢,失惊道:“莫不谁在你行说甚来?”吴用
道:“此是人之常情,更待多说?古人云:‘富与贵,人之所欲;贫与贱,人之所
恶。’观形察色,见貌知情。”宋江道:“军师,若是弟兄们但有异心,我当死于
九泉,忠心不改!”
次日早起,会集诸将,商议军机,大小人等都到帐前,宋江开话道:“俺是郓城小
吏出身,又犯大罪,托赖你众弟兄扶持,尊我为头,今日得为臣子。自古道:‘成
人不自在,自在不成人。’虽然朝廷出榜禁治,理合如此。汝诸将士,无故不得入
城。我等山间林下,卤莽军汉极多。倘或因而惹事,必然以法治罪,却又坏了声名。
如今不许我等入城去,倒是幸事。你们众人,若嫌拘束,但有异心,先当斩我首级,
然后你们自去行事。不然,吾亦无颜居世,必当自刎而死,一任你们自为!”众人
听了宋江之言,俱各垂泪设誓而散。有诗为证:
谁向西周怀好音,公明忠义不移心。
当时羞杀秦长脚,身在南朝心在金。
宋江诸将,自此之后,无事也不入城。看看上元节至,东京年例,大张灯火,庆赏
元宵,诸路尽做灯火,于各衙门点放。且说宋江营内浪子燕青,自与乐和商议:“如
今东京点放花灯火戏,庆赏丰年,今上天子,与民同乐。我两个更换些衣服,潜地
入城,看了便回。”只见有人说道:“你们看灯,也带挈我则个!”燕青看见,却
是黑旋风李逵。李逵道:“你们瞒着我,商量看灯,我已听了多时。”燕青道:“和
你去不打紧,只吃你性子不好,必要惹出事来。现今省院出榜,禁治我们,不许入
城。倘若和你入城去看灯,惹出事端,正中了他省院之计。”李逵道:“我今番再
不惹事便了,都依着你行!”燕青道:“明日换了衣巾,都打扮做客人相似,和你
入城去。”李逵大喜。次日都打扮做客人,伺候燕青,同入城去。不期乐和惧怕李
逵,潜与时迁先入城去了。燕青洒脱不开,只得和李逵入城看灯,不敢从陈桥门入
去,大宽转却从封丘门入城。两个手厮挽着,正投桑家瓦来。
来到瓦子前,听的勾栏内锣响,李逵定要入去,燕青只得和他挨在人丛里,听的上
面说平话,正说《三国志》,说到关云长刮骨疗毒。当时有云长左臂中箭,箭毒入
骨。医人华佗道:“若要此疾毒消,可立一铜柱,上置铁环,将臂膊穿将过去,用
索拴牢,割开皮肉,去骨三分,除却箭毒,却用油线缝拢,外用敷药贴了,内用长
托之剂,不过半月,可以平复如初。因此极难治疗。”关公大笑道:“大丈夫死生
不惧,何况只手?不用铜柱铁环,只此便割何妨!”随即叫取棋盘,与客弈棋,伸
起左臂,命华佗刮骨取毒,面不改色,对客谈笑自若。
正说到这里,李逵在人丛中高叫道:“这个正是好男子!”众人失惊,都看李逵,
燕青慌忙拦道:“李大哥,你怎地好村!勾栏瓦舍,如何使得大惊小怪这等叫!”
李逵道:“说到这里,不由人喝采!”燕青拖了李逵便走。
两个离了桑家瓦,转过串道,只见一个汉子飞砖掷瓦,去打一户人家。那人家道:
“清平世界,荡荡乾坤,散了二次,不肯还钱,颠倒打我屋里。”黑旋风听了,路
见不平,便要去打。燕青务死抱住,李逵睁着双眼,要和他厮打的意思。那汉子便
道:“俺自和他有帐讨钱,干你甚事?即日要跟张招讨下江南出征去,你休惹我。
到那里去也是死,要打便和你厮打,死在这里,也得一口好棺材。”李逵道:“却
是甚么下江南?不曾听的点兵调将。”燕青且劝开了闹,两个厮挽着,转出串道,
离了小巷,见一个小小茶肆,两个入去里面,寻副座头,坐了吃茶。对席有个老者,
便请会茶,闲口论闲话。燕青道:“请问丈丈,却才巷口一个军汉厮打,他说道要
跟张招讨下江南,早晚要去出征,请问端的那里去出征?”那老人道:“客人原来
不知。如今江南草寇方腊反了,占了八州二十五县,从睦州起,直至润州,自号为
一国,早晚来打扬州。因此朝廷已差下张招讨、刘都督去剿捕。”
燕青、李逵听了这话,慌忙还了茶钱,离了小巷,径奔出城,回到营中,来见军师
吴学究,报知此事。吴用见说,心中大喜,来对宋先锋说知江南方腊造反,朝廷已
遣张招讨领兵。宋江听了道:“我等诸将军马,闲居在此,甚是不宜。不若使人去
告知宿太尉,令其于天子前保奏,我等情愿起兵,前去征进。”当时会集诸将商议,
尽皆欢喜。
次日,宋江换了些衣服,带领燕青,自来说此一事。径入城中,直至太尉府前下马。
正值太尉在府,令人传报,太尉闻知,忙教请进。宋江来到堂上,再拜起居。宿太
尉道:“将军何事,更衣而来?”宋江禀道:“近因省院出榜,但凡出征官军,非
奉呼唤,不敢擅自入城。今日小将私步至此,上告恩相。听的江南方腊造反,占据
州郡,擅改年号,侵至润州,早晚渡江,来打扬州。宋江等人马久闲,在此屯扎不
宜。某等情愿部领兵马,前去征剿,尽忠报国,望恩相于天子前题奏则个!”宿太
尉听了,大喜道:“将军之言,正合吾意。下官当以一力保奏。将军请回,来早宿
某具本奏闻,天子必当重用。”宋江辞了太尉,自回营寨,与众兄弟说知。
却说宿太尉次日早朝入内,见天子在披香殿与百官文武计事,正说江南方腊作耗,
占据八州二十五县,改年建号,如此作反,自霸称尊,目今早晚兵犯扬州。天子乃
曰:“已命张招讨、刘都督征进,未见次第。”宿太尉越班奏曰:“想此草寇,既
成大患,陛下已遣张总兵、刘都督,再差征西得胜宋先锋,这两支军马为前部,可
去剿除,必干大功。”天子闻奏大喜,急令使臣宣省院官听圣旨。当下张招讨,从、
耿二参谋,亦行保奏,要调宋江这一干人马为前部先锋。省院官到殿,领了圣旨,
随即宣取宋先锋、卢先锋,直到披香殿下,朝见天子。拜舞已毕,天子降敕,封宋
江为平南都总管,征讨方腊正先锋;封卢俊义为兵马副总管,平南副先锋。各赐金
带一条,锦袍一领,金甲一副,名马一骑,彩缎二十五表里。其余正偏将佐,各赐
缎匹银两,待有功次,照名升赏,加受官爵。三军头目,给赐银两。都就于内务府
关支,定限目下出师起行。宋江、卢俊义领了圣旨,就辞了天子。皇上乃曰:“卿
等数内,有个能镌玉石印信金大坚,又有个能识良马皇甫端,留此二人,驾前听用。”
宋江、卢俊义承旨,再拜谢恩,出内上马回营。
宋江、卢俊义两个在马上欢喜,并马而行。出的城来,只见街市上一个汉子,手里
拿着一件东西,两条巧棒,中穿小索,以手牵动,那物便响。宋江见了,却不识的,
使军士唤那汉子问道:“此是何物?”那汉子答道:“此是胡敲也。用手牵动,自
然有声。”宋江乃作诗一首:
一声低了一声高,嘹亮声音透碧霄。
空有许多雄气力,无人提挈谩徒劳。
宋江在马上与卢俊义笑道:“这胡敲正比着我和你,空有冲天的本事,无人提挈,
何能振响!”卢俊义道:“兄长何故发此言?据我等胸中学识,不在古今名将之下。
如无本事,枉自有人提挈,亦作何用?”宋江道:“贤弟差矣!我等若非宿太尉一
力保奏,如何能够天子重用,为人不可忘本!”卢俊义自觉失言,不敢回话。
两个回到营寨,升帐而坐。当时会集诸将,除女将琼英因怀孕染病,留下东京,着
叶清夫妇伏侍,请医调治外,其余将佐,尽教收拾鞍马衣甲,准备起身,征讨方腊。
后来琼英病痊,弥月,产下一个面方耳大的儿子,取名叫做张节。次后闻得丈夫被
贼将厉天闰杀死于独松关,琼英哀恸昏绝,随即同叶清夫妇,亲自到独松关,扶柩
到张清故乡彰德府安葬。叶清又因病故,琼英同安氏老妪,苦守孤儿。张节长大,
跟吴大败金兀于和尚原,杀得兀亟须髯而遁。因此张节得封官爵,归家养
母,以终天年,奏请表扬其母贞节。此是琼英等贞节孝义的结果。
话休絮繁。再说宋江于奉诏讨方腊的次日,于内府关到赏赐缎匹银两,分诸将,
给散三军头目,便就起送金大坚、皇甫端去御前听用。宋江一面调拨战船先行,着
令水军头领整顿篙橹风帆,撑驾望大江进发,传令与马军头领,整顿弓、箭、枪、
刀、衣袍、铠甲。水陆并进,船骑同行,收拾起程。只见蔡太师差府干到营,索取
圣手书生萧让,要他代笔。次日,王都尉自来问宋江求要铁叫子乐和,闻此人善能
歌唱,要他府里使令。宋江只得依允,随即又起送了二人去讫。宋江自此去了五个
弟兄,心中好生郁郁不乐。当与卢俊义计议定了,号令诸军,准备出师。
却说这江南方腊造反已久,积渐而成,不想弄到许大事业。此人原是歙州山中樵夫,
因去溪边净手,水中照见自己头戴平天冠,身穿衮龙袍,以此向人说自家有天子福
分。因朱在吴中征取花石纲,百姓大怨,人人思乱,方腊乘机造反,就清溪县内
帮源洞中,起造宝殿、内苑、宫阙,睦州、歙州亦各有行宫,仍设文武职台,省院
官僚,内相外将,一应大臣。睦州即今时建德,宋改为严州;歙州即今时婺源,宋
改为徽州。这方腊直从这里占到润州,今镇江是也。共该八州二十五县。那八州:
歙州、睦州、杭州、苏州、常州、湖州、宣州、润州。那二十五县,都是这八州管
下。此时嘉兴、松江、崇德、海宁,皆是县治。方腊自为国王,独霸一方,非同小
可。原来方腊上应天书,推背图上道:“十千加一点,冬尽始称尊。纵横过浙水,
显迹在吴兴。”那十千,是万也;头加一点,乃方字也。冬尽,乃腊也;称尊者,
乃南面为君也。正应方腊二字。占据江南八郡,隔着长江天堑,又比淮西差多少来
去。
再说宋江选将出师,相辞了省院诸官,当有宿太尉、赵枢密亲来送行,赏劳三军。
水军头领已把战船从泗水入淮河,望淮安军坝,俱到扬州取齐。宋江、卢俊义谢了
宿太尉、赵枢密,将人马分作五起,取旱路投扬州来。于路无话,前军已到淮安县
屯扎。当有本州官员,置筵设席,等接宋先锋到来,请进城中管待,诉说:“方腊
贼兵浩大,不可轻敌。前面便是扬子大江,此是江南第一个险隘去处。隔江却是润
州。如今是方腊手下枢密吕师囊并十二个统制官守把住江岸。若不得润州为家,难
以抵敌。”
宋江听了,便请军师吴用计较良策,即目前面大江拦截,须用水军船只向前。吴用
道:“扬子江中,有金、焦二山,靠着润州城郭。可叫几个弟兄,前去探路,打听
隔江消息,用何船只,可以渡江。”宋江传令,教唤水军头领前来听令:“你众弟
兄,谁人与我先去探路,打听隔江消息?”只见帐下转过四员战将,尽皆愿往。不
是这几个人来探路,有分教:横尸似北固山高,流血染扬子江赤。直教:大军飞渡
乌龙阵,战舰平吞白雁滩。
毕竟宋江军马怎地去收方腊,且听下回分解。