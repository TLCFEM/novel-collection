\chapter{宋公明避暑疗军兵~乔道清回风烧贼寇}

话说王庆、段三娘与廖立斗不过六七合,廖立被王庆觑个破绽,一朴刀搠翻,
段三娘赶上,复一刀结果了性命。廖立做了半世强人,到此一场春梦。王庆提朴刀
喝道:“如有不愿顺者,廖立为样!”众喽罗见杀了廖立,谁敢抗拒,都投戈拜服。
王庆领众上山,来到寨中,此时已是东方发白。那山四面,都是生成的石室,如房
屋一般,因此叫做房山,属房州管下。当日王庆安顿了各人老小,计点喽罗,盘查
寨中粮草、金银、珍宝、锦帛、布匹等项,杀牛宰马,大赏喽罗,置酒与众人贺庆。
众人遂推王庆为寨主,一面打造军器,一面训练喽罗,准备迎敌官兵,不在话下。
且说当夜房州差来擒捉王庆的一行都头土兵人役,被王庆等杀散,有逃奔得脱的,
回州报知州尹张顾行说:“王庆等预先知觉,拒敌官兵,都头及报人黄达都被杀害。
那伙凶人,投奔西去。”张顾行大惊,次早计点土兵,杀死三十余名,伤者四十余
人。张顾行即日与本州镇守军官计议,添差捕盗官军及营兵,前去追捕。因强人凶
狠,官兵又损折了若干。房山寨喽罗日众,王庆等下山来打家劫舍。张顾行见贼势
猖獗,一面行下文书,仰属县知会守御本境,拨兵前来,协力收捕;一面再与本州
守御兵马都监胡有为计议剿捕。胡有为整点营中军兵,择日起兵前去剿捕。两营军
忽然鼓噪起来,却是为两个月无钱米关给,今日瘪着肚皮,如何去杀贼?张顾行闻
变,只得先将一个月钱米给散。只因这番给散,越激怒了军士,却是为何?当事的,
平日不将军士抚恤节制,直到鼓噪,方才给发请受,已是骄纵了军心。更有一桩可
笑处,今日有事,那扣头常例,又与平日一般克剥。他们平日受的克剥气多了,今
日一总发泄出来。军情汹汹,一时发作,把那胡有为杀死。张顾行见势头不好,只
护着印信,预先躲避。城中无主,又有本处无赖,附和了叛军,遂将良民焚劫。那
强贼王庆,见城中变起,乘势领众多喽罗来打房州。那些叛军及乌合奸徒,反随顺
了强人。因此王庆得志,遂被那厮占据了房州为巢穴。那张顾行到底躲避不脱,也
被杀害。

王庆劫掳房州仓库钱粮,遣李助、段二、段五,分头于房山寨及各处,立竖招军旗
号,买马招军,积草屯粮,远近村镇,都被劫掠。那些游手无赖,及恶逆犯罪的人,
纷纷归附。那时龚端、龚正,向被黄达讦告,家产荡尽,闻王庆招军,也来入了伙。
邻近州县,只好保守城池,谁人敢将军马剿捕?被强人两月之内,便集聚了二万余
人,打破邻近上津县、竹山县、郧乡县三个城池。邻近州县,申报朝廷,朝廷命就
彼处发兵剿捕。宋朝官兵,多因粮饷不足,兵失操练,兵不畏将,将不知兵。一闻
贼警,先是声张得十分凶猛,使士卒寒心,百姓丧胆。及至临阵对敌,将军怯懦,
军士馁弱。怎禁得王庆等贼众,都是拚着性命杀来,官军无不披靡。因此,被王庆
越弄得大了,又打破了南丰府。到后东京调来将士,非贿蔡京、童贯,即赂杨戬、
高俅,他们得了贿赂,那管甚么庸懦。那将士费了本钱,弄得权柄上手,恣意克剥
军粮,杀良冒功,纵兵掳掠,骚扰地方,反将赤子迫逼从贼。自此贼势渐大,纵兵
南下。李助献计,因他是荆南人,仍扮做星相入城,密纠恶少奸棍,里应外合,袭
破荆南城池。遂拜李助为军师,自称楚王。遂有江洋大盗,山寨强人,都来附和。
三四年间,占据了宋朝六座军州。王庆遂于南丰城中,建造宝殿、内苑、宫阙,僭
号改元;也学宋朝,伪设文武职台,省院官僚,内相外将。封李助为军师都丞相,
方翰为枢密,段二为护国统军大将,段五为辅国统军都督,范全为殿帅,龚端为宣
抚使,龚正为转运使——专管支纳出入、考算钱粮,丘翔为御营使,伪立段氏为妃。
自宣和元年作乱以来,至宣和五年春,那时宋江等正在河北征讨田虎,于壶关相拒
之日,那边淮西王庆又打破了云安军及宛州,一总被他占了八座军州。那八座乃是:
南丰

荆南

山南

云安
安德

东川

宛州

西京
那八处所属州县,共八十六处。王庆又于云安建造行宫,令施俊为留守官,镇守云
安军。

初时,王庆令刘敏等侵夺宛州时,那宛州邻近东京,蔡京等瞒不过天子,奏过道君
皇帝,敕蔡攸、童贯征讨王庆,来救宛州。蔡攸、童贯,兵无节制,暴虐士卒,军
心离散,因此,被刘敏等杀得大败亏输,所以陷了宛州,东京震恐。蔡攸、童贯惧
罪,只瞒着天子一个。贼将刘敏、鲁成等,胜了蔡攸、童贯,遂将鲁州、襄州围困。
却得宋江等平定河北班师,复奉诏征讨淮西。真是席不暇暖,马不停蹄,统领大兵
二十余万,向南进发。才渡黄河,省院又行文来催促陈安抚,宋江等兵马,星驰来
救鲁州、襄州。宋江等冒着暑热,汗马驰驱,由粟县、汜水一路行来,所过秋毫无
犯。大兵已到阳翟州界。贼人闻宋江兵到来,鲁州、襄州二处,都解围去了。

那时张清、琼英、叶清看剐了田虎,受了皇恩,奉诏协助宋江征讨王庆。张清等离
了东京,已到颖昌州半月余了。闻宋先锋兵到,三人到军前迎接。参见毕,备述蒙
恩褒封之事。宋江以下,称赞不已。宋江命张清等在军中听用。

宋江请陈安抚、侯参谋、罗武谕等驻扎阳翟城中,自己大军,不便入城。宋江传令,
教大军都屯扎于方城山树林深密阴荫处,以避暑热。又因军士跋涉千里,中暑疲困
者甚多,教安道全置办药料,医疗军士。再教军士搭盖凉庑,安顿马匹,令皇甫端
调治,刻剐鬣毛。吴用道:“大兵屯于丛林,恐敌人用火。”宋江道:“正要他用
火。”宋江却教军士再去于本山高冈凉荫树下,用竹篷茅草,盖一小小山棚。当有
河北降将乔道清会意,来禀宋江道:“乔某感先锋厚恩,今日愿略效微劳。”宋江
大喜,密授计于乔道清,往山棚中去了。宋江挑选军士强健者三万人,令张清、琼
英管领一万兵马,往东山麓埋伏;令孙安、卞祥也管领一万人马,往西山麓埋伏。
“只听我中军轰天炮响,一齐杀出”。将粮草都堆积于山南平麓,教李应、柴进领
五千军士看守。

分拨甫定,忽见公孙胜说道:“兄长筹画甚妙!但如此溽暑,军士往来疲病,倘贼
人以精锐突至,我兵虽十倍于众,必不能取胜。待贫道略施小术,先除了众人烦燥,
军马凉爽,自然强健。”说罢,便仗剑作法,脚踏魁罡二字,左手雷印,右手剑诀,
凝神观想,向巽方取了生气一口,念咒一遍。须臾,凉风飒飒,阴云冉冉,从本山
岭岫中喷薄出来,弥漫了方城山一座,二十余万人马,都在凉风爽气之中。除此山
外,依旧是销金铄铁般烈日,蜩蝉乱鸣,鸟雀藏匿。宋江以下众人,十分欢喜,称
谢公孙胜神功道德。如是六七日,又得安道全疗人,皇甫端调马,军兵马匹,渐渐
强健,不在话下。

且说宛州守将刘敏,乃贼中颇有谋略者,贼人称为刘智伯。他探知宋江兵马,屯扎
山林丛密处避暑。他道:“宋江这伙,终是水泊草寇,不知兵法,所以不能成大事。
待俺略施小计,管教那二十万军马,焦烂一半!”随即传令,挑选轻捷军士五千人,
各备火箭、火炮、火炬,再备战车二千辆,装载芦苇干柴,及硫黄焰硝引火之物。
每车一辆,令四人推送。此时是七月中旬新秋天气,刘敏引了鲁成、郑捷、寇猛、
顾岑四员副将,及铁骑一万,人披软战,马摘銮铃,在后接应。刘敏留下偏将韩、
班泽等,镇守城池。刘敏等众,薄暮离城,恰遇南风大作。刘敏大喜道:“宋江等
这伙人合败!”贼兵行至三更时分,才到方城山南二里外,忽然雾气弥漫山谷。刘
敏道:“天助俺成功!”教军士在后擂鼓呐喊助威,令五千军士,只向山林深密处,
只顾将火箭、火炮、火炬射打焚烧上去。教寇猛、毕胜,催趱推车军士,将火车点
着,向山麓下屯粮处烧来。众人正奋勇上前,忽的都叫道:“苦也!苦也!”却有
恁般奇事,南风正猛,一霎时,却怎么就转过北风!又听得山上霹雳般一声响亮,
被乔道清使了回风返火的法,那些火箭、火炬,都向南边贼阵里飞将来,却似千万
条金蛇火龙,烈焰腾腾的向贼兵飞扑将来,贼兵躲避不迭,都烧得焦头烂额。当下
宋军中有口号四句,单笑那刘敏,道是:
军机固难测,贼人妄擘划。
放火自烧军,好个刘智伯!
那时宋先锋教凌振将号炮施放,那炮直飞起半天里振响。东有张清、琼英,西有孙
安、卞祥,各领兵冲杀过来。贼兵大败亏输。鲁成被孙安一剑,挥为两段;郑捷被
琼英一石子,打下马来,张清再一枪,结果了性命;顾岑被卞祥搠死;寇猛被乱兵
所杀;二万三千人马,被火烧兵杀,折了一大半,其余四散逃窜;二千辆车,烧个
尽绝;只有刘敏同三四百败残军卒,向前逃奔,到宛州去了。宋军不曾烧毁半茎柴
草,也未常损折一个军卒,夺获马匹、衣甲、金鼓甚多。张清、孙安等,得胜回到
山寨献功。孙安献鲁成首级;张清、琼英献郑捷首级;卞祥献顾岑首级。宋江各各
赏劳,标写乔道清头功,及张清、琼英、孙安、卞祥功次。

吴用道:“兄长妙算,已丧贼胆,但宛州山水盘纡,丘原膏沃,地称陆海,若贼人
添拨兵将,以重兵守之,急切难克。目今金风却暑,玉露生凉,军马都已强健,当
乘我军威大振,城中单弱,速往攻之,必克。然须别分兵南北屯扎,以防贼人救兵
冲突。”宋江称善,依计传令,教关胜、秦明、杨志、黄信、孙立、宣赞、郝思文、
陈达、杨春、周通,统领兵马三万,屯扎宛州之东,以防贼人南来救兵;林冲、呼
延灼、董平、索超、韩滔、彭玘、单廷圭、魏定国、欧鹏、邓飞,领兵三万,屯扎
宛州之西,以拒贼人北来兵马。众将遵令,整点军马去了。当有河北降将孙安等一
十七员,一齐来禀道:“某等蒙先锋收录,深感先锋优礼。今某等愿为前部,前去
攻城,少报厚恩。”宋江依允,遂令张清、琼英统领孙安等十七员将佐,军马五万
为前部。那十七员乃是:
孙安

马灵

卞祥

山士奇

唐斌

文仲容
崔

金鼎

黄钺

梅玉

金祯

毕胜
潘迅

杨芳

冯升

胡避

叶清
当下张清遵令,统领将佐军兵,望宛州征进去了。

宋江同卢俊义、吴用等,管领其余将佐大兵,拔寨都起,离了方城山,望南进
发,到宛州十里外扎寨。令李云、汤隆、陶宗旺监造攻城器具,推送张清等军前备
用。张清等众将领兵马将宛州围得水泄不通。城中守将刘敏,是那夜中了宋江之计,
只逃脱得性命。到宛州,即差人往南丰王庆处申报,并行文邻近州县,求取救兵。
今日被宋兵围了城池,只令坚守城池,待救兵至,方可出击。宋兵攻打城池,一连
六七日,城垣坚固,急切不能得下。宛州城北临汝州,贼将张寿领救兵二万前来,
被林冲等杀其主将张寿,其余偏牙将士及军卒,都溃散去了。同日,又有宛州之南,
安昌、义阳等县救兵到来,被关胜等大败贼兵,擒其将柏仁、张怡,送到宋江大寨
正刑讫。二处斩获甚多。此时李云等已造就攻城器具。孙安、马灵等同心协力,令
军士囊土,四面拥堆距,逼近城垣。又选勇敢轻捷之士,用飞桥转关辘辒,越沟
堑,渡池濠,军士一齐奋勇登城,遂克宛州,活擒守将刘敏,其余偏牙将佐,杀死
二十余名,杀死军士五千余人,降者万人。宋江等大兵入城,将刘敏正法枭示,出
榜安民。标写关胜、林冲、张清,并孙安等众将功次。差人到阳翟州陈安抚处报捷,
并请陈安抚等移镇宛州。陈安抚闻报大喜,随即同了侯参谋、罗武谕来到宛州。宋
江等出郭迎接入城,陈安抚称赞宋江等功勋,是不必得说。

宋江在宛州料理军务,过了十余日,此时已是八月初旬,暑气渐退。宋江对吴用计
议道:“如今当取那一处城池?”吴用道:“此处南去山南军,南极湖湘,北控关
洛,乃是楚蜀咽喉之会。当先取此城,以分贼势。”宋江道:“军师所言,正合我
意。”遂留花荣、林冲、宣赞、郝思文、吕方、郭盛,辅助陈安抚等,管领兵马五
万,镇守宛州。陈安抚又留了圣手书生萧让,传令水军头领李俊等八员,统驾水军
船只,由泌水至山南城北汉江会集。宋江将陆兵分作三队,辞别陈安抚,统领众多
将佐,并军马一十五万,离了宛州,杀奔山南军来。真个是:万马奔驰天地怕,千
军踊跃鬼神愁。
毕竟宋兵如何攻取山南,且听下回分解。