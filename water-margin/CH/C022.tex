\chapter{阎婆大闹郓城县~朱仝义释宋公明}

话说当时众做公的拿住唐牛儿,解进县里来。知县听得有杀人的事,慌忙出来
升厅。众做公的把这唐牛儿簇拥在厅前。知县看时,只见一个婆子跪在左边,一个
汉子跪在右边。知县问道:“甚么杀人公事?”婆子告道:“老身姓阎。有个女儿
唤做婆惜,典与宋押司做外宅。昨夜晚间,我女儿和宋江一处吃酒,这个唐牛儿一
径来寻闹,叫骂出门,邻里尽知。今早宋江出去走了一遭,回来把我女儿杀了。老
身结扭到县前,这唐二又把宋江打夺了去,告相公做主。”知县道:“你这厮怎敢
打夺了凶身?”唐牛儿告道:“小人不知前后因依。只因昨夜去寻宋江搪碗酒吃,
被这阎婆叉小人出来。今早小人自出来卖糟姜,遇见阎婆结扭宋押司在县前。小人
见了,不合去劝他,他便走了。却不知他杀死他女儿的缘由。”知县喝道:“胡说!
宋江是个君子诚实的人,如何肯造次杀人?这人命之事,必然在你身上!左右在那
里?”便唤当厅公吏。

当下转上押司张文远来,见说阎婆告宋江杀了他女儿,“正是我的表子”。随
即取了各人口词,就替阎婆写了状子,叠了一宗案。便唤当地方仵作、行人,并地
厢、里正、邻佑一干人等,来到阎婆家,开了门,取尸首登场检验了。身边放着行
凶刀子一把。当日再三看验得,系是生前项上被刀勒死。众人登场了当,尸首把棺
木盛了,寄放寺院里,将一干人带到县里。

知县却和宋江最好,有心要出脱他,只把唐牛儿来再三推问。唐牛儿供道:“小
人并不知前后。”知县道:“你这厮如何隔夜去他家寻闹?一定你有干涉!”唐牛
儿告道:“小人一时撞去搪碗酒吃。”知县道:“胡说!打这厮!”左右两边狼虎
一般公人,把这唐牛儿一索捆翻了,打到三五十,前后语言一般。知县明知他不知
情,一心要救宋江,只把他来勘问。且叫取一面枷来钉了,禁在牢里。

那张文远上厅来禀道:“虽然如此,现有刀子是宋江的压衣刀,必须去拿宋江
来对问,便有下落。”知县吃他三回五次来禀,遮掩不住,只得差人去宋江下处捉
拿。宋江已自在逃去了,只拿得几家邻人来回话:“凶身宋江在逃,不知去向。”
张文远又禀道:“犯人宋江逃去,他父亲宋太公并兄弟宋清,现在宋家村居住,可
以勾追到官,责限比捕,跟寻宋江到官理问。”知县本不肯行移,只要朦胧做在唐
牛儿身上,日后自慢慢地出他。怎当这张文远立主文案,唆使阎婆上厅,只管来告。
知县情知阻当不住,只得押纸公文,差三两个做公的,去宋家庄勾追宋太公并兄弟
宋清。

公人领了公文,来到宋家村宋太公庄上。太公出来迎接,至草厅上坐定。公人
将出文书,递与太公看了。宋太公道:“上下请坐,容老汉告禀:老汉祖代务农,
守此田园过活。不孝之子宋江,自小忤逆,不肯本分生理,要去做吏,百般说他不
从。因此,老汉数年前,本县官长处告了他忤逆,出了他籍,不在老汉户内人数。
他自在县里住居,老汉自和孩儿宋清,在此荒村,守些田亩过活。他与老汉水米无
交,并无干涉。老汉也怕他做出事来,连累不便,因此在前官手里告了,执凭文帖,
在此存照。老汉取来,教上下看。”众公人都是和宋江好的,明知道这个是预先开
的门路,苦死不肯做冤家。众人回说道:“太公既有执凭,把将来我们看,抄去县
里回话。”太公随即宰杀些鸡鹅,置酒管待了众人,赍发了十数两银子,取出执凭
公文,教他众人抄了。众公人相辞了宋太公,自回县去回知县的话,说道:“宋太
公三年前出了宋江的籍,告了执凭文帖,见有抄白在此,难以勾捉。”知县又是要
出脱宋江的,便道:“既有执凭公文,他又别无亲族,只可出一千贯赏钱,行移诸
处,海捕捉拿便了。”

那张三又挑唆阎婆去厅上披头散发来告道:“宋江实是宋清隐藏在家,不令出
官。相公如何不与老身做主去拿宋江?”知县喝道:“他父亲已自三年前告了他忤
逆在官,出了他籍,现有执凭公文存照,如何拿得他父亲兄弟来比捕?”阎婆告道:
“相公,谁不知道他叫做孝义黑三郎?这执凭是个假的,只是相公做主则个!”知
县道:“胡说!前官手里押的印信公文,如何是假的?”阎婆在厅下叫屈叫苦,哽
哽咽咽地价哭告相公道:“人命大如天,若不肯与老身做主时,只得去州里告状。
只是我女儿死得甚苦!”那张三又上厅来替他禀道:“相公不与他行移拿人时,这
阎婆上司去告状,倒是利害。倘或来提问时,小吏难去回话。”知县情知有理,只
得押了一纸公文,便差朱仝、雷横二都头,当厅发落:“你等可带多人,去宋家村
宋大户庄上,搜捉犯人宋江来。”有诗为证:
不关心事总由他,路上何人怨折花?
为惜如花婆惜死,俏冤家做恶冤家。

朱、雷二都头领了公文,便来点起土兵四十余人,径奔宋家庄上来。宋太公得
知,慌忙出来迎接。朱仝、雷横二人说道:“太公休怪我们。上司差遣,盖不由己。
你的儿子押司现在何处?”宋太公道:“两位都头在上:我这逆子宋江,他和老汉
并无干涉。前官手里,已告开了他,现告的执凭在此。已与宋江三年多各户另籍,
不同老汉一家过活,亦不曾回庄上来。”朱仝道:“然虽如此,我们凭书请客,奉
帖勾人,难凭你说不在庄上。你等我们搜一搜看,好去回话。”便叫土兵三四十人,
围了庄院。“我自把定前门,雷都头,你先入去搜。”雷横便入进里面,庄前庄后
搜了一遍,出来对朱仝说道:“端的不在庄里。”朱仝道:“我只是放心不下,雷
都头,你和众弟兄把了门,我亲自细细地搜一遍。”宋太公道:“老汉是识法度的
人,如何敢藏在庄里?”朱仝道:“这个是人命的公事,你却嗔怪我们不得。”太
公道:“都头尊便,自细细地去搜。”朱仝道:“雷都头,你监着太公在这里,休
教他走动。”

朱仝自进庄里,把朴刀倚在壁边,把门来拴了,走入佛堂内去,把供床拖在一
边,揭起那片地板来。板底下有条索头,将索子头只一拽,铜铃一声响,宋江从地
窨子里钻将出来,见了朱仝,吃那一惊。朱仝道:“公明哥哥,休怪小弟今来捉你。
闲常时和你最好,有的事都不相瞒。一日酒中,兄长曾说道:‘我家佛座底下有个
地窨子,上面放着三世佛,佛堂内有片地板盖着,上面设着供床。你有些紧急之事,
可来这里躲避。’小弟那时听说,记在心里。今日本县知县,差我和雷横两个来时,
没奈何,要瞒生人眼目。相公也有觑兄长之心,只是被张三和这婆子在厅上发言发
语,道本县不做主时,定要在州里告状,因此上又差我两个来搜你庄上。我只怕雷
横执着,不会周全人,倘或见了兄长,没个做圆活处,因此小弟赚他在庄前,一径
自来和兄长说话。此地虽好,也不是安身之处,倘或有人知得,来这里搜着,如之
奈何?”宋江道:“我也自这般寻思。若不是贤兄如此周全,宋江定遭缧绁之厄。”
朱仝道:“休如此说。兄长却投何处去好?”宋江道:“小可寻思有三个安身之处:
一是沧州横海郡小旋风柴进庄上,二乃是青州清风寨小李广花荣处,三者是白虎山
孔太公庄上。他有两个孩儿:长男叫做毛头星孔明,次子叫做独火星孔亮,多曾来
县里相会。那三处在这里踌躇未定,不知投何处去好。”朱仝道:“兄长可以作急
寻思,当行即行。今晚便可动身,切勿迟延自误。”宋江道:“上下官司之事,全
望兄长维持,金帛使用,只顾来取。”朱仝道:“这事放心,都在我身上。兄长只
顾安排去路。”宋江谢了朱仝,再入地窨子去。

朱仝依旧把地板盖上,还将供床压了,开门拿朴刀,出来说道:“真个没在庄
里。”叫道:“雷都头,我们只拿了宋太公去如何?”雷横见说要拿宋太公去,寻
思:“朱仝那人,和宋江最好。他怎地颠倒要拿宋太公?这话以定是反说。他若再
提起,我落得做人情。”朱仝、雷横叫拢土兵,都入草堂上来。宋太公慌忙置酒管
待众人。朱仝道:“休要安排酒食。且请太公和四郎同到本县里走一遭。”雷横道:
“四郎如何不见?”宋太公道:“老汉使他去近村打些农器,不在庄里。宋江那厮,
自三年已前,把这逆子告出了户,现有一纸执凭公文在此存照。”朱仝道:“如何
说得过!我两个奉着知县台旨,叫拿你父子二人,自去县里回话。”雷横道:“朱
都头,你听我说:宋押司他犯罪过,其中必有缘故,也未便该死罪。既然太公已有
执凭公文,系是印信官文书,又不是假的,我们看宋押司日前交往之面,权且担负
他些个,只抄了执凭去回话便了。”朱仝寻思道:“我自反说,要他不疑。”朱仝
道:“既然兄弟这般说了,我没来由做甚么恶人?”宋太公谢了道:“深感二位都
头相觑。”随即排下酒食,犒赏众人。将出二十两银子,送与两位都头。朱仝、雷
横坚执不受,把来散与众人——四十个土兵——分了。抄了一张执凭公文,相别了
宋太公,离了宋家村。朱、雷二位都头,自引了一行人回县去了。

县里知县正值升厅,见朱仝、雷横回来了,便问缘由。两个禀道:“庄前庄后,
四围村坊,搜遍了二次,其实没这个人。宋太公卧病在床,不能动止,早晚临危;
宋清已自前月出外未回。因此只把执凭抄白在此。”知县道:“既然如此……”,
一面申呈本府,一面动了一纸海捕文书,不在话下。县里有那一等和宋江好的相交
之人,都替宋江去张三处说开。那张三也耐不过众人面皮,况且婆娘已死了,张三
又平常亦受宋江好处,因此也只得罢了。朱仝自凑些钱物,把与阎婆,教不要去州
里告状。这婆子也得了些钱物,没奈何,只得依允了。朱仝又将若干银两,教人上
州里去使用,文书不要驳将下来。又得知县一力主张,出一千贯赏钱,行移开了一
个海捕文书,只把唐牛儿问做成个“故纵凶身在逃”,脊杖二十,刺配五百里外。
干连的人,尽数保放宁家。这是后话。有诗为证:
一身狼狈为烟花,地窨藏身亦可拿。
临别叮咛好趋避,髯公端不愧朱家。

且说宋江,他是个庄农之家,如何有这地窨子?原来故宋时,为官容易,做吏
最难。为甚的为官容易?皆因那时朝廷奸臣当道,谗佞专权,非亲不用,非财不取。
为甚做吏最难?那时做押司的,但犯罪责,轻则刺配远恶军州,重则抄扎家产,结
果了残生性命,以此预先安排下这般去处躲身。又恐连累父母,教爹娘告了忤逆,
出了籍册,各户另居,官给执凭公文存照,不相来往,却做家私在屋里。宋时多有
这般算的。

且说宋江从地窨子出来,和父亲、兄弟商议:“今番不是朱仝相觑,须吃官司,
此恩不可忘报。如今我和兄弟两个,且去逃难。天可怜见,若遇宽恩大赦,那时回
来,父子相见。父亲可使人暗暗地送些金银去与朱仝,央他上下使用,及资助阎婆
些少,免得他上司去告扰。”太公道:“这事不用你忧心。你自和兄弟宋清,在路
小心。若到了彼处,那里使个得托的人寄封信来。”

当晚弟兄两个,拴束包裹。到四更时分起来,洗漱罢,吃了早饭,两个打扮动
身。宋江戴着白范阳毡笠儿,上穿白缎子衫,系一条梅红纵线绦,下面缠脚絣衬着
多耳麻鞋。宋清做伴当打扮,背了包裹,都出草厅前,拜辞了父亲宋太公。三人洒
泪不住,太公分付道:“你两个前程万里,休得烦恼。”宋江、宋清却分付大小庄
客,小心看家,早晚殷勤伏侍太公,休教饮食有缺。兄弟两个,各跨了一口腰刀,
都拿了一条朴刀,径出离了宋家村。

两个取路登程,正遇着秋末冬初天气。但见:
柄柄芰荷枯,叶叶梧桐坠。
蛩吟腐草中,雁落平沙地。
细雨湿枫林,霜重寒天气。
不是路行人,怎谙秋滋味。

话说宋江弟兄两个行了数程,在路上思量道:“我们却投奔兀谁的是?”宋清
答道:“我只闻江湖上人传说沧州横海郡柴大官人名字,说他是大周皇帝嫡派子孙,
只不曾拜识,何不只去投奔他?人都说仗义疏财,专一结识天下好汉,救助遭配的
人,是个现世的孟尝君。我两个只投奔他去。”宋江道:“我也心里是这般思想。
他虽和我常常书信来往,无缘分上不曾得会。”两个商量了,径望沧州路上来。途
中免不得登山涉水,过府冲州。但凡客商在路,早晚安歇,有两件事免不得:吃癞
碗,睡死人床。

且把闲话提过,只说正话。宋江弟兄两个,不则一日,来到沧州界分,问人道:
“柴大官人庄在何处?”问了地名,一径投庄前来,便问庄客:“柴大官人在庄上
也不?”庄客答道:“大官人在东庄上收租米,不在庄上。”宋江便问:“此间到
东庄有多少路?”庄客道:“有四十余里。”宋江道:“从何处落路去?”庄客道:
“不敢动问二位官人高姓?”宋江道:“我是郓城县宋江的便是。”庄客道:“莫
不是及时雨宋押司么?”宋江道:“便是。”庄客道:“大官人时常说大名,只怨
怅不能相会。既是宋押司时,小人引去。”庄客慌忙便领了宋江、宋清,径投东庄
来。没三个时辰,早来到东庄。宋江看时,端的好一所庄院,十分齐整。但见:

前迎阔港,后靠高峰。数千株槐柳成林,三五处厅堂待客。转屋角牛羊满地,
打麦场鹅鸭成群。饮馔豪华,赛过那孟尝食客;田园主管,不数他程郑家僮。正是
家有余粮鸡犬饱,户无差役子孙闲。

当下庄客便道:“二位官人且在此亭上坐一坐,待小人去通报大官人出来相接。”
宋江道:“好。”自和宋清在山亭上倚了朴刀,解下腰刀,歇了包裹,坐在亭子上。
那庄客入去不多时,只见那座中间庄门大开,柴大官人引着三五个伴当,慌忙跑将
出来,亭子上与宋江相见。

柴大官人见了宋江,拜在地下,口称道:“端的想杀柴进,天幸今日甚风吹得
到此,大慰平生渴仰之念,多幸!多幸!”宋江也拜在地下答道:“宋江疏顽小吏,
今日特来相投。”柴进扶起宋江来,口里说道:“昨夜灯花报,今早喜鹊噪,不想
却是贵兄来。”满脸堆下笑来。宋江见柴进接得意重,心里甚喜,便唤兄弟宋清,
也来相见了。柴进喝叫伴当收拾了宋押司行李,在后堂西轩下歇处。柴进携住宋江
的手,入到里面正厅上,分宾主坐定。柴进道:“不敢动问,闻知兄长在郓城县勾
当,如何得暇来到荒村敝处?”宋江答道:“久闻大官人大名,如雷灌耳。虽然节
次收得华翰,只恨贱役无闲,不能够相会。今日宋江不才,做出一件没出豁的事来,
弟兄二人寻思,无处安身,想起大官人仗义疏财,特来投奔。”柴进听罢,笑道:
“兄长放心!遮莫做下十恶大罪,既到敝庄,但不用忧心。不是柴进夸口,任他捕
盗官军,不敢正眼儿觑着小庄。”

宋江便把杀了阎婆惜的事,一一告诉了一遍。柴进笑将起来,说道:“兄长放
心。便杀了朝廷的命官,劫了府库的财物,柴进也敢藏在庄里。”说罢,便请宋江
弟兄两个洗浴。随即将出两套衣服、巾帻、丝鞋、净袜,教宋江弟兄两个换了出浴
的旧衣裳。两个洗了浴,都穿了新衣服。庄客自把宋江弟兄的旧衣裳送在歇宿处。
柴进邀宋江去后堂深处,已安排下酒食了,便请宋江正面坐地,柴进对席。宋清有
宋江在上,侧首坐了。

三人坐定,有十数个近上的庄客并几个主管,轮替着把盏,伏侍劝饮。柴进再
三劝宋江弟兄宽怀饮几杯,宋江称谢不已。酒至半酣,三人各诉胸中朝夕相爱之念。
看看天色晚了,点起灯烛。宋江辞道:“酒止。”柴进那里肯放,直吃到初更左侧。
宋江起身去净手。柴进唤一个庄客,提碗灯笼,引领宋江东廊尽头处去净手,便道:
“我且躲杯酒。”大宽转穿出前面廊下来,俄延走着,却转到东廊前面。宋江已有
八分酒,脚步趄了,只顾踏去。那廊下有一个大汉,因害疟疾,当不住那寒冷,把
一锨火在那里向。宋江仰着脸,只顾踏将去,正在火锨柄上,把那火锨里炭火,
都掀在那汉脸上。那汉吃了一惊,惊出一身汗来。

那汉气将起来,把宋江劈胸揪住,大喝道:“你是甚么鸟人?敢来消遣我!”
宋江也吃一惊。正分说不得,那个提灯笼的庄客,慌忙叫道:“不得无礼!这位是
大官人最相待的客官。”那汉道:“‘客官’,‘客官’!我初来时,也是‘客官’,
也曾相待的厚。如今却听庄客搬口,便疏慢了我,正是‘人无千日好,花无百日红’。”
却待要打宋江,那庄客撇了灯笼,便向前来劝。正劝不开,只见两三碗灯笼飞也似
来。柴大官人亲赶到说:“我接不着押司,如何却在这里闹?”

那庄客便把了火锨的事说一遍。柴进笑道:“大汉,你不认的这位奢遮的押
司?”那汉道:“奢遮,奢遮!他敢比不得郓城宋押司少些儿!”柴进大笑道:“大
汉,你认得宋押司不?”那汉道:“我虽不曾认的,江湖上久闻他是个及时雨宋公
明;且又仗义疏财,扶危济困,是个天下闻名的好汉。”柴进问道:“如何见的他
是天下闻名的好汉?”那汉道:“却才说不了,他便是真大丈夫,有头有尾,有始
有终,我如今只等病好时,便去投奔他。”柴进道:“你要见他么?”那汉道:“我
可知要见他哩!”柴进道:“大汉,远便十万八千里,近便只在面前。”柴进指着
宋江,便道:“此位便是及时雨宋公明。”那汉道:“真个也不是?”宋江道:“小
可便是宋江。”那汉定睛看了看,纳头便拜,说道:“我不是梦里么?与兄长相见!”
宋江道:“何故如此错爱?”那汉道:“却才甚是无礼,万望恕罪。有眼不识泰山!”
跪在地下,那里肯起来。宋江慌忙扶住道:“足下高姓大名?”柴进指着那汉,说
出他姓名,叫甚讳字。有分教:山中猛虎,见时魄散魂离;林下强人,撞着心惊胆
裂。正是:说开星月无光彩,道破江山水倒流。

毕竟柴大官人说出那汉还是何人,且听下回分解。