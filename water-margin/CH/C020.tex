\chapter{梁山泊义士尊晁盖~郓城县月夜走刘唐}

话说林冲杀了王伦,手拿尖刀,指着众人说道:“据林冲虽系禁军遭配到此,
今日为众豪杰至此相聚,争奈王伦心胸狭隘,嫉贤妒能,推故不纳,因此火并了这
厮,非林冲要图此位。据着我胸襟胆气,焉敢拒敌官军,剪除君侧元凶首恶?今有
晁兄,仗义疏财,智勇足备,方今天下人闻其名,无有不伏。我今日以义气为重,
立他为山寨之主,好么?”众人道:“头领言之极当。”晁盖道:“不可。自古‘强
兵不压主’。晁盖强杀,只是个远来新到的人,安敢便来占上?”林冲把手向前,
将晁盖推在交椅上,叫道:“今日事已到头,请勿推却。若有不从者,将王伦为例。”
再三再四,扶晁盖坐了。林冲喝叫众人就于亭前参拜了。一面使小喽罗去大寨里摆
下筵席,一面叫人抬过了王伦尸首,一面又着人去山前山后唤众多小头目,都来大
寨里聚义。

林冲等一行人,请晁盖上了轿马,都投大寨里来。到得聚义厅前,下了马,都
上厅来。众人扶晁天王去正中第一位交椅上坐定,中间焚起一炉香来。林冲向前道:
“小可林冲,只是个粗卤匹夫,不过只会些枪棒而已,无学无才,无智无术。今日
山寨,天幸得众豪杰相聚,大义既明,非比往日苟且。学究先生在此,便请做军师,
执掌兵权,调用将校,须坐第二位。”吴用答道:“吴某村中学究,胸次又无经纶
济世之才,虽只读些孙吴兵法,未曾有半粒微功,怎敢占上?”林冲道:“事已到
头,不必谦让。”吴用只得坐了第二位。林冲道:“公孙先生请坐第三位。”晁盖
道:“却使不得。若是这等推让之时,晁盖必须退位。”林冲道:“晁兄差矣!公
孙先生,名闻江湖,善能用兵,有鬼神不测之机,呼风唤雨之法,谁能及得?”公
孙胜道:“虽有些小之法,亦无济世之才,如何便敢占上?还是头领请坐。”林冲
道:“只今番克敌制胜,便见得先生妙法。正是鼎分三足,缺一不可,先生不必推
却。”公孙胜只得坐了第三位。

林冲再要让时,晁盖、吴用、公孙胜都不肯。三人俱道:“适蒙头领所说,鼎
分三足,以此不敢违命。我三人占上,头领再要让人时,晁盖等只得告退。”三人
扶住林冲,只得坐了第四位。晁盖道:“今番须请宋、杜二头领来坐。”那杜迁、
宋万见杀了王伦,寻思道:“自身本事低微,如何近的他们?不若做个人情。”苦
苦地请刘唐坐了第五位,阮小二坐了第六位,阮小五坐了第七位,阮小七坐了第八
位,杜迁坐了第九位,宋万坐了第十位,朱贵坐了第十一位。

梁山泊自此是十一位好汉坐定。山前山后,共有七八百人,都来厅前参拜了,
分立在两下。晁盖道:“你等众人在此,今日林教头扶我做山寨之主,吴学究做军
师,公孙先生同掌兵权,林教头等共管山寨。汝等众人,各依旧职,管领山前山后
事务,守备寨栅滩头,休教有失。各人务要竭力同心,共聚大义。”再教收拾两边
房屋,安顿了阮家老小,便教取出打劫得的生辰纲——金珠宝贝,——并自家庄上
过活的金银财帛,就当厅赏赐众小头目并众多小喽罗。当下椎牛宰马,祭祀天地神
明,庆贺重新聚义。众头领饮酒至半夜方散。次日,又办筵宴庆会,一连吃了数日
筵席。晁盖与吴用等众头领计议,整点仓廒,修理寨栅,打造军器——枪、刀、弓、
箭、衣甲、头盔——准备迎敌官军;安排大小船只,教演人兵水手上船厮杀,好做
提备,不在话下。自此梁山泊十一位头领聚义,真乃是交情浑似股肱,义气如同骨
肉。有诗为证:
古人交谊断黄金,心若同时谊亦深。
水浒请看忠义士,死生能守岁寒心。
因此林冲见晁盖作事宽洪,疏财仗义,安顿各家老小在山,蓦然思念妻子在京师,
存亡未保,遂将心腹备细诉与晁盖道:“小人自从上山之后,欲要搬取妻子上山来,
因见王伦心术不定,难以过活,一向蹉跎过了。流落东京,不知死活。”晁盖道:
“贤弟既有宝眷在京,如何不去取来完聚?你快写书,便教人下山去,星夜取上山
来,多少是好。”林冲当下写了一封书,叫两个自身边心腹小喽罗下山去了。

不过两个月,小喽罗还寨说道:“直至东京城内殿帅府前,寻到张教头家,闻
说娘子被高太尉威逼亲事,自缢身死,已故半载。张教头亦为忧疑,半月之前,染
患身故。止剩得女使锦儿,已招赘丈夫在家过活。访问邻里,亦是如此说。打听得
真实,回来报与头领。”林冲见说,潸然泪下,自此杜绝了心中挂念。晁盖等见说
了,怅然嗟叹。山寨中自此无话,每日只是操练人兵,准备抵敌官军。

忽一日,众头领正在聚义厅上商议事务,只见小喽罗报上山来说道:“济州府
差拨军官,带领约有一千人马,乘驾大小船四五百只,现在石碣村湖荡里屯住,特
来报知。”晁盖大惊,便请军师吴用商议道:“官军将至,如何迎敌?”吴用笑道:
“不须兄长挂心,吴某自有措置。自古道:‘水来土掩,兵到将迎。’”随即唤阮
氏三雄,附耳低言道:“如此如此……”又唤林冲、刘唐受计道:“你两个便这般
这般……。”再叫杜迁、宋万,也分付了。正是:
西迎项羽三千阵,今日先施第一功。

且说济州府尹点差团练使黄安并本府捕盗官一员,带领一千余人,拘集本处船
只,就石碣村湖荡调拨,分开船只作两路来取泊子。

且说团练使黄安,带领人马上船,摇旗呐喊,杀奔金沙滩来。看看渐近滩头,
只听得水面上呜呜咽咽吹将起来。黄安道:“这不是画角之声?且把船来分作两路,
去那芦花荡中湾住。”看时,只见水面上远远地三只船来。看那船时,每只船上只
有五个人:四个人摇着双橹,船头上立着一个人,头带绛红巾,都一样身穿红罗绣
袄,手里各拿着留客住,三只船上人,都一般打扮。于内有人认得的,便对黄安说
道:“这三只船上三个人,一个是阮小二,一个是阮小五,一个是阮小七。”黄安
道:“你众人与我一齐并力向前,拿这三个人!”两边有四五十只船,一齐发着喊,
杀奔前去。那三只船唿哨了一声,一齐便回。黄团练把手内枪拈搭动,向前来叫道:
“只顾杀这贼,我自有重赏。”那三只船前面走,背后官军船上,把箭射将去。那
三阮去船舱里,各拿起一片青狐皮来遮那箭矢。后面船只只顾赶。

赶不过二三里水港,黄安背后一只小船,飞也似划来报道:“且不要赶!我们
那一条杀入去的船只,都被他杀下水里去,把船都夺去了。”黄安问道:“怎的着
了那厮的手!”小船上人答道:“我们正行船时,只见远远地两只船来,每船上各
有五个人。我们并力杀去赶他,赶不过三四里水面,四下里小港钻出七八只小船来。
船上弩箭似飞蝗一般射将来,我们急把船回时,来到窄狭港口,只见岸上约有二三
十人,两头牵一条大篾索,横截在水面上。却待向前看索时,又被他岸上灰瓶、石
子,如雨点一般打将来。众官军只得弃了船只,下水逃命。我众人逃得出来,到旱
路边看时,那岸上人马皆不见了,马也被他牵去了;看马的军人,都杀死在水里。
我们芦花荡边,寻得这只小船儿,径来报与团练。”

黄安听得说了,叫苦不迭,便把白旗招动,教众船不要去赶,且一发回来。那
众船才拨得转头,未曾行动,只见背后那三只船,又引着十数只船,都只是这三五
个人,把红旗摇着,口里吹着胡哨,飞也似赶来。黄安却待把船摆开迎敌时,只听
得芦苇丛中炮响。黄安看时,四下里都是红旗摆满,慌了手脚。后面赶来的船上叫
道:“黄安留下了首级回去!”黄安把船尽力摇过芦苇岸边,却被两边小港里钻出
四五十只小船来,船上弩箭如雨点射将来。黄安就箭林里夺路时,只剩得三四只小
船了。黄安便跳过快船内,回头看时,只见后面的人,一个个都扑通的跳下水里去
了。有和船被拖去的,大半都被杀死。黄安驾着小快船,正走之间,只见芦花荡边
一只船上,立着刘唐,一挠钩搭住黄安的船,托地跳将过来,只一把拦腰提住,喝
道:“不要挣扎!”别的军人能识水者,水里被箭射死;不敢下水的,就船里都活
捉了。

黄安被刘唐扯到岸边,上了岸,远远地晁盖、公孙胜山边骑着马,挺着刀,引
五六十人,三二十匹马,齐来接应。一行人生擒活捉得一二百人,夺的船只,尽数
都收在山南水寨里安顿了。大小头领,一齐都到山寨。晁盖下了马,来到聚义厅上
坐定。众头领各去了戎装军器,团团坐下,捉那黄安绑在将军柱上;取过金银缎匹,
赏了小喽罗。点检共夺得六百余匹好马,这是林冲的功劳;东港是杜迁、宋万的功
劳;西港是阮氏三雄的功劳;捉得黄安,是刘唐的功劳。

众头领大喜,杀牛宰马,山寨里筵会。自酝的好酒,水泊里出的新鲜莲藕并鲜
鱼;山南树上,自有时新的桃、杏、梅、李、枇杷、山枣、柿、栗之类;自养的鸡、
猪、鹅、鸭等品物,不必细说。众头领只顾庆赏。新到山寨,得获全胜,非同小可。
有诗为证:
堪笑王伦妄自矜,庸才大任岂能胜!
一从火并归新主,会见梁山事业新。

正饮酒间,只见小喽罗报道:“山下朱头领使人到寨。”晁盖唤来问有甚事?
小喽罗道:“朱头领探听得一起客商,有数十人结联一处,今晚必从旱路经过,特
来报知。”晁盖道:“正没金帛使用,谁领人去走一遭?”三阮道:“我弟兄们去。”
晁盖道:“好兄弟,小心在意,速去早来。”三阮便下厅去,换了衣裳,跨了腰刀,
拿了朴刀、叉、留客住,点起一百余人上厅来,别了头领,便下山,就金沙滩把
船载过朱贵酒店里去了。晁盖恐三阮担负不下,又使刘唐点起一百余人,教领了下
山去接应,又分付道:“只可善取金帛财物,切不可伤害客商性命。”刘唐去了。
晁盖到三更,不见回报,又使杜迁、宋万引五十余人下山接应。

晁盖与吴用、公孙胜、林冲饮酒至天明,只见小喽罗报喜道:“亏得朱头领,
得了二十余辆车子金银财物,并四五十匹驴骡头口。”晁盖又问道:“不曾杀人么?”
小喽罗答道:“那许多客人,见我们来得头势猛了,都撇下车子、头口、行李,逃
命去了,并不曾伤害他一个。”晁盖见说大喜:“我等初到山寨,不可伤害于人。”
取一锭白银,赏了小喽罗,便叫将了酒果下山来,直接到金沙滩上。见众头领尽把
车辆扛上岸来,再叫撑船去载头口马匹,众头领大喜。把盏已毕,教人去请朱贵上
山来筵宴。

晁盖等众头领,都上到山寨聚义厅上,簸箕掌栲栳圈坐定。叫小喽罗扛抬过许
多财物在厅上,一包包打开,将彩帛衣服堆在一边,行货等物堆在一边,金银宝贝
堆在正面。众头领看了打劫得许多财物,心中欢喜,便叫掌库的小头目,每样取一
半,收贮在库,听候支用。这一半分做两分:厅上十一位头领,均分一分;山上山
下众人,均分一分。把这新拿到的军健,脸上刺了字号,选壮浪的分拨去各寨喂马
砍柴;软弱的,各处看车切草。黄安锁在后寨监房内。晁盖道:“我等今日初到山
寨,当初只指望逃灾避难,投托王伦帐下,为一小头目,多感林教头贤弟推让我为
尊,不想连得了两场喜事:第一赢得官军,收得许多人马船只,捉了黄安;二乃又
得了若干财物金银。此不是皆托众弟兄的才能?”众头领道:“皆托得大哥哥的福
荫,以此得采。”

晁盖再与吴用道:“俺们弟兄七人的性命,皆出于宋押司、朱都头两个。古人
道:‘知恩不报,非为人也!’今日富贵安乐,从何而来?早晚将些金银,可使人
亲到郓城县走一遭,此是第一件要紧的事务。再有白胜陷在济州大牢里,我们必须
要去救他出来。”吴用道:“兄长不必忧心,小生自有划。宋押司是个仁义之人,
紧地不望我们酬谢。然虽如此,礼不可缺,早晚待山寨粗安,必用一个兄弟自去。
白胜的事,可教蓦生人去那里使钱,买上嘱下,松宽他,便好脱身。我等且商量屯
粮,造船,制办军器,安排寨栅、城垣,添造房屋,整顿衣袍、铠甲,打造枪、刀、
弓、箭,防备迎敌官军。”晁盖道:“既然如此,全仗军师妙策指教。”吴用当下
调拨众头领,分派去办,不在话下。

且不说梁山泊自从晁盖上山,好生兴旺。却说济州府太守见黄安手下逃回的军
人,备说梁山泊杀死官军,生擒黄安一事;又说梁山泊好汉,十分英雄了得,无人
近傍得他,难以收捕;抑且水路难认,港汊多杂,以此不能取胜。府尹听了,只叫
得苦,向太师府干办说道:“何涛先折了许多人马,独自一个逃得性命回来,已被
割了两个耳朵,自回家将息,至今不能痊;去的五百人,无一个回来;因此又差团
练使黄安并本府捕盗官,带领军兵前去追捉,亦皆失陷。黄安已被活捉上山,杀死
官军,不知其数,又不能取胜,怎生是好!”太守肚里正怀着鬼胎,没个道理处,
只见承局来报说:“东门接官亭上,有新官到来,飞报到此。”

太守慌忙上马,来到东门外接官亭上,望见尘土起处,新官已到亭子前下马。
府尹接上亭子,相见已了,那新官取出中书省更替文书来,度与府尹。太守看罢,
随即和新官到州衙里,交割牌印,一应府库钱粮等项。当下安排筵席,管待新官。
旧太守备说梁山泊贼盗浩大,杀死官军一节。说罢,新官面如土色,心中思忖道:
“蔡太师将这件勾当抬举我,却是此等地面,这般府分!又没强兵猛将,如何收捕
得这伙强人?倘或这厮们来城里借粮时,却怎生奈何?”旧官太守次日收拾了衣装
行李,自回东京听罪,不在话下。

且说新官宗府尹到任之后,请将一员新调来镇守济州的军官来,当下商议招军
买马,集草屯粮,招募悍勇民夫,智谋贤士,准备收捕梁山泊好汉;一面申呈中书
省,转行牌仰附近州郡,并力剿捕;一面自行下文书所属州县,知会收剿,及仰属
县,着令守御本境。这个都不在话下。

且说本州孔目,差人赍一纸公文,行下所属郓城县,教守御本境,防备梁山泊
贼人。郓城县知县看了公文,教宋江迭成文案,行下各乡村,一体守备。宋江见了
公文,心内寻思道:“晁盖等众人,不想做下这般大事,犯了大罪,劫了生辰纲,
杀了做公的,伤了何观察,又损害了许多官军人马,又把黄安活捉上山。如此之罪,
是灭九族的勾当。虽是被人逼迫,事非得已,于法度上却饶不得。倘有疏失,如之
奈何?”自家一个心中纳闷。分付贴书后司张文远将此文书立成文案,行下各乡各
保。张文远自理会文卷,宋江却信步走出县来。

走不过三二十步,只听得背后有人叫声:“押司!”宋江转回头来看时,却是
做媒的王婆,引着一个婆子,却与他说道:“你有缘,做好事的押司来也!”宋江
转身来问道:“有甚么话说?”王婆拦住,指着阎婆对宋江说道:“押司不知,这
一家儿,从东京来,不是这里人家,嫡亲三口儿。夫主阎公,有个女儿婆惜。他那
阎公,平昔是个好唱的人,自小教得他那女儿婆惜,也会唱诸般耍令;年方一十八
岁,颇有些颜色。三口儿因来山东投奔一个官人不着,流落在此郓城县。不想这里
的人,不喜风流宴乐,因此不能过活,在这县后一个僻净巷内权住。昨日他的家公
因害时疫死了,这阎婆无钱津送,没做道理处,央及老身做媒。我道:‘这般时节,
那里有这等恰好?’又没借换处,正在这里走头没路的,只见押司打从这里过,以
此老身与这阎婆赶来,望押司可怜见他则个,作成一具棺材。”宋江道:“原来恁
地。你两个跟我来,去巷口酒店里,借笔砚写个帖子,与你去县东陈三郎家,取具
棺材。”宋江又问道:“你有结果使用么?”阎婆答道:“实不瞒押司说,棺材尚
无,那讨使用?”宋江道:“我再与你银子十两,做使用钱。”阎婆道:“便是重
生的父母,再长的爷娘,做驴做马,报答押司。”宋江道:“休要如此说。”随即
取出一锭银子,递与阎婆,自回下处去了。且说这婆子将了帖子,径来县东街陈三
郎家,取了一具棺材,回家发送了当,兀自余剩下五六两银子,娘儿两个,把来盘
缠,不在话下。

忽一朝,那阎婆因来谢宋江,见他下处,没有一个妇人家面,回来问间壁王婆
道:“宋押司下处,不见一个妇人面,他曾有娘子也无?”王婆道:“只闻宋押司
家里在宋家村住,却不曾见说他有娘子。在这县里做押司,只是客居。常常见他散
施棺材药饵,极肯济人贫苦,敢怕是未有娘子。”阎婆道:“我这女儿长得好模样,
又会唱曲儿,省得诸般耍笑,从小儿在东京时,只去行院人家串,那一个行院不爱
他!有几个上行首,要问我过房几次,我不肯。只因我两口儿,无人养老,因此不
过房与他。不想今来倒苦了他。我前日去谢宋押司,见他下处没娘子,因此央你与
我对宋押司说,他若要讨人时,我情愿把婆惜与他。我前日得你作成,亏了宋押司
救济,无可报答他,与他做个亲眷来往。”

王婆听了这话,次日来见宋江,备细说了这件事。宋江初时不肯,怎当这婆子
撮合山的嘴撺掇,宋江依允了,就在县西巷内,讨了一所楼房,置办些家火什物,
安顿了阎婆惜娘儿两个,在那里居住。没半月之间,打扮得阎婆惜满头珠翠,遍体
绫罗。正是:

花容袅娜,玉质娉婷。髻横一片乌云,眉扫半弯新月。金莲窄窄,湘裙微露不
胜情;玉笋纤纤,翠袖半笼无限意。星眼浑如点漆,酥胸真似截肪。金屋美人离御
苑,蕊珠仙子下尘寰。

宋江又过几日,连那婆子,也有若干头面衣服,端的养的婆惜丰衣足食。

初时宋江夜夜与婆惜一处歇卧,向后渐渐来得慢了。却是为何?原来宋江是个
好汉,只爱学使枪棒,于女色上不十分要紧。这阎婆惜水也似后生,况兼十八九岁,
正在妙龄之际,因此宋江不中那婆娘意。一日,宋江不合带后司贴书张文远来阎婆
惜家吃酒。这张文远,却是宋江的同房押司,那厮唤做小张三,生得眉清目秀,齿
白唇红;平昔只爱去三瓦两舍,飘蓬浮荡,学得一身风流俊俏;更兼品竹调丝,无
有不会。这婆惜是个酒色娼妓,一见张三,心里便喜,倒有意看上他。那张三见这
婆惜有意以目送情,等宋江起身净手,倒把言语来嘲惹张三。常言道:“风不来,
树不动;船不摇,水不浑。”那张三亦是个酒色之徒,这事如何不晓得?因见这婆
娘眉来眼去,十分有情,便记在心里。向后宋江不在时,这张三便去那里,假意儿
只做来寻宋江。那婆娘留住吃茶,言来语去,成了此事。谁想那婆娘自从和那张三
两个搭识上了,打得火块一般热。亦且这张三又是个惯弄此事的,岂不闻古人有言:
“一不将,二不带。”只因宋江千不合,万不合,带这张三来他家里吃酒,以此看
上了他。自古道:“风流茶说合,酒是色媒人。”正犯着这条款。阎婆惜自从和那
小张三两个搭上,并无半点儿情分在这宋江身上。宋江但若来时,只把言语伤他,
全不兜揽他些个。这宋江是个好汉,不以这女色为念,因此半月十日,去走得一遭。
那张三和这婆惜,如胶似漆,夜去明来,街坊上人也都知了。却有些风声吹在宋江
耳朵里。宋江半信不信,自肚里寻思道:“又不是我父母匹配的妻室,他若无心恋
我,我没来由惹气做甚么?我只不上门便了。”自此有几个月不去。阎婆累使人来
请,宋江只推事故不上门去。正是:
花娘有意随流水,义士无心恋落花。
婆爱钱财娘爱俏,一般行货两家茶。

话分两头。忽一日将晚,宋江从县里出来,去对过茶房里坐定吃茶,只见一个
大汉,头带白范阳毡笠儿,身穿一领黑绿罗袄,下面腿絣护膝,八搭麻鞋,腰里跨
着一口腰刀,背着一个大包,走得汗雨通流,气急喘促,把脸别转着看那县里。宋
江见了这个大汉走得跷蹊,慌忙起身赶出茶房来,跟着那汉走。约走了三二十步,
那汉回过头来,看了宋江,却不认得。宋江见了这人,略有些面熟,“莫不是那里
曾厮会来?”心中一时思量不起。那汉见宋江看了一回,也有些认得,立住了脚,
定睛看那宋江,又不敢问。宋江寻思道:“这个人好作怪!却怎地只顾看我?”宋
江亦不敢问他。只见那汉去路边一个篦头铺里问道:“大哥,前面那个押司是谁?”
篦头待诏应道:“这位是宋押司。”那汉提着朴刀,走到面前,唱个大喏,说道:
“押司认得小弟么?”宋江道:“足下有些面善。”那汉道:“可借一步说话。”
宋江便和那汉入一条僻净小巷。那汉道:“这个酒店里好说话。”

两个上到酒楼,拣个僻净阁儿里坐下。那汉倚了朴刀,解下包裹,撇在桌子底
下,那汉扑翻身便拜。宋江慌忙答礼道:“不敢拜问足下高姓?”那人道:“大恩
人,如何忘了小弟?”宋江道:“兄长是谁?真个有些面熟,小人失忘了。”那汉
道:“小弟便是晁保正庄上曾拜识尊颜蒙恩救了性命的赤发鬼刘唐便是。”宋江听
了大惊,说道:“贤弟,你好大胆!早是没做公的看见,险些儿惹出事来!”刘唐
道:“感承大恩,不惧一死,特地来酬谢。”宋江道:“晁保正弟兄们,近日如何?
兄弟,谁教你来?”刘唐道:“晁头领哥哥,再三拜上大恩人。得蒙救了性命,现
今做了梁山泊主都头领。吴学究做了军师,公孙胜同掌兵权。林冲一力维持,火并
了王伦。山寨里原有杜迁、宋万、朱贵,和俺弟兄七个,共是十一个头领。现今山
寨里聚集得七八百人,粮食不计其数。只想兄长大恩,无可报答,特使刘唐赍一封
书,并黄金一百两,相谢押司并朱、雷二都头。”刘唐打开包裹,取出书来,便递
与宋江。宋江看罢,便拽起褶子前襟,摸出招文袋,打开包儿时,刘唐取出金子放
在桌上。宋江把那封书——就取了一条金子和这书包了,——插在招文袋内,放下
衣襟,便道:“贤弟,将此金子依旧包了。”随即便唤量酒的打酒来,叫大块切一
盘肉来,铺下些菜蔬果子之类,叫量酒人筛酒与刘唐吃。

看看天色晚了,刘唐吃了酒,把桌上金子包打开,要取出来。宋江慌忙拦住道:
“贤弟,你听我说:你们七个弟兄初到山寨,正要金银使用,宋江家中颇有些过活,
且放在你山寨里,等宋江缺少盘缠时,却教兄弟宋清来取。今日非是宋江见外,于
内已受了一条。朱仝那人,也有些家私,不用与他,我自与他说知人情便了。雷横
这人,又不知我报与保正,况兼这人贪赌,倘或将些出去赌时,便惹出事来,不当
稳便,金子切不可与他。贤弟:我不敢留你相请去家中住,倘或有人认得时,不是
耍处!今夜月色必然明朗,你便可回山寨去,莫在此停搁。宋江再三申意众头领,
不能前来庆贺,切乞恕罪。”刘唐道:“哥哥大恩,无可报答,特令小弟送些人情
来与押司,微表孝顺之心。保正哥哥今做头领,学究军师号令非比旧日,小弟怎敢
将回去?到山寨中必然受责。”宋江道:“既是号令严明,我便写一封回书,与你
将去便了。”刘唐苦苦相央宋江收受,宋江那里肯接,随即取一幅纸来,借酒家笔
砚,备细写了一封回书,与刘唐收在包内。刘唐是个直性的人,见宋江如此推却,
想是不肯受了,便将金子依前包了。看看天色晚来,刘唐道:“既然兄长有了回书,
小弟连夜便去。”宋江道:“贤弟,不及相留,以心相照。”刘唐又下了四拜。宋
江教量酒人来道:“有此位官人留下白银一两在此,我明日却自来算。”刘唐背上
包裹,拿了朴刀,跟着宋江下楼来。离了酒楼,出到巷口,天色昏黄,是八月半天
气,月轮上来,宋江携住刘唐的手,分付道:“贤弟保重,再不可来!此间做公的
多,不是耍处。我更不远送,只此相别。”刘唐见月色明朗,拽开脚步,望西路便
走,连夜回梁山泊来。

再说宋江与刘唐别了,自慢慢行回下处来,一头走,一面肚里寻思道:“早是
没做公的看见,争些儿惹出一场大事来!”一头想:“那晁盖倒去落了草,直如此
大弄。”转不过两个弯,只听得背后有人叫一声:“押司,那里去来,好两日不见
面。”宋江回头看时,正是阎婆。不因这番,有分教:宋江小胆翻为大胆,善心变
做恶心。

毕竟宋江怎地发付阎婆,且听下回分解。