\chapter{睦州城箭射邓元觉~乌龙岭神助宋公明}

话说宋江因要救取解珍、解宝的尸,到于乌龙岭下,正中了石宝计策。四下里
伏兵齐起,前有石宝军马,后有邓元觉截住回路。石宝厉声高叫:“宋江不下马受
降,更待何时?”关胜大怒,拍马抡刀战石宝。两将交锋未定,后面喊声又起。脑
背后却是四个水军总管,一齐登岸,会同王𪟝、晁中,从岭上杀将下来。花荣急出,
当住后队,便和王𪟝交战。斗无数合,花荣便走。王𪟝、晁中乘势赶来,被花荣手
起,急放连珠二箭,射中二将,翻身落马。众军呐声喊,不敢向前,退后便走。四
个水军总管,见一连射死王𪟝、晁中,不敢向前,因此花荣抵敌得住。刺斜里又撞
出两阵军来:一队是指挥白钦,一队是指挥景德。这里宋江阵中二将齐出,吕方便
迎住白钦交战,郭盛便与景德相持,四下里分头厮杀,敌对死战。

宋江正慌促间,只听得南军后面喊杀连天,众军奔走。原来却是李逵引两个牌手——
项充、李衮,一千步军,从石宝马军后面杀来。邓元觉引军却待来救应时,背后撞
过鲁智深、武松,两口戒刀,横剁直砍,浑铁禅杖,一冲一戳。两个引一千步军,
直杀入来。随后又是秦明、李应、朱仝、燕顺、马麟、樊瑞、一丈青、王矮虎,各
带马军步军,舍死撞杀入来。四面宋兵,杀散石宝、邓元觉军马,救得宋江等回桐
庐县去,石宝也自收兵上岭去了。宋江在寨中称谢众将:“若非我兄弟相救,宋江
已与解珍、解宝同为泉下之鬼。”吴用道:“为是兄长此去,不合愚意,惟恐有失,
便遣众将相援。”宋江称谢不已。

且说乌龙岭上,石宝、邓元觉两个元帅在寨中商议道:“即日宋江兵马退在桐庐县
驻扎,倘或被他私越小路,度过岭后,睦州咫尺危矣。不若国师亲往清溪大内,面
见天子,奏请添调军马,守护这条岭隘,可保长久。”邓元觉道:“元帅之言极当,
小僧便往。”邓元觉随即上马,先来到睦州,见了右丞相祖士远说:“宋江兵强人
猛,势不可当,军马席卷而来,诚恐有失。小僧特来奏请添兵遣将,保守关隘。”
祖士远听了,便同邓元觉上马,离了睦州,一同到清溪县帮源洞中,先见了左丞相
娄敏中说过了,奏请添调军马。

次日早朝,方腊升殿,左右二丞相一同邓元觉朝见。拜舞已毕,邓元觉向前起居万
岁,便奏道:“臣僧元觉领着圣旨,与太子同守杭州。不想宋江军马兵强将勇,席
卷而来,势难迎敌,致被袁评事引诱入城,以致失陷杭州,太子贪战,出奔而亡。
今来元觉与元帅石宝退守乌龙岭关隘,近日连斩宋江四将,声势颇振。即日宋江已
进兵到桐庐驻扎,诚恐早晚贼人私越小路,透过关来,岭隘难保。请陛下早选良将,
添调精锐军马,同保乌龙岭关隘,以图退贼,克复城池。”方腊道:“各处军兵,
已都调尽。近日又为歙州昱岭上关隘甚紧,又分去了数万军兵。止有御林军马,寡
人要护御大内,如何四散调得开去?”邓元觉又奏道:“陛下不发救兵,臣僧无奈。
若是宋兵度岭之后,睦州焉能保守?”左丞相娄敏中出班奏曰:“这乌龙岭关隘,
亦是要紧去处。臣知御林军兵,总有三万,可分一万,跟国师去保守关隘。乞我王
圣鉴。”方腊不听娄敏中之言,坚执不肯调拨御林军马,去救乌龙岭。

当日朝罢,众人出内。娄丞相与众官商议,只教祖丞相睦州分一员将,拨五千军,
与国师去保乌龙岭。因此,邓元觉同祖士远回睦州来,选了五千精锐军马,首将一
员夏侯成,回到乌龙岭寨内,与石宝说知此事。石宝道:“既是朝廷不拨御林军马,
我等且守住关隘,不可出战。着四个水军总管,牢守滩头江岸边,但有船来,便去
杀退,不可进兵。”

且不说宝光国师同石宝、白钦、景德、夏侯成五个守住乌龙岭关隘。却说宋江自折
了将佐,只在桐庐县驻扎,按兵不动。一住二十余日,不出交战。忽有探马报道:
“朝廷又差童枢密赍赏赐,已到杭州。听知分兵两路,童枢密转差大将王禀,分赍
赏赐,投昱岭关卢先锋军前去了。童枢密即日便到,亲赍赏赐。”宋江见报,便与
吴用众将都离县二十里迎接。来到县治里,开读圣旨,便将赏赐分给众将。宋江等
参拜童枢密,随即设宴管待。童枢密问道:“征进之间,多听得损折将佐。”宋江
垂泪禀道:“往年跟随赵枢相北征辽虏,兵将全胜,端的不曾折了一个将校。自从
奉敕来征方腊,未离京师,首先去了公孙胜,驾前又留下了数人。进兵渡得江来,
但到一处,必折损数人。近又有八九个将佐,病倒在杭州,存亡未保。前面乌龙岭
厮杀二次,又折了几将。盖因山险水急,难以对阵,急切不能打透关隘。正在忧惶
之际,幸得恩相到此。”童枢密道:“今上天子,多知先锋建立大功,后闻损折将
佐,特差下官引大将王禀、赵谭,前来助阵。已使王禀赍赏往卢先锋处,分俵给散
众将去了。”随唤赵谭与宋江等相见,俱于桐庐县驻扎,饮宴管待已了。

次日,童枢密整点军马,欲要去打乌龙岭关隘。吴用谏道:“恩相未可轻动。且差
燕顺、马麟去溪僻小径去处,寻觅当村土居百姓,问其向道,别求小路,度得关那
边去。两面夹攻,彼此不能相顾,此关唾手可得。”宋江道:“此言极妙。”随即
差遣马麟、燕顺引数十个军健,去村落中寻访百姓问路。去了一日,至晚,引将一
个老儿来见宋江。宋江问道:“这老者是甚人?”马麟道:“这老的是本处土居人
户,都知这里路径溪山。”宋江道:“老者,你可指引我一条路径,过乌龙岭去,
我自重重赏你。”老儿告道:“老汉祖居是此间百姓,累被方腊残害,无处逃躲,
幸得天兵到此,万民有福,再见太平。老汉指引一条小路:过乌龙岭去,便是东管,
取睦州不远。便到北门,却转过西门,便是乌龙岭。”宋江听了大喜,随即叫取银
物,赏了引路老儿,留在寨中,又着人与酒饭管待。

次日,宋江请启童枢密守把桐庐县,自领正偏将一十二员,取小路进发。那十二员
是:花荣、秦明、鲁智深、武松、戴宗、李逵、樊瑞、王英、扈三娘、项充、李衮、
凌振。随行马步军兵一万人数,跟着引路老儿便行。马摘銮铃,军士衔枚疾走。至
小牛岭,已有一伙军兵拦路。宋江便叫李逵、项充、李衮冲杀入去,约有三五百守
路贼兵,都被李逵等杀尽。四更前后,已到东管。本处守把将伍应星,听得宋兵已
透过东管,思量部下只有二千人马,如何迎敌得,当时一哄都走了。径回睦州,报
与祖丞相等知道:“今被宋江军兵私越小路,已透过乌龙岭这边,尽到东管来了。”
祖士远听了大惊,急聚众将商议。宋江已令炮手凌振放起连珠炮。乌龙岭上寨中石
宝等听得大惊,急使指挥白钦引军探时,见宋江旗号,遍天遍地,摆满山林。急退
回岭上寨中,报与石宝等。石宝便道:“既然朝廷不发救兵,我等只坚守关隘,不
要去救。”邓元觉便道:“元帅差矣。如今若不调兵救应睦州,也自由可。倘或内
苑有失,我等亦不能保。你不去时,我自去救应睦州。”石宝苦劝不住,邓元觉点
了五千人马,绰了禅杖,带领夏侯成下岭去了。

且说宋江引兵到了东管,且不去打睦州,先来取乌龙岭关隘,却好正撞着邓元觉。
军马渐近,两军相迎,邓元觉当先出马挑战。花荣看见,便向宋江耳边低低道:“此
人则除如此如此可获。”宋江点头道是,就嘱付了秦明。两将都会意了。秦明首先
出马,便和邓元觉交战。斗到五六合,秦明回马便走,众军各自东西四散。邓元觉
看见秦明输了,倒撇了秦明,径奔来捉宋江。原来花荣已准备了,护持着宋江,只
待邓元觉来得较近,花荣满满地攀着弓,觑得亲切,照面门上飕地一箭。弓开满月,
箭发流星,正中邓元觉面门,坠下马去,被众军杀死,一齐卷杀拢来,南兵大败。
夏侯成抵敌不住,便奔睦州去了。宋兵直杀到乌龙岭上,擂木炮石,打将下来,不
能上去。宋兵却杀转来,先打睦州。

且说祖丞相见首将夏侯成逃来报说:“宋兵已度过东管,杀了邓国师,即日来打睦
州。”祖士远听了,便差人同夏侯成去清溪大内,请娄丞相入朝启奏:“现今宋兵
已从小路透过到东管,前来攻打睦州甚急,乞我王早发军兵救应,迟延必至失陷。”
方腊听了大惊,急宣殿前太尉郑彪,点与一万五千御林军马,星夜去救睦州。郑彪
奏道:“臣领圣旨,乞请天师同行策应,可敌宋江。”方腊准奏,便宣灵应天师包
道乙。当时宣诏天师,直至殿下面君。包道乙打了稽首。方腊传旨道:“今被宋江
兵马,看看侵犯寡人地面,累次陷了城池兵将。即日宋兵俱到睦州,可望天师阐扬
道法,护国救民,以保江山社稷。”包天师奏道:“主上宽心,贫道不才,凭胸中
之学识,仗陛下之洪福,一扫宋江兵马。”方腊大喜,赐坐设宴,管待包道乙。饮
筵罢,辞帝出朝。包天师便和郑彪、夏侯成商议起军。

原来这包道乙祖是金华山中人,幼年出家,学左道之法。向后跟了方腊谋叛造反,
但遇交锋,必使妖法害人,有一口宝剑,号为玄元混天剑,能飞百步取人。协助方
腊,行不仁之事。因此尊为灵应天师。那郑彪原是婺州兰溪县都头出身,自幼使得
枪棒惯熟,遭际方腊,做到殿帅太尉。酷爱道法,礼拜包道乙为师,学得他许多法
术在身,但遇厮杀之处,必有云气相随。因此,人呼为郑魔君。这夏侯成,亦是婺
州山中人,原是猎户出身,惯使钢叉,自来随着祖丞相管领睦州。当日三个在殿帅
府中,商议起军,门吏报道:“有司天太监浦文英来见。”天师问其来故,浦文英
说道:“闻知天师与太尉将军三位提兵去和宋兵战。文英夜观乾象,南方将星,皆
是无光,宋江等将星,尚有一半明朗者。天师此行虽好,只恐不利。何不回奏主上,
商量投拜为上,且解一国之厄。”包天师听了大怒,掣出玄元混天剑,把这浦文英
一剑挥为两段,急动文书,申奏方腊去讫,不在话下。史官有诗曰:
王气东南已渐消,犹凭左道用人妖。
文英既识真天命,何事捐生在伪朝?
当下便遣郑彪为先锋,调前部军马出城前进。包天师为中军,夏侯成为合后,军马
进发,来救睦州。

且说宋江兵将攻打睦州,未见次第,忽闻探马报来,清溪救军到了。宋江听罢,便
差王矮虎、一丈青两个出哨迎敌。夫妻二人带领三千马军,投清溪路上来,正迎着
郑彪,首先出马,便与王矮虎交战。两个更不打话,排开阵势,交马便斗。才到八
九合,只见郑彪口里念念有词,喝声道:“疾!”就头盔顶上,流出一道黑气来。
黑气之中,立着一个金甲天神,手持降魔宝杵,从半空里打将下来。王矮虎看见,
吃了一惊,手忙脚乱,失了枪法,被郑魔君一枪,戳下马去。一丈青看见戳了他丈
夫落马,急舞双刀去救时,郑彪便来交战。略战一合,郑彪回马便走。一丈青要报
丈夫之仇,急赶将来。郑魔君歇住铁枪,舒手去身边锦袋内,摸出一块镀金铜砖,
扭回身,看着一丈青面门上只一砖,打落下马而死。可怜能战佳人,到此一场春梦。
那郑魔君招转军马,却赶宋兵。宋兵大败,回见宋江,诉说王矮虎、一丈青都被郑
魔君戳打伤死,带去军兵,折其大半。宋江听得又折了王矮虎、一丈青,心中大怒,
急点起军马,引了李逵、项充、李衮,带了五千人马,前去迎敌。早见郑魔君军马
已到。宋江怒气填胸,当先出马,大喝郑彪道:“逆贼怎敢杀吾二将!”郑彪便提
枪出马,要战宋江。李逵见了大怒,掣起两把板斧,便飞奔出去,项充、李衮急舞
蛮牌遮护,三个直冲杀入郑彪怀里去。那郑魔君回马便走,三个直赶入南兵阵里去。
宋江恐折了李逵,急招起五千人马,一齐掩杀,南兵四散奔走。宋江且叫鸣金收兵,
两个牌手当得李逵回来,只见四下里乌云罩合,黑气漫天,不分南北东西,白昼如
夜。宋江军马,前无去路。但见:
阴云四合,黑雾漫天。下一阵风雨滂沱,起数声怒雷猛烈。山川震动,高低浑似天
崩;溪涧颠狂,左右却如地陷。悲
悲鬼哭,衮衮神号。定睛不见半分形,满耳惟闻千树响。
宋江军兵,当被郑魔君使妖法,黑暗了天地,迷踪失路,撞到一个去处,黑漫漫不
见一物,本部军兵,自乱起来。宋江仰天叹曰:“莫非吾当死于此地矣!”从巳时
直至未牌,方才黑雾消散,微有些光亮,看见一周遭都是金甲大汉,团团围住。宋
江见了,惊倒在地,口中只称:“乞赐早死!”不敢仰面,耳边只听得风雨之声。
手下众军将士,一个个都伏地受死,只等刀来砍杀。须臾,风雨过处,宋江却见刀
不砍来,有一人来搀宋江,口称:“请起!”宋江抬头仰脸看时,只见面前一个秀
才来扶。看那人时,怎生打扮,但见:
头裹乌纱软角唐巾,身穿白罗圆领凉衫。腰系乌犀金鞓束带,足穿四缝干皂朝靴。
面如傅粉,唇若涂朱。堂堂七尺之躯,楚楚三旬之上。若非上界灵官,定是九天进
士。

宋江见了失惊,起身叙礼,便问秀才高姓大名。那秀才答道:“小生姓邵名俊,土
居于此。今特来报知义士,方十三气数将尽,只在旬日可破。小生多曾与义士出力,
今虽受困,救兵已至,义士知否?”宋江再问道:“先生,方十三气数,何时可获?”
邵秀才把手一推,宋江忽然惊觉,乃是南柯一梦。醒来看时,面前一周遭大汉,却
原来都是松树。宋江大叫军将起来,寻路出去。此时云收雾敛,天朗气清,只听得
松树外面发喊将来。宋江便领起军兵,从里面杀出去时,早望见鲁智深、武松一路
杀来,正与郑彪交手。那包天师在马上,见武松使两口戒刀,步行直取郑彪,包道
乙便向鞘中掣出那口玄天混元剑来,从空飞下,正砍中武松左臂,血晕倒了。却得
鲁智深一条禅杖,忿力打入去,救得武松时,已自左臂砍得伶仃将断,却夺得他那
口混元剑。武松醒来,看见左臂已折,伶仃将断,一发自把戒刀割断了。宋江先叫
军校扶送回寨将息。鲁智深却杀入后阵去,正遇着夏侯成交战。两个斗了数合,夏
侯成败走,鲁智深一条禅杖,直打入去,南军四散。夏侯成便望山林中奔走。鲁智
深不舍,赶入深山里去了。

且说郑魔君那厮,又引兵赶将来,宋军阵内,李逵、项充、李衮三个见了,便舞起
蛮牌、飞刀、标枪、板斧,一齐冲杀入去。那郑魔君迎敌不过,越岭渡溪而走。三
个不识路径,只要立功,死命赶过溪去,紧追郑彪。溪西岸边,抢出三千军来,截
断宋兵。项充急回时,早被岸边两将拦住。便叫李逵、李衮时,已过溪赶郑彪去了。
不想前面溪涧又深,李衮先一交跌翻在溪里,被南军乱箭射死。项充急钻下岸来,
又被绳索绊翻,却待要挣扎,众军乱上,剁做肉泥。可怜李衮、项充到此,英雄怎
使!只有李逵独自一个,赶入深山里去了。溪边军马随后袭将去,未经半里,背后
喊声振起,却是花荣、秦明、樊瑞三将,引军来救,杀散南军,赶入深山,救得李
逵回来,只不见了鲁智深。众将齐来参见宋江,诉说追赶郑魔君,过溪厮杀,折了
项充、李衮,止救了李逵回来。宋江听罢,痛哭不止。整点军兵,折其一停。又不
见了鲁智深,武松已折了左臂。

宋江正哭之间,探马报道:“军师吴用和关胜、李应、朱仝、燕顺、马麟,提一万
军兵,从水路到来。”宋江迎见吴用等,便问来情。吴用答道:“童枢密自有随行
军马,并大将王禀、赵谭,都督刘光世又有军马,已到乌龙岭下。只留下吕方、郭
盛、裴宣、蒋敬、蔡福、蔡庆、杜兴、郁保四,并水军头领李俊、阮小五、阮小七、
童威、童猛等十三人,其余都跟吴用到此策应。”宋江诉说:“折了将佐,武松已
成了废人,鲁智深又不知去向,不由我不伤感。”吴用劝道:“兄长且宜开怀,即
日正是擒捉方腊之时,只以国家大事为重,不可忧损贵体。”宋江指着许多松树,
说梦中之事,与军师知道。吴用道:“既然有此灵验之梦,莫非此处坊隅庙宇,有
灵显之神,故来护佑兄长。”宋江道:“军师所见极当,就与足下进山寻访。”吴
用当与宋江信步行入山林,未及半箭之地,松树林中,早见一所庙宇,金书牌额,
上写:“乌龙神庙。”宋江、吴用入庙上殿看时,吃了一惊,殿上塑的龙君圣像,
正和梦中见者无异。宋江再拜恳谢道:“多蒙龙君神圣救护之恩,未能报谢,望乞
灵神助威。若平复了方腊,敬当一力申奏朝廷,重建庙宇,加封圣号。”宋江、吴
用拜罢,下阶看那石碑时,神乃唐朝一进士,姓邵名俊,应举不第,坠江而死,天
帝怜其忠直,赐作龙神。本处人民祈风得风,祈雨得雨,以此建立庙宇,四时享祭。
宋江看了,随即叫取乌猪白羊,祭祀已毕。出庙来再看备细,见周遭松树显化,可
谓异事。直至如今,严州北门外,有乌龙大王庙,亦名万松林。古迹尚存,有诗为
证:
忠心一点鬼神知,暗里维持信有之。
欲识龙君真姓字,万松林下读残碑。
且说宋江谢了龙君庇佑之恩,出庙上马,回到中军寨内,便与吴用商议打睦州之策。
坐至半夜,宋江觉道神思困倦,伏几而卧,只闻一人报曰:“有邵秀才相访。”宋
江急忙起身,出帐迎接时,只见邵龙君长揖宋江道:“昨日若非小生救护,义士已
被包道乙作起邪法,松树化人,擒获足下矣。适间深感祭奠之礼,特来致谢,就行
报知睦州来日可破,方十三旬日可擒。”宋江正待邀请入帐再问间,忽被风声一搅,
撒然觉来,又是一梦。

宋江急请军师圆梦,说知其事。吴用道:“既是龙君如此显灵,来日便可进兵,攻
打睦州。”宋江道:“言之极当。”至天明,传下军令,点起大队人马,攻取睦州。
便差燕顺、马麟守住乌龙岭这条大路,却调关胜、花荣、秦明、朱仝四员正将,当
先进兵,来取睦州,便望北门攻打。却令凌振施放九厢子母等火炮,直打入城去。
那火炮飞将起去,震的天崩地动,岳撼山摇,城中军马,惊得魂消魄丧,不杀自乱。
且说包天师、郑魔君后军,已被鲁智深杀散,追赶夏侯成,不知下落。那时已将军
马退入城中屯驻,却和右丞相祖士远、参政沈寿、佥书桓逸、元帅谭高、守将伍应
星等商议:“宋兵已至,何以解救?”祖士远道:“自古兵临城下,将至濠边,若
不死战,何以解之?打破城池,必被擒获,事在危急,尽须向前!”当下郑魔君引
着谭高、伍应星并牙将十数员,领精兵一万,开放城门,与宋江对敌。宋江教把军
马略退半箭之地,让他军马出城摆列。

那包天师拿着把交椅,坐在城头上,祖丞相、沈参政并桓佥书皆坐在敌楼上看。郑
魔君便挺枪跃马出阵,宋江阵上大刀关胜,出马舞刀,来战郑彪。二将交马,斗不
数合,那郑彪如何敌得关胜,只办得架隔遮拦,左右躲闪。这包道乙正在城头上看
了,便作妖法,口中念念有词,喝声道:“疾!”念着那助咒法,吹口气去,郑魔
君头上滚出一道黑气,黑气中间显出一尊金甲神人,手提降魔宝杵,望空打将下来。
南军队里,荡起昏邓邓黑云来。宋江见了,便唤混世魔王樊瑞来看,急令作法,并
自念天书上回风破暗的密咒秘诀。只见关胜头盔上,早卷起一道白云,白云之中,
也显出一尊神将,红发青脸,碧眼撩牙,骑一条乌龙,手执铁锤,去战郑魔君头上
那尊金甲神人,下面两军呐喊,二将交锋。战无数合,只见上面那骑乌龙的天将,
战退了金甲神人。下面关胜,一刀砍了郑魔君于马下。

包道乙见宋军中风起雷响,急待起身时,被凌振放起一个轰天炮,一个火弹子,正
打中包天师,头和身躯,击得粉碎,南兵大败。乘势杀入睦州。朱仝把元帅谭高一
枪戳在马下,李应飞刀杀死守将伍应星。睦州城下,见一火炮打中了包天师身躯,
南军都滚下城去了。宋江军马,已杀入城,众将一发向前,生擒了祖丞相、沈参政、
桓佥书,其余牙将,不问姓名,俱被宋兵杀死。

宋江等入城,先把火烧了方腊行宫,所有金帛,就赏与了三军众将,便出榜文安抚
了百姓。尚兀自点军未了,探马飞报将来:“西门乌龙岭上,马麟被白钦一标枪标
下去,石宝赶上,复了一刀,把马麟剁做两段。燕顺见了,便向前来战时,又被石
宝那厮一流星锤打死。石宝得胜,即日引军乘势杀来。”宋江听得又折了燕顺、马
麟,扼腕痛哭不尽。急差关胜、花荣、秦明、朱仝四员正将迎敌石宝、白钦,就要
取乌龙岭关隘。不是这四员将来乌龙岭厮杀,有分教:清溪县里,削平哨聚贼兵;
帮源洞中,活捉草头天子。直教宋江等:名标青史千年在,功播清时万古传。
毕竟宋江等怎地迎敌,且听下回分解。