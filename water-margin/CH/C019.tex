\chapter{林冲水寨大并火~晁盖梁山小夺泊}

话说当下何观察领了知府台旨下厅来,随即到机密房里,与众人商议。众多做
公的道:“若说这个石碣村湖荡,紧靠着梁山泊,都是茫茫荡荡,芦苇水港。若不
得大队官军,舟船人马,谁敢去那里捕捉贼人?”何涛听罢,说道:“这一论也是。”
再到厅上禀复府尹道:“原来这石碣村湖泊,正傍着梁山水泊,周围尽是深港水汊,
芦苇草荡。闲常时也兀自劫了人,莫说如今又添了那一伙强人在里面。若不起得大
队人马,如何敢去那里捕获得人?”府尹道:“既是如此说时,再差一员了得事的
捕盗巡检,点与五百官兵人马,和你一处去缉捕。”何观察领了台旨,再回机密房
来,唤集这众多做公的,整选了五百余人,各各自去准备什物器械。次日,那捕盗
巡检领了济州府帖文,与同何观察两个,点起五百军兵,同众多做公的,一齐奔石
碣村来。

且说晁盖、公孙胜,自从把火烧了庄院,带同十数个庄客,来到石碣村,半路
上撞见三阮弟兄,各执器械,却来接应到家,七个人都在阮小五庄上。那时阮小二
已把老小搬入湖泊里,七人商议要去投梁山泊一事。吴用道:“现今李家道口有那
旱地忽律朱贵在那里开酒店,招接四方好汉。但要入伙的,须是先投奔他。我们如
今安排了船只,把一应的物件装在船里,将些人情送与他引进。”

大家正在那里商议投奔梁山泊,只见几个打鱼的来报道:“官军人马,飞奔村
里来也!”晁盖便起身叫道:“这厮们赶来,我等休走!”阮小二道:“不妨!我
自对付他。叫那厮大半下水里去死,小半都搠杀他。”公孙胜道:“休慌!且看贫
道的本事!”晁盖道:“刘唐兄弟,你和学究先生,且把财赋老小,装载船里,径
撑去李家道口左侧相等;我们看些头势,随后便到。”阮小二选两只棹船,把娘和
老小,家中财赋,都装下船里。吴用、刘唐各押着一只,叫七八个伴当摇了船,先
到李家道口去等;又分付阮小五、阮小七撑驾小船,如此迎敌。两个各掉船去了。

且说何涛并捕盗巡检,带领官兵,渐近石碣村,但见河埠有船,尽数夺了;便
使会水的官兵,且下船里进发;岸上人马,船骑相迎,水陆并进。到阮小二家,一
齐呐喊,人兵并起,扑将入去,早是一所空房,里面只有些粗重家火。何涛道:“且
去拿几家附近渔户。”问时,说道:“他的两个兄弟阮小五、阮小七,都在湖泊里
住,非船不能去。”何涛与巡检商议道:“这湖泊里港汊又多,路径甚杂,抑且水
荡坡塘,不知深浅。若是四分五落去捉时,又怕中了这贼人奸计。我们把马匹都教
人看守在这村里,一发都下船里去。”当时捕盗巡检并何观察,一同做公的人等,
都下了船。那时捉的船,非止百十只,也有撑的,亦有摇的,一齐都望阮小五打鱼
庄上来。

行不到五六里水面,只听得芦苇中间,有人嘲歌。众人且住了船听时,那歌道:
打鱼一世蓼儿洼,不种青苗不种麻。
酷吏赃官都杀尽,忠心报答赵官家。
何观察并众人听了,尽吃一惊。只见远远地一个人,独棹一只小船儿唱将来。有认
得的指道:“这个便是阮小五。”何涛把手一招,众人并力向前,各执器械,挺着
迎将去。只见阮小五大笑骂道:“你这等虐害百姓的贼官,直如此大胆!敢来引老
爷做甚么!却不是来捋虎须!”何涛背后有会射弓箭的,搭上箭,曳满弓,一齐放
箭。阮小五见放箭来,拿着桦楸,翻筋斗钻下水里去。众人赶到跟前,拿个空。

又行不到两条港汊,只听得芦花荡里打唿哨,众人把船摆开,见前面两个人棹
着一只船来。船头上立着一个人,头戴青箬笠,身披绿蓑衣,手里拈着条笔管枪,
口里也唱着道:
老爷生长石碣村,禀性生来要杀人。
先斩何涛巡检首,京师献与赵王君。
何观察并众人听了,又吃一惊。一齐看时,前面那个人拈着枪,唱着歌,背后这个
摇着橹。有认得的说道:“这个正是阮小七。”何涛喝道:“众人并力向前,先拿
住这个贼!休教走了!”阮小七听得笑道:“泼贼!”便把枪只一点,那船便使转
来,望小港里串着走。众人发着喊,赶将去。这阮小七和那摇船的,飞也似摇着橹,
口里打着唿哨,串着小港汊中只顾走。

众官兵赶来赶去,看见那水港窄狭了,何涛道:“且住!把船且泊了,都傍岸
边。”上岸看时,只见茫茫荡荡,都是芦苇,正不见一些旱路。何涛心内疑惑,却
商议不定,便问那当村住的人,说道:“小人们虽是在此居住,也不知道这里有许
多去处。”何涛便教划着两只小船,船上各带三两个做公的,去前面探路。去了两
个时辰有余,不见回报。何涛道:“这厮们好不了事!”再差五个做公的,又划两
只船去探路。这几个做公的,划了两只船,又去了一个多时辰,并不见些回报。何
涛道:“这几个都是久惯做公的,四清六活的人,却怎地也不晓事,如何不着一只
船转来回报?不想这些带来的官兵,人人亦不知颠倒!”天色又看看晚了,何涛思
想:“在此不着边际,怎生奈何?我须用自去走一遭。”拣一只疾快小船,选了几
个老郎做公的,各拿了器械,桨起五六把桦楫,何涛坐在船头上,望这个芦苇港里
荡将去。

那时已是日没沉西,划得船开,约行了五六里水面,看见侧边岸上一个人,提
着把锄头走将来,何涛问道:“兀那汉子,你是甚人?这里是甚么去处?”那人应
道:“我是这村里庄家。这里唤做断头沟,没路了。”何涛道:“你曾见两只船过
来么?”那人道:“不是来捉阮小五的?”何涛道:“你怎地知得是来捉阮小五的?”
那人道:“他们只在前面乌林里厮打。”何涛道:“离这里还有多少路?”那人道:
“只在前面望得见便是。”何涛听得,便叫拢船,前去接应,便差两个做公的,拿
了叉上岸来。只见那汉提起锄头来,手到,把这两个做公的,一锄头一个,翻筋
斗都打下水里去。何涛见了吃一惊,急跳起身来时,却待奔上岸,只见那只船忽地
搪将开去,水底下钻起一个人来,把何涛两腿只一扯,扑通地倒撞下水里去。那几
个船里的却待要走,被这提锄头的赶将上船来,一锄头一个,排头打下去,脑浆也
打出来。这何涛被水底下这人倒拖上岸来,就解下他的膊来捆了。看水底下这人,
却是阮小七;岸上提锄头的那汉,便是阮小二。

弟兄两个,看着何涛骂道:“老爷弟兄三个,从来只爱杀人放火。量你这厮,
直得甚么!你如何大胆,特地引着官兵来捉我们!”何涛道:“好汉!小人奉上命差
遣,盖不由己。小人怎敢大胆,要来捉好汉?望好汉可怜见家中有个八十岁的老娘,
无人养赡,望乞饶恕性命则个!”阮家弟兄道:“且把他来捆做个粽子,撇在船舱
里。”把那几个尸首,都撺去水里去了。个个胡哨一声,芦苇丛中钻出四五个打鱼
的人来,都上了船。阮小二、阮小七各驾了一只船出来。

且说这捕盗巡检,领着官兵,都在那船里说道:“何观察他道做公的不了事,
自去探路,也去了许多时,不见回来。”那时正是初更左右,星光满天。众人都在
船上歇凉。忽然只见起一阵怪风,但见:

飞沙走石,卷水摇天。黑漫漫堆起乌云,昏邓邓催来急雨。倾翻荷叶,满波心
翠盖交加;摆动芦花,绕湖面白旗缭乱。吹折昆仑山顶树,唤醒东海老龙君。

那一阵怪风从背后吹将来,吹得众人掩面大惊,只叫得苦,把那缆船索都刮断
了。正没摆布处,只听得后面胡哨响;迎着风看时,只见芦花侧畔,射出一派火光
来。众人道:“今番却休了!”那大船小船,约有四五十只,正被这大风刮得你撞
我磕,捉摸不住,那火光却早来到面前。原来都是一丛小船,两只价帮住,上面满
满堆着芦苇柴草,刮刮杂杂烧着,乘着顺风直冲将来。那四五十只官船,屯塞做一
块,港汊又狭,又没回避处。那头等大船也有十数只,却被他火船推来,钻在大船
队里一烧。水底下原来又有人扶助着船烧将来,烧得大船上官兵都跳上岸来逃命奔
走,不想四边尽是芦苇野港,又没旱路;只见岸上芦苇又刮刮杂杂,也烧将起来。
那捕盗官兵,两头没处走。风又紧,火又猛,众官兵只得钻去,都奔烂泥里立地。

火光丛中,只见一只小快船,船尾上一个摇着船,船头上坐着一个先生,手里
明晃晃地拿着一口宝剑,口里喝道:“休教走了一个!”众兵都在烂泥里慌做一堆。
说犹未了,只见芦苇东岸,两个人引着四五个打鱼的,都手里明晃晃拿着刀枪走来。
这边芦苇西岸,又是两个人,也引着四五个打鱼的,手里也明晃晃拿着飞鱼钩走来。
东西两岸,四个好汉并这伙人,一齐动手,排头儿搠将来。无移时,把许多官兵都
搠死在烂泥里。

东岸两个,是晁盖、阮小五;西岸两个,是阮小二、阮小七;船上那个先生,
便是祭风的公孙胜。五位好汉,引着十数个打鱼的庄家,把这伙官兵,都搠死在芦
苇荡里。单单只剩得一个何观察,捆做粽子也似,丢在船舱里。阮小二提将上岸来,
指着骂道:“你这厮,是济州一个诈害百姓的蠢虫!我本待把你碎尸万段,却要你
回去对那济州府管事的贼驴说:俺这石碣村阮氏三雄,东溪村天王晁盖,都不是好
撩拨的!我也不来你城里借粮,他也休要来我这村中讨死!倘或正眼儿觑着,休道你
是一个小小州尹,也莫说蔡太师差干人来要拿我们,便是蔡京亲自来时,我也搠他
三二十个透明的窟窿。俺们放你回去,休得再来!传与你的那个鸟官人,教他休要
讨死!这里没大路,我着兄弟送你出路口去。”当时阮小七把一只小快船载了何涛,
直送他到大路口,喝道:“这里一直去,便有寻路处。别的众人都杀了,难道只恁
地好好放了你去,也吃你那州尹贼驴笑!且请下你两个耳朵来做表证!”阮小七身
边拔起尖刀,把何观察两个耳朵割下来,鲜血淋漓,插了刀,解了膊,放上岸去。
诗曰:
官兵尽付断头沟,要放何涛不便休。
留着耳朵听说话,旋将驴耳代驴头。
何涛得了性命,自寻路回济州去了。

且说晁盖、公孙胜和阮家三弟兄,并十数个打鱼的,一发都驾了五七只小船,
离了石碣村湖泊,径投李家道口来。到得那里,相寻着吴用、刘唐船只,合做一处。
吴用问起拒敌官兵一事,晁盖备细说了。吴用众人大喜。整顿船只齐了,一同来到
旱地忽律朱贵酒店里来相投。朱贵见了许多人来说投托入伙,慌忙迎接。吴用将来
历实说与朱贵听了,大喜,逐一都相见了,请入厅上坐定,忙叫酒保安排分例酒来,
管待众人。随即取出一张皮靶弓来,搭上一枝响箭,望着那对港芦苇中射去。响箭
到处,早见有小喽罗摇出一只船来。朱贵急写了一封书呈,备细写众豪杰入伙姓名
人数,先付与小喽罗赍了,教去寨里报知;一面又杀羊管待众好汉。

过了一夜,次日早起,朱贵唤一只大船,请众多好汉下船,就同带了晁盖等来
的船只,一齐望山寨里来。行了多时,早来到一处水口,只听的岸上鼓响锣鸣。晁
盖看时,只见七八个小喽罗,划出四只哨船来,见了朱贵,都声了喏,自依旧先去
了。

再说一行人来到金沙滩上岸,便留老小船只并打鱼的人在此等候。又见数十个
小喽罗,下山来接引到关上。王伦领着一班头领,出关迎接。晁盖等慌忙施礼,王
伦答礼道:“小可王伦,久闻晁天王大名,如雷灌耳。今日且喜光临草寨。”晁盖
道:“晁某是个不读书史的人,甚是粗卤,今日事在藏拙,甘心与头领帐下做一小
卒,不弃幸甚。”王伦道:“休如此说,且请到小寨,再有计议。”一行从人,都
跟着两个头领上山来。到得大寨聚义厅上,王伦再三谦让晁盖一行人上阶。晁盖等
七人,在右边一字儿立下;王伦与众头领,在左边一字儿立下。一个个都讲礼罢,
分宾主对席坐下。王伦唤阶下众小头目声喏已毕,一壁厢动起山寨中鼓乐。先叫小
头目去山下管待来的从人,关下另有客馆安歇。诗曰:
入伙分明是一群,相留意气便须亲。
如何待彼为宾客,只恐身难作主人。

且说山寨里宰了两头黄牛,十个羊,五个猪,大吹大擂筵席。众头领饮酒中间,
晁盖把胸中之事,从头至尾,都告诉王伦等众位。王伦听罢,骇然了半晌,心内踌
躇,做声不得,自己沉吟,虚应答筵宴。至晚席散,众头领送晁盖等众人关下客馆
内安歇,自有来的人伏侍。

晁盖心中欢喜,对吴用等六人说道:“我们造下这等迷天大罪,那里去安身?
不是这王头领如此错爱,我等皆已失所,此恩不可忘报!”吴用只是冷笑。晁盖道:
“先生何故只是冷笑?有事可以通知。”吴用道:“兄长性直,你道王伦肯收留我
们?兄长不看他的心,只观他的颜色动静规模。”晁盖道:“观他颜色怎地?”吴
用道:“兄长不见他早间席上与兄长说话,倒有交情;次后因兄长说出杀了许多官
兵捕盗巡检,放了何涛,阮氏三雄如此豪杰,他便有些颜色变了。虽是口中应答,
动静规模,心里好生不然。若是他有心收留我们,只就早上便议定了座位。杜迁、
宋万,这两个自是粗卤的人,待客之事,如何省得?只有林冲那人,原是京师禁军
教头,大郡的人,诸事晓得,今不得已,坐了第四位。早间见林冲看王伦答应兄长
模样,他自便有些不平之气,频频把眼瞅这王伦,心内自已踌躇。我看这人,倒有
顾盼之心,只是不得已。小生略放片言,教他本寨自相火并。”晁盖道:“全仗先
生妙策良谋,可以容身。”

当夜七人安歇了。次早天明,只见人报道:“林教头相访。”吴用便对晁盖道:
“这人来相探,中俺计了。”七个人慌忙起来迎接,邀请林冲入到客馆里面。吴用
向前称谢道:“夜来重蒙恩赐,拜扰不当。”林冲道:“小可有失恭敬。虽有奉承
之心,奈缘不在其位,望乞恕罪。”吴学究道:“我等虽是不才,非为草木,岂不
见头领错爱之心,顾盼之意,感恩不浅。”晁盖再三谦让林冲上坐,林冲那里肯,
推晁盖上首坐了,林冲便在下首坐定。吴用等六人一带坐下。晁盖道:“久闻教头
大名,不想今日得会。”林冲道:“小人旧在东京时,与朋友有礼节,不曾有误。
虽然今日能够得见尊颜,不得遂平生之愿,特地径来陪话。”晁盖称谢道:“深感
厚意。”

吴用便动问道:“小生旧日久闻头领在东京时,十分豪杰,不知缘何与高俅不
睦,致被陷害。后闻在沧州,亦被火烧了大军草料场,又是他的计策。向后不知谁
荐头领上山?”林冲道:“若说高俅这贼陷害一节,但提起,毛发植立!又不能报
得此仇!来此容身,皆是柴大官人举荐到此。”吴用道:“柴大官人,莫非是江湖
上人称为小旋风柴进的么?”林冲道:“正是此人。”晁盖道:“小可多闻人说柴
大官人仗义疏财,接纳四方豪杰,说是大周皇帝嫡派子孙,如何能够会他一面也好。”

吴用又对林冲道:“据这柴大官人,名闻寰海,声播天下的人,教头若非武艺
超群,他如何肯荐上山?非是吴用过称,理合王伦让这第一位头领坐。此天下之公
论,也不负了柴大官人之书信。”林冲道:“承先生高谈,只因小可犯下大罪,投
奔柴大官人,非他不留林冲,诚恐负累他不便,自愿上山。不想今日去住无门!非
在位次低微,且王伦只心术不定,语言不准,难以相聚。”吴用道:“王头领待人
接物,一团和气,如何心地倒恁窄狭?”林冲道:“今日山寨,天幸得众多豪杰到
此,相扶相助,似锦上添花,如旱苗得雨。此人只怀妒贤嫉能之心,但恐众豪杰势
力相压。夜来因见兄长所说众位杀死官兵一节,他便有些不然,就怀不肯相留的模
样,以此请众豪杰来关下安歇。”吴用便道:“既然王头领有这般之心,我等休要
待他发付,自投别处去便了。”林冲道:“众豪杰休生见外之心,林冲自有分晓。
小可只恐众豪杰生退去之意,特来早早说知。今日看他如何相待。若这厮语言有理,
不似昨日,万事罢论;倘若这厮今朝有半句话参差时,尽在林冲身上。”晁盖道:
“头领如此错爱,俺兄弟皆感厚恩。”吴用便道:“头领为我弟兄面上,倒教头领
与旧弟兄分颜。若是可容即容,不可容时,小生等登时告退。”林冲道:“先生差
矣!古人有言:‘惺惺惜惺惺,好汉惜好汉。’量这一个泼男女,腌畜生,终作
何用!众豪杰且请宽心。”林冲起身别了众人,说道:“少间相会。”众人相送出
来,林冲自上山去了。正是:
如何此处不留人,休言自有留人处。
应留人者怕人留,身苦难留留客住。
当日没多时,只见小喽罗到来相请,说道:“今日山寨里头领,相请众好汉,去山
南水寨亭上筵会。”晁盖道:“上复头领,少间便到。”小喽罗去了,晁盖问吴用
道:“先生,此一会如何?”吴学究笑道:“兄长放心,此一会倒有分做山寨之主。
今日林教头必然有火并王伦之意。他若有些心懒,小生凭着三寸不烂之舌,不由他
不火并。兄长身边各藏了暗器,只看小生把手来拈须为号,兄长便可协力。”晁盖
等众人暗喜。

辰牌已后,三四次人来催请。晁盖和众头领身边各各带了器械,暗藏在身上,
结束得端正,却来赴席。只见宋万亲自骑马,又来相请,小喽罗抬过七乘山轿,七
个人都上轿子,一径投南山水寨里来。到得山南看时,端的景物非常,直到寨后水
亭子前下了轿,王伦、杜迁、林冲、朱贵,都出来相接,邀请到那水亭子上,分宾
主坐定。看那水亭一遭景致时,但见:

四面水帘高卷,周回花压朱阑。满目香风,万朵芙蓉铺绿
水;迎眸翠色,千枝荷叶绕芳塘。华檐外阴阴柳影,锁窗前细细松声。江山秀气满
亭台,豪杰一群来聚会。

当下王伦与四个头领——杜迁、宋万、林冲、朱贵——坐在左边主位上;晁盖
与六个好汉——吴用、公孙胜、刘唐、三阮——坐在右边客席。阶下小喽罗轮番把
盏。酒至数巡,食供两次,晁盖和王伦盘话。但提起聚义一事,王伦便把闲话支吾
开去。吴用把眼来看林冲时,只见林冲侧坐交椅上,把眼瞅王伦身上。

看看饮酒至午后,王伦回头叫小喽罗取来。三四个人去不多时,只见一人捧个
大盘子,里放着五锭大银。王伦便起身把盏,对晁盖说道:“感蒙众豪杰到此聚义,
只恨敝山小寨,是一洼之水,如何安得许多真龙?聊备些小薄礼,万望笑留,烦投
大寨歇马,小可使人亲到麾下纳降。”晁盖道:“小子久闻大山招贤纳士,一径地
特来投托入伙,若是不能相容,我等众人自行告退。重蒙所赐白金,决不敢领。非
敢自夸丰富,小可聊有些盘缠使用。速请纳回厚礼,只此告别。”王伦道:“何故
推却?非是敝山不纳众位豪杰,奈缘只为粮少房稀,恐日后误了足下,众位面皮不
好,因此不敢相留。”说言未了,只见林冲双眉剔起,两眼圆睁,坐在交椅上大喝
道:“你前番我上山来时,也推道粮少房稀。今日晁兄与众豪杰到此山寨,你又发
出这等言语来,是何道理?”吴用便说道:“头领息怒。自是我等来的不是,倒坏
了你山寨情分。今日王头领以礼发付我们下山,送与盘缠,又不曾热赶将去,请头
领息怒,我等自去罢休。”林冲道:“这是笑里藏刀言清行浊的人!我其实今日放
他不过!”王伦喝道:“你看这畜生!又不醉了,倒把言语来伤触我,却不是反失
上下!”林冲大怒道:“量你是个落第穷儒,胸中又没文学,怎做得山寨之主!”
吴用便道:“晁兄,只因我等上山相投,反坏了头领面皮。只今办了船只,便当告
退。”

晁盖等七人便起身,要下亭子。王伦留道:“且请席终了去。”林冲把桌子只
一脚,踢在一边;抢起身来,衣襟底下掣出一把明晃晃刀来,的火杂杂。吴用便
把手将髭须一摸,晁盖、刘唐便上亭子来,虚拦住王伦叫道:“不要火并!”吴用
一手扯住林冲,便道:“头领不可造次!”公孙胜假意劝道:“休为我等坏了大义。”
阮小二便去帮住杜迁,阮小五便帮住宋万,阮小七帮住朱贵,吓得小喽罗们目瞪口
呆。

林冲拿住王伦骂道:“你是一个村野穷儒,亏了杜迁得到这里。柴大官人这等
资助你,给盘缠,与你相交,举荐我来,尚且许多推却。今日众豪杰特来相聚,
又要发付他下山去。这梁山泊便是你的!你这嫉贤妒能的贼,不杀了,要你何用!你
也无大量大才,也做不得山寨之主!”杜迁、宋万、朱贵本待要向前来劝,被这几
个紧紧帮着,那里敢动。王伦那时也要寻路走,却被晁盖、刘唐两个拦住。王伦见
头势不好,口里叫道:“我的心腹都在那里?”虽有几个身边知心腹的人,本待要
来救,见了林冲这般凶猛头势,谁敢向前?林冲即时拿住王伦,又骂了一顿,去心
窝里只一刀,察地搠倒在亭上。可怜王伦做了多年寨主,今日死在林冲之手,正
应古人言:“量大福也大,机深祸亦深。”有诗为证:
独据梁山志可羞,嫉贤傲士少宽柔。
只将寨主为身有,却把群英作寇仇。
酒席欢时生杀气,杯盘响处落人头。
胸怀褊狭真堪恨,不肯留贤命不留。

晁盖见杀了王伦,各掣刀在手。林冲早把王伦首级割下来,提在手里,吓得那
杜迁、宋万、朱贵都跪下说道:“愿随哥哥执鞭坠!”晁盖等慌忙扶起三人来。
吴用就血泊里曳过头把交椅来,便纳林冲坐地,叫道:“如有不伏者,将王伦为例!
今日扶林教头为山寨之主。”林冲大叫道:“先生差矣!我今日只为众豪杰义气为
重上头,火并了这不仁之贼,实无心要谋此位。今日吴兄却让此第一位与林冲坐,
岂不惹天下英雄耻笑?若欲相逼,宁死而已!弟有片言,不知众位肯依我么?”众人
道:“头领所言,谁敢不依?愿闻其言。”林冲言无数句,话不一席,有分教:断
金亭上,招多少断金之人;聚义厅前,开几番聚义之会。正是:替天行道人将至,
仗义疏财汉便来。

毕竟林冲对吴用说出甚言语来,且听下回分解。