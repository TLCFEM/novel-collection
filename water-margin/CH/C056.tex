\chapter{吴用使时迁盗甲~汤隆赚徐宁上山}

话说当时汤隆对众头领说道:“小可是祖代打造军器为生。先父因此艺上,遭
际老种经略相公,得做延安知寨。先朝曾用这连环甲马取胜。欲破阵时,须用钩镰
枪可破。汤隆祖传已有画样在此,若要打造,便可下手。汤隆虽是会打,却不会使。
若要会使的人,只除非是我那个姑舅哥哥。会使这钩镰枪法,只有他一个教头,他
家祖传习学,不教外人。或是马上,或是步行,都有法则,端的使动,神出鬼没!”
说言未了,林冲问道:“莫不是现做金枪班教师徐宁?”汤隆应道:“正是此人。”
林冲道:“你不说起,我也忘了。这徐宁的金枪法、钩镰枪法,端的是天下独步。
在京师时,多与我相会,较量武艺,彼此相敬相爱,只是如何能够得他上山来?”

汤隆道:“徐宁先祖留下一件宝贝,世上无对,乃是镇家之宝。汤隆比时,曾
随先父知寨往东京视探姑姑时,多曾见来。是一副雁翎砌就圈金甲。这一副甲,披
在身上,又轻又稳,刀剑箭矢,急不能透,人都唤做赛唐猊。多有贵公子要求一见,
造次不肯与人看。这副甲,是他的性命,用一个皮匣子盛着,直挂在卧房中梁上。
若是先对付得他这副甲来时,不由他不到这里。”吴用道:“若是如此,何难之有?
放着有高手弟兄在此,今次却用着鼓上蚤时迁去走一遭。”时迁随即应道:“只怕
无此一物在彼,若端的有时,好歹定要取了来。”汤隆道:“你若盗得甲来,我便
包办赚他上山。”

宋江问道:“你如何去赚他上山?”汤隆去宋江耳边低低说了数句,宋江笑道:
“此计大妙!”吴学究道:“再用得三个人,同上东京走一遭。一个到京收买烟火、
药料,并炮内用的药材;两个去取凌统领家老小。”彭玘见了,便起身禀道:“若
得一人到颍州取得小弟家眷上山,实拜成全之德。”宋江便道:“团练放心。便请
二位修书,小可自教人去。”便唤杨林,可将金银书信,带领伴当,前往颍州取彭
玘将军老小;薛永扮作使枪棒卖药的,往东京取凌统领老小;李云扮作客商,同往
东京收买烟火、药料等物;乐和随汤隆同行,又挈薛永往来作伴。一面先送时迁下
山去了。次后,且叫汤隆打起一把钩镰枪做样,却教雷横提调监督,原来雷横祖上
也是打铁出身。再说汤隆打起钩镰枪样子,教山寨里打军器的照着样子打造,自有
雷横提调监督,不在话下。大寨做个送路筵席,当下杨林、薛永、李云、乐和、汤
隆,辞别下山去了。次日又送戴宗下山,往来探听事情。这段话一时难尽。

这里且说时迁离了梁山泊,身边藏了暗器,诸般行头,在路迤逦来到东京,投
个客店安下了。次日踅进城来,寻问金枪班教师徐宁家,有人指点道:“入得班门
里,靠东第五家黑角子门便是。”时迁转入班门里,先看了前门,次后踅来,相了
后门,见是一带高墙,墙里望见两间小巧楼屋,侧首却是一根戗柱。时迁看了一回,
又去街坊问道:“徐教师在家里么?”人应道:“敢在内里随直未归。”时迁又问
道:“不知几时归?”人应道:“直到晚方归来,五更便去内里随班。”时迁叫了
相扰,且回客店里来,取了行头,藏在身边,分付店小二道:“我今夜多敢是不归,
照管房中则个。”小二道:“但放心自去,并不差池。”

时迁再入到城里,买了些晚饭吃了,却踅到金枪班徐宁家,左右看时,没一个
好安身去处。看看天色黑了,时迁捵入班门里面。是夜,寒冬天色,却无月光。时
迁看见土地庙后一株大柏树,便把两只腿夹定,一节节爬将上去树头顶,骑马儿坐
在枝柯上。悄悄望时,只见徐宁归来,望家里去了。又见班里两个人提着灯笼出来
关门,把一把锁锁了,各自归家去了。早听得谯楼禁鼓,却转初更。云寒星斗无光,
露散霜花渐白。时迁见班里静悄悄地,却从树上溜将下来,踅到徐宁后门边,从墙
上下来,不费半点气力,爬将过去,看里面时,却是个小小院子。时迁伏在厨房外
张时,见厨房下灯明,两个娅,兀自收拾未了。时迁却从戗柱上盘到膊风板边,
伏做一块儿,张那楼上时,见那金枪手徐宁和娘子对坐炉边向火,怀里抱着一个六
七岁孩儿。时迁看那卧房里时,见梁上果然有个大皮匣拴在上面,房门口挂着一副
弓箭,一口腰刀,衣架上挂着各色衣服。徐宁口里叫道:“梅香,你来与我折了衣
服。”下面一个娅上来,就侧首春台上,先折了一领紫绣圆领。又折一领官绿衬
里袄子,并下面五色花绣踢串,一个护项彩色锦帕,一条红绿结子,并手帕一包。
另用一个小黄帕儿,包着一条双獭尾荔枝金带,也放在包袱内,把来安在烘笼上。
——时迁多看在眼里。约至二更以后,徐宁收拾上床,娘子问道:“明日随直也不?”
徐宁道:“明日正是天子驾幸龙符宫,须用早起五更去伺候。”娘子听了,便分付
梅香道:“官人明日要起五更,出去随班;你们四更起来烧汤,安排点心。”时迁
自忖道:“眼见得梁上那个皮匣子,便是盛甲在里面。我若趁半夜下手便好;倘若
闹将起来,明日出不得城,却不误了大事?且捱到五更里下手不迟。”

听得徐宁夫妻两口儿上床睡了,两个娅在房门外打铺。房里桌上,却点着碗
灯。那五个人都睡着了。两个娅一日伏侍到晚,精神困倦,亦皆睡了。时迁溜下
来,去身边取个芦管儿,就窗棂眼里只一吹,把那碗灯早吹灭了。看看伏到四更左
侧,徐宁起来,便唤娅起来烧汤。那两个使女,从睡梦里起来,看房里没了灯,
叫道:“阿呀,今夜却没了灯!”徐宁道:“你不去后面讨灯,等几时!”那个梅
香开楼门。下胡梯响。时迁听得,却从柱上只一溜,来到后门边黑影里伏了。听得
娅正开后门出来,便去开墙门,时迁却潜入厨房里,贴身在厨桌下。梅香讨了灯
火入来看时,又去关门,却来灶前烧火。这个女使也起来生炭火上楼去。多时汤滚,
捧面汤上去,徐宁洗漱了,叫烫些热酒上来。娅安排肉食炊饼上去,徐宁吃罢,
叫把饭与外面当直的吃。时迁听得徐宁下来,叫伴当吃了饭,背着包袱,拿了金枪
出门。那个梅香点着灯,送徐宁出去,时迁却从厨桌下出来,便上楼去,从子边
直踅到梁上,却把身躯伏了。两个娅,又关闭了门户,吹灭了灯火,上楼来,脱
了衣裳,倒头便睡。

时迁听那两个娅睡着了,在梁上把那芦管儿指灯一吹,那灯又早灭了。时迁
却从梁上轻轻解了皮匣,正要下来,徐宁的娘子觉来,听得响,叫梅香道:“梁上
甚么响?”时迁做老鼠叫。娅道:“娘子不听得是老鼠叫?因厮打,这般响。”
时迁就便学老鼠厮打,溜将下来,悄悄地开了楼门,款款地背着皮匣,下得胡梯,
从里面直开到外门,来到班门口,已自有那随班的人出门,四更便开了锁。时迁得
了皮匣,从人队里,趁闹出去了,一口气奔出城外,到客店门前。此时天色未晓,
敲开店门,去房里取出行李,拴束做一担儿挑了,计算还了房钱,出离店肆,投东
便走。

行到四十里外,方才去食店里打火做些饭吃,只见一个人也撞将入来。时迁看
时,不是别人,却是神行太保戴宗。见时迁已得了物,两个暗暗说了几句话,戴宗
道:“我先将甲投山寨去,你与汤隆慢慢地来。”时迁打开皮匣,取出那副雁翎锁
子甲来,做一包袱包了。戴宗拴在身上,出了店门,作起神行法,自投梁山泊去了。

时迁却把空皮匣子明明的拴在担子上,吃了饭食,还了打火钱,挑上担儿,出
店门便走。到二十里路上,撞见汤隆,两个便入酒店里商量。汤隆道:“你只依我
从这条路去,但过路上酒店、饭店、客店,门上若见有白粉圈儿,你便可就在那店
里买酒买肉吃。客店之中,就便安歇。特地把这皮匣子放在他眼睛头。离此间一程
外等我。”时迁依计去了。汤隆慢慢地吃了一回酒,却投东京城里来。

且说徐宁家里,天明,两个娅起来,只见楼门也开了,下面中门大门都不关,
慌忙家里看时,一应物件都有,两个娅上楼来,对娘子说道:“不知怎的门户都
开了?却不曾失了物件。”娘子便道:“五更里听得梁上响,你说是老鼠厮打,你
且看那皮匣子没甚么事?”两个娅看了,只叫得苦:“皮匣子不知那里去了!”
那娘子听了,慌忙起来道:“快央人去龙符宫里,报与官人知道,教他早来跟寻!”
娅急急寻人去龙符宫报徐宁,连央了三四替人,都回来说道:“金枪班直随驾内
苑去了,外面都是亲军护御守把,谁人能够入去?直须等他自归。”徐宁妻子并两
个娅,如热子上蚂蚁,走头无路,不茶不饭,慌做一团。

徐宁直到黄昏时候,方才卸了衣袍服色,着当直的背了,将着金枪,径回家来。
到得班门口,邻舍说道:“娘子在家失盗,等候得观察,不见回来。”徐宁吃了一
惊,慌忙走到家里,两个娅迎门道:“官人五更出去,却被贼人闪将入来,单单
只把梁上那个皮匣子盗将去了。”徐宁听罢,只叫那连声的苦,从丹田底下直滚出
口角来。娘子道:“这贼正不知几时闪在屋里?”徐宁道:“别的都不打紧,这副
雁翎甲,乃是祖宗留传四代之宝,不曾有失。花儿王太尉曾还我三万贯钱,我不曾
舍得卖与他。恐怕久后军前阵后要用,生怕有些差池,因此拴在梁上。多少人要看
我的,只推没了。今次声张起来,枉惹他人耻笑,今却失去,如之奈何!”徐宁一
夜睡不着,思量道:“不知是甚么人盗了去!——也是曾知我这副甲的人。”娘子
想道:“敢是夜来灭了灯时,那贼已躲在家里了?必然是有人爱你的,将钱问你买
不得,因此使这个高手贼来盗了去。你可央人慢慢缉访出来,别作商议,且不要打
草惊蛇。”徐宁听了,到天明起来,坐在家中纳闷。好似:

蜀王春恨,宋玉秋悲,吕虔遗腰下之刀,雷焕失狱中之剑。珠亡照乘,璧碎连
城。王恺之珊瑚已毁,无可赔偿;裴航之玉杵未逢,难谐欢好。正是凤落荒坡凋锦
羽,龙居浅水失明珠。

这日徐宁正在家中纳闷,早饭时分,只听得有人扣门,当直的出去问了名姓,
入去报道:“有个延安府汤知寨儿子汤隆,特来拜望。”徐宁听罢,教请进客位里
相见。汤隆见了徐宁,纳头拜下,说道:“哥哥一向安乐?”徐宁答道:“闻知舅
舅归天去了,一者官身羁绊,二乃路途遥远,不能前来吊问。并不知兄弟信息,一
向正在何处?今次自何而来?”汤隆道:“言之不尽,自从父亲亡故之后,时乖运
蹇,一向流落江湖。今从山东径来京师,探望兄长。”徐宁道:“兄弟少坐。”便
叫安排酒食相待。汤隆去包袱内取出两锭蒜条金,重二十两,送与徐宁,说道:“先
父临终之日,留下这些东西,教寄与哥哥做遗念。为因无心腹之人,不曾捎来。今
次兄弟特地到京师纳还哥哥。”徐宁道:“感承舅舅如此挂念,我又不曾有半分孝
顺处,怎地报答!”汤隆道:“哥哥休恁地说。先父在日之时,常是想念哥哥这一
身武艺,只恨山遥水远,不能够相见一面,因此留这些物与哥哥做遗念。”徐宁谢
了汤隆,交收过了,且安排酒来管待。

汤隆和徐宁饮酒中间,徐宁只是眉头不展,面带忧容。汤隆起身道:“哥哥如
何尊颜有些不喜?心中必有忧疑不决之事。”徐宁叹口气道:“兄弟不知,一言难
尽,夜来家间被盗。”汤隆道:“不知失去了何物?”徐宁道:“单单只盗去了先
祖留下那副雁翎锁子甲,又唤做赛唐猊。昨夜失了这件东西,以此心下不乐。”汤
隆道:“哥哥那副甲,兄弟也曾见来,端的无比,先父常常称赞不尽。却是放在何
处被盗了去?”徐宁道:“我把一个皮匣子盛着,拴缚在卧房中梁上,正不知贼人
甚么时候入来盗了去。”汤隆问道:“却是甚等样皮匣子盛着?”徐宁道:“是个
红羊皮匣子盛着,里面又用香绵裹住。”汤隆假意失惊道:“红羊皮匣子?不是上
面有白线刺着绿云头如意,中间有狮子滚绣球的?”徐宁道:“兄弟,你那里见来?”
汤隆道:“小弟夜来离城四十里,在一个村店里沽些酒吃,见个鲜眼睛黑瘦汉子,
担儿上挑着。我见了,心中也自暗忖道:‘这个皮匣子,却是盛甚么东西的?’临
出门时,我问道:‘你这皮匣子作何用?’那汉子应道:‘原是盛甲的,如今胡乱
放些衣服。’必是这个人了。我见那厮却似闪肭了腿的,一步步挑着了走。何不我
们追赶他去?”徐宁道:“若是赶得着时,却不是天赐其便!”汤隆道:“既是如
此,不要耽搁,便赶去罢。”

徐宁听了,急急换上麻鞋,带了腰刀,提条朴刀,便和汤隆两个出了东郭门,
拽开脚步,迤逦赶来。前面见壁上有白圈酒店里,汤隆道:“我们且吃碗酒了赶,
就这里问一声。”汤隆入得门坐下,便问道:“主人家,借问一问,曾有个鲜眼黑
瘦汉子,挑个红羊皮匣子过去么?”店主人道:“昨夜晚,是有这般一个人挑着个
红羊皮匣子过去了,一似腿上吃跌了的,一步一攧走。”汤隆道:“哥哥,你听却
如何?”徐宁听了,做声不得。

两个连忙还了酒钱,出门便去。前面又见一个客店,壁上有那白圈,汤隆立住
了脚,说道:“哥哥,兄弟走不动了,和哥哥且就这客店里歇了。明日早去赶。”
徐宁道:“我却是官身,倘或点名不到,官司必然见责,如之奈何?”汤隆道:“这
个不用兄长忧心,嫂嫂必自推个事故。”当晚又在客店里问时,店小二答道:“昨
夜有一个鲜眼黑瘦汉子,在我店里歇了一夜,直睡到今日小日中,方才去了,口里
只问山东路程。”汤隆道:“恁地可以赶了。明日起个四更,定是赶着,拿住那厮,
便有下落。”当夜两个歇了,次日起个四更,离了客店,又迤逦赶来。汤隆但见壁
上有白粉圈儿,便做买酒买食吃了问路,处处皆说得一般。徐宁心中急切要那副甲,
只顾跟随着汤隆赶了去。看看天色又晚了,望见前面一所古庙,庙前树下,时迁放
着担儿,在那里坐地。汤隆看见,叫道:“好了!前面树下那个,不是哥哥盛甲的
匣子?”徐宁见了,抢向前来,一把揪住时迁,喝道:“你这厮好大胆!如何盗了
我这副甲来!”时迁道:“住,住!不要叫!是我盗了你这副甲来,你如今却是要怎
地?”徐宁喝道:“畜生无礼!倒问我要怎的!”时迁道:“你且看匣子里有甲也
无?”汤隆便把匣子打开看时,里面却是空的。徐宁道:“你这厮把我这副甲那里
去了!”时迁道:“你听我说:小人姓张,排行第一,泰安州人氏,本州有个财主,
要结识老种经略相公。知道你家有这副雁翎锁子甲,不肯货卖,特地使我同一个李
三,两人来你家偷盗,许俺们一万贯。不想我在你家柱子上跌下来,闪肭了腿,因
此走不动。先教李三把甲拿了去,只留得空匣在此。你若要奈何我时,便到官司,
只是拚着命,就打死我也不招,休想我指出别人来。若还肯饶我官司时,我和你去
讨这副甲来还你。”徐宁踌躇了半晌,决断不下。汤隆便道:“哥哥,不怕他飞了
去!只和他去讨甲!若无甲时,须有本处官司告理。”徐宁道:“兄弟也说的是。”
三个厮赶着,又投客店里来息了。徐宁、汤隆监住时迁一处宿歇。原来时迁故把些
绢帛扎缚了腿,只做闪肭了腿。徐宁见他又走不动,因此十分中只有五分防他。三
个又歇了一夜,次日早起来再行,时迁一路买酒买肉陪告。又行了一日。

次日,徐宁在路上心焦起来,不知毕竟有甲也无。正走之间,只见路旁边三四
个头口,拽出一辆空车子,背后一个人驾车。旁边一个客人,看着汤隆,纳头便拜。
汤隆问道:“兄弟因何到此?”那人答道:“郑州做了买卖,要回泰安州去。”汤
隆道:“最好。我三个要搭车子,也要到泰安州去走一遭。”那人道:“莫说三个
上车,再多些也不计较。”汤隆大喜,叫与徐宁相见。徐宁问道:“此人是谁?”
汤隆答道:“我去年在泰安州烧香,结识得这个兄弟,姓李,名荣,是个有义气的
人。”徐宁道:“既然如此,这张一又走不动,都上车子坐地。”只叫车客驾车子
行。四个人坐在车子上,徐宁问道:“张一,你且说与我那个财主姓名。”时迁吃
逼不过,三回五次推托,只得胡乱说道:“他是有名的郭大官人。”徐宁却问李荣
道:“你那泰安州曾有个郭大官人么?”李荣答道:“我那本州郭大官人,是个上
户财主,专好结识官宦来往,门下养着多少闲人。”徐宁听罢,心中想道:“既有
主坐,必不碍事。”又见李荣一路上说些枪棒,唱几个曲儿,不觉的又过了一日。

话休絮繁。看看到梁山泊只有两程多路,只见李荣叫车客把葫芦去沽些酒来,
买些肉来,就车子上吃三杯。李荣把出一个瓢来,先倾一瓢,来劝徐宁,徐宁一饮
而尽。李荣再叫倾酒,车客假做手脱,把这一葫芦酒,都倾翻在地下。李荣喝骂车
客再去沽些,只见徐宁口角流涎,扑地倒在车子上了。李荣是谁?却是铁叫子乐和。
三个从车上跳将下来,赶着车子,直送到旱地忽律朱贵酒店里。众人就把徐宁扛扶
下船,都到金沙滩上岸。宋江已有人报知,和众头领下山接着。徐宁此时麻药已醒,
众人又用解药解了。徐宁开眼见了众人,吃了一惊,便问汤隆道:“兄弟,你如何
赚我到这里?”汤隆道:“哥哥听我说:小弟今次闻知宋公明招接四方豪杰,因此
上在武冈镇拜黑旋风李逵做哥哥,投托大寨入伙。今被呼延灼用连环甲马冲阵,无
计可破,是小弟献此钩镰枪法。只除是哥哥会使。由此定这条计:使时迁先来盗了
你的甲,却教小弟赚哥哥上路,后使乐和假做李荣,过山时,下了蒙汗药,请哥哥
上山来坐把交椅。”徐宁道:“却是兄弟送了我也!”宋江执杯向前陪告道:“现
今宋江暂居水泊,专待朝廷招安,尽忠竭力报国。非敢贪财好杀,行不仁不义之事;
万望观察怜此真情,一同替天行道。”林冲也来把盏陪话道:“小弟亦到此间,多
说兄长清德,休要推却。”徐宁道:“汤隆兄弟,你却赚我到此,家中妻子,必被
官司擒捉,如之奈何!”宋江道:“这个不防。观察放心,只在小可身上,早晚便
取宝眷到此完聚。”晁盖、吴用、公孙胜,都来与徐宁陪话,安排筵席作庆。一面
选拣精壮小喽罗,学使钩镰枪法,一面使戴宗和汤隆星夜往东京,搬取徐宁老小。
旬日之间,杨林自颍州取到彭玘老小,薛永自东京取到凌振老小,李云收买到五车
烟火、药料回寨。更过数日,戴宗、汤隆取到徐宁老小上山。

徐宁见了妻子到来,吃了一惊,问是如何便到得这里。妻子答道:“自你转背,
官司点名不到,我使了些金银首饰,只推道患病在床,因此不来叫唤。忽见汤叔叔
赍着雁翎甲来;说道:‘甲便夺得来了,哥哥只是于路染病,将次死在客店里,叫
嫂嫂和孩儿便来看视。’把我赚上车子,我又不知路径,迤逦来到这里。”徐宁道:
“兄弟,好却好了,只可惜将我这副甲陷在家里了。”汤隆笑道:“好教哥哥欢喜,
打发嫂嫂上车之后,我便复翻身去赚了这甲,诱了这两人娅,收拾了家中应有细
软,做一担儿挑在这里。”徐宁道:“恁地时,我们不能够回东京去了。”汤隆道:
“我又教哥哥再知一件事:来在半路上,撞见一伙客人,我把哥哥的雁翎甲穿了,
搽画了脸,说哥哥名姓,劫了那伙客人的财物,这早晚东京已自遍行文书,捉拿哥
哥。”徐宁道:“兄弟,你也害得我不浅!”晁盖、宋江都来陪话道:“若不是如
此,观察如何肯在这里住?”随即拨定房屋,与徐宁安顿老小。众头领且商议破连
环马军之法。

此时雷横监造钩镰枪已都完备。宋江、吴用等启请徐宁,教众军健学使钩镰枪
法。徐宁道:“小弟今当尽情剖露,训练众军头目,拣选身材长壮之士。”众头领
都在聚义厅上看徐宁选军,说那个钩镰枪法。有分教:三千甲马登时破,一个英雄
指日降。

毕竟金枪徐宁怎的敷演钩镰枪法,且听下回分解。