\chapter{十节度议取梁山泊~宋公明一败高太尉}

再说梁山泊好汉,自从两赢童贯之后,宋江、吴用商议,必用着一个人,去东
京探听消息虚实,上山回报,预先准备军马交锋。言之未绝,只见神行太保戴宗道:
“小弟愿往。”宋江道:“探听军情,多亏煞兄弟一个,虽然贤弟去得,必须也用
一个相帮去最好。”李逵便道:“兄弟帮哥哥去走一遭。”宋江笑道:“你便是那
个不惹事的黑旋风!”李逵道:“今番去时,不惹事便了。”宋江喝退,一壁再问:
“有那个兄弟敢去走一遭?”赤发鬼刘唐禀道:“小弟帮戴宗哥哥去如何?”宋江
大喜道:“好!”当日两个收拾了行装,便下山去。
且不说戴宗、刘唐来东京打听消息,却说童贯和毕胜沿路收聚得败残军马四万余人,
比到东京,于路教众多管军的头领,各自部领所属军马,回营寨去了,只带御营军
马入城来。童贯卸了戎装衣甲,径投高太尉府中去商议。两个见了,各叙礼罢,请
入后堂深处坐定。童贯把大折两阵,结果了八路军官并许多军马,酆美又被活捉去
了,似此如之奈何,一一都告诉了。高太尉道:“枢相不要烦恼,这件事只瞒了今
上天子便了,谁敢胡奏?我和你去告禀太师,再作个道理。”
童贯和高俅上了马,径投蔡太师府内来。已有报知童枢密回了,蔡京料道不胜,又
听得和高俅同来,蔡京教唤入书院里来厮见。童贯拜了太师,泪如雨下。蔡京道:
“且休烦恼,我备知你折了军马之事。”高俅道:“贼居水泊,非船不能征进,枢
密只以马步军征剿,因此失利,中贼诡计。”童贯诉说折兵败阵之事,蔡京道:“你
折了许多军马,费了许多钱粮,又折了八路军官,这事怎敢教圣上得知?”童贯再
拜道:“望乞太师遮盖,救命则个!”蔡京道:“明日只奏道天气暑热,军士不伏
水土,权且罢战退兵。倘或震怒说道:‘似此心腹大患,不去剿灭,后必为殃。’
如此时,恁众官却怎地回答?”高俅道:“非是高俅夸口,若还太师肯保高俅领兵
亲去那里征讨,一鼓可平。”蔡京道:“若得太尉肯自去,可知是好,明日便当保
奏太尉为帅。”高俅又禀道:“只有一件,须得圣旨任便起军,并随造船只;或是
拘刷原用官船民船,或备官价收买木料,打造战船;水陆并进,船骑同行,方可指
日成功。”蔡京道:“这事容易。”
正话间,门吏报道:“酆美回来了。”童贯大喜。太师教唤进来,问其缘故。酆美
拜罢,叙说宋江但是活捉上山去的,尽数放回,不肯杀害,又与盘缠,令回乡里,
因此小将得见钧颜。高俅道:“这是贼人诡计,故意慢我国家。今后不点近处军马,
直去山东、河北拣选得用的人,跟高俅去。”蔡京道:“既然如此计议定了,来日
内里相见,面奏天子。”各自回府去了。
次日五更三点,都在侍班阁子里相聚。朝鼓响时,各依品从,分列丹墀,拜舞起居
已毕,文武分班,列于玉阶之下。只见蔡太师出班奏道:“昨遣枢密使童贯统率大
军进征梁山泊草寇,近因炎热,军马不伏水土,抑且贼居水洼,非船不行,马步军
兵,急不能进,因此权且罢战,各回营寨暂歇,别候圣旨。”天子乃云:“似此炎
热,再不复去矣。”蔡京奏道:“童贯可于泰乙宫听罪,别令一人为帅,再去征伐,
乞请圣旨。”天子曰:“此寇乃是心腹大患,不可不除,谁与寡人分忧?”高俅出
班奏曰:“微臣不材,愿效犬马之劳,去征剿此寇,伏取圣旨。”天子云:“既然
卿肯与寡人分忧,任卿择选军马。”高俅又奏:“梁山泊方圆八百余里,非仗舟船,
不能前进,臣乞圣旨,于梁山泊近处采伐木植,督工匠造船,或用官钱收买民船,
以为战伐之用。”天子曰:“委卿执掌,从卿处置,可行即行,慎勿害民。”高俅
奏道:“微臣安敢。只容宽限,以图成功。”天子令取锦袍金甲赐与高俅,另选吉
日出师。
当日百官朝退,童贯、高俅送太师到府,便唤中书省关房掾史,传奉圣旨,定夺拨
军。高太尉道:“前者有十节度使,多曾与国家建功,或征鬼方,或伐西夏并金、
辽等处,武艺精熟,请降钧帖,差拨为将。”蔡太师依允,便发十道札付文书,仰
各各部领所属精兵一万,前赴济州取齐,听候调用。十个节度使非同小可,每人领
军一万,克期并进。那十路军马:
河南河北节度使

王

焕
上党太原节度使

徐

京
京北弘农节度使

王文德
颍州汝南节度使

梅

展
中山安平节度使

张

开
江夏零陵节度使

杨

温
云中雁门节度使

韩存保
陇西汉阳节度使

李从吉
琅琊彭城节度使

项元镇
清河天水节度使

荆

忠
原来这十路军马,都是曾经训练精兵,更兼这十节度使,旧日都是绿林丛中出身,
后来受了招安,直做到许大官职,都是精锐勇猛之人,非是一时建了些少功名。当
日中书省定了程限,发十道公文,要这十路军马如期都到济州,迟慢者定依军令处
置。金陵建康府有一枝水军,为头统制官唤做刘梦龙。那人初生之时,其母梦见一
条黑龙飞入腹中,感而遂生;及至长大,善知水性,曾在西川峡江讨贼有功,升做
军官都统制,统领一万五千水军,棹船五百只,守住江南。高太尉要取这支水军并
船只星夜前来听调,又差一个心腹人,唤做牛邦喜,也做到步军校尉,教他去沿江
上下并一应河道内拘刷船只,都要来济州取齐,交割调用。高太尉帐前牙将极多,
于内两个最了得:一个唤做党世英,一个唤做党世雄。弟兄二人,现做统制官,各
有万夫不当之勇。高太尉又去御营内选拨精兵一万五千,通共各处军马一十三万,
先于诸路差官供送粮草,沿途交纳。高太尉连日整顿衣甲,制造旌旗,未及登程。
有诗为证:
轻事贪功愿领兵,兵权到手便留行。
幸因主帅迟迟去,多得三军数日生。
却说戴宗、刘唐在东京住了几日,打探得备细消息,星夜回还山寨,报说此事。宋
江听得高太尉亲自领兵,调天下军马一十三万、十节度使统领前来,心中惊恐,便
和吴用商议。吴用道:“仁兄勿忧,小生也久闻这十节度的名,多与朝廷建功,只
是当初无他的敌手,以此只显他的豪杰。如今放着这一班好弟兄,如狼似虎的人,
那十节度已是过时的人了,兄长何足惧哉!比及他十路军来,先教他吃我一惊。”
宋江道:“军师如何惊他?”吴用道:“他十路军马都到济州取齐,我这里先差两
个快厮杀的,去济州相近,接着来军,先杀一阵。——这是报信与高俅知道。”宋
江道:“叫谁去好?”吴用道:“差没羽箭张清、双枪将董平,此二人可去。”宋
江差二将各带一千马军,前去巡哨济州,相迎截杀各路军马;又拨水军头领,准备
泊子里夺船。山寨中头领预先调拨已定,且不细说,下来便知。
再说高太尉在京师俄延了二十余日,天子降敕,催促起军,高俅先发御营军马出城,
又选教坊司歌儿舞女三十余人,随军消遣。至日祭旗,辞驾登程,却好一月光景。
时值初秋天气,大小官员都在长亭饯别。高太尉戎装披挂,骑一匹金鞍战马,前面
摆着五匹玉辔雕鞍从马,左右两边,排着党世英、党世雄弟兄两个,背后许多殿帅
统制官、统军提辖、兵马防御、团练等官,参随在后。那队伍军马,十分摆布得整
齐。诗曰:
匿奸罔上非忠荩,好战全违旧典章。
不事怀柔服强暴,只驱良善敌刀枪。
那高太尉部领大军出城,来到长亭前下马,与众官作别,饮罢饯行酒,攀鞍上马,
登程望济州进发。于路上纵容军士,尽去村中纵横掳掠,黎民受害,非止一端。
却说十路军马陆续都到济州,有节度使王文德领着京北等处一路军马,星夜奔济州
来,离州尚有四十余里。当日催动人马,赶到一个去处,地名凤尾坡,坡下一座大
林。前军却好抹过林子,只听得一棒锣声响处,林子背后山坡脚边转出一彪军马来,
当先一将拦路。那员将顶盔挂甲,插箭弯弓,去那弓袋箭壶内侧插着小小两面黄旗,
旗上各有五个金字,写道:“英雄双枪将,风流万户侯。”两手两杆钢枪。此将
乃是梁山泊第一个惯冲头阵的勇将董平,因此人称为董一撞。董平勒定战马,截住
大路喝道:“来的是那里兵马?不早早下马受缚,更待何时?”这王文德兜住马,
呵呵大笑道:“瓶儿罐儿也有两个耳朵,你须曾闻我等十节度使累建大功,名扬天
下,大将王文德么?”董平大笑,喝道:“只你便是杀晚爷的大顽。”王文德听了
大怒,骂道:“反国草寇,怎敢辱吾!”拍马挺枪,直取董平。董平也挺双枪来迎。
两将斗到三十合,不分胜败。王文德料道赢不得董平,喝一声:“少歇再战。”各
归本阵。王文德分付众军,休要恋战,直冲过去。王文德在前,三军在后,大发声
喊,杀将过去。董平后面引军追赶。将过林子,正走之间,前面又冲出一彪军马来。
为首一员上将,正是没羽箭张清,在马上大喝一声:“休走!”手中拈定一个石子
打将来,望王文德头上便着。急待躲时,石子打中盔顶,王文德伏鞍而走,跑马奔
逃。两将赶来,看看赶上,只见侧首冲过一队军来。王文德看时,却是一般的节度
使杨温军马,齐来救应。因此,董平、张清不敢来追,自回去了。
两路军马同入济州歇定,太守张叔夜接待各路军马。数日之间,前路报来,高太尉
大军到了,十节度出城迎接,都相见了太尉,一齐护送入城,把州衙权为帅府,安
歇下了。高太尉传下号令,教十路军马,都向城外屯驻,伺候刘梦龙水军到来,一
同进发。这十路军马各自下寨,近山砍伐木植,人家搬掳门窗,搭盖窝铺,十分害
民。高太尉自在城中帅府内,定夺征进人马;无银两使用者,都充头哨出阵交锋;
有银两者,留在中军,虚功滥报。似此奸弊,非止一端。

高太尉在济州不过一二日,刘梦龙战船到了,参谒帅府。礼毕,高俅随即便唤
十节度使都到厅前,共议良策。王焕等禀复道:“太尉先教马步军去探路,引贼出
战,然后却调水路战船,去劫贼巢,令其两下不能相顾,可获群贼矣!”高太尉从
其所言。当时分拨王焕、徐京为前部先锋,王文德、梅展为合后收军,张开、杨温
为左军,韩存保、李从吉为右军,项元镇、荆忠为前后救应使。党世雄引领三千精
兵,上船协助刘梦龙水军船只,就行监战。诸军尽皆得令,整束了三日,请高太尉
看阅诸路军马。高太尉亲自出城,一一点看了,便遣大小三军,并水军,一齐进发,
径望梁山泊来。
且说董平、张清回寨,说知备细,宋江与众头领统率大军,下山不远,早见官军到
来。前军射住阵脚,两边拒定人马,只见先锋王焕出阵,使一条长枪,在马上厉声
高叫:“无端草寇,敢死村夫,认得大将王焕么?”对阵绣旗开处,宋江亲自出马,
与王焕声喏道:“王节度,你年纪高大了,不堪与国家出力,当枪对敌,恐有些一
差二误,枉送了你一世清名。你回去罢!另教年纪小的出来战。”王焕听得大怒,
骂道:“你这厮是个文面俗吏,安敢抗拒天兵!”宋江答道:“王节度,你休逞好
手,我这一班儿替天行道的好汉,不到得输与你!”王焕便挺枪戳将过来。宋江马
后,早有一将,銮铃响处,挺枪出阵。宋江看时,却是豹子头林冲,来战王焕。两
马相交,众军助喊,高太尉自临阵前,勒住马看。只听得两军呐喊喝采,果是马军
踏镫抬身看,步卒掀盔举眼观。两个施逞诸路枪法,但见:
一个屏风枪势如霹雳,一个水平枪勇若奔雷。一个朝天枪难防难躲,一个钻风枪怎
敌怎遮。这个恨不得枪戳透九霄云汉,那个恨不得枪刺透九曲黄河。一个枪如蟒离
岩洞,一个枪似龙跃波津。一个使枪的雄似虎吞羊,一个使枪的俊如雕扑兔。
王焕大战林冲,约有七八十合,不分胜败。两边各自鸣金,二将分开,各归本阵。
只见节度使荆忠到前军,马上欠身,禀复高太尉道:“小将愿与贼人决一阵,气请
钧旨。”高太尉便教荆忠出马交战。宋江马后鸾铃响处,呼延灼来迎。荆忠使一口
大杆刀,骑一匹瓜黄马,二将交锋,约斗二十合,被呼延灼卖个破绽,隔过大刀,
顺手提起钢鞭来,只一下,打个衬手,正着荆忠脑袋,打得脑浆迸流,眼珠突出,
死于马下。高俅看见折了一个节度使,火急便差项元镇骤马挺枪,飞出阵前,大喝:
“草贼敢战吾么?”宋江马后,双枪将董平撞出阵前,来战项元镇。两个斗不到十
合,项元镇霍地勒回马,拖了枪便走。董平拍马去赶,项元镇不入阵去,绕着阵脚,
落荒而走。董平飞马去追,项元镇带住枪,左手拈弓,右手搭箭,拽满弓,翻身背
射一箭。董平听得弓弦响,抬手去隔,一箭正中右臂,弃了枪,拨回马便走。项元
镇挂着弓,拈着箭,倒赶将来。呼延灼、林冲见了,两骑马各出,救得董平归阵。
高太尉指挥大军混战,宋江先教救了董平回山,后面军马,遮拦不住,都四散奔走。
高太尉直赶到水边,却调人去接应水路船只。
且说刘梦龙和党世雄布领水军,乘驾船只,迤前投梁山泊深处来,只见茫茫荡荡,
尽是芦苇蒹葭,密密遮定港汊。这里官船樯篙不断,相连十余里水面。正行之间,
只听得山坡上一声炮响,四面八方,小船齐出,那官船上军士,先有五分惧怯,看
了这等芦苇深处,尽皆慌了。怎禁得芦苇里面埋伏着小船,齐出冲断大队。官船前
后不相救应,大半官军,弃船而走。梁山泊好汉看见官军阵脚乱了,一齐鸣鼓摇船,
直冲上来。刘梦龙和党世雄急回船时,原来经过的浅港内,都被梁山泊好汉用小船
装载柴草,砍伐山中木植,填塞断了,那橹桨竟摇不动。众多军卒,尽弃了船只下
水。刘梦龙脱下戎装披挂,爬过水岸,拣小路走了。这党世雄不肯弃船,只顾叫水
军寻港汊深处摇去,不到二里,只见前面三只小船,船上是阮氏三雄,各人手执蓼
叶枪,挨近船边来,众多驾船军士,都跳下水里去了。党世雄自持铁搠,立在船头
上与阮小二交锋。阮小二也跳下水里去,阮小五、阮小七两个逼近身来。党世雄见
不是头,撇了铁搠,也跳下水里去了。只见水底下钻出船火儿张横来,一手揪住头
发,一手提定腰胯,滴溜溜丢上芦苇根头。先有十数个小喽罗躲在那里,铙钩套索
搭住,活捉上水浒寨来。
却说高太尉见水面上船只,都纷纷滚滚乱投山边去了,船上缚着的,尽是刘梦龙水
军的旗号,情知水路里又折了一阵,忙传军令,且教收兵回济州去,别作道理。五
军比及要退,又值天晚,只听得四下里火炮不住价响,宋江军马,不知几路杀将来。
高太尉只叫得:“苦了也。”正是:阴陵失路逢神弩,赤壁鏖兵遇怪风。
毕竟高太尉怎地脱身,且听下回分解。