\chapter{卢俊义分兵歙州道~宋公明大战乌龙岭}

话说当下张横听得道没了他兄弟张顺,烦恼得昏晕了,半晌却救得苏醒。宋江
道:“且扶在帐房里调治,却再问他海上事务。”宋江令裴宣、蒋敬写录众将功劳,
辰巳时分,都在营前聚集。李俊、石秀生擒吴值,三员女将生擒张道原,林冲蛇矛
戳死冷恭,解珍、解宝杀了崔,只走了石宝、邓元觉、王、晁中、温克让五人。
宋江便出榜安抚百姓,赏劳三军,把吴值、张道原解赴张招讨军前,斩首施行。献
粮袁评事申文保举作富阳县令,张招讨处关领空头官诰,不在话下。
众将都到城中歇下,左右报道:“阮小七从江里上岸,入城来了。”宋江唤到帐前
问时,说道:“小弟和张横并侯健、段景住带领水手,海边觅得船只,行至海盐等
处,指望便使入钱塘江来。不期风水不顺,打出大洋里去了。急使得回来,又被风
打破了船,众人都落在水里。侯健、段景住不识水性,落下去淹死海中,众多水手
各自逃生四散去了。小弟赴水到海口,进得赭山门,被潮直漾到半山,赴水回来。
却见张横哥哥在五云山江里,本待要上岸来,又不知他在那地里。昨夜望见城中火
起,又听得连珠炮响,想必是哥哥在杭州城厮杀,以此从江里上岸来。不知张横曾
到岸也不曾?”宋江说张横之事与阮小七知道,令和他自己两个哥哥相见了,依前
管领水军头领船只。宋江传令,先调水军头领,去江里收拾江船,伺候征进睦州。
想起张顺如此通灵显圣,去涌金门外靠西湖边建立庙宇,题名“金华太保”,宋江
亲去祭奠。后来收伏方腊有功于朝,宋江回京奏知此事,特奉圣旨,敕封为“金华
将军”,庙食杭州。
再说宋江在行宫内,因思渡江以来,损折许多将佐,心中十分悲怆。却去净慈寺修
设水陆道场七昼夜,判施斛食,济拔沉冥,超度众将,各设灵位享祭。做了好事已
毕,将方天定宫中一应禁物,尽皆毁坏,所有金银、宝贝、罗缎等项,分赏诸将军
校。杭州城百姓俱宁,设宴庆赏。当与军师从长计议,调兵收复睦州。此时已是四
月尽间,忽闻报道:“副都督刘光世并东京天使,都到杭州。”宋江当下引众将出
北关门迎接入城,就行宫开读圣旨:“敕先锋使宋江等收剿方腊,累建大功,敕赐
皇封御酒三十五瓶,锦衣三十五领,赏赐正将。其余偏将,照名支给赏赐缎匹。”
原来朝廷只知公孙胜不曾渡江收剿方腊,却不知折了许多头领。宋江见了三十五员
锦衣御酒,蓦然伤心,泪不能止。天使问时,宋江把折了众将的话,对天使说知。
天使道:“如此折将,朝廷怎知?下官回京,必当奏闻。”那时设宴款待天使,刘
光世主席,其余大小将佐,各依次序而坐。御赐酒宴,各各沾恩。现亡正偏将佐,
留下锦衣御酒赏赐,次日,设位遥空享祭。宋江将一瓶御酒,一领锦衣,去张顺庙
里,呼名享祭。锦衣就穿泥神身上,其余的都只遥空焚化。天使住了几日,送回京
师。
不觉迅速光阴,早过了数十日。张招讨差人赍文书来,催促先锋进兵。宋江与吴用
请卢俊义商议:“此去睦州,沿江直抵贼巢。此去歙州,却从昱岭关小路而去。今
从此处分兵征剿,不知贤弟兵取何处?”卢俊义道:“主兵遣将,听从哥哥严令,
安敢选择?”宋江道:“虽然如此,试看天命。”作两队分定人数,写成两处阄子,
焚香祈祷,各阄一处。宋江拈阄得睦州,卢俊义拈阄得歙州。宋江道:“方腊贼巢
正是清溪县帮源洞中。贤弟取了歙州,可屯住军马,申文飞报知会,约日同攻清溪
贼洞。”卢俊义便请宋公明酌量分调将佐军校。
先锋使宋江带领正偏将佐三十六员,攻取睦州并乌龙岭:
军师吴用

关胜

花荣

秦明

李应
戴宗
朱仝
李逵
鲁智深
武松
解珍
解宝
吕方
郭盛
樊瑞
马麟
燕顺
宋清
项充
李衮
王英
扈三娘
凌振
杜兴
蔡福
蔡庆
裴宣
蒋敬
郁保四
水军头领正偏将佐七员,部领船只,随军征进睦州:
李俊

阮小二

阮小五

阮小七

童猛
童威

孟康
副先锋卢俊义管领正偏将佐二十八员,收取歙州并昱岭关:
军师朱武

林冲

呼延灼

史进

杨雄
石秀
单廷
魏定国
孙立
黄信
欧鹏
杜迁
陈达
杨春
李忠
薛永
邹渊
李立
李云
邹润
汤隆
石勇
时迁
丁得孙
孙新
顾大嫂
张青
孙二娘
当下卢先锋部领正偏将校共计二十九员,随行军兵三万人马,择日辞了刘都督,别
了宋江,引兵望杭州,取山路经过临安县,进发登程去了。却说宋江等整顿船只军
马,分拨正偏将校,选日祭旗出师,水陆并进,船骑相迎。此时杭州城内瘟疫盛行,
已病倒六员将佐:是张横、穆弘、孔明、朱贵、杨林、白胜,患体未痊,不能征进,
就拨穆春、朱富看视病人,共是八员,寄留杭州。其余众将,尽随宋江攻取睦州,
共计三十七员,取路沿江望富阳县进发。
且不说两路军马起程,再说柴进同燕青,自秀州李亭别了宋先锋,行至海盐县前,
到海边趁船,使过越州,迤来到诸暨县,渡过渔浦,前到睦州界上。把关隘将校
拦住,柴进告道:“某乃是中原一秀士,能知天文地理,善会阴阳,识得六甲风云,
辨别三光气色,九流三教,无所不通,遥望江南有天子气而来,何故闭塞贤路?”
把关将校听得柴进言语不俗,便问姓名。柴进道:“某乃姓柯名引,一主一仆,投
上国而来,别无他故。”守将见说,留住柴进,差人径来睦州,报知右丞相祖士远、
参政沈寿、佥书桓逸、元帅谭高,四个跟前禀了。便使人接取柴进至睦州相见,各
叙礼罢。柴进一段话,耸动那四个,更兼柴进一表非俗,那里坦然不疑。右丞相祖
士远大喜,便叫佥书桓逸,引柴进去清溪大内朝觐。原来睦州、歙州,方腊都有行
宫大殿,内却有五府六部总制在清溪县帮源洞中。
且说柴进、燕青跟随桓逸来到清溪帝都,先来参见左丞相娄敏中。柴进高谈阔论,
一片言语,娄敏中大喜,就留柴进在相府管待。看了柴进、燕青出言不俗,知书通
礼,先自有八分欢喜。这娄敏中原是清溪县教学的先生,虽有些文章,苦不甚高,
被柴进这一段话,说得他大喜。过了一夜,次日早朝,等候方腊王子升殿。内列着
侍御、嫔妃、彩女,外列九卿四相,文武两班,殿前武士,金瓜长随侍从。当有左
丞相娄敏中出班启奏:“中原是孔夫子之乡。今有一贤士,姓柯名引,文武兼资,
智勇足备,善识天文地理,能辨六甲风云,贯通天地气色,三教九流,诸子百家,
无不通达。望天子气而来,现在朝门外,伺候我主传宣。”方腊道:“既有贤士到
来,便令白衣朝见。”各门大使传宣,引柴进到于殿下。拜舞起居,山呼万岁已毕,
宣入帘前。
方腊看见柴进一表非俗,有龙子龙孙气象,先有八分喜色。方腊问道:“贤士所言,
望天子气而来,在于何处?”柴进奏道:“臣柯引贱居中原,父母双亡,只身学业,
传先贤之秘诀,授祖师之玄文。近日夜观乾象,见帝星明朗,正照东吴。因此不辞
千里之劳,望气而来。特至江南,又见一缕五色天子之气,起自睦州。今得瞻天子
圣颜,抱龙凤之姿,挺天日之表,正应此气。臣不胜欣幸之至!”言讫再拜。方腊
道:“寡人虽有东南地土之分,近被宋江等侵夺城池,将近吾地,如之奈何?”柴
进奏道:“臣闻古人有言:‘得之易,失之易;得之难,失之难。’今陛下东南之
境,开基以来,席卷长驱,得了许多州郡。今虽被宋江侵了数处,不久气运复归于
圣上。陛下非止江南之境,他日中原社稷,亦属陛下。”
方腊见此等言语,心中大喜,敕赐锦墩命坐,管待御宴,加封为中书侍郎。自此柴
进每日得近方腊,无非用些阿谀美言谄佞,以取其事。未经半月,方腊及内外官僚,
无一人不喜柴进。次后,方腊见柴进署事公平,尽心喜爱,却令左丞相娄敏中做媒,
把金芝公主招赘柴进为驸马,封官主爵都尉。燕青改名云璧,人都称为云奉尉。柴
进自从与公主成亲之后,出入宫殿,都知内外备细。方腊但有军情重事,便宣柴进
至内宫计议。柴进时常奏说:“陛下气色真正,只被罡星冲犯,尚有半年不安。直
待并得宋江手下无了一员战将,罡星退度,陛下复兴基业,席卷长驱,直占中原之
地。”方腊道:“寡人手下爱将数员,尽被宋江杀死,似此奈何?”柴进又奏道:
“臣夜观天象,陛下气数,将星虽多数十位,不为正气,未久必亡。却有二十八宿
星象,正来辅助陛下,复兴基业。宋江伙内,亦有十数员来降。此也是数中星宿,
尽是陛下开疆展土之臣也!”方腊听了大喜。有诗为证:
蚕室当时惩太史,何人不罪李陵降?
谁知贵宠柯驸马,一念原来为宋江。
且不说柴进做了驸马,却说宋江部领大队人马军兵,离了杭州,望富阳县进发。时
有宝光国师邓元觉并元帅石宝、王、晁中、温克让五个,引了败残军马,守住富
阳县关隘,却使人来睦州求救。右丞相祖士远当差两员亲军指挥使,引一万军马,
前来策应。正指挥白钦、副指挥景德,两个都有万夫不当之勇,来到富阳县,和宝
光国师等合兵一处,占住山头。宋江等大队军马,已到七里湾,水军引着马军,一
发前进。石宝见了,上马带流星锤,拿劈风刀,离了富阳县山头,来迎宋江。
关胜正欲出马,吕方叫道:“兄长少停,看吕方和这厮斗几合。”宋江在门旗影里
看时,吕方一骑马,一枝戟,直取石宝,那石宝使劈风刀相迎。两个斗到五十合,
吕方力怯。郭盛见了,便持戟纵马,前来夹攻。那石宝一口刀战两枝戟,没半分漏
泄。正斗到至处,南边宝光国师急鸣锣收军。原来见大江里战船乘着顺风,都上滩
来,却来傍岸。怕他两处夹攻,因此鸣锣收军。吕方、郭盛缠住厮杀,那里肯放。
石宝又斗了三五合,宋兵阵上,朱仝一骑马,一条枪,又去夹攻。石宝战不过三将,
分开兵器便走。宋江鞭梢一指,直杀过富阳山岭。石宝军马,于路屯扎不住,直到
桐庐县界内。宋江连夜进兵,过白蜂岭下寨。当夜差遣解珍、解宝、燕顺、王矮虎、
一丈青取东路,李逵、项充、李衮、樊瑞、马麟取西路,各带一千步军,去桐庐县
劫寨。江里却教李俊、三阮、二童、孟康七人取水路进兵。
且说解珍等引着军兵杀到桐庐县时,已是三更天气。宝光国师正和石宝计议军务,
猛听的一声炮响,众人上马不迭。急看时,三路火起,诸将跟着石宝只顾逃命,那
里敢来迎敌。三路军马,横冲直撞杀将来。温克让上得马迟,便望小路而走,正撞
着王矮虎、一丈青。他夫妻二人一发上,把温克让横拖倒拽,活捉去了。李逵和项
充、李衮、樊瑞、马麟只顾在县里杀人放火。宋江见报,催趱军兵,拔寨都起,直
到桐庐县驻屯军马。王矮虎、一丈青献温克让请功。宋江教把温克让解赴杭州张招
讨前斩首,不在话下。
次日,宋江调兵,水陆并进,直到乌龙岭下,过岭便是睦州。此时宝光国师引着众
将,都上岭去把关隘,屯驻军马。那乌龙关隘,正靠长江,山峻水急,上立关防,
下排战舰。宋江军马近岭下屯驻,扎了寨栅。步军中差李逵、项充、李衮引五百牌
手,出哨探路。到得乌龙岭下,上面擂木炮石打将下来,不能前进,无计可施,回
报宋先锋。宋江又差阮小二、孟康、童猛、童威四个,先棹一半战船上滩。当下阮
小二带了两个副将,引一千水军,分作一百只船上,摇旗擂鼓,唱着山歌,渐近乌
龙岭边来。原来乌龙岭下那面靠山,却是方腊的水寨。那寨里也屯着五百只战船,
船上有五千来水军。为头的四个水军总管,名号浙江四龙。那四龙:

玉爪龙都总管成贵

锦鳞龙副总管翟源

冲波龙左副管乔正

戏珠龙右副管谢福
这四个总管,原是钱塘江里艄公,投奔方腊,却受三品职事。当日阮小二等乘驾船
只,从急流下水,摇上滩去。南军水寨里四个总管,已自知了,准备下五十连火排。
原来这火排,只是大松杉木穿成,排上都堆草把,草把内暗藏着硫黄焰硝引火之物,
把竹索编住,排在滩头。这里阮小二和孟康、童威、童猛四个,只顾摇上滩去。那
四个水军总管在上面看见了,各打一面乾红号旗,驾四只快船,顺水摇将下来。阮
小二看见,喝令水手放箭,那四只快船便回。阮小二便叫乘势赶上滩去,四只快船
傍滩住了,四个总管却跳上岸,许多水手们也都走了。阮小二望见滩上水寨里船广,
不敢上去,正在迟疑间,只见乌龙岭上把旗一招,金鼓齐鸣,火排一齐点着,望下
滩顺风冲将下来,背后大船一齐喊起,都是长枪挠钩,尽随火排下来。童威、童猛
见势大难近,便把船傍岸,弃了船只,爬过山边,上了山,寻路回寨。阮小二和孟
康兀自在船上迎敌,火排连烧将来。阮小二急下水时,后船赶上,一挠钩搭住。阮
小二心慌,怕吃他拿去受辱,扯出腰刀,自刎而亡。孟康见不是头,急要下水时,
火排上火炮齐发,一炮正打中孟康头盔,透顶打做肉泥。四个水军总管,却上火船
杀将下来。李俊和阮小五、阮小七都在后船,见前船失利,沿江岸杀来,只得急忙
转船,便随顺水放下桐庐岸来。
再说乌龙岭上宝光国师并元帅石宝,见水军总管得胜,乘势引军杀下岭来。水深不
能相赶,路远不能相追,宋兵复退在桐庐驻扎,南兵也收军上乌龙岭去了。
宋江在桐庐扎驻寨栅,又见折了阮小二、孟康,在帐中烦恼,寝食俱废,梦寐不安。
吴用与众将苦劝不得,阮小七、阮小五挂孝已了,自来谏劝宋江道:“我哥哥今日
为国家大事,折了性命,也强似死在梁山泊,埋没了名目。先锋主兵不须烦恼,且
请理国家大事。我弟兄两个,自去复仇。”宋江听了,稍稍回颜。次日,仍复整点
军马,再要进兵。吴用谏道:“兄长未可急性,且再寻思计策,度岭未迟。”只见
解珍、解宝便道:“我弟兄两个,原是猎户出身,巴山度岭得惯。我两个装做此间
猎户,爬上山去,放起一把火来,教那贼兵大惊,必然弃了关去。”吴用道:“此
计虽好,只恐这山险峻,难以进步,倘或失脚,性命难保。”解珍、解宝便道:“我
弟兄两个,自登州越狱上梁山泊,托哥哥福荫,做了许多年好汉,又受了国家诰命,
穿了锦袄子,今日为朝廷,便粉骨碎身,报答仁兄,也不为多。”宋江道:“贤弟
休说这凶话,只愿早早干了大功回京,朝廷不肯亏负我们。你只顾尽心竭力,与国
家出力。”
解珍、解宝便去拴束,穿了虎皮套袄,腰里各跨一口快刀,提了钢叉。两个来辞了
宋江,便取小路望乌龙岭上来。此时才有一更天气,路上撞着两个伏路小军。二人
结果了两个,到得岭下时,已有二更。听得岭上寨内更鼓分明,两个不敢从大路走,
攀藤揽葛,一步步爬上岭来。是夜月光明朗,如同白日。两个三停爬了二停之上,
望见岭上灯光闪闪。两个伏在岭门边听时,上面更鼓已打四更。解珍暗暗地叫兄弟
道:“夜又短,天色无多时了。我两个上去罢。”两个又攀援上去。正爬到岩壁崎
岖之处,悬崖峻之中,两个只顾爬上去,手脚都不闲,却把膊拴住钢叉,拖在
背后,刮得竹藤乱响,山岭上早吃人看见了。解珍正爬在山凹处,只听得上面叫声:
“着!”一挠钩正搭住解珍头髻。解珍急去腰里拔得刀出来时,上面已把他提得脚
悬了。解珍心慌,连忙一刀砍断挠钩,却从空里坠下来。可怜解珍做了半世好汉,
从这百十丈高岩上倒撞下来,死于非命。下面都是狼牙乱石,粉碎了身躯。解宝见
哥哥颠将下去,急退步下岭时,上头早滚下大小石块并短弩弓箭,从竹藤里射来。
可怜解宝为了一世猎户,做一块儿射死在乌龙岭边竹藤丛里,两个身死。
天明,岭上差人下来,将解珍、解宝尸首,就风化在岭上。探子听得备细,报与宋
先锋知道,解珍、解宝已死在乌龙岭。宋江听得又折了解珍、解宝,哭得几番昏晕,
便唤关胜、花荣点兵取乌龙岭关隘,与四个兄弟报仇。吴用谏道:“仁兄不可性急,
已死者皆是天命。若要取关,不可造次。须用神机妙策,智取其关,方可调兵遣将。”
宋江怒道:“谁想把我们弟兄手足,三停损了一停。不忍那贼们把我兄弟风化在岭
上,今夜必须提兵先去,夺尸首回来,具棺椁埋葬。”吴用阻道:“贼兵将尸风化,
诚恐有计,兄长未可造次。”宋江那里肯听军师谏劝,随即点起三千精兵,带领关
胜、花荣、吕方、郭盛四将,连夜进兵,到乌龙岭时,已是二更时分。小校报道:
“前面风化起两个人在那里,敢是解珍、解宝的尸首。”宋江纵马亲自来看时,见
两株树上,把竹竿挑起两个尸首,树上削去了一片皮,写两行大字在上,月黑不见
分晓。宋江令讨放炮火种,吹起灯来看时,上面写道:“宋江早晚也号令在此处。”
宋江看了大怒,却传令人上树去取尸首,只见四下里火把齐起,金鼓乱鸣,团团军
马围住。当前岭上,早乱箭射来。江里船内水军,都纷纷上岸来。宋江见了,叫声
苦,不知高低。急退军时,石宝当先截住去路,转过侧首,又是邓元觉杀将下来。
直使:规模有似马陵道,光景浑如落凤坡。
毕竟宋江军马怎地脱身,且听下回分解。