\chapter{呼延灼月夜赚关胜~宋公明雪天擒索超}

话说蒲东关胜,这人惯使口大刀,英雄盖世,义勇过人。当日辞了太师,统领
着一万五千人马,分为三队,离了东京,望梁山泊来。

话分两头。且说宋江与同众将,每日北京攻打城池不下。李成、闻达,那里敢
出对阵。索超箭疮深重,又未平复,更无人出战。宋江见攻打城子不破,心中纳闷:
离山已久,不见输赢。是夜在中军帐里闷坐,点上灯烛,取出玄女天书。正看之间,
猛然想起围城既久,不见有救军接应,戴宗回去,尚不见来,默然觉得神思恍惚,
寝食不安。忽小校报说:“军师来见。”吴用到得中军帐内,与宋江道:“我等众
军围许多时,如何杳无救军来到,城中又不出战?向有三骑马奔出城去,必是梁中
书使人去京师告急。他丈人蔡太师必然上紧遣兵,中间必有良将。倘用围魏救赵之
计,且不来解此处之危,反去取我梁山大寨,如之奈何?兄长不可不虑。我等先着
军士收拾,未可都退。”

正说之间,只见神行太保戴宗到来报说:“东京蔡太师拜请关菩萨玄孙蒲东郡
大刀关胜,引一彪军马,飞奔梁山泊来。寨中头领主张不定,请兄长军师早早收兵
回来,且解山寨之难。”吴用道:“虽然如此,不可急还。今夜晚间,先教步军前
行,留下两支军马,就飞虎峪两边埋伏。城中知道我等退军,必然追赶;若不如此,
我兵先乱。”宋江道:“军师言之极当。”传令便差小李广花荣,引五百军兵,去
飞虎峪左边埋伏;豹子头林冲,引五百军兵,去飞虎峪右边埋伏。再叫双鞭呼延灼,
引二十五骑马军,带着凌振,将了风火等炮,离城十数里远近,但见追兵过来,随
即施放号炮,令其两下伏兵齐去并杀追兵。一面传令前队退兵,倒拖旌旗,不鸣战
鼓,却如雨散云行,遇兵勿战,慢慢退回。步军队里,半夜起来,次第而行。直至
次日巳牌前后,方才尽退。

城上望见宋江军马,手拖旗幡,肩担刀斧,纷纷滚滚,拔寨都起,有还山之状。
城上看了仔细,报与梁中书知道:“梁山泊军马,今日尽数收兵,都回去了。”梁
中书听的,随即唤李成、闻达商议。闻达道:“想是京师救军去取他梁山泊,这厮
们恐失巢穴,慌忙归去。可以乘势追杀,必擒宋江。”说犹未了,城外报马到来,
赍东京文字,约会引兵去取贼巢。他若退兵,可以速追。梁中书便叫李成、闻达各
带一支军马,从东西两路追赶宋江军马。

且说宋江引兵退回,见城中调兵追赶,舍命便走。直退到飞虎峪那边,只听的
背后火炮齐响。李成、闻达吃了一惊,勒住战马看时,后面只见旗幡对刺,战鼓乱
鸣。李成、闻达火急回军,左手下撞出小李广花荣,右手下撞出豹子头林冲,各引
五百军马,两边杀来。措手不及,知道中了奸计,火速回军。前面又撞出呼延灼,
引着一支马军,大杀一阵,杀的李成、闻达金盔倒纳,衣甲飘零,退入城中,闭门
不出。宋江军马,次第而回。早转近梁山泊边,却好迎着丑郡马宣赞拦路。宋江约
住军兵,权且下寨,暗地使人从偏僻小路,赴水上山报知,约会水陆军兵,两下救
应。

且说水寨内头领船火儿张横,与兄弟浪里白跳张顺当时议定:“我和你弟兄两
个,自来寨中,不曾建功。只看着别人夸能说会,倒受他气。如今蒲东大刀关胜,
三路调军,打我寨栅,不若我和你两个,先去劫了他寨,捉得关胜,立这件大功,
众兄弟面前,也好争口气。”张顺道:“哥哥,我和你只管的些水军,倘或不相救
应,枉惹人耻笑。”张横道:“你若这般把细,何年月日能够建功?你不去便罢,
我今夜自去。”张顺苦谏不听。当夜张横点了小船五十余只,每船上只有三五人,
浑身都是软战,手执苦竹枪,各带蓼叶刀,趁着月光微明,寒露寂静,把小船直抵
旱路。此时约有二更时分。

却说关胜正在中军帐里点灯看书,有伏路小校悄悄来报:“芦花荡里,约有小
船四五十只,人人各执长枪,尽去芦苇里面两边埋伏,不知何意,特来报知。”关
胜听了,微微冷笑。当时暗传号令,教众军俱各如此准备。三军得令,各自潜伏。
且说张横将引三二百人,从芦苇中间藏踪蹑迹,直到寨边,拔开鹿角,径奔中军。
望见帐中灯烛荧煌,关胜手拈髭髯,坐看兵书。张横暗喜,手长枪,抢入帐房里
来。旁边一声锣响,众军喊动,如天崩地塌,山倒江翻,吓的张横倒拖长枪,转身
便走。四下里伏兵乱起,可怜会水张横,怎脱平川罗网。二三百人,不曾走的一个,
尽数被缚,推到帐前。关胜看了,笑骂:“无端草贼,安敢侮吾!”将张横陷车盛
了,其余者尽数监了;直等捉了宋江,一并解上京师。

不说关胜捉了张横,却说水寨内三阮头领正在寨中商议,使人去宋江哥哥处听
令,只见张顺到来,报说:“我哥哥因不听小弟苦谏,去劫关胜营寨,不料被捉,
囚车监了。”阮小七听了,叫将起来,说道:“我兄弟们同死同生,吉凶相救,你
是他嫡亲兄弟,却怎地教他独自去,被人捉了?你不去救,我弟兄三个自去救他。”
张顺道:“为不曾得哥哥将令,却不敢轻动。”阮小七道:“若等将令来时,你哥
哥吃他剁做八段。”阮小二、阮小五都道:“说的是。”张顺逆他三个不过,只得
依他。当夜四更,点起大小水寨头领,各架船一百余只,一齐杀奔关胜寨来。岸上
小军,望见水面上战船如蚂蚁相似,都傍岸边,慌忙报知主帅。关胜笑道:“无见
识贼奴,何足为虑!”随即唤首将,附耳低言,如此如此。

且说三阮在前,张顺在后,呐声喊,抢入寨来。只见寨内枪刀竖立,旌旗不倒,
并无一人。三阮大惊,转身便走。帐前一声锣响,左右两边,马军步军,分作八路,
簸箕掌,栲栳圈,重重迭迭,围裹将来。张顺见不是头,扑通的先跳下水去。三阮
夺路便走,急到的水边,后军赶上,挠钩齐下,套索飞来,把这活阎罗阮小七搭住,
横拖倒拽捉去了。阮小二、阮小五、张顺,却得混江龙李俊带的童威、童猛死救回
去。

不说阮小七被捉,囚在陷车之中。且说水军报上梁山泊来,刘唐便使张顺从水
路里直到宋江寨中,报说这个消息。宋江便与吴用商议,怎生退的关胜。吴用道:
“来日决战,且看胜败如何。”说犹未了,猛听得战鼓齐鸣,却是丑郡马宣赞部领
三军,直到大寨。宋江举众出迎,看了宣赞在门旗下勒战,便唤首将:“那个出马,
先拿这厮?”只见小李广花荣拍马持枪,直取宣赞。宣赞舞刀来迎,一来一往,一
上一下,斗到十合,花荣卖个破绽,回马便走。宣赞赶来,花荣就了事环带住钢枪,
拈弓取箭,侧坐雕鞍,轻舒猿臂,翻身一箭。宣赞听得弓弦响,却好箭来,把刀只
一隔,铮地一声响,射在刀面上。花荣见一箭不中,再取第二枝箭,看的较近,望
宣赞胸膛上射来。宣赞镫里藏身,又躲过了。宣赞见他弓箭高强,不敢追赶,霍地
勒回马,跑回本阵。花荣见他不赶,连忙便勒转马头,望宣赞赶来。又取第三枝箭,
望得宣赞后心较近,再射一箭。只听得铛地一声响,正射在背后护心镜上。宣赞慌
忙驰马入阵,便使人报与关胜。关胜得知,便唤小校:“快牵过战马来!”那匹马,
头至尾长一丈,蹄至脊高八尺,浑身上下,没一根杂毛,纯是火炭般赤。拴一副皮
甲,束三条肚带。关胜全装披挂,绰刀上马,直临阵前。门旗开处,便乃出马,有
《西江月》一首为证:

汉国功臣苗裔,三分良将玄孙。绣旗飘挂动天兵,金甲绿袍相称。

赤兔马
腾腾紫霞,青龙刀凛凛寒冰。蒲东郡内产豪英,义勇大刀关胜。

宋江看了关胜一表非俗,与吴用暗暗地喝采,回头与众多良将道:“将军英雄,
名不虚传!”说言未了,林冲忿怒,便道:“我等弟兄,自上梁山泊,大小五七十
阵,未尝挫了锐气,军师何故灭自己威风!”说罢,便挺枪出马,直取关胜。关胜
见了,大喝道:“水泊草寇,汝等怎敢背负朝廷!单要宋江与吾决战。”

宋江在门旗下喝住林冲,纵马亲自出阵,欠身与关胜施礼,说道:“郓城小吏
宋江到此谨参,惟将军问罪。”关胜道:“汝为小吏,安敢背叛朝廷?”宋江答道:
“盖为朝廷不明,纵容奸臣当道,谗佞专权,设除滥官污吏,陷害天下百姓。宋江
等替天行道,并无异心。”关胜大喝:“天兵到此,尚然抗拒,巧言令色,怎敢瞒
吾!若不下马受降,着你粉骨碎身!”霹雳火秦明听得大怒,手舞狼牙棍,纵坐下
马,直抢过来。关胜也纵马出迎,来斗秦明。林冲怕他夺了头功,猛可里飞抢过来,
径奔关胜。三骑马向征尘影里,转灯般厮杀。

宋江看了,恐伤关胜,便教鸣金收军。林冲、秦明回马阵前,说道:“正待擒
捉这厮,兄长何故收军罢战?”宋江道:“贤弟,我等忠义自守,以强欺弱,非所
愿也。纵使阵上捉他,此人不伏,亦乃惹人耻笑。吾看关胜英勇之将,世本忠臣,
乃祖为神,若得此人上山,宋江情愿让位。”林冲、秦明都不喜欢。当日两边各自
收兵。

且说关胜回到寨中,下马卸甲,心中暗忖道:“我力斗二将不过,看看输与他,
宋江倒收了军马,不知主何意?”却叫小军推出陷车中张横、阮小七过来,问道:
“宋江是个郓城小吏,你这厮们如何伏他?”阮小七应道:“俺哥哥山东、河北驰
名,都称做及时雨呼保义宋公明。你这厮不知礼义之人,如何省得!”关胜低头不
语,且教推过陷车。当晚寨中纳闷,坐卧不安,走出中军观看,月色满天,霜华遍
地,嗟叹不已。有伏路小校前来报说:“有个胡须将军,匹马单鞭,要见元帅。”
关胜道:“你不问他是谁!”小校道:“他又没衣甲军器,并不肯说姓名,只言要
见元帅。”关胜道:“既是如此,与我唤来。”没多时,来到帐中,拜见关胜。关
胜看了,有些面熟,灯光之下,略也认得,便问是谁。那人道:“乞退左右。”关
胜道:“不妨。”那人道:“小将呼延灼的便是。先前曾与朝廷统领连环马军,征
进梁山泊。谁想中贼奸计,失陷了军机,不能还乡。听得将军到来,不胜之喜。早
间宋江在阵上,林冲、秦明待捉将军,宋江火急收军,诚恐伤犯足下。此人素有归
顺之意,独奈众贼不从。暗与呼延灼商议,正要驱使众人归顺。将军若是听从,明
日夜间,轻弓短箭,骑着快马,从小路直入贼寨,生擒林冲等寇,解赴京师,共立
功勋。”关胜听罢大喜,请入帐,置酒相待。备说宋江专以忠义为主,不幸从贼无
辜。二人递相剖露衷情,并无疑心。

次日,宋江举众搦战,关胜与呼延灼商议:“今日可先赢首将,晚间可行此计。”
且说呼延灼借副衣甲穿了,彼各上马,都到阵前。宋江阵上大骂呼延灼道:“山寨
不曾亏负你半分,因何夤夜私去?”呼延灼回道:“汝等草寇,成何大事!”宋江
便令镇三山黄信出马,仗丧门剑,驱坐下马,直奔呼延灼。两马相交,斗不到十合,
呼延灼手起一鞭,把黄信打落马下。宋江阵上众军抢出来,扛了回去。关胜大喜,
令大小三军一齐掩杀。呼延灼道:“不可追掩。吴用那厮,广有神机,若还赶杀,
恐贼有计。”关胜听了,火急收军,都回本寨。到中军帐里,置酒相待,动问镇三
山黄信之事。呼延灼道:“此人原是朝廷命官,青州都监,与秦明、花荣一时落草。
今日先杀此贼,挫灭威风,今晚偷营,必然成事。”关胜大喜,传下将令:教宣赞、
郝思文两路接应;自引五百马军,轻弓短箭,叫呼延灼引路。至夜二更起身,三更
前后,直奔宋江寨中,炮响为号,里应外合,一齐进兵。

是夜月光如昼。黄昏时候,披挂已了,马摘鸾铃,人披软战,军卒衔枚疾走,
一齐乘马,呼延灼当先引路,众人跟着。转过山径,约行了半个更次,前面撞见三
五十个伏路小军,低声问道:“来的不是呼将军么?宋公明差我等在此迎接。”呼
延灼喝道:“休言语,随在我马后走!”呼延灼纵马先行,关胜乘马在后。又转过
一层山嘴,只见呼延灼把枪尖一指,远远地一碗红灯。关胜勒住马,问道:“有红
灯处是那里?”呼延灼道:“那里便是宋公明中军。”急催动人马。将近红灯,忽
听得一声炮响,众军跟定关胜,杀奔前来。到红灯之下看时,不见一个,便唤呼延
灼时,亦不见了。关胜大惊,知道中计,慌忙回马,听得四边山上,一齐鼓响锣鸣。
正是慌不择路,众军各自逃生。关胜连忙回马时,只剩得数骑马军跟着。转出山嘴,
又听得树林边脑后一声炮响,四下里挠钩齐出,把关胜拖下雕鞍,夺了刀马,卸去
衣甲,前推后拥,拿投大寨里来。

却说林冲、花荣自引一枝军马,截住郝思文,回头厮杀。月光之下,遥见郝思
文怎生打扮,有《西江月》为证:

千丈凌云豪气,一团筋骨精神。横枪跃马荡征尘,四海英雄难近。

身着战
袍锦绣,七星甲挂龙鳞。天丁元是郝思文,飞马当前出阵。
林冲大喝道:“你主将关胜,中计被擒,你这无名小将,何不下马受缚?”郝思文
大怒,直取林冲,二马相交,斗无数合,花荣挺枪助战,郝思文势力不加,回马便
走,肋后撞出个女将一丈青扈三娘,撒起红绵套索,把郝思文拖下马来。步军向前,
一齐捉住,解投大寨。

话分两处。这边秦明、孙立,自引一支军马去捉宣赞,当路正逢此人。那宣赞
怎生打扮,有《西江月》为证:

卷短黄须发,凹兜黑墨容颜。睁开怪眼似双环,鼻孔朝天仰面。

手内钢
刀耀雪,护身铠甲连环。海骝赤马锦鞍鞯,郡马英雄宣赞。
当下宣赞拍马大骂:“草贼匹夫,当吾者死,避我者生!”秦明大怒,跃马挥狼牙
棍,直取宣赞。二马相交,约斗数合。孙立侧首过来,宣赞慌张,刀法不依古格,
被秦明一棍,搠下马来。三军齐喊一声,向前捉住。再有扑天雕李应,引领大小军
兵,抢奔关胜寨内来,先救了张横、阮小七,并被擒水军人等,夺去一应粮草马匹,
却去招安四下败残人马。

宋江会众上山,此时东方渐明。忠义堂上分开坐次,早把关胜、宣赞、郝思文
分投解来。宋江见了,慌忙下堂,喝退军卒,亲解其缚,把关胜扶在正中交椅上,
纳头便拜,叩首伏罪,说道:“亡命狂徒,冒犯虎威,望乞恕罪。”关胜连忙答礼,
闭口无言,手脚无措。呼延灼亦向前来伏罪道:“小可既蒙将令,不敢不依,万望
将军免恕虚诳之罪。”关胜看了一班头领,义气深重,回顾与宣赞、郝思文道:“我
们被擒在此,所事若何?”二人答道:“并听将令。”关胜道:“无面还京,俺三
人愿早赐一死!”宋江道:“何故发此言?将军倘蒙不弃微贱,一同替天行道。若
是不肯,不敢苦留,只今便送回京。”关胜道:“人称忠义宋公明,话不虚传。今
日我等有家难奔,有国难投,愿在帐下,为一小卒。”宋江大喜。当日一面设筵庆
贺,一边使人招安逃窜败军,又得了五七千人马。军内有老幼者,随即给散银两,
便放回家;一边差薛永赍书往蒲东,搬取关胜老小,都不在话下。

宋江正饮宴间,默然想起卢员外、石秀陷在北京,潸然泪下。吴用道:“兄长
不必忧心,吴用自有措置。只过今晚,来日再起军兵,去打北京,必然成事。”关
胜便起身说道:“小将无可报答爱我之恩,愿为前部。”宋江大喜。次日早晨传令,
就教宣赞、郝思文,拨回旧有军马,便为前部先锋;其余原打北京头领,不缺一个。
再差李俊、张顺,将带水战盔甲随去,以次再望北京进发。

这里却说梁中书在城中,正与索超起病饮酒。只见探马报道:“关胜、宣赞、
郝思文并众军马,俱被宋江捉去,已入伙了。梁山泊军马,现今又到。”梁中书听
得,唬得目瞪痴呆,手脚无措。只见索超禀道:“前者中贼冷箭,今番且复此仇。”
梁中书随即赏了索超,便教引本部人马出城迎敌。李成、闻达随后调军接应。其时
正是仲冬天气,时候正冷,连日彤云密布,朔风乱吼。宋江兵到,索超直至飞虎峪
下寨。次日,引兵迎敌,宋江引前部吕方、郭盛,上高阜处看关胜厮杀。三通战鼓
罢,关胜出阵。只见对面索超出马。当时索超见了关胜,却不认得。随征军卒说道:
“这个来的,便是新背反的大刀关胜。”索超听了,并不打话,直抢过来,径奔关
胜。关胜也拍马舞刀来迎。两个斗无十合,李成正在中军,看见索超斧怯,战关胜
不下,自舞双刀出阵,夹攻关胜。这边宣赞、郝思文见了,各持兵器,前来助战,
五骑马搅做一块。宋江在高阜看见,鞭梢一指,大军卷杀过去,李成军马大败亏输,
杀得七断八绝,连夜退入城去,坚闭不出。宋江催兵直抵城下,扎住军马。次日,
索超亲引一支军马,出城冲突。吴用见了,便教军校迎敌戏战:“他若追来,乘势
便退。”此时索超又得了这一阵,欢喜入城。

当晚彤云四合,纷纷雪下,吴用已有计了,暗差步军去北京城外,靠山边河路
狭处,掘成陷坑,上用土盖。是夜雪急风严,平明看时,约有二尺深雪。城上望见
宋江军马,各有惧色,东西栅立不定。索超看了,便点三百军马,就时追出城来。
宋江军马四散奔波而走。却教水军头领李俊、张顺,身披软战,勒马横枪,前来迎
敌。却才与索超交马,弃枪便走,特引索超奔陷坑边来。索超是个性急的,那里照
顾。这里一边是路,一边是涧。李俊弃马,跳入涧中去了,向着前面,口里叫道:
“宋公明哥哥快走!”索超听了,不顾身体,飞马抢过阵来。山背后一声炮响,索
超连人和马,将下去。后面伏兵齐起,这索超便有三头六臂,也须七损八伤。正
是:烂银深盖藏圈套,碎玉平铺作陷坑。

毕竟急先锋索超性命如何,且听下回分解。