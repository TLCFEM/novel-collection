\chapter{梁山泊好汉劫法场~白龙庙英雄小聚义}

话说当时晁盖并众人听了,请问军师道:“这封书如何有脱卯处?”吴用说道:
“早间戴院长将去的回书,是我一时不仔细,见不到处,才使的那个图书,不是玉
箸篆文‘翰林蔡京’四字?只是这个图书,便是教戴宗吃官司。”金大坚便道:“小
弟每每见蔡太师书缄,并他的文章,都是这样图书。今次雕得无纤毫差错,如何有
破绽?”吴学究道:“你众位不知,如今江州蔡九知府是蔡太师儿子,如何父写书
与儿子,却使个讳字图书,因此差了。是我见不到处。此人到江州,必被盘诘,问
出实情,却是利害。”晁盖道:“快使人去赶唤他回来,别写如何?”吴学究道:
“如何赶得上?他作起神行法来,这早晚已走过五百里了。只是事不宜迟,我们只
得恁地,可救他两个。”晁盖道:“怎生去救?用何良策?”吴学究便向前与晁盖
耳边说道:“这般这般,如此如此。主将便可暗传下号令,与众人知道,只是如此
动身,休要误了日期。”众多好汉得了将令,各各拴束行头,连夜下山,望江州来,
不在话下。说话的如何不说计策出,管教下面便见。

且说戴宗扣着日期,回到江州,当厅下了回书。蔡九知府见了戴宗如期回来,
好生欢喜,先取酒来赏了三钟,亲自接了回书,便道:“你曾见我太师么?”戴宗
禀道:“小人只住得一夜便回了,不曾得见恩相。”知府拆开封皮,看见前面说信
笼内许多物件都收了。背后说妖人宋江,今上自要他看,可令牢固陷车,盛载密切,
差的当人员,连夜解上京师,沿途休教走失。书尾说黄文炳早晚奏过天子,必然自
有除授。蔡九知府看了,喜不自胜,叫取一锭二十五两花银赏了戴宗;一面分付教
合陷车,商量差人解发起身。戴宗谢了,自回下处,买了些酒肉,来牢里看觑宋江,
不在话下。

且说蔡九知府催并合成陷车,过得一二日,正要起程,只见门子来报道:“无
为军黄通判特来相探。”蔡九知府叫请至后堂相见,又送些礼物、时新酒果。知府
谢道:“累承厚意,何以克当。”黄文炳道:“村野微物,何足挂齿。”知府道:
“恭喜早晚必有荣除之庆。”黄文炳道:“公相何以知之?”知府道:“昨日下书
人已回,妖人宋江,教解京师。通判只在早晚奏过今上,升擢高任。家尊回书,备
说此事。”黄文炳道:“既是恁地,深感恩相主荐。那个人下书,真乃神行人也。”
知府道:“通判如不信时,就教观看家书,显得下官不谬。”黄文炳道:“小生只
恐家书不敢擅看。如若相托,求借一观。”知府便道:“通判乃心腹之交,看有何
妨。”便令从人取过家书,递与黄文炳看。

黄文炳接书在手,从头至尾读了一遍。卷过来,看了封皮,又见图书新鲜,黄
文炳摇着头道:“这封书不是真的。”知府道:“通判错矣。此是家尊亲手笔迹,
真正字体,如何不是真的?”黄文炳道:“公相容复:往常家书来时,曾有这个图
书么?”知府道:“往常来的家书,却不曾有这个图书,只是随手写的。今番以定
是图书匣在手边,就便印了这个图书在封皮上。”黄文炳道:“相公休怪小生多言,
这封书被人瞒过了相公。方今天下盛行苏、黄、米、蔡四家字体,谁不习学得?况
兼这个图书,是令尊恩相做翰林学士时使出来,法帖文字上,多有人曾见。如今升
转太师丞相,如何肯把翰林图书使出来?更兼亦是父寄书与子,须不当用讳字图书。
令尊太师恩相,是个识穷天下、高明远见的人,安肯造次错用?相公不信小生之言,
可细细盘问下书人,曾见府里谁来。若说不对,便是假书。休怪小生多说,因蒙错
爱至厚,方敢僭言。”蔡九知府听了,说道:“这事不难,此人自来不曾到东京,
一盘问便显虚实。”

知府留住黄文炳在屏风背后坐地,随即升厅,叫唤戴宗有委用的事。当下做公
的领了钧旨,四散去寻。有诗为证:
反诗假信事相牵,为与梁山盗结连。
不是黄蜂针痛处,蔡龟虽大总徒然。

且说戴宗自回到江州,先去牢里见了宋江,附耳低言,将前事说了,宋江心中
暗喜。次日,又有人请去酌杯,戴宗正在酒肆中吃酒,只见做公的四下来寻。当时
把戴宗唤到厅上,蔡九知府问道:“前日有劳你走了一遭,真个办事,不曾重重赏
你。”戴宗答道:“小人是承奉恩相差使的人,如何敢怠慢?”知府道:“我正连
日事忙,未曾问得你个仔细。你前日与我去京师,那座门入去?”戴宗道:“小人
到东京时,那日天色晚了,不知唤做甚么门。”知府又道:“我家府里门前,谁接
着你?留你在那里歇?”戴宗道:“小人到府前寻见一个门子,接了书入去。少刻,
门子出来,交收了信笼,着小人自去寻客店里歇了。次日早五更去府门前伺候时,
只见那门子回书出来。小人怕误了日期,那里敢再问备细,慌忙一径来了。”知府
再问道:“你见我府里那个门子,却是多少年纪?或是黑瘦,也白净肥胖?长大,也
是矮小?有须的,也是无须的?”戴宗道:“小人到府里时,天色黑了。次早回时,
又是五更时候,天色昏暗。不十分看得仔细。只觉不恁么长,中等身材,敢是有些
髭须。”

知府大怒,喝一声:“拿下厅去!”旁边走过十数个狱卒牢子,将戴宗驱翻在
当面。戴宗告道:“小人无罪。”知府喝道:“你这厮该死!我府里老门子王公已
死了数年,如今只是个小王看门,如何却道他年纪大,有髭髯?况兼门子小王不能
够入府堂里去,但有各处来的书信缄帖,必须经由府堂里张干办,方才去见李都管,
然后达知里面,才收礼物。便要回书,也须得伺候三日。我这两笼东西,如何没个
心腹的人出来,问你个常便备细,就胡乱收了。我昨日一时间仓卒,被你这厮瞒过
了。你如今只好好招说这封书那里得来!”戴宗道:“小人一时心慌,要赶程途,
因此不曾看得分晓。”蔡九知府喝道:“胡说!这贼骨头,不打如何肯招?左右与我
加力打这厮!”狱卒牢子情知不好,觑不得面皮,把戴宗捆翻,打得皮开肉绽,鲜
血迸流。戴宗捱不过拷打,只得招道:“端的这封书是假的。”知府道:“你这厮
怎地得这封假书来?”

戴宗告道:“小人路经梁山泊过,走出那一伙强人来,把小人劫了,绑缚上山,
要割腹剖心。去小人身上搜出书信看了,把信笼都夺了,却饶了小人。情知回乡不
得,只要山中乞死,他那里却写这封书与小人,回来脱身。一时怕见罪责,小人瞒
了恩相。”知府道:“是便是了,中间还有些胡说,眼见得你和梁山泊贼人通同造
意,谋了我信笼物件,却如何说这话?再打那厮!”戴宗由他拷讯,只不肯招和梁
山泊通情。蔡九知府再把戴宗拷讯了一回,语言前后相同,说道:“不必问了。取
具大枷枷了,下在牢里。”却退厅来称谢黄文炳道:“若非通判高见,下官险些儿
误了大事。”黄文炳又道:“眼见得这人也结连梁山泊,通同造意,谋叛为党,若
不祛除,必为后患。”知府道:“便把这两个问成了招状,立了文案,押去市曹斩
首,然后写表申朝。”黄文炳道:“相公高见极明。似此,一者朝廷见喜,知道相
公干这件大功;二者免得梁山泊草寇来劫牢。”知府道:“通判高见甚远,下官自
当动文书,亲自保举通判。”当日管待了黄文炳,送出府门,自回无为军去了。

次日,蔡九知府升厅,便叫当案孔目来分付道:“快教迭了文案,把这宋江、
戴宗的供状招款粘连了。一面写下犯由牌,教来日押赴市曹,斩首施行。自古谋逆
之人,决不待时,斩了宋江、戴宗,免致后患。”当案却是黄孔目,本人与戴宗颇
好,却无缘便救他,只替他叫得苦。当日禀道:“明日是个国家忌日,后日又是七
月十五日中元之节,皆不可行刑。大后日亦是国家景命。直至五日后,方可施行。”
一者天幸救济宋江,二乃梁山泊好汉未至。

蔡九知府听罢,依准黄孔目之言,直待第六日早晨,先差人去十字路口,打扫
了法场,饭后点起土兵和刀仗刽子,约有五百余人,都在大牢门前伺候。巳牌时候,
狱官禀了知府,亲自来做监斩官。黄孔目只得把犯由牌呈堂,当厅判了两个斩字,
便将片芦席贴起来。江州府众多节级牢子虽然和戴宗、宋江过得好,却没做道理救
得他,众人只替他两个叫苦。当时打扮已了,就大牢里把宋江、戴宗两个扎起,
又将胶水刷了头发,绾个鹅梨角儿,各插上一朵红绫子纸花。驱至青面圣者神案前,
各与了一碗长休饭、永别酒。吃罢,辞了神案,漏转身来,搭上利子。六七十个狱
卒早把宋江在前,戴宗在后,推拥出牢门前来。宋江和戴宗两个面面厮觑,各做声
不得。宋江只把脚来跌,戴宗低了头只叹气。江州府看的人,真乃压肩迭背,何止
一二千人。但见:

愁云荏苒,怨气氛氲。头上日色无光,四下悲风乱吼。缨枪对对,数声鼓响丧
三魂;棍棒森森,几下锣鸣催七魄。犯由牌高贴,人言此去几时回;白纸花双摇,
都道这番难再活。长休饭,嗓内难吞;永别酒,口中怎咽!狰狞刽子仗钢刀,丑恶
押牢持法器。皂纛旗下,几多魍魉跟随;十字街头,无限强魂等候。监斩官忙施号
令,仵作子准备扛尸。

刽子叫起恶杀,都来将宋江和戴宗前推后拥,押到市曹十字路口,团团枪棒围
住,把宋江面南背北,将戴宗面北背南,两个纳坐下,只等午时三刻,监斩官到来
开刀。那众人仰面看那犯由牌上写道:“江州府犯人一名宋江,故吟反诗,妄造妖
言,结连梁山泊强寇,通同造反,律斩。犯人一名戴宗,与宋江暗递私书,勾结梁
山泊强寇,通同谋叛,律斩。监斩官江州府知府蔡某。”那知府勒住马,只等报来。

只见法场东边一伙弄蛇的丐者,强要挨入法场里看,众土兵赶打不退。正相闹
间,只见法场西边一伙使枪棒卖药的,也强挨将入来。土兵喝道:“你那伙人好不
晓事,这是那里?强挨入来要看!”那伙使枪棒的说道:“你倒鸟村,我们冲州撞
府,那里不曾去,到处看出人。便是京师天子杀人,也放人看。你这小去处,砍得
两个人,闹动了世界,我们便挨入来看一看,打甚么鸟紧!”正和土兵闹将起来,
监斩官喝道:“且赶退去,休放过来!”闹犹未了,只见法场南边一伙挑担的脚夫,
又要挨将入来,土兵喝道:“这里出人,你挑那里去?”那伙人说道:“我们挑东
西送与知府相公去的,你们如何敢阻当我?”土兵道:“便是相公衙里人,也只得
去别处过一过。”那伙人就歇了担子,都掣了匾担,立在人丛里看。只见法场北边
一伙客商,推两辆车子过来,定要挨入法场上来。土兵喝道:“你那伙人那里去?”
客人应道:“我们要赶路程,可放我等过去。”土兵道:“这里出人,如何肯放你?
你要赶路程,从别路过去。”那伙客人笑道:“你倒说得好!俺们便是京师来的人,
不认得你这里鸟路,只是从这大路走。”土兵那里肯放?那伙客人齐齐地挨定了不
动,四下里吵闹不住。这蔡九知府见禁治不得,又见这伙客人都盘在车子上立定了
看。

没多时,法场中间人分开处,一个报,报道一声:“午时三刻!”监斩官便道:
“斩讫报来。”两势下刀棒刽子,便去开枷,行刑之人,执定法刀在手。说时迟,
一个个要见分明;那时快,闹攘攘一齐发作。只见那伙客人在车子上听得“斩”字,
数内一个客人便向怀中取出一面小锣儿,立在车子上当当地敲得两三声。四下里一
齐动手。有诗为证:
闲来乘兴入江楼,渺渺烟波接素秋。
呼酒谩浇千古恨,吟诗欲泻百重愁。
雁书不遂英雄志,失脚翻成狴犴囚。
搔动梁山诸义士,一齐云拥闹江州。

又见十字路口茶坊楼上一个虎形黑大汉,脱得赤条条的,两只手握两把板斧,
大吼一声,却似半天起个霹雳,从半空中跳将下来。手起斧落,早砍翻了两个行刑
的刽子,便望监斩官马前砍将来。众土兵急待把枪去搠时,那里拦当得住?众人且
簇拥蔡九知府逃命去了。

只见东边那伙弄蛇的丐者,身边都掣出尖刀,看着土兵便杀;西边那伙使枪棒
的,大发喊声,只顾乱杀将来,一派杀倒土兵狱卒;南边那伙挑担的脚夫,抡起匾
担,横七竖八,都打翻了土兵和那看的人;北边那伙客人,都跳下车来,推过车子,
拦住了人。两个客商钻将入来,一个背了宋江,一个背了戴宗。其余的人,也有取
出弓箭来射的,也有取出石子来打的,也有取出标枪来标的。原来扮客商的这伙,
便是晁盖、花荣、黄信、吕方、郭盛;那伙扮使枪棒的,便是燕顺、刘唐、杜迁、
宋万;扮挑担的,便是朱贵、王矮虎、郑天寿、石勇;那伙扮丐者的,便是阮小二、
阮小五、阮小七、白胜——这一行梁山泊共是十七个头领到来,带领小喽罗一百余
人,四下里杀将起来。

只见那人丛里那个黑大汉,抡两把板斧,一味地砍将来,晁盖等却不认得,只
见他第一个出力,杀人最多。晁盖猛省起来:戴宗曾说一个黑旋风李逵,和宋三郎
最好,是个莽撞之人。晁盖便叫道:“前面那好汉,莫不是黑旋风?”那汉那里肯
应,火杂杂地抡着大斧,只顾砍人。晁盖便叫背宋江、戴宗的两个小喽罗,只顾跟
着那黑大汉走。当下去十字街口,不问军官百姓,杀得尸横遍野,血流成渠,推倒
倾翻的,不计其数。众头领撇了车轮担仗,一行人尽跟了黑大汉,直杀出城来。背
后花荣、黄信、吕方、郭盛,四张弓箭,飞蝗般望后射来。那江州军民百姓,谁敢
近前。这黑大汉直杀到江边来,身上血溅满身,兀自在江边杀人。晁盖便挺朴刀叫
道:“不干百姓事,休只管伤人!”那汉那里来听叫唤,一斧一个,排头儿砍将去。
约莫离城沿江上也走了五七里路,前面望见尽是淘淘一派大江,却无了旱路。晁盖
看见,只叫得苦,那黑大汉方才叫道:“不要慌,且把哥哥背来庙里。”

众人都来看时,靠江边一所大庙,两扇门紧紧闭着。黑大汉两斧砍开,便抢入
来。晁盖众人看时,两边都是老桧苍松,林木遮映,前面牌额上四个金书大字,写
道:“白龙神庙。”小喽罗把宋江、戴宗背到庙里歇下,宋江方才敢开眼,见了晁
盖等众人,哭道:“哥哥,莫不是梦中相会?”晁盖便劝道:“恩兄不肯在山,致
有今日之苦。这个出力杀人的黑大汉是谁?”宋江道:“这个便是叫做黑旋风李逵。
他几番就要大牢里放了我,却是我怕走不脱,不肯依他。”晁盖道:“却是难得这
个人出力最多,又不怕刀斧箭矢。”花荣便叫:“且将衣服与俺二位兄长穿了。”

正相聚间,只见李逵提着双斧,从廊下走出来。宋江便叫住道:“兄弟那里去?”
李逵应道:“寻那庙祝,一发杀了,叵耐那厮不来接我们,倒把鸟庙门闭上了。我
指望拿他来祭门,却寻那厮不见。”宋江道:“你且来,先和我哥哥头领相见。”
李逵听了,丢了双斧,望着晁盖跪了一跪,说道:“大哥休怪铁牛粗卤。”与众人
都相见了,却认得朱贵是同乡人,两个大家欢喜。花荣便道:“哥哥,你教众人只
顾跟着李大哥走,如今来到这里,前面又是大江拦截住,断头路了,却又没一只船
接应,倘或城中官军赶杀出来,却怎生迎敌?将何接济?”李逵便道:“不要慌,
我与你们再杀入城去,和那个鸟蔡九知府一发都砍了便走。”戴宗此时方才苏醒,
便叫道:“兄弟,使不得莽性,城里有五七千军马,若杀入去,必然有失。”阮小
七便道:“远望隔江,那里有数只船在岸边,我兄弟三个赴水过去,夺那几只船过
来载众人如何?”晁盖道:“此计是最上着。”

当时阮家三弟兄都脱剥了衣服,各人插把尖刀,便钻入水里去。约莫赴开得半
里之际,只见江面上溜头流下三只棹船,吹风胡哨,飞也似摇将来。众人看时,见
那船上各有十数个人,都手里拿着军器,众人却慌将起来。宋江听得说了,便道:
“我命里这般合苦也!”奔出庙前看时,只见当头那只船上坐着一条大汉,倒提一
把明晃晃五股叉,头上挽个空心红,一点儿,下面拽起条白绢水,口里吹着胡
哨。宋江看时,不是别人,正是:

东去长江万里,内中一个雄夫。面如傅粉体如酥,履水如同平土。胆大能探禹
穴,心雄欲摘骊珠。翻波跳浪性如鱼,张顺名传千古。

当时张顺在船头上看见喝道:“你那伙是甚么人?敢在白龙庙里聚众?”宋江
挺身出庙前说道:“兄弟救我。”张顺等见是宋江,大叫道:“好了!”那三只棹
船飞也似摇到岸边,三阮看见,也赴过来。一行众人都上岸来到庙前。宋江看见张
顺自引十数个壮汉在那只船头上。张横引着穆弘、穆春、薛永,带十数个庄客在一
只船上。第三只船上,李俊引着李立、童威、童猛,也带十数个卖盐火家,都各执
枪棒上岸来。张顺见了宋江,喜从天降,便拜道:“自从哥哥吃官司,兄弟坐立不
安,又无路可救。近日又听得拿了戴院长。李大哥又不见面。我只得去寻了我哥哥,
引到穆太公庄上,叫了许多相识。今日我们正要杀入江州,要劫牢救哥哥,不想仁
兄已有好汉们救出,来到这里。不敢拜问,这伙豪杰,莫非是梁山泊义士晁天王么?”
宋江指着上首立的道:“这个便是晁盖哥哥,你等众位都来庙里叙礼则个。”张顺
等九人,晁盖等十七人,宋江、戴宗、李逵,共是二十九人,都入白龙庙聚会。这
个唤做白龙庙小聚会。

当下二十九筹好汉,各各讲礼已罢,只见小喽罗慌慌忙忙入庙来报道:“江州
城里鸣锣擂鼓,整顿军马,出城来追赶。远远望见旗蔽日,刀剑如麻,前面都是
带甲马军,后面尽是擎枪兵将,大刀阔斧,杀奔白龙庙路上来。”

李逵听了,大叫一声:“杀将去!”提了双斧,便出庙门,晁盖叫道:“一不
做,二不休,众好汉相助着晁某,直杀尽江州军马,方才回梁山泊去。”众英雄齐
声应道:“愿依尊命。”一百四五十人一齐呐喊,杀奔江州岸上来。有分教:血染
波红,尸如山积。直教:跳浪苍龙喷毒火,爬山猛虎吼天风。

毕竟晁盖等众好汉怎地脱身,且听下回分解。