\chapter{王教头私走延安府~九纹龙大闹史家村}

话说当时住持真人对洪太尉说道:“太尉不知,此殿中当初是祖老天师洞玄真
人传下法符,嘱付道:‘此殿内镇锁着三十六员天罡星,七十二座地煞星,共是一
百单八个魔君在里面。上立石碑,凿着龙章凤篆天符,镇住在此。若还放他出世,
必恼下方生灵。’如今太尉放他走了,怎生是好?”有诗为证:
千古幽扃一旦开,天罡地煞出泉台。
自来无事多生事,本为禳灾却惹灾。
社稷从今云扰扰,兵戈到处闹垓垓。
高俅奸佞虽堪恨,洪信从今酿祸胎。
当时洪太尉听罢,浑身冷汗,捉颤不住。急急收拾行李,引了从人,下山回京,真
人并道众送官已罢,自回宫内,修整殿宇,起竖石碑,不在话下。

再说洪太尉在途中分付从人,教把走妖魔一节,休说与外人知道,恐天子知而
见责。于路无话,星夜回至京师,进得汴梁城,闻人所说:“天师在东京禁院做了
七昼夜好事,普施符籙,禳救灾病,瘟疫尽消,军民安泰。天师辞朝,乘鹤驾云,
且回龙虎山去了。”洪太尉次日早朝,见了天子,奏说“天师乘鹤驾云,先到京师,
臣等驿站而来,才得到此。”仁宗准奏,赏赐洪信,复还旧职,亦不在话下。

后来仁宗天子在位共四十二年,晏驾,无有太子,传位濮安懿王允让之子,太
宗皇帝的孙,立帝号曰英宗。在位四年,传位与太子神宗。神宗在位一十八年,传
位与太子哲宗。那时天下尽皆太平,四方无事。

且说东京开封府汴梁宣武军,一个浮浪破落户子弟,姓高,排行第二,自小不
成家业,只好刺枪使棒,最是踢得好脚气毬,京师人口顺,不叫高二,却都叫他做
高毬。后来发迹,便将气毬那字去了毛傍,添作立人,便改作姓高,名俅。这人吹
弹歌舞,刺枪使棒,相扑顽耍,亦胡乱学诗、书、词、赋。若论仁、义、礼、智、
信、行、忠、良,却是不会,只在东京城里城外帮闲。因帮了一个生铁王员外儿子
使钱,每日三瓦两舍,风花雪月,被他父亲开封府里告了一纸文状,府尹把高俅断
了二十脊杖,迭配出界发放,东京城里人民不许容他在家宿食。高俅无计奈何,只
得来淮西临淮州,投奔一个开赌坊的闲汉柳大郎,名唤柳世权。他平生专好惜客养
闲人,招纳四方干隔涝汉子。高俅投托得柳大郎家,一住三年。

后来哲宗天子因拜南郊,感得风调雨顺,放宽恩大赦天下。那高俅在临淮州,
因得了赦宥罪犯,思量要回东京。这柳世权却和东京城里金梁桥下开生药铺的董将
士是亲戚,写了一封书札,收拾些人事盘缠,赍发高俅回东京,投奔董将士家过活。

当时高俅辞了柳大郎,背上包裹,离了临淮州,迤逦回到东京,径来金梁桥下
董生药家,下了这封信。董将士一见高俅,看了柳世权来书,自肚里寻思道:“这
高俅我家如何安着得他?若是个志诚老实的人,可以容他在家出入,也教孩儿们学
些好。他却是个帮闲的破落户,没信行的人。亦且当初有过犯来,被断配的人,旧
性必不肯改。若留住在家中,倒惹得孩儿们不学好了,待不收留他,又撇不过柳大
郎面皮。”当时只得权且欢天喜地,相留在家宿歇,每日酒食管待。住了十数日,
董将士思量出一个路数,将出一套衣服,写了一封书简,对高俅说道:“小人家下
萤火之光,照人不亮,恐后误了足下。我转荐足下与小苏学士处,久后也得个出身,
足下意内如何?”高俅大喜,谢了董将士。董将士使个人将着书简,引领高俅,径
到学士府内,门吏转报小苏学士,出来见了高俅,看了来书,知道高俅原是帮闲浮
浪的人,心下想道:“我这里如何安着得他?不如做个人情,荐他去驸马王晋卿府
里,做个亲随。人都唤他做小王都太尉,他便喜欢这样的人。”当时回了董将士书
札,留高俅在府里住了一夜。次日,写了一封书呈,使个干人,送高俅去那小王都
太尉处。

这太尉乃是哲宗皇帝妹夫,神宗皇帝的驸马。他喜爱风流人物,正用这样的人。
一见小苏学士差人持书送这高俅来,拜见了,便喜。随即写回书,收留高俅在府内
做个亲随。自此高俅遭际在王都尉府中出入,如同家人一般。自古道:“日远日疏,
日亲日近。”忽一日,小王都太尉庆诞生辰,分付府中安排筵宴,专请小舅端王。
这端王乃是神宗天子第十一子,哲宗皇帝御弟,现掌东驾,排号九大王,是个聪明
俊俏人物。这浮浪子弟门风帮闲之事,无一般不晓,无一般不会,更无一般不爱。
即如琴、棋、书、画,无所不通,踢毬打弹,品竹调丝,吹弹歌舞,自不必说。当
日王都尉府中,准备筵宴,水陆俱备。但见:

香焚宝鼎,花插金瓶。仙音院竞奏新声,教坊司频逞妙艺。水晶壶内,尽都是
紫府琼浆;琥珀杯中,满泛着瑶池玉液。玳瑁盘堆仙桃异果,玻璃碗供熊掌驼蹄。
鳞鳞脍切银丝,细细茶烹玉蕊。红裙舞女,尽随着象板鸾箫;翠袖歌姬,簇捧定龙
笙凤管。两行珠翠立阶前,一派笙歌临座上。

且说这端王来王都尉府中赴宴,都尉设席,请端王居中坐定,都尉对席相陪。
酒进数杯,食供两套,那端王起身净手,偶来书院里少歇,猛见书案上一对儿羊脂
玉碾成的镇纸狮子,极是做得好,细巧玲珑。端王拿起狮子,不落手看了一回道:
“好!”王都尉见端王心爱,便说道:“再有一个玉龙笔架,也是这个匠人一手做
的,却不在手头,明日取来,一并相送。”端王大喜道:“深谢厚意,想那笔架,
必是更妙。”王都尉道:“明日取出来,送至宫中便见。”端王又谢了。两个依旧
入席,饮宴至暮,尽醉方散。端王相别回宫去了。

次日,小王都太尉取出玉龙笔架,和两个镇纸玉狮子,着一个小金盒子盛了,
用黄罗包袱包了,写了一封书呈,却使高俅送去。高俅领了王都尉钧旨,将着两般
玉玩器,怀中揣着书呈,径投端王宫中来。把门官吏转报与院公。没多时,院公出
来问:“你是那个府里来的人?”高俅施礼罢,答道:“小人是王驸马府中,特送
玉玩器来进大王。”院公道:“殿下在庭心里和小黄门踢气毬,你自过去。”高俅
道:“相烦引进。”院公引到庭前,高俅看时,见端王头戴软纱唐巾,身穿紫绣龙
袍,腰系文武双穗绦。把绣龙袍前襟拽扎起,揣在绦儿边。足穿一双嵌金线飞凤靴,
三五个小黄门相伴着蹴气毬。高俅不敢过去冲撞,立在从人背后伺候。也是高俅合
当发迹,时运到来,那个气毬腾地起来,端王接个不着,向人丛里直滚到高俅身边。
那高俅见气毬来,也是一时的胆量,使个鸳鸯拐,踢还端王。端王见了大喜,便问
道:“你是甚人?”高俅向前跪下道:“小的是王都尉亲随,受东人使令,赍送两
般玉玩器来,进献大王,有书呈在此拜上。”端王听罢,笑道:“姐夫直如此挂心。”
高俅取出书呈进上。端王开盒子看了玩器,都递与堂候官收了去。

那端王且不理玉玩器下落,却先问高俅道:“你原来会踢气毬!你唤做甚么?”
高俅叉手跪覆道:“小的叫做高俅,胡乱踢得几脚。”端王道:“好!你便下场来
踢一回耍。”高俅拜道:“小的是何等样人,敢与恩王下脚!”端王道:“这是‘齐
云社’名为‘天下圆’,但踢何伤。”高俅再拜道:“怎敢!”三回五次告辞,端
王定要他踢,高俅只得叩头谢罪,解膝下场。才踢几脚,端王喝采。高俅只得把平
生本事都使出来,奉承端王。那身分模样,这气毬一似鳔胶粘在身上的。端王大喜,
那里肯放高俅回府去,就留在宫中过了一夜。次日,排个筵会,专请王都尉宫中赴
宴。

却说王都尉当日晚不见高俅回来,正疑思间,只见次日门子报道:“九大王差
人来传令旨,请太尉到宫中赴宴。”王都尉出来,见了那干人,看了令旨,随即上
马,来到九大王府前,下马入宫,来见了端王。端王大喜,称谢两般玉玩器。入席
饮宴间,端王说道:“这高俅踢得两脚好气毬,孤欲索此人做亲随如何?”王都尉
答道:“殿下既用此人,就留在宫中伏侍殿下。”端王欢喜,执杯相谢。二人又闲
话一回,至晚席散,王都尉自回驸马府去,不在话下。

且说端王自从索得高俅做伴之后,就留在宫中宿食。高俅自此遭际端王,每日
跟随,寸步不离。未及两个月,哲宗皇帝晏驾,无有太子,文武百官商议,册立端
王为天子,立帝号曰徽宗,便是玉清教主微妙道君皇帝。登基之后,一向无事,忽
一日,与高俅道:“朕欲要抬举你,但有边功,方可升迁,先教枢密院与你入名,
只是做随驾迁转的人。”后来没半年之间,直抬举高俅做到殿帅府太尉职事。正是:
不拘贵贱齐云社,一味模棱天下圆。
抬举高俅毬气力,全凭手脚会当权。

且说高俅得做了殿帅府太尉,选拣吉日良辰,去殿帅府里到任,所有一应合属
公吏衙将,都军监军,马步人等,尽来参拜,各呈手本,开报花名。高殿帅一一点
过,于内只欠一名八十万禁军教头王进,半月之前,已有病状在官,患病未痊,不
曾入衙门管事。高殿帅大怒,喝道:“胡说!既有手本呈来,却不是那厮抗拒官府,
搪塞下官!此人即系推病在家,快与我拿来。”随即差人到王进家来,捉拿王进。

且说这王进却无妻子,只有一个老母,年已六旬之上。牌头与教头王进说道:
“如今高殿帅新来上任,点你不着,军正司禀说染患在家,现有病患状在官。高殿
帅焦躁,那里肯信?定要拿你,只道是教头诈病在家,教头只得去走一遭。若还不
去,定连累小人了。”

王进听罢,只得捱着病来。进得殿帅府前,参见太尉,拜了四拜,躬身唱个喏,
起来立在一边。高俅道:“你那厮便是都军教头王升的儿子?”王进禀道:“小人
便是。”高俅喝道:“这厮,你爷是街市上使花棒卖药的,你省的甚么武艺?前官
没眼,参你做个教头,如何敢小觑我,不伏俺点视!你托谁的势,要推病在家,安
闲快乐!”王进告道:“小人怎敢,其实患病未痊。”高太尉骂道:“贼配军,你
既害病,如何来得?”王进又告道:“太尉呼唤,安敢不来!”高殿帅大怒,喝令
左右:“拿下!加力与我打这厮!”众多牙将都是和王进好的,只得与军正司同告
道:“今日太尉上任,好日头,权免此人这一次。”高太尉喝道:“你这贼配军,
且看众将之面,饶恕你今日,明日却和你理会。”王进谢罪罢,起来抬头看了,认
得是高俅。出得衙门,叹口气道:“俺的性命,今番难保了。俺道是甚么高殿帅,
却原来正是东京帮闲的‘圆社’高二。比先时曾学使棒,被我父亲一棒打翻,三四
个月将息不起,有此之仇。他今日发迹,得做殿帅府太尉,正待要报仇,我不想正
属他管。自古道:‘不怕官,只怕管。’俺如何与他争得?怎生奈何是好?”回到
家中,闷闷不已。对娘说知此事,母子二人,抱头而哭。娘道:“我儿,‘三十六
着,走为上着’。只恐没处走。”王进道:“母亲说得是,儿子寻思,也是这般计
较。只有延安府老种经略相公镇守边庭,他手下军官,多有曾到京师的,爱儿子使
枪棒,何不逃去投奔他们?那里是用人去处,足可安身立命。”正是:
用人之人,人始为用。
恃己自用,人为人送。
彼处得贤,此间失重。
若驱若引,可惜可痛。
当下娘儿两个商议定了。其母又道:“我儿,和你要私走,只恐门前两个牌军
是殿帅府拨来伏侍你的,他若得知,须走不脱。”王进道:“不妨,母亲放心。儿
子自有道理措置他。”

当下日晚未昏,王进先叫张牌入来,分付道:“你先吃了些晚饭,我使你一处
去干事。”张牌道:“教头使小人那里去?”王进道:“我因前日病患,许下酸枣
门外岳庙里香愿,明日早要去烧炷头香。你可今晚先去分付庙祝,教他来日早些开
庙门,等我来烧炷头香,就要三牲,献刘李王。你就庙里歇了等我。”张牌答应,
先吃了晚饭,叫了安置,望庙中去了。

当夜子母二人,收拾了行李、衣服、细软、银两,做一担儿打挟了。又装两个
料袋袱驼,拴在马上的。等到五更,天色未明,王进教起李牌,分付道:“你与我
将这些银两,去岳庙里,和张牌买个三牲煮熟,在那里等候。我买些纸烛,随后便
来。”李牌将银子望庙中去了。王进自去备了马,牵出后槽,将料袋袱驼搭上,把
索子拴缚牢了,牵在后门外,扶娘上了马。家中粗重都弃了,锁上前后门,挑了担
儿,跟在马后。趁五更天色未明,乘势出了西华门,取路望延安府来。

且说两个牌军,买了福物煮熟,在庙等到巳牌,也不见来。李牌心焦,走回到
家中寻时,见锁了门,两头无路。寻了半日,并无有人。看看待晚,岳庙里张牌疑
忌,一直奔回家来。又和李牌寻了一黄昏,看看黑了。两个见他当夜不归,又不见
他老娘。次日,两个牌军又去他亲戚之家访问,亦无寻处。两个恐怕连累,只得去
殿帅府首告:“王教头弃家在逃,子母不知去向。”高太尉见告,大怒道:“贼配
军在逃,看那厮待走那里去!”随即押下文书,行开诸州各府,捉拿逃军王进。二
人首告,免其罪责,不在话下。

且说王教头母子二人,自离了东京,在路免不得饥餐渴饮,夜住晓行,在路上
一月有余。忽一日,天色将晚,王进挑着担儿,跟在娘的马后,口里与母亲说道:
“天可怜见,惭愧了!我子母两个,脱了这天罗地网之厄,此去延安府不远了。高
太尉便要差人拿我,也拿不着了。”子母两个欢喜,在路上不觉错过了宿头。走了
这一晚,不遇着一处村坊,那里去投宿是好。正没理会处,只见远远地林子里闪出
一道灯光来。王进看了道:“好了,遮莫去那里陪个小心,借宿一宵,明日早行。”
当时转入林子里来看时,却是一所大庄院,一周遭都是土墙,墙外却有二三百株大
柳树。看那庄院,但见:

前通官道,后靠溪冈。一周遭青缕如烟,四下里绿阴似染。转屋角牛羊满地,
打麦场鹅鸭成群。田园广野,负佣庄客有千人;家眷轩昂,女使儿童难计数。正是:
家有余粮鸡犬饱,户多书籍子孙贤。

当时王教头来到庄前,敲门多时,只见一个庄客出来。王进放下担儿,与他施
礼。庄客道:“来俺庄上有甚事?”王进答道:“实不相瞒,小人母子二人,贪行
了些路程,错过了宿店,来到这里,前不巴村,后不巴店,欲投贵庄,借宿一宵,
明日早行。依例拜纳房金,万望周全方便。”庄客道:“既是如此,且等一等,待
我去问庄主太公,肯时,但歇不妨。”王进又道:“大哥方便。”庄客入去多时,
出来说道:“庄主太公教你两个入来。”王进请娘下了马。王进挑着担儿,就牵了
马,随庄客到里面打麦场上,歇下担儿,把马拴在柳树上。母子二人,直到草堂上
来见太公。

那太公年近六旬之上,须发皆白,头戴遮尘暖帽,身穿直缝宽衫,腰系皂丝绦,
足穿熟皮靴。王进见了便拜,太公连忙道:“客人休拜,你们是行路的人,辛苦风
霜,且坐一坐。”王进母子两个叙礼罢,都坐定。太公问道:“你们是那里来的?
如何昏晚到此?”王进答道:“小人姓张,原是京师人。今来消折了本钱,无可营
用,要去延安府投奔亲眷。不想今日路上贪行了些程途,错过了宿店,欲投贵庄,
假宿一宵,来日早行。房金依例拜纳。”太公道:“不妨,如今世上人那个顶着房
屋走哩!你母子二位,敢未打火?”叫庄客安排饭来。没多时,就厅上放开条桌子,
庄客托出一桶盘,四样菜蔬,一盘牛肉,铺放桌上,先烫酒来筛下。太公道:“村
落中无甚相待,休得见怪。”王进起身谢道:“小人母子无故相扰,此恩难报。”
太公道:“休这般说,且请吃酒。”一面劝了五七杯酒,搬出饭来。二人吃了,收
拾碗碟。太公起身,引王进子母到客房里安歇。王进告道:“小人母亲骑的头口,
相烦寄养,草料望乞应付,一并拜酬。”太公道:“这个不妨。我家也有头口骡马,
教庄客牵出后槽,一发喂养。”王进谢了,挑那担儿,到客房里来。庄客点上灯火,
一面提汤来洗了脚。太公自回里面去了。王进子母二人谢了庄客,掩上房门,收拾
歇息。

次日,睡到天晓,不见起来。庄主太公来到客房前过,听得王进子母在房里声
唤。太公问道:“客官,天晓,好起了。”王进听得,慌忙出房来,见太公施礼,
说道:“小人起多时了。夜来多多搅扰,甚是不当。”太公问道:“谁人如此声唤?”
王进道:“实不相瞒太公说:老母鞍马劳倦,昨夜心痛病发。”太公道:“既然如
此,客人休要烦恼,教你老母且在老夫庄上住几日。我有个医心疼的方,叫庄客去
县里撮药来,与你老母亲吃。教他放心,慢慢地将息。”王进谢了。

话休絮繁,自此王进子母二人在太公庄上服药。住了五七日,觉得母亲病患痊
了,王进收拾要行。当日因来后槽看马,只见空地上一个后生脱膊着,刺着一身青
龙,银盘也似一个面皮,约有十八九岁,拿条棒在那里使。王进看了半晌,不觉失
口道:“这棒也使得好了。只是有破绽,赢不得真好汉。”那后生听得大怒,喝道:
“你是甚么人?敢来笑话我的本事?俺经了七八个有名的师父,我不信倒不如你?你
敢和我一么?”

说犹未了,太公到来,喝那后生:“不得无礼!”那后生道:“叵耐这厮笑话
我的棒法。”太公道:“客人莫不会使枪棒?”王进道:“颇晓得些。敢问长上,
这后生是宅上何人?”太公道:“是老汉的儿子。”王进道:“既然是宅内小官人,
若爱学时,小人点拨他端正如何?”太公道:“恁地时,十分好。”便教那后生来
拜师父。那后生那里肯拜,心中越怒道:“阿爹,休听这厮胡说。若吃他赢得我这
条棒时,我便拜他为师。”王进道:“小官人若是不当村时,较量一棒耍子。”那
后生就空地当中,把一条棒使得风车儿似转,向王进道:“你来,你来!怕的不算
好汉!”王进只是笑,不肯动手。太公道:“客官既是肯教小顽时,使一棒何妨。”
王进笑道:“恐冲撞了令郎时,须不好看。”太公道:“这个不妨,若是打折了手
脚,也是他自作自受。”

王进道:“恕无礼。”去枪架上拿了一条棒在手里,来到空地,使个旗鼓。那
后生看了一看,拿条棒滚将入来,径奔王进。王进托地拖了棒便走,那后生抡着棒
又赶入来。王进回身,把棒望空地里劈将下来。那后生见棒劈来,用棒来隔。王进
却不打下来,将棒一掣,却望后生怀里直搠将来,只一缴,那后生的棒丢在一边,
扑地望后倒了。王进连忙撇了棒,向前扶住道:“休怪,休怪。”

那后生爬将起来,便去旁边掇条凳子,纳王进坐,便拜道:“我枉自经了许多
师家,原来不值半分。师父,没奈何,只得请教。”王进道:“我母子二人,连日
在此搅扰宅上,无恩可报,当以效力。”太公大喜,教那后生穿了衣裳,一同来后
堂坐下。叫庄客杀一个羊,安排了酒食果品之类,就请王进的母亲一同赴席。四个
人坐定,一面把盏,太公起身劝了一杯酒,说道:“师父如此高强,必是个教头,
小儿有眼不识泰山。”王进笑道:“奸不厮欺,俏不厮瞒,小人不姓张。俺是东京
八十万禁军教头王进的便是,这枪棒终日搏弄。为因新任一个高太尉,原被先父打
翻,今做殿帅府太尉,怀挟旧仇,要奈何王进。小人不合属他所管,和他争不得,
只得子母二人逃上延安府,去投托老种经略相公处勾当。不想来到这里,得遇长上
父子二位如此看待;又蒙救了老母病患,连日管顾,甚是不当。既然令郎肯学时,
小人一力奉教。只是令郎学的,都是花棒,只好看,上阵无用,小人从新点拨他。”
太公见说了,便道:“我儿,可知输了?快来再拜师父。”那后生又拜了王进。正
是:
好为师患负虚名,心服应难以力争。
只有胸中真本事,能令顽劣拜先生。
太公道:“教头在上,老汉祖居在这华阴县界,前面便是少华山。这村便唤做
史家村,村中总有三四百家,都姓史。老汉的儿子从小不务农业,只爱刺枪使棒,
母亲说他不得,怄气死了,老汉只得随他性子。不知使了多少钱财,投师父教他。
又请高手匠人与他刺了这身花绣,肩臂胸膛总有九条龙,满县人口顺,都叫他做九
纹龙史进。教头今日既到这里,一发成全了他亦好。老汉自当重重酬谢。”王进大
喜道:“太公放心,既然如此说时,小人一发教了令郎方去。”自当日为始,吃了酒
食,留住王教头母子二人在庄上。史进每日求王教头点拨十八般武艺,一一从头指
教。
那十八般武艺?
矛锤弓弩铳,鞭锏剑链挝。
斧钺并戈戟,牌棒与枪杈。

话说这史进每日在庄上管待王教头母子二人,指教武艺。史太公自去华阴县中
承当里正,不在话下。不觉荏苒光阴,早过半年之上,正是:
窗外日光弹指过,席间花影坐前移。
一杯未进笙歌送,阶下辰牌又报时。

前后得半年之上,史进打这十八般武艺,从新学得十分精熟。多得王进尽心指
教,点拨得件件都有奥妙。王进见他学得精熟了,自思:“在此虽好,只是不了。”
一日想起来,相辞要上延安府去。史进那里肯放,说道:“师父只在此间过了,小
弟奉养你母子二人,以终天年,多少是好!”王进道:“贤弟,多蒙你好心,在此
十分之好;只恐高太尉追捕到来,负累了你,不当稳便,以此两难。我一心要去延
安府,投着在老种经略处勾当,那里是镇守边庭,用人之际,足可安身立命。”

史进并太公苦留不住,只得安排一个筵席送行。托出一盘两个缎子、一百两花
银谢师。次日,王进收拾了担儿,备了马,子母二人,相辞史太公。王进请娘乘了
马,望延安府路途进发。史进叫庄客挑了担儿,亲送十里之程,心中难舍。史进当
时拜别了师父,洒泪分手,和庄客自回。王教头依旧自挑了担儿,跟着马,和娘两
个,自取关西路里去了。

话中不说王进去投军役,只说史进回到庄上,每日只是打熬气力,亦且壮年,
又没老小,半夜三更起来演习武艺,白日里只在庄后射弓走马。不到半载之间,史
进父亲太公,染病患症,数日不起。史进使人远近请医士看治,不能痊可,呜呼哀
哉,太公殁了。史进一面备棺椁盛殓,请僧修设好事,追斋理七,荐拔太公。又请
道士建立斋醮,超度生天,整做了十数坛好事功果道场,选了吉日良时,出丧安葬。
满村中三四百史家庄户,都来送丧挂孝,埋殡在村西山上祖坟内了。史进家自此无
人管业。史进又不肯务农,只要寻人使家生,较量枪棒。

自史太公死后,又早过了三四个月日。时当六月中旬,炎天正热。那一日,史
进无可消遣,捉个交床,坐在打麦场边柳阴树下乘凉。对面松林透过风来,史进喝
采道:“好凉风!”正乘凉哩,只见一个人探头探脑,在那里张望。史进喝道:“作
怪!谁在那里张俺庄上?”史进跳起身来,转过树背后,打一看时,认得是猎户劫
兔李吉。史进喝道:“李吉,张我庄内做甚么?莫不来相脚头?”李吉向前声喏道:
“大郎,小人要寻庄上矮丘乙郎吃碗酒,因见大郎在此乘凉,不敢过来冲撞。”

史进道:“我且问你:往常时,你只是担些野味,来我庄上卖,我又不曾亏了
你,如何一向不将来卖与我?敢是欺负我没钱?”李吉答道:“小人怎敢。一向没
有野味,以此不敢来。”史进说:“胡说!偌大一个少华山,恁地广阔,不信没有
个獐儿兔儿!”李吉道:“大郎原来不知:如今近日上面添了一伙强人,扎下一个
山寨,在上面聚集着五七百个小喽罗,有百十匹好马。为头那个大王,唤作神机军
师朱武,第二个唤做跳涧虎陈达,第三个唤做白花蛇杨春。这三个为头,打家劫舍,
华阴县里禁他不得,出三千贯赏钱召人拿他,谁敢上去惹他?因此上小人们不敢上
山打捕野味,那讨来卖?”史进道:“我也听得说有强人,不想那厮们如此大弄,
必然要恼人。李吉,你今后有野味时,寻些来。”李吉唱个喏,自去了。

史进归到厅前,寻思:“这厮们大弄,必要来薅恼村坊。既然如此,……”便
叫庄客拣两头肥水牛来杀了,庄内自有造下的好酒,先烧了一陌顺溜纸,便叫庄客
去请这当村里三四百史家庄户,都到家中草堂上序齿坐下,教庄客一面把盏劝酒。
史进对众人说道:“我听得少华山上有三个强人,聚集着五七百小喽罗,打家劫舍,
这厮们既然大弄,必然早晚要来俺村中罗唣。我今特请你众人来商议,倘若那厮们
来时,各家准备。我庄上打起梆子,你众人可各执枪棒,前来救应。你各家有事,
亦是如此。递相救护,共保村坊。如若强人自来,都是我来理会。”众人道:“我
等村农,只靠大郎做主,梆子响时,谁敢不来?”当晚众人谢酒,各自分散,回家
准备器械。自此史进修整门户墙垣,安排庄院,设立几处梆子,拴束衣甲,整顿刀
马,提防贼寇,不在话下。

且说少华山寨中三个头领,坐定商议,为头的神机军师朱武,那人原是定远人
氏,能使两口双刀,虽无十分本事,却精通阵法,广有谋略,有八句诗单道朱武好
处:
道服裁棕叶,云冠剪鹿皮。
脸红双眼俊,面白细髯垂。
阵法方诸葛,阴谋胜范蠡。
华山谁第一,朱武号神机。
第二个好汉姓陈,名达,原是邺城人氏,使一条出白点钢枪,亦有诗赞道:
力健声雄性粗卤,丈二长枪撒如雨。
邺中豪杰霸华阴,陈达人称跳涧虎。
第三个好汉姓杨,名春,蒲州解良县人氏,使一口大杆刀。亦有诗赞道:
腰长臂瘦力堪夸,到处刀锋乱撒花。
鼎立华山真好汉,江湖名播白花蛇。

朱武当与陈达、杨春说道:“如今我听知华阴县里出三千贯赏钱,召人捉我们。
诚恐来时,要与他厮杀。只是山寨钱粮欠少,如何不去劫掳些来,以供山寨之用,
聚积些粮食在寨里,防备官军来时,好和他打熬。”跳涧虎陈达道:“说得是。如
今便去华阴县里,先问他借粮,看他如何。”白花蛇杨春道:“不要华阴县去,只
去蒲城县,万无一失。”陈达道:“蒲城县人户稀少,钱粮不多,不如只打华阴县,
那里人民丰富,钱粮广有。”杨春道:“哥哥不知,若去打华阴县时,须从史家村
过。那个九纹龙史进是个大虫,不可去撩拨他。他如何肯放我们过去?”陈达道:
“兄弟好懦弱!一个村坊过去不得,怎地敢抵敌官军?”杨春道:“哥哥不可小觑
了他,那人端的了得。”朱武道:“我也曾闻他十分英雄,说这人真有本事,兄弟
休去罢。”陈达叫将起来,说道:“你两个闭了鸟嘴!长别人志气,灭自己威风。
他只是一个人,须不三头六臂,我不信。”喝叫小喽罗:“快备我的马来。如今便
去先打史家庄,后取华阴县。”朱武、杨春再三谏劝,陈达那里肯听,随即披挂上
马,点了一百四五十小喽罗,鸣锣擂鼓下山,望史家村去了。

且说史进正在庄前整制刀马,只见庄客报知此事。史进听得,就庄上敲起梆子
来。那庄前庄后,庄东庄西,三四百史家庄户,听得梆子响,都拖枪拽棒,聚起三
四百人,一齐都到史家庄上。看了史进头戴一字巾,身披朱红甲,上穿青锦祆,下
著抹绿靴,腰系皮膊,前后铁掩心,一张弓,一壶箭,手里拿一把三尖两刃四窍
八环刀。庄客牵过那匹火炭赤马,史进上了马,绰了刀,前面摆着三四十壮健的庄
客,后面列着八九十村蠢的乡夫,各史家庄户,都跟在后头,一齐呐喊,直到村北
路口。

那少华山陈达引了人马,飞奔到山坡下,便将小喽罗摆开。史进看时,见陈达
头戴干红凹面巾,身披裹金生铁甲,上穿一领红衲袄,脚穿一对吊墩靴,腰系七尺
攒线膊,坐骑一匹高头白马,手中横着丈八点钢矛。小喽罗两势下呐喊,二员将
就马上相见。陈达在马上看着史进,欠身施礼。史进喝道:“汝等杀人放火,打家
劫舍,犯着迷天大罪,都是该死的人。你也须有耳朵,好大胆,直来太岁头上动土!”
陈达在马上答道:“俺山寨里欠少些粮食,欲往华阴县借粮,经由贵庄,假一条路,
并不敢动一根草,可放我们过去,回来自当拜谢。”史进道:“胡说!俺家现当里
正,正要来拿你这伙贼,今日倒来经由我村中过,却不拿你,倒放你过去!本县知
道,须连累于我。”陈达道:“‘四海之内,皆兄弟也’,相烦借一条路。”史进
道:“甚么闲话!我便肯时,有一个不肯,你问得他肯便去。”陈达道:“好汉,
教我问谁?”史进道:“你问得我手里这口刀肯,便放你去。”陈达大怒道:“赶
人不要赶上,休得要逞精神!”

史进也怒,抡手中刀,骤坐下马,来战陈达。陈达也拍马挺枪,来迎史进。两
个交马,但见:

一来一往,一上一下。一来一往,有如深水戏珠龙;一上一下,却似半岩争食
虎。九纹龙忿怒,三尖刀只望顶门飞;跳涧虎生嗔,丈八矛不离心坎刺。好手中间
逞好手,红心里面夺红心。
史进、陈达两个斗了多时,史进卖个破绽,让陈达把枪望心窝里搠来,史进却
把腰一闪,陈达和枪攧入怀里来,史进轻舒猿臂,款扭狼腰,只一挟,把陈达轻轻
摘离了嵌花鞍,款款揪住了线膊,只一丢,丢落地,那匹战马拨风也似去了。史
进叫庄客将陈达绑缚了,众人把小喽罗一赶都走了。史进回到庄上,将陈达绑在庭
心内柱上,等待一发拿了那两个贼首,一并解官请赏。且把酒来赏了众人,教且权
散。众人喝采:“不枉了史大郎如此豪杰!”

休说众人欢喜饮酒,却说朱武、杨春两个,正在寨里猜疑,捉摸不定,且教小
喽罗再去打听消息,只见同去的人牵着空马,奔到山前,只叫道:“苦也!陈家哥
哥不听二位哥哥所说,送了性命。”朱武问其缘故,小喽罗备说交锋一节,怎当史
进英雄。朱武道:“我的言语不听,果有此祸。”杨春道:“我们尽数都去,与他
死拚如何?”朱武道:“亦是不可。他尚自输了,你如何拚得他过?我有一条苦计,
若救他不得,我和你都休。”杨春问道:“如何苦计?”朱武附耳低言说道:“只
除恁地。”杨春道:“好计!我和你便去,事不宜迟。”

再说史进正在庄上忿怒未消,只见庄客飞报道:“山寨里朱武、杨春自来了。”
史进道:“这厮合休,我教他两个一发解官。快牵马过来。”一面打起梆子,众人
早都到来。史进上了马,正待出庄门,只见朱武、杨春步行,已到庄前,两个双双
跪下,噙着两眼泪。史进下马来喝道:“你两个跪下如何说?”朱武哭道:“小人
等三个,累被官司逼迫,不得已上山落草,当初发愿道:‘不求同日生,只愿同日
死。’虽不及关、张、刘备的义气,其心则同。今日小弟陈达不听好言,误犯虎威,
已被英雄擒捉在贵庄,无计恳求,今来一径就死,望英雄将我三人,一发解官请赏,
誓不皱眉。我等就英雄手内请死,并无怨心。”史进听了,寻思道:“他们直恁义
气!我若拿他去解官请赏时,反教天下好汉们耻笑我不英雄。自古道:‘大虫不吃
伏肉。’”史进便道:“你两个且跟我进来。”朱武、杨春并无惧怯,随了史进,
直到后厅前跪下,又教史进绑缚。史进三回五次叫起来,他两个那里肯起来,惺惺
惜惺惺,好汉识好汉。史进道:“你们既然如此义气深重,我若送了你们,不是好
汉,我放陈达还你如何?”朱武道:“休得连累了英雄,不当稳便,宁可把我们去
解官请赏。”史进道:“如何使得?——你肯吃我酒食么?”朱武道:“一死尚然
不惧,何况酒肉乎?”有诗为证:
姓名各异死生同,慷慨偏多计较空。
只为衣冠无义侠,遂令草泽见奇雄。
当时史进大喜,解放陈达,就后厅上座,置酒设席,管待三人。朱武、杨春、
陈达拜谢大恩。酒至数杯,少添春色。酒罢,三人谢了史进,回山去了。史进送出
庄门,自回庄上。

却说朱武等三人归到寨中坐下,朱武道:“我们不是这条苦计,怎得性命在此?
虽然救了一人,却也难得史进为义气上放了我们。过几日备些礼物送去,谢他救命
之恩。”话休絮繁。过了十数日,朱武等三人收拾得三十两蒜条金,使两个小喽罗,
乘月黑夜送去史家庄上。当夜初更时分,小喽罗敲门,庄客报知史进,史进火急披
衣,来到庄前,问小喽罗:“有甚话说?”小喽罗道:“三个头领再三拜复:特地
使小校进些薄礼,酬谢大郎不杀之恩,不要推却,望乞笑留。”取出金子,递与史
进。初时推却,次后寻思道:“既然好意送来,受之为当。”叫庄客置酒,管待小
校吃了半夜酒,把些零碎银两,赏了小校,回山去了。又过半月有余,朱武等三人
在寨中商议掳掠得一串好大珠子,又使小喽罗连夜送来史家庄上,史进受了,不在
话下。

又过了半月,史进寻思道:“也难得这三个敬重我,我也备些礼物回奉他。”
次日,叫庄客寻个裁缝,自去县里买了三匹红锦,裁成三领锦袄子;又拣肥羊,煮
了三个,将大盒子盛了,委两个庄客去送。史进庄上,有个为头的庄客王四,此人
颇能答应官府,口舌利便,满庄人都叫他做赛伯当。史进教他同一个得力庄客,挑
了盒担,直送到山下。小喽罗问了备细,引到山寨里,见了朱武等三个头领,大喜,
受了锦袄子,并肥羊酒礼,把十两银子,赏了庄客。每人吃了十数碗酒,下山回归
庄内,见了史进,说道:“山上头领,多多上覆。”史进自此常常与朱武等三人往
来,不时间,只是王四去山寨里送物事,不则一日。寨里头领也频频地使人送金银
来与史进。

荏苒光阴,时遇八月中秋到来。史进要和三人说话,约至十五夜,来庄上赏月
饮酒。先使庄客王四,赍一封请书,直去少华山上,请朱武、陈达、杨春来庄上赴
席。王四驰书径到山寨里,见了三位头领,下了来书。朱武看了大喜,三个应允,
随即写封回书,赏了王四五两银子,吃了十来碗酒。王四下得山来,正撞着时常送
物事来的小喽罗,一把抱住,那里肯放。又拖去山路边村酒店里,吃了十数碗酒。
王四相别了回庄,一面走着,被山风一吹,酒却涌上来,踉踉跄跄,一步一攧。走
不到十里之路,见座林子,奔到里面,望着那绿茸茸莎草地上扑地倒了。原来摽兔
李吉正在那山坡下张兔儿,认得是史家庄上王四,赶入林子里来扶他,那里扶得动。
只见王四搭膊里突出银子来,李吉寻思道:“这厮醉了,那里讨得许多!何不拿他
些?”也是天罡星合当聚会,自然生出机会来。李吉解那搭膊,望地下只一抖,那
封回书和银子都抖出来。李吉拿起,颇识几字,将书拆开看时,见上面写着少华山
朱武、陈达、杨春,中间多有兼文带武的言语,却不识得,只认得三个名字。李吉
道:“我做猎户,几时能够发迹,算命道我今年有大财,却在这里。华阴县里现出
三千贯赏钱,捕捉他三个贼人。叵耐史进那厮,前日我去他庄上寻矮丘乙郎,他道
我来相脚头踩盘,你原来倒和贼人来往!”银子并书都拿去了,望华阴县里来出首。

却说庄客王四,一觉直睡到二更,方醒觉来,看见月光微微照在身上,吃了一
惊,跳将起来,却见四边都是松树。便去腰里摸时,搭膊和书都不见了。四下里寻
时,只见空膊在莎草地上,王四只管叫苦,寻思道:“银子不打紧,这封回书,
却怎生好?正不知被甚人拿去了?”眉头一纵,计上心来,自道:“若回去庄上说
脱了回书,大郎必然焦躁,定是赶我出去,不如只说不曾有回书,那里查照。”计
较定了,飞也似取路归来庄上,却好五更天气。史进见王四回来,问道:“你缘何
方才归来?”王四道:“托主人福荫,寨中三个头领,都不肯放,留住王四吃了半
夜酒,因此回来迟了。”史进又问:“曾有回书否?”王四道:“三个头领要写回
书,却是小人道:‘三位头领既然准来赴席,何必回书?小人又有杯酒,路上恐有
些失支脱节,不是耍处。’”史进听了大喜,说道:“不枉了诸人叫做赛伯当,真
个了得。”王四应道:“小人怎敢差迟,路上不曾住脚,一直奔回庄上。”史进道:
“既然如此,教人去县里买些果品、案酒伺候。”

不觉中秋节至,是日晴明得好。史进当日分付家中庄客,宰了一腔大羊,杀了
百十个鸡鹅,准备下酒食筵宴。看看天色晚来,怎见得好个中秋,但见:

午夜初长,黄昏已半,一轮月挂如银。冰盘如昼,赏玩正宜人。清影十分圆满,
桂花玉兔交馨。帘栊高卷,金杯频劝酒,欢笑贺升平。年年当此节,酩酊醉醺醺。
莫辞终夕饮,银汉露华新。

且说少华山上朱武、陈达、杨春三个头领,分付小喽罗看守寨栅,只带三五个
做伴,将了朴刀,各跨口腰刀,不骑鞍马,步行下山,径来到史家庄上。史进接着,
各叙礼罢,请入后园,庄内已安排下筵宴。史进请三位头领上坐,史进对席相陪,
便叫庄客把前后庄门拴了。一面饮酒,庄内庄客,轮流把盏,一边割羊劝酒。酒至
数杯,却早东边推起那轮明月,但见:

桂花离海峤,云叶散天衢。彩霞照万里如银,素魄映千山似水。影横旷野,惊
独宿之乌鸦;光射平湖,照双栖之鸿雁。冰轮展出三千里,玉兔平吞四百州。
史进正和三个头领在后园饮酒,赏玩中秋,叙说旧话新言,只听得墙外一声喊起,
火把乱明,史进大惊,跳起身来分付:“三位贤友且坐,待我去看。”喝叫庄客:
“不要开门!”掇条梯子,上墙打一看时,只见是华阴县县尉在马上,引着两个都
头,带着三四百土兵,围住庄院。史进和三个头领只管叫苦,外面火把光中,照见
钢叉、朴刀、五股叉、留客住,摆得似麻林一般。两个都头口里叫道:“不要走了
强贼。”不是这伙人来捉史进并三个头领,有分教:史进先杀了一两个人,结识了
十数个好汉,直使天罡地煞一齐相会。直教:芦花深处屯兵士,荷叶阴中治战船。

毕竟史进与三个头领怎地脱身,且听下回分解。