\chapter{扑天雕双修生死书~宋公明一打祝家庄}

话说当时杨雄扶起那人来,叫与石秀相见。石秀便问道:“这位兄长是谁?”
杨雄道:“这个兄弟,姓杜,名兴,祖贯是中山府人氏,因为他面颜生得粗莽,以
此人都叫他做鬼脸儿。上年间做买卖,来到蓟州,因一口气上,打死了同伙的客人,
吃官司,监在蓟州府里。杨雄见他说起拳棒都省得,一力维持救了他。不想今日在
此相会。”杜兴便问道:“恩人,为何公事来到这里?”杨雄附耳低言道:“我在
蓟州杀了人命,欲要投梁山泊去入伙。昨晚在祝家店投宿,因同一个来的火伴时迁,
偷了他店里报晓鸡吃,一时与店小二闹将起来,性起,把他店屋放火都烧了。我三
个连夜逃走,不提防背后赶来。我弟兄两个搠翻了他几个,不想乱草中间,舒出两
把挠钩,把时迁搭了去。我两个乱撞到此,正要问路,不想遇见贤弟。”杜兴道:
“恩人不要慌,我叫放时迁还你。”杨雄道:“贤弟少坐,同饮一杯。”

三人坐下,当下饮酒,杜兴便道:“小弟自从离了蓟州,多得恩人的恩惠,来
到这里,感承此间一个大官人见爱,收录小弟在家中,做个主管,每日拨万论千,
尽托付与杜兴身上,甚是信任,以此不想回乡去。”杨雄道:“此间大官人是谁?”
杜兴道:“此间独龙冈前面,有三座山冈,列着三个村坊。中间是祝家庄,西边是
扈家庄,东边是李家庄。这三处庄上,三村里算来,总有一二万军马人家。惟有祝
家庄最豪杰,为头家长,唤做祝朝奉,有三个儿子,名为祝氏三杰。长子祝龙,次
子祝虎,三子祝彪。又有一个教师,唤做铁棒栾廷玉,此人有万夫不当之勇。庄上
自有一二千了得的庄客。西边那个扈家庄,庄主扈太公,有个儿子,唤做飞天虎扈
成,也十分了得。惟有一个女儿最英雄,名唤一丈青扈三娘,使两口日月双刀,马
上如法了得。这里东村庄上,却是杜兴的主人,姓李,名应,能使一条浑铁点钢枪,
背藏飞刀五口,百步取人,神出鬼没。这三村结下生死誓愿,同心共意,但有吉凶,
递相救应。惟恐梁山泊好汉过来借粮,因此三村准备下抵敌他。如今小弟引二位到
庄上,见了李大官人,求书去搭救时迁。”杨雄又问道:“你那李大官人,莫不是
江湖上唤扑天雕的李应?”杜兴道:“正是他。”石秀道:“江湖上只听得说独龙
冈有个扑天雕李应是好汉,却原来在这里。多闻他真个了得,是好男子,我们去走
一遭。”杨雄便唤酒保,计算酒钱。杜兴那里肯要他还,便自招了酒钱。

三个离了村店,便引杨雄、石秀来到李家庄上。杨雄看时,真个好大庄院,外
面周回一遭阔港,粉墙傍岸,有数百株合抱不交的大柳树,门外一座吊桥,接着庄
门。入得门来,到厅前,两边有二十余座枪架,明晃晃的都插满军器。杜兴道:“两
位哥哥在此少等,待小弟入去报知,请大官人出来相见。”杜兴入去,不多时,只
见李应从里面出来。杨雄、石秀看时,果然好表人物,有《临江仙》词为证:

鹘眼鹰睛头似虎,燕颔猿臂狼腰,疏财仗义结英豪。爱骑雪白马,喜著绛红袍。

背上飞刀藏五把,点钢枪斜嵌银条,性刚谁敢犯分毫。李应真壮士,名号扑天雕。

当时李应出到厅前,杜兴引杨雄、石秀上厅拜见。李应连忙答礼,便教上厅请
坐,杨雄、石秀再三谦让,方才坐了。李应便教取酒来且相待。杨雄、石秀两个再
拜道:“望乞大官人致书与祝家庄,来救时迁性命,生死不敢有忘。”李应教请门
馆先生来商议,修了一封书缄,填写名讳,使个图书印记,便差一个副主管赍了,
备一匹快马,星火去祝家庄取这个人来。

那副主管领了东人书札,上马去了,杨雄、石秀拜谢罢。李应道:“二位壮士
放心,小人书去,便当放来。”杨雄、石秀又谢了。李应道:“且请去后堂,少叙
三杯等待。”两个随进里面,就具早膳相待。饭罢,吃了茶,李应问些枪法,见杨
雄、石秀说的有理,心中甚喜。

巳牌时分,那个副主管回来,李应唤到后堂问道:“去取的这人在那里?”主
管答道:“小人亲见朝奉,下了书,倒有放还之心,后来走出祝氏三杰,反焦躁起
来,书也不回,人也不放,定要解上州去。”李应失惊道:“他和我三家村里结生
死之交,书到便当依允,如何恁地起来?必是你说得不好,以致如此。杜主管,你
须自去走一遭,亲见祝朝奉,说个仔细缘由。”杜兴道:“小人愿去,只求东人亲
笔书缄,到那里方才肯放。”李应道:“说得是。”急取一幅花笺纸来,李应亲自
写了书札,封皮面上,使一个讳字图书,把与杜兴接了。后槽牵过一匹快马,备上
鞍辔,拿了鞭子,便出庄门,上马加鞭,奔祝家庄去了。李应道:“二位放心,我
这封亲笔书去,少刻定当放还。”杨雄、石秀深谢了,留在后堂饮酒等待。

看看天色待晚,不见杜兴回来,李应心中疑惑,再教人去接,只见庄客报道:
“杜主管回来了。”李应问道:“几个人回来?”庄客道:“只是主管独自一个跑
马回来。”李应摇着头道:“却又作怪!往常这厮,不是这等兜搭,今日缘何恁地?”
杨雄、石秀都跟出前厅来看时,只见杜兴下了马,入得庄门,见他模样,气得紫涨
了面皮,龇牙露嘴,半晌说不的话。有诗为证:
面貌天生本异常,怒时古怪更难当。
三分不像人模样,一似酆都焦面王。

李应出到厅前,连忙问道:“你且言备细缘故,怎么地来。”杜兴气定了,方
才道:“小人赍了东人书札,到他那里第三重门下,却好遇见祝龙、祝虎、祝彪弟
兄三个坐在那里,小人声了三个喏,祝彪喝道:‘你又来做甚么?’小人躬身禀道:
‘东人有书在此拜上。’祝彪那厮变了脸,骂道:‘你那主人恁地不晓人事!早晌
使个泼男女,来这里下书,要讨那个梁山泊贼人时迁。如今我正要解上州里去,又
来怎地?’小人说道:‘这个时迁不是梁山泊伙内人数,他自是蓟州来的客人。今
投见敝庄东人,不想误烧了官人店屋,明日东人自当依旧盖还,万望俯看薄面,高
抬贵手,宽恕宽恕。’祝家三个都叫道:‘不还,不还!’小人又道:‘官人请看
东人亲笔书札在此。’祝彪那厮接过书去,也不拆开来看,就手扯的粉碎,喝叫把
小人直叉出庄门。祝彪、祝虎发话道:‘休要惹老爷性发,把你那李应捉来,也做
梁山泊强寇解了去。’小人本不敢尽言,实被那三个畜生无礼,把东人百般秽骂,
便喝叫庄客来拿小人,被小人飞马走了。于路上气死小人,叵耐那厮枉与他许多年
结生死之交,今日全无些仁义。”诗曰:
徒闻似漆与如胶,利害场中忍便抛。
平日若无真义气,临时休说死生交。

李应听罢,心头那把无明业火,高举三千丈,按纳不下,大呼:“庄客,快备
我那马来!”杨雄、石秀谏道:“大官人息怒,休为小人们坏了贵处义气。”李应
那里肯听?便去房中披上一副黄金锁子甲,前后兽面掩心,穿一领大红袍,背胯边
插着飞刀五把,拿了点钢枪,戴上凤翅盔,出到庄前,点起三百悍勇庄客。杜兴也
披一副甲,持把枪上马,带领二十余骑马军。杨雄、石秀也抓扎起,挺着朴刀,跟
着李应的马,径奔祝家庄来。

日渐衔山时分,早到独龙冈前,便将人马排开。原来祝家庄又盖得好,占着这
座独龙山冈,四下一遭阔港。那庄正造在冈上,有三层城墙,都是顽石垒砌的,约
高二丈。前后两座庄门,两条吊桥。墙里四边,都盖窝铺,四下里遍插着枪刀军器,
门楼上排着战鼓铜锣。李应勒马,在庄前大叫:“祝家三子,怎敢毁谤老爷!”只
见庄门开处,拥出五六十骑马来,当先一骑似火炭赤的马上,坐着祝朝奉第三子祝
彪。怎生装束:

头戴缕金荷叶盔,身穿锁子梅花甲,腰悬锦袋弓和箭,手执纯钢刀与枪。马额
下垂照地红缨,人面上生撞天杀气。

李应见了祝彪,指着大骂道:“你这厮口边奶腥未退,头上胎发犹存,你爷与
我结生死之交,誓愿同心共意,保护村坊。你家但有事情,要取人时,早来早放;
要取物件,无有不奉。我今一个平人,二次修书来讨,你如何扯了我的书札,耻辱
我名,是何道理?”祝彪道:“俺家虽和你结生死之交,誓愿同心协意,共捉梁山
泊反贼,扫清山寨,你如何却结连反贼,意在谋叛?”李应喝道:“你说他是梁山
泊甚人?你这厮却冤平人做贼,当得何罪?”祝彪道:“贼人时迁已自招了,你休
要在这里胡说乱道,遮掩不过。你去便去,不去时,连你捉了,也做贼人解送。”

李应大怒,拍坐下马,挺手中枪,便奔祝彪。祝彪纵马去战李应。两个就独龙
冈前,一来一往,一上一下,斗了十七八合,祝彪战李应不过,拨回马便走。李应
纵马赶将去,祝彪把枪横担在马上,左手拈弓,右手取箭,搭上箭,拽满弓,觑得
较亲,背翻身一箭。李应急躲时,臂上早着。李应翻筋斗,坠下马来,祝彪便勒转
马来抢人。杨雄、石秀见了,大喝一声,拈两条朴刀,直奔祝彪马前杀将来。祝彪
抵当不住,急勒回马便走,早被杨雄一朴刀,戳在马后股上。那马负疼,壁直立起
来,险些儿把祝彪掀在马下,却得随从马上的人,都搭上箭射将来。杨雄、石秀见
了,自思又无衣甲遮身,只得退回不赶。杜兴也自把李应救起上马,先去了。杨雄、
石秀跟了众庄客也走了。祝家庄人马赶了二三里路,见天色晚来,也自回去了。

杜兴扶着李应,回到庄前,下了马,同入后堂坐。众宅眷都出来看视,拔了箭
矢,伏侍卸了衣甲,便把金疮药敷了疮口,连夜在后堂商议。杨雄、石秀与杜兴说
道:“既是大官人被那厮无礼,又中了箭,时迁亦不能够出来,都是我等连累大官
人了。我弟兄两个,只得上梁山泊去,恳告晁、宋二公并众头领,来与大官人报仇,
就救时迁。”因辞谢了李应。李应道:“非是我不用心,实出无奈。两位壮士,只
得休怪。”叫杜兴取些金银相赠,杨雄、石秀那里肯受。李应道:“江湖之上,二
位不必推却。”两个方才收受,拜辞了李应,杜兴送出村口,指与大路。杜兴作别
了,自回李家庄,不在话下。

且说杨雄、石秀取路投梁山泊来,早望见远远一处新造的酒店,那酒旗儿直挑
出来。两个入到店里,买些酒吃,就问路程。这酒店却是梁山泊新添设做眼的酒店,
正是石勇掌管。两个一面吃酒,一头动问酒保上梁山泊路程。石勇见他两个非常,
便来答应道:“你两位客人从那里来?要问上山去怎地?”杨雄道:“我们从蓟州
来。”石勇猛可想起道:“莫非足下是石秀么?”杨雄道:“我乃是杨雄,这个兄
弟是石秀。大哥如何得知石秀名?”石勇慌忙道:“小子不认得。前者戴宗哥哥到
蓟州回来,多曾称说兄长。闻名久矣,今得上山,且喜,且喜。”三个叙礼罢,杨
雄、石秀把上件事都对石勇说了。石勇随即叫酒保置办分例酒来相待。推开后面水
亭上窗子,拽起弓,放了一枝响箭。只见对港芦苇丛中,早有小喽罗摇过船来。石
勇便邀二位上船,直送到鸭嘴滩上岸。石勇已自先使人上山去报知。早见戴宗、杨
林下山来迎接。俱各叙礼罢,一同上至大寨里。

众头领知道有好汉上山,都来聚会,大寨坐下。戴宗、杨林引杨雄、石秀、上
厅参见晁盖、宋江,并众头领。相见已罢,晁盖细问两个踪迹,杨雄、石秀把本身
武艺,投托入伙先说了,众人大喜,让位而坐。杨雄渐渐说到有个来投托大寨同入
伙的时迁,不合偷了祝家店里报晓鸡,一时争闹起来,石秀放火烧了他店屋,时迁
被捉,李应二次修书去讨,怎当祝家三子坚执不放,誓愿要捉山寨里好汉,且又千
般辱骂,叵耐那厮十分无礼。不说万事皆休,才然说罢,晁盖大怒,喝叫:“孩儿
们将这两个与我斩讫报来!”正是:
杨雄石秀少商量,引带时迁行不臧。
豪杰心肠虽似火,绿林法度却如霜。
宋江慌忙劝道:“哥哥息怒,两个壮士,不远千里而来,同心协助,如何却要斩他?”
晁盖道:“俺梁山泊好汉,自从火并王伦之后,便以忠义为主,全施仁德于民。一
个个兄弟下山去,不曾折了锐气。新旧上山的兄弟们,各各都有豪杰的光彩。这厮
两个,把梁山泊好汉的名目去偷鸡吃,因此连累我等受辱。今日先斩了这两个,将
这厮首级去那里号令,便起军马去,就洗荡了那个村坊,不要输了锐气。孩儿们快
斩了报来。”宋江劝住道:“不然。哥哥不听这两位贤弟却才所说,那个鼓上蚤时
迁,他原是此等人,以致惹起祝家那厮来,岂是这二位贤弟要玷辱山寨?我也每每
听得有人说,祝家庄那厮,要和俺山寨敌对。即目山寨人马数多,钱粮缺少,非是
我等要去寻他,那厮倒来吹毛求疵,因而正好乘势去拿那厮。若打得此庄,倒有三
五年粮食。非是我们生事害他,其实那厮无礼。哥哥权且息怒,小可不才,亲领一
支军马,启请几位贤弟们下山,去打祝家庄。若不洗荡得那个村坊,誓不还山。一
是与山寨报仇,不折了锐气;二乃免此小辈被他耻辱;三则得许多粮食,以供山寨
之用;四者就请李应上山入伙。”吴学究道:“公明哥哥之言最好,岂可山寨自斩
手足之人?”戴宗便道:“宁乃斩了小弟,不可绝了贤路。”众头领力劝,晁盖方
才免了二人。杨雄、石秀也自谢罪。宋江抚谕道:“贤弟休生异心,此是山寨号令,
不得不如此。便是宋江,倘有过失,也须斩首,不敢容情。如今新近又立了铁面孔
目裴宣做军政司,赏功罚罪,已有定例。贤弟只得恕罪恕罪。”杨雄、石秀拜罢,
谢罪已了,晁盖叫去坐在杨林之下。山寨里都唤小喽罗来参贺新头领已毕,一面杀
牛宰马,且做庆喜筵席,拨定两所房屋,教杨雄、石秀安歇,每人拨十个小喽罗伏
侍。当晚席散,次日再备筵席,会众商量议事。

宋江教唤铁面孔目裴宣,计较下山人数,启请诸位头领,同宋江去打祝家庄,
定要洗荡了那个村坊。商量已定,除晁盖头领镇守山寨不动外,留下吴学究、刘唐,
并阮家三弟兄、吕方、郭盛,护持大寨。原拨定守滩、守关、守店有职事人员,俱
各不动。又拨新到头领孟康管造船只,顶替马麟监督战船。写下告示,将下山打祝
家庄头领分作两起:头一拨,宋江、花荣、李俊、穆弘、李逵、杨雄、石秀、黄信、
欧鹏、杨林,带领三千小喽罗,三百马军,披挂已了,下山前进;第二拨便是林冲、
秦明、戴宗、张横、张顺、马麟、邓飞、王矮虎、白胜,也带三千小喽罗,三百马
军,随后接应;再着金沙滩、鸭嘴滩二处小寨,只教宋万、郑天寿守把,就行接应
粮草。晁盖送路已了,自回山寨。

且说宋江并众头领径奔祝家庄来,于路无话。早来到独龙山前,尚有一里多路,
前军下了寨栅。宋江在中军帐里坐下,便和花荣商议道:“我听得说祝家庄里路径
甚杂,未可进兵,且先使两个入去探听路途曲折,知得顺逆路程,却才进去,与他
敌对。”李逵便道:“哥哥,兄弟闲了多时,不曾杀得一人,我便先去走一遭。”
宋江道:“兄弟,你去不得。若是破阵冲敌,用着你先去。这是做细作的勾当,用
你不着。”李逵笑道:“量这个鸟庄,何须哥哥费力,只兄弟自带三二百个孩儿杀
将去,把这个鸟庄上人都砍了,何须要人先去打听。”宋江喝道:“你这厮休胡说!
且一壁厢去,叫你便来。”李逵走开去了,自说道:“打死几个苍蝇,也何须大惊
小怪。”宋江便唤石秀来说道:“兄弟曾到彼处,可和杨林走一遭。”石秀便道:
“如今哥哥许多人马到这里,他庄上如何不提备,我们扮作甚么人入去好?”杨林
便道:“我自打扮了解魇的法师去,身边藏了短刀,手里擎着法环,于路摇将入去。
你只听我法环响,不要离了我前后。”石秀道:“我在蓟州原曾卖柴,我只是挑一
担柴进去卖便了。身边藏了暗器,有些缓急,匾担也用得着。”杨林道:“好,好。
我和你计较了,今夜打点,五更起来便行。”正是只为一鸡小忿,致令众虎相争,
所以古人有篇《西江月》道得好:

软弱安身之本,刚强惹祸之胎。无争无竞是贤才,亏我些儿何碍!

钝斧锤
砖易碎,快刀劈水难开。但看发白齿牙衰,惟有舌根不坏。

且说石秀挑着柴担先入去,行不到二十来里,只见路径曲折多杂,四下里弯环
相似,树木丛密,难认路头,石秀便歇下柴担不走。听得背后法环响得渐近,石秀
看时,却见杨林头带一个破笠子,身穿一领旧法衣,手里擎着法环,于路摇将进来。
石秀见没人,叫住杨林说道:“看见路径弯杂难认,不知那里是我前日跟随李应来
时的路。天色已晚,他们众人都是熟路,正看不仔细。”杨林道:“不要管他路径
曲直,只顾拣大路走便了。”石秀又挑了柴,只顾望大路先走,见前面一村人家,
数处酒店肉店。石秀挑着柴,便望酒店门前歇了,只见各店内都把刀枪插在门前,
每人身上穿一领黄背心,写个大“祝”字,往来的人,亦各如此。石秀见了,便看
着一个年老的人,唱个喏,拜揖道:“丈人,请问此间是何风俗?为甚都把刀枪插
在当门?”那老人道:“你是那里来的客人?原来不知,只可快走。”石秀道:“小
人是山东贩枣子的客人,消折了本钱,回乡不得,因此担柴来这里卖,不知此间乡
俗地理。”老人道:“只可快走别处躲避,这里早晚要大厮杀也。”石秀道:“此
间这等好村坊去处,怎地了大厮杀?”老人道:“客人,你敢真个不知,我说与你。
俺这里唤做祝家村,冈上便是祝朝奉衙里。如今恶了梁山泊好汉,现今引领军马在
村口,要来厮杀。却怕我这村里路杂,未敢入来,现今驻扎在外面。如今祝家庄上
行号令下来,每户人家,要我们精壮后生准备着,但有令传来,便去策应。”石秀
道:“丈人村中,总有多少人家?”老人道:“只我这祝家村,也有一二万人家,
东西还有两村人接应。东村唤做扑天雕李应李大官人,西村唤扈太公庄,有个女儿,
唤做扈三娘,绰号一丈青,十分了得。”石秀道:“似此,如何却怕梁山泊做甚么?”
那老人道:“若是我们初来时,不知路的,也要吃捉了。”石秀道:“丈人,怎地
初来时要吃捉了?”老人道:“我这村里的路,有首诗说道:‘好个祝家庄,尽是
盘陀路。容易入得来,只是出不去。’”石秀听罢,便哭起来,扑翻身便拜,向那
老人道:“小人是个江湖上折了本钱,归乡不得的人,倘或卖了柴出去,撞见厮杀,
走不脱,却不是苦?爷爷,怎地可怜见小人,情愿把这担柴相送爷爷,只指小人出
去的路罢。”那老人道:“我如何白要你的柴?我就买你的。你且入来,请你吃些
酒饭。”

石秀便谢了,挑着柴,跟那老人入到屋里。那老人筛下两碗白酒,盛一碗糕糜,
叫石秀吃了。石秀再拜谢道:“爷爷指教出去的路径。”那老人道:“你便从村里
走去,只看有白杨树,便可转弯,不问路道阔狭,但有白杨树的转弯,便是活路,
没那树时,都是死路,如有别的树木转弯,也不是活路。若还走差了,左来右去,
只走不出去。更兼死路里地下埋藏着竹签铁蒺藜,若是走差了,踏着飞签,准定吃
捉了,待走那里去!”石秀拜谢了,便问:“爷爷高姓?”那老人道:“这村里姓
祝的最多,惟有我复姓钟离,土居在此。”石秀道:“酒饭小人都吃够了,改日当
厚报。”

正说之间,只听得外面闹吵。石秀听得道拿了一个细作。石秀吃了一惊,跟那
老人出来看时,只见七八十个军人背绑着一个人过来。石秀看时,却是杨林,剥得
赤条条的,索子绑着。石秀看了,只暗暗地叫苦,悄悄假问老人道:“这个拿了的
是甚么人?为甚事绑了他?”那老人道:“你不见说他是宋江那里来的细作?”石
秀又问道:“怎地吃他拿了?”那老人道:“说这厮也好大胆,独自一个来做细作,
打扮做个解魇法师,闪入村里来。却又不认这路,只拣大路走了,左来右去,只走
了死路,又不晓的白杨树转弯抹角的消息。人见他走得差了,来路跷蹊,报与庄上
官人们来捉他,这厮方才又掣出刀来,手起伤了四五个人。当不住这里人多,一发
上,因此吃拿了,有人认得他从来是贼,叫做锦豹子杨林。”

说言未了,只听得前面喝道,说是庄上三官人巡绰过来。石秀在壁缝里张时,
看见前面摆着二十对缨枪,后面四五个人骑战马,都弯弓插箭,又有三五对青白哨
马,中间拥着一个年少的壮士,坐在一匹雪白马上,全副披挂了弓箭,手执一条银
枪。石秀自认得他,特地问老人道:“过去相公是谁?”那老人道:“这个正是祝
朝奉第三子,唤做祝彪,定着西村扈家庄一丈青为妻。弟兄三个,只有他第一了得。”
石秀拜谢道:“老爷爷指点寻路出去。”那老人道:“今日晚了,前面倘或厮杀,
枉送了你性命。”石秀道:“爷爷,可救一命则个。”那老人道:“你且在我家歇
一夜,明日打听得没事,便可出去。”石秀拜谢了,坐在他家,只听得门前四五替
报马报将来,排门分付道:“你那百姓,今夜只看红灯为号,齐心并力,捉拿梁山
泊贼人,解官请赏。”叫过去了。石秀问道:“这个人是谁?”那老人道:“这个
官人是本处捕盗巡检,今夜约会要捉宋江。”石秀见说,心中自忖了一回,讨个火
把,叫了安置,自去屋后草窝里睡了。

却说宋江军马在村口屯驻,不见杨林、石秀出来回报,随后又使欧鹏去到村口,
出来回报道:“听得那里讲动,说道捉了一个细作,小弟见路径又杂难认,不敢深
入重地。”宋江听罢,忿怒道:“如何等得回报了进兵?又吃拿了一个细作,必然
陷了两个兄弟,我们今夜只顾进兵,杀将入去,也要救他两个兄弟。未知你众头领
意下如何?”只见李逵便道:“我先杀入去,看是如何?”宋江听得,随即便传将
令,教军士都披挂了。李逵、杨雄前一队做先锋,使李俊等引军做合后,穆弘居左,
黄信在右,宋江、花荣、欧鹏等中军头领,摇旗呐喊,擂鼓鸣锣,大刀阔斧,杀奔
祝家庄来。比及杀到独龙冈上,是黄昏时分,宋江催趱前军打庄。先锋李逵脱得赤
条条的,挥两把夹钢板斧,火剌剌地杀向前来。到得庄前看时,已把吊桥高高地拽
起了,庄门里不见一点火。李逵便要下水过去,杨雄扯住道:“使不得。关闭庄门,
必有计策。待哥哥来,别有商议。”李逵那里忍得住,拍着双斧,隔岸大骂道:“那
鸟祝太公老贼!你出来,黑旋风爷爷在这里!”庄上只是不应。宋江中军人马到来,
杨雄接着,报说庄上并不见人马,亦无动静。宋江勒马看时,庄上不见刀枪人马,
心中疑惑,猛省道:“我的不是了。天书上明明戒说,临敌休急暴。是我一时见不
到,只要救两个兄弟,以此连夜进兵,不期深入重地。直到了他庄前,不见敌军,
他必有计策,快教三军且退。”李逵叫道:“哥哥,军马到这里了,休要退兵,我
与你先杀过去,你们都跟我来。”

说犹未了,庄上早知,只听得祝家庄里一个号炮,直飞起半天里去,那独龙冈
上千百把火把,一齐点着,那门楼上弩箭如雨点般射将来。宋江急取旧路回军,只
见后军头领李俊人马先发起喊来,说道:“来的旧路都阻塞了,必有埋伏。”宋江
教军马四下里寻路走。李逵挥起双斧,往来寻人厮杀,不见一个敌军。只见独龙冈
上山顶又放一个炮来,响声未绝,四下里喊声震地,惊的宋公明目睁口呆,罔知所
措。你便有文韬武略,怎逃出地网天罗?正是:安排缚虎擒龙计,要捉惊天动地人。

毕竟宋公明并众头领怎地脱身,且听下回分解。