\chapter{梁山泊十面埋伏~宋公明两赢童贯}

话说当日宋江阵中,前部先锋三队军马赶过对阵,大刀阔斧,杀得童贯三军人
马大败亏输,星落云散,七损八伤。军士抛金弃鼓,撇戟丢枪,觅子寻爷,呼兄唤
弟,折了万余人马,退三十里外扎住。吴用在阵中鸣金收军,传令道:“且未可尽
情追杀,略报个信与他。”梁山泊人马都收回山寨,各自献功请赏。

且说童贯输了一阵,折了人马,早扎寨栅安歇下,心中忧闷,会集诸将商议。酆美、
毕胜二将道:“枢相休忧,此寇知得官军到来,预先摆布下这座阵势。官军初到,
不知虚实,因此中贼奸计。想此草寇,只是倚山为势,多设军马,虚张声势,一时
失了地利。我等且再整练马步将士,停歇三日,养成锐气,将息战马,三日后将全
部军将分作长蛇之阵,俱是步军杀将去。此阵如长山之蛇,击首则尾应,击尾则首
应,击中则首尾皆应,都要连络不断,决此一阵,必见大功。”童贯道:“此计大
妙,正合吾意。”即时传下将令,整肃三军,训练已定。

第三日,五更造饭,军将饱食,马带皮甲,人披铁铠,大刀阔斧,弓弩上弦,正是
枪刀流水急,人马撮风行。大将酆美、毕胜当先引军,浩浩荡荡,杀奔梁山泊来。
八路军马,分于左右,前面发三百铁甲哨马前去探路,回来报与童贯中军知道,说:
“前日战场上,并不见一个军马。”童贯听了心疑,自来前军问酆美、毕胜道:“退
兵如何?”酆美答道:“休生退心,只顾冲突将去。长蛇阵摆定,怕做甚么?”官
军迤逦前行,直进到水泊边,竟不见一个军马,但见隔水茫茫荡荡,都是芦苇烟火,
远远地遥望见水浒寨山顶上一面杏黄旗在那里招,亦不见些动静。

童贯与酆美、毕胜勒马在万军之前,遥望见对岸水面上芦林中一只小船,船上一个
人,头戴青箬笠,身披绿蓑衣,斜倚着船背,岸西独自钓鱼。童贯的步军,隔着岸
叫那渔人,问道:“贼在那里?”那渔人只不应。童贯叫能射箭的放箭。两骑马直
近岸边滩头来,近水兜住马,扳弓搭箭,望那渔人后心,飕地一箭去。那枝箭正射
到箬笠上,当地一声响,那箭落下水里去了。这一个马军放一箭,正射到蓑衣上,
当地一声响,那箭也落下水里去了。那两个马军是童贯军中第一惯射弓箭的。两个
吃了一惊,勒回马,上来欠身禀童贯道:“两箭皆中,只是射不透,不知他身上穿
着甚的?”童贯再拨三百能射硬弓的哨路马军,来滩头摆开,一齐望着那渔人放箭。
那乱箭射去,渔人不慌,多有落在水里的,也有射着船上的。但射着蓑衣箬笠的,
都落下水里去。童贯见射他不死,便差会水的军汉脱了衣甲,赴水过去,捉那渔人,
早有三五十人赴将开去。那渔人听得船尾水响,知有人来,不慌不忙,放下鱼钓,
取棹竿拿在身边,近船来的,一棹竿一个,太阳上着的,脑袋上着的,面门上着的,
都打下水里去了。后面见沉了几个,都赴转岸上,去寻衣甲。

童贯看见大怒,教拨五百军汉下水去,定要拿这渔人;若有回来的,一刀两段。五
百军人脱了衣甲,呐声喊,一齐都跳下水里,赴将过去。那渔人回转船头,指着岸
上童贯大骂道:“乱国贼臣,害民的禽兽,来这里纳命,犹自不知死哩!”童贯大
怒,喝教马军放箭。那渔人呵呵大笑,说道:“兀那里有军马到了。”把手指一指,
弃了蓑衣箬笠,翻身攒入水底下去了。那五百军正赴到船边,只听得在水中乱叫,
都沉下去了。那渔人正是浪里白跳张顺,头上箬笠,上面是箬叶裹着,里面是铜打
成的;蓑衣里面,一片熟铜打就,披着如龟壳相似,可知道箭矢射不入。张顺攒下
水底,拔出腰刀,只顾排头价戳人,都沉下去,血水滚将起来。有乖的赴了开去,
逃得性命。

童贯在岸上看得呆了,身边一将指道:“山顶上那面黄旗正在那里磨动。”童
贯定睛看了,不解何意,众将也没做道理处。酆美道:“把三百铁甲哨马,分作两
队,教去两边山后出哨,看是如何。”却才分到山前,只听得芦苇中一个轰天雷炮
飞起,火烟缭乱。两边哨马齐回来报:“有伏兵到了。”童贯在马上,那一惊不小。
酆美、毕胜两边差人,教军士休要乱动,数十万军都掣刀在手。前后飞马来叫道:
“如有先走的便斩!”按住三军人马。童贯且与众将立马望时,山背后鼓声震地,
喊杀喧天,早飞出一彪军马,都打着黄旗,当先有两员骁将领兵。怎见得那队军马
整齐:
黄旗拥出万山中,烁烁金光射碧空。
马似怒涛冲石壁,人如烈火撼天风。
鼓声震动森罗殿,炮力掀翻泰华宫。
剑队暗藏插翅虎,枪林飞出美髯公。

两骑黄鬃马上,两员英雄头领:上首美髯公朱仝,下首插翅虎雷横,带领五千人马,
直杀奔官军。童贯令大将酆美、毕胜当先迎敌。两个得令,便骤马挺枪出阵,大骂:
“无端草贼,不来投降,更待何时!”雷横在马上大笑,喝道:“匹夫死在眼前,
尚且不知,怎敢与吾决战?”毕胜大怒,拍马挺枪,直取雷横,雷横也使枪来迎。
两马相交,军器并举,二将约战到二十余合,不分胜败。酆美见毕胜战久,不能取
胜,拍马舞刀,径来助战。朱仝见了,大喝一声,飞马抡刀,来战酆美。四匹马两
对儿在阵前厮杀。童贯看了,喝采不迭。斗到涧深里,只见朱仝、雷横卖个破绽,
拨回马头,望本阵便走。酆美、毕胜两将不舍,拍马追将过去。对阵军发声喊,望
山后便走,童贯叫尽力追赶过山脚去,只听得山顶上画角齐鸣,众将抬头看时,前
后两个炮直飞起来。童贯知有伏兵,把军马约住,教不要去赶。

只见山顶上闪出那面杏黄旗来,上面绣着“替天行道”四字。童贯踅过山那边看时,
见山头上一簇杂彩绣旗开处,显出那个郓城县盖世英雄山东呼保义宋江来。背后便
是军师吴用、公孙胜、花荣、徐宁,金枪手,银枪手,众多好汉。童贯见了大怒,
便差人马上山来拿宋江。大军人马分为两路,却待上山,只听得山顶上鼓乐喧天,
众好汉都笑。童贯越添心上怒,咬碎口中牙,喝道:“这贼怎敢戏吾!我当自擒这
厮。”酆美谏道:“枢相,彼必有计,不可亲临险地,且请回军,来日却再打听虚
实,方可进兵。”童贯道:“胡说!事已到这里,岂可退军?教星夜与贼交锋。今已
见贼,势不容退。”语犹未绝,只听得后军呐喊,探子报道:“正西山后冲出一彪
军来,把后军杀开做两处。”童贯大惊,带了酆美、毕胜,急回来救应后军时,东
边山后鼓声响处,又早飞出一队人马来。一半是红旗,一半是青旗,捧着两员大将,
引五千军马杀将来。那红旗军随红旗,青旗军随青旗,队伍端的整齐。但见:
对对红旗间翠袍,争飞战马转山腰。
日烘旗帜青龙见,风摆旌旗朱雀摇。
二队精兵皆勇猛,两员上将显英豪。
秦明手舞狼牙棍,关胜斜横偃月刀。

那红旗队里头领是霹雳火秦明,青旗队里头领是大刀关胜。二将在马上杀来,大喝
道:“童贯早纳下首级!”童贯大怒,便差酆美来战关胜,毕胜去斗秦明。童贯见
后军发喊得紧,又教鸣金收军,且休恋战,延便且退。朱仝、雷横引黄旗军又杀将
来,两下里夹攻,童贯军兵大乱,酆美、毕胜保护着童贯,逃命而走。正行之间,
刺斜里又飞出一彪军马来,接住了厮杀。那队军马,一半是白旗,一半是黑旗,黑
白旗中,也捧着两员虎将,引五千军马,拦住去路。这队军端的齐整:
炮似轰雷山石裂,绿林深处显戈矛。
素袍兵出银河涌,玄甲军来黑气浮。
两股鞭飞风雨响,一条枪到鬼神愁。
左边大将呼延灼,右手英雄豹子头。

那黑旗队里头领是双鞭呼延灼,白旗队里头领是豹子头林冲。二将在马上大喝道:
“奸臣童贯,待走那里去?早来受死!”一冲直杀入军中来。那睢州都监段鹏举接
住呼延灼交战,洳州都监马万里接着林冲厮杀。这马万里与林冲斗不到数合,气力
不加,却待要走,被林冲大喝一声,慌了手脚,着了一矛,戳在马下。段鹏举看见
马万里被林冲搠死,无心恋战,隔过呼延灼双鞭,霍地拨回马便走。呼延灼奋勇赶
将入来,两军混战,童贯只教夺路且回。只听得前军喊声大举,山背后飞出一彪步
军,直杀入垓心里来。当先一僧一行者,领着军兵,大叫道:“休教走了童贯!”
那和尚不修经忏,专好杀人,单号花和尚,双名鲁智深。这行者景阳冈曾打虎,水
浒寨最英雄,有名行者武松。这两个杀入阵来。怎见得,有《西江月》为证:
鲁智深一条禅杖,武行者两口钢刀。钢刀飞出火光飘,禅杖来如铁炮。

禅杖打
开脑袋,钢刀截断人腰。两般军器不相饶,百万军中显耀。
童贯众军被鲁智深、武松引领步军一冲,早四分五落。官军人马,前无去路,后没
退兵,只得引酆美、毕胜撞透重围,杀条血路,奔过山背后来。正方喘息,又听得
炮声大震,战鼓齐鸣,看两员猛将当先,一簇步军拦路。怎见得:
两头蛇腥风难近,双尾蝎毒气齐喷。钢叉一对世无伦,较猎场中声震。

左手解
珍出众,右手解宝超群。数千铁甲虎狼军,搅碎长蛇大阵。
来的步军头领解珍、解宝,各拈五股钢叉,又引领步军杀入阵内,童贯人马遮拦不
住,突围而走,五面马军步军一齐追杀,赶得官军星落云散,酆美、毕胜力保童贯
而走。见解珍、解宝兄弟两个,挺起钢叉,直冲到马前。童贯急忙拍马,望刺斜里
便走,背后酆美、毕胜赶来救应;又得唐州都监韩天麟、邓州都监王义,四个并力,
杀出垓心。方才进步,喘息未定,只见前面尘起,叫杀连天,绿丛丛林子里又早飞
出一彪人马,当先两员猛将,拦住去路。那两个是谁,但见:
一个宣花大斧,一个出白银枪。枪如毒蟒露梢长,斧起处似开山神将。一个风流俊
骨,一个猛烈刚肠。董平国士更无双,急先锋索超谁让。

这两员猛将:双枪将董平、急先锋索超,两个更不打话,飞马直取童贯。王义挺枪
去迎,被索超手起斧落,砍于马下。韩天麟来救,被董平一枪搠死。酆美、毕胜死
保护童贯,奔马逃命。四下里金鼓乱响,正不知何处军来。童贯拢马上坡看时,四
面八方四队马军,两胁两队步军,栲栳圈,簸箕掌,梁山泊军马大队齐齐杀来,童
贯军马如风落云散,东零西乱。正看之间,山坡下一簇人马出来,认的旗号是陈州
都监吴秉彝、许州都监李明。这两个引着些断枪折戟,败残军马,踅转琳琅山躲避。
看见招呼时,正欲上坡,急调人马,又见山侧喊声起来,飞过一彪人马赶出,两把
认旗招,马上两员猛将,各执兵器,飞奔官军。这两个是谁,有《临江仙》词为
证:
盔上长缨飘火焰,纷纷乱撒猩红,胸中豪气吐长虹。战袍裁蜀锦,铠甲镀金铜。

两
口宝刀如雪练,垓心抖擞威风,左冲右突显英雄。军班青面兽,好汉九纹龙。
这两员猛将,正是杨志、史进,两骑马,两口刀,却才截住吴秉彝、李明两个军官
厮杀。李明挺枪向前,来斗杨志;吴秉彝使方天戟,来战史进。两对儿在山坡下一
来一往,盘盘旋旋,各逞平生武艺。童贯在山坡上勒住马,观之不定。四个人约斗
到三十余合,吴秉彝用戟奔史进心坎上戳将来,史进只一闪,那枝戟从肋窝里放个
过,吴秉彝连人和马抢近前来,被史进手起刀落,只见一条血颡光连肉,顿落金鍪
在马边,吴秉彝死于坡下。李明见先折了一个,却待也要拨回马走时,被杨志大喝
一声,惊得魂消魄散,胆颤心寒,手中那条枪,不知颠倒。杨志把那口刀从顶门上
劈将下来,李明只一闪,那刀正剁着马的后胯下,那马后蹄将下去,把李明闪下
马来。弃了手中枪,却待奔走,这杨志手快,随复一刀,砍个正着。可怜李明半世
军官,化作南柯一梦。两员官将,皆死于坡下。杨志、史进追杀败军,正如砍瓜截
瓠相似。

童贯和酆美、毕胜在山坡上看了,不敢下来,身无所措,三个商量道:“似此如何
杀得出去?”酆美道:“枢相且宽心,小将望见正南上尚兀自有大队官军扎住在那
里,旗不倒,可以解救。毕都统保守枢相在山头,酆美杀开条路,取那枝军马来,
保护枢相出去。”童贯道:“天色将晚,你可善觑方便,疾去早来。”酆美提着大
杆刀,飞马杀下山来,冲开条路,直到南边。看那队军马时,却是嵩州都监周信,
把军兵团团摆定,死命抵住。垓心里看见那酆美来,便接入阵内,问:“枢相在那
里?”酆美道:“只在前面山坡上,专等你这枝军马去救护杀出来。事不宜迟,火
速便起。”周信听说罢,便教传令,马步军兵,都要相顾,休失队伍,齐心并力。
二员大将当先,众军助喊,杀奔山坡边来。行不到一箭之地,刺斜里一枝军到,酆
美舞刀,径出迎敌,认得是睢州都监段鹏举,三个都相见了,合兵一处。杀到山坡
下,毕胜下坡迎接上去,见了童贯,一处商议道:“今晚便杀出去好?却捱到来朝
去好?”酆美道:“我四人死保枢相,只就今晚杀透重围出去,可脱贼寇。”

看看近夜,只听得四边喊声不绝,金鼓乱鸣。约有二更时候,星月光亮,酆美当先,
众军官簇拥童贯在中间,一齐并力,杀下山坡来。只听得四下里乱叫道:“不要走
了童贯!”众官军只望正南路冲杀过来。看看混战到四更左右,杀出垓心,童贯在
马上以手加额,顶礼天地神明道:“惭愧,脱得这场大难!”催赶出界,奔济州去。
却才欢喜未尽,只见前面山坡边一带火把,不计其数;背后喊声又起,看见火把光
中两条好汉,拈着两口朴刀,引出一员骑白马的英雄大将,在马上横着一条点钢枪。
那人是谁,有《临江仙》词为证:
马步军中推第一,天罡数内为尊,上天降下恶星辰。眼珠如点漆,面部似镌银。

丈
二钢枪无敌手,身骑快马腾云,人材武艺两超群。梁山卢俊义,河北玉麒麟。
那马上的英雄大将,正是玉麒麟卢俊义。马前这两个使朴刀的好汉:一个是病关索
杨雄,一个是拚命三郎石秀,在火把光中引着三千余人,抖擞精神,拦住去路。卢
俊义在马上大喝道:“童贯不下马受缚,更待何时?”童贯听得,对众道:“前有
伏兵,后有追兵,似此如之奈何?”酆美道:“小将舍条性命,以报枢相,汝等众
官,紧保枢相,夺路望济州去,我自战住此贼。”酆美拍马舞刀,直奔卢俊义。两
马相交,斗不到数合,被卢俊义把枪只一逼,逼过大刀,抢入身去,劈腰提住,一
脚蹬开战马,把酆美活捉去了。杨雄、石秀便来接应,众军齐上,横拖倒拽捉了去。
毕胜和周信、段鹏举舍命保童贯,冲杀拦路军兵,且战且走;背后卢俊义赶来,童
贯败军,忙忙似丧家之狗,急急如漏网之鱼。天晓脱得追兵,望济州来。正走之间,
前面山坡背后又冲出一队步军来,那军都是铁掩心甲,绛红罗头巾。当先四员步军
头领,毕竟是谁:
黑旋风双持板斧,丧门神单仗龙泉。项充、李衮在旁边,手舞团牌体健。斩虎须投
大穴,诛龙必向深渊。三军威势振青天,恶鬼眼前活现。

这李逵抡两把板斧,鲍旭仗一口宝剑,项充、李衮各舞蛮牌遮护,却似一团火块,
从地皮上滚将来,杀得官军四分五落而走。童贯与众将且战且走,只逃性命。李逵
直砍入马军队里,把段鹏举马脚砍翻,掀将下来,就势一斧,劈开脑袋;再复一斧,
砍断咽喉,眼见得段鹏举不活了。且说败残官军将次捱到济州,真乃是头盔斜掩耳,
护项半兜腮,马步三军没了气力,人困马乏。奔到一条溪边,军马都且去吃水,只
听得对溪一声炮响,箭矢如飞蝗一般射将过来。官军急上溪岸去,树林边转出一彪
军马来。为头马上三个英雄是谁:
舞动一条玉蟒,撒开万点飞星。东昌骠骑是张清,没羽箭谁人敢近!飞枪的枪无虚
发,飞叉的叉不容情。两员虎将势纵横,左右马前帮定。

原来这没羽箭张清和龚旺、丁得孙带领三百余骑马军。那一队骁骑马军,都是铜铃
面具,雉尾红缨,轻弓短箭,绣旗花枪。三将为头,直冲将来。嵩州都监周信见张
清军马少,便来迎敌;毕胜保着童贯而走。周信纵马挺枪来迎,只见张清左手纳住
枪,右手似招宝七郎之形,口中喝一声道:“着!”去周信鼻凹上只一石子打中,
翻身落马;龚旺、丁得孙旁边飞马来相助,将那两条叉戳定咽喉,好似霜摧边地草,
雨打上林花,周信死于马下。童贯止和毕胜逃命,不敢入济州,引了败残军马,连
夜投东京去了,于路收拾逃难军马下寨。

原来宋江有仁有德,素怀归顺之心,不肯尽情追杀;惟恐众将不舍,要追童贯,火
急差戴宗传下将令,布告众头领,收拾各路军马步卒,都回山寨请功。各处鸣金收
军而回,鞍上将都敲金镫,步下卒齐唱凯歌,纷纷尽入梁山泊,个个同回宛子城。
宋江、吴用、公孙胜先到水浒寨中,忠义堂上坐下,令裴宣验看各人功赏。卢俊义
活捉酆美,解上寨来,跪在堂前。宋江自解其缚,请入堂内上坐,亲自捧杯陪话,
奉酒压惊。众头领都到堂上,是日杀牛宰马,重赏三军,留酆美住了两日,备办鞍
马,送下山去。酆美大喜。宋江陪话道:“将军阵前阵后,冒渎威严,切乞恕罪。
宋江等本无异心,只要归顺朝廷,与国家出力,被这不公不法之人逼得如此,望将
军回朝,善言解救。倘得他日重见恩光,生死不忘大德。”酆美拜谢不杀之恩,登
程下山。宋江令人直送出界,回京不在话下。

宋江回到忠义堂上,再与吴用等众头领商量。原来今次用此十面埋伏之计,都是吴
用机谋布置,杀得童贯胆寒心碎,梦里也怕,大军三停折了二停。吴用道:“童贯
回到京师,奏了官家,如何不再起兵来?必得一人直投东京,探听虚实,回报山寨,
预作准备。”宋江道:“军师此论,正合吾心。你弟兄中,不知那个敢去?”只见
坐次之中一个人应道:“兄弟愿往。”众人看了,都道:“须是他去,必干大事。”
不是这个人去,有分教:重施谋略,再败官军。且是:冲阵马亡青嶂下,戏波船陷
绿蒲中。
毕竟梁山泊是谁人前去打听,且听下回分解。