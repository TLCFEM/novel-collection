\chapter{宁海军宋江吊孝~涌金门张顺归神}

话说当下费保对李俊道:“小弟虽是个愚卤匹夫,曾闻聪明人道:‘世事有成
必有败,为人有兴必有衰。’哥哥在梁山泊,勋业到今,已经数十余载,更兼百战
百胜。去破辽国时,不曾损折了一个兄弟;今番收方腊,眼见挫动锐气,天数不久。
为何小弟不愿为官?为因世情不好。有日太平之后,一个个必然来侵害你性命。自
古道:‘太平本是将军定,不许将军见太平。’此言极妙。今我四人既已结义了哥
哥三人,何不趁此气数未尽之时,寻个了身达命之处,对付些钱财,打了一只大船,
聚集几人水手,江海内寻个净办处安身,以终天年,岂不美哉!”李俊听罢,倒地
便拜,说道:“仁兄,重蒙教导,指引愚迷,十分全美。只是方腊未曾剿得,宋公
明恩义难抛,行此一步未得。今日便随贤弟去了,全不见平生相聚的义气。若是众
位肯姑待李俊,容待收伏方腊之后,李俊引两个兄弟,径来相投,万望带挈。是必
贤弟们先准备下这条门路。若负今日之言,天实厌之,非为男子也!”那四个道:
“我等准备下船只,专望哥哥到来,切不可负约!”李俊、费保结义饮酒,都约定
了,誓不负盟。

次日,李俊辞别了费保四人,自和童威、童猛回来参见宋先锋,俱说费保等四人不
愿为官,只愿打鱼快活。宋江又嗟叹了一回,传令整点水陆军兵起程。吴江县已无
贼寇,直取平望镇,长驱而进,前望秀州而来。本州守将段恺闻知苏州三大王方貌
已死,只思量收拾走路。使人探知大军离城不远,遥望水陆路上,旌旗蔽日,船马
相连,吓得魂消胆丧。前队大将关胜、秦明已到城下,便分调水军船只,围住西门。
段恺在城上叫道:“不须攻击,准备纳降。”随即开放城门,段恺香花灯烛,牵羊
担酒,迎接宋先锋入城,直到州治歇下。

段恺为首参见了,宋江抚慰段恺,复为良臣,便出榜安民。段恺称说:“恺等原是
睦州良民,累被方腊残害,不得已投顺部下。今得天兵到此,安敢不降?”宋江备
问:“杭州宁海军城池,是甚人守据?有多少人马良将?”段恺禀道:“杭州城郭
阔远,人烟稠密,东北旱路,南面大江,西面是湖,乃是方腊大太子南安王方天定
守把,部下有七万余军马,二十四员战将,四个元帅,共是二十八员。为首两个最
了得:一个是歙州僧人,名号宝光如来,俗姓邓,法名元觉,使一条禅杖,乃是浑
铁打就的,可重五十余斤,人皆称为国师;又一个,乃是福州人氏,姓石名宝,惯
使一个流星锤,百发百中,又能使一口宝刀,名为劈风刀,可以裁铜截铁,遮莫三
层铠甲,如劈风一般过去。外有二十六员,都是遴选之将,亦皆悍勇。主帅切不可
轻敌。”

宋江听罢,赏了段恺,便教去张招讨军前说知备细。后来段恺就跟了张招讨行军,
守把苏州,却委副都督刘光世来秀州守御,宋先锋却移兵在欈李亭下寨。当与诸将
筵宴赏军,商议调兵攻取杭州之策。只见小旋风柴进起身道:“柴某自蒙兄长高唐
州救命已来,一向累蒙仁兄顾爱,坐享荣华,不曾报得恩义。今愿深入方腊贼巢,
去做细作,或得一阵功勋,报效朝廷,也与兄长有光。未知尊意肯容否?”宋江大
喜道:“若得大官人肯去直入贼巢,知得里面溪山曲折,可以进兵,生擒贼首方腊,
解上京师,方表微功,同享富贵。只恐贤弟路程劳苦,去不得。”柴进道:“情愿
舍死一往,只是得燕青为伴同行最好。此人晓得诸路乡谈,更兼见机而作。”宋江
道:“贤弟之言,无不依允。只是燕青拨在卢先锋部下,便可行文取来。”正商议
未了,闻人报道:“卢先锋特使燕青到来报捷。”宋江见报,大喜说道:“贤弟此
行,必成大功矣!恰限燕青到来,也是吉兆。”柴进也喜。

燕青到寨中,上帐拜罢宋江,吃了酒食。问道:“贤弟水路来?旱路来?”燕青答
道:“乘传到此。”宋江又问道:“戴宗回时,说道已进兵攻取湖州,其事如何?”
燕青禀道:“自离宣州,卢先锋分兵两处:先锋自引一半军马攻打湖州,杀死伪留
守弓温并手下副将五员,收伏了湖州,杀散了贼兵,安抚了百姓,一面行文申复张
招讨,拨统制守御,特令燕青来报捷。主将所分这一半人马,叫林冲引领前去,攻
取独松关,都到杭州聚会。小弟来时,听得说独松关路上每日厮杀,取不得关,先
锋又同朱武去了,嘱付委呼延将军统领军兵守住湖州,待中军招讨调拨得统制到来,
护境安民,才一面进兵,攻取德清县,到杭州会合。”宋江又问道:“湖州守御取
德清,并调去独松关厮杀,两处分的人将,你且说与我姓名,共是几人去,并几人
跟呼延灼来。”燕青道:“有单在此。分去独松关厮杀取关,现有正偏将佐二十三
员:
先锋卢俊义

朱武

林冲

董平

张清
解珍

解宝

吕方

郭盛

欧鹏
邓飞

李忠

周通

邹渊

邹润
孙新

顾大嫂

李立

白胜

汤隆
朱贵

朱富

时迁
现在湖州守御,即日进兵德清县,现有正偏将佐一十九员:
呼延灼

索超

穆弘

雷横

杨雄
刘唐

单廷圭

魏定国

陈达

杨春
薛永

杜迁

穆春

李云

石勇
龚旺

丁得孙

张青

孙二娘
这两处将佐,通计四十二员。小弟来时,那里商议定了,目下进兵。”
宋江道:“既然如此,两路进兵攻取最好。却才柴大官人要和你去方腊贼巢里面去
做细作,你敢去么?”燕青道:“主帅差遣,安敢不从?小弟愿陪侍柴大官人去。”
柴进甚喜,便道:“我扮做个白衣秀才,你扮做个仆者。一主一仆,背着琴剑书箱
上路去,无人疑忌。直去海边寻船,使过越州,却取小路去诸暨县,就那里穿过山
路,取睦州不远了。”商议已定,择一日,柴进、燕青辞了宋先锋,收拾琴剑书箱,
自投海边,寻船过去,不在话下。

且说军师吴用再与宋江道:“杭州南半边,有钱塘大江,通达海岛。若得几个人驾
小船从海边去进赭山门,到南门外江边放起号炮,竖立号旗,城中必慌。你水军中
头领,谁人去走一遭?”说犹未了,张横、三阮道:“我们都去。”宋江道:“杭
州西路,又靠着湖泊,亦要水军用度,你等不可都去。”吴用道:“只可叫张横同
阮小七,驾船将引侯健、段景住去。”当时拨了四个人,引着三十余个水手,将带
了十数个火炮号旗,自来海边寻船,望钱塘江里进发。

看官听说,这回话都是散沙一般。先人书会留传,一个个都要说到,只是难做一时
说;慢慢敷演关目,下来便见。看官只牢记关目头行,便知衷曲奥妙。

再说宋江分调兵将已了,回到秀州,计议进兵,攻取杭州,忽听得东京有使命赍捧
御酒赏赐到州。宋江引大小将校迎接入城,谢恩已罢,作御酒供宴,管待天使。饮
酒中间,天使又将出太医院奏准,为上皇乍感小疾,索取神医安道全回京,驾前委
用,降下圣旨,就令来取。宋江不敢阻当。次日,管待天使已了,就行起送安道全
赴京。宋江等送出十里长亭饯行,安道全自同天使回京。有诗赞曰:
安子青囊艺最精,山东行散有声名。
人夸脉得仓公妙,自负丹如蓟子成。
刮骨立看金镞出,解肌时见刃痕平。
梁山结义坚如石,此别难忘手足情。
再说宋江把颁降到赏赐,分俵众将,择日祭旗起军,辞别刘都督、耿参谋,上马进
兵,水陆并行,船骑同发。路至崇德县,守将闻知,奔回杭州去了。
且说方腊太子方天定聚集诸将,在行宫议事。今时龙翔宫基址,乃是旧日行宫。方
天定手下有四员大将。那四员:
宝光如来国师

邓元觉

南离大将军元帅

石宝
镇国大将军

厉天闰

护国大将军

司行方
这四个皆称元帅大将军名号,是方腊加封。又有二十四员偏将。那二十四员:
厉天佑

吴值

赵毅

黄爱

晁中
汤逢士

王彧

薛斗南

冷恭

张俭
元兴

姚义

温克让

茅迪

王仁
崔逦

廉明

徐白

张道原

凤仪
张韬

苏泾

米泉

贝应夔
这二十四个,皆封为将军。共是二十八员,在方天定行宫,聚集计议。方天定说道:
“即日宋江水陆并进过江南来,平折了与他三个大郡。止有杭州,是南国之屏障。
若有亏失,睦州焉能保守?前者司天太监浦文英奏,是‘罡星侵入吴地,就里为祸
不小’,正是这伙人了。今来犯吾境界,汝等诸官,各受重爵,务必赤心报国,休
得怠慢。”众将启奏方天定道:“主上宽心!放着许多精兵良将,未曾与宋江对敌。
目今虽是折陷了数处州郡,皆是不得其人,以致如此。今闻宋江、卢俊义分兵三路,
来取杭州,殿下与国师谨守宁海军城郭,作万年基业。臣等众将,各各分调迎敌。”
太子方天定大喜,传下令旨,也分三路军马,前去策应,只留国师邓元觉同保城池。
分去那三员元帅?乃是:

护国元帅司行方,引四员首将,救应德清:
薛斗南

黄爱

徐白

米泉

镇国元帅厉天闰,引四员首将,救应独松关:
厉天佑

张俭

张韬

姚义

南离元帅石宝,引八员首将,总军出郭迎敌大队人马:
温克让

赵毅

冷恭

王仁
张道原

吴值

廉明

凤仪
三员大将,分调三路,各引军三万。分拨人马已定,各赐金帛,催促起身。元帅司
行方,引了一枝军马,救应德清州,望余杭州进发。

且不说两路军马策应去了。却说这宋先锋大队军兵,迤逦前进,来至临平山,望见
山顶一面红旗,在那里磨动。宋江当下差正将二员——花荣、秦明,先来哨路,随
即催趱战船车过长安坝来。花荣、秦明两个带领了一千军马,转过山嘴,早迎着南
军石宝军马。手下两员首将当先,望见花荣、秦明,一齐出马。一个是王仁,一个
是凤仪,各挺一条长枪,便奔将来。宋军中花荣、秦明,便把军马摆开出战。秦明
手舞狼牙大棍,直取凤仪;花荣挺枪来战王仁。四马相交,斗过十合,不分胜败。
秦明、花荣观见南军后有接应,都喝一声:“少歇!”各回马还阵。花荣道:“且
休恋战,快去报哥哥来,别作商议。”后军随即飞报去中军。

宋江引朱仝、徐宁、黄信、孙立四将,直到阵前。南军王仁、凤仪,再出马交锋,
大骂:“败将敢再出来交战!”秦明大怒,舞起狼牙棍,纵马而出,和凤仪再战。
王仁却搦花荣出战。只见徐宁一骑马,便挺枪杀去。花荣与徐宁是一副一正——金
枪手、银枪手,花荣随即也纵马,便出在徐宁背后,拈弓取箭在手,不等徐宁、王
仁交手,觑得较亲,只一箭,把王仁射下马去,南军尽皆失色。凤仪见王仁被箭射
下马来,吃了一惊,措手不及,被秦明当头一棍打着,攧下马去,南兵漫散奔走。
宋军冲杀过去,石宝抵当不住,退回皋亭山来,直近东新桥下寨。当日天晚,策立
不定,南兵且退入城去。

次日,宋先锋军马已过了皋亭山,直抵东新桥下寨,传令教分调本部军兵,作三路
夹攻杭州。那三路军兵将佐是谁?
一路分拨步军头领正偏将,从汤镇路去取东门,是:
朱仝

史进

鲁智深

武松

王英

扈三娘

一路分拨水军头领正偏将,从北新桥取古塘,截西路,打靠湖城门:
李俊

张顺

阮小二

阮小五

孟康
中路马步水三军,分作三队进发,取北关门、艮山门。前队正偏将是:
关胜

花荣

秦明

徐宁

郝思文

凌振
第二队总兵主将宋先锋、军师吴用,部领人马正偏将是:
戴宗

李逵

石秀

黄信

孙立

樊瑞
鲍旭

项充

李衮

马麟

裴宣

蒋敬
燕顺

宋清

蔡福

蔡庆

郁保四
第三队水路陆路助战策应。正偏将是:
李应

孔明

杜兴

杨林

童威

童猛
当日宋江分拨大小三军已定,各自进发。
有话即长,无话即短。且说中路大队军兵,前队关胜直哨到东新桥,不见一个南军。
关胜心疑,退回桥外,使人回复宋先锋。宋江听了,使戴宗传令,分付道:“且未
可轻进。每日轮两个头领出哨。”头一日,是花荣、秦明,第二日徐宁、郝思文,
一连哨了数日,又不见出战。此日又该徐宁、郝思文,两个带了数十骑马,直哨到
北关门来,见城门大开着,两个来到吊桥边看时,城上一声擂鼓响,城里早撞出一
彪军马来。徐宁、郝思文急回马时,城西偏路喊声又起,一百余骑马军,冲在前面。
徐宁并力死战,杀出马军队里,回头不见了郝思文。再回来看时,见数员将校,把
郝思文活捉了入城去。徐宁急待回身,项上早中了一箭,带着箭飞马走时,六将背
后赶来。路上正逢着关胜,救得回来,血晕倒了。六员南将,已被关胜杀退,自回
城里去了。慌忙报与宋先锋知道。

宋江急来看徐宁时,七窍流血。宋江垂泪,便唤随军医士治疗,拔去箭矢,用金枪
药敷贴。宋江且教扶下战船内将息,自来看视。当夜三四次发昏,方知中了药箭。
宋江仰天叹道:“神医安道全已被取回京师,此间又无良医可救,必损吾股肱也!”
伤感不已。吴用来请宋江回寨,主议军情,勿以兄弟之情,误了国家重事。宋江使
人送徐宁到秀州去养病,不想箭中药毒,调治不痊。且说宋江又差人去军中打听郝
思文消息,次日,只见小军来报道:“杭州北关门城上,把竹竿挑起郝思文头来示
众。”方知道被方天定碎剐了,宋江见报,好生伤感。后半月徐宁已死,申文来报。
宋江因折了二将,按兵不动,且守住大路。

却说李俊等引兵到北新桥住扎,分军直到古塘深山去处探路,听得飞报道:“折了
郝思文,徐宁中箭而死。”李俊与张顺商议道:“寻思我等这条路道,第一要紧是
去独松关,湖州、德清二处冲要路口,抑且贼兵都在这里出没。我们若当住他咽喉
道路,被他两面来夹攻,我等兵少,难以迎敌。不若一发杀入西山深处,却好屯扎。
西湖水面,好做我们战场;山西后面通接西溪,却又好做退步。”便使小校报知先
锋,请取军令。次后引兵直过桃源岭西山深处,在今时灵隐寺屯驻;山北面西溪山
口,亦扎小寨,在今时古塘深处。前军却来唐家瓦出哨。

当日张顺对李俊说道:“南兵都已收入杭州城里去了,我们在此屯兵,今经半月之
久,不见出战,只在山里,几时能够获功?小弟今欲从湖里没水过去,从水门中暗
入城去,放火为号,哥哥便可进兵取他水门。就报与主将先锋,教三路一齐打城。”
李俊道:“此计虽好,恐兄弟独力难成。”张顺道:“便把这命报答先锋哥哥许多
年好情分,也不多了。”李俊道:“兄弟且慢去,待我先报与哥哥,整点人马策应。”
张顺道:“我这里一面行事,哥哥一面使人去报。比及兄弟到得城里,先锋哥哥已
自知了。”

当晚,张顺身边藏了一把蓼叶尖刀,饱吃了一顿酒食,来到西湖岸边,看见那三面
青山,一湖绿水,远望城郭,四座禁门,临着湖岸。那四座门:钱塘门、涌金门、
清波门、钱湖门。看官听说:原来这杭州旧宋以前,唤做清河镇。钱王手里,改为
杭州宁海军,设立十座城门:东有菜市门、荐桥门;南有候潮门、嘉会门;西有钱
湖门、清波门、涌金门、钱塘门;北有北关门、艮山门。高宗车驾南渡之后,建都
于此,唤做花花临安府,又添了三座城门。目今方腊占据时,还是钱王旧都,城子
方圆八十里,虽不比南渡以后,安排得十分的富贵,从来江山秀丽,人物奢华,所
以相传道:“上有天堂,下有苏杭。”怎见得:
江浙昔时都会,钱塘自古繁华。休言城内风光,且说西湖景物:有一万顷碧澄澄掩
映琉璃,列三千面青娜娜参差翡翠。春风湖上,艳桃浓李如描;夏日池中,绿盖红
莲似画。秋云涵如,看南国嫩菊堆金;冬雪纷飞,观北岭寒梅破玉。九里松青烟细
细,六桥水碧响泠泠。晓霞连映三天竺,暮云深锁二高峰。风生在猿呼洞口,雨飞
来龙井山头。三贤堂畔,一条鳌背侵天;四圣观前,百丈祥云缭绕。苏公堤东坡古
迹,孤山路和靖旧居。访友客投灵隐去,簪花人逐净慈来。平昔只闻三岛
远,岂知湖北胜蓬莱?
苏东坡学士有诗赞道:
湖光潋滟晴偏好,山色空蒙雨亦奇。
若把西湖比西子,淡妆浓抹也相宜。
又有古词名《浣溪沙》为证:

湖上朱桥响画轮,溶溶春水浸春云,碧琉璃滑净无尘。

当路游丝迎醉客,
入花黄鸟唤行人,日斜归去奈何春!
这西湖,故宋时果是景致无比,说之不尽。张顺来到西陵桥上,看了半晌。时当春
暖,西湖水色拖蓝,四面山光叠翠。张顺看了道:“我身生在浔阳江上,大风巨浪,
经了万千,何曾见这一湖好水,便死在这里,也做个快活鬼!”说罢,脱下布衫,
放在桥下,头上挽着个穿心红的儿,下面着腰生绢水裙,系一条膊,挂一口尖
刀,赤着脚,钻下湖里去,却从水底下摸将过湖来。

此时已是初更天气,月色微明,张顺摸近涌金门边,探起头来,在水面上听时,城
上更鼓,却打一更四点。城外静悄悄地,没一个人;城上女墙边,有四五个人在那
里探望。张顺再伏在水里去了。又等半回,再探起头来看时,女墙边悄不见一个人。
张顺摸到水口边看时,一带都是铁窗棂隔着;摸里面时,都是水帘护定,帘子上有
绳索,索上缚着一串铜铃。张顺见窗棂牢固,不能够入城,舒只手入去,扯那水帘
时,牵得索子上铃响,城上人早发起喊来。张顺从水底下,再钻入湖里伏了。听得
城上人马下来,看那水帘时,又不见有人,都在城上说道:“铃子响得跷蹊,莫不
是个大鱼顺水游来,撞动了水帘。”众军汉看了一回,并不见一物,又各自去睡了。
张顺再听时,城楼上已打三更,打了好一回更点,想必军人各自去东倒西歪睡熟了。
张顺再钻向城边去,料是水里入不得城。爬上岸来看时,那城上不见一个人在上面,
便欲要爬上城去,且又寻思道:“倘或城上有人,却不干折了性命,我且试探一试
探。”摸些土块,掷撒上城去。有不曾睡的军士,叫将起来,再下来看水门时,又
没动静。再上城来敌楼上看湖面上时,又没一只船只。原来西湖上船只,已奉方天
定令旨,都收入清波门外和净慈港内,别门俱不许泊船。众人道:“却是作怪?”
口里说道:“定是个鬼!我们各自睡去,休要睬他!”口里虽说,却不去睡,尽伏
在女墙边。

张顺又听了一个更次,不见些动静,却钻到城边来听,上面更鼓不响。张顺不敢便
上去,又把些土石抛掷上城去,又没动静。张顺寻思道:“已是四更,将及天亮,
不上城去,更待几时?”却才爬到半城,只听得上面一声梆子响,众军一齐起。张
顺从半城上跳下水池里去,待要趁水没时,城上踏弩硬弓、苦竹箭、鹅卵石,一齐
都射打下来。可怜张顺英雄,就涌金门外水池中身死。诗曰:
曾闻善战死兵戎,善溺终然丧水中。
瓦罐不离井上破,劝君莫但逞英雄。
话分两头,却说宋江日间已接了李俊飞报,说张顺没水入城,放火为号,便转报与
东门军士去了。当夜宋江在帐中和吴用议事,到四更,觉道神思困倦,退了左右,
在帐中伏几而卧。猛然一阵冷风,宋江起身看时,只见灯烛无光,寒气逼人。定睛
看时,见一个似人非人,似鬼非鬼,立于冷气之中。看那人时,浑身血污着,低低
道:“小弟跟随哥哥许多年,恩爱至厚。今以杀身报答,死于涌金门下枪箭之中,
今特来辞别哥哥。”宋江道:“这个不是张顺兄弟?”回过脸来这边,又见三四个,
都是鲜血满身,看不仔细。宋江大哭一声,蓦然觉来,乃是南柯一梦。帐外左右听
得哭声,入来看时,宋江道:“怪哉!”叫请军师圆梦。

吴用道:“兄长却才困倦暂时,有何异梦?”宋江道:“适间冷气过处,分明见张
顺一身血污,立在此间,告道:‘小弟跟着哥哥许多年,蒙恩至厚。今以杀身报答,
死于涌金门下枪箭之中,特来辞别。’转过脸来,这面又立着三四个带血的人,看
不分晓,就哭觉来。”吴用道:“早间李俊报说,张顺要过湖里去,越城放火为号,
莫不只是兄长记心,却得这恶梦?”宋江道:“只想张顺是个精灵的人,必然死于
无辜。”吴用道:“西湖到城边,必是险隘,想端的送了性命。张顺魂来,与兄长
托梦。”宋江道:“若如此时,这三四个又是甚人?”和吴学究议论不定,坐而待
旦,绝不见城中动静,心中越疑。看看午后,只见李俊使人飞报将来说:“张顺去
涌金门越城,被箭射死于水中,现今西湖城上把竹竿挑起头来,挂着号令。”
宋江见报了,又哭的昏倒;吴用等众将亦皆伤感。原来张顺为人甚好,深得弟兄情
分。宋江道:“我丧了父母,也不如此伤悼,不由我连心透骨苦痛!”吴用及众将
劝道:“哥哥以国家大事为念,休为弟兄之情,自伤贵体。”宋江道:“我必须亲
自到湖边,与他吊孝。”吴用谏道:“兄长不可亲临险地,若贼兵知得,必来攻击。”
宋江道:“我自有计较。”随即点李逵、鲍旭、项充、李衮四个,引五百步军去探
路,宋江随后带了石秀、戴宗、樊瑞、马麟,引五百军士,暗暗地从西山小路里去
李俊寨里。李俊等接着,请到灵隐寺中方丈内歇下。宋江又哭了一场,便请本寺僧
人,就寺里诵经,追荐张顺。

次日天晚,宋江叫小军去湖边扬一首白,上写道:“亡弟正将张顺之魂。”插于
水边。西陵桥上,排下许多祭物,却分付李逵道:“如此如此。”埋伏在北山路口;
樊瑞、马麟、石秀左右埋伏;戴宗随在身边。只等天色相近一更时分,宋江挂了白
袍,金盔上盖着一层孝绢,同戴宗并五七个僧人,却从小行山转到西陵桥上。军校
已都列下黑猪白羊,金银祭物,点起灯烛荧煌,焚起香来。宋江在当中证盟,朝着
涌金门下哭奠,戴宗立在侧边。先是僧人摇铃诵咒,摄招呼名,祝赞张顺魂魄,降
坠神。次后戴宗宣读祭文,宋江亲自把酒浇奠,仰天望东而哭。

正哭之间,只听得桥下两边一声喊起,南北两山一齐鼓响,两彪军马来拿宋江。正
是:只因恩义如天大,惹起兵戈卷地来。
毕竟宋江、戴宗怎地迎敌,且听下回分解。