\chapter{宋公明夜度益津关~吴学究智取文安县}

话说当下欧阳侍郎奏道:“宋江这伙,都是梁山泊英雄好汉。如今宋朝童子皇
帝,被蔡京、童贯、高俅、杨四个贼臣弄权,嫉贤妒能,闭塞贤路,非亲不进,
非财不用,久后如何容的他们!论臣愚意,郎主可加官爵,重赐金帛,多赏轻裘肥
马。臣愿为使臣,说他来降俺大辽国。郎主若得这伙军马来,觑中原如同反掌。臣
不敢自专,乞郎主圣鉴不错。”郎主听罢,便道:“你也说的是。你就为使臣,将
带一百八骑好马,一百八匹好缎子,俺的敕命一道,封宋江为镇国大将军,总领辽
兵大元帅,赐与金一提,银一秤,权当信物。教把众头目的姓名都抄将来,尽数封
他官爵。”

只见班部中兀颜都统军出来启奏郎主道:“宋江这一伙草贼,招安他做甚?放着奴
婢手下有二十八宿将军、十一曜大将,有的是强兵猛将,怕不赢他?若是这伙蛮子
不退呵,奴婢亲自引兵去剿杀这厮。”国主道:“你便是了的好汉,如插翅大虫,
再添的这伙呵,你又加生两翅。你且休得阻当。”辽主不听兀颜之言,再有谁敢多
言。原来这兀颜光都统军,正是辽国第一员上将,十八般武艺,无有不通,兵书战
策,尽皆熟闲。年方三十五六,堂堂一表,凛凛一躯,八尺有余身材,面白唇红,
须黄眼碧,威仪猛勇。上阵时,仗条浑铁点钢枪,杀到浓处,不时掣出腰间铁简,
使的铮铮有声,端的是有万夫不当之勇。

且不说兀颜统军谏奏,却说那欧阳侍郎领了辽国敕旨,将了许多礼物马匹,上了马,
径投蓟州来。宋江正在蓟州作养军士,听的辽国有使命至,未审来意吉凶,遂取玄
女之课,当下一卜,卜得个上上之兆。便与吴用商议道:“封中上上之兆,多是辽
国来招安我们,似此如之奈何?”吴用道:“若是如此时,正可将计就计,受了他
招安。将此蓟州与卢先锋管了,却取他霸州。若更得了他霸州,不愁他辽国不破。
即今取了他檀州,先去辽国一只左手。此事容易,只是放些先难后易,令他不疑。”
且说那欧阳侍郎已到城下,宋江传令,教开城门,放他进来。欧阳侍郎入到城中,
至州衙前下马,直到厅上。叙礼罢,分宾主而坐。宋江便问:“侍郎来意何干?”
欧阳侍郎道:“有件小事,上达钧听,乞屏左右。”宋江遂将左右喝退,请进后堂
深处说话。

欧阳侍郎至后堂,欠身与宋江道:“俺大辽国久闻将军大名,争奈山遥水远,无由
拜见威颜。又闻将军在梁山大寨,替天行道,众弟兄同心协力。今日宋朝奸臣们闭
塞贤路,有金帛投于门下者,便得高官重用;无贿赂投于门下者,总有大功于国,
空被沉埋,不得升赏。如此奸党弄权,谗佞侥幸,嫉贤妒能,赏罚不明,以致天下
大乱。江南、两浙、山东、河北,盗贼并起,草寇猖狂,良民受其涂炭,不得聊生。
今将军统十万精兵,赤心归顺,止得先锋之职,又无升受品爵。众弟兄劬劳报国,
俱各白身之士。遂命引兵直抵沙漠,受此劳苦,与国建功,朝廷又无恩赐。此皆奸
臣之计。若沿途掳掠金珠宝贝,令人馈送浸润与蔡京、童贯、高俅、杨四个贼臣,
可保官爵,恩命立至。若还不肯如此行事,将军纵使赤心报国,建大功勋,回到朝
廷,反坐罪犯。欧某今奉大辽国主特遣小官赍敕命一道,封将军为辽邦镇国大将军,
总领兵马大元帅。赠金一提,银一秤,彩缎一百八匹,名马一百八骑。便要抄录一
百八位头领姓名赴国,照名钦授官爵。非来诱说将军,此是国主久闻将军盛德,特
遣欧某前来预请将军众将,同意协心,辅助本国。”

宋江听罢,便答道:“侍郎言之极是。争奈宋江出身微贱,郓城小吏,犯罪在逃,
权居梁山水泊,避难逃灾。宋天子三番降诏,赦罪招安,虽然官小职微,亦未曾立
得功绩,以报朝廷赦罪之恩。今蒙郎主赐我以厚爵,赠之以重赏,然虽如此,未敢
拜受,请侍郎且回。即今溽暑炎热,权令军马停歇,暂且借国王这两个城子屯兵,
守待早晚秋凉,再作商议。”欧阳侍郎道:“将军不弃,权且受下辽王金帛、彩缎、
鞍马。俺回去,慢慢地再来说话,未为晚矣。”宋江道:“侍郎不知我等一百八人,
耳目最多,倘或走透消息,先惹其祸。”欧阳侍郎道:“兵权执掌,尽在将军手内,
谁敢不从?”宋江道:“侍郎不知就里。我等弟兄中间,多有性直刚勇之士。等我
调和端正,众所同心,却慢慢地回话,亦未为迟。”有诗为证:
金帛重驮出蓟州,熏风回首不胜羞。
辽王若问归降事,云在青山月在楼。
于是令备酒肴相待,送欧阳侍郎出城上马去了。

宋江却请军师吴用商议道:“适来辽国侍郎这一席话如何?”吴用听了,长叹一声,
低首不语,肚里沉吟。宋江便问道:“军师何故叹气?”吴用答道:“我寻思起来,
只是兄长以忠义为主,小弟不敢多言。我想欧阳侍郎所说这一席话,端的是有理。
目今宋朝天子,至圣至明,果被蔡京、童贯、高俅、杨四个奸臣专权,主上听信。
设使日后纵有成功,必无升赏。我等三番招安,兄长为尊,只得个先锋虚职。若论
我小子愚意,弃宋从辽,岂不为胜,只是负了兄长忠义之心。”宋江听罢,便道:
“军师差矣!若从辽国,此事切不可提。纵使宋朝负我,我忠心不负宋朝。久后纵
无功赏,也得青史上留名。若背正顺逆,天不容恕!吾辈当尽忠报国,死而后已!”
吴用道:“若是兄长存忠义于心,只就这条计上,可以取他霸州。目今盛暑炎天,
且当暂停,将养军马。”宋江、吴用计议已定,且不与众人说。同众将屯驻蓟州,
待过暑热。

次日,与公孙胜在中军闲话,宋江问道:“久闻先生师父罗真人,乃盛世之高士。
前番因打高唐州,要破高廉邪法,特地使戴宗、李逵来寻足下,说:‘尊师罗真人,
术法灵验。’敢烦贤弟,来日引宋江去法座前,焚香参拜,一洗尘俗。未知尊意如
何?”公孙胜便道:“贫道亦欲归望老母,参省本师。为见兄长连日屯兵未定,不
敢开言。今日正欲要禀仁兄,不想兄长要去。来日清晨,同往参礼本师,贫道就行
省视老母。”

次日,宋江暂委军师掌管军马。收拾了名香净果,金珠彩缎,将带花荣、戴宗、吕
方、郭盛、燕顺、马麟六个头领。宋江与公孙胜共八骑马,带领五千步卒,取路投
九宫县二仙山来。宋江等在马上,离了蓟州,来到山峰深处。但见青松满径,凉气
,炎暑全无,端的好座佳丽之山。公孙胜在马上道:“有名唤做呼鱼鼻山。”
宋江看那山时,但见:
四围,八面玲珑。重重晓色映晴霞,沥沥琴声飞瀑布。溪涧中漱玉飞琼,石壁
上堆蓝叠翠。白云洞口,紫藤高挂绿萝垂;碧玉峰前,丹桂悬崖青蔓袅。引子苍猿
献果,呼群麋鹿衔花。千峰竞秀,夜深白鹤听仙经;万壑争流,风暖幽禽相对语。
地僻红尘飞不到,山深车马几曾来。

当下公孙胜同宋江直至紫虚观前,众人下马,整顿衣巾。小校托着信香礼物,径到
观里鹤轩前面。观里道众,见了公孙胜,俱各向前施礼,同来见宋江,亦施礼罢。
公孙胜便问:“吾师何在?”道众道:“师父近日只在后面退居静坐,少曾到观。”
公孙胜听了,便和宋公明径投后山退居内来。转进观后,崎岖径路,曲折阶衢。行
不到一里之间,但见荆棘为篱,外面都是青松翠柏,篱内尽是瑶草琪花。中有三间
雪洞,罗真人在内端坐诵经。童子知有客来,开门相接。公孙胜先进草庵鹤轩前,
礼拜本师已毕,便禀道:“弟子旧友,山东宋公明,受了招安,今奉敕命,封先锋
之职,统兵来破辽虏,今到蓟州,特地要来参礼我师,现在此间。”罗真人见说,
便教请进。

宋江进得草庵,罗真人降阶迎接。宋江再三恳请罗真人,坐受拜礼。罗真人道:“将
军国家上将,贫道乃山野村夫,何敢当此?”宋江坚意谦让,要礼拜他。罗真人方
才肯坐。宋江先取信香炉中焚,参礼了八拜,便呼花荣等六个头领,俱各礼拜已
了。罗真人都教请坐,命童子烹茶献果已罢。罗真人乃曰:“将军上应星魁,外合
列曜,一同替天行道,今则归顺宋朝,此清名万载不磨矣!”宋江道:“江乃郓城
小吏,逃罪上山,感谢四方豪杰,望风而来。同声相应,同气相求,恩如骨肉,情
若股肱。天垂景象,方知上应天星地曜,会合一处。今奉诏命,统领大兵,征进辽
国,径涉仙境,夙生有缘,得一瞻拜。万望真人指迷前程之事,不胜万幸。”罗真
人道:“蒙将军不弃,折节下问。出家人违俗已久,心如死灰,无可效忠,幸勿督
过。”宋江再拜求教。罗真人道:“将军少坐,当具素斋。天色已晚,就此荒山草
榻,权宿一宵,来早回马。未知尊意若何?”宋江便道:“宋江正欲我师指教,点
悟愚迷,安忍便去?”随即唤从人托过金珠彩缎,上献罗真人。罗真人乃曰:“贫
道僻居野叟,寄形宇内,纵使受此金珠,亦无用处。随身自有布袍遮体,绫锦彩缎,
亦不曾穿。将军统数万之师,军前赏赐,日费浩繁,所赐之物,乞请纳回。”宋江
再拜,望请收纳。罗真人坚执不受,当即供献素斋,斋罢,又吃了茶。罗真人令公
孙胜回家省母,明早却来,随将军回城。当晚留宋江庵中闲话。宋江把心腹之事,
备细告知罗真人,愿求指迷。罗真人道:“将军一点忠义之心,与天地均同,神明
必相护佑。他日生当封侯,死当庙食,决无疑虑。只是将军一生命薄,不得全美。”
宋江告道:“我师,莫非宋江此身不得善终?”罗真人道:“非也!将军亡必正寝,
死必归坟。只是所生命薄,为人好处多磨,忧中少乐。得意浓时,便当退步,切勿
久恋富贵。”宋江再告:“我师,富贵非宋江之意,但愿弟兄常常完聚,虽居贫贱,
亦满微心。只求大家安乐。”罗真人笑道:“大限到来,岂容汝等留恋乎?”宋江
再拜,求罗真人法语。罗真人命童子取过纸笔,写下八句法语,度与宋江。那八句
说道是:
忠心者少,义气者稀。幽燕功毕,明月虚辉。
始逢冬暮,鸿雁分飞。吴头楚尾,官禄同归。

宋江看毕,不晓其意,再拜恳告:“乞我师金口剖决,指引迷愚。”罗真人道:“此
乃天机,不可泄漏。他日应时,将军自知。夜深更静,请将军观内暂宿一宵,来日
再会。贫道当年寝寐,未曾还的,再欲赴梦去也。将军勿罪!”宋江收了八句法语,
藏在身边,辞了罗真人,来观内宿歇。众道众接至方丈,宿了一宵。

次日清晨,来参真人,其时公孙胜已到草庵里了。罗真人叫备素馔斋饭相待。早馔
已毕,罗真人再与宋江道:“将军在上,贫道一言可禀。这个徒弟公孙胜,本从贫
道山中出家,远绝尘俗,正当其理。奈缘是一会下星辰,不由他不来。今俗缘日短,
道行日长。若今日便留下,在此伏侍贫道,却不见了弟兄往日情分。从今日跟将军
去干大功,如奏凯还京,此时相辞,却望将军还放。一者使贫道有传道之人,二乃
免他老母倚门之望。将军忠义之士,必举忠义之行。未知将军雅意肯纳贫道否?”
宋江道:“师父法旨,弟子安敢不听?况公孙胜先生与江弟兄,去住从他,焉敢阻
当?”罗真人同公孙胜都打个稽首道:“谢承将军金诺。”当下众人拜辞罗真人,
罗真人直送宋江等出庵相别。罗真人道:“将军善加保重,早得建节封侯。”宋江
拜别,出到观前。所有乘坐马匹,在观中喂养,从人已牵在观外俟候。众道士送宋
江等出到观外相别。宋江教牵马至半山平坦之处,与公孙胜等一同上马,再回蓟州。
一路无话,早到城中州衙前下马。黑旋风李逵接着说道:“哥哥去望罗真人,怎生
不带兄弟去走一遭?”戴宗道:“罗真人说,你要杀他,好生怪你。”李逵道:“他
也奈何的我也够了!”众人都笑。宋江入进衙内,众人都到后堂。宋江取出罗真人
那八句法语,递与吴用看详,不晓其意。众人反复看了,亦不省的。公孙胜道:“兄
长,此乃天机玄语,不可泄漏。收取过了,终身受用,休得只顾猜疑。师父法语,
过后方知。”宋江遂从其说,藏于天书之内。

自此之后,屯驻军马,在蓟州一月有余,并无军情之事。至七月半后,檀州赵枢密
行文书到来,说奉朝廷敕旨,催兵出战。宋江接得枢密院付,便与军师吴用计议,
前到玉田县,合会卢俊义等,操练军马,整顿军器,分拨人员已定,再回蓟州,祭
祀旗纛,选日出师。闻左右报道:“辽国有使来到。”宋江出接,却是欧阳侍郎,
便请入后堂。叙礼已罢,宋江问道:“侍郎来意如何?”欧阳侍郎道:“乞退左右。”
宋江随即喝散军士。侍郎乃言:“俺大辽国主好生慕公之德。若蒙将军慨然归顺,
肯助大辽,必当建节封侯。全望早成大义,免俺国主悬望之心。”宋江答道:“这
里也无外人,亦当尽忠告诉侍郎。不知前番足下来时,众军皆知其意。内中有一半
人,不肯归顺。若是宋江便随侍郎出幽州,朝见郎主时,有副先锋卢俊义,必然引
兵追赶。若就那里城下厮并,不见了我弟兄们日前的义气。我今先带些心腹之人,
不拣那座城子,借我躲避。他若引兵赶来,知我下落,那时却好回避他。他若不听,
却和他厮并,也未迟。他若不知我等下落时,他军马回报东京,必然别生支节。我
等那时朝见郎主,引领大辽军马,却来与他厮杀,未为晚矣!”

欧阳侍郎听了宋江这一席言语,心中甚喜,便回道:“俺这里紧靠霸州,有两个隘
口:一个唤做益津关,两边都是险峻高山,中间只一条驿路;一个是文安县,两面
都是恶山,过的关口,便是县治。这两座去处,是霸州两扇大门。将军若是如此,
可往霸州躲避。本州是俺辽国国舅康里定安守把。将军可就那里,与国舅同住,却
看这里如何。”宋江道:“若得如此,宋江星夜使人回家,搬取老父,以绝根本。
侍郎可暗地使人来引宋江去。只如此说,今夜我等收拾也。”欧阳侍郎大喜,别了
宋江,上马去了。有诗为证:
国士从胡志可伤,常山骂贼姓名香。
宋江若肯降辽国,何似梁山作大王。

当日宋江令人去请卢俊义、吴用、朱武到蓟州,一同计较智取霸州之策,下来便见。
宋江酌量已定,卢俊义领令去了。吴用、朱武暗暗分付众将,如此如此而行。宋江
带去人数,林冲、花荣、朱仝、刘唐、穆弘、李逵、樊瑞、鲍旭、项充、李衮、吕
方、郭盛、孔明、孔亮,共计一十五员头领,止带一万来军校。拨定人数,只等欧
阳侍郎来到便行。望了两日,只见欧阳侍郎飞马而来,对宋江道:“俺郎主知道将
军实是好心的人,既蒙归顺,怕他宋兵做甚么?俺大辽国,有的是好兵好将,强人
壮马相助。你既然要取令大人,不放心时,且请在霸州与国舅作伴,俺却差人去取
未迟。”宋江听了,与侍郎道:“愿去的军将,收拾已完备,几时可行?”欧阳侍
郎道:“则今夜便行,请将军传令。”宋江随即分付下去,都教马摘銮铃,军卒衔
枚疾走,当晚便行。一面管待来使。黄昏左侧,开城西门便出,欧阳侍郎引数十骑
在前领路。宋江引一支军马,随后便行。约行过二十余里,只见宋江在马上猛然失
声,叫声:“苦也!”说道:“约下军师吴学究同来归顺大辽,不想来的慌速,不
曾等的他来。军马慢行,却快使人取接他来。”当时已是三更左侧,前面已是益津
关隘口。欧阳侍郎大喝一声:“开门!”当下把关的军将,开放关口,军马人将,
尽数度关,直到霸州。天色将晓,欧阳侍郎请宋江入城,报知国舅康里定安。
原来这国舅,是大辽郎主皇后亲兄,为人最有权势,更兼胆勇过人。将着两员侍郎,
守住霸州:一个唤做金福侍郎,一个唤做叶清侍郎。听的报道宋江来降,便叫军马
且在城外下寨,只教为头的宋先锋请进城来。欧阳侍郎便同宋江入城,来见定安国
舅。国舅见了宋江,一表非俗,便乃降阶而接,请至后堂,叙礼罢,请在上坐。宋
江答道:“国舅乃金枝玉叶,小将是投降之人,怎消受国舅殊礼重待?宋江将何报
答?”定安国舅道:“多听得将军的名传寰海,威镇中原,声名闻于大辽。俺的国
主,好生慕爱。”宋江道:“小将比领国舅的福荫,宋江当尽心报答郎主大恩。”
定安国舅大喜,忙叫安排庆贺筵宴。一面又叫椎牛宰马,赏劳三军。城中选了一所
宅子,教宋江、花荣等安歇,方才教军马尽数入城屯扎。花荣等众将,都来见了国
舅等众人。番将同宋江一处安歇已了,宋江便请欧阳侍郎分付道:“可烦侍郎差人
报与把关的军汉,怕有军师吴用来时,分付便可教他进关来,我和他一处安歇。昨
夜来得仓卒,不曾等候得他。我一时与足下只顾先来了,正忘了他。军情主事,少
他不得。更兼军师文武足备,智谋并优,六韬三略,无有不会。”欧阳侍郎听了,
随即便传下言语,差人去与益津关、文安县二处把关军将说知:“但有一个秀才模
样的人,姓吴名用,便可放他过来。”

且说文安县得了欧阳侍郎的言语,便差人转出益津关上,报知就里,说与备细。上
关来望时,只见尘头蔽日,土雾遮天,有军马奔上关来。把关将士准备擂木炮石,
安排对敌,只见山前一骑马上,坐着一人,秀才模样,背后一个行脚僧,一个行者,
随后又有数十个百姓,都赶上关来。马到关前,高声大叫:“我是宋江手下军师吴
用,欲待来寻兄长,被宋兵追赶得紧,你可开关救我!”把关将道:“想来正是此
人。”随即开关,放入吴学究来。只见那两个行脚僧人、行者,也挨入关。关上人
当住,那行者早撞在门里了。和尚便道:“俺两个出家人,被军马赶的紧,救咱们
则个!”把关的军,定要推出关去。那和尚发作,行者焦躁,大叫道:“俺不是出
家人,俺是杀人的太岁鲁智深、武松的便是!”花和尚抡起铁禅杖,拦头便打。武
行者掣出双戒刀,就便杀人,正如砍瓜切菜一般。那数十个百姓,便是解珍、解宝、
李立、李云、杨林、石勇、时迁、段景住、白胜、郁保四这伙人,早奔关里,一发
夺了关口。卢俊义引着军兵,都赶到关上,一齐杀入文安县来。把关的官员,那里
迎敌的住。这伙都到文安县取齐。

却说吴用飞马奔到霸州城下,守门的番官报入城来。宋江与欧阳侍郎在城边相接,
便教引见国舅康里定安。吴用说道:“吴用不合来的迟了些个。正出城来,不想卢
俊义知觉,直赶将来,追到关前。小生今入城来,此时不知如何。”又见流星探马
报来说道:“宋兵夺了文安县,军马杀近霸州。”定安国舅便教点兵,出城迎敌,
宋江道:“未可调兵,等他到城下,宋江自用好言招抚他。如若不从,却和他厮并
未迟。”只见探马又报将来说:“宋兵离城不远!”定安国舅与宋江一齐上城看望。
见宋兵整整齐齐,都摆列在城下。卢俊义顶盔挂甲,跃马横枪,点军调将,耀武扬
威,立马在门旗之下,高声大叫道:“只教反朝廷的宋江出来!”宋江立在城楼下
女墙边,指着卢俊义说道:“兄弟,所有宋朝赏罚不明,奸臣当道,谗佞专权,我
已顺了大辽国主。汝可同心,也来帮助我,同扶大辽郎主,不失了梁山许多时相聚
之意。”卢俊义大骂道:“俺在北京安家乐业,你来赚我上山。宋天子三番降诏,
招安我们,有何亏负你处?你怎敢反背朝廷?你那短见无能之人,早出来打话,见个
胜败输赢!”宋江大怒,喝教开城门,便差林冲、花荣、朱仝、穆弘四将齐出,活
拿这厮。卢俊义一见了四将,约住军校,跃马横枪,直取四将,全无惧怯。林冲等
四将斗了二十余合,拨回马头,望城中便走。卢俊义把枪一招,后面大队军马,一
齐赶杀入来。林冲、花荣占住吊桥,回身再杀,诈败佯输,诱引卢俊义抢入城中。
背后三军,齐声呐喊,城中宋江等诸将,一齐兵变,接应入城,四方混杀,人人束
手,个个归心。定安国舅气的目睁口呆,罔知所措,与众等侍郎束手被擒。

宋江引军到城中,诸将都至州衙内来,参见宋江。宋江传令,先请上定安国舅并欧
阳侍郎、金福侍郎、叶清侍郎,并皆分坐,以礼相待。宋江道:“汝辽国不知就里,
看的俺们差矣!我这伙好汉,非比啸聚山林之辈。一个个乃是列宿之臣,岂肯背主
降辽?只要取汝霸州,特地乘此机会。今已成功,国舅等请回本国,切勿忧疑,俺
无杀害之心。但是汝等部下之人,并各家老小,俱各还本国。霸州城子,已属天朝,
汝等勿得再来争执。今后刀兵到处,无有再容。”宋江号令已了,将城中应有番官,
尽数驱遣起身,随从定安国舅都回幽州。宋江一面出榜安民,令副先锋卢俊义将引
一半军马,回守蓟州,宋江等一半军将,守住霸州。差人赍奉军帖,飞报赵枢密,
得了霸州。赵安抚听了大喜,一面写表申奏朝廷。

且说定安国舅与同三个侍郎,带领众人归到燕京,来见郎主,备细奏说宋江诈降一
事,因此被那伙蛮子占了霸州。辽主听了大怒,喝骂欧阳侍郎:“都是你这奴婢佞
臣,往来搬斗,折了俺的霸州紧要的城池,教俺燕京如何保守?快与我拿去斩了!”
班部中转出兀颜统军,启奏道:“郎主勿忧,量这厮何须国主费力。奴婢自有个道
理,且免斩欧阳侍郎。若是宋江知得,反被他耻笑。”辽主准奏,赦了欧阳侍郎。
兀颜统军奏道:“奴婢引起部下二十八宿将军,十一曜大将,前去布下阵势,把这
些蛮子,一鼓儿平收。”说言未绝,班部中却转出贺统军前来奏道:“郎主不用忧
心,奴婢自有个见识。常言道:‘杀鸡焉用牛刀。’那里消得正统军自去,只贺某
聊施小计,教这一伙蛮子,死无葬身之地!”郎主听了,大喜道:“俺的爱卿,愿
闻你的妙策。”贺统军启口摇舌,说这妙计,有分教,卢俊义来到一个去处,马无
料草,人绝口粮。直教:三军骁勇齐消魄,一代英雄也皱眉。
毕竟贺统军道出甚计来,且听下回分解。