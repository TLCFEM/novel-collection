\chapter{李逵打死殷天锡~柴进失陷高唐州}

话说当下朱仝对众人说道:“若要我上山时,你只杀了黑旋风,与我出了这口
气,我便罢。”李逵听了大怒道:“教你咬我鸟!晁、宋二位哥哥将令,干我屁事!”
朱仝怒发,又要和李逵厮并,三个又劝住了。朱仝道:“若有黑旋风时,我死也不
上山去!”柴进道:“恁地也却容易,我自有个道理,只留下李大哥在我这里便了。
你们三个自上山去,以满晁、宋二公之意。”朱仝道:“如今做下这件事了,知府
必然行移文书,去郓城县追捉,拿我家小,如之奈何?”吴学究道:“足下放心,
此时多敢宋公明已都取宝眷在山上了。”朱仝方才有些放心。柴进置酒相待,就当
日送行。三个临晚辞了柴大官人便行。柴进叫庄客备三骑马送出关外,临别时,吴
用又分付李逵道:“你且小心,只在大官人庄上住几时,切不可胡乱惹事累人。待
半年三个月,等他性定,却来取你还山,多管也来请柴大官人入伙。”三个自上马
去了。

不说柴进和李逵回庄,且只说朱仝随吴用、雷横来梁山泊入伙,行了一程,出
离沧州地界,庄客自骑了马回去。三个取路投梁山泊来,于路无话,早到朱贵酒店
里,先使人上山寨报知。晁盖、宋江引了大小头目,打鼓吹笛,直到金沙滩迎接,
一行人都相见了。各人乘马回到山上大寨前下了马,都到聚义厅上,叙说旧话。朱
仝道:“小弟今蒙呼唤到山,沧州知府必然行移文书去郓城县捉我老小,如之奈何?”
宋江大笑道:“我教长兄放心,尊嫂并令郎已取到这里多日了。”朱仝又问道:“现
在何处?”宋江道:“奉养在家父太公歇处,兄长请自己去问慰便了。”朱仝大喜。
宋江着人引朱仝直到宋太公歇所,见了一家老小,并一应细软行李,妻子说道:“近
日有人赍书来,说你已在山寨入伙了,因此收拾星夜到此。”朱仝出来拜谢了众人。
宋江便请朱仝、雷横山顶下寨,一面且做筵席,连日庆贺新头领,不在话下。

却说沧州知府至晚不见朱仝抱小衙内回来,差人四散去寻了半夜,次日有人见
杀死在林子里,报与知府知道。府尹听了大怒,亲自到林子里看了,痛哭不已,备
办棺木烧化。次日升厅,便行移公文,诸处缉捕捉拿朱仝正身。郓城县已自申报朱
仝妻子挈家在逃,不知去向,行开各州县出给赏钱捕获,不在话下。

只说李逵在柴进庄上住了一个来月,忽一日,见一个人赍一封书火急奔庄上来,
柴大官人却好迎着,接书看了,大惊道:“既是如此,我只得去走一遭。”李逵便
问道:“大官人有甚紧事?”柴进道:“我有个叔叔柴皇城,现在高唐州居住,今
被本州知府高廉的老婆兄弟殷天锡那厮,来要占花园,怄了一口气,卧病在床,早
晚性命不保,必有遗嘱的言语分付,特来唤我。想叔叔无儿无女,必须亲身去走一
遭。”李逵道:“既是大官人去时,我也跟大官人去走一遭如何?”柴进道:“大
哥肯去时,就同走一遭。”柴进即便收拾行李,选了十数匹好马,带了几个庄客。
次日五更起来,柴进、李逵并从人,都上了马,离了庄院,望高唐州来。

不一日,来到高唐州,入城直至柴皇城宅前下马,留李逵和从人在外面厅房内。
柴进自径入卧房里来看视那叔叔柴皇城时,但见:

面如金纸,体似枯柴。悠悠无七魄三魂,细细只一丝两气。牙关紧急,连朝水
米不沾唇;心膈膨胀,尽日药丸难下肚。丧门吊客已随身,扁鹊卢医难下手。
柴进看了柴皇城,自坐在叔叔榻前,放声恸哭。皇城的继室出来劝柴进道:“大官
人鞍马风尘不易,初到此间,且休烦恼。”柴进施礼罢,便问事情。继室答道:“此
间新任知府高廉,兼管本州兵马,是东京高太尉的叔伯兄弟,倚仗他哥哥势,要在
这里无所不为。带将一个妻舅殷天锡来,人尽称他做殷直阁。那厮年纪却小,又倚
仗他姐夫高廉的权势,在此间横行害人。有那等献勤的卖科,对他说我家宅后有个
花园水亭,盖造得好。那厮带将许多奸诈不及的三二十人,径入家里来宅子后看了,
便要发遣我们出去,他要来住。皇城对他说道:‘我家是金枝玉叶,有先朝丹书铁
券在门,诸人不许欺侮。你如何敢夺占我的住宅,赶我老小那里去?’那厮不容所
言,定要我们出屋。皇城去扯他,反被这厮推抢殴打,因此受这口气,一卧不起,
饮食不吃,服药无效,眼见得上天远,入地近。今日得大官人来家做个主张,便有
些山高水低,也更不忧。”柴进答道:“尊婶放心,只顾请好医士调治叔叔,但有
门户,小侄自使人回沧州家里,去取丹书铁券来,和他理会。便告到官府今上御前,
也不怕他!”继室道:“皇城干事,全不济事,还是大官人理论是得。”

柴进看视了叔叔一回,却出来和李逵并带来人从说知备细。李逵听了,跳将起
来说道:“这厮好无道理!我有大斧在这里,教他吃我几斧,却再商量。”柴进道:
“李大哥,你且息怒,没来由,和他粗卤做甚么?他虽是倚势欺人,我家放着有护
持圣旨,这里和他理论不得,须是京师也有大似他的,放着明明的条例,和他打官
司。”李逵道:“条例,条例,若还依得,天下不乱了!我只是前打后商量。那厮
若还去告,和那鸟官一发都砍了!”柴进笑道:“可知朱仝要和你厮并,见面不得。
这里是禁城之内,如何比得你小寨里横行?”李逵道:“禁城便怎地?江州无为军
偏我不曾杀人?”柴进道:“等我看了头势,用着大哥时,那时相央,无事只在房
里请坐。”正说之间,里面侍妾慌忙来请大官人看视皇城。

柴进入到里面卧榻前,只见皇城阁着两眼泪,对柴进说道:“贤侄志气轩昂,
不辱祖宗。我今日被殷天锡怄死,你可看骨肉之面,亲赍书往京师拦驾告状,与我
报仇,九泉之下,也感贤侄亲意。保重!保重!再不多嘱!”言罢,便放了命。柴进
痛哭了一场。继室恐怕昏晕,劝住柴进道:“大官人烦恼有日,且请商量后事。”
柴进道:“誓书在我家里,不曾带得来,星夜教人去取,须用将往东京告状。叔叔
尊灵,且安排棺椁盛殓,成了孝服,却再商量。”柴进教依官制,备办内棺外椁,
依礼铺设灵位,一门穿了重孝,大小举哀。李逵在外面听得堂里哭泣,自己磨拳擦
掌价气,问从人都不肯说。宅里请僧修设好事功果。

至第三日,只见这殷天锡骑着一匹撺行的马,将引闲汉三二十人,手执弹弓、
川弩、吹筒、气球、拈竿、乐器,城外游玩了一遭,带五七分酒,佯醉假颠,径来
到柴皇城宅前,勒住马,叫里面管家的人出来说话。柴进听得说,挂着一身孝服,
慌忙出来答应。那殷天锡在马上问道:“你是他家甚么人?”柴进答道:“小可是
柴皇城亲侄柴进。”殷天锡道:“前日我分付道,教他家搬出屋去,如何不依我言
语?”柴进道:“便是叔叔卧病,不敢移动,夜来已自身故,待断七了搬出去。”
殷天锡道:“放屁!我只限你三日便要出屋,三日外不搬,先把你这厮枷号起,先
吃我一百讯棍!”柴进道:“直阁休恁相欺!我家也是龙子龙孙,放着先朝丹书铁
券,谁敢不敬?”殷天锡喝道:“你将出来我看!”柴进道:“现在沧州家里,已
使人去取来。”殷天锡大怒道:“这厮正是胡说!便有誓书铁券,我也不怕,左右
与我打这厮!”

众人却待动手,原来黑旋风李逵在门缝里都看见,听得喝打柴进,便拽开房门,
大吼一声,直抢到马边,早把殷天锡揪下马来,一拳打翻。那二三十人却待抢他,
被李逵手起,早打倒五六个,一哄都走了。李逵拿殷天锡提起来,拳头脚尖一发上,
柴进那里劝得住。看那殷天锡时,呜呼哀哉,伏惟尚飨。有诗为证:
惨刻侵谋倚横豪,岂知天理竟难逃。
李逵猛恶无人敌,不见阎罗不肯饶。

李逵将殷天锡打死在地,柴进只叫得苦,便教李逵且去后堂商议。柴进道:“眼
见得便有人到这里,你安身不得了。官司我自支吾,你快走回梁山泊去。”李逵道:
“我便走了,须连累你。”柴进道:“我自有誓书铁券护身,你便去是,事不宜迟。”
李逵取了双斧,带了盘缠,出后门,自投梁山泊去了。

不多时,只见二百余人各执刀杖枪棒,围住柴皇城家。柴进见来捉人,便出来
说道:“我同你们府里分诉去。”众人先缚了柴进,便入家里搜捉行凶黑大汉不见,
只把柴进绑到州衙内,当厅跪下。知府高廉听得打死了他的舅子殷天锡,正在厅上
咬牙切齿忿恨,只待拿人来。早把柴进驱翻在厅前阶下,高廉喝道:“你怎敢打死
了我殷天锡?”柴进告道:“小人是柴世宗嫡派子孙,家门有先朝太祖誓书铁券,
现在沧州居住。为是叔叔柴皇城病重,特来看视,不幸身故,现今停丧在家。殷直
阁将带三二十人到家,定要赶逐出屋,不容柴进分说,喝令众人殴打,被庄客李大
救护,一时行凶打死。”高廉喝道:“李大现在那里?”柴进道:“心慌逃走了。”
高廉道:“他是个庄客,不得你的言语,如何敢打死人!你又故纵他逃走了,却来
瞒昧官府。你这厮,不打如何肯招?牢子下手,加力与我打这厮!”柴进叫道:“庄
客李大救主,误打死人,非干我事!放着先朝太祖誓书,如何便下刑法打我?”高
廉道:“誓书有在那里?”柴进道:“已使人回沧州去取来也。”高廉大怒,喝道:
“这厮正是抗拒官府,左右腕头加力,好生痛打!”众人下手,把柴进打得皮开肉
绽,鲜血迸流,只得招做使令庄客李大打死殷天锡,取面二十五斤死囚枷钉了,发
下牢里监收。殷天锡尸首检验了,自把棺木殡葬,不在话下。这殷夫人要与兄弟报
仇,教丈夫高廉抄扎了柴皇城家私,监禁下人口,占住了房屋围院,柴进自在牢中
受苦。有诗为证:
脂唇粉面毒如蛇,铁券金书空里花。
可怪祖宗能让位,子孙犹不保身家。

却说李逵连夜回梁山泊,到得寨里,来见众头领。朱仝一见李逵,怒从心起,
掣条朴刀,径奔李逵。黑旋风拔出双斧,便斗朱仝。晁盖、宋江,并众头领,一齐
向前劝住。宋江与朱仝陪话道:“前者杀了小衙内,不干李逵之事。却是军师吴学
究因请兄长不肯上山,一时定的计策。今日既到山寨,便休记心,只顾同心协助,
共兴大义,休教外人耻笑。”便叫李逵兄弟与朱仝陪话。李逵睁着怪眼,叫将起来,
说道:“他直恁般做得起!我也多曾在山寨出气力,他又不曾有半点之功,却怎地
倒教我陪话!”宋江道:“兄弟,却是你杀了小衙内,虽是军师严令,论齿序他也
是你哥哥,且看我面,与他伏个礼,我却自拜你便了。”李逵吃宋江央及不过,便
道:“我不是怕你,为是哥哥逼我,没奈何了,与你陪话。”李逵吃宋江逼住了,
只得撇了双斧,拜了朱仝两拜,朱仝方才消了这口气。山寨里晁头领且教安排筵席,
与他两个和解。

李逵说起:“柴大官人因去高唐州看亲叔叔柴皇城病症,却被本州高知府妻舅
殷天锡,要夺屋宇花园,殴骂柴进,吃我打死了殷天锡那厮。”宋江听罢,失惊道:
“你自走了,须连累柴大官人吃官司。”吴学究道:“兄长休惊,等戴宗回山,便
有分晓。”李逵问道:“戴宗哥哥那里去了?”吴用道:“我怕你在柴大官人庄上
惹事不好,特地教他来唤你回山。他到那里,不见你时,必去高唐州寻你。”说言
未绝,只见小校来报戴院长回来了。宋江便去迎接,到了堂上坐下,便问柴大官人
一事。戴宗答道:“去到柴大官人庄上,已知同李逵投高唐州去了。径奔那里去打
听,只见满城人传道殷天锡因争柴皇城庄屋,被一个黑大汉打死了,现今负累了柴
大官人陷于缧绁,下在牢里。柴皇城一家人口家私,尽都抄扎了。柴大官人性命,
早晚不保。”晁盖道:“这个黑厮又做出来了,但到处便惹口面。”李逵道:“柴
皇城被他打伤,怄气死了,又来占他房屋,又喝教打柴大官人,便是活佛,也忍不
得!”晁盖道:“柴大官人自来与山寨有恩,今日他有危难,如何不下山去救他?
我亲自去走一遭。”宋江道:“哥哥是山寨之主,如何可便轻动?小可和柴大官人
旧来有恩,情愿替哥哥下山。”吴学究道:“高唐州城池虽小,人物稠穰,军广粮
多,不可轻敌。烦请林冲、花荣、秦明、李俊、吕方、郭盛、孙立、欧鹏、杨林、
邓飞、马麟、白胜,十二个头领,部引马步军兵五千,作前队先锋;军中主帅宋公
明、吴用,并朱仝、雷横、戴宗、李逵、张顺、杨雄、石秀,十个头领,部引马步
军兵三千策应。”共该二十二位头领,辞了晁盖等众人,离了山寨,望高唐州进发。
端的好整齐,但见:

绣旗飘号带,画角间铜锣。三股叉,五股叉,灿灿秋霜;点钢枪,芦叶枪,纷
纷瑞雪。蛮牌遮路,强弓硬弩当先;火炮随车,大戟长戈拥后。鞍上将似南山猛虎,
人人好斗能争;坐下马如北海苍龙,骑骑能冲敢战。端的枪刀流水急,果然人马撮
风行。

梁山泊前军已到高唐州地界,早有军卒报知高廉。高廉听了,冷笑道:“你这
伙草贼,在梁山泊窝藏,我兀自要来剿捕你,今日你倒来就缚,此是天教我成功。
左右,快传下号令,整点军马出城迎敌,着那众百姓上城守护。”这高知府上马管
军,下马管民,一声号令下去,那帐前都统、监军、统领、统制、提辖军职一应官
员,各各部领军马,就教场里点视已罢,诸将便摆布出城迎敌。高廉手下有三百体
己军士,号为飞天神兵,一个个都是山东、河北、江西、湖南、两淮、两浙选来的
精壮好汉。那三百飞天神兵怎生结束,但见:

头披乱发,脑后撒一把烟云;身挂葫芦,背上藏千条火焰。黄抹额齐分八卦,
豹皮甲尽按四方。熟铜面具似金装,镔铁滚刀如扫帚。掩心铠甲,前后竖两面青铜;
照眼旌旗,左右列千
层黑雾。疑是天蓬离斗府,正如月孛下云衢。

那知府高廉亲自引了三百神兵,披甲背剑,上马出到城外,把部下军官周回排
成阵势,却将三百神兵列在中军,摇旗呐喊,擂鼓鸣金,只等敌军到来。却说林冲、
花荣、秦明引领五千人马到来。两军相迎,旗鼓相望,各把强弓硬弩射住阵脚。两
军中吹动画角,发起擂鼓。花荣、秦明,带同十个头领,都到阵前,把马勒住。头
领林冲横丈八蛇矛,跃马出阵,厉声高叫:“高唐州纳命的出来!”高廉把马一纵,
引着三十余个军官,都出到门旗下,勒住马,指着林冲骂道:“你这伙不知死的叛
贼,怎敢直犯俺的城池?”林冲喝道:“你这个害民强盗,我早晚杀到京师,把你
那厮欺君贼臣高俅,碎尸万段,方是愿足。”高廉大怒,回头问道:“谁人出马先
捉此贼去?”军官队里转出一个统制官,姓于,名直,拍马抡刀,竟出阵前。林冲
见了,径奔于直,两个战不到五合,于直被林冲心窝里一蛇矛刺着,翻筋斗颠下马
去。高廉见了大惊,“再有谁人出马报仇?”军官队里又转出一个统制官,姓温,
双名文宝,使一条长枪,骑一匹黄骠马,銮铃响,珂鸣,早出到阵前,四只马蹄
荡起征尘,直奔林冲。秦明见了,大叫:“哥哥稍歇,看我立斩此贼。”林冲勒住
马,收了点钢矛,让秦明战温文宝。两个约斗十合之上,秦明放个门户,让他枪搠
入来,手起棍落,把温文宝削去半个天灵盖,死于马上,那马跑回本阵去了。两阵
军相对,齐呐声喊。

高廉见连折二将,便去背上掣出那口太阿宝剑来,口中念念有词,喝声道:“疾!”
只见高廉队中卷起一道黑气。那道气散至半空里,飞沙走石,撼地摇天,刮起怪风,
径扫过对阵来。林冲、秦明、花荣等众将,对面不能相顾,惊得那坐下马乱窜咆哮,
众人回身便走。高廉把剑一挥,指点那三百神兵,从阵里杀将出来,背后官军协助,
一掩过来,赶得林冲等军马星落云散,七断八续,呼兄唤弟,觅子寻爷,五千军兵
折了一千余人,直退回五十里下寨。高廉见人马退去,也收了本部军兵,入高唐州
城里安下。

却说宋江中军人马到来,林冲等接着,且说前事。宋江、吴用听了大惊,与军
师道:“是何神术,如此利害?”吴学究道:“想是妖法,若能回风返火,便可破
敌。”宋江听罢,打开天书看时,第三卷上有回风返火破阵之法。宋江大喜,用心
记了咒语并秘诀,整点人马,五更造饭吃了,摇旗擂鼓,杀进城下来。

有人报入城中,高廉再点了得胜人马,并三百神兵,开放城门,布下吊桥,出
来摆成阵势。宋江带剑纵马出阵前,望见高廉军中一簇皂旗,吴学究道:“那阵内
皂旗,便是使神师计的军兵。但恐又使此法,如何迎敌?”宋江道:“军师放心,
我自有破阵之法。诸军众将勿得惊疑,只顾向前杀去。”高廉分付大小将校:“不
要与他强敌挑斗,但见牌响,一齐并力擒获宋江,我自有重赏。”两军喊声起处,
高廉马鞍鞒上挂着那面聚兽铜牌,上有龙章凤篆,手里拿着宝剑,出阵前。宋江指
着高廉骂道:“昨夜我不曾到,兄弟们误折一阵,今日我必要把你诛尽杀绝。”高
廉喝道:“你这伙反贼,快早早下马受缚,省得我腥手污脚!”言罢,把剑一挥,
口中念念有词,喝声道:“疾!”黑气起处,早卷起怪风来。宋江不等那风到,口
中也念念有词,左手捏诀,右手提剑一指,说声道:“疾!”那阵风不望宋江阵里
来,倒望高廉神兵队里去了。宋江却待招呼人马杀将过去,高廉见回了风,急取铜
牌,把剑敲动,向那神兵队里卷一阵黄沙,就中军走出一群猛兽。但见:

狻猊舞爪,狮子摇头。闪金獬豸逞威雄,奋锦貔貅施勇猛。豺狼作对吐獠牙,
直奔雄兵;虎豹成群张巨口,来喷劣马。带刺野猪冲阵入,卷毛恶犬撞人来。如龙
大蟒扑天飞,吞象顽蛇钻地落。

高廉铜牌响处,一群怪兽毒虫直冲过来,宋江阵里众多人马惊呆了。宋江撇了
剑,拨回马先走,众头领簇捧着,尽都逃命,大小军校,你我不能相顾,夺路而走。
高廉在后面把剑一挥,神兵在前,官军在后,一齐掩杀将来。宋江人马,大败亏输。
高廉赶杀二十余里,鸣金收军,城中去了。宋江来到土坡下,收住人马,扎下寨栅,
虽是损折了些军卒,却喜众头领都有。屯住军马,便与军师吴用商议道:“今番打
高唐州,连折了两阵,无计可破神兵,如之奈何?”吴学究道:“若是这厮会使神
师计,他必然今夜要来劫寨,可先用计提备,此处只可屯扎些少军马,我等去旧寨
内驻扎。”宋江传令,只留下杨林、白胜看寨,其余人马,退去旧寨内将息。

且说杨林、白胜引人离寨半里草坡内埋伏,等到一更时分。但见:

云生四野,雾涨八方。摇天撼地起狂风,倒海翻江飞急雨。雷公忿怒,倒骑火
兽逞神威;电母生嗔,乱掣金蛇施圣力。大树和根拔去,深波彻底卷干。若非灌口
斩蛟龙,疑是泗州降水母。
当夜风雷大作,杨林、白胜引着三百余人伏在草里看时,只见高廉步走,引领三百
神兵,吹风唿哨,杀入寨里来,见是空寨,回身便走。杨林、白胜呐声喊,高廉只
怕中了计,四散便走,三百神兵各自奔逃。杨林、白胜乱放弩箭,只顾射去,一箭
正中高廉左肩,众军四散,冒雨赶杀。高廉引领了神兵去得远了,杨林、白胜人少,
不敢深入。少刻,雨过云收,复见一天星斗,月光之下,草坡前搠翻射死拿得神兵
二十余人,解赴宋公明寨内。具说雷雨风云之事。宋江、吴用见说,大惊道:“此
间只隔得五里远近,却又无雨无风!”众人议道:“正是妖法只在本处,离地只有
三四十丈,云雨气味,是左近水泊中摄将来的。”杨林说:“高廉也自披发仗剑,
杀入寨中,身上中了我一弩箭,回城中去了。为是人少,不敢去追。”宋江分赏杨
林、白胜,把拿来的中伤神兵斩了,分拨众头领下了七八个小寨,围绕大寨,提备
再来劫寨,一面使人回山寨,取军马协助。

且说高廉自中了箭,回到城中养病,令军士守护城池,晓夜提备,“且休与他
厮杀,待我箭疮平复起来,捉宋江未迟。”

却说宋江见折了人马,心中忧闷,和军师吴用商量道:“只这回高廉尚且破不
得,倘或别添他处军马,并力来劫,如之奈何?”吴学究道:“我想要破高廉妖法,
只除非依我如此如此。若不去请这个人来,柴大官人性命,也是难救。高唐州城子,
永不能得。”正是:要除起雾兴云法,须请通天彻地人。

毕竟吴学究说这个人是谁,且听下回分解。