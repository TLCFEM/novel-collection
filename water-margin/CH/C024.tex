\chapter{王婆贪贿说风情~郓哥不忿闹茶肆}

话说当日武都头回转身来,看见那人,扑翻身便拜。那人原来不是别人,正是
武松的嫡亲哥哥武大郎。武松拜罢,说道:“一年有余不见哥哥,如何却在这里?”
武大道:“二哥,你去了许多时,如何不寄封书来与我?我又怨你,又想你。”武
松道:“哥哥如何是怨我,想我?”武大道:“我怨你时,当初你在清河县里,要
便吃酒醉了,和人相打,时常吃官司,教我要便随衙听候,不曾有一个月净办,常
教我受苦:这个便是怨你处。想你时,我近来取得一个老小,清河县人,不怯气都
来相欺负,没人做主;你在家时,谁敢来放个屁?我如今在那里安不得身,只得搬
来这里赁房居住:因此便是想你处。”

看官听说:原来武大与武松,是一母所生两个。武松身长八尺,一貌堂堂,浑
身上下,有千百斤气力,不恁地,如何打得那个猛虎?这武大郎,身不满五尺,面
目丑陋,头脑可笑。清河县人见他生得短矮,起他一个诨名,叫做三寸丁谷树皮。

那清河县里有一个大户人家,有个使女,小名唤做潘金莲,年方二十余岁,颇
有些颜色,因为那个大户要缠他,这女使只是去告主人婆,意下不肯依从。那个大
户以此记恨于心,却倒赔些房奁,不要武大一文钱,白白地嫁与他。自从武大娶得
那妇人之后,清河县里有几个奸诈的浮浪子弟们,却来他家里恼。原来这妇人,
见武大身材短矮,人物猥,不会风流。这婆娘倒诸般好,为头的爱偷汉子。有诗
为证:
金莲容貌更堪题,笑蹙春山八字眉。
若遇风流清子弟,等闲云雨便偷期。

却说那潘金莲过门之后,武大是个懦弱依本分的人,被这一班人不时间在门前
叫道:“好一块羊肉,倒落在狗口里!”因此武大在清河县住不牢,搬来这阳谷县
紫石街赁房居住,每日仍旧挑卖炊饼。

此日正在县前做买卖,当下见了武松,武大道:“兄弟,我前日在街上听得人
沸沸地说道:‘景阳冈上一个打虎的壮士,姓武,县里知县参他做个都头。’我也
八分猜道是你,原来今日才得撞见。我且不做买卖,一同和你家去。”武松道:“哥
哥家在那里?”武大用手指道:“只在前面紫石街便是。”武松替武大挑了担儿,
武大引着武松,转弯抹角,一径望紫石街来。

转过两个弯,来到一个茶坊间壁,武大叫一声:“大嫂开门。”只见芦帘起处,
一个妇人出到帘子下应道:“大哥,怎地半早便归?”武大道:“你的叔叔在这里,
且来厮见。”武大郎接了担儿入去,便出来道:“二哥,入屋里来,和你嫂嫂相见。”
武松揭起帘子,入进里面,与那妇人相见。武大说道:“大嫂,原来景阳冈上打死
大虫新充做都头的,正是我这兄弟。”那妇人叉手向前道:“叔叔万福。”武松道:
“嫂嫂请坐。”武松当下推金山,倒玉柱,纳头便拜。那妇人向前扶住武松道:“叔
叔,折杀奴家。”武松道:“嫂嫂受礼。”那妇人道:“奴家也听得说道:‘有个
打虎的好汉,迎到县前来。’奴家也正待要去看一看。不想去得迟了,赶不上,不
曾看见,原来却是叔叔。且请叔叔到楼上去坐。”武松看那妇人时,但见:

眉似初春柳叶,常含着雨恨云愁;脸如三月桃花,暗藏着风情月意。纤腰袅娜,
拘束的燕懒莺慵;檀口轻盈,勾引得蜂狂蝶乱。玉貌妖娆花解语,芳容窈窕玉生香。

当下那妇人叫武大请武松上楼,主客席里坐地。三个人同到楼上坐了,那妇人
看着武大道:“我陪侍着叔叔坐地,你去安排些酒食来,管待叔叔。”武大应道:
“最好。二哥,你且坐一坐,我便来也。”武大下楼去了。那妇人在楼上,看了武
松这表人物,自心里寻思道:“武松与他是嫡亲一母兄弟,他又生的这般长大。我
嫁得这等一个,也不枉了为人一世!你看我那三寸丁谷树皮,三分像人,七分似鬼,
我直恁地晦气!据着武松,大虫也吃他打倒了,他必然好气力。说他又未曾婚娶,
何不叫他搬来我家里住?不想这段因缘,却在这里!”

那妇人脸上堆下笑来问武松道:“叔叔,来这里几日了?”武松答道:“到此
间十数日了。”妇人道:“叔叔在那里安歇?”武松道:“胡乱权在县衙里安歇。”
那妇人道:“叔叔,恁地时,却不便当。”武松道:“独自一身,容易料理,早晚
自有土兵伏侍。”妇人道:“那等人伏侍叔叔,怎地顾管得到,何不搬来一家里住?
早晚要些汤水吃时,奴家亲自安排与叔叔吃,不强似这伙腌人。叔叔便吃口清汤,
也放心得下。”武松道:“深谢嫂嫂。”那妇人道:“莫不别处有婶婶,可取来厮
会也好。”武松道:“武二并不曾婚娶。”妇人又问道:“叔叔青春多少?”武松
道:“虚度二十五岁。”那妇人道:“长奴三岁。叔叔今番从那里来?”武松道:
“在沧州住了一年有余,只想哥哥在清河县住,不想却搬在这里。”那妇人道:“一
言难尽!自从嫁得你哥哥,吃他忒善了,被人欺负,清河县里住不得,搬来这里。
若得叔叔这般雄壮,谁敢道个不字!”武松道:“家兄从来本分,不似武二撒泼。”
那妇人笑道:“怎地这般颠倒说?常言道:‘人无刚骨,安身不牢。’奴家平生快
性,看不得这般三答不回头,四答和身转的人。”武松道:“家兄却不到得惹事,
要嫂嫂忧心。”

正在楼上说话未了,武大买了些酒肉果品归来,放在厨下,走上楼来叫道:“大
嫂,你下来安排。”那妇人应道:“你看那不晓事的,叔叔在这里坐地,却教我撇
了下来。”武松道:“嫂嫂请自便。”那妇人道:“何不去叫间壁王干娘安排便了?
只是这般不见便!”

武大自去央了间壁王婆,安排端正了,都搬上楼来,摆在桌子上,无非是些鱼
肉果菜之类,随即烫酒上来。武大叫妇人坐了主位,武松对席,武大打横。三个人
坐下,武大筛酒在各人面前。那妇人拿起酒来道:“叔叔休怪,没甚管待,请酒一
杯。”武松道:“感谢嫂嫂,休这般说。”武大只顾上下筛酒烫酒,那里来管别事。
那妇人笑容可掬,满口儿叫:“叔叔,怎地鱼和肉也不吃一块儿?”拣好的递将过
来。武松是个直性的汉子,只把做亲嫂嫂相待。谁知那妇人是个使女出身,惯会小
意儿。武大又是个善弱的人,那里会管待人。

那妇人吃了几杯酒,一双眼只看着武松的身上,武松吃他看不过,只低了头,
不恁么理会。当日吃了十数杯酒,武松便起身。武大道:“二哥,再吃几杯了去。”
武松道:“只好恁地,却又来望哥哥。”都送下楼来。那妇人道:“叔叔是必搬来
家里住。若是叔叔不搬来时,教我两口儿也吃别人笑话,亲兄弟难比别人。大哥,
你便打点一间房,请叔叔来家里过活,休教邻舍街坊道个不是。”武大道:“大嫂
说的是。二哥,你便搬来,也教我争口气。”武松道:“既是哥哥、嫂嫂恁地说时,
今晚有些行李,便取了来。”那妇人道:“叔叔是必记心,奴这里专望。”那妇人
情意十分殷勤,正是:
叔嫂通言礼禁严,手援须识是从权。
英雄只念连枝树,淫妇偏思并蒂莲。

武松别了哥嫂,离了紫石街,径投县里来,正值知县在厅上坐衙。武松上厅来
禀道:“武松有个亲兄,搬在紫石街居住。武松欲就家里宿歇,早晚衙门中听候使
唤。不敢擅去,请恩相钧旨。”知县道:“这是孝悌的勾当,我如何阻你?你可每
日来县里伺候。”武松谢了,收拾行李铺盖。有那新制的衣服,并前者赏赐的物件,
叫个土兵挑了,武松引到哥哥家里。那妇人见了,却比半夜里拾金宝的一般欢喜,
堆下笑来。武大叫个木匠,就楼上整了一间房,铺下一张床,里面放一条桌子,安
两个杌子,一个火炉。武松先把行李安顿了,分付土兵自回去,当晚就哥嫂家里歇
卧。

次日早起,那妇人慌忙起来,烧洗面汤,舀漱口水。叫武松洗漱了口面,裹了
巾帻,出门去县里画卯。那妇人道:“叔叔画了卯,早些个归来吃饭,休去别处吃。”
武松道:“便来也。”径去县里画了卯,伺候了一早晨,回到家里。那妇人洗手剔
甲,齐齐整整,安排下饭食,三口儿共桌儿吃。武松吃了饭,那妇人双手捧一盏茶,
递与武松吃。武松道:“教嫂嫂生受,武松寝食不安。县里拨一个土兵来使唤。”
那妇人连声叫道:“叔叔却怎地这般见外?自家的骨肉,又不伏侍了别人。便拨一
个土兵来使用,这厮上锅上灶地不干净,奴眼里也看不得这等人。”武松道:“恁
地时,却生受嫂嫂。”话休絮烦。自从武松搬将家里来,取些银子与武大,教买饼
馓茶果,请邻舍吃茶。众邻舍斗分子来与武松人情,武大又安排了回席,都不在话
下。

过了数日,武松取出一匹彩色缎子与嫂嫂做衣裳。那妇人笑嘻嘻道:“叔叔,
如何使得!既然叔叔把与奴家,不敢推辞,只得接了。”武松自此只在哥哥家里宿
歇。武大依前上街挑卖炊饼。武松每日自去县里画卯,承应差使。不论归迟归早,
那妇人顿羹顿饭,欢天喜地伏侍武松,武松倒过意不去。那妇人常把些言语来撩拨
他,武松是个硬心直汉,却不见怪。

有话即长,无话即短。不觉过了一月有余,看看是十一月天气。连日朔风紧起,
四下里彤云密布,又早纷纷扬扬,飞下一天大雪来。怎见得好雪,正是:
眼波飘瞥任风吹,柳絮沾泥若有私。
粉态轻狂迷世界,巫山云雨未为奇。
当日那雪,直下到一更天气,却似银铺世界,玉碾乾坤。次日,武松清早出去县里
画卯,直到日中未归。武大被这妇人赶出去做买卖,央及间壁王婆,买下些酒肉之
类,去武松房里簇了一盆炭火,心里自想道:“我今日着实撩斗他一撩斗,不信他
不动情。”那妇人独自一个,冷冷清清立在帘儿下等着,只见武松踏着那乱琼碎玉
归来。那妇人揭起帘子,陪着笑脸迎接道:“叔叔寒冷。”武松道:“感谢嫂嫂忧
念。”入得门来,便把毡笠儿除将下来。那妇人双手去接,武松道:“不劳嫂嫂生
受。”自把雪来拂了,挂在壁上;解了腰里缠袋,脱了身上鹦哥绿丝衲祆,入房
里搭了。那妇人便道:“奴等一早起,叔叔怎地不归来吃早饭?”武松道:“便是
县里一个相识,请吃早饭。却才又有一个作杯,我不奈烦,一直走到家来。”那妇
人道:“恁地,叔叔向火。”武松道:“好。”便脱了油靴,换了一双袜子,穿了
暖鞋,掇个杌子,自近火边坐地。

那妇人把前门上了拴,后门也关了,却搬些按酒、果品、菜蔬,入武松房里来,
摆在桌子上。武松问道:“哥哥那里去未归?”妇人道:“你哥哥每日自出去做买
卖,我和叔叔自饮三杯。”武松道:“一发等哥哥家来吃。”妇人道:“那里等的
他来?等他不得。”说犹未了,早暖了一注子酒来。武松道:“嫂嫂坐地,等武二
去烫酒正当。”妇人道:“叔叔,你自便。”那妇人也掇个杌子,近火边坐了。火
头边桌儿上,摆着杯盘。那妇人拿盏酒,擎在手里,看着武松道:“叔叔满饮此杯。”
武松接过手来,一饮而尽。那妇人又筛一杯酒来说道:“天色寒冷,叔叔饮个成双
杯儿。”武松道:“嫂嫂自便。”接来又一饮而尽。武松却筛一杯酒,递与那妇人
吃。妇人接过酒来吃了,却拿注子再斟酒来,放在武松面前。

那妇人脸上堆着笑容说道:“我听得一个闲人说道:叔叔在县前东街上,养着
一个唱的,敢端的有这话么?”武松道:“嫂嫂休听外人胡说,武二从来不是这等
人。”妇人道:“我不信,只怕叔叔口头不似心头。”武松道:“嫂嫂不信时,只
问哥哥。”那妇人道:“他晓的甚么!晓的这等事时,不卖炊饼了。叔叔且请一杯。”
连筛了三四杯酒饮了。那妇人也有三杯酒落肚,只管把闲话来说。武松也知了八九
分,自家只把头来低了。

那妇人起身去烫酒,武松自在房里拿起火箸簇火。那妇人暖了一注子酒来到房
里,一只手拿着注子,一只手便去武松肩胛上只一捏,说道:“叔叔,只穿这些衣
裳不冷?”武松已自有五分不快意,也不应他。那妇人见他不应,劈手便来夺火箸,
口里道:“叔叔,你不会簇火,我与你拨火,只要一似火盆常热便好。”武松有八
分焦燥,只不做声。那妇人不看武松焦燥,便放了火箸,却筛一盏酒来,自呷了一
口,剩了大半盏,看着武松道:“你若有心,吃我这半盏儿残酒。”

武松劈手夺来,泼在地下,说道:“嫂嫂休要恁地不识羞耻!”把手只一推,
争些儿把那妇人推一交。武松睁起眼来道:“武二是个顶天立地、噙齿戴发男子汉,
不是那等败坏风俗、没人伦的猪狗,嫂嫂休要这般不识廉耻,为此等的勾当。倘有
些风吹草动,武二眼里认的是嫂嫂,拳头却不认的是嫂嫂!再来休要恁地!”那妇
人通红了脸,便收拾了杯盘盏碟,口里说道:“我自作乐耍子,不值得便当真起来,
好不识人敬重!”搬了家火,自向厨下去了。有诗为证:
酒作媒人色胆张,贪淫不顾坏纲常。
席间便欲求云雨,激得雷霆怒一场。

却说潘金莲勾搭武松不动,反被抢白一场。武松自在房里气忿忿地。天色却早,
未牌时分,武大挑了担儿,归来推门,那妇人慌忙开门。武大进来,歇了担儿,随
到厨下,见老婆双眼哭的红红的。武大道:“你和谁闹来?”那妇人道:“都是你
不争气,教外人来欺负我。”武大道:“谁人敢来欺负你?”妇人道:“情知是有
谁!争奈武二那厮,我见他大雪里归来,连忙安排酒请他吃,他见前后没人,便把
言语来调戏我。”武大道:“我的兄弟不是这等人,从来老实,休要高做声,吃邻
舍家笑话!”

武大撇了老婆,来到武松房里叫道:“二哥,你不曾吃点心,我和你吃些个。”
武松只不则声。寻思了半晌,再脱了丝鞋,依旧穿上油膀靴,着了上盖,带上毡笠
儿,一头系缠袋,一面出门。武大叫道:“二哥那里去?”也不应,一直地只顾去
了。

武大回到厨下来问老婆道:“我叫他又不应,只顾望县前这条路走了去,正是
不知怎地了?”那妇人骂道:“糊突桶,有甚么难见处!那厮羞了,没脸儿见你,
走了出去。我猜他已定叫个人来搬行李,不要在这里宿歇。”武大道:“他搬了去,
须吃别人笑话。”那妇人道:“混沌魍魉,他来调戏我,倒不吃别人笑。你要便自
和他道话,我却做不的这样的人。你还了我一纸休书来,你自留他便了。”武大那
里敢再开口。

正在家中两口儿絮聒,只见武松引了一个土兵,拿着条匾担,径来房里,收拾
了行李,便出门去。武大赶出来叫道:“二哥,做甚么便搬了去?”武松道:“哥
哥不要问,说起来,装你的幌子。你只由我自去便了。”武大那里敢再问备细,由
武松搬了去。那妇人在里面喃喃呐呐的骂道:“却也好!人只道一个亲兄弟做都头,
怎地养活了哥嫂,却不知反来嚼咬人!正是‘花木瓜,空好看’。你搬了去,倒谢
天地,且得冤家离眼前。”武大见老婆这等骂,正不知怎地,心中只是咄咄不乐,
放他不下。

自从武松搬了去县衙里宿歇,武大自依然每日上街挑卖炊饼。本待要去县里寻
兄弟说话,却被这婆娘千叮万嘱分付,教不要去兜揽他,因此武大不敢去寻武松。

拈指间,岁月如流,不觉雪晴,过了十数日。却说本县知县自到任已来,却得
二年半多了,赚得好些金银,欲待要使人送上东京去,与亲眷处收贮使用,谋个升
转,却怕路上被人劫了去,须得一个有本事的心腹人去便好。猛可想起武松来:“须
是此人可去,有这等英雄了得!”当日便唤武松到衙内商议道:“我有一个亲戚,
在东京城里住,欲要送一担礼物去,就捎封书问安则个。只恐途中不好行,须是得
你这等英雄好汉,方去得。你可休辞辛苦,与我去走一遭,回来我自重重赏你。”
武松应道:“小人得蒙恩相抬举,安敢推故?既蒙差遣,只得便去。小人也自来不
曾到东京,就那里观看光景一遭。相公明日打点端正了便行。”知县大喜,赏了三
杯,不在话下。

且说武松领下知县言语,出县门来,到得下处,取了些银两,叫了个土兵,却
上街来买了一瓶酒并鱼肉果品之类,一径投紫石街来,直到武大家里。武大恰好卖
炊饼了回来,见武松在门前坐地,叫土兵去厨下安排。那妇人余情不断,见武松把
将酒食来,心中自想道:“莫不这厮思量我了,却又回来?那厮以定强不过我,且
慢慢地相问他。”

那妇人便上楼去,重匀粉面,再整云鬟,换些艳色衣服穿了,来到门前迎接武
松。那妇人拜道:“叔叔,不知怎地错见了?好几日并不上门,教奴心里没理会处。
每日叫你哥哥来县里寻叔叔陪话,归来只说道:‘没寻处。’今日且喜得叔叔家来,
没事坏钱做甚么?”武松答道:“武二有句话,特来要和哥哥、嫂嫂说知则个。”
那妇人道:“既是如此,楼上去坐地。”

三个人来到楼上客位里,武松让哥嫂上首坐了,武松掇个杌子,横头坐了。土
兵搬将酒肉上楼来,摆在桌子上。武松劝哥哥、嫂嫂吃酒。那妇人只顾把眼来睃武
松,武松只顾吃酒。酒至五巡,武松讨付劝杯,叫土兵筛了一杯酒,拿在手里,看
着武大道:“大哥在上,今日武二蒙知县相公差往东京干事,明日便要起程,多是
两个月,少是四五十日便回。有句话,特来和你说知:你从来为人懦弱,我不在家,
恐怕被外人来欺负。假如你每日卖十扇笼炊饼,你从明日为始,只做五扇笼出去卖;
每日迟出早归,不要和人吃酒。归到家里,便下了帘子,早闭上门,省了多少是非
口舌。如若有人欺负你,不要和他争执,待我回来,自和他理论。大哥依我时,满
饮此杯。”武大接了酒道:“我兄弟见得是,我都依你说。”吃过了一杯酒。

武松再筛第二杯酒,对那妇人说道:“嫂嫂是个精细的人,不必用武松多说。
我哥哥为人质朴,全靠嫂嫂做主看觑他。常言道:‘表壮不如里壮。’嫂嫂把得家
定,我哥哥烦恼做甚么?岂不闻古人言:‘篱牢犬不入。’”那妇人听了这话,被
武松说了这一篇,一点红从耳朵边起,紫了面皮,指着武大便骂道:“你这个腌
混沌!有甚么言语,在外人处说来,欺负老娘!我是一个不戴头巾男子汉,叮叮当
当响的婆娘!拳头上立得人,膊上走得马,人面上行的人,不是那等搠不出的鳖
老婆!自从嫁了武大,真个蝼蚁也不敢入屋里来,有甚么篱笆不牢,犬儿钻得入来!
你胡言乱语,一句句都要下落;丢下砖头瓦儿,一个个也要着地。”武松笑道:“若
得嫂嫂这般做主最好!只要心口相应,却不要心头不似口头。既然如此,武二都记
得嫂嫂说的话了,请饮过此杯。”那妇人推开酒盏,一直跑下楼来,走到半胡梯上
发话道:“你既是聪明伶俐,却不道‘长嫂为母’!我当初嫁武大时,曾不听得说
有甚么阿叔,那里走得来!‘是亲不是亲,便要做乔家公’。自是老娘晦气了,鸟
撞着许多事!”哭下楼去了。有诗为证:
良言逆听即为仇,笑眼登时有泪流。
只是两行淫祸水,不因悲苦不因羞。

且说那妇人做出许多奸伪张致,那武大、武松弟兄两个吃了几杯。武松拜辞哥
哥,武大道:“兄弟去了,早早回来,和你相见。”口里说,不觉眼中堕泪。武松
见武大眼中垂泪,便说道:“哥哥便不做得买卖也罢,只在家里坐地。盘缠兄弟自
送将来。”武大送武松下楼来,临出门,武松又道:“大哥,我的言语,休要忘了。”

武松带了土兵,自回县前来收拾。次日早起来,拴束了包裹,来见知县。那知
县已自先差下一辆车儿,把箱笼都装载车子上。点两个精壮土兵,县衙里拨两个心
腹伴当,都分付了。那四个跟了武松,就厅前拜辞了知县,曳扎起,提了朴刀,监
押车子,一行五人,离了阳谷县,取路望东京去了。

话分两头。只说武大郎自从武松说了去,整整的吃那婆娘骂了三四日。武大忍
气吞声,由他自骂,心里只依着兄弟的言语,真个每日只做一半炊饼出去卖,未晚
便归。一脚歇了担儿,便去除了帘子,关上大门,却来家里坐地。那妇人看了这般,
心内焦躁,指着武大脸上骂道:“混沌浊物,我倒不曾见日头在半天里,便把着丧
门关了,也须吃别人道我家怎地禁鬼!听你那兄弟鸟嘴,也不怕别人笑耻。”武大
道:“由他们笑道说我家禁鬼。我的兄弟说的是好话,省了多少是非。”那妇人道:
“呸!浊物!你是个男子汉,自不做主,却听别人调遣。”武大摇手道:“由他。他
说的话,是金子言语。”自武松去了十数日,武大每日只是晏出早归;归到家里,
便关了门。那妇人也和他闹了几场。向后闹惯了,不以为事。自此这妇人约莫到武
大归时,先自去收了帘子,关上大门。武大见了,自心里也喜,寻思道:“恁地时
却好!”

又过了三二日,冬已将残,天色回阳微暖。当日武大将次归来,那妇人惯了,
自先向门前来叉那帘子。也是合当有事,却好一个人从帘子边走过。自古道:“没
巧不成话。”这妇人正手里拿叉竿不牢,失手滑将倒去,不端不正,却好打在那人
头巾上。那人立住了脚,正待要发作;回过脸来看时,是个生的妖娆的妇人,先自
酥了半边,那怒气直钻过爪洼国去了,变作笑吟吟的脸儿。这妇人情知不是,叉手
深深地道个万福,说道:“奴家一时失手,官人休怪。”那人一头把手整头巾,一
面把腰曲着地还礼道:“不妨事。娘子请尊便。”却被这间壁的王婆见了。那婆子
正在茶局子里水帘底下看见了,笑道:“兀谁教大官人打这屋檐边过?打得正好!”
那人笑道:“倒是小人不是。冲撞娘子,休怪。”那妇人答道:“官人不要见责。”
那人又笑着,大大地唱个肥喏道:“小人不敢。”那一双眼,却只在这妇人身上,
临动身,也回了七八遍头,自摇摇摆摆,踏着八字脚去了。这妇人自收了帘子叉竿
归去,掩上大门,等武大归来。诗曰:
篱不牢时犬会钻,收帘对面好相看。
王婆莫负能勾引,须信叉竿是钓竿。

再说来人姓甚名谁?那里居住?原来只是阳谷县一个破落户财主,就县前开着个
生药铺。从小也是一个奸诈的人,使得些好拳棒。近来暴发迹,专在县里管些公事,
与人放刁把滥,说事过钱,排陷官吏。因此,满县人都饶让他些个。那人复姓西门,
单讳一个庆字,排行第一,人都唤他做西门大郎。近来发迹有钱,人都称他做西门
大官人。

不多时,只见那西门庆一转踅入王婆茶坊里来,便去里边水帘下坐了。王婆笑
道:“大官人却才唱得好个大肥喏!”西门庆也笑道:“干娘,你且来,我问你:
间壁这个雌儿,是谁的老小?”王婆道:“他是阎罗大王的妹子,五道将军的女儿,
问他怎地?”西门庆道:“我和你说正话,休要取笑。”王婆道:“大官人怎么不
认得?他老公便是每日在县前卖熟食的。”西门庆道:“莫非是卖枣糕徐三的老婆?”
王婆摇手道:“不是。若是他的,正是一对儿。大官人再猜。”西门庆道:“可是
银担子李二的老婆?”王婆摇头道:“不是。若是他的时,也倒是一双。”西门庆
道:“倒敢是花膊陆小乙的妻子?”王婆大笑道:“不是。若他的时,也又是好
一对儿。大官人再猜一猜。”西门庆道:“干娘,我其实猜不着。”王婆哈哈笑道:
“好教大官人得知了笑一声。他的盖老,便是街上卖炊饼的武大郎。”西门庆跌脚
笑道:“莫不是人叫他三寸丁谷树皮的武大郎?”王婆道:“正是他。”西门庆听
了,叫起苦来说道:“好块羊肉,怎地落在狗口里!”王婆道:“便是这般苦事。
自古道:‘骏马却驮痴汉走,美妻常伴拙夫眠。’月下老偏生要是这般配合!”西
门庆道:“王干娘,我少你多少茶钱?”王婆道:“不多,由他歇些时却算。”西
门庆又道:“你儿子跟谁出去?”王婆道:“说不得。跟一个客人淮上去,至今不
归,又不知死活。”西门庆道:“却不叫他跟我?”王婆笑道:“若得大官人抬举
他,十分之好。”西门庆道:“等他归来,却再计较。”再说了几句闲话,相谢起
身去了。

约莫未及两个时辰,又踅将来王婆店门口帘边坐地,朝着武大门前。半歇,王
婆出来道:“大官人,吃个梅汤?”西门庆道:“最好多加些酸。”王婆做了一个
梅汤,双手递与西门庆。西门庆慢慢地吃了,盏托放在桌子上。西门庆道:“王干
娘,你这梅汤做得好,有多少在屋里?”王婆笑道:“老身做了一世媒,那讨一个
在屋里?”西门庆道:“我问你梅汤,你却说做媒,差了多少。”王婆道:“老身
只听的大官人问这媒做得好,老身只道说做媒。”西门庆道:“干娘,你既是撮合
山,也与我做头媒,说头好亲事,我自重重谢你。”王婆道:“大官人,你宅上大
娘子得知时,婆子这脸,怎吃得耳刮子?”西门庆道:“我家大娘子最好,极是容
得人。现今也讨几个身边人在家里,只是没一个中得我意的。你有这般好的,与我
主张一个,便来说不妨。就是回头人也好,只要中得我意。”王婆道:“前日有一
个倒好,只怕大官人不要。”西门庆道:“若好时,你与我说成了,我自谢你。”
王婆道:“生得十二分人物,只是年纪大些。”西门庆道:“便差一两岁,也不打
紧。真个几岁?”王婆道:“那娘子戊寅生,属虎的,新年恰好九十三岁。”西门
庆笑道:“你看这风婆子,只要扯着风脸取笑。”西门庆笑了起身去。

看看天色晚了,王婆却才点上灯来,正要关门,只见西门庆又踅将来,径去帘
底下那座头上坐了,朝着武大门前只顾望。王婆道:“大官人,吃个和合汤如何?”
西门庆道:“最好。干娘放甜些。”王婆点一盏和合汤,递与西门庆吃。坐个一晚,
起身道:“干娘记了帐目,明日一发还钱。”王婆道:“不妨,伏惟安置,来日早
请过访。”西门庆又笑了去。

当晚无事。次日清早,王婆却才开门,把眼看门外时,只见这西门庆又在门前
两头来往踅。王婆见了道:“这个刷子踅得紧!你看我着些甜糖抹在这厮鼻子上,
只叫他不着。那厮会讨县里人便宜,且教他来老娘手里纳些败缺。”原来这个开
茶坊的王婆,也是不依本分的。端的这婆子:

开言欺陆贾,出口胜隋何。只鸾孤凤,霎时间交仗成双;寡妇鳏男,一席话搬
唆捉对。略施妙计,使阿罗汉抱住比丘尼;稍用机关,教李天王搂定鬼子母。甜言
说诱,男如封涉也生心;软语调和,女似麻姑能动念。教唆得织女害相思,调弄得
嫦娥寻配偶。

且说王婆却才开得门,正在茶局子里生炭,整理茶锅。张见西门庆从早晨在门
前踅了几遭,一径奔入茶房里来,水帘底下,望着武大门前帘子里坐了看。王婆只
做不看见,只顾在茶局里煽风炉子,不出来问茶。西门庆叫道:“干娘,点两盏茶
来。”王婆应道:“大官人来了。连日少见,且请坐。”便浓浓的点两盏姜茶,将
来放在桌子上。西门庆道:“干娘相陪我吃个茶。”王婆哈哈笑道:“我又不是影
射的。”西门庆也笑了一回,问道:“干娘,间壁卖甚么?”王婆道:“他家卖拖
蒸河漏子热烫温和大辣酥。”西门庆笑道:“你看这婆子只是风。”王婆笑道:“我
不风,他家自有亲老公。”西门庆道:“干娘,和你说正经话:说他家如法做得好
炊饼,我要问他做三五十个,不知出去在家?”王婆道:“若要买炊饼,少间等他
街上回了买,何消得上门上户?”西门庆道:“干娘说的是。”吃了茶,坐了一回,
起身道:“干娘记了帐目。”王婆道:“不妨事。老娘牢牢写在帐上。”西门庆笑
了去。

王婆只在茶局子里张时,冷眼睃见西门庆又在门前踅过东去,又看一看;走过
西来,又睃一睃;走了七八遍,径踅入茶坊里来。王婆道:“大官人稀行,好几时
不见面。”西门庆笑将起来,去身边摸出一两来银子,递与王婆,说道:“干娘权
收了做茶钱。”婆子笑道:“何消得许多?”西门庆道:“只顾放着。”婆子暗暗
地喜欢道:“来了,这刷子当败。”且把银子来藏了,便道:“老身看大官人有些
渴,吃个宽煎叶儿茶如何?”西门庆道:“干娘如何便猜得着?”婆子道:“有甚
么难猜。自古道:‘入门休问荣枯事,观着容颜便得知。’老身异样跷蹊作怪的事,
都猜得着。”西门庆道:“我有一件心上的事,干娘若猜的着时,输与你五两银子。”
王婆笑道:“老娘也不消三智五猜,只一智便猜个十分。大官人,你把耳朵来。你
这两日脚步紧,赶趁得频,以定是记挂着隔壁那个人。我这猜如何?”西门庆笑起
来道:“干娘,你端的智赛隋何,机强陆贾!不瞒干娘说:我不知怎地吃他那日叉
帘子时,见了这一面,却似收了我三魂七魄的一般,只是没做个道理入脚处。不知
你会弄手段么?”王婆哈哈的笑起来道:“老身不瞒大官人说:我家卖茶,叫做鬼
打更。三年前六月初三下雪的那一日,卖了一个泡茶,直到如今不发市,专一靠些
杂趁养口。”

西门庆问道:“怎地叫做杂趁?”王婆笑道:“老身为头是做媒,又会做牙婆,
也会抱腰,也会收小的,也会说风情,也会做马泊六。”西门庆道:“干娘端的与
我说得这件事成,便送十两银子与你做棺材本。”王婆道:“大官人,我知道还有
一件事打搅,也多是地不得。”西门庆说:“你且道甚么一件事打搅?”王婆道:
“大官人,休怪老身直言:但凡捱光最难,十分光时,使钱到九分九厘,也有难成
就处。我知你从来悭吝,不肯胡乱便使钱。只这一件打搅。”西门庆道:“这个极
容易医治,我只听你的言语便了。”

王婆道:“若是大官人肯使钱时,老身有一条计,便教大官人和这雌儿会一面。
只不知官人肯依我么?”西门庆道:“不拣怎地,我都依你。干娘有甚妙计?”王
婆笑道:“今日晚了,且回去。过半年三个月,却来商量。”西门庆便跪下道:“干
娘休要撒科,你作成我则个!”

王婆笑道:“大官人却又慌了。老身那条计,是个上着。虽然入不得武成王庙。
端的强似孙武子教女兵,十捉九着。大官人,我今日对你说:这个人原是清河县大
户人家讨来的养女,却做得一手好针线。大官人,你便买一匹白绫,一匹蓝绸,一
匹白绢,再用十两好绵,都把来与老身。我却走将过去,问他讨茶吃,却与这雌儿
说道:‘有个施主官人,与我一套送终衣料,特来借历头,央及娘子与老身拣个好
日,去请个裁缝来做。’他若见我这般说,不睬我时,此事便休了。他若说:‘我
替你做。’不要我叫裁缝时,这便有一分光了。我便请他家来做。他若说:‘将来
我家里做。’不肯过来,此事便休了。他若欢天喜地说:‘我来做,就替你裁。’
这光便有二分了。若是肯来我这里做时,却要安排些酒食点心请他。第一日,你也
不要来。第二日,他若说不便,当时定要将家去做,此事便休了。他若依前肯过我
家做时,这光便有三分了。这一日,你也不要来。到第三日晌午前后,你整整齐齐
打扮了来,咳嗽为号。你便在门前说道:‘怎地连日不见王干娘?’我便出来,请
你入房里来。若是他见你入来,便起身跑了归去,难道我拖住他?此事便休了。他
若见你入来,不动身时,这光便有四分了。坐下时,便对雌儿说道:‘这个便是与
我衣料的施主官人。亏煞他!’我夸大官人许多好处,你便卖弄他的针线。若是他
不来兜揽应答,此事便休了。他若口里应答说话时,这光便有五分了。我却说道:
‘难得这个娘子与我作成出手做。亏煞你两个施主:一个出钱的,一个出力的。不
是老身路歧相央,难得这个娘子在这里,官人好做个主人,替老身与娘子浇手。’
你便取出银子来央我买。若是他抽身便走时,不成扯住他?此事便休了。他若是不
动身时,事务易成,这光便有六分了。我却拿了银子,临出门对他道:‘有劳娘子
相待大官人坐一坐。’他若也起身走了家去时,我也难道阻当他?此事便休了。若
是他不起身走动时,此事又好了,这光便有七分了。等我买得东西来,摆在桌子上,
我便道:‘娘子且收拾生活,吃一杯儿酒,难得这位官人坏钞。’他若不肯和你同
桌吃时,走了回去,此事便休了。若是他只口里说要去,却不动身时,此事又好了,
这光便有八分了。待他吃的酒浓时,正说得入港,我便推道没了酒,再叫你买,你
便又央我去买。我只做去买酒,把门曳上,关你和他两个在里面。他若焦躁,跑了
归去,此事便休了。他若由我曳上门,不焦躁时,这光便有九分了。只欠一分光了
便完就。这一分倒难。大官人,你在房里,着几句甜净的话儿,说将入去。你却不
可躁暴,便去动手动脚,打搅了事,那时我不管你。先假做把袖子在桌上拂落一双
箸去,你只做去地下拾箸,将手去他脚上捏一捏,他若闹将起来,我自来搭救,此
事也便休了,再也难得成。若是他不做声时,此是十分光了。他必然有意,这十分
事做得成。这条计策如何?”

西门庆听罢,大喜道:“虽然上不得凌烟阁,端的好计!”王婆道:“不要忘
了许我的十两银子!”西门庆道:“‘但得一片橘皮吃,莫便忘了洞庭湖!’这条
计几时可行?”王婆道:“只在今晚,便有回报。我如今趁武大未归,走过去细细
地说诱他。你却便使人将绫绸绢匹并绵子来。”西门庆道:“得干娘完成得这件事,
如何敢失信?”作别了王婆,便去市上绸绢铺里买了绫绸绢缎,并十两清水好绵。
家里叫个伴当,取包袱包了,带了五两碎银,径送入茶坊里。王婆接了这物,分付
伴当回去。诗曰:
岂是风流胜可争?迷魂阵里出奇兵。
安排十面捱光计,只取亡身入陷坑。

这王婆开了后门,走过武大家里来。那妇人接着请去楼上坐地。那王婆道:“娘
子怎地不过贫家吃茶?”那妇人道:“便是这几日身体不快,懒走去的。”王婆道:
“娘子家里有历日么?借与老身看一看,要选个裁衣日。”那妇人道:“干娘裁甚
么衣裳?”王婆道:“便是老身十病九痛,怕有些山高水低,头先要制办些送终衣
服。难得近处一个财主,见老身这般说,布施与我一套衣料,绫绸绢缎,又与若干
好绵,放在家里一年有余,不能够做。今年觉道身体好生不济,又撞着如今闰月,
趁这两日要做;又被那裁缝勒,只推生活忙,不肯来做。老身说不得这等苦!”
那妇人听了笑道:“只怕奴家做得不中干娘意;若不嫌时,奴出手与干娘做如何?”
那婆子听了这话,堆下笑来说道:“若得娘子贵手做时,老身便死来也得好处去。
久闻娘子好手针线,只是不敢来相央。”那妇人道:“这个何妨。既是许了干娘,
务要与干娘做了。将历头去叫人拣个黄道好日,奴便与你动手。”王婆道:“若得
娘子肯与老身做时,娘子是一点福星,何用选日?老身也前日央人看来,说道明日
是个黄道好日。老身只道裁衣不用黄道日了,不记他。”那妇人道:“归寿衣正要
黄道日好,何用别选日?”王婆道:“既是娘子肯作成老身时,大胆只是明日起动
娘子到寒家则个。”那妇人道:“干娘,不必。将过来做不得?”王婆道:“便是
老身也要看娘子做生活则个;又怕家里没人看门前。”那妇人道:“既是干娘恁地
说时,我明日饭后便来。”那婆子千恩万谢下楼去了。当晚回复了西门庆的话,约
定后日准来。当夜无语。次日清早,王婆收拾房里干净了,买了些线索,安排了些
茶水,在家里等候。

且说武大吃了早饭,打当了担儿,自出去做道路。那妇人把帘儿挂了,从后门
走过王婆家里来。那婆子欢喜无限,接入房里坐下,便浓浓地点道茶,撒上些出白
松子、胡桃肉,递与这妇人吃了。抹得桌子干净,便将出那绫绸绢缎来。妇人将尺
量了长短,裁得完备,便缝起来。婆子看了,口里不住声价喝采道:“好手段!老
身也活了六七十岁,眼里真个不曾见这般好针线。”那妇人缝到日中,王婆便安排
些酒食请他,下了一斤面,与那妇人吃了。再缝了一歇,将次晚来,便收拾起生活,
自归去。

恰好武大归来,挑着空担儿进门,那妇人曳开门,下了帘子,武大入屋里来,
看见老婆面色微红,便问道:“你那里吃酒来?”那妇人应道:“便是间壁王干娘,
央我做送终的衣裳,日中安排些点心请我。”武大道:“阿呀!不要吃他的,我们
也有央及他处。他便央你做得件把衣裳,你便自归来吃些点心,不值得搅恼他。你
明日倘或再去做时,带了些钱在身边,也买些酒食与他回礼。常言道:‘远亲不如
近邻。’休要失了人情。他若是不肯要你还礼时,你便只是拿了家来,做去还他。”
那妇人听了,当晚无话。有诗为证:
可奈虔婆设计深,大郎混沌不知因。
带钱买酒酬奸诈,却把婆娘白送人。

且说王婆子设计已定,赚潘金莲来家。次日饭后,武大自出去了,王婆便踅过
来相请。去到他房里,取出生活,一面缝将起来。王婆自一边点茶来吃了。不在话
下。看看日中,那妇人取出一贯钱付与王婆说道:“干娘,奴和你买杯酒吃。”王
婆道:“阿呀!那里有这个道理?老身央及娘子在这里做生活,如何颠倒教娘子坏
钱?”那妇人道:“却是拙夫分付奴来。若还干娘见外时,只是将了家去做还干娘。”
那婆子听了,连声道:“大郎直恁地晓事。既然娘子这般说时,老身权且收下。”
这婆子生怕打脱了这事,自又添钱去买些好酒好食、希奇果子来,殷勤相待。看官
听说:但凡世上妇人,由你十八分精细,被人小意儿过纵,十个九个着了道儿。再
说王婆安排了点心,请那妇人吃了酒食,再缝了一歇,看看晚来,千恩万谢归去了。

话休絮繁。第三日早饭后,王婆只张武大出去了,便走过后头来叫道:“娘子,
老身大胆……”那妇人从楼上下来道:“奴却待来也。”两个厮见了,来到王婆房
里坐下,取过生活来缝。那婆子随即点盏茶来,两个吃了。那妇人看看缝到晌午前
后。却说西门庆巴不到这一日,裹了顶新头巾,穿了一套整整齐齐衣服,带了三五
两碎银子,径投这紫石街来。到得茶坊门首,便咳嗽道:“王干娘,连日如何不见?”
那婆子瞧科,便应道:“兀谁叫老娘?”西门庆道:“是我。”那婆子赶出来,看
了笑道:“我只道是谁,却原来是施主大官人。你来得正好,且请你入去看一看。”
把西门庆袖子一拖,拖进房里,看着那妇人道:“这个便是那施主,与老身这衣料
的官人。”西门庆见了那妇人,便唱个喏。那妇人慌忙放下生活,还了万福。

王婆却指着这妇人对西门庆道:“难得官人与老身缎匹,放了一年,不曾做得。
如今又亏杀这位娘子出手与老身做成全了。真个是布机也似好针线,又密又好,其
实难得!大官人,你且看一看。”西门庆把起来看了喝采,口里说道:“这位娘子
怎地传得这手好生活,神仙一般的手段!”那妇人笑道:“官人休笑话!”西门庆
问王婆道:“干娘,不敢问,这位是谁家宅上娘子?”王婆道:“大官人,你猜。”
西门庆道:“小人如何猜得着?”王婆吟吟的笑道:“便是间壁的武大郎的娘子。
前日叉竿打得不疼,大官人便忘了?”那妇人赤着脸便道:“那日奴家偶然失手,
官人休要记怀。”西门庆道:“说那里话。”王婆便接口道:“这位大官人,一生
和气,从来不会记恨,极是好人。”西门庆道:“前日小人不认得,原来却是武大
郎的娘子。小人只认的大郎一个养家经纪人,且是在街上做些买卖,大大小小,不
曾恶了一个人;又会赚钱,又且好性格,真个难得这等人。”王婆道:“可知哩!
娘子自从嫁得这个大郎,但是有事,百依百随。”那妇人应道:“拙夫是无用之人,
官人休要笑话。”西门庆道:“娘子差矣!古人道:‘柔软是立身之本,刚强是惹
祸之胎。’似娘子的大郎所为良善时,‘万丈水无涓滴漏’。”王婆打着撺鼓儿道:
“说的是。”

西门庆奖了一回,便坐在妇人对面。王婆又道:“娘子,你认的这个官人么?”
那妇人道:“奴不认的。”婆子道:“这个大官人,是这本县一个财主,知县相公
也和他来往,叫做西门大官人。万万贯钱财,开着个生药铺在县前。家里钱过北斗,
米烂陈仓;赤的是金,白的是银,圆的是珠,光的是宝。也有犀牛头上角,亦有大
象口中牙。”那婆子只顾夸奖西门庆,口里假嘈。那妇人就低了头缝针线。西门庆
得见潘金莲十分情思,恨不就做一处。王婆便去点两盏茶来,递一盏与西门庆,一
盏递与这妇人,说道:“娘子相待大官人则个。”吃罢茶,便觉有些眉目送情。王
婆看着西门庆,把一只手在脸上摸,西门庆心里瞧科,已知有五分了。

王婆便道:“大官人不来时,老身也不敢来宅上相请。一者缘法,二乃来得恰
好。常言道:‘一客不烦二主。’大官人便是出钱的,这位娘子便是出力的。不是
老身路歧相烦,难得这位娘子在这里,官人好做个主人,替老身与娘子浇手。”西
门庆道:“小人也见不到,这里有银子在此。”便取出来,和帕子递与王婆,备办
些酒食。那妇人便道:“不消生受得。”口里说,却不动身。王婆将了银子便去,
那妇人又不起身。婆子便出门,又道:“有劳娘子相陪大官人坐一坐。”那妇人道:
“干娘,免了。”却亦是不动身。也是因缘,却都有意了。西门庆这厮一双眼只看
着那妇人;这婆娘一双眼也把来偷睃西门庆,见了这表人物,心中倒有五七分意了,
又低着头自做生活。

不多时,王婆买了些现成的肥鹅、熟肉、细巧果子归来,尽把盘子盛了;果子
菜蔬,尽都装了,搬来房里桌子上,看着那妇人道:“娘子且收拾过生活,吃一杯
儿酒。”那妇人道:“干娘自便,相待大官人,奴却不当。”依旧原不动身。那婆
子道:“正是专与娘子浇手,如何却说这话?”王婆将盘馔都摆在桌子上,三人坐
定,把酒来斟。这西门庆拿起酒盏来说道:“娘子,满饮此杯。”那妇人谢道:“多
感官人厚意。”王婆道:“老身知得娘子洪饮,且请开怀吃两盏儿。”有诗为证:
从来男女不同筵,卖俏迎奸最可怜。
不记都头昔日语,犬儿今已到篱边。
又诗曰:
须知酒色本相连,饮食能成男女缘。
不必都头多嘱付,开篱日待犬来眠。

却说那妇人接酒在手,那西门庆拿起箸来道:“干娘,替我劝娘子请些个。”
那婆子拣好的递将过来,与那妇人吃。一连斟了三巡酒,那婆子便去烫酒来。

西门庆道:“不敢动问娘子青春多少?”那妇人应道:“奴家虚度二十三岁。”
西门庆道:“小人痴长五岁。”那妇人道:“官人将天比地。”王婆便插口道:“好
个精细的娘子,不惟做得好针线,诸子百家皆通。”西门庆道:“却是那里去讨?
武大郎好生有福!”王婆便道:“不是老身说是非,大官人宅里枉有许多,那里讨
一个赶得上这娘子的!”西门庆道:“便是这等一言难尽!只是小人命薄,不曾招
得一个好的。”王婆道:“大官人先头娘子须好。”西门庆道:“休说!若是我先
妻在时,却不怎地家无主,屋倒竖。如今枉自有三五七口人吃饭,都不管事。”那
妇人问道:“官人恁地时,殁了大娘子得几年了?”西门庆道:“说不得。小人先
妻,是微末出身,却倒百伶百俐,是件件都替的小人。如今不幸他殁了,已得三年,
家里的事,都七颠八倒。为何小人只是走了出来?在家里时,便要怄气!”那婆子
道:“大官人,休怪老身直言:你先头娘子,也没有武大娘子这手针线。”西门庆
道:“便是小人先妻,也没此娘子这表人物。”那婆子笑道:“官人,你养的外宅
在东街上,如何不请老身去吃茶?”西门庆道:“便是唱慢曲儿的张惜惜。我见他
是路歧人,不喜欢。”婆子又道:“官人,你和李娇娇却长久。”西门庆道:“这
个人,现今取在家里。若得他会当家时,自册正了他多时。”王婆道:“若有这般
中的官人意的来宅上说,没妨事么?”西门庆道:“我的爹娘俱已没了,我自主张,
谁敢道个‘不’字!”王婆道:“我自说耍,急切那里有中得官人意的?”西门庆
道:“做甚么了便没!只恨我夫妻缘分上薄,自不撞着。”

西门庆和这婆子,一递一句,说了一回。王婆便道:“正好吃酒,却又没了。
官人休怪老身差拨,再买一瓶儿酒来吃如何?”西门庆道:“我手帕里有五两来碎
银子,一发撒在你处,要吃时只顾取来,多的干娘便就收了。”那婆子谢了官人,
起身睃这粉头时,一钟酒落肚,哄动春心;又自两个言来语去,都有意了,只低了
头,却不起身。那婆子满脸堆下笑来说道:“老身去取瓶儿酒来,与娘子再吃一杯
儿。有劳娘子相待大官人坐一坐。注子里有酒没?便再筛两盏儿,和大官人吃。老
身直去县前那家,有好酒买一瓶来,有好歇儿耽搁。”那妇人口里说道:“不用了。”
坐着却不动身。婆子出到房门前,便把索儿缚了房门,却来当路坐了。

且说西门庆自在房里,便斟酒来劝那妇人,却把袖子在桌上一拂,把那双箸拂
落地下。也是缘法凑巧,那双箸正落在妇人脚边。西门庆连忙蹲身下去拾,只见那
妇人尖尖的一双小脚儿,正在箸边。西门庆且不拾箸,便去那妇人绣花鞋儿上捏
一把。那妇人便笑将起来,说道:“官人休要罗唣!你真个要勾搭我?”西门庆便
跪下道:“只是娘子作成小生。”那妇人便把西门庆搂将起来。

当下只见王婆推开房门入来,怒道:“你两个做得好事!”西门庆和那妇人都
吃了一惊。那婆子便道:“好呀,好呀!我请你来做衣裳,不曾叫你来偷汉子!武大
得知,须连累我,不若我先去出首。”回身便走。那妇人扯住裙儿道:“干娘饶恕
则个!”西门庆道:“干娘低声!”王婆笑道:“若要我饶恕你们,都要依我一件
事。”那妇人便道:“休说一件,便是十件,奴也依干娘。”王婆道:“你从今日
为始,瞒着武大,每日不要失约负了大官人,我便罢休;若是一日不来,我便对你
武大说。”那妇人道:“只依着干娘便了。”王婆又道:“西门大官人,你自不用
老身说得。这十分好事,已都完了。所许之物,不可失信。你若负心,我也要对武
大说。”西门庆道:“干娘放心,并不失信。”三人又吃几杯酒,已是下午的时分,
那妇人便起身道:“武大那厮将归来,奴自回去。”便踅过后门归家,先去下了帘
子,武大恰好进门。

且说王婆看着西门庆道:“好手段么?”西门庆道:“端的亏了干娘!我到家
里,便取一锭银送来与你,所许之物,岂敢昧心!”王婆道:“‘眼望旌节至,专
等好消息。’不要叫老身‘棺材出了讨挽歌郎钱’。”西门庆笑了去,不在话下。

那妇人自当日为始,每日踅过王婆家里来,和西门庆做一处,恩情似漆,心意
如胶。自古道:“好事不出门,恶事传千里。”不到半月之间,街坊邻舍,都知得
了,只瞒着武大一个不知。有诗为证:
半晌风流有何益,一般滋味不须夸。
他时祸起萧墙内,悔杀今朝恋野花。

断章句,话分两头。且说本县有个小的,年方十五六岁,本身姓乔。因为做军
在郓州生养的,就取名叫做郓哥。家中止有一个老爹。那小厮生得乖觉,自来只靠
县前这许多酒店里卖些时新果品,时常得西门庆赍发他些盘缠。其日,正寻得一篮
儿雪梨,提着来绕街寻问西门庆。又有一等的多口人说道:“郓哥,你若要寻他,
我教你一处去寻。”郓哥道:“聒噪阿叔,叫我去寻得他见,赚得三五十钱养活老
爹也好。”那多口的道:“西门庆他如今刮上了卖炊饼的武大老婆,每日只在紫石
街上王婆茶房里坐地,这早晚多定正在那里。你小孩子家,只顾撞入去不妨。”

那郓哥得了这话,谢了阿叔指教。这小猴子提了篮儿,一直望紫石街走来,径
奔入茶坊里去,却好正见王婆坐在小凳儿上绩绪。郓哥把篮儿放下,看着王婆道:
“干娘拜揖。”那婆子问道:“郓哥,你来这里做甚么?”郓哥道:“要寻大官人,
赚三五十钱,养活老爹。”婆子道:“甚么大官人?”郓哥道:“干娘情知是那个,
便只是他那个。”婆子道:“便是大官人,也有个姓名。”郓哥道:“便是两个字
的。”婆子道:“甚么两个字的?”郓哥道:“干娘只是要作耍。我要和西门大官
人说句话。”望里面便走。那婆子一把揪住道:“小猴子,那里去?人家屋里,各
有内外。”郓哥道:“我去房里便寻出来。”王婆道:“含鸟猢狲,我屋里那得甚
么西门大官人!”郓哥道:“干娘,不要独吃自呵!也把些汁水与我呷一呷!我有甚
么不理会得!”婆子便骂道:“你那小猢狲,理会得甚么!”郓哥道:“你正是‘马
蹄刀木杓里切菜’,水泄不漏,半点儿也没得落地。直要我说出来,只怕卖炊饼的
哥哥发作!”

那婆子吃他这两句道着他真病,心中大怒,喝道:“含鸟猢狲,也来老娘屋里
放屁辣臊!”郓哥道:“我是小猢狲,你是马泊六!”那婆子揪住郓哥,凿上两个
栗暴。郓哥叫道:“做甚么便打我!”婆子骂道:“贼猢狲,高则声,大耳刮子打
出你去!”郓哥道:“老咬虫,没事得便打我!”这婆子一头叉,一头大栗暴凿,
直打出街上去,雪梨篮儿也丢出去。那篮雪梨四分五落,滚了开去。这小猴子打那
虔婆不过,一头骂,一头哭,一头走,一头街上拾梨儿,指着那王婆茶坊里骂道:
“老咬虫,我教你不要慌!我不去说与他!不做出来不信!”提了篮儿,径奔去寻
这个人。正是从前作过事,没兴一齐来。直教:掀翻狐兔窝中草,惊起鸳鸯沙上眠。

毕竟这郓哥寻甚么人,且听下回分解。