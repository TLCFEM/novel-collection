\chapter{戴宗智取公孙胜~李逵斧劈罗真人}

话说当下吴学究对宋公明说道:“要破此法,只除非快教人去蓟州寻取公孙胜
来,便可破得。”宋江道:“前番戴宗去了几时,全然打听不着,却那里去寻?”
吴用道:“只说蓟州,有管下多少县治、镇市、乡村,他须不曾寻得到。我想公孙
胜,他是个清高的人,必然在个名山洞府、大川真境居住。今番教戴宗可去绕蓟州
管下县道名山仙境去处,寻觅一遭,不愁不见他。”宋江听罢,随即叫请戴院长商
议:可往蓟州寻取公孙胜。戴宗道:“小可愿往,只是得一个做伴的去方好。”吴
用道:“你作起神行法来,谁人赶得你上?”戴宗道:“若是同伴的人,我也把甲
马拴在他腿上,教他也走得许多路程。”李逵便道:“我与戴院长做伴走一遭。”
戴宗道:“你若要跟我去,须要一路上吃素,都听我的言语。”李逵道:“这个有
甚难处?我都依你便了。”宋江、吴用分付道:“路上小心在意,休要惹事。若得
见了,早早回来。”李逵道:“我打死了殷天锡,却教柴大官人吃官司。我如何不
要救他?今番并不敢惹事了。”二人各藏了暗器,拴缚了包裹,拜辞宋江并众人,
离了高唐州,取路投蓟州来。

走了二十余里,李逵立住脚道:“大哥,买碗酒吃了走也好。”戴宗道:“你
要跟我作神行法,须要只吃素酒。且向前面去。”李逵答道:“便吃些肉,也打甚
么紧。”戴宗道:“你又来了。今日已晚,且寻客店宿了,明日早行。”两个又走
了三十余里,天色昏黑,寻着一个客店歇了,烧起火来做饭,沽一角酒来吃。李逵
搬一碗素饭,并一碗菜汤,来房里与戴宗吃。戴宗道:“你如何不吃饭?”李逵应
道:“我且未要吃饭哩。”戴宗寻思道:“这厮必然瞒着我背地里吃荤。”戴宗自
把素饭吃了,却悄悄地来后面张时,见李逵讨两角酒,一盘牛肉,在那里自吃。戴
宗道:“我说甚么?且不要道破他,明日小小地耍他耍便了。”戴宗自去房里睡了。
李逵吃了一回酒肉,恐怕戴宗说他,自暗暗的来房里睡了。

到五更时分,戴宗起来叫李逵打火,做些素饭吃了,各分行李在背上,算还了
房客钱,离了客店。行不到二里多路,戴宗说道:“我们昨日不曾使神行法,今日
须要赶程途,你先把包裹拴得牢了,我与你作法,行八百里便住。”戴宗取四个甲
马,去李逵两只腿上也缚了,分付道:“你前面酒食店里等我。”戴宗念念有词,
吹口气在李逵腿上,李逵拽开脚步,浑如驾云的一般,飞也似去了。戴宗笑道:“且
着他忍一日饿。”戴宗也自拴上甲马,随后赶来。李逵不省得这法,只道和他走路
一般。只听耳朵边风雨之声,两边房屋树木,一似连排价倒了的,脚底下如云催雾
趱。李逵怕将起来,几遍待要住脚,两条腿那里收拾得住,却似有人在下面推的相
似,脚不点地,只管的走去了。看见酒肉饭店,又不能够入去买吃,李逵只得叫:
“爷爷,且住一住!”看看走到红日平西,肚里又饥又渴,越不能够住脚,惊得一
身臭汗,气喘做一团。戴宗从背后赶来,叫道:“李大哥,怎的不买些点心吃了去?”
李逵应道:“哥哥,救我一救,饿杀铁牛也!”戴宗怀里摸出几个炊饼来自吃。李
逵叫道:“我不能够住脚买吃,你与我两个充饥。”戴宗道:“兄弟,你走上来与
你吃。”李逵伸着手,只隔一丈来远近,只接不着。李逵叫道:“好哥哥,等我一
等。”戴宗道:“便是今日有些跷蹊,我的两条腿也不能够住。”李逵道:“阿也!
我的这鸟脚不由我半分,自这般走了去,只好把大斧砍了那下半截下来。”戴宗道:
“只除是恁的般方好。不然,直走到明年正月初一日,也不能住。”李逵道:“好
哥哥,休使道儿耍我,砍了腿下来,你却笑我。”戴宗道:“你敢是昨夜不依我?
今日连我也走不得住,你自走去。”李逵叫道:“好爷爷,你饶我住一住!”戴宗
道:“我的这法,不许吃荤,第一戒的是牛肉。若还吃了一块牛肉,直要走十万里,
方才得住。”李逵道:“却是苦也!我昨夜不合瞒着哥哥,真个偷买几斤牛肉吃了,
正是怎么好!”戴宗道:“怪得今日连我的这腿也收不住,只用去天尽头走一遭了,
慢慢地却得三五年,方才回得来。”李逵听罢,叫起撞天屈来。

戴宗笑道:“你从今已后,只依得我一件事,我便罢得这法。”李逵道:“老
爹,我今都依你便了。”戴宗道:“你如今敢再瞒着我吃荤么?”李逵道:“今后
但吃荤,舌头上生碗来大疔疮!我见哥哥要吃素,铁牛却吃不得,因此上瞒着哥哥,
今后并不敢了。”戴宗道:“既是恁地,饶你这一遍!”退后一步,把衣袖去李逵
腿上只一拂,喝声:“住!”李逵却似钉住了的一般,两只脚立定地下,挪移不动。
戴宗道:“我先去,你且慢慢的来。”李逵正待抬脚,那里移得动,拽也拽不起,
一似生铁铸就了的。李逵大叫道:“又是苦也!晚夕怎地得去?”便叫道:“哥哥
救我一救。”戴宗转回头来笑道:“你今番依我说么?”李逵道:“你是我亲爷,
却是不敢违了你的言语。”戴宗道:“你今番却要依我。”便把手绾了李逵,喝声:
“起!”两个轻轻地走了去。李逵道:“哥哥,可怜见铁牛,早歇了罢!”前面到
一个客店,两个且来投宿。戴宗、李逵入到房里去,腿上都卸下甲马来,取出几陌
纸钱烧送了,问李逵道:“今番却如何?”李逵道:“这两条腿,方才是我的了。”
戴宗道:“谁着你夜来私买酒肉吃?”李逵道:“为是你不许我吃荤,偷了些吃,
也吃你耍得我好了。”

戴宗叫李逵安排些素酒素饭吃了,烧汤洗了脚,上床歇了。睡到五更起来,洗
漱罢,吃了饭,还了房钱,两个又上路。行不到三里多路,戴宗取出甲马道:“兄
弟,今日与你只缚两个,教你慢行些。”李逵道:“亲爷,我不要缚了。”戴宗道:
“你既依我言语,我和你干大事,如何肯弄你?你若不依我,教你一似夜来只钉住
在这里。只等我去蓟州寻见了公孙胜,回来放你。”李逵慌忙叫道:“我依,我依!”
戴宗与李逵当日各缚两个甲马,作起神行法,扶着李逵两个一同走。原来戴宗的法,
要行便行,要住便住。李逵从此那里敢违他言语,于路上只是买些素酒素饭,吃了
便行。话休絮繁。两个用神行法,不旬日,迤来蓟州城外客店里歇了。

次日,两个入城来,戴宗扮做主人,李逵扮做仆者。绕城中寻了一日,并无一
个认得公孙胜的,两个自回店里歇了。次日,又去城中小街狭巷,寻了一日,绝无
消耗。李逵心焦,骂道:“这个乞丐道人,却鸟躲在那里!我若见时,脑揪将去见
哥哥。”戴宗说道:“你又来了,若不听我言语,我又教你吃苦。”李逵笑道:“我
自这般说耍。”戴宗又埋怨了一回,李逵不敢回话。两个又来店里歇了。

次日早起,却去城外近村镇市寻觅。戴宗但见老人,便施礼拜问公孙胜先生家
在那里居住,并无一人认得。戴宗也问过数十处。当日晌午时分,两个走得肚饥,
路旁边见一个素面店,两个直入来,买些点心吃。只见里面都坐满,没一个空处,
戴宗、李逵立在当路。过卖问道:“客官要吃面时,和这老人合坐一坐。”戴宗见
个老丈,独自一个占着一付大座头,便与他施礼,唱个喏,两个对面坐了。李逵坐
在戴宗肩下,分付过卖造四个壮面来。戴宗道:“我吃一个,你吃三个不少么?”
李逵道:“不济事。一发做六个来,我都包办。”过卖见了也笑。等了半日,不见
把面来。李逵却见都搬入里面去了,心中已有五分焦躁。只见过卖却搬一个热面,
放在合坐老人面前。那老人也不谦让,拿起面来便吃。那分面却热,老儿低着头,
伏桌儿吃。李逵性急,见不搬面来,叫一声:“过卖!”骂道:“却教老爷等了这
半日。”把那桌子只一拍,溅那老人一脸热汁,那分面都泼翻了。老儿焦躁,便来
揪住李逵,喝道:“你是何道理,打翻我面?”李逵捻起拳头,要打老儿。

戴宗慌忙喝住,与他陪话道:“丈丈休和他一般见识,小可赔丈丈一分面。”
那老人道:“客官不知:老汉路远,早要吃了面回去听讲,迟时误了程途。”戴宗
问道:“丈丈何处人氏?却听谁人讲甚么?”老儿答道:“老汉是本处蓟州管下九
宫县二仙山下人氏。因来这城中买些好香回去,听山上罗真人讲说长生不老之法。”
戴宗寻思道:“莫不公孙胜也在那里?”便问老人道:“丈丈贵庄,曾有个公孙胜
么?”老人道:“客官问别人定不知,多有人不认的他。老汉和他是邻舍。他只有
个老母在堂。这个先生,一向云游在外,此时唤做公孙一清。如今出姓,都只叫他
清道人,不叫做公孙胜。此是俗名,无人认得。”戴宗道:“正是‘踏破铁鞋无觅
处,得来全不费工夫’。”戴宗又拜问丈丈道:“九宫县二仙山离此间多少路?清
道人在家么?”老人道:“二仙山只离本县四十五里便是。清道人,他是罗真人上
首徒弟。他本师不放离左右。”戴宗听了大喜,连忙催趱面来吃,和那老儿一同吃
了,算还面钱,同出店肆,问了路途。戴宗道:“丈丈先行。小可买些香纸,也便
来也。”老人作别去了。

戴宗、李逵回到客店里,取了行李包裹,再拴上甲马,离了客店,两个取路投
九宫县二仙山来。戴宗使起神行法,四十五里,片时到了。二人来到县前,问二仙
山时,有人指道:“离县投东,只有五里便是。”两个又离了县治,投东而行。果
然行不到五里,早望见那座仙山,委实秀丽。但见:

青山削翠,碧岫堆云。两崖分虎踞龙盘,四面有猿啼鹤唳。朝看云封山顶,暮
观日挂林梢。流水潺漫,涧内声声鸣玉;飞泉瀑布,洞中隐隐奏瑶琴。若非道侣
修行,定有仙翁炼药。

当下戴宗、李逵来到二仙山下,见个樵夫,戴宗与他施礼,说道:“借问此间
清道人家在何处居住?”樵夫指道:“只过这东山嘴,门外有条小石桥的便是。”
两个抹过山嘴来,见有十数间草房,一周围矮墙,墙外一座小小石桥。两个来到桥
边,见一个村姑提一篮新果子出来。戴宗施礼问道:“娘子从清道人家出来,清道
人在家么?”村姑答道:“在屋后炼丹。”戴宗心中暗喜,分付李逵道:“你且去
树背后躲一躲,待我自入去,见了他,却来叫你。”戴宗自入到里面看时,一带三
间草房,门上悬挂一个芦帘。戴宗咳嗽了一声,只见一个白发婆婆从里面出来。戴
宗看那婆婆,但见:

苍然古貌,鹤发酡颜。眼昏似秋月笼烟,眉白如晓霜映日。青裙素服,依稀紫
府元君;布袄荆钗,仿佛骊山老姥。形如天上翔云鹤,貌似山中傲雪松。

戴宗当下施礼道:“告禀老娘:小可欲求清道人相见一面。”婆婆问道:“官
人高姓?”戴宗道:“小可姓戴,名宗,从山东到此。”婆婆道:“孩儿出外云游,
不曾还家。”戴宗道:“小可是旧时相识,要说一句紧要的话,求见一面。”婆婆
道:“不在家里,有甚话说,留下在此不妨。待回家,自来相见。”戴宗道:“小
可再来。”就辞了婆婆,却来门外对李逵道:“今番须用着你。方才他娘说道,不
在家里,如今你可去请他。他若说不在时,你便打将起来,却不得伤犯他老母。我
来喝住,你便罢。”

李逵先去包裹里取出双斧,插在两胯下,入的门里,叫一声:“着个出来!”
婆婆慌忙迎着问道:“是谁?”见了李逵睁着双眼,先有八分怕他,问道:“哥哥
有甚话说?”李逵道:“我是梁山泊黑旋风。奉着哥哥将令,教我来请公孙胜。你
叫他出来,佛眼相看;若还不肯出来,放一把鸟火,把你家当都烧做白地。莫言不
是。早早出来!”婆婆道:“好汉莫要恁地。我这里不是公孙胜家,自唤做清道人。”
李逵道:“你只叫他出来,我自认得他鸟脸。”婆婆道:“出外云游未归。”李逵
拔出大斧,先砍翻一堵壁。婆婆向前拦住,李逵道:“你不叫你儿子出来,我只杀
了你。”拿起斧来便砍,把那婆婆惊倒在地。只见公孙胜从里面走将出来,叫道:
“不得无礼!”有诗为证:
药炉丹灶学神仙,遁迹深山了万缘。
不是凶神来屋里,公孙安肯出堂前。
戴宗便来喝道:“铁牛,如何吓倒老母!”戴宗连忙扶起。李逵撇了大斧,便唱个
喏道:“阿哥休怪。不恁地,你不肯出来。”公孙胜先扶娘入去了,却出来拜请戴
宗、李逵,邀进一间净室坐下,问道:“亏二位寻得到此。”戴宗道:“自从师父
下山之后,小可先来蓟州寻了一遍,并无打听处,只纠合得一伙弟兄上山。今次宋
公明哥哥,因去高唐州救柴大官人,致被知府高廉两三阵用妖法赢了,无计奈何,
只得教小可和李逵来寻请足下。绕遍蓟州,并无寻处。偶因素面店中,得个此间老
丈指引到此。却见村姑说足下在家烧炼丹药,老母只是推却,因此使李逵激出师父
来。这个太莽了些,望乞恕罪。哥哥在高唐州界上,度日如年。请师父便可行程,
以见始终成全大义之美。”公孙胜道:“贫道幼年飘荡江湖,多与好汉们相聚。自
从梁山泊分别回乡,非是昧心:一者母亲年老,无人奉侍;二乃本师罗真人留在屋
前,恐怕有人寻来,故改名清道人,隐藏在此。”戴宗道:“今者宋公明正在危急
之际,师父慈悲,只得去走一遭。”公孙胜道:“干碍老母无人养赡,本师罗真人
如何肯放。其实去不得了。”戴宗再拜恳告,公孙胜扶起戴宗,说道:“再容商议。”
公孙胜留戴宗、李逵在净室里坐定,安排些素酒素食相待。

三个吃了一回,戴宗又苦苦哀告道:“若是师父不肯去时,宋公明必被高廉捉
了,山寨大义,从此休矣!”公孙胜道:“且容我去禀问本师真人。若肯容许,便
一同去。”戴宗道:“只今便去启问本师。”公孙胜道:“且宽心住一宵,明日早
去。”戴宗道:“哥哥在彼一日,如度一年,烦请师父同往一遭。”公孙胜便起身,
引了戴宗、李逵,离了家里,取路上二仙山来。此时已是秋残冬初时分,日短夜长,
容易得晚,来到半山腰,却早红轮西坠。松阴里面一条小路,直到罗真人观前,见
有朱红牌额,上写三个金字,书着“紫虚观”。三人来到观前,看那二仙山时,果
然是好座仙境。但见:

青松郁郁,翠柏森森。一群白鹤听经,数个青衣碾药。青梧翠竹,洞门深锁碧
窗寒;白雪黄芽,石室云封丹灶暖。野鹿
衔花穿径去,山猿擎果度岩来。时闻道士谈经,每见仙翁论法。虚皇坛畔,天风吹
下步虚声;礼斗殿中,鸾背忽来环韵。只此便为真紫府,更于何处觅蓬莱?
三人就着衣亭上,整顿衣服,从廊下入来,径投殿后松鹤轩里去。两个童子,看见
公孙胜领人入来,报知罗真人,传法旨,教请三人入来。当下公孙胜引着戴宗、李
逵,到松鹤轩内,正值真人朝真才罢,坐在云床上。公孙胜向前行礼起居,躬身侍
立。戴宗、李逵看那罗真人时,端的有神游八极之表。但见:

星冠攒玉叶,鹤氅缕金霞。长髯广颊,修行到无漏之天;碧眼方瞳,服食造长
生之境。每啖安期之枣,曾尝方朔之桃。气满丹田,端的绿筋紫脑;名登玄,定
知苍肾青肝。正是三更步月鸾声远,万里乘云鹤背高。

戴宗当下见了,慌忙下拜。李逵只管着眼看。罗真人问公孙胜道:“此二位何
来?”公孙胜道:“便是昔日弟子曾告我师,山东义友是也。今为高唐州知府高廉
显逞异术,有兄宋江特令二弟来此,呼唤弟子。未敢擅便,故来禀问我师。”罗真
人道:“吾弟子既脱火坑,学炼长生,何得再慕此境?”戴宗再拜道:“容乞暂请
公孙先生下山,破了高廉,便送还山。”罗真人道:“二位不知:此非出家人闲管
之事。汝等自下山去商议。”

公孙胜只得引了二人,离了松鹤轩,连晚下山来。李逵问道:“那老仙先生说
甚么?”戴宗道:“你偏不听得?”李逵道:“便是不省得这般鸟则声。”戴宗道:
“便是他的师父说道教他休去。”李逵听了,叫起来道:“教我两个走了许多路程,
千难万难寻见了,却放出这个屁来!莫要引老爷性发,一只手捻碎你这道冠儿,一
只手提住腰胯,把那老贼道倒直撞下山去!”戴宗瞅着道:“你又要钉住了脚!”
李逵道:“不敢,不敢!我自这般说一声儿耍。”

三个再到公孙胜家里,当夜安排些晚饭吃了。公孙胜道:“且权宿一宵,明日
再去恳告本师。若肯时,便去。”戴宗至夜叫了安置,两个收拾行李,都来净室里
睡了。两个睡到五更左侧,李逵悄悄地爬将起来。听得戴宗的睡着,自己寻思
道:“却不是干鸟气么?你原是山寨里人,却来问甚么鸟师父!明朝那厮又不肯,却
不误了哥哥的大事?我忍不得了,只是杀了那个老贼道,教他没问处,只得和我去。”

李逵当时摸了两把板斧,悄悄地开了房门,乘着星月明朗,一步步摸上山来。
到得紫虚观前,却见两扇大门关了。旁边篱墙苦不甚高,李逵腾地跳将过去,开了
大门,一步步摸入里面来。直至松鹤轩前,只听隔窗有人看诵玉枢宝经之声。李逵
爬上来,舐破窗纸张时,见罗真人独自一个坐在云床上。面前桌儿上烧着一炉好香,
点着两枝画烛,朗朗诵经。李逵道:“这贼道却不是当死!”一踅踅过门边来,把
手只一推,呀的两扇亮齐开。李逵抢将入去,提起斧头,便望罗真人脑门上劈将
下来,砍倒在云床上,流出白血来。李逵看了,笑道:“眼见的这贼道是童男子身,
颐养得元阳真气,不曾走泄,正没半点的红。”李逵再仔细看时,连那道冠儿劈做
两半,一颗头直砍到项下。李逵道:“今番且除了一害,不烦恼公孙胜不去。”便
转身出了松鹤轩,从侧首廊下奔将出来,只见一个青衣童子拦住李逵,喝道:“你
杀了我本师,待走那里去!”李逵道:“你这个小贼道,也吃我一斧!”手起斧落,
把头早砍下台基边去。二人都被李逵砍了,李逵笑道:“只好撒开。”径取路出了
观门,飞也似奔下山来。到得公孙胜家里,闪入来,闭上了门,净室里听戴宗时,
兀自未觉,李逵依然原又去睡了。直到天明,公孙胜起来安排早饭,相待两个吃了。
戴宗道:“再请先生同引我二人上山,恳告真人。”李逵听了,暗暗地冷笑。

三个依原旧路,再上山来。入到紫虚观里松鹤轩中,见两个童子。公孙胜问道:
“真人何在?”童子答道:“真人坐在云床上养性。”李逵听说,吃了一惊,把舌
头伸将出来,半日缩不入去。三个揭起帘子,入来看时,见罗真人坐在云床上中间。
李逵暗暗想道:“昨夜莫非是错杀了?”罗真人便道:“汝等三人又来何干?”戴
宗道:“特来哀告我师慈悲,救取众人免难。”罗真人道:“这黑大汉是谁?”戴
宗答道:“是小可义弟,姓李,名逵。”真人笑道:“本待不教公孙胜去,看他的
面上,教他去走一遭。”戴宗拜谢,李逵自暗暗寻思道:“那厮知道我要杀他,却
又鸟说!”

只见罗真人道:“我教你三人片时便到高唐州如何?”三个谢了,戴宗寻思:
“这罗真人又强似我的神行法。”真人唤道童取三个手帕来。戴宗道:“上告我师:
却是怎生教我们便能够到高唐州?”罗真人便起身道:“都跟我来。”三个人随出
观门外石岩上来。先取一个红手帕,铺在石上道:“吾弟子可登。”公孙胜双脚在
上面,罗真人把袖一拂,喝声道:“起!”那手帕化做一片红云,载了公孙胜,冉
冉腾空便起,离山约有二十余丈。罗真人喝声:“住!”那片红云不动。却铺下一
个青手帕,教戴宗踏上。喝声:“起!”那手帕却化作一片青云,载了戴宗,起在
半空里去了。那两片青红二云,如芦席大,起在天上转,李逵看得呆了。罗真人却
把一个白手帕铺在石上,唤李逵踏上。李逵笑道:“你不是耍,若跌下来,好个大
疙瘩。”罗真人道:“你见二人么?”李逵立在手帕上,罗真人说一声“起!”那
手帕化做一片白云,飞将起去。李逵叫道:“阿呀!我的不稳,放我下来。”罗真
人把右手一招,那青红二云平平坠将下来。戴宗拜谢,侍立在面前,公孙胜侍立在
左手。李逵在上面叫道:“我也要撒尿撒屎,你不着我下来,我劈头便撒下来也!”
罗真人问道:“我等自是出家人,不曾恼犯了你,你因何夜来越墙而过,入来把斧
劈我?若是我无道德,已被杀了,又杀了我一个道童。”李逵道:“不是我,你敢
错认了?”罗真人笑道:“虽然只是砍了我两个葫芦,其心不善,且教你吃些磨难。”
把手一招喝声:“去!”一阵恶风,把李逵吹入云端里。只见两个黄巾力士,押着
李逵,耳边只听得风雨之声,不觉径到蓟州地界,得魂不着体,手脚摇战。忽听
得刮剌剌地响一声,却从蓟州府厅屋上骨碌碌滚将下来。

当日正值府尹马士弘坐衙,厅前立着许多公吏人等,看见半天里落下一个黑大
汉来,众皆吃惊。马知府见了,叫道:“且拿这厮过来!”当下十数个牢子狱卒,
把李逵驱至当面。马府尹喝道:“你这厮是那里妖人?如何从半天里吊将下来?”
李逵吃跌得头破额裂,半晌说不出话来。马知府道:“必然是个妖人,教去取些法
物来。”牢子节级将李逵捆翻,驱下厅前草地里,一个虞候,掇一盆狗血,没头一
淋;又一个提一桶尿粪来,望李逵头上直浇到脚底下。李逵口里、耳朵里,都是尿
屎。李逵叫道:“我不是妖人,我是跟罗真人的伴当。”原来蓟州人都知道罗真人
是个现世的活神仙,因此不肯下手伤他,再驱李逵到厅前,早有吏人禀道:“这蓟
州罗真人,是天下有名的得道活神仙。若是他的从者,不可加刑。”马府尹笑道:
“我读千卷之书,每闻今古之事,未见神仙有如此徒弟,即系妖人。牢子,与我加
力打那厮!”众人只得拿翻李逵,打得一佛出世,二佛涅。马知府喝道:“你那
厮快招了妖人,便不打你。”李逵只得招做“妖人李二。”取一面大枷钉了,押下
大牢里去。李逵来到死囚狱里,说道:“我是直日神将,如何枷了我?好歹教你这
蓟州一城人都死。”那押牢节级、禁子,都知罗真人道德清高,谁不钦服,都来问
李逵:“你端的是甚么人?”李逵道:“我是罗真人亲随直日神将,因一时有失,
恶了真人,把我撇在此间,教我受此苦难,三两日必来取我。你们若不把些酒食来
将息我时,我教你们众人全家都死。”那节级、牢子见了他说,倒都怕他,只得买
酒买肉请他吃。李逵见他们害怕,越说起风话来。牢里众人越怕了,又将热水来与
他洗浴了,换些干净衣裳。李逵道:“若还缺了我酒食,我便飞了去,教你们受苦。”
牢里禁子只得倒陪告他。李逵陷在蓟州牢里不提。

且说罗真人把上项的事,一一说与戴宗。戴宗只是苦苦哀告,求救李逵。罗真
人留住戴宗在观里宿歇,动问山寨里事务。戴宗诉说晁天王、宋公明仗义疏财,专
只替天行道,誓不损害忠臣烈士,孝子贤孙,义夫节妇,许多好处。罗真人听罢甚
喜。一住五日,戴宗每日磕头礼拜,求告真人,乞救李逵。罗真人道:“这等人只
可驱除了,休带回去。”戴宗告道:“真人不知:李逵虽是愚蠢,不省理法,也有
些小好处:第一,鲠直,分毫不肯苟取于人;第二,不会阿谄于人,虽死,其忠不
改;第三,并无淫欲邪心,贪财背义,敢勇当先。因此宋公明甚是爱他。不争没了
这个人回去,教小可难见兄长宋公明之面。”罗真人笑道:“贫道已知这人是上界
天杀星之数。为是下土众生作业太重,故罚他下来杀戮。吾亦安肯逆天,坏了此人?
只是磨他一会,我叫取来还你。”戴宗拜谢。

罗真人叫一声:“力士安在?”就鹤轩前起一阵风。风过处,一尊黄巾力士出
现。但见:

面如红玉,须似皂绒。仿佛有一丈身材,纵横有千斤气力。黄巾侧畔,金环日
耀喷霞光;绣袄中间,铁甲霜铺吞月影。常在坛前护法,每来世上降魔。
那个黄巾力士上告:“我师有何法旨?”罗真人道:“先差你押去蓟州的那人,罪
业已满。你还去蓟州牢里取他回来,速去速回。”力士声喏去了。约有半个时辰,
从虚空里把李逵撇将下来。

戴宗连忙扶住李逵,问道:“兄弟这两日在那里?”李逵看了罗真人,只管磕
头拜说道:“铁牛不敢了也!”罗真人道:“你从今已后,可以戒性,竭力扶持宋
公明,休生歹心。”李逵再拜道:“敢不遵依真人言语?”戴宗道:“你正去那里
走了这几日?”李逵道:“自那日一阵风,直刮我去蓟州府里,从厅屋脊上直滚下
来,被他府里众人拿住。那个马知府,道我是妖人,捉翻我捆了,却教牢子狱卒,
把狗血和尿屎,淋我一头一身;打得我两腿肉烂,把我枷了,下在大牢里去。众人
问我,是何神从天上落下来?我因说是罗真人的亲随直日神将,因有些过失,罚受
此苦。过二三日,必来取我。虽是吃了一顿棍棒,却也诈得些酒食,那厮们惧怕
真人,却与我洗浴,换了一身衣裳。方才正在亭心里诈酒肉吃,只见半空里跳下这
个黄巾力士,把枷锁开了,喝我闭眼,一似睡梦中,直扶到这里。”公孙胜道:“师
父似这般的黄巾力士,有一千余员,都是本师真人的伴当。”李逵听了叫道:“活
佛,你何不早说,免教我做了这般不是!”只顾下拜。戴宗也再拜恳告道:“小可
端的来的多日了,高唐州军马甚急,望乞师父慈悲,放公孙先生同弟子去救哥哥宋
公明,破了高廉,便送还山。”罗真人道:“我本不教他去,今为汝大义为重,权
教他去走一遭。我有片言,汝当记取。”公孙胜向前跪听真人指教。正是:满还济
世安邦愿,来作乘鸾跨凤人。

毕竟罗真人对公孙胜说出甚话来,且听下回分解。