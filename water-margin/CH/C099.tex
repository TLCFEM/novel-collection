\chapter{花和尚解脱缘缠井~混江龙水灌太原城}

话说田虎接得叶清申文,拆开付与近侍识字的:“读与寡人听。”书中说:“臣
邬梨招赘全羽为婿。此人十分骁勇,杀退宋兵,宋江等退守昭德府。臣邬梨即日再
令臣女郡主琼英,同全羽领兵恢复昭德城。谨遣总管叶清报捷,并以婚配事奉闻,
乞大王恕臣擅配之罪。”田虎听罢,减了七分忧色,随即传令,封全羽为中兴平南
先锋郡马之职,仍令叶清同两个伪指挥使,赍领令旨,及花红、锦缎、银两,到襄
垣县封赏郡马。叶清拜辞田虎,同两个伪指挥使,望襄垣进发,不题。
却说前日神行太保戴宗,奉宋公明将令,往各府州县,传遍军帖已毕,投汾阳府卢
俊义处探听去了。其各府州县新官,陆续已到。各路守城将佐,随即交与新官治理,
诸将统领军马,次第都到昭德府。第一队是卫州守将关胜、呼延灼,同壶关守将孙
立、朱仝、燕顺、马麟,抱犊山守将文仲容、崔,军马到来,入城参见陈安抚、
宋江已毕,说:“水军头领李俊探听得潞城已克,即同张横、张顺、阮小二、阮小
五、阮小七、童威、童猛,统驾水军船只,自卫河出黄河,由黄河到潞城县东潞水,
聚集听调。”当下宋江置酒叙阔。次日,令关胜、呼延灼、文仲容、崔,领兵马
到潞城,传令水军头领李俊等,“协同汝等及索超等人马,进兵攻取榆社、大谷等
县,抄出威胜州贼巢之后,不得疏虞!恐贼计穷,投降金人。”关胜等遵令去了。
次后,陵川县守城将士李应、柴进,高平县守城将士史进、穆弘,盖州守城将士花
荣、董平、杜兴、施恩,各各交代与新官,领军马到来,参见已毕,称说花荣等将,
在盖州镇守,北将山士奇从壶关战败,领了败残军士,纠合浮山县军马,来寇盖州,
被花荣等两路伏兵齐发,活擒山士奇,杀死二千余人,山士奇遂降。其余军将,四
散逃窜。当下花荣等引山士奇另参宋先锋,宋江令置酒接风相叙。宋江等军马,只
在昭德城中屯住,佯示惧怕张清、琼英之意,以坚田虎之心,不在话下。

且说卢俊义等已克汾阳府,田豹败走到孝义县,恰遇马灵兵到。那马灵是涿州
人,素有妖术:脚踏风火二轮,日行千里,因此人称他做神驹子;又有金砖法,打
人最是利害;凡上阵时,额上又现出一只妖眼,因此人又称他做小华光。术在乔道
清之下。他手下有偏将二员,乃是武能、徐瑾。那二将都学了马灵的妖术。当下马
灵与田豹合兵一处,统领武能、徐瑾、索贤、党世隆、凌光、段仁、苗成、陈宣,
并三万雄兵,到汾阳城北十里外扎寨。南军将佐,连日与马灵等交战不利。卢俊义
引兵退入汾阳城中,不敢与他厮杀,只愁北军来攻城池。正在纳闷,忽有守东门军
士飞报将来,说宋先锋特差公孙胜、乔道清,领兵马二千,前来助战。
卢俊义忙教开门请进。相见已毕,卢俊义揖公孙胜上坐,乔道清次之,置酒管待。
卢俊义诉说:“马灵术法利害,被他打伤了雷横、郑天寿、杨雄、石秀、焦挺、邹
渊、邹润、龚旺、丁得孙、石勇数员将佐。卢某正在束手无策,却得二位先生到此。”
乔道清说道:“小道与吾师为此禀过宋先锋,特到此拿他。”说还未毕,只见守城
军飞报将来,说马灵领兵杀奔东门来,武能、徐瑾领兵杀至西门,田豹同索贤、党
世隆、凌光、段仁领兵杀奔北门来。公孙胜听报,说道:“贫道出东门敌马灵,乔
贤弟出西门擒武能、徐瑾,卢先锋领兵出北门,迎敌田豹。”卢俊义又教黄信、杨
志、欧鹏、邓飞,四将统领兵马,助一清先生。当下戴宗闻马灵会神行,也要同公
孙胜出去,卢俊义依允。再令陈达、杨春、李忠、周通,领兵马助乔先生。卢俊义
同秦明、宣赞、郝思文、韩滔、彭,领兵出南门,迎敌田豹。当日汾阳城外,东
西北三面,旗幡蔽日,金鼓振天,同时厮杀。
不说卢俊义、乔道清两路厮杀,且说神驹子马灵,领兵摇旗擂鼓,辱骂搦战。只见
城门开处,放下吊桥,南军将佐,拥出城来,将军马一字儿排开,如长蛇之阵。马
灵纵马挺戟大喝道:“你们这伙鸟败汉,可速还俺们的城池!若稍延挨,教你片甲
不留!”欧鹏、邓飞两马并出,大喝道:“你的死期到了!”欧鹏拈铁枪,邓飞舞
铁链,二人拍马直抢马灵,马灵挺戟来迎。三将斗到十合之上,马灵手取金砖,正
欲望欧鹏打来。此时公孙胜已是骤马上前,仗剑作法。那时马灵手起,这边公孙胜
把剑一指,猛可的霹雳也似一声响亮,只见红光罩满,公孙胜满剑都是火焰,马灵
金砖堕地,就地一滚,即时消灭。公孙胜真个法术通灵,转眼间,南阵将士、军卒、
器械,浑身都是火焰,把一个长蛇阵,变的火龙相似。马灵金砖法,被公孙胜神火
克了。公孙胜把麈尾招动,军马首尾合杀拢来,北军大败亏输,杀得星落云散,七
断八续,军士三停内折了二停。马灵战败逃生,幸得会使神行法,脚踏风火二轮,
望东飞去。南阵里神行太保戴宗,已是拴缚停当甲马,也作起神行法,手挺朴刀,
赶将上去。顷刻间,马灵已去了二十余里,戴宗止行得十六七里,看看望不见马灵
了。前面马灵正在飞行,却撞着一个胖大和尚,劈面抢来,把马灵一禅杖打翻,顺
手牵羊,早把马灵擒住。
那和尚正在盘问马灵,戴宗早已赶到,只见和尚擒住马灵。戴宗上前看那和尚时,
却是花和尚鲁智深。戴宗惊问道:“吾师如何到这里?”鲁智深道:“这里是甚么
所在?”戴宗道:“此处是汾阳府城东郭。这个是北将马灵,适被公孙一清在阵上
破了妖法,小弟追赶上来。那厮行得快,却被吾师擒住,真个从天而降!”鲁智深
笑道:“洒家虽不是天上下来,也在地上出来。”当下二人缚了马灵,三人脚踏实
地,径望汾阳府来。戴宗再问鲁智深来历,鲁智深一头走,一头说道:“前日田虎
差一个鸟婆娘到襄垣城外厮杀。他也会飞石子,便将许多头领打伤,洒家在阵上杀
入去,正要拿那鸟婆娘,不提防茂草丛中,藏着一穴。洒家双脚落空,只一交颠下
穴去,半晌方到穴底,幸得不曾跌伤。洒家看穴中时,旁边又有一穴,透出亮光来。
洒家走进去观看,却是奇怪,一般有天有日,亦有村庄房舍。其中人民,也是在那
里忙忙的营干,见了洒家,都只是笑。洒家也不去问,也只顾抢入去。过了人烟辏
集的所在,前面静悄悄的旷野,无人居住。洒家行了多时,只见一个草庵,听的庵
中木鱼咯咯地响。洒家走进去看时,与洒家一般的一个和尚,盘膝坐地念经。洒家
问他的出路,那和尚答道:‘来从来处来,去从去处去。’洒家不省那两句话,焦
躁起来。那和尚笑道:‘你知道这个所在么?’洒家道:‘那里知道恁般鸟所在。’
那和尚又笑道:‘上至非非想,下至无间地,三千大千,世界广远,人莫能知。’
又道:‘凡人皆有心,有心必有念;地狱天堂,皆生于念。是故三界惟心,万法惟
识,一念不生,则六道俱销,轮回斯绝。’洒家听他这段话说得明白,望那和尚唱
了个大喏。那和尚大笑道:‘你一入缘缠井,难出欲迷天,我指示你的去路。’那
和尚便领洒家出庵,才走得三五步,便对洒家说道:‘从此分手,日后再会。’用
手向前指道:‘你前去可得神驹。’洒家回头,不见了那和尚,眼前忽的一亮,又
是一般景界,却遇着这个人。洒家见他走的蹊跷,被洒家一禅杖打翻,却不知为何
已到这里。此处节气,又与昭德府那边不同:桃李只有恁般大叶,却无半朵花蕊。”
戴宗笑道:“如今已是三月下旬,桃李多落尽了。”鲁智深不肯信,争让道:“如
今正是二月下旬,适才落井,只停得一回儿,却怎么便是三月下旬?”戴宗听说,
十分惊异。二人押着马灵,一径来到汾阳城。
此时公孙胜已是杀退北军,收兵入城。卢俊义、秦明、宣赞、郝思文、韩滔、彭,
杀了索贤、党世隆、凌光三将,直追田彪、段仁至十里外,杀散北军。田彪同段仁、
陈宣、苗成,领败残兵,望北去了。卢俊义收兵回城,又遇乔道清破了武能、徐瑾,
同陈达、杨春、李忠、周通,领兵追赶到来。被南军两路合杀,北兵大败,死者甚
众。武能被杨春一大杆刀砍下马来,徐瑾被郝思文刺死,夺获马匹、衣甲、金鼓、
鞍辔无数。卢俊义与乔道清合兵一处,奏凯进城。卢俊义刚到府治,只见鲁智深、
戴宗将马灵解来。卢俊义大喜,忙问:“鲁智深为何到此?宋哥哥与邬梨那厮厮杀,
胜败如何?”鲁智深再将前面堕井及宋江与邬梨交战的事,细述一遍,卢俊义以下
诸将,惊讶不已。
当下卢俊义亲释马灵之缚。马灵在路上已听了鲁智深这段话,又见卢俊义如此意气,
拜伏愿降。卢俊义赏劳三军将士。次日,晋宁府守城将佐,已有新官交代,都到汾
阳听用。卢俊义教戴宗、马灵往宋先锋处报捷,即日与副军师朱武计议征进,不题。
且说马灵传受戴宗日行千里之法,二人一日便到宋先锋军前,入寨参见,备细报捷。
宋江听了鲁智深这段话,惊讶喜悦,亲自到陈安抚处,参见报捷,不在话下。
再说田豹同段仁、陈宣、苗成统领败残军卒,急急如丧家之狗,忙忙似漏网之鱼,
到威胜见田虎,哭诉那丧师失地之事。又有伪枢密院官急入内启奏道:“大王,两
日流星报马,将羽书雪片也似报来,说统军大将马灵已被擒拿;关胜、呼延灼兵马,
已围榆社县;卢俊义等兵马,已破介休县城池;独有襄垣县邬国舅处,屡有捷音,
宋兵不敢正视。”田虎闻报大惊,手足无措。文武多官计议,欲北降金人。当有伪
右丞相太师卞祥,叱退多官,启奏道:“宋兵纵有三路,我这威胜,万山环列,粮
草足支二年,御林卫驾等精兵二十余万。东有武乡,西有沁源二县,各有精兵五万。
后有太原县、祈县、临县、大谷县,城池坚固,粮草充足,尚可战守。古语有云:
‘宁为鸡口,无为牛后。’”
田虎踌躇未答,又报总管叶清到来。田虎即令召进,叶清拜舞毕,称说:“郡主郡
马屡次斩获,兵威大振,兵马直抵昭德府。正要围城,因邬国舅偶患风寒,不能管
摄兵马。乞大王添差良将精兵,协助郡主郡马,恢复昭德府。”当有伪都督范权启
奏道:“臣闻郡主郡马,甚是骁勇,宋兵不敢正视。若得大王御驾亲征,又有雄兵
猛将助他,必成中兴大功。臣愿助太子监国。”田虎准奏。原来范权之女,有倾国
之姿。范权献与田虎,田虎十分宠幸。因此,范权说的,无有不从。今日范权受了
叶清重赂,又见宋兵势大,他便乘机卖国。
当下田虎拨付卞祥将佐十员,精兵三万,前往迎敌卢俊义、花荣等兵马。又令伪太
尉房学度,也统领将佐十员,精兵三万,往榆社迎敌关胜等兵马。田虎亲自统领伪
尚书李天锡、郑之瑞、枢密薛时、林昕、都督胡英、唐昌,及殿帅、御林护驾教头、
团练使、指挥使、将军、较尉等众,挑选精兵十万,择日祭旗兴师,杀牛宰马,犒
赏三军。再传令旨,教兄弟田豹、田彪同都督范权等,及文武多官,辅太子田定监
国。叶清得了这个消息,密差心腹,星夜驰至襄垣城中,报知张清、琼英。张清令
解珍、解宝将绳索悬挂出城,星夜往报宋先锋知会去了。
却说卞祥伺候兵符,挑选军马,盘桓了三日,方才统领樊玉明、鱼得源、傅祥、顾
恺、寇琛、管琰、冯翊、吕振、吉文炳、安士隆等偏牙各项将佐,军马三万,出了
威胜州东门。军分两队,前队是樊玉明、鱼得源、冯翊、顾恺,领兵马五千。刚到
沁源县,地名绵山,山坡下一座大林,前军却好抹过林子,只听得一棒锣声响处,
林子背后山坡脚边,撞出一彪军来,却是宋公明得了张清消息,密差花荣、董平、
林冲、史进、杜兴、穆弘,领精勇骑兵五千,人披软战,马摘銮铃,星夜疾驰到此。
军中一将,骤马当先,两手两杆钢枪。此将乃是宋军中第一个惯冲头阵的双枪将
董平,大喝道:“来的是那里兵马?不早早受缚,更待何时?”樊玉明大骂:“水
洼草寇,何故侵夺俺这里城池?”董平大怒,喝道:“天兵到此,兀是抗拒!”拍
马挺双枪,直抢樊玉明。那边樊玉明纵马拈枪来迎。二将斗到二十余合,樊玉明力
怯,遮架不住,被董平一枪,刺中咽喉,翻身落马。那边冯翊大怒,挺条浑铁枪,
飞马直抢董平。那边小李广花荣,骤马接住厮杀。二将斗到十合之上,花荣拨马,
望本阵便走。冯翊纵马赶来,却被花荣带住花枪,拈弓搭箭,扯得那弓满满的,扭
转身躯,觑定冯翊较亲,只一箭,正中冯翊面门,头盔倒卓,两脚蹬空,扑通的撞
下马来。花荣拨转马,再一枪,结果了性命。董平、林冲、史进、穆弘、杜兴,招
动兵马,一齐卷杀过来。顾恺早被林冲搠翻。鱼得源堕马,被人马践踏身死。北兵
大败亏输,五千军马,杀死大半,其余四散逃窜。花荣等兵士,夺了金鼓马匹,追
杀北兵,至五里外,却遇卞祥大兵到来。
那卞祥是庄家出身,他两条臂膊,有水牛般气力,武艺精熟,乃是贼中上将。当下
两军相对,旗鼓相望,两阵里画角齐鸣,鼍鼓迭擂。北将卞祥,立马当先,头顶凤
翅金盔,身挂鱼鳞银甲,九尺长短身材,三牙掩口髭须,面方肩阔,眉竖眼圆,跨
匹冲波战马,提把开山大斧。左右两边,排着傅祥、管琰、寇琛、吕振四个伪统制
官;后面又有伪统军、提辖、兵马防御、团练等官,参随在后。队伍军马,十分摆
布得整齐。南阵里九纹龙史进骤马出阵,大喝:“来将何人?快下马受缚,免污刀
斧!”卞祥呵呵大笑道:“瓶儿罐儿,也有两个耳朵。你须曾闻得我卞祥的名字么?”
史进喝道:“助逆匹夫,天兵到此,兀是抗拒!”拍马舞三尖两刃八环刀,直抢卞
祥。卞祥也抡大斧来迎。二马相交,两器并举,刀斧纵横,马蹄撩乱,斗到三十余
合,不分胜败。这边花荣爱卞祥武艺高强,却不肯放冷箭,只拍马挺枪,上前助战。
卞祥力敌二将,又斗了三十余合,不分胜败。北阵中将士,恐卞祥有失,急鸣金收
兵。花荣、董平,见天色已晚,又寡不敌众,也不追赶,亦收兵向南,两军自去十
余里扎寨。
是夜南风大作,浓云泼墨,夜半,大雨震雷。此时田虎统领众多官员将佐军马,已
离了威胜城池百余里,天晚扎寨。帐中自有随行军中内侍姬妾,及范美人在帐中欢
宴。是夜也遇了大雨。自此霖雨一连五日不止,上面张盖的天雨盖都漏,下面又是
水渌渌的,军士不好炊爨立脚,角弓软,箭翎脱,各营军马,都在营中兀守,不在
话下。
且说索超、徐宁、单廷、魏定国、汤隆、唐斌、耿恭等将,接得关胜、呼延灼、
文仲容、崔陆兵,及水军头领李俊等水军船只,众将计议,留单廷、魏定国镇
守潞城,关胜等将佐水陆并进,船骑同行,打破榆社县,再留索超、汤隆,镇守城
池。关胜等众,乘胜长驱,势如破竹,又克了大谷县,杀了守城将佐,其余牙将军
兵,降者无算。关胜安抚军民,赏劳将士,差人到宋先锋处报捷。次日,关胜等同
时也遇了大雨,在城屯扎,不能前进。忽报:“卢先锋留下宣赞、郝思文、吕方、
郭盛,管领兵马,镇守汾阳府。卢俊义等已克了介休、平遥两县,再留韩滔、彭
镇守介休县,孔明、孔亮镇守平遥县,卢先锋统领众多将佐军马,现围太原县城池,
也因雨阻,不能攻打。”恰好水军头领李俊在城,听了此报,忙对关胜说道:“卢
先锋等今遇天雨连绵,流水大至,使三军不得稽留,倘贼人选死士出城冲击,奈何!
小弟有一计,欲到卢先锋处商议。”关胜依允。
当下混江龙李俊,即刻辞了关胜出城,教童威、童猛统管水军船只,自己同了二张、
三阮,带领水军二千,戴笠披蓑,冒雨冲风,间道疾驰到卢俊义军前,入寨参见。
不及寒温,即与卢俊义密语片晌。卢俊义大喜,随即传令军士,冒雨砍木作筏,李
俊等分头行事去了,不题。
且说太原城中守城将士张雄,伪授殿帅之职,项忠、徐岳,伪授都统制之职,这三
个人是贼中最好杀的。手下军卒,个个凶残淫暴,城中百姓,受暴虐不过,弃了家
产,四散逃亡,十停中已去了七八停。张雄等今被大兵围困,负固不服。张雄与项
忠、徐岳计议:目今天雨,宋兵欲掠无所,水地不利,薪刍既寡,军无稽留之心,
急出击之,必获全胜。此时是四月上旬,张雄正欲分兵出四门,冲击宋兵,忽听得
四面锣声振响。张雄忙上敌楼望城外时,只见宋军冒雨穿屐,俱登高阜山冈。张雄
正在惊疑,又听得智伯渠边,及东西三处,喊声振天,如千军万马狂奔驰骤之声。
霎时间,洪波怒涛飞至,却如秋中八月潮汹涌,天上黄河水泻倾。真个是:功过智
伯城三板,计胜淮阴沙几囊。
毕竟不知这水势如何底止,且听下回分解。