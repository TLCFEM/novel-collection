\chapter{燕青智扑擎天柱~李逵寿张乔坐衙}

话说这燕青,他虽是三十六星之末,却机巧心灵,多见广识,了身达命,都强
似那三十五个。当日燕青禀宋江道:“小乙自幼跟着卢员外学得这身相扑,江湖上
不曾逢着对手,今日幸遇此机会,三月二十八日又近了,小乙并不要带一人,自去
献台上,好歹攀他攧一交。若是输了攧死,永无怨心;倘或赢时,也与哥哥增些光
彩。这日必然有一场好闹,哥哥却使人救应。”宋江说道:“贤弟,闻知那人身长
一丈,貌若金刚,约有千百斤气力,你这般瘦小身材,纵有本事,怎地近傍得他?”
燕青道:“不怕他长大身材,只恐他不着圈套。常言道:‘相扑的有力使力,无力
斗智。’非是燕青敢说口,临机应变,看景生情,不倒的输与他那呆汉。”卢俊义
便道:“我这小乙,端的自小学成好一身相扑,随他心意,叫他去。至期,卢某自
去接应他回来。”宋江问道:“几时可行?”燕青答道:“今日是三月二十四日了,
来日拜辞哥哥下山,路上略宿一宵,二十六日赶到庙上,二十七日在那里打探一日,
二十八日却好和那厮放对。”当日无事,次日宋江置酒与燕青送行。众人看燕青时,
打扮得村村朴朴,将一身花绣,把衲袄包得不见。扮做山东货郎,腰里插着一把串
鼓儿,挑一条高肩杂货担子,诸人看了都笑。宋江道:“你既然装做货郎担儿,你
且唱个山东货郎转调歌与我众人听。”燕青一手拈串鼓,一手打板,唱出货郎太平
歌,与山东人不差分毫来去,众人又笑。酒至半酣,燕青辞了众头领下山,过了金
沙滩,取路往泰安州来。

当日天晚,正待要寻店安歇,只听得背后有人叫道:“燕小乙哥,等我一等!”
燕青歇下担子看时,却是黑旋风李逵。燕青道:“你赶来怎地?”李逵道:“你相
伴我去荆门镇走了两遭,我见你独自个来,放心不下,不曾对哥哥说知,偷走下山,
特来帮你。”燕青道:“我这里用你不着,你快早早回去。”李逵焦躁起来,说道:
“你便是真个了得的好汉,我好意来帮你,你倒翻成恶意!我却偏鸟要去!”燕青
寻思,怕坏了义气,便对李逵说道:“和你去。不争那里圣帝生日,都是四山五岳
的人聚会,认得你的颇多,你依的我三件事,便和你同去。”李逵道:“依得。”
燕青道:“从今路上和你前后各自走,一脚到客店里,入得店门,你便自不要出来,
这是第一件了。第二件,到得庙上客店里,你只推病,把被包了头脸,假做打睡,
更不要做声。第三件,当日庙上,你挨在稠人中看争交时,不要大惊小怪。大哥,
依得么?”李逵道:“有甚难处!都依你便了。”

当晚,两个投客店安歇。次日五更起来,还了房钱,同行到前面打火吃了饭。
燕青道:“李大哥,你先走半里,我随后来也。”那条路上,只见烧香的人来往不
绝,多有讲说任原的本事,两年在泰岳无对,今年又经三年了。燕青听得,有在心
里。申牌时候,将近庙上,旁边众人都立定脚,仰面在那里看。燕青歇下担儿,分
开人丛,也挨向前看时,只见两条红标柱,恰与坊巷牌额一般相似,上立一面粉牌,
写道:“太原相扑擎天柱任原。”旁边两行小字道:“拳打南山猛虎,脚踢北海苍
龙。”燕青看了,便扯匾担,将牌打得粉碎,也不说什么,再挑了担儿,望庙上去
了。看的众人,多有好事的,飞报任原说,今年有劈牌放对的。

且说燕青前面迎着李逵,便来寻客店安歇。原来庙上好生热闹,不算一百二十
行经商买卖,只客店也有一千四五百家,延接天下香官。到菩萨圣节之时,也没安
着人处,许多客店,都歇满了。燕青、李逵只得就市梢头赁一所客店安下,把担子
歇了,取一床夹被,教李逵睡着。店小二来问道:“大哥是山东货郎,来庙上赶趁,
怕敢出房钱不起?”燕青打着乡谈说道:“你好小觑人!一间小房,值得多少,便
比一间大房钱,没处去了。别人出多少房钱,我也出多少还你。”店小二道:“大
哥休怪,正是要紧的日子,先说得明白最好。”燕青道:“我自来做买卖,倒不打
紧,那里不去歇了,不想路上撞见了这个乡中亲戚,现患气病,因此只得要讨你店
中歇。我先与你五贯铜钱,央及你就锅中替我安排些茶饭,临起身一发酬谢你。”
小二哥接了铜钱,自去门前安排茶饭,不在话下。

没多时候,只听得店门外热闹,二三十条大汉走入店里来,问小二哥道:“劈
牌定对的好汉,在那房里安歇?”店小二道:“我这里没有。”那伙人道:“都说
在你店中。”小二哥道:“只有两眼房,空着一眼,一眼是个山东货郎,扶着一个
病汉赁了。”那一伙人道:“正是那个货郎儿劈牌定对。”店小二道:“休道别人
取笑!那货郎儿是一个小小后生,做得甚用!”那伙人齐道:“你只引我们去张一
张。”店小二指道:“那角落头房里便是。”众人来看时,见紧闭着房门,都去窗
子眼里张时,见里面床上两个人脚厮抵睡着。众人寻思不下,数内有一个道:“既
是敢来劈牌,要做天下对手,不是小可的人,怕人算他,以定是假装害病的。”众
人道:“正是了,都不要猜,临期便见。”不到黄昏前后,店里何止三二十伙人来
打听,分说得店小二口唇也破了。当晚搬饭与二人吃,只见李逵从被窝里钻出头来,
小二哥见了,吃一惊,叫声:“阿呀!这个是争交的爷爷了!”燕青道:“争交的
不是他,他自病患在身,我便是径来争交的。”小二哥道:“你休要瞒我,我看任
原吞得你在肚里。”燕青道:“你休笑我,我自有法度,教你们大笑一场,回来多
把利物赏你。”小二哥看着他们吃了晚饭,收了碗碟,自去厨头洗刮,心中只是不
信。

次日,燕青和李逵吃了些早饭,分付道:“哥哥,你自拴了房门高睡。”燕青
却随了众人,来到岱岳庙里看时,果然是天下第一。但见:

庙居泰岱,山镇乾坤。为山岳之至尊,乃万神之领袖。山
头伏槛,直望见弱水蓬莱;绝顶攀松,尽都是密云薄雾。楼台森耸,疑是金乌展翅
飞来;殿阁棱层,恍觉玉兔腾身走到。雕梁画栋,碧瓦朱檐。凤扉亮映黄纱,龟
背绣帘垂锦带。遥观圣象,九旒冕舜目尧眉;近睹神颜,衮龙袍汤肩禹背。九天司
命,芙蓉冠掩映绛纱衣;炳灵圣公,赭黄袍偏称蓝田带。左侍下玉簪珠履,右侍下
紫绶金章。阖殿威严,护驾三千金甲将;两廊猛勇,勤王十万铁衣兵。五岳楼相接
东宫,仁安殿紧连北阙。蒿里山下,判官分七十二司;白骡庙中,土神按二十四气。
管火池铁面太尉,月月通灵;掌生死五道将军,年年显圣。御香不断,天神飞马报
丹书;祭祀依时,老幼望风皆获福。嘉宁殿祥云杳霭,正阳门瑞气盘旋。万民朝拜
碧霞君,四远归依仁圣帝。
当时燕青游玩了一遭,却去草参亭参拜了四拜,问烧香的道:“这相扑任教师在那
里歇?”便有好事人说:“在迎恩桥下那个大客店里便是,他教着二三百个上足徒
弟。”燕青听了,径来迎恩桥下看时,见桥边栏杆子上坐着二三十个相扑子弟,面
前遍插铺金旗牌,锦绣帐额,等身靠背。燕青闪入客店里去,看见任原坐在亭心上,
真乃有揭谛仪容,金刚貌相。坦开胸脯,显存孝打虎之威;侧坐胡床,有霸王拔山
之势。在那里看徒弟相扑。数内有人认得燕青曾劈牌来,暗暗报与任原。只见任原
跳将起来,着膀子,口里说道:“今年那个合死的,来我手里纳命。”燕青低了
头,急出店门,听得里面都笑。急回到自己下处,安排些酒食,与李逵同吃了一回。
李逵道:“这们睡,闷死我也!”燕青道:“只有今日一晚,明日便见雌雄。”当
时闲话,都不必说。

三更前后,听得一派鼓乐响,乃是庙上众香官与圣帝上寿。四更前后,燕青、
李逵起来,问店小二先讨汤洗了面,梳光了头,脱去了里面衲袄,下面牢拴了腿绷
护膝,匾扎起了熟绢水裈,穿了多耳麻鞋,上穿汗衫,搭膊系了腰。两个吃了早饭,
叫小二分付道:“房中的行李,你与我照管。”店小二应道:“并无失脱,早早得
胜回来。”只这小客店里,也有三二十个烧香的,都对燕青道:“后生,你自斟酌,
不要枉送了性命。”燕青道:“当下小人喝采之时,众人可与小人夺些利物。”众
人都有先去了的。李逵道:“我带了这两把板斧去也好。”燕青道:“这个却使不
得,被人看破,误了大事。”当时两个杂在人队里,先去廊下,做一块儿伏了。

那日烧香的人,真乃亚肩迭背,偌大一个东岳庙,一涌便满了,屋脊梁上都是
看的人。朝着嘉宁殿,扎缚起山棚,棚上都是金银器皿,锦绣缎匹,门外拴着五头
骏马,全付鞍辔。知州禁住烧香的人,看这当年相扑献圣。一个年老的部署,拿着
竹批,上得献台,参神已罢,便请今年相扑的对手,出马争交。说言未了,只见人
如潮涌,却早十数对哨棒过来,前面列着四把绣旗。那任原坐在轿上,这轿前轿后
三二十对花膊的好汉,前遮后拥,来到献台上。部署请下轿来,开了几句温暖的
呵会。任原道:“我两年到岱岳,夺了头筹,白白拿了若干利物,今年必用脱膊。”
说罢,见一个拿水桶的上来。任原的徒弟,都在献台边,一周遭都密密地立着。且
说任原先解了膊,除了巾帻,虚笼着蜀锦袄子,喝了一声参神喏,受了两口神水,
脱下锦袄,百十万人齐喝一声采。看那任原时,怎生打扮:

头绾一窝穿心红角子,腰系一条绛罗翠袖。三串带儿拴十二个玉蝴蝶牙子扣儿,
主腰上排数对金鸳鸯踅褶衬衣。护膝中有铜裆铜裤,缴臁内有铁片铁环。扎腕牢拴,
踢鞋紧系。世间架海擎天柱,岳下降魔斩将人。

那部署道:“教师两年在庙上不曾有对手,今年是第三番了,教师有甚言语,
安复天下众香官?”任原道:“四百座军州,七千余县治,好事香官,恭敬圣帝,
都助将利物来,任原两年白受了,今年辞了圣帝还乡,再也不上山来了。东至日出,
西至日没,两轮日月,一合乾坤,南及南蛮,北济幽燕,敢有出来和我争利物的么?”
说犹未了,燕青捺着两边人的肩臂,口中叫道:“有,有!”从人背上直飞抢到献
台上来。众人齐发声喊。那部署接着问道:“汉子,你姓甚名谁?那里人氏?你从何
处来?”燕青道:“我是山东张货郎,特地来和他争利物。”那部署道:“汉子,
性命只在眼前,你省得么?你有保人也无?”燕青道:“我就是保人,死了要谁偿
命?”部署道:“你且脱膊下来看。”燕青除了头巾,光光的梳着两个角儿,脱下
草鞋,赤了双脚,蹲在献台一边,解了腿绷护膝,跳将起来,把布衫脱将下来,吐
个架子,则见庙里的看官如搅海翻江相似,迭头价喝采,众人都呆了。

任原看了他这花绣,急健身材,心里倒有五分怯他。殿门外月台上本州太守坐
在那里弹压,前后皂衣公吏环立七八十对,随即使人来叫燕青下献台,来到面前。
太守见了他这身花绣,一似玉亭柱上铺着软翠,心中大喜,问道:“汉子,你是那
里人氏?因何到此?”燕青道:“小人姓张,排行第一,山东莱州人氏。听得任原
搦天下人相扑,特来和他争交。”知州道:“前面那匹全副鞍马,是我出的利物,
把与任原;山棚上应有物件,我主张分一半与你,你两个分了罢,我自抬举你在我
身边。”

燕青道:“相公,这利物倒不打紧,只要攧翻他,教众人取笑,图一声喝采。”
知州道:“他是一个金刚般一条大汉,你敢近他不得!”燕青道:“死而无怨。”
再上献台来,要与任原定对。部署问他先要了文书,怀中取出相扑社条,读了一遍,
对燕青道:“你省得么?不许暗算。”燕青冷笑道:“他身上都有准备,我单单只
这个水裈儿,暗算他甚么?”知州又叫部署来分付道:“这般一个汉子,俊俏后生,
可惜了!你去与他分了这扑。”部署随即上献台,又对燕青道:“汉子,你留了性
命还乡去罢,我与你分了这扑。”燕青道:“你好不晓事,知是我赢我输!”众人
都和起来。只见分开了数万香官,两边排得似鱼鳞一般,廊庑屋脊上也都坐满,只
怕遮着了这对相扑。任原此时有心恨不得把燕青丢去九霄云外,跌死了他。部署道:
“既然你两个要相扑,今年且赛这对献圣。都要小心着,各各在意。”净净地献台
上只三个人,此时宿露尽收,旭日初起,部署拿着竹批,两边分付已了,叫声:“看
扑!”

这个相扑,一来一往,最要说得分明,说时迟,那时疾,正如空中星移电掣相
似,些儿迟慢不得。当时燕青做一块儿蹲在右边,任原先在左边立个门户,燕青只
不动弹。初时献台上各占一半,中间心里合交。任原见燕青不动弹,看看逼过右边
来,燕青只瞅他下三面。任原暗忖道:“这人必来算我下三面。你看我不消动手,
只一脚踢这厮下献台去。”任原看看逼将入来,虚将左脚卖个破绽,燕青叫一声:
“不要来!”任原却待奔他,被燕青去任原左胁下穿将过去。任原性起,急转身又
来拿燕青,被燕青虚跃一跃,又在右胁下钻过去。大汉转身终是不便,三换换得脚
步乱了。燕青却抢将入去,用右手扭住任原,探左手插入任原交裆,用肩胛顶住他
胸脯,把任原直托将起来,头重脚轻,借力便旋四五旋,旋到献台边,叫一声:“下
去!”把任原头在下,脚在上,直撺下献台来。这一扑,名唤做鹁鸽旋,数万的香
官看了,齐声喝采。那任原的徒弟们见攧翻了他师父,先把山棚拽倒,乱抢了利物。
众人乱喝打时,那二三十徒弟抢入献台来,知州那里治押得住。不想旁边恼犯了这
个太岁,却是黑旋风李逵看见了,睁圆怪眼,倒竖虎须,面前别无器械,便把杉剌
子葱般拔断,拿两条杉木在手,直打将来。

香官数内有人认得李逵的,说将出名姓来,外面做公人的齐入庙里大叫道:“休
教走了梁山泊黑旋风!”那知府听得这话,从顶门上不见了三魂,脚底下疏失了七
魄,便望后殿走了。四下里的人涌并围将来,庙里香官各自奔走。李逵看任原时,
跌得昏晕,倒在献台边,口内只有些游气。李逵揭块石板,把任原头打得粉碎。两
个从庙里打将出来,门外弓箭乱射入来,燕青、李逵只得爬上屋去,揭瓦乱打。

不多时,只听得庙门前喊声大举,有人杀将入来。当头一个,头戴白范阳毡笠
儿,身穿白缎子袄,跨口腰刀,挺条朴刀,那汉是北京玉麒麟卢俊义。后面带着史
进、穆弘、鲁智深、武松、解珍、解宝七筹好汉,引一千余人,杀开庙门,入来策
应。燕青、李逵见了,便从屋上跳将下来,跟着大队便走。李逵便去客店里拿了双
斧,赶来厮杀。这府里整点得官军来时,那伙好汉已自去得远了。官兵已知梁山泊
人众难敌,不敢来追赶。却说卢俊义便叫收拾李逵回去,行了半日,路上又不见了
李逵。卢俊义又笑道:“正是招灾惹祸,必须使人寻他上山。”穆弘道:“我去寻
他回寨。”卢俊义道:“最好。”

且不说卢俊义引众还山,却说李逵手持双斧,直到寿张县。当日午衙方散,李
逵来到县衙门口,大叫入来:“梁山泊黑旋风爹爹在此!”吓得县中人手足都麻木
了,动弹不得。原来这寿张县贴着梁山泊最近,若听得“黑旋风李逵”五个字,端
的医得小儿夜啼惊哭,今日亲身到来,如何不怕!当时李逵径去知县椅子上坐了,
口中叫道:“着两个出来说话,不来时,便放火!”廊下房内众人商量:“只得着
几个出去答应;不然,怎地得他去?”数内两个吏员出来厅上拜了四拜,跪着道:
“头领到此,必有指使。”李逵道:“我不来打搅你县里人,因往这里经过,闲耍
一遭,请出你知县来,我和他厮见。”两个去了,出来回话道:“知县相公却才见
头领来,开了后门,不知走往那里去了。”

李逵不信,自转入后堂房里来寻,却见有那幞头衣衫匣子在那里放着。李逵扭
开锁,取出幞头,插上展角,将来戴了,把绿袍公服穿上,把角带系了,再寻皂靴,
换了麻鞋,拿着槐简,走出厅前,大叫道:“吏典人等都来参见!”众人没奈何,
只得上去答应。李逵道:“我这般打扮也好么?”众人道:“十分相称。”李逵道:
“你们令史祗候都与我排衙了便去;若不依我,这县都翻做白地。”众人怕他,只
得聚集些公吏人来,擎着牙杖骨朵,打了三通擂鼓,向前声喏。

李逵呵呵大笑,又道:“你众人内也着两个来告状。”吏人道:“头领坐在此
地,谁敢来告状?”李逵道:“可知人不来告状,你这里自着两个装做告状的来告。
我又不伤他,只是取一回笑耍。”公吏人等商量了一会,只得着两个牢子装做厮打
的来告状,县门外百姓都放来看。两个跪在厅前,这个告道:“相公可怜见,他打
了小人。”那个告:“他骂了小人,我才打他。”李逵道:“那个是吃打的?”原
告道:“小人是吃打的。”又问道:“那个是打了他的?”被告道:“他先骂了,
小人是打他来。”李逵道:“这个打了人的是好汉,先放了他去。这个不长进的,
怎地吃人打了,与我枷号在衙门前示众。”

李逵起身,把绿袍抓扎起,槐简揣在腰里,掣出大斧,直看着枷了那个原告人,
号令在县门前,方才大踏步去了,也不脱那衣靴。县门前看的百姓,那里忍得住笑。
正在寿张县前走过东,走过西,忽听得一处学堂读书之声,李逵揭起帘子,走将入
去,吓得那先生跳窗走了。众学生们哭的哭,叫的叫,跑的跑,躲的躲。李逵大笑,
出门来,正撞着穆弘。穆弘叫道:“众人忧得你苦,你却在这里风!快上山去!”
那里由他,拖着便走。李逵只得离了寿张县,径奔梁山泊来。有诗为证:
牧民县令每猖狂,自幼先生教不良。
应遣铁牛巡历到,琴堂闹了闹书堂。

二人渡过金沙滩,来到寨里,众人见了李逵这般打扮都笑。到得忠义堂上,宋
江正与燕青庆喜,只见李逵放下绿袍,去了双斧,摇摇摆摆,直至堂前,执着槐
简,来拜宋江。拜不得两拜,把这绿袍踏裂,绊倒在地,众人都笑。宋江骂道:
“你这厮忒大胆!不曾着我知道,私走下山,这是该死的罪过!但到处便惹起事端,
今日对众弟兄说过,再不饶你!”李逵喏喏连声而退。梁山泊自此人马平安,都无
甚事,每日在山寨中教演武艺,操练人马,令会水者上船习学。各寨中添造军器、
衣袍、铠甲、枪刀、弓箭、牌弩、旗帜,不在话下。

且说泰安州备将前事申奏东京,进奏院中,又有收得各处州县申奏表文,皆为
宋江等反乱,骚扰地方。此时道君皇帝有一个月不曾临朝视事,当日早朝,正是三
下静鞭鸣御阙,两班文武列金阶,殿头官喝道:“有事出班早奏,无事卷帘退朝。”
进奏院卿出班奏曰:“臣院中收得各处州县累次表文,皆为宋江等部领贼寇,公然
直进府州,劫掠库藏,抢掳仓廒,杀害军民,贪厌无足,所到之处,无人可敌。若
不早为剿捕,日后必成大患。”天子乃云:“上元夜此寇闹了京国,今又往各处骚
扰,何况那里附近州郡。朕已累次差遣枢密院进兵,至今不见回奏。”旁有御史大
夫崔靖出班奏曰:“臣闻梁山泊上立一面大旗,上书‘替天行道’四字,此是曜民
之术。民心既服,不可加兵。即目辽兵犯境,各处军马遮掩不及,若要起兵征伐,
深为不便。以臣愚意,此等山间亡命之徒,皆犯官刑,无路可避,遂乃啸聚山林,
恣为不道。若降一封丹诏,光禄寺颁给御酒珍羞,差一员大臣,直到梁山泊,好言
抚谕,招安来降,假此以敌辽兵,公私两便。伏乞陛下圣鉴。”天子云:“卿言甚
当,正合朕意。”便差殿前太尉陈宗善为使,赍擎丹诏御酒,前去招安梁山泊大小
人数。是日朝散,陈太尉领了诏敕,回家收拾。不争陈太尉奉诏招安,有分教:香
醪翻做烧身药,丹诏应为引战书。

毕竟陈太尉怎地来招安宋江,且听下回分解。