\chapter{宋江兵打北京城~关胜议取梁山泊}

话说当时石秀和卢俊义两个,在城内走投没路,四下里人马合来,众做公的把
挠钩搭住,套索绊翻,可怜悍勇英雄,方信寡不敌众。两个当下尽被捉了,解到梁
中书面前,叫押过劫法场的贼来。石秀押在厅下,睁圆怪眼,高声大骂:“你这败
坏国家害百姓的贼,我听着哥哥将令:早晚便引军来,打你城子,踏为平地,把你
砍做三截!先教老爷来和你们说知。”石秀在厅前千贼万贼价骂,厅上众人都唬呆
了。梁中书听了,沉吟半晌,叫取大枷来,且把二人枷了,监放死囚牢里,分付蔡
福在意看管,休教有失。蔡福要结识梁山泊好汉,把他两个做一处牢里关着,每日
好酒好肉与他两个吃,因此不曾吃苦,倒将养得好了。却说梁中书唤本州新任王太
守当厅发落,就城中计点被伤人数。杀死的有七八十个,跌伤头面,磕损皮肤,撞
折腿脚者,不计其数。报名在官,梁中书支给官钱,医治烧化了当。

次日,城里城外报说将来:“收得梁山泊没头帖子数十张,不敢隐瞒,只得呈
上。”梁中书看了,吓得魂飞天外,魄散九霄。帖子上写道:

梁山泊义士宋江,仰示大名府,布告天下。今为大宋朝滥官当道,污吏专权,
殴死良民,涂炭万姓。北京卢俊义乃豪杰之士,今者启请上山,一同替天行道。如
何妄徇奸贿,杀害善良!特令石秀先来报知,不期俱被擒捉。如是存得二人性命,
献出淫妇奸夫,吾无侵扰;倘若故伤羽翼,屈坏股肱,便当拔寨兴师,同心雪恨,
大兵到处,玉石俱焚。剿除奸诈,殄灭愚顽,天地咸扶,鬼神共,谈笑入城,并
无轻恕。义夫节妇,孝子顺孙,好义良民,清慎官吏,切勿惊惶,各安职业。谕众
知悉。
当时梁中书看了没头告示,便唤王太守到来商议:“此事如何剖决?”王太守是个
善懦之人,听得说了这话,便禀梁中书道:“梁山泊这一伙,朝廷几次尚且收捕他
不得,何况我这里一郡之力?倘若这亡命之徒,引兵到来,朝廷救兵不迭,那时悔
之晚矣!若论小官愚意:且姑存此二人性命,一面写表,申奏朝廷;二即奉书呈上
蔡太师恩相知道;三者可教本处军马出城下寨,提备不虞。如此,可保北京无事,
军民不伤。若将这两个一时杀坏,诚恐寇兵临城,一者无兵解救,二者朝廷见怪,
三乃百姓惊慌,城中扰乱,深为未便。”梁中书听了道:“知府言之极当。”先唤
押牢节级蔡福来,便道:“这两个贼徒,非同小可。你若是拘束得紧,诚恐丧命;
若教你宽松,又怕他走了。你弟兄两个,早早晚晚,可紧可慢,在意坚固管候发落,
休得时刻怠慢。”蔡福听了,心中暗喜:“如此发放,正中下怀。”领了钧旨,自
去牢中安慰他两个,不在话下。

只说梁中书便唤兵马都监大刀闻达、天王李成两个,都到厅前商议。梁中书备
说梁山泊没头告示,王太守所言之事。两个都监听罢,李成便道:“量这伙草寇,
如何敢擅离巢穴?相公何必有劳神思?李某不才,食禄多矣,无功报德,愿施犬马之
劳,统领军卒,离城下寨。草寇不来,别作商议;如若那伙强寇年衰命尽,擅离巢
穴,领众前来,不是小将夸口,定令此贼片甲不回!”梁中书听了大喜,随即取金
花绣缎,赏劳二将。两个辞谢,别了梁中书,各回营寨安歇。

次日,李成升帐,唤大小官军上帐商议。旁边走过一人,威风凛凛,相貌堂堂,
便是急先锋索超,又出头相见。李成传令道:“宋江草寇,早晚临城,要来打俺北
京,你可点本部军兵,离城三十五里下寨,我随后却领军来。”索超得了将令,次
日点起本部军兵,至三十五里,地名飞虎峪,靠山下了寨栅。次日,李成引领正偏
将,离城二十五里,地名槐树坡,下了寨栅。周围密布枪刀,四下深藏鹿角,三面
掘下陷坑。众军摩拳擦掌,诸将协力同心,只等梁山泊军马到来,便要建功。

话分两头。原来这没头帖子,却是吴学究闻得燕青、杨雄报信,又叫戴宗打听
得卢员外、石秀都被擒捉,因此虚写告示,向没人处撇下及桥梁道路上贴放,只要
保全卢俊义、石秀二人性命。戴宗回到梁山泊,把上项事备细与众头领说知。宋江
听罢大惊,就忠义堂上打鼓集众,大小头领,各依次序而坐。宋江开话对吴学究道:
“当初军师好意,启请卢员外上山来聚义,今日不想却教他受苦,又陷了石秀兄弟。
当用何计可救?”吴用道:“兄长放心。小生不才,愿献一计,乘此机会,就取北
京钱粮,以供山寨之用。明日是个吉辰,请兄长分一半头领,把守山寨,其余尽随
我等去打城池。”宋江道:“军师之言极当。”便唤铁面孔目裴宣,派拨大小军兵,
来日起程。黑旋风李逵便道:“我这两把大斧,多时不曾发市,听得打州劫县,他
也在厅边欢喜。哥哥拨与我五百小喽罗,抢到北京,把梁中书砍做肉泥,拿住李固
和那婆娘,碎尸万段。救取卢员外、石秀二人性命,是我心愿。”宋江道:“兄弟
虽然勇猛,这北京非比别处州府,且梁中书又是蔡太师女婿;更兼手下有李成、闻
达,都有万夫不当之勇,不可轻敌。”李逵大叫道:“哥哥这般长别人志气,灭自
己威风!且看兄弟去如何。若还输了,誓不回山。”吴用道:“既然你要去,便教
做先锋,点与五百好汉相随,就充头阵,来日下山。”当晚宋江和吴用商议,拨定
了人数。裴宣写了告示,送到各寨,各依拨次施行,不得时刻有误。

此时秋末冬初天气,征夫容易披挂,战马易得肥满,军卒久不临阵,皆生战斗
之心;各恨不平,尽想报仇之念。得蒙差遣,欢天喜地,收拾枪刀,拴束鞍马,摩
拳擦掌,时刻下山。第一拨:当先哨路黑旋风李逵,部领小喽罗五百。第二拨:两
头蛇解珍、双尾蝎解宝、毛头星孔明、独火星孔亮,部领小喽罗一千。第三拨:女
头领一丈青扈三娘,副将母夜叉孙二娘、母大虫顾大嫂,部领小喽罗一千。第四拨:
扑天雕李应,副将九纹龙史进、小尉迟孙新,部领小喽罗一千。中军主将都头领宋
江,军师吴用。簇帐头领四员:小温侯吕方、赛仁贵郭盛、病尉迟孙立、镇三山黄
信。前军头领霹雳火秦明,副将百胜将韩滔、天目将彭。后军头领豹子头林冲,
副将铁笛仙马麟、火眼狻猊邓飞。左军头领双鞭呼延灼,副将摩云金翅欧鹏、锦毛
虎燕顺。右军头领小李广花荣,副将跳涧虎陈达、白花蛇杨春。并带炮手轰天雷凌
振,接应粮草。探听军情头领一员,神行太保戴宗。军兵分拨已定,平明,各头领
依次而行,当日进发。只留下副军师公孙胜并刘唐、朱仝、穆弘四个头领,统领马
步军兵,守把山寨。三关水寨中,自有李俊等守把,不在话下。

却说索超正在飞虎峪寨中坐地,只见流星报马前来报说:“宋江军马大小人兵,
不计其数,离寨约有二三十里,将近到来。”索超听的,飞报李成槐树坡寨内。李
成听了,一面报马入城,一面自备了战马,直到前寨。索超接着,说了备细。次日
五更造饭,平明拔寨都起,前到庾家疃,列成阵势,摆开一万五千人马。李成、索
超全副披挂,门旗下勒住战马。平东一望,远远地尘土起处,约有五百余人,飞奔
前来。李成鞭梢一指,军健脚踏硬弩,手拽强弓。梁山泊好汉在庾家疃一字儿摆成
阵势。只见:

人人都带茜红巾,个个齐穿绯衲袄。鹭鸶腿紧系脚绷,虎狼腰牢拴裹肚。三股
叉直迸寒光,四棱简横拖冷雾。柳叶枪,火尖枪,密布如麻;青铜刀,偃月刀,纷
纷似雪。满地红旗飘火焰,半空赤帜耀霞光。

东阵上只见一员好汉,当前出马,乃是黑旋风李逵,手双斧,睁圆怪眼,咬
碎钢牙,高声大叫:“认得梁山泊好汉黑旋风么?”李成在马上看了,与索超大笑
道:“每日只说梁山泊好汉,原来只是这等腌草寇,何足为道!先锋,你看么?何
不先捉此贼?”索超笑道:“割鸡焉用牛刀,自有战将建功,不必主将挂念。”言
未绝,索超马后一员首将,姓王,名定,手拈长枪,引领部下一百马军,飞奔冲将
过来。李逵胆勇过人,虽是带甲遮护,怎当马军一冲,当时四下奔走。索超引军直
赶过庾家疃来,只见山坡背后,锣鼓喧天,早撞出两彪军马:左有解珍、孔亮,右
有孔明、解宝,各领五百小喽罗,冲杀将来。索超见他有接应军马,方才吃惊,不
来追赶,勒马便回。李成问道:“如何不拿贼来?”索超道:“赶过山去,正要拿
他,原来这厮们倒有接应人马,伏兵齐起,难以下手。”李成道:“这等草寇,何
足惧哉!”将引前部军兵,尽数杀过庾家疃来。只见前面摇旗呐喊,擂鼓鸣锣,又
是一彪军马:当先一骑马上,却是一员女将,结束得十分标致。有《念奴娇》为证:

玉雪肌肤,芙蓉模样,有天然标格。金铠辉煌鳞甲动,银渗红罗抹额。玉手纤
纤,双持宝刃,恁英雄赫。眼溜秋波,万种妖娆堪摘。

谩驰宝马当前,霜刃
如风,要把官兵斩馘。粉面尘飞,征袍汗湿,杀气腾胸腋。战士消魂,敌人丧胆,
女将中间奇特。得胜归来,隐隐笑生双颊。

且说这扈三娘引军,红旗上金书大字“女将一丈青”,左有顾大嫂,右有孙二
娘,引一千余军马,尽是七长八短汉,四山五岳人。李成看了道:“这等军人,作
何用处!先锋与我向前迎敌,我却分兵勒捕四下草寇。”索超领了将令,手金蘸
斧,拍坐下马,杀奔前来。一丈青勒马回头,望山凹里便走。李成分开人马,四下
里赶杀。正赶之间,只听的喊声震地,雾气遮天,一彪人马飞也似追来。李成急急
退兵十四五里,首尾不能管顾。急退入庾家疃时,左冲出解珍、孔亮,部领人马,
赶杀将来;右冲出孔明、解宝,部领人马,又杀到来。三员女将,拨转马头,随后
杀来,赶的李成军马四分五落。急待回寨,黑旋风李逵当先拦住。李成、索超冲开
人马,夺路而去。比及回寨,大折一阵。宋江军马也不追赶,一面收兵暂歇,扎下
营寨。

且说李成、索超,慌忙差人入城。报知梁中书,连夜再差闻达速领本部军马前
来助战。李成接着,就槐树坡寨内商议退兵之策。闻达笑道:“疥癞之疾,何足挂
意!闻某不才,来日愿决一阵,务要全胜。”当夜商议定了,传令与军士得知,四
更造饭,五更披挂,平明进兵。战鼓三通,拔寨都起,前到庾家疃。早见宋江军马,
泼风也似价来。但见:
征云冉冉飞晴空,征尘漠漠迷西东。
十万貔貅声震地,车厢火炮如雷轰。
鼙鼓冬冬撼山谷,旌旗猎猎摇天风。
枪影摇空翻玉蟒,剑光耀日飞苍龙。
六师鹰扬鬼神泣,三军英勇貅虎同。
罡星煞曜降凡世,天蓬丁甲离青穹。
银盔金甲濯冰雪,强弓硬弩真难攻。
人人只欲尽忠义,擒王斩将非邀功。
大刀闻达不知量,狂言逞技真雕虫!
飞虎峪中兵四起,星驰电逐无前锋。
闭关收拾残戈甲,有如脱兔潜葭蓬。
当日大刀闻达,便教将军马摆开,强弓硬弩,射住阵脚。花腔鼍鼓擂,杂彩绣旗摇。
宋江阵中,早已捧出一员大将,红旗银字,大书“霹雳火秦明”。怎生打扮:

头戴朱红漆笠,身穿绛色袍鲜,连环锁甲兽吞肩。抹绿战靴云嵌,凤翅明盔耀
日,狮蛮宝带腰悬。狼牙混棍手中拈,凛
凛英雄罕见。
秦明勒马,厉声高叫:“北京滥官污吏听着!多时要打你这城子,诚恐害了百姓良
民。好好将卢俊义、石秀送将过来,淫妇奸夫,一同解出,我便退兵罢战,誓不相
侵。若是执迷不悟,便教昆冈火起,玉石俱焚,只在目前。有话早说,休得俄延!”
说犹未了,闻达大怒,便问首将:“谁与我力擒此贼?”说言未了,脑后铃鸾响处,
一员大将,当先出马。怎生打扮:

耀日兜鍪晃晃,连环铁甲重重,团花点翠锦袍红,金带成双凤。鹊画弓藏袋
内,狼牙箭插壶中。雕鞍稳定五花龙,大斧手中摩弄。
这个是北京上将,姓索,名超,因为此人性急,人皆呼他为急先锋。出到阵前,高
声喝道:“你这厮是朝廷命官,国家有何负你?你好人不做,却去落草为贼!我今拿
住你时,碎尸万段,死有余辜!”这个秦明,又是一个性急的人,听了这话,正是
炉中添炭,火上浇油,拍马向前,抡狼牙棍直奔将来;索超纵马,直挺秦明。二匹
劣马相交,两般军器并举,众军呐喊。斗过二十余合,不分胜败。宋江军中,先锋
队里转过韩滔,就马上拈弓搭箭,觑的索超较亲,飕地只一箭,正中索超左臂。撇
了大斧,回马望本阵便走。宋江鞭梢一指,大小三军,一齐卷杀过来。杀的尸横遍
野,流血成河,大败亏输。直追过庾家疃,随即夺了槐树坡小寨。当晚闻达直奔飞
虎峪,计点军兵,三停去一。宋江就槐树坡寨内屯扎,吴用道:“军兵败走,心中
必怯。若不乘势追赶,诚恐养成勇气,急忙难得。”宋江道:“军师之言极当。”
随即传令,当晚就将精锐得胜军将,分作四路,连夜进发,杀奔城来。

再说闻达奔到飞虎峪,忙忙似丧家之犬,急急如漏网之鱼,正在寨中商议计策,
小校来报:“近山上一带火起!”闻达带领军兵,上马看时,只见东边山上,火把
不知其数,照的遍山遍野通红。闻达便引军兵迎敌,山后又是马军来到,当先首将
小李广花荣,引副将杨春、陈达,横杀将来。闻达措手不及,领兵便回飞虎峪。西
边山上,火把不知其数,当先首将双鞭呼延灼,引副将欧鹏、燕顺,冲击将来。后
面喊声又起,却是首将霹雳火秦明,引副将韩滔、彭,并力杀来。闻达军马大乱,
拔寨都起。只见前面喊声又起,火光晃耀,却是轰天雷凌振,将带副手,从小路直
转飞虎峪那边,放起炮来。闻达引军夺路,奔城而去。只见前面鼓声响处,早有一
彪军马拦路,火光丛中,闪出首将豹子头林冲,引副将马麟、邓飞,截住归路。四
下里战鼓齐鸣,烈火竞起,众军乱撺,各自逃生。闻达手舞大刀,杀开条路走,正
撞着李成,合兵一处,且战且走。战到天明,已至城下。梁中书听的这个消息,惊
的三魂荡荡,七魄幽幽,连忙点军出城,接应败残人马,紧闭城门,坚守不出。次
日,宋江军马追来,直抵东门下寨,准备攻城。

且说梁中书在留守司聚众商议,难以解救。李成道:“贼兵临城,事在告急,
若是迟延,必至失陷。相公可修告急家书,差心腹之人,星夜赶上京师,报与蔡太
师知道,早奏朝廷,调遣精兵前来救应,此是上策;第二,作紧行文,关报邻近府
县,亦教早早调兵接应;第三,北京城内,着仰大名府起差民夫上城,同心协助,
守护城池,准备擂木炮石,踏弩硬弓,灰瓶金汁,晓夜提备,如此可保无虞。”梁
中书道:“家书随便修下,谁人去走一遭?”当日差下首将王定,全副披挂;又差
数个马军,领了密书,放开城门吊桥,望东京飞报声息,及关报邻近府分,发兵救
应。先仰王太守起集民夫上城守护,不在话下。

且说宋江分调众将,引军围城,东西北三面下寨,只空南门不围,每日引军攻
打一面。向山寨中催取粮草,为久屯之计,务要打破北京,救取卢员外、石秀二人。
李成、闻达连日提兵出城交战,不能取胜。索超箭疮将息,未得痊可。

不说宋江军兵打城,且说首将王定赍领密书,三骑马直到东京太师府前下马。
门吏转报入去,太师教唤王定进来,直到后堂拜罢,呈上密书。蔡太师拆开封皮看
了,大惊,问其备细。王定把卢俊义的事,一一说了:“如今宋江领兵围城,声势
浩大,不可抵敌。”庾家疃、槐树坡、飞虎峪三处厮杀,尽皆说罢。蔡京道:“鞍
马劳困,你且去馆驿内安下,待我会官商议。”王定又禀道:“太师恩相:大名危
如累卵,破在旦夕,倘或失陷,河北县郡,如之奈何?望太师恩相早早发兵剿除!”
蔡京道:“不必多说,你且退去。”王定去了。

太师随即差当日府干,请枢密院官急来商议军情重事。不移时,东厅枢密使童
贯引三衙太尉,都到节堂参见太师。蔡京把大名危急之事,备细说了一遍:“如今
将何计策,用何良将,可退贼兵,以保城郭?”说罢,众官互相厮觑,各有惧色。
只见那步司太尉背后转出一人,乃是衙门防御使保义,姓宣,名赞,掌管兵马。此
人生的面如锅底,鼻孔朝天,卷发赤须,彪形八尺;使口钢刀,武艺出众。先前在
王府曾做郡马,人呼为丑郡马。因对连珠箭赢了番将,郡王爱他武艺,招做女婿。
谁想郡主嫌他丑陋,怀恨而亡,因此不得重用,只做得个兵马保义使。童贯是个阿
谀谄佞之徒,与他不能相下,常有嫌疑之心。当时此人忍不住,出班来禀太师道:
“小将当初在乡中,有个相识。此人乃是汉末三分义勇武安王嫡派子孙,姓关,名
胜,生的规模与祖上云长相似,使一口青龙偃月刀,人称为大刀关胜。现做蒲东巡
检,屈在下僚。此人幼读兵书,深通武艺,有万夫不当之勇。若以礼币请他,拜为
上将,可以扫清水寨,殄灭狂徒,保国安民。乞取钧旨。”蔡京听罢大喜,就差宣
赞为使,赍了文书,鞍马连夜星火前往蒲东,礼请关胜赴京计议。众官皆退。

话休絮繁。宣赞领了文书,上马进发,带将三五个从人,不则一日,来到蒲东
巡检司前下马。当日关胜正和郝思文在衙内论说古今兴废之事,闻说东京有使命至,
关胜忙与郝思文出来迎接。各施礼罢,请到厅上坐地。关胜问道:“故人久不相见,
今日何事,远劳亲自到此?”宣赞回言:“为因梁山泊草寇攻打北京,宣某在太师
面前,一力保举兄长有安邦定国之策,降兵斩将之才,特奉朝廷敕旨,太师钧命,
彩币鞍马,礼请起行。兄长勿得推却,便请收拾赴京。”关胜听罢大喜,与宣赞说
道:“这个兄弟,姓郝,双名思文,是我拜义弟兄。当初他母亲梦井木犴投胎,因
而有孕,后生此人,因此人唤他做井木犴。这兄弟十八般武艺无有不能。得蒙太师
呼唤,一同前去,协力报国,有何不可?”宣赞喜诺,就行催请登程。

当下关胜分付老小,一同郝思文,将引关西汉十数个人,收拾刀马、盔甲、行
李,跟随宣赞连夜起程。来到东京,径投太师府前下马。门吏转报蔡太师得知,教
唤进。宣赞引关胜、郝思文直到节堂,拜见已罢,立在阶下。蔡京看了关胜,端的
好表人材:堂堂八尺五六身躯,细细三柳髭须,两眉入鬓,凤眼朝天;面如重枣,
唇若涂朱。太师大喜,便问:“将军青春多少?”关胜答道:“小将三旬有二。”
蔡太师道:“梁山泊草寇围困北京城郭,请问良将,愿施妙策,以解其围。”关胜
禀道:“久闻草寇占住水洼,惊群动众。今擅离巢穴,自取其祸。若救北京,虚劳
人力。乞假精兵数万,先取梁山,后拿贼寇,教他首尾不能相顾。”太师见说大喜,
与宣赞道:“此乃围魏救赵之计,正合吾心。”随即唤枢密院官,调拨山东、河北
精锐军兵一万五千,教郝思文为先锋,宣赞为合后,关胜为领兵指挥使,步军太尉
段常接应粮草。犒赏三军,限日下起行,大刀阔斧,杀奔梁山泊来。直教龙离大海,
不能驾雾腾云;虎到平川,怎地张牙舞爪?正是:贪观天上中秋月,失却盘中照殿
珠。

毕竟宋江军马怎地结果,且听下回分解。