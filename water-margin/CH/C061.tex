\chapter{吴用智赚玉麒麟~张顺夜闹金沙渡}

话说这龙华寺僧人,说出三绝玉麒麟卢俊义名字与宋江,吴用道:“小生凭三
寸不烂之舌,直往北京说卢俊义上山,如探囊取物,手到拈来,只是少一个粗心大
胆的伴当,和我同去。”说犹未了,只见黑旋风李逵高声叫道:“军师哥哥,小弟
与你走一遭。”宋江喝道:“兄弟,你且住着!若是上风放火,下风杀人,打家劫
舍,冲州撞府,合用着你。这是做细作的勾当,你性子又不好,去不的。”李逵道:
“你们都道我生的丑,嫌我,不要我去。”宋江道:“不是嫌你,如今大名府做公
的极多,倘或被人看破,枉送了你的性命。”李逵叫道:“不妨。我定要去走一遭。”
吴用道:“你若依的我三件事,便带你去;若依不的,只在寨中坐地。”李逵道:
“莫说三件,便是三十件也依你!”吴用道:“第一件,你的酒性如烈火,自今日
去,便断了酒,回来你却开;第二件,于路上做道童打扮,随着我,我但叫你,不
要违拗;第三件最难,你从明日为始,并不要说话,只做哑子一般。依的这三件,
便带你去。”李逵道:“不吃酒,做道童,却依得;闭着这个嘴不说话,却是憋杀
我!”吴用道:“你若开口,便惹出事来。”李逵道:“也容易,我只口里衔着一
文铜钱便了!”宋江道:“兄弟,你坚执要去,若有疏失,休要怨我。”李逵道:
“不妨,不妨。我这两把板斧拿了去,少也砍他娘千百个鸟头才罢。”众头领都笑,
那里劝的住。当日忠义堂上做筵席送路。至晚,各自去歇息。次日清早,吴用收拾
了一包行李,教李逵打扮做道童,挑担下山。宋江与众头领都在金沙滩送行,再三
分付吴用小心在意,休教李逵有失。吴用、李逵别了众人下山,宋江等回寨。

且说吴用、李逵二人往北京去,行了四五日路程,每日天晚投店安歇,平明打
火上路,于路上,吴用被李逵怄的苦。行了几日,赶到北京城外店肆里歇下。当晚,
李逵去厨下做饭,一拳打的店小二吐血。小二哥来房里告诉吴用道:“你家哑道童
忒狠:小人烧火迟了些,就打的小人吐血。”吴用慌忙与他陪话,把十数贯钱与他
将息,自埋怨李逵,不在话下。

过了一夜,次日天明,起来安排些饭食吃了。吴用唤李逵入房中分付道:“你
这厮苦死要来,一路上怄死我也!今日入城,不是耍处,你休送了我的性命!”李
逵道:“不敢,不敢。”吴用道:“我再和你打个暗号:若是我把头来摇时,你便
不可动弹。”李逵应承了。两个就店里打扮入城:吴用戴一顶乌绉纱抹眉头巾,穿
一领皂沿边白绢道服,系一条杂彩吕公绦,着一双方头青布履,手里拿一逼赛黄金
熟铜铃杵。李逵戗几根蓬松黄发,绾两枚浑骨丫髻,黑虎躯穿一领粗布短褐袍,飞
熊腰勒一条杂色短须绦,穿一双蹬山透土靴,担一条过头木拐棒,挑着个纸招儿,
上写着:“讲命谈天,卦金一两。”

吴用、李逵两个打扮了,锁上房门,离了店肆,望北京城南门来。行无一里,
却早望见城门,端的好个北京!但见:

城高地险,堑阔濠深。一周回鹿角交加,四下里排叉密布。鼓楼雄壮,缤纷杂
彩旗幡;堞道坦平,簇摆刀枪剑戟。钱粮浩大,人物繁华。东西院鼓乐喧天,南北
店货财满地。千员猛将统层城,百万黎民居上国。
此时天下各处盗贼生发,各州府县俱有军马守把。惟此北京,是河北第一个去处,
更兼又是梁中书统领大军镇守,如何不摆得整齐?

且说吴用、李逵两个,摇摇摆摆,却好来到城门下,守门的约有四五十军士,
簇捧着一个把门的官人在那里坐定。吴用向前施礼,军士问道:“秀才那里来?”
吴用答道:“小生姓张,名用。这个道童姓李。江湖上卖卦营生,今来大郡,与人
讲命。”身边取出假文引,教军士看了。众人道:“这个道童的鸟眼,恰像贼一般
看人!”李逵听得,正待要发作,吴用慌忙把头来摇,李逵便低了头。吴用向前与
把门军士陪话道:“小生一言难尽!这个道童,又聋又哑,只有一分蛮气力;却是
家生的孩儿,没奈何带他出来。这厮不省人事,望乞恕罪!”辞了便行。李逵跟在
背后,脚高步低,望市心里来。吴用手中摇着铃杵,口里念四句口号道:“甘罗发
早子牙迟,彭祖颜回寿不齐。范丹贫穷石崇富,八字生来各有时。”吴用又道:“乃
时也,运也,命也。知生,知死,知贵,知贱。若要问前程,先赐银一两。”说罢,
又摇铃杵。北京城内小儿约有五六十个,跟着看了笑。却好转到卢员外解库门首,
自歌自笑,去了复又回来,小儿们哄动。

卢员外正在解库厅前坐地,看着那一班主管收解,只听得街上喧哄,唤当直的
问道:“如何街上热闹?”当直的报复:“员外,端的好笑!街上一个别处来的算
命先生,在街上卖卦,要银一两算一命,谁人舍的?后头一个跟的道童,且是生的
渗濑,走又走的没样范,小的们跟定了笑。”卢俊义道:“既出大言,必有广学。
当直的,与我请他来。”当直的慌忙去叫道:“先生,员外有请。”吴用道:“是
何人请我?”当直的道:“卢员外相请。”吴用便与道童跟着转来,揭起帘子,入
到厅前,教李逵只在鹅项椅上坐定等候。

吴用转过前来,见卢员外时,那人生的如何?有《满庭芳》词为证:

目炯双瞳,眉分八字,身躯九尺如银。威风凛凛,仪表似天神。惯使一条棍棒,
护身龙、绝技无伦。京城内、家传清白,积祖富豪门。

杀场临敌处,冲开万马,
扫退千军。更忠肝贯日,壮气凌云。慷慨疏财仗义,论英名、播满乾坤。卢员外,
双名俊义,绰号玉麒麟。
当时吴用向前施礼,卢俊义欠身答礼问道:“先生贵乡何处?尊姓高名?”吴用答
道:“小生姓张,名用,自号谈天口。祖贯山东人氏,能算皇极先天数,知人生死
贵贱。卦金白银一两,方才算命。”卢俊义请入后堂小阁儿里,分宾坐定。茶汤已
罢,叫当直的取过白银一两,奉作命金:“烦先生看贱造则个。”吴用道:“请贵
庚月日下算。”卢俊义道:“先生,君子问灾不问福,不必道在下豪富,只求推算
目下行藏则个。在下今年三十二岁,甲子年,乙丑月,丙寅日,丁卯时。”吴用取
出一把铁算子来,排在桌上,算了一回,拿起算子桌上一拍,大叫一声:“怪哉!”
卢俊义失惊问道:“贱造主何吉凶?”吴用道:“员外若不见怪,当以直言。”卢
俊义道:“正要先生与迷人指路,但说不妨。”吴用道:“员外这命,目下不出百
日之内,必有血光之灾:家私不能保守,死于刀剑之下。”卢俊义笑道:“先生差
矣。卢某生于北京,长在豪富之家;祖宗无犯法之男,亲族无再婚之女;更兼俊义
作事谨慎,非理不为,非财不取,如何能有血光之灾?”吴用改容变色,急取原银
付还,起身便走,嗟叹而言:“天下原来都要人阿谀谄佞!罢,罢!分明指与平川路,
却把忠言当恶言。小生告退。”

卢俊义道:“先生息怒。前言特地戏耳,愿听指教。”吴用道:“小生直言,
切勿见怪!”卢俊义道:“在下专听,愿勿隐匿。”吴用道:“员外贵造,一向都
行好运。但今年时犯岁君,正交恶限。目今百日之内,尸首异处。此乃生来分定,
不可逃也。”卢俊义道:“可以回避否?”吴用再把铁算子搭了一回,便回员外道:
“只除非去东南方巽地上,一千里之外,方可免此大难。虽有些惊恐,却不伤大体。”
卢俊义道:“若是免的此难,当以厚报。”吴用道:“命中有四句卦歌,小生说与
员外,写于壁上。日后应验,方知小生灵处。”卢俊义叫取笔砚来,便去白粉壁上
写。吴用口歌四句:“芦花丛里一扁舟,俊杰俄从此地游。义士若能知此理,反躬
逃难可无忧。”当时卢俊义写罢,吴用收拾起算子,作揖便行。卢俊义留道:“先
生少坐,过午了去。”吴用答道:“多蒙员外厚意,误了小生卖卦,改日再来拜会。”
抽身便起。卢俊义送到门首,李逵拿了拐棒,走出门外。吴学究别了卢俊义,引了
李逵,径出城来。回到店中,算还房宿饭钱,收拾行李包裹,李逵挑出卦牌。出离
店肆,对李逵说道:“大事了也!我们星夜赶回山寨,安排圈套,准备机关,迎接
卢俊义,他早晚便来也!”

且不说吴用、李逵还寨,却说卢俊义自从算卦之后,寸心如割,坐立不安,也
是天罡星合当聚会,听了这算命的话,一日耐不得,便叫当直的,去唤众主管商议
事务。少刻都到,那一个为头管家私的主管,姓李,名固。这李固原是东京人,因
来北京投奔相识不着,冻倒在卢员外门前。卢俊义救了他性命,养在家中。因见他
勤谨,写的算的,教他管顾家间事务。五年之内,直抬举他做了都管。一应里外家
私,都在他身上,手下管着四五十个行财管干,一家内都称他做李都管。当日大小
管事之人,都随李固来堂前声喏。

卢员外看了一遭,便道:“怎生不见我那一个人?”说犹未了,阶前走过一人
来。但见:

六尺以上身材,二十四五年纪,三牙掩口细髯,十分腰细膀阔。带一顶木瓜心
攒顶头巾,穿一领银丝纱团领白衫,系一条蜘蛛斑红线压腰,着一双土黄皮油膀夹
靴。脑后一对挨兽金环,护项一枚香罗手帕,腰间斜插名人扇,鬓畔常簪四季花。
这人是北京土居人氏,自小父母双亡,卢员外家中养的他大。为见他一身雪练也似
白肉,卢俊义叫一个高手匠人,与他刺了这一身遍体花绣,却似玉亭柱上铺着软翠。
若赛锦体,由你是谁,都输与他。不则一身好花绣,更兼吹的、弹的、唱的、舞的、
拆白道字、顶真续麻,无有不能,无有不会。亦是说的诸路乡谈,省的诸行百艺的
市语。更且一身本事,无人比的:拿着一张川弩,只用三枝短箭,郊外落生,并不
放空,箭到物落;晚间入城,少杀也有百十个虫蚁。若赛锦标社,那里利物,管取
都是他的。亦且此人百伶百俐,道头知尾。本身姓燕,排行第一,官名单讳个青字。
北京城里人口顺,都叫他做浪子燕青。曾有一篇《沁园春》词单道着燕青的好处,
但见:

唇若涂朱,睛如点漆,面似堆琼。有出人英武,凌云志气,资禀聪明。仪表天
然磊落,梁山上端的夸能。伊州古调,唱出绕梁声。果然是艺苑专精,风月丛中第
一名。
听鼓板喧云,笙声嘹亮,畅叙幽情。棍棒参差,揎拳飞脚,四百军州到处惊。人都
羡英雄领袖,浪子燕青。
原来这燕青是卢俊义家心腹人,也上厅声喏了,做两行立住:李固立在左边,燕青
立在右边。

卢俊义开言道:“我夜来算了一命,道我有百日血光之灾,只除非出去东南上
一千里之外躲避。我想东南方有个去处,是泰安州,那里有东岳泰山天齐仁圣帝金
殿,管天下人民生死灾厄。我一者去那里烧炷香,消灾灭罪;二者躲过这场灾晦;
三者做些买卖,观看外方景致。李固,你与我觅十辆太平车子,装十辆山东货物,
你就收拾行李,跟我去走一遭。燕青小乙看管家里,库房钥匙只今日便与李固交割。
我三日之内,便要起身。”李固道:“主人误矣。常言道:‘卖卜卖卦,转回说话。’
休听那算命的胡言乱语,只在家中,怕做甚么?”卢俊义道:“我命中注定了,你
休逆我。若有灾来,悔却晚矣。”燕青道:“主人在上,须听小乙愚言:这一条路,
去山东泰安州,正打从梁山泊边过。近年泊内,是宋江一伙强人在那里打家劫舍,
官兵捕盗,近他不得。主人要去烧香,等太平了去。休信夜来那个算命的胡讲。倒
敢是梁山泊歹人,假装做阴阳人,来煽惑主人。小乙可惜夜来不在家里,若在家时,
三言两语,盘倒那先生,到敢有场好笑。”卢俊义道:“你们不要胡说,谁人敢来
赚我!梁山泊那伙贼男女,打甚么紧!我观他如同草芥,兀自要去特地捉他,把日前
学成武艺,显扬于天下,也算个男子大丈夫!”

说犹未了,屏风背后走出娘子来,乃是卢员外的浑家,年方二十五岁,姓贾,
嫁与卢俊义,才方五载。娘子贾氏便道:“丈夫,我听你说多时了。自古道:‘出
外一里,不如屋里。’休听那算命的胡说,撇下海阔一个家业,耽惊受怕,去虎穴
龙潭里做买卖。你且只在家内,清心寡欲,高居静坐,自然无事。”卢俊义道:“你
妇人家省得甚么?宁可信其有,不可信其无,自古祸出师人口,必主吉凶。我既主
意定了,你都不得多言多语!”

燕青又道:“小人靠主人福荫,学得些个棒法在身。不是小乙说嘴,帮着主人
去走一遭,路上便有些个草寇出来,小人也敢发落的三五十个开去。留下李都管看
家,小人伏侍主人走一遭。”卢俊义道:“便是我买卖上不省的,要带李固去。他
须省的,又替我大半气力。因此留你在家看守。自有别人管帐,只教你做个桩主。”
李固又道:“小人近日有些脚气的症候,十分走不的多路。”卢俊义听了,大怒道:
“‘养兵千日,用在一朝。’我要你跟我去走一遭,你便有许多推故。若是那一个
再阻我的,教他知我拳头的滋味。”李固吓得面如土色,众人谁敢再说,各自散了。

李固只的忍气吞声,自去安排行李,讨了十辆太平车子,唤了十个脚夫,四五
十拽车头口,把行李装上车子,行货拴缚完备。卢俊义自去结束。第三日烧了神福,
给散了家中大男小女,一个个都分付了。当晚先叫李固引两个当直的尽收拾了出城,
李固去了。娘子看了车仗,流泪而去。次日五更,卢俊义起来沐浴罢,更换一身新
衣服,吃了早膳,取出器械,到后堂里辞别了祖先香火。临时出门上路,分付娘子
好生看家,多便三个月,少只四五十日便回。贾氏道:“丈夫路上小心,频寄书信
回来。”说罢,燕青在面前拜了。卢俊义分付道:“小乙在家,凡事向前,不可出
去三瓦两舍打哄。”燕青道:“主人如此出行,小乙怎敢怠慢?”

卢俊义提了棍棒,出到城外,有诗一首,单道卢俊义这条好棒:
挂壁悬崖欺瑞雪,撑天柱地撼狂风。
虽然身上无牙爪,出水巴山秃尾龙。
李固接着,卢俊义道:“你可引两个伴当先去。但有干净客店,先做下饭等候。车
仗脚夫,到来便吃,省得耽搁了路程。”李固也提条杆棒,先和两个伴当去了。卢
俊义和数个当直的随后押着车仗行,但见途中山明水秀,路阔坡平,心中欢喜道:
“我若是在家,那里见这般景致!”行了四十余里,李固接着主人,吃点心中饭罢,
李固又先去了。再行四五十里,到客店里,李固接着车仗人马宿食。卢俊义来到店
房内,倚了棍棒,挂了毡笠儿,解下腰刀,换了鞋袜,宿食皆不必说。次日清早起
来,打火做饭,众人吃了,收拾车辆头口,上路又行。

自此在路夜宿晓行,已经数日,来到一个客店里宿食,天明要行,只见店小二
哥对卢俊义说道:“好教官人得知:离小人店不得二十里路,正打梁山泊边口子前
过去。山上宋公明大王,虽然不害来往客人,官人须是悄悄过去,休得大惊小怪。”
卢俊义听了道:“原来如此。”便叫当直的取下了衣箱,打开锁,去里面提出一个
包,内取出四面白绢旗,问小二哥讨了四根竹竿,每一根缚起一面旗来,每面栲栳
大小几个字,写道:
慷慨北京卢俊义,远驮货物离乡地。
一心只要捉强人,那时方表男儿志。
李固等众人看了,一齐叫起苦来。店小二问道:“官人莫不和山上宋大王是亲么?”
卢俊义道:“我自是北京财主,却和这贼们有甚么亲!我特地要来捉宋江这厮!”
小二哥道:“官人低声些,不要连累小人,不是耍处!你便有一万人马,也近他不
的。”卢俊义道:“放屁!你这厮们都和那贼人做一路!”店小二叫苦不迭,众车
脚夫都痴呆了。李固跪在地下告道:“主人可怜见众人,留了这条性命回乡去,强
似做罗天大醮!”卢俊义喝道:“你省的甚么!这等燕雀,安敢和鸿鹄厮并?我思量
平生学的一身本事,不曾逢着买主,今日幸然逢此机会,不就这里发卖,更待何时!
我那车子上叉袋里,已准备下一袋熟麻索,倘或这贼们当死合亡,撞在我手里,一
朴刀一个砍翻,你们众人,与我便缚在车子上。撇了货物不打紧,且收拾车子捉人,
把这贼首解上京师,请功受赏,方表我平生之愿。若你们一个不肯去的,只就这里
把你们先杀了。”前面摆四辆车子,上插了四把绢旗;后面六辆车子,随从了行。
那李固和众人,哭哭啼啼,只得依他。卢俊义取出朴刀,装在杆棒上,三个丫儿扣
牢了,赶着车子,奔梁山泊路上来。李固等见了崎岖山路,行一步,怕一步,卢俊
义只顾赶着要行。从清早起来,行到巳牌时分,远远地望见一座大林,有千百株合
抱不交的大树。却好行到林子边,只听得一声胡哨响,吓的李固和两个当直的没躲
处。卢俊义教把车仗押在一边。车夫众人都躲在车子底下叫苦。卢俊义喝道:“我
若搠翻,你们与我便缚!”说犹未了,只见林子边走出四五百小喽罗来,听得后面
锣声响处,又有四五百小喽罗截住后路。林子里一声炮响,托地跳出一筹好汉。怎
地模样,但见:
茜红头巾,金花斜袅;
铁甲凤盔,锦衣绣袄。
血染髭髯,虎威雄暴;
大斧一双,人皆吓倒。

当下李逵手双斧,厉声高叫:“卢员外,认得哑道童么?”卢俊义猛省,喝
道:“我时常有心要来拿你这伙强盗,今日特地到此,快教宋江那厮下山投拜!倘
或执迷,我片时间教你人人皆死,个个不留!”李逵呵呵大笑道:“员外,你今日
中了俺的军师妙计,快来坐把交椅!”卢俊义大怒,着手中朴刀,来斗李逵,李
逵抡起双斧来迎。两个斗不到三合,李逵托地跳出圈子外来,转过身,望林子里便
走。卢俊义挺着朴刀,随后赶去,李逵在林木丛中东闪西躲。引得卢俊义性发,破
一步,抢入林来,李逵飞奔乱松丛中去了。

卢俊义赶过林子这边,一个人也不见了。却待回身,只听得松林旁边转出一伙
人来,一个人高声大叫:“员外不要走,认的俺么?”卢俊义看时,却是一个胖大
和尚:身穿皂直裰,倒提铁禅杖。卢俊义喝道:“你是那里来的和尚!”鲁智深大
笑道:“洒家是花和尚鲁智深,今奉军师将令,着俺来迎接员外上山。”卢俊义焦
躁,大骂:“秃驴敢如此无礼!”拈手中宝刀,直取那和尚。鲁智深抡起铁禅杖来
迎。两个斗不到三合,鲁智深拨开朴刀,回身便走,卢俊义赶将去。

正赶之间,喽罗里走出行者武松,抡两口戒刀,直奔将来。卢俊义不赶和尚,
来斗武松。又不到三合,武松拔步便走。卢俊义哈哈大笑:“我不赶你。你这厮们
何足道哉!”说犹未了,只见山坡下一个人在那里叫道:“卢员外,你如何省得!
岂不闻‘人怕落荡,铁怕落炉?’哥哥定下的计策,你待走那里去!”卢俊义喝道:
“你这厮是谁!”那人笑道:“小可便是赤发鬼刘唐。”卢俊义骂道:“草贼休走!”
挺手中朴刀,直取刘唐。方才斗得三合,刺斜里一个人大叫道:“好汉没遮拦穆弘
在此!”当时刘唐、穆弘,两个两条朴刀,双斗卢俊义。正斗之间,不到三合,只
听的背后脚步响。卢俊义喝声:“着!”刘唐、穆弘跳退数步。卢俊义便转身斗背
后的好汉,却是扑天雕李应。三个头领,丁字脚围定。卢俊义全然不慌,越斗越健。

正好步斗,只听得山顶上一声锣响,三个头领各自卖个破绽,一齐拔步去了。
卢俊义又斗得一身臭汗,不去赶他;再回林子边,来寻车仗人伴时,十辆车子,人
伴头口,都不见了。卢俊义便向高阜处,四下里打一望,只见远远地山坡下,一伙
小喽罗,把车仗头口,赶在前面,将李固一干人,连连串串,缚在后面,鸣锣擂鼓,
解投松树那边去。

卢俊义望见,心如火炽,气似烟生,提着朴刀,直赶将去。约莫离山坡不远,
只见两筹好汉喝一声道:“那里去!”一个是美髯公朱仝,一个是插翅虎雷横。卢
俊义见了,高声骂道:“你这伙草贼,好好把车仗人马还我!”朱仝手拈长须大笑
道:“卢员外,你还恁地不晓事?中了俺军师妙计,便肋生双翅,也飞不出去。快
来大寨坐把交椅。”卢俊义听了大怒,挺起朴刀,直奔二人,朱仝、雷横各将兵器
相迎。斗不到三合,两个回身便走。卢俊义寻思道:“须是赶翻一个,却才讨得车
仗。”舍着性命,赶转山坡,两个好汉,都不见了。只听得山顶上鼓板吹箫,仰面
看时,风刮起那面杏黄旗来,上面绣着“替天行道”四字。转过来打一望,望见红
罗销金伞下,盖着宋江,左有吴用,右有公孙胜。一行部从二百余人,一齐声喏道:
“员外,别来无恙!”

卢俊义见了越怒,指名叫骂山上。吴用劝道:“员外且请息怒。宋公明久慕威
名,特令吴某亲诣门墙,迎员外上山,一同替天行道,请休见责。”卢俊义大骂:
“无端草贼,怎敢赚我!”宋江背后转过小李广花荣,拈弓取箭,看着卢俊义喝道:
“卢员外休要逞能,先教你看花荣神箭!”说犹未了,飕地一箭,正中卢俊义头上
毡笠儿的红缨。吃了一惊,回身便走。山上鼓声震地,只见霹雳火秦明、豹子头林
冲,引一彪军马,摇旗呐喊,从山东边杀出来;又见双鞭将呼延灼、金枪手徐宁,
也领一彪军马,摇旗呐喊,从山西边杀出来,吓得卢俊义走投没路。看看天色将晚,
脚又疼,肚又饥,正是慌不择路,望山僻小径只顾走。约莫黄昏时分,烟迷远水,
雾锁深山,星月微明,不分丛莽。正走之间,不到天尽头,须到地尽处,看看走到
鸭嘴滩头,只一望时,都见满目芦花,茫茫烟水。卢俊义看见,仰天长叹道:“是
我不听好人言,今日果有惶事。”

正烦恼间,只见芦苇里面一个渔人,摇着一只小船出来,那渔人倚定小船叫道:
“客官好大胆!这是梁山泊出没的去处,半夜三更,怎地来到这里!”卢俊义道:
“便是我迷踪失路,寻不着宿头,你救我则个!”渔人道:“此间大宽转有一个市
井,却用走三十余里向开路程,更兼路杂,最是难认;若是水路去时,只有三五里
远近。你舍得十贯钱与我,我便把船载你过去。”卢俊义道:“你若渡得我过去,
寻得市井客店,我多与你些银两。”那渔人摇船傍岸,扶卢俊义下船,把铁篙撑开。
约行三五里水面,只听得前面芦苇丛中橹声响,一只小船飞也似来,船上有两个人:
前面一个人,赤条条地拿着一条水篙,后面那个摇着橹。前面的人横定篙,口里唱
着山歌道:
生来不会读诗书,且就梁山泊里居。
准备窝弓射猛虎,安排香铒钓鳌鱼。
卢俊义听得,吃了一惊,不敢做声。又听得右边芦苇丛中,也是两个人,摇一只小
船出来;后面的摇着橹,有咿哑之声;前面横定篙,口里也唱山歌道:
乾坤生我泼皮身,赋性从来要杀人。
万两黄金浑不爱,一心要捉玉麒麟。
卢俊义听了,只叫得苦。只见当中一只小船,飞也似摇将来,船头上立着一个人,
倒提铁钻木篙,口里亦唱着山歌道:
芦花丛里一扁舟,俊杰俄从此地游。
义士若能知此理,反躬逃难可无忧。
歌罢,三只船一齐唱喏。中间是阮小二,左边是阮小五,右边是阮小七。那三只小
船,一齐撞将来。卢俊义听了,心内转惊,自想又不识水性,连声便叫渔人:“快
与我拢船近岸!”那渔人哈哈大笑,对卢俊义说道:“上是青天,下是绿水;我生
在浔阳江,来上梁山泊;三更不改名,四更不改姓,绰号混江龙李俊的便是!员外
若还不肯降时,枉送了你性命!”卢俊义大惊,喝一声说道:“不是你,便是我!”
拿着朴刀,望李俊心窝里搠将来,李俊见朴刀搠将来,拿定棹牌,一个背抛筋斗,
扑通的翻下水去了。那只船滴溜溜在水面上转,朴刀又搠将下水去了。只见船尾一
个人从水底下钻出来,叫一声,乃是浪里白跳张顺,把手挟住船梢,脚踏水浪,把
船只一侧,船底朝天,英雄落水。正是:铺排打凤牢龙计,坑陷惊天动地人。

毕竟卢俊义性命如何,且听下回分解。