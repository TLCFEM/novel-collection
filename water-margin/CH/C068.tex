\chapter{宋公明夜打曾头市~卢俊义活捉史文恭}

话说当时段景住跑来,对林冲等说道:“我与杨林、石勇,前往北地买马,到
彼选得壮窜有筋力好毛片骏马,买了二百余匹。回至青州地面,被一伙强人,为头
一个唤做险道神郁保四,聚集二百余人,尽数把马劫夺,解送曾头市去了。石勇、
杨林,不知去向。小弟连夜逃来,报知此事。”关胜见说,叫且回山寨与哥哥相见
了,却商议此事。众人且过渡来,都到忠义堂上,见了宋江。关胜引单廷、魏定
国,与大小头领俱各相见了。李逵把下山杀了韩伯龙,遇见焦挺、鲍旭,同去打破
凌州之事,说了一遍。宋江听罢,又添四个好汉,正在欢喜。段景住备说夺马一事,
宋江听了,大怒道:“前者夺我马匹,今又如此无礼。晁天王的冤仇未曾报得,旦
夕不乐,若不去报此仇,惹人耻笑。”吴用道:“即日春暖,正好厮杀。前者进兵,
失其地利,如今必用智取。”宋江道:“此仇深入骨髓,不报得,誓不还山。”吴
用道:“且教时迁,他会飞檐走壁,可去探听消息一遭,回来却作商量。”时迁听
命去了,无三二日,只见杨林、石勇逃得回寨,备说曾头市史文恭口出大言,要与
梁山泊势不两立。宋江见说,便要起兵,吴用道:“再待时迁回报,却去未迟。”
宋江怒气填胸,要报此仇,片时忍耐不住,又使戴宗飞去打听,立等回报。

不过数日,却是戴宗先回来,说:“这曾头市要与凌州报仇,欲起军马。现今
曾头市口扎下大寨,又在法华寺内做中军帐,数百里遍插旌旗,不知何路可进。”
次日,时迁回寨报说:“小弟直到曾头市里面,探知备细,现今扎下五个寨栅:曾
头市前面,二千余人守住村口。总寨内是教师史文恭执掌,北寨是曾涂与副教师苏
定,南寨是次子曾密,西寨是三子曾索,东寨是四子曾魁,中寨是第五子曾升,与
父亲曾弄守把。这个青州郁保四,身长一丈,腰阔数围,绰号险道神,将这夺的许
多马匹,都喂养在法华寺内。”

吴用听罢,便教会集诸将,一同商议:“既然他设五个寨栅,我这里分调五支
军将,可作五路去打他五个寨栅。”卢俊义便起身道:“卢某得蒙救命上山,未能
报效,今愿尽命向前,未知尊意若何?”宋江大喜,便道:“员外如肯下山,便为
前部。”吴用谏道:“员外初到山寨,未经战阵,山岭崎岖,乘马不便,不可为前
部先锋。别引一支军马,前去平川埋伏,只听中军炮响,便来接应。”吴用主意,
只恐卢俊义捉得史文恭时,宋江不负晁盖遗言,让位与他,因此不允他为前部先锋。
宋江大意,只要卢俊义建功,乘此机会,教他为山寨之主。吴用不肯,立主叫卢员
外带同燕青,引领五百步军,平川小路听号。再分调五路军马:曾头市正南大寨,
差马军头领霹雳火秦明、小李广花荣,副将马麟、邓飞,引军三千攻打;曾头市正
东大寨,差步军头领花和尚鲁智深、行者武松,副将孔明、孔亮,引军三千攻打;
曾头市正北大寨,差马军头领青面兽杨志、九纹龙史进,副将杨春、陈达,引军三
千攻打;曾头市正西大寨,差步军头领美髯公朱仝、插翅虎雷横,副将邹渊、邹润,
引军三千攻打;曾头市正中总寨,都头领宋公明,军师吴用、公孙胜,随行副将吕
方、郭盛、解珍、解宝、戴宗、时迁,领军五千攻打;合后步军头领黑旋风李逵、
混世魔王樊瑞,副将项充、李衮,引马步军兵五千。其余头领,各守山寨。

不说宋江部领五军兵将大进。且说曾头市探事人探知备细,报入寨中。曾长官
听了,便请教师史文恭、苏定,商议军情重事。史文恭道:“梁山泊军马来时,只
是多使陷坑,方才捉得他强兵猛将。这伙草寇,须是这条计,以为上策。”曾长官
便差庄客人等,将了锄头铁锹,去村口掘下陷坑数十处,上面虚浮土盖,四下里埋
伏了军兵,只等敌军到来。又去曾头市北路,也掘下十数处陷坑。比及宋江军马起
行时,吴用预先暗使时迁又去打听。数日之间,时迁回来报说:“曾头市寨南寨北,
尽都掘下陷坑,不计其数,只等俺军马到来。”吴用见说,大笑道:“不足为奇!”
引军前进,来到曾头市相近。此时日午时分,前队望见一骑马来,项带铜铃,尾拴
雉尾;马上一人,青巾白袍,手执短枪。前队望见,便要追赶。吴用止住,便教军
马就此下寨,四面掘了濠堑,下了铁蒺藜,传下令去:教五军各自分头下寨,一般
掘下濠堑,下了蒺藜。一住三日,曾头市不出交战。吴用再使时迁扮作伏路小军,
去曾头市寨中,探听他不出何意,所有陷坑,暗暗地记着,离寨多少路远,总有几
处。时迁去了一日,都知备细,暗地使了记号,回报军师。次日,吴用传令:教前
队步军,各执铁锄,分作两队。又把粮车一百有余,装载芦苇干柴,藏在中军。当
晚传令与各寨诸军头领,来日巳牌,只听东西两路步军先去打寨,再教攻打曾头市
北寨的杨志、史进,把马军一字儿摆开,如若那边擂鼓摇旗,虚张声势,切不可进。
吴用传令已了。

再说曾头市史文恭只要引宋江军马打寨,便着他陷坑,寨前路狭,待走那里去。
次日巳牌,听得寨前炮响,追兵大队,都到南门。次后,只见东寨边来报道:“一
个和尚抡着铁禅杖,一个行者舞起双戒刀,攻打前后。”史文恭道:“这两个必是
梁山泊鲁智深、武松。”犹恐有失,便分人去帮助曾魁。只见西寨边又来报道:“一
个长髯大汉,一个虎面贼人,旗号上写着美髯公朱仝、插翅虎雷横,前来攻打甚急。”
史文恭听了,又分拨人去帮助曾索。又听得寨前炮响,史文恭按兵不动,只要等他
入来,塌了陷坑,山后伏兵齐起,接应捉人。这里吴用却调马军,从山背后两路抄
到寨前,前面步军,只顾看寨,又不敢去;两边伏兵,都摆在寨前;背后吴用军马
赶来,尽数逼下坑去。史文恭却待出来,吴用鞭梢一指,军寨中锣响,一齐排出百
余辆车子来,尽数把火点着,上面芦苇干柴,硫黄焰硝,一齐着起,烟火迷天。比
及史文恭军马出来,尽被火车横拦当住,只得回避,急待退军。公孙胜早在阵中,
挥剑作法,借起大风,刮得火焰卷入南门,早把敌楼排栅,尽行烧毁。已自得胜,
鸣金收军,四下里入寨,当晚权歇。史文恭连夜修整寨门,两下当住。

次日,曾涂对史文恭计议道:“若不先斩贼首,难以追灭。”嘱咐教师史文恭
牢守寨栅,曾涂率领军兵,披挂上马,出阵搦战。宋江在中军,闻和曾涂搦战,带
领吕方、郭盛,相随出到前军。门旗影里,看见曾涂,心怀旧恨,用鞭指道:“谁
与我先捉这厮,报往日之仇?”小温侯吕方拍坐下马,挺手中方天画戟,直取曾涂。
两马交锋,军器并举,斗到三十合已上,郭盛在门旗下,看见两个中间,将及输了
一个。原来吕方本事,敌不得曾涂,三十合已前,兀自抵敌不住,三十合已后,戟
法乱了,只办得遮架躲闪。郭盛只恐吕方有失,便骤坐下马,拈手中方天画戟,飞
出阵来,夹攻曾涂。三骑马在阵前,绞做一团。原来两枝戟上,都拴着金钱豹尾。
吕方、郭盛要捉曾涂,两枝戟齐举,曾涂眼明,便用枪只一拨,却被两条豹尾搅住
朱缨,夺扯不开,三个各要掣出军器使用。小李广花荣在阵中看见,恐怕输了两个,
便纵马出来,左手拈起雕弓,右手急取箭,搭上箭,拽满弓,望着曾涂射来。这
曾涂却好掣出枪来,那两枝戟兀自搅做一团。说时迟,那时疾,曾涂掣枪,便望吕
方项根搠来。花荣箭早先到,正中曾涂左臂,翻身落马,头盔倒卓,两脚蹬空。吕
方、郭盛双戟并施,曾涂死于非命。十数骑马军,飞奔回来,报知史文恭,转报中
寨。

曾长官听得大哭。只见旁边恼犯了一个壮士曾升,武艺绝高,使两口飞刀,人
莫敢近。当时听了大怒,咬牙切齿,喝教:“备我马来,要与哥哥报仇!”曾长官
拦当不住,全身披挂,绰刀上马,直奔前寨。史文恭接着劝道:“小将军不可轻敌。
宋江军中,智勇猛将极多。若论史某愚意:只宜坚守五寨,暗地使人前往凌州,便
教飞奏朝廷,调兵选将,多拨官军,分作两处征剿:一打梁山泊,一保曾头市,令
贼无心恋战,必欲退兵,急奔回山。那时史某不才,与汝兄弟一同追杀,必获大功。”
说言未了,北寨副教师苏定到来,见说坚守一节,也道:“梁山泊吴用那厮诡计多
谋,不可轻敌,只宜退守;待救兵到来,从长商议。”曾升叫道:“杀我亲兄,此
冤不报,更待何时!直等养成贼势,退敌则难!”史文恭、苏定阻当不住。曾升上
马,带领数十骑马军,飞奔出寨搦战。宋江闻知,传令前军迎敌。当时秦明得令,
舞起狼牙棍,正要出阵斗这曾升,只见黑旋风李逵,手板斧,直奔军前,不问事
由,抢出垓心。对阵有人认的,说道:“这个是梁山泊黑旋风李逵。”曾升见了,
便叫放箭。原来李逵但是上阵,便要脱膊,全得项充、李衮蛮牌遮护。此时独自抢
来,被曾升一箭,腿上正着,身如泰山,倒在地下。曾升背后马军,齐抢过来,宋
江阵上秦明、花荣,飞马向前死救,背后马麟、邓飞、吕方、郭盛,一齐接应归阵。
曾升见了宋江阵上人多,不敢再战,以此领兵还寨。宋江也自收军驻扎。

次日,史文恭、苏定只是主张不要对阵,怎禁得曾升催并道:“要报兄仇。”
史文恭无奈,只得披挂上马。那匹马便是先前夺的段景住的千里龙驹照夜玉狮子马。
宋江引诸将摆开阵势迎敌。对阵史文恭出马,怎生打扮:
头上金盔耀日光,身披铠甲赛冰霜。
坐骑千里龙驹马,手执朱缨丈二枪。
斯时史文恭出马,横杀过来,宋江阵上秦明要夺头功,飞奔坐下马来迎。二骑相交,
军器并举。约斗二十余合,秦明力怯,望本阵便走。史文恭奋勇赶来,神枪到处,
秦明后腿股上早着,倒下马来。吕方、郭盛、马麟、邓飞,四将齐出,死命来救。
虽然救得秦明,军兵折了一阵。收回败军,离寨十里驻扎。宋江叫把车子载了秦明,
一面使人送回山寨将息,再与吴用商量:教取大刀关胜、金枪手徐宁,并要单廷、
魏定国四位下山,同来协助。宋江自己焚香祈祷,占卜一课。吴用看了卦象,便道:
“虽然此处可破,今夜必主有贼兵入寨。”宋江道:“可以早作准备。”吴用道:
“请兄长放心,只顾传下号令:先去报与三寨头领,今夜起东西二寨,便教解珍在
左,解宝在右,其余军马各于四下里埋伏。”已定。

是夜,天清月白,风静云闲,史文恭在寨中对曾升道:“贼兵今日输了两将,
必然惧怯,乘虚正好劫寨。”曾升见说,便教请北寨苏定、南寨曾密、西寨曾索,
引兵前来,一同劫寨。二更左侧,潜地出哨,马摘鸾铃,人披软战,直到宋江中军
寨内,见四下无人,劫着空寨,急叫中计,转身便走。左手下撞出两头蛇解珍,右
手下撞出双尾蝎解宝,后面便是小李广花荣,一发赶上,曾索在黑地里,被解珍一
钢叉,搠于马下。放起火来,后寨发喊,东西两边,进兵攻打寨栅。混战了半夜,
史文恭夺路得回。

曾长官又见折了曾索,烦恼倍增。次日要史文恭写书投降。史文恭也有八分惧
怯,随即写书,速差一人赍擎,直到宋江大寨。小校报知,曾头市有人下书,宋江
传令,教唤入来。小校将书呈上,宋江拆开看时,写道:

曾头市主曾弄顿首,再拜宋公明统军头领麾下:日昨小男,倚仗一时之勇,误
有冒犯虎威。向日天王率众到来,理合就当归附。奈何无端部卒,施放冷箭,更兼
夺马之罪,虽百口何辞!原之实非本意。今顽犬已亡,遣使讲和。如蒙罢战休兵,
将原夺马匹,尽数纳还,更赍金帛,犒劳三军,免致两伤。谨此奉书,伏乞照察。
宋江看罢来书,心中大怒,扯书骂道:“杀吾兄长,焉肯干休?只待洗荡村坊,是
吾本愿!”下书人俯伏在地,凛颤不已。吴用慌忙劝道:“兄长差矣。我等相争,
皆为气耳。既是曾家差人下书讲和,岂为一时之忿,以失大义?”随即便写回书,
取银十两,赏了来使。回还本寨,将书呈上。曾长官与史文恭拆开看时,上面写道:

梁山泊主将宋江,手书回复曾头市主曾弄帐前:国以信
而治天下,将以勇而镇外邦,人无礼而何为,财非义而不取。梁山泊与曾头市,自
来无仇,各守边界。奈缘尔将行一时之恶,惹数载之冤。若要讲和,便须发还二次
原夺马匹,并要夺马凶徒郁保四,犒劳军士金帛。忠诚既笃,礼数休轻。如或更变,
别有定夺。
曾长官与史文恭看了,俱各惊忧。次日,曾长官又使人来说:“若肯讲和,各请一
人质当。”宋江不肯,吴用便道:“无伤。”随即便差时迁、李逵、樊瑞、项充、
李衮五人,前去为信。临行时,吴用叫过时迁,附耳低言:“如此如此,休得有误。”
不说五人去了,却说关胜、徐宁、单廷、魏定国到了。当时见了众人,就在中军
扎驻。

且说时迁引四个好汉,来见曾长官,时迁向前说道:“奉哥哥将令,差时迁引
李逵等四人前来讲和。”史文恭道:“吴用差遣五个人来,必然有谋。”李逵大怒,
揪住史文恭便打,曾长官慌忙劝住。时迁道:“李逵虽然粗卤,却是俺宋公明哥哥
心腹之人,特使他来,休得疑惑。”曾长官中心只要讲和,不听史文恭之言,便教
置酒相待,请去法华寺寨中安歇,拨五百军人前后围住。却使曾升带同郁保四,来
宋江大寨讲和。二人到中军相见了,随后将原夺二次马匹,并金帛一车,送到大寨。
宋江看罢道:“这马都是后次夺的。正有先前段景住送来那匹千里白龙驹照夜玉狮
子马,如何不见将来?”曾升道:“是师父史文恭乘坐着,以此不曾将来。”宋江
道:“你疾忙快写书去,教早早牵那匹马来还我。”曾升便写书,叫从人还寨,讨
这匹马来。史文恭听得,回道:“别的马将去不吝,这匹马却不与他。”从人往复
去了几遭,宋江定死要这匹马。史文恭使人来说道:“若还定要我这匹马时,着他
即便退军,我便送来还他。”

宋江听得这话,便与吴用商量。尚然未决,忽有人来报道:“青州、凌州两路
有军马到来。”宋江道:“那厮们知得,必然变卦。”暗传下号令,就差关胜、单
廷、魏定国,去迎青州军马;花荣、马麟、邓飞,去迎凌州军马。暗地叫出郁保
四来,用好言抚恤他,十分恩义相待,说道:“你若肯建这场功劳,山寨里也教你
做个头领。夺马之仇,折箭为誓,一齐都罢。你若不从,曾头市破在旦夕,任从你
心。”郁保四听言,情愿投拜,从命帐下。吴用授计与郁保四道:“你只做私逃还
寨,与史文恭说道:‘我和曾升去宋江寨中讲和,打听得真实了:如今宋江大意,
只要赚这匹千里马,实无心讲和,若还与了他,必然翻变。如今听得青州、凌州两
路救兵到了,十分心慌,正好乘势用计,不可有误。’他若信从了,我自有处置。”

郁保四领了言语,直到史文恭寨里,把前事具说一遍。史文恭领了郁保四来见
曾长官,备说宋江无心讲和,可以乘势劫他寨栅。曾长官道:“我那曾升当在那里,
若还翻变,必然被他杀害。”史文恭道:“打破他寨,好歹救了。今晚传令与各寨,
尽数都起,先劫宋江大寨。如断去蛇首,众贼无用,回来却杀李逵等五人未迟。”
曾长官道:“教师可以善用良计。”当下传令与北寨苏定、东寨曾魁、南寨曾密,
一同劫寨。郁保四却闪入法华寺大寨内,看了李逵等五人,暗与时迁走透这个消息。

再说宋江同吴用说道:“未知此计若何?”吴用道:“如是郁保四不回,便是
中俺之计。他若今晚来劫我寨,我等退伏两边,却教鲁智深、武松,引步军杀入他
东寨;朱仝、雷横,引步军杀入他西寨;却令杨志、史进,引马军截杀北寨:此名
番犬伏窝之计,百发百中。”

当晚却说史文恭带了苏定、曾密、曾魁,尽数起发。是夜月色朦胧,星辰昏暗。
史文恭、苏定当先,曾密、曾魁押后,马摘鸾铃,人披软战,尽都来到宋江总寨。
只见寨门不关,寨内并无一人,又不见些动静,情知中计,即便回身。急望本寨去
时,只见曾头市里锣鼓炮响,却是时迁爬去法华寺钟楼上撞起钟来,声响为号,东
西两门,火炮齐响,喊声大举,正不知多少军马,杀将入来。却说法华寺中李逵、
樊瑞、项充、李衮,一齐发作,杀将出来。史文恭等急回到寨时,寻路不见。曾长
官见寨中大闹,又听得梁山泊大军两路杀将入来,就在寨里自缢而死。曾密径奔西
寨,被朱仝一朴刀搠死。曾魁要奔东寨时,乱军中马践为泥。苏定死命奔出北门,
却有无数陷坑,背后鲁智深、武松,赶杀将来,前逢杨志、史进,乱箭射死苏定。
后头撞来的人马,都入陷坑中去,重重迭迭,陷死不知其数。

且说史文恭得这千里马,行得快,杀出西门,落荒而走。此时黑雾遮天,不分
南北。约行了二十余里,不知何处。只听得树林背后,一齐锣响,撞出四五百军来。
当先一将,手提杆棒,望马脚便打。那匹马是千里龙驹,见棒来时,从头上跳过去
了。史文恭正走之间,只见阴云冉冉,冷气飕飕,黑雾漫漫,狂风飒飒,虚空中一
人,当住去路。史文恭疑是神兵,勒马便回,东西南北,四边都是晁盖阴魂缠住。
史文恭再回旧路,却撞着浪子燕青,又转过玉麒麟卢俊义来,喝一声:“强贼,待
走那里去!”腿股上只一朴刀,搠下马来,便把绳索绑了,解投曾头市来。燕青牵
了那匹千里龙驹,径到大寨。

宋江看了,心中一喜一怒:喜者得卢员外建功,怒者恨史文恭射杀晁天王,仇
人相见,分外眼睁。先把曾升就本处斩首,曾家一门老少,尽数不留。抄掳到金银
财宝,米麦粮食,尽行装载上车,回梁山泊,给散各都头领,犒赏三军。且说关胜
领军杀退青州军马,花荣领兵杀散凌州军马,都回来了。大小头领,不缺一个。又
得了这匹千里龙驹照夜玉狮子马,其余物件,尽不必说。陷车内囚了史文恭,便收
拾军马,回梁山泊来。所过州县村坊,并无侵扰。回到山寨忠义堂上,都来参见晁
盖之灵。宋江传令:教圣手书生萧让作了祭文,令大小头领,人人挂孝,个个举哀,
将史文恭剖腹剜心,享祭晁盖已罢。宋江就忠义堂上,与众弟兄商议立梁山泊之主。

吴用便道:“兄长为尊,卢员外为次,其余众弟兄,各依旧位。”宋江道:“向
者晁天王遗言:‘但有人捉得史文恭者,不拣是谁,便为梁山泊之主。’今日卢员
外生擒此贼,赴山祭献晁兄,报仇雪恨,正当为尊,不必多说。”卢俊义道:“小
弟德薄才疏,怎敢承当此位!若得居末,尚自过分。”宋江道:“非宋某多谦,有
三件不如员外处:第一件,宋江身材黑矮,貌拙才疏;员外堂堂一表,凛凛一躯,
有贵人之相。第二件,宋江出身小吏,犯罪在逃,感蒙众弟兄不弃,暂居尊位;员
外生于富贵之家,长有豪杰之誉,虽然有些凶险,累蒙天。第三件,宋江文不能
安邦,武又不能附众,手无缚鸡之力,身无寸箭之功;员外力敌万人,通今博古,
天下谁不望风而服。尊兄有如此才德,正当为山寨之主。他时归顺朝廷,建功立业,
官爵升迁,能使弟兄们尽生光彩。宋江主张已定,休得推托。”

卢俊义拜于地下,说道:“兄长枉自多谈,卢某宁死,实难从命。”吴用劝道:
“兄长为尊,卢员外为次,人皆所伏。兄长若如是再三推让,恐冷了众人之心。”
原来吴用已把眼视众人,故出此语。只见黑旋风李逵大叫道:“我在江州舍身拚命,
跟将你来,众人都饶让你一步。我自天也不怕!你只管让来让去,做甚鸟!我便杀将
起来,各自散伙!”武松见吴用以目示人,也发作叫道:“哥哥手下许多军官,受
朝廷诰命的,也只是让哥哥,如何肯从别人?”刘唐便道:“我们起初七个上山,
那时便有让哥哥为尊之意,今日却要让别人!”鲁智深大叫道:“若还兄长推让别
人,洒家们各自撒开!”宋江道:“你众人不必多说,我自有个道理,尽天意,看
是如何,方才可定。”吴用道:“有何高见,便请一言。”宋江道:“有两件事。”
正是:教梁山泊内,重添两个英雄;东平府中,又惹一场灾祸。直教:天罡尽数投
山寨,地煞空群聚水涯。

毕竟宋江说出那两件事来,且听下回分解。