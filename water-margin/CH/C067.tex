\chapter{宋江赏马步三军~关胜降水火二将}

话说当下梁中书、李成、闻达,慌速寻得败残军马,投南便走。正行之间,又
撞着两队伏兵,前后掩杀。李成当先,闻达在后,护着梁中书,并力死战,撞透重
围,脱得大难,头盔不整,衣甲飘零,虽是折了人马,且喜三人逃得性命,投西去
了。樊瑞引项充、李衮,乘势追赶不上,自与雷横、施恩、穆春等,同回北京城内
听令。

再说军师吴用,在城中传下将令,一面出榜安民,一面救灭了火。梁中书、李
成、闻达、王太守各家老小,杀的杀了,走的走了,也不来追究。便把大名府库藏
打开,应有金银宝物,缎匹绫锦,都装载上车子;又开仓廒,将粮米济满城百姓
了,余者亦装载上车,将回梁山泊仓用。号令众头领人马,都皆完备。把李固、贾
氏钉在陷车内,将军马标拨作三队,回梁山泊来。正是:鞍上将敲金镫响,马前军
唱凯歌回。却叫戴宗先去报宋公明。宋江会集诸将,下山迎接,都到忠义堂上。宋
江见了卢俊义,纳头便拜,卢俊义慌忙答礼。宋江道:“我等众人,欲请员外上山,
同聚大义,不想却遭此难,几被倾送,寸心如割。皇天垂,今日再得相见,大慰
平生。”卢俊义拜谢道:“上托兄长虎威,深感众头领之德,齐心并力,救拔贱体,
肝胆涂地,难以报答。”便请蔡福、蔡庆拜见宋江,言说:“在下若非此二人,安
得残生到此!”称谢不尽。当下宋江要卢员外为尊,卢俊义拜道:“卢某是何等之
人,敢为山寨之主?若得与兄长执鞭坠镫,愿为一卒,报答救命之恩,实为万幸!”
宋江再三拜请,卢俊义那里肯坐。

只见李逵道:“哥哥若让别人做山寨之主,我便杀将起来。”武松道:“哥哥
只管让来让去,让得弟兄们心肠冷了。”宋江大喝道:“汝等省得甚么?不得多言!”
卢俊义慌忙拜道:“若是兄长苦苦相让,着卢某安身不牢。”李逵叫道:“今朝都
没事了,哥哥便做皇帝,教卢员外做丞相,我们都做大官,杀去东京,夺了鸟位,
却不强似在这里鸟乱!”宋江大怒,喝骂李逵。吴用劝道:“且教卢员外东边耳房
安歇,宾客相待。等日后有功,却再让位。”宋江方才欢喜,就叫燕青一处安歇。
另拨房屋,叫蔡福、蔡庆安顿老小。关胜家眷,薛永已取到山寨。宋江便叫大设筵
宴,犒赏马步水三军,令大小头目,并众喽罗军健,各自成团作队去吃酒。忠义堂
上,设宴庆贺。大小头领,相谦相让,饮酒作乐。卢俊义起身道:“淫妇奸夫,擒
捉在此,听候发落。”宋江笑道:“我正忘了,叫他两个过来。”众军把陷车打开,
拖出堂前,李固绑在左边将军柱上,贾氏绑在右边将军柱上。宋江道:“休问这厮
罪恶,请员外自行发落。”卢俊义手拿短刀,自下堂来,大骂泼妇贼奴,就将二人
割腹剜心,凌迟处死。抛弃尸首,上堂来拜谢众人。众头领尽皆作贺,称赞不已。

且不说梁山泊大设筵宴,犒赏马步水三军。却说北京梁中书探听得梁山泊军马
退去,再和李成、闻达引领败残军马,入城来看觑老小时,十损八九,众皆号哭不
已。比及邻近起军追赶梁山泊人马时,已自去得远了,且教各自收军。梁中书的夫
人,躲得在后花园中,逃得性命,便叫丈夫写表申奏朝廷,写书教太师知道:早早
调兵遣将,剿除贼寇报仇。抄写民间被杀死者五千余人,中伤者不计其数,各部军
马,总折却三万有余。首将赍了奏文密书上路,不则一日,来到东京太师府前下马。
门吏转报,太师教唤入来,首将直至节堂下拜见了,呈上密书申奏,诉说打破北京,
贼寇浩大,不能抵敌。蔡京初意,亦欲苟且招安,功归梁中书身上,自己亦有荣宠;
今见事体败坏,难遮掩,便欲主战,因大怒道:“且教首将退去!”次日五更,景
阳钟响,待漏院众集文武群臣,蔡太师为首,直临玉阶,面奏道君皇帝。天子览奏,
大惊。有谏议大夫赵鼎出班奏道:“前者往往调兵征发,皆折兵将,盖因失其地利,
以致如此。以臣愚意,不若降敕赦罪招安,诏取赴阙,命作良臣,以防边境之害。”
蔡京听了大怒,喝叱道:“汝为谏议大夫,反灭朝廷纲纪,猖獗小人,罪合赐死!”
天子曰:“如此,目下便令出朝。”当下革了赵鼎官爵,罢为庶人,当朝谁敢再奏。
有诗为证:
玺书招抚是良谋,却把忠言作寇仇。
一自老成人去后,梁山军马不能收。
天子又问蔡京道:“似此贼势猖獗,可遣谁人剿捕?”蔡太师奏道:“臣量这等山
野草贼,安用大军,臣举凌州有二将:一人姓单,名廷;一人姓魏,名定国,现
任本州团练使。伏乞陛下圣旨,星夜差人,调此一枝人马,克日扫清水泊。”天子
大喜,随即降写敕符,着枢密院调遣。天子驾起,百官退朝,众官暗笑。次日,蔡
京会省院差官,赍捧圣旨敕符,投凌州来。

再说宋江水浒寨内,将北京所得的府库金宝钱物,给赏与马步水三军,连日杀
牛宰马,大排筵宴,庆赏卢员外;虽无庖凤烹龙,端的肉山酒海。众头领酒至半酣,
吴用对宋江等说道:“今为卢员外打破北京,杀损人民,劫掠府库,赶得梁中书等
离城逃奔,他岂不写表申奏朝廷?况他丈人是当朝太师,怎肯干罢?必然起军发马,
前来征讨。”宋江道:“军师所虑,最为得理。何不使人连夜去北京探听虚实,我
这里好做准备。”吴用笑道:“小弟已差人去了,将次回也。”正在筵会之间,商
议未了,只见原差探事人到来,报说:“北京梁中书果然申奏朝廷,要调兵征剿。
有谏议大夫赵鼎奏请招安,致被蔡京喝骂,削了赵鼎官职。如今奏过天子,差人赍
捧敕符,往凌州调遣单廷、魏定国两个团练使,起本州军马,前来征讨。”宋江
便道:“似此如何迎敌?”吴用道:“等他来时,一发捉了。”关胜起身对宋江、
吴用道:“关某自从上山,深感仁兄厚待,不曾出得半分气力。单廷、魏定国,
蒲城多曾相会。久知单廷那厮,善能用水浸兵之法,人皆称为圣水将军。魏定国
这厮,精熟火攻兵法,上阵专能用火器取人,因此呼为‘神火将军’。凌州是本境,
兼管本州兵马,取此二人为部下。小弟不才,愿借五千军兵,不等他二将起行,先
在凌州路上接住。他若肯降时,带上山来;若不肯投降,必当擒来,奉献兄长,亦
不须用众头领张弓挟矢,费力劳神。不知尊意若何?”宋江大喜,便叫宣赞、郝思
文二将,就跟着一同前去。关胜带了五千军马,来日下山。次早,宋江与众头领在
金沙滩寨前饯行,关胜三人引兵去了。

众头领回到忠义堂上,吴用便对宋江说道:“关胜此去,未保其心,可以再差
良将,随后监督,就行接应。”宋江道:“吾观关胜义气凛然,始终如一,军师不
必多疑。”吴用道:“只恐他心不似兄长之心。可再叫林冲、杨志领兵,孙立、黄
信为副将,带领五千人马,随即下山。”李逵便道:“我也去走一遭。”宋江道:
“此一去用你不着,自有良将建功。”李逵道:“兄弟若闲,便要生病,若不叫我
去时,独自也要去走一遭。”宋江喝道:“你若不听我的军令,割了你头!”李逵
见说,闷闷不已,下堂去了。不说林冲、杨志领兵下山,接应关胜。次日,只见小
军来报:“黑旋风李逵昨夜二更,拿了两把板斧,不知那里去了!”宋江见报,只
叫得苦:“是我夜来冲撞了他这几句言语,多管是投别处去了!”吴用道:“兄长,
非也。他虽粗卤,义气倒重,不到得投别处去。多管是过两日便来,兄长放心。”
宋江心慌,先使戴宗去赶,后着时迁、李云、乐和、王定六四个首将,分四路去寻。

且说李逵是夜提着两把板斧下山,抄小路径投凌州去。一路上自寻思道:“这
两个鸟将军,何消得许多军马去征他!我且抢入城中,一斧一个都砍杀了,也教哥
哥吃一惊!也和他们争得一口气!”走了半日,走得肚饥,原来贪慌下山,不曾带
得盘缠。多时不做这买卖,寻思道:“只得寻个鸟出气的。”正走之间,看见路旁
一个村酒店,李逵便入去里面坐下,连打了三角酒、二斤肉吃了,起身便走。酒保
拦住讨钱。李逵道:“待我前头去寻得些买卖,却把来还你!”说罢,便动身。只
见外面走入个彪形大汉来,喝道:“你这黑厮,好大胆!谁开的酒店,你来白吃,
不肯还钱!”李逵睁着眼道:“老爷不拣那里,只是白吃!”那汉道:“我对你说
时,惊得你尿流屁滚!老爷是梁山泊好汉韩伯龙的便是!本钱都是宋江哥哥的。”李
逵听了暗笑:“我山寨里那里认得这个鸟人!”原来韩伯龙曾在江湖上打家劫舍,
要来上梁山泊入伙,却投奔了旱地忽律朱贵,要他引见宋江。因是宋公明生发背疮,
在寨中又调兵遣将,多忙少闲,不曾见得,朱贵权且教他在村中卖酒。当时李逵去
腰间拔出一把板斧,看着韩伯龙道:“把斧头为当。”韩伯龙不知是计,舒手来接,
见李逵手起,望面门上只一斧,地砍着。可怜韩伯龙做了半世强人,死在李逵
之手。两三个火家,只恨爷娘少生了两只脚,望深村里走了。李逵就地下掳掠了盘
缠,放火烧了草屋,望凌州去了。

行不得一日,正走之间,官道旁边,只见走过一条大汉,直上直下相李逵。李
逵见那人看他,便道:“你那厮看老爷怎地?”那汉便答道:“你是谁的老爷?”
李逵便抢将入来。那汉子手起一拳,打个塔墩,李逵寻思:“这汉子倒使得好拳!”
坐在地下,仰着脸问道:“你这汉子,姓甚名谁?”那汉道:“老爷没姓,要厮打
便和你厮打!你敢起来!”李逵大怒,正待跳将起来,被那汉子肋罗里只一脚,又
踢了一交。李逵叫道:“赢他不得。”爬将起来便走。那汉叫住问道:“这黑汉子,
你姓甚名谁?那里人氏?”李逵道:“我说与你,休要吃惊。我是梁山泊黑旋风李
逵的便是。”那汉道:“你端的是不是?不要说谎。”李逵道:“你不信,只看我
这两把板斧。”那汉道:“你既是梁山泊好汉,独自一个投那里去?”李逵道:“我
和哥哥别口气,要投凌州去杀那姓单姓魏的两个。”那汉道:“我听得你梁山泊已
有军马去了,你且说是谁?”李逵道:“先是大刀关胜领兵,随后便是豹子头林冲、
青面兽杨志,领军策应。”那汉听了,纳头便拜。李逵道:“你端的姓甚名谁?”
那汉道:“小人原是中山府人氏,祖传三代,相扑为生。却才手脚,父子相传,不
教徒弟。平生最无面目,到处投人不着,山东、河北都叫我做没面目焦挺。近日打
听得寇州地面,有座山,名为枯树山。山上有个强人,平生只好杀人,世人把他比
做丧门神,姓鲍名旭。他在那山里,打家劫舍,我如今待要去那里入伙。”

李逵道:“你有这等本事,如何不来投奔俺哥哥宋公明?”焦挺道:“我多时
要投奔大寨入伙,却没条门路。今日得遇兄长,愿随哥哥。”李逵道:“我却要和
宋公明哥哥争口气了下山来,不杀得一个人,空着双手,怎地回去?你和我去枯树
山,说了鲍旭,同去凌州杀得单、魏二将,便好回山。”焦挺道:“凌州一府城池,
许多军马在彼,我和你只两个,便有十分本事,也不济事,枉送了性命;不如单去
枯树山说了鲍旭,都去大寨入伙,此为上计。”

两个正说之间,背后时迁赶将来,叫道:“哥哥忧得作苦,便请回山。如今分
四路去赶你也。”李逵引着焦挺,且教与时迁厮见了。时迁劝李逵回山:“宋公明
哥哥等你。”李逵道:“你且住!我和焦挺商量定了,先去枯树山说了鲍旭,方才
回来。”时迁道:“使不得。哥哥等你,即便回寨。”李逵道:“你若不跟我去,
你自先回山寨,报与哥哥知道,我便回也。”时迁惧怕李逵,自回山寨去了。焦挺
却和李逵自投寇州来,望枯树山去了。

话分两头。却说关胜与同宣赞、郝思文引领五千军马接来,相近凌州。且说凌
州太守,接得东京调兵的敕旨并蔡太师札付,便请兵马团练单廷、魏定国商议。
二将受了札付,随即选点军兵,关领军器,拴束鞍马,整顿粮草,指日起行。忽闻
报说:“蒲东大刀关胜引军到来,侵犯本州。”单廷、魏定国听得大怒,便收拾
军马,出城迎敌。两军相近,旗鼓相望。门旗下关胜出马。那边阵内鼓声响处,圣
水将军出马。怎生打扮:

戴一顶浑铁打就四方铁帽,顶上撒一颗斗来大小黑缨。披一付熊皮砌就嵌缝沿
边乌油铠甲,穿一领皂罗绣就点翠团花秃袖征袍,着一双斜皮踢镫嵌线云跟靴,系
一条碧钉就迭胜狮蛮带。一张弓,一壶箭。骑一匹深乌马,使一条黑杆枪。
前面打一把引军按北方皂纛旗,上书七个银字:“圣水将军单廷。”又见这边鸾
铃响处,转出这员神火将军魏定国来出马。怎生打扮:

戴一顶朱红缀嵌点金束发盔,顶上撒一把扫帚长短赤缨。披一副摆连环吞兽面
猊铠,穿一领绣云霞飞怪兽绛红袍,着一双刺麒麟间翡翠云缝锦跟靴。带一张描
金雀画宝雕弓,悬一壶凤翎凿山狼牙箭。骑坐一匹胭脂马,手使一口熟铜刀。
前面打一把引军按南方红绣旗,上书七个银字:“神火将军魏定国。”两员虎将,
一齐出到阵前。关胜见了,在马上说道:“二位将军,别来久矣!”单廷、魏定
国大笑,指着关胜骂道:“无才小辈,背反狂夫!上负朝廷之恩,下辱祖宗名目,
不知死活!引军到来,有何礼说?”关胜答道:“你二将差矣。目今主上昏昧,奸
臣弄权,非亲不用,非仇不弹。兄长宋公明仁德施恩,替天行道,特令关某等到来,
招请二位将军。倘蒙不弃,便请过来,同归山寨。”单、魏二将听得大怒,骤马齐
出。一个是北方一朵乌云,一个如南方一团烈火,飞出阵前。关胜却待去迎敌,左
手下飞出宣赞,右手下奔出郝思文,两对儿阵前厮杀。刀对刀,迸万道寒光;枪搠
枪,起一天杀气。关胜遥见神火将越斗越精神,圣水将无半点惧色。正斗之间,两
将拨转马头,望本阵便走。郝思文、宣赞随即追赶,冲入阵中。只见魏定国转入左
边,单廷转过右边。随后宣赞赶着魏定国,郝思文追住单廷。

且说宣赞正赶之间,只见四五百步军,都是红旗红甲,一字儿围裹将来,挠钩
齐下,套索飞来,和人连马,活捉去了。再说郝思文追住单廷到右边,只见五百
来步军,尽是黑旗黑甲,一字儿裹转来,脑后众军齐上,把郝思文生擒活捉去了。
可怜二将英雄,到此翻成画饼。一面把人解入凌州,一面仍率五百精兵,卷杀过来。
关胜举手无措,大败输亏,望后便退。随即单廷、魏定国拍马在背后追来。关胜
正走之间,只见前面冲出二将。关胜看时,左有林冲,右有杨志,从两肋窝里撞将
出来,杀散凌州军马。关胜收住本部残兵,与林冲、杨志相见,合兵一处。随后孙
立、黄信一同见了,权且下寨。

却说水火二将,捉得宣赞、郝思文,得胜回到城中,张太守接着,置酒作贺;
一面教人做造陷车,装了二人,差一员偏将,带领三百步军,连夜解上东京,申达
朝廷。且说偏将带领三百人马,监押宣赞、郝思文上东京来,迤前行,来到一个
去处。只见满山枯树,遍地芦芽,一声锣响,撞出一伙强人,当先一个,手双斧,
声喝如雷,正是梁山泊黑旋风李逵。后面带着这个好汉,端的是谁,正是:
相扑丛中人尽伏,拽拳飞脚如刀毒。
劣性发时似山倒,焦挺从来没面目。
李逵、焦挺两个好汉,引着小喽罗,拦住去路,也不打话,便抢陷车。偏将急待要
走,背后又撞出一个好汉,正是:
狰狞丑脸如锅底,双睛迭暴露狼唇。
放火杀人提阔剑,鲍旭名唤丧门神。
这个好汉,正是丧门神鲍旭,向前把偏将手起剑落,砍下马来,其余人等,撇下陷
车,尽皆逃命去了。李逵看时,却是宣赞、郝思文,便问了备细来由。宣赞见李逵
亦问:“你怎生在此?”李逵说道:“为是哥哥不肯教我来厮杀,独自个私走下山
来,先杀了韩伯龙,后撞见焦挺,引我在此。鲍旭一见如故,便如亲兄弟一般接待。
却才商议,正欲去打凌州,却有小喽罗山头上望见这伙人马,监押陷车到来。只道
官兵捕盗,不想却是你二位。”鲍旭邀请到寨内,杀牛置酒相待。郝思文道:“兄
弟既然有心上梁山泊入伙,不若将引本部人马,就同去凌州,并力攻打,此为上策。”
鲍旭道:“小可与李兄正如此商议。足下之言,说的最是。我山寨之中,也有三二
百匹好马。”带领五七百小喽罗,五筹好汉,一齐来打凌州。

却说逃难军士奔回来,报与张太守说道:“半路里有强人夺了陷车,杀了偏将。”
单廷、魏定国听得大怒,便道:“这番拿着,便在这里施刑。”只听得城外关胜
引兵搦战。单廷争先出马,开城门,放下吊桥,引五百玄甲军,飞奔出城迎敌。
门旗开处,圣水将军单廷出马,大骂关胜道:“辱国败将,何不就死!”关胜听
了,舞刀拍马。两个斗不到五十余合,关胜勒转马头,慌忙便走,单廷随即赶将
来。约赶十余里,关胜回头喝道:“你这厮不下马受降,更待何时!”单廷挺枪,
直取关胜后心。关胜使出神威,拖起刀背,只一拍,喝一声:“下去!”单廷落
马。关胜下马,向前扶起,叫道:“将军恕罪!”单廷惶恐伏礼,乞命受降。关
胜道:“某与宋公明哥哥面前,多曾举你。特来相招二位将军,同聚大义。”单廷
答道:“不才愿施犬马之力,同共替天行道。”两个说罢,并马而行。林冲接见
二人并马行来,便问其故。关胜不说输赢,答道:“山僻之内,诉旧论新,招请归
降。”林冲等众皆大喜。单廷回至阵前,大叫一声,五百玄甲军兵,一哄过来;
其余人马,奔入城中去了,连忙报知太守。

魏定国听了,大怒,次日领起军马,出城交战。单廷与同关胜、林冲,直临
阵前。只见门旗开处,神火将军魏定国出马,见了单廷顺了关胜,大骂:“忘恩
背主,负义匹夫!”关胜大怒,拍马向前迎敌。二马相交,军器并举。两将斗不到
十合,魏定国望本阵便走。关胜却欲要追,单廷大叫道:“将军不可去赶。”关
胜连忙勒住战马。说犹未了,凌州阵内,早飞出五百火兵,身穿绛衣,手执火器,
前后拥出有五十辆火车,车上都满装芦苇引火之物。军人背上,各拴铁葫芦一个,
内藏硫黄焰硝,五色烟药,一齐点着,飞抢出来。人近人倒,马过马伤。关胜军兵
四散奔走,退四十余里扎住。魏定国收转军马回城,看见本州烘烘火起,烈烈烟生。
原来却是黑旋风李逵与同焦挺、鲍旭带领枯树山人马,都去凌州背后,打破北门,
杀入城中,放起火来,劫掳仓库钱粮。魏定国知了,不敢入城,慌速回军,被关胜
随后赶上追杀,首尾不能相顾。凌州已失,魏定国只得退走,奔中陵县屯驻。关胜
引军把县四下围住,便令诸将调兵攻打。魏定国闭门不出。

单廷便对关胜、林冲等众位说道:“此人是一勇之夫,攻击得紧,他宁死,
必不辱。事宽即完,急难成效。小弟愿往县中,不避刀斧,用好言招抚此人,束手
来降,免动干戈。”关胜见说,大喜,随即叫单廷单人匹马到县。小校报知,魏
定国出来相见了。单廷用好言说道:“如今朝廷不明,天下大乱,天子昏昧,奸
臣弄权,我等归顺宋公明,且居水泊。久后奸臣退位,那时去邪归正,未为晚矣。”
魏定国听罢,沉吟半晌,说道:“若是要我归顺,须是关胜亲自来请,我便投降;
他若是不来,我宁死不辱!”单廷即便上马回来,报与关胜。关胜见说,便道:
“大丈夫作事,何故疑惑?”便与单廷匹马单刀而去。林冲谏道:“兄长,人心
难忖,三思而行。”关胜道:“好汉作事无妨。”直到县衙。魏定国接着,大喜,
愿拜投降,同叙旧情,设筵管待。当日带领五百火兵,都来大寨,与林冲、杨志并
众头领,俱各相见已了,即便收军回梁山泊来。宋江早使戴宗接着,对李逵说道:
“只为你偷走下山,教众兄弟赶了许多路,如今时迁、乐和、李云、王定六四个,
先回山去了。我如今先去报知哥哥,免至悬望。”

不说戴宗先去了,且说关胜等军马回到金沙滩边,水军头领棹船接济军马,陆
续过渡,只见一个人气急败坏跑将来。众人看时,却是金毛犬段景住。林冲便问道:
“你和杨林、石勇去北地里买马,如何这等慌速跑来?”段景住言无数句,话不一
席,有分教,宋江调拨军兵,来打这个去处,重报旧仇,再雪前恨。正是:情知语
是钩和线,从头钓出是非来。

毕竟段景住说出甚言语来,且听下回分解。