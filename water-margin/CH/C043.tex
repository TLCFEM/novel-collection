\chapter{假李逵剪径劫单人~黑旋风沂岭杀四虎}

话说李逵道:“哥哥,你且说那三件事?”宋江道:“你要去沂州沂水县搬取
母亲,第一件,径回,不可吃酒;第二件,因你性急,谁肯和你同去,你只自悄悄
地取了娘便来;第三件,你使的那两把板斧,休要带去,路上小心在意,早去早回。”
李逵道:“这三件事,有甚么依不得!哥哥放心,我只今日便行,我也不住了。”
当下李逵拽扎得爽利,只跨一口腰刀,提条朴刀,带了一锭大银,三五个小银子,
吃了几杯酒,唱个大喏,别了众人,便下山来,过金沙滩去了。

晁盖、宋江与众头领送行已罢,回到大寨里聚义厅上坐定。宋江放心不下,对
众人说道:“李逵这个兄弟,此去必然有失。不知众兄弟们,谁是他乡中人?可与
他那里探听个消息。”杜迁便道:“只有朱贵原是沂州沂水县人,与他是乡里。”
宋江听罢,说道:“我却忘了。前日在白龙庙聚会时,李逵已自认得朱贵是同乡人。”
宋江便着人去请朱贵,小喽罗飞报下山来,直至店里,请的朱贵到来。宋江道:“今
有李逵兄弟前往家乡搬取老母。因他酒性不好,为此不肯差人与他同去,诚恐路上
有失。今知贤弟是他乡中人,你可去他那里探听,走一遭。”朱贵答道:“小弟是
沂州沂水县人,现在一个兄弟唤做朱富,在本县西门外开着个酒店。这李逵他是本
县百丈村董店东住。有个哥哥,唤做李达,专与人家做长工。这李逵自小凶顽,因
打死了人,逃走在江湖上,一向不曾回归。如今着小弟去那里探听也不妨,只怕店
里无人看管。小弟也多时不曾还乡,亦就要回家探望兄弟一遭。”宋江道:“这个
看店,不必你忧心,我自教侯健、石勇替你暂管几时。”朱贵领了这言语,相辞了
众头领下山来,便走到店里,收拾包裹,交割铺面与石勇、侯健,自奔沂州去了。

这里宋江与晁盖在寨中,每日筵席,饮酒快乐,与吴学究看习天书,不在话下。

且说李逵独自一个离了梁山泊,取路来到沂水县界。于路,李逵端的不吃酒,
因此不惹事,无有话说。行至沂水县西门外,见一簇人围着榜看,李逵也立在人丛
中,听得读道:“榜上第一名正贼宋江,系郓城县人;第二名从贼戴宗,系江州两
院押狱;第三名从贼李逵,系沂州沂水县人。”李逵在背后听了,正待指手画脚,
没做奈何处,只见一个人抢向前来,拦腰抱住,叫道:“张大哥,你在这里做甚么?”
李逵扭过身看时,认得是旱地忽律朱贵。李逵问道:“你如何也来这里?”朱贵道:
“你且跟我来说话。”

两个一同来西门外近村一个酒店内,直入到后面一间静房中坐了。朱贵指着李
逵道:“你好大胆!那榜上明明写着赏一万贯钱捉宋江,五千钱捉戴宗,三千钱捉
李逵,你却如何立在那里看榜?倘或被眼疾手快的拿了送官,如之奈何?宋公明哥哥
只怕你惹事,不肯教人和你同来,又怕你到这里做出怪来,续后特使我赶来探听你
的消息。我迟下山来一日,又先到你一日,你如何今日才到这里?”李逵道:“便
是哥哥分付,教我不要吃酒,以此路上走得慢了。你如何认得这个酒店里?你是这
里人,家在那里住?”朱贵道:“这个酒店,便是我兄弟朱富家里。我原是此间人,
因在江湖上做客,消折了本钱,就于梁山泊落草,今次方回。”又叫兄弟朱富来与
李逵相见了。朱富置酒管待李逵。李逵道:“哥哥分付,教我不要吃酒,今日我已
到乡里了,便吃两碗儿,打甚么鸟紧!”朱贵不敢阻当他,由他吃。

当夜直吃到四更时分,安排些饭食,李逵吃了,趁五更晓星残月,霞光明朗,
便投村里去。朱贵分付道:“休从小路去,只从大朴树转弯,投东大路,一直往百
丈村去,便是董店东。快取了母亲来,和你早回山寨去。”李逵道:“我自从小路
去,却不近?大路走,谁耐烦!”朱贵道:“小路走,多大虫,又有乘势夺包裹的
剪径贼人。”李逵应道:“我却怕甚鸟!”戴上毡笠儿,提了朴刀,跨了腰刀,别
了朱贵、朱富,便出门投百丈村来。

约行了数十里,天色渐渐微明,去那露草之中,赶出一只白兔儿来,望前路去
了。李逵赶了一直,笑道:“那畜生倒引了我一程路。”有诗为证:
山径崎岖静复深,西风黄叶满疏林。
偶因逐兔过前界,不记仓忙行路心。
正走之间,只见前面有五十来株大树丛杂,时值新秋,叶儿正红。李逵来到树林边
厢,只见转过一条大汉,喝道:“是会的留下买路钱,免得夺了包裹。”李逵看那
人时,戴一顶红绢抓儿头巾,穿一领粗布衲袄,手里拿着两把板斧,把黑墨搽在
脸上。李逵见了,大喝一声:“你这厮是甚么鸟人?敢在这里剪径!”那汉道:“若
问我名字,吓碎你心胆,老爷叫做黑旋风。你留下买路钱并包裹,便饶了你性命,
容你过去。”李逵大笑道:“没你娘鸟兴!你这厮是甚么人?那里来的?也学老爷名
目,在这里胡行。”李逵挺起手中朴刀,来奔那汉,那汉那里抵当得住,却待要走,
早被李逵腿股上一朴刀,搠翻在地,一脚踏住胸脯,喝道:“认得老爷么?”那汉
在地下叫道:“爷爷,饶恁孩儿性命!”李逵道:“我正是江湖上的好汉黑旋风李
逵,便是你这厮辱莫老爷名字。”那汉道:“小人虽然姓李,不是真的黑旋风。为
是爷爷江湖上有名目,提起好汉大名,神鬼也怕,因此小人盗学爷爷名目,胡乱在
此剪径。但有孤单客人经过,听得说了黑旋风三个字,便撇了行李,逃奔了去,以
此得这些利息,实不敢害人。小人自己的贱名叫做李鬼,只在这前村住。”李逵道:
“叵耐这厮无礼,却在这里夺人的包裹行李,坏我的名目,学我使两把板斧,且教
他先吃我一斧。”劈手夺过一把斧来便砍,李鬼慌忙叫道:“爷爷杀我一个,便是
杀我两个。”李逵听得,住了手问道:“怎的杀你一个,便是杀你两个?”李鬼道:
“小人本不敢剪径,家中因有个九十岁的老母,无人养赡,因此小人单题爷爷大名
唬吓人,夺些单身的包裹,养赡老母。其实并不曾敢害了一个人。如今爷爷杀了小
人,家中老母,必是饿杀。”

李逵虽是个杀人不眨眼的魔君,听的说了这话,自肚里寻思道:“我特地归家
来取娘,却倒杀了一个养娘的人,天地也不我。罢,罢!我饶了你这厮性命。”
放将起来,李鬼手提着斧,纳头便拜。李逵道:“只我便是真黑旋风,你从今已后,
休要坏了俺的名目。”李鬼道:“小人今番得了性命,自回家改业,再不敢倚着爷
爷名目,在这里剪径。”李逵道:“你有孝顺之心,我与你十两银子做本钱,便去
改业。”李逵便取出一锭银子,把与李鬼,拜谢去了。

李逵自笑道:“这厮却撞在我手里。既然他是个孝顺的人,必去改业,我若杀
了他,也不合天理。我也自去休。”拿了朴刀,一步步投山僻小路而来。诗曰:
李逵迎母却逢伤,李鬼何曾为养娘。
可见世间忠孝处,事情言语贵参详。

走到巳牌时分,看看肚里又饥又渴,四下里都是山径小路,不见有一个酒店饭
店。正走之间,只见远远在山凹里露出两间草屋。李逵见了,奔到那人家里来,只
见后面走出一个妇人来,髻鬓边插一簇野花,搽一脸胭脂铅粉。李逵放下朴刀道:
“嫂子,我是过路客人,肚中饥饿,寻不着酒食店,我与你一贯足钱,央你回些酒
饭吃。”那妇人见了李逵这般模样,不敢说没,只得答道:“酒便没买处,饭便做
些与客人吃了去。”李逵道:“也罢。只多做些个,正肚中饥出鸟来。”那妇人道:
“做一升米不少么?”李逵道:“做三升米饭来吃。”那妇人向厨中烧起火来,便
去溪边淘了米,将来做饭。

李逵却转过屋后山边来净手,只见一个汉子手脚从山后归来。李逵转过屋
后听时,那妇人正要上山讨菜,开后门,见了,便问道:“大哥,那里闪了腿?”
那汉子应道:“大嫂,我险些儿和你不厮见了,你道我晦鸟气么?指望出去等个单
身的过,整整等了半个月,不曾发市,甫能今日抹着一个,你道是谁?原来正是那
真黑旋风。却恨撞着那驴鸟,我如何敌得他过?倒吃他一朴刀,搠翻在地,定要杀
我,吃我假意叫道:‘你杀我一个,却害了我两个。’他便问我缘故,我便告道:
‘家中有个九十岁的老娘,无人养赡,定是饿死。’那驴鸟真个信我,饶了我性命,
又与我一个银子做本钱,教我改了业养娘。我恐怕他省悟了,赶将来,且离了那林
子里僻静处睡了一回,从后山走回家来。”那妇人道:“休要高声。却才一个黑大
汉来家中,教我做饭,莫不正是他。如今在门前坐地,你去张一张看。若是他时,
你去寻些麻药来,放在菜内,教那厮吃了,麻翻在地,我和你却对付了他,谋得他
些金银,搬往县里住,去做些买卖,却不强似在这里剪径!”

李逵已听得了,便道:“叵耐这厮,我倒与了他一个银子,又饶了性命,他倒
又要害我。这个正是情理难容!”一转踅到后门边。这李鬼恰待出门,被李逵劈
揪住,那妇人慌忙自望前门走了。李逵捉住李鬼,按翻在地,身边掣出腰刀,早割
下头来。拿着刀,却奔前门寻那妇人时,正不知走那里去了。再入屋内来,去房中
搜看,只见有两个竹笼,盛些旧衣裳,底下搜得些碎银两并几件钗环,李逵都拿了。
又去李鬼身边搜了那锭小银子,都打缚在包裹里。却去锅里看时,三升米饭早熟了,
只没菜蔬下饭。李逵盛饭来吃了一回,看看自笑道:“好痴汉,放着好肉在面前,
却不会吃。”拔出腰刀,便去李鬼腿上割下两块肉来,把些水洗净了,灶里抓些炭
火来便烧。一面烧,一面吃。吃得饱了,把李鬼的尸首拖放屋下,放了把火,提了
朴刀,自投山路里去了。

比及赶到董店东时,日已平西。径奔到家中,推开门,入进里面,只听得娘在
床上问道:“是谁人来?”李逵看时,见娘双眼都盲了,坐在床上念佛。李逵道:
“娘,铁牛来家了。”娘道:“我儿,你去了许多时,这几年正在那里安身?你的
大哥,只是在人家做长工,止博得些饭食吃,养娘全不济事。我时常思量你,眼泪
流干,因此瞎了双目。你一向正是如何?”李逵寻思道:“我若说在梁山泊落草,
娘定不肯去,我只假说便了。”李逵应道:“铁牛如今做了官,上路特来取娘。”
娘道:“恁地却好也!只是你怎生和我去得?”李逵道:“铁牛背娘到前路,却觅
一辆车儿载去。”娘道:“你等大哥来,却商议。”李逵道:“等做甚么?我自和
你去便了。”恰待要行,只见李达提了一罐子饭来。

入得门,李逵见了,便拜道:“哥哥,多年不见。”李达骂道:“你这厮归来
则甚?又来负累人。”娘便道:“铁牛如今做了官,特地家来取我。”李达道:“娘
呀!休信他放屁。当初他打杀了人,教我披枷带锁,受了万千的苦。如今又听得他
和梁山泊贼人通同,劫了法场,闹了江州,现在梁山泊做了强盗。前日江州行移公
文到来,着落原籍追捕正身,却要捉我到官比捕,又得财主替我官司分理,说他兄
弟已自十来年不知去向,亦不曾回家,莫不是同名同姓的人冒供乡贯?又替我上下
使钱,因此不吃官司杖限追要。现今出榜赏三千钱捉他。你这厮不死,却走家来胡
说乱道!”李逵道:“哥哥不要焦躁,一发和你同上山去快活,多少是好。”李达
大怒,本待要打李逵,却又敌他不过,把饭罐撇在地下,一直去了。

李逵道:“他这一去,必然报人来捉我,却是脱不得身,不如及早走罢。我大
哥从来不曾见这大银,我且留下一锭五十两的大银子,放在床上。大哥归来见了,
必然不赶来。”李逵便解下腰包,取一锭大银,放在床上,叫道:“娘,我自背你
去休。”娘道:“你背我那里去?”李逵道:“你休问我,只顾去快活便了。我自
背你去不妨。”李逵当下背了娘,提了朴刀,出门望小路里便走。

却说李达奔来财主家报了,领着十来个庄客,飞也似赶到家里看时,不见了老
娘,只见床上留下一锭大银子。李达见了这锭大银,心中忖道:“铁牛留下银子,
背娘去那里藏了。必是梁山泊有人和他来,我若赶去,倒吃他坏了性命。想他背娘,
必去山寨里快活。”众人不见了李逵,都没做理会处。李达却对众庄客说道:“这
铁牛背娘去,不知往那条路去了,这里小路甚杂,怎地去赶他?”众庄客见李达没
理会处,俄延了半晌,也各自回去了,不在话下。

这里只说李逵怕李达领人赶来,背着娘只望乱山深处僻静小路而走。看看天色
晚了,但见:

暮烟横远岫,宿雾锁奇峰。慈鸦撩乱投林,百鸟喧呼傍树。行行雁阵,坠长空
飞入芦花;点点萤光,明野径偏依腐草。卷起金风飘败叶,吹来霜气布深山。
当下李逵背娘到岭下,天色已晚了。娘双眼不明,不知早晚。李逵却自认得这条岭,
唤做沂岭。过那边去,方才有人家。娘儿两个,趁着星明月朗,一步步捱上岭来。
娘在背上说道:“我儿,那里讨口水来我吃也好。”李逵道:“老娘,且待过岭去,
借了人家安歇了,做些饭吃。”娘道:“我日中吃了些干饭,口渴的当不得。”李
逵道:“我喉咙里也烟发火出。你且等我背你到岭上,寻水与你吃。”娘道:“我
儿,端的渴杀我也!救我一救!”李逵道:“我也困倦的要不得。”李逵看看捱得
到岭上,松树边一块大青石上,把娘放下,插了朴刀在侧边,分付娘道:“耐心坐
一坐,我去寻水来你吃。”李逵听得溪涧里水响,闻声寻将去,盘过了两三处山脚,
到得那涧边看时,一溪好水。怎见得,有诗为证:
穿崖透壑不辞劳,远望方知出处高。
溪涧岂能留得住,终归大海作波涛。

李逵来到溪边,捧起水来,自吃了几口,寻思道:“怎生能够得这水去,把与
娘吃?”立起身来,东观西望,远远地山顶上见个庵儿,李逵道:“好了。”攀藤
揽葛,上到庵前,推开门看时,却是个泗州大圣祠堂。面前有个石香炉。李逵用手
去掇,原来却是和座子凿成的。李逵拔了一回,那里拔得动,一时性起来,连那座
子掇出,前面石阶上一磕,把那香炉磕将下来,拿了再到溪边,将这香炉水里浸了,
拔起乱草,洗得干净,挽了半香炉水,双手擎来,再寻旧路,夹七夹八走上岭来。

到得松树里边,石头上不见了娘,只见朴刀插在那里。李逵叫娘吃水,杳无踪
迹,叫了几声不应。李逵心慌,丢了香炉,定住眼四下里看时,并不见娘。走不到
三十余步,只见草地上一团血迹。李逵见了,心里越疑惑,趁着那血迹寻将去。寻
到一处大洞口,只见两个小虎儿在那里舐一条人腿。正是:
假黑旋风真捣鬼,生时欺心死烧腿。
谁知娘腿亦遭伤,饿虎饿人皆为嘴。
李逵心里忖道:“我从梁山泊归来,特为老娘来取他,千辛万苦,背到这里,却把
来与你吃了。那鸟大虫拖着这条人腿,不是我娘的是谁的?”心头火起,赤黄须竖
立起来,将手中朴刀挺起来,搠那两个小虎。这小大虫被搠得慌,也张牙舞爪钻向
前来,被李逵手起,先搠死了一个,那一个望洞里便钻了入去。李逵赶到洞里,也
搠死了。李逵却钻入那大虫洞内,伏在里面张外面时,只见那母大虫张牙舞爪望窝
里来。李逵道:“正是你这业畜吃了我娘。”放下朴刀,胯边掣出腰刀。那母大虫
到洞口,先把尾去窝里一剪,便把后半截身躯坐将入去。李逵在窝内看得仔细,把
刀朝母大虫尾底下尽平生气力舍命一戳,正中那母大虫粪门。李逵使得力重,和那
刀靶,也直送入肚里去了。那母大虫吼了一声,就洞口带着刀,跳过涧边去了。李
逵却拿了朴刀,就洞里赶将出来,那老虎负疼,直抢下山石岩下去了。李逵恰待要
赶,只见就树边卷起一阵狂风,吹得败叶树木如雨一般打将下来。自古道:“云生
从龙,风生从虎。”那一阵风起处,星月光辉之下,大吼了一声,忽地跳出一只吊
睛白额虎来。那大虫望李逵势猛一扑,那李逵不慌不忙,趁着那大虫的势力,手起
一刀,正中那大虫颔下。那大虫不曾再展再扑:一者护那疼痛,二者伤着他那气管。
那大虫退不够五七步,只听得响一声,如倒半壁山,登时间死在岩下。

那李逵一时间杀了子母四虎,还又到虎窝边,将着刀复看了一遍,只恐还有大
虫,已无有踪迹。李逵也困乏了,走向泗州大圣庙里,睡到天明。次日早晨,李逵
却来收拾亲娘的两腿及剩的骨殖,把布衫包裹了,直到泗州大圣庵后掘土坑葬了。
李逵大哭了一场,有诗为证:
沂岭西风九月秋,雌雄虎子聚林丘。
因将老母残躯啖,致使英雄血泪流。
猛拚一身探虎穴,立诛四虎报冤仇。
泗州庙后亲埋葬,千古传名李铁牛。

这李逵肚里又饥又渴,不免收拾包裹,拿了朴刀,寻路慢慢的走过岭来。只见
五七个猎户都在那里收窝弓弩箭,见了李逵一身血污,行将下岭来,众猎户吃了一
惊,问道:“你这客人莫非是山神土地,如何敢独自过岭来?”李逵见问,自肚里
寻思道:“如今沂水县出榜,赏三千贯钱捉我,我如何敢说实话?只谎说罢。”答
道:“我是客人。昨夜和娘过岭来,因我娘要水吃,我去岭下取水,被那大虫把我
娘拖去吃了。我直寻到虎窝里,先杀了两个小虎,后杀了两个大虎,泗州大圣庙里
睡到天明,方才下来。”众猎户齐叫道:“不信你一个人如何杀得四个虎?便是李
存孝和子路也只打得一个。这两个小虎且不打紧,那两个大虎非同小可。我们为这
两个畜生,不知都吃了几顿棍棒。这条沂岭自从有了这窝虎在上面,整三五个月,
没人敢行。我们不信,敢是你哄我?”李逵道:“我又不是此间人,没来由哄你做
甚么?你们不信,我和你上岭去,寻讨与你。就带些人去扛了下来。”众猎户道:
“若端的有时,我们自重重的谢你。却是好也!”

众猎户打起胡哨来,一霎时聚起三五十人,都拿了挠钩枪棒,跟着李逵,再上
岭来。此时天大明朗,都到那山顶上。远远望见窝边果然杀死两个小虎,一个在窝
内,一个在外面;一只母大虫死在山岩边,一只雄虎死在泗州大圣庙前。众猎户见
了杀死四个大虫,尽皆欢喜,便把索子抓缚起来,众人扛抬下岭,就邀李逵同去请
赏,一面先使人报知里正上户,都来迎接着,抬到一个大户人家,唤做曹太公庄上。
那人原是闲吏,专一在乡放刁把滥。近来暴有几贯浮财,只是为人行短。当时曹太
公亲自接来相见了,邀请李逵到草堂上坐定,动问那杀虎的缘由。李逵却把夜来同
娘到岭上要水吃,因此杀死大虫的话,说了一遍。众人都呆了。曹太公动问壮士高
姓名讳,李逵答道:“我姓张,无名,只唤做张大胆。”诗曰:
人言只有假李逵,从来再无李逵假。
如何李四冒张三,谁假谁真皆作耍。
曹太公道:“真乃是大胆壮士,不恁地胆大,如何杀的四个大虫!”一壁厢叫安排
酒食管待,不在话下。

且说当村里得知沂岭上杀了四个大虫,抬在曹太公家,讲动了村坊道店,哄的
前村后村,山僻人家,大男幼女,成群拽队,都来看虎,入见曹太公,相待着打虎
的壮士,在厅上吃酒。数中却有李鬼的老婆,逃在前村爹娘家里,随着众人也来看
虎,却认得李逵的模样,慌忙来家对爹娘说道:“这个杀虎的黑大汉,便是杀我老
公,烧了我屋的。他正是梁山泊黑旋风李逵。”爹娘听得,连忙来报知里正。里正
听了道:“他既是黑旋风时,正是岭后百丈村打死了人的李逵,逃走在江州,又做
出事来,行移到本县原籍追捉,如今官司出三千贯赏钱拿他,他却走在这里!”暗
地使人去请得曹太公到来商议。曹太公推道更衣,急急的到里正家里。正说这个杀
虎的壮士,便是岭后百丈村里的黑旋风李逵,现今官司着落拿他。曹太公道:“你
们要打听得仔细。倘不是时,倒惹得不好;若真个是时,却不妨。要拿他时也容易,
只怕不是他时却难。”里正道:“现有李鬼的老婆认得他。曾来李鬼家做饭吃,杀
了李鬼。”曹太公道:“既是如此,我们且只顾置酒请他,却问他:‘今番杀了大
虫,还是要去县请功,只是要村里讨赏?’若还他不肯去县里请功时,便是黑旋风
了,着人轮换把盏,灌得醉了,缚在这里,却去报知本县,差都头来取去,万无一
失。”有诗为证:
常言芥投针孔,窄路每遇冤家。
李鬼鬼魂不散,旋风风色非佳。
打虎功思县赏,杀人身被官拿。
试看螳螂黄雀,劝君得意休夸。

众人道:“说得是。”里正与众人商量定了。曹太公回家来款住李逵,一面且
置酒来相待,便道:“适间抛撇,请勿见怪。且请壮士解下腰间包裹,放下朴刀,
宽松坐一坐。”李逵道:“好,好!我的腰刀已搠在雌虎肚里了,只有刀鞘在这里。
若是开剥时,可讨来还我。”曹太公道:“壮士放心,我这里有的是好刀,相送一
把与壮士悬带。”李逵解了腰刀、尖刀,并缠袋、包裹,都递与庄客收贮,便把朴
刀倚在壁边。曹太公叫取大盘肉、大壶酒来。众多大户并里正、猎户人等,轮番把
盏,大碗大钟,只顾劝李逵。曹太公又请问道:“不知壮士要将这虎解官请功,只
是在这里讨些赍发!”李逵道:“我是过往客人,忙些个,偶然杀了这窝猛虎,不
须去县里请功。只此有些赍发,便罢;若无,我也去了。”曹太公道:“如何敢轻
慢了壮士?少刻村中敛取盘缠相送。我这里自解虎到县里去。”李逵道:“布衫先
借一领与我换了上盖。”曹太公道:“有,有。”当时便取一领细青布衲袄,就与
李逵换了身上的血污衣裳。只见门前鼓响笛鸣,都将酒来,与李逵把盏作庆,一杯
冷,一杯热。李逵不知是计,只顾开怀畅饮,全不记宋江分付的言语。不两个时辰,
把李逵灌得酩酊大醉,立脚不住。众人扶到后堂空屋下,放翻在一条板凳上,就取
两条绳子,连板凳绑住了。便叫里正带人,飞也似去县里报知。就引李鬼老婆去做
原告,补了一纸状子。

此时哄动了沂水县里,知县听得大惊,连忙升厅问道:“黑旋风拿住在那里?
这是谋叛的人,不可走了。”原告人并猎户答应道:“现缚在本乡曹大户家,为是
无人禁得他,诚恐有失,路上走了,不敢解来。”知县随即叫唤本县都头去取来。
就厅前转过一个都头来声喏,那人是谁,有诗为证:
面阔眉浓须鬓赤,双睛碧绿似番人。
沂水县中青眼虎,豪杰都头是李云。
当下知县唤李云上厅来,分付道:“沂岭下曹大户庄上拿住黑旋风李逵,你可多带
人去,密地解来,休要哄动村坊,被他走了。”李都头领了台旨,下厅来,点起三
十个老郎土兵,各带了器械,便奔沂岭村中来。

这沂水县是个小去处,如何掩饰得过?此时街市上讲动了,说道:“拿着了闹
江州的黑旋风。如今差李都头去拿来。”朱贵在东庄门外朱富家听了这个消息,慌
忙来后面对兄弟朱富说道:“这黑厮又做出来了,如何解救?宋公明特为他,诚恐
有失,差我来打听消息。如今他吃拿了,我若不救得他时,怎的回寨去见哥哥,似
此怎生是好?”朱富道:“大哥且不要慌。这李都头一身好本事,有三五十人近他
不得,我和你只两个同心合意,如何敢近傍他?只可智取,不可力敌。李云日常时
最是爱我,常常教我使些器械,我却有个道理对他,只是在这里安不得身了。今晚
煮了三二十斤肉,将十数瓶酒,把肉大块切了,却将些蒙汗药拌在里面,我两个五
更带数个火家挑着,去半路里僻静处等候他解来时,只做与他把酒贺喜,将众人都
麻翻了,却放李逵,如何?”朱贵道:“此计大妙。事不宜迟,可以整顿,及早便
去。”朱富道:“只是李云不会吃酒,便麻翻了,终久醒得快。还有件事:倘或日
后得知,须在此安身不得。”朱贵道:“兄弟,你在这里卖酒,也不济事。不如带
领老小,跟我上山,一发入了伙,论秤分金银,换套穿衣服,却不快活?今夜便叫
两个火家觅了一辆车儿,先送妻子和细软行李起身,约在十里牌等候,都去上山。
我如今包裹内带得一包蒙汗药在这里,李云不会吃酒时,肉里多糁些,逼着他多吃
些,也麻倒了,救得李逵同上山去,有何不可。”朱富道:“哥哥说得是。”便叫
人去觅下了一辆车儿,打拴了三五个包箱,捎在车儿上,家中粗物都弃了,叫浑家
和儿女上了车子,分付两个火家,跟着车子,只顾先去。

且说朱贵、朱富当夜煮熟了肉,切做大块,将药来拌了,连酒装做两担,带了
二三十个空碗。又有若干菜蔬,也把药来拌了。恐有不吃肉的,也教他着手。两担
酒肉,两个火家各挑一担。弟兄两个,自提了些果盒之类,四更前后,直接将来僻
静山路口坐。等到天明,远远地只听得敲着锣响,朱贵接到路口。

且说那三十来个土兵自村里吃了半夜酒,四更前后,把李逵背剪绑了,解将来。
后面李都头坐在马上,看看来到面前。朱富便向前拦住,叫道:“师父且喜,小弟
将来接力。”桶内舀一壶酒来,斟一大钟,上劝李云。朱贵托着肉来,火家捧过果
盒。李云见了,慌忙下马,跳向前来,说道:“贤弟,何劳如此远接。”朱富道:
“聊表徒弟孝顺之心。”李云接过酒来,到口不吃,朱富跪下道:“小弟已知师父
不饮酒。今日这个喜酒,也饮半盏儿。”李云推却不过,略呷了两口。朱富便道:
“师父不饮酒,须请些肉。”李云道:“夜间已饱,吃不得了。”朱富道:“师父
行了许多路,肚里也饥了。虽不中吃,胡乱请些,也免小弟之羞。”拣两块好的,
递将过来。李云见他如此殷勤,只得勉意吃了两块。朱富把酒来劝上户、里正,并
猎户人等,都劝了三钟,朱贵便叫土兵、庄客众人都来吃酒。这伙男女那里顾个冷
热、好吃不好吃,酒肉到口,只顾吃,正如这风卷残云,落花流水,一齐上来,抢
着吃了。

李逵光着眼,看了朱贵兄弟两个,已知用计,故意道:“你们也请我吃些。”
朱贵喝道:“你是歹人,有何酒肉与你吃?这般杀才,快闭了口!”李云看着土兵,
喝道叫走,只见一个个都面面厮觑,走动不得,口颤脚麻,都跌倒了。李云急叫:
“中了计了。”恰待向前,不觉自家也头重脚轻,晕倒了,软做一堆,睡在地下。
当时朱贵、朱富各夺了一条朴刀,喝声:“孩儿们休走!”两个挺起朴刀,来赶这
伙不曾吃酒肉的庄客,并那看的人。走得快的,走了;走得迟的,就搠死在地。李
逵大叫一声,把那绑缚的麻绳都挣断了,便夺过一条朴刀来杀李云。朱富慌忙拦住
叫道:“不要害他。他是我的师父,为人最好,你只顾先走。”李逵应道:“不杀
得曹太公老驴,如何出得这口气!”李逵赶上,手起一朴刀,先搠死曹太公,并李
鬼的老婆,续后里正也杀了。性起来,把猎户排头儿一味价搠将去,那三十来个土
兵都被搠死了。这看的人和众庄客只恨爹娘少生两只脚,都望深村野路逃命去了。

李逵还只顾寻人要杀,朱贵喝道:“不干看的人事,休只管伤人。”慌忙拦住,
李逵方才住了手,就土兵身上剥了两件衣服穿上。三个人提着朴刀,便要从小路里
走。朱富道:“不好,却是我送了师父性命。他醒时,如何见的知县,必然赶来。
你两个先行,我等他一等。我想他日前教我的恩义,且是为人忠直,等他赶来,就
请他一发上山入伙,也是我的恩义,免得教回县去吃苦。”朱贵道:“兄弟,你也
见的是,我便先去跟了车子行,留李逵在路旁帮你等他。只有李云那厮吃的药少,
没一个时辰便醒。若是他不赶来时,你们两个休执迷等他。”朱富道:“这是自然
了。”

当下朱贵前行去了。只说朱富和李逵坐在路旁边等候,果然不到一个时辰,只
见李云挺着一条朴刀,飞也似赶来,大叫道:“强贼休走!”李逵见他来的凶,跳
起身,挺着朴刀,来斗李云,恐伤朱富。正是有分教:梁山泊内添双虎,聚义厅前
庆四人。

毕竟黑旋风斗青眼虎,二人胜败如何,且听下回分解。