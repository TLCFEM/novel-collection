\chapter{张清琼英双建功~陈观宋江同奏捷}

话说太原县城池,被混江龙李俊,乘大雨后水势暴涨,同二张、三阮,统领水
军,约定时刻,分头决引智伯渠及晋水,灌浸太原城池。顷刻间,水势汹涌。但见:
骤然飞急水,忽地起洪波。军卒乘木筏冲来,将士驾天潢飞至。神号鬼哭,昏昏日
色无光;岳撼山崩,浩浩波声若怒。城垣尽倒,窝铺皆休。旗帜随波,不见青红交
杂;兵戈汩浪,难排霜雪争叉。僵尸如鱼鳖沉浮,热血与波涛并沸。须臾树木连根
起,顷刻榱题贴水飞。
当时城中鼎沸,军民将士,见水突至,都是水渌渌的爬墙上屋,攀木抱梁,老弱肥
胖的,只好上台上桌。转眼间,连桌凳也浮起来,房屋倾圮,都做了水中鱼鳖。城
外李俊、二张、三阮,乘着飞江、天浮,逼近城来,恰与城垣高下相等。军士攀缘
上城,各执利刃,砍杀守城士卒。又有军士乘木筏冲来,城垣被冲,无不倾倒。张
雄正在城楼上叫苦不迭,被张横、张顺从飞江上城,手执朴刀,喊一声,抢上楼来,
一连砍翻了十余个军卒,众人乱窜逃生。张雄躲避不迭,被张横一朴刀砍翻,张顺
赶上前,察的一刀,剁下头来。比及水势四散退去,城内军民,沉溺的,压杀的,
已是无数。梁柱门扇,窗什物,尸骸顺流壅塞南城。城中只有避暑宫,乃是北齐
神武帝所建,基址高固,当下附近军民,一齐抢上去,挨挤践踏,死的也有二千余
人。连那高阜及城垣上,一总所存军民,仅千余人。城外百姓,却得卢先锋密唤里
保,传谕居民,预先摆布,锣声一响,即时都上高阜。况城外四散空阔,水势去的
快,因此城外百姓,不致湮没。
当下混江龙李俊,领水军据了西门;船火儿张横,同浪里白跳张顺,夺了北门;立
地太岁阮小二、短命二郎阮小五,占了东门;活阎罗阮小七,夺了南门。四门俱竖
起宋军旗号。至晚水退,现出平地,李俊等大开城门,请卢先锋等军马入城。城中
鸡犬不闻,尸骸山积。虽是张雄等恶贯满盈,李俊这条计策,也忒惨毒了。那千余
人,四散的跪在泥水地上,插烛也似磕头乞命。卢俊义查点这伙人中,只有十数个
军卒,其余都是百姓。项忠、徐岳爬在帅府后傍屋的大桧树上,见水退,溜将下来,
被南军获住,解到卢先锋处。卢俊义教斩首示众。给发本县府库中银两,赈济城内
外被水百姓。差人往宋先锋处报捷。一面令军士埋葬尸骸,修筑城垣房屋,召民居
住。
不说卢俊义在太原县抚绥料理,再说太原未破时,田虎统领十万大军,因雨在铜
山南屯扎,探马报来,邬国舅病亡,郡主、郡马即退军到襄垣,殡殓国舅。田虎大
惊,差人在襄垣城中传旨,着琼英在城中镇守,着全羽前来听用,并问为何差往襄
垣人役,都不来回奏。
次日雨霁,平明时分,流星探马飞报将来,说宋江差孙安、马灵,领兵前来拒敌。
田虎听报,大怒道:“孙安、马灵,都受我高官厚禄,今日反叛,情理难容。待寡
人亲自去问他。卿等努力,如有擒得二人者,千金赏,万户侯。”当下田虎亲自驱
兵向前,与宋兵相对。北军观看宋军旗号,原来是病尉迟孙立、铁笛仙马麟。北阵
前金瓜密布,铁斧齐排,剑戟成行,旗幡作队。那九曲飞龙赭黄伞下,玉辔金鞍,
银鬃白马上,坐着那个草头大王田虎。出到阵前,亲自监战。南阵后,宋江统领吴
用、孙新、顾大嫂、王英、扈三娘、孙立、朱仝、燕顺兵马又到。宋江也亲自督战。
田虎闻说是宋江,方欲遣将出阵,擒捉宋江,只听得飞马报道:“关胜等连破榆社、
大谷两个城池;西路卢俊义军马又打破平遥、介休两县,被他引水灌了太原城池,
城中兵将,不留一个;右丞相卞祥扎寨绵山,与花荣等相持,被卢俊义从太原领兵,
后面杀来。卞丞相当不得两面夹攻,大败亏输,卞祥被卢俊义活捉过阵去。卢俊义
同关胜合兵一处,将沁源县围得铁桶相似。”田虎听罢,大惊无措,忙传令旨,便
教收军,退保威胜城内。
当下李天锡等押住阵脚,薛时、林昕、胡英、唐昌保护田虎先行。只听的铜山北,
炮声振响,被宋江密教鲁智深、刘唐、鲍旭、项充、李衮,统领精勇步兵,抄出铜
山北,分两路杀奔前来。田虎急驱御林军马来战,忽被马灵、孙安领兵马从东铲
斜里杀来。马灵脚踏风火二轮,将金砖望北军乱打。孙安挥双剑砍杀。二将领兵,
突入北阵,如入无人之境,把北军冲做两截。北军虽有十万之众,被吴用筹画这三
路兵马,横冲直撞,纵横乱杀,北军大败,杀得星落云散,七断八续。当下伪尚书
李天锡等保护田虎,望东冲杀逃奔,却被鲁智深等领着标枪、团牌、飞刀手,冲开
血路,杀奔前来。又把李天锡、郑之瑞、薛时、林昕等军马,冲散奔西。田虎手下,
虽是御林军马,挑选那最精勇的,他们自来与官军斗敌,从未曾见有恁般凶猛的,
今日如何抵当得住!
当下田虎左右,只有都督胡英、唐昌、总管叶清,及金吾较尉等将,领着五千败残
军马,拥护奔逃。正在危急,忽的又有一彪军马,从东突至。田虎见了,仰天大叹
道:“天丧我也!”北军看那彪军马中,当先一个俊庞年少将军,头戴青巾绩,身
穿绿战袍,手执梨花枪,坐匹高头雪白卷毛马,旗号上写的分明,乃是“中兴平南
先锋郡马全羽”。那时叶清紧随田虎,看了旗号,奏知田虎。田虎传旨,快教郡马
救驾。那全郡马近前,下马跪奏道:“臣启大王:甲在身,不能俯伏,臣该万死。”
田虎道:“赦卿无罪。”全郡马又奏道:“事在危急,奉请大王到襄垣城中,权避
敌锋。待臣同郡主杀退宋兵,再请大王到威胜大内,计议良策,恢复基业。”田虎
大喜,传下令旨,即望襄垣进发。全郡马在后面,抵当追赶的兵将。田虎等众,已
到襄垣城下,背后喊杀连天,追赶将来。襄垣城上守城将士看见,连忙开城门,放
吊桥。胡英引兵在前,军士听见后面赶来,一拥抢进城去,也顾不得甚么大王。
胡英刚进得城门,猛听得一声梆子响,两边伏兵齐发,将胡英及三千余人,都赶入
陷坑中去,被军士把长枪乱搠,可怜三千余人,不留半个。城中大叫:“田虎要活
的!”田虎见城中变起,方知是计,急勒马望北奔走。张清、叶清拍马赶来,田虎
那匹好马行得快,张清、叶清领军士追赶不上,已离了一箭之地,只见田虎马前,
忽地起阵旋风,风中现出一个女子,大叫道:“奸贼田虎,我仇家夫妇,都被汝害
了,今日走到那里去?”就女子身旁,又起一阵阴风,望田虎劈面滚来,那女子寂
然不见。田虎坐下马,忽然惊跃嘶鸣,田虎落马堕地,被张清、叶清赶上,跳下马
来,同军士一拥上前擒住。唐昌领众挺枪骤马来救。张清见唐昌抢来,疾忙上马,
拈一石子飞来,正中唐昌面门,撞下马去。张清大叫道:“我不是甚么全羽,乃是
天朝宋先锋部下没羽箭张清。”那时李逵、武松,领五百步兵,从城内抢出来,二
人大吼一声,把那殿帅将军、金吾较尉等二千余人,杀的星落云散。张清刺杀了唐
昌,缚了田虎,簇拥入城,闭了城门,待宋先锋杀退北兵,方可解去。鲁智深追赶
到来,见田虎已捉入城去,鲁智深等复向西杀到铜山侧。此时已是酉牌时分。
宋江等三路军马与北兵鏖战一日,杀死军士二万余人。北军无主,四面八方,乱窜
逃生。范美人及姬妾等项,都被乱兵所杀。李天锡、郑之瑞、薛时、林昕,领三万
余人,上铜山据住。宋江领兵四面围困。鲁智深来报,田虎已被张清擒捉。宋江
以手加额,忙传将令,差军星夜疾驰到襄垣,教武松等坚闭城门,看守田虎。教张
清领兵速到威胜,策应琼英等。
原来琼英已奉吴军师密计,同解珍、解宝、乐和、段景住、王定六、郁保四、蔡福、
蔡庆,带领五千军马,尽着北军旗号,伏于武乡县城外石盘山侧。琼英等探知田虎
与我兵厮杀,琼英领众人星夜疾驰到威胜城下。是日天晚,已是暮霞敛彩,新月垂
钩,琼英在城下莺声娇啭叫道:“我乃郡主,保护大王到此,快开城门!”当下守
城军卒,飞报王宫内里。田豹、田彪闻报,上马疾驰到南城,忙上城楼观看,果见
赭黄伞下,那匹雕鞍银鬃白马上,坐着大王,马前一个女将,旗上大书“郡主琼英”,
后面有尚书都督等官,远远跟随。只见琼英高声叫道:“胡都督等与宋兵战败,我
特保护大王到此。教官员速出城接驾!”田豹等见是田虎,即令开了城门,出城迎
接。二人才到马前,只听马上的大王大喝道:“武士与寡人拿下二贼。”军士一拥
上前,将二人擒住。田豹、田彪大叫:“我二人无罪!”急要挣扎时,已被军士将
绳索绑缚了。原来这个田虎,乃是吴用教孙安拣择南军中与田虎一般面貌的一个军
卒,依着田虎妆束。后面尚书都督,却是解珍、解宝等数人假扮的。当下众人各掣
出兵器,王定六、郁保四、蔡福、蔡庆领五百余人,将田豹、田彪连夜解往襄垣去
了。
城上见捉了田豹、田彪,又见将二人押解向南,情知有诈,急出城来抢时,却被琼
英要杀田定,不顾性命,同解珍、解宝一拥抢入城来。守门将士上前来斗敌,被琼
英飞石子打去,一连伤了六七个人。解珍、解宝帮助琼英厮杀,城外乐和、段景住,
急教军士卸下北军打扮,个个是南军号衣,一齐抢入城来,夺了南门。乐和、段景
住挺朴刀,领军上城,杀散军士,竖起宋军旗号。城中一时鼎沸起来,尚有许多伪
文武官员,及王亲国戚等众,急引兵来厮杀。琼英这四千余人,深入巢穴,如何抵
敌?却得张清领八千余人到来,驱兵入城,见琼英、解珍、解宝与北兵正在鏖战,
张清上前飞石,连打四员北将,杀退北军。张清对琼英道:“不该深入重地,又且
众寡不敌。”琼英道:“欲报父仇,虽粉骨碎身,亦所不辞!”张清道:“田虎已
被我擒捉在襄垣了。”琼英方才喜欢。
正欲引兵出城,也是天厌贼众之恶,又得卢俊义打破沁源城池,统领大兵到来,见
了南门旗号,急驱兵马入城,与张清合兵一处,赶杀北军。秦明、杨志、杜迁、宋
万,领兵夺了东门;欧鹏、邓飞、雷横、杨林,夺了西门;黄信、陈达、杨春、周
通,领兵夺了北门;杨雄、石秀、焦挺、穆春、郑天寿、邹渊、邹润,领步兵,大
刀阔斧,从王宫前面砍杀入去;龚旺、丁得孙、李立、石勇、陶宗旺,领步兵,从
后宰门砍杀入去。杀死王宫内院嫔妃、姬妾、内侍人等无算。田定闻变,自刎身死。
张清、琼英、张青、孙二娘、唐斌、文仲容、崔、耿恭、曹正、薛永、李忠、朱
富、时迁、白胜,分头去杀伪尚书、伪殿帅、伪枢密以下等众,及伪封的王亲国戚
等贼徒,正是:
金阶殿下人头滚,玉砌朝门热血喷。
莫道不分玉与石,为庆为殃心自扪。
当下宋兵在威胜城中,杀的尸横市井,血满沟渠。卢俊义传令,不得杀害百姓,连
忙差人先往宋先锋处报捷。当夜宋兵直闹至五更方息,军将降者甚多。
天明,卢俊义计点将佐,除神机军师朱武在沁源城中镇守外,其余将佐,都无伤损。
只有降将耿恭,被人马践踏身死。众将都来献功。焦挺将田定死尸驼来,琼英咬牙
切齿,拔佩刀割了首级,把他尸骸支解。此时邬梨老婆倪氏已死,琼英寻了叶清妻
子安氏,辞别卢俊义,同张清到襄垣,将田虎等押解到宋先锋处。卢俊义正在料理
军务,忽有探马报来,说北将房学度将索超、汤隆围困在榆社县。卢俊义即教关胜、
秦明、雷横、陈达、杨春、杨林、周通,领兵去解救索超等。
次日,宋江已破李天锡等于铜山,一面差人申报陈安抚说:“贼巢已破,贼首已
擒,
请安抚到威胜城中料理。”宋江统领大兵,已到威胜城外,卢俊义等迎接入城。宋
江出榜,安抚百姓。卢俊义将卞祥解来。宋江见卞祥状貌魁伟,亲释其缚,以礼相
待。卞祥见宋江如此意气,感激归降。次日,张清、琼英、叶清将田虎、田豹、田
彪,囚载陷车,解送到来。琼英同了张清,双双的拜见伯伯宋先锋。琼英拜谢王英
等昔日冒犯之罪。宋江叫将田虎等监在一边,待大军班师,一同解送东京献俘。即
教置酒,与张清、琼英庆贺。当日有威胜属县武乡守城将士方顺等,将军民户口册
籍、仓库钱粮,前来献纳。宋江赏劳毕,仍令方顺依旧镇守。宋江在威胜城一连过
了两日,探马报到,说关胜等到榆社县,同索超、汤隆内外夹攻,杀了北将房学度。
北军死者五千余人,其余军士都降。宋江大喜,对众将道:“都赖众兄弟之力,得
成平寇之功。”即细细标写众将功劳,及张清、琼英擒贼首、捣贼巢的大功。又过
了三四日,关胜兵马方到,又报陈安抚兵马也到了。
宋江统领将佐,出郭迎接入城,参见已毕,陈安抚称赞道:“将军等五月之内,成
不世之功。下官一闻擒捉贼首,先将表文差人马上驰往京师奏凯,朝廷必当重封官
爵。”宋江再拜称谢。
次日,琼英来禀,欲往太原石室山,寻觅母亲尸骸埋葬,宋江即命张清、叶清同去,
不题。
宋江禀过陈安抚,将田虎宫殿院宇,珠轩翠屋,尽行烧毁。又与陈安抚计议,发仓
廪,赈济各处遭兵被火居民。修书申呈宿太尉,写表申奏朝廷,差戴宗即日起行。
戴宗擎赍表文书札,赶上陈安抚差的赍奏官,一同入进东京,先到宿太尉府前,依
先寻了杨虞候,将书呈递。宿太尉大喜,明日早朝,并陈安抚表文,一同上达天听。
道君皇帝龙颜喜悦,敕宋江等料理候代,班师回京,封官受爵。戴宗得了这个消息,
即日拜辞宿太尉,离了东京,明日未牌时分,便到威胜城中,报知陈安抚、宋先锋。
陈观、宋江一面教把生擒到贼徒伪官等众,除留田虎、田豹、田彪,另行解赴东京,
其余从贼,都就威胜市曹斩首施行。所有未收去处,乃是晋宁所属蒲、解等州县。
贼役赃官,得知田虎已被擒获,一半逃散,一半自行投首。陈安抚尽皆准首,复为
良民。就行出榜去各处招抚,以安百姓。其余随从贼徒,不伤人者,亦准其自首投
降,复为乡民,给还产业田园。克复州县已了,各调守御官军,护境安民,不在话
下。
再说道君皇帝已降诏敕,差官赍领,到河北谕陈观等。次日,临幸武学,百官先集,
蔡京于坐上谈兵,众皆拱听。内中却有一官,仰着面孔,看视屋角,不去睬他。蔡
京大怒,连忙查问那官员姓名。正是:一人向隅,满坐不乐。只因蔡京查这个官员
姓名,直教:天罡地煞临轸翼,猛将雄兵定楚郢。
毕竟蔡京查问那官员是谁,且听下回分解。