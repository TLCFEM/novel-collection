\chapter{及时雨会神行太保~黑旋风斗浪里白跳}

话说当时宋江别了差拨,出抄事房来,到点视厅上看时,见那节级掇条凳子坐
在厅前,高声喝道:“那个是新配到囚徒?”牌头指着宋江道:“这个便是。”那
节级便骂道:“你这黑矮杀才,倚仗谁的势要,不送常例钱来与我?”宋江道:“‘人
情人情,在人情愿。’你如何逼取人财?好小哉相!”两边看的人听了,倒捏两把
汗。那人大怒,喝骂:“贼配军,安敢如此无礼!颠倒说我小哉!那兜驮的,与我背
起来,且打这厮一百讯棍!”两边营里众人都是和宋江好的,见说要打他,一哄都
走了,只剩得那节级和宋江。那人见众人都散了,肚里越怒,拿起讯棍,便奔来打
宋江。宋江说道:“节级,你要打我,我得何罪?”那人大喝道:“你这贼配军,
是我手里行货,轻咳嗽便是罪过。”宋江道:“你便寻我过失,也不到得该死。”
那人怒道:“你说不该死,我要结果你也不难,只似打杀一个苍蝇。”宋江冷笑道:
“我因不送得常例钱便该死时,结识梁山泊吴学究的,却该怎地?”那人听了这话,
慌忙丢了手中讯棍,便问道:“你说甚么?”宋江又答道:“自说那结识军师吴学
究的,你问我怎的?”那人慌了手脚,拖住宋江问道:“你正是谁?那里得这话来?”
宋江笑道:“小可便是山东郓城县宋江。”那人听了大惊,连忙作揖说道:“原来
兄长正是及时雨宋公明。”宋江道:“何足挂齿!”那人便道:“兄长,此间不是
说话处,未敢下拜。同往城里叙怀,请兄长便行。”宋江道:“好,节级少待,容
宋江锁了房门便来。”

宋江慌忙到房里取了吴用的书,自带了银两,出来锁上房门,分付牌头看管,
便和那人离了牢城营内,奔入江州城里来,去一个临街酒肆中楼上坐下。那人问道:
“兄长何处见吴学究来?”宋江怀中取出书来,递与那人。那人拆开封皮,从头读
了,藏在袖内,起身望着宋江便拜。宋江慌忙答礼道:“适间言语冲撞,休怪,休
怪!”那人道:“小弟只听得说有个姓宋的发下牢城营里来。往常时,但是发来的
配军,常例送银五两,今番已经十数日,不见送来,今日是个闲暇日头,因此下来
取讨,不想却是仁兄。恰才在营内甚是言语冒渎了哥哥,万望恕罪!”宋江道:“差
拨亦曾常对小可说起大名。宋江有心要拜识尊颜,又不知足下住处,亦无因入城,
特地只等尊兄下来,要与足下相会一面,以此耽误日久。不是为这五两银子不舍得
送来,只想尊兄必是自来,故意延挨。今日幸得相见,以慰平生之愿。”

说话的,那人是谁?便是吴学究所荐的江州两院押牢节级戴院长戴宗。那时故
宋时金陵一路节级,都称呼“家长”,湖南一路节级,都称呼做“院长”。原来这
戴院长有一等惊人的道术,但出路时,赍书飞报紧急军情事,把两个甲马拴在两只
腿上,作起神行法来,一日能行五百里;把四个甲马拴在腿上,便一日能行八百里。
因此人都称做神行太保戴宗。有《临江仙》为证:

面阔唇方神眼突,瘦长清秀人材,皂纱巾畔翠花开。黄旗书令字,红串映宣牌。

健足欲追千里马,罗衫常惹尘埃,神行太保术奇哉!程途八百里,朝去暮还来。

当下戴院长与宋公明说罢了来情去意,戴宗、宋江俱各大喜。两个坐在阁子里,
叫那卖酒的过来,安排酒果、肴馔、菜蔬来,就酒楼上两个饮酒。宋江诉说一路上
遇见许多好汉,众人相会的事务,戴宗也倾心吐胆,把和这吴学究相交来往的事,
告诉了一遍。

两个正说到心腹相爱之处,才饮得两三杯酒,只听楼下喧闹起来,过卖连忙走
入阁子来,对戴宗说道:“这个人只除非是院长说得他下,没奈何,烦院长去解拆
则个。”戴宗问道:“在楼下作闹的是谁?”过卖道:“便是时常同院长走的那个
唤做铁牛李大哥,在底下寻主人家借钱。”戴宗笑道:“又是这厮在下面无礼,我
只道是甚么人。兄长少坐,我去叫了这厮上来。”

戴宗便起身下去,不多时,引着一个黑凛凛大汉上楼来。宋江看见,吃了一惊,
便问道:“院长,这大哥是谁?”戴宗道:“这个是小弟身边牢里一个小牢子,姓
李,名逵,祖贯是沂州沂水县百丈村人氏。本身一个异名,唤做黑旋风李逵。他乡
中都叫他做李铁牛。因为打死了人,逃走出来,虽遇赦宥,流落在此江州,不曾还
乡。为他酒性不好,多人惧他。能使两把板斧,及会拳棍,现今在此牢里勾当。”
有诗为证:
家住沂州翠岭东,杀人放火恣行凶。
不搽煤墨浑身黑,似着朱砂两眼红。
闲向溪边磨巨斧,闷来岩畔斫乔松。
力如牛猛坚如铁,撼地摇天黑旋风。

李逵看着宋江问戴宗道:“哥哥,这黑汉子是谁?”戴宗对宋江笑道:“押司,
你看这厮恁么粗卤,全不识些体面。”李逵便道:“我问大哥:怎地是粗卤?”戴
宗道:“兄弟,你便请问这位官人是谁便好,你倒却说‘这黑汉子是谁’,这不是
粗卤,却是甚么?我且与你说知:这位仁兄,便是闲常你要去投奔他的义士哥哥。”
李逵道:“莫不是山东及时雨黑宋江?”戴宗喝道:“咄!你这厮敢如此犯上,直
言叫唤,全不识些高低,兀自不快下拜等几时?”李逵道:“若真个是宋公明,我
便下拜;若是闲人,我却拜甚鸟!节级哥哥,不要瞒我拜了,你却笑我。”宋江便
道:“我正是山东黑宋江。”李逵拍手叫道:“我那爷!你何不早说些个,也教铁
牛欢喜。”扑翻身躯便拜。宋江连忙答礼,说道:“壮士大哥请坐。”戴宗道:“兄
弟,你便来我身边坐了吃酒。”李逵道:“不耐烦小盏吃,换个大碗来筛。”宋江
便问道:“却才大哥为何在楼下发怒?”李逵道:“我有一锭大银,解了十两小银
使用了,却问这主人家挪借十两银子,去赎那大银出来,便还他,自要些使用。叵
耐这鸟主人不肯借与我,却待要和那厮放对,打得他家粉碎,却被大哥叫了我上来。”
宋江道:“只用十两银子去取,再要利钱么?”李逵道:“利钱已有在这里了,只
要十两本钱去讨。”宋江听罢,便去身边取出一个十两银子,把与李逵,说道:“大
哥,你将去赎来用度。”戴宗要阻当时,宋江已把出来了。李逵接得银子,便道:
“却是好也!两位哥哥只在这里等我一等,赎了银子便来送还,就和宋哥哥去城外
吃碗酒。”宋江道:“且坐一坐,吃几碗了去。”李逵道:“我去了便来。”推开
帘子,下楼去了。

戴宗道:“兄长休借这银与他便好,却才小弟正欲要阻,兄长已把在他手里了。”
宋江道:“却是为何?”戴宗道:“这厮虽是耿直,只是贪酒好赌。他却几时有一
锭大银解了,兄长吃他赚漏了这个银去。他慌忙出门,必是去赌。若还赢得时,便
有的送来还哥哥;若是输了时,那里讨这十两银来还兄长?戴宗面上须不好看。”
宋江笑道:“院长尊兄何必见外,量这些银两,何足挂齿,由他去赌输了罢。我看
这人倒是个忠直汉子。”戴宗道:“这厮本事自有,只是心粗胆大不好。在江州牢
里,但吃醉了时,却不奈何罪人,只要打一般强的牢子。我也被他连累得苦。专一
路见不平,好打强汉,以此江州满城人都怕他。”诗曰:
贿赂公行法枉施,罪人多受不平亏。
以强凌弱真堪恨,天使拳头付李逵。

宋江道:“俺们再饮两杯,却去城外闲玩一遭。”戴宗道:“小弟也正忘了和
兄长去看江景则个。”宋江道:“小可也要看江州的景致,如此最好。”

且不说两个再饮酒,只说李逵得了这个银子,寻思道:“难得宋江哥哥,又不
曾和我深交,便借我十两银子,果然仗义疏财,名不虚传。如今来到这里,却恨我
这几日赌输了,没一文做好汉请他。如今得他这十两银子,且将去赌一赌,倘或赢
得几贯钱来,请他一请也好看。”当时李逵慌忙跑出城外小张乙赌房里来,便去场
上将这十两银子撇在地下,叫道:“把头钱过来我博。”那小张乙得知李逵从来赌
直,便道:“大哥且歇这一博,下来便是你博。”李逵道:“我要先赌这一博。”
小张乙道:“你便傍猜也好。”李逵道:“我不傍猜,只要博这一博,五两银子做
一注。”有那一般赌的,却待要博,被李逵擗手夺过头钱来,便叫道:“我博兀谁?”
小张乙道:“便博我五两银子。”李逵叫一声,地博一个叉。小张乙便拿了银
子过来,李逵叫道:“我的银子是十两。”小张乙道:“你再博我五两,快,便还
了你这锭银子。”李逵又拿起头钱,叫声:“快!”的又博个叉。小张乙笑道:
“我叫你休抢头钱,且歇一博,不听我口,如今一连博上两个叉。”李逵道:“我
这银子是别人的。”小张乙道:“遮莫是谁的,也不济事了。你既输了,却说甚么?”
李逵道:“没奈何,且借我一借,明日便送来还你。”小张乙道:“说甚么闲话?
自古赌钱场上无父子,你明明地输了,如何倒来革争?”李逵把布衫拽起在前面,
口里喝道:“你们还我也不还?”小张乙道:“李大哥,你闲常最赌的直,今日如
何恁么没出豁?”李逵也不答应他,便就地下掳了银子,又抢了别人赌的十来两银
子,都搂在布衫兜里,睁起双眼,就道:“老爷闲常赌直,今日权且不直一遍。”
小张乙急待向前夺时,被李逵一指一交。十二三个赌博的一齐上,要夺那银子,被
李逵指东打西,指南打北。李逵把这伙人打得没地躲处,便出到门前,把门的问道:
“大郎那里去?”被李逵提在一边,一脚踢开了门,便走。那伙人随后赶将出来,
都只在门前叫道:“李大哥,你恁地没道理,都抢了我们众人的银子去!”只在门
前叫喊,没一个敢近前来讨。诗曰:
世人无事不嬲帐,直道只用在赌上。
李逵不直亦不妨,又为赌贼作榜样。

李逵正走之时,听得背后一人赶上来,扳住肩臂喝道:“你这厮如何却抢掳别
人财物?”李逵口里应道:“干你鸟事!”回过脸来看时,却是戴宗,背后立着宋
江。李逵见了,惶恐满面,便道:“哥哥休怪,铁牛闲常只是赌直,今日不想输了
哥哥的银子,又没得些钱来相请哥哥,喉急了,时下做出这些不直来。”宋江听了,
大笑道:“贤弟但要银子使用,只顾来问我讨。今日既是明明地输与他了,快把来
还他。”李逵只得从布衫兜里取出来,都递在宋江手里。宋江便叫过小张乙前来,
都付与他。小张乙接过来说道:“二位官人在上,小人只拿了自己的,这十两原银,
虽是李大哥两博输与小人,如今小人情愿不要他的,省的记了冤仇。”宋江道:“你
只顾将去,不要记怀。”小张乙那里肯。宋江便道:“他不曾打伤了你们么?”小
张乙道:“讨头的,拾钱的,和那把门的,都被他打倒在里面。”宋江道:“既是
恁的,就与他众人做将息钱,兄弟自不敢来了,我自着他去。”小张乙收了银子,
拜谢了回去。

宋江道:“我们和李大哥吃三杯去。”戴宗道:“前面靠江有那琵琶亭酒馆,
是唐朝白乐天古迹。我们去亭上酌三杯,就观江景则个。”宋江道:“可于城中买
些肴馔之物将去。”戴宗道:“不用,如今那亭上有人在里面卖酒。”宋江道:“恁
地时却好。”当时三人便望琵琶亭上来。到得亭子上看时,一边靠着浔阳江,一边
是店主人家房屋。琵琶亭上有十数付座头,戴宗便拣一付干净座头,让宋江坐了头
位,戴宗坐在对席,肩下便是李逵。三个坐定,便叫酒保铺下菜蔬、果品、海鲜、
按酒之类,酒保取过两樽玉壶春酒,此是江州有名的上色好酒,开了泥头。宋江纵
目观看那江时,端的是景致非常。但见:

云外遥山耸翠,江边远水翻银。隐隐沙汀,飞起几行鸥鹭;悠悠小蒲,撑回数
只渔舟。翻翻雪浪拍长空,拂拂凉风吹水面。紫霄峰上接穹苍,琵琶亭半临江岸。
四围空阔,八面玲
珑。栏干影浸玻璃,窗外光浮玉璧。昔日乐天声价重,当年司马泪痕多。

当时三人坐下,李逵便道:“酒把大碗来筛,不耐烦小盏价吃。”戴宗喝道:
“兄弟好村,你不要做声,只顾吃酒便了。”宋江分付酒保道:“我两个面前放两
只盏子,这位大哥面前放个大碗。”酒保应了,下去取只碗来,放在李逵面前,一
面筛酒,一面铺下肴馔。李逵笑道:“真个好个宋哥哥,人说不差了,便知做兄弟
的性格。结拜得这位哥哥,也不枉了。”酒保斟酒,连筛了五七遍。宋江因见了这
两人,心中欢喜,吃了几杯,忽然心里想要鱼辣汤吃,便问戴宗道:“这里有好鲜
鱼么?”戴宗笑道:“兄长,你不见满江都是渔船,此间正是鱼米之乡,如何没有
鲜鱼?”宋江道:“得些辣鱼汤醒酒最好。”戴宗便唤酒保,教造三分加辣点红白
鱼汤来。顷刻造了汤来,宋江看见道:“美食不如美器,虽是个酒肆之中,端的好
整齐器皿。”拿起箸来,相劝戴宗、李逵吃,自也吃了些鱼,呷了几口汤汁。李逵
也不使箸,便把手去碗里捞起鱼来,和骨头都嚼吃了。宋江看见,忍笑不住,呷了
两口汁,便放下箸不吃了。戴宗道:“兄长,已定这鱼腌了,不中仁兄吃。”宋江
道:“便是不才酒后,只爱口鲜鱼汤吃,这个鱼真是不甚好。”戴宗应道:“便是
小弟也吃不得,是腌的,不中吃。”李逵嚼了自碗里鱼,便道:“两位哥哥都不吃,
我替你们吃了。”便伸手去宋江碗里捞将过来吃了,又去戴宗碗里也捞过来吃了,
滴滴点点淋一桌子汁水。

宋江见李逵把三碗鱼汤和骨头都嚼吃了,便叫酒保来分付道:“我这大哥想是
肚饥,你可去大块肉切二斤来与他吃,少刻一发算钱还你。”酒保道:“小人这里
只卖羊肉,却没牛肉,要肥羊尽有。”李逵听了,便把鱼汁劈脸泼将去,淋那酒保
一身。戴宗喝道:“你又做甚么!”李逵应道:“叵耐这厮无礼,欺负我只吃牛肉,
不卖羊肉与我吃。”酒保道:“小人问一声,也不多话。”宋江道:“你去只顾切
来,我自还钱。”酒保忍气吞声去切了二斤羊肉,做一盘,将来放在桌子上。李逵
见了,也不谦让,大把价揸来只顾吃,拈指间把这二斤羊肉都吃了。宋江看了道:
“壮哉,真好汉也!”李逵道:“这宋大哥便知我的鸟意,吃肉不强似吃鱼。”戴
宗叫酒保来问道:“却才鱼汤,家生甚是整齐,鱼却腌了,不中吃。别有甚好鲜鱼
时,另造些辣汤来,与我这位官人醒酒。”酒保答道:“不敢瞒院长说,这鱼端的
是昨夜的。今日的活鱼还在船内,等鱼牙主人不来,未曾敢卖动,因此未有好鲜鱼。”
李逵跳起来道:“我自去讨两尾活鱼来与哥哥吃。”戴宗道:“你休去,只央酒保
去回几尾来便了。”李逵道:“船上打鱼的,不敢不与我,值得甚么!”戴宗拦当
不住,李逵一直去了。戴宗对宋江说道:“兄长休怪小弟引这等人来相会,全没些
个体面,羞辱杀人!”宋江道:“他生性是恁的,如何教他改得?我倒敬他真实不
假。”两个自在琵琶亭上笑语说话取乐。诗曰:
湓江烟景出尘寰,江上峰峦拥髻鬟。
明月琵琶人不见,黄芦苦竹暮潮还。

却说李逵走到江边看时,见那渔船一字排着,约有八九十只,都缆系在绿杨树
下。船上渔人,有斜枕着船梢睡的,有在船头上结网的,也有在水里洗浴的。此时
正是五月半天气,一轮红日,将及沉西,不见主人来开舱卖鱼。李逵走到船边,喝
一声道:“你们船上活鱼把两尾来与我。”那渔人应道:“我们等不见渔牙主人来,
不敢开舱。你看,那行贩都在岸上坐地。”李逵道:“等甚么鸟主人!先把两尾鱼
来与我。”那渔人又答道:“纸也未曾烧,如何敢开舱?那里先拿鱼与你?”李逵
见他众人不肯拿鱼,便跳上一只船去,渔人那里拦当得住。李逵不省得船上的事,
只顾便把竹笆篾一拔,渔人在岸上只叫得:“罢了!”李逵伸手去板底下一绞摸
时,那里有一个鱼在里面。原来那大江里渔船,船尾开半截大孔,放江水出入,养
着活鱼,却把竹笆篾拦住,以此船舱里活水往来,养放活鱼,因此江州有好鲜鱼。
这李逵不省得,倒先把竹笆篾提起了,将那一舱活鱼都走了。李逵又跳过那边船上
去拔那竹篾,那七八十渔人都奔上船,把竹篙来打李逵。李逵大怒,焦躁起来,便
脱下布衫,里面单系着一条棋子布手巾儿,见那乱竹篙打来,两只手一驾,早抢了
五六条在手里,一似扭葱般都扭断了。渔人看见,尽吃一惊,却都去解了缆,把船
撑开去了。李逵忿怒,赤条条地拿两截折竹篙,上岸来赶打行贩,都乱纷纷地挑了
担走。

正热闹里,只见一个人从小路里走出来,众人看见叫道:“主人来了,这黑大
汉在此抢鱼,都赶散了渔船。”那人道:“甚么黑大汉,敢如此无礼!”众人把手
指道:“那厮兀自在岸边寻人厮打。”那人抢将过去,喝道:“你这厮吃了豹子心
大虫胆,也不敢来搅乱老爷的道路!”李逵看那人时,六尺五六身材,三十二三年
纪,三柳掩口黑髯,头上裹顶青纱万字巾,掩映着穿心红一点儿,上穿一领白布
衫,腰系一条绢搭膊,下面青白枭脚,多耳麻鞋,手里提条行秤。那人正来卖鱼,
见了李逵在那里横七竖八打人,便把秤递与行贩接了,赶上前来大喝道:“你这厮
要打谁?”李逵也不回话,抡过竹篙,却望那人便打。那人抢入去,早夺了竹篙,
李逵便一把揪住那人头发,那人便奔他下三面,要跌李逵。怎敌得李逵水牛般气力?
直推将开去,不能够拢身,那人便望肋下擢得几拳,李逵那里着在意里?那人又飞
起脚来踢,被李逵直把头按将下去,提起铁锤般大小拳头,去那人脊梁上擂鼓也似
打。那人怎生挣扎?李逵正打哩,一个人在背后劈腰抱住,一个人便来帮住手,喝
道:“使不得,使不得!”李逵回头看时,却是宋江、戴宗。李逵便放了手,那人
略得脱身,一道烟走了。

戴宗埋冤李逵道:“我教你休来讨鱼,又在这里和人厮打。倘或一拳打死了人,
你不去偿命坐牢?”李逵应道:“你怕我连累你,我自打死了一个,我自去承当。”
宋江便道:“兄弟休要论口,拿了布衫,且去吃酒。”李逵向那柳树根头拾起布衫,
搭在膊上,跟了宋江、戴宗便走。行不得十数步,只听的背后有人叫骂道:“黑
杀才今番来和你见个输赢。”李逵回转头来看时,便是那人,脱得赤条条地,匾扎
起一条水儿,露出一身雪练也似白肉,头上除了巾帻,显出那个穿心一点红俏
儿来,在江边独自一个把竹篙撑着一只渔船赶将来,口里大骂道:“千刀万剐的黑
杀才,老爷怕你的,不算好汉!走的,不是好男子!”李逵听了大怒,吼了一声,
撇了布衫,抢转身来,那人便把船略拢来,凑在岸边,一手把竹篙点定了船,口里
大骂着。李逵也骂道:“好汉便上岸来。”那人把竹篙去李逵腿上便搠,撩拨得李
逵火起,托地跳在船上。说时迟,那时快,那人只要诱得李逵上船,便把竹篙望岸
边一点,双脚一蹬,那只渔船,一似狂风飘败叶,箭也似投江心里去了。

李逵虽然也识得水,却不甚高,当时慌了手脚。那个人也不叫骂,撇了竹篙,
叫声:“你来,今番和你定要见个输赢。”便把李逵膊拿住,口里说道:“且不
和你厮打,先教你吃些水!”两只脚把船只一晃,船底朝天,英雄落水,两个好汉
“扑通”地都翻筋斗撞下江里去。宋江、戴宗急赶至岸边,那只船已翻在江里,两
个只在岸上叫苦。江岸边早拥上三五百人,在柳阴树下看,都道:“这黑大汉今番
却着道儿,便挣扎得性命,也吃了一肚皮水。”宋江、戴宗在岸边看时,只见江面
开处,那人把李逵提将起来,又淹将下去,两个正在江心里面清波碧浪中间,一个
显浑身黑肉,一个露遍体霜肤。两个打做一团,绞做一块,江岸上那三五百人没一
个不喝采。但见:

一个是沂水县成精异物,一个是小孤山作怪妖魔。这个是酥团结就肌肤,那个
如炭屑凑成皮肉。一个是马灵官白蛇托化,一个是赵元帅黑虎投胎。这个似万万锤
打就银人,那个如千千火炼成铁汉。一个是五台山银牙白象,一个是九曲河铁甲老
龙。这个如布漆罗汉显神通,那个似玉碾金刚施勇猛。一个盘旋良久,汗流遍体迸
真珠;一个揪扯多时,水浸浑身倾
墨汁。那个学华光教主,向碧波深处显形骸;这个像黑煞天神,在雪浪堆中呈面目。
正是玉龙搅暗天边日,黑鬼掀开水底天。
当时宋江、戴宗看见李逵被那人在水里揪住,浸得眼白,又提起来,又纳下去,何
止淹了数十遭,正是:
舟行陆地力能为,拳到江心无可施。
真是黑风吹白浪,铁牛儿作水牛儿。

宋江见李逵吃亏,便叫戴宗央人去救。戴宗问众人道:“这白大汉是谁?”有
认得的说道:“这个好汉便是本处卖鱼主人,唤做张顺。”宋江听得,猛省道:“莫
不是绰号浪里白跳的张顺?”众人道:“正是,正是!”宋江对戴宗说道:“我有
他哥哥张横的家书在营里。”戴宗听了,便向岸边高声叫道:“张二哥不要动手,
有你令兄张横家书在此。这黑大汉是俺们兄弟,你且饶了他,上岸来说话。”张顺
在江心里见是戴宗叫他,却也时常认得,便放了李逵,赴到岸边,爬上岸来,看着
戴宗唱个喏道:“院长休怪小人无礼。”戴宗道:“足下可看我面,且去救了我这
兄弟上来,却教你相会一个人。”张顺再跳下水里,赴将开去,李逵正在江里探头
探脑,假挣扎水。张顺早到分际,带住了李逵一只手,自把两条腿踏着水浪,
如行平地,那水浸不过他肚皮,淹着脐下,摆了一只手,直托李逵上岸来,江边看
的人个个喝采。宋江看得呆了。半晌,张顺、李逵都到岸上,李逵喘做一团,口里
只吐白水。戴宗道:“且都请你们到琵琶亭上说话。”张顺讨了布衫穿着,李逵也
穿了布衫,四个人再到琵琶亭上来。

戴宗便对张顺道:“二哥,你认得我么?”张顺道:“小人自识得院长,只是
无缘,不曾拜会。”戴宗指着李逵问张顺道:“足下日常曾认得他么?今日倒冲撞
了你。”张顺道:“小人如何不认的李大哥?只是不曾交手。”李逵道:“你也淹
得我勾了。”张顺道:“你也打得我好了。”戴宗道:“你两个今番却做个至交的
弟兄。常言道:‘不打不成相识。’”李逵道:“你路上休撞着我。”张顺道:“我
只在水里等你便了。”四人都笑起来,大家唱个无礼喏。

戴宗指着宋江对张顺道:“二哥,你曾认得这位兄长么?”张顺看了道:“小
人却不认得,这里亦不曾见。”李逵跳起身来道:“这哥哥便是黑宋江。”张顺道:
“莫非是山东及时雨郓城宋押司?”戴宗道:“正是公明哥哥。”张顺纳头便拜道:
“久闻大名,不想今日得会,多听的江湖上来往的人说兄长清德,扶危济困,仗义
疏财。”宋江答道:“量小可何足道哉!前日来时,揭阳岭下混江龙李俊家里住了
几日,后在浔阳江上,因穆弘相会,得遇令兄张横,修了一封家书,寄来与足下,
放在营内,不曾带得来。今日便和戴院长并李大哥来这里琵琶亭吃三杯,就观江景。
宋江偶然酒后思量些鲜鱼汤醒酒,怎当的他定要来讨鱼,我两个阻他不住。只听得
江岸上发喊热闹,叫酒保看时,说道是黑大汉和人厮打,我两个急急走来劝解,不
想却与壮士相会。今日宋江一朝得遇三位豪杰,岂非天幸!且请同坐,菜酌三杯。”
再唤酒保重整杯盘,再备肴馔。张顺道:“既然哥哥要好鲜鱼吃,兄弟去取几尾来。”
宋江道:“最好。”李逵道:“我和你去讨。”戴宗喝道:“又来了,你还吃的水
不快活。”张顺笑将起来,绾了李逵手说道:“我今番和你去讨鱼,看别人怎地!”
正是:
上殿相争似虎,落水斗亦如龙。
果然不失和气,斯为草泽英雄。

两个下琵琶亭来,到得江边,张顺略哨一声,只见江上渔船都撑拢来到岸边,
张顺问道:“那个船里有金色鲤鱼?”只见这个应道:“我船上来。”那个应道:
“我船里有。”一霎时却凑拢十数尾金色鲤鱼来。张顺选了四尾大的,把柳条穿了,
先教李逵将来亭上整理。张顺自点了行贩,分付小牙子去把秤卖鱼,张顺却自来琵
琶亭上陪侍宋江。宋江谢道:“何须许多,但赐一尾,也十分够了。”张顺答道:
“些小微物,何足挂齿!兄长食不了时,将回行馆做下饭。”两个序齿,李逵年长,
坐了第三位,张顺坐第四位。再叫酒保讨两樽玉壶春上色酒来,并些海鲜、按酒、
果品之类。张顺分付酒保,把一尾鱼做辣汤,用酒蒸,一尾叫酒保切。

四人饮酒中间,各叙胸中之事,正说得入耳,只见一个女娘,年方二八,穿一
身纱衣,来到跟前,深深的道了四个万福,顿开喉音便唱。李逵正待要卖弄胸中许
多豪杰的事务,却被他唱起来一搅,三个且都听唱,打断了他的话头。李逵怒从心
起,跳起身来,把两个指头去那女娘子额上一点,那女子大叫一声,蓦然倒地。众
人近前看时,只见那女娘桃腮似土,檀口无言。那酒店主人一发向前拦住四人,要
去经官告理。正是:怜香惜玉无情绪,煮鹤焚琴惹是非。

毕竟宋江等四人在酒店里怎地脱身,且听下回分解。