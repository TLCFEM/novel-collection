\chapter{谋坟地阴险产逆~蹈春阳妖艳生奸}

话说蔡京在武学中查问那不听他谈兵,仰视屋角的这个官员,姓罗名,祖贯
云南军达州人,现做武学谕。当下蔡京怒气填胸,正欲发作,因天子驾到报来,蔡
京遂放下此事,率领百官,迎接圣驾进学,拜舞山呼。道君皇帝讲武已毕,当有武
学谕罗,不等蔡京开口,上前俯伏,先启奏道:“武学谕小臣罗,冒万死,谨
将淮西强贼王庆造反情形,上达圣聪。王庆作乱淮西,五年于兹,官军不能抵敌。
童贯、蔡攸,奉旨往淮西征讨,全军覆没;惧罪隐匿,欺诳陛下,说军士水土不服,
权且罢兵,以致养成大患。王庆势愈猖獗,前月又将臣乡云安军攻破,掳掠淫杀,
惨毒不忍言说,通共占据八座军州,八十六个州县。蔡京经体赞元,其子蔡攸,如
是复军杀将,辱国丧师,今日圣驾未临时,犹俨然上坐谈兵,大言不惭,病狂丧心!
乞陛下速诛蔡京等误国贼臣,选将发兵,速行征剿,救生民于涂炭,保社稷以无疆,
臣民幸甚!天下幸甚!”
道君皇帝闻奏大怒,深责蔡京等隐匿之罪。当被蔡京等巧言宛奏天子,不即加罪,
起驾还宫。次日,又有亳州太守侯蒙到京听调,上书直言童贯、蔡攸丧师辱国之罪;
并荐举:“宋江等才略过人,屡建奇功,征辽回来,又定河北,今已奏凯班师。目
今王庆猖獗,乞陛下降敕,将宋江等先行褒赏,即着这支军马,征讨淮西,必成大
功。”徽宗皇帝准奏,随即降旨下省院,议封宋江等官爵。
省院官同蔡京等商议,回奏:“王庆打破宛州,昨有禹州、载州、莱县三处申文告
急。那三处是东京所属州县,邻近神京,乞陛下敕陈、宋江等,不必班师回京,
着他统领军马,星夜驰援禹州等处。臣等保举侯蒙为行军参谋。罗素有韬略,着
他同侯蒙到陈军前听用。宋江等正在征剿,未便升受,待淮西奏凯,另行酌议封
赏。”原来蔡京知王庆那里兵强将猛,与童贯、杨、高俅计议,故意将侯蒙、罗
送到陈那里,只等宋江等败绩,侯蒙、罗怕他走上天去!那时却不是一网打
尽。
话不絮繁。却说那四个贼臣的条议,道君皇帝一一准奏,降旨写敕,就着侯蒙、罗
,赍捧诏敕,及领赏赐金银、缎匹、袍服、衣甲、马匹、御酒等物,即日起行,
驰往河北,宣谕宋江等。又敕该部将河北新复各府州县所缺正佐官员,速行推补,
勒限星驰赶任。道君皇帝剖断政事已毕,复被王黼、蔡攸二人,劝帝到艮岳娱乐去
了,不题。
且说侯蒙赍领诏敕及赏赐将士等物,满满的装载三十五车,离了东京,望河北进发。
于路无话,不则一日,过了壶关山、昭德府,来到威胜州,离城尚有二十余里,遇
着宋兵押解贼首到来。却是宋江先接了班师诏敕,恰遇琼英葬母回来。宋江将琼英
母子及叶清贞孝节义的事,擒元凶贼首的功,并乔道清、孙安等降顺天朝,有功员
役,都备细写表,申奏朝廷,就差张清、琼英、叶清,领兵押解贼首先行。当下张
清上前,与侯参谋、罗相见已毕。张清得了这个消息,差人驰往陈安抚、宋先锋
处报闻。陈、宋江率领诸将,出郭迎接,侯蒙等捧赍圣旨入城,摆列龙亭香案。
陈安抚及宋江以下诸将,整整齐齐,朝北跪着,裴宣喝拜。拜罢,侯蒙面南,立于
龙亭之左,将诏书宣读道:
制曰:朕以敬天法祖,缵绍洪基,惟赖杰宏股肱,赞大业。迩来边庭多儆,国祚少
宁,尔先锋使宋江等,跋履山川,逾越险阻,先成平虏之功,次奏静寇之绩,朕实
嘉赖。今特差参谋侯蒙,赍捧诏书,给赐安抚陈,及宋江、卢俊义等金银、袍
缎、名马、衣甲、御酒等物,用彰尔功。兹者又因强贼王庆,作敌淮西,倾覆我城
池,芟夷我人民,虔刘我边陲,荡摇我西京,仍敕陈为安抚,宋江为平西都先锋,
卢俊义为平西副先锋,侯蒙为行军参谋。诏书到日,即统领军马,星驰先救宛州。
尔等将士,协力尽忠,功奏荡平,定行封赏。其三军头目,如钦赏未敷,着陈就
于河北州县内丰盈库藏中挪撮给赏,造册奏闻。尔其钦哉!特谕。

宣和五年四月

日
侯蒙读罢丹诏,陈及宋江等山呼万岁,再拜谢恩已毕。侯蒙取过金银、缎匹等项,
依次照名给散:陈安抚及宋江、卢俊义,各黄金五百两,锦缎十表里,锦袍一套,
名马一匹,御酒二瓶;吴用等三十四员,各赏白金二百两,彩缎四表里,御酒一瓶;
朱武等七十二员,各赐白金一百两,御酒一瓶;余下金银,陈安抚设处凑足,散
军兵已毕。宋江复令张清、琼英、叶清,押解田虎、田豹、田彪,到京师献俘去了。
公孙胜来禀,乞兄长修五龙山龙神庙中五条龙像。宋江依允,差匠修塑。
宋江差戴宗、马灵往谕各路守城将士,一等新官到来,即行交代,勒兵前来,征剿
王庆。宋江又料理了数日,各处新官皆到,诸路守城将佐,统领军兵,陆续到来。
宋江将钦赏银两,散已毕,宋江令萧让、金大坚镌勒碑石,记叙其事。正值五月
五日天中节,宋江教宋清大排筵席,庆贺太平,请陈安抚上坐,新任太守及侯蒙、
罗,并本州佐贰等官次之;宋江以下,除张清晋京外,其一百单七人,及河北降
将乔道清、孙安、卞祥等一十七员,整整齐齐,排坐两边。当下席间,陈、侯蒙、
罗称赞宋江等功勋;宋江、吴用等感激三位知己,或论朝事,或诉衷曲,觥筹交
错,灯烛辉煌,直饮至夜半方散。次日,宋江与吴用计议,整点兵马,辞别州官,
离了威胜,同陈等众,望南进发。所过地方,秋毫无犯。百姓香花灯烛,络绎道
路,拜谢宋江等剪除贼寇,我们百姓,得再见天日之恩。
不说宋江等望南征进,再说没羽箭张清同琼英、叶清,将陷车囚解田虎等,已到东
京,先将宋江书札,呈达宿太尉,并送金珠珍玩。宿太尉转达上皇,天子大嘉琼英
母子贞孝,降敕特赠琼英母宋氏为介休贞节县君,着彼处有司,建造坊祠,表扬贞
节,春秋享祀。封琼英为贞孝宜人,叶清为正排军,钦赏白银五十两,表扬其义。
张清复还旧日原职。仍着三人协助宋江,征讨淮西,功成升赏。道君皇帝敕下法司,
将反贼田虎、田豹、田彪,押赴市曹,凌迟碎剐。当下琼英带得父母小像,禀过监
斩官,将仇申、宋氏小像,悬挂法场中,像前摆张桌子,等到午时三刻,田虎开刀
碎剐后,琼英将田虎首级,摆在桌上,滴血祭奠父母,放声大哭。此时琼英这段事,
东京已传遍了,当日观者如垛,见琼英哭得悲恸,无不感泣。琼英祭奠已毕,同张
清、叶清望阙谢恩。三人离了东京,径望宛州进发,来助宋江,征讨王庆,不在话
下。
看官牢记话头,仔细听着,且把王庆自幼至长的事,表白出来。那王庆原来是东京
开封府内一个副排军。他父亲王砉,是东京大富户,专一打点衙门,唆结讼,放
刁把滥,排陷良善,因此人都让他些个。他听信了一个风水先生,看中了一块阴地,
当出大贵之子。这块地,就是王砉亲戚人家葬过的,王砉与风水先生设计陷害。王
砉出尖,把那家告纸谎状,官司累年,家产荡尽,那家敌王砉不过,离了东京,远
方居住。后来王庆造反,三族皆夷,独此家在远方,官府查出是王砉被害,独得保
全。王砉夺了那块坟地,葬过父母,妻子怀孕弥月。王砉梦虎入室,蹲踞堂西,忽
被狮兽突入,将虎衔去。王砉觉来,老婆便产王庆。那王庆从小浮浪,到十六七岁,
生得身雄力大,不去读书,专好斗鸡走马,使枪抡棒。那王砉夫妻两口儿,单单养
得王庆一个,十分爱恤,自来护短,凭他惯了,到得长大,如何拘管得下。王庆赌
的是钱儿,宿的是娼儿,吃的是酒儿。王砉夫妇,也有时训诲他。王庆逆性发作,
将父母詈骂,王砉无可奈何,只索由他。过了六七年,把个家产费得罄尽,单靠着
一身本事,在本府充做个副排军。一有钱钞在手,三兄四弟,终日大酒大肉价同吃;
若是有些不如意时节,拽出拳头便打。所以众人又惧怕他,又喜欢他。
一日,王庆五更入衙画卯,干办完了执事,闲步出城南,到玉津圃游玩。此时是徽
宗政和六年,仲春天气,游人如蚁,军马如云,正是:
上苑花开堤柳眠,游人队里杂婵娟。
金勒马嘶芳草地,玉楼人醉杏花天。
王庆独自闲耍了一回,向那圃中一颗傍池的垂杨上,将肩胛斜倚着,欲等个相识到
来,同去酒肆中吃三杯进城。无移时,只见池北边十来个干办、虞候、伴当、养娘
人等,簇着一乘轿子,轿子里面,如花似朵的一个年少女子。那女子要看景致,不
用竹帘。那王庆好的是女色,见了这般标致的女子,把个魂灵都吊下来。认得那伙
干办、虞候,是枢密童贯府中人。当下王庆远远地跟着轿子,随了那伙人来到艮岳。
那艮岳在京城东北隅,即道君皇帝所筑,奇峰怪石,古木珍禽,亭榭池馆,不可胜
数。外面朱垣绯户,如禁门一般,有内相禁军看守,等闲人脚指头儿也不敢踅到门
前。
那簇人歇下轿,养娘扶女子出了轿,径望艮岳门内,袅袅娜娜,妖妖娆娆走进去。
那看门禁军内侍,都让开条路,让他走进去了。原来那女子是童贯之弟童贳之女,
杨的外孙。童贯抚养为己女,许配蔡攸之子,却是蔡京的孙儿媳妇了,小名叫做
娇秀,年方二八。他禀过童贯,乘天子两日在李师师家娱乐,欲到艮岳游玩。童贯
预先分付了禁军人役,因此不敢拦阻。那娇秀进去了两个时辰,兀是不见出来。王
庆那厮,呆呆地在外面守着,肚里饥饿,踅到东街酒店里,买些酒肉,忙忙地吃了
六七杯,恐怕那女子去了,连帐也不算,向便袋里摸出一块二钱重的银子,丢与店
小二道:“少停便来算帐。”
王庆再踅到艮岳前,又停了一回,只见那女子同了养娘,轻移莲步,走出艮岳来,
且不上轿,看那艮岳外面的景致。王庆踅上前去看那女子时,真个标致。有《混江
龙》词为证:
丰资毓秀,那里个金屋堪收?点樱桃小口,横秋水双眸。若不是昨夜晴开新月皎,
怎能得今朝肠断小梁州。芳芬绰约蕙兰俦,香飘雅丽芙蓉袖,两下里心猿都被月引
花钩。
王庆看到好处,不觉心头撞鹿,骨软筋麻,好便似雪狮子向火,霎时间酥了半边。
那娇秀在人丛里,睃见王庆的相貌:
凤眼浓眉如画,微须白面红颜。顶平额阔满天仓,七尺身材壮健。善会偷香窃玉,
惯的卖俏行奸。凝眸呆想立人前,俊俏风流无限。
那娇秀一眼睃着王庆风流,也看上了他。当有干办、虞候,喝开众人,养娘扶娇秀
上轿,众人簇拥着,转东过西,却到酸枣门外岳庙里来烧香。王庆又跟随到岳庙里,
人山人海的,挨挤不开,众人见是童枢密处虞候、干办,都让开条路。那娇秀下轿
进香,王庆挨踅上前,却是不能近身,又恐随从人等叱咤,假意与庙祝厮熟,帮他
点烛烧香,一双眼不住的溜那娇秀,娇秀也把眼来频睃。原来蔡攸的儿子,生来是
憨呆的。那娇秀在家,听得几次媒婆传说是真,日夜叫屈怨恨。今日见了王庆风流
俊俏,那小鬼头儿春心也动了。
当下童府中一个董虞候,早已瞧科,认得排军王庆。董虞候把王庆劈脸一掌打去,
喝道:“这个是甚么人家的宅眷!你是开封府一个军健,你好大胆,如何也在这里
挨挨挤挤。待俺对相公说了,教你这颗驴头,安不牢在颈上!”王庆那敢则声,抱
头鼠窜,奔出庙门来,一口唾,叫声道:“碎!我直恁这般呆!癞虾蟆怎想吃天鹅
肉!”当晚忍气吞声,惭愧回家。谁知那娇秀回府,倒是日夜思想,厚贿侍婢,反
去问那董虞候,教他说王庆的详细。侍婢与一个薛婆子相熟,同他做了马泊六,悄
地勾引王庆从后门进来,人不知,鬼不觉,与娇秀勾搭。王庆那厮,喜出望外,终
日饮酒。
光阴荏苒,过了三月,正是乐极生悲。王庆一日吃得烂醉如泥,在本府正排军张斌
面前,露出马脚,遂将此事彰扬开去,不免吹在童贯耳朵里。童贯大怒,思想要寻
罪过摆拨他,不在话下。
且说王庆因此事发觉,不敢再进童府去了。一日在家闲坐,此时已是五月下旬,天
气炎热,王庆掇条板凳,放在天井中乘凉,方起身入屋里去拿扇子,只见那条板凳
四脚搬动,从天井中走将入来。王庆喝声道:“奇怪!”飞起右脚,向板凳只一脚
踢去。王庆叫声道:“阿也苦也!”不踢时,万事皆休,一踢时,立至。正是:
天有不测风云,人有旦夕祸福。
毕竟王庆踢这板凳,为何叫苦起来,且听下回分解。