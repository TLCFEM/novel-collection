\chapter{忠义堂石碣受天文~梁山泊英雄排座次}

话说宋公明,一打东平,两打东昌,回归山寨,计点大小头领,共有一百八员,
心中大喜。遂对众兄弟道:“宋江自从闹了江州上山之后,皆赖托众弟兄英雄扶助,
立我为头。今者共聚得一百八员头领,心中甚喜。自从晁盖哥哥归天之后,但引兵
马下山,公然保全。此是上天护佑,非人之能。纵有被掳之人,陷于缧绁,或是中
伤回来,且都无事。今者一百八人皆在面前聚会,端的古往今来,实为罕有。从前
兵刃到处,杀害生灵,无可禳谢。我心中欲建一罗天大醮,报答天地神明眷佑之恩:
一则祈保众弟兄身心安乐;二则惟愿朝廷早降恩光,赦免逆天大罪,众当竭力捐躯,
尽忠报国,死而后已;三则上荐晁天王早生天界,世世生生,再得相见。就行超度
横亡、恶死,火烧、水溺,一应无辜被害之人,俱得善道。我欲行此一事,未知众
弟兄意下如何?”

众头领都称道:“此是善果好事,哥哥主见不差。”吴用便道:“先请公孙胜
一清主行醮事,然后令人下山,四远邀请得道高士,就带醮器赴寨,仍使人收买一
应香烛纸马、花果祭仪、素馔净食,并合用一应物件。”商议选定四月十五日为始,
七昼夜好事。山寨广施钱财,督并干办。日期已近,向那忠义堂前挂起长,四首
堂上扎缚三层高台。堂内铺设七宝三清圣像,两班设二十八宿十二宫辰,一切主醮
星官真宰。堂外仍设监坛崔、卢、邓、窦神将。摆列已定,设放醮器齐备,请到道
众,连公孙胜共是四十九员。是日晴明的好,天和气朗,月白风清。宋江、卢俊义
为首,吴用与众头领为次拈香。公孙胜作高功,主行斋事,关发一应文书符命,不
在话下。当日醮筵,但见:

香腾瑞霭,花簇锦屏。一千条画烛流光,数百盏银灯散彩。对对高张羽盖,重
重密布幢。风清三界步虚声,月冷九天垂沆瀣。金钟撞处,高功表进奏虚皇;玉
鸣时,都讲登坛朝玉帝。绛绡衣星辰灿烂,芙蓉冠金碧交加。监坛神将狰狞,直
日功曹勇猛。道士齐宣宝忏,上瑶台酌水献花;真人密诵灵章,按法剑踏罡布斗。
青龙隐隐来黄道,白鹤翩翩下紫宸。
当日公孙胜与那四十八员道众,都在忠义堂上做醮,每日三朝,至第七日满散。宋
江要求上天报应,特教公孙胜专拜青词,奏闻天帝,每日三朝。却好至第七日三更
时分,公孙胜在虚皇坛第一层,众道士在第二层,宋江等众头领在第三层,众小头
目并将校都在坛下。众皆恳求上苍,务要拜求报应。

是夜三更时候,只听得天上一声响,如裂帛相似,正是西北乾方天门上。众人
看时,直竖金盘:两头尖,中间阔,又唤做天门开,又唤做天眼开,里面毫光射人
眼目,霞彩缭绕,从中间卷出一块火来,如栲栳之形,直滚下虚皇坛来。那团火绕
坛滚了一遭,竟钻入正南地下去了。此时天眼已合,众道士下坛来,宋江随即叫人
将铁锹锄头掘开泥土,根寻火块。那地下掘不到三尺深浅,只见一个石碣,正面两
侧,各有天书文字。有诗为证:
忠义英雄迥结台,感通上帝亦奇哉!
人间善恶皆招报,天眼何时不大开!

当下宋江且教化纸满散。平明,斋众道士,各赠与金帛之物,以充衬资。方才
取过石碣看时,上面乃是龙章凤篆蝌蚪之书,人皆不识。众道士内有一人姓何,法
讳玄通,对宋江说道:“小道家间祖上留下一册文书,专能辨验天书,那上面自古
都是蝌蚪文字,以此贫道善能辨认,译将出来,便知端的。”宋江听了大喜,连忙
捧过石碣,教何道士看了,良久说道:“此石都是义士大名镌在上面:侧首一边是
‘替天行道’四字,一边是‘忠义双全’四字;顶上皆有星辰南北二斗;下面却是
尊号。若不见责,当以从头一一敷宣。”宋江道:“幸得高士指迷,缘分不浅,若
蒙见教,实感大德。唯恐上天见责之言,请勿藏匿,万望尽情剖露,休遗片言。”
宋江唤过圣手书生萧让,用黄纸誊写。何道士乃言:“前面有天书三十六行,皆是
天罡星;背后也有天书七十二行,皆是地煞星,下面注着众义士的姓名。”观看良
久,教萧让从头至后,尽数抄誊。石碣前面,书梁山泊天罡星三十六员:
天魁星呼保义宋江

天罡星玉麒麟卢俊义
天机星智多星吴用

天闲星入云龙公孙胜
天勇星大刀关胜

天雄星豹子头林冲
天猛星霹雳火秦明

天威星双鞭呼延灼
天英星小李广花荣

天贵星小旋风柴进
天富星扑天雕李应

天满星美髯公朱仝
天孤星花和尚鲁智深

天伤星行者武松
天立星双枪将董平

天捷星没羽箭张清
天暗星青面兽杨志

天佑星金枪手徐宁
天空星急先锋索超

天速星神行太保戴宗
天异星赤发鬼刘唐

天杀星黑旋风李逵
天微星九纹龙史进

天究星没遮拦穆弘
天退星插翅虎雷横

天寿星混江龙李俊
天剑星立地太岁阮小二

天平星船火儿张横
天罪星短命二郎阮小五

天损星浪里白跳张顺
天败星活阎罗阮小七

天牢星病关索杨雄
天慧星拚命三郎石秀

天暴星两头蛇解珍
天哭星双尾蝎解宝

天巧星浪子燕青
石碣背面,书地煞星七十二员:
地魁星神机军师朱武

地煞星镇三山黄信
地勇星病尉迟孙立

地杰星丑郡马宣赞
地雄星井木犴郝思文

地威星百胜将韩滔
地英星天目将彭玘

地奇星圣水将单廷圭
地猛星神火将魏定国

地文星圣手书生萧让
地正星铁面孔目裴宣

地阔星摩云金翅欧鹏
地阖星火眼狻猊邓飞

地强星锦毛虎燕顺
地暗星锦豹子杨林

地轴星轰天雷凌振
地会星神算子蒋敬

地佐星小温侯吕方
地佑星赛仁贵郭盛

地灵星神医安道全
地兽星紫髯伯皇甫端

地微星矮脚虎王英
地慧星一丈青扈三娘

地暴星丧门神鲍旭
地然星混世魔王樊瑞

地猖星毛头星孔明
地狂星独火星孔亮

地飞星八臂那吒项充
地走星飞天上圣李衮

地巧星玉臂匠金大坚
地明星铁笛仙马麟

地进星出洞蛟童威
地退星翻江蜃童猛

地满星玉竿孟康
地遂星通臂猿侯健

地周星跳涧虎陈达
地隐星白花蛇杨春

地异星白面郎君郑天寿
地理星九尾龟陶宗旺

地俊星铁扇子宋清
地乐星铁叫子乐和

地捷星花项虎龚旺
地速星中箭虎丁得孙

地镇星小遮拦穆春
地稽星操刀鬼曹正

地魔星云里金刚宋万
地妖星摸着天杜迁

地幽星病大虫薛永
地伏星金眼彪施恩

地空星小霸王周通
地僻星打虎将李忠

地全星鬼脸儿杜兴
地孤星金钱豹子汤隆

地角星独角龙邹润
地短星出林龙邹渊

地藏星笑面虎朱富
地囚星旱地忽律朱贵

地平星铁臂膊蔡福
地损星一枝花蔡庆

地奴星催命判官李立
地察星青眼虎李云

地恶星没面目焦挺
地丑星石将军石勇

地数星小尉迟孙新
地阴星母大虫顾大嫂

地刑星菜园子张青
地壮星母夜叉孙二娘

地劣星活闪婆王定六
地健星险道神郁保四

地耗星白日鼠白胜
地贼星鼓上蚤时迁

地狗星金毛犬段景住

当时何道士辨验天书,教萧让写录出来。读罢,众人看了,俱惊讶不已。宋江
与众头领道:“鄙猥小吏,原来上应星魁,众多弟兄也原来都是一会之人。上天显
应,合当聚义。今已数足,上苍分定位数,为大小二等。天罡、地煞星辰,都已分
定次序,众头领各守其位,各休争执,不可逆了天言。”众人皆道:“天地之意,
物理数定,谁敢违拗?”宋江遂取黄金五十两,酬谢何道士。其余道众收得经资,
收拾醮器,四散下山去了。有诗为证:
月明风冷醮坛深,鸾鹤空中送好音。
地煞天罡排姓字,激昂忠义一生心。

且不说众道士回家去了,只说宋江与军师吴学究、朱武等计议,堂上要立一面
牌额,大书“忠义堂”三字;断金亭也换个大牌扁。前面册立三关,忠义堂后建筑
雁台一座,顶上正面大厅一所,东西各设两房。正厅供养晁天王灵位。东边房内,
宋江、吴用、吕方、郭盛;西边房内,卢俊义、公孙胜、孔明、孔亮。第二坡左一
代房内,朱武、黄信、孙立、萧让、裴宣;右一代房内,戴宗、燕青、张清、安道
全、皇甫端。忠义堂左边,掌管钱粮仓廒收放,柴进、李应、蒋敬、凌振;右边花
荣、樊瑞、项充、李衮。山前南路第一关,解珍、解宝守把;第二关,鲁智深、武
松守把;第三关,朱仝、雷横守把。东山一关,史进、刘唐守把;西山一关,杨雄、
石秀守把;北山一关,穆弘、李逵守把。六关之外,置立八寨:有四旱寨,四水寨。
正南旱寨,秦明、索超、欧鹏、邓飞;正东旱寨,关胜、徐宁、宣赞、郝思文;正
西旱寨,林冲、董平、单廷圭、魏定国;正北旱寨,呼延灼、杨志、韩滔、彭玘。
东南水寨,李俊、阮小二;西南水寨,张横、张顺;东北水寨,阮小五、童威;西
北水寨,阮小七、童猛。其余各有执事。从新置立旌旗等项。山顶上立一面杏黄旗,
上书“替天行道”四字。忠义堂前绣字红旗二面:一书“山东呼保义”,一书“河
北玉麒麟”。外设飞龙飞虎旗、飞熊飞豹旗、青龙白虎旗、朱雀玄武旗,黄钺白旄,
青皂盖,绯缨黑纛。中军器械外,又有四斗五方旗、三才九曜旗、二十八宿旗、
六十四卦旗、周天九宫八卦旗、一百二十四面镇天旗。尽是侯健制造。金大坚铸造
兵符印信。一切完备,选定吉日良时,杀牛宰马,祭献天地神明,挂上忠义堂、断
金亭牌额,立起“替天行道”杏黄旗。

宋江当日大设筵宴,亲捧兵符印信,颁布号令:“诸多大小兄弟,各各管领,
悉宜遵守,毋得违误,有伤义气。如有故违不遵者,定依军法治之,决不轻恕。”

计开:

梁山泊总兵都头领二员:

呼保义宋江

玉麒麟卢俊义

掌管机密军师二员:

智多星吴用

入云龙公孙胜

同参赞军务头领一员:

神机军师朱武

掌管钱粮头领二员:

小旋风柴进

扑天雕李应

马军五虎将五员:

大刀关胜

豹子头林冲

霹雳火秦明

双鞭呼延灼

双枪将董平

马军八虎骑兼先锋使八员:

小李广花荣

金枪手徐宁

青面兽杨志

急先锋索超

没羽箭张清

美髯公朱仝

九纹龙史进

没遮拦穆弘

马军小彪将兼远探出哨头领一十六员:

镇三山黄信

病慰迟孙立

丑郡马宣赞

井木犴郝思文

百胜将韩滔

天目将彭玘

圣水将单廷圭

神火将魏定国

摩云金翅欧鹏

火眼狻猊邓飞

锦毛虎燕顺

铁笛仙马麟

跳涧虎陈达

白花蛇杨春

锦豹子杨林

小霸王周通

步军头领一十员:

花和尚鲁智深

行者武松

赤发鬼刘唐

插翅虎雷横

黑旋风李逵

浪子燕青

病关索杨雄

拚命三郎石秀

两头蛇解珍

双尾蝎解宝

步军将校一十七员:

混世魔王樊端

丧门神鲍旭

八臂那吒项充

飞天大圣李衮

病大虫薛永

金眼彪施恩

小遮拦穆春

打虎将李忠

白面郎君郑天寿

云里金刚宋万

摸着天杜迁

出林龙邹渊

独角龙邹润

花项虎龚旺

中箭虎丁得孙

没面目焦挺

石将军石勇

四寨水军头领八员:

混江龙李俊

船火儿张横

浪里白跳张顺

立地太岁阮小二

短命二郎阮小五

活阎罗阮小七

出洞蛟童威

翻江蜃童猛

四店打听声息,邀接来宾头领八员:

东山酒店

小尉迟孙新

母大虫顾大嫂

西山酒店

菜园子张青

母夜叉孙二娘

南山酒店

旱地忽律朱贵

鬼脸儿杜兴

北山酒店

催命判官李立

活闪婆王定六

总探声息头领一员:

神行太保戴宗

军中走报机密步军头领四员:

铁叫子乐和

鼓上蚤时迁

金毛犬段景住

白日鼠白胜

守护中军马军骁将二员:

小温侯吕方

赛仁贵郭盛

守护中军步军骁将二员:

毛头星孔明

独火星孔亮

专管行刑刽子二员:

铁臂膊蔡福

一枝花蔡庆

专掌三军内采事马军头领二员:

矮脚虎王英

一丈青扈三娘

掌管监造诸事头领一十六员:

行文走檄调兵遣将一员

圣手书生萧让

定功赏罚军政司一员

铁面孔目裴宣

考算钱粮支出纳入一员

神算子蒋敬

监造大小战船一员


玉竿孟康

专造一应兵符印信一员

玉臂匠金大坚

专造一应旌旗袍袄一员

通臂猿侯健

专攻医兽一应马匹一员

紫髯伯皇甫端

专治诸疾内外科医士一员

神医安道全

监督打造一应军器铁甲一员

金钱豹子汤隆

专造一应大小号炮一员

轰天雷凌振

起造修缉房舍一员


青眼虎李云

屠宰牛马猪羊牲口一员

操刀鬼曹正

排设筵宴一员


铁扇子宋清

监造供应一切酒醋一员

笑面虎朱富

监筑梁山泊一应城垣一员

九尾龟陶宗旺

专一把捧帅字旗一员

险道神郁保四

宣和二年四月初一日,梁山泊大聚会分调人员告示。

当日梁山泊宋公明传令已了,分调众头领已定,各各领了兵符印信,筵宴已毕,
人皆大醉,众头领各归所拨寨分,中间有未定执事者,都于雁台前后驻扎听调。有
篇言语,单道梁山泊的好处,怎见得:

八方共域,异姓一家。天地显罡煞之精,人境合杰灵之美。千里面朝夕相见,
一寸心死生可同。相貌语言,南北东西虽各别;心情肝胆,忠诚信义并无差。其人
则有帝子神孙,富豪将吏,并三教九流,乃至猎户渔人,屠儿刽子,都一般儿哥弟
称呼,不分贵贱;且又有同胞手足,捉对夫妻,与叔侄郎舅,以及跟随主仆,争斗
冤仇,皆一样的酒筵欢乐,无问亲疏。或精灵,或粗卤,或村朴,或风流,何尝相
碍,果然认性同居;或笔舌,或刀枪,或奔驰,或偷骗,各有偏长,真是随才器使。
可恨的是假文墨,没奈何着一个圣手书生,聊存风雅;最恼的是大头巾,幸喜得先
杀却白衣秀士,洗尽酸悭。地方四五百里,英雄一百八人。昔时常说江湖上闻名,
似古楼钟声声传播;今日始知星辰中列姓,如念珠子个个连牵。在晁盖恐托胆称王,
归天及早;惟宋江肯呼群保义,把寨为头。休言啸聚山林,早愿瞻依廊庙。

梁山泊忠义堂上,号令已定,各各遵守。宋江拣了吉日良时,焚一炉香,鸣鼓
聚众,都到堂上。宋江对众道:“今非昔比,我有片言。今日既是天罡地曜相会,
必须对天盟誓,各无异心,死生相托,患难相扶,一同保国安民。”众皆大喜。各
人拈香已罢,一齐跪在堂上,宋江为首誓曰:“宋江鄙猥小吏,无学无能,荷天地
之盖载,感日月之照临,聚弟兄于梁山,结英雄于水泊,共一百八人,上符天数,
下合人心。自今已后,若是各人存心不仁,削绝大义,万望天地行诛,神人共戮,
万世不得人身,亿载永沉末劫。但愿共存忠义于心,同著功勋于国,替天行道,保
境安民。神天鉴察,报应昭彰。”誓毕,众皆同声共愿,但愿生生相会,世世相逢,
永无断阻。当日歃血誓盟,尽醉方散。看官听说,这里方才是梁山泊大聚义处。有
诗为证:
光耀飞离土窟间,天罡地煞降尘寰。
说时豪气侵肌冷,讲处英雄透胆寒。
仗义疏财归水泊,报仇雪恨上梁山。
堂前一卷天文字,付与诸公仔细看。

起头分拨已定,话不重言。原来泊子里好汉,但闲便下山,或带人马,或只是
数个头领各自取路去。途次中若是客商车辆人马,任从经过;若是上任官员,箱里
搜出金银来时,全家不留。所得之物,解送山寨,纳库公用,其余些小,就便分了。
折莫便是百十里,三二百里,若有钱粮广积害民的大户,便引人去公然搬取上山,
谁敢阻当。但打听得有那欺压良善暴富小人,积攒得些家私,不论远近,令人便去
尽数收拾上山。如此之为,大小何止千百余处。为是无人可以当抵,又不怕你叫起
撞天屈来,因此不曾显露,所以无有话说。

再说宋江自盟誓之后,一向不曾下山,不觉炎威已过,又早秋凉,重阳节近。
宋江便叫宋清安排大筵席,会众兄弟同赏菊花,唤做菊花之会。但有下山的兄弟们,
不论远近,都要招回寨来赴筵。至日,肉山酒海,先行给散马步水三军一应小头目
人等,各令自去打团儿吃酒。且说忠义堂上遍插菊花,各依次坐,分头把盏。堂前
两边筛锣击鼓,大吹大擂,语笑喧哗,觥筹交错,众头领开怀痛饮。马麟品箫,乐
和唱曲,燕青弹筝,各取其乐。不觉日暮,宋江大醉,叫取纸笔来,一时乘着酒兴,
作《满江红》一词。写毕,令乐和单唱这首词,道是:

喜遇重阳,更佳酿、今朝新熟。见碧水丹山,黄芦苦竹。
头上尽教添白发,鬓边不可无黄菊。愿樽前、长叙弟兄情。如金玉。

统豺虎,
御边幅。号令明,军威肃。中心愿,平虏保民安国。日月常悬忠烈胆,风尘障却奸
邪目。望天王降诏早招安,心方足。
乐和唱这个词,正唱到“望天王降诏早招安”,只见武松叫道:“今日也要招安,
明日也要招安去,冷了弟兄们的心!”黑旋风便睁圆怪眼,大叫道:“招安,招安,
招甚鸟安!”只一脚,把桌子踢起,攧做粉碎。宋江大喝道:“这黑厮怎敢如此无
礼!左右与我推去,斩讫报来!”众人都跪下告道:“这人酒后发狂,哥哥宽恕。”
宋江答道:“众贤弟请起,且把这厮监下。”众人皆喜。有几个当刑小校,向前来
请李逵。李逵道:“你怕我敢挣扎?哥哥杀我也不怨,剐我也不恨,除了他,天也
不怕。”说了,便随着小校去监房里睡。

宋江听了他说,不觉酒醒,忽然发悲。吴用劝道:“兄长既设此会,人皆欢乐
饮酒,他是个粗卤的人,一时醉后冲撞,何必挂怀,且陪众兄弟尽此一乐。”宋江
道:“我在江州,醉后误吟了反诗,得他气力来,今日又作《满江红》词,险些儿
坏了他性命!早是得众兄弟谏救了。他与我身上情分最重,因此潸然泪下。”便叫
武松:“兄弟,你也是个晓事的人,我主张招安,要改邪归正,为国家臣子,如何
便冷了众人的心?”鲁智深便道:“只今满朝文武,多是奸邪,蒙蔽圣聪,就比俺
的直裰染做皂了,洗杀怎得干净?招安不济事,便拜辞了,明日一个个各去寻趁罢。”
宋江道:“众弟兄听说:今皇上至圣至明,只被奸臣闭塞,暂时昏昧,有日云开见
日,知我等替天行道,不扰良民,赦罪招安,同心报国,青史留名,有何不美!因
此只愿早早招安,别无他意。”众皆称谢不已。当日饮酒,终不畅怀。席散,各回
本寨。

次日清晨,众人来看李逵时,尚兀自未醒,众头领睡里唤起来说道:“你昨日
大醉,骂了哥哥,今日要杀你。”李逵道:“我梦里也不敢骂他!他要杀我时,便
由他杀了罢。”众弟兄引着李逵,去堂上见宋江请罪。宋江喝道:“我手下许多人
马,都似你这般无礼,不乱了法度?且看众兄弟之面,寄下你项上一刀,再犯必不
轻恕。”李逵喏喏连声而退,众人皆散。

一向无事,渐近岁终。那一日久雪初晴,只见山下有人来报,离寨七八里,拿
得莱州解灯上东京去的一行人,在关外听候将令。宋江道:“休要执缚,好生叫上
关来。”没多时,解到堂前:两个公人,八九个灯匠,五辆车子。为头的这一个告
道:“小人是莱州承差公人,这几个都是灯匠。年例:东京着落本州,要灯三架,
今年又添两架,乃是玉棚玲珑九华灯。”宋江随即赏与酒食,叫取出灯来看。那做
灯匠人将那玉棚灯挂起,安上四边结带,上下通计九九八十一盏,从忠义堂上挂起,
直垂到地。宋江道:“我本待都留了你的,惟恐教你吃苦,不当稳便,只留下这碗
九华灯在此,其余的你们自解官去。酬烦之资,白银二十两。”众人再拜,恳谢不
已,下山去了。

宋江教把这碗灯点在晁天王孝堂内。次日,对众头领说道:“我生长在山东,
不曾到京师,闻知今上大张灯火,与民同乐,庆赏元宵,自冬至后,便造起灯,至
今才完。我如今要和几个兄弟私去看灯一遭便回。”吴用谏道:“不可,如今东京
做公的最多,倘有疏失,如之奈何!”宋江道:“我日间只在客店里藏身,夜晚入
城看灯,有何虑焉?”众人苦谏不住,宋江坚执要行。正是:猛虎直临丹凤阙,杀
星夜犯卧牛城。

毕竟宋江怎地去东京看灯,且听下回分解。