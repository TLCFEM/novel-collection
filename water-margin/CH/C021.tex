\chapter{虔婆醉打唐牛儿~宋江怒杀阎婆惜}

话说宋江别了刘唐,乘着月色满街,信步自回下处来。却好的遇着阎婆,赶上
前来叫道:“押司,多日使人相请,好贵人,难见面!便是小贱人有些言语高低,
伤触了押司,也看得老身薄面,自教训他与押司陪话。今晚老身有缘,得见押司,
同走一遭去。”宋江道:“我今日县里事务忙,摆拨不开,改日却来。”阎婆道:
“这个使不得。我女儿在家里专望,押司胡乱温顾他便了。直恁地下得!”宋江道:
“端的忙些个,明日准来。”阎婆道:“我今晚要和你去。”便把宋江衣袖扯住了,
发话道:“是谁挑拨你?我娘儿两个,下半世过活,都靠着押司。外人说的闲是闲
非,都不要听他,押司自做个主张。我女儿但有差错,都在老身身上。押司胡乱去
走一遭。”宋江道:“你不要缠,我的事务分拨不开在这里。”阎婆道:“押司便
误了些公事,知县相公不到得便责罚你。这回错过,后次难逢。押司只得和老身去
走一遭,到家里自有告诉。”宋江是个快性的人,吃那婆子缠不过,便道:“你放
了手,我去便了。”阎婆道:“押司不要跑了去,老人家赶不上。”宋江道:“直
恁地这等?”两个厮跟着来到门前,正是:
酒不醉人人自醉,花不迷人人自迷。
直饶今日能知悔,何不当初莫去为?

宋江立住了脚,阎婆把手一拦,说道:“押司来到这里,终不成不入去了。”
宋江进到里面凳子上坐了,那婆子是乖的,自古道:“老虔婆如何出得他手?”只
怕宋江走去,便帮在身边坐了,叫道:“我儿,你心爱的三郎在这里。”那阎婆惜
倒在床上,对着盏孤灯,正在没可寻思处,只等这小张三来。听得娘叫道:“你的
心爱的三郎在这里。”那婆娘只道是张三郎,慌忙起来,把手掠一掠云髻,口里喃
喃的骂道:“这短命,等得我苦也!老娘先打两个耳刮子着!”飞也似跑下楼来,
就子眼里张时,堂前琉璃灯却明亮,照见是宋江,那婆娘复翻身转又上楼去,依
前倒在床上。

阎婆听得女儿脚步下楼来了,又听得再上楼去了,婆子又叫道:“我儿,你的
三郎在这里,怎地倒走了去?”那婆惜在床上应道:“这屋里多远,他不会来。他
又不瞎,如何自不上来?直等我来迎接他,没了当絮絮聒聒地!”阎婆道:“这贱
人真个望不见押司来,气苦了。恁地说,也好教押司受他两句儿。”婆子笑道:“押
司,我同你上楼去。”宋江听了那婆娘说这几句,心里自有五分不自在,被这婆子
来扯,勉强只得上楼去。

原来是一间六椽楼屋。前半间安一副春台,桌凳;后半间铺着卧房,贴里安一
张三面棱花的床,两边都是栏干,上挂着一顶红罗幔帐;侧首放个衣架,搭着手巾;
这边放着个洗手盆;一张金漆桌子上,放一个锡灯台;边厢两个杌子;正面壁上挂
一幅仕女;对床排着四把一字交椅。

宋江来到楼上,阎婆便拖入房里去。宋江便向杌子上朝着床边坐了。阎婆就床
上拖起女儿来,说道:“押司在这里。我儿,你只是性气不好,把言语来伤触他,
恼得押司不上门,闲时却在家里思量。我如今不容易请得他来,你却不起来陪句话
儿,颠倒使性!”婆惜把手拓开,说那婆子:“你做甚么这般鸟乱!我又不曾做了
歹事!他自不上门,教我怎地陪话!”

宋江听了,也不做声。婆子便推过一把交椅,在宋江肩下,便推他女儿过来,
说道:“你且和三郎坐一坐。不陪话便罢,不要焦躁。你两个多时不见,也说一句
有情的话儿。”那婆娘那里肯过来,便去宋江对面坐了。宋江低了头不做声。婆子
看女儿时,也别转了脸。阎婆道:“没酒没浆,做甚么道场?老身有一瓶儿好酒在
这里,买些果品来,与押司陪话。我儿,你相陪押司坐地,不要怕羞,我便来也。”
宋江自寻思道:“我吃这婆子钉住了,脱身不得。等他下楼去,我随后也走了。”
那婆子瞧见宋江要走的意思,出得房门去,门上却有屈戌,便把房门拽上,将屈戌
搭了。宋江暗忖道:“那虔婆倒先算了我。”

且说阎婆下楼来,先去灶前点起个灯,灶里现成烧着一锅脚汤,再凑上些柴头,
拿了些碎银子,出巷口去买得些时新果品、鲜鱼、嫩鸡、肥之类。归到家中,都
把盘子盛了;取酒倾在盆里,舀半旋子,在锅里烫热了,倾在酒壶里。收拾了数盆
菜蔬,三只酒盏,三双箸,一桶盘托上楼来,放在春台上。开了房门,搬将入来,
摆在桌子上。看宋江时,只低着头;看女儿时,也朝着别处。

阎婆道:“我儿起来把盏酒。”婆惜道:“你们自吃,我不耐烦!”婆子道:
“我儿,爷娘手里从小儿惯了你性儿,别人面上须使不得。”婆惜道:“不把盏便
怎地?终不成飞剑来取了我头!”那婆子倒笑起来,说道:“又是我的不是了。押
司是个风流人物,不和你一般见识。你不把酒便罢,且回过脸来吃盏酒儿。”婆惜
只不回过头来。那婆子自把酒来劝宋江,宋江勉意吃了一盏。婆子笑道:“押司莫
要见责。闲话都打迭起,明日慢慢告诉。外人见押司在这里,多少干热的不怯气,
胡言乱语,放屁辣臊,押司都不要听,且只顾吃酒。”筛了三盏在桌子上,说道:
“我儿不要使小孩儿的性,胡乱吃一盏酒。”婆惜道:“没得只顾缠我!我饱了,
吃不得。”阎婆道:“我儿,你也陪侍你的三郎吃盏酒使得。”

婆惜一头听了,一面肚里寻思:“我只心在张三身上,兀谁耐烦相伴这厮!若
不把他灌得醉了,他必来缠我。”婆惜只得勉意拿起酒来,吃了半盏。婆子笑道:
“我儿只是焦躁,且开怀吃两盏儿睡。押司也满饮几杯。”宋江被他劝不过,连饮
了三五杯。婆子也连连吃了几杯,再下楼去烫酒。

那婆子见女儿不吃酒,心中不悦,才见女儿回心吃酒,欢喜道:“若是今夜兜
得他住,那人恼恨都忘了。且又和他缠几时,却再商量。”婆子一头寻思,一面自
在灶前吃了三大钟酒,觉得有些痒麻上来,却又筛了一碗吃,旋了大半旋,倾在注
子里,爬上楼来,见那宋江低着头不做声,女儿也别转着脸弄裙子。这婆子哈哈地
笑道:“你两个又不是泥塑的,做甚么都不做声?押司,你不合是个男子汉,只得
装些温柔,说些风话儿耍。”宋江正没做道理处,口里只不做声,肚里好生进退不
得。阎婆惜自想道:“你不来睬我,指望老娘一似闲常时,来陪你话,相伴你耍笑?
我如今却不耍!”那婆子吃了许多酒,口里只管夹七带八嘈,正在那里张家长,李
家短,说白道绿。有诗为证:
只要孤老不出门,花言巧语弄精魂。
几多聪慧遭他陷,死后应须拔舌根。

却有郓城县一个卖糟的唐二哥,叫做唐牛儿,如常在街上只是帮闲,常常得
宋江赍助他。但有些公事去告宋江,也落得几贯钱使。宋江要用他时,死命向前。
这一日晚,正赌钱输了,没做道理处,却去县前寻宋江,奔到下处寻不见。街坊都
道:“唐二哥,你寻谁?这般忙?”唐牛儿道:“我喉急了,要寻孤老,一地里不
见他。”众人道:“你的孤老是谁?”唐牛儿道:“便是县里宋押司。”众人道:
“我方才见他和阎婆两个过去,一路走着。”唐牛儿道:“是了。这阎婆惜贼贱虫,
他自和张三两个打得火块也似热,只瞒着宋押司一个,他敢也知些风声,好几时不
去了。今晚必然吃那老咬虫假意儿缠了去。我正没钱使,喉急了,胡乱去那里寻几
贯钱使,就帮两碗酒吃。”一径奔到阎婆门前,见里面灯明,门却不关。入到胡梯
边,听得阎婆在楼上呵呵地笑。唐牛儿捏脚捏手,上到楼上,板壁缝里张时,见宋
江和婆惜两个都低着头;那婆子坐在横头桌子边,口里七十三八十四只顾嘈。

唐牛儿闪将入来,看着阎婆和宋江、婆惜,唱了三个喏,立在边头。宋江寻思
道:“这厮来的最好。”把嘴望下一努。唐牛儿是个乖的人,便瞧科,看着宋江便
说道:“小人何处不寻过,原来却在这里吃酒耍,好吃得安稳!”宋江道:“莫不
是县里有甚么要紧事?”唐牛儿道:“押司,你怎地忘了?便是早间那件公事,知
县相公在厅上发作,着四五替公人来下处寻押司,一地里又没寻处,相公焦躁做一
片。押司便可动身。”宋江道:“恁地要紧,只得去。”便起身要下楼,吃那婆子
拦住道:“押司不要使这科分。这唐牛儿捻泛过来,你这精贼也瞒老娘!正是‘鲁
班手里调大斧’!这早晚知县自回衙去,和夫人吃酒取乐,有甚么事务得发作?你这
般道儿,只好瞒魍魉,老娘手里说不过去。”

唐牛儿便道:“真个是知县相公紧等的勾当,我却不会说谎。”阎婆道:“放
你娘狗屁!老娘一双眼,却是琉璃葫芦儿一般,却才见押司努嘴过来,叫你发科,
你倒不撺掇押司来我屋里,颠倒打抹他去,常言道:‘杀人可恕,情理难容。’”
这婆子跳起身来,便把那唐牛儿劈脖子只一叉,踉踉跄跄,直从房里叉下楼来。唐
牛儿道:“你做甚么便叉我?”婆子喝道:“你不晓得破人买卖衣饭,如杀父母妻
子,你高做声,便打你这贼乞丐!”唐牛儿钻将过来道:“你打!”这婆子乘着酒
兴,叉开五指,去那唐牛儿脸上连打两掌,直出帘子外去,婆子便扯帘子,撇放
门背后,却把两扇门关上,拿栓拴了,口里只顾骂。

那唐牛儿吃了这两掌,立在门前大叫道:“贼老咬虫,不要慌!我不看宋押司
面皮,教你这屋里粉碎!教你双日不着单日着!我不结果了你,不姓唐!”拍着胸大
骂了去。

婆子再到楼上,看着宋江道:“押司没事睬那乞丐做甚么?那厮一地里去搪酒
吃,只是搬是搬非。这等倒街卧巷的横死贼,也来上门上户欺负人!”宋江是个真
实的人,吃这婆子一篇道着了真病,倒抽身不得。婆子道:“押司不要心里见责,
老身只恁地知重得了。我儿和押司只吃这杯。我猜着你两个多时不见,以定要早睡,
收拾了罢休。”婆子又劝宋江吃两杯,收拾杯盘下楼来,自去灶下去。

宋江在楼上,自肚里寻思说:“这婆子女儿,和张三两个有事,我心里半信不
信,眼里不曾见真实。待要去来,只道我村。况且夜深了,我只得权睡一睡,且看
这婆娘怎地,今夜与我情分如何。”只见那婆子又上楼来说道:“夜深了,我叫押
司两口儿早睡。”那婆娘应道:“不干你事,你自去睡。”婆子笑下楼来,口里道:
“押司安置。今夜多欢,明日慢慢地起。”婆子下楼来,收拾了灶上,洗了脚手,
吹灭灯,自去睡了。

却说宋江坐在杌子上,只指望那婆娘似比先时,先来偎倚陪话,胡乱又将就几
时。谁想婆惜心里寻思道:“我只思量张三,吃他搅了,却似眼中钉一般,那厮倒
直指望我一似先前时来下气,老娘如今却不要耍。只见说撑船就岸,几曾有撑岸就
船?你不来睬我,老娘倒落得!”

看官听说,原来这色最是怕人。若是他有心恋你时,身上便有刀剑水火,也拦
他不住,他也不怕;若是他无心恋你时,你便身坐在金银堆里,他也不睬你。常言
道:“佳人有意村夫俏,红粉无心浪子村。”宋公明是个勇烈大丈夫,为女色的手
段却不会。这阎婆惜被那张三小意儿百依百随,轻怜重惜,卖俏迎奸,引乱这婆娘
的心,如何肯恋宋江?

当夜两个在灯下,坐着对面,都不做声,各自肚里踌躇,却似等泥干掇入庙。
看看天色夜深,窗间月上,但见:

银河耿耿,玉漏迢迢。穿窗斜月映寒光,透户凉风吹夜气。谯楼禁鼓,一更未
尽一更催;别院寒砧,千捣将残千捣起。画檐间叮当铁马,敲碎旅客孤怀;银台上
闪烁清灯,偏照闺人长叹。贪淫妓女心如火,仗义英雄气似虹。

当下宋江坐在杌子上睃那婆娘时,复地叹口气。约莫也是二更天气,那婆娘不
脱衣裳,便上床去,自倚了绣枕,扭过身,朝里壁自睡了。宋江看了,寻思道:“可
奈这贱人全不睬我些个,他自睡了。我今日吃这婆子言来语去,央了几杯酒,打熬
不得,夜深只得睡了罢。”把头上巾帻除下,放在桌子上,脱下上盖衣裳,搭在衣
架上,腰里解下鸾带,上有一把解衣刀和招文袋,却挂在床边栏干子上,脱去了丝
鞋净,便上床去那婆娘脚后睡了。

半个更次,听得婆惜在脚后冷笑。宋江心里气闷,如何睡得着!自古道:“欢
娱嫌夜短,寂寞恨更长。”看看三更交半夜,酒却醒了。捱到五更,宋江起来,面
桶里冷水洗了脸,便穿了上盖衣裳,带了巾帻,口里骂道:“你这贼贱人好生无礼!”
婆惜也不曾睡着,听得宋江骂时,扭过身来回道:“你不羞这脸。”宋江忍那口气,
便下楼来。阎婆听得脚步响,便在床上说道:“押司且睡歇,等天明去。没来由起
五更做甚么?”宋江也不应,只顾来开门。婆子又道:“押司出去时,与我拽上门。”
宋江出得门来,就拽上了。忍那口气没出处,一直要奔回下处来。却从县前过,见
一碗灯明,看时,却是卖汤药的王公来到县前赶早市。

那老儿见是宋江来,慌忙道:“押司如何今日出来得早?”宋江道:“便是夜
来酒醉,错听更鼓。”王公道:“押司必然伤酒,且请一盏醒酒二陈汤。”宋江道:
“最好。”就凳上坐了。那老子浓浓的奉一盏二陈汤,递与宋江吃。宋江吃了,蓦
然想起道:“时常吃他的汤药,不曾要我还钱。我旧时曾许他一具棺材,不曾与得
他。想起昨日有那晁盖送来的金子,受了他一条,在招文袋里,何不就与那老儿做
棺材钱,教他欢喜?”宋江便道:“王公,我日前曾许你一具棺木钱,一向不曾把
得与你。今日我有些金子在这里,把与你,你便可将去陈三郎家,买了一具棺材,
放在家里。你百年归寿时,我却再与你些送终之资。”王公道:“恩主时常觑老汉,
又蒙与终身寿具,老子今世不能报答,后世做驴做马,报答押司。”宋江道:“休
如此说。”便揭起背子前襟去取那招文袋时,吃了一惊道:“苦也!昨夜正忘在那
贱人的床头栏干子上,我一时气起来,只顾走了,不曾系得在腰里。这几两金子值
得甚么?须有晁盖寄来的那一封书,包着这金。我本欲在酒楼上刘唐前烧毁了,他
回去说时,只道我不把他来为念。正要将到下处来烧,却被这阎婆缠将我去。昨晚
要就灯下烧时,恐怕露在贱人眼里,因此不曾烧得。今早走得慌,不期忘了。我常
时见这婆娘看些曲本,颇识几字,若是被他拿了,倒是利害!”便起身道:“阿公
休怪。不是我说谎,只道金子在招文袋里,不想出来得忙,忘了在家。我去取来与
你。”王公道:“休要去取。明日慢慢的与老汉不迟。”宋江道:“阿公,你不知
道,我还有一件物事,做一处放着,以此要去取。”宋江慌慌急急,奔回阎婆家里
来,正是:
合是英雄有事来,天教遗失箧中财。
已知着爱皆冤对,岂料酬恩是祸胎!

且说这阎婆惜听得宋江出门去了,爬将起来,口里自言自语道:“那厮搅了老
娘一夜睡不着。那厮含脸,只指望老娘陪气下情。我不信你,老娘自和张三过得好,
谁耐烦睬你!你不上门来倒好!”口里说着,一头铺被,脱下上截袄儿,解了下面
裙子,袒开胸前,脱下截衬衣。床面前灯却明亮,照见床头栏干子上拖下条紫罗鸾
带。婆惜见了,笑道:“黑三那厮乞嚯不尽,忘了鸾带在这里。老娘且捉了,把来
与张三系。”便用手去一提,提起招文袋和刀子来,只觉袋里有些重,便把手抽开,
望桌子上只一抖,正抖出那包金子和书来。这婆娘拿起来看时,灯下照见是黄黄的
一条金子。婆惜笑道:“天教我和张三买物事吃。这几日我见张三瘦了,我也正要
买些东西和他将息。”将金子放下,却把那纸书展开来灯下看时,上面写着晁盖并
许多事务。婆惜道:“好呀!我只道:‘吊桶落在井里’,原来也有‘井落在吊桶
里’。我正要和张三两个做夫妻。单单只多你这厮,今日也撞在我手里!原来你和
梁山泊强贼通同往来,送一百两金子与你。且不要慌,老娘慢慢地消遣你。”就把
这封书依原包了金子,还插在招文袋里,“不怕你教五圣来摄了去”。正在楼上自
言自语,只听得楼下呀地门响。婆子问道:“是谁?”宋江道:“是我。”婆子道:
“我说早哩,押司却不信要去,原来早了又回来。且再和姐姐睡一睡,到天明去。”
宋江也不回话,一径奔上楼来。

那婆娘听得是宋江回来,慌忙把鸾带、刀子、招文袋,一发卷做一块,藏在被
里,紧紧地靠了床里壁,只做假睡着。宋江撞到房里,径去床头栏干上取时,
却不见了。宋江心内自慌,只得忍了昨夜的气,把手去摇那妇人道:“你看我日前
的面,还我招文袋。”那婆惜假睡着,只不应。宋江又摇道:“你不要急燥,我自
明日与你陪话。”婆惜道:“老娘正睡哩,是谁搅我?”宋江道:“你情知是我,
假做甚么?”婆惜扭转身道:“黑三,你说甚么?”宋江道:“你还了我招文袋。”
婆惜道:“你在那里交付与我手里,却来问我讨?”宋江道:“忘了在你脚后小栏
干上。这里又没人来,只是你收得。”婆惜道:“呸!你不见鬼来!”宋江道:“夜
来是我不是了,明日与你陪话。你只还了我罢,休要作耍。”婆惜道:“谁和你作
耍?我不曾收得!”宋江道:“你先时不曾脱衣裳睡,如今盖着被子睡,以定是起
来铺被时拿了。”

只见那婆惜柳眉踢竖,星眼圆睁,说道:“老娘拿是拿了,只是不还你!你使
官府的人,便拿我去做贼断。”宋江道:“我须不曾冤你做贼。”婆惜道:“可知
老娘不是贼哩!”宋江见这话,心里越慌,便说道:“我须不曾歹看承你娘儿两个,
还了我罢!我要去干事。”婆惜道:“闲常也只嗔老娘和张三有事。他有些不如你
处,也不该一刀的罪犯,不强似你和打劫贼通同。”宋江道:“好姐姐,不要叫,
邻舍听得,不是耍处。”婆惜道:“你怕外人听得,你莫做不得!这封书,老娘牢
牢地收着。若要饶你时,只依我三件事便罢!”宋江道:“休说三件事,便是三十
件事也依你。”婆惜道:“只怕依不得。”宋江道:“当行即行。敢问那三件事?”

阎婆惜道:“第一件,你可从今日便将原典我的文书来还我;再写一纸,任从
我改嫁张三,并不敢再来争执的文书。”宋江道:“这个依得。”婆惜道:“第二
件,我头上带的,我身上穿的,家里使用的,虽都是你办的,也委一纸文书,不许
你日后来讨。”宋江道:“这个也依得。”阎婆惜又道:“只怕你第三件依不得。”
宋江道:“我已两件都依你,缘何这件依不得?”婆惜道:“有那梁山泊晁盖送与
你的一百两金子,快把来与我,我便饶你这一场天字第一号官司,还你这招文袋里
的款状。”宋江道:“那两件倒都依得。这一百两金子,果然送来与我,我不肯受
他的,依前教他把了回去。若端的有时,双手便送与你。”婆惜道:“可知哩!常
言道:‘公人见钱,如蝇子见血。’他使人送金子与你,你岂有推了转去的?这话
却似放屁!做公人的,‘那个猫儿不吃腥’?‘阎罗王面前,须没放回的鬼’!你待
瞒谁!便把这一百两金子与我,值得甚么!你怕是贼赃时,快熔过了与我。”宋江道:
“你也须知我是老实的人,不会说谎。你若不信,限我三日,我将家私变卖一百两
金子与你。你还了我招文袋。”婆惜冷笑道:“你这黑三倒乖,把我一似小孩儿般
捉弄。我便先还了你招文袋,这封书,歇三日却问你讨金子,正是‘棺材出了,讨
挽歌郎钱。’我这里一手交钱,一手交货。你快把来两相交割。”宋江道:“果然
不曾有这金子。”婆惜道:“明朝到公厅上,你也说不曾有这金子?”

宋江听了公厅两字,怒气直起,那里按纳得住,睁着眼道:“你还也不还!”
那妇人道:“你恁地狠,我便还你不迭!”宋江道:“你真个不还!”婆惜道:“不
还!再饶你一百个不还!若要还时,在郓城县还你!”

宋江便来扯那婆惜盖的被。妇人身边却有这件物,倒不顾被,两手只紧紧地抱
住胸前。宋江扯开被来,却见这鸾带头正在那妇人胸前拖下来。宋江道:“原来却
在这里!”一不做,二不休,两手便来夺。那婆娘那里肯放,宋江在床边舍命的夺,
婆惜死也不放。宋江恨命只一拽,倒拽出那把压衣刀子在席上,宋江便抢在手里。
那婆娘见宋江抢刀在手,叫:“黑三郎杀人也!”只这一声,提起宋江这个念头来。
那一肚皮气,正没出处。婆惜却叫第二声时,宋江左手早按住那婆娘,右手却早刀
落,去那婆惜颡子上只一勒,鲜血飞出,那妇人兀自吼哩。宋江怕他不死。再复一
刀,那颗头,伶伶仃仃,落在枕头上。但见:

手到处青春丧命,刀落时红粉亡身。七魄悠悠,已赴森罗殿上;三魂渺渺,应
归枉死城中。紧闭星眸,直挺挺尸横席上;半开檀口,湿津津头落枕边。从来美兴
一时休,此日娇容堪恋否。

宋江一时怒起,杀了阎婆惜,取过招文袋,抽出那封书来,便就残灯下烧了,
系上鸾带,走下楼来。那婆子在下面睡,听他两口儿论口,倒也不着在意里。只听
得女儿叫一声“黑三郎杀人也!”正不知怎地,慌忙跳起来,穿了衣裳,奔上楼来,
却好和宋江打个胸厮撞。阎婆问道:“你两口儿做甚么闹?”宋江道:“你女儿忒
无礼,被我杀了!”婆子笑道:“却是甚话?便是押司生的眼凶,又酒性不好,专
要杀人,押司休取笑老身。”宋江道:“你不信时,去房里看,我真个杀了。”婆
子道:“我不信。”推开房门看时,只见血泊里挺着尸首。婆子道:“苦也!却是
怎地好?”宋江道:“我是烈汉!一世也不走,随你要怎地。”婆子道:“这贱人
果是不好,押司不错杀了,只是老身无人养赡。”宋江道:“这个不妨,既是你如
此说时,你却不用忧心。我颇有家计,只教你丰衣足食便了,快活过半世。”阎婆
道:“恁地时却是好也,深谢押司。我女儿死在床上,怎地断送?”宋江道:“这
个容易。我去陈三郎家,买一具棺材与你。仵作行人入殓时,我自分付他来。我再
取十两银子与你结果。”婆子谢道:“押司只好趁天未明时讨具棺材盛了,邻舍街
坊都不要见影。”宋江道:“也好。你取纸笔来,我写个票子与你去取。”阎婆道:
“票子也不济事,须是押司自去取,便肯早早发来。”宋江道:“也说得是。”

两个下楼来,婆子去房里拿了锁钥,出到门前,把门锁了,带了钥匙。宋江与
阎婆两个投县前来。此时天色尚早,未明,县门却才开。那婆子约莫到县前左侧,
把宋江一把结住,发喊叫道:“有杀人贼在这里!”吓得宋江慌做一团,连忙掩住
口道:“不要叫。”那里掩得住。县前有几个做公的走将拢来,看时,认得是宋江,
便劝道:“婆子闭嘴!押司不是这般的人,有事只消得好说。”阎婆道:“他正是
凶首,与我捉住,同到县里。”原来宋江为人最好,上下爱敬,满县人没一个不让
他。因为做公的都不肯下手拿他,又不信这婆子说。有诗为证:
好人有难皆怜惜,奸恶无灾尽诧憎。
可见生平须自检,临时情义始堪凭。

正在那里没个解救,恰好唐牛儿托一盘子洗净的糟姜来县前赶趁,正见这婆子
结扭住宋江在那里叫冤屈。唐牛儿见是阎婆一把扭结住宋江,想起昨夜的一肚子鸟
气来,便把盘子放在卖药的老王凳子上,钻将过来,喝道:“老贼虫,你做甚么结
扭住押司?”婆子道:“唐二,你不要来打夺人去,要你偿命也!”唐牛儿大怒,
那里听他说,把婆子手一拆,拆开了,不问事由,叉开五指,去阎婆脸上只一掌,
打个满天星。那婆子昏撒了,只得放手。宋江得脱,往闹里一直走了。

婆子便一把去结扭住唐牛儿叫道:“宋押司杀了我的女儿,你却打夺去了。”
唐牛儿慌道:“我那里得知!”阎婆叫道:“上下替我捉一捉杀人贼则个!不时,
须要带累你们。”众做公的,只碍宋江面皮,不肯动手,拿唐牛儿时,须不耽搁。
众人向前,一个带住婆子,三四个拿住唐牛儿,把他横拖倒拽,直推进郓城县里来。
正是:祸福无门,惟人自召;披麻救火,惹焰烧身。

毕竟唐牛儿被阎婆结住,怎地脱身,且听下回分解。