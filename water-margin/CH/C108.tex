\chapter{乔道清兴雾取城~小旋风藏炮击贼}

话说杨志、孙安、卞祥正追赶奚胜,到伊阙山侧,不提防山坡后有贼将埋伏,
领一万骑兵突出,与杨志等大杀一阵。奚胜得脱,领败残兵进城去了。孙安奋勇厮
并,杀死贼将二人,却是众寡不敌,这千余甲马骑兵,都被贼兵驱入深谷中去。那
谷四面都是峭壁,却无出路,被贼兵搬运木石,塞断谷口。贼人进城,报知龚端。
龚端差二千兵把住谷口,杨志、孙安等,便是插翅也飞不出来。

不说杨志等被困,且说卢俊义等得破奚胜六花阵,大半亏马灵用金砖术,打翻若干
贼兵,更兼众将勇猛,得获全胜,杀了贼中猛将三员,乘势驱兵,夺了龙门关,斩
级万余,夺获马匹、盔甲、金鼓无算,贼兵退入城中去了。卢俊义计点军马,只不
见了冲头阵的杨志、孙安、卞祥一千军马。当下卢俊义教解珍、解宝、邹渊、邹润,
各领一千人马,分四路去寻,至日暮,却无影响。

次日,卢俊义按兵不动,再令解珍等去寻访。解宝领一支军,攀藤附葛,爬山越岭,
到伊阙山东最高的一个山岭上。望见山岭之西,下面深谷中,隐隐的有一簇人马,
被树林丛密遮蔽了,不能够看得详细。又且高下悬隔,声唤不闻。解宝领军卒下山,
寻个居民访问,那里有一个人家,都因兵乱迁避去了。次后到一个最深僻的山凹平
旷处,方才有几家穷苦的村农,见了若干军马,都慌做一团。解宝道:“我们是朝
廷兵马,来此剿捕贼寇的。”那些人听说是官兵,更是慌张。解宝用好言抚慰说道:
“我们军将是宋先锋部下。”那些人道:“可是那杀鞑子,擒田虎,不骚扰地方的
宋先锋么?”解宝道:“正是。”那些村农跪拜道:“可知道将军等不来抓鸡缚狗!
前年也有官兵到此剿捕贼人,那些军士与强盗一般掳掠。因此,我等避到这个所在
来。今日得将军到此,使我们再见天日。”解宝把那杨志等一千人马,不知下落,
并那岭西深谷去处,问访众人。那些人都道:“这个谷叫谷,只有一条进去的
路。”农人遂引解宝等来到谷口,恰好邹渊、邹润两支军马,也寻到来。合兵一处,
杀散贼兵,一同上前,搬开木石,解宝、邹渊领兵马进谷。此时已是深秋天气,果
然好个深岩幽谷。但见:
玉露雕伤枫树林,深岩邃谷气萧森。
岭巅云雾连天涌,壁峭松筠接地阴。
杨志、孙安、卞祥与一千军士,马罢人困,都在树林下,坐以待毙。见了解宝等人
马,众人都喜跃欢呼。解宝将带来的干粮,分散杨志等众人,先且充饥。食罢,众
军一齐出谷。解宝叫村农随到大寨,来见卢先锋。卢俊义大喜,取银两米谷,赈济
穷民。村农磕头感激,千恩万谢去了。随后解珍这支军马,也回寨了。是日天晚歇
息,一宿无话。

次早,卢俊义正与朱武调遣兵马,攻取城池,忽有流星探马报将来说:“王庆差伪
都督杜壆领十二员将佐,兵马二万,前来救援,兵马已到三十里外了。”卢俊义闻
报,教朱武、杨志、孙立、单廷圭、魏定国,同乔道清、马灵,管领兵马二万,列
阵于大寨前,以当城中贼兵突出;教解珍、解宝、穆春、薛永,管领军马五千,看
守山寨。卢俊义亲自统领其余将佐,军马三万五千,迎敌杜壆。当有浪子燕青禀道:
“主人今日不宜亲自临阵。”卢俊义道:“却是为何?”燕青道:“小人昨夜,有
不祥的梦兆。”卢俊义道:“梦寐之事,何足凭信。既以身许国,也顾不得利害。”
燕青道:“若是主人决意要行,乞拨五百步兵,与小人自去行事。”卢俊义笑道:
“小乙,你待要怎么?”燕青道:“主人勿管,只拨与小人便了。”卢俊义道:“便
拨与你,看你做出甚事来!”随即拨五百步兵与燕青。燕青领了自去,卢俊义冷笑
不止。统领众将兵马,离了大寨,由平泉桥经过。那平泉中多奇异的石子,乃唐朝
李德裕旧庄,只见燕青引着众人,在那里砍伐树木。卢俊义心下虽是好笑,忙忙地
要去厮杀,无暇去问他。兵马过了龙门关西十里外,向西列阵等候。至一个时辰,
贼兵方到。

两阵相对,擂鼓呐喊。西阵里偏将卫鹤,舞大杆刀,拍马当先。宋阵中山士奇跃马
挺枪,更不打话,接住厮杀。两骑马在阵前斗过三十合,山士奇挺枪刺中卫鹤的战
马后腿,那马后蹄将下去,把卫鹤闪下马来,山士奇又一枪戳死。西阵中酆泰大
怒,舞两条铁简,拍马直抢山士奇。二将斗到十合之上,卞祥见山士奇斗不过酆泰,
拈枪拍马助战。被酆泰大喝一声,只一简,把山士奇打下马来,再加一简,结果了
性命,拍马舞剑来迎。怎奈卞祥更是勇猛。酆泰马头才到,大喝一声,一枪刺中酆
泰心窝,死于马下。两军大喊。西阵主帅杜壆,见连折了二将,心如火炽,气若烟
生,挺一条丈八蛇矛,骤马亲自出阵。宋阵主帅卢俊义也亲自出阵,与杜壆斗过五
十合,不分胜败。杜壆那条蛇矛,神出鬼没。孙安见卢先锋不能取胜,挥剑拍马助
战。贼将卓茂,舞条狼牙棍,纵马来迎。与孙安斗不上四五合,孙安奋神威,将卓
茂一剑,斩于马下。拨转马,骤上前,挥剑来砍杜壆。杜壆见他杀了卓茂,措手不
及,被孙安手起剑落,砍断右臂,翻身落马,卢俊义再一枪,结果了性命。卢俊义
等驱兵卷杀过去,贼兵大败。

忽地西南上铲斜小路里,冲出一队骑兵,当先马上一将,状貌粗黑丑恶,一头蓬松
短发,顶个铁道冠,穿领绛征袍,坐匹赤炭马,仗剑指挥众军,弯环踢跳,飞奔前
来。卢俊义等看是贼兵号衣,驱兵一拥上前冲杀。那将不来与你厮杀,口中喃喃呐
呐地念了两句,望正南离位上砍了一剑,转眼间,贼将口中喷出火来。须臾,平空
地上,腾腾火炽,烈烈烟生,望宋军烧将来。卢俊义走避不迭,宋军大败,弃下金
鼓、马匹,乱窜奔逃。走不迭的,都烧得焦头烂额。军士死者,五千余人。众将保
护着卢俊义,奔走到平泉桥。军士争先上桥,登时把桥挤踏得倾圮下来。幸得燕青
砍伐树木,于桥两旁,刚搭得完浮桥,军士得渡,全活者二万人。卢俊义与卞祥两
骑马落后,行至桥边,被贼将赶上,一口火望卞祥喷来。卞祥满身是火,烧损坠马,
被贼兵所杀。卢俊义幸得浮桥接济,驰窜去了。

贼将领兵追杀到来,却得前军报知乔道清。乔道清单骑仗剑,迎着贼将。那贼将见
乔道清迎上来,再把剑望南砍去,那火比前番更是炽焰。乔道清捏诀念咒,把剑望
坎方一指,使出三昧神水的法。霎时间,有千百道黑气,飞迎前来,却变成瀑布飞
泉,又如亿兆斛的琼珠玉屑,望贼将泼去,灭了妖火。那贼将见破了妖术,拨马逃
奔,战马踏着一块水石,马蹄后失,把那贼将闪下马来。乔道清飞马赶上,挥剑砍
为两段。那五千骑兵,掀翻跌伤者,五百余人。乔道清仗剑大喝道:“如肯归降,
都留下驴头!”贼人见乔道清如此法力,都下马投戈,拜伏乞命。乔道清再用好言
抚慰,枭了贼将首级,率领降贼,来见卢先锋献捷。卢俊义感谢不已,并称赞燕青
功劳,众将问降贼,方晓得那妖人姓寇名灭,惯用妖火烧人。人因他貌相丑恶,叫
他做毒焰鬼王。昔年助王庆造反的,不知往那里去了二年,近日又到南丰说:“宋
兵势大,待俺去剿他。”因此,王庆差他星驰到此。龚端、奚胜望见救兵输了,不
敢出来厮杀,只添兵坚守城池。当下乔道清说:“这里城池深固,急切不能得破。
今夜待贫道略施小术,助先锋成功,以报二位先锋厚恩。”卢俊义道:“愿闻神术。”
乔道清附耳低言说道:“如此,如此。”卢俊义大喜,随即调遣将士,各去行事,
准备攻城。一面教军士以礼殡葬山士奇、卞祥,卢俊义亲自设祭。

是夜二更时分,乔道清出来仗剑作法。须臾雾起,把西京一座城池,周回都遮漫了,
守城军士,咫尺不辨,你我不能相顾。宋兵乘黑暗里,从飞桥转关辘辒上,攀缘上
女墙,只听得一声炮响,重雾忽然光敛,城上四面,都是宋兵,各向身边取出火种,
燃点火炬,上下照耀,如同白昼一般。守城军士,先是惊得麻木了,都动弹不得,
被宋兵掣出兵器砍杀,贼兵坠城死者无算。龚端、奚胜见变起仓卒,急引兵来救应,
已被宋军夺了四门。卢俊义大驱兵马进城,龚端、奚胜都被乱兵杀死,其余偏牙将
佐头目俱降,军士降服者三万人,百姓秋毫无犯。

天明,卢俊义出榜安民,标录乔道清大功,重赏三军将士,差马灵到宋先锋处报捷。
马灵遵令去了,至晚便来回话说:“宋先锋等攻打荆南,连日与贼人交战,大败南
丰救兵,主帅谢宁被擒。宋先锋因戎事焦劳,染病在营中,数日军务,都是吴军师
统握。”卢俊义闻报,郁郁不乐,连忙料理军务,将西京城池。交与乔道清、马灵
统兵镇守。卢俊义次日,辞别乔道清、马灵,统领朱武等二十员将佐,离了西京,
急急忙忙望荆南进发。不则一日,兵马已到荆南城北大寨中,卢俊义等入寨问候。
宋江亏神医安道全疗治,病势已减了六七分,卢俊义等甚是喜慰。

正在叙阔,各述军务,忽有逃回军士报说:“唐斌正护送萧让等,离大寨行至三十
里,忽被荆南贼将縻貹、马勥,领一万精兵,从斜僻小路抄出,乘先锋卧病,要来
劫大寨之后,正遇着我们人马。唐斌力敌二将,怎奈众寡不敌,更兼縻貹十分勇猛。
唐斌被縻貹杀死,萧让、裴宣、金大坚都被活捉去。他们正要来劫寨,探听得卢先
锋等大兵到来,贼人只掳了萧让等遁去。”宋江听罢,不觉失声哭道:“萧让等性
命休矣!”病势仍旧沉重。卢俊义等众将,都来劝解。

卢俊义问道:“萧让等到何处去?”宋江呜咽答道:“萧让知我有病,特辞了陈安
抚来看视我,并奉陈安抚命,即取金大坚、裴宣到宛州,要他们写勒碑石,及查勘
文卷。我今日特差唐斌,领一千人马护送他三个去。不料被贼人捉掳,三人必被杀
害!”宋江遂教卢俊义帮助吴用,攻打城池,拿住縻貹、马勥报仇,卢俊义等遵令,
来到城北军前。众人与吴学究叙礼毕,卢俊义连忙说萧让等被掳之事。吴用大惊道:
“苦也!断送了这三个人!”传令教众将围城,并力攻打城池。众将遵令,四面攻
城。吴用又令军汉上云梯,望城中高叫道:“速将萧让、金大坚、裴宣送出来!若
稍迟延,打破城池,不论军民,尽行屠戮!”

却说城中守将梁永伪授留守之职,同正偏将佐,在城镇守。那縻貹、马勥都战败,
逃遁到此。当日捉了萧让等三人,因宋兵尚未围城,縻貹叫开城门进城,将萧让等
解到帅府献功。梁永颇闻得圣手书生的名目,教军士解放绑缚,要他降服。萧让、
裴宣、金大坚三人睁眼大骂道:“无知逆贼,汝等看我们是何等样人?逆贼快把我
三人一刀两段罢了!这六个膝盖骨,休想有半个儿着地!即日宋先锋打破城池,拿你
们这伙鼠辈,碎尸万段!”梁永大怒,叫军汉:“打那三个奴狗跪着!”军汉拿起
杆棒便打,只打得跌仆,那里有一个肯跪。三人骂不绝口。梁永道:“你们要一刀
两段,俺偏要慢慢地摆布你。”喝叫军士:“将这三个奴狗,立枷在辕门外,只顾
打他两腿,打折了驴腿,自然跪将下来。”军汉得令,便来套枷絣扒摆布。
帅府前军士居民,都来看宋军中人物,内中早恼怒了一个真正有男子气的须眉丈夫。
那男子姓萧,双名叫嘉穗,寓居帅府南街纸张铺间壁。他高祖萧憺,字僧达,南北
朝时人,为荆南刺史。江水败堤,萧憺亲率将吏,冒雨修筑。雨甚水壮,将吏请少
避之,萧憺道:“王尊欲以身塞河,我独何心哉?”言毕,而水退堤立。是岁,嘉
禾生,一茎六穗,萧嘉穗取名在此。那萧嘉穗偶游荆南,荆南人思慕其上祖仁德,
把萧嘉穗十分敬重。那萧嘉穗襟怀豪爽,志气高远,度量宽宏,膂力过人,武艺精
熟,乃是十分有胆气的人。凡遇有肝胆者,不论贵贱,都交给他。适遇王庆作乱,
侵夺城池,萧嘉穗献计御贼。当事的不肯用他计策,以致城陷。贼人下令,凡百姓
只许入城,并不许一个出去。萧嘉穗在城中,日夜留心图贼,却是单丝不成线。今
日见贼人将萧让等三个絣扒,又听得宋兵为萧让等攻城紧急,军民都有惊恐之状。
萧嘉穗想了一回道:“机会在此。只此一著,可以保全城中几许生灵。”忙归寓所。
此时已是申牌时分,连忙叫小厮磨了一碗墨汁,向间壁纸铺里买了数张皮料厚棉纸,
在灯下濡墨挥毫,大书特书的写道:

城中都是宋朝良民,必不肯甘心助贼。宋先锋是朝廷良将,杀鞑子,擒田虎,
到处莫敢撄其锋。手下将佐一百单八人,情同股肱。辕门前絣扒的三人,义不屈膝,
宋先锋等英雄忠义可知。今日贼人若害了这三人,城中兵微将寡,早晚打破城池,
玉石俱焚。城中军民,要保全性命的,都跟我去杀贼!
萧嘉穗将那数张纸都写完了,悄地探听消息,只听得百姓们都在家里哭泣。萧嘉穗
道:“民心如此,我计成矣!”挨到昧爽时分,踅出寓所,将写下的数张字纸,抛
向帅府前左右街市闹处。

少顷,天明,军士居民,这边方拾一张来看,那边又有人拾了一张,登时聚着数簇
军民观看。早有巡风军卒,抢一张去,飞报与梁永知道。梁永大惊,急差宣令官出
府传令,教军士谨守辕门及各营,着一面严行缉捕奸细。那萧嘉穗身边藏一把宝刀,
挨入人丛中,也来观看,将纸上言语,高声朗诵了两遍,军民都错愕相顾。那宣令
官奉着主将的令,骑着马,五六个军汉,跟随到各营传令。萧嘉穗抢上前,大吼一
声,一刀砍断马足,宣令官撞下马去,一刀剁下头来。萧嘉穗左手抓了人头,右手
提刀,大呼道:“要保全性命的,都跟萧嘉穗去杀贼!”帅府前军士,平素认得萧
嘉穗,又晓得他是铁汉,霎时有五六百人,拥着他结做一块。

萧嘉穗见军士聚拢来,复连声大呼道:“百姓有胆量的,都来相助!”声音响振数
百步。那时四面响应,百姓都抢棍棒,拔杉剌,折桌脚,拈指间已有五六千人。迭
声呐喊,萧嘉穗当先,领众抢入帅府。那梁永平日暴虐军民,鞭挞士卒,护卫军将,
都恨入骨髓。一闻变起,都来相助,赶入去,把梁永等一家老小都杀了。萧嘉穗领
众军民人等,拥出帅府,此时已有二万余人。把萧让、裴宣、金大坚放了絣扒,都
打开了枷。萧嘉穗选三个有膂力的人,背着萧让等三人。萧嘉穗当先,抓了梁永首
级,赶到北门,杀死守门将马勥,赶散把门军士,开城门,放吊桥。

那时吴用正到北门,亲督将士攻城,听的城中呐喊,又是开城门,只道贼人出来冲
击,忙教军马退下三四箭之地,列阵迎敌。只见萧嘉穗抓着人头,背后三个军汉,
背负萧让等,过了吊桥,忙奔前来。吴用正在惊讶,萧让等高叫道:“吴军师,实
亏这个壮士,激聚众民,杀了贼将,救我等出来。”吴用听了,又惊又喜。萧嘉穗
对吴用道:“事在仓卒,不及叙礼。请军师快领兵入城!”那吊桥边已有若干军民,
都齐声叫道:“请宋先锋入城!”吴用见诸色人等,都有在里面,遂传令教将士统
军马入城,如有妄杀一人者,同伍皆斩。北城上守城军士,看见事势如此,都投戈
下城。其东西南三面守城军士,闻了这个消息,都捆缚了守城贼将,大开城门,香
花灯烛,迎接宋兵入城。只有縻貹那厮勇猛,人近他不得,出西门,杀出重围走了。
吴用差人飞报宋江。宋江闻报,把那忧国家、哭兄弟的病证退了九分九厘,欣喜雀
跃,同众将拔寨都起。大军来到荆南城中,宋江升坐帅府,安抚军民,慰劳将士。
宋江请萧嘉穗到帅府,问了姓名,扶他上坐。宋江纳头便拜道:“壮士豪举,诛锄
叛逆,保全生灵,兵不血刃,克复城池,又救了宋某的三个兄弟,宋江合当下拜。”
萧嘉穗答拜不迭道:“此非萧某之能,皆众军民之力也!”宋江听了这句,愈加钦
敬。宋江以下将佐,都叙礼毕。城中军士,将贼将解来。宋江问愿降者,尽行免罪。
因此满城欢声雷动,降服数万人。恰好水军头领李俊等,统领水军船只。到了汉江,
都来参见。

宋江教置酒款待萧壮士。宋江亲自执杯劝酒,说道:“足下鸿才茂德,宋某回朝,
面奏天子,一定优擢。”萧嘉穗道:“这个倒不必,萧某今日之举,非为功名富贵。
萧某少负不羁之行,长无乡曲之誉,是孤陋寡闻的一个人。方今谗人高张,贤士无
名,虽材怀随和,行若由夷的,终不能达九重。萧某见若干有抱负的英雄,不计生
死,赴公家之难者,倘举事一有不当,那些全躯保妻子的,随而媒孽其短,身家性
命,都在权奸掌握之中。象萧某今日,无官守之责,却似那闲云野鹤,何天之不可
飞耶!”这一席话,说得宋江以下,无不嗟叹。座中公孙胜、鲁智深、武松、燕青、
李俊、童威、童猛、戴宗、柴进、樊瑞、朱武、蒋敬等这十余个人,把萧壮士这段
话,更是点头玩味。当晚酒散,萧嘉穗辞谢出府。次早,宋江差戴宗到陈安抚处报
捷。宋江亲自到萧壮士寓所,特地拜望,却是一个空寓。间壁纸铺里说:“萧嘉穗
今早天未明时,收拾了琴剑书囊,辞别了小人,不知往那里去了。”后人有诗赞萧
憺祖孙之德云:
冒雨修堤萧僧达,波狂涛怒心不怛。
恪诚止水堤功成,六穗嘉禾一茎发。
贤孙豪俊侔厥翁,咄叱民从贼首𢫬。
泽及生灵哲保身,闲云野鹤真超脱。
宋江回到帅府,对众头领说萧嘉穗飘然而去,众将无不叹息。至晚,戴宗回报,说
宛州、山南两处所属未克州县,陈安抚、侯参谋授方略与罗戬及林冲、花荣等,俱
各讨平。朝廷已差若干新官到来,各行交代讫。陈安抚已率领诸将起程,即日便到。
宋江与吴用计议:“待陈安抚到这里镇守,我们好起大兵,前去剿灭渠魁。”宋江
却在荆南调摄五六日,病已全愈。一日,报陈安抚等兵马到来,宋江等接入城中。
参见毕,陈安抚大赏三军将士。次后山南守将史进等,已将州务交代新官,随后也
到。宋江将州务请陈安抚治理。宋江等拜别陈安抚,统领大军,水陆并进,战骑同
行,来剿南丰贼人巢穴。此时一百单八个英雄,都在一处,又有河北降将孙安等十
一人,军马二十余万,连战连捷,兵威大振,所到地方,贼人望风降顺。宋江将复
过州县,呈报陈安抚。陈瓘差罗统领将士兵马,前来镇守。

宋江等水陆大兵,长驱直至南丰地界。哨马报到,说侦探得贼人王庆将李助为统军
大元帅,就本处调选水陆兵马五万。又调云安、东川、安德三路各兵马二万,都是
本处伪兵马都监刘以敬、上官义等统领。数十员猛将,及十一万雄兵,前来拒敌。
王庆亲自督征。宋江闻报,与吴用计议道:“贼兵倾巢而来,必是抵死厮并。我将
何策胜之?”吴用道:“兵法只是‘多方以误之’这一句。俺们如今将士都在一处,
多分调几路前去厮杀,教他应接不暇。”宋江依议传令,分调兵将。

先一日,有扑天雕李应、小旋风柴进,奉宋先锋将令,统领马步头领单廷圭、魏定
国、施恩、薛永、穆春、李忠,领兵五千,护送粮草车仗,并缎帛、火炮、车辆。
在大兵之后,地名龙门山,南麓下傍山有一村庄,四围都是高泥冈子,却象个土城,
三面有路出入。居民空下草瓦房数百间,居民因避兵迁避去了。是晚,东北风大作,
浓云泼墨,李应、柴进见天色已暮,恐天雨沾湿了粮草,教军士拆开门扇,把车辆
推送屋里。军士方欲造饭食息,忽见病大虫薛永领兵巡哨,捉了一个奸细,来报柴
进说:“审问得奸细说,贼人縻貹,领精兵一万,今夜二更,要来劫烧粮草,现今
伏在龙门山中。”

原来,那龙门山两崖对峙如门,其中可通舟楫,树木丛密。李应听说,便对柴进道:
“待小弟去庄前,等那鸟败贼,杀他片甲不回。”柴进道:“那縻貹十分勇猛,不
可力敌。况且我这里兵少,待小弟略施小计,拚五六车火炮,百十车柴薪,与唐斌
等报仇,把那奸细杀了。教军士将粮草、火炮、车辆,教李应领兵三千,都备弓弩
火箭,护卫粮车。在黄昏时候,尽数出了土冈,望南先行,却留下百十辆柴薪车,
四散列于西南下风头草房茅檐边。将百十辆空车,五六处结队摆列,上面略放些粮
米。各处藏下火炮,及铺放硫黄焰硝灌过的干柴。教施恩、薛永、穆春、李忠领兵
二千,埋于东泥冈路口。教单廷圭领马兵一千,于庄南路口,等候贼人到来,都是
恁般恁般,依我行事。”柴进同神火将军魏定国,领步兵三百人,都带火种火器,
上山埋伏于丛密树林里。

等到二更时分,贼将縻貹果然同了二个偏将,领着万余军马,人披软战,马摘銮铃,
掩旗息鼓,疾驰到南土冈门口来。单廷圭见贼兵来,教军士燃点火把,接住厮杀。
单廷圭与縻貹斗不到四五合,单廷圭拨马领兵退入去。那縻貹是有勇无谋的人,领
兵一径抢进来。薛永、施恩见南路举火,即教李忠、穆春分兵一千,疾驰到庄南,
把住路口。那时贼兵都喊杀连天抢入去,只望东北上风头杀来,乃是空屋,不见粮
草。縻貹领兵四面搜索,看见下风头只有一二百辆粮草车,有五六百军士看守,见
贼兵来,发声喊,都奔散了。縻貹道:“原来不多粮草!”叫军士打火把照看,中
间车队里,每队有两辆缎匹车。那些贼兵见了,便去乱抢。縻貹急要止遏时,又被
山上将火箭火把乱打射下来,草房柴车上,都燔烧起来。贼兵发喊,急躲避时,早
被火炮药线引着火,传递得快,如轰雷般打击出来,贼兵奔走不迭的,都被火炮击
死。拈指间,烘烘火起,烈烈烟生。但见:

风随火势,火趁风威。千枝火箭掣金蛇,万个轰雷震火焰。骊山顶上,料应褒
姒逞英雄;扬子江头,不弱周郎施妙计。氤氲紫雾腾天起,闪烁红霞贯地来。必必
剥剥响不绝,浑如除夜放炮竹。

当下火势昌炽,炮声震响,如天摧地烈之声。须臾,百十间草房,变做烟团火块。
縻貹被火炮击死,贼兵击死大半,焦头烂额者无数,又被单廷圭、施恩等三路追杀
进来,二个偏将都被杀死,一万人马,只有千余人从土冈上爬出去,逃脱性命。天
明,柴进等仍与李应等合兵一处,将粮草运送大寨来。宋先锋正升帐,遣调兵马杀
贼,只见马军拴束马匹,步军安排器械,正是:旌旗红展一天霞,刀剑白铺千里雪。

毕竟宋江等如何厮杀,且听下回分解。