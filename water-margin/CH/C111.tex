\chapter{张顺夜伏金山寺~宋江智取润州城}

话说这九千三百里扬子大江,远接三江,却是汉阳江、浔阳江、扬子江。从四
川直至大海,中间通着多少去处,以此呼为万里长江。地分吴、楚,江心内有两座
山:一座唤做金山,一座唤做焦山。金山上有一座寺,绕山起盖,谓之寺里山。焦
山上一座寺,藏在山凹里,不见形势,谓之山里寺。这两座山,生在江中,正占着
楚尾吴头,一边是淮东扬州,一边是浙西润州,今时镇江是也。
且说润州城郭,却是方腊手下东厅枢密使吕师囊守把江岸。此人原是歙州富户,因
献钱粮与方腊,官封为东厅枢密使。幼年曾读兵书战策,惯使一条丈八蛇矛,武艺
出众。部下管领着十二个统制官,名号江南十二神,协同守把润州江岸。那十二神:
擎天神

福州沈刚

游弋神

歙州潘文得
遁甲神

睦州应明

六丁神

明州徐统
霹雳神

越州张近仁

巨灵神

杭州沈泽
太白神

湖州赵毅

太岁神

宣州高可立
吊客神

常州范畴

黄神

润州卓万里
豹尾神

江州和潼

丧门神

苏州沈
话说枢密使吕师囊,统领着五万南兵,据住江岸。甘露亭下,摆列着战船三千余只,
江北岸却是瓜洲渡口,净荡荡地无甚险阻。
此时先锋使宋江兵马战船,水陆并进,已到淮安了,约至扬州取齐。当日宋先锋在
帐中与军师吴用等商议:“此去大江不远,江南岸便是贼兵守把,谁人与我先去探
路一遭,打听隔江消息,可以进兵?”帐下转过四员战将,皆云愿往。那四个?一
个是小旋风柴进,一个是浪里白跳张顺,一个是拚命三郎石秀,一个是活阎罗阮小
七。宋江道:“你四人分作两路:张顺和柴进,阮小七和石秀,可直到金、焦二山
上宿歇,打听润州贼巢虚实,前来扬州回话。”四人辞了宋江,各带了两个伴当,
扮做客人,取路先投扬州来。此时一路百姓,听得大军来征剿方腊,都挈家搬在村
里躲避了。四个人在扬州城里分别,各办了些干粮。石秀自和阮小七带了两个伴当,
投焦山去了。
却说柴进和张顺也带了两个伴当,将干粮捎在身边,各带把锋快尖刀,提了朴刀,
四个奔瓜洲来。此时正是初春天气,日暖花香,到得扬子江边,凭高一望,淘淘雪
浪,滚滚烟波,是好江景也。有诗为证:
万里烟波万里天,红霞遥映海东边。
打鱼舟子浑无事,醉拥青蓑自在眠。
这柴进二人,望见北固山下一代都是青白二色旌旗,岸边一字儿摆着许多船只,江
北岸上,一根木头也无。柴进道:“瓜洲路上,虽有屋宇,并无人住,江上又无渡
船,怎生得知隔江消息?”张顺道:“须得一间屋儿歇下,看兄弟赴水过去对江金
山脚下,打听虚实。”柴进道:“也说得是。”当下四个人奔到江边,见一带数间
草房,尽皆关闭,推门不开。张顺转过侧首,掇开一堵壁子,钻将入去,见个白头
婆婆,从灶边走起来。张顺道:“婆婆,你家为甚不开门?”那婆婆答道:“实不
瞒客人说,如今听得朝廷起大军来与方腊厮杀。我这里正是风门水口,有些人家,
都搬了别处去躲,只留下老身在这里看屋。”张顺道:“你家男子汉那里去了?”
婆婆道:“村里去望老小去了。”张顺道:“我有四个人,要渡江过去,那里有船
觅一只?”婆婆道:“船却那里去讨?近日吕枢密听得大军来和他厮杀,都把船只
拘管过润州去了。”张顺道:“我四人自有粮食,只借你家宿歇两日,与你些银子
作房钱,并不搅扰你。”婆婆道:“歇却不妨,只是没有床席。”张顺道:“我们
自有措置。”婆婆道:“客人,只怕早晚有大军来!”张顺道:“我们自有回避。”
当时开门,放柴进和伴当入来,都倚了朴刀,放了行李,取些干粮烧饼出来吃了。
张顺再来江边,望那江景时,见金山寺正在江心里。但见:

江吞鳌背,山耸龙鳞。烂银盘涌出青螺,软翠帷远拖素练。遥观金殿,受八面
之天风;远望钟楼,倚千层之石壁。梵塔高侵沧海日,讲堂低映碧波云。无边阁,
看万里征帆;飞步亭,纳一天爽气。郭璞墓中龙吐浪,金山寺里鬼移灯。
张顺在江边看了一回,心中思忖道:“润州吕枢密必然时常到这山上,我且今夜去
走一遭,必知消息。”回来和柴进商量道:“如今来到这里,一只小船也没,怎知
隔江之事?我今夜把衣服打拴了两个大银,顶在头上,直赴过金山寺去,把些财赂
与那和尚,讨个虚实,回报先锋哥哥。你只在此间等候。”柴进道:“早干了事便
回。”
是夜星月交辉,风恬浪静,水天一色。黄昏时分,张顺脱膊了,扁扎起一腰白绢水
儿,把这头巾衣服裹了两个大银,拴缚在头上,腰间带一把尖刀,从瓜洲下水,
直赴开江心中来。那水淹不过他胸脯,在水中如走旱路,看看赴到金山脚下,见石
峰边缆着一只小船。张顺爬到船边,除下头上衣包,解了湿衣,扎拭了身上,穿上
衣服,坐在船中。听得润州更鼓,正打三更。张顺伏在船内望时,只见上溜头一只
小船,摇将过来。张顺看了道:“这只船来得跷蹊,必有奸细!”便要放船开去,
不想那只船一条大索锁了,又无橹篙。张顺只得又脱了衣服,拔出尖刀,再跳下江
里,直赴到那船边。
船上两个人摇着橹,只望北岸,不提防南边,只顾摇。张顺却从水底下一钻,钻到
船边,扳住船舷,把尖刀一削,两个摇橹的撒了橹,倒撞下江里去了。张顺早跳在
船上。那船舱里钻出两个人来,张顺手起一刀,砍得一个下水去,那个吓得倒入舱
里去。张顺喝道:“你是甚人?那里来的船只?实说,我便饶你!”那人道:“好汉
听禀:小人是此间扬州城外定浦村陈将士家干人,使小人过润州投拜吕枢密那里献
粮,准了,使个虞候和小人同回,索要白粮五万石、船三百只,作进奉之礼。”张
顺道:“那个虞候,姓甚名谁?现在那里?”干人道:“虞候姓叶名贵,却才好汉
砍下江里去的便是。”张顺道:“你却姓甚?甚么名字?几时过去投拜?船里有甚物
件?”干人道:“小人姓吴名成,今年正月初七日渡江。吕枢密直教小人去苏州,
见了御弟三大王方貌,关了号色旌旗三百面,并主人陈将士官诰,封做扬州府尹,
正授中明大夫名爵,更有号衣一千领,及吕枢密付一道。”张顺又问道:“你的
主人姓甚名字?有多少人马?”吴成道:“人有数千,马有百十余匹。嫡亲有两个
孩儿,好生了得。长子陈益,次子陈泰。主人将士,叫做陈观。”张顺都问了备细
来情去意,一刀也把吴成剁下水里去了。船尾上装起橹来,径摇到瓜洲。
柴进听橹声响,急忙出来看时,见张顺摇只船来,柴进便问来由。张顺把前事一一
说了,柴进大喜,去船舱里,取出一包袱文书,并三百面红绢号旗,杂色号衣一千
领,做两担打迭了。张顺道:“我却去取了衣裳来。”把船再摇到金山脚下,取了
衣裳、巾帻、银子,再摇到瓜洲岸边,天色方晓,重雾罩地。张顺把船砍漏,推开
江里去沉了。来到屋下,把三二两银子与了婆婆,两个伴当挑了担子,径回扬州来。
此时宋先锋军马俱屯扎在扬州城外,本州官员迎接宋先锋入城馆驿内安下,连日筵
宴,供给军士。
却说柴进、张顺伺候席散,在馆驿内见了宋江,备说陈观父子交结方腊,早晚诱引
贼兵渡江,来打扬州。天幸江心里遇见,教主帅成这件功劳。宋江听了大喜,便请
军师吴用商议用甚良策。吴用道:“既有这个机会,觑润州城易如反掌!先拿了陈
观,大事便定。只除如此如此。”即时唤浪子燕青,扮做叶虞候,教解珍、解宝扮
做南军。问了定浦村路头,解珍、解宝挑着担子,燕青都领了备细言语,三个出扬
州城来,取路投定浦村。离城四十余里,早问到陈将士庄前。见门首二三十庄客,
都整整齐齐,一般打扮。但见:
攒竹笠子,上铺着一把黑缨;细线衲袄,腰系着八尺红绢。牛膀鞋,登山似箭;獐
皮袜,护脚如绵。人人都带雁翎刀,个个尽提鸦嘴搠。
当下燕青改作浙人乡谈,与庄客唱喏道:“将士宅上有么?”庄客道:“客人那里
来?”燕青道:“从润州来。渡江错走了路,半日盘旋,问得到此。”庄客见说,
便引入客房里去,教歇了担子,带燕青到后厅来见陈将士。燕青便下拜道:“叶贵
就此参见!”拜罢,陈将士问道:“足下何处来?”燕青打浙音道:“回避闲人,
方敢对相公说。”陈将士道:“这几个都是我心腹人,但说不妨。”燕青道:“小
人姓叶名贵,是吕枢密帐前虞候。正月初七日,接得吴成密书,枢密甚喜,特差叶
贵送吴成到苏州,见御弟三大王,备说相公之意。三大王使人启奏,降下官诰,就
封相公为扬州府尹。两位直阁舍人,待吕枢密相见了时,再定官爵。今欲使令吴成
回程,谁想感冒风寒病症,不能动止。枢密怕误了大事,特差叶贵送到相公官诰,
并枢密文书、关防牌面、号旗三百面、号衣一千领,克日定时,要相公粮食船只,
前赴润州江岸交割。”便取官诰文书递与陈将士,看了大喜,忙摆香案,望南谢恩
已了,便唤陈益、陈泰出来相见。
燕青叫解珍、解宝取出号衣号旗,入后厅交付。陈将士便邀燕青请坐。燕青道:“小
人是个走卒,相公处如何敢坐?”陈将士道:“足下是那壁恩相差来的人,又与小
官赍诰敕,怎敢轻慢?权坐无妨。”燕青再三谦让了,远远地坐下。陈将士叫取酒
来,把盏劝燕青,燕青推却道:“小人天戒不饮酒。”待他把过三两巡酒,两个儿
子都来与父亲庆贺递酒。燕青把眼使叫解珍、解宝行事。解宝身边取出不按君臣的
药头,张人眼慢,放在酒壶里。燕青便起身说道:“叶贵虽然不曾将酒过江,借相
公酒果,权为上贺之意。”便斟一大钟酒,上劝陈将士满饮此杯。随即便劝陈益、
陈泰两个,各饮了一杯。当面有几个心腹庄客,都被燕青劝了一杯。燕青那嘴一努,
解珍出来外面寻了火种,身边取出号旗号炮,就庄前放起。左右两边,已有头领等
候,只听号炮响,前来策应。
燕青在堂里,见一个个都倒了,身边掣出短刀,和解宝一齐动手,早都割下头来。
庄门外哄动十个好汉,从前面打将入来。那十员将佐:花和尚鲁智深、行者武松、
九纹龙史进、病关索杨雄、黑旋风李逵、八臂那吒项充、飞天大圣李衮、丧门神鲍
旭、锦豹子杨林、病大虫薛永。门前众庄客,那里迎敌得住?里面燕青、解珍、解
宝早提出陈将士父子首级来。庄门外又早一彪人马官军到来,为首六员将佐。那六
员:美髯公朱仝、急先锋索超、没羽箭张清、混世魔王樊瑞、打虎将李忠、小霸王
周通。当下六员首将引一千军马,围住庄院,把陈将士一家老幼,尽皆杀了。拿住
庄客,引去浦里看时,傍庄傍港,泊着三四百只船,却满满装载粮米在内。众将得
了数目,飞报主将宋江。
宋江听得杀了陈将士,便与吴用计议进兵。收拾行李,辞了总督张招讨,部领大队
人马,亲到陈将士庄上,分拨前队将校,上船行计,一面使人催趱战船过去。吴用
道:“选三百只快船,船上各插着方腊降来的旗号。着一千军汉,各穿了号衣,其
余三四千人,衣服不等。”三百只船内,埋伏二万余人,更差穆弘扮做陈益,李俊
扮做陈泰,各坐一只大船,其余船分拨将佐。
第一拨船上,穆弘、李俊管领。穆弘身边,拨与十个偏将簇拥着。那十个:
项充

李衮

鲍旭

薛永

杨林
杜迁

宋万

邹渊

邹润

石勇
李俊身边,也拨与十个偏将簇拥着。那十个:
童威

童猛

孔明

孔亮

郑天寿
李立

李云

施恩

白胜

陶宗旺
第二拨船上,差张横、张顺管领。张横船上,拨与四个偏将簇拥着。那四个:
曹正

杜兴

龚旺

丁得孙
张顺船上,拨与四个偏将簇拥着。那四个:
孟康

侯健

汤隆

焦挺
第三拨船上便差十员正将管领,也分作两船进发。那十个:
史进

雷横

杨雄

刘唐

蔡庆
张清

李逵

解珍

解宝

柴进
这三百船上,分派大小正偏将佐,共计四十二员渡江。次后,宋江等却把战船装载
马匹,游龙飞鲸等船一千只,打着宋朝先锋使宋江旗号,大小马步将佐,一发载船
渡江。两个水军头领,一个是阮小二,一个是阮小五,总行催督。
且不说宋江中军渡江,却说润州北固山上,哨见对港三百来只战船,一齐出浦,船
上却插着护送衣粮先锋红旗号,南军连忙报入行省里来。吕枢密聚集十二个统制官,
都全副披挂,弓弩上弦,刀剑出鞘,带领精兵,自来江边观看。见前面一百只船,
先傍岸拢来。船上望着两个为头的,前后簇拥着的,都披着金锁子号衣,一个个都
是那彪形大汉。吕枢密下马,坐在银交椅上,十二个统制官两行把住江岸。
穆弘、李俊见吕枢密在江岸上坐地,起身声喏。左右虞候喝令住船,一百只船,一
字儿抛定了锚。背后那二百只船,乘着顺风,都到了。分开在两下拢来,一百只在
左,一百只在右,做三下均匀摆定了。客帐司下船来问道:“船从那里来?”穆弘
答道:“小人姓陈名益,兄弟陈泰,父亲陈观,特遣某等弟兄献纳白米五万石、船
三百只、精兵五千,来谢枢密恩相保奏之恩。”客帐司道:“前日枢密相公使叶虞
候去来,现在何处?”穆弘道:“虞候和吴成各染伤寒时疫,现在庄上养病,不能
前来。今将关防文书在此呈上。”
客帐司接了文书,上江岸来禀复吕枢密道:“扬州定浦村陈府尹男陈益、陈泰,纳
粮献兵,呈上原赍去关防文书在此。”吕枢密看果是原领公文,传钧旨,教唤二人
上岸。客帐司唤陈益、陈泰上来参见。穆弘、李俊上得岸来,随后二十个偏将,都
跟上去。排军喝道:“卿相在此,闲杂人不得近前!”二十个偏将都立住了。穆弘、
李俊躬身叉手,远远侍立。客帐司半晌方才引一人过去参拜了,跪在面前。吕枢密
道:“你父亲陈观,如何不自来?”穆弘禀道:“父亲听知是梁山泊宋江等领兵到
来,诚恐贼人下乡扰搅,在家支吾,未敢擅离。”吕枢密道:“你两个那个是兄?”
穆弘道:“陈益是兄。”吕枢密道:“你弟兄两个,曾习武艺么?”穆弘道:“托
赖恩相福荫,颇曾训练。”吕枢密道:“你将来白粮,怎地装载?”穆弘道:“大
船装粮三百石,小船装粮二百石。”吕枢密道:“你两个来到,恐有他意!”穆弘
道:“小人父子,一片孝顺之心,怎敢怀半点外意?”吕枢密道:“虽然是你好心,
吾观你船上军汉模样非常,不由人不疑。你两个只在这里,吾差四个统制官,引一
百军人下船搜看,但有分外之物,决不轻恕。”穆弘道:“小人此来,指望恩相重
用,何必见疑!”
吕师囊正欲点四个统制下船搜看,只见探马报道:“有圣旨到南门外了,请枢相便
上马迎接。”吕枢密急上了马,便分付道:“且与我把住江岸,这两个陈益、陈泰
随将我来。”穆弘把眼看李俊一觉。等吕枢密先行去了,穆弘、李俊随后招呼二十
个偏将,便入城门。守门将校喝道:“枢密相公只叫这两个为头的入来。其余人伴,
休放进去!”穆弘、李俊过去了,二十个偏将都被挡住在城边。
且说吕枢密到南门外,接着天使,便问道:“缘何来得如此要急?”那天使是方腊
面前引进使冯喜,悄悄地对吕师囊道:“近日司天太监浦文英奏道:‘夜观天象,
有无数罡星入吴地分野,中间杂有一半无光,就里为祸不小。’天子特降圣旨,教
枢密紧守江岸。但有北边来的人,须要仔细盘诘,磨问实情。如是形影奇异者,随
即诛杀,勿得停留。”吕枢密听了大惊:“却才这一班人,我十分疑忌,如今却得
这话。且请到城中开读。”冯喜同吕枢密都到行省,开读圣旨已了,只见飞马又报:
“苏州又有使命,赍擎御弟三大王令旨到来。”言说:“你前日扬州陈将士投降一
节,未可准信,诚恐有诈。近奉圣旨,近来司天监内照见罡星入于吴地分野,可以
牢守江岸。我早晚自差人到来监督。”吕枢密道:“大王亦为此事挂心,下官已奉
圣旨。”随即令人牢守江面,来的船上人,一个也休放上岸,一面设宴管待两个使
命。
却说那三百只船上人,见半日没些动静。左边一百只船上张横、张顺,带八个偏将,
提军器上岸;右边一百只船上十员正将,都拿了枪刀,钻上岸来;守江面南军,拦
当不住。黑旋风李逵和解珍、解宝,便抢入城。守门官军急出拦截,李逵抡起双斧,
一砍一剁,早杀翻两个把门官军。城边发起喊来,解珍、解宝各挺钢叉入城,都一
时发作,那里关得城门迭?李逵横身在门底下,寻人砍杀,先至城边二十个偏将,
各夺了军器,就杀起来。
吕枢密急使人传令来,教牢守江面时,城门边已自杀入城了。十二个统制官,听得
城边发喊,各提动军马时,史进、柴进早招起三百只船内军兵,脱了南军的号衣,
为首先上岸,船舱里埋伏军兵,一齐都杀上岸来。为首统制官沈刚、潘文得两路军
马来保城门时,沈刚被史进一刀剁下马去,潘文得被张横刺斜里一枪搠倒。众军混
杀,那十个统制官,都望城子里退入去,保守家眷。穆弘、李俊在城中听得消息,
就酒店里夺得火种,便放起火来。吕枢密急上马时,早得三个统制官到来救应。城
里降天也似火起。瓜洲望见,先发一彪军马,过来接应。城里四门,混战良久,城
上早竖起宋先锋旗号。四面八方,混杀人马,难以尽说,下来便见。
且说江北岸,早有一百五十只战船傍岸,一齐牵上战马,为首十员战将登岸,都是
全付披挂。那十员大将:关胜、呼延灼、花荣、秦明、郝思文、宣赞、单廷、韩
滔、彭、魏定国。正偏战将一十员,部领二千军马,冲杀入城。此时吕枢密方才
大败,引着中伤人马,径奔丹徒县去了。大军夺得润州,且教救灭了火,分拨把住
四门,却来江边,迎接宋先锋船。正见江面上游龙、飞鲸船只,乘着顺风,都到南
岸。大小将佐迎接宋先锋入城,预先出榜,安抚百姓,点本部将佐,都到中军请功。
史进献沈刚首级,张横献潘文得首级,刘唐献沈泽首级,孔明、孔亮生擒卓万里,
项充、李衮生擒和潼,郝思文箭射死徐统。得了润州,杀了四个统制官,生擒两个
统制官,杀死牙将官兵,不计其数。
宋江点本部将佐,折了三个偏将,都是乱军中被箭射死,马踏身亡。那三个?一个
是云里金刚宋万,一个是没面目焦挺,一个是九尾龟陶宗旺。宋江见折了三将,心
中烦恼,怏怏不乐。吴用劝道:“生死人之分定。虽折了三个兄弟,且喜得了江南
第一个险隘州郡,何故烦恼,有伤玉体?要与国家干功,且请理论大事。”宋江道:
“我等一百八人,天文所载,上应星曜。当初梁山泊发愿,五台山设誓,但愿同生
同死。回京之后,谁想道先去了公孙胜,御前留了金大坚、皇甫端,蔡太师又用了
萧让,王都尉又要了乐和。今日方渡江,又折了我三个弟兄。想起宋万这人,虽然
不曾立得奇功,当初梁山泊开创之时,多亏此人。今日作泉下之客!”宋江传令,
叫军士就宋万死处,搭起祭仪,列了银钱,排下乌猪白羊,宋江亲自祭祀奠酒。就
押生擒到伪统制卓万里、和潼,就那里斩首沥血,享祭三位英魂。宋江回府治里,
支给功赏,一面写了申状,使人报捷,亲请张招讨,不在话下。沿街杀的死尸,尽
教收拾出城烧化,收拾三个偏将尸骸,葬于润州东门外。
且说吕枢密折了大半人马,引着六个统制官,退守丹徒县,那里敢再进兵。申将告
急文书,去苏州报与三大王方貌求救。闻有探马报来,苏州差元帅邢政领军到来了。
吕枢密接见邢元帅,问慰了,来到县治,备说陈将士诈降缘由,以致透漏宋江军马
渡江。“今得元帅到此,可同恢复润州”。邢政道:“三大王为知罡星犯吴地,特
差下官领军到来,巡守江面。不想枢密失利,下官与你报仇,枢密当以助战。”次
日,邢政引军来恢夺润州。
却说宋江在润州衙内与吴用商议,差童威、童猛引百余人去焦山,寻取石秀、阮小
七,一面调兵出城,来取丹徒县。点五千军马,为首差十员正将。那十人:关胜、
林冲、秦明、呼延灼、董平、花荣、徐宁、朱仝、索超、杨志。当下十员正将部领
精兵五千,离了润州,望丹徒县来。关胜等正行之次,路上正迎着邢政军马。两军
相对,各把弓箭射住阵脚,排成阵势。南军阵上,邢政挺枪出马,六个统制官,分
在两下。宋军阵中关胜见了,纵马舞青龙偃月刀来战邢政。两员将斗到十四五合,
一将翻身落马。正是:瓦罐不离井上破,将军必在阵前亡。
毕竟二将厮杀,输了的是谁,且听下回分解。