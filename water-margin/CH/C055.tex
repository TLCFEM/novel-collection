\chapter{高太尉大兴三路兵~呼延灼摆布连环马}

话说高太尉问呼延灼道:“将军所保何人,可为先锋?”呼延灼禀道:“小人
举保陈州团练使,姓韩,名滔,原是东京人氏,曾应过武举出身。使一条枣木槊,
人呼为百胜将军。此人可为正先锋。又有一人,乃是颍州团练使,姓彭,名,亦
是东京人氏,乃累代将门之子。使一口三尖两刃刀,武艺出众,人呼为天目将军。
此人可为副先锋。”高太尉听了大喜道:“若是韩、彭二将为先锋,何愁狂寇!”
当日高太尉就殿帅府押了两道牒文,着枢密院差人,星夜往陈、颍二州,调取韩滔、
彭,火速赴京。不旬日之间,二将已到京师,径来殿帅府,参见了太尉并呼延灼。
次日,高太尉带领众人,都往御教场中,操演武艺。看军了当,却来殿帅府,会同
枢密院官,计议军机重事。高太尉问道:“你等三路,总有多少人马?”呼延灼答
道:“三路军马,计有五千,连步军,数及一万。”高太尉道:“你三人亲自回州,
拣选精锐马军三千,步军五千,约会起程,收剿梁山泊。”呼延灼禀道:“此三路
马步军兵,都是训练精熟之士,人强马壮,不必殿帅忧虑。但恐衣甲未全,只怕误
了日期,取罪不便,乞恩相宽限。”高太尉道:“既是如此说时,你三人可就京师
甲仗库内,不拘数目,任意选拣衣甲盔刀,关领前去。务要军马整齐,好与对敌。
出师之日,我自差官来点视。”呼延灼领了钧旨,带人往甲仗库关支。呼延灼选讫
铁甲三千副,熟皮马甲五千副,铜铁头盔三千顶,长枪二千根,滚刀一千把,弓箭
不计其数,火炮铁炮五百余架,都装载上车。临辞之日,高太尉又拨与战马三千匹。
三个将军,各赏了金银缎匹,三军尽关了粮赏。呼延灼和韩滔、彭,都与了必胜
军状,辞别了高太尉并枢密院等官,三人上马,都投汝宁州来。于路无话。

到得本州,呼延灼便道:“韩滔、彭,各往陈、颍二州起军,前来汝宁会合。”
不到半月之上,三路兵马,都已完足。呼延灼便把京师关到衣甲盔刀、旗枪鞍马,
并打造连环、铁铠、军器等物,分三军已了,伺候出军。高太尉差到殿帅府两员
军官,前来点视。犒赏三军已罢,呼延灼摆布三路兵马出城,端的是:

鞍上人披铁铠,坐下马带铜铃。旌旗红展一天霞,刀剑白铺千里雪。弓弯鹊画,
飞鱼袋半露龙梢;笼插雕翎,狮子壶紧拴豹尾。人顶深盔垂护项,微漏双睛;马披
重甲带朱缨,单悬四足。开路人兵,齐担大斧;合后军将,尽拈长枪。数千甲马离
州城,三个将军来水泊。
当下起军,摆布兵马出城,前军开路韩滔,中军主将呼延灼,后军催督彭,马步
三军人等,浩浩荡荡,杀奔梁山泊来。

却说梁山泊远探报马,径到大寨,报知此事。聚义厅上,当中晁盖、宋江,上
首军师吴用,下首法师公孙胜并众头领,各与柴进贺喜,终日筵宴,听知报道:“汝
宁州双鞭呼延灼,引着军马到来征进。”众皆商议迎敌之策。吴用便道:“我闻此
人,祖乃开国功臣河东名将呼延赞之后,嫡派子孙。此人武艺精熟,使两条铜鞭,
人不可近。必用能征敢战之将,先以力敌,后用智擒。”说言未了,黑旋风李逵便
道:“我与你去捉这厮!”宋江道:“你如何去得?我自有调度:可请霹雳火秦明
打头阵,豹子头林冲打第二阵,小李广花荣打第三阵,一丈青扈三娘打第四阵,病
尉迟孙立打第五阵;将前面五阵,一队队战罢如纺车般转作后军。我亲自带引十个
弟兄,引大队人马押后。左军五将——朱仝、雷横、穆弘、黄信、吕方;右军五将——
杨雄、石秀、欧鹏、马麟、郭盛。水路中可请李俊、张横、张顺、阮家三弟兄,驾
船接应。却教李逵与杨林,引步军分作两路,埋伏救应。”宋江调拨已定,前军秦
明早引人马下山,向平原旷野之处,列成阵势。此时虽是冬天,却喜和暖。等候了
一日,早望见官军到来,先锋队里,百胜将韩滔领兵扎下寨栅,当晚不战。

次日天晓,两军对阵,三通画鼓,出到阵前。马上横着狼牙棍,望对阵门旗开
处,先锋将韩滔横槊勒马,大骂秦明道:“天兵到此,不思早早投降,还敢抗拒,
不是讨死!我直把你水泊填平,梁山踏碎,生擒活捉你这伙反贼解京,碎尸万段!”
秦明本是性急的人,听了也不打话,便拍马舞起狼牙棍,直取韩滔。韩滔挺槊跃马,
来战秦明。两个斗到二十余合,韩滔力怯,只待要走。背后中军主将呼延灼已到,
见韩滔战秦明不下,便从中军舞起双鞭,纵坐下那匹御赐踢雪乌骓,咆哮嘶喊,来
到阵前,秦明见了,欲待来战呼延灼,第二拨豹子头林冲已到,便叫:“秦统制少
歇,看我战三百合,却理会!”林冲挺起蛇矛,直奔呼延灼,秦明自把军马从左边
踅向山坡后去。这里呼延灼自战林冲。两个正是对手:枪来鞭去花一团,鞭去枪来
锦一簇。两个斗到五十合之上,不分胜败。第三拨小李广花荣军到,阵门下大叫道:
“林将军少息,看我擒捉这厮!”林冲拨转马便走。呼延灼因见林冲武艺高强,也
回本阵。林冲自把本部军马一转,转过山坡后去,让花荣挺枪出马。呼延灼后军也
到,天目将彭横着那三尖两刃四窍八环刀,骤着五明千里黄花马,出阵大骂花荣
道:“反国逆贼,何足为道!与吾并个输赢!”花荣大怒,也不答话,便与彭交
马。两个战二十余合,呼延灼看见彭力怯,纵马舞鞭,直奔花荣。斗不到三合,
第四拨一丈青扈三娘人马已到,大叫:“花将军少歇,看我捉这厮。”花荣也引军
望右边踅转山坡下去了。彭来战一丈青未定,第五拨病尉迟孙立军马早到,勒马
于阵前摆着,看这扈三娘去战彭。两个正在征尘影里,杀气阴中:一个使大杆刀,
一个使双刀。两个斗到二十余合,一丈青把双刀分开,回马便走。彭要逞功劳,
纵马赶来,一丈青便把双刀挂在马鞍鞒上,袍底下取出红锦套索,上有二十四个金
钩,等彭马来得近,扭过身躯,把套索望空一撒,看得亲切,彭措手不及,早
拖下马来。孙立喝教众军一发向前,把彭捉了。

呼延灼看见大怒,忿力向前来救,一丈青便拍马来迎敌。呼延灼恨不得一口水
吞了那一丈青。两个斗到十合之上,急切赢不得一丈青,呼延灼心中想道:“这个
泼妇人在我手里斗了许多合,倒恁地了得!”心忙意急,卖个破绽,放他入来,却
把双鞭只一盖,盖将下来。那双刀却在怀里,提起右手铜鞭,望一丈青顶门上打下
来。却被一丈青眼明手快,早起刀只一隔,右手那口刀,望上直飞起来。却好那一
鞭打将下来,正在刀口上,“铮”地一声响,火光迸散,一丈青回马望本阵便走。
呼延灼纵马赶来,病尉迟孙立见了,便挺枪纵马向前,迎住厮杀。背后宋江却好引
十对良将都到,列成阵势。一丈青自引了人马,也投山坡下去了。

宋江见活捉得天目将彭心中甚喜,且来阵前看孙立与呼延灼交战。孙立也把
枪带住,手腕上绰起那条竹节钢鞭,来迎呼延灼。两个都使钢鞭,那更一般打扮:
病尉迟孙立是交角铁幞头,大红罗抹额,百花点翠皂罗袍,乌油戗金甲,骑一匹乌
骓马,使一条竹节虎眼鞭,赛过尉迟恭;这呼延灼却是冲天角铁幞头,锁金黄罗抹
额,七星打钉皂罗袍,乌油对嵌铠甲,骑一匹御赐踢雪乌骓,使两条水磨八棱钢鞭,
左手的重十二斤,右手重十三斤,真似呼延赞。两个在阵前左盘右旋,斗到三十余
合,不分胜败。宋江看了,喝采不已。有诗为证:
各跨乌骓健似龙,呼延赞对尉迟恭。
双鞭遇敌真奇事,更好同归水浒中。

官军阵里韩滔,见说折了彭,便去后军队里,尽起军马,一发向前厮杀。宋
江只怕冲将过来,便把鞭梢一指,十个头领,引了大小军士,掩杀过去。背后四路
军兵,分作两路夹攻拢来。呼延灼见了,急收转本部军马,各敌个住。为何不能全
胜?却被呼延灼阵里都是连环马官军。马带马甲,人披铁铠。马带甲,只露得四蹄
悬地;人披铠,只露着一对眼睛。宋江阵上虽有甲马,只是红缨面具,铜铃雉尾而
已。这里射将箭去,那里甲都护住了。那三千马军,各有弓箭,对面射来,因此不
敢近前。宋江急叫鸣金收军,呼延灼也退二十余里下寨。

宋江收军,退到山西下寨,屯住军马,且教左右群刀手,簇拥彭过来。宋江
望见,便起身喝退军士,亲解其缚,扶入帐中,分宾而坐。宋江便拜。彭连忙答
礼拜道:“小子被擒之人,理合就死,何故将军以宾礼待之?”宋江道:“某等众
人,无处容身,暂占水泊,权时避难,造恶甚多。今者朝廷差遣将军前来收捕,本
合延颈就缚。但恐不能存命,因此负罪交锋,误犯虎威,敢乞恕罪。”彭答道:
“素知将军仗义行仁,扶危济困,不想果然如此义气!倘蒙存留微命,当以捐躯保
奏。”宋江道:“某等众兄弟也只待圣主宽恩,赦宥重罪,忘生报国,万死不辞。”
诗曰:
忠为君王恨贼臣,义连兄弟且藏身。
不因忠义心如一,安得团圆百八人。
宋江当日就将天目将彭,使人送上大寨,教与晁天王相见,留在寨里。这里自一
面犒赏三军并众头领,计议军情。再说呼延灼收军下寨,自和韩滔商议,如何取胜
梁山水泊。韩滔道:“今日这厮们见俺催军近前,他便慌忙掩击过来,明日尽数驱
马军向前,必获大胜。”呼延灼道:“我已如此安排下了,只要和你商量相通。”
随即传下将令:“教三千匹马军,做一排摆着,每三十匹一连,却把铁环连锁;但
遇敌军,远用箭射,近则使枪,直冲入去;三千连环马军,分作一百队锁定;五千
步军,在后策应。明日休得挑战,我和你押后掠阵。但若交锋,分作三面冲将过去。”
计策商量已定,次日天晓出战。

却说宋江次日把军马分作五队在前,后军十将簇拥,两路伏兵,分于左右。秦
明当先,搦呼延灼出马交战,只见对阵但只呐喊,并不交锋。为头五军,都一字儿
摆在阵前:中是秦明,左是林冲、一丈青,右是花荣,孙立在后。随即宋江引十将
也到,重重迭迭,摆着人马。看对阵时,约有一千步军,只是擂鼓发喊,并无一人
出马交锋。宋江看了,心中疑惑,暗传号令:“教后军且退。”却纵马直到花荣队
里窥望。猛听对阵里连珠炮响,一千步军,忽然分作两下,放出三面连环马军,直
冲将来;两边把弓箭乱射,中间尽是长枪。宋江看了大惊,急令众军把弓箭施放,
那里抵敌得住。每一队三十匹马,一齐跑发,不容你不向前走。那连环马军,漫山
遍野,横冲直撞将来。前面五队军马望见,便乱跑了,策立不定;后面大队人马,
拦当不住,各自逃生。宋江飞马慌忙便走,十将拥护而行。背后早有一队连环马军
追将来,却得伏兵李逵、杨林引人从芦苇中杀出来,救得宋江。逃至水边,却有李
俊、张横、张顺、三阮六个水军头领,摆下战船接应。宋江急急上船,便传将令:
教分头去救应众头领下船。那连环马直赶到水边,乱箭射来,船上却有傍牌遮护,
不能损伤。慌忙把船棹到鸭嘴滩头,尽行上岸。就水寨里整点人马,折其大半,却
喜众头领都全;虽然折了些马匹,都救得性命。少刻,只见石勇、时迁、孙新、顾
大嫂,都逃命上山,却说:“步军冲杀将来,把店屋平拆了去。我等若无号船接应,
尽被擒捉。”宋江一一亲自抚慰,计点众头领时,中箭者六人:林冲、雷横、李逵、
石秀、孙新、黄信;小喽罗中伤带箭者,不计其数。晁盖闻知,同吴用、公孙胜下
山来动问。宋江眉头不展,面带忧容。吴用劝道:“哥哥休忧,胜败乃兵家常事,
何必挂心?别生良策,可破连环军马。”晁盖便传号令:分付水军,牢固寨栅船只,
保守滩头,晓夜提备,请宋公明上山安歇。宋江不肯上山,只就鸭嘴滩寨内驻扎,
只教带伤头领上山养病。

却说呼延灼大获全胜,回到本寨,开放连环马,都次第前来请功。杀死者不计
其数,生擒的五百余人,夺得战马三百余匹。随即差人前去京师报捷,一面犒赏三
军。

却说高太尉正在殿帅府坐衙,门上报道:“呼延灼收捕梁山泊得胜,差人报捷。”
心中大喜。次日早朝,越班奏闻天子。徽宗甚喜,敕赏黄封御酒十瓶,锦袍一领;
差官一员,赍钱十万贯,前去行营赏军。高太尉领了圣旨,同到殿帅府,随即差官
赍捧前去。

却说呼延灼已知有天使到,与韩滔出二十里外迎接。接到寨中,谢恩受赏已毕,
置酒管待天使。一面令韩先锋钱赏军,且将捉到五百余人,囚在寨中,待拿得贼
首,一并解赴京师,示众施行。天使问:“彭团练如何失陷?”呼延灼道:“为因
贪捉宋江,深入重地,致被擒捉。今次群贼必不敢再来。小可分兵攻打,务要肃清
山寨,扫尽水洼,擒获众贼,拆毁巢穴。但恨四面是水,无路可进。遥观寨栅,只
除非得火炮飞打,以碎贼巢。久闻东京有个炮手凌振,名号轰天雷。此人善造火炮,
能去十四五里远近,石炮落处,天崩地陷,山倒石裂。若得此人,可以攻打贼巢。
更兼他深通武艺,弓马熟娴。若得天使回京,于太尉前言知此事,可以急急差遣到
来,克日可取贼巢。”使命应允。次日起程,于路无话。回到京师,来见高太尉,
备说呼延灼求索炮手凌振,要建大功。高太尉听罢,传下钧旨,教唤甲仗库副炮手
凌振那人来。原来凌振祖贯燕陵人,是宋朝盛世第一个炮手,人都呼他是轰天雷。
更兼武艺精熟。曾有四句诗赞凌振的好处:
强火发时城郭碎,烟云散处鬼神愁。
金轮子母轰天振,炮手名闻四百州。

当下凌振来参见了高太尉,就受了行军统领官文凭,便教收拾鞍马军器起身。
且说凌振把应用的烟火、药料,就将做下的诸色火炮,并一应的炮石、炮架,装载
上车;带了随身衣甲盔刀行李等件,并三四十个军汉,离了东京,取路投梁山泊来。
到得行营,先来参见主将呼延灼,次见先锋韩滔,备问水寨远近路程。山寨峻去
处,安排三等炮石攻打:第一是风火炮,第二是金轮炮,第三是子母炮。先令军健
整顿炮架,直去水边竖起,准备放炮。

却说宋江在鸭嘴滩上小寨内,和军师吴学究商议破阵之法,无计可施。有探细
人来报道:“东京新差一个炮手,号作轰天雷凌振,即日在于水边竖起架子,安排
施放火炮,攻打寨栅。”吴学究道:“这个不妨。我山寨四面都是水泊,港汊甚多,
宛子城离水又远,纵有飞天火炮,如何能够打得到城边?且弃了鸭嘴滩小寨,看他
怎地设法施放,却做商议。”当下宋江弃了小寨,便都起身,且上关来。晁盖、公
孙胜接到聚义厅上,问道:“似此如何破敌?”动问未绝,早听得山下炮响。一连
放了三个火炮,两个打在水里,一个直打到鸭嘴滩边小寨上。宋江见说,心中展转
忧闷,众头领尽皆失色。吴学究道:“若得一人,诱引凌振到水边,先捉了此人,
方可商议破敌之法。”晁盖道:“可着李俊、张横、张顺、三阮,六人棹船如此行
事,岸上朱仝、雷横如此接应。”

且说六个水军头领,得了将令,分作两队:李俊和张横先带了四五十个会水的
军士,用两只快船,从芦苇深处,悄悄过去;背后张顺、三阮,掌四十余只小船接
应。再说李俊、张横上到对岸,便去炮架子边呐声喊,把炮架推翻。军士慌忙报与
凌振知道,凌振便带了风火二炮,拿枪上马,引了一千余人赶将来。李俊、张横领
人便走。凌振追至芦苇滩边,看见一字儿摆开四十余只小船,船上共有百十余个水
军。李俊、张横早跳在船上,故意不把船开。看看人马到来,呐声喊,都跳下水里
去了。凌振人马已到,便来抢船。朱仝、雷横却在对岸呐喊擂鼓。凌振夺得许多船
只,叫军健尽数上船,便杀过去。船才行到波心之中,只见岸上朱仝、雷横鸣起锣
来;水底下早钻起四五十水军,尽把船尾楔子拔了,水都滚入船里来;外边就势扳
翻船,军健都撞在水里。凌振急待回船,船尾舵橹,已自被拽下水底去了。两边却
钻上两个头领来,把船只一扳,仰合转来,凌振却被合下水里去。水底下却是阮小
二,一把抱住,直拖到对岸来。岸上早有头领接着,便把索子绑了,先解上山来。
水中生擒二百余人,一半水中死,些少逃得性命回去。诗曰:
怎许船军便渡河,不施火炮却如何。
空说半天轰霹雳,却愁尺水起风波。

呼延灼得知,急领军马赶将来时,船都已过鸭嘴滩去了。箭又射不着,人都不
见了,只忍得气。呼延灼恨了半晌,只得引了人马回去。且说众头领捉得轰天雷凌
振,解上山寨,先使人报知。宋江便同满寨头领下第二关迎接,见了凌振,连忙亲
解其缚,便埋怨众人道:“我叫你们礼请统领上山,如何恁的无礼!”凌振拜谢不
杀之恩,宋江便与他把盏已了,自执其手,相请上山。到大寨,见了彭已做了头
领,凌振闭口无言。彭劝道:“晁、宋二头领,替天行道,招纳豪杰,专等招安,
与国家出力。既然我等到此,只得从命。”宋江却又陪话,凌振答道:“小的在此
趋侍不妨,争奈老母妻子,都在京师,倘或有人知觉,必遭诛戮,如之奈何!”宋
江道:“但请放心,限日取还统领。”凌振谢道:“若得头领如此周全,死而瞑目。”
晁盖道:“且教做筵席庆贺。”

次日,厅上大聚会众头领。饮酒之间,宋江与众人商议破连环马之策。正无良
法,只见金钱豹子汤隆起身道:“小人不材,愿献一计。除是得这般军器和我一个
哥哥,可以破得连环甲马。”吴学究便问道:“贤弟,你且说用何等军器?你这个
令亲哥哥是谁?”汤隆不慌不忙,叉手向前,说出这般军器和那个人来。有分教:
四五个头领直往京师,三千余马军尽遭毒手。正是:计就玉京擒獬豸,谋成金阙捉
狻猊。

毕竟汤隆对众说出那般军器,甚么人来,且听下回分解。