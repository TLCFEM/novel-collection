\chapter{梁山泊分金大买市~宋公明全伙受招安}

话说燕青在李师师家遇见道君皇帝,告得一道本身赦书,次后见了宿太尉,又
和戴宗定计,去高太尉府中赚出萧让、乐和。四个人等城门开时,随即出城,径赶
回梁山泊来,报知上项事务。且说李师师当夜不见燕青来家,心中亦有些疑虑。却
说高太尉府中亲随人,次日供送茶饭与萧让、乐和,就房中不见了二人,慌忙报知
都管。都管便来花园中看时,只见柳树边拴着两条粗索,已知走了二人,只得报知
太尉。高俅听罢,吃了一惊,越添忧闷,只在府中推病不出。

次日五更,道君皇帝设朝,驾坐文德殿。文武班齐,天子宣命卷帘,旨令左右近臣,
宣枢密使童贯出班。问道:“你去岁统十万大军,亲为招讨,征进梁山泊,胜败如
何?”童贯跪下,便奏道:“臣旧岁统率大军,前去征进,非不效力,奈缘暑热,
军士不伏水土,患病者众,十死二三,臣见军马艰难,以此权且收兵罢战,各归本
营操练。所有御林军,于路病患,多有损折。次后降诏,此伙贼人,不伏招抚。及
高俅以舟师征进,亦中途抱病而返。”天子大怒,喝道:“都是汝等妒贤嫉能,奸
佞之臣,瞒着寡人行事!你去岁统兵征伐梁山泊,如何只两阵,被寇兵杀的人马辟
易,片甲只骑无还,遂令王师败绩。次后高俅那厮,废了州郡多少钱粮,陷害了许
多兵船,折了若干军马,自己又被寇活捉上山,宋江等不肯杀害,放将回来。寡人
闻宋江这伙,不侵州府,不掠良民,只待招安,与国家出力,都是汝等不才贪佞之
臣,枉受朝廷爵禄,坏了国家大事!汝掌管枢密,岂不自惭?本当拿问,姑免这次,
再犯不饶!”童贯默默无言,退在一边。

天子又问:“你大臣中,谁可前去招抚梁山泊宋江等一班人众?”圣宣未了,有殿
前太尉宿元景出班跪下,奏道:“臣虽不才,愿往一遭。”天子大喜:“寡人御笔
亲书丹诏。”便叫抬上御案,拂开诏纸,天子就御案上亲书丹诏。左右近臣,捧过
御宝,天子自行用讫。又命库藏官,教取金牌三十六面,银牌七十二面,红锦三十
六匹,绿锦七十二匹,黄封御酒一百八瓶,尽付与宿太尉。又赠正从表里二十四匹,
金字招安御旗一面,限次日便行。宿太尉就文德殿辞了天子。百官朝罢,童枢密羞
惭满面,回府推病,不敢入朝。高太尉闻知,恐惧无措,亦不敢入朝。有诗为证:
一封恩诏出明光,看梁山尽束装。
知道怀柔胜征伐,悔教赤子受痍伤。

且说宿太尉打担了御酒、金银牌面、缎匹表里之物,上马出城,打起御赐金字黄旗,
众官相送出南熏门,投济州进发,不在话下。却说燕青、戴宗、萧让、乐和四个连
夜到山寨,把上件事都说与宋公明并头领知道。燕青便取出道君皇帝御笔亲写赦书,
与宋江等众人看了。吴用道:“此回必有佳音。”宋江焚起好香,取出九天玄女课
来,望空祈祷祝告了,卜得个上上大吉之兆。宋江大喜:“此事必成。再烦戴宗、
燕青前去探听虚实,作急回报,好做准备。”戴宗、燕青去了数日,回来报说:“朝
廷差宿太尉亲赍丹诏,更有御酒、金银牌面、红绿锦缎表里,前来招安,早晚到也!”
宋江听罢大喜,在忠义堂上忙传将令,分拨人员,从梁山泊直抵济州地面,扎缚起
二十四座山棚,上面都是结彩悬花,下面陈设笙箫鼓乐。各处附近州郡,雇倩乐人,
分拨于各山棚去处,迎接诏敕。每一座山棚上,拨一个小头目监管。一壁教人分投
买办果品海味、按酒干食等项,准备筵宴茶饭席面。

且说宿太尉奉敕来梁山泊招安,一干人马迤逦都到济州。太守张叔夜出郭迎接入城,
馆驿中安下。太守起居宿太尉已毕,把过接风酒。张叔夜禀道:“朝廷颁诏敕来招
安,已是二次,盖因不得其人,误了国家大事。今者太尉此行,必与国家立大功也!”
宿太尉乃言:“天子近闻梁山泊一伙,以义为主,不侵州郡,不害良民,口称替天
行道,今差下官赍到天子御笔亲书丹诏,敕赐金牌三十六面,银牌七十二面,红锦
三十六匹,绿锦七十二匹,黄封御酒一百八瓶,表里二十四匹,来此招安,礼物轻
否?”张叔夜道:“这一班人,非在礼物轻重,要图忠义报国,扬名后代。若得太
尉早来如此,也不教国家损兵折将,虚耗了钱粮。此一伙义士归降之后,必与朝廷
建功立业。”宿太尉道:“下官在此专待,有烦太守亲往山寨报知,着令准备迎接。”
张叔夜答道:“小官愿往。”随即上马出城,带了十数个从人,径投梁山泊来。到
得山下,早有小头目接着,报上寨里来。宋江听罢,慌忙下山,迎接张太守上山。
到忠义堂上,相见罢,张叔夜道:“义士恭喜!朝廷特遣殿前宿太尉,赍擎丹诏,
御笔亲书,前来招安。敕赐金牌、表里、御酒、缎匹,现在济州城内。义士可以准
备迎接诏旨。”宋江大喜,以手加额道:“宋江等再生之幸!”当时留请张太守茶
饭。张叔夜道:“非是下官拒意,惟恐太尉见怪回迟。”宋江道:“略奉一杯,非
敢为礼。”张叔夜坚执便行。宋江忙教托出一盘金银相送。张太守见了,便道:“这
个决不敢受。”宋江道:“些少微物,聊表寸心。若事毕之后,尚容图报。”张叔
夜道:“深感义士厚意,且留于大寨,却来请领,亦未为晚。”太守可谓廉以律己
者矣!有诗为证:
济州太守世无双,不爱黄金爱宋江。
信是清廉能服众,非关威势可招降。

宋江便差大小军师吴用、朱武,并萧让、乐和四个,跟随张太守下山,直往济州来
参见宿太尉。约至后日,众多大小头目离寨三十里外,伏道相迎。当时吴用等跟随
太守张叔夜连夜下山,直到济州。次日,来馆驿中参见宿太尉,拜罢,跪在面前。
宿太尉教平身起来,俱各命坐。四个谦让,那里敢坐。太尉问其姓氏,吴用答道:
“小生吴用,在下朱武、萧让、乐和,奉兄长宋公明命,特来迎接恩相。兄长与弟
兄,后日离寨三十里外,伏道迎接。”宿太尉大喜,便道:“加亮先生,自从华州
一别之后,已经数载,谁想今日得与重会!下官知汝弟兄之心,素怀忠义,只被奸
臣闭塞,谗佞专权,使汝众人,下情不能上达。目今天子悉已知之,特命下官赍到
天子御笔亲书丹诏、金银牌面、红绿锦缎、御酒表里,前来招安。汝等勿疑,尽心
受领。”吴用等再拜称谢道:“山野狂夫,有劳恩相降临。感蒙天恩,皆出太尉之
赐。众弟兄刻骨铭心,难以补报。”张叔夜一面设宴管待。

到第三日清晨,济州装起香车三座,将御酒另一处龙凤盒内抬着;金银牌面、红绿
锦缎,另一处扛抬;御书丹诏,龙亭内安放。宿太尉上了马,靠龙亭东行,太守张
叔夜骑马在后相陪;吴用等四人,乘马跟着;大小人伴,一齐簇拥。前面马上,打
着御赐销金黄旗,金鼓旗幡队伍开路,出了济州,迤逦前行。未及十里,早迎着山
棚。宿太尉在马上看了,见上面结彩悬花,下面笙箫鼓乐,迫道迎接。再行不过数
十里,又是结彩山棚。前面望见香烟拂道,宋江、卢俊义跪在面前,背后众头领齐
齐都跪在地下迎接恩诏。宿太尉道:“都教上马。”一同迎至水边,那梁山泊千百
只战船,一齐渡将过去,直至金沙滩上岸。三关之上,三关之下,鼓乐喧天,军士
导从,仪卫不断,异香缭绕,直至忠义堂前下马。香车龙亭,抬放忠义堂上。中间
设着三个几案,都用黄罗龙凤桌围围着。正中设万岁龙牌,将御书丹诏放在中间;
金银牌面,放在左边;红绿锦缎,放在右边;御酒表里,亦放于前。金炉内焚着好
香。宋江、卢俊义邀请宿太尉、张太守上堂设坐。左边立着萧让、乐和,右边立着
裴宣、燕青。宋江、卢俊义等,都跪在堂前。裴宣喝拜。拜罢,萧让开读诏文。
制曰:朕自即位以来,用仁义以治天下,公赏罚以定干戈,求贤未尝少怠,爱民如
恐不及,遐迩赤子,咸知朕心。切念宋江、卢俊义等,素怀忠义,不施暴虐,归顺
之心已久,报效之志凛然。虽犯罪恶,各有所由,察其衷情,深可怜悯。朕今特差
殿前太尉宿元景,赍捧诏书,亲到梁山水泊,将宋江等大小人员所犯罪恶,尽行赦
免。给降金牌三十六面、红锦三十六匹,赐与宋江等上头领;银牌七十二面、绿锦
七十二匹,赐与宋江部下头目。赦书到日,莫负朕心,早早归顺,必当重用。故兹
诏敕,想宜悉知。

宣和四年春二月

日诏示
萧让读罢丹诏,宋江等山呼万岁,再拜谢恩已毕,宿太尉取过金银牌面、红绿锦缎,
令裴宣依次照名给散已罢。叫开御酒,取过银酒海,都倾在里面,随即取过旋杓舀
酒,就堂前温热,倾在银壶内。宿太尉执着金钟,斟过一杯酒来,对众头领道:“宿
元景虽奉君命,特赍御酒到此,命赐众头领,诚恐义士见疑,元景先饮此杯,与众
义士看,勿得疑虑。”众头领称谢不已。宿太尉饮毕,再斟酒来,先劝宋江,宋江
举杯跪饮。然后卢俊义、吴用、公孙胜,陆续饮酒,遍劝一百单八名头领,俱饮一
杯。宋江传令,教收起御酒,却请太尉居中而坐,众头领拜复起居。宋江进前称谢
道:“宋江昨者西岳得识台颜,多感太尉恩厚,于天子左右力奏,救拔宋江等再见
天日之光,铭心刻骨,不敢有忘。”宿太尉道:“元景虽知义士等忠义凛然,替天
行道,奈缘不知就里委曲之事,因此,天子左右未敢题奏,以致误了许多时。前
者收到闻参谋书,又蒙厚礼,方知有此衷情。其日天子在披香殿上,官家与元景闲
论,问起义士,以此元景奏知此事。不期天子已知备细,与某所奏相同。次日,天
子驾坐文德殿,就百官之前,痛责童枢密、深怪高太尉累次无功,亲命取过文房四
宝,天子御笔亲书丹诏,特差宿某亲到大寨,启请众头领。烦望义士早早收拾朝京,
休负圣天子宣召抚安之意。”众皆大喜,拜手称谢。礼毕,张太守推说地方有事,
别了太尉,自回城内去了。

这里且说宋江,教请出闻参谋相见,宿太尉欣然话旧,满堂欢喜。当请宿太尉居中
上坐,闻参谋对席相陪。堂上堂下,皆列位次,大设筵宴,轮番把盏。厅前大吹大
擂。虽无炮龙烹凤,端的是肉山酒海。当日尽皆大醉,各扶归幕次安歇。次日又排
筵宴,各各倾心露胆,讲说平生之怀。第三日,再排席面,请宿太尉游山,至暮尽
醉方散。倏尔已经数日,宿太尉要回,宋江等坚意相留。宿太尉道:“义士不知就
里,元景奉天子敕旨而来,到此间数日之久,荷蒙英雄慨然归顺,大义俱全。若不
急回,诚恐奸臣相妒,别生异议。”宋江等道:“太尉既然如此,不敢苦留。今日
尽此一醉,来早拜送恩相下山。”当时会集大小头领,尽来集义饮宴。吃酒中间,
众皆称谢。宿太尉又用好言抚恤,至晚方散。次日清晨,安排车马,宋江亲捧一盘
金珠到宿太尉幕次,再拜上献。宿太尉那里肯受。宋江再三献纳,方才收了。打迭
衣箱,拴束行李鞍马,准备起程。其余跟来人数,连日自是朱武、乐和管待,依例
饮馔,酒量高低,并皆厚赠金银财帛,众人皆喜。仍将金宝赍送闻参谋,亦不肯受。
宋江坚执奉承,才肯收纳。宋江遂请闻参谋随同宿太尉回京师。梁山泊大小头领,
金鼓细乐,相送太尉下山,渡过金沙滩,俱送过三十里外,众皆下马,与宿太尉把
盏饯行。

宋江当先执盏擎杯道:“太尉恩相回见天颜,善言保奏。”宿太尉回道:“义士但
且放心,只早早收拾朝京为上。军马若到京师来,可先使人到我府中通报。俺先奏
闻天子,使人持节来迎,方见十分公气。”宋江道:“恩相容复:小可水洼,自从
王伦上山开创之后,却是晁盖上山,今至宋江,已经数载,附近居民,扰害不浅。
小可愚意,今欲罄竭资财,买市十日,收拾已了,便当尽数朝京,安敢迟滞。亦望
太尉将此愚衷,上达天听,以宽限次。”宿太尉应允,别了众人,带了开诏一干人
马,自投济州而去。

宋江等却回大寨,到忠义堂上,鸣鼓聚众。大小头领坐下,诸多军校都到堂前。宋
江传令:“众弟兄在此,自从王伦开创山寨以来,次后晁天王上山建业,如此兴旺。
我自江州得众兄弟相救到此,推我为尊,已经数载。今日喜得朝廷招安,重见天日
之面,早晚要去朝京,与国家出力。今来汝等众人,但得府库之物,纳于库中公用,
其余所得之资,并从均分。我等一百八人,上应天星,生死一处。今者天子宽恩降
诏,赦罪招安,大小众人,尽皆释其所犯。我等一百八人,早晚朝京面圣,莫负天
子洪恩。汝等军校,也有自来落草的,也有随众上山的,亦有军官失陷的,亦有掳
掠来的。今次我等受了招安,俱赴朝廷。你等如愿去的,作数上名进发;如不愿去
的,就这里报名相辞。我自赍发你等下山,任从生理。”

宋江号令已罢,着落裴宣、萧让照数上名。号令一下,三军各各自去商议。当下辞
去的,也有三五千人。宋江皆赏钱物,赍发去了。愿随去充军者,作数报官。次日,
宋江又令萧让写了告示,差人四散去贴,晓示临近州郡乡镇村坊,各各报知,仍请
诸人到山买市十日。其告示曰:
梁山泊义士宋江等,谨以大义布告四方。向因聚众山林,多扰四方百姓。今日幸蒙
天子宽仁厚德,特降诏敕,赦免本罪,招安归降,朝暮朝觐,无以酬谢,就本身买
市十日。倘蒙不外,赍价前来,一一报答,并无虚谬。特此告知,远近居民,勿疑
辞避,惠然光临,不胜万幸。

宣和四年三月

日梁山泊义士宋江等谨请

萧让写毕告示,差人去附近州郡及四散村坊,尽行贴遍。发库内金珠、宝贝、彩缎、
绫罗、纱绢等项,分散各头领并军校人员。另选一分,为上国进奉,其余堆集山寨,
尽行招人买市十日。于三月初三日为始,至十三日止,宰下牛羊,酝造酒醴,但到
山寨里买市的人,尽以酒食管待,犒劳从人。至期,四方居民,担囊负笈,雾集云
屯,俱至山寨。宋江传令,以一举十,俱各欢喜,拜谢下山。一连十日,每日如此。
十日已外,住罢买市,号令大小,收拾赴京朝觐。宋江便要起送各家老小还乡。吴
用谏道:“兄长未可,且留众宝眷在此山寨。待我等朝觐面君之后,承恩已定,那
时发遣各家老小还乡未迟。”宋江听罢道:“军师之言极当。”再传将令,教头领
即便收拾,整顿军士。宋江等随即火速起身,早到济州,谢了太守张叔夜。太守即
设筵宴,管待众多义士,赏劳三军人马。宋江等辞了张太守,出城进发,带领众多
军马,径投东京来。先令戴宗、燕青前来京师宿太尉府中报知。太尉见说,随即便
入内里奏知天子:“宋江等众军马朝京。”天子闻奏大喜,便差太尉并御驾指挥使
一员,手持旌旄节钺,出城迎接。当下宿太尉领圣旨出郭。

且说宋江军马在路,甚是摆的整齐。前面打着两面红旗:一面上书“顺天”二字,
一面上书“护国”二字。众头领都是戎装披挂,惟有吴学究纶巾羽服,公孙胜鹤氅
道袍,鲁智深烈火僧衣,武行者香皂直裰;其余都是战袍金铠,本身服色。在路非
止一日,来到京师城外,前逢御驾指挥使持节迎着军马。宋江闻知,领众头领前来
参见宿太尉已毕,且把军马屯驻新曹门外,下了寨栅,听候圣旨。

且说宿太尉并御驾指挥使入城,回奏天子说:“宋江等军马,俱屯在新曹门外,听
候圣旨。”天子乃曰:“寡人久闻梁山泊宋江等有一百八人,上应天星,更兼英雄
勇猛。今已归降,到于京师。寡人来日引百官登宣德楼,可教宋江等俱依临敌披挂
戎装服色,休带大队人马,只将三五百马步军进城,自东过西,寡人亲要观看。也
教在城军民,知此英雄豪杰,为国良臣。然后却令卸其衣甲,除去军器,都穿所赐
锦袍,从东华门而入,就文德殿朝见。”御驾指挥使直至行营寨前,口传圣旨与宋
江等知道。次日,宋江传令,教铁面孔目裴宣选拣彪形大汉五七百步军,前面打着
金鼓旗幡,后面摆着枪刀斧钺,中间竖着“顺天”、“护国”二面红旗,军士各悬
刀剑弓矢,众人各各都穿本身披挂,戎装袍甲,摆成队伍,从东郭门而入。只见东
京百姓军民,扶老挈幼,迫路观看,如睹天神。是时天子引百官在宣德楼上,临轩
观看。见前面摆列金鼓旗幡,枪刀斧钺,各分队伍;中有踏白马军,打起“顺天”、
“护国”二面红旗,外有二三十骑马上随军鼓乐;后面众多好汉,簇簇而行。怎见
得英雄好汉,入城朝觐,但见:
风清玉陛,露挹金盘。东方旭日初升,北阙珠帘半卷。南熏门外,百八员义士归心;
宣德楼前,亿万岁君王刮目。肃威仪乍行朝典,逞精神犹整军容。风雨日星,并识
天颜之霁;电雷霹雳,不烦天讨之威。帝阙前万灵咸集:有圣、有仙、有那吒、有
金刚、有阎罗、有判官、有门神、有太岁,乃至夜叉鬼魔,共仰道君皇帝。凤楼下
百兽来朝:为彪、为豹、为麒麟、为狻猊、为犴、为金翅、为雕鹏、为龟猿,以
及犬鼠蛇蝎,皆知宋主人王。五龙夹日,是为入云龙、混江龙、出林龙、九纹龙、
独角龙,如出洞蛟、翻江蜃,自逐队朝天。众虎离山,是为插翅虎、跳涧虎、锦毛
虎、花项虎、青眼虎、笑面虎、矮脚虎、中箭虎,若病大虫、母大虫,亦随班行礼。
原称公侯伯子的,应谙朝仪;谁知尘舞山呼,亦许园丁、医算、匠作、船工之辈。
凡生毛发须髯的,自堪宠命;岂意绯袍紫绶,并加妇人、浪子、和尚、行者之身。
拟空名,则太保、军师、郡马、孔目、郎将、先锋,官衔早列;比古人,则霸王、
李广、关索、温侯、尉迟、仁贵,当代重生。有那生得好的,如白面郎插一枝花,
擎着笛扇鼓幡,欲歌且舞;看这生得丑的,似青面兽蒙鬼脸儿,拿着枪刀鞭箭,会
战能征。长的比险道神,身长一丈;狠的像石将军,力镇三山。发可赤,眼可青,
俱各抱丹心一片;摸得天,跳得浪,决不走邪佞两途。喜近君王,不似昔时无面目;
恩宽防御,果然此日没遮拦。试看全伙里舞枪弄棒的书生,犹胜满朝中欺君害民的
官吏。义
士今欣遇主,皇家始庆得人!

且说道君皇帝同百官在宣德楼上,看了梁山泊宋江等这一行部从,喜动龙颜,心中
大悦,与百官道:“此辈好汉,真英雄也!”叹羡不已。命殿头官传旨,教宋江等
各换御赐锦袍见帝。殿头官领命传与宋江等,向东华门外脱去戎装带,穿了御赐
红绿锦袍,悬带金银牌面,各带朝天巾帻,抹绿朝靴。惟公孙胜将红锦裁成道袍,
鲁智深缝做僧衣,武行者改作直裰,皆不忘君赐也。宋江、卢俊义为首,吴用、公
孙胜为次,引领众人从东华门而入。当日整肃朝仪,陈设銮驾,辰牌时候,天子驾
升文德殿。仪礼司官引宋江等依次入朝,排班行礼。殿头官赞拜舞起居,山呼万岁
已毕,天子欣喜,敕令宣上文德殿来,照依班次赐坐,命排御筵。敕光禄寺摆宴,
良酝署进酒,珍羞署进食,掌醢署造饭,大官署供膳,教坊司奏乐。天子亲御宝座
陪宴,只见:
九重门启,鸣哕哕之鸾声;阊阖天开,睹巍巍之龙衮。筵开玳瑁,七宝器黄金嵌就;
炉列麒麟,百和香龙脑修成。玻璃盏间琥珀钟,玛瑙杯联珊瑚。赤瑛盘内,高堆
麟脯鸾肝;紫玉碟中,满驼蹄熊掌。桃花汤洁,缕塞北之黄羊;银丝脍鲜,剖江
南之赤鲤。黄金盏满泛香醪,紫霞杯滟浮琼液。五俎八簋,百味庶羞。糖浇就甘甜
狮仙,面制成香酥定胜。方当酒进五巡,正是汤陈三献。教坊司凤鸾韶舞,礼乐司
排长伶官。朝鬼门道,分明开说,头一个装外的,黑漆幞头,有如明镜,描花罗,
俨若生成;第二个戏色的,系离水犀角腰带,裹红花绿叶罗巾,黄衣长衬短靴,
衫袖襟密排山水样;第三个末色的,裹结络球头帽子,着役迭胜罗衫,最先来提
掇甚分明,念几段杂文真罕有;第四个净色的,语言动众,颜色繁过,依院本填腔
调曲,按格范打诨发科;第五个贴净的,忙中九伯,眼目张
狂,队额角涂一道明戗,劈面门抹两色蛤粉。裹一顶油油腻腻旧头巾,穿一领邋邋
遢遢泼戏袄,吃六棒板不嫌疼,打两杖麻鞭浑似耍。这五人引领着六十四回队舞
优人,百二十名散做乐工,搬演杂剧,装孤打撺。个个青巾桶帽,人人红带花袍。
吹龙笛,击鼍鼓,声震云霄;弹锦瑟,抚银筝,韵惊鱼鸟。吊百戏众口喧哗,纵谐
语齐声喝采。装扮的是:太平年万国来朝,雍熙世八仙庆寿。搬演的是:玄宗梦游
广寒殿,狄青夜夺昆仑关。也有神仙道侣,亦有孝子顺孙。观之者,真可坚其心志;
听之者,足以养其性情。须臾间,八个排长,簇拥着四个美人,歌舞双行,吹弹并
举。歌的是:《朝天子》、《贺圣朝》、《感皇恩》、《殿前欢》,治世之音;舞
的是:《醉回回》、《活观音》、《柳青娘》、《鲍老儿》,淳正之态。果然道:
百宝装腰带,珍珠络臂鞲;笑时花近眼,舞罢锦缠头。大宴已成,众乐齐举。主上
无为千万寿,天颜有喜万方同。有诗为证:
九重凤阙新开宴,千岁龙墀旧赐衣。
盖世功名能自立,矢心忠义岂相违。

且说天子赐宋江等筵宴,至暮方散。谢恩已罢,宋江等俱各簪花出内,在西华门外
各各上马,回归本寨。次日入城,礼仪司引至文德殿谢恩,喜动龙颜,天子欲加官
爵,敕令宋江等来日受职。宋江等谢恩,出朝回寨,不在话下。又说枢密院官,具
本上奏:“新降之人,未效功劳,不可辄便加爵,可待日后征讨,建立功勋,量加
官赏。现今数万之众,逼城下寨,甚为不宜。陛下可将宋江等所部军马,原是京师
有被陷之将,仍还本处。外路军兵,各归原所。其余人众,分作五路,山东、河北
分调开去,此为上策。”次日,天子命御驾指挥使,直至宋江营中,口传圣旨,令
宋江等分开军马,各归原所。众头领听得,心中不悦,回道:“我等投降朝廷,都
不曾见些官爵,便要将俺弟兄等分遣调开。俺等众头领生死相随,誓不相舍,端的
要如此,我们只得再回梁山泊去。”宋江急忙止住,遂用忠言恳求来使,烦乞善言
回奏。那指挥使回到朝廷,那里敢隐蔽,只得把上项所言奏闻天子。天子大惊,急
宣枢密院官计议。有枢密使童贯奏道:“这厮们虽降,其心不改,终贻大患。以臣
愚意,不若陛下传旨,赚入京城,将此一百八人,尽数剿除,然后分散他的军马,
以绝国家之患。”天子听罢,圣意沉吟未决。向那御屏风背后,转出一大臣,紫袍
象简,高声喝道:“四边狼烟未息,中间又起祸胎,都是汝等庸恶之臣,坏了圣朝
天下!”正是:只凭立国安邦口,来救惊天动地人。
毕竟御屏风后喝的那员大臣是谁,且听下回分解。