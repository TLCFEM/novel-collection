\chapter{宋公明奉诏破大辽~陈桥驿滴泪斩小卒}

话说当年有辽国郎主,起兵前来侵占山后九州边界。兵分四路而入,劫掳山东、
山西,抢掠河南、河北。各处州县,申达表文,奏请朝廷求救,先经枢密院,然后
得到御前。所有枢密童贯,同太师蔡京、太尉高俅、杨商议,纳下表章不奏,只
是行移邻近州府,催趱各处径调军马,前去策应,正如担雪填井一般。此事人皆尽
知,只瞒着天子一个。适来四个贼臣设计,教枢密童贯启奏,将宋江等众,要行陷
害。不期那御屏风后,转出一员大臣来喝住,正是殿前都太尉宿元景,便向殿前启
奏道:“陛下,宋江这伙好汉方始归降,一百八人,恩同手足,意若同胞,他们决
不肯便拆散分开,虽死不舍相离。如何今又要害他众人性命?此辈好汉,智勇非同
小可。倘或城中翻变起来,将何解救?现今辽国兴兵十万之众,侵占山后九州所属
县治。各处申达表文求救,累次调兵前去征剿交锋,如汤泼蚁。贼势浩大,所遣官
军,又无良策,每每只是折兵损将,瞒着陛下不奏。以臣愚见,正好差宋江等全伙
良将,部领所属军将人马,直抵本境,收伏辽贼,令此辈好汉建功,进用于国,实
有便益。微臣不敢自专,乞请圣鉴。”天子听罢宿太尉所奏,龙颜大喜,询问众官,
俱言有理。天子大骂枢密院童贯等官:“都是汝等谗佞之徒、误国之辈,妒贤嫉能,
闭塞贤路,饰词矫情,坏尽朝廷大事!姑恕情罪,免其追问。”天子亲书诏敕,赐
宋江为破辽都先锋,卢俊义为副先锋,其余诸将,待建功之后,加官受爵。就差太
尉宿元景亲赍诏敕,去宋江军前行营开读。天子退朝,百官皆散。

且说宿太尉领了圣旨出朝,径到宋江行寨军前开读。宋江等忙排香案迎接,跪听诏
敕已罢,众皆大喜。宋江等拜谢宿太尉道:“某等众人,正欲如此,与国家出力,
建功立业,以为忠臣。今得太尉恩相,力赐保奏,恩同父母。只有梁山泊晁天王灵
位,未曾安厝;亦有各家老小家眷,未曾发送还乡;所有城垣,未曾拆毁,战船亦
未曾将来。有烦恩相题奏,乞降圣旨,宽限旬日,还山了此数事,整顿器具、枪刀、
甲马,便当尽忠报国。”宿太尉听罢大喜,回奏天子。即降圣旨,敕赐库内取金一
千两、银五千两、彩缎五千匹,颁赐众将,就令太尉于库藏开支,去行营俵散与众
将。原有老小者,赏赐给付与老小养赡终身;原无老小者,给付本人,自行收受。
宋江奉敕,谢恩已毕,给散众人收讫。宿太尉回朝,分付宋江道:“将军还山,可
速去快来,先使人报知下官,不可迟误!”

再说宋江聚众商议,所带还山人数是谁。宋江与同军师吴用、公孙胜、林冲、刘唐、
杜迁、宋万、朱贵、宋清、阮家三弟兄,马步水军一万余人回去。其余大队人马,
都随卢先锋在京师屯扎。宋江与吴用、公孙胜等于路无话,回到梁山泊忠义堂上坐
下,便传将令,教各家老小眷属,收拾行李,准备起程。一面叫宰杀猪羊牲口,香
烛钱马,祭献晁天王,然后焚化灵牌。随即将各家老小,各各送回原所州县,上车
乘马,俱已去了。然后教自家庄客,送老小、宋太公并家眷人口,再回郓城县宋家
村,复为良民。随即叫阮家三弟兄拣选合用船只,其余不堪用的小船,尽行给散与
附近居民收用。山中应有屋宇房舍,任从居民搬拆。三关城垣,忠义等屋,尽行拆
毁。一应事务,整理已了,收拾人马火速还京。

一路无话,早到东京。卢俊义等接至大寨。先使燕青入城,报知宿太尉,要辞天子,
引领大军起程。宿太尉见报,入内奏知天子。次日,引宋江于武英殿朝见天子,龙
颜欣悦,赐酒已罢,玉音道:“卿等休辞道途跋涉,军马驱驰,与寡人征虏破辽,
早奏凯歌而回,朕当重加录用。其众将校,量功加爵。卿勿怠焉!”宋江叩头称谢,
端简启奏:“臣乃鄙猥小吏,误犯刑典,流递江州。醉后狂言,临刑弃市,众力救
之,无处逃避,遂乃潜身水泊,苟延微命。所犯罪恶,万死难逃。今蒙圣上宽恤收
录,大敷旷荡之恩,得蒙赦免本罪。臣披肝沥胆,尚不能补报皇上之恩。今奉诏命,
敢不竭力尽忠,死而后已!”天子大喜,再赐御酒,教取描金鹊画弓箭一副,名马
一匹,全副鞍辔,宝刀一口,赐与宋江。宋江叩首谢恩,辞陛出内,将领天子御赐
宝刀、鞍马、弓箭,就带回营,传令诸军将校,准备起行。

且说徽宗天子次早令宿太尉传下圣旨,教中书省院官二员,就陈桥驿与宋江先锋犒
劳三军,每名军士酒一瓶、肉一斤,对众关支,毋得克减。中书省得了圣旨,一面
连更晓夜,整顿酒肉,差官二员,前去给散。

再说宋江传令诸军,便与军师吴用计议,将军马分作二起进程:令五虎八彪将引军
先行,十骠骑将在后,宋江、卢俊义、吴用、公孙胜统领中军。水军头领三阮、李
俊、张横、张顺,带领童威、童猛、孟康、王定六并水手头目人等,撑驾战船,自
蔡河内出黄河,投北进发。宋江催趱三军,取陈桥驿大路而进。号令军将,毋得动
扰乡民。有诗为证:
招摇旌旆出天京,受命专师事远征。
请看梁山军纪律,何如太尉御营兵。

且说中书省差到二员厢官,在陈桥驿给散酒肉,赏劳三军。谁想这伙官员,贪滥无
厌,徇私作弊,克减酒肉。都是那等谗佞之徒,贪爱贿赂的人,却将御赐的官酒,
每瓶克减只有半瓶;肉一斤,克减六两。前队军马,尽行给散过了;后军散到一队
皂军之中,都是头上黑盔,身披玄甲,却是项充、李衮所管的牌手。那军汉中一个
军校,接得酒肉过来看时,酒只半瓶,肉只十两,指着厢官骂道:“都是你这等好
利之徒,坏了朝廷恩赏!”厢官喝道:“我怎的是好利之徒?”那军校道:“皇帝
赐俺一瓶酒,一斤肉,你都克减了。不是我们争嘴,堪恨你这厮们无道理,佛面上
去刮金!”厢官骂道:“你这大胆,剐不尽、杀不绝的贼!梁山泊反性,尚不改!”
军校大怒,把这酒和肉劈脸都打将去。厢官喝道:“捉下这个泼贼!”那军校就团
牌边掣出刀来。厢官指着手大骂道:“腌草寇,拔刀敢杀谁?”军校道:“俺在
梁山泊时,强似你的好汉,被我杀了万千。量你这等贼官,直些甚鸟?”厢官喝道:
“你敢杀我?”那军校走入一步,手起一刀飞去,正中厢官脸上,剁着扑地倒了。
众人发声喊,都走了。那军汉又赶将入来,再剁了几刀,眼见的不能够活了。众军
汉簇住了不行。

当下项充、李衮飞报宋江。宋江听得大惊,便与吴用商议,此事如之奈何。吴学究
道:“省院官甚是不喜我等,今又做得这件事来,正中了他的机会。只可先把那军
校斩首号令,一面申复省院,勒兵听罪。急急可叫戴宗、燕青悄悄进城,备细告知
宿太尉。烦他预先奏知委曲,令中书省院谗害不得,方保无事。”宋江计议定了,
飞马亲到陈桥驿边。那军校立在死尸边不动。宋江自令人于馆驿内搬出酒肉,赏劳
三军,都教进前;却唤这军校直到馆驿中,问其情节。那军校答道:“他千梁山泊
反贼,万梁山泊反贼,骂俺们杀剐不尽,因此一时性起,杀了他,专待将军听罪。”
宋江道:“他是朝廷命官,我兀自惧他,你如何便把他来杀了?须是要连累我等众
人!俺如今方始奉诏去破大辽,未曾见尺寸之功,倒做了这等的勾当,如之奈何?”
那军校叩首伏死。宋江哭道:“我自从上梁山泊以来,大小兄弟,不曾坏了一个。
今日一身入官所管,寸步也由我不得。虽是你强气未灭,使不的旧时性格。”这军
校道:“小人只是伏死。”宋江令那军校痛饮一醉,教他树下缢死,却斩头来号令。
将厢官尸首,备棺椁盛贮,然后动文书申呈中书省院,不在话下。

再说戴宗、燕青潜地进城,径到宿太尉府内,备细诉知衷情。当晚宿太尉入内,将
上项事务,奏知天子。次日,皇上于文德殿设朝,当有中书省院官出班奏曰:“新
降将宋江部下兵卒,杀死省院差去监散酒肉命官一员,乞圣旨拿问。”天子曰:“寡
人待不委你省院来,事却该你这衙门!你们又委用不得其人,以致惹起事端。赏军
酒肉,大破小用,军士有名无实,以致如此。”省院等官又奏道:“御酒之物,谁
敢克减?”是时天威震怒,喝道:“寡人已自差人暗行体察,深知备细,尔等尚自
巧言令色,对朕支吾!寡人御赐之酒,一瓶克半瓶,赐肉一斤,只有十两,以致壮
士一怒,目前流血!”天子喝问:“正犯安在?”省院官奏道:“宋江已自将本犯
斩首号令示众,申呈本院,勒兵听罪。”天子曰:“他既斩了正犯军士,宋江禁治
不严之罪,权且纪录,待破辽回日,量功理会。”省院官默默无言而退。天子当时
传旨,差官前去,催督宋江起程,所杀军校,就于陈桥驿枭首示众。

却说宋江正在陈桥驿勒兵听罪,只见驾上差官来到,着宋江等进兵征辽,违犯军校,
枭首示众。宋江谢恩已毕,将军校首级,挂于陈桥驿号令,将尸埋了。宋江大哭一
场,垂泪上马,提兵望北而进。每日兵行六十里,扎营下寨,所过州县,秋毫无犯。
沿路无话。将次相近辽境,宋江便请军师吴用商议道:“即日辽兵四路侵犯,我等
分兵前去征讨的是?只打城池的是?”吴用道:“若是分兵前去,奈缘地广人稀,
首尾不能救应。不如只是打他几个城池,却再商量。若还攻击得紧,他自然收兵。”
宋江道:“军师此计甚高!”随即唤过段景住来,分付道:“你走北路甚熟,可引
领军马前进。近的是甚州县?”段景住禀道:“前面便是檀州,正是辽国紧要隘口。
有条水路,港汊最深,唤做潞水,团团绕着城池。这潞水直通渭河,须用战船征进。
宜先趱水军头领船只到了,然后水陆并进,船骑相连,可取檀州。”宋江听罢,便
使戴宗催促水军头领李俊等,晓夜趱船至潞水取齐。

却说宋江整点人马水军船只,约会日期,水陆并行,杀投檀州来。且说檀州城内守
把城池番官,却是辽国洞仙侍郎手下四员猛将,一个唤做阿里奇,一个唤做咬儿惟
康,一个唤做楚明玉,一个唤做曹明济。此四员战将,皆有万夫不当之勇。闻知宋
朝差宋江全伙到来,一面写表申奏郎主,一面关报邻近蓟州、霸州、涿州、雄州救
应,一面调兵出城迎敌。便差阿里奇、楚明玉两个引兵出战。

且说大刀关胜在于前部先锋,引军杀近檀州所属密云县来。县官闻的,飞报与两个
番将说道:“宋朝军马,大张旗号,乃是梁山泊新受招安宋江这伙。”阿里奇听了
笑道:“既是这伙草寇,何足道哉!”传令教番兵扎掂已了,来日出密云县与宋江
交锋。

次日,宋江听报辽兵已近,即时传令将士,交锋要看头势,休要失支脱节。众将得
令,披挂上马。宋江、卢俊义,俱各戎装擐带,亲在军前监战。远远望见辽兵盖地
而来,黑洞洞遮天蔽日,都是皂雕旗。两下齐把弓弩射住阵脚。只见对阵皂旗开处,
正中间捧出一员番将,骑着一匹达马,弯环踢跳。宋江看那番将时,怎生打扮,但
见:
戴一顶三叉紫金冠,冠口内拴两根雉尾。穿一领衬甲白罗袍,袍背上绣三个凤凰。
披一副连环镔铁铠,系一条嵌宝狮蛮带,著一对云根鹰爪靴,挂一条护项销金帕,
带一张鹊画铁胎弓,悬一壶雕翎子箭。手掿梨花点钢枪,坐骑银色拳花马。

那番官旗号上写的分明:“大辽上将阿里奇”。宋江看了,与诸将道:“此番将不
可轻敌!”言未绝,金枪手徐宁出战,横着钩镰枪,骤坐下马,直临阵前。番将阿
里奇见了,大骂道:“宋朝合败,命草寇为将,敢来侵犯大国,尚不知死!”徐宁
喝道:“辱国小将,敢出秽言!”两军呐喊。徐宁与阿里奇抢到垓心交战,两马相
逢,兵器并举。二将斗不过三十余合,徐宁敌不住番将,望本阵便走。花荣急取弓
箭在手。那番将正赶将来,张清又早按住鞍鞒,探手去锦袋内取个石子,看着番将
较亲,照面门上只一石子,正中阿里奇左眼,翻筋斗落于马下。这里花荣、林冲、
秦明、索超,四将齐出,先抢了那匹好马,活捉了阿里奇归阵。副将楚明玉见折了
阿里奇,急要向前去救时,被宋江大队军马,前后掩杀将来,就弃了密云县,大败
亏输,奔檀州来。宋江且不追赶,就在密云县屯扎下营。看番将阿里奇时,打破眉
梢,损其一目,负痛身死。宋江传令,教把番官尸骸烧化。功绩簿上,标写张清第
一功。就将阿里奇连环镔铁铠、出白梨花枪、嵌宝狮蛮带、银色拳花马,并靴、袍、
弓、箭,都赐了张清。是日且就密云县中,众皆作贺,设宴饮酒,不在话下。

次日,宋江升帐,传令起军,都离密云县,直抵檀州来。却说檀州洞仙侍郎听得报
来,折了一员正将,坚闭城门,不出迎敌。又听的报有水军战船,在于城下,遂乃
引众番将,上城观看。只见宋江阵中猛将,摇旗呐喊,耀武扬威,掿战厮杀。洞仙
侍郎见了说道:“似此,怎不输了小将军阿里奇?”当下副将楚明玉答应道:“小
将军那里是输与那厮?蛮兵先输了,俺小将军赶将过去,被那里一个穿绿的蛮子,
一石子打下马去。那厮队里四个蛮子,四条枪,便来攒住了。俺这壁厢措手不及,
以此输与他了。”洞仙侍郎道:“那个打石子的蛮子,怎地模样?”左右有认得的,
指着说道:“城下兀那个带青包巾,现今披着小将军的衣甲,骑着小将军的马,那
个便是。”洞仙侍郎攀着女墙边看时,只见张清已自先见了,趱马向前,只一石子
飞来。左右齐叫一声躲时,那石子早从洞仙侍郎耳根边擦过,把耳轮擦了一片皮。
洞仙侍郎负疼道:“这个蛮子,直这般利害!”下城来,一面写表申奏大辽郎主,
一面行报外境各州提备。

却说宋江引兵在城下,一连打了三五日,不能取胜,再引军马回密云县屯驻,帐中
坐下,计议破城之策。只见戴宗报来,取到水军头领,乘驾战船,都到潞水。宋江
便教李俊等到军中商议。李俊等都到帐前参见宋江。宋江道:“今次厮杀,不比在
梁山泊时,可要先探水势深浅,方可进兵。我看这条潞水,水势甚急,倘或一失,
难以救应。尔等宜仔细,不可托大!将船只盖伏的好着,只扮作运粮船相似。你等
头领,各带暗器,潜伏于船内。止着三五人撑驾摇橹,岸上着两人牵拽,一步步挨
到城下,把船泊在两岸,待我这里进兵。城中知道,必开水门来抢粮船。尔等伏兵
却起,夺他水门,可成大功。”李俊等听令去了。只见探水小校报道:“西北上有
一彪军马卷杀而来,都打着皂雕旗,约有一万余人,望檀州来了。”吴用道:“必
是辽国调来救兵。我这里先差几将拦截厮杀,杀的散时,免令城中得他壮胆。”宋
江便差张清、董平、关胜、林冲,各带十数个小头领,五千军马,飞奔前来。

原来辽国郎主闻知,说是梁山泊宋江这伙好汉领兵杀至檀州,围了城子,特差这两
个皇侄前来救应:一个唤做耶律国珍,一个唤做国宝。两个乃是辽国上将,又是皇
侄,皆有万夫不当之勇。引起一万番兵,来救檀州。看看至近,迎着宋兵。两边摆
开阵势,两员番将,一齐出马。但见:
头戴妆金嵌宝三叉紫金冠,身披锦边珠嵌锁子黄金铠。身上猩猩血染战红袍,袍上
斑斑锦织金翅雕。腰系白玉带,背插虎头牌。左边袋内插雕弓,右手壶中攒硬箭。
手中掿丈二绿沉枪,坐下骑九尺银鬃马。

那番将是弟兄两个,都一般打扮,都一般使枪。宋兵迎着,摆开阵势。双枪将董平
出马,厉声高叫:“来者甚处番贼?”那耶律国珍大怒,喝道:“水洼草寇,敢来
犯吾大国,倒问俺那里来的?”董平也不再问,跃马挺枪,直抢耶律国珍。那番家
年少的将军,性气正刚,那里肯饶人一步,挺起钢枪,直迎过来。二马相交,三枪
乱举。二将正在征尘影里,杀气丛中,使双枪的,另有枪法;使单枪的,各用神机。
两个斗过五十合,不分胜败。那耶律国宝见哥哥战了许多时,恐怕力怯,就中军筛
起锣来。耶律国珍正斗到热处,听的鸣锣,急要脱身,被董平两条枪绞住,那里肯
放。耶律国珍此时心忙,枪法慢了些,被董平右手逼过绿沉枪,使起左手枪来,望
番将项根上只一枪,搠个正着。可怜耶律国珍,金冠倒卓,两脚登空,落于马下。
兄弟耶律国宝看见哥哥落马,便抢出阵来,一骑马,一条枪,奔来救取。宋兵阵上
没羽箭张清,见他过来,这里那得放空,在马上约住梨花枪,探只手去锦袋内拈出
一个石子,把马一拍,飞出阵前。这耶律国宝飞也似来,张清迎头扑将去。两骑马
隔不的十来丈远近,番将不提防,只道他来交战,只见张清手起,喝声道:“着!”
那石子望耶律国宝面上打个正着,翻筋斗落马。关胜、林冲拥兵掩杀。辽兵无主,
东西乱窜。只一阵,杀散辽兵万余人马,把两个番官,全副鞍马,两面金牌,收拾
宝冠袍甲,仍割下两颗首级,当时夺了战马一千余匹,解到密云县来见宋江献纳。
宋江大喜,赏劳三军,书写董平、张清第二功,等打破檀州,一并申奏。

宋江与吴用商议,到晚写下军帖,差调林冲、关胜引领一彪军马,从西北上去取檀
州。再调呼延灼、董平,也引一彪军马,从东北上进兵。却教卢俊义引一彪军马,
从西南上取路。“我等中军,从东南路上去,只听的炮响,一齐进发。”却差炮手
凌振及李逵、樊瑞、鲍旭,并牌手项充、李衮,将带滚牌军一千余人,直去城下,
施放号炮。至二更为期,水陆并进。各路军兵,都要厮应。号令已了,诸军各各准
备取城。

且说洞仙侍郎正在檀州坚守,专望救兵到来。却有皇侄败残人马,逃命奔入城中,
备细告说:两个皇侄大王,耶律国珍被个使双枪的害了,耶律国宝被个戴青包巾的,
使石子打下马来拿去。洞仙侍郎跌脚骂道:“又是这蛮子!不争损了二位皇侄,教
俺有甚面目去见郎主?拿住那个青包巾的蛮子时,碎碎的割那厮!”至晚,番兵报
洞仙侍郎道:“潞水河内,有五七百只粮船,泊在两岸,远远处又有军马来也!”
洞仙侍郎听了道:“那蛮子不识俺的水路,错把粮船直行到这里。岸上人马,一定
是来寻粮船。”便差三员番将楚明玉、曹明济、咬儿惟康前来分付道:“那宋江等
蛮子,今晚又调许多人马来,却有若干粮船在俺河里。可教咬儿惟康引一千军马出
城冲突,却教楚明玉、曹明济开放水门,从紧溜里放船出去。三停之内,截他二停
粮船,便是汝等干大功也!”不知成败何如,有诗为证:
妙算从来迥不同,檀州城下列艨艟。
侍郎不识兵家意,反自开门把路通。

再说宋江人马,当晚黄昏左侧,李逵、樊瑞为首,将引步军在城下大骂。洞仙侍郎
叫咬儿惟康催趱军马,出城冲杀。城门开处,放下吊桥,辽兵出城。却说李逵、樊
瑞、鲍旭、项充、李衮五个好汉引一千步军,尽是悍勇刀牌手,就吊桥边冲住,番
军人马,那里能够出的城来。凌振却在军中,搭起炮架,准备放炮,只等时候来到。
由他城上放箭,自有牌手左右遮抵着。鲍旭却在后面呐喊。虽是一千余人,却是万
余人的气象。洞仙侍郎在城中见军马冲突不出,急叫楚明玉、曹明济开了水门抢船。
此时宋江水军头领都已先自伏在船中准备,未曾动弹。见他水门开了,一片片绞起
闸板,放出战船来。凌振得了消息,便先点起一个风火炮来。炮声响处,两边战船,
厮迎将来,抵敌番船。左边踊出李俊、张横、张顺,右边踊出阮家三弟兄,都使着
战船,杀入番船队里。番将楚明玉、曹明济见战船踊跃而来,抵敌不住,料道有埋
伏军兵,急待要回船,早被这里水手军兵,都跳过船来,只得上岸而走。宋江水军
那六个头领,先抢了水门。管门番将,杀的杀了,走的走了。这楚明玉、曹明济各
自逃命去了。水门上预先一把火起,凌振又放一个车箱炮来。那炮直飞在半天里响。
洞仙侍郎听的火炮连天声响,吓的魂不附体。李逵、樊瑞、鲍旭引领牌手项充、李
衮等众,直杀入城。洞仙侍郎和咬儿惟康在城中,看见城门已都被夺了,又见四路
宋兵一齐都杀到来,只得上马,弃了城池,出北门便走。未及二里,正撞着大刀关
胜、豹子头林冲,拦住去路。正是:天罗密布难移步,地网高张怎脱身。
毕竟洞仙侍郎怎的逃生,且听下回分解。