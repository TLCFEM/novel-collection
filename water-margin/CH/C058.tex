\chapter{三山聚义打青州~众虎同心归水泊}

当有武松引孔亮拜告鲁智深、杨志,求救哥哥孔明,并叔叔孔宾。鲁智深便要
聚集三山人马,前去攻打。杨志道:“若要打青州,须用大队军马,方可打得。俺
知梁山泊宋公明大名,江湖上都唤他做及时雨宋江,更兼呼延灼是他那里仇人。俺
们弟兄和孔家弟兄的人马,都并做一处,洒家这里,再等桃花山人马齐备,一面且
去攻打青州。孔亮兄弟,你可亲身星夜去梁山泊,请下宋公明来,并力攻城,此为
上计。亦且宋三郎与你至厚,你们弟兄心下如何?”鲁智深道:“正是如此。我只
见今日也有人说宋三郎好,明日也有人说宋三郎好,可惜洒家不曾相会。众人说他
的名字,聒得洒家耳朵也聋了,想必其人是个真男子,以致天下闻名。前番和花知
寨在清风山时,洒家有心要去和他厮会,及至洒家去时,又听得说道去了,以此无
缘不得相见。罢了!孔亮兄弟,你要救你哥哥时,快亲自去那里告请他们,洒家等
先在这里和那撮鸟们厮杀。”孔亮交付小喽罗与了鲁智深,只带一个伴当,扮做客
商,星夜投梁山泊来。

且说鲁智深、杨志、武松三人,去山寨里唤将施恩、曹正,再带一二百人下山
来相助。桃花山李忠、周通得了消息,便带本山人马,尽数起点,只留三五十个小
喽罗看守寨栅;其余都带下山来,青州城下聚集,一同攻打城池,不在话下。

却说孔亮自离了青州,迤来到梁山泊边催命判官李立酒店里买酒吃,问路。
李立见他两个来得面生,便请坐地,问道:“客人从那里来?”孔亮道:“从青州
来。”李立问道:“客人要去梁山泊寻谁?”孔亮答道:“有个相识在山上,特来
寻他。”李立道:“山上寨中,都是大王住处,你如何去得?”孔亮道:“便是要
寻宋大王。”李立道:“既是来寻宋头领,我这里有分例。”便叫火家快去安排分
例酒来相待。孔亮道:“素不相识,如何见款?”李立道:“客官不知,但是来寻
山寨头领,必然是社火中人故旧交友,岂敢有失祗应!便当去报。”孔亮道:“小
人便是白虎山前庄户孔亮的便是。”李立道:“曾听得宋公明哥哥说大名来,今日
且喜上山。”二人饮罢分例酒,随即开窗,就水亭上放了一枝响箭。见对港芦苇深
处,早有小喽罗棹过船来。到水亭下,李立便请孔亮下了船,一同摇到金沙滩上岸,
却上关来。孔亮看见三关雄壮,枪刀剑戟如林,心下想道:“听得说梁山泊兴旺,
不想做下这等大事业!”已有小喽罗先去报知,宋江慌忙下来迎接。

孔亮见了,连忙下拜。宋江问道:“贤弟缘何到此?”孔亮拜罢,放声大哭。
宋江道:“贤弟心中有何危厄不决之难,但请尽说不妨。便当不避水火,力为救解,
与汝相助,贤弟且请起来。”孔亮道:“自从师父离别之后,老父亡化,哥哥孔明
与本乡上户争些闲气起来,杀了他一家老小,官司来捕捉得紧。因此反上白虎山,
聚得五七百人,打家劫舍。青州城里,却有叔父孔宾,被慕容知府捉了,重枷钉在
狱中。因此我弟兄两个去打城子,指望救取叔叔孔宾。谁想去到城下,正撞了一个
使双鞭的呼延灼。哥哥与他交锋,致被他捉了,解送青州,下在牢里,存亡未保,
小弟又被他追杀一阵。次日,正撞着武松,说起师父大名来,他便引我去拜见同伴
的:一个是花和尚鲁智深,一个是青面兽杨志。他二人一见如故,便商议救兄一事。
他道:‘我请鲁、杨二头领并桃花山李忠、周通,聚集三山人马,攻打青州;你可
连夜快去梁山泊内,告你师父宋公明,来救你叔兄两个。’以此今日一径到此。”
宋江道:“此是易为之事,你且放心。先来拜见晁头领,共同商议。”

宋江便引孔亮参见晁盖、吴用、公孙胜,并众头领,备说呼延灼走在青州,投
奔慕容知府,今来捉了孔明,以此孔亮来到,恳告求救。晁盖道:“既然他两处好
汉,尚兀自仗义行仁,今者三郎和他至爱交友,如何不去?三郎贤弟,你连次下山
多遍,今番权且守寨,愚兄替你走一遭。”宋江道:“哥哥是山寨之主,不可轻动。
这个是兄弟的事。既是他远来相投,小可若自不去,恐他弟兄们心下不安;小可情
愿请几位弟兄同走一遭。”说言未了,厅上厅下一齐都道:“愿效犬马之劳,跟随
同去。”宋江大喜。当日设筵管待孔亮。饮筵之间,宋江唤铁面孔目裴宣定拨下山
人数,分作五军起行:前军便差花荣、秦明、燕顺、王矮虎,开路作先锋;第二队,
便差穆弘、杨雄、解珍、解宝;中军便是主将宋江、吴用、吕方、郭盛;第四队便
是朱仝、柴进、李俊、张横;后军便差孙立、杨林、欧鹏、凌振,摧军作合后。梁
山泊点起五军,共计二十个头领,马步军兵二千人马。其余头领,自与晁盖守把寨
栅。当下宋江别了晁盖,自同孔亮下山来。梁山人马分作五军起发,正是:

初离水泊,浑如海内纵蛟龙;乍出梁山,却似风中奔虎豹。五军并进,前后列
二十辈英雄;一阵同行,首尾分三千名士卒。绣彩旗如云似雾,蘸钢刀灿雪铺霜。
鸾铃响,战马奔驰;画鼓振,征夫踊跃。卷地黄尘霭霭,漫天土雨蒙蒙。宝纛旗中,
簇拥着多智足谋吴学究;碧油幢下,端坐定替天行道宋公明。过去鬼神皆拱手,回
来民庶尽歌谣。

话说宋江引了梁山泊二十个头领,三千人马,分作五军前进,于路无事,所过
州县,秋毫无犯。已到青州,孔亮先到鲁智深等军中,报知众好汉,安排迎接。宋
江中军到了,武松引鲁智深、杨志、李忠、周通、施恩、曹正,都来相见了。宋江
让鲁智深坐地,鲁智深道:“久闻阿哥大名,无缘不曾拜会,今日且喜认得阿哥。”
宋江答道:“不才何足道哉!江湖上义士,甚称吾师清德。今日得识慈颜,平生甚
幸。”杨志也起身再拜道:“杨志旧日经过梁山泊,多蒙山寨重义相留;为是洒家
愚迷,不曾肯住。今日幸得义士壮观山寨,此是天下第一好事。”宋江答道:“制
使威名,播于江湖,只恨宋江相会太晚。”鲁智深便令左右置酒管待,一一都相见
了。

次日,宋江问:“青州一节,近日胜败如何?”杨志道:“自从孔亮去了,前
后也交锋三五次,各无输赢。如今青州只凭呼延灼一个;若是拿得此人,觑此城子,
如汤泼雪。”吴学究笑道:“此人不可力敌,可用智擒。”宋江道:“用何智可获
此人?”吴学究道:“只除如此如此。”宋江大喜道:“此计大妙!”当日分拨了
人马。次早起军,前到青州城下,四面尽着军马围住,擂鼓摇旗,呐喊搦战。城里
慕容知府见报,慌忙教请呼延灼商议:“今次群贼又去报知梁山泊宋江到来,似此
如之奈何?”呼延灼道:“恩相放心。群贼到来,先失地利。这厮们只好在水泊里
张狂,今却擅离巢穴,一个来,捉一个,那厮们如何施展得?请恩相上城,看呼延
灼厮杀。”呼延灼连忙披挂衣甲上马,叫开城门,放下吊桥,领了一千人马,近城
摆开。宋江阵中,一将出马。那人手狼牙棍,厉声高骂知府:“滥官,害民贼徒!
把我全家诛戮,今日正好报仇雪恨!”慕容知府认得秦明,便骂道:“你这厮是朝
廷命官,国家不曾负你,缘何敢造反,若拿住你时,碎尸万段!可先下手拿这贼!”
呼延灼听了,舞起双鞭,纵马直取秦明。秦明也出马,舞动狼牙大棍,来迎呼延灼。
二将交马,正是对手。有《西江月》为证:

鞭舞两条龙尾,棍横一串狼牙。三军看得眼睛花,二将纵横交马。

使棍的
军班领袖,使鞭的将种堪夸。天昏地惨日扬沙,这厮杀鬼神须怕。
两个斗到四五十合,不分胜败。慕容知府见斗得多时,恐怕呼延灼有失,慌忙鸣金
收军入城。秦明也不追赶,退回本阵。宋江教众头领军校,且退十五里下寨。

却说呼延灼回到城中,下马来见慕容知府,说道:“小将正要拿那秦明,恩相
如何收军?”知府道:“我见你斗了许多合,但恐劳困,因此收军暂歇。秦明那厮,
原是我这里统制,与花荣一同背反,这厮亦不可轻敌。”呼延灼道:“恩相放心,
小将必要擒此背义之贼!适间和他斗时,棍法已自乱了。来日教恩相看我立斩此贼!”
知府道:“既是将军如此英雄,来日若临敌之时,可杀开条路,送三个人出去:一
个教他去往东京求救,两个教他去邻近府州,会合起兵,相助剿捕。”呼延灼道:
“恩相高见极明。”当日知府写了求救文书,选了三个军官,都发放了当。

只说呼延灼回到歇处,卸了衣甲暂歇。天色未明,只听的军校来报道:“城北
门外土坡上,有三骑私自在那里看城:中间一个穿红袍骑白马的;两边两个,只认
得右边的是小李广花荣,左边那个道妆打扮。”呼延灼道:“那个穿红的,眼见是
宋江了;道妆的,必是军师吴用。你们且休惊动了他,便点一百马军,跟我捉这三
个。”

呼延灼连忙披挂上马,提了双鞭,带领一百余骑马军,悄悄地开了北门,放下
吊桥,引军赶上坡来。宋江、吴用、花荣三个,只顾呆了脸看城。呼延灼拍马上坡,
三个勒转马头,慢慢走去。呼延灼奋力赶到前面几株枯树边厢,宋江、吴用、花荣
三个齐齐的勒住马。呼延灼方才赶到枯树边,只听得呐声喊,呼延灼正踏着陷坑,
人马都跌将下坑去了。两边走出五六十个挠钩手,先把呼延灼钩将起来,绑缚了拿
去,后面牵着那匹马。这许多赶来的马军,却被花荣拈弓搭箭,射倒当头五七个,
后面的勒转马,一哄都走了。

宋江回到寨里坐,左右群刀手,却把呼延灼推将过来。宋江见了,连忙起身,
喝叫:“快解了绳索!”亲自扶呼延灼上帐坐定,宋江拜见。呼延灼道:“何故如
此?”宋江道:“小可宋江怎敢背负朝廷?盖为官吏污滥,威逼得紧,误犯大罪,
因此权借水泊里随时避难,只待朝廷赦罪招安。不想起动将军,致劳神力。实慕将
军虎威。今者误有冒犯,切乞恕罪!”呼延灼道:“被擒之人,万死尚轻,义士何
故重礼陪话?”宋江道:“量宋江怎敢坏得将军性命?皇天可表寸心。”只是恳告
哀求。呼延灼道:“兄长尊意,莫非教呼延灼往东京告请招安,到山赦罪?”宋江
道:“将军如何去得?高太尉那厮,是个心地匾窄之徒,忘人大恩,记人小过。将
军折了许多军马钱粮,他如何不见你罪责?如今韩滔、彭、凌振,已多在敝山入
伙。倘蒙将军不弃山寨微贱,宋江情愿让位与将军;等朝廷见用,受了招安,那时
尽忠报国,未为晚矣。”

呼延灼沉思了半晌,一者是天罡之数,自然义气相投;二者见宋江礼貌甚恭,
语言有理,叹了一口气,跪下在地道:“非是呼延灼不忠于国,实感兄长义气过人,
不容呼延灼不依,愿随鞭镫。事既如此,决无还理。”有诗为证:
亲承天语净狼烟,不着先鞭愿执鞭。
岂昧忠心翻作贼,降魔殿内有因缘。
宋江大喜,请呼延灼和众头领相见了,叫问李忠、周通,讨这匹踢雪乌骓马,送将
军骑坐。众人再商议救孔明之计,吴用道:“只除教呼延灼将军赚开城门,垂手可
得!更兼绝了呼延灼将军念头。”宋江听了,来与呼延灼陪话道:“非是宋江贪劫
城池,实因孔明叔侄,陷在缧绁之中,非将军赚开城门,必不可得。”呼延灼答道:
“小将既蒙兄长收录,理当效力。”当晚点起秦明、花荣、孙立、燕顺、吕方、郭
盛、解珍、解宝、欧鹏、王英,十个头领,都扮作军士衣服模样,跟了呼延灼,共
是十一骑军马,来到城边,直至濠堑上,大呼:“城上开门,我逃得性命回来!”

城上人听得是呼延灼声音,慌忙报与慕容知府。此时知府为折了呼延灼,正纳
闷间,听得报说呼延灼逃得回来,心中欢喜,连忙上马,奔到城上。望见呼延灼有
十数骑马跟着,又不见面颜,只认得呼延灼声音,知府问道:“将军如何走得回来?”
呼延灼道:“我被那厮的陷坑捉了我到寨里,却有原跟我的头目,暗地盗这匹马与
我骑,就跟我来了。”知府只听得呼延灼说了,便叫军士开了城门,放下吊桥。十
个头领跟到城门里,迎着知府,早被秦明一棍,把慕容知府打下马来。解珍、解宝
便放起火来;欧鹏、王矮虎奔上城,把军士杀散。宋江大队人马,见城上火起,一
齐拥将入来。宋江急急传令:休教残害百姓,且收仓库钱粮。就大牢里救出孔明,
并他叔叔孔宾一家老小,便教救灭了火。把慕容知府一家老幼,尽皆斩首,抄扎家
私,分众军。天明,计点在城百姓被火烧之家,给散粮米救济。把府库金帛,仓
廒米粮,装载五六百车;又得了二百余匹好马,就青州府里做个庆喜筵席,请三山
头领同归大寨。李忠、周通使人回桃花山,尽数收拾人马钱粮下山,放火烧毁寨栅。
鲁智深也使施恩、曹正回二龙山,与张青、孙二娘收拾人马钱粮,也烧了宝珠寺寨
栅。数日之间,三山人马都皆完备。

宋江领了大队人马,班师回山。先叫花荣、秦明、呼延灼、朱仝四将开路,所
过州县,分毫不扰。乡村百姓,扶老挈幼,烧香罗拜迎接。数日之间,已到梁山泊
边。众多水军头领,具舟迎接。晁盖引领山寨马步头领,都在金沙滩迎接。直至大
寨,向聚义厅上列位坐定。大排筵庆贺新到山寨头领,呼延灼、鲁智深、杨志、武
松、施恩、曹正、张青、孙二娘、李忠、周通、孔明、孔亮:共十二位新上山头领。
坐间,林冲说起相谢鲁智深相救一事。鲁智深动问道:“洒家自与教头沧州别后,
曾知阿嫂信息否?”林冲答道:“小可自火并王伦之后,使人回家搬取老小,已知
拙妇被高太尉逆子所逼,随即自缢而死,妻父亦为忧疑,染病而亡。”杨志举起旧
日王伦手内上山相会之事,众人皆道:“此皆注定,非偶然也!”晁盖说起黄泥冈
劫取生辰纲一事,众皆大笑。次日轮流做筵席,不在话下。

且说宋江见山寨又添了许多人马,如何不喜,便叫汤隆做铁匠总管,提督打造
诸般军器,并铁叶连环等甲;侯健管做旌旗袍服总管,添造三才、九曜、四斗、五
方、二十八宿等旗,飞龙、飞虎、飞熊、飞豹旗,黄钺白旄,朱缨皂盖。山边四面
筑起墩台。重造西路南路二处酒店,招接往来上山好汉,一就探听飞报军情。山西
路酒店,今令张青、孙二娘夫妻二人,原是酒家,前去看守;山南路酒店,仍令孙
新、顾大嫂夫妻看守;山东路酒店,依旧朱贵、乐和;山北路酒店,还是李立、时
迁。三关上添造寨栅,分调头领看守。部领已定,各各遵依,不在话下。

忽一日,花和尚鲁智深来对宋公明说道:“智深有个相识,李忠兄弟也曾认的,
唤做九纹龙史进,现在华州华阴县少华山上,和那一个神机军师朱武,又有一个跳
涧虎陈达,一个白花蛇杨春,四个在那里聚义。洒家常常思念他。昔日在瓦罐寺救
助洒家,思念不曾有忘。洒家要去那里探望他一遭,就取他四个同来入伙,未知尊
意如何?”宋江道:“我也曾闻得史进大名,若得吾师去请他来,最好。虽然如此,
不可独自去,可烦武松兄弟相伴走一遭。他是行者,一般出家人,正好同行。”武
松应道:“我和师父去。”当日便收拾腰包行李,鲁智深只做禅和子打扮,武松妆
做随侍行者。两个相辞了众头领下山,过了金沙滩,晓行夜住,不止一日,来到华
州华阴县界,径投少华山来。

且说宋江自鲁智深、武松去后,一时容他下山,常自放心不下,便唤神行太保
戴宗随后跟来,探听消息。

再说鲁智深、武松两个,来到少华山下,伏路小喽罗出来拦住问道:“你两个
出家人那里来?”武松便答道:“这山上有史大官人么?”小喽罗说道:“既是要
寻史大王的,且在这里少等。我上山报知头领,便下来迎接。”武松道:“你只说
鲁智深到来相探。”小喽罗去不多时,只见神机军师朱武,并跳涧虎陈达、白花蛇
杨春,三个下山来接鲁智深、武松,却不见有史进。鲁智深便问道:“史大官人在
那里?却如何不见他?”朱武近前上复道:“吾师不是延安府鲁提辖么?”鲁智深
道:“洒家便是。这行者便是景阳冈打虎都头武松。”三个慌忙剪拂道:“闻名久
矣!听知二位在二龙山扎寨,今日缘何到此?”鲁智深道:“俺们如今不在二龙山
了,投托梁山泊宋公明大寨入伙,今者特来寻史大官人。”朱武道:“既是二位到
此,且请到山寨中,容小可备细告诉。”鲁智深道:“有话便说,待一待,谁鸟耐
烦?”武松道:“师父是个性急的人,有话便说何妨。”

朱武道:“小人等三个在此山寨,自从史大官人上山之后,好生兴旺。近日史
大官人下山,因撞见一个画匠,原是北京大名府人氏,姓王,名义。因许下西岳华
山金天圣帝庙内妆画影壁,前去还愿。因为带将一个女儿,名唤玉娇枝同行,却被
本州贺太守——原是蔡太师门人,那厮为官贪滥,非理害民。一日,因来庙里行香,
不想正见了玉娇枝有些颜色,累次着人来说,要娶他为妾。王义不从,太守将他女
儿强夺了去为妾,又把王义刺配远恶军州。路经这里过,正撞见史大官人,告说这
件事。史大官人把王义救在山上,将两个防送公人杀了,直去府里要刺贺太守;被
人知觉,倒吃拿了,现监在牢里。又要聚起军马,扫荡山寨,我等正在这里无计可
施!”

鲁智深听了道:“这撮鸟敢如此无礼!倒恁么利害!洒家与你结果了那厮!”朱
武道:“且请二位到寨里商议。”一行五个头领,都到少华山寨中坐下,便叫王义
见鲁智深、武松,诉说贺太守贪酷害民,强占良家女子。朱武等一面杀牛宰马,管
待鲁智深、武松。饮筵间,鲁智深想道:“贺太守那厮好没道理,我明日与你去州
里打死那厮罢!”武松道:“哥哥不得造次。我和你星夜回梁山泊去报知,请宋公
明领大队人马来打华州,方可救得史大官人。”鲁智深叫道:“等俺们去山寨里叫
得人来,史家兄弟性命不知那里去了。”武松道:“便杀了太守,也怎地救得史大
官人?武松却决不肯放鲁智深去!”朱武又劝道:“吾师且息怒。武都头也论得是。”
鲁智深焦躁起来,便道:“都是你这般慢性的人,以此送了俺史家兄弟;你也休去
梁山泊报知,看洒家去如何!”众人那里劝得住,当晚又谏不从。明早起个四更,
提了禅杖,带了戒刀,径奔华州去了。武松道:“不听人说,此去必然有失。”朱
武随即差两个精细的小喽罗,前去打听消息。

却说鲁智深奔到华州城里,路旁借问州衙在那里,人指道:“只过州桥,投东
便是。”鲁智深却好来到浮桥上,只见人都道:“和尚且躲一躲,太守相公过来。”
鲁智深道:“俺正要寻他,却正好撞在洒家手里!那厮多敢是当死!”贺太守头踏
一对对摆将过来,看见太守那乘轿子,却是暖轿;轿窗两边,各有十个虞候簇拥着,
人人手执鞭枪铁链,守护两下。鲁智深看了寻思道:“不好打那撮鸟,若打不着,
倒吃他笑。”贺太守却在轿窗眼里,看见了鲁智深欲进不进。过了渭桥,到府中下
了轿,便叫两个虞候分付道:“你与我去请桥上那个胖大和尚到府里赴斋。”虞候
领了言语,来到桥上,对鲁智深说道:“太守相公请你赴斋。”鲁智深想道:“这
厮合当死在洒家手里。俺却才正要打他,只怕打不着,让他过去了。俺要寻他,他
却来请洒家。”鲁智深便随了虞候,径到府里。太守已自分付下了,一见鲁智深进
到厅前,太守叫放了禅杖,去了戒刀,请后堂赴斋。鲁智深初时不肯,众人说道:
“你是出家人,好不晓事,府堂深处,如何许你带刀杖入去?”鲁智深想:“只俺
两个拳头,也打碎了那厮脑袋!”廊下放了禅杖、戒刀,跟虞候入来。贺太守正在
后堂坐定。把手一招,喝声:“捉下这秃贼!”两边壁衣内,走出三四十个做公的
来,横拖倒拽,捉了鲁智深。你便是那吒太子,怎逃地网天罗?火首金刚,难脱龙
潭虎窟!正是:飞蛾投火身倾丧,怒鳖吞钩命必伤。

毕竟鲁智深被贺太守拿下,性命如何,且听下回分解。