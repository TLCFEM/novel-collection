\chapter{宋江智取无为军~张顺活捉黄文炳}

话说江州城外白龙庙中,梁山泊好汉劫了法场,救得宋江、戴宗。正是晁盖、
花荣、黄信、吕方、郭盛、刘唐、燕顺、杜迁、宋万、朱贵、王矮虎、郑天寿、石
勇、阮小二、阮小五、阮小七、白胜,共是一十七人,领带着八九十个悍勇壮健小
喽罗。浔阳江上来接应的好汉张顺、张横、李俊、李立、穆弘、穆春、童威、童猛、
薛永九筹好汉,也带四十余人,都是江面上做私商的火家,撑驾三只大船,前来接
应。城里黑旋风李逵引众人杀至浔阳江边。两路救应,通共有一百四五十人,都在
白龙庙里聚义。只听得小喽罗报道:“江州城里军兵擂鼓,摇旗鸣锣,发喊追赶到
来。”

那黑旋风李逵听得,大吼了一声,提两把板斧,先出庙门,众好汉呐声喊,都
挺手中军器,齐出庙来迎敌。刘唐、朱贵先把宋江、戴宗护送上船,李俊同张顺、
三阮整顿船只。就江边看时,见城里出来的官军,约有五七千马军,当先都是顶盔
衣甲,全副弓箭,手里都使长枪,背后步军簇拥,摇旗呐喊,杀奔前来。这里李逵
当先,抡着板斧,赤条条地飞奔砍将入去,背后便是花荣、黄信、吕方、郭盛四将
拥护。花荣见前面的军马都扎住了枪,只怕李逵着伤,偷手取弓箭出来,搭上箭,
拽满弓,望着为头领的一个马军,飕地一箭,只见翻筋斗射下马去。那一伙马军,
吃了一惊,各自奔命,拨转马头便走,倒把步军先冲倒了一半。这里众多好汉们一
齐冲突将去,杀得那官军尸横野烂,血染江红,直杀到江州城下,城上策应官军早
把擂木炮石打将下来。官军慌忙入城,关上城门,好几日不敢出来。众多好汉拖转
黑旋风,回到白龙庙前下船。晁盖整点众人完备,都叫分头下船,开江便走。

却值顺风,拽起风帆,三只大船载了许多人马头领,却投穆太公庄上来。一帆
顺风,早到岸边埠头,一行众人,都上岸来。穆弘邀请众好汉到庄内堂上,穆太公
出来迎接,宋江等众人都相见了。太公道:“众头领连夜劳神,具请客房中安歇,
将息贵体。”各人且去房里暂歇将养,整理衣服器械。当日穆弘叫庄客宰了一头黄
牛,杀了十数个猪、羊、鸡、鹅、鱼、鸭,珍肴异馔,排下筵席,管待众头领。饮
酒中间,说起许多情节。晁盖道:“若非是二哥众位把船相救,我等皆被陷于缧绁。”
穆太公道:“你等如何却打从那条路上来?”李逵道:“我自只拣人多处杀将去,
他们自要跟我来,我又不曾叫他!”众人听了,都大笑。

宋江起身与众人道:“小人宋江,若无众好汉相救时,和戴院长皆死于非命。
今日之恩,深于沧海,如何报答得众位?只恨黄文炳那厮搜根剔齿,几番唆毒,要
害我们。这冤仇如何不报?怎地启请众位好汉,再做个天大人情,去打了无为军,
杀得黄文炳那厮,也与宋江消了这口无穷之恨。那时回去如何?”晁盖道:“我们
众人偷营劫寨,只可使一遍,如何再行得?似此奸贼已有提备,不若且回山寨去,
聚起大队人马,一发和学究、公孙二先生,并林冲、秦明,都来报仇,也未为晚。”
宋江道:“若是回山去了,再不能够得来。一者山遥路远,二乃江州必然申开明文,
各处谨守。不要痴想,只是趁这个机会,便好下手,不要等他做了准备。”花荣道:
“哥哥见得是。虽然如此,只是无人识得路境,不知他地理如何。先得个人去那里
城中探听虚实,也要看无为军出没的路径去处,就要认黄文炳那贼的住处了,然后
方好下手。”薛永便起身说道:“小弟多在江湖上行,此处无为军最熟,我去探听
一遭如何?”宋江道:“若得贤弟去走一遭最好。”薛永当日别了众人自去了。

只说宋江自和众头领在穆弘庄上商议要打无为军一事,整顿军器枪刀,安排弓
弩箭矢,打点大小船只等项。提备已了,只见薛永去了两日,带将一个人回到庄上
来,拜见宋江。宋江便问道:“兄弟,这位壮士是谁?”薛永答道:“这人姓侯,
名健,祖居洪都人氏。做得第一手裁缝,端的是飞针走线。更兼惯习枪棒,曾拜薛
永为师。人见他黑瘦轻捷,因此唤他做通臂猿。现在这无为军城里黄文炳家做生活。
小弟因见了,就请在此。”宋江大喜,便教同坐商议,那人也是一座地煞星之数,
自然义气相投。宋江便问江州消息,无为军路径如何,薛永说道:“如今蔡九知府
计点官军、百姓被杀死有五百余人,带伤中箭者,不计其数。现今差人星夜申奏朝
廷去了。城门日中后便关,出入的好生盘问得紧。原来哥哥被害一事,倒不干蔡九
知府事,都是黄文炳那厮三回五次,点拨知府,教害二位。如今见劫了法场,城中
甚慌,晓夜提备。小弟又去无为军打听,正撞见侯健这个兄弟出来吃饭,因是得知
备细。”

宋江道:“侯兄何以知之?”侯健道:“小人自幼只爱习学枪棒,多得薛师父
指教,因此不敢忘恩。近日黄通判特取小人来他家做衣服,因出来遇见师父,提起
仁兄大名,说起此一节事来。小人要结识仁兄,特来报知备细。这黄文炳有个嫡亲
哥哥,唤做黄文烨,与这文炳是一母所生二子。这黄文烨平生只是行善事,修桥补
路,塑佛斋僧,扶危济困,救拔贫苦,那无为军城中,都叫他黄佛子。这黄文炳虽
是罢闲通判,心里只要害人,惯行歹事,无为军都叫他做黄蜂刺。他弟兄两个分开
做两处住,只在一条巷内出入,靠北门里便是他家。黄文炳贴着城住,黄文烨近着
大街。小人在他那里做生活,却听得黄通判回家来说这件事:‘蔡九知府已被瞒过
了,却是我点拨他,教知府先斩了,然后奏去。’黄文烨听得说时,只在背后骂说
道:‘又做这等短命促掐的事。于你无干,何故定要害他?倘或有天理之时,报应
只在目前,却不是反招其祸。’这两日听得劫了法场,好生吃惊,昨夜去江州探望
蔡九知府,与他计较,尚兀自未回来。”宋江道:“黄文炳隔着他哥哥家多少路?”
侯健道:“原是一家分开的,如今只隔着中间一个菜园。”宋江道:“黄文炳家多
少人口?有几房头?”侯健道:“男子妇人通有四五十口。”宋江道:“天教我报
仇,特地送这个人来。虽是如此,全靠众弟兄维持。”众人齐声应道:“当以死向
前,正要驱除这等赃滥奸恶之人,与哥哥报仇雪恨。”宋江又道:“只恨黄文炳那
贼一个,却与无为军百姓无干。他兄既然仁德,亦不可害他,休教天下人骂我等不
仁。众弟兄去时,不可分毫侵害百姓。今去那里,我有一计,只望众人扶助扶助。”
众头领齐声道:“专听哥哥指教。”

宋江道:“有烦穆太公对付八九十个叉袋,又要百十束芦柴,用着五只大船,
两只小船。央及张顺、李俊驾两只小船,在江面上与他如此行;五只大船上,用着
张横、三阮、童威和识水的人护船,此计方可。”穆弘道:“此间芦苇、油柴、布
袋都有,我庄上的人都会使水驾船,便请哥哥行事。”宋江道:“却用侯家兄弟引
着薛永并白胜,先去无为军城中藏了,来日三更二点为期,且听门外放起带铃鹁鸽,
便教白胜上城策应,先插一条白绢号带,近黄文炳家,便是上城去处。再又教石勇、
杜迁扮做丐者,去城门边左近埋伏,只看火为号,便要下手杀把门军士。李俊、张
顺只在江面上往来巡绰,等候策应。”

宋江分拨已定。薛永、白胜、侯健先自走了。随后再是石勇、杜迁扮做丐者,
身边各藏了短刀暗器,也去了。这里自一面扛抬沙土布袋和芦苇、油柴,上船装载。
众好汉至期各各拴束了,身上都准备了器械,船仓里埋伏军汉,众头领分拨下船。
晁盖、宋江、花荣在童威船上,燕顺、王矮虎、郑天寿在张横船上,戴宗、刘唐、
黄信在阮小二船上,吕方、郭盛、李立在阮小五船上,穆弘、穆春、李逵在阮小七
船上。只留下朱贵、宋万在穆太公庄,看理江州城里消息。先使童猛棹一只打渔快
船,前去探路。小喽罗并军健都伏在仓里,大家庄客、水手,撑驾船只,当夜密地
望无为军来。此时正是七月尽天气,夜凉风静,月白江清,水影山光,上下一碧。
昔日参寥子有首诗,题这江景,道是:
洪涛滚滚烟波杳,月淡风清九江晓。
欲从舟子问如何,但觉庐山眼中小。

是夜初更前后,大小船只都到无为江岸边,拣那有芦苇深处,一字儿缆定了船
只,只见童猛回船来报道:“城里并无些动静。”宋江便叫手下众人,把这沙土布
袋和芦苇干柴都搬上岸,望城边来。听那更鼓时,正打二更。宋江叫小喽罗各各
了沙土布袋并芦柴,就城边堆垛了。众好汉各挺手中军器,只留张横、三阮、两童
守船接应,其余头领都奔城边来。望城上时,约离北门有半里之路,宋江便叫放起
带铃鹁鸽。只见城上一条竹竿,缚着白号带,风飘起来。宋江见了,便叫军士就这
城边堆起沙土布袋,分付军汉,一面挑担芦苇、油柴上城。只见白胜已在那里接应
等候,把手指与众军汉道:“只那条巷便是黄文炳住处。”宋江问白胜道:“薛永、
侯健在那里?”白胜道:“他两个潜入黄文炳家里去了,只等哥哥到来。”宋江又
问道:“你曾见石勇、杜迁么?”白胜道:“他两个在城门边左近伺候。”宋江听
罢,引了众好汉下城来,径到黄文炳门前。只见侯健闪在房檐下,宋江唤来,附耳
低言道:“你去将菜园门开了,放他军士把芦苇、油柴堆放里面,可教薛永寻把火
来点着,却去敲黄文炳门道:‘间壁大官人家失火,有箱笼什物搬来寄顿。’敲得
门开,我自有摆布。”

宋江教众好汉分几个把住两头。侯健先去开了菜园门,军汉把芦柴搬来,堆在
里面。侯健就讨了火种,递与薛永,将来点着。侯健便闪出来,却去敲门叫道:“间
壁大官人家失火,有箱笼搬来寄顿,快开门则个。”里面听得,便起来看时,望见
隔壁火起,连忙开门出来。晁盖、宋江等呐声喊,杀将入去。众好汉亦各动手,见
一个,杀一个,见两人,杀一双,把黄文炳一门内外大小四五十口,尽皆杀了,不
留一人,只不见了文炳一个。众好汉把他从前酷害良民积攒下许多家私金银,收拾
俱尽。大哨一声,众多好汉都扛了箱笼家财,却奔城上来。

且说石勇、杜迁见火起,各掣出尖刀,便杀把门军人,又见前街邻舍拿了水桶
梯子,都来救火。石勇、杜迁大喝道:“你那百姓,休得向前!我们是梁山泊好汉
数千在此,来杀黄文炳一门良贱,与宋江、戴宗报仇,不干你百姓事。你们快回家
躲避了,休得出来闲管事。”众邻舍还有不信的,立住了脚看,只见黑旋风李逵抡
起两把板斧,着地卷将来,众邻舍方才呐声喊,抬了梯子水桶,一哄都走了。这边
后巷也有几个守门军汉,带了些人,了麻搭火钩,都奔来救火。早被花荣张起弓,
当头一箭,射翻了一个,大喝道:“要死的,便来救火。”那伙军汉一齐都退去了。
只见薛永拿着火把,便就黄文炳家里前后点着,乱乱杂杂火起。看那火时,但见:

黑云匝地,红焰飞天。律律走万道金蛇,焰腾腾散千团火块。狂风相助,雕
梁画栋片时休。炎焰涨空,大厦高堂弹指没。这不是火,却是:文炳心头恶,触恼
丙丁神。害人施毒焰,惹火自烧身。

当时石勇、杜迁已杀倒把门军士,李逵砍断铁锁,大开了城门,一半人从城上
出去,一半人从城门下出去。张横、三阮、两童都来接应,合做一处,扛抬财物上
船。无为军已知江州被梁山泊好汉劫了法场,杀死无数的人,如何敢出来追赶,只
得回避了。这宋江一行众好汉只恨拿不着黄文炳,都上了船去,摇开了,自投穆弘
庄上来,不在话下。

却说江州城里望见无为军火起,蒸天价红,满城中讲动,只得报知本府。这黄
文炳正在府里议事,听得报说了,慌忙来禀知府道:“敝乡失火,急欲回家看觑。”
蔡九知府听得,忙叫开城门,差一只官船相送。黄文炳谢了知府,随即出来,带了
从人,慌速下船,摇开江面,望无为军来。看见火势猛烈,映得江面上都红,艄公
说道:“这火只是北门里火。”黄文炳见说了,心里越慌。看看摇到江心里,只见
一只小船从江面上摇过去了,不多时,又是一只小船摇将过来,却不径过,望着官
船直撞将来。从人喝道:“甚么船,敢如此直撞来!”只见那小船上一个大汉跳起
来,手里拿着挠钩,口里应道:“去江州报失火的船。”黄文炳便钻出来问道:“那
里失火?”那大汉道:“北门里黄通判家,被梁山泊好汉杀了一家人口,劫了家私,
如今正烧着哩!”黄文炳失口叫声苦,不知高低。那汉听了,一挠钩搭住了船,便
跳过来。黄文炳是个乖觉的人,早瞧了八分,便奔船梢后走,望江里踊身便跳。忽
见江面上一只船,水底下早钻过一个人,把黄文炳劈腰抱住,拦头揪起,扯上船来。
船上那个大汉早来接应,便把麻索绑了。水底下活捉了黄文炳的,便是浪里白跳张
顺,船上把挠钩的,便是混江龙李俊,两个好汉立在船上。那摇官船的艄公只顾下
拜。李俊说道:“我不杀你们,只要捉黄文炳这厮,你们自回去说与蔡九知府那贼
驴知道,俺梁山泊好汉们权寄下他那颗驴头,早晚便要来取。”艄公战抖抖的道:
“小人去说。”李俊、张顺拿了黄文炳过自己的小船上,放那官船去了。

两个好汉棹了两只快船,径奔穆弘庄上,早摇到岸边,望见一行头领,都在岸
上等候,搬运箱笼上岸。见说拿得黄文炳,宋江不胜之喜,众好汉一齐心中大喜,
说:“正要此人见面。”李俊、张顺早把黄文炳带上岸来,众人看了,监押着,离
了江岸,到穆太公庄上来。朱贵、宋万接着众人,入到庄里草厅上坐下。宋江把黄
文炳剥了湿衣服,绑在柳树上,请众头领团团坐定。宋江叫取一壶酒来,与众人把
盏。上自晁盖,下至白胜,共是三十位好汉,都把遍了。宋江大骂黄文炳:“你这
厮!我与你往日无冤,近日无仇,你如何只要害我,三回五次,教唆蔡九知府杀我
两个?你既读圣贤之书,如何要做这等毒害的事?我又不与你有杀父之仇,你如何定
要谋我?你哥哥黄文烨,与你这厮一母所生,他怎恁般修善,久闻你那城中都称他
做黄佛子,我昨夜分毫不曾侵犯他。你这厮在乡中只是害人,交结权势,浸润官长,
欺压良善,我知道无为军人民都叫你做黄蜂刺,我今日且替你拔了这个刺。”黄文
炳告道:“小人已知过失,只求早死。”晁盖喝道:“你那贼驴,怕你不死!你这
厮早知今日,悔不当初。”宋江便问道:“那个兄弟替我下手?”只见黑旋风李逵
跳起身来说道:“我与哥哥动手割这厮。我看他肥胖了,倒好烧吃。”晁盖道:“说
得是,教取把尖刀来,就讨盆炭火来,细细地割这厮烧来下酒,与我贤弟消这怨气。”

李逵拿起尖刀,看着黄文炳笑道:“你这厮在蔡九知府后堂且会说黄道黑,拨
置害人,无中生有撺掇他。今日你要快死,老爷却要你慢死。”便把尖刀先从腿上
割起,拣好的,就当面炭火上炙来下酒。割一块,炙一块,无片时,割了黄文炳,
李逵方才把刀割开胸膛,取出心肝,把来与众头领做醒酒汤。众多好汉看割了黄文
炳,都来草堂上与宋江贺喜。有诗为证:
文炳趋炎巧计乖,却将忠义苦挤排。
奸谋未遂身先死,难免刳心炙肉灾。

只见宋江先跪在地下,众头领慌忙都跪下,齐道:“哥哥有甚事,但说不妨,
兄弟们敢不听。”宋江便道:“小可不才,自小学吏。初世为人,便要结识天下好
汉。奈缘力薄才疏,不能接待,以遂平生之愿。自从刺配江州,多感晁头领并众豪
杰苦苦相留,宋江因见父亲严训,不曾肯住。正是天赐机会,于路直至浔阳江上,
又遭际许多豪杰。不想小可不才,一时间酒后狂言,险累了戴院长性命。感谢众位
豪杰不避凶险,来虎穴龙潭,力救残生,又蒙协助,报了冤仇。如此犯下大罪,闹
了两座州城,必然申奏去了。今日不由宋江不上梁山泊投托哥哥去,未知众位意下
若何?如是相从者,只今收拾便行;如不愿去的,一听尊命。只恐事发,反遭负累,
烦可寻思。”说言未绝,李逵跳将起来,便叫道:“都去,都去!但有不去的,吃
我一鸟斧,砍做两截便罢!”宋江道:“你这般粗卤说话,全在各人弟兄们心肯意
肯,方可同去。”众人议论道:“如今杀死了许多官军人马,闹了两处州郡,他如
何不申奏朝廷,必然起军马来擒获。今若不随哥哥去,同死同生,却投那里去?”

宋江大喜,谢了众人。当日先叫朱贵和宋万前回山寨里去报知,次后分作五起
进程:头一起,便是晁盖、宋江、花荣、戴宗、李逵;第二起,便是刘唐、杜迁、
石勇、薛永、侯健;第三起,便是李俊、李立、吕方、郭盛、童威、童猛;第四起,
便是黄信、张顺、张横、阮家三弟兄;第五起,便是燕顺、王矮虎、穆弘、穆春、
郑天寿、白胜。五起二十八个头领,带了一干人等,将这所得黄文炳家财各各分开,
装载上车子。穆弘带了太公并家小人等,将应有家财金宝装载车上。庄客数内有不
愿去的,都赍发他些银两,自投别主去;佣工有愿去的,一同便往。前四起陆续去
了,已自行动。穆弘收拾庄内已了,放起十数个火把,烧了庄院,撇下了田地,自
投梁山泊来。

且不说五起人马登程,节次进发,只隔二十里而行。先说第一起晁盖、宋江、
花荣、戴宗、李逵五骑马,带着车仗人伴,在路行了三日,前面来到一个去处,地
名唤做黄门山。宋江在马上与晁盖说道:“这座山生得形势怪恶,莫不有大伙在内?
可着人催后面人马上来,一同过去。”说犹未了,只见前面山嘴上锣鸣鼓响。宋
江道:“我说么!且不要走动,等后面人马到来,好和他厮杀。”花荣便拈弓搭箭
在手,晁盖、戴宗各执朴刀,李逵拿着双斧,拥护着宋江,一齐趱马向前。只见山
坡边闪出三五百个小喽罗,当先簇拥出四筹好汉,各挺军器在手,高声喝道:“你
等大闹了江州,劫掠了无为军,杀害了许多官军百姓,待回梁山泊去?我四个等你
多时。会事的只留下宋江,都饶了你们性命。”宋江听得,便挺身出去,跪在地下,
说道:“小可宋江被人陷害,冤屈无伸,今得四方豪杰救了性命,小可不知在何处
触犯了四位英雄,万望高抬贵手,饶恕残生。”

那四筹好汉见了宋江跪在前面,都慌忙滚鞍下马,撇了军器,飞奔前来,拜倒
在地下,说道:“俺弟兄四个只闻山东及时雨宋公明大名,想杀也不能够见面。俺
听知哥哥在江州为事吃官司,我弟兄商议定了,正要来劫牢,只是不得个实信。前
日使小喽罗直到江州来打听,回来说道:‘已有多少好汉闹了江州,劫了法场,救
出往揭阳镇去了;后又烧了无为军,劫掠黄通判家。’料想哥哥必从这里来。节次
使人路中来探望,犹恐未真,故反作此一番诘问。冲撞哥哥,万勿见罪。今日幸见
仁兄,小寨里略备薄酒粗食,权当接风。请众好汉同到敝寨盘桓片时。”

宋江大喜,扶起四位好汉,逐一请问大名。为头的那人姓欧,名鹏,祖贯是黄
州人氏,守把大江军户,因恶了本官,逃走在江湖上绿林中,熬出这个名字,唤做
摩云金翅。第二个好汉姓蒋,名敬,祖贯是湖南潭州人氏,原是落科举子出身,科
举不第,弃文就武,颇有谋略,精通书算,积万累千,纤毫不差,亦能刺枪使棒,
布阵排兵,因此人都唤他做神算子。第三个好汉姓马,名麟,祖贯是南京建康人氏,
原是小番子闲汉出身,吹得双铁笛,使得好大滚刀,百十人近他不得,因此人都唤
他做铁笛仙。第四个好汉姓陶,名宗旺,祖贯是光州人氏,庄家田户出身,惯使一
把铁锹,有的是气力,亦能使枪抡刀,因此人都唤做九尾龟。怎见得四个好汉英雄,
有《西江月》为证:

力壮身强无赛,行时捷似飞腾,摩云金翅是欧鹏,首位黄山排定。

幼恨毛
锥失利,长从韬略搜精,如神算法善行兵,文武全才蒋敬。

铁笛一声山裂,铜刀两口神惊,马麟形貌更狰狞,厮杀场中超乘。

宗旺力
如猛虎,铁锹到处无情,神龟九尾喻多能,都是英雄头领。

这四筹好汉接住宋江,小喽罗早捧过果盒,一大壶酒,两大盘肉,托过来把盏。
先递晁盖、宋江,次递花荣、戴宗、李逵,与众人都相见了。一面递酒。没两个时
辰,第二起头领又到了,一个个尽都相见。把盏已遍,邀请众位上山。两起十位头
领先来到黄门山寨内,那四筹好汉便叫椎牛宰马管待。却教小喽罗陆续下山,接请
后面那三起十八位头领上山来筵宴。未及半日,三起好汉已都来到了,尽在聚义厅
上筵席相会。宋江饮酒中间,在席上开话道:“今次宋江投奔了哥哥晁天王,上梁
山泊去,一同聚义,未知四位好汉肯弃了此处,同往梁山泊大寨相聚否?”四个好
汉齐答道:“若蒙二位义士不弃贫贱,情愿执鞭坠镫。”宋江、晁盖大喜,便说道:
“既是四位肯从大义,便请收拾起程。”众多头领俱各欢喜。在山寨住了一日,过
了一夜。次日,宋江、晁盖仍旧做头一起,下山进发先去;次后依例而行,只隔着
二十里远近。四筹好汉收拾起财帛金银等项,带领了小喽罗三五百人,便烧毁了寨
栅,随作第六起登程。宋江又合得这四个好汉,心中甚喜,于路在马上对晁盖说道:
“小弟来江湖上走了这几遭,虽是受了些惊恐,却也结识得这许多好汉。今日同哥
哥上山去,这回只得死心塌地,与哥哥同死同生。”一路上说着闲话,不觉早来到
朱贵酒店里了。

且说四个守山寨的头领吴用、公孙胜、林冲、秦明和两个新来的萧让、金大坚,
已得朱贵、宋万先回报知,每日差小头目棹船出来酒店里迎接,一起起都到金沙滩
上岸,擂鼓吹笛,众好汉们都乘马轿,迎上寨来。到得关下,军师吴学究等六人,
把了接风酒,都到聚义厅上,焚起一炉好香。晁盖便请宋江为山寨之主,坐第一把
交椅。宋江那里肯,便道:“哥哥差矣!感蒙众位不避刀斧,救拔宋江性命,哥哥
原是山寨之主,如何却让不才?若要坚执如此相让,宋江情愿就死。”晁盖道:“贤
弟如何这般说!当初若不是贤弟担那血海般干系,救得我等七人性命上山,如何有
今日之众?你正是山寨之恩主。你不坐,谁坐?”宋江道:“仁兄,论年齿,兄长
也大十岁,宋江若坐了,岂不自羞。”再三推晁盖坐了第一位,宋江坐了第二位,
吴学究坐了第三位,公孙胜坐了第四位。宋江道:“休分功劳高下,梁山泊一行旧
头领去左边主位上坐,新到头领去右边客位上坐,待日后出力多寡,那时另行定夺。”
众人齐道:“哥哥言之极当。”左边一带,是林冲、刘唐、阮小二、阮小五、阮小
七、杜迁、宋万、朱贵、白胜;右边一带,论年甲次序,互相推让,花荣、秦明、
黄信、戴宗、李逵、李俊、穆弘、张横、张顺、燕顺、吕方、郭盛、萧让、王矮虎、
薛永、金大坚、穆春、李立、欧鹏、蒋敬、童威、童猛、马麟、石勇、侯健、郑天
寿、陶宗旺。共是四十位头领坐下。大吹大擂,且吃庆喜筵席。

宋江说起江州蔡九知府捏造谣言一事,说与众人:“叵耐黄文炳那厮,事又不
干他己,却在知府面前胡言乱道,解说道:‘耗国因家木’,耗散国家钱粮的人,
必是家头着个‘木’字,不是个‘宋’字?‘刀兵点水工’,兴动刀兵之人,必是
三点水着个‘工’字,不是个‘江’字?这个正应宋江身上。那后两句道:‘纵横
三十六,播乱在山东。’合主宋江造反在山东。以此拿了小可。不期戴院长又传了
假书,以此黄文炳那厮撺掇知府,只要先斩后奏。若非众好汉救了,焉得到此!”
李逵跳将起来道:“好哥哥,正应着天上的言语,虽然吃了他些苦,黄文炳那贼也
吃我割得快活。放着我们有许多军马,便造反,怕怎地?晁盖哥哥便做了大皇帝,
宋江哥哥便做了小皇帝,吴先生做个丞相,公孙道士便做个国师,我们都做个将军,
杀去东京,夺了鸟位,在那里快活,却不好?不强似这个鸟水泊里?”戴宗连忙喝
道:“铁牛,你这厮胡说!你今日既到这里,不可使你那在江州性儿,须要听两位
头领哥哥的言语号令,亦不许你胡言乱语,多嘴多舌。再如此多言插口,先割了你
这颗头来为令,以警后人。”李逵道:“阿哎!若割了我这颗头,几时再长的一个
出来?我只吃酒便了。”众多好汉都笑。

宋江又题起拒敌官军一事,说道:“那时小可初闻这个消息,好不惊恐,不期
今日轮到宋江身上。”吴用道:“兄长当初若依了弟兄之言,只住山上快活,不到
江州,不省了多少事?这都是天数注定如此。”宋江道:“黄安那厮,如今在那里?”
晁盖道:“那厮住不够两三个月,便病死了。”宋江嗟叹不已。当日饮酒,各各尽
欢。晁盖先叫安顿穆太公一家老小。叫取过黄文炳的家财,赏劳了众多出力的小喽
罗。取出原将来的信笼,交还戴院长收用。戴宗那里肯要,定教收放库内,公支使
用。晁盖叫众多小喽罗参拜了新头领李俊等,都参见了。连日山寨里杀牛宰马,作
庆贺筵席,不在话下。

再说晁盖教向山前山后各拨定房屋居住,山寨里再起造房舍,修理城垣。至第
三日,酒席上宋江起身对众头领说道:“宋江还有一件大事,正要禀众弟兄:小可
今欲下山走一遭,乞假数日,未知众位肯否?”晁盖便问道:“贤弟今欲要往何处,
干甚么大事?”宋江不慌不忙,说出这个去处。有分教:枪刀林里,再逃一遍残生;
山岭边旁,传授千年勋业。正是:只因玄女书三卷,留得清风史数篇。

毕竟宋公明要往何处去走一遭,且听下回分解。