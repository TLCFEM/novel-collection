\chapter{病关索大闹翠屏山~拚命三火烧祝家店}

话说当下众邻舍结住王公,直到蓟州府里首告。知府却才升厅,一行人跪下告
道:“这老子挑着一担糕粥,泼翻在地下,看时,却有两个死尸在地下:一个是和
尚,一个是头陀,俱各身上无一丝,头陀身边有刀一把。”老子告道:“老汉每日
常卖糕糜营生,只是五更出来赶趁。今朝起得早了些个,和这铁头猴子只顾走,不
看下面,一交绊翻,碗碟都打碎了,只见两个死尸血渌渌的在地上,一时失惊,叫
起来,倒被邻舍扯住到官。望相公明镜,可怜见辩察。”知府随即取了供词,行下
公文,委当方里甲,带了仵作公人,押了邻舍、王公一干人等,下来检验尸首,明
白回报。众人登场看检已了,回州禀复知府:“被杀死僧人系是报恩寺黎裴如海,
旁边头陀,系是寺后胡道。和尚不穿一丝,身上三四道搠伤致命方死;胡道身边见
有凶刀一把,只见项上有勒死痕伤一道,想是胡道掣刀搠死和尚,惧罪自行勒死。”
知府叫拘本寺僧鞫问缘故,俱各不知情由,知府也没个决断,当案孔目禀道:“眼
见得这和尚裸形赤体,必是和那头陀干甚不公不法的事,互相杀死,不干王公之事。
邻舍都教召保听候,尸首着仰本寺住持即备棺木盛殓,放在别处,立个互相杀死的
文书便了。”知府道:“也说得是。”随即发落了一干人等,不在话下。

蓟州城里有些好事的子弟,做成一调儿,道是:

叵耐秃囚无状,做事直恁狂荡。暗约娇娥,要为夫妇,永
同鸳帐。怎禁贯恶满盈,玷辱诸多和尚,血泊内横尸里巷。今日赤条条甚么模样,
立雪齐腰,投岩喂虎,全不想祖师经上。目莲救母生天,这贼秃为婆娘身丧。
后来书会们备知了这件事,拿起笔来,又做了这只《临江仙》词,教唱道:

淫行沙门招杀报,暗中不爽分毫。头陀尸首亦蹊,一丝真不挂,立地吃屠刀。

大和尚此时精血丧,小和尚昨夜风骚。空门里刎颈见相交,拚死争同穴,残生送两
条。
这件事,满城都讲动了。那妇人也惊得呆了,自不敢说,只是肚里暗暗地叫苦。

杨雄在蓟州府里,有人告道杀死和尚、头陀,心里早瞧了七八分,寻思:“此
一事,准是石秀做出来的。我前日一时间错怪了他,我今日闲些,且去寻他,问他
个真实。”正走过州桥前来,只听得背后有人叫道:“哥哥,那里去?”杨雄回过
头来,见是石秀,便道:“兄弟,我正没寻你处。”石秀道:“哥哥且来我下处,
和你说话。”把杨雄引到客店里小房内,说道:“哥哥,兄弟不说谎么?”杨雄道:
“兄弟,你休怪我。是我一时愚蠢,不是了,酒后失言,反被那婆娘瞒过了,怪兄
弟相闹不得。我今特来寻贤弟,负荆请罪。”石秀道:“哥哥,兄弟虽是个不才小
人,却是顶天立地的好汉,如何肯做这等之事?怕哥哥日后中了奸计,因此来寻哥
哥,有表记教哥哥看。”将过和尚、头陀的衣裳,“尽剥在此。”杨雄看了,心头
火起,便道:“兄弟休怪。我今夜碎割了这贱人,出这口恶气。”石秀笑道:“你
又来了。你既是公门中勾当的人,如何不知法度?你又不曾拿得他真奸,如何杀得
人?倘或是小弟胡说时,却不错杀了人。”杨雄道:“似此怎生罢休得?”石秀道:
“哥哥只依着兄弟的言语,教你做个好男子。”杨雄道:“贤弟,你怎地教我做个
好男子?”石秀道:“此间东门外有一座翠屏山,好生僻静。哥哥到明日,只说道,
我多时不曾烧香,我今来和大嫂同去,把那妇人赚将出来,就带了迎儿同到山上。
小弟先在那里等候着,当头对面,把这是非都对得明白了,哥哥那时写与一纸休书,
弃了这妇人,却不是上着?”杨雄道:“兄弟,何必说得,你身上清洁,我已知了,
都是那妇人谎说。”石秀道:“不然,我也要哥哥知道他往来真实的事。”杨雄道:
“既然兄弟如此高见,必然不差,我明日准定和那贱人来,你却休要误了。”石秀
道:“小弟不来时,所言俱是虚谬。”

杨雄当下别了石秀,离了客店,且去府里办事,至晚回来,并不提起,亦不说
甚,只和每日一般。次日天明起来,对那妇人说道:“我昨夜梦见神人叫我,说有
旧愿不曾还得。向日许下东门外岳庙里那炷香愿,未曾还得,今日我闲些,要去还
了,须和你同去。”那妇人道:“你便自去还了罢,要我去何用?”杨雄道:“这
愿心却是当初说亲时许下的,必须要和你同去。”那妇人道:“既是恁地,我们早
吃些素饭,烧汤沐浴了去。”杨雄道:“我去买香纸,顾轿子,你便洗浴了,梳头
插带了等我,就叫迎儿也去走一遭。”

杨雄又来客店里,相约石秀:“饭罢便来,兄弟休误。”石秀道:“哥哥,你
若抬得来时,只教在半山里下了轿,你三个步行上来,我自在上面一个僻处等你,
不要带闲人上来。”杨雄约了石秀,买了纸烛,归来吃了早饭。那妇人不知此事,
只顾打扮的齐齐整整,迎儿也插带了,轿夫扛轿子,早在门前伺候。杨雄道:“泰
山看家,我和大嫂烧香了便回。”潘公道:“多烧香,早去早回。”

那妇人上了轿子,迎儿跟着,杨雄也随在后面。出得东门来,杨雄低低分付轿
夫道:“与我抬上翠屏山去,我自多还你些轿钱。”不到两个时辰,早来到翠屏山
上。原来这座翠屏山,却在蓟州东门外二十里,都是人家的乱坟,上面一望,尽是
青草白杨,并无庵舍寺院。当下杨雄把那妇人抬到半山,叫轿夫歇下轿子,拔去葱
管,搭起轿帘,叫那妇人出轿来。妇人问道:“却怎地来这山里?”杨雄道:“你
只顾且上去。轿夫只在这里等候,不要来,少刻一发打发你酒钱。”轿夫道:“这
个不妨,小人自只在此间伺候便了。”杨雄引着那妇人并迎儿,三个人上了四五层
山坡,只见石秀坐在上面。那妇人道:“香纸如何不将来?”杨雄道:“我自先使
人将上去了。”把妇人一引,引到一处古墓里,石秀便把包裹、腰刀、杆棒,都放
在树根前,来道:“嫂嫂拜揖。”那妇人连忙应道:“叔叔怎地也在这里?”一头
说,一面肚里吃了一惊。石秀道:“在此专等多时。”杨雄道:“你前日对我说道:
叔叔多遍把言语调戏你,又将手摸着你胸前,问你有孕也无。今日这里无人,你两
个对的明白。”那妇人道:“哎呀!过了的事,只顾说甚么?”石秀睁着眼来道:
“嫂嫂,你怎么说?这须不是闲话,正要哥哥面前对个明白。”那妇人道:“叔叔,
你没事自把儿提做甚么?”石秀道:“嫂嫂,你休要硬诤,教你看个证见。”便
去包裹里,取出海黎并头陀的衣服来,撒放地下道:“你认得么?”那妇人看了,
飞红了脸,无言可对。石秀“飕”地掣出腰刀,便与杨雄说道:“此事只问迎儿,
便知端的。”

杨雄便揪过那丫头跪在面前,喝道:“你这小贱人!快好好实说:怎地在和尚
房里入奸,怎生约会把香桌儿为号,如何教头陀来敲木鱼。实对我说,饶你这条性
命;但瞒了一句,先把你剁做肉泥!”迎儿叫道:“官人,不干我事,不要杀我,
我说与你。”却把僧房中吃酒,上楼看佛牙,赶他下楼来看潘公酒醒说起,“两个
背地里约下,第三日教头陀来化斋饭,叫我取铜钱布施与他,娘子和他约定:但是
官人当牢上宿,要我掇香桌儿放在后门外,便是暗号。头陀来看了,却去报知和尚。
当晚海黎扮做俗人,带顶头巾入来,五更里只听那头陀来敲木鱼响,高声念佛为
号,叫我开后门放他出去。但是和尚来时,瞒我不得,只得对我说了。娘子许我一
副钏镯,一套衣裳,我只得随顺了。似此往来,通有数十遭,后来便吃杀了。又与
我几件首饰,教我对官人说石叔叔把言语调戏一节。这个我眼里不曾见,因此不敢
说。只此是实,并无虚谬。”

迎儿说罢,石秀便道:“哥哥得知么?这般言语,须不是兄弟教他如此说。请
哥哥却问嫂嫂备细缘由。”杨雄揪过那妇人来,喝道:“贼贱人!丫头已都招了,
便你一些儿休赖,再把实情对我说了,饶了这贱人一条性命。”那妇人说道:“我
的不是了。你看我旧日夫妻之面,饶恕了我这一遍。”石秀道:“哥哥含糊不得,
须要问嫂嫂一个明白备细缘由。”杨雄喝道:“贱人,你快说!”那妇人只得把偷
和尚的事,从做道场夜里说起,直至往来,一一都说了。石秀道:“你却怎地对哥
哥倒说我来调戏你?”那妇人道:“前日他醉了骂我,我见他骂得跷蹊,我只猜是
叔叔看见破绽,说与他。到五更里,又提起来问叔叔如何,我却把这段话来支吾,
实是叔叔并不曾恁地。”石秀道:“今日三面说得明白了,任从哥哥心下如何措置。”
杨雄道:“兄弟,你与我拔了这贱人的头面,剥了衣裳,我亲自伏侍他。”石秀便
把那妇人头面首饰衣服都剥了,杨雄割两条裙带来,亲自用手把妇人绑在树上。石
秀也把迎儿的首饰都去了,递过刀来说道:“哥哥,这个小贱人,留他做甚么?一
发斩草除根。”杨雄应道:“果然,兄弟把刀来,我自动手。”迎儿见头势不好,
却待要叫,杨雄手起一刀,挥作两段。那妇人在树上叫道:“叔叔劝一劝。”石秀
道:“嫂嫂,哥哥自来伏侍你。”杨雄向前,把刀先挖出舌头,一刀便割了,且教
那妇人叫不的。杨雄却指着骂道:“你这贼贱人!我一时间误听不明,险些被你瞒
过了。一者坏了我兄弟情分,二乃久后必然被你害了性命。不如我今日先下手为强。
我想你这婆娘心肝五脏怎地生着,我且看一看。”一刀从心窝里直割到小肚子下,
取出心肝五脏,挂在松树上。杨雄又将这妇人七事件分开了,却将头面衣服都拴在
包裹里了。

杨雄道:“兄弟,你且来,和你商量一个长便。如今一个奸夫,一个淫妇,都
已杀了,只是我和你投那里去安身?”石秀道:“兄弟已寻思下了,自有个所在,
请哥哥便行,不可耽迟。”杨雄道:“却是那里去?”石秀道:“哥哥杀了人,兄
弟又杀人,不去投梁山泊入伙,却投那里去?”杨雄道:“且住!我和你又不曾认
得他那里一个人,如何便肯收录我们?”石秀道:“哥哥差矣!如今天下江湖上皆
闻山东及时雨宋公明招贤纳士,结识天下好汉,谁不知道?放着我和你一身好武艺,
愁甚不收留!”杨雄道:“凡事先难后易,免得后患,我却不合是公人,只恐他疑
心,不肯安着我们。”石秀笑道:“他不是押司出身?我教哥哥一发放心。前者哥
哥认义兄弟那一日,先在酒店里和我吃酒的那两个人,一个是梁山泊神行太保戴宗,
一个是锦豹子杨林。他与兄弟十两一锭银子,尚兀自在包里,因此可去投托他。”
杨雄道:“既有这条门路,我去收拾了些盘缠便走。”石秀道:“哥哥,你也这般
搭缠。倘或入城事发拿住,如何脱身?放着包裹里现有若干钗钏首饰,兄弟又有些
银两,再有三五个人,也够用了,何须又去取讨。惹起是非来,如何解救?这事少
时便发,不可迟滞,我们只好望山后走。”

石秀便背上包裹,拿了杆棒,杨雄插了腰刀在身边,提了朴刀,却待要离古墓,
只见松树后走出一个人来叫道:“清平世界,荡荡乾坤,把人割了,却去投奔梁山
泊入伙,我听得多时了。”杨雄、石秀看时,那人纳头便拜。杨雄却认得这人,姓
时,名迁,祖贯是高唐州人氏,流落在此,只一地里做些飞檐走壁,跳篱骗马的勾
当。曾在蓟州府里吃官司,却是杨雄救了他。人都叫作鼓上蚤。有诗为证:
骨软身躯健,眉浓眼目鲜。
形容如怪族,行走似飞仙。
夜静穿墙过,更深绕屋悬。
偷营高手客,鼓上蚤时迁。
当时杨雄便问时迁:“你如何在这里?”时迁道:“节级哥哥听禀:小人近日没甚
道路,在这山里掘些古坟,觅两分东西。因见哥哥在此行事,不敢出来冲撞,却听
说去投梁山泊入伙。小人如今在此,只做得些偷鸡盗狗的勾当,几时是了。跟随的
二位哥哥上山去,却不好?未知尊意肯带挈小人么?”石秀道:“既是好汉中人物,
他那里如今招纳壮士,那争你一个?若如此说时,我们一同去。”时迁道:“小人
却认得小路去。”当下引了杨雄、石秀,三个人自取小路下后山,投梁山泊去了。

却说这两个轿夫在半山里等到红日平西,不见三个下来,分付了,又不敢上去。
挨不过了,不免信步寻上山来,只见一群老鸦成团打块在古墓上。两个轿夫上去看
时,原来却是老鸦夺那肚肠吃,以此聒噪。轿夫看了,吃那一惊,慌忙回家报与潘
公,一同去蓟州府里首告。知府随即差委一员县尉,带了仵作行人,来翠屏山检验
尸首已了,回复知府,禀道:“检得一口妇人潘巧云,割在松树边;使女迎儿,杀
死在古墓下。坟边遗下一堆妇人与和尚、头陀衣服。”知府听了,想起前日海和尚、
头陀的事,备细询问潘公。那老子把这僧房酒醉一节,和这石秀出去的缘由,细说
了一遍。知府道:“眼见得这妇人与和尚通奸,那女使、头陀做脚。想石秀那厮,
路见不平,杀死头陀、和尚;杨雄这厮,今日杀了妇人、女使无疑,定是如此。只
拿得杨雄、石秀,便知端的。”当即行移文书,出给赏钱,捕获杨雄、石秀,其余
轿夫人等,各放回听候。潘公自去买棺木,将尸首殡葬,不在话下。

再说杨雄、石秀、时迁离了蓟州地面,在路夜宿晓行,不则一日。行到郓州地
面,过得香林洼,早望见一座高山,不觉天色渐渐晚了,看见前面一所靠溪客店,
三个人行到门首看时,但见:

前临官道,后傍大溪。数百株垂柳当门,一两树梅花傍屋。荆榛篱落,周回绕
定茅茨;芦苇帘栊,前后遮藏土炕。右壁厢一行,书写“庭幽暮接五湖宾”;左势
下七字,题道“户敞朝迎三岛客”。虽居野店荒村外,亦有高车驷马来。

当日黄昏时候,店小二却待关门,只见这三个人撞将入来,小二问道:“客人
来路远,以此晚了。”时迁道:“我们今日走了一百里以上路程,因此到得晚了。”
小二哥放他三个入来安歇,问道:“客人不曾打火么?”时迁道:“我们自理会。”
小二道:“今日没客歇,灶上有两只锅干净,客人自用不妨。”时迁问道:“店里
有酒肉卖么?”小二道:“今日早起有些肉,都被近村人家买了去,只剩得一瓮酒
在这里,并无下饭。”时迁道:“也罢,先借五升米来做饭,却理会。”小二哥取
出米来与时迁,就淘了,做起一锅饭来,石秀自在房中安顿行李,杨雄取出一只钗
儿,把与店小二,先回他这瓮酒来吃,明日一发算帐。小二哥收了钗儿,便去里面
掇出那瓮酒来开了,将一碟儿熟菜放在桌子上。时迁先提一桶汤来,叫杨雄、石秀
洗了脚手,一面筛酒来,就来请小二哥一处坐地吃酒,放下四只大碗,斟下酒来吃。

石秀看见店中檐下,插着十数把好朴刀,问小二哥道:“你家店里怎的有这军
器?”小二哥应道:“都是主人家留在这里。”石秀道:“你家主人是甚么样人?”
小二道:“客人,你是江湖上走的人,如何不知我这里的名字?前面那座高山,便
唤做独龙山。山前有一座另巍巍冈子,便唤做独龙冈,上面便是主人家住宅。这里
方圆三十里,却唤做祝家庄。庄主太公祝朝奉有三个儿子,称为祝氏三杰。庄前庄
后,有五七百人家,都是佃户,各家分下两把朴刀与他。这里唤作祝家店。常有数
十个家人来店里上宿,以此分下朴刀在这里。”石秀道:“他分军器在店里何用?”
小二道:“此间离梁山泊不远,只恐他那里贼人来借粮,因此准备下。”石秀道:
“与你些银两,回与我一把朴刀用如何?”小二哥道:“这个却使不得,器械上都
编着字号。我小人吃不得主人家的棍棒,我这主人法度不轻。”石秀笑道:“我自
取笑你,你却便慌。且只顾吃酒。”小二道:“小人吃不得了,先去歇了,客人自
便宽饮几杯。”小二哥去了。

杨雄、石秀又自吃了一回酒,只见时迁道:“哥哥要肉吃么?”杨雄道:“店
小二说没了肉卖,你又那里得来?”时迁嘻嘻的笑着,去灶上提出一只老大公鸡来。
杨雄问道:“那里得这鸡来?”时迁道:“兄弟却才去后面净手,见这只鸡在笼里,
寻思没甚与哥哥吃酒,被我悄悄把去溪边杀了,提桶汤去后面,就那里得干净,
煮得熟了,把来与二位哥哥吃。”杨雄道:“你这厮还是这等贼手贼脚。”石秀笑
道:“还不改本行。”三个笑了一回,把这鸡来手撕开吃了,一面盛饭来吃。

只见那店小二略睡一睡,放心不下,爬将起来,前后去照管,只见厨桌上有些
鸡毛和鸡骨头,却去灶上看时,半锅肥汁,小二慌忙去后面笼里看时,不见了鸡,
连忙出来问道:“客人,你们好不达道理,如何偷了我店里报晓的鸡吃!”时迁道:
“见鬼了。耶耶,我自路上买得这只鸡来吃,何曾见你的鸡!”小二道:“我店里
的鸡,却那里去了?”时迁道:“敢被野猫拖了,黄猩子吃了,鹞鹰扑了去,我却
怎地得知!”小二道:“我的鸡才在笼里,不是你偷了是谁?”石秀道:“不要争,
直几钱,赔了你便罢。”店小二道:“我的是报晓鸡,店内少他不得,你便赔我十
两银子也不济,只要还我鸡。”石秀大怒道:“你诈哄谁?老爷不赔你,便怎地?”
店小二笑道:“客人,你们休要在这里讨野火吃!只我店里不比别处客店,拿你到
庄上,便做梁山泊贼寇解了去。”石秀听了,大骂道:“便是梁山泊好汉,你怎么
拿了我去请赏!”杨雄也怒道:“好意还你些钱,不赔你,怎地拿我去!”小二叫
一声:“有贼!”只见店里赤条条地走出三五个大汉来,径奔杨雄、石秀来,被石
秀手起,一拳一个,都打翻了。小二哥正待要叫,被时迁一掌,打肿了脸,作声不
得。这几个大汉都从后门走了。杨雄道:“兄弟,这厮们以定去报人来,我们快吃
了饭走了罢。”三个当下吃饱了,把包裹分开腰了,穿上麻鞋,跨了腰刀,各人去
枪架上拣了一条好朴刀。石秀道:“左右只是左右,不可放过了他。”便去灶前寻
了把草,灶里点个火,望里面四下着。看那草房被风一煽,刮刮杂杂火起来。那
火顷刻间天也似般大。三个拽开脚步,望大路便走。正是:
只为偷儿攘一鸡,从教杰士竞追。
梁山水泊兴波浪,祝氏山庄化作泥。

三个人行了两个更次,只见前面后面火把不计其数,约有一二百人,发着喊,
赶将来。石秀道:“且不要慌,我们且拣小路走。”杨雄道:“且住!一个来,杀
一个;两个来,杀一双。待天色明朗却走。”说犹未了,四下里合拢来。杨雄当先,
石秀在后,时迁在中,三个挺着朴刀,来战庄客。那伙人初时不知,抡着枪棒赶来。
杨雄手起朴刀,早戳翻了五七个。前面的便走,后面的急待要退,石秀赶入去,又
戳翻了六七人。四下里庄客见说杀伤了十数人,都是要性命的,思量不是头,都退
了去。三个得一步,赶一步。正走之间,喊声又起,枯草里舒出两把挠钩,正把时
迁一挠钩搭住,拖入草窝去了。石秀急转身来救时迁,背后又舒出两把挠钩来,却
得杨雄眼快,便把朴刀一拨,两把挠钩拨开去了,将朴刀望草里便戳,发声喊,都
走了。两个见捉了时迁,怕深入重地,亦无心恋战,顾不得时迁了,只四下里寻路
走罢。见远远的火把乱明,小路上又无丛林树木,照得有路便走,一直望东边去了。
众庄客四下里赶不着,自救了带伤的人去,将时迁背剪绑了,押送祝家庄来。

且说杨雄、石秀走到天明,望见一座村落酒店,石秀道:“哥哥,前头酒肆里
买碗酒饭吃了去,就问路程。”两个便入村店里来,倚了朴刀,对面坐下,叫酒保
取些酒来,就做些饭吃。酒保一面铺下菜蔬、案酒,烫将酒来。方欲待吃,只见外
面一个大汉奔走入来,生得阔脸方腮,眼鲜耳大,貌丑形粗,穿一领茶褐绸衫,戴
一顶万字头巾,系一条白绢搭膊,下面穿一双油膀靴,叫道:“大官人教你们挑担
来庄上纳。”店主人连忙应道:“装了担,少刻便送到庄上。”那人分付了,便转
身,又说道:“快挑来。”却待出门,正从杨雄、石秀面前过,杨雄却认得他,便
叫一声:“小郎,你如何却在这里?不看我一看?”那人回转头来,看了一看,却
也认得,便叫道:“恩人如何来到这里?”望着杨雄便拜。不是杨雄撞见了这个人,
有分教:三庄盟誓成虚谬,众虎咆哮起祸殃。

毕竟杨雄、石秀遇见的那人是谁,且听下回分解。