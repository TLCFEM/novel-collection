\chapter{施恩三入死囚牢~武松大闹飞云浦}

话说当时武松踏住蒋门神在地下道:“若要我饶你性命,只依我三件事便罢!”
蒋门神便道:“好汉但说,蒋忠都依。”武松道:“第一件,要你便离了快活林,
将一应家火什物,随即交还原主金眼彪施恩。谁教你强夺他的?”蒋门神慌忙应道:
“依得,依得。”武松道:“第二件,我如今饶了你起来,你便去央请快活林为头
为脑的英雄豪杰,都来与施恩陪话。”蒋门神道:“小人也依得。”武松道:“第
三件,你从今日交割还了,便要你离了这快活林,连夜回乡去,不许你在孟州住!
在这里不回去时,我见一遍,打你一遍,我见十遍,打十遍;轻则打你半死,重则
结果了你命。你依得么?”蒋门神听了,要挣扎性命,连声应道:“依得,依得,
蒋忠都依。”武松就地下提起蒋门神来,看时,打得脸青嘴肿,脖子歪在半边,额
角头流出鲜血来。武松指着蒋门神说道:“休言你这厮鸟蠢汉,景阳冈上那只大虫,
也只三拳两脚,我兀自打死了!量你这个,值得甚的!快交割还他。但迟了些个,再
是一顿,便一发结果了你这厮!”蒋门神此时方才知是武松,只得喏喏连声告饶。
正说之间,只见施恩早到,带领着三二十个悍勇军健,都来相帮;却见武松赢了蒋
门神,不胜之喜,团团拥定武松。武松指着蒋门神道:“本主已自在这里了。你一
面便搬,一面快去请人来陪话。”蒋门神答道:“好汉,且请去店里坐地。”

武松带一行人都到店里看时,满地都是酒浆,这两个鸟男女,正在缸里扶墙摸
壁挣扎。那妇人方才从缸里爬得出来,头脸都吃磕破了,下半截淋淋漓漓都拖着酒
浆,那几个火家酒保,走得不见影了。

武松与众人入到店里坐下,喝道:“你等快收拾起身!”一面安排车子,收拾
行李,先送那妇人去了。一面叫不着伤的酒保,去镇上请十数个为头的豪杰,都来
店里,替蒋门神与施恩陪话。尽把好酒开了,有的是按酒,都摆列了桌面,请众人
坐地。武松叫施恩在蒋门神上首坐定。各人面前放只大碗,叫把酒只顾筛来。

酒至数碗,武松开话道:“众位高邻都在这里,小人武松自从阳谷县杀了人,
配在这里,便听得人说道:‘快活林这座酒店,原是小施管营造的屋宇等项买卖,
被这蒋门神倚势豪强,公然夺了,白白地占了他的衣饭。’你众人休猜道是我的主
人,他和我并无干涉。我从来只要打天下这等不明道德的人。我若路见不平,真乃
拔刀相助,我便死也不怕。今日我本待把蒋家这厮,一顿拳脚打死,就除了一害;
我看你众高邻面上,权寄下这厮一条性命。只今晚便叫他投外府去。若不离了此间,
再撞见我时,景阳冈上大虫,便是模样。”众人才知道他是景阳冈上打虎的武都头,
都起身替蒋门神陪话道:“好汉息怒。教他便搬了去,奉还本主。”那蒋门神吃他
一吓,那里敢再做声。施恩便点了家火什物,交割了店肆。蒋门神羞惭满面,相谢
了众人,自唤了一辆车儿,就装了行李,起身去了,不在话下。

且说武松邀众高邻,直吃得尽醉方休。至晚,众人散了,武松一觉,直睡到次
日辰牌方醒。却说施老管营听得儿子施恩重霸得快活林酒店,自骑了马,直来店里,
相谢武松,连日在店内饮酒作贺。快活林一境之人,都知武松了得,那一个不来拜
见武松?自此重整店面,开张酒肆,老管营自回安平寨理事。施恩使人打听蒋门神
带了老小,不知去向。这里只顾自做买卖,且不去理他,就留武松在店里居住。自
此施恩的买卖,比往常加增三五分利息,各店里并各赌坊兑坊,加利倍送闲钱来与
施恩。施恩得武松争了这口气,把武松似爷娘一般敬重。施恩似此重霸得孟州道快
活林,不在话下。正是:
夺人道路人还夺,义气多时利亦多。
快活林中重快活,恶人自有恶人磨。

荏苒光阴,早过了一月之上。炎威渐退,玉露生凉,金风去暑,已及深秋。有
话即长,无话即短。当日施恩正和武松在店里闲坐说话,论些拳棒枪法,只见店门
前两三个军汉,牵着一匹马,来店里寻问主人道:“那个是打虎的武都头?”施恩
却认得是孟州守御兵马都监张蒙方衙内亲随人。施恩便向前问道:“你等寻武都头
则甚?”那军汉说道:“奉都监相公钧旨:闻知武都头是个好男子,特地差我们将
马来取他,相公有钧帖在此。”施恩看了,寻思道:“这张都监是我父亲的上司官,
属他调遣。今者武松又是配来的囚徒,亦属他管下,只得教他去。”施恩便对武松
道:“兄长,这几位郎中,是张都监相公处差来取你。他既着人牵马来,哥哥心下
如何?”武松是个刚直的人,不知委曲,便道:“他既是取我,只得走一遭,看他
有甚话说。”随即换了衣裳巾帻,带了个小伴当,上了马,一同众人,投孟州城里
来。

到得张都监宅前,下了马,跟着那军汉,直到厅前参见那张都监。那张蒙方在
厅上,见了武松来,大喜道:“教进前来相见。”武松到厅下,拜了张都监,叉手
立在侧边。张都监便对武松道:“我闻知你是个大丈夫,男子汉,英雄无敌,敢与
人同死同生。我帐前现缺恁地一个人,不知你肯与我做亲随体己人么?”武松跪下
称谢道:“小人是个牢城营内囚徒。若蒙恩相抬举,小人当以执鞭随镫,伏侍恩相。”
张都监大喜,便叫取果盒酒出来。张都监亲自赐了酒,叫武松吃的大醉。就前厅廊
下,收拾一间耳房,与武松安歇。次日,又差人去施恩处,取了行李来,只在张都
监家宿歇。早晚都监相公,不住地唤武松进后堂与酒与食,放他穿房入户,把做亲
人一般看待。又叫裁缝与武松彻里彻外做秋衣。武松见了,也自欢喜,心内寻思道:
“难得这个都监相公,一力要抬举我。自从到这里住了,寸步不离,又没工夫去快
活林与施恩说话。虽是他频频使人来相看我,多管是不能够入宅里来。”

武松自从在张都监宅里,相公见爱;但是人有些公事来央浼他的,武松对都监
相公说了,无有不依。外人俱送些金银、财帛、缎匹等件。武松买个柳藤箱子,把
这送的东西,都锁在里面,不在话下。

时光迅速,却早又是八月中秋。怎见得中秋好景,但见:

玉露泠泠,金风淅淅。井畔梧桐落叶,池中菡萏成房。新雁声悲,寒蛩韵急。
舞风杨柳半摧残,带雨芙蓉逞娇艳。秋色平分催节序,月轮端正照山河。
当时张都监向后堂深处鸳鸯楼下,安排筵宴,庆赏中秋,叫唤武松到里面饮酒。武
松见夫人宅眷,都在席上,吃了一杯,便待转身出来。张都监唤住武松问道:“你
那里去?”武松答道:“恩相在上:夫人宅眷在此饮宴,小人理合回避。”张都监
大笑道:“差了,我敬你是个义士,特地请将你来一处饮酒,如自家一般,何故却
要回避?”便教坐了。武松道:“小人是个囚徒,如何敢与恩相坐地?”张都监道:
“义士,你如何见外?此间又无外人,便坐不妨。”武松三回五次,谦让告辞,张
都监那里肯放,定要武松一处坐地。武松只得唱个无礼喏,远远地斜着身坐下。张
都监着丫、养娘相劝,一杯两盏。看看饮过五七杯酒,张都监叫抬上果桌饮酒,
又进了一两套食,次说些闲话,问了些枪法。张都监道:“大丈夫饮酒,何用小杯!”
叫取大银赏钟斟酒与义士吃。连珠箭劝了武松几锺。看看月明光彩,照入东窗。武
松吃的半醉,却都忘了礼数,只顾痛饮。张都监叫唤一个心爱的养娘,叫做玉兰,
出来唱曲。那玉兰生得如何,但见:

脸如莲萼,唇似樱桃。两弯眉画远山青,一对眼明秋水润。纤腰袅娜,绿罗裙
掩映金莲;素体馨香,绛纱袖轻笼玉笋。
凤钗斜插笼云髻,象板高擎立玳筵。
那张都监指着玉兰道:“这里别无外人,只有我心腹之人武都头在此。你可唱个中
秋对月时景的曲儿,教我们听则个。”玉兰执着象板,向前各道个万福,顿开喉咙,
唱一只东坡学士中秋《水调歌》,唱道是:

明月几时有?把酒问青天:不知天上宫阙,今夕是何年?我欲乘风归去,只恐琼
楼玉宇,高处不胜寒。起舞弄清影,何似在人间。

高卷珠帘,低绮户,照无眠。
不应有恨,何事常向别时圆?人有悲欢离合,月有阴晴圆缺,此事古难全。但愿人
长久,万里共婵娟。

这玉兰唱罢,放下象板,又各道了一个万福,立在一边。张都监又道:“玉兰,
你可把一巡酒。”这玉兰应了,便拿了一副劝盘,丫斟酒,先递了相公,次劝了
夫人,第三便劝武松饮酒。张都监叫斟满着。武松那里敢抬头?起身远远地接过酒
来,唱了相公、夫人两个大喏,拿起酒来,一饮而尽,便还了盏子。张都监指着玉
兰对武松道:“此女颇有些聪明伶俐,善知音律,极能针指。如你不嫌低微,数日
之间,择了良时,将来与你做个妻室。”武松起身再拜道:“量小人何者之人,怎
敢望恩相宅眷为妻?枉自折武松的草料。”张都监笑道:“我既出了此言,必要与
你。你休推故阻,我必不负约。”

当时一连又饮了十数杯酒。约莫酒涌上来,恐怕失了礼节,便起身拜谢了相公、
夫人,出到前厅廊下房门前。开了门,觉道酒食在腹,未能便睡,去房里脱了衣裳,
除了巾帻,拿条哨棒来厅心里,月明下,使几回棒,打了几个轮头。仰面看天时,
约莫三更时分。武松进到房里,却待脱衣去睡,只听得后堂里一片声叫起“有贼”
来。武松听得道:“都监相公如此爱我,他后堂内里有贼,我如何不去救护?”武
松献勤,提了一条哨棒,径抢入后堂里来。只见那个唱的玉兰,慌慌张张走出来指
道:“一个贼奔入后花园里去了!”武松听得这话,提着哨棒,大踏步直赶入花园
里去寻时,一周遭不见。复翻身却奔出来,不提防黑影里撇出一条板凳,把武松一
交绊翻,走出七八个军汉,叫一声:“捉贼!”就地下把武松一条麻索绑了。武松
急叫道:“是我!”那众军汉那里容他分说?只见堂里灯烛荧煌,张都监坐在厅上,
一片声叫道:“拿将来!”众军汉把武松一步一棍,打到厅前。

武松叫道:“我不是贼,是武松。”张都监看了大怒,变了面皮,喝骂道:“你
这个贼配军!本是个强盗,贼心贼肝的人,我倒要抬举你一力成人,不曾亏负了你
半点儿,却才教你一处吃酒,同席坐地,我指望要抬举,与你个官,你如何却做这
等的勾当?”武松大叫道:“相公,非干我事!我来捉贼,如何倒把我捉了做贼?武
松是个顶天立地的好汉,不做这般的事。”张都监喝道:“你这厮休赖!且把他押
去他房里,搜看有无赃物。”众军汉把武松押着,径到他房里,打开他那柳藤箱子
看时,上面都是些衣服,下面却是些银酒器皿,约有一二百两赃物。武松见了,也
自目睁口呆,只叫得屈。众军汉把箱子抬出厅前,张都监看了大骂道:“贼配军,
如此无礼,赃物正在你箱子里搜出来,如何赖得过!常言道:‘众生好度人难度!’
原来你这厮外貌像人,倒有这等贼心贼肝!既然赃证明白,没话说了。”连夜便把
赃物封了,且叫送去机密房里监收,天明却和这厮说话。武松大叫冤屈,那里肯容
他分说?众军汉扛了赃物,将武松送到机密房里收管了。张都监连夜使人去对知府
说了,押司孔目上下都使用了钱。

次日天明,知府方才坐厅,左右缉捕观察,把武松押至当厅,赃物都扛在厅上。
张都监家心腹人,赍着张都监被盗的文书,呈上知府看了。那知府喝令左右把武松
一索捆翻。牢子节级将一束问事狱具放在面前。武松却待开口分说,知府喝道:“这
厮原是远流配军,如何不做贼,以定是一时见财起意。既是赃证明白,休听这厮胡
说,只顾与我加力打!”那牢子狱卒,拿起批头竹片,雨点地打下来。武松情知不
是话头,只得屈招做:“本月十五日,一时见本官衙内许多银酒器皿,因而起意,
至夜乘势窃取入己。”与了招状。知府道:“这厮正是见财起意,不必说了,且取
枷来钉了监下。”牢子将过长枷,把武松枷了,押下死囚牢里监禁了。诗曰:
都监贪污实可嗟,出妻献婢售奸邪。
如何太守心堪买,也把平人当贼拿。

且说武松下到大牢里,寻思道:“叵耐张都监那厮,安排这般圈套坑陷我。我
若能够挣得性命出去时,却又理会。”牢子狱卒,把武松押在大牢里,将他一双脚
昼夜匣着;又把木钮钉住双手,那里容他些松宽。

话里却说施恩,已有人报知此事,慌忙入城来和父亲商议。老管营道:“眼见
得是张团练替蒋门神报仇,买嘱张都监,却设出这条计策陷害武松。必然是他着人
去上下都使了钱,受了人情贿赂,众人以此不由他分说,必然要害他性命。我如今
寻思起来,他须不该死罪。只是买求两院押牢节级,便好可以存他性命。在外却又
别作商议。”施恩道:“现今当牢节级姓康的,和孩儿最过得好。只得去求浼他如
何?”老管营道:“他是为你吃官司,你不去救他,更待何时?”

施恩将了一二百两银子,径投康节级,却在牢未回。施恩教他家着人去牢里说
知。不多时,康节级归来与施恩相见。施恩把上件事一一告诉了一遍。康节级答道:
“不瞒兄长说:此一件事,皆是张都监和张团练两个,同姓结义做兄弟。现今蒋门
神躲在张团练家里,却央张团练买嘱这张都监,商量设出这条计来,一应上下之人,
都是蒋门神用贿赂,我们都接了他钱。厅上知府,一力与他作主,定要结果武松性
命,只有当案一个叶孔目不肯,因此不敢害他。这人忠直仗义,不肯要害平人,以
此武松还不吃亏。今听施兄所说了,牢中之事,尽是我自维持,如今便去宽他,今
后不教他吃半点儿苦。你却快央人去,只嘱叶孔目,要求他早断出去,便可救得他
性命。”施恩取一百两银子与康节级。康节级那里肯受,再三推辞,方才收了。

施恩相别出门来,径回营里,又寻一个和叶孔目知契的人,送一百两银子与他,
只求早早紧急决断。那叶孔目已知武松是个好汉,亦自有心周全他,已把那文案做
得活着,只被这知府受了张都监贿赂嘱托,不肯从轻。勘来武松窃取人财,又不得
死罪,因此互相延挨,只要牢里谋他性命。今来又得了这一百两银子,亦知是屈陷
武松,却把这文案都改得轻了,尽出豁了武松,只待限满决断。有诗为证:
赃吏纷纷据要津,公然白日受黄金。
西厅孔目心如水,不把真心作贼心。

且说施恩于次日安排了许多酒馔,甚是齐备,来央康节级引领,直进大牢里看
视武松,见面送饭。此时武松已自得康节级看觑,将这刑禁都放宽了。施恩又取三
二十两银子,分与众小牢子。取酒食叫武松吃了,施恩附耳低言道:“这场官司,
明明是都监替蒋门神报仇,陷害哥哥。你且宽心,不要忧念。我已央人和叶孔目说
通了,甚有周全你的好意。且待限满断决你出去,却再理会。”此时武松得松宽了,
已有越狱之心;听得施恩说罢,却放了那片心。施恩在牢里安慰了武松,归到营中。

过了两日,施恩再备些酒食钱财,又央康节级引领入牢里,与武松说话。相见
了,将酒食管待,又分了些零碎银子与众人做酒钱。回归家来,又央浼人上下去
使用,催趱打点文书。过得数日,施恩再备了酒肉,做了几件衣裳,再央康节级维
持,相引将来牢里,请众人吃酒,买求看觑武松,叫他更换了些衣服,吃了酒食。
出入情熟,一连数日,施恩来了大牢里三次。却不提防被张团练家心腹人见了,回
去报知。那张团练便去对张都监说了其事。张都监却再使人送金帛来与知府,就说
与此事。那知府是个赃官,接受了贿赂,便差人常常下牢里来闸看,但见闲人,便
要拿问。施恩得知了,那里敢再去看觑?武松却自得康节级和众牢子自照管他。施
恩自此早晚只去得康节级家里讨信,得知长短,都不在话下。

看看前后将及两月。有这当案叶孔目一力主张,知府处早晚说开就里。那知府
方才知道张都监接受了蒋门神若干银子,通同张团练,设计排陷武松,自心里想道:
“你倒赚了银两,教我与你害人!”因此心都懒了,不来管看。

捱到六十日限满,牢中取出武松,当厅开了枷。当案叶孔目读了招状,就拟下
罪名,脊杖二十,刺配恩州牢城,原盗赃物,给还本主。张都监只得着家人当官领
了赃物。当厅把武松断了二十脊杖,刺了金印,取一面七斤半铁叶盘头枷钉了,押
一纸公文,差两个壮健公人,防送武松,限了时日要起身。那两个公人,领了牒文,
押解了武松出孟州衙门便行。原来武松吃断棒之时,却得老管营使钱通了,叶孔目
又看觑他,知府亦知他被陷害,不十分来打重,因此断得棒轻。

武松忍着那口气,带上行枷,出得城来,两个公人监在后面。约行得一里多路,
只见官道旁边酒店里钻出施恩来,看着武松道:“小弟在此专等。”武松看施恩时,
又包着头,络着手臂。武松问道:“我好几时不见你,如何又做恁地模样?”施恩
答道:“实不相瞒哥哥说:小弟自从牢里三番相见之后,知府得知了,不时差人下
来牢里点闸,那张都监又差人在牢门口左右两边巡看着,因此小弟不能够再进大牢
里看望兄长,只到得康节级家里讨信。半月之前,小弟正在快活林中店里,只见蒋
门神那厮,又领着一伙军汉到来厮打。小弟被他又痛打一顿,也要小弟央浼人陪话,
却被他仍复夺了店面,依旧交还了许多家火什物。小弟在家将息未起,今日听得哥
哥断配恩州,特有两件绵衣,送与哥哥路上穿着。煮得两只熟鹅在此,请哥哥吃了
两块去。”

施恩便邀两个公人,请他入酒肆,那两个公人那里肯进酒店里去?便发言发语
道:“武松这厮,他是个贼汉,不争我们吃你的酒食,明日官府上须惹口舌。你若
怕打,快走开去!”施恩见不是话头,便取十来两银子,送与他两个公人。那厮两
个,那里肯接?恼忿忿地,只要催促武松上路。施恩讨两碗酒,叫武松吃了,把一
个包裹拴在武松腰里,把这两只熟鹅挂在武松行枷上。施恩附耳低言道:“包裹里
有两件绵衣,一帕子散碎银子,路上好做盘缠。也有两只八搭麻鞋在里面。只是要
路上仔细提防,这两个贼男女,不怀好意。”武松点头道:“不须分付,我已省得
了。再着两个来,也不惧他。你自回去将息。且请放心,我自有措置。”施恩拜辞
了武松,哭着去了,不在话下。

武松和两个公人上路,行不到数里之上,两个公人悄悄地商议道:“不见那两
个来。”武松听了,自暗暗地寻思,冷笑道:“没你娘鸟兴,那厮倒来扑复老爷!”
武松右手却吃钉住在行枷上,左手却散着。武松就枷上取下那熟鹅来,只顾自吃,
也不睬那两个公人。又行了四五里路,再把这只熟鹅除来,右手扯着,把左手撕来,
只顾自吃。行不过五里路,把这两只熟鹅都吃尽了。约莫离城也有八九里多路,只
见前面路边,先有两个人,提着朴刀,各跨口腰刀,先在那里等候。见了公人监押
武松到来,便帮着一路走。武松又见这两个公人,与那两个提朴刀的挤眉弄眼,打
些暗号。武松早睃见,自瞧了八分尴尬,只安在肚里,却且只做不见。

又走不数里多路,只见前面来到一处济济荡荡鱼浦,四面都是野港阔河。五个
人行至浦边一条阔板桥,一座牌楼上有牌额写着道“飞云浦”三字。武松见了,假
意问道:“这里地名,唤做甚么去处?”两个公人应道:“你又不眼瞎,须见桥边
牌额上写道‘飞云浦’。”武松站住道:“我要净手则个。”那两个提朴刀的走近
一步,却被武松叫声:“下去!”一飞脚早踢中,翻筋斗踢下水去了。这一个急待
转身,武松右脚早起,扑通地也踢下水里去。那两个公人慌了,望桥下便走。武松
喝一声:“那里去!”把枷只一扭,折做两半个,赶将下桥来。那两个先自惊倒了
一个。武松奔上前去,望那一个走的后心上,只一拳打翻,就水边拿起朴刀来,赶
上去,搠上几朴刀,死在地下,却转身回来,把那个惊倒的,也搠几刀。

这两个踢下水去的,才挣得起,正待要走,武松追着,又砍倒一个,赶入一步,
劈头揪住一个喝道:“你这厮实说,我便饶你性命!”那人道:“小人两个是蒋门
神徒弟。今被师父和张团练定计,使小人两个来相帮防送公人,一处来害好汉。”
武松道:“你师父蒋门神今在何处?”那人道:“小人临来时,和张团练都在张都
监家里后堂鸳鸯楼上吃酒,专等小人回报。”武松道:“原来恁地,却饶你不得。”
手起刀落,也把这人杀了。解下他腰刀来,拣好的带了一把,将两个尸首,都撺在
浦里。又怕那两个不死,提起朴刀,每人身上又搠了几刀。立在桥上看了一会,思
量道:“虽然杀了四个贼男女,不杀得张都监、张团练、蒋门神,如何出得这口恨
气!”提着朴刀,踌躇了半晌,一个念头,竟奔回孟州城里来。

不因这番,有分教:武松杀几个贪夫,出一口怨气。定教:画堂深处尸横地,
红烛光中血满楼。

毕竟武松再回孟州城来,怎地结果,且听下回分解。