\chapter{浔阳楼宋江吟反诗~梁山泊戴宗传假信}

话说当下李逵把指头捺倒了那女娘,酒店主人拦住说道:“四位官人如何是
好?”主人心慌,便叫酒保过卖都向前来救他,就地下把水喷,看看苏醒,扶将
起来。看时,额角上抹脱了一片油皮,因此那女子晕昏倒了,救得醒来,千好万好。
他的爹娘听得说是黑旋风,先是惊得呆了半晌,那里敢说一言?看那女子,已自说
得话了,娘母取个手帕,自与他包了头,收拾了钗。宋江问道:“你姓甚么?那
里人家?”那老妇人道:“不瞒官人说,老身夫妻两口儿,姓宋,原是京师人。只
有这个女儿,小字玉莲,他爹自教得他几个曲儿,胡乱叫他来这琵琶亭上卖唱养口。
为他性急,不看头势,不管官人说话,只顾便唱,今日这哥哥失手,伤了女儿些个,
终不成经官动词,连累官人。”宋江见他说得本分,便道:“你着甚人跟我到营里,
我与你二十两银子,将息女儿,日后嫁个良人,免在这里卖唱。”那夫妻两口儿便
拜谢道:“怎敢指望许多!”宋江道:“我说一句是一句,并不会说慌。你便叫你
老儿自跟我去讨与他。”那夫妻二人拜谢道:“深感官人救济。”

戴宗埋怨李逵道:“你这厮要便与人合口,又教哥哥坏了许多银子。”李逵道:
“只指头略擦得一擦,他自倒了,不曾见这般鸟女子恁地娇嫩。你便在我脸上打一
百拳,也不妨。”宋江等众人都笑起来。张顺便叫酒保去说,这席酒钱我自还他。
酒保听得道:“不妨,不妨!只顾去。”宋江那里肯,便道:“兄弟,我劝二位来
吃酒,倒要你还钱!”张顺苦死要还,说道:“难得哥哥会面,仁兄在山东时,小
弟哥儿两个也兀自要来投奔哥哥,今日天幸得识尊颜,权表薄意,非足为礼。”戴
宗道:“公明兄长,既然是张二哥相敬之心,只得曲允。”宋江道:“既然兄弟还
了,改日却另置杯复礼。”张顺大喜,就将了两尾鲤鱼,和戴宗、李逵带了这个宋
老儿,都送宋江离了琵琶亭,来到营里,五个人都进抄事房里坐下。宋江先取两锭
小银二十两,与了宋老儿,那老儿拜谢了去,不在话下。天色已晚,张顺送了鱼,
宋江取出张横书,付与张顺,相别去了。宋江又取出五十两一锭大银对李逵道:“兄
弟,你将去使用。”戴宗、李逵也自作别,赶入城去了。

只说宋江把一尾鱼送与管营,留一尾自吃。宋江因见鱼鲜,贪爱爽口,多吃了
些,至夜四更,肚里绞肠刮肚价疼;天明时,一连泻了二十来遭,昏晕倒了,睡在
房中。宋江为人最好,营里众人都来煮粥烧汤,看觑伏侍他。次日,张顺因见宋江
爱鱼吃,又将得好金色大鲤鱼两尾送来,就谢宋江寄书之义,却见宋江破腹,泻倒
在床,众囚徒都在房里看视。张顺见了,要请医人调治。宋江道:“自贪口腹,吃
了些鲜鱼,坏了肚腹,你只与我赎一贴止泻六和汤来吃便好了。”叫张顺把这两尾
鱼,一尾送与王管营,一尾送与赵差拨。张顺送了鱼,就赎了一贴六和汤药来与宋
江了自回去,不在话下。营内自有众人煎药伏侍。次日,戴宗、李逵备了酒肉,径
来抄事房看望宋江。只见宋江暴病才可,吃不得酒肉,两个自在房面前吃了,直至
日晚,相别去了,亦不在话下。

只说宋江自在营中将息了五七日,觉得身体没事,病症已痊,思量要入城中去
寻戴宗。又过了一日,不见他一个来。次日早膳罢,辰牌前后,揣了些银子,锁上
房门,离了营里。信步出街来,径走入城,去州衙前左边寻问戴院长家。有人说道:
“他又无老小,只在城隍庙间壁观音庵里歇。”宋江听了,寻访直到那里,已自锁
了门出去了。却又来寻问黑旋风李逵时,多人说道:“他是个没头神,又无家室,
只在牢里安身。没地里的巡检,东边歇两日,西边歪几时,正不知他那里是住处。”
宋江又寻问卖鱼牙子张顺时,亦有人说道:“他自在城外村里住,便自卖鱼时,也
只在城外江边,只除非讨赊钱入城来。”

宋江听罢,又寻出城来,直要问到那里,独自一个闷闷不已,信步再出城外来,
看见那一派江景非常,观之不足。正行到一座酒楼前过,仰面看时,旁边竖着一根
望竿,悬挂着一个青布酒旆子,上写道:“浔阳江正库。”雕檐外一面牌额,上有
苏东坡大书“浔阳楼”三字。宋江看了,便道:“我在郓城县时,只听得说江州好
座浔阳楼,原来却在这里!我虽独自一个在此,不可错过,何不且上楼去自己看玩
一遭?”宋江来到楼前看时,只见门边朱红华表,柱上两面白粉牌,各有五个大字,
写道:“世间无比酒,天下有名楼。”宋江便上楼来,去靠江占一座阁子里坐了。
凭阑举目看时,端的好座酒楼。但见:

雕檐映日,画栋飞云。碧阑干低接轩窗,翠帘幕高悬户牖。消磨醉眼,倚青天
万迭云山;勾惹吟魂,翻瑞雪一江烟水。白苹渡口,时闻渔父鸣榔;红蓼滩头,每
见钓翁击楫。楼畔绿槐啼野鸟,门前翠柳系花骢。

宋江看罢,喝采不已。酒保上楼来问道:“官人还是要待客,只是自消遣?”
宋江道:“要待两位客人,未见来,你且先取一樽好酒,果品、肉食只顾卖来,鱼
便不要。”酒保听了,便下楼去。少时,一托盘把上楼来,一樽蓝桥风月美酒,摆
下菜蔬,时新果品、按酒,列几般肥羊、嫩鸡、酿鹅、精肉,尽使朱红盘碟。宋江
看了,心中暗喜,自夸道:“这般整齐肴馔,济楚器皿,端的是好个江州!我虽是
犯罪远流到此,却也看了些真山真水。我那里虽有几座名山古迹,却无此等景致。”
独自一个,一杯两盏,倚阑畅饮,不觉沉醉,猛然蓦上心来,思想道:“我生在山
东,长在郓城,学吏出身,结识了多少江湖好汉,虽留得一个虚名,目今三旬之上,
名又不成,功又不就,倒被文了双颊,配来在这里。我家乡中老父和兄弟,如何得
相见?”不觉酒涌上来,潸然泪下,临风触目,感恨伤怀。忽然做了一首《西江月》
词,便唤酒保索借笔砚来。起身观玩,见白粉壁上多有先人题咏,宋江寻思道:“何
不就书于此?倘若他日身荣,再来经过,重睹一番,以记岁月,想今日之苦。”乘
着酒兴,磨得墨浓,蘸得笔饱,去那白粉壁上挥毫便写道:

自幼曾攻经史,长成亦有权谋。恰如猛虎卧荒丘,潜伏爪牙忍受。

不幸刺
文双颊,那堪配在江州。他年若得报冤仇,血染浔阳江口!
宋江写罢,自看了大喜大笑,一面又饮了数杯酒,不觉欢喜,自狂荡起来,手舞足
蹈,又拿起笔来,去那《西江月》后再写下四句诗,道是:
心在山东身在吴,飘蓬江海谩嗟吁。
他时若遂凌云志,敢笑黄巢不丈夫!
宋江写罢诗,又去后面大书五字道:“郓城宋江作。”写罢,掷笔在桌上,又自歌
了一回。再饮过数杯酒,不觉沉醉,力不胜酒,便唤酒保计算了,取些银子算还,
多的都赏了酒保,拂袖下楼来。踉踉跄跄,取路回营里来。开了房门,便倒在床上,
一觉直睡到五更。酒醒时,全然不记得昨日在浔阳江楼上题诗一节。当时害酒,自
在房里睡卧,不在话下。

且说这江州对岸,另有个城子,唤做无为军,却是个野去处。城中有个在闲通
判,姓黄,双名文炳。这人虽读经书,却是阿谀谄佞之徒,心地匾窄,只要嫉贤妒
能,胜如己者害之,不如己者弄之,专在乡里害人。闻知这蔡九知府是当朝蔡太师
儿子,每每来浸润他,时常过江来谒访知府,指望他引荐出职,再欲做官。也是宋
江命运合当受苦,撞了这个对头。

当日这黄文炳在私家闲坐,无可消遣,带了两个仆人,买了些时新礼物,自家
一只快船渡过江来,径去府里探望蔡九知府。恰恨撞着府里公宴,不敢进去。却再
回船,正好那只船仆人已缆在浔阳楼下。黄文炳因见天气暄热,且去楼上闲玩一回。
信步入酒库里来,看了一遭,转到酒楼上,凭栏消遣,观见壁上题咏甚多,也有做
得好的,亦有歪谈乱道的。黄文炳看了冷笑。正看到宋江题《西江月》词,并所吟
四句诗,大惊道:“这个不是反诗?谁写在此?”后面却书道“郓城宋江作”五个
大字。黄文炳再读道:“自幼曾攻经史,长成亦有权谋。”冷笑道:“这人自负不
浅。”又读道:“恰如猛虎卧荒丘,潜伏爪牙忍受。”黄文炳道:“那厮也是个不
依本分的人。”又读:“不幸刺文双颊,那堪配在江州。”黄文炳道:“也不是个
高尚其志的人,看来只是个配军。”又读道:“他年若得报冤仇,血染浔阳江口。”
黄文炳道:“这厮报仇兀谁?却要在此生事!量你是个配军,做得甚用!”又读诗道:
“心在山东身在吴,飘蓬江海谩嗟吁。”黄文炳道:“这两句兀自可恕。”又读道:
“他时若遂凌云志,敢笑黄巢不丈夫!”黄文炳摇着头道:“这厮无礼,他却要赛
过黄巢,不谋反待怎地?”再看了“郓城宋江作”。黄文炳道:“我也多曾闻这个
名字,那人多管是个小吏。”便唤酒保来问道:“作这两篇诗词,端的是何人题下
在此?”酒保道:“夜来一个人独自吃了一瓶酒,醉后疏狂,写在这里。”黄文炳
道:“约莫甚么样人?”酒保道:“面颊上有两行金印,多管是牢城营内人。生得
黑矮肥胖。”黄文炳道:“是了。”就借笔砚取幅纸来抄了,藏在身边,分付酒保
休要刮去了。

黄文炳下楼,自去船中歇了一夜。次日饭后,仆人挑了盒仗,一径又到府前,
正值知府退堂在衙内,使人入去报复。多样时,蔡九知府遣人出来,邀请在后堂。
蔡九知府却出来与黄文炳叙罢寒温已毕,送了礼物,分宾坐下。黄文炳禀说道:“文
炳夜来渡江到府拜望,闻知公宴,不敢擅入,今日重复拜见恩相。”蔡九知府道:
“通判乃是心腹之交,径入来同坐何妨!下官有失迎迓。”左右执事人献茶。茶罢,
黄文炳道:“相公在上,不敢拜问,不知近日尊府太师恩相曾使人来否?”知府道:
“前日才有书来。”黄文炳道:“不敢动问,京师近日有何新闻?”知府道:“家
尊写来书上分付道:近日太史院司天监奏道,夜观天象,罡星照临吴、楚,敢有作
耗之人,随即体察剿除。更兼街市小儿谣言四句道:‘耗国因家木,刀兵点水工。
纵横三十六,播乱在山东。’因此嘱付下官,紧守地方。”黄文炳寻思了半晌,笑
道:“恩相,事非偶然也!”黄文炳袖中取出所抄之诗,呈与知府道:“不想却在
此处。”蔡九知府看了道:“这是个反诗,通判那里得来?”黄文炳道:“小生夜
来不敢进府,回至江边,无可消遣,却去浔阳楼上避热闲玩,观看前人吟咏,只见
白粉壁上,新题下这篇。”知府道:“却是何等样人写下?”

黄文炳回道:“相公,上面明题着姓名,道是‘郓城宋江作’。”知府道:“这
宋江却是甚么人?”黄文炳道:“他分明写着‘不幸刺文双颊,那堪配在江州’。
眼见得只是个配军,牢城营犯罪的囚徒。”知府道:“量这个配军,做得甚么!”
黄文炳道:“相公不可小觑了他。恰才相公所言尊府恩相家书说小儿谣言,正应在
本人身上。”知府道:“何以见得?”黄文炳道:“‘耗国因家木’,耗散国家钱
粮的人,必是‘家’头着个‘木’字,明明是个‘宋’字;第二句‘刀兵点水工’,
兴起刀兵之人,水边着个‘工’字,明是个‘江’字。这个人姓宋,名江,又作下
反诗,明是天数,万民有福。”知府又问道:“何谓‘纵横三十六,播乱在山东’?”
黄文炳答道:“或是六六之年,或是六六之数;‘播乱在山东’,今郓城县正是山
东地方。这四句谣言,已都应了。”知府又道:“不知此间有这个人么?”黄文炳
回道:“小生夜来问那酒保时,说道这人只是前日写下了去。这个不难,只取牢城
营文册一查,便见有无。”知府道:“通判高见极明。”便唤从人叫库子取过牢城
营里文册簿来看。当时从人于库内取至文册,蔡九知府亲自检看,见后面果有五月
间新配到囚徒一名“郓城县宋江”。黄文炳看了道:“正是应谣言的人,非同小可。
如是迟缓,诚恐走透了消息,可急差人捕获,下在牢里,却再商议。”知府道:“言
之极当。”随即升厅,叫唤两院押牢节级过来。厅下戴宗声喏。知府道:“你与我
带了做公的人,快下牢城营里,捉拿浔阳楼吟反诗的犯人郓城县宋江来,不可时刻
违误。”

戴宗听罢,吃了一惊,心里只叫得苦。随即出府来,点了众节级牢子,都叫各
去家里取了各人器械:“来我下处间壁城隍庙里取齐。”戴宗分付了众人,各自归
家去,戴宗却自作起神行法,先来到牢城营里,径入抄事房,推开门看时,宋江正
在房里,见是戴宗入来,慌忙迎接,便道:“我前日入城来,那里不寻遍。因贤弟
不在,独自无聊,自去浔阳楼上饮了一瓶酒。这两日迷迷不好,正在这里害酒。”
戴宗道:“哥哥,你前日却写下甚言语在楼上?”宋江道:“醉后狂言,谁个记得。”
戴宗道:“却才知府唤我当厅发落,叫多带从人,‘拿捉浔阳楼上题反诗的犯人郓
城县宋江正身赴官’。兄弟吃了一惊,先去稳住众做公的在城隍庙等候。如今我特
来先报知哥哥,却是怎地好?如何解救?”宋江听罢,搔头不知痒处,只叫得苦:
“我今番必是死也!”戴宗道:“我教仁兄一着解手,未知如何?如今小弟不敢耽
搁,回去便和人来捉你,你可披乱了头发,把尿屎泼在地上,就倒在里面,诈作风
魔。我和众人来时,你便口里胡言乱语,只做失心风便好,我自去替你回复知府。”
宋江道:“感谢贤弟指教,万望维持则个。”

戴宗慌忙别了宋江,回到城里,径来城隍庙,唤了众做公的,一直奔入牢城营
里来,假意喝问:“那个是新配来的宋江?”牌头引众人到抄事房里,只见宋江披
散头发,倒在尿屎坑里滚,见了戴宗和做公的人来,便说道:“你们是甚么鸟人?”
戴宗假意大喝一声:“捉拿这厮!”宋江白着眼,却乱打将来,口里乱道:“我是
玉皇大帝的女婿。丈人教我领十万天兵来杀你江州人,阎罗大王做先锋,五道将军
做合后,与我一颗金印,重八百余斤,杀你这般鸟人!”众做公的道:“原来是个
失心风的汉子,我们拿他去何用?”戴宗道:“说得是。我们且去回话,要拿时再
来。”众人跟了戴宗回到州衙里,蔡九知府在厅上专等回报。戴宗和众做公的在厅
下回复知府道:“原来这宋江是个失心风的人。尿屎秽污全不顾,口里胡言乱语,
浑身臭粪不可当,因此不敢拿来。”

蔡九知府正待要问缘故时,黄文炳早在屏风背后转将出来,对知府道:“休信
这话。本人作的诗词,写的笔迹,不是有风症的人,其中有诈。好歹只顾拿来。便
走不动,扛也扛将来。”蔡九知府道:“通判说得是。”便发落戴宗:“你们不拣
怎地,只与我拿得来。”

戴宗领了钧旨,只叫得苦!再将带了众人下牢城营里来,对宋江道:“仁兄,
事不谐矣。兄长只得去走一遭。”便把一个大竹箩,扛了宋江,直抬到江州府里,
当厅歇下。知府道:“拿过这厮来。”众做公的把宋江押于阶下。宋江那里肯跪,
睁着眼,见了蔡九知府道:“你是甚么鸟人,敢来问我!我是玉皇大帝的女婿。丈
人教我引十万天兵,杀你江州人,阎罗大王做先锋,五道将军做合后,有一颗金印,
重八百余斤。你也快躲了我,不时,教你们都死!”

蔡九知府看了,没做理会处。黄文炳又对知府道:“且唤本营差拨并牌头来问,
这人来时有风,近日却才风?若是来时风,便是真症候;若是近日才风,必是诈风。”
知府道:“言之极当。”便差人唤到管营、差拨,问他两个时,那里敢隐瞒,只得
直说道:“这人来时不见有风病,敢只是近日举发此症。”知府听了,大怒。唤过
牢子狱卒,把宋江捆翻,一连打上五十下,打得宋江一佛出世,二佛涅,皮开肉
绽,鲜血淋漓。戴宗看了,只叫得苦,又没做道理救他处。宋江初时也胡言乱语,
次后吃拷打不过,只得招道:“自不合一时酒后,误写反诗,别无主意。”蔡九知
府即取了招状,将一面二十五斤死囚枷枷了,推放大牢里收禁。宋江吃打得两腿走
不动,当厅钉了,直押赴死囚牢里来。却得戴宗一力维持,分付了众小牢子,都教
好觑此人。戴宗自安排饭食,供给宋江,不在话下。

再说蔡九知府退厅,邀请黄文炳到后堂称谢道:“若非通判高明远见,下官险
些儿被这厮瞒过了。”黄文炳又道:“相公在上,此事也不宜迟。只好急急修一封
书,便差人星夜上京师,报与尊府恩相知道,显得相公干了这件国家大事。就一发
禀道:‘若要活的,便着一辆陷车解上京;如不要活的,恐防路途走失,就于本处
斩首号令,以除大害。’便是今上得知必喜。”蔡九知府道:“通判所言有理,下
官即日也要使人回家,书上就荐通判之功,使家尊面奏天子,早早升授富贵城池,
去享荣华。”黄文炳拜谢道:“小生终身皆依托门下,自当衔环背鞍之报。”黄文
炳就撺掇蔡九知府写了家书,印上图书。黄文炳问道:“相公差那个心腹人去?”
知府道:“本州自有个两院节级,唤做戴宗,会使神行法,一日能行八百里路程,
只来早便差此人径往京师,只消旬日,可以往回。”黄文炳道:“若得如此之快,
最好,最好!”蔡九知府就后堂置酒,管待了黄文炳,次日相辞知府,自回无为军
去了。

且说蔡九知府安排两个信笼,打点了金珠宝贝玩好之物,上面都贴了封皮。次
日早晨,唤过戴宗到后堂嘱付道:“我有这般礼物,一封家书,要送上东京太师府
里去,庆贺我父亲六月十五日生辰。日期将近,只有你能干去得。你休辞辛苦,可
与我星夜去走一遭,讨了回书便转来,我自重重的赏你。你的程途,都在我心上。
我已料着你神行的日期,专等你回报。切不可沿途耽搁,有误事情。”

戴宗听了,不敢不依。只得领了家书、信笼,便拜辞了知府,挑回下处安顿了,
却来牢里对宋江说道:“哥哥放心,知府差我上京师去,只旬日之间便回。就太师
府里使些见识,解救哥哥的事。每日饭食,我自分付在李逵身上,委着他安排送来,
不教有缺。仁兄且宽心守耐几日。”宋江道:“望烦贤弟救宋江一命则个。”戴宗
叫过李逵,当面分付道:“你哥哥误题了反诗,在这里吃官司,未知如何。我如今
又吃差往东京去,早晚便回。哥哥饭食,朝暮全靠着你看觑他则个。”李逵应道:
“吟了反诗,打甚么鸟紧!万千谋反的,倒做了大官。你自放心东京去,牢里谁敢
奈何他!好便好,不好,我使老大斧头砍他娘!”戴宗临行又嘱付道:“兄弟小心,
不要贪酒,失误了哥哥饭食。休得出去醉了,饿着哥哥。”李逵道:“哥哥,你
自放心去。若是这等疑忌时,兄弟从今日就断了酒,待你回来却开。早晚只在牢里
伏侍宋江哥哥,有何不可?”戴宗听了,大喜道:“兄弟若得如此发心,坚意守看
哥哥更好。”当日作别自去了。李逵真个不吃酒,早晚只在牢里伏侍宋江,寸步不
离。

不说李逵自看觑宋江,且说戴宗回到下处,换了腿、护膝、八答麻鞋,穿上
杏黄衫,整了膊,腰里插了宣牌,换了巾帻,便袋里藏了书信盘缠,挑上两个信
笼,出到城外,身边取出四个甲马,去两只腿上,每只各拴两个,口里念起神行法
咒语来。怎见得神行法效验?

仿佛浑如驾雾,依稀好似腾云。如飞两脚荡红尘,越岭登山去紧。顷刻才离乡
镇,片时又过州城。金钱甲马果通神,千里如同眼近。
当日戴宗离了江州,一日行到晚,投客店安歇,解下甲马,取数陌金纸烧送了。过
了一宿,次日早起来,吃了酒食,离了客店,又拴上四个甲马,挑起信笼,放开脚
步便行。端的是耳边风雨之声,脚不点地。路上略吃些素饭、素酒、点心又走。看
看日暮,戴宗早歇了,又投客店宿歇一夜。次日起个五更,赶早凉行,拴上甲马,
挑上信笼,又走。约行过了三二百里,已是巳牌时分,不见一个干净酒店。

此时正是六月初旬天气,蒸得汗雨淋漓,满身蒸湿,又怕中了暑气。正饥渴之
际,早望见前面树林侧首一座傍水临湖酒肆,戴宗拈指间走到跟前,看时,干干净
净有二十付座头,尽是红油桌凳,一带都是槛窗。戴宗挑着信笼入到里面,拣一付
稳便座头,歇下信笼,解下腰里膊,脱下杏黄衫,喷口水晾在窗栏上。戴宗坐下,
只见个酒保来问道:“上下,打几角酒?要甚么肉食下酒,或猪、羊、牛肉?”戴
宗道:“酒便不要多,与我做口饭来吃。”酒保又道:“我这里卖酒卖饭,又有馒
头粉汤。”戴宗道:“我却不吃荤腥,有甚么素汤下饭?”酒保道:“加料麻辣
豆腐如何?”戴宗道:“最好,最好!”酒保去不多时,一碗豆腐,放两碟菜蔬,
连筛三大碗酒来。戴宗正饥又渴,一上把酒和豆腐都吃了,却待讨饭吃,只见天旋
地转,头晕眼花,就凳边便倒。酒保叫道:“倒了!”只见店里走出一个人来,怎
生模样。但见:
臂阔腿长腰细,待客一团和气。
梁山作眼英雄,旱地忽律朱贵。

当下朱贵从里面出来,说道:“且把信笼将入去,先搜那厮身边,有甚东西。”
便有两个火家去他身上搜看,只见便袋里搜出一个纸包,包着一封书,取过来,递
与朱头领。朱贵扯开,却是一封家书,见封皮上面写道:“平安家信,百拜奉上父
亲大人膝下,男蔡德章谨封。”朱贵便拆开,从头看去,见上面写道:“现今拿得
应谣言题反诗山东宋江监收在牢一节,听候施行。”

朱贵看罢,惊得呆了,半晌则声不得。火家正把戴宗扛起来,背入杀人作房里
去开剥,只见凳头边溜下膊,上挂着朱红绿漆宣牌。朱贵拿起来看时,上面雕着
银字道是:“江州两院押牢节级戴宗”。朱贵看了道:“且不要动手,我常听的军
师说这江州有个神行太保戴宗,是他至爱相识。莫非正是此人?如何倒送书去害宋
江?这一段事,却又天幸撞在我手里。”叫火家:“且与我把解药救醒他来,问个
虚实缘由。”

当时火家把水调了解药,扶起来,灌将下去。须臾之间,只见戴宗舒眉展眼,
便爬起来。却见朱贵拆开家书在手里看,戴宗便喝道:“你是甚人?好大胆,却把
蒙汗药麻翻了我!如今又把太师府书信擅开拆,毁了封皮,却该甚罪?”朱贵笑道:
“这封鸟书,打甚么不紧!休说拆开了太师府书札,俺这里兀自要和大宋皇帝做个
对头的。”戴宗听了大惊,便问道:“好汉,你却是谁?愿求大名。”朱贵答道:
“俺这里行不更名,坐不改姓,梁山泊好汉旱地忽律朱贵的便是。”戴宗道:“既
然是梁山泊头领时,定然认得吴学究先生。”朱贵道:“吴学究是俺大寨里军师,
执掌兵权。足下如何认得他?”戴宗道:“他和小可至爱相识。”朱贵道:“兄长
莫非是军师常说的江州神行太保戴院长么?”戴宗道:“小可便是。”朱贵又问道:
“前者宋公明断配江州,经过山寨,吴军师曾寄一封书与足下,如今却缘何倒去害
宋三郎性命?”戴宗道:“宋公明和我又是至爱兄弟,他如今为吟了反诗,救他不
得。我如今正要往京师寻门路救他,如何肯害他性命?”朱贵道:“你不信,请看
蔡九知府的来书。”戴宗看了,自吃一惊,却把吴学究初寄的书,与宋公明相会的
话,并宋江在浔阳楼醉后误题反诗一事,备细说了一遍。朱贵道:“既然如此,请
院长亲到山寨里,与众头领商议良策,可救宋公明性命。”

朱贵慌忙叫备分例酒食,管待了戴宗,便向水亭上,觑着对港,放了一枝号箭。
响箭到处,早有小喽罗摇过船来。朱贵便同戴宗带了信笼下船,到金沙滩上岸,引
至大寨。吴用见报,连忙下关迎接。见了戴宗,叙礼道:“间别久矣!今日甚风吹
得到此?且请到大寨里来,与众头领相见了。”朱贵说起戴宗来的缘故,如今宋公
明现监在彼。晁盖听得,慌忙请戴院长坐地,备问宋三郎吃官司为甚么事起。戴宗
却把宋江吟反诗的事,一一说了。晁盖听罢大惊,便要起请众头领点了人马,下山
去打江州,救取宋三郎上山。吴用谏道:“哥哥不可造次!江州离此间路远,军马
去时,诚恐因而惹祸,打草惊蛇,倒送宋公明性命。此一件事,不可力敌,只可智
取。吴用不才,略施小计,只在戴院长身上,定要救宋三郎性命。”晁盖道:“愿
闻军师妙计。”吴学究道:“如今蔡九知府却差院长送书上东京,去讨太师回报,
只这封书上,将计就计,写一封假回书,教院长回去。书上只说,‘教把犯人宋江
切不可施行,便须密切差的当人员解赴东京,问了详细,定行处决示众,断绝童谣’。
等他解来此间经过,我这里自差人下山夺了。此计如何?”晁盖道:“倘若不从这
里过时,却不误了大事!”公孙胜便道:“这个何难。我们自着人去远近探听,遮
莫从那里过,务要等着,好歹夺了。只怕不能够他解来。”

晁盖道:“好却是好,只是没人会写蔡京笔迹。”吴学究道:“吴用已思量心
里了。如今天下盛行四家字体,是苏东坡、黄鲁直、米元章、蔡京四家字体。——
苏、黄、米、蔡,宋朝‘四绝’。小生曾和济州城里一个秀才做相识。那人姓萧,
名让。因他会写诸家字体,人都唤他做圣手书生;又会使枪弄棒,舞剑抡刀。吴用
知他写得蔡京笔迹,不若央及戴院长就到他家赚道:‘泰安州岳庙里要写道碑文,
先送五十两银子在此,作安家之资。’便要他来。随后却使人赚了他老小上山,就
教本人入伙,如何?”晁盖道:“书有他写,便好了,也须要使个图书印记。”吴
学究又道:“小生再有个相识,亦思量在肚里了。这人也是中原一绝,现在济州城
里居住。本身姓金,双名大坚,开得好石碑文,剔得好图书、玉石、印记;亦会枪
棒厮打。因为他雕得好玉石,人都称他做玉臂匠。也把五十两银去,就赚他来镌碑
文;到半路上,却也如此行便了。这两个人,山寨里亦有用他处。”晁盖道:“妙
哉!”当日且安排筵席,管待戴宗,就晚歇了。

次日早饭罢,烦请戴院长打扮做太保模样,将了一二百两银子,拴上甲马便下
山,把船渡过金沙滩上岸,拽开脚步,奔到济州来。没两个时辰,早到城里,寻问
圣手书生萧让住处,有人指道:“只在州衙东首文庙前居住。”戴宗径到门首,咳
嗽一声,问道:“萧先生有么?”只见一个秀才从里面出来。见了戴宗,却不认得,
便问道:“太保何处?有甚见教?”戴宗施礼罢,说道:“小可是泰安州岳庙里打
供太保,今为本庙重修五岳楼,本州上户要刻道碑文,特地教小可赍白银五十两,
作安家之资,请秀才便挪尊步,同到庙里作文则个。选定了日期,不可迟滞。”萧
让道:“小生只会作文及书丹,别无甚用。如要立碑,还用刊字匠作。”戴宗道:
“小可再有五十两白银,就要请玉臂匠金大坚刻石。拣定了好日,万望指引,寻了
同行。”

萧让得了五十两银子,便和戴宗同来寻请金大坚。正行过文庙,只见萧让把手
指道:“前面那个来的,便是玉臂匠金大坚。”当下萧让唤住金大坚,教与戴宗相
见,具说泰安州岳庙里重修五岳楼,众上户要立道碑文碣石之事,这太保特地各赍
五十两银子,来请我和你两个去。金大坚见了银子,心中欢喜。两个邀请戴宗就酒
肆中市沽三杯,置些蔬食,管待了。戴宗就付与金大坚五十两银子,作安家之资,
又说道:“阴阳人已拣定了日期,请二位今日便烦动身。”萧让道:“天气暄热,
今日便动身,也行不多路,前面赶不上宿头。只是来日起个五更,挨门出去。”金
大坚道:“正是如此说。”两个都约定了来早起身,各自归家收拾动用。萧让留戴
宗在家宿歇。

次日五更,金大坚持了包裹行头,来和萧让、戴宗三人同行。离了济州城里,
行不过十里多路,戴宗道:“二位先生慢来,不敢催逼,小可先去报知众上户来接
二位。”拽开步数,争先去了。这两个背着些包裹,自慢慢而行。看看走到未牌时
候,约莫也走过了七八十里路,只见前面一声胡哨响,山城坡下跳出一伙好汉,约
有四五十人,当头一个好汉,正是那清风山王矮虎,大喝一声道:“你两个是甚么
人?那里去?孩儿们拿这厮取心来吃酒。”萧让告道:“小人两个是上泰安州刻石镌
文的,又没一分财赋,止有几件衣服。”王矮虎喝道:“俺不要你财赋衣服,只要
你两个聪明人的心肝做下酒。”萧让和金大坚焦躁,倚仗各人胸中本事,便挺着杆
棒,径奔王矮虎。王矮虎也挺朴刀来斗两个。三人各使手中器械,约战了五七合,
王矮虎转身便走。两个却待去赶,听得山上锣声又响,左边走出云里金刚宋万,右
边走出摸着天杜迁,背后却是白面郎君郑天寿。各带三十余人,一发上,把萧让、
金大坚横拖倒拽,捉投林子里来。

四筹好汉道:“你两个放心,我们奉着晁天王的将令,特来请你二位上山入伙。”
萧让道:“山寨里要我们何用?我两个手无缚鸡之力,只好吃饭。”杜迁道:“吴
军师一来与你相识,二乃知你两个武艺本事,特使戴宗来宅上相请。”萧让、金大
坚都面面厮觑,做声不得。当时都到旱地忽律朱贵酒店里,相待了分例酒食,连夜
唤船,便送上山来。到得大寨,晁盖、吴用并头领众人都相见了,一面安排筵席相
待,且说修蔡京回书一事,“因请二位上山入伙,共聚大义”。两个听了,都扯住
吴学究道:“我们在此趋侍不妨,只恨各家都有老小在彼,明日官司知道,必然坏
了。”吴用道:“二位贤弟不必忧心,天明时便有分晓。”当夜只顾吃酒歇了。

次日天明,只见小喽罗报道:“都到了。”吴学究道:“请二位贤弟亲自去接
宝眷。”萧让、金大坚听得,半信半不信。两个下至半山,只见数乘轿子抬着两家
老小上山来。两个惊得呆了,问其备细。老小说道:“你昨日出门之后,只见这一
行人将着轿子来,说家长只在城外客店里中了暑风,快叫取老小来看救。出得城时,
不容我们下轿,直抬到这里。”两家都一般说。萧让听了,与金大坚两个闭口无言,
只得死心塌地,再回山寨入伙,安顿了两家老小。

吴学究却请出来,与萧让商议写蔡京字体回书,去救宋公明。金大坚便道:“从
来雕得蔡京的诸样图书名讳字号。”当时两个动手完成,安排了回书,备个筵席,
便送戴宗起程,分付了备细书意。戴宗辞了众头领,相别下山,小喽罗已把船只渡
过金沙滩,送至朱贵酒店里。戴宗取四个甲马,拴在腿上,作别朱贵,拽开脚步,
登程去了。

且说吴用送了戴宗过渡,自同众头领再回大寨筵席。正饮酒间,只见吴学究叫
声苦,不知高低。众头领问道:“军师何故叫苦?”吴用便道:“你众人不知:是
我这封书,倒送了戴宗和宋公明性命也。”众头领大惊,连忙问道:“军师书上却
是怎地差错?”吴学究道:“是我一时只顾其前,不顾其后,书中有个老大脱卯。”
萧让便道:“小生写的字体和蔡太师字体一般,语句又不曾差了。请问军师,不知
那一处脱卯?”金大坚又道:“小生雕的图书,亦无纤毫差错,怎地见得有脱卯处?”
吴学究迭两个指头,说出这个差错脱卯处。有分教:众好汉大闹江州城,鼎沸白龙
庙。直教:弓弩丛中逃性命,刀枪林里救英雄。

毕竟军师吴学究说出怎生脱卯来,且听下回分解。