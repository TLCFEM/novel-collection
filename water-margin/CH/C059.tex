\chapter{吴用赚金铃吊挂~宋江闹西岳华山}

话说贺太守把鲁智深赚到后堂内,喝声:“拿下!”众多做公的,把鲁智深簇
拥到厅阶下。贺太守喝道:“你这秃驴,从那里来?”鲁智深应道:“洒家有甚罪
犯?”太守道:“你只实说,谁教你来刺我?”鲁智深道:“俺是出家人,你却如
何问俺这话?”太守喝道:“却才见你这秃驴,意欲要把禅杖打我轿子,却又思量,
不敢下手。你这秃驴好好招了。”鲁智深道:“洒家又不曾杀你,你如何拿住洒家,
妄指平人?”太守喝骂:“几曾见出家人自称洒家。这秃驴必是个关西五路打家劫
舍的强盗,来与史进那厮报仇,不打如何肯招。左右好生加力打那秃驴。”鲁智深
大叫道:“不要打伤老爷。我说与你,俺是梁山泊好汉花和尚鲁智深。我死倒不打
紧,洒家的哥哥宋公明得知,下山来时,你这颗驴头趁早儿都砍了送去。”贺太守
听了大怒,把鲁智深拷打了一回,教取面大枷来钉了,押下死囚牢里去。一面申闻
都省,乞请明降。禅杖、戒刀,封入府堂里去了。

此时闹动了华州一府。小喽罗得了这个消息,飞报上山来。武松大惊道:“我
两个来华州干事,折了一个,怎地回去见众头领。”正没理会处,只见山下小喽罗
报道:“有个梁山泊差来的头领,唤做神行太保戴宗,现在山下。”武松慌忙下来
迎接上山,和朱武等三人都相见了,诉说鲁智深不听谏劝失陷一事。戴宗听了,大
惊道:“我不可久停了!就便回梁山泊报与哥哥知道,早遣兵将,前来救取!”武
松道:“小弟在这里专等,万望兄长早去急来。”戴宗吃了些素食,作起神行法,
再回梁山泊来。三日之间,已到山寨;见了晁、宋二头领,便说鲁智深因救史进,
要刺贺太守被陷一事。宋江听罢,失惊道:“既然两个兄弟有难,如何不救?我今
不可耽搁。便须点起人马,作三队而行。”前军点五员先锋:花荣、秦明、林冲、
杨志、呼延灼,引领一千甲马,二千步军先行,逢山开路,遇水叠桥;中军领兵主
将宋公明、军师吴用、朱仝、徐宁、解珍、解宝,共是六个头领,马步军兵二千;
后军主掌粮草,李应、杨雄、石秀、李俊、张顺,共是五个头领押后,马步军兵二
千,共计七千人马,离了梁山泊,直取华州来。在路趱行,不止一日,早过了半路,
先使戴宗去报少华山上。朱武等三人,安排下猪羊牛马,酝造下好酒等候。

再说宋江军马三队都到少华山下,武松引了朱武、陈达、杨春三人,下山拜请
宋江、吴用,并众头领,都到山寨里坐下。宋江备问城中之事,朱武道:“两个头
领,已被贺太守监在牢里,只等朝廷明降发落。”宋江与吴用说道:“怎地定计去
救取史进、鲁智深?”朱武说道:“华州城郭广阔,濠沟深远,急切难打;只除非
得里应外合,方可取得。”吴学究道:“明日且去城边看那城池如何,却再商量。”
宋江饮酒到晚,巴不得天明,要去看城。吴用谏道:“城中监着两只大虫在牢里,
如何不做提备?白日未可去看。今夜月色必然明朗,申牌前后下山,一更时分,可
到那里窥望。”

当日捱到午后,宋江、吴用、花荣、秦明、朱仝,共是五骑马下山,迤前行。
初更时分,已到华州城外。在山坡高处,立马望华州城里时,正是二月中旬天气,
月华如昼,天上无一片云彩;看见华州周围有数座城门,城高地壮,堑濠深阔。看
了半晌,远远地望见那西岳华山时,端的是好座名山。但见:

峰名仙掌,观隐云台。上连玉女洗头盆,下接天河分派水。乾坤皆秀,尖峰仿
佛接云根;山岳推尊,怪石巍峨侵斗柄。青如澄黛,碧若浮蓝。张僧繇妙笔画难成,
李龙眠天机描不就。深沉洞府,月光飞万道金霞;岩崖,日影动千条紫焰。
旁人遥指,云池波内藕如船;故老传闻,玉井水中花十丈。巨灵神忿怒,劈开山顶
逞神通;陈处士清高,结就茅庵来盹睡。千古传名推华岳,万年香火祀金天。
宋江等看了西岳华山,见城池厚壮,形势坚牢,无计可施。吴用道:“且回寨里去,
再作商议。”五骑马连夜回到少华山上。宋江眉头不展,面带忧容。吴学究道:“且
差十数个精细小喽罗下山,去远近探听消息。”

两日内,忽有一人上山来报道:“如今朝廷差个殿司太尉,将领御赐金铃吊挂
来西岳降香,从黄河入渭河而来。”吴用听了,便道:“哥哥休忧,计在这里了。”
便叫李俊、张顺:“你两个与我如此如此而行。”李俊道:“只是无人识得地境,
得一个引领路道最好。”白花蛇杨春便道:“小弟相帮同去如何?”宋江大喜。三
个下山去了。次日,吴学究请宋江、李应、朱仝、呼延灼、花荣、秦明、徐宁,共
七个人,悄悄止带五百余人下山。径到渭河渡口,李俊、张顺、杨春已夺下十数只
大船在彼。吴用便叫花荣、秦明、徐宁、呼延灼四个埋伏在岸上;宋江、吴用、朱
仝、李应下在船里;李俊、张顺、杨春把船都去滩头藏了。

众人等候了一夜。次日天明,听得远远地锣鸣鼓响,三只官船到来,船上插着
一面黄旗,上写“钦奉圣旨西岳降香太尉宿元景”。宋江看了,心中暗喜道:“昔
日玄女有言,‘遇宿重重喜’,今日既见此人,必有主意。”太尉官船将近河口,
朱仝、李应各执长枪,立在宋江、吴用背后。太尉船到当港截住。船里走出紫衫银
带虞候二十余人,喝道:“你等甚么船只,敢当港拦截住大臣?”宋江执着骨朵,
躬身声喏。吴学究立在船头上说道:“梁山泊义士宋江,谨参祗候。”船上客帐司
出来答道:“此是朝廷太尉,奉圣旨去西岳降香。汝等是梁山泊乱寇,何故拦截!”
吴用道:“俺们义士,只要求见太尉尊颜,有告复的事。”客帐司道:“你等是何
等人,敢造次要见太尉!”两边虞候喝道:“低声!”宋江说道:“暂请太尉到岸
上,自有商量的事。”客帐司道:“休胡说!太尉是朝廷命臣,如何与你商量?”
宋江道:“太尉不肯相见,只怕孩儿们惊了太尉。”朱仝把枪上小号旗只一招动,
岸上花荣、秦明、徐宁、呼延灼,引出马军来,一齐搭上弓箭,都到河口,摆列在
岸上。那船上艄公,都惊得钻入舱里去了。客帐司人慌了,只得入去禀复,宿太尉
只得出到船头上坐定。

宋江躬身唱喏道:“宋江等不敢造次。”宿太尉道:“义士何故如此邀截船只?”
宋江道:“某等怎敢邀截太尉?只欲求请太尉上岸,别有禀复。”宿太尉道:“我
今特奉圣旨,自去西岳降香,与义士有何商议?朝廷大臣,如何轻易登岸?”宋江
道:“太尉不肯时,只怕下面伴当亦不相容。”李应把号带枪一招,李俊、张顺、
杨春一齐撑出船来。宿太尉看见大惊。李俊、张顺明晃晃掣出尖刀在手,早跳过船
来,手起先把两个虞候下水里去。宋江连忙喝道:“休得胡做,惊了贵人!”李
俊、张顺扑地也跳下水去,早把两个虞候又送上船来。张顺、李俊在水面上如登平
地,托地又跳上船来,吓得宿太尉魂不着体。宋江喝道:“孩儿们且退去,休得惊
着贵人,俺自慢慢地请太尉登岸。”宿太尉道:“义士有甚事?就此说不妨。”宋
江道:“这里不是说话处,谨请太尉到山寨告禀,并无损害之心;若怀此念,西岳
神灵诛灭!”到此时候,不容太尉不上岸,宿太尉只得离船上了岸。众人牵过一匹
马来,扶策太尉上了马,不得已随众同行。宋江先叫花荣、秦明陪奉太尉上山。宋
江随后也上了马,分付教把船上一应人等,并御香、祭物、金铃吊挂,齐齐收拾上
山;只留下李俊、张顺,带领一百余人看船。

一行众头领都到山上,宋江下马入寨,把宿太尉扶在聚义厅上当中坐定,众头
领两边侍立着。宋江下了四拜,跪在面前,告复道:“宋江原是郓城县小吏,为被
官司所逼,不得已哨聚山林,权借梁山水泊避难,专等朝廷招安,与国家出力。今
有两个兄弟,无事被贺太守生事陷害,下在牢里。欲借太尉御香、仪从,并金铃吊
挂,去赚华州;事毕并还,于太尉身上,并无侵犯。乞太尉钧鉴。”宿太尉道:“不
争你将了御香等物去,明日事露,须连累下官。”宋江道:“太尉回京,都推在宋
江身上便了。”宿太尉看了那一班人模样,怎生推托得?只得应允了。宋江执盏擎
杯,设筵拜谢。就把太尉带来的人穿的衣服都借穿了。于小喽罗数内,选拣一个俊
俏的,剃了髭须,穿了太尉的衣服,扮做宿元景;宋江、吴用扮做客帐司;解珍、
解宝、杨雄、石秀扮做虞候;小喽罗都是紫衫银带,执着旌节,旗、仪仗、法物,
擎抬了御香、祭礼、金铃吊挂;花荣、徐宁、朱仝、李应扮做四个衙兵;朱武、陈
达、杨春款住太尉并跟随一应人等,置酒管待。却教秦明、呼延灼引一队人马,林
冲、杨志引一队人马,分作两路取城;教武松预先去西岳门下伺候,只听号起行事。

话休絮繁。且说一行人等,离了山寨,径到河口下船而行,不去报与华州太守,
一径奔西岳庙来。戴宗先去报知云台观观主,并庙里职事人等,直至船边,迎接上
岸。香花灯烛,幢宝盖,摆列在前;先请御香上了香亭,庙里人夫扛抬了,导引
金铃吊挂前行。观主拜见了太尉,吴学究道:“太尉一路染病不快,且把轿子来。”
左右人等,扶策太尉上轿,径到岳庙里官厅内歇下。客帐司吴学究对观主道:“这
是特奉圣旨,赍捧御香、金铃吊挂,来与圣帝供养;缘何本州官员轻慢,不来迎接?”
观主答道:“已使人去报了,敢是便到。”说犹未了,本州先使一员推官,带领做
公的五七十人,将着酒果,来见太尉。原来那扮太尉的小喽罗虽然模样相似,却语
言发放不得,因此只教妆做染病,把靠褥围定在床上坐。推官看了,见来的旌节、
门旗、牙仗等物,都是内府制造出的,如何不信?客帐司假意出入,禀复了两遭,
却引推官入去,远远地阶下参拜了。那假太尉只把手指,并不听得说甚么。吴用引
到面前,埋怨推官道:“太尉是天子前近幸大臣,不辞千里之遥,特奉圣旨到此降
香,不想于路染病未痊,本州众官,如何不来远接!”推官答道:“前路官司虽有
文书到州,不见近报,因此有失迎迓。不期太尉先到庙里,本是太守便来,奈缘少
华山贼人,纠合梁山泊草盗,要打城池,每日在彼提防,以此不敢擅离。特差小官
先来贡献酒礼,太守随后便来参见。”吴学究道:“太尉涓滴不饮,只叫太守快来
商议行礼。”推官随即教取酒来,与客帐司亲随人把盏了。吴学究又入去禀一遭,
将了钥匙出来,引着推官去看金铃吊挂,开了锁,就香帛袋中取出那御赐金铃吊挂
来,叫推官看,便把条竹竿叉起。看时,果然制造得无比。但见:

浑金打就,五彩妆成。双悬缨络金铃,上挂珠玑宝盖。黄罗密布,中间八爪玉
龙盘;紫带低垂,外壁双飞金凤递。对嵌珊瑚玛瑙,重围琥珀珍珠。碧琉璃掩映绛
纱灯,红菡萏参差青翠叶。堪宜金屋琼楼挂,雅称瑶台宝殿悬。
这一对金铃吊挂乃是东京内府高手匠人做成的,浑是七宝珍珠嵌造,中间点着碗红
纱灯笼,乃是圣帝殿上正中挂的,不是内府降来,民间如何做得?吴用叫推官看了,
再收入柜匣内锁了。又将出中书省许多公文,付与推官,便叫太守来商议,拣日祭
祀。推官和众多做公的,都见了许多物件文凭,便辞了客帐司,径回到华州府里,
来报贺太守。

却说宋江暗暗地喝采道:“这厮虽然奸猾,也骗得他眼花心乱了。”此时武松
已在庙门下了。吴学究又使石秀藏了尖刀,也来庙门下,相帮武松行事,却又叫戴
宗扮虞候。云台观主进献素斋,一面教执事人等安排铺陈岳庙。宋江闲步看那西岳
庙时,果然是盖造的好,殿宇非凡,真乃人间天上。宋江来到正殿上,拈香再拜,
暗暗祈祷已罢,回至官厅前,门人报道:“贺太守来也。”宋江便叫花荣、徐宁、
朱仝、李应四个衙兵,各执着器械,分列在两边;解珍、解宝、杨雄、戴宗,各带
暗器,侍立在左右。

却说贺太守将带三百余人,来到庙前下马,簇拥入来,假客帐司吴学究、宋江
见贺太守带着三百余人,都是带刀公吏人等入来,吴学究喝道:“朝廷太尉在此,
闲杂人不许近前!”众人立住了脚,贺太守独自进前来拜见太尉。客帐司道:“太
尉教请太守入来厮见。”贺太守入到官厅前,望着假太尉便拜。吴学究道:“太守,
你知罪么?”太守道:“贺某不知太尉到来,伏乞恕罪。”吴学究道:“太尉奉敕
到此西岳降香,如何不来远接?”太守答道:“不曾有近报到州,有失迎迓。”吴
学究喝声:“拿下!”解珍、解宝弟兄两个,身边早掣出短刀来,一脚把贺太守踢
翻,便割了头。宋江喝道:“兄弟们动手!”早把那跟来的人三百余人,惊得呆了,
正走不动。花荣等一发向前,把那一干人,算子般都倒在地下,有一半抢出庙门下,
武松、石秀舞刀杀将入来,小喽罗四下赶杀,三百余人不剩一个回去。续后到庙里
的,都被张顺、李俊杀了。

宋江急叫收了御香、吊挂下船,都赶到华州时,早见城中两路火起,一齐杀将
入来,先去牢中救了史进、鲁智深;就打开库藏,取了财帛,装载上车。一行人离
了华州,上船回到少华山上,都来拜见宿太尉,纳还了御香、金铃吊挂、旌节、门
旗、仪仗等物,拜谢了太尉恩相。宋江教取一盘金银相送太尉;随从人等,不分高
低,都与了金银;就山寨里做了个送路筵席,谢承太尉。众头领直送下山,到河口
交割了一应什物船只,一些不少,还了原来的人等。

宋江谢别了宿太尉,回到少华山上,便与四筹好汉商议,收拾山寨钱粮,放火
烧了寨栅。一行人等,军马粮草,都望梁山泊来。

且说宿太尉下船来,到华州城中,已知被梁山泊贼人杀死军兵人马,劫了府库
钱粮;城中杀死军校一百余人,马匹尽皆掳去。西岳庙中,又杀了许多人性命;便
叫本州推官动文书申达中书省起奏,都做“宋江先在途中劫了御香、吊挂,因此赚
知府到庙,杀害性命”。宿太尉到庙里焚了御香,把这金铃吊挂分付与了云台观主,
星夜急急自回京师,奏知此事,不在话下。

再说宋江救了史进、鲁智深,带了少华山四个好汉,仍旧作三队,分人马,
向梁山泊来,所过州县,秋毫无犯。先使戴宗前来上山报知,晁盖并众头领下山迎
接宋江等,一同到山寨里聚义厅上,都相见已罢,一面做庆喜筵席。

次日,史进、朱武、陈达、杨春,各以己财做筵宴,拜谢晁、宋二公并众头领。
过了数日,话休絮烦。忽一日,有旱地忽律朱贵上山报说:“徐州沛县芒砀山中,
新有一伙强人,聚集着三千人马。为头一个先生,姓樊,名瑞,绰号混世魔王,能
呼风唤雨,用兵如神。手下两个副将:一个姓项,名充,绰号八臂那吒,能使一面
团牌,牌上插飞刀二十四把,手中仗一条铁标枪。又有一个姓李,名衮,绰号飞天
大圣,也使一面团牌,牌上插标枪二十四根,手中使一口宝剑。——这三个结为兄
弟,占住芒砀山,打家劫舍。三个商量了,要来吞并俺梁山泊大寨。小弟听得说,
不得不报。”宋江听了,大怒道:“这贼怎敢如此无礼!我便再下山走一遭!”只
见九纹龙史进便起身道:“小弟等四个初到大寨,无半米之功,情愿引本部人马,
前去收捕这伙强人。”宋江大喜。当下史进点起本部人马,与同朱武、陈达、杨春,
都披挂了,来辞宋江下山。把船渡过金沙滩,上路径奔芒砀山来。三日之内,早望
见那座山,乃是昔日汉高祖斩蛇起义之处。三军人马来到山下,早有伏路小喽罗上
山报知。

且说史进把少华山带来的人马摆开,史进全身披挂,骑一匹火炭赤马,当先出
阵。怎见得史进的英雄,但见:

久在华州城外住,出身原是庄农,学成武艺惯心胸。三尖刀似雪,浑赤马如龙。

体挂连环镔铁铠,战袍风猩红,雕青镌玉更玲珑。江湖称史进,绰号九纹龙。
当时史进首先出马,手中横着三尖两刃刀;背后三个头领,中间的便是神机军师朱
武。那人原是定远县人氏,平生足智多谋,亦能使两口双刀,出到阵前,亦有八句
诗单道朱武好处:
道服裁棕叶,云冠剪鹿皮。
脸红双眼俊,面目细髯垂。
智可张良比,才将范蠡欺。
今堪副吴用,朱武号神机。
上首马上坐着一筹好汉,手中横着一条出白点钢枪,绰号跳涧虎陈达,原是邺城人
氏。当时提枪跃马,出到阵前,也有一首诗单道着陈达好处:
每见力人能虎跳,亦知猛虎跳山溪。
果然陈达人中虎,跃马腾枪奋鼓鼙。
下首马上坐着一筹好汉,手中使一口大杆刀,绰号白花蛇杨春,原是解良县蒲城人
氏。当下挺刀立马,守住阵门,也有一首诗单题杨春的好处:
杨春名姓亦奢遮,劫客多年在少华。
伸臂展腰长有力,能吞巨象白花蛇。

四个好汉勒马在阵前,望不多时,只见芒砀山上飞下一彪人马来,当先两个好
汉:为头那一个,便是徐州沛县人氏,姓项,名充,绰号八臂那吒,使一面团牌,
背插飞刀二十四把,百步取人,无有不中,右手仗一条标枪,后面打着一面认军旗,
上书“八臂那吒”,步行下山。有八句诗单题项充:
铁帽深遮顶,铜环半掩腮。
傍牌悬兽面,飞刃插龙胎。
脚到如风火,身先降祸灾。
那吒号八臂,此是项充来。
次后那个,便是邳县人氏,姓李,名衮,绰号飞天大圣,会使一面团牌,背插二十
四把标枪,亦能百步取人;左手挽牌,右手伏剑;后面打着一面认军旗,上书“飞
天大圣”,出到阵前。有八句诗单道李衮:
缨盖盔兜顶,袍遮铁掩襟。
胸藏拖地胆,毛盖杀人心。
飞刃齐攒玉,蛮牌满画金。
飞天号大圣,李衮众人钦。

当下两个步行下山,见了对阵史进、朱武、陈达、杨春四骑马在阵前,并不打
话,小喽罗筛起锣来,两个好汉舞动团牌,齐上直滚入阵来。史进等拦当不住,后
军先走,史进前军抵敌,朱武等中军呐喊,乱窜起来,正所谓人住马不住,杀得退
走三四十里。史进险些儿中了飞刀,杨春转身得迟,被一飞刀,战马着伤,弃了马,
逃命走了。史进点军,折了一半,和朱武等商议,欲要差人回梁山泊求救。正忧疑
之间,只见军士来报:“北边大路上,尘头起处,约有二千军马到来。”史进等直
迎来时,却是梁山泊旗号,当先马上两员上将:一个是小李广花荣,一个是金枪手
徐宁。史进接着,备说项充、李衮蛮牌滚动,军马遮拦不住。花荣道:“宋公明哥
哥见兄长来了,放心不下,好生懊悔,特遣我两个到来帮助。”史进等大喜,合兵
一处下寨。次日天晓,正欲起兵对敌,军士报道:“北边大路上又有军马到来。”
花荣、徐宁、史进一齐上马接时,却是宋公明亲自和军师吴学究、公孙胜、柴进、
朱仝、呼延灼、穆弘、孙立、黄信、吕方、郭盛,带领三千人马来到。史进备说项
充、李衮飞刀、标枪、滚牌难近,折了人马一事。宋江大惊,吴用道:“且把军马
扎下寨栅,别作商议。”宋江性急,要起兵剿捕,直到山下。

此时天色已晚,望见芒砀山上,都是青色灯笼,公孙胜看了,便道:“此寨中
青色灯笼,必有个会行妖法之人在内,我等且把军马退去,来日贫道献一个阵法,
要捉此二人。”宋江大喜,传令教军马且退二十里扎住营寨。次日清晨,公孙胜献
出这个阵法,有分教:魔王拱手上梁山,神将倾心归水泊。

毕竟公孙胜献出甚么阵法来,且听下回分解。