\chapter{小霸王醉入销金帐~花和尚大闹桃花村}

话说当日智真长老道:“智深,你此间决不可住了。我有一个师弟,现在东京
大相国寺住持,唤做智清禅师。我与你这封书,去投他那里,讨个职事僧做。我夜
来看了,赠汝四句偈言,你可终身受用,记取今日之言。”智深跪下道:“洒家愿
听偈言。”长老道:“遇林而起,遇山而富,遇水而兴,遇江而止。”鲁智深听了
四句偈言,拜了长老九拜。背了包裹、腰包、肚包,藏了书信,辞了长老并众僧人,
离了五台山,径到铁匠间壁客店里歇了,等候打了禅杖、戒刀完备就行。寺内众僧
得鲁智深去了,无一个不欢喜。长老教火工道人自来收拾打坏了的金刚、亭子。过
不得数日,赵员外自将若干钱物来五台山,再塑起金刚,重修起半山亭子,不在话
下。有诗为证:
禅林辞去入禅林,知己相逢义断金。
且把威风惊贼胆,漫将妙理悦禅心。
绰名久唤花和尚,道号亲名鲁智深。
俗愿了时终证果,眼前争奈没知音。

再说这鲁智深就客店里住了几日,等得两件家生都已完备,做了刀鞘,把戒刀
插放鞘内,禅杖却把漆来裹了。将些碎银子赏了铁匠,背了包裹,跨了戒刀,提了
禅杖,作别了客店主人并铁匠,行程上路。过往人看了,果然是个莽和尚。但见:

皂直裰背穿双袖,青圆绦斜绾双头。鞘内戒刀,藏春冰三尺;
肩头禅杖,横铁蟒一条。鹭鹚腿紧系脚絣,蜘蛛肚牢拴衣钵。
嘴缝边攒千条断头铁线,胸脯上露一带盖胆寒毛。生成食肉餐
鱼脸,不是看经念佛人。

且说鲁智深自离了五台山文殊院,取路投东京来。行了半月之上,于路不投寺
院去歇,只是客店内打火安身,白日间酒肆里买吃。

一日正行之间,贪看山明水秀,不觉天色已晚。但见:

山影深沉,槐阴渐没。绿杨郊外,时闻鸟雀归林;红杏村中,每见牛羊入圈。
落日带烟生碧雾,断霞映水散红光。溪边钓叟移舟去,野外村童跨犊归。
鲁智深因见山水秀丽,贪行了半日,赶不上宿头,路中又没人作伴,那里投宿
是好?又赶了三二十里田地,过了一条板桥,远远地望见一簇红霞,树木丛中,闪着
一所庄院,庄后重重叠叠,都是乱山。鲁智深道:“只得投庄上去借宿。”径奔到庄前
看时,见数十个庄家,忙忙急急,搬东搬西。鲁智深到庄前,倚了禅杖,与庄客打
个问讯。庄客道:“和尚,日晚来我庄上做甚的?”智深道:“洒家赶不上宿头,
欲借贵庄投宿一宵,明早便行。”庄客道:“我庄上今夜有事,歇不得。”智深道:
“胡乱借洒家歇一夜,明日便行。”庄客道:“和尚快走,休在这里讨死!”智深
道:“也是怪哉!歇一夜,打甚么不紧?怎地便是讨死?”庄家道:“去便去,不去
时,便捉来缚在这里。”鲁智深大怒道:“你这厮村人,好没道理!俺又不曾说甚
的,便要绑缚洒家。”庄家们也有骂的,也有劝的。

鲁智深提起禅杖,却待要发作,只见庄里走出一个老人来。鲁智深看那老人时,
似年近六旬之上。拄一条过头拄杖,走将出来,喝问庄客:“你们闹甚么?”庄客
道:“可奈这个和尚要打我们。”智深便道:“小僧是五台山来的和尚,要上东京
去干事,今晚赶不上宿头,借贵庄投宿一宵,庄家那厮无礼,要绑缚洒家。”那老
人道:“既是五台山来的僧人,随我进来。”智深跟那老人直到正堂上,分宾主坐
下。那老人道:“师父,休要怪。庄家们不省得师父是活佛去处来的,他作寻常一
例相看。老汉从来敬信佛天三宝,虽是我庄上今夜有事,权且留师父歇一宵了去。”
智深将禅杖倚了,起身打个问讯,谢道:“感承施主,小僧不敢动问贵庄高姓?”
老人道:“老汉姓刘,此间唤做桃花村,乡人都叫老汉做桃花庄刘太公。敢问师父
俗姓,唤做甚么讳字?”智深道:“俺的师父是智真长老,与俺取了个讳字。因洒
家姓鲁,唤做鲁智深。”太公道:“师父请吃些晚饭,不知肯吃荤腥也不?”鲁智
深道:“洒家不忌荤酒,遮莫甚么浑清白酒,都不拣选;牛肉狗肉,但有便吃。”
太公道:“既然师父不忌荤酒,先叫庄客取酒肉来。”没多时,庄客掇张桌子,放
下一盘牛肉,三四样菜蔬,一双箸,放在鲁智深面前。智深解下腰包、肚包,坐定。
那庄客旋了一壶酒,拿一只盏子,筛下酒与智深吃。这鲁智深也不谦让,也不推辞,
无一时,一壶酒,一盘肉,都吃了。太公对席看见,呆了半晌。庄客搬饭来,又吃
了,抬过桌子。

太公分付道:“胡乱教师父在外面耳房中歇一宵,夜间如若外面热闹,不可出
来窥望。”智深道:“敢问贵庄今夜有甚事?”太公道:“非是你出家人闲管的事。”
智深道:“太公缘何模样不甚喜欢?莫不怪小僧来搅扰你么?明日洒家算还你房钱便
了。”太公道:“师父听说,我家时常斋僧布施,那争师父一个?只是我家今夜小
女招夫,以此烦恼。”鲁智深呵呵大笑道:“‘男大须婚,女大必嫁’。这是人伦
大事,五常之礼,何故烦恼?”太公道:“师父不知,这头亲事,不是情愿与的。”
智深大笑道:“太公,你也是个痴汉,既然不两相情愿,如何招赘做个女婿?”太
公道:“老汉止有这个小女,如今方得一十九岁,被此间有座山,唤做桃花山,近
来山上有两个大王,扎了寨栅,聚集着五七百人,打家劫舍。此间青州官军捕盗,
禁他不得,因来老汉庄上讨进奉,见了老汉女儿,撇下二十两金子、一匹红锦为定
礼,选着今夜好日,晚间来入赘老汉庄上。又和他争执不得,只得与他,因此烦恼,
非是争师父一个人。”智深听了道:“原来如此。小僧有个道理,教他回心转意,
不要娶你女儿如何?”太公道:“他是个杀人不眨眼魔君,你如何能够得他回心转
意?”智深道:“洒家在五台山智真长老处,学得说因缘,便是铁石人,也劝得他
转。今晚可教你女儿别处藏了,俺就你女儿房内说因缘,劝他便回心转意。”太公
道:“好却甚好,只是不要捋虎须。”智深道:“洒家的不是性命!你只依着俺行。”
太公道:“却是好也!我家有福,得遇这个活佛下降。”庄客听得,都吃一惊。

太公问智深:“再要饭吃么?”智深道:“饭便不要吃,有酒再将些来吃。”
太公道:“有,有!”随即叫庄客取一只熟鹅,大碗斟将酒来,叫智深尽意吃了三
二十碗,那只熟鹅也吃了。叫庄客将了包裹,先安放房里,提了禅杖,带了戒刀,
问道:“太公,你的女儿躲过了不曾?”太公道:“老汉已把女儿寄送在邻舍庄里
去了。”智深道:“引洒家新妇房内去。”太公引至房边,指道:“这里面便是。”
智深道:“你们自去躲了。”太公与众庄客自出外面安排筵席。智深把房中桌椅等
物,都掇过了;将戒刀放在床头,禅杖把来倚在床边,把销金帐子下了,脱得赤条
条地,跳上床去坐了。

太公见天色看看黑了,叫庄客前后点起灯烛荧煌,就打麦场上放下一条桌子,
上面摆着香花灯烛。一面叫庄客大盘盛着肉,大壶温着酒。约莫初更时分,只听得
山边锣鸣鼓响。这刘太公怀着鬼胎,庄家们都捏着两把汗,尽出庄门外看时,只见
远远地四五十火把,照曜如同白日,一簇人马,飞奔庄上来。但见:

雾锁青山影里,滚出一伙没头神;烟迷绿树林边,摆着几行争食鬼。人人凶恶,
个个狰狞。头巾都戴茜根红,衲袄尽披枫叶赤。缨枪对对,围遮定吃人心肝的
小魔王;梢棒双双,簇捧着不养爹娘的真太岁。夜间罗刹去迎亲,山上大虫来
下马。

刘太公看见,便叫庄客大开庄门,前来迎接。只见前遮后拥,明晃晃的都是器
械旗枪,尽把红绿绢帛缚着。小喽罗头巾边乱插着野花。前面摆着四五对红纱灯笼,
照着马上那个大王。怎生打扮,但见:

头戴撮尖干红凹面巾,鬓傍边插一枝罗帛象生花,上穿一领围虎体挽绒金绣绿
罗袍,腰系一条称狼身销金包肚红搭膊,着一双对掩云跟牛皮靴,骑一匹高头
卷毛大白马。
那大王来到庄前下了马,只见众小喽罗齐声贺道:“帽儿光光,今夜做个新郎;衣
衫窄窄,今夜做个娇客。”刘太公慌忙亲捧台盏,斟下一杯好酒,跪在地下,众庄
客都跪着。那大王把手来扶道:“你是我的丈人,如何倒跪我?”太公道:“休说
这话,老汉只是大王治下管的人户。”那大王已有七八分醉了,呵呵大笑道:“我
与你家做个女婿,也不亏负了你。你的女儿匹配我也好。”刘太公把了下马杯,来
到打麦场上,见了香花灯烛,便道:“泰山,何须如此迎接?”那里又饮了三杯,
来到厅上,唤小喽罗教把马去系在绿杨树上。小喽罗把鼓乐就厅前擂将起来,大王
上厅坐下,叫道:“丈人,我的夫人在那里?”太公道:“便是怕羞,不敢出来。”
大王笑道:“且将酒来,我与丈人回敬。”那大王把了一杯,便道:“我且和夫人
厮见了,却来吃酒未迟。”那刘太公一心只要那和尚劝他,便道:“老汉自引大王
去。”拿了烛台,引着大王,转入屏风背后,直到新人房前。太公指与道:“此间
便是,请大王自入去。”太公拿了烛台,一直去了。未知凶吉如何,先办一条走路。

那大王推开房门,见里面黑洞洞地。大王道:“你看我那丈人,是个做家的人,
房里也不点碗灯,由我那夫人黑地里坐地。明日叫小喽罗山寨里扛一桶好油来与他
点。”鲁智深坐在帐子里都听得,忍住笑,不做一声。那大王摸进房中,叫道:“娘
子,你如何不出来接我?你休要怕羞,我明日要你做压寨夫人。”一头叫娘子,一
头摸来摸去。一摸摸着销金帐子,便揭起来,探一只手入去摸时,摸着鲁智深的肚
皮,被鲁智深就势劈头巾带角儿揪住,一按按将下床来。那大王却待挣扎,鲁智深
把右手捏起拳头,骂一声:“直娘贼!”连耳根带脖子只一拳,那大王叫一声:“做
甚么便打老公?”鲁智深喝道:“教你认的老婆!”拖倒在床边,拳头脚尖一齐上,
打得大王叫救人。刘太公惊得呆了,只道这早晚正说因缘劝那大王,却听的里面叫
救人。太公慌忙把着灯烛,引了小喽罗,一齐抢将入来。众人灯下打一看时,只见
一个胖大和尚,赤条条不着一丝,骑翻大王在床面前打。为头的小喽罗叫道:“你
众人都来救大王。”众小喽罗一齐拖枪拽棒,打将入来救时,鲁智深见了,撇下大
王,床边绰了禅杖,着地打将出来。小喽罗见来得凶猛,发声喊都走了。刘太公只
管叫苦。打闹里,那大王爬出房门,奔到门前,摸着空马,树上折枝柳条,托地跳
在马背上,把柳条便打那马,却跑不去。大王道:“苦也!这马也来欺负我。”再
看时,原来心慌,不曾解得缰绳,连忙扯断了,骑着【扌产】马飞走。出得庄门,大骂:
“刘太公老驴休慌,不怕你飞了。”把马打上两柳条,拨喇喇地驮了大王上山去。

刘太公扯住鲁智深道:“和尚,你苦了老汉一家儿了!”鲁智深说道:“休怪
无礼。且取衣服和直裰来,洒家穿了说话。”庄家去房里取来,智深穿了。太公道:
“我当初只指望你说因缘,劝他回心转意,谁想你便下拳打他这一顿,定是去报山
寨里大队强人来杀我家。”智深道:“太公休慌。俺说与你:洒家不是别人,俺是
延安府老种经略相公帐前提辖官,为因打死了人,出家做和尚,休道这两个鸟人,
便是一二千军马来,洒家也不怕他。你们众人不信时,提俺禅杖看。”庄客们那里
提得动。智深接过来手里,一似拈灯草一般使起来。太公道:“师父休要走了去,
却要救护我们一家儿使得。”智深道:“甚么闲话!俺死也不走。”太公道:“且
将些酒来师父吃,休得要抵死醉了。”鲁智深道:“洒家一分酒,只有一分本事,
十分酒,便有十分的气力。”太公道:“恁地时最好。我这里有的是酒肉,只顾教
师父吃。”

且说这桃花山大头领坐在寨里,正欲差人下山来探听做女婿的二头领如何,只
见数个小喽罗气急败坏,走到山寨里叫道:“苦也!苦也!”大头领连忙问道:“有
甚么事,慌做一团?”小喽罗道:“二哥哥吃打坏了。”大头领大惊,正问备细,
只见报道:“二哥哥来了。”大头领看时,只见二头领红巾也没了,身上绿袍扯得
粉碎,下得马倒在厅前,口里说道:“哥哥救我一救。”大头领问道:“怎么来?”
二头领道:“兄弟下得山,到他庄上,入进房里去。叵耐那老驴把女儿藏过了,却
教一个胖和尚躲在女儿床上。我却不提防,揭起帐子摸一摸,吃那厮揪住,一顿拳
头脚尖,打得一身伤损。那厮见众人入来救应,放了手,提起禅杖打将出去。因此
我得脱了身,拾得性命。哥哥与我做主报仇。”大头领道:“原来恁地。你去房中
将息,我与你去拿那贼秃来。”喝叫左右:“快备我的马来!”众小喽罗都去。大
头领上了马,绰枪在手,尽数引了小喽罗,一齐呐喊下山去了。

再说鲁智深正吃酒哩,庄客报道:“山上大头领尽数都来了。”智深道:“你
等休慌。洒家但打翻的,你们只顾缚了,解去官司请赏。取俺的戒刀来。”鲁智深
把直裰脱了,拽扎起下面衣服,跨了戒刀,大踏步提了禅杖,出到打麦场上。只见
大头领在火把丛中,一骑马抢到庄前,马上挺着长枪,高声喝道:“那秃驴在那里?
早早出来决个胜负。”智深大怒,骂道:“腌打脊泼才,叫你认得洒家!”抡起
禅杖,着地卷将来。那大头领逼住枪,大叫道:“和尚且休要动手,你的声音好厮
熟,你且通个姓名。”鲁智深道:“洒家不是别人,老种经略相公帐前提辖鲁达的
便是,如今出了家,做和尚,唤做鲁智深。”那大头领呵呵大笑,滚鞍下马,撇了
枪,扑翻身便拜道:“哥哥别来无恙,可知二哥着了你手。”鲁智深只道赚他,托
地跳退数步,把禅杖收住,定睛看时,火把下认得,不是别人,却是江湖上使枪棒
卖药的教头打虎将李忠。原来强人下拜,不说此二字,为军中不利,只唤做剪拂,
此乃吉利的字样。李忠当下剪拂了起来,扶住鲁智深道:“哥哥缘何做了和尚?”
智深道:“且和你到里面说话。”刘太公见了,又只叫苦:“这和尚原来也是一路!”

鲁智深到里面,再把直裰穿了,和李忠都到厅上叙旧。鲁智深坐在正面,唤刘
太公出来,那老儿不敢向前。智深道:“太公休怕,他也是俺的兄弟。”那老儿见
说是兄弟,心里越慌,又不敢不出来。李忠坐了第二位,太公坐了第三位。鲁智深
道:“你二位在此,俺自从渭州三拳打死了镇关西,逃走到代州雁门县,因见了洒
家赍发他的金老。那老儿不曾回东京去,却随个相识,也在雁门县住。他那个女儿,
就与了本处一个财主赵员外。和俺厮见了,好生相敬。不想官司追捉得洒家要紧,
那员外陪钱去送俺五台山智真长老处落发为僧。洒家因两番酒后闹了僧堂,本师长
老与俺一封书,教洒家去东京大相国寺,投了智清禅师,讨个职事僧做。因为天晚,
到这庄上投宿,不想与兄弟相见。却才俺打的那汉是谁?你如何又在这里?”李忠
道:“小弟自从那日与哥哥在渭州酒楼上同史进三人分散,次日听得说哥哥打死了
郑屠。我去寻史进商议,他又不知投那里去了。小弟听得差人缉捕,慌忙也走了,
却从这山下经过。却才被哥哥打的那汉,先在这里桃花山扎寨,唤做小霸王周通。
那时引人下山来和小弟厮杀,被我赢了,他留小弟在山上为寨主,让第一把交椅,
教小弟坐了,以此在这里落草。”

智深道:“既然兄弟在此,刘太公这头亲事,再也休题。他止有这个女儿,要
养终身;不争被你把了去,教他老人家失所。”太公见说了,大喜,安排酒食出来,
管待二位。小喽罗们每人两个馒头,两块肉,一大碗酒,都教吃饱了。太公将出原
定的金子缎匹。鲁智深道:“李家兄弟,你与他收了去,这件事都在你身上。”李
忠道:“这个不妨事。且请哥哥去小寨住几时,刘太公也走一遭。”太公叫庄客安
排轿子,抬了鲁智深,带了禅杖、戒刀、行李。李忠也上了马,太公也乘了一乘小
轿,却早天色大明。众人上山来,智深、太公到得寨前,下了轿子,李忠也下了马,
邀请智深入到寨中,向这聚义厅上,三人坐定,李忠叫请周通出来。周通见了和尚,
心中怒道:“哥哥却不与我报仇,倒请他来寨里,让他上面坐!”李忠道:“兄弟,
你认得这和尚么?”周通道:“我若认得他时,须不吃他打了。”李忠笑道:“这
和尚便是我日常和你说的三拳打死镇关西的,便是他。”周通把头摸一摸,叫声:
“阿呀!”扑翻身便剪拂。鲁智深答礼道:“休怪冲撞。”

三个坐定,刘太公立在面前,鲁智深便道:“周家兄弟,你来听俺说,刘太公
这头亲事,你却不知他只有这个女儿,养老送终,承祀香火,都在他身上。你若娶
了,教他老人家失所,他心里怕不情愿。你依着洒家,把来弃了,别选一个好的。
原定的金子缎匹,将在这里。你心下如何?”周通道:“并听大哥言语,兄弟再不
敢登门。”智深道:“大丈夫作事,却休要翻悔!”周通折箭为誓。刘太公拜谢了,
纳还金子缎匹,自下山回庄去了。

李忠、周通椎牛宰马,安排筵席,管待了数日。引鲁智深山前山后观看景致,
果是好座桃花山,生得凶怪,四围险峻,单单只一条路上去,四下里漫漫都是乱草。
智深看了道:“果然好险隘去处。”住了几日,鲁智深见李忠、周通不是个慷慨之
人,作事悭吝,只要下山。两个苦留,那里肯住,只推道:“俺如今既出了家,如
何肯落草?”李忠、周通道:“哥哥既然不肯落草,要去时,我等明日下山,但得
多少,尽送与哥哥作路费。”次日,山寨里一面杀羊宰猪,且做送路筵席,安排整
顿,却将金银酒器,设放在桌上。正待入席饮酒,只见小喽罗报来说:“山下有两
辆车,十数个人来也。”李忠、周通见报了,点起众多小喽罗,只留一两个伏侍鲁
智深饮酒。两个好汉道:“哥哥只顾请自在吃几杯,我两个下山去取得财来,就与
哥哥送行。”分付已罢,引领众人下山去了。

且说这鲁智深寻思道:“这两个人好生悭吝,现放着有许多金银,却不送与俺,
直等要去打劫得别人的,送与洒家。这个不是把官路当人情,只苦别人!洒家且教
这厮吃俺一惊。”便唤这几个小喽罗近前来筛酒吃。方才吃得两盏,跳起身来,两
拳打翻两个小喽罗,便解搭膊做一块儿捆了,口里都塞了些麻核桃。便取出包裹打
开,没要紧的都撇了,只拿了桌上金银酒器,都踏匾了,拴在包裹;胸前度牒袋内
藏了真长老的书信;跨了戒刀,提了禅杖,顶了衣包,便出寨来。到山后打一望时,
都是险峻之处,却寻思:“洒家从前山去时,以定吃那厮们撞见,不如就此间乱草
处滚将下去。”先把戒刀和包裹拴了,望下丢落去,又把禅杖也撺落去。却把身望
下只一滚,骨碌碌直滚到山脚边,并无伤损。诗曰:
绝险曾无鸟道开,欲行且止自疑猜。
光头包裹从高下,瓜熟纷纷落蒂来。

当时鲁智深从险峻处滚下,跳将起来,寻了包裹,跨了戒刀,拿了禅杖,拽开
脚手,取路便走。

再说李忠、周通下到山边,正迎着那数十个人,各有器械。李忠、周通挺着枪,
小喽罗呐着喊,抢向前来喝道:“兀那客人,会事的留下买路钱!”那客人内有一
个便拈着朴刀来斗李忠,一来一往,一去一回,斗了十余合,不分胜负。周通大怒,
赶向前来喝一声,众小喽罗一齐都上,那伙客人抵当不住,转身便走。有那走得迟
的,尽被搠死七八个。劫了车子财物,和着凯歌,慢慢地上山来。到得寨里,打一
看时,只见两个小喽罗捆做一块在亭柱边,桌子上金银酒器,都不见了。周通解了
小喽罗,问其备细,鲁智深那里去了。小喽罗说道:“把我两个打翻捆缚了,卷了
若干器皿,都拿了去。”周通道:“这贼秃不是好人,倒着了那厮手脚,却从那里
去了?”团团寻踪迹,到后山,见一带荒草平平地都滚倒了。周通看了道:“这秃
驴倒是个老贼!这般险峻山冈,从这里滚了下去。”李忠道:“我们赶上去问他讨,
也羞那厮一场。”周通道:“罢,罢!贼去了关门,那里去赶?便赶得着时,也问他
取不成。倘有些不然起来,我和你又敌他不过,后来倒难厮见了;不如罢手,后来
倒好相见。我们且自把车子上包裹打开,将金银缎匹分作三分,我和你各捉一分,
一分赏了众小喽罗。”李忠道:“是我不合引他上山,折了你许多东西,我的这一
分都与了你。”周通道:“哥哥,我同你同死同生,休恁地计较。”看官牢记话头,
这李忠、周通自在桃花山打劫。

再说鲁智深离了桃花山,放开脚步,从早晨直走到午后,约莫走下五六十里多
路,肚里又饥,路上又没个打火处,寻思:“早起只顾贪走,不曾吃得些东西,却
投那里去好?”东观西望,猛然听得远远地铃铎之声,鲁智深听得道:“好了!不
是寺院,便是宫观,风吹得檐前铃铎之声,洒家且寻去那里投奔。”不是鲁智深投
那个去处,有分教:到那里断送了十余条性命生灵,一把火烧了有名的灵山古迹。
直教:黄金殿上生红焰,碧玉堂前起黑烟。

毕竟鲁智深投甚么寺观来,且听下回分解。