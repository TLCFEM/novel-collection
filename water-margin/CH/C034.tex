\chapter{镇三山大闹青州道~霹雳火夜走瓦砾场}

话说那黄信上马,手中横着这口丧门剑;刘知寨也骑着马,身上披挂些戎衣,
手中拿一把叉。那一百四五十军汉寨兵,各执着缨枪棍棒,腰下都带短刀利剑,两
下鼓,一声锣,解宋江和花荣望青州来。

众人都离了清风寨,行不过三四十里路头,前面见一座大林子。正来到那山嘴
边,前头寨兵指道:“林子里有人窥望。”都立住了脚。黄信在马上问道:“为甚
不行?”军汉答道:“前面林子里有人窥看。”黄信喝道:“休睬他,只顾走!”

看看渐近林子前,只听得当当的二三十面大锣,一齐响起来。那寨兵人等,都
慌了手脚,只待要走。黄信喝道:“且住,都与我摆开。”叫道:“刘知寨,你压
着囚车。”刘高在马上,死应不得,只口里念道:“救苦救难天尊。”便许下十万
卷经,三百座寺,救一救。惊的脸如成精的东瓜,青一回,黄一回。这黄信是个武
官,终有些胆量,便拍马向前看时,只见林子四边齐齐的分过三五百个小喽罗来,
一个个身长力壮,都是面恶眼凶,头裹红巾,身穿衲袄,腰悬利剑,手执长枪,早
把一行人围住。

林子中跳出三个好汉来:一个穿青,一个穿绿,一个穿红。都戴着一顶销金万
字头巾,各跨一口腰刀,又使一把朴刀,当住去路。中间是锦毛虎燕顺,上首是矮
脚虎王英,下首是白面郎君郑天寿。三个好汉大喝道:“来往的到此当住脚,留下
三千两买路黄金,任从过去。”黄信在马上大喝道:“你那厮们,不得无礼,镇三
山在此!”三个好汉睁着眼,大喝道:“你便是镇万山,也要三千两买路黄金;没
时,不放你过去。”黄信说道:“我是上司取公事的都监,有甚么买路钱与你?”
那三个好汉笑道:“莫说你是上司一个都监,便是赵官家驾过,也要三千贯买路钱;
若是没有,且把公事人当在这里,待你取钱来赎。”黄信大怒,骂道:“强贼,怎
敢如此无礼!”喝叫左右擂鼓鸣锣。黄信拍马舞剑,直奔燕顺。三个好汉一齐挺起
朴刀,来战黄信。

黄信见三个好汉都来并他,奋力在马上斗了十合,怎地当得他三个住?亦且刘
高是个文官,又向前不得,见了这般势头,只待要走。黄信怕吃他三个拿了,坏了
名声,只得一骑马,扑喇喇跑回旧路,三个头领,挺着朴刀赶将来。黄信那里顾得
众人,独自飞马奔回清风镇去了。众军见黄信回马时,已自发声喊,撇了囚车,都
四散走了。只剩得刘高,见势头不好,慌忙勒转马头,连打三鞭。那马正待跑时,
被那小喽罗拽起绊马索,早把刘高的马掀翻,倒撞下来。众小喽罗一发向前,拿了
刘高,抢了囚车,打开车辆,花荣已把自己的囚车掀开了,便跳出来,将这缚索都
挣断了,却打碎那个囚车,救出宋江来。自有那几个小喽罗,已自反剪了刘高,又
向前去抢得他骑的马,亦有三匹驾车的马,却剥了刘高的衣服,与宋江穿了,把马
先送上山去。这三个好汉,一同花荣并小喽罗,把刘高赤条条的绑了押回山寨来。

原来这三位好汉,为因不知宋江消息,差几个能干的小喽罗下山,直来清风镇
上探听,闻人说道:“都监黄信掷盏为号,拿了花知寨并宋江,陷车囚了,解投青
州来。”因此报与三个好汉得知,带了人马,大宽转兜出大路来,预先截住去路,
小路里亦差人伺候。因此救了两个,拿得刘高,都回山寨里来。

当晚上的山时,已是二更时分,都到聚义厅上相会,请宋江、花荣当中坐定,
三个好汉对席相陪,一面且备酒食管待。燕顺分付,叫孩儿们各自都去吃酒。花荣
在厅上称谢三个好汉,说道:“花荣与哥哥皆得三位壮士救了性命,报了冤仇,此
恩难报。只是花荣还有妻小妹子在清风寨中,必然被黄信擒捉,却是怎生救得?”
燕顺道:“知寨放心:料应黄信不敢便拿恭人,若拿时,也须从这条路里经过。我
明日弟兄三个下山,去取恭人和令妹还知寨。”便差小喽罗下山,先去探听。花荣
谢道:“深感壮士大恩。”宋江便道:“且与我拿过刘高那厮来。”燕顺便道:“把
他绑在将军柱上,割腹取心,与哥哥庆喜。”花荣道:“我亲自下手割这厮。”

宋江骂道:“你这厮,我与你往日无冤,近日无仇,你如何听信那不贤的妇人
害我!今日擒来,有何理说?”花荣道:“哥哥问他则甚?”把刀去刘高心窝里只
一剜,那颗心献在宋江面前,小喽罗自把尸首拖在一边。宋江道:“今日虽杀了这
厮滥污匹夫,只有那个淫妇,不曾杀得,出那口大气。”王矮虎便道:“哥哥放心,
我明日自下山去,拿那妇人,今番还我受用。”众皆大笑。当夜饮酒罢,各自歇息。
次日起来,商议打清风寨一事。燕顺道:“昨日孩儿们走得辛苦了,今日歇他一日,
明日早下山去也未迟。”宋江道:“也见得是,正要将息人强马壮,不在促忙。”

不说山寨整点军马起程,且说都监黄信一骑马奔回清风镇上大寨内,便点寨兵
人马,紧守四边栅门。黄信写了申状,叫两个教军头目,飞马报与慕容知府。知府
听得飞报军情紧急公务,连夜升厅,看了黄信申状:反了花荣,结连清风山强盗,
时刻清风寨不保,事在告急,早遣良将保守地方。知府看了大惊,便差人去请青州
指挥司总管本州兵马秦统制,急来商议军情重事。那人原是山后开州人氏,姓秦,
讳个明字,因他性格急躁,声若雷霆,以此人都呼他做霹雳火秦明。祖是军官出身,
使一条狼牙棒,有万夫不当之勇。

那人听得知府请唤,径到府里来见知府,各施礼罢。那慕容知府将出那黄信的
飞报申状来,教秦统制看了,秦明大怒道:“红头子敢如此无礼!不须公祖忧心,
不才便起军马,不拿了这贼,誓不再见公祖!”慕容知府道:“将军若是迟慢,恐
这厮们去打清风寨。”秦明答道:“此事如何敢迟误?只今连夜便去点起人马,来
日早行。”知府大喜,忙叫安排酒肉干粮,先去城外等候赏军。秦明见说反了花荣,
怒忿忿地上马,奔到指挥司里,便点起一百马军、四百步军,先叫出城去取齐,摆
布了起身。

却说慕容知府先在城外寺院里蒸下馒头,摆了大碗,烫下酒,每一个人三碗酒,
两个馒头,一斤熟肉。方才备办得了,却望见军马出城,看那军马时,摆得整齐。
但见:

烈烈旌旗似火,森森戈戟如麻。阵分八卦摆长蛇,委实神惊鬼怕。枪见绿沉紫
焰,旗飘绣带红霞,马蹄来往乱交加。乾坤生杀气,成败属谁家。

当日清早,秦明摆布军马,出城取齐,引军红旗上大书“兵马总管秦统制”,
领兵起行。慕容知府看见秦明全副披挂了出城来,果是英雄无比。但见:

盔上红缨飘烈焰,锦袍血染猩猩,连环锁甲砌金星。云根靴抹绿,龟背铠堆银。
坐下马如同獬豸,狼牙棒密嵌铜钉,怒时两目便圆睁。性如霹雳火,虎将是秦明。

当下霹雳火秦明在马上出城来,见慕容知府在城外赏军,慌忙叫军汉接了军器,
下马来和知府相见,施礼罢,知府把了盏,将些言语嘱付总管道:“善觑方便,早
奏凯歌。”赏军已罢,放起信炮,秦明辞了知府,飞身上马,摆开队伍,催趱军兵,
大刀阔斧,径奔清风寨来。原来这清风镇却在青州东南上,从正南取清风山较近,
可早到山北小路。

却说清风山寨里这小喽罗们探知备细,报上山来。山寨里众好汉正待要打清风
寨去,只听的报道:“秦明引兵马到来。”都面面厮觑,俱各骇然。花荣便道:“你
众位俱不要慌。自古兵临告急,必须死敌,教小喽罗饱吃了酒饭,只依着我行。先
须力敌,后用智取,如此如此,好么?”宋江道:“好计!正是如此行。”当日宋
江、花荣先定了计策,便叫小喽罗各自去准备。花荣自选了一骑好马,一副衣甲,
弓箭铁枪,都收拾了等候。

再说秦明领兵来到清风山下,离山十里,下了寨栅。次日五更造饭,军士吃罢,
放起一个信炮,直奔清风山来,拣空阔去处,摆开人马,发起擂鼓,只听见山上锣
声震天响,飞下一彪人马出来。秦明勒住马,横着狼牙棒,睁着眼看时,却见众小
喽罗簇拥着小李广花荣下山来。到得山坡前,一声锣响,列成阵势,花荣在马上擎
着铁枪,朝秦明声个喏。秦明大喝道:“花荣,你祖代是将门之子,朝廷命官,教
你做个知寨,掌握一境地方,食禄于国,有何亏你处?却去结连贼寇,反背朝廷。
我今特来捉你,会事的下马受缚,免得腥手污脚。”花荣陪着笑道:“总管容复听
禀:量花荣如何肯反背朝廷?实被刘高这厮,无中生有,官报私仇,逼迫得花荣有
家难奔,有国难投,权且躲避在此,望总管详察救解。”秦明道:“你兀自不下马
受缚,更待何时?地花言巧语,煽惑军心。”喝叫左右两边擂鼓。秦明抡动狼牙
棒,直奔花荣。花荣大笑道:“秦明,你这厮原来不识好人饶让。我念你是个上司
官,你道俺真个怕你!”便纵马挺枪,来战秦明。两个就清风山下厮杀,真乃是棋
逢敌手难藏幸,将遇良材好用功。这两个将军比试,但见:

一对南山猛虎,两条北海苍龙。龙怒时头角峥嵘,虎斗处爪牙狞恶。爪牙狞恶,
似银钩不离锦毛团;头角峥嵘,如铜叶振摇金色树。翻翻复复,点钢枪没半米放闲;
往往来来,狼牙棒有千般解数。狼牙棒当头劈下,离顶门只隔分毫;点钢枪用力刺
来,望心坎微争半指。使点钢枪的壮士,威风上逼斗牛寒;舞狼牙棒的将军,怒气
起如云电发。一个是扶持社稷天蓬将,一个是整顿江山黑煞神。

当下秦明和花荣两个交手,斗到四五十合,不分胜败。花荣连斗了许多合,卖
个破绽,拨回马望山下小路便走。秦明大怒,赶将来。花荣把枪去了事环上带住,
把马勒个定,左手拈起弓,右手拔箭,拽满弓,扭过身躯,望秦明盔顶上只一箭,
正中盔上,射落斗来大那颗红缨,却似报个信与他。秦明吃了一惊,不敢向前追赶,
霍地拨回马,恰待赶杀,众小喽罗一哄地都上山去了。花荣自从别路,也转上山寨
去了。

秦明见他都走散了,心中越怒道:“叵耐这草寇无礼!”喝叫鸣锣擂鼓,取路
上山。众军齐声呐喊,步军先上山来。转过三两个山头,只见上面擂木、炮石、灰
瓶、金汁,从峻处打将下来。向前的退步不迭,早打倒三五十个,只得再退下山
来。

秦明是个性急的人,心头火起,那里按纳得住?带领军马,绕山下来,寻路上
山。寻到午牌时分,只见西山边锣响,树林丛中闪出一对红旗军来。秦明引了人马,
赶将去时,锣也不响,红旗都不见了。秦明看那路时,又没正路,都只是几条砍柴
的小路,却把乱树折木,交叉当了路口,又不能上去得。

正待差军汉开路,只见军汉来报道:“东山边锣响,一阵红旗军出来。”秦明
引了人马,飞也似奔过东山边来,看时,锣也不鸣,红旗也不见了。秦明纵马去四
下里寻路时,都是乱树折木,断塞了砍柴的路径。

只见探事的又来报道:“西边山上锣又响,红旗军又出来了。”秦明拍马再奔
来西山边,看时,又不见一个人,红旗也没了。秦明是个急性的人,恨不得把牙齿
都咬碎了。

正在西山边气忿忿的,又听得东山边锣声震地价响,急带了人马,又赶过来东
山边,看时,又不见有一个贼汉,红旗都不见了。

秦明气满胸脯,又要赶军汉上山寻路,只听得西山边又发起喊来。秦明怒气冲
天,大驱兵马,投西山边来,山上山下看时,并不见一个人。秦明喝叫军汉,两边
寻路上山,数内有一个军人禀说道:“这里都不是正路,只除非东南上有一条大路,
可以上去。若是只在这里寻路上去时,惟恐有失。”秦明听了,便道:“既有那条
大路时,连夜赶将去。”便驱一行军马奔东南角上来。

看看天色晚了,又走得人困马乏;巴得到那山下时,正欲下寨造饭,只见山上
火把乱起,锣鼓乱鸣。秦明转怒,引领四五十马军跑上山来。只见山上树林内乱箭
射将下来,又射伤了些军士,秦明只得回马下山,且教军士只顾造饭。恰才举得火
着,只见山上有八九十把火光,呼风唿哨下来。秦明急待引军赶时,火把一齐都灭
了。当夜虽有月光,亦被阴云笼罩,不甚明朗。秦明怒不可当,便叫军士点起火把,
烧那树木,只听得山嘴上鼓笛之声。秦明纵马上来看时,见山顶上点着十余个火把,
照见花荣陪侍着宋江在上面饮酒。秦明看了,心中没出气处,勒着马,在山下大骂。
花荣回言道:“秦统制,你不必焦躁,且回去将息着,我明日和你并个你死我活的
输赢便罢。”秦明大叫道:“反贼!你便下来,我如今和你并个三百合,却再做理
会。”花荣笑道:“秦总管,你今日劳困了,我便赢得你,也不为强。你且回去,
明日却来。”秦明越怒,只管在山下骂,本待寻路上山,却又怕花荣的弓箭,因此
只在山坡下骂。

正叫骂之间,只听得本部下军马发起喊来。秦明急回到山下看时,只见这边山
上,火炮火箭,一齐烧将下来;背后二三十个小喽罗做一群,把弓弩在黑影里射人。
众军马发喊,一齐都拥过那边山侧深坑里去躲。此时已有三更时分,众军马正躲得
弩箭时,只叫得苦,上溜头滚下水来,一行人马却都在溪里,各自挣扎性命。爬得
上岸的,尽被小喽罗挠钩搭住,活捉上山去了;爬不上岸的,尽淹死在溪里。

且说秦明此时怒气冲天,脑门粉碎,却见一条小路在侧边,秦明把马一拨,抢
上山来。走不到三五十步,和人连马攧下陷坑里去。两边埋伏下五十个挠钩手,把
秦明搭将起来,剥了浑身战袄、衣甲、头盔、军器,拿条绳索绑了,把马也救起来,
都解上清风山来。

原来这般圈套,都是花荣和宋江的计策。先使小喽罗或在东,或在西,引诱的
秦明人困马乏,策立不定。预先又把这土布袋填住两溪的水,等候夜深,却把人马
逼赶溪里去,上面却放下水来。那急流的水都结果了军马。你道秦明带出的五百人
马,一大半淹死在水中,都送了性命;生擒活捉得一百五七十人,夺了七八十匹好
马,不曾逃得一个回去。次后陷马坑里活捉了秦明。

当下一行小喽罗捉秦明到山寨里,早是天明时候。五位好汉坐在聚义厅上,小
喽罗缚绑秦明解在厅前。花荣见了,连忙跳离交椅,接下厅来,亲自解了绳索,扶
上厅来,纳头拜在地下。秦明慌忙答礼,便道:“我是被擒之人,由你们碎尸而死,
何故却来拜我?”花荣跪下道:“小喽罗不识尊卑,误有冒渎,切乞恕罪。”随即
便取衣服与秦明穿了。秦明问花荣道:“这位为头的好汉,却是甚人?”花荣道:
“这位是花荣的哥哥,郓城县宋押司宋江的便是。这三位是山寨之主:燕顺、王英、
郑天寿。”秦明道:“这三位我自晓得,这宋押司莫不是唤做山东及时雨宋公明么?”
宋江答道:“小人便是。”秦明连忙下拜道:“闻名久矣,不想今日得会义士!”
宋江慌忙答礼不迭。秦明见宋江腿脚不便,问道:“兄长如何贵足不便?”宋江却
把自离郓城县起头,直至刘知寨拷打的事故,从头对秦明说了一遍。秦明只把头来
摇道:“若听一面之词,误了多少缘故。容秦明回州去对慕容知府说知此事。”燕
顺相留且住数日,随即便叫杀牛宰马,安排筵席饮宴。拿上山的军汉,都藏在山后
房里,也与他酒食管待。

秦明吃了数杯,起身道:“众位壮士,既是你们的好情分,不杀秦明,还了我
盔甲、马匹、军器,回州去。”燕顺道:“总管差矣。你既是引了青州五百兵马,
都没了,如何回得州去?慕容知府如何不见你罪责?不如权在荒山草寨住几时。本不
堪歇马,权就此间落草,论秤分金银,整套穿衣服,不强似受那大头巾的气?”秦
明听罢,便下厅道:“秦明生是大宋人,死是大宋鬼。朝廷教我做到兵马总管,兼
受统制使官职,又不曾亏了秦明,我如何肯做强人,背反朝廷?你们众位要杀时,
便杀了我,休想我随顺你们!”花荣赶下厅来拖住道:“秦兄长息怒,听小弟一言:
我也是朝廷命官之子,无可奈何,被逼迫的如此。总管既是不肯落草,如何相逼得
你随顺?只且请少坐,席终了时,小弟讨衣甲、头盔、鞍马、军器还兄长去。”秦
明那里肯坐?花荣又劝道:“总管夜来劳神费力了一日一夜,人也尚自当不得,那
匹马如何不喂得他饱了去?”秦明听了,肚内寻思,也说得是。再上厅来,坐了饮
酒。那五位好汉轮番把盏,陪话劝酒。秦明一则软困,二乃吃众好汉劝不过,开怀
吃得醉了,扶入帐房睡了。这里众人自去行事,不在话下。

且说秦明一觉直睡到次日辰牌方醒,跳将起来,洗漱罢,便要下山。众好汉都
来相留道:“总管,且吃早饭动身,送下山去。”秦明性急的人,便要下山。众人
慌忙安排些酒食管待了,取出头盔、衣甲,与秦明披挂了,牵过那匹马来,并狼牙
棒,先叫人在山下伺候,五位好汉都送秦明下山来,相别了,交还马匹军器。

秦明上了马,拿着狼牙棒,趁天色大明,离了清风山,取路飞奔青州来。到得
十里路头,恰好巳牌前后,远远地望见烟尘乱起,并无一个人来往。秦明见了,心
中自有八分疑忌,到得城外看时,原来旧有数百人家,却都被火烧做白地,一片瓦
砾场上,横七竖八,杀死的男子妇人,不计其数。秦明看了大惊,打那匹马在瓦砾
场上,跑到城边,大叫开门时,只见门边吊桥高拽起了,都摆列着军士旌旗,擂木
炮石。秦明勒着马大叫:“城上放下吊桥,度我入城。”城上早有人看见是秦明,
便擂起鼓来,呐着喊。秦明叫道:“我是秦总管,如何不放我入城?”只见慕容知
府立在城上女墙边大喝道:“反贼,你如何不识羞耻!昨夜引人马来打城子,把许
多好百姓杀了,又把许多房屋烧了,今日兀自又来赚哄城门。朝廷须不曾亏负了你,
你这厮倒如何行此不仁!已自差人奏闻朝廷去了,早晚拿住你时,把你这厮碎尸万
段!”秦明大叫道:“公祖差矣!秦明因折了人马,又被这厮们捉了上山去,方才
得脱,昨夜何曾来打城子?”知府喝道:“我如何不认的你这厮的马匹、衣甲、军
器、头盔?城上众人明明地见你指拨红头子杀人放火,你如何赖得过?便做你输了被
擒,如何五百军人没一个逃得回来报信?你如今指望赚开城门取老小,你的妻子,
今早已都杀了。你若不信,与你头看。”军士把枪将秦明妻子首级挑起在枪上,教
秦明看。秦明是个性急的人,看了浑家首级,气破胸脯,分说不得,只叫得苦屈。
城上弩箭如雨点般射将下来,秦明只得回避,看见遍野处火焰,尚兀自未灭。

秦明回马在瓦砾场上,恨不得寻个死处,肚里寻思了半晌,纵马再回旧路。行
不得十来里,只见林子里转出一伙人马来,当先五匹马上五个好汉,不是别人,宋
江、花荣、燕顺、王英、郑天寿,随从一二百小喽罗。宋江在马上欠身道:“总管
何不回青州?独自一骑投何处去?”秦明见问,怒气道:“不知是那个天不盖、地
不载、该剐的贼,装做我去打了城子,坏了百姓人家房屋,杀害良民,倒结果了我
一家老小,闪得我如今上天无路,入地无门!我若寻见那人时,直打碎这条狼牙棒
便罢!”宋江便道:“总管息怒,既然没了夫人,不妨,小人自当与总管做媒。我
有个好见识,请总管回去,这里难说。且请到山寨里告禀,一同便往。”秦明只得
随顺,再回清风山来。于路无话,早到山亭前下马,众人一齐都进山寨内,小喽罗
已安排酒果肴馔在聚义厅上,五个好汉,邀请秦明上厅,都让他中间坐定。五个好
汉齐齐跪下,秦明连忙答礼,也跪在地。

宋江开话道:“总管休怪,昨日因留总管在山,坚意不肯,却是宋江定出这条
计来,叫小卒似总管模样的,却穿了足下的衣甲、头盔,骑着那马,横着狼牙棒,
直奔青州城下,点拨红头子杀人。燕顺、王矮虎带领五十余人助战,只做总管去家
中取老小。因此杀人放火,先绝了总管归路的念头。今日众人特地请罪。”秦明见
说了,怒气于心,欲待要和宋江等厮并,却又自肚里寻思:一则是上界星辰契合,
二乃被他们软困,以礼待之,三则又怕斗他们不过。因此只得纳了这口气,便说道:
“你们弟兄虽是好意,要留秦明,只是害得我忒毒些个,断送了我妻小一家人口。”
宋江答道:“不恁地时,兄长如何肯死心塌地?若是没了嫂嫂夫人,宋江恰知得花
知寨有一妹,甚是贤慧,宋江情愿主婚,陪备财礼,与总管为室如何?”秦明见众
人如此相敬相爱,方才放心归顺。

众人都让宋江在居中坐了,秦明上首,花荣肩下,三位好汉依次而坐,大吹大
擂饮酒,商议打清风寨一事。秦明道:“这事容易,不须众弟兄费心。黄信那人,
亦是治下;二者是秦明教他的武艺;三乃和我过的最好。明日我便先去叫开栅门,
一席话,说他入伙投降,就取了花知寨宝眷,拿了刘高的泼妇,与仁兄报仇雪恨,
作进见之礼如何?”宋江大喜道:“若得总管如此慨然相许,却是多幸多幸!”当
日筵席散了,各自歇息。次日早起来,吃了早饭,都各各披挂了。秦明上马,先下
山来,拿了狼牙棒,飞奔清风镇来。

却说黄信自到清风镇上,发放镇上军民,点起寨兵,晓夜提防,牢守栅门,又
不敢出战,累累使人探听,不见青州调兵策应。当日只听得报道:“栅外有秦统制
独自一骑马到来,叫开栅门。”黄信听了,便上马飞奔门边看时,果是一人一骑,
又无伴当。黄信便叫开栅门,放下吊桥,迎接秦总管入来,直到大寨公厅前下马,
请上厅来,叙礼罢,黄信便问道:“总管缘何单骑到此?”秦明当下先说了损折军
马等情,后说:“山东及时雨宋公明疏财仗义,结识天下好汉,谁不钦敬他?如今
现在清风山上,我今次也在山寨入了伙。你又无老小,何不听我言语,也去山寨入
伙,免受那文官的气。”黄信答道:“既然恩官在彼,黄信安敢不从?只是不曾听
得说有宋公明在山上,今次却说及时雨宋公明,自何而来?”秦明笑道:“便是你
前日解去的郓城虎张三便是,他怕说出真名姓,惹起自己的官司,以此只认说是张
三。”黄信听了,跌脚道:“若是小弟得知是宋公明时,路上也自放了他。一时见
不到处,只听了刘高一面之词,险不坏了他性命。”秦明、黄信两个正在公廨内商
量起身,只见寨兵报道:“有两路军马,鸣锣擂鼓,杀奔镇上来。”秦明、黄信听
得,都上了马,前来迎敌。军马到得栅门边望时,只见:尘土蔽日,杀气遮天,两
路军兵投镇上,四条好汉下山来。

毕竟秦明、黄信怎地迎敌,且听下回分解。