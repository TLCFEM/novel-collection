\chapter{混江龙太湖小结义~宋公明苏州大会垓}

话说当下众将救起宋江,半晌方才苏醒,对吴用等说道:“我们今番必然收伏
不得方腊了!自从渡江以来,如此不利,连连损折了我八个弟兄。”吴用劝道:“主
帅休说此言,恐懈军心。当初破大辽之时,大小完全回京,皆是天数。今番折了兄
弟们,此是各人寿数。眼见得渡江以来,连得了三个大郡:润州、常州、宣州。此
乃皆是天子洪福齐天,主将之虎威,如何不利?先锋何故自丧志气?”宋江道:“虽
然天数将尽,我想一百八人,上应列宿,又合天文所载,兄弟们如手足之亲。今日
听了这般凶信,不由我不伤心。”吴用再劝道:“主将请休烦恼,勿伤贵体。且请
理会调兵接应,攻打无锡县。”宋江道:“留下柴大官人与我做伴。别写军帖,使
戴院长与我送去,回复卢先锋,着令进兵攻打湖州,早至杭州聚会。”吴用教裴宣
写了军帖回复,使戴宗往宣州去了,不在话下。

却说吕师囊引着许定逃回至无锡县,正迎着苏州三大王发来救应军兵,为头是六军
指挥使卫忠,带十数个牙将,引兵一万,来救常州,合兵一处,守住无锡县。吕枢
密诉说金节献城一事,卫忠道:“枢密宽心,小将必然再要恢复常州。”只见探马
报道:“宋军至近,早作准备。”卫忠便引兵上马,出北门外迎敌。早见宋兵军马
势大,为头是黑旋风李逵,引着鲍旭、项充、李衮,当先直杀过来。卫忠力怯,军
马不曾摆成行列,大败而走;急退入无锡县时,四个早随马后赶入县治。吕枢密便
奔南门而走。关胜引着兵马,已夺了无锡县;卫忠、许定亦望南门走了,都回苏州
去了。关胜等得了县治,便差人飞报宋先锋。宋江与众头领都到无锡县,便出榜安
抚了本处百姓,复为良民,引大队军马,都屯住在本县,却使人申请张、刘二总兵,
镇守常州。

且说吕枢密会同卫忠、许定三个,引了败残军马,奔苏州城来告三大王求救,诉说
宋军势大,迎敌不住,兵马席卷而来,以致失陷城池。三大王大怒,喝令武士,推
转吕枢密,斩讫报来。卫忠等告说:“宋江部下军将,皆是惯战兵马,多有勇烈好
汉了得的人,更兼步卒都是梁山泊小喽罗,多曾惯斗,因此难敌。”方貌道:“权
且寄下你项上一刀,与你五千军马,首先出哨。我自分拨大将,随后便来策应。”
吕师囊拜谢了,全身披挂,手执丈八蛇矛,上马引军,首先出城。

却说三大王聚集手下八员战将,名为八骠骑,一个个都是身长力壮,武艺精熟的人。
那八员:
飞龙大将军刘赟

飞虎大将军张威
飞熊大将军徐方

飞豹大将军郭世广
飞天大将军邬福

飞云大将军苟正
飞山大将军甄诚

飞水大将军昌盛
当下三大王方貌,亲自披挂,手持方天画戟,上马出阵,监督中军人马,前来交战。
马前摆列着那八员大将,背后整整齐齐有三二十个副将,引五万南兵人马,出阊阖
门来,迎敌宋军。前部吕师囊引着卫忠、许定,已过寒山寺了,望无锡县而来。宋
江已使人探知,尽引许多正偏将佐,把军马调出无锡县,前进十里余路。两军相遇,
旗鼓相望,各列成阵势。吕师囊忿那口气,跃坐下马,横手中矛,亲自出阵,要与
宋江交战。宋江在门旗下见了,回头问道:“谁人敢拿此贼?”说犹未了,金枪手
徐宁挺起手中金枪,骤坐下马,出到阵前,便和吕枢密交战。二将交锋,左右助喊,
约战了二十余合,吕师囊露出破绽来,被徐宁肋下刺着一枪,搠下马去。两军一齐
呐喊。黑旋风李逵手挥双斧,丧门神鲍旭挺仗飞刀,项充、李衮各舞枪牌,杀过阵
来,南兵大乱。

宋江驱兵赶杀,正迎着方貌大队人马,两边各把弓箭射住阵脚,各列成阵势。南军
阵上,一字摆开八将。方貌在中军听得说杀了吕枢密,心中大怒,便横戟出马来,
大骂宋江道:“量你等只是梁山泊一伙打家劫舍的草贼,宋朝合败,封你为先锋,
领兵侵入吾地,我今直把你诛尽杀绝,方才罢兵!”宋江在马上指道:“你这厮只
是睦州一伙村夫,量你有甚福禄,妄要图王霸业。不如及早投降,免汝一死!天兵
到此,尚自巧言抗拒!我若不把你杀尽,誓不回军!”方貌喝道:“且休与你论口,
我手下有八员猛将在此,你敢拨八个出来厮杀么?”宋江笑道:“若是我两个并你
一个,也不算好汉。你使八个出来,我使八员首将,和你比试本事,便见输赢。但
是杀下马的,各自抬回本阵,不许暗箭伤人,亦不许抢掳尸首。如若不见输赢,不
得混战,明日再约厮杀。”方貌听了,便叫八将出来,各执兵器,骤马向前。宋江
道:“诸将相让马军出战。”说言未绝,八将齐出,那八人:关胜、花荣、徐宁、
秦明、朱仝、黄信、孙立、郝思文。宋江阵内,门旗开处,左右两边,分出八员首
将,齐齐骤马,直临阵上。两军中花腔鼓擂,杂彩旗摇,各家放了一个号炮,两军
助着喊声,十六骑马齐出,各自寻着敌手,捉对儿厮杀。那十六员将佐,如何见得
寻着对手,配合交锋?关胜战刘赟,秦明战张威,花荣战徐方,徐宁战邬福,朱仝
战苟正,黄信战郭世广,孙立战甄诚,郝思文战昌盛,真乃是难描难画。但见:

征尘乱起,杀气横生。人人欲作那吒,个个争为敬德。三十二条臂膊,如织锦
穿梭;六十四只马蹄,似追风走雹。队旗错杂,难分赤白青黄;兵器交加,莫辨枪
刀剑戟。试看旋转烽烟里,真似元宵走马灯。
这十六员猛将,都是英雄,用心相敌,斗到三十合之上,数中一将,翻身落马,赢
得的是谁?美髯公朱仝,一枪把苟正刺下马来。两阵上各自鸣金收军,七对将军分
开,两下各回本阵。
三大王方貌,见折了一员大将,寻思不利,引兵退回苏州城内。宋江当日催趱军马,
直近寒山寺下寨,升赏朱仝。裴宣写了军状,申复张招讨,不在话下。

且说三大王方貌退兵入城,坚守不出,分调诸将,守把各门,深栽鹿角。城上列着
踏弩硬弓,擂木炮石,窝铺内熔煎金汁,女墙边堆垛灰瓶,准备牢守城池。

次日,宋江见南兵不出,引了花荣、徐宁、黄信、孙立,带领三千余骑马军,前来
看城。见苏州城郭,一周遭都是水港环绕,墙垣坚固,想道:“急不能够打得城破。”
回到寨中,和吴用计议攻城之策。有人报道:“水军头领正将李俊,从江阴来见主
将。”宋江教请入帐中。见了李俊,宋江便问沿海消息。李俊答道:“自从拨领水
军,一同石秀等杀至江阴、太仓沿海等处,守将严勇、副将李玉部领水军船只,出
战交锋。严勇在船上被阮小二一枪搠下水去,李玉已被乱箭射死,因此得了江阴、
太仓。即日石秀、张横、张顺去取嘉定,三阮去取常熟,小弟特来报捷。”宋江见
说大喜,赏赐了李俊,着令自往常州,去见张、刘二招讨,投下申状。

且说这李俊径投常州来,见了张招讨、刘都督,备说收复了江阴、太仓海岛去处,
杀了贼将严勇、李玉。张招讨给与了赏赐,令回宋先锋处听调。李俊回到寒山寺寨
中,来见宋先锋。宋江因见苏州城外水面空阔,必用水军船只厮杀,因此就留下李
俊,教整点船只,准备行事。李俊说道:“容俊去看水面阔狭,如何用兵,却作道
理。”宋江道:“是。”李俊去了两日,回来说道:“此城正南上相近太湖,兄弟
欲得备舟一只,投宜兴小港,私入太湖里去,出吴江,探听南边消息,然后可以进
兵,四面夹攻,方可得破。”宋江道:“贤弟此言极当!只是没有副手与你同去。”
随即便拨李大官人带同孔明、孔亮、施恩、杜兴四个,去江阴、太仓、昆山、常熟、
嘉定等处,协助水军,收复沿海县治,便可替回童威、童猛,来帮助李俊行事。李
应领了军帖,辞别宋江,引四员偏将投江阴去了。不过两日,童威、童猛回来,参
见宋先锋。宋江抚慰了,就叫随从李俊,乘驾小船前去探听南边消息。

且说李俊带了童威、童猛,驾起一叶扁舟,两个水手摇橹,五个人径奔宜兴小港里
去,盘旋直入太湖中来。看那太湖时,果然水天空阔,万顷一碧。但见:

天连远水,水接遥天。高低水影无尘,上下天光一色。双双野鹭飞来,点破碧
琉璃;两两轻鸥惊起,冲开青翡翠。春光淡荡,溶溶波皱鱼鳞;夏雨滂沱,滚滚浪
翻银屋。秋蟾皎洁,金蛇游走波澜;冬雪纷飞,玉蝶弥漫天地。混沌凿开元气窟,
冯夷独占水晶宫。
有诗为证:
溶溶漾漾白鸥飞,绿净春深好染衣。
南去北来人自老,夕阳常送钓船归。
当下李俊和童威、童猛并两个水手,驾着一叶小船,径奔太湖,渐近吴江,远远望
见一派渔船,约有四五十只。李俊道:“我等只做买鱼,去那里打听一遭。”五个
人一径摇到那打鱼船边,李俊问道:“渔翁,有大鲤鱼吗?”渔人道:“你们要大
鲤鱼,随我家里去卖与你。”李俊摇着船,跟那几只鱼船去。没多时,渐渐到一个
处所。看时,团团一遭,都是驼腰柳树,篱落中有二十余家。那渔人先把船来缆了,
随即引李俊、童威、童猛三人上岸,到一个庄院里。一脚入得庄门,那人嗽了一声,
两边钻出七八条大汉,都拿着挠钩,把李俊三人一齐搭住,径捉入庄里去,不问事
情,便把三人都绑在桩木上。

李俊把眼看时,只见草厅上坐着四个好汉。为头那个赤须黄发,穿着领青绸衲袄;
第二个瘦长短髯,穿着一领黑绿盘领木绵衫;第三个黑面长须;第四个骨脸阔腮,
扇圈胡须。两个都一般穿着领青衲袄子,头上各带黑毡笠儿,身边都倚着军器。为
头那个喝问李俊道:“你等这厮们,都是那里人氏?来我这湖泊里做甚么?”李俊
应道:“俺是扬州人,来这里做客,特来买鱼。”那第四个骨脸的道:“哥哥休问
他,眼见得是细作了。只顾与我取他心肝来吃酒。”李俊听得这话,寻思道:“我
在浔阳江上,做了许多年私商,梁山泊内又妆了几年的好汉,却不想今日结果性命
在这里!罢,罢,罢!”叹了口气,看着童威、童猛道:“今日是我连累了兄弟两
个,做鬼也只是一处去!”童威、童猛道:“哥哥休说这话,我们便死也够了。只
是死在这里,埋没了兄长大名。”三面厮觑着,腆起胸脯受死。

那四个好汉却看了他们三个,说了一回,互相厮觑道:“这个为头的人,必不是以
下之人。”那为头的好汉又问道:“你三个正是何等样人?可通个姓名,教我们知
道。”李俊又应道:“你们要杀便杀,我等姓名,至死也不说与你,枉惹的好汉们
耻笑!”那为头的见说了这话,便跳起来,把刀都割断了绳索,放起这三个人来。
四个渔人,都扶他至屋内请坐。为头那个纳头便拜,说道:“我等做了一世强人,
不曾见你这般好义气人物!好汉,三位老兄正是何处人氏?愿闻大名姓字。”李俊道:
“眼见得你四位大哥,必是个好汉了。便说与你,随你们拿我三个那里去。我三个
是梁山泊宋公明手下副将。我是混江龙李俊。这两个兄弟:一个是出洞蛟童威,一
个是翻江蜃童猛。今来受了朝廷招安,新破辽国,班师回京,又奉敕命,来收方腊。
你若是方腊手下人员,便解我三人去请赏,休想我们挣扎!”

那四个听罢,纳头便拜,齐齐跪道:“有眼不识泰山,却才甚是冒渎,休怪!休怪!
俺四个兄弟,非是方腊手下,原旧都在绿林丛中讨衣吃饭。今来寻得这个去处,地
名唤做榆柳庄,四下里都是深港,非船莫能进。俺四个只着打鱼的做眼,太湖里面
寻些衣食。近来一冬,都学得些水势,因此无人敢来侵傍。俺们也久闻你梁山泊宋
公明招集天下好汉,并兄长大名,亦闻有个浪里白跳张顺,不想今日得遇哥哥!”
李俊道:“张顺是我弟兄,亦做同班水军头领,现在江阴地面,收捕贼人。改日同
他来,却和你们相会。愿求你等四位大名。”为头那一个道:“小弟们因在绿林丛
中走,都有异名,哥哥勿笑!小弟是赤须龙费保,一个是卷毛虎倪云,一个是太湖
蛟卜青,一个是瘦脸熊狄成。”

李俊听说了四个姓名,大喜道:“列位从此不必相疑,喜得是一家人!俺哥哥宋公
明现做收方腊正先锋,即日要取苏州,不得次第,特差我三个人来探路。今既得遇
你四位好汉,可随我去见俺先锋,都保你们做官,待收了方腊,朝廷升用。”费保
道:“容复:若是我四个要做官时,方腊手下,也得个统制做了多时。所以不愿为
官,只求快活。若是哥哥要我四人帮助时,水里水里去,火里火里去;若说保我做
官时,其实不要。”李俊道:“既是恁地,我等只就这里结义为兄弟如何?”四个
好汉见说大喜,便叫宰了一口猪,一羊,致酒设席,结拜李俊为兄。李俊叫童威、
童猛都结义了。

七个人在榆柳庄上商议,说宋公明要取苏州一事,“方貌又不肯出战,城池四面是
水,无路可攻,舟船港狭,难以准敌,似此怎得城子破?”费保道:“哥哥且宽心
住两日。杭州不时间有方腊手下人来苏州公干,可以乘势智取城郭。小弟使几个打
鱼的去缉听,若还有人来时,便定计策。”李俊道:“此言极妙!”费保便唤几个
渔人,先行去了,自同李俊每日在庄上饮酒。在那里住了两三日,只见打鱼的回来
报道:“平望镇上有十数只递运船只,船尾上都插着黄旗,旗上写着‘承造王府衣
甲’,眼见的是杭州解来的。每只船上,只有五七人。”李俊道:“既有这个机会,
万望兄弟们助力。”费保道:“只今便往。”李俊道:“但若是那船上走了一个,
其计不谐了。”费保道:“哥哥放心,都在兄弟身上。”随即聚集六七十只打鱼小
船。七筹好汉,各坐一只,其余都是渔人,各藏了暗器,尽从小港透入大江,四散
接将去。

当夜,星月满天,那十只官船,都湾在江东龙王庙前。费保船先到,忽起一声号哨,
六七十只鱼船,一齐拢来,各自帮住大船。那官船里人急钻出来,早被挠钩搭住,
三个五个,做一串儿缚了。及至跳得下水的,都被挠钩搭上船来。尽把小船带住官
船,都移入太湖深处;直到榆柳庄时,已是四更天气。闲杂之人,都缚做一串,把
大石头坠定,抛在太湖里淹死。捉得两个为头的来问时,原来是守把杭州方腊大太
子南安王方天定手下库官,特奉令旨,押送新造完铁甲三千副,解赴苏州三大王方
貌处交割。李俊问了姓名,要了一应关防文书,也把两个库官杀了。李俊道:“须
是我亲自去和哥哥商议,方可行此一件事。”费保道:“我着人把船渡哥哥,从小
港里到军前觉近便。”就叫两个渔人,摇一只快船送出去。李俊分付童威、童猛并
费保等,且教把衣甲船只,悄悄藏在庄后港内,休得吃人知觉了。费保道:“无事。”
自来打并船只。

却说李俊和两个渔人,驾起一叶快船,径取小港,掉到军前寒山寺上岸。来至寨中,
见了宋先锋,备说前事。吴用听了大喜道:“若是如此,苏州唾手可得。便请主将
传令,就差李逵、鲍旭、项充、李衮,带领冲阵牌手二百人,跟随李俊回太湖庄上,
与费保等四位好汉,如此行计,约在第二日进发。”李俊领了军令,带同一行人,
直到太湖边来。三个先过湖去,却把船只接取李逵等一干人,都到榆柳庄上。李俊
引着李逵、鲍旭、项充、李衮四个,和费保等相见了。费保看见李逵这般相貌,都
皆骇然。邀取二百余人,在庄上置备酒食相待。

到第三日,众人商议定了。费保扮做解衣甲正库官,倪云扮做副使,都穿了南官的
号衣,将带了一应关防文书,众渔人都装做官船上艄公水手,却藏黑旋风等二百余
人将校在船舱里。卜青、狄成押着后船,都带了放火的器械。却欲要行动,只见渔
人又来报道:“湖面上有一只船,在那里摇来摇去。”李俊道:“又来作怪!”急
急自去看时,船头上立着两个人,看来却是神行太保戴宗和轰天雷凌振。李俊唿了
一声号哨,那只船飞也似奔来庄上,到得岸边,上岸来,都相见了。李俊问:“二
位何来?甚事见报?”戴宗道:“哥哥急使李逵来了,正忘却一件大事,特地差我
与凌振赍一百号炮在船里,湖面上寻赶不上,这里又不敢拢来傍岸,教兄弟明早卯
时进城,到得里面,便放这一百个火炮为号。”李俊道:“最好!”便就船里,搬
过炮笼炮架来,都藏埋衣甲船内。费保等闻知是戴宗,又置酒设席管待。凌振带来
十个炮手,都埋伏摆在第三只船内。当夜四更,离庄望苏州来,五更已后,到得城
下。

守门军士在城上望见南国旗号,慌忙报知管门大将,却是飞豹大将军郭世广,亲自
上城来,问了小校备细,接取关防文书,吊上城来看了。郭世广使人赍至三大王府
里,辨看了来文,又差人来监视,却才教放入城门。郭世广直在水门边坐地,再叫
人下船看时,满满地堆着铁甲号衣,因此一只只都放入城去。放过十只船了,便关
水门。三大王差来的监视官员,引着五百军,在岸上跟定,便着湾住了船。李逵、
鲍旭、项充、李衮从船舱里钻出来。监视官见了四个人形容粗丑,急待问是甚人时,
项充、李衮早舞起团牌,飞出一把刀来,把监视官剁下马去。那五百军欲待上船,
被李逵掣起双斧,早跳在岸上,一连砍翻十数个,那五百军人都走了。船里众好汉
并牌手二百余人,一齐上岸,便放起火来。凌振就岸边撒开炮架,搬出号炮,连放
了十数个。那炮震得城楼也动,四下里打将入去。

三大王方貌正在府中计议,听的火炮接连响,惊得魂不附体。各门守将,听得城中
炮响不绝,各引兵奔城中来。各门飞报:“南军都被冷箭射死,宋军已上城了。”
苏州城内,鼎沸起来,正不知多少宋军入城。黑旋风李逵和鲍旭引着两个牌手,在
城里横冲直撞,追杀南兵。李俊、戴宗引着费保四人,护持凌振,只顾放炮。宋江
已调三路军将取城。宋兵杀入城来,南军漫散,各自逃生。

且说三大王方貌急急披挂上马,引了五七百铁甲军夺路,待要杀出南门,不想
正撞见黑旋风李逵这一伙,杀得铁甲军东西乱窜,四散奔走。小巷里又撞出鲁智深,
抡起铁禅杖打将来。方貌抵当不住,独自跃马,再回府来。乌鹊桥下转出武松,赶
上一刀,掠断了马脚,方貌倒攧将下来,被武松再复一刀砍了,提首级径来中军,
参见先锋请功。此时宋江已进城中王府坐下,令诸将各自去城里搜杀南军,尽皆捉
获。单只走了刘赟一个,领了些败残军兵,投秀州去了。有诗为证:
神器从来不可干,僭王称号讵能安?
武松立马诛方貌,留与凶顽做样看。

宋江到王府坐下,便传下号令,休教杀害良民百姓,一面教救灭了四下里火,便出
安民文榜,晓谕军民。次后聚集诸将,到府请功。已知武松杀了方貌,朱仝生擒徐
方,史进生擒了甄诚,孙立鞭打死张威,李俊枪刺死昌盛,樊瑞杀死邬福,宣赞和
郭世广鏖战,你我相伤,都死于饮马桥下。其余都擒得牙将,解来请功。宋江见折
了丑郡马宣赞,伤悼不已,便使人安排花棺彩椁,迎去虎丘山下殡葬。把方貌首级
并徐方、甄诚,解赴常州张招讨军前施行。张招讨就将徐方、甄诚碎剐于市,方貌
首级,解赴京师。回将许多赏赐来苏州,给散众将。张招讨移文申状,请刘光世镇
守苏州,却令宋先锋沿便进兵,收捕贼寇。只见探马报道:“刘都督、耿参谋来守
苏州。”当日众将都跟着宋先锋迎接刘光世等官入城王府安下。参贺已了,宋江众
将自来州治议事,使人去探沿海水军头领消息如何。却早报说沿海诸处县治,听得
苏州已破,群贼各自逃散,海僻县道,尽皆平静了。宋江大喜,申达文书到中军报
捷,请张招讨晓谕旧官复职,另拨中军统制,前去各处守御安民,退回水军头领正
偏将佐,来苏州调用。

数日之间,统制等官各自分投去了。水军头领都回苏州,诉说三阮打常熟,折了施
恩;又去攻取昆山,折了孔亮;石秀、李应等尽皆回了;施恩、孔亮不识水性,一
时落水,俱被淹死。宋江见又折了二将,心中大忧,嗟叹不已。武松念起旧日恩义,
也大哭了一场。

且说费保等四人来辞宋先锋,要回去。宋江坚意相留不肯,重赏了四人,再令李俊
送费保等回榆柳庄去。李俊当时又和童威、童猛送费保等四人到榆柳庄上,费保等
又治酒设席相款。饮酒中间,费保起身与李俊把盏,说出几句言语来,有分教:李
俊离却中原之境,别立化外之基。正是:了身达命蟾离壳,立业成名鱼化龙。
毕竟费保与李俊说出甚言语来,且听下回分解。