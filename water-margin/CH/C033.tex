\chapter{宋江夜看小鳌山~花荣大闹清风寨}

话说这清风山离青州不远,只隔得百里来路。这清风寨却在青州三岔路口,地
名清风镇。因为这三岔路上,通三处恶山,因此特设这清风寨在这清风镇上。那里
也有三五千人家,却离这清风山只有一站多路,当日三位头领自上山去了。

只说宋公明独自一个,背着些包裹,迤逦来到清风镇上,便借问花知寨住处。
那镇上人答道:“这清风寨衙门,在镇市中间。南边有个小寨,是文官刘知寨住宅;
北边那个小寨,正是武官花知寨住宅。”宋江听罢,谢了那人,便投北寨来。到得
门首,见有几个把门军汉,问了姓名,入去通报。只见寨里走出那个少年的军官来,
拖住宋江便拜。那人生得如何,但见:

齿白唇红双眼俊,两眉入鬓常清,细腰宽膀似猿形。能骑乖劣马,爱放海东青。
百步穿杨神臂健,弓开秋月分明,雕翎箭发迸寒星。人称小李广,将种是花荣。
出来的年少将军不是别人,正是清风寨武知寨小李广花荣。那花荣怎生打扮,但见:
身上战袍金翠绣,腰间玉带嵌山犀。
渗青巾帻双环小,文武花靴抹绿低。

花荣见宋江拜罢,喝叫军汉接了包裹、朴刀、腰刀,扶住宋江,直到正厅上,
便请宋江当中凉床上坐了。花荣又纳头拜了四拜,起身道:“自从别了兄长之后,
屈指又早五六年矣,常常念想。听得兄长杀了一个泼烟花,官司行文书各处追捕。
小弟闻得,如坐针毡,连连写了十数封书,去贵庄问信,不知曾到也不?今日天赐,
幸得哥哥到此,相见一面,大慰平生。”说罢又拜。宋江扶住道:“贤弟休只顾讲
礼,请坐了,听在下告诉。”花荣斜坐着。宋江把杀阎婆惜一事,和投奔柴大官人,
并孔太公庄上遇见武松,清风山上被捉,遇燕顺……等事,细细地都说了一遍。花
荣听罢,答道:“兄长如此多磨难,今日幸得仁兄到此,且住数年,却又理会。”
宋江道:“若非兄弟宋清寄书来孔太公庄上时,在下也特地要来贤弟这里走一遭。”
花荣便请宋江去后堂里坐,唤出浑家崔氏,来拜伯伯。拜罢,花荣又叫妹子出来拜
了哥哥。便请宋江更换衣裳鞋袜,香汤沐浴,在后堂安排筵席洗尘。

当日筵宴上,宋江把救了刘知寨恭人的事,备细对花荣说了一遍。花荣听罢,
皱了双眉说道:“兄长没来由,救那妇人做甚么?正好教灭这厮的口!”宋江道:
“却又作怪!我听得说是清风寨知寨的恭人,因此把做贤弟同僚面上,特地不顾王
矮虎相怪,一力要救他下山。你却如何恁的说?”花荣道:“兄长不知:不是小弟
说口,这清风寨是青州紧要去处,若还是小弟独自在这里守把时,远近强人,怎敢
把青州搅得粉碎!近日除将这个穷酸饿醋来做个正知寨,这厮又是文官,又没本事,
自从到任,把此乡间些少上户诈骗,乱行法度,无所不为。小弟是个武官副知寨,
每每被这厮怄气,恨不得杀了这滥污贼禽兽。兄长却如何救了这厮的妇人?打紧这
婆娘极不贤,只是调拨他丈夫行不仁的事,残害良民,贪图贿赂,正好叫那贱人受
些玷辱。兄长错救了这等不才的人。”宋江听了,便劝道:“贤弟差矣!自古道:
‘冤仇可解不可结。’他和你是同僚官,虽有些过失,你可隐恶而扬善。贤弟休如
此浅见。”花荣道:“兄长见得极明。来日公廨内见刘知寨时,与他说过救了他老
小之事。”宋江道:“贤弟若如此,也显你的好处。”花荣夫妻几口儿,朝暮臻臻
至至,献酒供食,伏侍宋江。当晚安排床帐,在后堂轩下请宋江安歇。次日,又备
酒食筵宴管待。

话休絮烦。宋江自到花荣寨里,吃了四五日酒。花荣手下有几个体己人,一日
换一个,拨些碎银子在他身边,每日教相陪宋江去清风镇街上,观看市井喧哗,村
落宫观寺院,闲走乐情。自那日为始,这体己人相陪着闲走,邀宋江去市井上闲玩。
那清风镇上也有几座小勾栏,并茶坊酒肆,自不必说得。当日宋江与这体己人,在
小勾栏里闲看了一回,又去近村寺院道家宫观游赏一回,请去市镇上酒肆中饮酒。
临起身时,那体己人取银两还酒钱。宋江那里肯要他还钱?却自取碎银还了。宋江
归来,又不对花荣说。那个同饮的人欢喜,又落得银子,又得身闲,自此每日拨一
个相陪,和宋江去闲走。每日又只是宋江使钱。自从到寨里,无一个不敬爱他的。
宋江在花荣寨里,住了将及一月有余,看看腊尽春回,又早元宵节近。

且说这清风寨镇上居民,商量放灯一事,准备庆赏元宵。科敛钱物,去土地大
王庙前扎缚起一座小鳌山,上面结彩悬花,张挂五六百碗花灯,土地大王庙内,逞
赛诸般社火。家家门前,扎起灯棚,赛悬灯火。市镇上,诸行百艺都有。虽然比不
得京师,只此也是人间天上。当下宋江在寨里和花荣饮酒,正值元宵。是日晴明得
好,花荣到巳牌前后,上马去公廨内点起数百个军士,教晚间去市镇上弹压;又点
差许多军汉,分头去四下里守把栅门。未牌时分回寨来,邀宋江吃点心。宋江对花
荣说道:“听闻此间市镇上今晚点放花灯,我欲去看看。”花荣答道:“小弟本欲
陪侍兄长,奈缘我职役在身,不能够闲步同往。今夜兄长自与家间二三人去看灯,
早早的便回。小弟在家专待家宴三杯,以庆佳节。”宋江道:“最好。”却早天色
向夜,东边推出那轮明月上来。正是:
玉漏铜壶且莫催,星桥火树彻明开。
鳌山高耸青云上,何处游人不看来!

当晚宋江和花荣家亲随体己人两三个,跟随着缓步徐行。到这清风镇上看灯时,
只见家家门前,搭起灯棚,悬挂花灯,灯上画着许多故事,也有剪彩飞白牡丹花灯,
并芙蓉荷花异样灯火。四五个人,手厮挽着,来到大王庙前,看那小鳌山时,但见:

山石穿双龙戏水,云霞映独鹤朝天。金莲灯,玉梅灯,晃一片琉璃;荷花灯,
芙蓉灯,散千团锦绣。银蛾斗彩,双双随绣带香球;雪柳争辉,缕缕拂华幡翠。
村歌社鼓,花灯影里竞喧阗;织妇蚕奴,画烛光中同赏玩。虽无佳丽风流曲,尽贺
丰登大有年。
当下宋江等四人在鳌山前看了一回,迤逦投南走。不过五七百步,只见前面灯烛荧
煌,一伙人围住在一个大墙院门首热闹。锣声响处,众人喝采。宋江看时,却是一
伙舞鲍老的。宋江矮矬,人背后看不见。那相陪的体己人,却认的社火队里,便教
分开众人,让宋江看。那跳鲍老的身躯扭得村村势势的,宋江看了,呵呵大笑。

只见这墙院里面,却是刘知寨夫妻两口儿,和几个婆娘在里面看。听得宋江笑
声,那刘知寨的老婆,于灯下却认的宋江,便指与丈夫道:“兀那个黑矮汉子,便
是前日清风山抢掳下我的贼头。”刘知寨听了,吃一惊,便唤亲随六七人,叫捉那
个笑的黑汉子。宋江听得,回身便走。走不过十余家,众军汉赶上,把宋江捉住,
拿了来,恰似皂雕追紫燕,正如猛虎啖羊羔。拿到寨里,用四条麻索绑了,押至厅
前。那三个体己人,见捉了宋江去,自跑回来报与花荣知道。

且说刘知寨坐在厅上,叫解过那厮来,众人把宋江簇拥在厅前跪下。刘知寨喝
道:“你这厮是清风山打劫强贼,如何敢擅自来看灯!今被擒获,有何理说?”宋
江告道:“小人自是郓城县客人张三,与花知寨是故友。来此间多日了,从不曾在
清风山打劫。”刘知寨老婆,却从屏风背后转将出来,喝道:“你这厮兀自赖哩!
你记得教我叫你做大王时?”宋江告道:“恭人差矣。那时小人不对恭人说来:‘小
人自是郓城县客人,亦被掳掠在此间,不能够下山去。’”刘知寨道:“你既是客
人,被掳劫在那里,今日如何能够下山来,却到我这里看灯?”那妇人便说道:“你
这厮在山上时,大剌剌的坐在中间交椅上,由我叫大王,那里睬人!”宋江道:“恭
人,全不记我一力救你下山,如何今日倒把我强扭做贼!”那妇人听了大怒,指着
宋江骂道:“这等赖皮赖骨,不打如何肯招!”刘知寨道:“说得是。”喝叫取过
批头来打那厮。一连打了两料,打得宋江皮开肉绽,鲜血迸流。便叫把铁锁锁了,
明日合个囚车,把郓城虎张三解上州里去。

却说相陪宋江的体己人,慌忙奔回来报知花荣。花荣听罢大惊,连忙写一封书,
差两个能干亲随人,去刘知寨处取。亲随人赍了书,急忙到刘知寨门前。把门军士
入去报复道:“花知寨差人在门前下书。”刘高叫唤至当厅。那亲随人将书呈上,
刘高拆开封皮读道:

花荣拜上僚兄相公座前:所有薄亲刘丈,近日从济州来,因看灯火,误犯尊威,
万乞情恕放免,自当造谢。草字不恭,烦乞照察不宣。
刘高看了大怒,把书扯的粉碎,大骂道:“花荣这厮无礼!你是朝廷命官,如何却
与强贼通同,也来瞒我。这贼已招是郓城县张三,你却如何写道是刘丈?俺须不是
你侮弄的。你写他姓刘,是和我同姓,恁的我便放了他!”喝令左右把下书人推将
出去。那亲随人被赶出寨门,急急归来,禀复花荣知道。花荣听了,只叫得:“苦
了哥哥!快备我的马来!”

花荣披挂,拴束了弓箭,绰枪上马,带了三五十名军汉,都拖枪拽棒,直奔到
刘高寨里来。把门军人见了,那里敢拦当;见花荣头势不好,尽皆吃惊,都四散走
了。花荣抢到厅前,下了马,手中拿着枪,那三五十人,都摆在厅前。花荣口里叫
道:“请刘知寨说话。”刘高听得,惊的魂飞魄散,惧怕花荣是个武官,那里敢出
来相见?花荣见刘高不出来,立了一回,喝叫左右去两边耳房里搜人。那三五十军
汉一齐去搜时,早从廊下耳房里寻见宋江,被麻索高吊起在梁上,又使铁索锁着,
两腿打得肉绽。几个军汉便把绳索割断,铁锁打开,救出宋江。花荣便叫军士先送
回家里去。花荣上了马,绰枪在手,口里发话道:“刘知寨,你便是个正知寨,待
怎的奈何了花荣!谁家没个亲眷!你却甚么意思?我的一个表兄,直拿在家里,强扭
做贼。好欺负人,明日和你说话。”花荣带了众人,自回到寨里来看视宋江。

却说刘知寨见花荣救了人去,急忙点起一二百人,也叫来花荣寨夺人。那二百
人内,新有两个教头。为首的教头,虽然了得些枪刀,终不及花荣武艺,不敢不从
刘高,只得引了众人,奔花荣寨里来。把门军士入去报知花荣。此时天色未甚明亮,
那二百来人拥在门首,谁敢先入去,都惧怕花荣了得。看看天大明了,却见两扇大
门不关,只见花知寨在正厅上坐着,左手拿着弓,右手挽着箭。众人都拥在门前,
花荣竖起弓,大喝道:“你这军士们,不知冤各有头,债各有主。刘高差你来,休
要替他出色。你那两个新参教头,还未见花知寨的武艺,今日先教你众人看花知寨
弓箭,然后你那厮们要替刘高出色,不怕的入来。看我先射大门上左边门神的骨朵
头!”搭上箭,拽满弓,只一箭,喝声:“着!”正射中门神骨朵头。众人看了,
都吃一惊。花荣又取第二枝箭,大叫道:“你们众人,再看我这第二枝箭,要射右
边门神的头盔上朱缨。”飕的又一箭,不偏不斜,正中缨头上。那两枝箭却射定在
两扇门上。花荣再取第三枝箭,喝道:“你众人看我第三枝箭,要射你那队里穿白
的教头心窝。”那人叫声:“哎呀!”便转身先走。众人发声喊,一齐都走了。

花荣且叫闭上寨门,却来后堂看觑宋江。花荣说道:“小弟误了哥哥,受此之
苦。”宋江答道:“我却不妨,只恐刘高那厮,不肯和你干休。我们也要计较个长
便。”花荣道:“小弟舍着弃了这道官诰,和那厮理会。”宋江道:“不想那妇人
将恩作怨,教丈夫打我这一顿。我本待自说出真名姓来,却又怕阎婆惜事发,因此
只说郓城客人张三。叵耐刘高无礼,要把我做郓城虎张三,解上州去,合个囚车盛
我。要做清风山贼首时,顷刻便是一刀一剐。不得贤弟自来力救,便有铜唇铁舌,
也和他分辩不得。”花荣道:“小弟寻思,只想他是读书人,须念同姓之亲,因此
写了‘刘丈’,不想他直恁没些人情。如今既已救了来家,且却又理会。”宋江道:
“贤弟差矣!既然仗你豪势,救了人来,凡事要三思。自古道:‘吃饭防噎,行路
防跌。’他被你公然夺了人来,急使人来抢,又被你一吓,尽都散了,我想他如何
肯干罢,必然要和你动文书。今晚我先走上清风山去躲避,你明日却好和他白赖,
终久只是文武不和相殴的官司。我若再被他拿出去时,你便和他分说不过。”花荣
道:“小弟只是一勇之夫,却无兄长的高明远见。只恐兄长伤重了,走不动。”宋
江道:“不妨。事急难以搁,我自捱到山下便了。”当日敷贴了膏药,吃了些酒
肉,把包裹都寄在花荣处。黄昏时分,便使两个军汉,送出栅外去了。宋江自连夜
捱去,不在话下。

再说刘知寨见军士一个个都散回寨里来,说道:“花知寨十分英勇了得,谁敢
去近前当他弓箭!”两个教头道:“着他一箭时,射个透明窟窿,却是都去不得。”
刘高那厮,终是个文官,意思深狠,有些算计,当下刘高寻思起来:“想他这一夺
去,必然连夜放他上清风山去了,明日却来和我白赖。便争竞到上司,也只是文武
不和斗殴之事,我却如何奈何的他?我今夜差二三十军汉,去五里路头等候。倘若
天幸捉着时,将来悄悄的关在家里,却暗地使人连夜去州里,报知军官下来取,就
和花荣一发拿了,都害了他性命。那时我独自霸着这清风寨,省得受那厮们的气。”
当晚点了二十余人,各执枪棒,连夜去了。约莫有二更时候,去的军汉,背剪绑得
宋江到来。刘知寨见了,大喜道:“不出吾之所料。且与我囚在后院里,休教一个
人得知。”连夜便写了实封申状,差两个心腹之人,星夜来青州府飞报。次日,花
荣只道宋江上清风山去了,坐视在家,心里自道:“我且看他怎的!”竟不来睬着。
刘高也只做不知,两下都不说着。

且说这青州府知府,正值升厅公座。那知府复姓慕容,双名彦达,是今上徽宗
天子慕容贵妃之兄。倚托妹子的势,要在青州横行,残害良民,欺罔僚友,无所不
为。正欲回衙早饭,只见左右公人,接上刘知寨申状,飞报贼情公事。知府接来,
看了刘高的文书,吃了一惊,便道:“花荣是个功臣之子,如何结连清风山强贼?
这罪犯非小,未委虚的。”便教唤那本州兵马都监,来到厅上,分付他去。

原来那个都监姓黄,名信。为他本身武艺高强,威镇青州,因此称他为镇三山。
那青州地面,所管下有三座恶山:第一便是清风山,第二便是二龙山,第三便是桃
花山。这三处都是强人草寇出没的去处。黄信却自夸要捉尽三山人马,因此唤做镇
三山。这兵马都监黄信上厅来,领了知府的言语,出来点起五十个壮健军汉,披挂
了衣甲,马上擎着那口丧门剑,连夜便下清风寨来,径到刘高寨前下马。刘知寨出
来接着,请到后堂,叙礼罢,一面安排酒食管待,一面犒赏军士。后面取出宋江来,
教黄信看了。黄信道:“这个不必问了。连夜合个囚车,把这厮盛在里面。”头上
抹了红绢,插一个纸旗,上写着“清风山贼首郓城虎张三”。宋江那里敢分辩,只
得由他们安排。黄信再问刘高道:“你拿得张三时,花荣知也不知?”刘高道:“小
官夜来二更拿了他,悄悄的藏在家里,花荣只道去了,安坐在家。”黄信道:“既
是恁的,却容易。明早安排一副羊酒,去大寨里公厅上摆着;却教四下里埋伏下三
五十人,预备着。我却自去花荣家请得他来,只推道:‘慕容知府听得你文武不和,
因此特差我来置酒劝谕。’赚到公厅,只看我掷盏为号,就下手拿住了,一同解上
州里去。此计如何?”刘高喝采道:“还是相公高见,此计大妙。却似‘瓮中捉鳖,
手到拿来’。”

当夜定了计策,次日天晓,先去大寨左右两边帐幕里,预先埋伏了军士,厅上
虚设着酒食筵宴。早饭前后,黄信上了马,只带三两个从人,来到花荣寨前。军人
入去传报,花荣问道:“来做甚么?”军汉答道:“只听得教报道黄都监特来相探。”
花荣听罢,便出来迎接。黄信下马,花荣请至厅上,叙礼罢,便问道:“都监相公,
有何公干到此?”黄信道:“下官蒙知府呼唤,发落道:为是你清风寨内,文武官
僚不和,未知为甚缘由,知府诚恐二位因私仇而误公事,特差黄某赍到羊酒前来,
与你二位讲和。已安排在大寨公厅上,便请足下上马同往。”花荣笑道:“花荣如
何敢欺罔刘高?他又是个正知寨。只是本人累累要寻花荣的过失,不想惊动知府,
有劳都监下临草寨,花荣将何以报?”黄信附耳低言道:“知府只为足下一人。倘
有些刀兵动时,他是文官,做得何用?你只依着我行。”花荣道:“深谢都监过爱。”
黄信便邀花荣同出门首上马。花荣道:“且请都监少叙三杯了去。”黄信道:“待
说开了,畅饮何妨。”花荣只得叫备马。

当时两个并马而行,直来到大寨,下了马,黄信携着花荣的手,同上公厅来,
只见刘高已自先在公厅上。三个人都相见了。黄信叫取酒来,从人已自先把花荣的
马牵将出去,闭了寨门。花荣不知是计,只想黄信是一般武官,必无歹意。黄信擎
一盏酒来,先劝刘高道:“知府为因听得你文武二官,同僚不和,好生忧心。今日
特委黄信到来,与你二公陪话。烦望只以报答朝廷为重,再后有事,和同商议。”
刘高答道:“量刘高不才,颇识些理法,直教知府恩相,如此挂心。我二人也无甚
言语争执,此是外人妄传。”黄信大笑道:“妙哉!”刘高饮过酒,黄信又斟第二
杯酒,来劝花荣道:“虽然是刘知寨如此说了,想必是闲人妄传,故是如此,且请
饮一杯。”花荣接过酒吃了,刘高拿副台盏,斟一盏酒,回劝黄信道:“动劳都监
相公降临敝地,满饮此杯。”黄信接过酒来,拿在手里,把眼四下一看,有十数个
军汉,簇上厅来。黄信把酒盏望地下一掷,只听得后堂一声喊起,两边帐幕里,走
出三五十个壮健军汉,一发上,把花荣拿倒在厅前。黄信喝道:“绑了!”

花荣一片声叫道:“我得何罪?”黄信大笑,喝道:“你兀自敢叫哩!你结连
清风山强贼,一同背反朝廷,当得何罪!我念你往日面皮,不去惊动拿你家老小。”
花荣叫道:“也须有个证见。”黄信道:“还你一个证见,教你看真赃真贼,我不
屈你。左右,与我推将来。”无移时,一辆囚车,一个纸旗儿,一条红抹额,从外
面推将入来。花荣看时,却是宋江。目睁口呆,面面厮觑,做声不得。黄信喝道:
“这须不干我事,现有告人刘高在此。”花荣道:“不妨,不妨,这是我的亲眷。
他自是郓城县人,你要强扭他做贼,到上司自有分辩处。”黄信道:“你既然如此
说时,我只解你上州里,你自去分辩。”便叫刘知寨点起一百寨兵防送。花荣便对
黄信说道:“都监赚我来,虽然捉了我,便到朝廷,和他还有分辩。可看我和都监
一般武职官面,休去我衣服,容我坐在囚车里。”黄信道:“这一件容易,便依着
你。就叫刘知寨一同去州里折辩明白,休要枉害人性命。”

当时黄信与刘高都上了马,监押着两辆囚车,并带三五十军士,一百寨兵,簇
拥着车子,取路奔青州府来。有分教:火焰堆里,送数百间屋宇人家;刀斧丛中,
杀一二千残生性命。正是:生事事生君莫恕,害人人害汝休嗔。

毕竟解宋江投青州来,怎地脱身,且听下回分解。