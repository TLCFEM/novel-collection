\chapter{宋公明兵打蓟州城~卢俊义大战玉田县}

话说洞仙侍郎见檀州已失,只得奔走出城,同咬儿惟康拥护而行。正撞着林冲、
关胜,大杀一阵,那里有心恋战,望刺斜里死命撞出去。关胜、林冲要抢城子,也
不来追赶,且奔入城。

却说宋江引大队军马入檀州,赶散番军,一面出榜,安抚百姓军民,秋毫不许有犯。
传令教把战船尽数收入城中。一面赏劳三军,及将在城辽国所用官员,有姓者仍前
委用,无姓番官尽行发遣出城,还于沙漠。一面写表申奏朝廷,得了檀州,尽将府
库财帛金宝,解赴京师,写书申呈宿太尉,题奏此事。

天子闻奏,龙颜大喜。随即降旨,钦差东京府同知赵安抚统领二万御营军马,前来
监战。却说宋江等听的报来,引众将出郭远远迎接,入到檀州府内歇下,权为行军
帅府。诸将头目,尽来参见,施礼已毕。原来这赵安抚,祖是赵家宗派,为人宽仁
厚德,作事端方,亦是宿太尉于天子前保奏,特差此人上边监督兵马。这赵安抚见
了宋江仁德,十分欢喜,说道:“圣上已知你等众将用心,军士劳苦,特差下官前
来军前监督,就赍赏赐金银缎匹二十五车,但有奇功,申奏朝廷,请降官封。将军
今已得了州郡,下官再当申达朝廷。众将皆须尽忠竭力,早成大功,班师回京,天
子必当重用。”宋江等拜谢道:“请烦安抚相公镇守檀州,小将等分兵攻取辽国紧
要州郡,教他首尾不能相顾。”一面将赏赐俵散军将,一面勒回各路军马听调,攻
取辽国州郡。有杨雄禀道:“前面便是蓟州相近。此处是个大郡,钱粮极广,米麦
丰盈,乃是辽国库藏。打了蓟州,诸处可取。”宋江听罢,便请军师吴用商议。

却说洞仙侍郎与咬儿惟康正往东走,撞见楚明玉、曹明济,引着些败残军马,一同
投奔蓟州。入的城来,见了御弟大王耶律得重,诉说:“宋江兵将浩大,内有一个
使石子的蛮子,十分了得。那石子百发百中,不放一个空,最会打人。两位皇侄并
小将阿里奇,尽是被他石子打死了。”耶律大王道:“既是这般,你且在这里帮俺
杀那蛮子。”说犹未了,只见流星探马报将来,说道:“宋江兵分两路,来打蓟州,
一路杀至平峪县,一路杀至玉田县。”御弟大王听了,随即便教洞仙侍郎:“将引
本部军马,把住平峪县口,不要和他厮杀。俺先引兵,且拿了玉田县的蛮子,却从
背后抄将过来,平峪县的蛮子,走往那里去?一边关报霸州、幽州,教两路军马,
前来接应。”原来这蓟州,却是辽国郎主差御弟耶律得重守把。部领四个孩儿:长
子宗云,次子宗电,三子宗雷,四子宗霖。手下十数员战将,一个总兵大将,唤做
宝密圣,一个副总兵,唤做天山勇,守住着蓟州城池。当时御弟大王,嘱付宝密圣
守城,亲引大军,将带四个孩儿,并副总兵天山勇,飞奔玉田县来。

且说宋江引兵前至平峪县,见前面把住关隘,未敢进兵,就平峪县西屯住。

却说卢俊义引许多战将,三万人马,前到玉田县,早与辽兵相近。卢俊义便与军师
朱武商议道:“目今与辽兵相近,只是吴人不识越境,到他地理生疏,何策可取?”
朱武答道:“若论愚意,未知他地理,诸军不可擅进。可将队伍摆为长蛇之势,首
尾相应,循环无端,如此则不愁地理生疏。”卢先锋道:“军师所言,正合吾意。”
遂乃催兵前进。远远望见辽兵盖地而来,但见:
黄沙漫漫,黑雾浓浓。皂雕旗展一派乌云,拐子马荡半天杀气。青毡笠帽,似千池
荷叶弄轻风;铁打兜鍪,如万顷海洋凝冻日。人人衣襟左掩,个个发搭齐肩。连环
铁铠重披,刺纳
战袍紧系。番军壮健,黑面皮碧眼黄须;达马咆哮,阔膀膊钢腰铁脚。羊角弓攒沙
柳箭,虎皮袍衬窄雕鞍。生居边塞,长成会拽硬弓;世本朔方,养大能骑劣马。铜
羯鼓军前打,芦叶胡笳马上吹。

那御弟大王耶律得重引兵先到玉田县,将军马摆开阵势。宋军中朱武上云梯看
了,下来回报卢先锋道:“番人布的阵,乃是五虎靠山阵,不足为奇。”朱武再上
将台,把号旗招动,左盘右旋,调拨众军,也摆一个阵势。卢俊义看了不识。问道:
“此是何阵势?”朱武道:“此乃是鲲化为鹏阵。”卢俊义道:“何为鲲化为鹏?”
朱武道:“北海有鱼,其名曰鲲,能化大鹏,一飞九万里。此阵远观近看,只是个
小阵,若来攻时,便变做大阵,因此唤做鲲化为鹏。”卢俊义听了,称赞不已。
对阵敌军鼓响,门旗开处,那御弟大王亲自出马,四个孩儿分在左右,都是一般披
挂。但见:
头戴铁缦笠戗箭番盔,上拴纯黑球缨。身衬宝圆镜柳叶细甲,系条狮蛮金带。踏
靴半弯鹰嘴,梨花袍锦绣盘龙。各挂强弓硬弩,都骑骏马雕鞍。腰间尽插锟剑,
手内齐拿扫帚刀。

中间御弟大王,两边四个小将军,身上两肩胛,都悬着小小明镜,镜边对嵌着皂缨。
四口宝刀,四骑快马,齐齐摆在阵前。那御弟大王背后,又是层层摆列,自有许多
战将。那四员小将军高声大叫:“汝等草贼,何敢犯吾边界!”卢俊义听的,便问
道:“两军临敌,那个英雄当先出战?”说犹未了,只见大刀关胜舞起青龙偃月刀,
争先出马。那边番将耶律宗云,舞刀拍马,来迎关胜。两个斗不上五合,耶律宗霖
拍马舞刀,便来协助。呼延灼见了,举起双鞭,直出迎住厮杀。那两个耶律宗电、
耶律宗雷弟兄,挺刀跃马,齐出交战。这里徐宁、索超,各举兵器相迎。四对儿在
阵前厮杀,绞做一团,打做一块。

正斗之间,没羽箭张清看见,悄悄的纵马趱向阵前。却有檀州败残的军士,认的张
清,慌忙报知御弟大王道:“这对阵穿绿战袍的蛮子,便是惯飞石子的。他如今趱
马出阵来,又使前番手段。”天山勇听了便道:“大王放心,教这蛮子吃俺一弩箭!”
原来那天山勇,马上惯使漆抹弩,一尺来长铁翎箭,有名唤做一点油。那天山勇在
马上把了事环带住,趱马出阵,教两个副将在前面影射着,三骑马悄悄直趱至阵前。
张清又先见了,偷取石子在手,看着那番官当头的,只一石子,急叫:“着!”早
从盔上擦过。那天山勇却闪在这将马背后,安的箭稳,扣的弦正,觑着张清较亲,
直射将来。张清叫声:“阿也!”急躲时,射中咽喉,翻身落马。双枪将董平、九
纹龙史进,将引解珍、解宝,死命去救回。卢先锋看了,急教拔出箭来,血流不止,
项上便束缚兜住。随即叫邹渊、邹润扶张清上车子,护送回檀州,教神医安道全调
治。

车子却才去了,只见阵前喊声又起,报道:“西北上有一彪军马飞奔杀来,并不打
话,横冲直撞,赶入阵中。”卢俊义见箭射了张清,无心恋战。四将各佯输诈败,
退回去了。四个番将,乘势赶来。西北上来的番军,刺斜里又杀将来。对阵的大队
番军,山倒也似踊跃将来,那里变的阵法。三军众将,隔的七断八续,你我不能相
救,只留卢俊义一骑马,一条枪,倒杀过那边去了。天色傍晚,四个小将军却好回
来,正迎着卢俊义。一骑马,一条枪,力敌四个番将,并无半点惧怯。约斗了一个
时辰,卢俊义得便处,卖个破绽,耶律宗霖把刀砍将入来,被卢俊义大喝一声,那
番将措手不及,着一枪,刺下马去。那三个小将军,各吃了一惊,皆有惧色,无心
恋战,拍马去了。卢俊义下马,拔刀割了耶律宗霖首级,拴在马项下。翻身上马,
望南而行,又撞见一伙辽兵,约有一千余人。被卢俊义又撞杀入去,辽兵四散奔走。
再行不到数里,又撞见一彪军马。

此夜月黑,不辨是何处的人马,只听的语音,却是宋朝人说话。卢俊义便问:“来
军是谁?”却是呼延灼答应。卢俊义大喜,合兵一处。呼延灼道:“被辽兵冲散,
不能救应。小将撞开阵势,和韩滔、彭玘直杀到此,不知诸将如何。”卢俊义又说:
“力敌四将,被我杀了一个,三个走了。次后又撞着一千余人,亦被我杀散。来到
这里,不想迎着将军。”两个并马,带着从人,望南而行。不过十数里路,前面早
有军马拦路。呼延灼道:“黑夜怎地厮杀,待天明决一死战!”对阵听的,便问道:
“来者莫非呼延灼将军?”呼延灼认的声音,是大刀关胜,便叫道:“卢头领在此!”
众头领都下马,且来草地上坐下。卢俊义、呼延灼说了本身之事。关胜道:“阵前
失利,你我不相救应。我和宣赞、郝思文、单廷圭、魏定国五骑马寻条路走,然后
收拾的军兵一千余人,来到这里。不识地理,只在此伏路,待天明却行。不想撞着
哥哥。”合兵一处,众人捱到天晓,迤逦望南再行。将次到玉田县,见一彪人马哨
路。看时,却是双枪将董平、金枪手徐宁,弟兄们都扎住玉田县中,辽兵尽行赶散,
说道:“侯健、白胜两个去报宋公明,只不见了解珍、解宝、杨林、石勇。”卢俊
义教且进兵,在玉田县界检点众将军校,不见了五千余人,心中烦恼。巳牌时分,
有人报道:“解珍、解宝、杨林、石勇,将领二千余人来了。”卢俊义又唤来问时,
解珍道:“俺四个倒撞过去了。深入重地,迷踪失路,急切不敢回转。今早又撞见
辽兵,大杀了一场,方才到得这里。”卢俊义叫将耶律宗霖首级于玉田县号令,抚
谕三军百姓。

未到黄昏前后,军士们正要收拾安歇,只见伏路小校来报道:“辽兵不知多少,四
面把县围了。”卢俊义听的大惊,引了燕青上城看时,远近火把,有十里厚薄。一
个小将军当先指点,正是耶律宗云,骑着一匹劣马,在火把中间,催趱三军。燕青
道:“昨日张清中他一冷箭,今日回礼则个!”燕青取出弩子,一箭射去,正中番
将鼻凹,番将落马。众兵急救时,宗云已自伤闷不醒。番军早退五里。

卢俊义县中与众将商议:“虽然放了一冷箭,辽兵稍退,天明必来攻,围裹的铁桶
相似,怎生救解?”朱武道:“宋公明若得知这个消息,必然来救。里应外合,方
可免难。”众人捱到天明,望见辽兵四面摆的无缝。只见东南上尘土起,兵马数万
人而来,众将皆望南兵。朱武道:“此必是宋公明军马到了!等他收军,齐望南杀
去,这里尽数起兵,随后一掩。”

且说对阵辽兵,从辰时直围到未牌,正待困倦,却被宋江军马杀来,抵当不住,尽
数收拾都去。朱武道:“不就这里追赶,更待何时!”卢俊义当即传令,开县四门,
尽领军马,出城追杀,辽兵大败,杀的星落云散,七断八续,辽兵四散败走。宋江
赶的辽兵去远,到天明鸣金收军,进玉田县。卢先锋合兵一处,诉说攻打蓟州。留
下柴进、李应、李俊、张横、张顺、阮家三弟兄、王矮虎、一丈青、孙新、顾大嫂、
张青、孙二娘、裴宣、萧让、宋清、乐和、安道全、皇甫端、童威、童猛、王定六,
都随赵枢密在檀州守御。其余诸将,分作左右二军。宋先锋总领左军人马四十八员:
军师吴用、公孙胜、林冲、花荣、秦明、杨志、朱仝、雷横、刘唐、李逵、鲁智深、
武松、杨雄、石秀、黄信、孙立、欧鹏、邓飞、吕方、郭盛、樊瑞、鲍旭、项充、
李衮、穆弘、穆春、孔明、孔亮、燕顺、马麟、施恩、薛永、宋万、杜迁、朱贵、
朱富、凌振、汤隆、蔡福、蔡庆、戴宗、蒋敬、金大坚、段景住、时迁、郁保四、
孟康。卢先锋总领右军人马三十七员:军师朱武、关胜、呼延灼、董平、张清、索
超、徐宁、燕青、史进、解珍、解宝、韩滔、彭玘、宣赞、郝思文、单廷圭、魏定
国、陈达、杨春、李忠、周通、陶宗旺、郑天寿、龚旺、丁得孙、邹渊、邹润、李
立、李云、焦挺、石勇、侯健、杜兴、曹正、杨林、白胜。分兵已罢,作两路来取
蓟州:宋先锋引军取平峪县进发,卢俊义引兵取玉田县进发。赵安抚与二十三将,
镇守檀州,不在话下。

且说宋江见军士连日辛苦,且教暂歇。攻打蓟州,自有计较了。先使人往檀州,问
张清箭疮如何。神医安道全使人回话道:“虽然外损皮肉,却不伤内,请主将放心。
调理的脓水干时,自然无事。即目炎天,军士多病,已禀过赵枢密相公,遣萧让、
宋清前往东京收买药饵,就向太医院关支暑药。皇甫端亦要关给官局内啖马的药材
物料,都委萧让、宋清去了。就报先锋知道。”宋江听的,心中颇喜,再与卢先锋
计较,先打蓟州。宋江道:“我未知你在玉田县受围时,已自先商量下计了。有公
孙胜原是蓟州人,杨雄亦曾在那府里做节级,石秀、时迁亦在那里住的久远。前日
杀退辽兵,我教时迁、石秀也只做败残军马,杂在里面,必然都投蓟州城内住扎。
他两个若入的城中,自有去处。时迁曾献计道:‘蓟州城有一座大寺,唤叫宝严寺,
廊下有法轮宝藏,中间是大雄宝殿,前有一座宝塔,直耸云霄。’石秀说道:‘教
他去宝塔顶上躲着,每日饭食,我自对付来与他吃。只等城外哥哥军马攻打得紧急
时,然后却就宝严寺塔上,放起火来为号。’时迁自是个惯飞檐走壁的人,那里不
躲了身子?石秀临期自去州衙内放火,他两个商量已定,自去了。我这里一面收拾
进兵。”有《西江月》为证:
山后辽兵侵境,中原宋帝兴军。水乡取出众天星,奉诏去邪归正。暗地时迁放火,
更兼石秀同行。等闲打破永平城,千载功勋可敬!

次日,宋江引兵,撇了平峪县,与卢俊义合兵一处,催起军马,径奔蓟州来。
且说御弟大王自折了两个孩儿,不胜懊恨,便同大将宝密圣、天山勇、洞仙侍郎等
商议道:“前次涿州、霸州两路救兵,各自分散前去。如今宋江合兵在玉田县,早
晚进兵来打蓟州,似此怎生奈何?”大将宝密圣道:“宋江兵若不来,万事皆休。
若是那伙蛮子来时,小将自出去与他相敌。若不活拿他几个,这厮们那里肯退?”
洞仙侍郎道:“那蛮子队有那个穿绿袍的,惯使石子,好生利害,可以提防他。”
天山勇道:“这个蛮子,已被俺一弩箭,射中咽喉,多是死了也!”洞仙侍郎道:
“除了这个蛮子,别的都不打紧。”正商议间,小校来报,宋江军马杀奔蓟州来。
御弟大王连忙整点三军人马,教宝密圣、天山勇火速出城迎敌。离城三十里外,与
宋江对敌。

各自摆开阵势,番将宝密圣横槊出马。宋江在阵前见了,便问道:“斩将夺旗,乃
见头功!”说犹未了,只见豹子头林冲便出阵前来,与番将宝密圣大战。两个斗了
三十余合,不分胜败。林冲要见头功,持丈八蛇矛,斗到间深里,暴雷也似大叫一
声,拨过长枪,用蛇矛去宝密圣脖项上刺中,一矛搠下马去。宋江大喜。两军发喊。
番将天山勇见刺了宝密圣,横枪便出。宋江阵里,徐宁挺钩镰枪直迎将来。二马相
交,斗不到二十来合,被徐宁手起一枪,把天山勇搠于马下。宋江见连赢了二将,
心中大喜,催军混战。辽兵大败,望蓟州奔走。宋江军马赶了十数里,收兵回来。
当日宋江扎下营寨,赏劳三军,次日传令,拔寨都起,直抵蓟州。第三日,御弟大
王见折了二员大将,十分惊慌,又见报道:“宋军到了!”忙与洞仙侍郎道:“你
可引这支军马,出城迎敌,替俺分忧也好。”洞仙侍郎不敢不依,只得引了咬儿惟
康、楚明玉、曹明济,领起一千军马,就城下摆开。宋江军马渐近城边,雁翅般排
将来。门旗开处,索超横担大斧,出马阵前。番兵队里,咬儿惟康便抢出阵来。两
个并不打话,二将相交,斗到二十余合。番将终是胆怯,无心恋战,只得要走。索
超纵马赶上,双手抡起大斧,觑着番将脑门上劈将下来,把这咬儿惟康脑袋,劈做
两半个。洞仙侍郎见了,慌忙叫楚明玉、曹明济快去策应。这两个已自八分胆怯,
因吃逼不过,只得挺起手中枪,向前出阵。宋江军中九纹龙史进,见番军中二将双
出,便舞刀拍马,直取二将。史进逞起英雄,手起刀落,先将楚明玉砍于马下。这
曹明济急待要走,史进赶上一刀,也砍于马下。史进纵马杀入辽军阵内,宋江见了,
鞭梢一指,驱兵大进,直杀到吊桥边。耶律得重见了,越添愁闷,便教紧闭城门,
各将上城紧守。一面申奏郎主,一面差人往霸州、幽州求救。

且说宋江与吴用计议道:“似此城中紧守,如何摆布?”吴用道:“既城中已有石
秀、时迁在里面,如何耽搁的长远?教四面竖起云梯炮架,即便攻城。再教凌振将
火炮四下里施放,打将入去。攻击得紧,其城必破。”宋江即便传令,四面连夜攻
城。

再说御弟大王,见宋兵四下里攻击得紧,尽驱蓟州在城百姓,上城守护。当下石秀
在城中宝严寺内,守了多日,不见动静。只见时迁来报道:“城外哥哥军马,打得
城子紧。我们不就这里放火,更待何时?”石秀见说了,便和时迁商议,先从宝塔
上放起一把火来,然后去佛殿上烧着。时迁道:“你快去州衙内放火。在南门要紧
的去处,火着起来,外面见了,定然加力攻城,愁他不破!”两个商量了,都自有
引火的药头、火刀,火石、火筒、烟煤,藏在身边。当日晚来,宋江军马打城甚紧。
却说时迁,他是个飞檐走壁的人,跳墙越城,如登平地。当时先去宝严寺塔上,点
起一把火来。那宝塔最高,火起时,城里城外,那里不看见。火光照的三十余里远
近,似火钻一般。然后却来佛殿上放火。那两把火起,城中鼎沸起来。百姓人民,
家家老幼慌忙,户户儿啼女哭,大小逃生。石秀直爬去蓟州衙门庭屋上●风板里,
点起火来。蓟州城中,见三处火起,知有细作,百姓那里有心守护城池,已都阻当
不住,各自逃归看家。没多时,山门里又一把火起,却是时迁出宝严寺来,又放了
一把火。那御弟大王,见了城中无半个更次,四五路火起,知宋江有人在城里。慌
慌急急,收拾军马,带了老小,并两个孩儿,装载上车,开了北门便走。宋江见城
中军马慌乱,催促军兵,卷杀入城。城里城外,喊杀连天,早夺了南门。洞仙侍郎
见寡不敌众,只得跟随御弟大王,投北门而走。

宋江引大队军马,入蓟州城来,便传下将令,先教救灭了四边风火。天明出榜,安
抚蓟州百姓。将三军人马,尽数收入蓟州屯住,赏劳三军诸将。功绩簿上,标写石
秀、时迁功次,便行文书,申复赵安抚知道得了蓟州大郡,请相公前来驻扎。赵安
抚回文书来说道:“我在檀州,权且屯扎,教宋先锋且守住蓟州。即目炎暑,天气
暄热,未可动兵。待到天气微凉,再作计议。”宋江得了回文,便教卢俊义分领原
拨军将,于玉田县屯扎,其余大队军兵,守住蓟州。待到天气微凉,别行听调。
却说御弟大王耶律得重与洞仙侍郎将带老小,奔回幽州,直至燕京来见大辽郎主。
且说辽国郎主升坐金殿,聚集文武两班臣僚,朝参已毕。有门大使奏道:“蓟州
御弟大王,回至门下。”郎主闻奏,忙教宣召,宣至殿下。那耶律得重与洞仙侍郎,
俯伏御阶之下,放声大哭。郎主道:“俺的爱弟,且休烦恼,有甚事务,当以尽情
奏知寡人。”那耶律得重奏道:“宋朝童子皇帝,差调宋江领兵前来征讨,军马势
大,难以抵敌。送了臣的两个孩儿,杀了檀州四员大将。宋军席卷而来,又失陷了
蓟州,特来殿前请死!”

大辽国主听了,传圣旨道:“卿且起来,俺的这里好生商议。”郎主道:“引兵的
那蛮子,是甚人?这等喽罗!”班部中右丞相太师褚坚出班奏道:“臣闻宋江这伙
原是梁山泊水浒寨草寇,却不肯杀害良民,专一替天行道,只杀滥官污吏、诈害
百姓的人。后来童贯、高俅,引兵前去收捕,被宋江只五阵,杀的片甲不回。他这
伙好汉,剿捕他不得。童子皇帝遣使三番降诏去招安,他后来都投降了。只把宋江
封为先锋使,又不曾实授官职,其余都是白身人。今日差将他来,便和俺们厮杀。
他道有一百八人,应天上星宿。这伙人好生了得,郎主休要小觑了他。”郎主道:
“你这等话说时,恁地怎生是好?”班部丛中转出一员官,乃是欧阳侍郎,袍拂
地,象简当胸奏道:“郎主万岁!臣虽不才,愿献小计,可退宋兵。”郎主大喜道:
“你既有好的见识,当下便说。”欧阳侍郎言无数句,话不一席,有分教:宋江名
标青史,事载丹书。正是:护国谋成欺吕望,顺天功就赛张良。
毕竟欧阳侍郎奏出甚事来,且听下回分解。