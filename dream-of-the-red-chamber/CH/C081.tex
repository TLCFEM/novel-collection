\chapter{占旺相四美钓游鱼~奉严词两番入家塾}

且说迎春归去之后,邢夫人像没有这事,倒是王夫人抚养了一场,却甚实伤感,
在房中自己叹息了一回。只见宝玉走来请安,看见王夫人脸上似有泪痕,也不敢坐,
只在傍边站着。王夫人叫他坐下,宝玉才捱上炕来,就在王夫人身
旁坐了。王夫人见他呆呆的瞅着,似有欲言不言的光景,便道:“你又为什么这样
呆呆的?”宝玉道:“并不为什么。只是昨儿听见二姐姐这种光景,我实在替他受
不得。虽不敢告诉老太太,却这两夜只是睡不着。我想咱们这样人家的姑娘,那里
受得这样的委屈?况且二姐姐是个最懦弱的人,向来不会和人拌嘴,偏偏儿的遇见
这样没人心的东西,竟一点儿不知道女人的苦处!”说着,几乎滴下泪来。王夫人
道:“这也是没法儿的事。俗语说的:‘嫁出去的女孩儿,泼出去的水。’叫我能
怎么样呢?”宝玉道:“我昨儿夜里倒想了一个主意:咱们索性回明了老太太,把
二姐姐接回来,还叫他紫菱洲住着,仍旧我们姐妹弟兄们一块儿吃,一块儿玩,省
得受孙家那混帐行子的气。等他来接,咱们硬不叫他去。由他接一百回,咱们留一
百回。只说是老太太的主意。这个岂不好呢?”王夫人听了,又好笑又好恼,说道:
“你又发了呆气了!混说的是什么?大凡做了女孩儿,终久是要出门子的。嫁到人家
去,娘家那里顾得?也只好看他自己的命运,碰的好就好,碰的不好也就没法儿。
你难道没听见人说‘嫁鸡随鸡,嫁狗随狗’,那里个个都像你大姐姐做娘娘呢?况
且你二姐姐是新媳妇,孙姑爷也还是年轻的人,各人有各人的脾气,新来乍到,自
然要有些扭的。过几年,大家摸着脾气儿,生儿长女以后,那就好了。你断断不
许在老太太跟前说起半个字,我知道了是不依你的。快去干你的去罢,别在这里混
说了。”说的宝玉也不敢作声,坐了一回,无精打彩的出来了。着一肚子闷气,
无处可泄,走到园中,一径往潇湘馆来。刚进了门,便放声大哭起来。

黛玉正在梳洗才毕,见宝玉这个光景,倒吓了一跳,问:“是怎么了?合谁怄
了气了?”连问几声。宝玉低着头,伏在桌子上呜呜咽咽,哭的说不出话来。黛玉
便在椅子上怔怔的瞅着他,一会子问道:“到底是别人合你怄了气了,还是我得罪
了你呢?”宝玉摇手道:“都不是,都不是。”黛玉道:“那么着,为什么这么伤
心起来?”宝玉道:“我只想着,咱们大家越早些死的越好,活着真真没有趣儿。”
黛玉听了这话,更觉惊讶,道:“这是什么话?你真正发了疯不成?”宝玉道:“也
并不是我发疯。我告诉你,你也不能不伤心。前儿二姐姐回来的样子和那些话,你
也都听见看见了。我想人到了大的时候,为什么要嫁?嫁出去,受人家这般苦楚!还
记得咱们初结海棠社的时候,大家吟诗做东道,那时候何等热闹。如今宝姐姐家去
了,连香菱也不能过来,二姐姐又出了门子了,几个知心知意的人都不在一处,弄
得这样光景!我原打算去告诉老太太,接二姐姐回来,谁知太太不依,倒说我呆、
混说。我又不敢言语。这不多几时,你瞧瞧,园中光景,已经大变了。若再过几年,
又不知怎么样了。故此,越想不由的人心里难受起来。”黛玉听了这番言语,把头
渐渐的低了下去,身子渐渐的退至炕上,一言不发,叹了口气,便向里躺下去了。

紫鹃刚拿进茶来,见他两个这样,正在纳闷,只见袭人来了,进来看见宝玉,
便道:“二爷在这里呢么?老太太那里叫呢。我估量着二爷就是在这里。”黛玉听
见是袭人,便欠身起来让坐。黛玉的两个眼圈儿已经哭的通红了。宝玉看见,道:
“妹妹,我刚才说的,不过是些呆话,你也不用伤心了。要想我的话时,身子更要
保重才好。你歇歇儿罢。老太太那边叫我,我看看去就来。”说着,往外走了。袭
人悄问黛玉道:“你两个人又为什么?”黛玉道:“他为他二姐姐伤心;我是刚才
眼睛发痒揉的,并不为什么。”袭人也不言语,忙跟了宝玉出来,各自散了。宝玉
来到贾母那边,贾母却已经歇晌,只得回到怡红院。

到了午后,宝玉睡了中觉起来,甚觉无聊,随手拿了一本书看。袭人见他看书,
忙去沏茶伺候。谁知宝玉拿的那本书却是《古乐府》,随手翻来,正看见曹孟德“对
酒当歌,人生几何”一首,不觉刺心。因放下这一本,又拿一本看时,却是晋文。
翻了几页,忽然把书掩上,托着腮只管痴痴的坐着。袭人倒了茶来,见他这般光景,
便道:“你为什么又不看了?”宝玉也不答言,接过茶来,喝了一口,便放下了。
袭人一时摸不着头脑,也只管站在傍边,呆呆的看着他。忽见宝玉站起来,嘴里咕
咕哝哝的说道:“好一个‘放浪形骸之外’!”袭人听了,又好笑,又不敢问他,
只得劝道:“你若不爱看这些书,不如还到园里逛逛,也省得闷出毛病来。”那宝
玉一面口中答应,只管出着神,往外走了。

一时走到沁芳亭,但见萧疏景象,人去房空。又来至蘅芜院,更是香草依然,
门窗掩闭。转过藕香榭来,远远的只见几个人,在蓼溆一带栏干上靠着,有几个小
丫头蹲在地下找东西。宝玉轻轻的走在假山背后听着。只听一个说道:“看他上
来不上来。”好似李纹的语音。一个笑道:“好,下去了。我知道他不上来的。”
这个却是探春的声音。一个又道:“是了。姐姐你别动,只管等着,他横竖上来。”
一个又说:“上来了。”这两个是李绮邢岫烟的声儿。宝玉忍不住,拾了一块小砖
头儿,往那水里一撂,“咕咚”一声。四个人都吓了一跳,惊讶道:“这是谁这么
促狭?唬了我们一跳!”宝玉笑着从山子后直跳出来,笑道:“你们好乐啊!怎么不
叫我一声儿?”探春道:“我就知道再不是别人,必是二哥哥这么淘气。没什么说
的,你好好儿的赔我们的鱼罢。刚才一个鱼上来,刚刚儿的要钓着,叫你唬跑了。”
宝玉笑道:“你们在这里玩,竟不找我,我还要罚你们呢。”大家笑了一回。

宝玉道:“咱们大家今儿钓鱼,占占谁的运气好?看谁钓得着就是他今年的运
气好,钓不着就是他今年运气不好。咱们谁先钓?”探春便让李纹,李纹不肯。探
春笑道:“这样就是我先钓。”回头向宝玉说道:“二哥哥,你再赶走了我的鱼,
我可不依了。”宝玉道:“头里原是我要唬你们玩,这会子你只管钓罢。”探春把
丝绳抛下,没十来句话的工夫,就有一个杨叶窜儿吞着钩子,把漂儿坠下去。探春
把竿一挑,往地下一撩,却是活迸的。侍书在满地上乱抓,两手捧着搁在小磁坛内,
清水养着。探春把钓竿递与李纹。李纹也把钓竿垂下,但觉丝儿一动,忙挑起来,
却是个空钩子。又垂下去半晌,钩丝一动,又挑起来,还是空钩子。李纹把那钩子
拿上来一瞧,原来往里钩了。李纹笑道:“怪不得钓不着。”忙叫素云把钩子敲好
了,换上新虫子,上边贴好了苇片儿。垂下去一会儿,见苇片直沉下去,急忙提起
来,倒是一个二寸长的鲫瓜儿。李纹笑着道:“宝哥哥钓罢。”宝玉道:“索性三
妹妹和邢妹妹钓了我再钓。”岫烟却不答言。只见李绮道:“宝哥哥先钓罢。”说
着,水面上起了一个泡儿。探春道:“不必尽着让了。你看那鱼都在三妹妹那边呢,
还是三妹妹快着钓罢。”李绮笑着接了钓竿儿,果然沉下去就钓了一个。然后岫烟
来钓着了一个,随将竿子仍旧递给探春,探春才递与宝玉。宝玉道:“我是要做姜
太公的。”便走下石矶,坐在池边钓起来。岂知那水里的鱼,看见人影儿,都躲到
别处去了。宝玉抡着钓竿,等了半天,那钓丝儿动也不动。刚有一个鱼儿在水边吐
沫,宝玉把竿子一,又唬走了。急的宝玉道:“我最是个性儿急的人,他偏性儿
慢,这可怎么样呢?好鱼儿,快来罢,你也成全成全我呢。”说的四人都笑了。一
言未了,只见钓丝微微一动。宝玉喜极,满怀用力往上一兜,把钓竿往石上一碰,
折作两段,丝也振断了,钩子也不知往那里去了。众人越发笑起来。探春道:“再
没见像你这样卤人!”

正说着,只见麝月慌慌张张的跑来说:“二爷,老太太醒了,叫你快去呢。”
五个人都唬了一跳。探春便问麝月道:“老太太叫二爷什么事?”麝月道:“我也
不知道。就只听见说是什么闹破了,叫宝玉来问;还要叫琏二奶奶一块儿查问呢。”
吓得宝玉发了一回呆,说道:“不知又是那个丫头遭了瘟了。”探春道:“不知什
么事,二哥哥你快去。有什么信儿,先叫麝月来告诉我们一声儿。”说着便同李纹、
李绮、岫烟走了。

宝玉走到贾母房中,只见王夫人陪着贾母摸牌。宝玉看见无事,才把心放下了
一半。贾母见他进来,便问道:“你前年那一次得病的时候,后来亏了一个疯和尚
和个瘸道士治好了的。那会子病里你觉得是怎么样?”宝玉想了一回道:“我记得
得病的时候儿,好好的站着,倒像背地里有人把我拦头一棍,疼的眼睛前头漆黑,
看见满屋子里都是些青面獠牙、拿刀举棒的恶鬼。躺在炕上,觉着脑袋上加了几个
脑箍似的。以后便疼的任什么不知道了。到好的时候,又记得堂屋里一片金光,直
照到我床上来,那些鬼都跑着躲避,就不见了。我的头也不疼了,心上也就清楚了。”
贾母告诉王夫人道:“这个样儿也就差不多了。”

说着凤姐也进来了,见了贾母,又回身见过了王夫人,说道:“老祖宗要问我
什么?”贾母道:“你那年中了邪的时候儿,你还记得么?”凤姐儿笑道:“我也
不很记得了。但觉自己身子不由自主,倒像有什么人拉拉扯扯,要我杀人才好。有
什么拿什么,见什么杀什么,自己原觉很乏,只是不能住手。”贾母道:“好的时
候儿呢?”凤姐道:“好的时候好像空中有人说了几句话似的,却不记得说什么来
着。”贾母道:“这么看起来,竟是他了。他姐儿两个病中的光景合才说的一样。
这老东西竟这样坏心!宝玉枉认了他做干妈!倒是这个和尚道人,阿弥陀佛,才是救
宝玉性命的。只是没有报答他。”凤姐道:“怎么老太太想起我们的病来呢?”贾
母道:“你问你太太去,我懒怠说。”王夫人道:“才刚老爷进来,说起宝玉的干
妈竟是个混帐东西,邪魔外道的。如今闹破了,被锦衣府拿住送入刑部监,要问死
罪的了。前几天被人告发的。那个人叫做什么潘三保,有一所房子,卖给斜对过当
铺里。这房子加了几倍价钱,潘三保还要加,当铺里那里还肯?潘三保便买嘱了这
老东西,——因他常到当铺里去,那当铺里人的内眷都和他好的,——他就使了个
法儿,叫人家的内人便得了邪病,家翻宅乱起来。他又去说,这个病他能治,就用
些神马纸钱烧献了,果然见效。他又向人家内眷们要了十几两银子。岂知老佛爷有
眼,应该败露了。这一天急要回去,掉了一个绢包儿。当铺里人捡起来一看,里头
有许多纸人,还见四丸子很香的香。正诧异着呢,那老东西倒回来找这绢包儿,这
里的人就把他拿住。身边一搜,搜出一个匣子,里面有象牙刻的一男一女,不穿衣
裳,光着身子的两个魔王,还有七根朱红绣花针。立时送到锦衣府去,问出许多官
员家大户太太姑娘们的隐情事来。所以知会了营里,把他家中一抄,抄出好些泥塑
的煞神,几匣子闷香。炕背后空屋子里挂着一盏七星灯,灯下有几个草人,有头上
戴着脑箍的,有胸前穿着钉子的,有项上拴着锁子的。柜子里无数纸人儿。底下几
篇小帐,上面记着某家验过,应找银若干。得人家油钱香分也不计其数。”

凤姐道:“咱们的病一准是他。我记得咱们病后,那老妖精向赵姨娘那里来过
几次,和赵姨娘讨银子,见了我,就脸上变貌变色,两眼黧鸡似的。我当初还猜了
几遍,总不知什么原故。如今说起来,却原来都是有因的。但只我在这里当家,自
然惹人恨怨,怪不得别人治我,宝玉可合人有什么仇呢?忍得下这么毒手!”贾母
道:“焉知不因我疼宝玉,不疼环儿,竟给你们种了毒了呢。”王夫人道:“这老
货已经问了罪,决不好叫他来对证。没有对证,赵姨娘那里肯认帐?事情又大,闹
出来外面也不雅。等他自作自受,少不得要自己败露的。”贾母道:“你这话说的
也是。这样事没有对证也难作准。只是佛爷菩萨看的真,他们姐儿两个如今又比谁
不济了呢?罢了,过去的事,凤哥儿也不必提了。今日你合你太太都在我这边吃了
晚饭再过去罢。”遂叫鸳鸯琥珀等传饭。凤姐赶忙笑道:“怎么老祖宗倒操起心来?”
王夫人也笑了。只见外头几个媳妇伺候。凤姐连忙告诉小丫头子传饭:“我合太太
都跟着老太太吃。”

正说着,只见玉钏儿走来对王夫人道:“老爷要找一件什么东西,请太太伺候
了老太太的饭完了,自己去找一找呢。”贾母道:“你去罢,保不住你老爷有要紧
的事。”王夫人答应着,便留下凤姐儿伺候,自己退了出来。回至房中,合贾政说
了些闲话,把东西找出来了。贾政便问道:“迎儿已经回去了?他在孙家怎么样?”
王夫人道:“迎丫头一肚子眼泪,说孙姑爷凶横的了不得。”因把迎春的话述了一
遍。贾政叹道:“我原知不是对头,无奈大老爷已说定了,叫我也没法。不过迎丫
头受些委屈罢了。”王夫人道:“这还是新媳妇,只指望他以后好了好。”说着,
“嗤”的一笑。贾政道:“笑什么?”王夫人道:“我笑宝玉儿早起,特特的到这
屋里来,说的都是些小孩子话。”贾政道:“他说什么?”王夫人把宝玉的言语笑
述了一遍。贾政也忍不住的笑,因又说道:“你提宝玉,我正想起一件事来了。这
孩子天天放在园里,也不是事。生女儿不得济,还是别人家的人;生儿若不济事,
关系非浅。前日倒有人和我提起一位先生来,学问人品都是极好的,也是南边人。
但我想南边先生,性情最是和平。咱们城里的孩子,个个踢天弄井,鬼聪明倒是有
的,可以搪塞就搪塞过去了,胆子又大。先生再要不肯给没脸,一日哄哥儿似的,
没的白耽误了。所以老辈子不肯请外头的先生,只在本家择出有年纪再有点学问的
请来掌家塾。如今儒大太爷虽学问也只中平,但还弹压的住这些小孩子们,不至以
颟顸了事。我想宝玉闲着总不好,不如仍旧叫他家塾中读书去罢了。”王夫人道:
“老爷说的很是。自从老爷外任去了,他又常病,竟耽搁了好几年。如今且在家学
里温习温习,也是好的。”贾政点头,又说些闲话不提。

且说宝玉次日起来,梳洗已毕,早有小厮们传进话来,说:“老爷叫二爷说话。”
宝玉忙整理了衣裳,来至贾政书房中,请了安,站着。贾政道:“你近来作些什么
功课?虽有几篇字,也算不得什么。我看你近来的光景,越发比头几年散荡了,况
且每每听见你推病,不肯念书。如今可大好了?我还听见你天天在园子里和姐妹们
玩玩笑笑,甚至和那些丫头们混闹,把自己的正经事总丢在脑袋后头。就是做得几
句诗词,也并不怎么样,有什么稀罕处?比如应试选举,到底以文章为主。你这上
头倒没有一点儿工夫!我可嘱咐你:自今日起,再不许做诗做对的了,单要习学八
股文章。限你一年,若毫无长进,你也不用念书了,我也不愿有你这样的儿子了。”
遂叫李贵来,说:“明儿一早,传焙茗跟了宝玉去收拾应念的书籍,一齐拿过来我
看看。亲自送他到家学里去。”喝命宝玉:“去罢!明日起早来见我。”

宝玉听了,半日竟无一言可答,因回到怡红院来。袭人正在着急听信,见说取
书,倒也喜欢。独是宝玉要人即刻送信给贾母,欲叫拦阻。贾母得信,便命人叫过
宝玉来,告诉他说:“只管放心先去,别叫你老子生气。有什么难为你,有我呢。”
宝玉没法,只得回来,嘱咐了丫头们:“明日早早叫我,老爷要等着送我到家学里
去呢。”袭人等答应了,同麝月两个倒替着醒了一夜。

次日一早,袭人便叫醒宝玉,梳洗了,换了衣裳,打发小丫头子传了焙茗在二
门上伺候,拿着书籍等物。袭人又催了两遍,宝玉只得出来,过贾政书房中来,先
打听老爷过来了没有。书房中小厮答应:“方才一位清客相公请老爷回话,里边说
‘梳洗呢’,命清客相公出去候着去了。”宝玉听了,心里稍稍安顿,连忙到贾政
这边来。恰好贾政着人来叫,宝玉便跟着进去。贾政不免又吩咐几句话,带了宝玉,
上了车,焙茗拿着书籍,一直到家塾中来。早有人先抢一步,回代儒说:“老爷来
了。”代儒站起身来,贾政早已走入,向代儒请了安。代儒拉着手问了好,又问:
“老太太近日安么?”宝玉过来也请了安。贾政站着,请代儒坐了,然后坐下。贾
政道:“我今日自己送他来,因要求托一番。这孩子年纪也不小了,到底要学个成
人的举业,才是终身立身成名之事。如今他在家中,只是和些孩子们混闹。虽懂得
几句诗词,也是胡诌乱道的;就是好了,也不过是风云月露,与一生的正事毫无关
涉。”代儒道:“我看他相貌也还体面,灵性也还去得,为什么不念书,只是心野
贪玩?诗词一道,不是学不得的,只要发达了以后,再学还不迟呢。”贾政道:“原
是如此。目今只求叫他读书、讲书、作文章。倘或不听教训,还求太爷认真的管教
管教他,才不至有名无实的,白耽误了他的一世。”说毕站起来,又作了一个揖,
然后说了些闲话,才辞了出去。代儒送至门首,说:“老太太前替我问好请安罢。”
贾政答应着,自己上车去了。

代儒回身进来,看见宝玉在西南角靠窗户摆着一张花梨小桌,右边堆下两套旧
书,薄薄儿的一本文章,叫焙茗将纸墨笔砚都搁在抽屉里藏着。代儒道:“宝玉,
我听见说你前儿有病,如今可大好了?”宝玉站起来道:“大好了。”代儒道:“如
今论起来,你可也该用功了。你父亲望你成人,恳切的很。你且把从前念过的书打
头儿理一遍,每日早起理书,饭后写字,晌午讲书,念几遍文章就是了。”宝玉答
应了个“是”。回身坐下时,不免四面一看。见昔时金荣辈不见了几个,又添了几
个小学生,都是些粗俗异常的。忽然想起秦钟来,如今没有一个做得伴、说句知心
话儿的。心上凄然不乐,却不敢作声,只是闷着看书。代儒告诉宝玉道:“今日头
一天,早些放你家去罢。明日要讲书了。但是你又不是很愚夯的,明日我倒要你先
讲一两章书我听,试试你近来的工课何如,我才晓得你到怎么个分儿上头。”说的
宝玉心中乱跳。

欲知明日讲解何如,且听下回分解。