\chapter{记微嫌舅兄欺弱女~惊谜语妻妾谏痴人}

话说邢王二夫人听尤氏一段话,明知也难挽回。王夫人只得说道:“姑娘要行
善,这也是前生的夙根,我们也实在拦不住。只是咱们这样人家的姑娘出了家,不
成个事体。如今你嫂子说了,准你修行,也是好处。却有一句话要说:那头发可以
不剃的,只要自己的心真,那在头发上头呢?你想妙玉也是带发修行的。不知他怎
样凡心一动,才闹到那个分儿。姑娘执意如此,我们就把姑娘住的房子便算了姑娘
的静室。所有服侍姑娘的人,也得叫他们来问。他若愿意跟的,就讲不得说亲配人;
若不愿意跟的,另打主意。”惜春听了,收了泪,拜谢了邢王二夫人,李纨、尤氏
等。王夫人说了,便问彩屏等:“谁愿跟姑娘修行?”彩屏等回道:“太太们派谁就
是谁。”

王夫人知道不愿意,正在想人。袭人立在宝玉身后,想来宝玉必要大哭,防着
他的旧病。岂知宝玉叹道:“真真难得!”袭人心里更自伤悲。宝钗虽不言语,遇事
试探,见他执迷不醒,只得暗中落泪。王夫人才要叫了众丫头来问,忽见紫鹃走上
前去,在王夫人面前跪下,回道:“刚才太太问跟四姑娘的姐姐,太太看着怎么样?”
王夫人道:“这个如何强派得人的?谁愿意,他自然就说出来了。”紫鹃道:“姑娘修
行,自然姑娘愿意,并不是别的姐姐们的意思。我有句话回太太:我也并不是拆开
姐姐们,各人有各人的心。我服侍林姑娘一场,林姑娘待我也是太太们知道的,实
在恩重如山,无以可报。他死了,我恨不得跟了他去,但只他不是这里的人,我又
受主子家的恩典,难以从死。如今四姑娘既要修行,我就求太太们将我派了跟着姑
娘,伏侍姑娘一辈子,不知太太们准不准?若准了,就是我的造化了。”邢王二夫人
尚未答言,只见宝玉听到那里,想起黛玉,一阵心酸,眼泪早下来了。众人才要问
他时,他又哈哈的大笑,走上来道:“我不该说的。这紫鹃蒙太太派给我屋里,我
才敢说:求太太准了他罢,全了他的好心。”王夫人道:“你头里姊妹出了嫁,还哭
得死去活来;如今看见四妹妹要出家,不但不劝,倒说‘好事’。你如今到底是怎
么个意思?我索性不明白了。”宝玉道:“四妹妹修行是已经准了的,四妹妹也是一
定的主意了?若是真呢,我有一句话告诉太太;若是不定呢,我就不敢混说了。”惜
春道:“二哥哥说话也好笑,一个人主意不定,便扭得过太太们来了?我也是像紫鹃
的话:容我呢,是我的造化;不容我呢还有一个死呢:那怕什么?二哥哥既有话,
只管说。”宝玉道:“我这也不算什么泄漏了,这也是一定的。我念一首诗给你们听
听罢。”众人道:“人家苦得很的时候,你倒来做诗怄人。”宝玉道:“不是做诗,我
到过一个地方儿看了来的。你们听听罢。”众人道:“使得。你就念念,别顺着嘴儿
胡诌。”宝玉也不分辩,便说道:
勘破三春景不长,缁衣顿改昔年妆。
可怜绣户侯门女,独卧青灯古佛旁。
李纨宝钗听了,诧异道:“不好了!这个人入了魔了。”王夫人听了这话,点头叹息,
便问:“宝玉,你到底是那里看来的?”宝玉不便说出来,回道:“太太也不必问我,
自有见的地方。”王夫人回过味来,细细一想,便更哭起来道:“你说前儿是玩话,
怎么忽然有这首诗?罢了,我知道了。你们叫我怎么样呢?我也没有法儿了,也只得
由着你们去罢,但只等我合上了眼,各自干各自的就完了!”

宝钗一面劝着,这个心比刀绞更甚,也掌不住,便放声大哭起来。袭人已经哭
的死去活来,幸亏秋纹扶着。宝玉也不啼哭,也不相劝,只不言语。贾兰贾环听到
那里,各自走开。李纨竭力的解说:“总是宝兄弟见四妹妹修行,他想来是痛极了,
不顾前后的疯话,这也作不得准。独有紫鹃的事情,准不准,好叫他起来。”王夫
人道:“什么依不依?横竖一个人的主意定了,那也是扭不过来的。可是宝玉说的,
也是一定的了!”紫鹃听了磕头。惜春又谢了王夫人。紫鹃又给宝玉宝钗磕了头,
宝玉念声:“阿弥陀佛!难得,难得!不料你倒先好了。”宝钗虽然有把持,也难掌住。
只有袭人也顾不得王夫人在上,便痛哭不止,说:“我也愿意跟了四姑娘去修行。”
宝玉笑道:“你也是好心,但是你不能享这个清福的。”袭人哭道:“这么说,我是
要死的了?”宝玉听到那里,倒觉伤心,只是说不出来。

因时已五更,宝玉请王夫人安歇。李纨等各自散去。彩屏等暂且伏侍惜春回去,
后来指配了人家。紫鹃终身伏侍,毫不改初。此是后话。

且言贾政扶了贾母灵柩,一路南行,因遇着班师的兵将船只过境,河道拥挤,
不能速行,在道实在心焦。幸喜遇见了海疆的官员,闻得镇海统制钦召回京,想来
探春一定回家,略略解些烦心。只打听不出起程的日期,心里又是烦躁。想到盘费
算来不敷,不得已写书一封,差人到赖尚荣任上借银五百,叫人沿途迎来,应付需
用。过了数日,贾政的船才行得十数里,那家人回来,迎上船只,将赖尚荣的禀启
呈上。书内告了多少苦处,备上白银五十两。贾政看了大怒,即命家人:“立刻送
还!将原书发回,叫他不必费心。”那家人无奈,只得回到赖尚荣任所。赖尚荣接到
原书银两,心中烦闷,知事办得不周到,又添了一百,央来人带回,帮着说些好话。
岂知那人不肯带回,撂下就走。赖尚荣心下不安,立刻修书到家,回明他父亲,叫
他设法告假,赎出身来。于是赖家托了贾蔷贾芸等在王夫人面前乞恩放出。贾蔷明
知不能,过了一日,假说王夫人不依的话,回覆了。赖家一面告假,一面差人到赖
尚荣任上,叫他告病辞官。王夫人并不知道。

那贾芸听见贾蔷的假话,心里便没想头。连日在外又输了好些银钱,无所抵偿,
便和贾环借贷。贾环本是一个钱没有的,虽是赵姨娘有些积蓄,早被他弄光了,那
能照应人家?便想起凤姐待他刻薄,趁着贾琏不在家,要摆布巧姐出气,遂把这个
当叫贾芸来上,故意的埋怨贾芸道:“你们年纪又大,放着弄银钱的事又不敢办,
倒和我没有钱的人商量。”贾芸道:“三叔你这话说的倒好笑。咱们一块儿玩,一块
儿闹,那里有有钱的事?”贾环道:“不是前儿有人说是外藩要买个偏房?你们何不
和王大舅商量,把巧姐说给他呢?”贾芸道:“叔叔,我说句招你生气的话:外藩
花了钱买人,还想能和咱们走动么?”贾环在贾芸耳边说了些话。贾芸虽然点头,
只道贾环是小孩子的话,也不当事。恰好王仁走来说道:“你们两个人商量些什么?
瞒着我吗?”贾芸便将贾环的话附耳低言的说了。王仁拍手道:“这倒是一宗好事,
又有银子。只怕你们不能。若是你们敢办,我是亲舅舅,做得主的。只要环老三在
大太太跟前那么一说,我找邢大舅再一说,太太们问起来,你们打伙儿说好就是了。”

贾环等商议定了,王仁便去找邢大舅,贾芸便去回邢王二夫人,说得锦上添花。
王夫人听了,虽然入耳,只是不信。邢夫人听得邢大舅知道,心里愿意,便打发人
找了邢大舅来问他。那邢大舅已经听了王仁的话,又可分肥,便在邢夫人跟前说道:
“若说这位郡王,是极有体面的。若应了这门亲事,虽说不是正配,管保一过了门,
姐夫的官早复了,这里的声势又好了。”邢夫人本是没主意的人,被傻大舅一番假
话哄得心动,请了王仁来一问,更说得热闹。于是邢夫人倒叫人出去追着贾芸去说。
王仁即刻找了人去到外藩公馆说了。那外藩不知底细,便要打发人来相看。贾芸又
钻了相看的人,说明:“原是瞒着合宅的,只说是王府相亲。等到成了,他祖母作
主,亲舅舅的保山,是不怕的。”那相看的人应了。贾芸便送信与邢夫人,并回了
王夫人。那李纨宝钗等不知原故,只道是件好事,也都欢喜。

那日果然来了几个女人,都是艳妆丽服。邢夫人接了进去,叙了些闲话。那来
人本知是个诰命,也不敢怠慢。邢夫人因事未定,也没有和巧姐说明,只说有亲戚
来瞧,叫他去见。巧姐到底是个小孩子,那管这些,便跟了奶妈过来。平儿不放心,
也跟着来。只见有两个宫人打扮的,见了巧姐,便浑身上下一看,更又起身来拉着
巧姐的手又瞧了一遍,略坐了一坐就走了。倒把巧姐看得羞臊。回到房中纳闷,想
来没有这门亲戚,便问平儿。平儿先看见来头,却也猜着八九:“必是相亲的。但
是二爷不在家,大太太作主,到底不知是那府里的。若说是对头亲,不该这样相看。
瞧那几个人的来头,不像是本支王府,好像是外头路数。如今且不必和姑娘说明,
且打听明白再说。”

平儿心下留神打听,那些丫头婆子都是平儿使过的,平儿一问,所有听见外头
的风声都告诉了。平儿便吓的没了主意,虽不和巧姐说,便赶着去告诉了李纨宝钗,
求他二人告诉王夫人。王夫人知道这事不好,便和邢夫人说知。怎奈邢夫人信了兄
弟并王仁的话,反疑心王夫人不是好意,便说:“孙女儿也大了。现在琏儿不在家,
这件事我还做得主。况且他亲舅爷爷和他亲舅舅打听的,难道倒比别人不真么?我
横竖是愿意的。倘有什么不好,我和琏儿也抱怨不着别人。”王夫人听了这些话,
心下暗暗生气,勉强说些闲话,便走了出来告诉了宝钗,自己落泪。宝玉劝道:“太
太别烦恼。这件事,我看来是不成的。这又是巧姐儿命里所招,只求太太不管就是
了。”王夫人道:“你一开口就是疯话!人家说定了就要接过去。若依平儿的话,你
琏二哥哥不抱怨我么?别说自己的侄孙女儿,就是亲戚家的,也是要好才好。邢姑
娘是我们作媒的,配了你二大舅子,如今和和顺顺的过日子,不好么?那琴姑娘,
梅家娶了去,听见说是丰衣足食的,很好。就是史姑娘,是他叔叔的主意,头里原
好,如今姑爷痨病死了,你史妹妹立志守寡,也就苦了。若是巧姐儿错给了人家儿,
可不是我的心坏?”正说着,平儿过来瞧宝钗,并探听邢夫人的口气。王夫人将邢
夫人的话说了一遍。平儿呆了半天,跪下求道:“巧姐儿终身,全仗着太太!若信了
人家的话,不但姑娘一辈子受了苦,便是琏二爷回来,怎么说呢?”王夫人道:“你
是个明白人,起来听我说:巧姐儿到底是大太太孙女儿,他要作主,我能够拦他么?”
宝玉劝道:“无妨碍的,只要明白就是了。”平儿生怕宝玉疯癫嚷出来,也并不言语,
回了王夫人,竟自去了。

这里王夫人想到烦闷,一阵心痛,叫丫头扶着,勉强回到自己房中躺下,不叫
宝玉宝钗过来,说睡睡就好的。自己却也烦闷。听见说李婶娘来了,也不及接待。
只见贾兰进来请了安,回道:“今早爷爷那里打发人带了一封书子来,外头小子们
传进来的。我母亲接了,正要过来,因我老娘来了,叫我先呈给太太瞧,回来我母
亲就过来来回太太。还说我老娘要过来呢。”说着,一面把书子呈上。王夫人一面
接书,一面问道:“你老娘来作什么?”贾兰道:“我也不知道。我只听见我老娘说:
我三姨儿的婆婆家有什么信儿来了。”王夫人听了,想起来还是前次给甄宝玉说了
李绮,后来放定下茶,想来此时甄家要娶过门,所以李婶娘来商量这件事情。便点
点头儿,一面拆开书信,见上面写着道:

近因沿途俱系海疆凯旋船只,不能迅速前行。闻探姐随翁婿来都,不知曾有信
否?前接到琏侄手禀,知大老爷身体欠安,亦不知已有确信否?宝玉兰儿场期已近,
务须实心用功,不可怠惰。老太太灵柩抵家,尚需日时。我身体平善,不必挂念。
此谕宝玉等知道。月日手书。蓉儿另禀。
王夫人看了,仍旧递给贾兰,说:“你拿去给你二叔叔瞧瞧,还交给你母亲罢。”正
说着,李纨同李婶娘过来,请安问好毕,王夫人让了坐。李婶娘便将甄家要娶李绮
的话说了一遍。大家商议了一会子。李纨因问王夫人道:“老爷的书子,太太看过
了么?”王夫人道:“看过了。”贾兰便拿着给他母亲瞧。李纨看了道:“三姑娘出
了门好几年,总没有来,如今要回京了,太太也放了好些心。”王夫人道:“我本是
心痛,看见探丫头要回来了,心里略好些,只是不知几时才到?”李婶娘便问了贾
政在路好。李纨因向贾兰道:“哥儿瞧见了?场期近了,你爷爷惦记的什么似的。你
快拿了去给二叔叔瞧去罢。”李婶娘道:“他们爷儿两个又没进过学,怎么能下场
呢?”王夫人道:“他爷爷做粮道的起身时,给他们爷儿两个援了例监了。”李婶娘
点头。贾兰一面拿着书子出来,来找宝玉。

却说宝玉送了王夫人去后,正拿着《秋水》一篇在那里细玩。宝钗从里间走出,
见他看的得意忘言,便走过来一看。见是这个,心里着实烦闷,细想:“他只顾把
这些出世离群的话当作一件正经事,终久不妥!”看他这种光景,料劝不过来,便
坐在宝玉傍边,怔怔的瞅着。宝玉见他这般,便道:“你这又是为什么?”宝钗道:
“我想你我既为夫妇,你便是我终身的倚靠,却不在情欲之私。论起荣华富贵,原
不过是过眼烟云;但自古圣贤,以人品根柢为重——”宝玉也没听完,把那本书搁
在旁边,微微的笑道:“据你说‘人品根柢’,又是什么‘古圣贤’,你可知古圣贤
说过,‘不失其赤子之心’?那赤子有什么好处?不过是无知无识无贪无忌。我们生
来已陷溺在贪嗔痴爱中,犹如污泥一般,怎么能跳出这般尘网?如今才晓得‘聚散
浮生’四字,古人说了,不曾提醒一个。既要讲到人品根柢,谁是到那太初一步地
位的?”宝钗道:“你既说‘赤子之心’,古圣贤原以忠孝为赤子之心,并不是遁世
离群、无关无系为赤子之心。尧、舜、禹、汤、周、孔,时刻以救民济世为心,所
谓赤子之心,原不过是‘不忍’二字。若你方才所说的忍于抛弃天伦,还成什么道
理?”宝玉点头笑道:“尧舜不强巢许,武周不强夷齐。”宝钗不等他说完,便道:
“你这个话,益发不是了。古来若都是巢、许、夷、齐,为什么如今人又把尧、舜、
周、孔称为圣贤呢?况且你自比夷齐,更不成话。夷齐原是生在殷商末世,有许多
难处之事,所以才有托而逃。当此圣世,咱们世受国恩,祖父锦衣玉食;况你自有
生以来,自去世的老太太,以及老爷太太,视如珍宝。你方才所说,自己想一想,
是与不是?”宝玉听了,也不答言,只有仰头微笑。宝钗因又劝道:“你既理屈词
穷,我劝你从此把心收一收,好好的用用功,但能博得一第,便是从此而止,也不
枉天恩祖德了。”宝玉点了点头,叹了口气,说道:“一第呢其实也不是什么难事。
倒是你这个‘从此而止’,‘不枉天恩祖德’,却还不离其宗。”宝钗未及答言,袭人
过来说道:“刚才二奶奶说的古圣先贤,我们也不懂。我只想着我们这些人,从小
儿辛辛苦苦跟着二爷,不知陪了多少小心,论起理来原该当的,但只二爷也该体谅
体谅。况且二奶奶替二爷在老爷太太跟前行了多少孝道,就是二爷不以夫妻为事,
也不可太辜负了人心。至于神仙那一层,更是谎话,谁见过有走到凡间来的神仙呢?
那里来的这么个和尚,说了些混话,二爷就信了真!二爷是读书的人,难道他的话
比老爷太太还重么?”宝玉听了,低头不语。

袭人还要说时,只听外面脚步走响,隔着窗户问道:“二叔在屋里呢么?”宝
玉听了是贾兰的声音,便站起来笑道:“你进来罢。”宝钗也站起来。贾兰进来,笑
容可掬的给宝玉宝钗请了安,问了袭人的好,袭人也问了好,便把书子呈给宝玉瞧。
宝玉接在手中看了,便道:“你三姑姑回来了?”贾兰道:“爷爷既如此写,自然是
回来的了。”宝玉点头不语,默默如有所思。贾兰便问:“叔叔看见了:爷爷后头写
着,叫咱们好生念书呢。叔叔这成子只怕总没作文章罢?”宝玉笑道:“我也要作
几篇熟一熟手,好去诓这个功名。”贾兰道:“叔叔既这样,就拟几个题目,我跟着
叔叔作作,也好进去混场。别到那时交了白卷子,惹人笑话;不但笑话我,人家连
叔叔都要笑话了。”宝玉道:“你也不至如此。”说着,宝钗命贾兰坐下。宝玉仍坐
在原处,贾兰侧身坐了。两个谈了一回文,不觉喜动颜色。宝钗见他爷儿两个谈得
高兴,便仍进屋里去了,心中细想:“宝玉此时光景,或者醒悟过来了。只是刚才
说话,他把那‘从此而止’四字单单的许可,这又不知是什么意思了?”宝钗尚自
犹豫。惟有袭人看他爱讲文章,提到下场,更又欣然,心里想道:“阿弥陀佛!好容
易讲《四书》似的才讲过来了。”这里宝玉和贾兰讲文,莺儿沏过茶来。贾兰站起
来接了,又说了一会子下场的规矩,并请甄宝玉在一处的话,宝玉也甚似愿意。

一时贾兰回去,便将书子留给宝玉了。那宝玉看着书子,笑嘻嘻走进来,递给
麝月收了,便出来将那本《庄子》收了。把几部向来最得意的,如《参同契》、《元
命苞》、《五灯会元》之类,叫出麝月、秋纹、莺儿等都搬了搁在一边。宝钗见他这
番举动,甚为罕异,因欲试探他,便笑问道:“不看他倒是正经,但又何必搬开呢。”
宝玉道:“如今才明白过来了。这些书都算不得什么。我还要一火焚之,方为干净。”
宝钗听了,更欣喜异常。只听宝玉口中微吟道:“内典语中无佛性,金丹法外有仙
舟。”宝钗也没很听真,只听得“无佛性”,“有仙舟”几个字,心中转又狐疑,且
看他作何光景。宝玉便命麝月秋纹等收拾一间静室,把那些语录名稿及应制诗之类
都找出来,搁在静室中,自己却当真静静的用起功来。宝钗这才放了心。

那袭人此时真是闻所未闻,见所未见,便悄悄的笑着向宝钗道:“到底奶奶说
话透彻!只一路讲究,就把二爷劝明白了。就只可惜迟了一点儿,临场太近了。”宝
钗点头微笑道:“功名自有定数,中与不中,倒也不在用功的迟早。但愿他从此一
心巴结正路,把从前那些邪魔永不沾染,就是好了。”说到这里,见房里无人,便
悄说道:“这一番悔悟过来固然很好,但只一件:怕又犯了前头的旧病,和女孩儿
们打起交道来,也是不好。”袭人道:“奶奶说的也是。二爷自从信了和尚,才把这
些姐妹冷淡了;如今不信和尚,真怕又要犯了前头的旧病呢。我想:奶奶和我,二
爷原不大理会。紫鹃去了,如今只他们四个。这里头就是五儿有些个狐媚子,听见
说,他妈求了大奶奶和奶奶,说要讨出去给人家儿呢,但是这两天到底在这里呢。
麝月秋纹虽没别的,只是二爷那几年也都有些顽顽皮皮的。如今算来,只有莺儿二
爷倒不大理会,况且莺儿也稳重。我想倒茶弄水,只叫莺儿带着小丫头们伏侍就够
了,不知奶奶心里怎么样?”宝钗道:“我也虑的是这个,你说的倒也罢了。”从此
便派莺儿带着小丫头伏侍。那宝玉却也不出房门,天天只差人去给王夫人请安。王
夫人听见他这番光景,那一种欣慰之情更不待言了。

到了八月初三这一日,正是贾母的冥寿。宝玉早晨过来磕了头,便回去,仍到
静室中去了。饭后,宝钗袭人等都和姊妹们跟着邢王二夫人在前面屋里说闲话儿。
宝玉自在静室,冥心危坐。忽见莺儿端了一盘瓜果进来,说:“太太叫人送来给二
爷吃的,这是老太太的克什。”宝玉站起来答应了,复又坐下,便道:“搁在那里罢。”
莺儿一面放下瓜果,一面悄悄向宝玉道:“太太那里夸二爷呢。”宝玉微笑。莺儿又
道:“太太说了:二爷这一用功,明儿进场中了出来,明年再中了进士,作了官,
老爷太太可就不枉了盼二爷了。”宝玉也只点头微笑。莺儿忽然想起那年给宝玉打
络子的时候宝玉说的话来,便道:“真要二爷中了,那可是我们姑奶奶的造化了。
二爷还记得那一年在园子里,不是二爷叫我打梅花络子时说的:我们姑奶奶后来带
着我不知到那一个有造化的人家儿去呢?如今二爷可是有造化的罢咧!”宝玉听到这
里,又觉尘心一动,连忙敛神定息,微微的笑道:“据你说来,我是有造化的,你
们姑娘也是有造化的,你呢?”莺儿把脸飞红了,勉强笑道:“我们不过当丫头一
辈子罢咧,有什么造化呢。”宝玉笑道:“果然能够一辈子是丫头,你这个造化比我
们还大呢。”莺儿听见这话,似乎又是疯话了,恐怕自己招出宝玉的病根来,打算
着要走。只见宝玉笑着说道:“傻丫头,我告诉你罢。”

未知宝玉又说出什么话来,且听下回分解。