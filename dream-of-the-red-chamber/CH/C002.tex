\chapter{贾夫人仙逝扬州城~冷子兴演说荣国府}

却说封肃听见公差传唤,忙出来陪笑启问,那些人只嚷:“快请出甄爷来。”
封肃忙陪笑道:“小人姓封,并不姓甄。只有当日小婿姓甄,今已出家一二年了,
不知可是问他?”那些公人道:“我们也不知什么‘真’‘假’,既是你的女婿,
就带了你去面太爷便了。”大家把封肃推拥而去,封家各各惊慌,不知何事。至
二更时分,封肃方回来,众人忙问端的。——“原来新任太爷姓贾名化,本湖州人
氏,曾与女婿旧交,因在我家门首看见娇杏丫头买线,只说女婿移住此间,所以来
传。我将缘故回明,那太爷感伤叹息了一回;又问外孙女儿,我说看灯丢了。太爷
说:‘不妨,待我差人去,务必找寻回来。’说了一回话,临走又送我二两银子。”
甄家娘子听了,不觉感伤。一夜无话。

次日,早有雨村遣人送了两封银子、四匹锦缎,答谢甄家娘子;又一封密书与
封肃,托他向甄家娘子要那娇杏作二房。封肃喜得眉开眼笑,巴不得去奉承太爷,
便在女儿前一力撺掇。当夜用一乘小轿,便把娇杏送进衙内去了。雨村欢喜自不必
言,又封百金赠与封肃,又送甄家娘子许多礼物,令其且自过活,以待访寻女儿下
落。却说娇杏那丫头便是当年回顾雨村的,因偶然一看便弄出这段奇缘,也是意想
不到之事。谁知他命运两济,不承望自到雨村身边,只一年便生一子,又半载雨村
嫡配忽染疾下世,雨村便将他扶作正室夫人。正是:
偶因一回顾,便为人上人。

原来雨村因那年士隐赠银之后,他于十六日便起身赴京。大比之期,十分得意,
中了进士,选入外班,今已升了本县太爷。虽才干优长,未免贪酷,且恃才侮上,
那同寅皆侧目而视。不上一年,便被上司参了一本,说他貌似有才,性实狡猾,又
题了一两件徇庇蠹役、交结乡绅之事,龙颜大怒,即命革职。部文一到,本府各官
无不喜悦。那雨村虽十分惭恨,面上却全无一点怨色,仍是嘻笑自若;交代过了公
事,将历年所积的宦囊,并家属人等,送至原籍安顿妥当了,却自己担风袖月,游
览天下胜迹。

那日偶又游至维扬地方,闻得今年盐政点的是林如海。这林如海姓林名海,表
字如海,乃是前科的探花,今已升兰台寺大夫,本贯姑苏人氏,今钦点为巡盐御史,
到任未久。原来这林如海之祖也曾袭过列侯的,今到如海,业经五世。起初只袭三
世,因当今隆恩盛德,额外加恩,至如海之父又袭了一代,到了如海便从科第出身。
虽系世禄之家,却是书香之族。只可惜这林家支庶不盛,人丁有限,虽有几门,却
与如海俱是堂族,没甚亲支嫡派的。今如海年已五十,只有一个三岁之子,又于去
岁亡了,虽有几房姬妾,奈命中无子,亦无可如何之事。只嫡妻贾氏生得一女,乳
名黛玉,年方五岁,夫妻爱之如掌上明珠。见他生得聪明俊秀,也欲使他识几个字,
不过假充养子,聊解膝下荒凉之叹。

且说贾雨村在旅店偶感风寒,愈后又因盘费不继,正欲得一个居停之所以为息
肩之地。偶遇两个旧友认得新盐政,知他正要请一西席教训女儿,遂将雨村荐进衙
门去。这女学生年纪幼小,身体又弱,工课不限多寡,其馀不过两个伴读丫鬟,故
雨村十分省力,正好养病。看看又是一载有馀,不料女学生之母贾氏夫人一病而亡。
女学生奉侍汤药,守丧尽礼,过于哀痛,素本怯弱,因此旧病复发,有好些时不曾
上学。雨村闲居无聊,每当风日晴和,饭后便出来闲步。

这一日偶至郊外,意欲赏鉴那村野风光。信步至一山环水漩、茂林修竹之处,
隐隐有座庙宇,门巷倾颓,墙垣剥落。有额题曰“智通寺”。门旁又有一副旧破的
对联云:
身后有馀忘缩手,
眼前无路想回头。
雨村看了,因想道:“这两句文虽甚浅,其意则深。也曾游过些名山大刹,倒不曾
见过这话头,其中想必有个翻过筋斗来的也未可知,何不进去一访。”走入看时,
只有一个龙钟老僧在那里煮粥。雨村见了,却不在意;及至问他两句话,那老僧既
聋且昏,又齿落舌钝,所答非所问。雨村不耐烦,仍退出来,意欲到那村肆中沽饮
三杯,以助野趣。于是移步行来。刚入肆门,只见座上吃酒之客有一人起身大笑,
接了出来,口内说:“奇遇,奇遇!”雨村忙看时,此人是都中古董行中贸易姓冷
号子兴的,旧日在都相识。雨村最赞这冷子兴是个有作为大本领的人,这子兴又借
雨村斯文之名,故二人最相投契。雨村忙亦笑问:“老兄何日到此?弟竟不知。今
日偶遇,真奇缘也。”子兴道:“去年岁底到家,今因还要入都,从此顺路找个敝
友说一句话。承他的情,留我多住两日。我也无甚紧事,且盘桓两日,待月半时也
就起身了。今日敝友有事,我因闲走到此,不期这样巧遇!”一面说一面让雨村同
席坐了,另整上酒肴来。

二人闲谈慢饮,叙些别后之事。雨村因问:“近日都中可有新闻没有?”子兴
道:“倒没有什么新闻,倒是老先生的贵同宗家出了一件小小的异事。”雨村笑道:
“弟族中无人在都,何谈及此?”子兴笑道:“你们同姓,岂非一族?”雨村问:
“是谁家?”子兴笑道:“荣国贾府中,可也不玷辱老先生的门楣了!”雨村道:
“原来是他家。若论起来,寒族人丁却自不少,东汉贾复以来,支派繁盛,各省皆
有,谁能逐细考查?若论荣国一支,却是同谱。但他那等荣耀,我们不便去认他,
故越发生疏了。”子兴叹道:“老先生休这样说。如今的这荣、宁两府,也都萧索
了,不比先时的光景!”雨村道:“当日宁荣两宅人口也极多,如何便萧索了呢?”
子兴道:“正是,说来也话长。”雨村道:“去岁我到金陵时,因欲游览六朝遗迹,
那日进了石头城,从他宅门前经过。街东是宁国府,街西是荣国府,二宅相连,竟
将大半条街占了。大门外虽冷落无人,隔着围墙一望,里面厅殿楼阁也还都峥嵘轩
峻,就是后边一带花园里,树木山石,也都还有葱蔚洇润之气,那里像个衰败之
家?”子兴笑道:“亏你是进士出身,原来不通。古人有言:‘百足之虫,死而不
僵。’如今虽说不似先年那样兴盛,较之平常仕宦人家,到底气象不同。如今人口
日多,事务日盛,主仆上下都是安富尊荣,运筹谋画的竟无一个,那日用排场,又
不能将就省俭。如今外面的架子虽没很倒,内囊却也尽上来了。这也是小事。更有
一件大事:谁知这样钟鸣鼎食的人家儿,如今养的儿孙,竟一代不如一代了!”

雨村听说,也道:“这样诗礼之家,岂有不善教育之理?别门不知,只说这宁
荣两宅,是最教子有方的,何至如此?”子兴叹道:“正说的是这两门呢。等我告
诉你:当日宁国公是一母同胞弟兄两个。宁公居长,生了两个儿子。宁公死后,长
子贾代化袭了官,也养了两个儿子:长子名贾敷,八九岁上死了,只剩了一个次子
贾敬,袭了官,如今一味好道,只爱烧丹炼汞,别事一概不管。幸而早年留下一个
儿子,名唤贾珍,因他父亲一心想作神仙,把官倒让他袭了。他父亲又不肯住在家
里,只在都中城外和那些道士们胡羼。这位珍爷也生了一个儿子,今年才十六岁,
名叫贾蓉。如今敬老爷不管事了,这珍爷那里干正事?只一味高乐不了,把那宁国
府竟翻过来了,也没有敢来管他的人。再说荣府你听:方才所说异事就出在这里。
自荣公死后,长子贾代善袭了官,娶的是金陵世家史侯的小姐为妻。生了两个儿子,
长名贾赦,次名贾政。如今代善早已去世,太夫人尚在。长子贾赦袭了官,为人却
也中平,也不管理家事;惟有次子贾政,自幼酷喜读书,为人端方正直。祖父钟爱,
原要他从科甲出身,不料代善临终遗本一上,皇上怜念先臣,即叫长子袭了官;又
问还有几个儿子,立刻引见,又将这政老爷赐了个额外主事职衔,叫他入部习学,
如今现已升了员外郎。这政老爷的夫人王氏,头胎生的公子名叫贾珠,十四岁进学,
后来娶了妻、生了子,不到二十岁,一病就死了。第二胎生了一位小姐,生在大年
初一,就奇了。不想隔了十几年,又生了一位公子,说来更奇:一落胞胎,嘴里便
衔下一块五彩晶莹的玉来,还有许多字迹。你道是新闻不是?”

雨村笑道:“果然奇异,只怕这人的来历不小。”子兴冷笑道:“万人都这样
说,因而他祖母爱如珍宝。那周岁时,政老爷试他将来的志向,便将世上所有的东
西摆了无数叫他抓。谁知他一概不取,伸手只把些脂粉钗环抓来玩弄,那政老爷便
不喜欢,说将来不过酒色之徒,因此不甚爱惜。独那太君还是命根子一般。——说
来又奇:如今长了十来岁,虽然淘气异常,但聪明乖觉,百个不及他一个;说起孩
子话来也奇,他说:‘女儿是水做的骨肉,男子是泥做的骨肉。我见了女儿便清爽,
见了男子便觉浊臭逼人。’你道好笑不好笑?将来色鬼无疑了!”

雨村罕然厉色道:“非也!可惜你们不知道这人的来历,大约政老前辈也错以
淫魔色鬼看待了。若非多读书识事,加以致知格物之功、悟道参玄之力者,不能知
也。”子兴见他说得这样重大,忙请教其故。雨村道:“天地生人,除大仁大恶,
馀者皆无大异。若大仁者则应运而生,大恶者则应劫而生,运生世治,劫生世危。
尧、舜、禹、汤、文、武、周、召、孔、孟、董、韩、周、程、朱、张,皆应运而
生者;蚩尤、共工、桀、纣、始皇、王莽、曹操、桓温、安禄山、秦桧等,皆应劫
而生者。大仁者修治天下,大恶者扰乱天下。清明灵秀,天地之正气,仁者之所秉
也;残忍乖僻,天地之邪气,恶者之所秉也。今当祚永运隆之日,太平无为之世,
清明灵秀之气所秉者,上自朝廷,下至草野,比比皆是。所馀之秀气漫无所归,遂
为甘露、为和风,洽然溉及四海。彼残忍乖邪之气,不能荡溢于光天化日之下,遂
凝结充塞于深沟大壑之中。偶因风荡,或被云摧,略有摇动感发之意,一丝半缕误
而逸出者,值灵秀之气适过,正不容邪,邪复妒正,两不相下;如风水雷电地中既
遇,既不能消,又不能让,必致搏击掀发。既然发泄,那邪气亦必赋之于人。假使
或男或女偶秉此气而生者,上则不能为仁人为君子,下亦不能为大凶大恶。置之千
万人之中,其聪俊灵秀之气,则在千万人之上;其乖僻邪谬不近人情之态,又在千
万人之下。若生于公侯富贵之家,则为情痴情种。若生于诗书清贫之族,则为逸士
高人。纵然生于薄祚寒门,甚至为奇优,为名娼,亦断不至为走卒健仆,甘遭庸夫
驱制。如前之许由、陶潜、阮籍、嵇康、刘伶、王谢二族、顾虎头、陈后主、唐明
皇、宋徽宗、刘庭芝、温飞卿、米南宫、石曼卿、柳耆卿、秦少游,近日倪云林、
唐伯虎、祝枝山,再如李龟年、黄幡绰、敬新磨、卓文君、红拂、薛涛、崔莺、朝
云之流,此皆易地则同之人也。”

子兴道:“依你说,‘成则公侯败则贼’了?”雨村道:“正是这意。你还不
知,我自革职以来,这两年遍游各省,也曾遇见两个异样孩子,所以方才你一说这
宝玉,我就猜着了八九也是这一派人物。不用远说,只这金陵城内钦差金陵省体仁
院总裁甄家,你可知道?”子兴道:“谁人不知!这甄府就是贾府老亲,他们两家
来往极亲热的。就是我也和他家往来非止一日了。”雨村笑道:“去岁我在金陵,
也曾有人荐我到甄府处馆。我进去看其光景,谁知他家那等荣贵,却是个富而好礼
之家,倒是个难得之馆。但是这个学生虽是启蒙,却比一个举业的还劳神,说起来
更可笑,他说:‘必得两个女儿陪着我读书,我方能认得字,心上也明白,不然我
心里自己糊涂。’又常对着跟他的小厮们说:‘这女儿两个字极尊贵极清净的,比
那瑞兽珍禽、奇花异草更觉希罕尊贵呢,你们这种浊口臭舌万万不可唐突了这两个
字,要紧,要紧!但凡要说的时节,必用净水香茶漱了口方可;设若失错,便要凿
牙穿眼的。’其暴虐顽劣,种种异常;只放了学进去,见了那些女儿们,其温厚和
平、聪敏文雅,竟变了一个样子。因此他令尊也曾下死笞楚过几次,竟不能改。每
打的吃疼不过时,他便‘姐姐’‘妹妹’的乱叫起来。后来听得里面女儿们拿他取
笑:‘因何打急了只管叫姐妹作什么?莫不叫姐妹们去讨情讨饶?你岂不愧些!’他
回答的最妙,他说:‘急痛之时,只叫姐姐妹妹字样,或可解疼也未可知,因叫了
一声,果觉疼得好些。遂得了秘法,每疼痛之极,便连叫姐妹起来了。’你说可笑
不可笑?为他祖母溺爱不明,每因孙辱师责子,我所以辞了馆出来的。这等子弟必
不能守祖父基业、从师友规劝的。只可惜他家几个好姊妹都是少有的!”

子兴道:“便是贾府中现在三个也不错。政老爷的长女名元春,因贤孝才德,
选入宫作女史去了。二小姐乃是赦老爷姨娘所出,名迎春。三小姐政老爷庶出,名
探春。四小姐乃宁府珍爷的胞妹,名惜春。因史老夫人极爱孙女,都跟在祖母这边,
一处读书,听得个个不错。”雨村道:“更妙在甄家风俗,女儿之名亦皆从男子之
名,不似别人家里另外用这些‘春’‘红’‘香’‘玉’等艳字。何得贾府亦落此
俗套?”子兴道:“不然。只因现今大小姐是正月初一所生,故名‘元春’,馀者
都从了‘春’字;上一排的却也是从弟兄而来的。现有对证:目今你贵东家林公的
夫人,即荣府中赦、政二公的胞妹,在家时名字唤贾敏。不信时你回去细访可知。”
雨村拍手笑道:“是极。我这女学生名叫黛玉,他读书凡‘敏’字他皆念作‘密’
字,写字遇着‘敏’字亦减一二笔。我心中每每疑惑,今听你说,是为此无疑矣。
怪道我这女学生言语举止另是一样,不与凡女子相同。度其母不凡,故生此女,今
知为荣府之外孙,又不足罕矣!可惜上月其母竟亡故了。”子兴叹道:“老姊妹三
个,这是极小的,又没了!长一辈的姊妹一个也没了。只看这小一辈的,将来的东
床何如呢。”

雨村道:“正是。方才说政公已有一个衔玉之子,又有长子所遗弱孙,这赦老
竟无一个不成?”子兴道:“政公既有玉儿之后,其妾又生了一个,倒不知其好歹。
只眼前现有二子一孙,却不知将来何如。若问那赦老爷,也有一子,名叫贾琏,今
已二十多岁了,亲上做亲,娶的是政老爷夫人王氏内侄女,今已娶了四五年。这位
琏爷身上现捐了个同知,也是不喜正务的,于世路上好机变,言谈去得,所以目今
现在乃叔政老爷家住,帮着料理家务。谁知自娶了这位奶奶之后,倒上下无人不称
颂他的夫人,琏爷倒退了一舍之地:模样又极标致,言谈又爽利,心机又极深细,
竟是个男人万不及一的。”雨村听了笑道:“可知我言不谬了。你我方才所说的这
几个人,只怕都是那正邪两赋而来一路之人,未可知也。”

子兴道:“正也罢,邪也罢,只顾算别人家的账,你也吃杯酒才好。”雨村道:
“只顾说话,就多吃了几杯。”子兴笑道:“说着别人家的闲话,正好下酒,即多
吃几杯何妨。”雨村向窗外看道:“天也晚了,仔细关了城,我们慢慢进城再谈,
未为不可。”于是二人起身,算还酒钱。方欲走时,忽听得后面有人叫道:“雨村
兄恭喜了!特来报个喜信的。”

雨村忙回头看时,——要知是谁,且听下回分解。