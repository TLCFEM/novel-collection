\chapter{闲取乐偶攒金庆寿~不了情暂撮土为香}

话说王夫人因见贾母那日在大观园不过着了些风寒,不是什么大病,请医生吃
了两剂药也就好了,命凤姐来,吩咐他预备给贾政带送东西。正商议着,只见贾母
打发人来叫,王夫人忙引着凤姐儿过来。王夫人又请问:“这会子可又觉大安些?”
贾母道:“今日可大好了。方才你们送来野鸡崽子汤,我尝了一尝,倒有味儿,又
吃了两块肉,心里很受用。”王夫人笑道:“这是凤丫头孝敬老太太的,算他的孝心
虔,不枉了素日老太太疼他。”贾母点头笑道:“难为他想着。若是还有生的,再炸
上两块,咸浸浸的,喝粥有味儿。那汤虽好,就只不对稀饭。”凤姐听了,连忙答
应,命人到大厨房传话。

这里贾母又向王夫人笑道:“我打发人找你来,不为别的:初二日是凤丫头的
生日。上两年我原想着替他做生日,偏到跟前又有事就混过去了。今年人又齐全,
料着又没事,咱们大家好生乐一天。”王夫人笑道:“我也想着呢。既是老太太高兴,
何不就商议定了?”贾母笑道:“我想往年不拘谁做生日,都是各自送各自的礼,
这个也俗了,也觉太生分。今儿我出个新法子,又不生分,又可以取乐儿。”王夫
人忙道:“老太太怎么想着好,就是怎么样行。”贾母笑道:“我想着咱们也学那小
家子,大家凑个分子,多少尽着这钱去办,你说好不好?”王夫人道:“这个很好,
但不知怎么个凑法儿?”贾母听说,一发高兴起来,忙遣人去请薛姨妈邢夫人等,
又叫请姑娘们并宝玉,和那府里的尤氏和赖大家的,及有些头脸管事的媳妇也都叫
了来。众丫头婆子见贾母十分高兴,也都高兴,忙忙的各自分头去请的请,传的传。
没顿饭的工夫,老的少的,上的下的,乌压压挤了一屋子。只薛姨妈和贾母对坐,
邢夫人王夫人只坐在房门前两张椅子上,宝钗姐妹等五六个人坐在炕上,宝玉坐在
贾母怀前,底下满满的站了一地。贾母忙命拿几张小杌子来,给赖大母亲等几个高
年有体面的嬷嬷坐了。贾府风俗:年高伏侍过父母的家人比年轻的主子还有体面呢,
所以尤氏凤姐等只管地下站着,那赖大的母亲等三四个老嬷嬷告了罪,都坐在小杌
子上。

贾母笑着把方才一夕话说与众人听了,众人谁不凑这趣儿呢。再也有和凤姐儿
好,情愿这样的。也有怕凤姐儿,巴不得奉承他的。况且都是拿的出来的,所以一
闻此言都欣然应诺。贾母先道:“我出二十两。”薛姨妈笑道:“我随着老太太,也
是二十两。”邢夫人王夫人笑道:“我们不敢和老太太并肩,自然矮一等,每人十六
两罢了。”尤氏李纨也笑道:“我们自然又矮一等,每人十二两罢。”贾母忙和李纨
道:“你寡妇失业的,那里还拉你出这个钱,我替你出了罢。”凤姐忙笑道:“老太
太别高兴,且算一算账再揽事。老太太身上已有两分呢。这会子又替大嫂子出十二
两,说着高兴,一会子回想又心疼了!过后儿又说:‘都是为凤丫头花了钱。’使个
巧法子,哄着我拿出三四倍子来暗里补上,我还做梦呢!”说的众人都笑了。贾母
笑道:“依你怎么样呢?”凤姐笑道:“生日没到,我这会子已经折受的不受用了。
我一个钱也不出,惊动这些人,实在不安,不如大嫂子这分我替他出了罢。我到那
一日多吃些东西,就享了福了。”邢夫人等听了,都说很是,贾母方允了。

凤姐儿又笑道:“我还有一句话呢:我想老祖宗自己二十两,又有林妹妹宝兄
弟的两分子;姨妈自己二十两,又有宝妹妹的一分子:这倒也公道。只是二位太太
每位十六两,自己又少,又不替人出,这有些不公道。老祖宗吃了亏了!”贾母听
了,呵呵大笑道:“到底是我的凤丫头向着我,这说的很是。要不是你,我叫他们
又哄了去了。”凤姐笑道:“老祖宗只把他哥儿两个交给两位太太,一位占一个罢,
派每位替出一分就是了。”贾母忙说:“这很公道,就是这样。”赖大的母亲忙站起
来笑道:“这可反了,我替二位太太生气!在那边是儿子媳妇,在这边是内侄女儿,
倒不向着婆婆姑姑,倒向着别人,这儿媳妇倒成了陌路人,‘内’侄女儿倒成了‘外’
侄女儿了!”说的贾母和众人都大笑起来了。

赖大的母亲因又问道:“少奶奶们十二两,我们自然也该矮一等了?”贾母听
说,道:“这使不得。你们虽该矮一等,我知道你们这几个都是财主,位虽低些,
钱却比他们多。你们和他们一例才使得。”众嬷嬷听了,连忙答应。贾母又道:“姑
娘们不过应个景儿,每人照一个月的月例就是了。”又回头叫鸳鸯来:“你们也凑几
个人,商议凑了来。”鸳鸯答应着,去不多时,带了平儿、袭人、彩霞等,还有几
个丫头来,也有二两的,也有一两的。贾母因问平儿:“你难道不替你主子做生日?
还入在这里头?”平儿笑道:“我那个私自另外的有了,这是公中的,也该出一分。”
贾母笑道:“这才是好孩子。”

凤姐又笑道:“上下都全了;还有二位姨奶奶,他出不出也问一声儿。尽到他
们是理,不然他们只当小看了他们了。”贾母听说,忙说:“可是呢。怎么倒忘了他
们?只怕他们不得闲儿,叫个丫头问问去。”说着,早有丫头去了。半日回来说道:
“每位也出二两。”贾母喜欢道:“拿笔砚来算明,共计多少。”尤氏因悄悄的骂凤
姐道:“我把你这没足够的小蹄子儿!这么些婆婆婶子凑银子给你做生日,你还不
够,又拉上两个苦瓠子。”凤姐也悄悄的笑道:“你少胡说,一会子离了这里,我才
和你算账!他们两个为什么苦呢?有了钱也是白填还别人,不如拘了来咱们乐。”

说着早已合了,共凑了一百五十两有零。贾母道:“一天戏酒用不了。”尤氏道:
“既不请客,酒席又不多,两三日的用度都够了。头等,戏不用钱,省在这上头。”
贾母道:“凤丫头说那一班好,就传那一班。”凤姐道:“咱们家的班子都听熟了,
倒是花几个钱叫一班来听听罢。”贾母道:“这件事我交给珍哥媳妇了,越发叫凤丫
头别操一点心儿,受用一日才算。”尤氏答应着。又说了一回话,都知贾母乏了,
才渐渐的散出来。

尤氏等送出邢夫人王夫人二人散去,因往凤姐房里来,商议怎么办生日的话。
凤姐儿道:“你不用问我,你只看老太太的眼色儿行事就完了。”尤氏笑道:“你这
么个阿物儿,也忒行了大运了。我当有什么事叫我们去,原来单为这个!出了钱不
算,还叫我操心,你怎么谢我?”凤姐笑道:“别扯臊!我又没叫你来,谢你什么?
你怕操心,你这会子就回老太太去,再派一个就是了。”尤氏笑道:“你瞧瞧,把他
兴的这个样儿!我劝你收着些儿好,太满了就要流出来了。”二人又说了一回方散。

次日,将银子送到宁国府来,尤氏方才起来梳洗,因问:“是谁送过来的?”
丫头们回说:“林妈。”尤氏便命:“叫了他来。”丫头们走至下房,叫了林之孝家的
过来。尤氏命他脚踏上坐了,一面忙着梳洗,一面问他:“这一包银子共多少?”
林之孝家的回说:“这是我们底下人的银子,凑了先送过来。老太太和太太们的还
没有呢。”正说着,丫头们回说:“那府里的姨太太打发人送了分子来了。”尤氏笑
骂道:“小蹄子们,专会记得这些没要紧的话!昨儿不过是老太太一时高兴,故意儿
的学那小家子凑分子,你们就记得了,到了你们嘴里当正经话说。还不快接进来呢!”
丫头们笑着忙接银子进来,一共两封,连宝钗、黛玉的都有了。尤氏问:“还少谁
的?”林之孝家的道:“还少老太太、太太、姑娘们的,我们底下姑娘们的。”尤氏
道:“还有你们大奶奶的呢?”林之孝家的道:“奶奶过去,这银子都从二奶奶手里
发,一共都有了。”

说着,尤氏梳洗了,命人伺候车辆。一时来至荣府,先来见凤姐,只见凤姐已
将银子封好,正要送去。尤氏问:“都齐了么?”凤姐笑道:“都有了!快拿去罢,
丢了我不管。”尤氏笑道:“我有些信不及,倒要当面点一点。”说着,果然按数一
点,只没有李纨的一分。尤氏笑道:“我说你闹鬼呢!怎么你大嫂子的没有?”凤姐
笑道:“那么些还不够?就短一分儿也罢了。等不够了,我再找给你。”尤氏道:“昨
儿你在人跟前做情,今儿又来和我赖,这我可不依你。我只和老太太要去。”凤姐
笑道:“我看你利害,明儿有了事,我也丁是丁卯是卯的,你也别抱怨!”尤氏笑道:
“只这一分儿不给也罢了,要不看你素日孝敬我,我本来依你么?”说着,把平儿
的一分也拿出来,说道:“平儿来把你的收了去,等不够了,我替你添上。”平儿会
意,笑道:“奶奶先使着,若剩下了,再赏我一样。”尤氏笑道:“只许你主子作弊,
就不许我作情吗?”平儿只得收了。尤氏又道:“我看着你主子这么细致,弄这些
钱,那里使去?使不了,明儿带了棺材里使去!”一面说着,一面又往贾母处来。先
请了安,大概说了两句话,便走到鸳鸯房中,和鸳鸯商议,只听鸳鸯的主意行事,
何以讨贾母喜欢。二人计议妥当。尤氏临走时,也把鸳鸯的二两银子还他,说:“这
还使不了呢。”说着,一径出来,又至王夫人跟前说了一回话,因王夫人进了佛堂,
把彩云的一分也还了他。凤姐儿不在跟前,一时把周赵二人的也还了。他两个还不
敢收,尤氏道:“你们可怜见的,那里有这些闲钱?凤丫头便知道了,有我应着呢。”
二人听说,千恩万谢的收了。

转眼已是九月初二日,园中人都打听得尤氏办得十分热闹,不但有戏,连耍百
戏并说书的女先儿全有,都打点着取乐玩耍。李纨又向众姐妹道:“今儿是正经社
日,可别忘了。宝玉也不来,想必他不知,又贪住什么玩意儿,把这事又忘了。”
说着,便命丫头:“去瞧做什么呢,快请了来。”丫头去了半日,回说:“花大姐姐
说,今儿一早就出门去了。”众人听了都诧异,说:“再没有出门之理。这丫头糊涂!”
因又命翠墨去。一时翠墨回来,说:“可不真出门了!说有个朋友死了,出去探丧去
了。”探春道:“断然没有的事。凭他什么,再没有今日出门之理。你叫袭人来,我
问他。”刚说着,只见袭人走来,李纨等都说道:“今儿凭他有什么事,也不该出门。
头一件,你二奶奶的生日,老太太都这么高兴,两府上下都凑热闹儿,他倒走了?
第二件,又是头一社的正日子,也不告假,就私自去了!”袭人叹道:“昨儿晚上就
说了,今儿一早有要紧的事,到北静王府里去,就赶着回来。劝他别去,他必不依。
今儿一早起来,又要素衣裳穿,想必是北静王府里要紧的什么人没了也未可知。”
李纨等道:“若果如此,也该去走走,只是也该回来了。”说着,大家又商议:“咱
们只管作诗,等他来罚他。”刚说着,只见贾母已打发人来请,便都往前头去了。
袭人回明宝玉的事,贾母不乐,便命人接去。

原来宝玉心里有件心事,于头一日就吩咐焙茗:“明日一早出门,备两匹马在
后门口等着,不用别人跟着。说给李贵:我往北府里去了,倘或要有人找我,叫他
拦住不用找。只说北府里留下了,横竖就来的。”焙茗也摸不着头脑,只得依言说
了,今儿一早果然备了两匹马,在园后门等着。天亮了,只见宝玉遍体纯素,从角
门出来,一语不发跨上马,一弯腰顺着街就蹭下去了。焙茗也只得跨上马,加鞭赶
上,在后面忙问:“往那里去?”宝玉道:“这条路是往那里去的?”焙茗道:“这
是出北门的大道。出去了冷清清,没有什么玩的。”宝玉听说,点头道:“正要冷清
清的地方。”说着,越发加了两鞭,那马早已转了两个弯子,出了城门。焙茗越发
不得主意,只得紧紧的跟着。

一气跑了七八里路出来,人烟渐渐稀少,宝玉方勒住马,回头问焙茗道:“这
里可有卖香的?”焙茗道:“香倒有,不知是那一样?”宝玉想到别的香不好,须
得檀、芸、降三样。焙茗笑道:“这三样可难得。”宝玉为难。焙茗见他为难,因问
道:“要香做什么使?我见二爷时常带的小荷包儿有散香,何不找找?”一句提醒了
宝玉,便回手衣襟上挂着个荷包摸了一摸,竟有两星沉速,心内喜欢:“只是不恭
些。”再想:“自己亲身带的,倒比买的又好些。”于是又问炉炭,焙茗道:“这可罢
了,荒郊野外,那里有?既用这些,何不早说,带了来岂不便宜?”宝玉道:“糊涂
东西!要可以带了来,又不这样没命的跑了。”焙茗想了半日,笑道:“我得了个主
意,不知二爷心下如何。我想来二爷不止用这个,只怕还要用别的,这也不是事。
如今我们索性往前再走二里,就是水仙庵了。”宝玉听了,忙问:“水仙庵就在这里?
更好了。我们就去。”说着就加鞭前行,一面回头向焙茗道:“这水仙庵的姑子长往
咱们家去,这一去到那里和他借香炉使使,他自然是肯的。”焙茗道:“别说是咱们
家的香火,就是平白不认识的庙里,和他借,他也不敢驳回。只是一件,我常见二
爷最厌这水仙庵的,如何今儿又这样喜欢了?”宝玉道:“我素日最恨俗人不知原
故混供神,混盖庙。这都是当日有钱的老公们和那些有钱的愚妇们,听见有个神,
就盖起庙来供着,也不知那神是何人,因听些野史小说便信真了。比如这水仙庵里
面,因供的是洛神,故名水仙庵。殊不知古来并没有个洛神,那原是曹子建的谎话,
谁知这起愚人就塑了像供着。今儿却合我的心事,故借他一用。”

说着,早已来至门前。那老姑子见宝玉来了,事出意外,竟像天上掉下个活龙
来的一般,忙上来问好,命老道来接马。宝玉进去,也不拜洛神之像,却只管赏鉴。
虽是泥塑的,却真有那“翩若惊鸿,婉若游龙”、“荷出渌波,日映朝霞”的姿态。
宝玉不觉滴下泪来。老姑子献了茶,宝玉因和他借香炉烧香。那姑子去了半日,连
香供纸马都预备了来。宝玉说道:“一概不用。”命焙茗捧着炉出至后园中,拣一块
干净地方儿,竟拣不出。焙茗道:“那井台上如何?”宝玉点头。

一齐来至井台上,将炉放下,焙茗站过一旁。宝玉掏出香来焚上,含泪施了半
礼,回身命收了去。焙茗答应,且不收,忙爬下磕了几个头,口内祝道:“我焙茗
跟二爷这几年,二爷的心事我没有不知道的,只有今儿这一祭祀,没有告诉我,我
也不敢问。只是受祭的阴魂,虽不知名姓,想来自然是那人间有一、天上无双,极
聪明清雅的一位姐姐妹妹了。二爷的心事难出口,我替二爷祝赞你:你若有灵有圣,
我们二爷这样想着你,你也时常来望候望候二爷,未尝不可。你在阴间,保佑二爷
来生也变个女孩儿,和你们一处玩耍,岂不两下里都有趣了。”说毕又磕了几个头,
才爬起来。

宝玉听他没说完,便掌不住笑了。因踢他道:“别胡说,看人听见笑话。”焙茗
起来,收过香炉,和宝玉走着,因道:“我已经合姑子说了二爷还没用饭,叫他收
拾了些东西,二爷勉强吃些。我知道今儿里头大排筵宴,热闹非常,二爷为此才躲
了来的。横竖在这里清净一天,也就尽乐了;要不吃东西,断使不得。”宝玉道:“戏
酒不吃,这随便的吃些也不妨。”焙茗道:“这才是。还有一说:咱们来了,必有人
不放心。若没有人不放心,便晚些进城何妨?若有人不放心,二爷须得进城回家去
才是。第一老太太、太太也放了心,第二礼也尽了,不过这么着。就是家去听戏喝
酒,也并不是爷有意,原是陪着父母尽个孝道儿。要单为这个,不顾老太太、太太
悬心,就是才受祭的阴魂儿也不安哪。二爷想我这话怎么样?”宝玉笑道:“你的
意思我猜着了。你想着只你一个跟了我出来,回来你怕担不是,所以拿这大题目来
劝我。我才来了,不过为尽个礼,再去吃酒看戏,并没说一日不进城。这已经完了
心愿,赶着进城,大家放心就是了。”焙茗道:“这更好。”

说着二人来至禅堂,果然那姑子收拾了一桌好素菜。宝玉胡乱吃了些,焙茗也
吃了。二人便上马,仍回旧路。焙茗在后面,只嘱咐:“二爷好生骑着。这马总没
大骑,手提紧着些儿。”一面说着,早已进了城,仍从后门进去,忙忙来至怡红院
中。袭人等都不在屋里,只有几个老婆子看屋子,见他来了,都喜的眉开眼笑道:
“阿弥陀佛,可来了!没把花姑娘急疯了呢。上头正坐席呢,二爷快去罢。”宝玉听
说,忙将素衣脱了,自己找了颜色吉服换上,便问道:“都在什么地方坐席呢?”
老婆子们回道:“在新盖的大花厅上呢。”

宝玉听了,一径往花厅上来,耳内早隐隐闻得箫管歌吹之声。刚到穿堂那边,
只见玉钏儿独坐在廊檐下垂泪,一见宝玉来了,便长出了一口气,砸着嘴儿说道:
“嗳!凤凰来了,快进去罢!再一会子不来,可就都反了。”宝玉陪笑道:“你猜我往
那里去了?”玉钏儿把身一扭,也不理他,只管拭泪,宝玉只得怏怏的进去了。到
了花厅上,见了贾母、王夫人等,众人真如得了“凤凰”一般。贾母先问道:“你
往那里去了,这早晚才来?还不给你姐姐行礼去呢!”因笑着又向凤姐儿道:“你兄
弟不知好歹,就有要紧的事,怎么也不说一声儿就私自跑了,这还了得!明儿再这
样,等你老子回家,必告诉他打你。”凤姐儿笑着道:“行礼倒是小事,宝兄弟明儿
断不可不言语一声儿,也不传人跟着就出去。街上车马多,头一件叫人不放心。再
也不像咱们这样人家出门的规矩。”这里贾母又骂跟的人:“为什么都听他的话,说
往那里去就去了,也不回一声儿!”一面又问:“他到底往那里去了?可吃了什么没
有?唬着了没有?”宝玉只回说:“北静王的一个爱妾没了,今日给他道恼去。我见
他哭的那样,不好撇下他就回来,所以多等了会子。”贾母道:“以后再私自出门,
不先告诉我,一定叫你老子打你!”宝玉连忙答应着。贾母又要打跟的人。众人又
劝道:“老太太也不必生气了,他已经答应不敢了,况且回来又没事,大家该放心
乐一会子了。”贾母先不放心,自然着急发狠;今见宝玉回来,喜且有余,那里还
恨?也就不提了。还怕他不受用,或者别处没吃饭,路上着了惊恐,反又百般的哄
他。袭人早已过来伏侍,大家仍旧听戏。

当日演的是《荆钗记》,贾母薛姨妈等都看的心酸落泪,也有笑的,也有恨的,
也有骂的。

要知端底,下回分解。