\chapter{村老老是信口开河~情哥哥偏寻根究底}

话说众人见平儿来了,都说:“你们奶奶做什么呢,怎么不来了?”平儿笑道:
“他那里得空儿来?因为说没得好生吃,又不得来,所以叫我来问还有没有,叫我
再要几个拿了家去吃罢。”湘云道:“有,多着呢!”忙命人拿盒子装了十个极大的。
平儿道:“多拿几个团脐的。”众人又拉平儿坐,平儿不肯,李纨瞅着他笑道:“偏
叫你坐!”因拉他身旁坐下,端了一杯酒,送到他嘴边。平儿忙喝了一口,就要走,
李纨道:“偏不许你去!显见得你只有凤丫头,就不听我的话了。”说着,又命嬷嬷
们:“先送了盒子去,就说我留下平儿了。”那婆子一时拿了盒子回来,说:“二奶
奶说:‘叫奶奶和姑娘们别笑话要嘴吃。这个盒子里,方才舅太太那里送来的菱粉
糕和鸡油卷儿,给奶奶姑娘们吃的。’”又向平儿道:“说了:‘使唤你来,你就贪住
嘴不去了,叫你少喝钟儿罢。’”平儿笑道:“多喝了,又把我怎么样?”一面说,
一面只管喝,又吃螃蟹。李纨揽着他笑道:“可惜这么个好体面模样儿,命却平常,
只落得屋里使唤。不知道的人,谁不拿你当做奶奶太太看?”平儿一面和宝钗湘云
等吃喝着,一面回头笑道:“奶奶,别这么摸的我怪痒痒的。”李氏道:“嗳哟!这硬
的是什么?”平儿道:“是钥匙。”李氏道:“有什么要紧的东西怕人偷了去,这么
带在身上?我成日家和人说:有个唐僧取经,就有个白马来驮着他;刘智远打天下,
就有个瓜精来送盔甲;有个凤丫头,就有个你。你就是你奶奶的一把总钥匙,还要
这钥匙做什么?”平儿笑道:“奶奶吃了酒,又拿我来打趣着取笑儿了。”

宝钗笑道:“这倒是真话。我们没事评论起来,你们这几个,都是百个里头挑
不出一个来的。妙在各人有各人的好处。”李纨道:“大小都有个天理:比如老太太
屋里,要没鸳鸯姑娘,如何使得?从太太起,那一个敢驳老太太的回?他现敢驳回,
偏老太太只听他一个人的话。老太太的那些穿带的,别人不记得,他都记得。要不
是他经管着,不知叫人诳骗了多少去呢!况且他心也公道,虽然这样,倒常替人上
好话儿,还倒不倚势欺人的。”惜春笑道:“老太太昨日还说呢,他比我们还强呢!”
平儿道:“那原是个好的,我们那里比得上他?”宝玉道:“太太屋里的彩霞,是个
老实人。”探春道:“可不是‘老实’!心里可有数儿呢。太太是那么佛爷似的,事
情上不留心,他都知道。凡一应事,都是他提着太太行,连老爷在家出外去的一应
大小事,他都知道,太太忘了,他背后告诉太太。”李纨道:“那也罢了。”指着宝
玉道:“这一个小爷屋里,要不是袭人,你们度量到个什么田地?凤丫头就是个楚霸
王,也得两只膀子好举千斤鼎,他不是这丫头,他就得这么周到了?”平儿道:“先
时赔了四个丫头来,死的死,去的去,只剩下我一个孤鬼儿了。”李纨道:“你倒是
有造化的,凤丫头也是有造化的。想当初你大爷在日,何曾也没两个人?你们看,
我还是那容不下人的?天天只是他们不如意,所以你大爷一没了,我趁着年轻都打
发了。要是有一个好的守的住,我到底也有个膀臂了。”说着不觉眼圈儿红了。

众人都道:“这又何必伤心,不如散了倒好。”说着,便都洗了手,大家约着往
贾母王夫人处问安。众婆子丫头打扫亭子,收洗杯盘。袭人便和平儿一同往前去。
袭人因让平儿到屋里坐坐,再喝碗茶去。平儿回说:“不喝茶了,再来罢。”一面说,
一面便要出去。袭人又叫住,问道:“这个月的月钱,连老太太、太太屋里还没放,
是为什么?”平儿见问,忙转身至袭人跟前,又见无人,悄悄说道:“你快别问!横
竖再迟两天就放了。”袭人笑道:“这是为什么,唬的你这个样儿?”平儿悄声告诉
他道:“这个月的月钱,我们奶奶早已支了,放给人使呢。等别处利钱收了来,凑
齐了才放呢。因为是你,我才告诉你,可不许告诉一个人去!”袭人笑道:“他难道
还短钱使?还没个足厌?何苦还操这心?”平儿笑道:“何曾不是呢。他这几年,只
拿着这一项银子翻出有几百来了。他的公费月例又使不着,十两八两零碎攒了,又
放出去,单他这体己利钱,一年不到,上千的银子呢。”袭人笑道:“拿着我们的钱,
你们主子奴才赚利钱,哄的我们呆等着!”平儿道:“你又说没良心的话,你难道还
少钱?”袭人道:“我虽不少,只是我也没处儿使去,就只预备我们那一个。”平儿
道:“你倘若有紧要事用银钱使时,我那里还有几两银子,你先拿来使,明日我扣
下你的就是了。”袭人道:“此时也用不着。怕一时要用起来不够了,我打发人去取
就是了。”

平儿答应着,一径出了园门,只见凤姐那边打发人来找平儿,说:“奶奶有事
等你。”平儿道:“有什么事这么要紧?我叫大奶奶拉扯住说话儿,我又没逃了,这
么连三接四的叫人来找!”那丫头说道:“这又不是我的主意,姑娘这话自己和奶奶
说去。”平儿啐道:“好了,你们越发上脸了!”说着走来。只见凤姐儿不在屋里,
忽见上回来打抽丰的刘老老和板儿来了,坐在那边屋里,还有张材家的周瑞家的陪
着。又有两三个丫头在地下,倒口袋里的枣儿、倭瓜并些野菜。众人见他进来,都
忙站起来。刘老老因上次来过,知道平儿的身分,忙跳下地来,问:“姑娘好?”
又说:“家里都问好。早要来请姑奶奶的安、看姑娘来的,因为庄家忙,好容易今
年多打了两石粮食,瓜果菜蔬也丰盛,这是头一起摘下来的,并没敢卖呢,留的尖
儿,孝敬姑奶奶、姑娘们尝尝。姑娘们天天山珍海味的,也吃腻了,吃个野菜儿,
也算我们的穷心。”平儿忙道:“多谢费心。”又让坐,自己坐了,又让:“张婶子周
大娘坐了。”命小丫头子:“倒茶去。”周瑞张材两家的因笑道:“姑娘今日脸上有些
春色,眼圈儿都红了。”平儿笑道:“可不是,我原不喝,大奶奶和姑娘们只是拉着
死灌,不得已喝了两钟,脸就红了。”张材家的笑道:“我倒想着要喝呢,又没人让
我。明日再有人请姑娘,可带了我去罢。”说着,大家都笑了。周瑞家的道:“早起
我就看见那螃蟹了,一斤只好秤两个三个,这么两三大篓,想是有七八十斤呢。”
周瑞家的又道:“要是上上下下,只怕还不够!”平儿道:“那里都吃?不过都是有名
儿的吃两个子。那些散众儿的,也有摸着的,也有摸不着的。”刘老老道:“这些螃
蟹,今年就值五分一斤,十斤五钱,五五二两五,三五一十五,再搭上酒菜,一共
倒有二十多两银子。阿弥陀佛!这一顿的银子,够我们庄家人过一年了!”

平儿因问:“想是见过奶奶了?”刘老老道:“见过了,叫我们等着呢。”说着,
又往窗外看天气,说道:“天好早晚了,我们也去罢,别出不去城才是饥荒呢。”周
瑞家的道:“等着我替你瞧瞧去。”说着,一径去了,半日方来,笑道:“可是老老
的福来了,竟投了这两个人的缘了。”平儿等问:“怎么样?”周瑞家的笑道:“二
奶奶在老太太跟前呢,我原是悄悄的告诉二奶奶:‘刘老老要家去呢,怕晚了赶不
出城去。’二奶奶说:‘大远的,难为他扛了些东西来,晚了就住一夜,明日再去。’
这可不是投上二奶奶的缘了吗?这也罢了,偏老太太又听见了,问:‘刘老老是谁?’
二奶奶就回明白了。老太太又说:‘我正想个积古的老人家说话儿,请了来我见见。’
这可不是想不到的投上缘了?”说着,催刘老老下来前去。刘老老道:“我这生像
儿,怎么见得呢?好嫂子,你就说我去了罢!”平儿忙道:“你快去罢,不相干的。
我们老太太最是惜老怜贫的,比不得那个狂三诈四的那些人。想是你怯上,我和周
大娘送你去。”说着,同周瑞家的带了刘老老往贾母这边来。二门口该班的小厮们,
见了平儿出来都站起来,有两个又跑上来,赶着平儿叫“姑娘”。平儿问道:“又说
什么?”那小厮笑道:“这会子也好早晚了,我妈病着,等我去请大夫。好姑娘,
我讨半日假,可使得?”平儿道:“你们倒好,都商量定了,一天一个,告假又不
回奶奶,只和我胡缠。前日住儿去了,二爷偏叫他,叫不着,我应起来了,还说我
做了情了。你今日又来了。”周瑞家的道:“当真的他妈病了,姑娘也替他应着放了
他罢。”平儿道:“明日一早来。听着,我还要使你呢。再睡的日头晒着屁股再来!
你这一去,带个信儿给旺儿,就说奶奶的话,问他那剩的利钱,明日要还不交来,
奶奶不要了,索性送他使罢。”那小厮欢天喜地,答应去了。

平儿等来至贾母房中。彼时大观园中姐妹们都在贾母前承奉,刘老老进去,只
见满屋里珠围翠绕、花枝招展的,并不知都系何人。只见一张榻上,独歪着一位老
婆婆,身后坐着一个纱罗裹的美人一般的个丫鬟在那里捶腿,凤姐儿站着正说笑。
刘老老便知是贾母了,忙上来,陪着笑,拜了几拜,口里说:“请老寿星安!”贾母
也忙欠身问好,又命周瑞家的端过椅子来坐着。那板儿仍是怯人,不知问候。贾母
道:“老亲家,你今年多大年纪了?”刘老老忙起身答道:“我今年七十五了。”贾
母向众人道:“这么大年纪了,还这么硬朗。比我大好几岁呢!我要到这个年纪,还
不知怎么动不得呢。”刘老老笑道:“我们生来是受苦的人,老太太生来是享福的。
我们要也这么着,那些庄家活也没人做了。”贾母道:“眼睛牙齿还好?”刘老老道:
“还都好,就是今年左边的槽牙活动了。”贾母道:“我老了,都不中用了,眼也花,
耳也聋,记性也没了。你们这些老亲戚,我都不记得了。亲戚们来了,我怕人笑话,
我都不会。不过嚼的动的吃两口,睡一觉,闷了时和这些孙子孙女儿玩笑会子就完
了。”刘老老笑道:“这正是老太太的福了。我们想这么着不能。”贾母道:“什么福,
不过是老废物罢咧!”说的大家都笑了。贾母又笑道:“我才听见凤哥儿说,你带了
好些瓜菜来,我叫他快收拾去了。我正想个地里现结的瓜儿菜儿吃,外头买的不像
你们地里的好吃。”刘老老笑道:“这是野意儿,不过吃个新鲜。依我们倒想鱼肉吃,
只是吃不起。”贾母又道:“今日既认着了亲,别空空的就去,不嫌我这里,就住一
两天再去。我们也有个园子,园子里头也有果子。你明日也尝尝,带些家去,也算
是看亲戚一趟。”凤姐儿见贾母喜欢,也忙留道:“我们这里虽不比你们的场院大,
空屋子还有两间,你住两天,把你们那里的新闻故事儿,说些给我们老太太听听。”
贾母笑道:“凤丫头别拿他取笑儿,他是屯里人,老实,那里搁的住你打趣?”说
着,又命人去先抓果子给板儿吃。板儿见人多了,又不敢吃。贾母又命拿些钱给他,
叫小么儿们带他外头玩去。刘老老吃了茶,便把些乡村中所见所闻的事情说给贾母
听,贾母越发得了趣味。正说着,凤姐儿便命人请刘老老吃晚饭,贾母又将自己的
菜拣了几样,命人送过去给刘老老吃。

凤姐知道合了贾母的心,吃了饭便又打发过来。鸳鸯忙命老婆子带了刘老老去
洗了澡,自己去挑了两件随常的衣裳叫给刘老老换上。那刘老老那里见过这般行事?
忙换了衣裳出来,坐在贾母榻前,又搜寻些话出来说。彼时宝玉姐妹们也都在这里
坐着,他们何曾听见过这些话,自觉比那些瞽目先生说的书还好听。那刘老老虽是
个村野人,却生来的有些见识,况且年纪老了,世情上经历过的,见头一件贾母高
兴,第二件这些哥儿姐儿都爱听,便没话也编出些话来讲。因说道:“我们村庄上
种地种菜,每年每日,春夏秋冬,风里雨里,那里有个坐着的空儿?天天都是在那
地头上做歇马凉亭,什么奇奇怪怪的事不见呢!就像旧年冬天,接连下了几天雪,
地下压了三四尺深。我那日起的早,还没出屋门,只听外头柴草响,我想着必定有
人偷柴草来了。我巴着窗户眼儿一瞧,不是我们村庄上的人——”贾母道:“必定
是过路的客人们冷了,见现成的柴火抽些烤火,也是有的。”刘老老笑道:“也并不
是客人,所以说来奇怪。老寿星打量什么人?原来是一个十七八岁极标致的个小姑
娘儿,梳着溜油儿光的头,穿着大红袄儿,白绫子裙儿。”刚说到这里,忽听外面
人吵嚷起来,又说:“不相干,别唬着老太太!”贾母等听了,忙问:“怎么了?”
丫鬟回说:“南院子马棚里走了水了,不相干,已经救下去了。”贾母最胆小的,听
了这话,忙起身扶了人出至廊上来瞧时,只见那东南角上火光犹亮。贾母唬得口内
念佛,又忙命人去火神跟前烧香。王夫人等也忙都过来请安,回说:“已经救下去
了。老太太请进去罢。”贾母足足的看着火光熄了,方领众人进来。

宝玉且忙问刘老老:“那女孩儿大雪地里做什么抽柴火?倘或冻出病来呢?”贾
母道:“都是才说抽柴火,惹出事来了,你还问呢!别说这个了,说别的罢。”宝玉
听说,心内虽不乐,也只得罢了。刘老老便又想了想,说道:“我们庄子东边庄上
有个老奶奶子,今年九十多岁了。他天天吃斋念佛,谁知就感动了观音菩萨,夜里
来托梦,说:‘你这么虔心,原本你该绝后的,如今奏了玉皇,给你个孙子。’原来
这老奶奶只有一个儿子,这儿子也只一个儿子,好容易养到十七八岁上,死了,哭
的什么儿似的。后起间,真又养了一个,今年才十三四岁,长得粉团儿似的,聪明
伶俐的了不得呢。这些神佛是有的不是!”这一席话暗合了贾母王夫人的心事,连
王夫人也都听住了。

宝玉心中只惦记抽柴的事,因闷的心中筹画。探春因问他:“昨日扰了史大妹
妹,咱们回去商议着邀一社,又还了席,也请老太太赏菊何如?”宝玉笑道:“老
太太说了,还要摆酒还史妹妹的席,叫咱们做陪呢。等吃了老太太的,咱们再请不
迟。”探春道:“越往前越冷了,老太太未必高兴。”宝玉道:“老太太又喜欢下雨下
雪的,咱们等下头场雪,请老太太赏雪不好吗?咱们雪下吟诗,也更有趣了。”黛玉
笑道:“咱们雪下吟诗,依我说,还不如弄一捆柴火,雪下抽柴,还更有趣儿呢!”
说着,宝钗等都笑了。宝玉瞅了他一眼,也不答话。

一时散了,背地里宝玉到底拉了刘老老,细问那女孩儿是谁。刘老老只得编了
告诉他:“那原是我们庄子北沿儿地埂子上,有个小祠堂儿,供的不是神佛,当先
有个什么老爷——”说着,又想名姓。宝玉道:“不拘什么名姓,也不必想了,只
说原故就是了。”刘老老道:“这老爷没有儿子,只有一位小姐,名字叫什么若玉,
知书儿识字的,老爷太太爱的像珍珠儿。可惜了儿的,这小姐儿长到十七岁了,一
病就病死了。”宝玉听了,跌足叹惜,又问:“后来怎么样?”刘老老道:“因为老
爷太太疼的心肝儿似的,盖了那祠堂,塑了个像儿,派了人烧香儿拨火的。如今年
深日久了,人也没了,庙也烂了,那泥胎儿可就成了精咧。”宝玉忙道:“不是成精,
规矩这样人是不死的。”刘老老道:“阿弥陀佛!是这么着吗?不是哥儿说,我们还当
他成了精了呢。他时常变了人出来闲逛。我才说抽柴火的,就是他了。我们村庄上
的人商量着还要拿榔头砸他呢。”宝玉忙道:“快别如此。要平了庙,罪过不小!”
刘老老道:“幸亏哥儿告诉我,明日回去,拦住他们就是了。”宝玉道:“我们老太
太、太太都是善人,就是合家大小也都好善喜舍,最爱修庙塑神的。我明日做一个
疏头,替你化些布施,你就做香头,攒了钱,把这庙修盖,再装塑了泥像,每月给
你香火钱烧香,好不好?”刘老老道:“若这样时,我托那小姐的福,也有几个钱
使了。”宝玉又问他地名庄名,来往远近,坐落何方,刘老老便顺口诌了出来。

宝玉信以为真,回至房中,盘算了一夜。次日一早,便出来给了焙茗几百钱,
按着刘老老说的方向地名,着焙茗去先踏看明白,回来再作主意。那焙茗去后,宝
玉左等也不来,右等也不来,急的热地里的蚰蜒似的。好容易等到日落,方见焙茗
兴兴头头的回来了。宝玉忙问:“可找着了?”焙茗笑道:“爷听的不明白,叫我好
找!那地名坐落,不像爷听的一样,所以找了一天,找到东北角田埂子上,才有一
个破庙。”宝玉听说,喜的眉开眼笑,忙说道:“刘老老有年纪的人,一时错记了也
是有的。你且说你见的。”焙茗道:“那庙门却倒也朝南开,也是稀破的。我找的正
没好气,一见这个,我说可好了,连忙进去。一看泥胎,唬的我又跑出来了,活像
真的似的!”宝玉喜的笑道:“他能变化人了,自然有些生气。”焙茗拍手道:“那里
是什么女孩儿?竟是一位青脸红发的瘟神爷!”宝玉听了,啐了一口,骂道:“真是
个没用的杀材,这点子事也干不来!”焙茗道:“爷又不知看了什么书,或者听了谁
的混帐语,信真了,把这件没头脑的事派我去碰头。怎么说我没用呢?”宝玉见他
急了,忙抚慰他道:“你别急,改日闲了,你再找去。要是他哄我们呢,自然没了;
要竟是有的,你岂不也积了阴骘呢?我必重重的赏你。”说着,只见二门上的小厮来
说:“老太太屋里的姑娘们站在二门口找二爷呢。”

不知何事,下回分解。