\chapter{听曲文宝玉悟禅机~制灯谜贾政悲谶语}

话说贾琏听凤姐儿说有话商量,因止步问:“什么话?”凤姐道:“二十一是
薛妹妹的生日,你到底怎么样?”贾琏道:“我知道怎么样?你连多少大生日都料
理过了,这会子倒没有主意了!”凤姐道:“大生日是有一定的则例。如今他这生
日,大又不是,小又不是,所以和你商量。”贾琏听了,低头想了半日,道:“你
竟糊涂了。现有比例,那林妹妹就是例。往年怎么给林妹妹做的,如今也照样给薛
妹妹做就是了。”凤姐听了冷笑道:“我难道这个也不知道!我也这么想来着。但
昨日听见老太太说,问起大家的年纪生日来,听见薛大妹妹今年十五岁,虽不算是
整生日,也算得将笄的年分儿了。老太太说要替他做生日,自然和往年给林妹妹做
的不同了。”贾琏道:“这么着,就比林妹妹的多增些。”凤姐道:“我也这么想
着,所以讨你的口气儿。我私自添了,你又怪我不回明白了你了。”贾琏笑道:“罢!
罢!这空头情我不领。你不盘察我就够了,我还怪你?”说着,一径去了,不在话
下。

且说湘云住了两日,便要回去,贾母因说:“等过了你宝姐姐的生日,看了戏,
再回去。”湘云听了,只得住下,又一面遣人回去,将自己旧日作的两件针线活计
取来,为宝钗生辰之仪。

谁想贾母自见宝钗来了,喜他稳重和平,正值他才过第一个生辰,便自己捐资
二十两,唤了凤姐来,交与他备酒戏。凤姐凑趣,笑道:“一个老祖宗,给孩子们
作生日,不拘怎么着,谁还敢争?又办什么酒席呢?既高兴,要热闹,就说不得自己
花费几两老库里的体己。这早晚找出这霉烂的二十两银子来做东,意思还叫我们赔
上!果然拿不出来也罢了,金的银的圆的扁的压塌了箱子底,只是累我们。老祖
宗看看,谁不是你老人家的儿女?难道将来只有宝兄弟顶你老人家上五台山不成?那
些东西只留给他!我们虽不配使,也别太苦了我们。这个够酒的够戏的呢?”说的
满屋里都笑起来。贾母亦笑道:“你们听听这嘴!我也算会说的了,怎么说不过这
猴儿?你婆婆也不敢强嘴,你就和我啊的!”凤姐笑道:“我婆婆也是一样的
疼宝玉,我也没处诉冤!倒说我强嘴!”说着,又引贾母笑了一会。贾母十分喜悦。
到晚上,众人都在贾母前,定省之馀,大家娘儿们说笑时,贾母因问宝钗爱听何戏,
爱吃何物。宝钗深知贾母年老之人,喜热闹戏文,爱吃甜烂之物,便总依贾母素喜
者说了一遍。贾母更加喜欢。次日,先送过衣服玩物去,王夫人、凤姐、黛玉等诸
人皆有随分的,不须细说。至二十一日,贾母内院搭了家常小巧戏台,定了一班新
出的小戏,昆弋两腔俱有。就在贾母上房摆了几席家宴酒席,并无一个外客,只有
薛姨妈、史湘云、宝钗是客,馀者皆是自己人。这日早起,宝玉因不见黛玉,便到
他房中来寻,只见黛玉歪在炕上。宝玉笑道:“起来吃饭去。就开戏了,你爱听那
一出?我好点。”黛玉冷笑道:“你既这么说,你就特叫一班戏,拣我爱的唱给我
听,这会子犯不上借着光儿问我。”宝玉笑道:“这有什么难的,明儿就叫一班子,
也叫他们借着咱们的光儿。”一面说,一面拉他起来,携手出去。

吃了饭,点戏时,贾母一面先叫宝钗点,宝钗推让一遍,无法,只得点了一出
《西游记》。贾母自是喜欢。又让薛姨妈,薛姨妈见宝钗点了,不肯再点。贾母便
特命凤姐点。凤姐虽有邢王二夫人在前,但因贾母之命,不敢违拗,且知贾母喜热
闹更喜谑笑科诨,便先点了一出,却是《刘二当衣》。贾母果真更又喜欢。然后便
命黛玉点,黛玉又让王夫人等先点。贾母道:“今儿原是我特带着你们取乐,咱们
只管咱们的,别理他们。我巴巴儿的唱戏摆酒,为他们呢?他们白听戏白吃已经便
宜了,还让他们点戏呢!”说着,大家都笑。黛玉方点了一出。然后宝玉、史湘云、
迎、探、惜、李纨等俱各点了,按出扮演。

至上酒席时,贾母又命宝钗点,宝钗点了一出《山门》。宝玉道:“你只好点
这些戏。”宝钗道:“你白听了这几年戏,那里知道这出戏,排场词藻都好呢。”
宝玉道:“我从来怕这些热闹戏。”宝钗笑道:“要说这一出‘热闹’,你更不知
戏了。你过来,我告诉你,这一出戏是一套《北点绛唇》,铿锵顿挫,那音律不用
说是好了,那词藻中有只《寄生草》,极妙,你何曾知道!”宝玉见说的这般好,
便凑近来央告:“好姐姐,念给我听听。”宝钗便念给他听道:

漫英雄泪,相离处士家。谢慈悲剃度在莲台下。没缘法转眼分离乍。赤条条
来去无牵挂。那里讨烟蓑雨笠卷单行?一任俺芒鞋破钵随缘化!
宝玉听了,喜的拍膝摇头,称赏不已;又赞宝钗无书不知。黛玉把嘴一撇道:“安
静些看戏吧!还没唱《山门》,你就《妆疯》了。”说的湘云也笑了。于是大家看
戏,到晚方散。

贾母深爱那做小旦的和那做小丑的,因命人带进来,细看时,益发可怜见的。
因问他年纪,那小旦才十一岁,小丑才九岁,大家叹息了一回。贾母令人另拿些肉
果给他两个,又另赏钱。凤姐笑道:“这个孩子扮上活像一个人,你们再瞧不出来。”
宝钗心内也知道,却点头不说;宝玉也点了点头儿不敢说。湘云便接口道:“我知
道,是像林姐姐的模样儿。”宝玉听了,忙把湘云瞅了一眼。众人听了这话,留神
细看,都笑起来了,说:“果然像他!”一时散了。

晚间,湘云便命翠缕把衣包收拾了。翠缕道:“忙什么?等去的时候包也不迟。”
湘云道:“明早就走,还在这里做什么?——看人家的脸子!”宝玉听了这话,忙
近前说道:“好妹妹,你错怪了我。林妹妹是个多心的人。别人分明知道,不肯说
出来,也皆因怕他恼。谁知你不防头就说出来了,他岂不恼呢?我怕你得罪了人,
所以才使眼色。你这会子恼了我,岂不辜负了我?要是别人,那怕他得罪了人,与
我何干呢?”湘云摔手道:“你那花言巧语别望着我说。我原不及你林妹妹。别人
拿他取笑儿都使得,我说了就有不是。我本也不配和他说话:他是主子姑娘,我是
奴才丫头么。”宝玉急的说道:“我倒是为你为出不是来了。我要有坏心,立刻化
成灰,教万人拿脚踹!”湘云道:“大正月里,少信着嘴胡说这些没要紧的歪话!
你要说,你说给那些小性儿、行动爱恼人、会辖治你的人听去,别叫我啐你。”说
着,进贾母里间屋里,气忿忿的躺着去了。

宝玉没趣,只得又来找黛玉。谁知才进门,便被黛玉推出来了,将门关上。宝
玉又不解何故,在窗外只是低声叫好妹妹好妹妹,黛玉总不理他。宝玉闷闷的垂头
不语。紫鹃却知端底,当此时料不能劝。那宝玉只呆呆的站着。黛玉只当他回去了,
却开了门,只见宝玉还站在那里。黛玉不好再闭门,宝玉因跟进来,问道:“凡事
都有个原故,说出来人也不委屈。好好的就恼,到底为什么起呢?”黛玉冷笑道:
“问我呢!我也不知为什么。我原是给你们取笑儿的,——拿着我比戏子,给众人
取笑儿!”宝玉道:“我并没有比你,也并没有笑你,为什么恼我呢?”黛玉道:
“你还要比,你还要笑?你不比不笑,比人家比了笑了的还利害呢!”宝玉听说,
无可分辩。黛玉又道:“这还可恕。你为什么又和云儿使眼色儿?这安的是什么心?
莫不是他和我玩,他就自轻自贱了?他是公侯的小姐,我原是民间的丫头。他和我
玩,设如我回了口,那不是他自惹轻贱?你是这个主意不是?你却也是好心,只是那
一个不领你的情,一般也恼了。你又拿我作情,倒说我‘小性儿、行动肯恼人’。
你又怕他得罪了我,——我恼他与你何干,他得罪了我又与你何干呢?”

宝玉听了,方知才和湘云私谈,他也听见了。细想自己原为怕他二人恼了,故
在中间调停,不料自己反落了两处的数落,正合着前日所看《南华经》内“巧者劳
而智者忧,无能者无所求,蔬食而遨游,泛若不系之舟”,又曰“山木自寇,源泉
自盗”等句,因此越想越无趣。再细想来:“如今不过这几个人,尚不能应酬妥协,
将来犹欲何为?”想到其间,也不分辩,自己转身回房。黛玉见他去了,便知回思
无趣,赌气去的,一言也不发,不禁自己越添了气,便说:“这一去,一辈子也别
来了,也别说话!”那宝玉不理,竟回来,躺在床上,只是闷闷的。袭人虽深知原
委,不敢就说,只得以别事来解说,因笑道:“今儿听了戏,又勾出几天戏来。宝
姑娘一定要还席的。”宝玉冷笑道:“他还不还,与我什么相干?”袭人见这话不
似往日,因又笑道:“这是怎么说呢?好好儿的大正月里,娘儿们姐儿们都喜喜欢
欢的,你又怎么这个样儿了?”宝玉冷笑道:“他们娘儿们姐儿们喜欢不喜欢,也
与我无干。”袭人笑道:“大家随和儿,你也随点和儿不好?”宝玉道:“什么‘大
家彼此’?他们有‘大家彼此’,我只是赤条条无牵挂的!”说到这句,不觉泪下。
袭人见这景况,不敢再说。宝玉细想这一句意味,不禁大哭起来。翻身站起来,至
案边,提笔立占一偈云:

你证我证,心证意证。是无有证,斯可云证。无可云证,是立足境。
写毕,自己虽解悟,又恐人看了不解,因又填一只《寄生草》,写在偈后。又念了
一遍,自觉心中无有挂碍,便上床睡了。

谁知黛玉见宝玉此番果断而去,假以寻袭人为由,来看动静。袭人回道:“已
经睡了。”黛玉听了,就欲回去,袭人笑道:“姑娘请站着,有一个字帖儿,瞧瞧
写的是什么话。”便将宝玉方才所写的拿给黛玉看。黛玉看了,知是宝玉为一时感
忿而作,不觉又可笑又可叹。便向袭人道:“作的是个玩意儿,无甚关系的。”说
毕,便拿了回房去。

次日,和宝钗湘云同看。宝钗念其词曰:

无我原非你,从他不解伊。肆行无碍凭来去。茫茫着甚悲愁喜,纷纷说甚亲疏
密。从前碌碌却因何?到如今回头试想真无趣!
看毕,又看那偈语,因笑道:“这是我的不是了。我昨儿一支曲子,把他这个话惹
出来。这些道书机锋,最能移性的,明儿认真说起这些疯话,存了这个念头,岂不
是从我这支曲子起的呢?我成了个罪魁了!”说着,便撕了个粉碎,递给丫头们,
叫快烧了。黛玉笑道:“不该撕了,等我问他,你们跟我来,包管叫他收了这个痴
心。”

三人说着,过来见了宝玉。黛玉先笑道:“宝玉,我问你:至贵者宝,至坚者
玉。尔有何贵?尔有何坚?”宝玉竟不能答。二人笑道:“这样愚钝,还参禅呢!”
湘云也拍手笑道:“宝哥哥可输了。”黛玉又道:“你道‘无可云证,是立足境’,
固然好了,只是据我看来,还未尽善。我还续两句云:‘无立足境,方是干净。’”
宝钗道:“实在这方悟彻。当日南宗六祖惠能初寻师至韶州,闻五祖弘忍在黄梅,
他便充作火头僧。五祖欲求法嗣,令诸僧各出一偈,上座神秀说道:‘身是菩提树,
心如明镜台。时时勤拂拭,莫使有尘埃。’惠能在厨房舂米,听了道:‘美则美矣,
了则未了。’因自念一偈曰:‘菩提本非树,明镜亦非台。本来无一物,何处染尘
埃?’五祖便将衣钵传给了他。今儿这偈语亦同此意了。只是方才这句机锋,尚未
完全了结,这便丢开手不成?”黛玉笑道:“他不能答就算输了,这会子答上了也
不为出奇了。只是以后再不许谈禅了。连我们两个人所知所能的,你还不知不能呢,
还去参什么禅呢!”宝玉自己以为觉悟,不想忽被黛玉一问,便不能答;宝钗又比
出语录来,此皆素不见他们所能的。自己想了一想:“原来他们比我的知觉在先,
尚未解悟,我如今何必自寻苦恼。”想毕,便笑道:“谁又参禅,不过是一时的玩
话儿罢了。”说罢,四人仍复如旧。

忽然人报娘娘差人送出一个灯谜来,命他们大家去猜,猜后每人也作一个送进
去。四人听说,忙出来至贾母上房,只见一个小太监,拿了一盏四角平头白纱灯,
专为灯谜而制,上面已有了一个,众人都争看乱猜。小太监又下谕道:“众小姐猜
着,不要说出来,每人只暗暗的写了,一齐封送进去,候娘娘自验是否。”宝钗听
了,近前一看,是一首七言绝句,并无新奇,口中少不得称赞,只说“难猜”,故
意寻思。其实一见早猜着了。宝玉、黛玉、湘云、探春四个人也都解了,各自暗暗
的写了。一并将贾环贾兰等传来,一齐各揣心机猜了,写在纸上,然后各人拈一物
作成一谜,恭楷写了,挂于灯上。

太监去了,至晚出来,传谕道:“前日娘娘所制,俱已猜着,惟二小姐与三爷
猜的不是。小姐们作的也都猜了,不知是否?”说着,也将写的拿出来,也有猜着
的,也有猜不着的。太监又将颁赐之物送与猜着之人,每人一个宫制诗筒,一柄茶
筅,独迎春贾环二人未得。迎春自以为玩笑小事,并不介意;贾环便觉得没趣。且
又听太监说:“三爷所作这个不通,娘娘也没猜,叫我带回问三爷是个什么。”众
人听了,都来看他作的是什么,——写道:
大哥有角只八个,二哥有角只两根。
大哥只在床上坐,二哥爱在房上蹲。
众人看了,大发一笑。贾环只得告诉太监说:“是一个枕头,一个兽头。”太监记
了,领茶而去。

贾母见元春这般有兴,自己一发喜乐,便命速作一架小巧精致围屏灯来,设于
堂屋,命他姊妹们各自暗暗的做了,写出来粘在屏上;然后预备下香茶细果以及各
色玩物,为猜着之贺。贾政朝罢,见贾母高兴,况在节间,晚上也来承欢取乐。上
面贾母、贾政、宝玉一席;王夫人、宝钗、黛玉、湘云又一席,迎春、探春、惜春
三人又一席,俱在下面。地下老婆丫鬟站满。李宫裁王熙凤二人在里间又一席。贾
政因不见贾兰,便问:“怎么不见兰哥儿?”地下女人们忙进里间问李氏,李氏起
身笑着回道:“他说方才老爷并没叫他去,他不肯来。”女人们回复了贾政,众人
都笑说:“天生的牛心拐孤!”贾政忙遣贾环和个女人将贾兰唤来,贾母命他在身
边坐了,抓果子给他吃,大家说笑取乐。往常间只有宝玉长谈阔论,今日贾政在这
里,便唯唯而已。馀者,湘云虽系闺阁弱质,却素喜谈论,今日贾政在席,也自
口禁语;黛玉本性娇懒,不肯多话;宝钗原不妄言轻动,便此时亦是坦然自若:故
此一席,虽是家常取乐,反见拘束。

贾母亦知因贾政一人在此所致,酒过三巡,便撵贾政去歇息。贾政亦知贾母之
意,撵了他去好让他姊妹兄弟们取乐,因陪笑道:“今日原听见老太太这里大设春
灯雅谜,故也备了彩礼酒席,特来入会。何疼孙子孙女之心,便不略赐与儿子半
点?”贾母笑道:“你在这里,他们都不敢说笑,没的倒叫我闷的慌。你要猜谜儿,
我说一个你猜,猜不着是要罚的。”贾政忙笑道:“自然受罚。若猜着了,也要领
赏呢。”贾母道:“这个自然。”便念道:
猴子身轻站树梢。打一果名。
贾政已知是荔枝,故意乱猜,罚了许多东西,然后方猜着了,也得了贾母的东西。
然后也念一个灯谜与贾母猜。念道:
身自端方,体自坚硬。
虽不能言,有言必应。打一用物。
说毕,便悄悄的说与宝玉,宝玉会意,又悄悄的告诉了贾母。贾母想了一想,果然
不差,便说:“是砚台。”贾政笑道:“到底是老太太,一猜就是。”回头说:“快
把贺彩献上来。”地下妇女答应一声,大盘小盒,一齐捧上。贾母逐件看去,都是
灯节下所用所玩新巧之物,心中甚喜,遂命:“给你老爷斟酒。”宝玉执壶,迎春
送酒。贾母因说:“你瞧瞧那屏上,都是他姐儿们做的,再猜一猜我听。”

贾政答应,起身走至屏前,只见第一个是元妃的,写着道:
能使妖魔胆尽摧,身如束帛气如雷。
一声震得人方恐,回首相看已化灰。打一玩物。
贾政道:“这是爆竹吗?”宝玉答道:“是。”贾政又看迎春的,道:
天运人功理不穷,有功无运也难逢。
因何镇日纷纷乱?只为阴阳数不通。打一用物。
贾政道:“是算盘?”迎春笑道:“是。”又往下看,是探春的,道:
阶下儿童仰面时,清明妆点最堪宜。
游丝一断浑无力,莫向东风怨别离。打一玩物。
贾政道:“好像风筝。”探春道:“是。”贾政再往下看,是黛玉的,道:
朝罢谁携两袖烟?琴边衾里两无缘。
晓筹不用鸡人报,五夜无烦侍女添。
焦首朝朝还暮暮,煎心日日复年年。
光阴荏苒须当惜,风雨阴晴任变迁。打一用物。
贾政道:“这个莫非是更香?”宝玉代言道:“是。”贾政又看道:
南面而坐,北面而朝。
“象忧亦忧,象喜亦喜。”打一用物。
贾政道:“好,好!如猜镜子,妙极!”宝玉笑回道:“是。”贾政道:“这一个
却无名字,是谁做的?”贾母道:“这个大约是宝玉做的?”贾政就不言语。往下
再看宝钗的,道是:
有眼无珠腹内空,荷花出水喜相逢。
梧桐叶落分离别,恩爱夫妻不到冬。打一用物。
贾政看完,心内自忖道:“此物还倒有限,只是小小年纪,作此等言语,更觉不祥。
看来皆非福寿之辈。”想到此处,甚觉烦闷,大有悲戚之状,只是垂头沉思。贾母
见贾政如此光景,想到他身体劳乏,又恐拘束了他众姊妹,不得高兴玩耍,便对贾
政道:“你竟不必在这里了,歇着去罢。让我们再坐一会子,也就散了。”贾政一
闻此言,连忙答应几个“是”,又勉强劝了贾母一回酒,方才退出去了。回至房中,
只是思索,翻来覆去,甚觉凄惋。

这里贾母见贾政去了,便道:“你们乐一乐罢。”一语未了,只见宝玉跑至围
屏灯前,指手画脚,信口批评:“这个这一句不好。”“那个破的不恰当。”如同
开了锁的猴儿一般。黛玉便道:“还像方才大家坐着,说说笑笑,岂不斯文些儿?”
凤姐儿自里间屋里出来,插口说道:“你这个人,就该老爷每日合你寸步儿不离才
好。刚才我忘了,为什么不当着老爷,撺掇着叫你作诗谜儿?这会子不怕你不出汗
呢。”说的宝玉急了,扯着凤姐儿厮缠了一会。贾母又和李宫裁并众姊妹等说笑了
一会子,也觉有些困倦,听了听,已交四鼓了。因命将食物撤去,赏给众人,遂起
身道:“我们歇着罢。明日还是节呢,该当早些起来。明日晚上再玩罢。”于是众
人方慢慢的散去。

未知次日如何,且听下回分解。