\chapter{王凤姐弄权铁槛寺~秦鲸卿得趣馒头庵}

话说宝玉举目见北静王世荣头上戴着净白簪缨银翅王帽,穿着江牙海水五爪龙
白蟒袍,系着碧玉红带,面如美玉,目似明星,真好秀丽人物。宝玉忙抢上来参
见,世荣从轿内伸手搀住。见宝玉戴着束发银冠,勒着双龙出海抹额,穿着白蟒箭
袖,围着攒珠银带,面若春花,目如点漆。北静王笑道:“名不虚传,果然如‘宝’
似‘玉’。”问:“衔的那宝贝在那里?”宝玉见问,连忙从衣内取出,递与北静
王细细看了,又念了那上头的字,因问:“果灵验否?”贾政忙道:“虽如此说,
只是未曾试过。”北静王一面极口称奇,一面理顺彩绦,亲自与宝玉带上,又携手
问宝玉几岁,现读何书。宝玉一一答应。北静王见他语言清朗,谈吐有致,一面又
向贾政笑道:“令郎真乃龙驹凤雏,非小王在世翁前唐突,将来‘雏凤清于老凤声’,
未可量也。”贾政陪笑道:“犬子岂敢谬承金奖。赖藩郡馀恩,果如所言,亦荫生
辈之幸矣。”北静王又道:“只是一件:令郎如此资质,想老太夫人自然钟爱。但
吾辈后生,甚不宜溺爱,溺爱则未免荒失了学业。昔小王曾蹈此辙,想令郎亦未必
不如是也。若令郎在家难以用功,不妨常到寒邸,小王虽不才,却多蒙海内众名士
凡至都者,未有不垂青目的。是以寒邸高人颇聚,令郎常去谈谈会会,则学问可以
日进矣。”贾政忙躬身答道:“是。”北静王又将腕上一串念珠卸下来,递与宝玉
道:“今日初会,仓卒无敬贺之物,此系圣上所赐苓香念珠一串,权为贺敬之礼。”
宝玉连忙接了,回身奉与贾政。贾政带着宝玉谢过了。于是贾赦、贾珍等一齐上来,
叩请回舆。北静王道:“逝者已登仙界,非你我碌碌尘寰中人。小王虽上叨天恩,
虚邀郡袭,岂可越仙而进呢?”贾赦等见执意不从,只得谢恩回来,命手下人掩
乐停音,将殡过完,方让北静王过去。不在话下。

且说宁府送殡,一路热闹非常。刚至城门,又有贾赦、贾政、贾珍诸同寅属下
各家祭棚接祭,一一的谢过,然后出城,竟奔铁槛寺大路而来。彼时贾珍带着贾蓉
来到诸长辈前让坐轿上马,因而贾赦一辈的各自上了车轿,贾珍一辈的也将要上
马。凤姐因惦记着宝玉,怕他在郊外纵性不服家人的话,贾政管不着,惟恐有闪失,
因此命小厮来唤他。宝玉只得到他车前。凤姐笑道:“好兄弟,你是个尊贵人,和
女孩儿似的人品,别学他们猴在马上。下来,咱们姐儿两个同坐车好不好?”宝玉
听说,便下了马,爬上凤姐车内,二人说笑前进。

不一时,只见那边两骑马直奔凤姐车来,下马扶车回道:“这里有下处,奶奶
请歇歇更衣。”凤姐命请邢王二夫人示下,那二人回说:“太太们说不歇了,叫奶
奶自便。”凤姐便命歇歇再走。小厮带着轿马岔出人群,往北而来。宝玉忙命人去
请秦钟。那时秦钟正骑着马随他父亲的轿,忽见宝玉的小厮跑来请他去打尖。秦钟
远看着宝玉所骑的马,搭着鞍笼,随着凤姐的车往北而去,便知宝玉同凤姐一车,
自己也带马赶上来,同入一庄门内。

那庄农人家,无多房舍,妇女无处回避。那些村姑野妇见了凤姐、宝玉、秦钟
的人品衣服,几疑天人下降。凤姐进入茅屋,先命宝玉等出去玩玩。宝玉会意,因
同秦钟带了小厮们各处游玩。凡庄家动用之物,俱不曾见过的,宝玉见了,都以为
奇,不知何名何用。小厮中有知道的,一一告诉了名色并其用处。宝玉听了,因点
头道:“怪道古人诗上说:‘谁知盘中餐,粒粒皆辛苦!’正为此也。”一面说,
一面又到一间房内。见炕上有个纺车儿,越发以为稀奇。小厮们又说:“是纺线织
布的。”宝玉便上炕摇转。只见一个村妆丫头,约有十七八岁,走来说道:“别弄
坏了!”众小厮忙上来吆喝。宝玉也住了手,说道:“我因没有见过,所以试一试
玩儿。”那丫头道:“你不会转,等我转给你瞧。”秦钟暗拉宝玉道:“此卿大有
意趣。”宝玉推他道:“再胡说,我就打了!”说着,只见那丫头纺起线来,果然
好看。忽听那边老婆子叫道:“二丫头,快过来!”那丫头丢了纺车,一径去了。

宝玉怅然无趣。只见凤姐打发人来,叫他两个进去。凤姐洗了手,换了衣服,
问他换不换,宝玉道:“不换。”也就罢了。仆妇们端上茶食果品来,又倒上香茶
来,凤姐等吃了茶,待他们收拾完备,便起身上车。外面旺儿预备赏封赏了那庄户
人家,那妇人等忙来谢赏。宝玉留心看时,并不见纺线之女。走不多远,却见这二
丫头怀里抱着个小孩子,同着两个小女孩子,在村头站着瞅他。宝玉情不自禁,然
身在车上,只得眼角留情而已。一时电卷风驰,回头已无踪迹了。

说笑间,已赶上大殡。早又前面法鼓金铙,幢幡宝盖,铁槛寺中僧众摆列路旁。
少时到了寺中,另演佛事,重设香坛,安灵于内殿偏室之中,宝珠安理寝室为伴。
外面贾珍款待一应亲友,也有坐住的,也有告辞的,一一谢了乏;从公、侯、伯、
子、男,一起一起的散,至未末方散尽了。里面的堂客皆是凤姐接待,先从诰命散
起,也到未正上下方散完了。只有几个近亲本族,等做过三日道场方去的。那时邢
王二夫人知凤姐必不能回家,便要带了宝玉同进城去。那宝玉乍到郊外,那里肯回
去?只要跟着凤姐住着,王夫人只得交与凤姐而去。

原来这铁槛寺是宁荣二公当日修造的,现今还有香火地亩,以备京中老了人
口,在此停灵。其中阴阳两宅俱是预备妥贴的,好为送灵人口寄居。不想如今后人
繁盛,其中贫富不一,或性情参商。有那家道艰难的,便住在这里了,有那有钱有
势尚排场的,只说这里不方便,一定另外或村庄或尼庵寻个下处,为事毕宴退之所。
即今秦氏之丧,族中诸人,也有在铁槛寺的,也有别寻下处的。凤姐也嫌不方便,
因遣人来和馒头庵的姑子静虚说了,腾出几间房来预备。——原来这馒头庵和水月
寺一势,因他庙里做的馒头好,就起了这个浑号,离铁槛寺不远。当下和尚工课已
完,奠过晚茶,贾珍便命贾蓉请凤姐歇息。凤姐见还有几个妯娌们陪着女亲,自己
便辞了众人,带着宝玉秦钟往馒头庵来。只因秦邦业年迈多病,不能在此,只命秦
钟等待安灵罢,所以秦钟只跟着凤姐宝玉。一时到了庵中,静虚带领智善、智能两
个徒弟出来迎接,大家见过。凤姐等至净室更衣净手毕,因见智能儿越发长高了,
模样儿越发出息的水灵了,因说道:“你们师徒怎么这些日子也不往我们那里去?”
静虚道:“可是这几日因胡老爷府里产了公子,太太送了十两银子来这里,叫请几
位师父念三日《血盆经》,忙的就没得来请奶奶的安。”

不言老尼陪着凤姐。且说那秦钟宝玉二人正在殿上玩耍,因见智能儿过来,宝
玉笑道:“能儿来了。”秦钟说:“理他作什么?”宝玉笑道:“你别弄鬼儿!那
一日在老太太屋里,一个人没有,你搂着他作什么呢?这会子还哄我!”秦钟笑道:
“这可是没有的话。”宝玉道:“有没有也不管你,你只叫他倒碗茶来我喝,就撂
过手。”秦钟笑道:“这又奇了,你叫他倒去,还怕他不倒?何用我说呢!”宝玉
道:“我叫他倒的是无情意的,不及你叫他倒的是有情意的。”秦钟没法,只得说
道:“能儿倒碗茶来。”那能儿自幼在荣府走动,无人不识,常和宝玉秦钟玩笑,
如今长大了,渐知风月,便看上了秦钟人物风流,那秦钟也爱他妍媚,二人虽未上
手,却已情投意合了。智能走去倒了茶来。秦钟笑说:“给我。”宝玉又叫:“给
我。”智能儿抿着嘴儿笑道:“一碗茶也争,难道我手上有蜜!”宝玉先抢着了,
喝着,方要问话,只见智善来叫智能去摆果碟子,一时来请他两个去吃果茶。他两
个那里吃这些东西?略坐坐仍出来玩耍。

凤姐也便回至净室歇息,老尼相伴。此时众婆子媳妇见无事,都陆续散了自去
歇息,跟前不过几个心腹小丫头,老尼便趁机说道:“我有一事,要到府里求太太,
先请奶奶的示下。”凤姐问道:“什么事?”老尼道:“阿弥陀佛!只因当日我先
在长安县善才庵里出家的时候儿,有个施主姓张,是大财主。他的女孩儿小名金哥,
那年都往我庙里来进香,不想遇见长安府太爷的小舅子李少爷。那李少爷一眼看见
金哥就爱上了,立刻打发人来求亲,不想金哥已受了原任长安守备公子的聘定。张
家欲待退亲,又怕守备不依,因此说已有了人家了。谁知李少爷一定要娶,张家正
在没法,两处为难;不料守备家听见此信,也不问青红皂白,就来吵闹,说:‘一
个女孩儿你许几家子人家儿?’偏不许退定礼,就打起官司来。女家急了,只得着
人上京找门路,赌气偏要退定礼。我想如今长安节度云老爷,和府上相好,怎么求
太太和老爷说说,写一封书子,求云老爷和那守备说一声,不怕他不依。要是肯行,
张家那怕倾家孝顺,也是情愿的。”凤姐听了笑道:“这事倒不大。只是太太再不
管这些事。”老尼道:“太太不管,奶奶可以主张了。”凤姐笑道:“我也不等银
子使,也不做这样的事。”静虚听了,打去妄想。半晌叹道:“虽这么说,只是张
家已经知道求了府里。如今不管,张家不说没工夫、不希图他的谢礼,倒像府里连
这点子手段也没有似的。”

凤姐听了这话,便发了兴头,说道:“你是素日知道我的,从来不信什么阴司
地狱报应的,凭是什么事,我说要行就行。你叫他拿三千两银子来,我就替他出这
口气。”老尼听说,喜之不胜,忙说:“有!有!这个不难。”凤姐又道:“我比不
得他们扯篷拉纤的图银子。这三千两银子,不过是给打发说去的小厮们作盘缠,使
他赚几个辛苦钱儿,我一个钱也不要。就是三万两我此刻还拿的出来。”老尼忙答
应道:“既如此,奶奶明天就开恩罢了。”凤姐道:“你瞧瞧我忙的,那一处少的
了我?我既应了你,自然给你了结啊。”老尼道:“这点子事要在别人,自然忙的
不知怎么样;要是奶奶跟前,再添上些,也不够奶奶一办的。俗语说的:‘能者多
劳。’太太见奶奶这样才情,越发都推给奶奶了。只是奶奶也要保重贵体些才是。”
一路奉承,凤姐越发受用了,也不顾劳乏,更攀谈起来。

谁想秦钟趁黑晚无人,来寻智能儿。刚到后头房里,只见智能儿独在那儿洗茶
碗,秦钟便搂着亲嘴。智能儿急的跺脚说:“这是做什么!”就要叫唤。秦钟道:
“好妹妹,我要急死了!你今儿再不依我,我就死在这里。”智能儿道:“你要怎
么样,除非我出了这牢坑,离了这些人,才好呢。”秦钟道:“这也容易,只是‘远
水解不得近渴’。”说着一口吹了灯,满屋里漆黑,将智能儿抱到炕上。那智能儿
百般的扎挣不起来,又不好嚷,不知怎么样就把中衣儿解下来了。这里刚才入港,
说时迟,那时快,猛然间一个人从身后冒冒失失的按住,也不出声。二人唬的魂飞
魄散。只听“嗤”的一笑,这才知是宝玉。秦钟连忙起来抱怨道:“这算什么?”
宝玉道:“你倒不依?咱们就嚷出来。”羞的智能儿趁暗中跑了。宝玉拉着秦钟出
来道:“你可还强嘴不强?”秦钟笑道:“好哥哥,你只别嚷,你要怎么着都使的。”
宝玉笑道:“这会子也不用说,等一会儿睡下咱们再慢慢儿的算帐。”

一时宽衣安歇的时节,凤姐在里间,宝玉秦钟在外间,满地下皆是婆子们打铺
坐更。凤姐因怕通灵玉失落,等宝玉睡下,令人拿来在自己枕边。却不知宝玉和
秦钟如何算帐,未见真切,此系疑案,不敢创纂。

且说次日一早,便有贾母王夫人打发了人来看宝玉,命多穿两件衣服,无事宁
可回去。宝玉那里肯?又兼秦钟恋着智能儿,调唆宝玉求凤姐再住一天。凤姐想了
一想,丧仪大事虽妥,还有些小事,也可以再住一日:一则贾珍跟前送了满情,二
则又可以完了静虚的事,三则顺了宝玉的心。因此便向宝玉道:“我的事都完了。
你要在这里逛,少不得索性辛苦了。明儿是一定要走的了。”宝玉听说,千姐姐万
姐姐的央求:“只住一日,明儿必回去的。”于是又住了一夜。凤姐便命悄悄将昨
日老尼之事说与来旺儿。旺儿心中俱已明白,急忙进城,找着主文的相公,假托贾
琏所嘱,修书一封,连夜往长安县来。不过百里之遥,两日工夫,俱已妥协。那节
度使名唤云光,久悬贾府之情,这些小事岂有不允之理,给了回书。旺儿回来,不
在话下。

且说凤姐等又过了一日,次日方别了老尼,着他三日后往府里去讨信。那秦钟
和智能儿两个,百般的不忍分离,背地里设了多少幽期密约,只得含恨而别,俱不
用细述。凤姐又到铁槛寺中照望一番。宝珠执意不肯回家,贾珍只得派妇女相伴。

后事如何,且听下回分解。