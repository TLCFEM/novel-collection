\chapter{大观园试才题对额~荣国府归省庆元宵}

话说秦钟既死,宝玉痛哭不止,李贵等好容易劝解半日方住,归时还带馀哀。
贾母帮了几十两银子,外又另备奠仪,宝玉去吊祭。七日后便送殡掩埋了,别无记
述。只有宝玉日日感悼,思念不已,然亦无可如何了。又不知过了几时才罢。

这日贾珍等来回贾政:“园内工程俱已告竣,大老爷已瞧过了,只等老爷瞧了,
或有不妥之处,再行改造,好题匾额对联。”贾政听了,沉思一会,说道:“这匾
对倒是一件难事。论礼该请贵妃赐题才是,然贵妃若不亲观其景,亦难悬拟。若直
待贵妃游幸时再行请题,若大景致,若干亭榭,无字标题,任是花柳山水,也断不
能生色。”众清客在旁笑答道:“老世翁所见极是。如今我们有个主意:各处匾对
断不可少,亦断不可定。如今且按其景致,或两字、三字、四字,虚合其意拟了来,
暂且做出灯匾对联悬了,待贵妃游幸时,再请定名,岂不两全?”贾政听了道:“所
见不差。我们今日且看看去,只管题了,若妥便用;若不妥,将雨村请来,令他再
拟。”众人笑道:“老爷今日一拟定佳,何必又待雨村。”贾政笑道:“你们不知:
我自幼于花鸟山水题咏上就平平的,如今上了年纪,且案牍劳烦,于这怡情悦性的
文章更生疏了。便拟出来,也不免迂腐,反使花柳园亭因而减色,转没意思。”众
清客道:“这也无妨。我们大家看了公拟,各举所长,优则存之,劣则删之,未为
不可。”贾政道:“此论极是。且喜今日天气和暖,大家去逛逛。”说着,起身引
众人前往。贾珍先去园中知会。

可巧近日宝玉因思念秦钟,忧伤不已,贾母常命人带他到新园子里来玩耍。此
时也才进去,忽见贾珍来了,和他笑道:“你还不快出去呢,一会子老爷就来了。”
宝玉听了,带着奶娘小厮们,一溜烟跑出园来。方转过弯,顶头看见贾政引着众客
来了,躲之不及,只得一旁站住。贾政近来闻得代儒称赞他专能对对,虽不喜读书,
却有些歪才,所以此时便命他跟入园中,意欲试他一试。宝玉未知何意,只得随往。

刚至园门,只见贾珍带领许多执事人旁边侍立。贾政道:“你且把园门关上,
我们先瞧外面,再进去。”贾珍命人将门关上。贾政先秉正看门,只见正门五间,
上面筒瓦泥鳅脊,那门栏窗俱是细雕时新花样,并无朱粉涂饰。一色水磨群墙,
下面白石台阶,凿成西番莲花样。左右一望,雪白粉墙,下面虎皮石砌成纹理,不
落富丽俗套,自是喜欢。遂命开门进去。只见一带翠嶂挡在面前。众清客都道:“好
山,好山!”贾政道:“非此一山,一进来园中所有之景悉入目中,更有何趣?”
众人都道:“极是。非胸中大有丘壑,焉能想到这里。”说毕,往前一望,见白石
,或如鬼怪,或似猛兽,纵横拱立。上面苔藓斑驳,或藤萝掩映,其中微露羊
肠小径。贾政道:“我们就从此小径游去,回来由那一边出去,方可遍览。”

说毕,命贾珍前导,自己扶了宝玉,逶迤走进山口。抬头忽见山上有镜面白石
一块,正是迎面留题处。贾政回头笑道:“诸公请看,此处题以何名方妙?”众人
听说,也有说该题“叠翠”二字的,也有说该题“锦嶂”的,又有说“赛香炉”的,
又有说“小终南”的,种种名色,不止几十个。原来众客心中,早知贾政要试宝玉
的才情,故此只将些俗套敷衍。宝玉也知此意。贾政听了,便回头命宝玉拟来。宝
玉道:“尝听见古人说:‘编新不如述旧,刻古终胜雕今。’况这里并非主山正景,
原无可题,不过是探景的一进步耳。莫如直书古人‘曲径通幽’这旧句在上,倒也
大方。”众人听了,赞道:“是极,好极!二世兄天分高,才情远,不似我们读腐
了书的。”贾政笑道:“不当过奖他。他年小的人,不过以一知充十用,取笑罢了。
再俟选拟。”

说着,进入石洞,只见佳木茏葱,奇花烂漫,一带清流,从花木深处泻于石隙
之下。再进数步,渐向北边,平坦宽豁,两边飞楼插空,雕甍绣槛,皆隐于山坳树
杪之间。俯而视之,但见青溪泻玉,石磴穿云,白石为栏,环抱池沼,石桥三港,
兽面衔吐。桥上有亭,贾政与诸人到亭内坐了,问:“诸公以何题此?”诸人都说:
“当日欧阳公《醉翁亭记》有云‘有亭翼然’,就名‘翼然’罢。”贾政笑道:“‘翼
然’虽佳,但此亭压水而成,还须偏于水题为称。依我拙裁,欧阳公句:‘泻于两
峰之间’,竟用他这一个‘泻’字。”有一客道:“是极,是极。竟是‘泻玉’二
字妙。”贾政拈须寻思,因叫宝玉也拟一个来。宝玉回道:“老爷方才所说已是。
但如今追究了去,似乎当日欧阳公题酿泉用一‘泻’字则妥,今日此泉也用‘泻’
字,似乎不妥。况此处既为省亲别墅,亦当依应制之体,用此等字亦似粗陋不雅。
求再拟蕴藉含蓄者。”贾政笑道:“诸公听此论何如?方才众人编新,你说‘不如
述古’;如今我们述古,你又说粗陋不妥。你且说你的。”宝玉道:“用‘泻玉’
二字,则不若‘沁芳’二字,岂不新雅?”贾政拈须点头不语。众人都忙迎合,称
赞宝玉才情不凡。贾政道:“匾上二字容易。再作一副七言对来。”宝玉四顾一望,
机上心来,乃念道:
绕堤柳借三篙翠,
隔岸花分一脉香。
贾政听了,点头微笑。众人又称赞了一番。

于是出亭过池,一山一石,一花一木,莫不着意观览。忽抬头见前面一带粉垣,
数楹修舍,有千百竿翠竹遮映。众人都道:“好个所在!”于是大家进入,只见进
门便是曲折游廊,阶下石子漫成甬路,上面小小三间房舍,两明一暗,里面都是合
着地步打的床几椅案。从里间房里,又有一小门,出去却是后园,有大株梨花,阔
叶芭蕉,又有两间小小退步。后院墙下忽开一隙,得泉一派,开沟尺许,灌入墙内,
绕阶缘屋至前院,盘旋竹下而出。贾政笑道:“这一处倒还好,若能月夜至此窗下
读书,也不枉虚生一世。”说着便看宝玉,唬的宝玉忙垂了头。众人忙用闲话解说。
又二客说:“此处的匾该题四个字。”贾政笑问:“那四字?”一个道是:“淇水
遗风。”贾政道:“俗。”又一个道是:“睢园遗迹。”贾政道:“也俗。”贾珍
在旁说道:“还是宝兄弟拟一个罢。”贾政道:“他未曾做,先要议论人家的好歹,
可见是个轻薄东西。”众客道:“议论的是,也无奈他何。”贾政忙道:“休如此
纵了他。”因说道:“今日任你狂为乱道,等说出议论来,方许你做。方才众人说
的,可有使得的没有?”宝玉见问,便答道:“都似不妥。”贾政冷笑道:“怎么
不妥?”宝玉道:“这是第一处行幸之所,必须颂圣方可。若用四字的匾,又有古
人现成的,何必再做?”贾政道:“难道‘淇水’、‘睢园’不是古人的?”宝玉
道:“这太板了。莫若‘有凤来仪’四字。”众人都哄然叫妙。贾政点头道:“畜
生,畜生!可谓‘管窥蠡测’矣。”因命:“再题一联来。”宝玉便念道:
宝鼎茶闲烟尚绿,
幽窗棋罢指犹凉。
贾政摇头道:“也未见长。”说毕,引人出来。

方欲走时,忽想起一事来,问贾珍道:“这些院落屋宇,并几案桌椅都算有了。
还有那些帐幔帘子并陈设玩器古董,可也都是一处一处合式配就的么?”贾珍回
道:“那陈设的东西早已添了许多,自然临期合式陈设。帐幔帘子,昨日听见琏兄
弟说,还不全。那原是一起工程之时就画了各处的图样,量准尺寸,就打发人办去
的;想必昨日得了一半。”贾政听了,便知此事不是贾珍的首尾,便叫人去唤贾琏。
一时来了,贾政问他:“共有几宗?现今得了几宗?尚欠几宗?”贾琏见问,忙向靴
筒内取出靴掖里装的一个纸折略节来,看了一看,回道:“妆蟒洒堆、刻丝弹墨并
各色绸绫大小幔子一百二十架,昨日得了八十架,下欠四十架。帘子二百挂,昨日
俱得了。外有猩猩毡帘二百挂,湘妃竹帘一百挂,金丝藤红漆竹帘一百挂,黑漆竹
帘一百挂,五彩线络盘花帘二百挂,每样得了一半,也不过秋天都全了。椅搭、桌
围、床裙、杌套,每分一千二百件,也有了。”

一面说,一面走,忽见青山斜阻。转过山怀中,隐隐露出一带黄泥墙,墙上皆
用稻茎掩护。有几百枝杏花,如喷火蒸霞一般。里面数楹茅屋,外面却是桑、榆、
槿、柘各色树稚新条,随其曲折,编就两溜青篱。篱外山坡之下,有一土井,旁有
桔槔辘轳之属;下面分畦列亩,佳蔬菜花,一望无际。贾政笑道:“倒是此处有些
道理。虽系人力穿凿,却入目动心,未免勾引起我归农之意。我们且进去歇息歇息。”
说毕,方欲进去,忽见篱门外路旁有一石,亦为留题之所。众人笑道:“更妙,更
妙!此处若悬匾待题,则田舍家风一洗尽矣。立此一碣,又觉许多生色,非范石湖
田家之咏不足以尽其妙。”贾政道:“诸公请题。”众人云:“方才世兄云:‘编
新不如述旧。’此处古人已道尽矣:莫若直书‘杏花村’为妙。”贾政听了,笑向
贾珍道:“正亏提醒了我。此处都好,只是还少一个酒幌,明日竟做一个来,就依
外面村庄的式样,不必华丽,用竹竿挑在树梢头。”贾珍答应了,又回道:“此处
竟不必养别样雀鸟,只养些鹅、鸭、鸡之类,才相称。”贾政与众人都说好。

贾政又向众人道:“‘杏花村’固佳,只是犯了正村名,直待请名方可。”众
客都道:“是呀!如今虚的,却是何字样好呢?”大家正想,宝玉却等不得了,也
不等贾政的话,便说道:“旧诗云:‘红杏梢头挂酒旗。’如今莫若且题以‘杏帘
在望’四字。”众人都道:“好个‘在望’!又暗合‘杏花村’意思。”宝玉冷笑
道:“村名若用‘杏花’二字,便俗陋不堪了。唐人诗里,还有‘柴门临水稻花香’,
何不用‘稻香村’的妙?”众人听了,越发同声拍手道妙。贾政一声断喝:“无知
的畜生!你能知道几个古人,能记得几首旧诗,敢在老先生们跟前卖弄!方才任你胡
说,也不过试你的清浊,取笑而已,你就认真了!”

说着,引众人步入茆堂,里面纸窗木榻,富贵气象一洗皆尽。贾政心中自是欢
喜,却瞅宝玉道:“此处如何?”众人见问,都忙悄悄的推宝玉教他说好。宝玉不
听人言,便应声道:“不及‘有凤来仪’多了。”贾政听了道:“咳!无知的蠢物,
你只知朱楼画栋、恶赖富丽为佳,那里知道这清幽气象呢?终是不读书之过!”宝
玉忙答道:“老爷教训的固是,但古人云‘天然’二字,不知何意?”众人见宝玉
牛心,都怕他讨了没趣;今见问“天然”二字,众人忙道:“哥儿别的都明白,如
何‘天然’反要问呢?天然者,天之自成,不是人力之所为的。”宝玉道:“却又
来!此处置一田庄,分明是人力造作成的:远无邻村,近不负郭,背山无脉,临水
无源,高无隐寺之塔,下无通市之桥,峭然孤出,似非大观,那及前数处有自然之
理、自然之趣呢?虽种竹引泉,亦不伤穿凿。古人云‘天然图画’四字,正恐非其
地而强为其地,非其山而强为其山,即百般精巧,终不相宜……”未及说完,贾政
气的喝命:“出去!”才出去,又喝命:“回来!”命:“再题一联,若不通,
一并打嘴巴!”宝玉吓的战兢兢的,半日,只得念道:
新绿涨添浣葛处,
好云香护采芹人。

贾政听了,摇头道:“更不好。”一面引人出来,转过山坡,穿花度柳,抚石
依泉,过了荼架,入木香棚,越牡丹亭,度芍药圃,到蔷薇院,傍芭蕉坞里盘旋
曲折。忽闻水声潺潺,出于石洞;上则萝薜倒垂,下则落花浮荡。众人都道:“好
景,好景!”贾政道:“诸公题以何名?”众人道:“再不必拟了,恰恰乎是‘武
陵源’三字。”贾政笑道:“又落实了,而且陈旧。”众人笑道:“不然就用‘秦
人旧舍’四字也罢。”宝玉道:“越发背谬了。‘秦人旧舍’是避乱之意,如何使
得?莫若‘蓼汀花溆’四字。”贾政听了道:“更是胡说。”

于是贾政进了港洞,又问贾珍:“有船无船?”贾珍道:“采莲船共四只,座
船一只,如今尚未造成。”贾政笑道:“可惜不得入了!”贾珍道:“从山上盘道
也可以进去的。”说毕,在前导引,大家攀藤抚树过去。只见水上落花愈多,其水
愈加清溜,溶溶荡荡,曲折萦纡。池边两行垂柳,杂以桃杏遮天,无一些尘土。忽
见柳阴中又露出一个折带朱栏板桥来,度过桥去,诸路可通,便见一所清凉瓦舍,
一色水磨砖墙,清瓦花堵。那大主山所分之脉皆穿墙而过。贾政道:“此处这一所
房子,无味的很。”因而步入门时,忽迎面突出插天的大玲珑山石来,四面群绕各
式石块,竟把里面所有房屋悉皆遮住。且一树花木也无,只见许多异草,或有牵藤
的,或有引蔓的,或垂山岭,或穿石脚,甚至垂檐绕柱,萦砌盘阶,或如翠带飘摇,
或如金绳蟠屈,或实若丹砂,或花如金桂,味香气馥,非凡花之可比。贾政不禁道:
“有趣!只是不大认识。”有的说:“是薜荔藤萝。”贾政道:“薜荔藤萝那得有
此异香?”宝玉道:“果然不是。这众草中也有藤萝薜荔。那香的是杜若蘅芜,那
一种大约是兰,这一种大约是金葛,那一种是金草,这一种是玉藤,红的自
然是紫芸,绿的定是青芷。想来那《离骚》、《文选》所有的那些异草:有叫作什
么霍纳姜汇的,也有叫作什么纶组紫绛的。还有什么石帆、清松、扶留等样的,见
于左太冲《吴都赋》。又有叫作什么绿荑的,还有什么丹椒、蘼芜、风莲,见于《蜀
都赋》。如今年深岁改,人不能识,故皆象形夺名,渐渐的唤差了,也是有的。”
未及说完,贾政喝道:“谁问你来?”唬的宝玉倒退,不敢再说。

贾政因见两边俱是超手游廊,便顺着游廊步入,只见上面五间清厦,连着卷棚,
四面出廊,绿窗油壁,更比前清雅不同。贾政叹道:“此轩中煮茗操琴,也不必再
焚香了。此造却出意外,诸公必有佳作新题以颜其额,方不负此。”众人笑道:“莫
若‘兰风蕙露’贴切了。”贾政道:“也只好用这四字。其联云何?”一人道:“我
想了一对,大家批削改正。道是:‘麝兰芳霭斜阳院,杜若香飘明月洲。’”众人
道:“妙则妙矣!只是‘斜阳’二字不妥。”那人引古诗“蘼芜满院泣斜阳”句,
众人云:“颓丧,颓丧!”又一人道:“我也有一联,诸公评阅评阅。”念道:“三
径香风飘玉蕙,一庭明月照金兰。”贾政拈须沉吟,意欲也题一联。忽抬头见宝玉
在旁不敢作声,因喝道:“怎么你应说话时又不说了!还要等人请教你不成?”宝
玉听了回道:“此处并没有什么‘兰麝’、‘明月’、‘洲渚’之类,若要这样着
迹说来,就题二百联也不能完。”贾政道:“谁按着你的头,教你必定说这些字样
呢?”宝玉道:“如此说,则匾上莫若‘蘅芷清芬’四字。对联则是:‘吟成豆蔻
诗犹艳,睡足荼梦亦香。’”贾政笑道:“这是套的‘书成蕉叶文犹绿’,不足
为奇。”众人道:“李太白‘凤凰台’之作,全套‘黄鹤楼’。只要套得妙。如今
细评起来,方才这一联竟比‘书成蕉叶’尤觉幽雅活动。”贾政笑道:“岂有此理。”

说着,大家出来。走不多远,则见崇阁巍峨,层楼高起,面面琳宫合抱,迢迢
复道萦纡。青松拂檐,玉兰绕砌;金辉兽面,彩焕螭头。贾政道:“这是正殿了。
只是太富丽了些!”众人都道:“要如此方是。虽然贵妃崇尚节俭,然今日之尊,
礼仪如此,不为过也。”一面说,一面走,只见正面现出一座玉石牌坊,上面龙蟠
螭护,玲珑凿就。贾政道:“此处书以何文?”众人道:“必是‘蓬莱仙境’方妙。”
贾政摇头不语。宝玉见了这个所在,心中忽有所动,寻思起来,倒像在那里见过的
一般,却一时想不起那年那日的事了。贾政又命他题咏,宝玉只顾细思前景,全无
心于此了。众人不知其意,只当他受了这半日折磨,精神耗散,才尽词穷了,再要
牛难逼迫着了急,或生出事来,倒不便。遂忙都劝贾政道:“罢了,明日再题罢了。”
贾政心中也怕贾母不放心,遂冷笑道:“你这畜生,也竟有不能之时了。也罢,限
你一日,明日题不来,定不饶你。这是第一要紧处所,要好生作来!”

说着,引人出来,再一观望,原来自进门至此,才游了十之五六。又值人来回,
有雨村处遣人回话。贾政笑道:“此数处不能游了。虽如此,到底从那一边出去,
也可略观大概。”说着,引客行来,至一大桥,水如晶帘一般奔入。原来这桥边是
通外河之闸,引泉而入者。贾政因问:“此闸何名?”宝玉道:“此乃沁芳源之正
流,即名‘沁芳闸’。”贾政道:“胡说,偏不用‘沁芳’二字。”

于是一路行来,或清堂,或茅舍,或堆石为垣,或编花为门,或山下得幽尼佛
寺,或林中藏女道丹房,或长廊曲洞,或方厦圆亭:贾政皆不及进去。因半日未尝
歇息,腿酸脚软,忽又见前面露出一所院落来,贾政道:“到此可要歇息歇息了。”
说着一径引入,绕着碧桃花,穿过竹篱花障编就的月洞门,俄见粉垣环护,绿柳周
垂。贾政与众人进了门,两边尽是游廊相接,院中点衬几块山石,一边种几本芭蕉,
那一边是一树西府海棠,其势若伞,丝垂金缕,葩吐丹砂。众人都道:“好花,好
花!海棠也有,从没见过这样好的。”贾政道:“这叫做‘女儿棠’,乃是外国之
种,俗传出‘女儿国’,故花最繁盛,——亦荒唐不经之说耳。”众人道:“毕竟
此花不同,‘女国’之说,想亦有之。”宝玉云:“大约骚人咏士以此花红若施脂,
弱如扶病,近乎闺阁风度,故以‘女儿’命名,世人以讹传讹,都未免认真了。”
众人都说:“领教!妙解!”一面说话,一面都在廊下榻上坐了。贾政因道:“想
几个什么新鲜字来题?”一客道:“‘蕉鹤’二字妙。”又一个道:“‘崇光泛彩’
方妙。”贾政与众人都道:“好个‘崇光泛彩’!”宝玉也道:“妙。”又说:“只
是可惜了!”众人问:“如何可惜?”宝玉道:“此处蕉棠两植,其意暗蓄‘红’
‘绿’二字在内,若说一样,遗漏一样,便不足取。”贾政道:“依你如何?”宝
玉道:“依我,题‘红香绿玉’四字,方两全其美。”贾政摇头道:“不好,不好!”

说着,引人进入房内。只见其中收拾的与别处不同,竟分不出间隔来。原来四
面皆是雕空玲珑木板,或“流云百蝠”,或“岁寒三友”,或山水人物,或翎毛花
卉,或集锦,或博古,或万福万寿,各种花样,皆是名手雕镂五彩,销金嵌玉的。
一一,或贮书,或设鼎,或安置笔砚,或供设瓶花,或安放盆景。其式样或
圆或方,或葵花蕉叶,或连环半璧,真是花团锦簇,剔透玲珑。倏尔五色纱糊,竟
系小窗;倏尔彩绫轻覆,竟系幽户。且满墙皆是随依古董玩器之形抠成的槽子,如
琴、剑、悬瓶之类,俱悬于壁,却都是与壁相平的。众人都赞:“好精致!难为怎
么做的!”原来贾政走进来了,未到两层,便都迷了旧路,左瞧也有门可通,右瞧
也有窗隔断,及到跟前,又被一架书挡住,回头又有窗纱明透门径。及至门前,忽
见迎面也进来了一起人,与自己的形相一样,——却是一架大玻璃镜。转过镜去,
一发见门多了。贾珍笑道:“老爷随我来,从这里出去就是后院,出了后院倒比先
近了。”引着贾政及众人转了两层纱厨,果得一门出去,院中满架蔷薇。转过花障,
只见青溪前阻。众人诧异:“这水又从何而来?”贾珍遥指道:“原从那闸起流至
那洞口,从东北山凹里引到那村庄里,又开一道岔口,引至西南上,共总流到这里,
仍旧合在一处,从那墙下出去。”众人听了,都道:“神妙之极!”说着,忽见大
山阻路,众人都迷了路,贾珍笑道:“跟我来。”乃在前导引,众人随着,由山脚
下一转,便是平坦大路,豁然大门现于面前,众人都道:“有趣,有趣!搜神夺巧,
至于此极!”于是大家出来。

那宝玉一心只记挂着里边姊妹们,又不见贾政吩咐,只得跟到书房。贾政忽想
起来道:“你还不去,看老太太惦记你。难道还逛不足么?”宝玉方退了出来。至
院外,就有跟贾政的小厮上来抱住,说道:“今日亏了老爷喜欢,方才老太太打发
人出来问了几遍,我们回说老爷喜欢;要不然,老太太叫你进去了,就不得展才了。
人人都说你才那些诗比众人都强,今儿得了彩头,该赏我们了。”宝玉笑道:“每
人一吊。”众人道:“谁没见那一吊钱!把这荷包赏了罢。”说着,一个个都上来
解荷包,解扇袋,不容分说,将宝玉所佩之物,尽行解去。又道:“好生送上去罢。”
一个个围绕着,送至贾母门前。那时贾母正等着他,见他来了,知道不曾难为他,
心中自是喜欢。

少时袭人倒了茶来,见身边佩物一件不存,因笑道:“带的东西必又是那起没
脸的东西们解了去了。”黛玉听说,走过来一瞧,果然一件没有,因向宝玉道:“我
给你的那个荷包也给他们了?你明儿再想我的东西,可不能够了!”说毕,生气回
房,将前日宝玉嘱咐他没做完的香袋儿,拿起剪子来就铰。宝玉见他生气,便忙赶
过来,早已剪破了。宝玉曾见过这香袋,虽未完工,却十分精巧,无故剪了,却也
可气。因忙把衣领解了,从里面衣襟上将所系荷包解下来了,递与黛玉道:“你瞧
瞧,这是什么东西?我何从把你的东西给人来着?”黛玉见他如此珍重,带在里面,
可知是怕人拿去之意,因此自悔莽撞剪了香袋,低着头一言不发。宝玉道:“你也
不用铰,我知你是懒怠给我东西。我连这荷包奉还,何如?”说着掷向他怀中而去。
黛玉越发气的哭了,拿起荷包又铰。宝玉忙回身抢住,笑道:“好妹妹饶了他罢!”
黛玉将剪子一摔,拭泪说道:“你不用合我好一阵歹一阵的,要恼就撂开手。”说
着赌气上床,面向里倒下拭泪。禁不住宝玉上来“妹妹”长“妹妹”短赔不是。

前面贾母一片声找宝玉。众人回说:“在林姑娘房里。”贾母听说道:“好,
好!让他姐妹们一处玩玩儿罢。才他老子拘了他这半天,让他松泛一会子罢。只别
叫他们拌嘴。”众人答应着。

黛玉被宝玉缠不过,只得起来道:“你的意思不叫我安生,我就离了你。”说
着往外就走。宝玉笑道:“你到那里我跟到那里。”一面仍拿着荷包来带上。黛玉
伸手抢道:“你说不要,这会子又带上,我也替你怪臊的!”说着“嗤”的一声笑
了。宝玉道:“好妹妹,明儿另替我做个香袋儿罢!”黛玉道:“那也瞧我的高兴
罢了。”一面说,一面二人出房,到王夫人上房中去了。可巧宝钗也在那里。

此时王夫人那边热闹非常。原来贾蔷已从姑苏采买了十二个女孩子、并聘了教
习以及行头等事来了,那时薛姨妈另于东北上一所幽静房舍居住,将梨香院另行修
理了,就令教习在此教演女戏;又另派了家中旧曾学过歌唱的众女人们,———如
今皆是皤然老妪,着他们带领管理。其日月出入银钱等事,以及诸凡大小所需之物
料帐目,就令贾蔷总理。

又有林之孝来回:“采访聘买得十二个小尼姑、小道姑,都到了。连新做的二
十分道袍也有了。外又有一个带发修行的,本是苏州人氏,祖上也是读书仕宦之家,
因自幼多病,买了许多替身,皆不中用,到底这姑娘入了空门,方才好了,所以带
发修行。今年十八岁,取名妙玉。如今父母俱已亡故,身边只有两个老嬷嬷、一个
小丫头伏侍,文墨也极通,经典也极熟,模样又极好。因听说长安都中有观音遗迹
并贝叶遗文,去年随了师父上来,现在西门外牟尼院住着。他师父精演先天神数,
于去冬圆寂了。遗言说他:‘不宜回乡,在此静候,自有结果。’所以未曾扶灵回
去。”王夫人便道:“这样我们何不接了他来?”林之孝家的回道:“若请他,他
说:‘侯门公府,必以贵势压人,我再不去的。’”王夫人道:“他既是宦家小姐,
自然要性傲些。就下个请帖请他何妨。”林之孝家的答应着出去,叫书启相公写个
请帖去请妙玉,次日遣人备车轿去接。

不知后来如何,且听下回分解。