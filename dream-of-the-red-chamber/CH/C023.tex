\chapter{西厢记妙词通戏语~牡丹亭艳曲警芳心}

话说贾母次日仍领众人过节。那元妃却自幸大观园回宫去后,便命将那日所有
的题咏,命探春抄录妥协,自己编次优劣,又令在大观园勒石,为千古风流雅事。
因此贾政命人选拔精工,大观园磨石镌字。贾珍率领贾蓉贾蔷等监工。因贾蔷又管
着文官等十二个女戏子并行头等事,不得空闲,因此又将贾菖、贾菱、贾萍唤来监
工。一日烫蜡钉朱,动起手来。这也不在话下。

且说那玉皇庙并达摩庵两处,一班的十二个小沙弥并十二个小道士,如今挪出
大观园来,贾政正想发到各庙去分住。不想后街上住的贾芹之母杨氏,正打算到贾
政这边谋一个大小事件与儿子管管,也好弄些银钱使用,可巧听见这件事,便坐车
来求凤姐。凤姐因见他素日嘴头儿乖滑,便依允了。想了几句话,便回了王夫人说:
“这些小和尚小道士万不可打发到别处去,一时娘娘出来,就要应承的。倘或散了,
若再用时,可又费事。依我的主意,不如将他们都送到家庙铁槛寺去,月间不过派
一个人拿几两银子去买柴米就是了。说声用,走去叫一声就来,一点儿不费事。”
王夫人听了,便商之于贾政。贾政听了笑道:“倒是提醒了我。就是这样。”即时
唤贾琏。贾琏正同凤姐吃饭,一闻呼唤,放下饭便走。凤姐一把拉住,笑道:“你
先站住,听我说话:要是别的事,我不管;要是为小和尚小道士们的事,好歹你依
着我这么着。”如此这般,教了一套话。贾琏摇头笑道:“我不管!你有本事你说
去。”凤姐听说,把头一梗,把筷子一放,腮上带笑不笑的瞅着贾琏道:“你是真
话,还是玩话儿?”贾琏笑道:“西廊下五嫂子的儿子芸儿求了我两三遭,要件事
管管,我应了,叫他等着。好容易出来这件事,你又夺了去!”凤姐儿笑道:“你
放心。园子东北角上,娘娘说了,还叫多多的种松柏树,楼底下还叫种些花草儿。
等这件事出来,我包管叫芸儿管这工程就是了。”贾琏道:“这也罢了。”因又悄
悄的笑道:“我问你,我昨儿晚上不过要改个样儿,你为什么就那么扭手扭脚的
呢?”凤姐听了,把脸飞红,“嗤”的一笑,向贾琏啐了一口,依旧低下头吃饭。
贾琏笑着一径去了。

走到前面见了贾政,果然为小和尚的事。贾琏便依着凤姐的话,说道:“看来
芹儿倒出息了,这件事竟交给他去管,横竖照里头的规例,每月支领就是了。”贾
政原不大理论这些小事,听贾琏如此说,便依允了。贾琏回房告诉凤姐,凤姐即命
人去告诉杨氏,贾芹便来见贾琏夫妻,感谢不尽。凤姐又做情先支三个月的费用,
叫他写了领字,贾琏画了押,登时发了对牌出去,银库上按数发出三个月的供给来,
白花花三百两。贾芹随手拈了一块与掌平的人,叫他们“喝了茶罢”。于是命小厮
拿了回家,与母亲商议。登时雇车坐上,又雇了几辆车子至荣国府角门前,唤出二
十四个人来,坐上车子,一径往城外铁槛寺去了。当下无话。

如今且说那元妃在宫中编次《大观园题咏》,忽然想起那园中的景致,自从幸
过之后,贾政必定敬谨封锁,不叫人进去,岂不辜负此园?况家中现有几个能诗会
赋的姊妹们,何不命他们进去居住,也不使佳人落魄,花柳无颜。却又想宝玉自幼
在姊妹丛中长大,不比别的兄弟,若不命他进去,又怕冷落了他,恐贾母王夫人心
上不喜,须得也命他进去居住方妥。命太监夏忠到荣府下一道谕:“命宝钗等在园
中居住,不可封锢;命宝玉也随进去读书。”贾政王夫人接了谕命。夏忠去后,便
回明贾母,遣人进去各处收拾打扫,安设帘幔床帐。

别人听了,还犹自可,惟宝玉喜之不胜。正和贾母盘算要这个要那个,忽见丫
鬟来说:“老爷叫宝玉。”宝玉呆了半晌,登时扫了兴,脸上转了色,便拉着贾母
扭的扭股儿糖似的,死也不敢去。贾母只得安慰他道:“好宝贝,你只管去,有我
呢。他不敢委屈了你。况你做了这篇好文章,想必娘娘叫你进园去住,他吩咐你几
句话,不过是怕你在里头淘气。他说什么,你只好生答应着就是了。”一面安慰,
一面唤了两个老嬷嬷来,吩咐:“好生带了宝玉去,别叫他老子唬着他。”老嬷嬷
答应了。宝玉只得前去,一步挪不了三寸,蹭到这边来。

可巧贾政在王夫人房中商议事情,金钏儿、彩云、彩凤、绣鸾、绣凤等众丫鬟
都廊檐下站着呢,一见宝玉来,都抿着嘴儿笑他。金钏儿一把拉着宝玉,悄悄的说
道:“我这嘴上是才擦的香香甜甜的胭脂,你这会子可吃不吃了?”彩云一把推开
金钏儿,笑道:“人家心里发虚,你还怄他!趁这会子喜欢,快进去罢。”宝玉只
得挨门进去。原来贾政和王夫人都在里间呢。赵姨娘打起帘子来,宝玉挨身而入,
只见贾政和王夫人对坐在炕上说话儿,地下一溜椅子,迎春、探春、惜春、贾环四
人都坐在那里。一见他进来,探春惜春和贾环都站起来。

贾政一举目见宝玉站在跟前,神彩飘逸,秀色夺人,又看看贾环人物委琐,举
止粗糙,忽又想起贾珠来。再看看王夫人只有这一个亲生的儿子,素爱如珍;自己
的胡须将已苍白:因此上把平日嫌恶宝玉之心不觉减了八九分。半晌说道:“娘娘
吩咐说:你日日在外游嬉,渐次疏懒了工课,如今叫禁管你和姐妹们在园里读书。
你可好生用心学习,再不守分安常,你可仔细着!”宝玉连连答应了几个“是”。
王夫人便拉他在身边坐下。他姊弟三人依旧坐下,王夫人摸索着宝玉的脖项说道:
“前儿的丸药都吃完了没有?”宝玉答应道:“还有一丸。”王夫人道:“明儿再
取十丸来,天天临睡时候,叫袭人伏侍你吃了再睡。”宝玉道:“从太太吩咐了,
袭人天天临睡打发我吃的。”贾政便问道:“谁叫‘袭人’?”王夫人道:“是个
丫头。”贾政道:“丫头不拘叫个什么罢了,是谁起这样刁钻名字?”王夫人见贾
政不喜欢了,便替宝玉掩饰道:“是老太太起的。”贾政道:“老太太如何晓得这
样的话?一定是宝玉。”宝玉见瞒不过,只得起身回道:“因素日读诗,曾记古人
有句诗云:‘花气袭人知昼暖’,因这丫头姓‘花’,便随意起的。”王夫人忙向
宝玉说道:“你回去改了罢。老爷也不用为这小事生气。”贾政道:“其实也无妨
碍,不用改。只可见宝玉不务正,专在这些浓词艳诗上做工夫。”说毕,断喝了一
声:“作孽的畜生,还不出去!”王夫人也忙道:“去罢,去罢。怕老太太等吃饭
呢。”

宝玉答应了,慢慢的退出去,向金钏儿笑着伸伸舌头,带着两个老嬷嬷,一溜
烟去了。刚至穿堂门前,只见袭人倚门而立,见宝玉平安回来,堆下笑来,问道:
“叫你做什么?”宝玉告诉:“没有什么,不过怕我进园淘气,吩咐吩咐。”一面
说,一面回至贾母跟前回明原委。只见黛玉正在那里,宝玉便问他:“你住在那一
处好?”黛玉正盘算这事,忽见宝玉一问,便笑道:“我心里想着潇湘馆好。我爱
那几竿竹子,隐着一道曲栏,比别处幽静些。”宝玉听了,拍手笑道:“合了我的
主意了,我也要叫你那里住。我就住怡红院,咱们两个又近,又都清幽。”二人正
计议着,贾政遣人来回贾母,说是:“二月二十二日是好日子,哥儿姐儿们就搬进
去罢。这几日便遣人进去分派收拾。”宝钗住了蘅芜院,黛玉住了潇湘馆,迎春住
了缀锦楼,探春住了秋掩书斋,惜春住了蓼风轩,李纨住了稻香村,宝玉住了怡红
院。每一处添两个老嬷嬷,四个丫头;除各人的奶娘亲随丫头外,另有专管收拾打
扫的。至二十二日,一齐进去,登时园内花招绣带,柳拂香风,不似前番那等寂寞
了。

闲言少叙,且说宝玉自进园来,心满意足,再无别项可生贪求之心,每日只和
姊妹丫鬟们一处,或读书,或写字,或弹琴下棋,作画吟诗,以至描鸾刺凤,斗草
簪花,低吟悄唱,拆字猜枚,无所不至,倒也十分快意。他曾有几首四时即事诗,
虽不算好,却是真情真景。《春夜即事》云:
霞绡云幄任铺陈,隔巷蛙声听未真。
枕上轻寒窗外雨,眼前春色梦中人。
盈盈烛泪因谁泣,点点花愁为我嗔。
自是小鬟妖懒惯,拥衾不耐笑言频。
《夏夜即事》云:
倦绣佳人幽梦长,金笼鹦鹉唤茶汤。
窗明麝月开宫镜,室霭檀云品御香。
琥珀杯倾荷露滑,玻璃槛纳柳风凉。
水亭处处齐纨动,帘卷朱楼罢晚妆。
《秋夜即事》云:
绛芸轩里绝喧哗,桂魄流光浸茜纱。
苔锁石纹容睡鹤,井飘桐露湿栖鸦。
抱衾婢至舒金凤,倚槛人归落翠花。
静夜不眠因酒渴,沉烟重拨索烹茶。
《冬夜即事》云:
梅魂竹梦已三更,锦衾睡未成。
松影一庭惟见鹤,梨花满地不闻莺。
女奴翠袖诗怀冷,公子金貂酒力轻。
却喜侍儿知试茗,扫将新雪及时烹。

不说宝玉闲吟,且说这几首诗,当时有一等势利人,见是荣国府十二三岁的公
子做的,抄录出来,各处称颂;再有等轻薄子弟,爱上那风流妖艳之句,也写在扇
头壁上,不时吟哦赏赞。因此上竟有人来寻诗觅字,倩画求题。这宝玉一发得意了,
每日家做这些外务。谁想静中生动,忽一日,不自在起来,这也不好,那也不好,
出来进去只是发闷。园中那些女陔子,正是混沌世界天真烂熳之时,坐卧不避,嬉
笑无心,那里知宝玉此时的心事?那宝玉不自在,便懒在园内,只想外头鬼混,却
痴痴的又说不出什么滋味来。茗烟见他这样,因想与他开心,左思右想皆是宝玉玩
烦了的,只有一件,不曾见过。想毕便走到书坊内,把那古今小说,并那飞燕、合
德、则天、玉环的“外传”,与那传奇角本,买了许多,孝敬宝玉。宝玉一看,如
得珍宝。茗烟又嘱咐道:“不可拿进园去,叫人知道了,我就‘吃不了兜着走’了。”
宝玉那里肯不拿进去?踟蹰再四,单把那文理雅道些的,拣了几套进去,放在床顶
上,无人时方看;那粗俗过露的,都藏于外面书房内。

那日正当三月中浣,早饭后,宝玉携了一套《会真记》,走到沁芳闸桥那边桃
花底下一块石上坐着,展开《会真记》,从头细看。正看到“落红成阵”,只见一
阵风过,树上桃花吹下一大斗来,落得满身满书满地皆是花片。宝玉要抖将不来,
恐怕脚步践踏了,只得兜了那花瓣儿,来至池边,抖在池内。那花瓣儿浮在水面,
飘飘荡荡,竟流出沁芳闸去了。回来只见地下还有许多花瓣。宝玉正踟蹰间,只听
背后有人说道:“你在这里做什么?”宝玉一回头,却是黛玉来了,肩上担着花锄,
花锄上挂着纱囊,手内拿着花帚。宝玉笑道:“来的正好,你把这些花瓣儿都扫起
来,撂在那水里去罢。我才撂了好些在那里了。”黛玉道:“撂在水里不好,你看
这里的水干净,只一流出去,有人家的地方儿什么没有?仍旧把花遭塌了。那畸角
儿上我有一个花冢,如今把他扫了,装在这绢袋里,埋在那里;日久随土化了,岂
不干净。”

宝玉听了,喜不自禁,笑道:“待我放下书,帮你来收拾。”黛玉道:“什么
书?”宝玉见问,慌的藏了,便说道:“不过是《中庸》《大学》。”黛玉道:“你
又在我跟前弄鬼。趁早儿给我瞧瞧,好多着呢!”宝玉道:“妹妹,要论你我是不
怕的,你看了好歹别告诉人。真是好文章!你要看了,连饭也不想吃呢!”一面说,
一面递过去。黛玉把花具放下,接书来瞧,从头看去,越看越爱,不顿饭时,已看
了好几出了。但觉词句警人,馀香满口。一面看了,只管出神,心内还默默记诵。
宝玉笑道:“妹妹,你说好不好?”黛玉笑着点头儿。宝玉笑道:“我就是个‘多
愁多病的身’,你就是那‘倾国倾城的貌’。”黛玉听了,不觉带腮连耳的通红了,
登时竖起两道似蹙非蹙的眉,瞪了一双似睁非睁的眼,桃腮带怒,薄面含嗔,指着
宝玉道:“你这该死的,胡说了!好好儿的,把这些淫词艳曲弄了来,说这些混帐
话,欺负我。我告诉舅舅、舅母去!”说到“欺负”二字,就把眼圈儿红了,转身
就走。宝玉急了,忙向前拦住道:“好妹妹,千万饶我这一遭儿罢!要有心欺负你,
明儿我掉在池子里,叫个癞头鼋吃了去,变个大忘八,等你明儿做了‘一品夫人’
病老归西的时候儿,我往你坟上替你驼一辈子碑去。”说的黛玉“扑嗤”的一声笑
了,一面揉着眼,一面笑道:“一般唬的这么个样儿,还只管胡说。呸!原来也是
个‘银样蜡枪头’。”宝玉听了,笑道:“你说说,你这个呢?我也告诉去。”黛
玉笑道:“你说你会‘过目成诵’,难道我就不能‘一目十行’了?”宝玉一面收
书,一面笑道:“正经快把花儿埋了罢,别提那些个了。”二人便收拾落花。

正才掩埋妥协,只见袭人走来,说道:“那里没找到?摸在这里来了!那边大老
爷身上不好,姑娘们都过去请安去了,老太太叫打发你去呢。快回去换衣裳罢。”
宝玉听了,忙拿了书,别了黛玉,同袭人回房换衣不提。

这里黛玉见宝玉去了,听见众姐妹也不在房中,自己闷闷的。正欲回房,刚走
到梨香院墙角外,只听见墙内笛韵悠扬,歌声婉转,黛玉便知是那十二个女孩子演
习戏文。虽未留心去听,偶然两句吹到耳朵内,明明白白一字不落道:“原来是姹
紫嫣红开遍,似这般都付与断井颓垣。”黛玉听了,倒也十分感慨缠绵,便止步侧
耳细听。又唱道是:“良辰美景奈何天,赏心乐事谁家院。”听了这两句,不觉点
头自叹,心下自思:“原来戏上也有好文章,可惜世人只知看戏,未必能领略其中
的趣味。”想毕,又后悔不该胡想,耽误了听曲子。再听时,恰唱到:“只为你如
花美眷,似水流年。”黛玉听了这两句,不觉心动神摇。又听道“你在幽闺自怜”
等句,越发如醉如痴,站立不住,便一蹲身坐在一块山子石上,细嚼“如花美眷,
似水流年”八个字的滋味。忽又想起前日见古人诗中,有“水流花谢两无情”之句;
再词中又有“流水落花春去也,天上人间”之句;又兼方才所见《西厢记》中“花
落水流红,闲愁万种”之句:都一时想起来,凑聚在一处。仔细忖度,不觉心痛神
驰,眼中落泪。

正没个开交处,忽觉身背后有人拍了他一下,及至回头看时,未知是谁,下回
分解。