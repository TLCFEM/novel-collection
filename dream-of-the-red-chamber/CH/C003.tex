\chapter{托内兄如海荐西宾~接外孙贾母惜孤女}

却说雨村忙回头看时,不是别人,乃是当日同僚一案参革的张如圭。他系此地
人,革后家居,今打听得都中奏准起复旧员之信,他便四下里寻情找门路,忽遇见
雨村,故忙道喜。二人见了礼,张如圭便将此信告知雨村,雨村欢喜,忙忙叙了两
句,各自别去回家。冷子兴听得此言,便忙献计,令雨村央求林如海,转向都中去
央烦贾政。雨村领其意而别,回至馆中,忙寻邸报看真确了,次日面谋之如海。如
海道:“天缘凑巧,因贱荆去世,都中家岳母念及小女无人依傍,前已遣了男女船
只来接,因小女未曾大痊,故尚未行,此刻正思送女进京。因向蒙教训之恩,未经
酬报,遇此机会岂有不尽心图报之理。弟已预筹之,修下荐书一封,托内兄务为周
全,方可稍尽弟之鄙诚;即有所费,弟于内家信中写明,不劳吾兄多虑。”雨村一
面打恭,谢不释口,一面又问:“不知令亲大人现居何职?只怕晚生草率,不敢进
谒。”如海笑道:“若论舍亲,与尊兄犹系一家,乃荣公之孙:大内兄现袭一等将
军之职,名赦,字恩侯;二内兄名政,字存周,现任工部员外郎,其为人谦恭厚道,
大有祖父遗风,非膏粱轻薄之流。故弟致书烦托,否则不但有污尊兄清操,即弟亦
不屑为矣。”雨村听了,心下方信了昨日子兴之言,于是又谢了林如海。如海又说:
“择了出月初二日小女入都,吾兄即同路而往,岂不两便?”雨村唯唯听命,心中
十分得意。如海遂打点礼物并饯行之事,雨村一一领了。

那女学生原不忍离亲而去,无奈他外祖母必欲其往,且兼如海说:“汝父年已
半百,再无续室之意,且汝多病,年又极小,上无亲母教养,下无姊妹扶持。今去
依傍外祖母及舅氏姊妹,正好减我内顾之忧,如何不去?”黛玉听了,方洒泪拜别,
随了奶娘及荣府中几个老妇登舟而去。雨村另有船只,带了两个小童,依附黛玉而
行。

一日到了京都,雨村先整了衣冠,带着童仆,拿了宗侄的名帖至荣府门上投了。
彼时贾政已看了妹丈之书,即忙请入相会。见雨村像貌魁伟,言谈不俗,且这贾政
最喜的是读书人,礼贤下士,拯溺救危,大有祖风,况又系妹丈致意,因此优待雨
村,更又不同。便极力帮助,题奏之日,谋了一个复职。不上两月,便选了金陵应
天府,辞了贾政,择日到任去了,不在话下。

且说黛玉自那日弃舟登岸时,便有荣府打发轿子并拉行李车辆伺候。这黛玉尝
听得母亲说,他外祖母家与别人家不同。他近日所见的这几个三等的仆妇,吃穿用
度已是不凡,何况今至其家,都要步步留心,时时在意,不要多说一句话,不可多
行一步路,恐被人耻笑了去。自上了轿,进了城,从纱窗中瞧了一瞧,其街市之繁
华,人烟之阜盛,自非别处可比。又行了半日,忽见街北蹲着两个大石狮子,三间
兽头大门,门前列坐着十来个华冠丽服之人,正门不开,只东西两角门有人出入。
正门之上有一匾,匾上大书“敕造宁国府”五个大字。黛玉想道:“这是外祖的长
房了。”又往西不远,照样也是三间大门,方是“荣国府”,却不进正门,只由西
角门而进。轿子抬着走了一箭之远,将转弯时便歇了轿,后面的婆子也都下来了,
另换了四个眉目秀洁的十七八岁的小厮上来,抬着轿子,众婆子步下跟随。至一垂
花门前落下,那小斯俱肃然退出,众婆子上前打起轿帘,扶黛玉下了轿。黛玉扶着
婆子的手进了垂花门,两边是超手游廊,正中是穿堂,当地放着一个紫檀架子大理
石屏风。转过屏风,小小三间厅房,厅后便是正房大院。正面五间上房,皆是雕梁
画栋,两边穿山游廊厢房,挂着各色鹦鹉画眉等雀鸟。台阶上坐着几个穿红着绿的
丫头,一见他们来了,都笑迎上来道:“刚才老太太还念诵呢!可巧就来了。”于
是三四人争着打帘子。一面听得人说:“林姑娘来了!”

黛玉方进房,只见两个人扶着一位鬓发如银的老母迎上来。黛玉知是外祖母
了,正欲下拜,早被外祖母抱住,搂入怀中,“心肝儿肉”叫着大哭起来。当下侍
立之人无不下泪,黛玉也哭个不休。众人慢慢解劝,那黛玉方拜见了外祖母。贾母
方一一指与黛玉道:“这是你大舅母。这是二舅母。这是你先前珠大哥的媳妇珠大
嫂子。”黛玉一一拜见。贾母又叫:“请姑娘们。今日远客来了,可以不必上学去。”
众人答应了一声,便去了两个。不一时,只见三个奶妈并五六个丫鬟,拥着三位姑
娘来了。第一个肌肤微丰,身材合中,腮凝新荔,鼻腻鹅脂,温柔沉默,观之可亲。
第二个削肩细腰,长挑身材,鸭蛋脸儿,俊眼修眉,顾盼神飞,文彩精华,见之忘
俗。第三个身量未足,形容尚小。其钗环裙袄,三人皆是一样的妆束。黛玉忙起身
迎上来见礼,互相厮认,归了坐位。丫鬟送上茶来。不过叙些黛玉之母如何得病,
如何请医服药,如何送死发丧。不免贾母又伤感起来,因说:“我这些女孩儿,所
疼的独有你母亲。今一旦先我而亡,不得见面,怎不伤心!”说着携了黛玉的手又
哭起来。众人都忙相劝慰,方略略止住。

众人见黛玉年纪虽小,其举止言谈不俗,身体面貌虽弱不胜衣,却有一段风流
态度,便知他有不足之症。因问:“常服何药?为何不治好了?”黛玉道:“我自
来如此,从会吃饭时便吃药,到如今了,经过多少名医,总未见效。那一年我才三
岁,记得来了一个癞头和尚,说要化我去出家。我父母自是不从,他又说:‘既舍
不得他,但只怕他的病一生也不能好的!若要好时,除非从此以后总不许见哭声,
除父母之外,凡有外亲一概不见,方可平安了此一生。’这和尚疯疯癫癫说了这些
不经之谈,也没人理他。如今还是吃人参养荣丸。”贾母道:“这正好,我这里正
配丸药呢,叫他们多配一料就是了。”

一语未完,只听后院中有笑语声,说:“我来迟了,没得迎接远客!”黛玉思
忖道:“这些人个个皆敛声屏气如此,这来者是谁,这样放诞无礼?”心下想时,
只见一群媳妇丫鬟拥着一个丽人从后房进来。这个人打扮与姑娘们不同,彩绣辉
煌,恍若神妃仙子。头上戴着金丝八宝攒珠髻,绾着朝阳五凤挂珠钗,项上戴着赤
金盘螭缨络圈,身上穿着缕金百蝶穿花大红云缎窄袄,外罩五彩刻丝石青银鼠
褂,下着翡翠撒花洋绉裙。一双丹凤三角眼,两弯柳叶掉梢眉,身量苗条,体格风
骚,粉面含春威不露,丹唇未启笑先闻。黛玉连忙起身接见。贾母笑道:“你不认
得他:他是我们这里有名的一个泼辣货,南京所谓‘辣子’,你只叫他‘凤辣子’
就是了。”黛玉正不知以何称呼,众姊妹都忙告诉黛玉道:“这是琏二嫂子。”黛
玉虽不曾识面,听见他母亲说过:大舅贾赦之子贾琏,娶的就是二舅母王氏的内侄
女;自幼假充男儿教养,学名叫做王熙凤。黛玉忙陪笑见礼,以“嫂”呼之。

这熙凤携着黛玉的手,上下细细打量一回,便仍送至贾母身边坐下,因笑道:
“天下真有这样标致人儿!我今日才算看见了!况且这通身的气派竟不像老祖宗的
外孙女儿,竟是嫡亲的孙女儿似的,怨不得老祖宗天天嘴里心里放不下。只可怜我
这妹妹这么命苦,怎么姑妈偏就去世了呢!”说着便用帕拭泪。贾母笑道:“我才
好了,你又来招我。你妹妹远路才来,身子又弱,也才劝住了,快别再提了。”熙
凤听了,忙转悲为喜道:“正是呢!我一见了妹妹,一心都在他身上,又是喜欢,
又是伤心,竟忘了老祖宗了,该打,该打!”又忙拉着黛玉的手问道:“妹妹几岁
了?可也上过学?现吃什么药?在这里别想家,要什么吃的、什么玩的,只管告诉我。
丫头老婆们不好,也只管告诉我。”黛玉一一答应。一面熙凤又问人:“林姑娘的
东西可搬进来了?带了几个人来?你们赶早打扫两间屋子,叫他们歇歇儿去。”说话
时已摆了果茶上来,熙凤亲自布让。又见二舅母问他:“月钱放完了没有?”熙凤
道:“放完了。刚才带了人到后楼上找缎子,找了半日也没见昨儿太太说的那个。
想必太太记错了。”王夫人道:“有没有,什么要紧。”因又说道:“该随手拿出
两个来给你这妹妹裁衣裳啊。等晚上想着再叫人去拿罢。”熙凤道:“我倒先料着
了。知道妹妹这两日必到,我已经预备下了,等太太回去过了目,好送来。”王夫
人一笑,点头不语。

当下茶果已撤,贾母命两个老嬷嬷带黛玉去见两个舅舅去。维时贾赦之妻邢氏
忙起身笑回道:“我带了外甥女儿过去,到底便宜些。”贾母笑道:“正是呢。你
也去罢,不必过来了。”那邢夫人答应了,遂带着黛玉和王夫人作辞,大家送至穿
堂。垂花门前早有众小厮拉过一辆翠幄清油车来,邢夫人携了黛玉坐上,众老婆们
放下车帘,方命小厮们抬起。拉至宽处,驾上驯骡,出了西角门往东,过荣府正门,
入一黑油漆大门内,至仪门前方下了车。邢夫人挽着黛玉的手进入院中,黛玉度其
处必是荣府中之花园隔断过来的。进入三层仪门,果见正房、厢房、游廊,悉皆小
巧别致,不似那边的轩峻壮丽,且院中随处之树木山石皆好。及进入正室,早有许
多艳妆丽服之姬妾丫鬟迎着。邢夫人让黛玉坐了,一面令人到外书房中请贾赦。一
时回来说:“老爷说:‘连日身上不好,见了姑娘彼此伤心,暂且不忍相见。劝姑
娘不必伤怀想家,跟着老太太和舅母,是和家里一样的。姐妹们虽拙,大家一处作
伴,也可以解些烦闷。或有委屈之处,只管说,别外道了才是。’”黛玉忙站起身
来,一一答应了。再坐一刻便告辞,邢夫人苦留吃过饭去。黛玉笑回道:“舅母爱
惜赐饭,原不应辞,只是还要过去拜见二舅舅,恐去迟了不恭,异日再领:望舅母
容谅。”邢夫人道:“这也罢了。”遂命两个嬷嬷用方才坐来的车送过去。于是黛
玉告辞。邢夫人送至仪门前,又嘱咐了众人几句,眼看着车去了方回来。

一时黛玉进入荣府,下了车,只见一条大甬路直接出大门来。众嬷嬷引着便往
东转弯,走过一座东西穿堂、向南大厅之后,仪门内大院落,上面五间大正房,两
边厢房鹿顶,耳门钻山,四通八达,轩昂壮丽,比各处不同。黛玉便知这方是正内
室。进入堂屋,抬头迎面先见一个赤金九龙青地大匾,匾上写着斗大三个字,是“荣
禧堂”;后有一行小字:“某年月日书赐荣国公贾源”,又有“万几宸翰”之宝。
大紫檀雕螭案上设着三尺多高青绿古铜鼎,悬着待漏随朝墨龙大画,一边是錾金
彝,一边是玻璃盆。地下两溜十六张楠木圈椅。又有一副对联,乃是乌木联牌镶着
錾金字迹,道是:
座上珠玑昭日月,
堂前黼黻焕烟霞。
下面一行小字是:“世教弟勋袭东安郡王穆莳拜手书。”原来王夫人时常居坐宴息
也不在这正室中,只在东边的三间耳房内。于是嬷嬷们引黛玉进东房门来。临窗大
炕上铺着猩红洋毯,正面设着大红金钱蟒引枕,秋香色金钱蟒大条褥,两边设一对
梅花式洋漆小几,左边几上摆着文王鼎,鼎旁匙箸香盒,右边几上摆着汝窑美人觚,
里面插着时鲜花草。地下面西一溜四张大椅,都搭着银红撒花椅搭,底下四副脚踏;
两边又有一对高几,几上茗碗瓶花俱备。其馀陈设,不必细说。老嬷嬷让黛玉上炕
坐。炕沿上却也有两个锦褥对设。黛玉度其位次,便不上炕,只就东边椅上坐了。
本房的丫鬟忙捧上茶来。黛玉一面吃了,打量这些丫鬟们妆饰衣裙、举止行动,果
与别家不同。

茶未吃了,只见一个穿红绫袄青绸掐牙背心的一个丫鬟走来笑道:“太太说:
请林姑娘到那边坐罢。”老嬷嬷听了,于是又引黛玉出来,到了东南三间小正房内。
正面炕上横设一张炕桌,上面堆着书籍茶具,靠东壁面西设着半旧的青缎靠背引
枕。王夫人却坐在西边下首,亦是半旧青缎靠背坐褥,见黛玉来了,便往东让。黛
玉心中料定这是贾政之位,因见挨炕一溜三张椅子上也搭着半旧的弹花椅袱,黛玉
便向椅上坐了。王夫人再三让他上炕,他方挨王夫人坐下。王夫人因说:“你舅舅
今日斋戒去了,再见罢。只是有句话嘱咐你:你三个姐妹倒都极好,以后一处念书
认字,学针线,或偶一玩笑,却都有个尽让的。我就只一件不放心:我有一个孽根
祸胎,是家里的‘混世魔王’,今日因往庙里还愿去,尚未回来,晚上你看见就知
道了。你以后总不用理会他,你这些姐姐妹妹都不敢沾惹他的。”黛玉素闻母亲说
过,有个内侄乃衔玉而生,顽劣异常,不喜读书,最喜在内帏厮混,外祖母又溺爱,
无人敢管。今见王夫人所说,便知是这位表兄,一面陪笑道:“舅母所说,可是衔
玉而生的?在家时记得母亲常说,这位哥哥比我大一岁,小名就叫宝玉,性虽憨顽,
说待姊妹们却是极好的。况我来了,自然和姊妹们一处,弟兄们是另院别房,岂有
沾惹之理?”王夫人笑道:“你不知道原故:他和别人不同,自幼因老太太疼爱,
原系和姐妹们一处娇养惯了的。若姐妹们不理他,他倒还安静些;若一日姐妹们和
他多说了一句话,他心上一喜,便生出许多事来。所以嘱咐你别理会他。他嘴里一
时甜言蜜语,一时有天没日,疯疯傻傻,只休信他。”黛玉一一的都答应着。

忽见一个丫鬟来说:“老太太那里传晚饭了。”王夫人忙携了黛玉出后房门,
由后廊往西。出了角门,是一条南北甬路,南边是倒座三间小小抱厦厅,北边立着
一个粉油大影壁,后有一个半大门,小小一所房屋。王夫人笑指向黛玉道:“这是
你凤姐姐的屋子。回来你好往这里找他去,少什么东西只管和他说就是了。”这院
门上也有几个才总角的小厮,都垂手侍立。王夫人遂携黛玉穿过一个东西穿堂,便
是贾母的后院了。于是进入后房门,已有许多人在此伺候,见王夫人来,方安设桌
椅。贾珠之妻李氏捧杯,熙凤安箸,王夫人进羹。贾母正面榻上独坐,两旁四张空
椅。熙凤忙拉黛玉在左边第一张椅子上坐下,黛玉十分推让。贾母笑道:“你舅母
和嫂子们是不在这里吃饭的。你是客,原该这么坐。”黛玉方告了坐,就坐了。贾
母命王夫人也坐了。迎春姊妹三个告了坐方上来,迎春坐右手第一,探春左第二,
惜春右第二。旁边丫鬟执着拂尘、漱盂、巾帕,李纨、凤姐立于案边布让;外间伺
候的媳妇丫鬟虽多,却连一声咳嗽不闻。饭毕,各各有丫鬟用小茶盘捧上茶来。当
日林家教女以惜福养身,每饭后必过片时方吃茶,不伤脾胃;今黛玉见了这里许多
规矩,不似家中,也只得随和些,接了茶。又有人捧过漱盂来,黛玉也漱了口,又
盥手毕。然后又捧上茶来,这方是吃的茶。贾母便说:“你们去罢,让我们自在说
说话儿。”王夫人遂起身,又说了两句闲话儿,方引李、凤二人去了。

贾母因问黛玉念何书。黛玉道:“刚念了《四书》。”黛玉又问姊妹们读何书,
贾母道:“读什么书,不过认几个字罢了。”一语未了,只听外面一阵脚步响,丫
鬟进来报道:“宝玉来了。”黛玉心想,这个宝玉不知是怎样个惫懒人呢。及至进
来一看,却是位青年公子:头上戴着束发嵌宝紫金冠,齐眉勒着二龙戏珠金抹额,
一件二色金百蝶穿花大红箭袖,束着五彩丝攒花结长穗宫绦,外罩石青起花八团倭
缎排穗褂,登着青缎粉底小朝靴。面若中秋之月,色如春晓之花,鬓若刀裁,眉如
墨画,鼻如悬胆,睛若秋波,虽怒时而似笑,即视而有情。项上金螭缨络,又有
一根五色丝绦,系着一块美玉。黛玉一见便吃一大惊,心中想道:“好生奇怪,倒
像在那里见过的,何等眼熟!”只见这宝玉向贾母请了安,贾母便命:“去见你娘
来。”即转身去了。一回再来时,已换了冠带,头上周围一转的短发都结成小辫,
红丝结束,共攒至顶中胎发,总编一根大辫,黑亮如漆,从顶至梢,一串四颗大珠,
用金八宝坠脚。身上穿着银红撒花半旧大袄,仍旧带着项圈、宝玉、寄名锁、护身
符等物,下面半露松绿撒花绫裤,锦边弹墨袜,厚底大红鞋。越显得面如傅粉,唇
若施脂,转盼多情,语言若笑。天然一段风韵,全在眉梢;平生万种情思,悉堆眼
角。看其外貌最是极好,却难知其底细,后人有《西江月》二词,批的极确。词曰:

无故寻愁觅恨,有时似傻如狂。纵然生得好皮囊,腹内原
来草莽。潦倒不通庶务,愚顽怕读文章。行为偏僻性乖张,那管世人诽谤。
又曰:

富贵不知乐业,贫穷难耐凄凉。可怜辜负好时光,于国于家无望。天下无能第
一,古今不肖无双。寄言纨与膏粱:莫效此儿形状!

却说贾母见他进来,笑道:“外客没见就脱了衣裳了,还不去见你妹妹呢。”
宝玉早已看见了一个袅袅婷婷的女儿,便料定是林姑妈之女,忙来见礼。归了坐细
看时,真是与众各别。只见:

两弯似蹙非蹙笼烟眉,一双似喜非喜含情目。态生两靥之愁,娇袭一身之病。
泪光点点,娇喘微微。闲静似娇花照水,行动如弱柳扶风。心较比干多一窍,病如
西子胜三分。
宝玉看罢,笑道:“这个妹妹我曾见过的。”贾母笑道:“又胡说了,你何曾见过?”
宝玉笑道:“虽没见过,却看着面善,心里倒像是远别重逢的一般。”贾母笑道:
“好,好!这么更相和睦了。”

宝玉便走向黛玉身边坐下,又细细打量一番,因问:“妹妹可曾读书?”黛玉
道:“不曾读书,只上了一年学,些须认得几个字。”宝玉又道:“妹妹尊名?”
黛玉便说了名,宝玉又道:“表字?”黛玉道:“无字。”宝玉笑道:“我送妹妹
一字:莫若‘颦颦’二字极妙。”探春便道:“何处出典?”宝玉道:“《古今人
物通考》上说:‘西方有石名黛,可代画眉之墨。’况这妹妹眉尖若蹙,取这个字
岂不美?”探春笑道:“只怕又是杜撰。”宝玉笑道:“除了《四书》,杜撰的也
太多呢。”因又问黛玉:“可有玉没有?”众人都不解。黛玉便忖度着:“因他有
玉,所以才问我的。”便答道:“我没有玉。你那玉也是件稀罕物儿,岂能人人皆
有?”宝玉听了,登时发作起狂病来,摘下那玉就狠命摔去,骂道:“什么罕物!
人的高下不识,还说灵不灵呢!我也不要这劳什子!”吓的地下众人一拥争去拾玉。
贾母急的搂了宝玉道“孽障!你生气要打骂人容易,何苦摔那命根子!”宝玉满面
泪痕哭道:“家里姐姐妹妹都没有,单我有,我说没趣儿;如今来了这个神仙似的
妹妹也没有,可知这不是个好东西。”贾母忙哄他道:“你这妹妹原有玉来着。因
你姑妈去世时,舍不得你妹妹,无法可处,遂将他的玉带了去,一则全殉葬之礼,
尽你妹妹的孝心;二则你姑妈的阴灵儿也可权作见了你妹妹了。因此他说没有,也
是不便自己夸张的意思啊。你还不好生带上,仔细你娘知道!”说着便向丫鬟手中
接来亲与他带上。宝玉听如此说,想了一想,也就不生别论。

当下奶娘来问黛玉房舍,贾母便说:“将宝玉挪出来,同我在套间暖阁里,把
你林姑娘暂且安置在碧纱厨里。等过了残冬,春天再给他们收拾房屋,另作一番安
置罢。”宝玉道:“好祖宗,我就在碧纱厨外的床上很妥当。又何必出来,闹的老
祖宗不得安静呢?”贾母想一想说:“也罢了。”每人一个奶娘并一个丫头照管,
馀者在外间上夜听唤。一面早有熙凤命人送了一顶藕合色花帐并锦被缎褥之类。黛
玉只带了两个人来,一个是自己的奶娘王嬷嬷,一个是十岁的小丫头,名唤雪雁。
贾母见雪雁甚小,一团孩气,王嬷嬷又极老,料黛玉皆不遂心,将自己身边一个二
等小丫头名唤鹦哥的与了黛玉。亦如迎春等一般,每人除自幼乳母外,另有四个教
引嬷嬷,除贴身掌管钗钏盥沐两个丫头外,另有四五个洒扫房屋来往使役的小丫
头。当下王嬷嬷与鹦哥陪侍黛玉在碧纱厨内,宝玉乳母李嬷嬷并大丫头名唤袭人的
陪侍在外面大床上。原来这袭人亦是贾母之婢,本名蕊珠,贾母因溺爱宝玉,恐宝
玉之婢不中使,素喜蕊珠心地纯良,遂与宝玉。宝玉因知他本姓花,又曾见旧人诗
句有“花气袭人”之句,遂回明贾母,即把蕊珠更名袭人。

却说袭人倒有些痴处:伏侍贾母时,心中只有贾母;如今跟了宝玉,心中又只
有宝玉了。只因宝玉性情乖僻,每每规谏,见宝玉不听,心中着实忧郁。是晚宝玉
李嬷嬷已睡了,他见里面黛玉鹦哥犹未安歇,他自卸了妆,悄悄的进来,笑问:“姑
娘怎么还不安歇?”黛玉忙笑让:“姐姐请坐。”袭人在床沿上坐了。鹦哥笑道:
“林姑娘在这里伤心,自己淌眼抹泪的,说:‘今儿才来了,就惹出你们哥儿的病
来。倘或摔坏了那玉,岂不是因我之过!’所以伤心,我好容易劝好了。”袭人道:
“姑娘快别这么着!将来只怕比这更奇怪的笑话儿还有呢。若为他这种行状你多心
伤感,只怕你还伤感不了呢。快别多心。”黛玉道:“姐姐们说的,我记着就是了。”
又叙了一回,方才安歇。

次早起来省过贾母,因往王夫人处来。正值王夫人与熙凤在一处拆金陵来的书
信,又有王夫人的兄嫂处遣来的两个媳妇儿来说话。黛玉虽不知原委,探春等却晓
得是议论金陵城中居住的薛家姨母之子——表兄薛蟠,倚财仗势,打死人命,现在
应天府案下审理。如今舅舅王子腾得了信,遣人来告诉这边,意欲唤取进京之意。

毕竟怎的,下回分解。