\chapter{王熙凤毒设相思局~贾天祥正照风月鉴}

话说凤姐正与平儿说话,只见有人回说:“瑞大爷来了。”凤姐命:“请进来
罢。”贾瑞见请,心中暗喜,见了凤姐,满面陪笑,连连问好。凤姐儿也假意殷勤
让坐让茶。贾瑞见凤姐如此打扮,越发酥倒,因饧了眼问道:“二哥哥怎么还不回
来?”凤姐道:“不知什么缘故。”贾瑞笑道:“别是路上有人绊住了脚,舍不得
回来了罢?”凤姐道:“可知男人家见一个爱一个也是有的。”贾瑞笑道:“嫂子
这话错了,我就不是这样人。”凤姐笑道:“像你这样的人能有几个呢,十个里也
挑不出一个来!”贾瑞听了,喜的抓耳挠腮,又道:“嫂子天天也闷的很。”凤姐
道:“正是呢,只盼个人来说话解解闷儿。”贾瑞笑道:“我倒天天闲着。若天天
过来替嫂子解解闷儿,可好么?”凤姐笑道:“你哄我呢!你那里肯往我这里来?”
贾瑞道:“我在嫂子面前若有一句谎话,天打雷劈!只因素日闻得人说,嫂子是个
利害人,在你跟前一点也错不得,所以唬住我了。我如今见嫂子是个有说有笑极疼
人的,我怎么不来?死了也情愿。”凤姐笑道:“果然你是个明白人,比蓉儿兄弟
两个强远了。我看他那样清秀,只当他们心里明白,谁知竟是两个糊涂虫,一点不
知人心。”

贾瑞听这话,越发撞在心坎上,由不得又往前凑一凑,觑着眼看凤姐的荷包,
又问:“戴着什么戒指?”凤姐悄悄的道:“放尊重些,别叫丫头们看见了。”贾
瑞如听纶音佛语一般,忙往后退。凤姐笑道:“你该去了。”贾瑞道:“我再坐一
坐儿,好狠心的嫂子!”凤姐儿又悄悄的道:“大天白日人来人往,你就在这里也
不方便。你且去,等到晚上起了更你来,悄悄的在西边穿堂儿等我。”贾瑞听了,
如得珍宝,忙问道:“你别哄我。但是那里人过的多,怎么好躲呢?”凤姐道:“你
只放心,我把上夜的小厮们都放了假,两边门一关,再没别人了。”贾瑞听了,喜
之不尽,忙忙的告辞而去,心内以为得手。

盼到晚上,果然黑地里摸入荣府,趁掩门时钻入穿堂。果见漆黑无一人来往,
贾母那边去的门已倒锁了,只有向东的门未关。贾瑞侧耳听着,半日不见人来。忽
听咯噔一声,东边的门也关上了。贾瑞急的也不敢则声,只得悄悄出来,将门撼了
撼,关得铁桶一般。此时要出去亦不能了,南北俱是大墙,要跳也无攀援。这屋内
又是过堂风,空落落的,现是腊月天气,夜又长,朔风凛凛,侵肌裂骨,一夜几乎
不曾冻死。好容易盼到早晨,只见一个老婆子先将东门开了进来,去叫西门,贾瑞
瞅他背着脸,一溜烟抱了肩跑出来。幸而天气尚早,人都未起,从后门一径跑回家
去。

原来贾瑞父母早亡,只有他祖父代儒教养。那代儒素日教训最严,不许贾瑞多
走一步,生怕他在外吃酒赌钱,有误学业。今忽见他一夜不归,只料定他在外非饮
即赌,嫖娼宿妓,那里想到这段公案?因此也气了一夜。贾瑞也捻着一把汗,少不
得回来撒谎,只说:“往舅舅家去了,天黑了,留我住了一夜。”代儒道:“自来
出门非禀我不敢擅出,如何昨日私自去了?据此也该打,何况是撒谎!”因此发狠,
按倒打了三四十板,还不许他吃饭,叫他跪在院内读文章,定要补出十天工课来方
罢。贾瑞先冻了一夜,又挨了打,又饿着肚子,跪在风地里念文章,其苦万状。

此时贾瑞邪心未改,再不想到凤姐捉弄他。过了两日,得了空儿,仍找寻凤姐。
凤姐故意抱怨他失信,贾瑞急的起誓。凤姐因他自投罗网,少不的再寻别计令他知
改,故又约他道:“今日晚上,你别在那里了,你在我这房后小过道儿里头那间空
屋子里等我。——可别冒撞了!”贾瑞道:“果真么?”凤姐道:“你不信就别来!”
贾瑞道:“必来,必来!死也要来的。”凤姐道:“这会子你先去罢。”贾瑞料定
晚间必妥,此时先去了。凤姐在这里便点兵派将,设下圈套。

那贾瑞只盼不到晚,偏偏家里亲戚又来了,吃了晚饭才去,那天已有掌灯时候;
又等他祖父安歇,方溜进荣府,往那夹道中屋子里来等着,热锅上蚂蚁一般。只是
左等不见人影,右听也没声响,心中害怕,不住猜疑道:“别是不来了,又冻我一
夜不成?”正自胡猜,只见黑的进来一个人。贾瑞便打定是凤姐,不管青红皂
白,那人刚到面前,便如饿虎扑食、猫儿捕鼠的一般抱住,叫道:“亲嫂子,等死
我了!”说着,抱到屋里炕上就亲嘴扯裤子,满口里“亲爹”“亲娘”的乱叫起来。
那人只不做声,贾瑞便扯下自己的裤子来,硬帮帮就想顶入。忽然灯光一闪,只见
贾蔷举着个蜡台,照道:“谁在这屋里呢?”只见炕上那人笑道:“瑞大叔要我
呢!”

贾瑞不看则已,看了时真臊的无地可入。你道是谁?却是贾蓉。贾瑞回身要跑,
被贾蔷一把揪住道:“别走!如今琏二婶子已经告到太太跟前,说你调戏他,他暂
时稳住你在这里。太太听见气死过去了,这会子叫我来拿你。快跟我走罢!”贾瑞
听了,魂不附体,只说:“好侄儿!你只说没有我,我明日重重的谢你!”贾蔷道:
“放你不值什么,只不知你谢我多少?况且口说无凭,写一张文契才算。”贾瑞道:
“这怎么落纸呢?”贾蔷道:“这也不妨,写个赌钱输了,借银若干两,就完了。”
贾瑞道:“这也容易。”贾蔷翻身出来,纸笔现成,拿来叫贾瑞写。他两个做好做
歹,只写了五十两银子,画了押,贾蔷收起来。然后撕掳贾蓉。贾蓉先咬定牙不依,
只说:“明日告诉族中的人评评理。”贾瑞急的至于磕头。贾蔷做好做歹的,也写
了一张五十两欠契才罢。贾蔷又道:“如今要放你,我就担着不是。老太太那边的
门早已关了。老爷正在厅上看南京来的东西,那一条路定难过去。如今只好走后门。
要这一走,倘或遇见了人,连我也不好。等我先去探探,再来领你。这屋里你还藏
不住,少时就来堆东西,等我寻个地方。”说毕,拉着贾瑞,仍息了灯,出至院外,
摸着大台阶底下,说道:“这窝儿里好。只蹲着,别哼一声。等我来再走。”说毕,
二人去了。

贾瑞此时身不由己,只得蹲在那台阶下。正要盘算,只听头顶上一声响,哗喇
喇一净桶尿粪从上面直泼下来,可巧浇了他一身一头。贾瑞掌不住“嗳哟”一声,
忙又掩住口,不敢声张,满头满脸皆是尿屎,浑身冰冷打战。只见贾蔷跑来叫:“快
走,快走!”贾瑞方得了命,三步两步从后门跑到家中,天已三更,只得叫开了门。
家人见他这般光景,问:“是怎么了?”少不得撒谎说:“天黑了,失脚掉在茅厕
里了。”一面即到自己房中更衣洗濯。心下方想到凤姐玩他,因此发一回狠。再想
想凤姐的模样儿标致,又恨不得一时搂在怀里。胡思乱想,一夜也不曾合眼。自此
虽想凤姐,只不敢往荣府去了。

贾蓉等两个常常来要银子,他又怕祖父知道。正是相思尚且难禁,况又添了债
务,日间工课又紧;他二十来岁的人,尚未娶亲,想着凤姐不得到手,自不免有些
“指头儿告了消乏”;更兼两回冻恼奔波:因此三五下里夹攻,不觉就得了一病:
心内发膨胀,口内无滋味,脚下如绵,眼中似醋,黑夜作烧,白日常倦,下溺遗精,
嗽痰带血,诸如此症,不上一年都添全了。于是不能支持,一头躺倒,合上眼还只
梦魂颠倒,满口胡话,惊怖异常。百般请医疗治,诸如肉桂、附子、鳖甲、麦冬、
玉竹等药吃了有几十斤下去,也不见个动静。

倏又腊尽春回,这病更加沉重。代儒也着了忙,各处请医疗治,皆不见效。因
后来吃“独参汤”,代儒如何有这力量,只得往荣府里来寻。王夫人命凤姐秤二两
给他。凤姐回说:“前儿新近替老太太配了药,那整的太太又说留着送杨提督的太
太配药,偏偏昨儿我已经叫人送了去了。”王夫人道:“就是咱们这边没了,你叫
个人往你婆婆那里问问,或是你珍大哥哥那里有,寻些来凑着给人家。吃好了,救
人一命,也是你们的好处。”凤姐应了,也不遣人去寻。只将些渣末凑了几钱,命
人送去,只说:“太太叫送来的,再也没了。”然后向王夫人说:“都寻了来了,
共凑了二两多,送去了。”

那贾瑞此时要命心急,无药不吃,只是白花钱不见效。忽然这日有个跛足道人
来化斋,口称专治冤孽之症。贾瑞偏偏在内听见了,直着声叫喊,说:“快去请进
那位菩萨来救命!”一面在枕头上磕头。众人只得带进那道士来。贾瑞一把拉住,
连叫“菩萨救我!”那道士叹道:“你这病非药可医。我有个宝贝与你,你天天看
时,此命可保矣。”说毕,从搭裢中取出个正面反面皆可照人的镜子来,——背上
錾着“风月宝鉴”四字,——递与贾瑞道:“这物出自太虚幻境空灵殿上,警幻仙
子所制,专治邪思妄动之症,有济世保生之功。所以带他到世上来,单与那些聪明
俊秀、风雅王孙等照看。千万不可照正面,只照背面,要紧,要紧!三日后我来收
取,管叫你病好。”说毕,徉长而去。众人苦留不住。

贾瑞接了镜子,想道:“这道士倒有意思,我何不照一照试试?”想毕,拿起
那“宝鉴”来,向反面一照。只见一个骷髅儿,立在里面。贾瑞忙掩了,骂那道士:
“混帐!如何吓我!我倒再照照正面是什么?”想着,便将正面一照,只见凤姐站在
里面点手儿叫他。贾瑞心中一喜,荡悠悠觉得进了镜子,与凤姐云雨一番,凤姐仍
送他出来。到了床上,“嗳哟”了一声,一睁眼,镜子从新又掉过来,仍是反面立
着一个骷髅。贾瑞自觉汗津津的,底下已遗了一滩精。心中到底不足,又翻过正面
来,只见凤姐还招手叫他,他又进去:如此三四次。到了这次,刚要出镜子来,只
见两个人走来,拿铁锁把他套住,拉了就走。贾瑞叫道:“让我拿了镜子再走——”
只说这句就再不能说话了。

旁边伏侍的人只见他先还拿着镜子照,落下来,仍睁开眼拾在手内,末后镜子
掉下来,便不动了。众人上来看时,已经咽了气了,身子底下冰凉精湿遗下了一大
滩精。这才忙着穿衣抬床。代儒夫妇哭的死去活来,大骂道士:“是何妖道!”遂
命人架起火来烧那镜子。只听空中叫道:“谁叫他自己照了正面呢!你们自己以假
为真,为何烧我此镜?”忽见那镜从房中飞出。代儒出门看时,却还是那个跛足道
人,喊道:“还我的风月宝鉴来!”说着,抢了镜子,眼看着他飘然去了。

当下代儒没法,只得料理丧事,各处去报。三日起经,七日发引,寄灵铁槛寺
后。一时贾家众人齐来吊问。荣府贾赦赠银二十两,贾政也是二十两,宁府贾珍亦
有二十两,其馀族中人贫富不一,或一二两、三四两不等。外又有各同窗家中分资,
也凑了二三十两。代儒家道虽然淡薄,得此帮助,倒也丰丰富富完了此事。

谁知这年冬底,林如海因为身染重疾,写书来特接黛玉回去。贾母听了,未免
又加忧闷,只得忙忙的打点黛玉起身。宝玉大不自在,争奈父女之情,也不好拦阻。
于是贾母定要贾琏送他去,仍叫带回来。一应土仪盘费,不消絮说,自然要妥贴的。
作速择了日期,贾琏同着黛玉辞别了众人,带领仆从,登舟往扬州去了。

要知端的,且听下回分解。