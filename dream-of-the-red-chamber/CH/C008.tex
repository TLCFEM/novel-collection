\chapter{贾宝玉奇缘识金锁~薛宝钗巧合认通灵}

话说宝玉和凤姐回家,见过众人,宝玉便回明贾母要约秦钟上家塾之事,自己
也有个伴读的朋友,正好发愤;又着实称赞秦钟人品行事,最是可人怜爱的。凤姐
又在一旁帮着说:“改日秦钟还来拜见老祖宗呢。”说的贾母喜欢起来。凤姐又趁
势请贾母一同过去看戏。贾母虽年高,却极有兴头。后日,尤氏来请,遂带了王夫
人、黛玉、宝玉等过去看戏。至晌午,贾母便回来歇息。王夫人本好清净,见贾母
回来,也就回来了。然后凤姐坐了首席,尽欢至晚而罢。

却说宝玉送贾母回来,待贾母歇了中觉,还要回去看戏,又恐搅的秦氏等人不
便。因想起宝钗近日在家养病,未去看视,意欲去望他。若从上房后角门过去,恐
怕遇见别事缠绕,又怕遇见他父亲,更为不妥,宁可绕个远儿。当下众嬷嬷丫鬟伺
候他换衣服,见不曾换,仍出二门去了,众嬷嬷丫鬟只得跟随出来。还只当他去那
边府中看戏,谁知到了穿堂儿,便向东北边绕过厅后而去。偏顶头遇见了门下清客
相公詹光、单聘仁二人走来,一见了宝玉,便都赶上来笑着,一个抱着腰,一个拉
着手,道:“我的菩萨哥儿!我说做了好梦呢,好容易遇见你了!”说着,又唠叨
了半日才走开。老嬷嬷叫住,因问:“你们二位是往老爷那里去的不是?”二人点
头道:“是。”又笑着说:“老爷在梦坡斋小书房里歇中觉呢,不妨事的。”一面
说,一面走了,说的宝玉也笑了。于是转弯向北奔梨香院来。可巧管库房的总领吴
新登和仓上的头目名叫戴良的,同着几个管事的头目,共七个人从帐房里出来,一
见宝玉,赶忙都一齐垂手站立。独有一个买办名唤钱华,因他多日未见宝玉,忙上
来打千儿请宝玉的安,宝玉含笑伸手叫他起来。众人都笑说:“前儿在一处看见二
爷写的斗方儿,越发好了,多早晚赏我们几张贴贴。”宝玉笑道:“在那里看见了?”
众人道:“好几处都有,都称赞的了不得,还和我们寻呢!”宝玉笑道:“不值什
么,你们说给我的小么儿们就是了。”一面说,一面前走,众人待他过去,方都各
自散了。

闲言少述。且说宝玉来至梨香院中,先进薛姨妈屋里来,见薛姨妈打点针黹与
丫鬟们呢。宝玉忙请了安,薛姨妈一把拉住,抱入怀中笑说:“这么冷天,我的儿,
难为你想着来!快上炕来坐着罢。”命人沏滚滚的茶来。宝玉因问:“哥哥没在家
么?”薛姨妈叹道:“他是没笼头的马,天天逛不了,那里肯在家一日呢?”宝玉
道:“姐姐可大安了?”薛姨妈道:“可是呢,你前儿又想着打发人来瞧他。他在
里间不是,你去瞧。他那里比这里暖和,你那里坐着,我收拾收拾就进来和你说话
儿。”

宝玉听了,忙下炕来到了里间门前,只见吊着半旧的红绸软帘。宝玉掀帘一步
进去,先就看见宝钗坐在炕上作针线,头上挽着黑漆油光的儿,蜜合色的棉袄,
玫瑰紫二色金银线的坎肩儿,葱黄绫子棉裙:一色儿半新不旧的,看去不见奢华,
惟觉雅淡。罕言寡语,人谓装愚;安分随时,自云守拙。宝玉一面看,一面问:“姐
姐可大愈了?”宝钗抬头看见宝玉进来,连忙起身含笑答道:“已经大好了,多谢
惦记着。”说着,让他在炕沿上坐下,即令莺儿:“倒茶来。”一面又问老太太姨
娘安,又问别的姐妹们好。一面看宝玉头上戴着累丝嵌宝紫金冠,额上勒着二龙捧
珠抹额,身上穿着秋香色立蟒白狐腋箭袖,系着五色蝴蝶鸾绦,项上挂着长命锁、
记名符,另外有那一块落草时衔下来的宝玉。宝钗因笑说道:“成日家说你的这块
玉,究竟未曾细细的赏鉴过,我今儿倒要瞧瞧。”说着便挪近前来。宝玉亦凑过去,
便从项上摘下来,递在宝钗手内。宝钗托在掌上,只见大如雀卵,灿若明霞,莹润
如酥,五色花纹缠护。

看官们须知道,这就是大荒山中青埂峰下的那块顽石幻相。后人有诗嘲云:
女娲炼石已荒唐,又向荒唐演大荒。
失去本来真面目,幻来新就臭皮囊。
好知运败金无彩,堪叹时乖玉不光。
白骨如山忘姓氏,无非公子与红妆。
通灵宝玉正面

通灵宝玉反面

那顽石亦曾记下他这幻相并癞僧所镌篆文,今亦按图画于后面。但其真体最小,方
从胎中小儿口中衔下,今若按式画出,恐字迹过于微细,使观者大废眼光,亦非畅
事,所以略展放些,以便灯下醉中可阅。今注明此故,方不至以胎中之儿口有多大、
怎得衔此狼蠢大之物为诮。

宝钗看毕,又从新翻过正面来细看,口里念道:“莫失莫忘,仙寿恒昌。”念
了两遍,乃回头向莺儿笑道:“你不去倒茶,也在这里发呆作什么?”莺儿也嘻嘻
的笑道:“我听这两句话,倒像和姑娘项圈上的两句话是一对儿。”宝玉听了,忙
笑道:“原来姐姐那项圈上也有字?我也赏鉴赏鉴。”宝钗道:“你别听他的话,
没有什么字。”宝玉央及道:“好姐姐,你怎么瞧我的呢!”宝钗被他缠不过,因
说道:“也是个人给了两句吉利话儿,錾上了,所以天天带着。不然沉甸甸的,有
什么趣儿?”一面说,一面解了排扣,从里面大红袄儿上将那珠宝晶莹、黄金灿烂
的璎珞摘出来。宝玉忙托着锁看时,果然一面有四个字,两面八个字,共成两句吉
谶。亦曾按式画下形相。
金锁正面

金锁反面

宝玉看了,也念了两遍,又念自己的两遍,因笑问:“姐姐,这八个字倒和我的是
一对儿。”莺儿笑道:“是个癞头和尚送的,他说必须錾在金器上——”宝钗不等
他说完,便嗔着:“不去倒茶!”一面又问宝玉从那里来。

宝玉此时与宝钗挨肩坐着,只闻一阵阵的香气,不知何味,遂问:“姐姐熏的
是什么香?我竟没闻过这味儿。”宝钗道:“我最怕熏香。好好儿的衣裳,为什么
熏他?”宝玉道:“那么着这是什么香呢?”宝钗想了想,说:“是了,是我早起
吃了冷香丸的香气。”宝玉笑道:“什么‘冷香丸’,这么好闻?好姐姐,给我一
丸尝尝呢。”宝钗笑道:“又混闹了。一个药也是混吃的?”

一语未了,忽听外面人说:“林姑娘来了。”话犹未完,黛玉已摇摇摆摆的进
来,一见宝玉,便笑道:“哎哟!我来的不巧了。”宝玉等忙起身让坐。宝钗笑道:
“这是怎么说?”黛玉道:“早知他来,我就不来了。”宝钗道:“这是什么意思?”
黛玉道:“什么意思呢:来呢一齐来,不来一个也不来;今儿他来,明儿我来,间
错开了来,岂不天天有人来呢?也不至太冷落,也不至太热闹。姐姐有什么不解的
呢?”宝玉因见他外面罩着大红羽缎对襟褂子,便问:“下雪了么?”地下老婆们
说:“下了这半日了。”宝玉道:“取了我的斗篷来。”黛玉便笑道:“是不是?
我来了他就该走了!”宝玉道:“我何曾说要去,不过拿来预备着。”宝玉的奶母
李嬷嬷便说道:“天又下雪,也要看时候儿,就在这里和姐姐妹妹一处玩玩儿罢。
姨太太那里摆茶呢。我叫丫头去取了斗篷来,说给小么儿们散了罢?”宝玉点头。
李嬷嬷出去,命小厮们:“都散了罢。”

这里薛姨妈已摆了几样细巧茶食,留他们喝茶吃果子。宝玉因夸前日在东府里
珍大嫂子的好鹅掌。薛姨妈连忙把自己糟的取了来给他尝。宝玉笑道:“这个就酒
才好!”薛姨妈便命人灌了上等酒来。李嬷嬷上来道:“姨太太,酒倒罢了。”宝
玉笑央道:“好妈妈,我只喝一钟。”李妈道:“不中用,当着老太太、太太,那
怕你喝一坛呢。不是那日我眼错不见,不知那个没调教的只图讨你的喜欢,给了你
一口酒喝,葬送的我挨了两天骂!姨太太不知道他的性子呢,喝了酒更弄性。有一
天老太太高兴,又尽着他喝;什么日子又不许他喝。何苦我白赔在里头呢?”薛姨
妈笑道:“老货!只管放心喝你的去罢。我也不许他喝多了。就是老太太问,有我
呢!”一面命小丫头:“来,让你奶奶去也吃一杯搪搪寒气。”那李妈听如此说,
只得且和众人吃酒去。这里宝玉又说:“不必烫暖了,我只爱喝冷的。”薛姨妈道:
“这可使不得:吃了冷酒,写字手打颤儿。”宝钗笑道:“宝兄弟,亏你每日家杂
学旁收的,难道就不知道酒性最热,要热吃下去,发散的就快;要冷吃下去,便凝
结在内。拿五脏去暖他,岂不受害?从此还不改了呢。快别吃那冷的了。”宝玉听
这话有理,便放下冷的,令人烫来方饮。

黛玉磕着瓜子儿,只管抿着嘴儿笑。可巧黛玉的丫鬟雪雁走来给黛玉送小手炉
儿,黛玉因含笑问他说:“谁叫你送来的?难为他费心。那里就冷死我了呢!”雪
雁道:“紫鹃姐姐怕姑娘冷,叫我送来的。”黛玉接了,抱在怀中,笑道:“也亏
了你倒听他的话!我平日和你说的,全当耳旁风,怎么他说了你就依,比圣旨还快
呢。”宝玉听这话,知是黛玉借此奚落,也无回复之词,只嘻嘻的笑了一阵罢了。
宝钗素知黛玉是如此惯了的,也不理他。薛姨妈因笑道:“你素日身子单弱,禁不
得冷,他们惦记着你倒不好?”黛玉笑道:“姨妈不知道:幸亏是姨妈这里,倘或
在别人家,那不叫人家恼吗?难道人家连个手炉也没有,巴巴儿的打家里送了来?不
说丫头们太小心,还只当我素日是这么轻狂惯了的呢。”薛姨妈道:“你是个多心
的,有这些想头。我就没有这些心。”

说话时,宝玉已是三杯过去了,李嬷嬷又上来拦阻。宝玉正在个心甜意洽之时,
又兼姐妹们说说笑笑,那里肯不吃?只得屈意央告:“好妈妈,我再吃两杯就不吃
了。”李嬷嬷道:“你可仔细今儿老爷在家,提防着问你的书!”宝玉听了此话,
便心中大不悦,慢慢的放下酒,垂了头。黛玉忙说道:“别扫大家的兴。舅舅若叫,
只说姨妈这里留住你。这妈妈,他又该拿我们来醒脾了!”一面悄悄的推宝玉,叫
他赌赌气,一面咕哝说:“别理那老货,咱们只管乐咱们的。”那李妈也素知黛玉
的为人,说道:“林姐儿,你别助着他了。你要劝他只怕他还听些。”黛玉冷笑道:
“我为什么助着他?——我也不犯着劝他。你这妈妈太小心了!往常老太太又给他酒
吃,如今在姨妈这里多吃了一口,想来也不妨事。必定姨妈这里是外人,不当在这
里吃,也未可知。”李嬷嬷听了,又是急,又是笑,说道:“真真这林姐儿,说出
一句话来,比刀子还利害。”宝钗也忍不住笑着把黛玉腮上一拧,说道:“真真的
这个颦丫头一张嘴,叫人恨又不是,喜欢又不是。”薛姨妈一面笑着,又说:“别
怕,别怕,我的儿!来到这里没好的给你吃,别把这点子东西吓的存在心里,倒叫
我不安。只管放心吃,有我呢!索性吃了晚饭去。要醉了,就跟着我睡罢。”因命:
“再烫些酒来。姨妈陪你吃两杯,可就吃饭罢。”宝玉听了,方又鼓起兴来。李嬷
嬷因吩咐小丫头:“你们在这里小心着,我家去换了衣裳就来。”悄悄的回薛姨妈
道:“姨太太别由他尽着吃了。”说着便家去了。

这里虽还有两三个老婆子,都是不关痛痒的,见李妈走了,也都悄悄的自寻方
便去了。只剩了两个小丫头,乐得讨宝玉的喜欢。幸而薛姨妈千哄万哄,只容他吃
了几杯,就忙收过了。作了酸笋鸡皮汤,宝玉痛喝了几碗,又吃了半碗多碧粳粥;
一时薛林二人也吃完了饭,又酽酽的喝了几碗茶。薛姨妈才放了心。雪雁等几个人,
也吃了饭进来伺候。黛玉因问宝玉道:“你走不走?”宝玉乜斜倦眼道:“你要走
我和你同走。”黛玉听说,遂起身道:“咱们来了这一日,也该回去了。”说着,
二人便告辞。小丫头忙捧过斗笠来,宝玉把头略低一低,叫他戴上。那丫头便将这
大红猩毡斗笠一抖,才往宝玉头上一合,宝玉便说:“罢了罢了!好蠢东西,你也
轻些儿。难道没见别人戴过?等我自己戴罢。”黛玉站在炕沿上道:“过来,我给
你戴罢。”宝玉忙近前来。黛玉用手轻轻笼住束发冠儿,将笠沿掖在抹额之上,把
那一颗核桃大的绛绒簪缨扶起,颤巍巍露于笠外。整理已毕,端详了一会,说道:
“好了,披上斗篷罢。”宝玉听了,方接了斗篷披上。薛姨妈忙道:“跟你们的妈
妈都还没来呢,且略等等儿。”宝玉道:“我们倒等着他们!有丫头们跟着就是了。”
薛姨妈不放心,吩咐两个女人送了他兄妹们去。

他二人道了扰,一径回至贾母房中。贾母尚未用晚饭,知是薛姨妈处来,更加
喜欢。因见宝玉吃了酒,遂叫他自回房中歇着,不许再出来了。又令人好生招呼着。
忽想起跟宝玉的人来,遂问众人:“李奶子怎么不见?”众人不敢直说他家去了,
只说:“才进来了,想是有事,又出去了。”宝玉踉跄着回头道:“他比老太太还
受用呢,问他作什么!没有他只怕我还多活两日儿。”一面说,一面来至自己卧室。
只见笔墨在案。晴雯先接出来,笑道:“好啊,叫我研了墨,早起高兴,只写了三
个字,扔下笔就走了,哄我等了这一天。快来给我写完了这些墨才算呢!”宝玉方
想起早起的事来,因笑道:“我写的那三个字在那里呢?”晴雯笑道:“这个人可
醉了。你头里过那府里去,嘱咐我贴在门斗儿上的。我恐怕别人贴坏了,亲自爬高
上梯,贴了半天,这会子还冻的手僵着呢!”宝玉笑道:“我忘了。你手冷,我替
你渥着。”便伸手拉着晴雯的手,同看门斗上新写的三个字。

一时黛玉来了,宝玉笑道:“好妹妹,你别撒谎,你看这三个字那一个好?”
黛玉仰头看见是“绛芸轩”三字,笑道:“个个都好,怎么写的这样好了!明儿也
替我写个匾。”宝玉笑道:“你又哄我了。”说着又问:“袭人姐姐呢?”晴雯向
里间炕上努嘴儿。宝玉看时,见袭人和衣睡着。宝玉笑道:“好啊!这么早就睡了。”
又问晴雯道:“今儿我那边吃早饭,有一碟子豆腐皮儿的包子。我想着你爱吃,和
珍大奶奶要了,只说我晚上吃,叫人送来的。你可见了没有?”晴雯道:“快别提
了。一送来我就知道是我的。偏才吃了饭,就搁在那里。后来李奶奶来了看见,说:
‘宝玉未必吃了,拿去给我孙子吃罢。’就叫人送了家去了。”正说着,茜雪捧上
茶来。宝玉还让:“林妹妹喝茶。”众人笑道:“林姑娘早走了,还让呢。”宝玉
吃了半盏,忽又想起早晨的茶来,问茜雪道:“早起沏了碗枫露茶,我说过那茶是
三四次后才出色,这会子怎么又斟上这个茶来?”茜雪道:“我原留着来着,那会
子李奶奶来了,喝了去了。”宝玉听了,将手中茶杯顺手往地下一摔,豁琅一声打
了个粉碎,泼了茜雪一裙子。又跳起来问着茜雪道:“他是你那一门子的‘奶奶’,
你们这么孝敬他?不过是我小时候儿吃过他几日奶罢了,如今惯的比祖宗还大!撵出
去大家干净!”说着立刻便要去回贾母。

原来袭人未睡,不过是故意儿装睡,引着宝玉来怄他玩耍。先听见说字问包子,
也还可以不必起来;后来摔了茶钟动了气,遂连忙起来解劝。早有贾母那边的人来
问:“是怎么了?”袭人忙道:“我才倒茶,叫雪滑倒了,失手砸了钟子了。”一
面又劝宝玉道:“你诚心要撵他也好,我们都愿意出去,不如就势儿连我们一齐撵
了,你也不愁没有好的来伏侍你。”宝玉听了,方才不言语了。袭人等便搀至炕上,
脱了衣裳,不知宝玉口内还说些什么,只觉口齿缠绵,眉眼愈加饧涩,忙伏侍他睡
下。袭人摘下那“通灵宝玉”来,用绢子包好,在褥子底下,恐怕次日带时冰了
他的脖子。那宝玉到枕就睡着了。彼时李嬷嬷等已进来了,听见醉了,也就不敢上
前,只悄悄的打听睡着了,方放心散去。

次日醒来,就有人回:“那边小蓉大爷带了秦钟来拜。”宝玉忙接出去,领了
拜见贾母。贾母见秦钟形容标致,举止温柔,堪陪宝玉读书,心中十分喜欢,便留
茶留饭,又叫人带去见王夫人等。众人因爱秦氏,见了秦钟是这样人品,也都欢喜,
临去时都有表礼。贾母又给了一个荷包和一个金魁星,取“文星和合”之意。又嘱
咐他道:“你家住的远,或一时冷热不便,只管住在我们这里。只和你宝二叔在一
处,别跟着那不长进的东西们学。”秦钟一一的答应,回家禀知他父亲。

他父亲秦邦业现任营缮司郎中,年近七旬,夫人早亡,因年至五旬时尚无儿女,
便向养生堂抱了一个儿子和一个女儿。谁知儿子又死了,只剩下个女儿,小名叫做
可儿,又起个官名叫做兼美。长大时,生得形容袅娜,性格风流,因素与贾家有些
瓜葛,故结了亲。秦邦业却于五十三岁上得了秦钟,今年十二岁了;因去岁业师回
南,在家温习旧课,正要与贾亲家商议附往他家塾中去。可巧遇见宝玉这个机会,
又知贾家塾中司塾的乃现今之老儒贾代儒,秦钟此去,可望学业进益,从此成名,
因十分喜悦。只是宦囊羞涩,那边都是一双富贵眼睛:少了拿不出来。因是儿子的
终身大事所关,说不得东并西凑,恭恭敬敬封了二十四两贽见礼,带了秦钟到代儒
家来拜见,然后听宝玉拣的好日子一同入塾。塾中从此闹起事来。

未知如何,下回分解。