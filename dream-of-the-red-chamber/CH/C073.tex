\chapter{痴丫头误拾绣春囊~懦小姐不问累金凤}

话说那赵姨娘和贾政说话,忽听外面一声响,不知何物,忙问时,原来是外间
窗屉不曾扣好,滑了屈戌掉下来。赵姨娘骂了丫头几句,自己带领丫鬟上好,方进
来打发贾政安歇,不在话下。

却说怡红院中宝玉方才睡下,丫鬟们正欲各散安歇,忽听有人来敲院门。老婆
子开了,见是赵姨娘房内的丫头名唤小鹊的,问他作什么,小鹊不答,直往里走,
来找宝玉。只见宝玉才睡下,晴雯等犹在床边坐着,大家玩笑。见他来了,都问:
“什么事,这时候又跑了来?”小鹊连忙悄向宝玉道:“我来告诉你个信儿,方才
我们奶奶咕咕唧唧的,在老爷前不知说了你些个什么,我只听见‘宝玉’二字。我
来告诉你,仔细明儿老爷和你说话罢。”一面说着,回身就走。袭人命人留他吃茶,
因怕关门,遂一直去了。宝玉听了,知道赵姨娘心术不端,合自己仇人似的,又不
知他说些什么,便如孙大圣听见了紧箍儿咒的一般,登时四肢五内一齐皆不自在起
来。想来想去,别无他法,且理熟了书预备明儿盘考,只能书不舛错,就有别事也
可搪塞。一面想罢,忙披衣起来要读书。心中又自后悔:“这些日子,只说不提了,
偏又丢生了。早知该天天好歹温习些。”如今打算打算,肚子里现可背诵的,不过
只有《学》、《庸》、二《论》还背得出来。至上本《孟子》,就有一半是夹生的,
若凭空提一句,断不能背;至下《孟子》,就有大半生的,算起《五经》来,因近
来做诗,常把《五经》集些,虽不甚熟,还可塞责。别的虽不记得,素日贾政幸未
叫读的,纵不知,也还不妨。至于古文,还是那几年所读过的几篇《左传》、《国
策》、《公羊》、《谷梁》、汉、唐等文,这几年未曾读得,不过一时之兴,随看
随忘,未曾下过苦功,如何记得?这是更难塞责的。更有时文八股一道,因平素深
恶,说这原非圣贤之制撰,焉能阐发圣贤之奥,不过是后人饵名钓禄之阶。虽贾政
当日起身,选了百十篇命他读的,不过是后人的时文,偶见其中一二股内,或承起
之中,有作的精致,或流荡、或游戏、或悲感稍能动性者,偶尔一读,不过供一时
之兴趣,究竟何曾成篇潜心玩索?如今若温习这个,又恐明日盘究那个;若温习那
个,又恐盘驳这个:一夜之工,亦不能全然温习。因此,越添了焦躁。

自己读书,不值紧要,却累着一房丫鬟们都不能睡。袭人等在旁剪烛斟茶,那
些小的都困倦起来,前仰后合。晴雯骂道:“什么小蹄子们!一个个黑家白日挺尸
挺不够,偶然一次睡迟了些,就装出这个腔调儿来了。再这么着,我拿针扎你们两
下子!”话犹未了,只听外间咕咚一声,急忙看时,原来是个小丫头坐着打盹,一
头撞到壁上,从梦中惊醒。却正是晴雯说这话之时,他怔怔的只当是晴雯打了他一
下子,遂哭着央说:“好姐姐,我再不敢了!”众人都笑起来。宝玉忙劝道:“饶
他罢。原该叫他们睡去。你们也该替换着睡。”袭人道:“小祖宗,你只顾你的罢!
统共这一夜的工夫,你把心暂且用在这几本书上,等过了这一关,由你再张罗别的,
也不算误了什么。”宝玉听他说的恳切,只得又读几句。麝月斟了一杯茶来润舌,
宝玉接茶吃了。因见麝月只穿着短袄,宝玉道:“夜静了冷,到底穿一件大衣裳才
是啊。”麝月笑指着书道:“你暂且把我们忘了,使不得吗?且把心搁在这上头些
罢。”

话犹未了,只听春燕秋纹从后房门跑进来,口内喊说:“不好了!一个人打墙
上跳下来了。”众人听说,忙问:“在那里?”即喝起人来,各处寻找。晴雯因见
宝玉读书苦恼,劳费一夜神思,明日也未必妥当,心下正要替宝玉想个主意,好脱
此难。忽然碰着这一惊,便生计向宝玉道:“趁这个机会,快装病,只说吓着了。”
这话正中宝玉心怀。因叫起上夜的来,打着灯笼各处搜寻,并无踪迹,都说:“小
姑娘们想是睡花了眼出去,风摇的树枝儿,错认了人。”晴雯便道:“别放屁!你
们查的不严,怕耽不是,还拿这话来支吾!刚才并不是一个人见的,宝玉和我们出
去,大家亲见的。如今宝玉吓得颜色都变了,满身发热,我这会子还要上房里取安
魂丸药去呢。太太问起来,是要回明白了的,难道依你说就罢了?”众人听了吓得
不敢则声,只得又各处去找。晴雯和秋纹二人果出去要药去,故意闹的众人皆知宝
玉着了惊,吓病了。王夫人听了,忙命人来看视给药,又吩咐各上夜人仔细搜查;
又一面叫查二门外邻园墙上夜的小厮们。于是园内灯笼火把,直闹了一夜。至五更
天,就传管家的细看查访。

贾母闻知宝玉被吓,细问原由,众人不敢再隐,只得回明。贾母道:“我不料
有此事。如今各处上夜的都不小心还是小事,只怕他们就是贼也未可知。”当下邢
夫人尤氏等都过来请安,李纨凤姐及姊妹等皆陪侍,听贾母如此说,都默无所答。
独探春出位笑道:“近因凤姐姐身子不好几日,园里的人,比先放肆许多。先前不
过是大家偷着一时半刻,或夜里坐更时三四个人聚在一处,或掷骰,或斗牌,小玩
意儿,不过为着熬困起见。如今渐次放诞,竟开了赌局,甚至头家局主,或三十吊
五十吊的大输赢。半月前竟有争斗相打的事。”贾母听了,忙说:“你既知道,为
什么不早回我来?”探春道:“我因想着太太事多,且连日不自在,所以没回,只
告诉大嫂子和管事的人们,戒饬过几次,近日好些了。”贾母忙道:“你姑娘家,
那里知道这里头的利害?你以为赌钱常事,不过怕起争端;不知夜间既耍钱,就保
不住不吃酒,既吃酒,就未免门户任意开锁,或买东西,其中夜静人稀,趁便藏贼
引盗,什么事做不出来?况且园内你姐儿们起居所伴者,皆系丫头媳妇们,贤愚混
杂。贼盗事小,倘有别事,略沾带些,关系非小!这事岂可轻恕?”探春听说,便
默然归坐。凤姐虽未大愈,精神未尝稍减,今见贾母如此说,便忙道:“偏偏我又
病了。”遂回头命人速传林之孝家的等总理家事的四个媳妇来了,当着贾母申饬了
一顿。贾母命:“即刻查了头家赌家来!有人出首者赏,隐情不告者罚。”

林之孝家的等见贾母动怒,谁敢徇私,忙去园内传齐,又一一盘查。虽然大家
赖一回,终不免水落石出。查得大头家三人,小头家八人,聚赌者统共二十多人,
都带来见贾母,跪在院内,磕响头求饶。贾母先问大头家名姓,和钱之多少。原来
这大头家,一个是林之孝家的两姨亲家,一个是园内厨房内柳家媳妇之妹,一个是
迎春之乳母。这是三个为首的,馀者不能多记。贾母便命将骰子纸牌一并烧毁,所
有的钱入官,分散与众人;将为首者每人打四十大板,撵出去,总不许再入;从者
每人打二十板,革去三月月钱,拨入圊厕行内。又将林之孝家的申饬了一番。林之
孝家的见他的亲戚又给他打嘴,自己也觉没趣;迎春在坐也觉没意思。黛玉、宝钗、
探春等见迎春的乳母如此,也是“物伤其类”的意思,遂都起身笑向贾母讨情,说:
“这个奶奶素日原不玩的,不知怎么,也偶然高兴;求看二姐姐面上,饶过这次罢。”
贾母道:“你们不知道。大约这些奶子们,一个个仗着奶过哥儿姐儿,原比别人有
些体面,他们就生事,比别人更可恶!专管调唆主子,护短偏向。我都是经过的。
况且要拿一个作法,恰好果然就遇见了一个。你们别管,我自有道理。”宝钗等听
说,只得罢了。一时贾母歇晌,大家散出,都知贾母生气,皆不敢回家,只得在此
暂候。尤氏到凤姐儿处来闲话了一回,因他也不自在,只得园内去闲谈。

邢夫人在王夫人处坐了一回,也要到园内走走。刚至园门前,只见贾母房内的
小丫头子名唤傻大姐的,笑嘻嘻走来,手内拿着个花红柳绿的东西,低头瞧着只管
走。不防迎头撞见邢夫人,抬头看见,方才站住。邢夫人因说:“这傻丫头又得个
什么爱巴物儿,这样喜欢?拿来我瞧瞧。”原来这傻大姐年方十四岁,是新挑上来
给贾母这边专做粗活的。因他生的体肥面阔,两只大脚,做粗活很爽利简捷,且心
性愚顽,一无知识,出言可以发笑。贾母喜欢,便起名为“傻大姐”,若有错失,
也不苛责他。无事时便入园内来玩耍,正往山石背后掏促织去,忽见一个五彩绣香
囊,上面绣的并非花鸟等物,一面却是两个人赤条条的相抱,一面是几个字。这痴
丫头原不认得是春意儿,心下打量:“敢是两个妖精打架?不就是两个人打架呢?”
左右猜解不来,正要拿去给贾母看呢,所以笑嘻嘻走回。忽见邢夫人如此说,便笑
道:“太太真个说的巧,真是个爱巴物儿。太太瞧一瞧。”说着便送过去。邢夫人
接来一看,吓得连忙死紧攥住,忙问:“你是那里得的?”傻大姐道:“我掏促织
儿,在山子石后头拣的。”邢夫人道:“快别告诉人!这不是好东西。连你也要打
死呢。因你素日是个傻丫头,以后再别提了。”这傻大姐听了,反吓得黄了脸,说:
“再不敢了。”磕了头,呆呆而去。

邢夫人回头看时,都是些女孩儿,不便递给他们,自己便在袖里。心内十分
罕异,揣摩此物从何而来,且不形于声色,到了迎春房里。迎春正因他乳母获罪,
心中不自在,忽报母亲来了,遂接入。奉茶毕,邢夫人因说道:“你这么大了,你
那奶妈子行此事,你也不说说他。如今别人都好好的,偏咱们的人做出这事来,什
么意思?”迎春低头弄衣带,半晌答道:“我说他两次,他不听,也叫我没法儿。
况因他是妈妈,只有他说我的,没有我说他的。”邢夫人道:“胡说。你不好了,
他原该说;如今他犯了法,你就该拿出姑娘的身分来。他敢不依,你就回我去才是。
如今直等外人共知,这可是什么意思!再者:放头儿,还只怕他巧语花言的和你借
贷些簪环衣裳做本钱。你这心活面软,未必不周济他些。若被他骗了去,我是一个
钱没有的,看你明日怎么过节?”迎春不语,只低着头。邢夫人见他这般,因冷笑
道:“你是大老爷跟前的人养的,这里探丫头是二老爷跟前的人养的,出身一样,
你娘比赵姨娘强十分,你也该比探丫头强才是。怎么你反不及他一点?倒是我无儿
女的一生干净,也不能惹人笑话!”人回:“琏二奶奶来了。”邢夫人听了,冷笑
两声,命人出去说:“请他自己养病,我这里不用他伺候。”接着又有探事的小丫
头来报说:“老太太醒了。”邢夫人方起身往前边来。

迎春送至院外方回。绣橘因说道:“如何?前儿我回姑娘:‘那一个攒珠累金
凤,竟不知那里去了。’回了姑娘,竟不问一声儿。我说:‘必是老奶奶拿去当了
银子放头儿了。’姑娘不信,只说司棋收着,叫问司棋。司棋虽病,心里却明白,
说:‘没有收起来,还在书架上匣里放着,预备八月十五要戴呢。’姑娘该叫人去
问老奶奶一声。”迎春道:“何用问?那自然是他拿了去摘了肩儿了。我只说他悄
悄的拿了出去,不过一时半晌,仍旧悄悄的放在里头,谁知他就忘了。今日偏又闹
出来,问他也无益。”绣橘道:“何曾是忘记?他是试准了姑娘的性格儿才这么着。
如今我有个主意:到二奶奶屋里,将此事回了,他或着人要,他或省事拿几吊钱来
替他赎了,如何?”迎春忙道:“罢,罢,省事些好。宁可没有了,又何必生事?”
绣橘道:“姑娘怎么这样软弱?都要省起事来,将来连姑娘还骗了去。我竟去的是。”
说着便走。迎春便不言语,只好由他。

谁知迎春的乳母之媳玉柱儿媳妇为他婆婆得罪,来求迎春去讨情,他们正说金
凤一事,且不进去。也因素日迎春懦弱,他们都不放在心上;如今见绣橘立意去回
凤姐,又看这事脱不过去,只得进来,陪笑先向绣橘说:“姑娘,你别去生事。姑
娘的金丝凤,原是我们老奶奶老糊涂了,输了几个钱,没的捞梢,所以借去,不想
今日弄出事来。虽然这样,到底主子的东西,我们不敢迟误,终久是要赎的。如今
还要求姑娘看着从小儿吃奶的情,往老太太那边去讨一个情儿,救出他来才好。”
迎春便说道:“好嫂子,你趁早打了这妄想。要等我去说情儿,等到明年,也是不
中用的。方才连宝姐姐林妹妹,大伙儿说情,老太太还不依,何况是我一个人?我
自己臊还臊不过来,还去讨臊去?”绣橘便说:“赎金凤是一件事,说情是一件事,
别绞在一处。难道姑娘不去说情,你就不赔了不成?嫂子且取了金凤来再说。”玉
柱儿家的听见迎春如此拒绝他,绣橘的话又锋利,无可回答,一时脸上过不去,也
明欺迎春素日好性儿,乃向绣橘说道:“姑娘,你别太张势了!你满家子算一算,
谁的妈妈奶奶不仗着主子哥儿姐儿得些便宜,偏咱们就这样‘丁是丁,卯是卯’的?
只许你们偷偷摸摸的哄骗了去。自从邢姑娘来了,太太吩咐一个月俭省出一两银子
来给舅太太去,这里饶添了邢姑娘的使费,反少了一两银子。时常短了这个,少了
那个,那不是我们供给?谁又要去?不过大家将就些罢了。算到今日少说也有三十两
了,我们这一向的钱岂不白填了限呢?”绣橘不待说完,便啐了一口,道:“做什
么你白填了三十两?我且和你算算账!姑娘要了些什么东西?”迎春听了这媳妇发邢
夫人之私意,忙止道:“罢,罢!不能拿了金凤来,你不必拉三扯四的乱嚷。我也
不要那凤了。就是太太问时,我只说丢了,也妨碍不着你什么,你出去歇歇儿去罢。
何苦呢?”一面叫绣橘倒茶来。绣橘又气又急,因说道:“姑娘虽不怕,我是做什
么的?把姑娘的东西丢了,他倒赖说姑娘使了他的钱,这如今竟要准折起来。倘或
太太问姑娘为什么使了这些钱,敢是我们就中取势?这还了得!”一行说,一行就
哭了。司棋听不过,只得勉强过来,帮着绣橘问着那媳妇。迎春劝止不住,自拿了
一本《太上感应篇》去看。

三人正没开交,可巧宝钗、黛玉、宝琴、探春等,因恐迎春今日不自在,都约
着来安慰。他们走至院中,听见几个人讲究,探春从纱窗内一看,只见迎春倚在床
上看书,若有不闻之状,探春也笑了。小丫头们忙打起帘子报道:“姑娘们来了。”
迎春放下书起身。那媳妇见有人来,且又有探春在内,不劝自止了,遂趁便就走。
探春坐下,便问:“才刚谁在这里说话,倒像拌嘴似的?”迎春笑道:“没有什么,
左不过他们小题大做罢了,何必问他?”探春笑道:“我才听见什么‘金凤’,又
是什么‘没有钱,只合我们奴才要’。谁和奴才要钱了?难道姐姐和奴才要钱不成?”
司棋绣橘道:“姑娘说的是了!姑娘何曾和他要什么了?”探春笑道:“姐姐既没
有和他要,必定是我们和他们要了不成?你叫他进来,我倒要问问他。”迎春笑道:
“这话又可笑。你们又无沾碍,何必如此?”探春道:“这倒不然。我和姐姐一样。
姐姐的事,和我一般。他说姐姐,即是说我;我那边有人怨我,姐姐听见,也是合
怨姐姐一样。咱们是主子,自然不理论那些钱财小事,只知想起什么要什么,也是
有的事。但不知累丝凤怎么又夹在里头?”那玉柱儿媳妇生恐绣橘等告出他来,遂
忙进来用话掩饰。探春深知其意,因笑道:“你们所以糊涂!如今你奶奶已得了不
是,趁此求二奶奶,把方才的钱未曾散人的拿出些来赎来就完了。比不得没闹出来,
大家都藏着留脸面。如今既是没了脸,趁此时,总有十个罪也只一人受罚,没有砍
两颗头的理。你依我说,竟是和二奶奶趁便说去。在这里大声小气,如何使得!”
这媳妇被探春说出真病,也无可赖了,只不敢往凤姐处自首。探春笑道:“我不听
见便罢,既听见,少不得替你们分解分解。”

谁知探春早使了眼色与侍书,侍书出去了。这里正说话,忽见平儿进来。宝琴
拍手笑道:“三姐姐敢是有驱神召将的符术?”黛玉笑道:“这倒不是道家法术,
倒是用兵最精的所谓‘守如处女,出如脱兔’,‘出其不备’的妙策。”二人取笑,
宝钗便使眼色与二人,遂以别话岔开。探春见平儿来了,遂问:“你奶奶可好些了?
真是病糊涂了,事事都不在心上,叫我们受这样委屈。”平儿忙道:“谁敢给姑娘
气受?姑娘吩咐我。”那玉柱儿媳妇方慌了手脚,遂上来赶着平儿叫:“姑娘坐下,
让我说原故,姑娘请听。”平儿正色道:“姑娘这里说话,也有你混插嘴的理吗!
你但凡知礼,该在外头伺候,也有外头的媳妇们无故到姑娘屋里来的?”绣橘道:
“你不知我们这屋里是没礼的,谁爱来就来。”平儿道:“都是你们不是!姑娘好
性儿,你们就该打出去,然后再回太太去才是。”柱儿媳妇见平儿出了言,红了脸,
才退出去。探春接着道:“我且告诉你:要是别人得罪了我,倒还罢了。如今这柱
儿媳妇和他婆婆,仗着是嬷嬷,又瞅着二姐姐好性儿,私自拿了首饰去赌钱,而且
还捏造假账,逼着去讨情,和这两个丫头在卧房里大嚷大叫,二姐姐竟不能辖治。
所以我看不过,才请你来问一声:还是他本是天外的人,不知道理?还是有谁主使
他如此,先把二姐姐制伏了,然后就要治我和四姑娘了?”平儿忙陪笑道:“姑娘
怎么今日说出这话来?我们奶奶如何担得起!”探春冷笑道:“俗语说的,‘物伤
其类,唇亡齿寒’,我自然有些心惊么。”

平儿问迎春道:“若论此事,本好处的。但只他是姑娘的奶嫂,姑娘怎么样呢?”
当下迎春只合宝钗看《感应篇》故事,究竟连探春的话也没听见,忽见平儿如此说,
仍笑道:“问我,我也没什么法子。他们的不是,自作自受,我也不能讨情,我也
不去加责,就是了。至于私自拿去的东西,送来我收下,不送来我也不要了。太太
们要来问我,可以隐瞒遮饰的过去,是他的造化;要瞒不住我也没法儿,没有个为
他们反欺枉太太们的理,少不得直说。你们要说我好性儿,没个决断;有好主意可
以八面周全,不叫太太们生气,任凭你们处治,我也不管。”众人听了,都好笑起
来。黛玉笑道:“真是‘虎狼屯于阶陛,尚谈因果’。要是二姐姐是个男人,一家
上下这些人,又如何裁治他们?”迎春笑道:“正是,多少男人衣租食税,及至事
到临头,尚且如此。况且‘太上’说的好,救人急难,最是阴骘事。我虽不能救人,
何苦来白白去和人结怨结仇,作那样无益有损的事呢?”一语未了,只听又有一人
来了。

不知是谁,下回分解。