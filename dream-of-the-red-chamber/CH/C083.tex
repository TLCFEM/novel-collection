\chapter{省宫闱贾元妃染恙~闹闺阃薛宝钗吞声}

话说探春湘云才要走时,忽听外面一个人嚷道:“你这不成人的小蹄子!你是
个什么东西,来这园子里头混搅!”黛玉听了,大叫一声道:“这里住不得了!”
一手指着窗外,两眼反插上去。原来黛玉住在大观园中,虽靠着贾母疼爱,然在别
人身上,凡事终是寸步留心。听见窗外老婆子这样骂着,在别人呢,一句是贴不上
的,竟像专骂着自己的。自思一个千金小姐,只因没了爹娘,不知何人指使这老婆
子来这般辱骂,那里委屈得来?因此,肝肠崩裂,哭的过去了。紫鹃只是哭叫:“姑
娘怎么样了?快醒来罢!”探春也叫了一回。半晌,黛玉回过这口气,还说不出话
来,那只手仍向窗外指着。

探春会意,开门出去,看见老婆子手中拿着拐棍,赶着一个不干不净的毛丫头
道:“我是为照管这园中的花果树木,来到这里,你作什么来了?等我家去,打你
一个知道。”这丫头扭着头,把一个指头探在嘴里,瞅着老婆子笑。探春骂道:“你
们这些人,如今越发没了王法了。这里是你骂人的地方儿吗?”老婆子见是探春,
连忙陪着笑脸儿说道:“刚才是我的外孙女儿,看见我来了,他就跟了来。我怕他
闹,所以才吆喝他回去,那里敢在这里骂人呢?”探春道:“不用多说了,快给我
都出去。这里林姑娘身上不大好,还不快去么!”老婆子答应了几个“是”,说着,
一扭身去了,那丫头也就跑了。

探春回来,看见湘云拉着黛玉的手只管哭,紫鹃一手抱着黛玉,一手给黛玉揉
胸口,黛玉的眼睛方渐渐的转过来了。探春笑道:“想是听见老婆子的话,你疑了
心了么?”黛玉只摇摇头儿。探春道:“他是骂他外孙女儿,我才刚也听见了。这
种东西说话再没有一点道理的,他们懂得什么避讳。”黛玉听了,叹了口气,拉着
探春的手道:“姐儿——”叫了一声,又不言语了。探春又道:“你别心烦。我来
看你,是姊妹们应该的。你又少人伏侍。只要你安心肯吃药,心上把喜欢事儿想想,
能够一天一天的硬朗起来,大家依旧结社做诗,岂不好呢。”湘云道:“可是三姐
姐说的,那么着不乐?”黛玉哽咽道:“你们只顾要我喜欢,可怜我那里赶得上这
日子?只怕不能够了。”探春道:“你这话说的太过了,谁没个病儿灾儿的?那里就
想到这里来了。你好生歇歇儿罢,我们到老太太那边,回来再看你。你要什么东西,
只管叫紫鹃告诉我。”黛玉流泪道:“好妹妹,你到老太太那里,只说我请安,身
上略有点不好,不是什么大病,也不用老太太烦心的。”探春答应道:“我知道,
你只管养着罢。”说着,才同湘云出去了。

这里紫鹃扶着黛玉躺在床上,地下诸事自有雪雁照料,自己只守着傍边看着黛
玉,又是心酸,又不敢哭泣。那黛玉闭着眼躺了半晌,那里睡得着,觉得园里头平
日只见寂寞,如今躺在床上,偏听得风声、虫鸣声、鸟语声、人走的脚步声,又像
远远的孩子们啼哭声,一阵一阵的聒噪的烦躁起来。因叫紫鹃:“放下帐子来。”
雪雁捧了一碗燕窝汤,递给紫鹃。紫鹃隔着帐子,轻轻问道:“姑娘,喝一口汤罢?”
黛玉微微应了一声。紫鹃复将汤递给雪雁,自己上来,搀扶黛玉坐起,然后接过汤
来,搁在唇边试了一试,一手搂着黛玉肩臂,一手端着汤送到唇边。黛玉微微睁眼
喝了两三口,便摇摇头儿不喝了。紫鹃仍将碗递给雪雁,轻轻扶黛玉睡下。静了一
时,略觉安顿。

只听窗外悄悄问道:“紫鹃妹妹在家么?”雪雁连忙出来,见是袭人,因悄悄
说道:“姐姐屋里坐着。”袭人也便悄悄问道:“姑娘怎么着?”一面走,一面雪
雁告诉夜间及方才之事。袭人听了这话,也唬怔了,因说道:“怪道刚才翠缕到我
们那边说你们姑娘病了,唬的宝二爷连忙打发我来,看看是怎么样。”正说着,只
见紫鹃从里间掀起帘子,望外看见袭人,招手儿叫他。袭人轻轻走过来,问道:“姑
娘睡着了吗?”紫鹃点点头儿,问道:“姐姐才听见说了?”袭人也点点头儿,蹙
着眉道:“终久怎么样好呢?那一位昨夜也把我唬了个半死儿!”紫鹃忙问:“怎
么了?”袭人道:“昨日晚上睡觉还是好好儿的,谁知半夜里一叠连声的嚷起心疼
来。嘴里胡说白道,只说好像刀子割了去的似的,直闹到打亮梆子以后才好些了。
你说唬人不唬人?今日不能上学,还要请大夫来吃药呢。”正说着,只听黛玉在帐
子里又咳嗽起来,紫鹃连忙过来捧痰盒儿接痰。黛玉微微睁眼问道:“你合谁说话
呢?”紫鹃道:“袭人姐姐来瞧姑娘来了。”说着,袭人已走到床前。黛玉命紫鹃
扶起,一手指着床边,让袭人坐下。袭人侧身坐了,连忙陪着笑劝道:“姑娘倒还
是躺着罢。”黛玉道:“不妨,你们快别这样大惊小怪的。刚才是说谁半夜里心疼
起来?”袭人道:“是宝二爷偶然魇住了,不是认真怎么样。”黛玉会意,知道是
袭人怕自己又悬心的原故,又感激,又伤心,因趁势问道:“既是魇住了,不听见
他还说什么?”袭人道:“也没说什么。”黛玉点点头儿,迟了半日,叹了一声,
才说道:“你们别告诉宝二爷说我不好,看耽搁了他的工夫,又叫老爷生气。”袭
人答应了,又劝道:“姑娘,还是躺躺歇歇罢。”黛玉点头,命紫鹃扶着歪下。袭
人不免坐在旁边,又宽慰了几句,然后告辞。回到怡红院,只说黛玉身上略觉不受
用,也没什么大病。宝玉才放了心。

且说探春湘云出了潇湘馆,一路往贾母这边来。探春因嘱咐湘云道:“妹妹回
来见了老太太,别像刚才那样冒冒失失的了。”湘云点头笑道:“知道了。我头里
是叫他唬的忘了神了。”说着已到贾母那边。探春因提起黛玉的病来。贾母听了,
自是心烦,因说道:“偏是这两个‘玉’儿多病多灾的。林丫头一来二去的大了,
他这个身子也要紧。我看那孩子太是个心细。”众人也不敢答言。贾母便向鸳鸯道:
“你告诉他们,明儿大夫来瞧了宝玉,叫他再到林姑娘那屋里去。”鸳鸯答应着出
来,告诉了婆子们。婆子们自去传话。这里探春湘云就跟着贾母吃了晚饭,然后同
回园中去,不提。

到了次日,大夫来了。瞧了宝玉,不过说饮食不调,着了点儿风邪,没大要紧,
疏散疏散就好了。这里王夫人凤姐等,一面遣人拿了方子回贾母,一面使人到潇湘
馆,告诉说:“大夫就过来。”紫鹃答应了,连忙给黛玉盖好被窝,放下帐子,雪
雁赶着收拾房里的东西。一时贾琏陪着大夫进来了,便说道:“这位老爷是常来的,
姑娘们不用回避。”老婆子打起帘子,贾琏让着,进入房中坐下。贾琏道:“紫鹃
姐姐,你先把姑娘的病势向王老爷说说。”王大夫道:“且慢说。等我诊了脉,听
我说了,看是对不对。若有不合的地方,姑娘们再告诉我。”紫鹃便向帐中扶出黛
玉的一只手来,搁在迎手上。紫鹃又把镯子连袖子轻轻的撸起,不叫压住了脉息。
那王大夫诊了好一会儿,又换那只手也诊了,便同贾琏出来,到外间屋里坐下,说
道:“六脉皆弦,因平日郁结所致。”说着,紫鹃也出来,站在里间门口。那王大
夫便向紫鹃道:“这病时常应得头晕,减饮食,多梦。每到五更,必醒个几次;即
日间听见不干自己的事,也必要动气,且多疑多惧。不知者疑为性情乖诞,其实因
肝阴亏损,心气衰耗,都是这个病在那里作怪。不知是否?”紫鹃点点头儿,向贾
琏道:“说的很是。”王太医道:“既这样,就是了。”说毕,起身同贾琏往外书
房去开方子。小厮们早已预备下一张梅红单帖;王太医吃了茶,因提笔先写道:

六脉弦迟,素由积郁。左寸无力,心气已衰。关脉独洪,肝邪偏旺。木气不能
疏达,势必上侵脾土,饮食无味;甚至胜所不胜,肺金定受其殃。气不流精,凝而
为痰;血随气涌,自然咳吐。理宜疏肝保肺,涵养心脾。虽有补剂,未可骤施。姑
拟“黑逍遥”以开其先,后用“归肺固金”以继其后。不揣固陋,俟高明裁服。
又将七味药与引子写了。贾琏拿来看时,问道:“血势上冲,柴胡使得么?”王大
夫笑道:“二爷但知柴胡是升提之品,为吐衄所忌,岂知用鳖血拌炒,非柴胡不足
宣少阳甲胆之气。以鳖血制之,使其不致升提,且能培养肝阴,制遏邪火。所以《内
经》说:‘通因通用,塞因塞用。’柴胡用鳖血拌炒,正是‘假周勃以安刘’的法
子。”贾琏点头道:“原来是这么着。这就是了。”王大夫又道:“先请服两剂,
再加减,或再换方子罢。我还有一点小事,不能久坐,容日再来请安。”说着,贾
琏送了出来,说道:“舍弟的药,就是那么着了?”王大夫道:“宝二爷倒没什么
大病,大约再吃一剂就好了。”说着上车而去。

这里贾琏一面叫人抓药,一面回到房中告诉凤姐黛玉的病原与大夫用的药,述
了一遍。只见周瑞家的走来,回了几件没要紧的事。贾琏听到一半,便说道:“你
回二奶奶罢,我还有事呢。”说着就走了。周瑞家的回完了这件事,又说道:“我
方才到林姑娘那边,看他那个病竟是不好呢。脸上一点血色也没有,摸了摸身上,
只剩了一把骨头。问问他,也没有话说,只是淌眼泪。回来紫鹃告诉我说:‘姑娘
现在病着,要什么自己又不肯要,我打算要问二奶奶那里支用一两个月的月钱。如
今吃药虽是公中的,零用也得几个钱。’我答应了他,替他来回奶奶。”凤姐低了
半日头,说道:“竟这么着罢,我送他几两银子使罢。也不用告诉林姑娘。这月钱
却是不好支的。一个人开了例,要是都支起来,那如何使得呢?你不记得赵姨娘和
三姑娘拌嘴了?也无非为的是月钱。况且近来你也知道,出去的多进来的少,总绕
不过弯儿来。不知道的还说我打算的不好,更有那一种嚼舌根的,说我搬运到娘家
去了。周嫂子,你倒是那里经手的人,这个自然还知道些。”周瑞家的道:“真正
委屈死人!这样大门头儿,除了奶奶这样心计儿当家罢了。别说是女人当不来,就
是三头六臂的男人还撑不住呢。还说这些个混账话。”说着又笑了一声道:“奶奶
还没听见呢,外头的人还更糊涂呢。前儿周瑞回家来,说起外头的人打量着咱们府
里不知怎么样有钱呢。也有说:‘贾府里的银库几间,金库几间,使的家伙都是金
子镶了、玉石嵌了的。’也有说:‘姑娘做了王妃,自然皇上家的东西分的了一半
子给娘家。前儿贵妃娘娘省亲回来,我们还亲见他带了几车金银回来,所以家里收
拾摆设的水晶宫似的。那日在庙里还愿,花了几万银子,只算是牛身上拔了一根毛
罢咧。’有人还说:‘他门前的狮子,只怕还是玉石的呢。园子里还有金麒麟,叫
人偷了一个去,如今剩下一个了。家里的奶奶姑娘不用说,就是屋里使唤的姑娘们,
也是一点儿不动的,喝酒下棋,弹琴画画,横竖有人伏侍呢,单管穿罗罩纱。吃的
带的,都是人家不认得的。那些哥儿姐儿们更不用说了,要天上的月亮,也有人去
拿下来给他玩。’还有歌儿呢,说是:‘宁国府,荣国府,金银财宝如粪土。吃不
穷,穿不穷,算来——’”说到这里,猛然咽住。原来那时歌儿说道是“算来总是
一场空”,这周瑞家的说溜了嘴,说到这里,忽然想起这话不好,因咽住了。

凤姐儿听了,已明白必是句不好的话了,也不便追问。因说道:“那都没要紧,
只是这‘金麒麟’的话从何而来?”周瑞家的笑道:“就是那庙里的老道士送给宝
二爷的小金麒麟儿。后来丢了几天,亏了史姑娘捡着,还了他,外头就造出这个谣
言来了。奶奶说这些人可笑不可笑?”凤姐道:“这些话倒不是可笑,倒是可怕的。
咱们一日难似一日,外面还是这么讲究。俗语儿说的,‘人怕出名猪怕壮’,况且
又是个虚名儿,终久还不知怎么样呢。”周瑞家的道:“奶奶虑的也是。只是满城
里茶坊酒铺儿以及各胡同儿都是这样说,况且不是一年了,那里握的住众人的
嘴?”凤姐点点头儿。因叫平儿称了几两银子,递给周瑞家的道:“你先拿去交给
紫鹃,只说我给他添补买东西的。若要官中的只管要去,别提这月钱的话。他也是
个伶透人,自然明白我的话。我得了空儿就去瞧姑娘去。”周瑞家的接了银子,答
应着自去,不提。

且说贾琏走到外面,只见一个小厮迎上来,回道:“大老爷叫二爷说话呢。”
贾琏急忙过来,见了贾赦。贾赦道:“方才风闻宫里头传了一个太医院御医、两个
吏目去看病,想来不是宫女儿下人了。这几天,娘娘宫里有什么信儿没有?”贾琏
道:“没有。”贾赦道:“你去问问二老爷和你珍大哥;不然,还该叫人去到太医
院里打听打听才是。”贾琏答应了,一面吩咐人往太医院去,一面连忙去见贾政贾
珍。贾政听了这话,因问道:“是那里来的风声?”贾琏道:“是大老爷才说的。”
贾政道:“你索性和你珍大哥到里头打听打听。”贾琏道:“我已经打发人往太医
院打听去了。”一面说着,一面退出来去找贾珍。只见贾珍迎面来了,贾琏忙告诉
贾珍。贾珍道:“我正为也听见这话,来回大老爷二老爷去呢。”于是两个人同着
来见贾政。贾政道:“如系元妃,少不得终有信的。”说着,贾赦也过来了。

到了晌午,打听的尚未回来,门上人进来回说:“有两个内相在外,要见二位
老爷呢。”贾赦道:“请进来。”门上的人领了老公进来。贾赦贾政迎至二门外,
先请了娘娘的安,一面同着进来,走至厅上,让了坐。老公道:“前日这里贵妃娘
娘有些欠安,昨日奉过旨意,宣召亲丁四人进里头探问。许各带丫头一人,馀皆不
用。亲丁男人,只许在宫门外递个职名请安听信,不得擅入。准于明日辰巳时进去,
申酉时出来。”贾政贾赦等站着听了旨意,复又坐下,让老公吃茶毕,老公辞了出
去。

贾赦贾政送出大门,回来先禀贾母。贾母道:“亲丁四人,自然是我和你们两
位太太了。那一个人呢?”众人也不敢答言。贾母想了想,道:“必得是凤姐儿,
他诸事有照应。你们爷儿们各自商量去罢。”贾赦贾政答应了出来,因派了贾琏贾
蓉看家外,凡“文”字辈至“草”字辈一应都去。遂吩咐家人预备四乘绿轿,十余
辆翠盖车,明儿黎明伺候。家人答应去了。贾赦贾政又进去回明贾母:“辰巳时进
去,申酉时出来。今日早些歇歇,明日好早些起来,收拾进宫。”贾母道:“我知
道,你们去罢。”赦政等退出。这里邢夫人、王夫人、凤姐儿也都说了一会子元妃
的病,又说了些闲话,才各自散了。

次日黎明,各屋子里丫头们将灯火俱已点齐,太太们各梳洗毕,爷们亦各整顿
好了。一到卯初,林之孝合赖大进来,至二门口回道:“轿车俱已齐备,在门外伺
候着呢。”不一时,贾赦邢夫人也过来了。大家用了早饭,凤姐先扶老太太出来,
众人围随,各带使女一人,缓缓前行。又命李贵等二人先骑马去外宫门接应,自己
家眷随后。“文”字辈至“草”字辈各自登车骑马,跟着众家人,一齐去了。贾琏
贾蓉在家中看家。

且说贾家的车辆轿马俱在外西垣门口歇下等着。一会儿,有两个内监出来,说
道:“贾府省亲的太太奶奶们着令入宫探问。爷们俱着令内宫门外请安,不得入见。”
门上人叫:“快进去。”贾府中四乘轿子跟着小内监前行,贾家爷们在轿后步行跟
着,令众家人在外等候。走近宫门口,只见几个老公在门上坐着,见他们来了,便
站起来说道:“贾府爷们至此。”贾赦贾政便捱次立定。轿子抬至宫门口,便都出
了轿,早有几个小内监引路,贾母等各有丫头扶着步行。走至元妃寝宫,只见奎壁
辉煌,琉璃照耀。又有两个小宫女儿传谕道:“只用请安,一概仪注都免。”贾母
等谢了恩,来至床前,请安毕,元妃都赐了坐。贾母等又告了坐。元妃便问贾母道:
“近日身上可好?”贾母扶着小丫头,颤颤巍巍站起来,答应道:“托娘娘洪福,
起居尚健。”元妃又向邢夫人王夫人问了好。邢王二夫人站着回了话。元妃又问凤
姐:“家中过的日子若何?”凤姐站起来回奏道:“尚可支持。”元妃道:“这几
年来,难为你操心。”凤姐正要站起来回奏,只见一个宫女传进许多职名,请娘娘
龙目。元妃看时,说是贾赦贾政等若干人。那元妃看了职名,心里一酸,止不住早
流下泪来。宫女儿递过绢子,元妃一面拭泪,一面传谕道:“今日稍安,令他们外
面暂歇。”贾母等站起来,又谢了恩。元妃含泪道:“父女弟兄,反不如小家子得
以常常亲近。”贾母等都忍着泪道:“娘娘不用悲伤,家中已托着娘娘的福多了。”
元妃又问:“宝玉近来若何?”贾母道:“近来颇肯念书。因他父亲逼得严紧,如
今文字也都做上来了。”元妃道:“这样才好。”遂命外宫赐宴。便有两个宫女儿,
四个小太监,引了到一座宫里。已摆得齐整,各按坐次坐了。不必细述。一时吃完
了饭,贾母带着他婆媳三人,谢过宴。又耽搁了一回,看看已近酉初,不敢羁留,
俱各辞了出来。元妃命宫女儿引道,送至内宫门,门外仍是四个小太监送出。贾母
等依旧坐着轿子出来,贾赦接着,大伙儿一齐回去。到家,又要安排明后日进宫,
仍令照应齐集,不提。

且说薛家金桂自赶出薛蟠去了,日间拌嘴没有对头,秋菱又住在宝钗那边去
了,只剩得宝蟾一人同住。既给与薛蟠作妾,宝蟾的意气又不比从前了,金桂看去,
更是一个对头,自己也后悔不来。一日,吃了几杯闷酒,躺在炕上,便要借那宝蟾
作个醒酒汤儿,因问着宝蟾道:“大爷前日出门,到底是到那里去?你自然是知道
的了。”宝蟾道:“我那里知道?他在奶奶跟前还不说,谁知道他那些事?”金桂
冷笑道:“如今还有什么‘奶奶’‘太太’的,都是你们的世界了。别人是惹不得
的,有人护庇着,我也不敢去虎头上捉虱子。你还是我的丫头,问你一句话,你就
和我摔脸子,说话!你既这么有势力,为什么不把我勒死了,你和秋菱不拘谁做
了奶奶,那不清净了么?偏我又不死,碍着你们的道儿!”宝蟾听了这话,那里受
得住,便眼睛直直的瞅着金桂道:“奶奶这些闲话只好说给别人听去!我并没合奶
奶说什么。奶奶不敢惹人家,何苦来拿着我们小软儿出气呢?正经的,奶奶又装听
不见,‘没事人一大堆’了。”说着,便哭天哭地起来。金桂越发性起,便爬下炕
来,要打宝蟾。宝蟾也是夏家的风气,半点儿不让。金桂将桌椅杯盏尽行打翻,那
宝蟾只管喊冤叫屈,那里理会他?

岂知薛姨妈在宝钗房中,听见如此吵嚷,便叫:“香菱,你过去瞧瞧,且劝劝
他们。”宝钗道:“使不得,妈妈别叫他去。他去了岂能劝他?那更是火上浇了油
了。”薛姨妈道:“既这么样,我自己过去。”宝钗道:“依我说,妈妈也不用去,
由着他们闹去罢。这也是没法儿的事了。”薛姨妈道:“这那里还了得!”说着,
自己扶了丫头,往金桂这边来。宝钗只得也跟着过去。又嘱咐香菱道:“你在这里
罢。”

母女同至金桂房门口,听见里头正还嚷哭不止。薛姨妈道:“你们是怎么着,
又这么家翻宅乱起来?这还像个人家儿吗?矮墙浅屋的,难道都不怕亲戚们听见笑话
了么?”金桂屋里接声道:“我倒怕人笑话呢!只是这里扫帚颠倒竖,也没主子,
也没奴才,也没大老婆没小老婆:都是混账世界了。我们夏家门子里没见过这样规
矩,实在受不得你们家这样委屈了。”宝钗道:“大嫂子,妈妈因听见闹得慌才过
来的,就是问的急了些,没有分清‘奶奶’‘宝蟾’两字,也没有什么。如今且先
把事情说开,大家和和气气的过日子,也省了妈妈天天为咱们操心哪。”薛姨妈道:
“是啊,先把事情说开了,你再问我的不是,还不迟呢。”金桂道:“好姑娘,好
姑娘!你是个大贤大德的,你日后必定有个好人家好女婿,决不像我这样守活寡,
举眼无亲,叫人家骑上头来欺负的。我是个没心眼儿的人,只求姑娘,我说话,别
往死里挑捡!我从小儿到如今,没有爹娘教导。再者,我们屋里老婆、汉子、大女
人、小女人的事,姑娘也管不得!”宝钗听了这话,又是羞,又是气,见他母亲这
样光景,又是疼不过,因忍了气说道:“大嫂子,我劝你少说句儿罢。谁挑捡你?
又是谁欺负你?别说是嫂子啊,就是秋菱,我也从来没有加他一点声气儿啊。”金
桂听了这几句话,更加拍着炕檐大哭起来说:“我那里比得秋菱?连他脚底下的泥
我还跟不上呢!他是来久了的,知道姑娘的心事,又会献勤儿。我是新来的,又不
会献勤儿,如何拿我比他?何苦来!天下有几个都是贵妃的命?行点好儿罢。别修的
像我嫁个糊涂行子,守活寡,那就是活活儿的现了眼了!”薛姨妈听到这里,万分
气不过,便站起身来道:“不是我护着自己的女孩儿,他句句劝你,你却句句怄他。
你有什么过不去,不用寻他,勒死我倒也是希松的!”宝钗忙劝道:“妈妈,你老
人家不用动气。咱们既来劝他,自己生气,倒多了一层气。不如且去,等嫂子歇歇
儿再说。”因吩咐宝蟾道:“你也别闹了。”说着,跟了薛姨妈便出来了。

走过院子里,只见贾母身边的丫头同着秋菱迎面走来。薛姨妈道:“你从那里
来?老太太身上可安?”那丫头道:“老太太身上好,叫来请姨太太安,还谢谢前
儿的荔枝,还给琴姑娘道喜。”宝钗道:“你多早晚来的?”那丫头道:“来了好
一会子了。”薛姨妈料他知道,红着脸说道:“这如今,我们家里闹的也不像个过
日子的人家了,叫你们那边听见笑话。”丫头道:“姨太太说那里的话?谁家没个
碟大碗小磕着碰着的呢。那是姨太太多心罢咧。”说着,跟了回到薛姨妈房中,略
坐了一回就去了。宝钗正嘱咐香菱些话,只听薛姨妈忽然叫道:“左胁疼痛的很。”
说着,便向炕上躺下。唬得宝钗香菱二人手足无措。

要知后事如何,下回分解。