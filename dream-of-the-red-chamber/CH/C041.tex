\chapter{贾宝玉品茶栊翠庵~刘老老醉卧怡红院}

话说刘老老两只手比着说道:“花儿落了结个大倭瓜。”众人听了,哄堂大笑
起来。于是吃过门杯,因又斗趣笑道:“今儿实说罢,我的手脚子粗,又喝了酒,
仔细失手打了这磁杯。有木头的杯取个来,我就失了手,掉了地下也无碍。”众人
听了又笑起来。凤姐儿听如此说,便忙笑道:“果真要木头的,我就取了来,可有
一句话先说下:这木头的可比不得磁的,那都是一套,定要吃遍一套才算呢。”刘
老老听了,心下道:“我方才不过是趣话取笑儿,谁知他果真竟有。我时常在
乡绅大家也赴过席,金杯银杯倒都也见过,从没见有木头杯的。哦是了,想必是小
孩子们使的木碗儿,不过诓我多喝两碗。别管他,横竖这酒蜜水儿似的,多喝点子
也无妨。”想毕,便说“取来再商量”。凤姐因命丰儿:“前面里间书架子上,有
十个竹根套杯取来。”丰儿听了才要去取,鸳鸯笑道:“我知道,你那十个杯还小;
况且你才说木头的,这会子又拿了竹根的来,倒不好看。不如把我们那里的黄杨根
子整的十个大套杯拿来,灌他十下子。”凤姐儿笑道:“更好了。”

鸳鸯果命人取来。刘老老一看,又惊又喜:惊的是一连十个挨次大小分下来,
那大的足足的像个小盆子,极小的还有手里的杯子两个大;喜的是雕镂奇绝,一色
山水树木人物,并有草字以及图印。因忙说道:“拿了那小的来就是了。”凤姐儿
笑道:“这个杯,没有这大量的,所以没人敢使他。老老既要,好容易找出来,必
定要挨次吃一遍才使得。”刘老老吓的忙道:“这个不敢!好姑奶奶,饶了我罢。”
贾母、薛姨妈、王夫人知道他有年纪的人,禁不起,忙笑道:“说是说,笑是笑,
不可多吃了,只吃这头一杯罢。”刘老老道:“阿弥陀佛!我还是小杯吃罢,把这
大杯收着,我带了家去,慢慢的吃罢。”说的众人又笑起来。

鸳鸯无法,只得命人满斟了一大杯,刘老老两手捧着喝。贾母薛姨妈都道:“慢
些,别呛了。”薛姨妈又命凤姐儿布个菜儿。凤姐笑道:“老老要吃什么,说出名
儿来,我夹了喂你。”刘老老道:“我知道什么名儿!样样都是好的。”贾母笑道:
“把茄鲞夹些喂他。”凤姐儿听说,依言夹些茄鲞送入刘老老口中,因笑道:“你
们天天吃茄子,也尝尝我们这茄子,弄的可口不可口。”刘老老笑道:“别哄我了,
茄子跑出这个味儿来了,我们也不用种粮食,只种茄子了。”众人笑道:“真是茄
子,我们再不哄你。”刘老老诧异道:“真是茄子?我白吃了半日。姑奶奶再喂我
些,这一口细嚼嚼。”凤姐儿果又夹了些放入他口内。刘老老细嚼了半日,笑道:
“虽有一点茄子香,只是还不像是茄子。告诉我是个什么法子弄的,我也弄着吃
去。”凤姐儿笑道:“这也不难。你把才下来的茄子把皮刨了,只要净肉,切成碎
钉子,用鸡油炸了。再用鸡肉脯子合香菌、新笋、蘑菇、五香豆腐干子、各色干果
子,都切成钉儿,拿鸡汤煨干了,拿香油一收,外加糟油一拌,盛在磁罐子里封严
了。要吃的时候儿,拿出来,用炒的鸡瓜子一拌,就是了。”刘老老听了,摇头吐
舌说:“我的佛祖!倒得多少只鸡配他,怪道这个味儿。”一面笑,一面慢慢的吃
完了酒,还只管细玩那杯子。凤姐笑道:“还不足兴,再吃一杯罢?”刘老老忙道:
“了不得,那就醉死了。我因为爱这样儿好看,亏他怎么做来着!”鸳鸯笑道:“酒
喝完了,到底这杯子是什么木头的?”刘老老笑道:“怨不得姑娘不认得,你们在
这金门绣户里,那里认的木头?我们成日家和树林子做街坊,困了枕着他睡,乏了
靠着他坐,荒年间饿了还吃他;眼睛里天天见他,耳朵里天天听他,嘴儿里天天说
他,所以好歹真假,我是认得的。让我认认。”一面说,一面细细端详了半日,道:
“你们这样人家,断没有那贱东西,那容易得的木头你们也不收着了。我掂着这么
体,这再不是杨木,一定是黄松做的。”众人听了,哄堂大笑起来。

只见一个婆子走来,请问贾母说:“姑娘们都到了藕香榭,请示下:就演罢,
还是再等一会儿呢?”贾母忙笑道:“可是倒忘了,就叫他们演罢。”那婆子答应
去了。不一时,只听得箫管悠扬,笙笛并发;正值风清气爽之时,那乐声穿林度水
而来,自然使人神怡心旷。宝玉先禁不住,拿起壶来斟了一杯,一口饮尽,复又斟
上;才要饮,只见王夫人也要饮,命人换暖酒,宝玉连忙将自己的杯捧了过来,送
到王夫人口边,王夫人便就他手内吃了两口。一时暖酒来了,宝玉仍归旧坐。王夫
人提了暖壶下席来,众人都出了席,薛姨妈也站起来,贾母忙命李凤二人接过壶来:
“让你姨妈坐了,大家才便。”王夫人见如此说,方将壶递与凤姐儿,自己归坐。
贾母笑道:“大家吃上两杯,今日实在有趣。”说着,擎杯让薛姨妈,又向湘云宝
钗道:“你姐妹两个也吃一杯。你林妹妹不大会吃,也别饶他。”说着自己也干了,
湘云、宝钗、黛玉也都吃了。当下刘老老听见这般音乐,且又有了酒,越发喜的手
舞足蹈起来。宝玉因下席过来,向黛玉笑道:“你瞧刘老老的样子。”黛玉笑道:
“当日圣乐一奏,百兽率舞,如今才一牛耳。”众姐妹都笑了。

须臾乐止,薛姨妈笑道:“大家的酒也都有了,且出去散散再坐罢。”贾母也
正要散散,于是大家出席,都随着贾母游玩。贾母因要带着刘老老散闷,遂携了刘
老老至山前树下,盘桓了半晌,又说给他这是什么树,这是什么石,这是什么花。
刘老老一一领会,又向贾母道:“谁知城里不但人尊贵,连雀儿也是尊贵的。偏这
雀儿到了你们这里,他也变俊了,也会说话了。”众人不解,因问:“什么雀儿变
俊了会说话?”刘老老道:“那廊上金架子上站的绿毛红嘴是鹦哥儿,我是认得的。
那笼子里的黑老鸹子,又长出凤头儿来,也会说话呢!”众人听了又都笑起来。

一时只见丫头们来请用点心,贾母道:“吃了两杯酒,倒也不饿。也罢,就拿
了来这里,大家随便吃些罢。”丫头听说,便去抬了两张几来,又端了两个小捧盒。
揭开看时,每个盒内两样。这盒内是两样蒸食:一样是藕粉桂花糖糕,一样是松瓤
鹅油卷。那盒内是两样炸的:一样是只有一寸来大的小饺儿。贾母因问:“什么馅
子?”婆子们忙回:“是螃蟹的。”贾母听了,皱眉说道:“这会子油腻腻的,谁
吃这个。”又看那一样,是奶油炸的各色小面果子。也不喜欢,因让薛姨妈,薛姨
妈只拣了块糕。贾母拣了个卷子,只尝了一尝,剩的半个,递给丫头了。刘老老因
见那小面果子儿都玲珑剔透,各式各样,又拣了一朵牡丹花样的,笑道:“我们乡
里最巧的姐儿们,剪子也不能铰出这么个纸的来。我又爱吃,又舍不得吃,包些家
去给他们做花样子去倒好。”众人都笑了。贾母笑道:“家去我送你一磁坛子,你
先趁热吃罢。”别人不过拣各人爱吃的拣了一两样就算了,刘老老原不曾吃过这些
东西,且都做的小巧,不显堆垛儿,他和板儿每样吃了些个,就去了半盘子。剩的,
凤姐又命攒了两盘,并一个攒盒,给文官儿等吃去。

忽见奶子抱了大姐儿来,大家哄他玩了一会。那大姐儿因抱着一个大柚子玩,
忽见板儿抱着一个佛手,大姐儿便要。丫鬟哄他取去,大姐儿等不得,便哭了。众
人忙把柚子给了板儿,将板儿的佛手哄过来给他才罢。那板儿因玩了半日佛手,此
刻又两手抓着些果子吃,又见这个柚子又香又圆,更觉好玩,且当球踢着玩去,也
就不要佛手了。

当下贾母等吃过了茶,又带了刘老老至栊翠庵来。妙玉相迎进去。众人至院中,
见花木繁盛,贾母笑道:“到底是他们修行的人,没事常常修理,比别处越发好看。”
一面说,一面便往东禅堂来。妙玉笑往里让,贾母道:“我们才都吃了酒肉,你这
里头有菩萨,冲了罪过。我们这里坐坐,把你的好茶拿来,我们吃一杯就去了。”
宝玉留神看他是怎么行事,只见妙玉亲自捧了一个海棠花式雕漆填金“云龙献寿”
的小茶盘,里面放一个成窑五彩小盖钟,捧与贾母。贾母道:“我不吃六安茶。”
妙玉笑说:“知道。这是‘老君眉’。”贾母接了,又问:“是什么水?”妙玉道:
“是旧年蠲的雨水。”贾母便吃了半盏,笑着递与刘老老,说:“你尝尝这个茶。”
刘老老便一口吃尽,笑道:“好是好,就是淡些,再熬浓些更好了。”贾母众人都
笑起来。然后众人都是一色的官窑脱胎填白盖碗。

那妙玉便把宝钗黛玉的衣襟一拉,二人随他出去。宝玉悄悄的随后跟了来。只
见妙玉让他二人在耳房内,宝钗便坐在榻上,黛玉便坐在妙玉的蒲团上。妙玉自向
风炉上煽滚了水,另泡了一壶茶。宝玉便轻轻走进来,笑道:“你们吃体己茶呢!”
二人都笑道:“你又赶了来撤茶吃!这里并没你吃的。”妙玉刚要去取杯,只见道
婆收了上面茶盏来,妙玉忙命:“将那成窑的茶杯别收了,搁在外头去罢。”宝玉
会意,知为刘老老吃了,他嫌腌不要了。又见妙玉另拿出两只杯来,一个旁边有
一耳,杯上镌着“”三个隶字,后有一行小真字,是“王恺珍玩”;又有“宋
元丰五年四月眉山苏轼见于秘府”一行小字。妙玉斟了一递与宝钗。那一只形似
钵而小,也有三个垂珠篆字,镌着“点犀”。妙玉斟了一与黛玉,仍将前番自
己常日吃茶的那只绿玉斗来斟与宝玉。宝玉笑道:“常言‘世法平等’:他两个就
用那样古玩奇珍,我就是个俗器了?”妙玉道:“这是俗器?不是我说狂话,只怕
你家里未必找的出这么一个俗器来呢!”宝玉笑道:“俗语说:‘随乡入乡’,到
了你这里,自然把这金珠玉宝一概贬为俗器了。”妙玉听如此说,十分欢喜,遂又
寻出一只九曲十环一百二十节蟠虬整雕竹根的一个大盏出来,笑道:“就剩了这一
个,你可吃的了这一海?”宝玉喜的忙道:“吃的了。”妙玉笑道:“你虽吃的了,
也没这些茶你遭塌。岂不闻一杯为品,二杯即是解渴的蠢物,三杯便是饮驴了。你
吃这一海,更成什么?”说的宝钗、黛玉、宝玉都笑了。妙玉执壶,只向海内斟了
约有一杯。宝玉细细吃了,果觉轻淳无比,赏赞不绝。妙玉正色道:“你这遭吃茶,
是托他两个的福,独你来了,我是不能给你吃的。”宝玉笑道:“我深知道,我也
不领你的情,只谢他二人便了。”妙玉听了,方说:“这话明白。”

黛玉因问:“这也是旧年的雨水?”妙玉冷笑道:“你这么个人,竟是大俗人,
连水也尝不出来!这是五年前我在玄墓蟠香寺住着,收的梅花上的雪,统共得了那
一鬼脸青的花瓮一瓮,总舍不得吃,埋在地下,今年夏天才开了。我只吃过一回,
这是第二回了。你怎么尝不出来?隔年蠲的雨水,那有这样清淳?如何吃得!”宝钗
知他天性怪僻,不好多话,亦不好多坐,吃过茶,便约着黛玉走出来。宝玉和妙玉
陪笑说道:“那茶杯虽然腌了,白撩了岂不可惜?依我说,不如就给了那贫婆子
罢,他卖了也可以度日。你说使得么?”妙玉听了,想了一想,点头说道:“这也
罢了。幸而那杯子是我没吃过的;若是我吃过的,我就砸碎了也不能给他。你要给
他,我也不管,你只交给他快拿了去罢。”宝玉道:“自然如此。你那里和他说话
去?越发连你都腌了。只交给我就是了。”妙玉便命人拿来递给宝玉。宝玉接了,
又道:“等我们出去了,我叫几个小么儿来河里打几桶水来洗地如何?”妙玉笑道:
“这更好了。只是你嘱咐他们,抬了水,只搁在山门外头墙根下,别进门来。”宝
玉道:“这是自然的。”说着,便袖着那杯递给贾母屋里的小丫头子拿着,说:“明
日刘老老家去,给他带去罢。”交代明白,贾母已经出来要回去。妙玉亦不甚留,
送出山门,回身便将门闭了,不在话下。

且说贾母因觉身上乏倦,便命王夫人和迎春姐妹陪着薛姨妈去吃酒,自己便往
稻香村来歇息。凤姐忙命人将小竹椅抬来,贾母坐上,两个婆子抬起,凤姐李纨和
众丫头婆子围随去了,不在话下。这里薛姨妈也就辞出。王夫人打发文官等出去,
将攒盒散给众丫头们吃去,自己便也乘空歇着,随便歪在方才贾母坐的榻上,命一
个小丫头放下帘子来,又命捶着腿,吩咐他:“老太太那里有信,你就叫我。”说
着也歪着睡着了。宝玉湘云等看着丫头们将攒盒搁在山石上,也有坐在山石上的,
也有坐在草地下的,也有靠着树的,也有傍着水的,倒也十分热闹。

一时又见鸳鸯来了,要带着刘老老逛,众人也都跟着取笑。一时来至省亲别墅
的牌坊底下,刘老老道:“嗳呀!这里还有大庙呢。”说着,便爬下磕头。众人笑
弯了腰。刘老老道:“笑什么?这牌楼上的字我都认得。我们那里这样庙宇最多,
都是这样的牌坊,那字就是庙的名字。”众人笑道:“你认得这是什么庙?”刘老
老便抬头指那字道:“这不是‘玉皇宝殿’!”众人笑的拍手打掌,还要拿他取笑
儿。刘老老觉的肚里一阵乱响,忙的拉着一个丫头,要了两张纸,就解裙子。众人
又是笑,又忙喝他:“这里使不得!”忙命一个婆子,带了东北角上去了。那婆子
指给他地方,便乐得走开去歇息。那刘老老因喝了些酒,他的脾气和黄酒不相宜,
且吃了许多油腻饮食发渴,多喝了几碗茶,不免通泻起来,蹲了半日方完。及出厕
来,酒被风吹,且年迈之人,蹲了半天,忽一起身,只觉眼花头晕,辨不出路径。
四顾一望,都是树木山石,楼台房舍,却不知那一处是往那一路去的了,只得顺着
一条石子路慢慢的走来。及至到了房子跟前又找不着门,再找了半日,忽见一带竹
篱。刘老老心中自忖道:“这里也有扁豆架子?”一面想,一面顺着花障走来,得
了个月洞门进去。

只见迎面一带水池,有七八尺宽,石头镶岸,里面碧波清水,上面有块白石横
架。刘老老便踱过石去,顺着石子甬路走去,转了两个弯子,只见有个房门。于是
进了房门,便见迎面一个女孩儿,满面含笑的迎出来。刘老老忙笑道:“姑娘们把
我丢下了,叫我碰头碰到这里来了。”说了,只觉那女孩儿不答。刘老老便赶来拉
他的手,咕咚一声却撞到板壁上,把头碰的生疼。细瞧了一瞧,原来是一幅画儿。
刘老老自忖道:“怎么画儿有这样凸出来的?”一面想,一面看,一面又用手摸去,
却是一色平的,点头叹了两声。一转身,方得了个小门,门上挂着葱绿撒花软帘,
刘老老掀帘进去。抬头一看,只见四面墙壁玲珑剔透,琴剑瓶炉皆贴在墙上,锦笼
纱罩,金彩珠光,连地下踩的砖皆是碧绿凿花,竟越发把眼花了。找门出去,那里
有门?左一架书、右一架屏。刚从屏后得了一个门,只见一个老婆子也从外面迎着
进来。刘老老诧异,心中恍惚:莫非是他亲家母?因问道:“你也来了,想是见我
这几日没家去?亏你找我来,那位姑娘带进来的?”又见他戴着满头花,便笑道:
“你好没见世面!见这里的花好,你就没死活戴了一头。”说着,那老婆子只是笑,
也不答言。刘老老便伸手去羞他的脸,他也拿手来挡,两个对闹着。刘老老一下子
却摸着了,但觉那老婆子的脸冰凉挺硬的,倒把刘老老唬了一跳。猛想起:“常听
见富贵人家有种穿衣镜,这别是我在镜子里头吗?”想毕,又伸手一抹,再细一看,
可不是四面雕空的板壁,将这镜子嵌在中间的,不觉也笑了。因说:“这可怎么出
去呢?”一面用手摸时,只听“硌磴”一声,又吓的不住的展眼儿。原来是西洋机
括,可以开合,不意刘老老乱摸之间,其力巧合,便撞开消息,掩过镜子,露出门
来。刘老老又惊又喜,遂走出来,忽见有一副最精致的床帐。他此时又带了七八分
酒,又走乏了,便一屁股坐在床上。只说歇歇,不承望身不由己,前仰后合的,朦
胧两眼,一歪身就睡倒在床上。

且说众人等他不见,板儿没了他老老,急的哭了。众人都笑道:“别是掉在茅
厕里了?快叫人去瞧瞧。”因命两个婆子去找。回来说:“没有。”众人纳闷。还
是袭人想道:“一定他醉了,迷了路,顺着这条路往我们后院子里去了。要进了花
障子,打后门进去,还有小丫头子们知道;若不进花障子,再往西南上去,可够他
绕会子好的了!我瞧瞧去。”说着便回来。进了怡红院,叫人,谁知那几个小丫头
已偷空玩去了。

袭人进了房门,转过集锦子,就听的鼾如雷,忙进来,只闻见酒屁臭气满
屋。一瞧,只见刘老老扎手舞脚的仰卧在床上。袭人这一惊不小,忙上来将他没死
活的推醒。那刘老老惊醒,睁眼看见袭人,连忙爬起来,道:“姑娘,我该死了!
好歹并没弄腌了床。”一面说,用手去掸。袭人恐惊动了宝玉,只向他摇手儿,
不叫他说话。忙将当地大鼎内贮了三四把百合香,仍用罩子罩上。所喜不曾呕吐。
忙悄悄的笑道:“不相干,有我呢。你跟我出来罢。”刘老老答应着,跟了袭人,
出至小丫头子们房中,命他坐下,因教他说道:“你说‘醉倒在山子石上,打了个
盹儿’就完了。”刘老老答应“是”。又给了他两碗茶吃,方觉酒醒了。因问道:
“这是那个小姐的绣房?这么精致!我就像到了天宫里的似的。”袭人微微的笑道:
“这个么,是宝二爷的卧房啊。”那刘老老吓的不敢做声。袭人带他从前面出去,
见了众人,只说:“他在草地下睡着了,带了他来的。”众人都不理会,也就罢了。

一时贾母醒了,就在稻香村摆晚饭。贾母因觉懒懒的,也没吃饭,便坐了竹椅
小敞轿,回至房中歇息,命凤姐儿等去吃饭。他姐妹方复进园来。

未知如何,且看下回分解。