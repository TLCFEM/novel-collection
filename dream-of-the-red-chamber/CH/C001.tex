\chapter{甄士隐梦幻识通灵~贾雨村风尘怀闺秀}

——此开卷第一回也。作者自云:曾历过一番梦幻之后,故将真事隐去,而借
通灵说此《石头记》一书也,故曰“甄士隐”云云。但书中所记何事何人?自己又
云:“今风尘碌碌,一事无成,忽念及当日所有之女子:一一细考较去,觉其行止
见识皆出我之上。我堂堂须眉诚不若彼裙钗,我实愧则有馀,悔又无益,大无可如
何之日也。当此日,欲将已往所赖天恩祖德,锦衣纨之时,饫甘餍肥之日,背父
兄教育之恩,负师友规训之德,以致今日一技无成、半生潦倒之罪,编述一集,以
告天下;知我之负罪固多,然闺阁中历历有人,万不可因我之不肖,自护己短,一
并使其泯灭也。所以蓬牖茅椽,绳床瓦灶,并不足妨我襟怀;况那晨风夕月,阶柳
庭花,更觉得润人笔墨。我虽不学无文,又何妨用假语村言敷演出来?亦可使闺阁
昭传。复可破一时之闷,醒同人之目,不亦宜乎?”故曰“贾雨村”云云。更于篇
中间用“梦”“幻”等字,却是此书本旨,兼寓提醒阅者之意。

看官你道此书从何而起?说来虽近荒唐,细玩颇有趣味。却说那女娲氏炼石补
天之时,于大荒山无稽崖炼成高十二丈、见方二十四丈大的顽石三万六千五百零一
块。那娲皇只用了三万六千五百块,单单剩下一块未用,弃在青埂峰下。谁知此石
自经锻炼之后,灵性已通,自去自来,可大可小。因见众石俱得补天,独自己无才
不得入选,遂自怨自愧,日夜悲哀。一日正当嗟悼之际,俄见一僧一道远远而来,
生得骨格不凡,丰神迥异,来到这青埂峰下,席地坐谈。见着这块鲜莹明洁的石头,
且又缩成扇坠一般,甚属可爱。那僧托于掌上,笑道:“形体倒也是个灵物了,只
是没有实在的好处。须得再镌上几个字,使人人见了便知你是件奇物,然后携你到
那昌明隆盛之邦、诗礼簪缨之族、花柳繁华地、温柔富贵乡那里去走一遭。”石头
听了大喜,因问:“不知可镌何字?携到何方?望乞明示。”那僧笑道:“你且莫问,
日后自然明白。”说毕,便袖了,同那道人飘然而去,竟不知投向何方。

又不知过了几世几劫,因有个空空道人访道求仙,从这大荒山无稽崖青埂峰下
经过。忽见一块大石,上面字迹分明,编述历历。空空道人乃从头一看,原来是无
才补天、幻形入世,被那茫茫大士、渺渺真人携入红尘、引登彼岸的一块顽石;上
面叙着堕落之乡、投胎之处,以及家庭琐事、闺阁闲情、诗词谜语,倒还全备。只
是朝代年纪,失落无考。后面又有一偈云:
无才可去补苍天,枉入红尘若许年。
此系身前身后事,倩谁记去作奇传?
空空道人看了一回,晓得这石头有些来历,遂向石头说道:“石兄,你这一段故事,
据你自己说来,有些趣味,故镌写在此,意欲闻世传奇。据我看来:第一件,无朝
代年纪可考;第二件,并无大贤大忠、理朝廷、治风俗的善政,其中只不过几个异
样女子,或情或痴,或小才微善。我纵然抄去,也算不得一种奇书。”石头果然答
道:“我师何必太痴!我想历来野史的朝代,无非假借汉、唐的名色;莫如我这石头
所记不借此套,只按自己的事体情理,反倒新鲜别致。况且那野史中,或讪谤君相,
或贬人妻女,奸淫凶恶,不可胜数;更有一种风月笔墨,其淫秽污臭最易坏人子弟。
至于才子佳人等书,则又开口‘文君’,满篇‘子建’,千部一腔,千人一面,且终
不能不涉淫滥。在作者不过要写出自己的两首情诗艳赋来,故假捏出男女二人名姓;
又必旁添一小人拨乱其间,如戏中的小丑一般。更可厌者,‘之乎者也’,非理即文,
大不近情,自相矛盾。竟不如我这半世亲见亲闻的几个女子,虽不敢说强似前代书
中所有之人,但观其事迹原委,亦可消愁破闷;至于几首歪诗,也可以喷饭供酒。
其间离合悲欢,兴衰际遇,俱是按迹循踪,不敢稍加穿凿,至失其真。只愿世人当
那醉馀睡醒之时,或避事消愁之际,把此一玩,不但是洗旧翻新,却也省了些寿命
筋力,不更去谋虚逐妄了。我师意为如何?”

空空道人听如此说,思忖半晌,将这《石头记》再检阅一遍。因见上面大旨不
过谈情,亦只是实录其事,绝无伤时诲淫之病,方从头至尾抄写回来,闻世传奇。
从此空空道人因空见色,由色生情,传情入色,自色悟空,遂改名情僧,改《石头
记》为《情僧录》。东鲁孔梅溪题曰《风月宝鉴》。后因曹雪芹于悼红轩中,披阅十
载,增删五次,纂成目录,分出章回,又题曰《金陵十二钗》,并题一绝。即此便
是《石头记》的缘起。诗云:
满纸荒唐言,一把辛酸泪。
都云作者痴,谁解其中味!

《石头记》缘起既明,正不知那石头上面记着何人何事?看官请听。按那石上
书云:当日地陷东南,这东南有个姑苏城,城中阊门,最是红尘中一二等富贵风流
之地。这阊门外有个十里街,街内有个仁清巷,巷内有个古庙,因地方狭窄,人皆
呼作“葫芦庙”。庙旁住着一家乡宦,姓甄名费字士隐,嫡妻封氏,性情贤淑,深
明礼义。家中虽不甚富贵,然本地也推他为望族了。因这甄士隐禀性恬淡,不以功
名为念,每日只以观花种竹、酌酒吟诗为乐,倒是神仙一流人物。只是一件不足:
年过半百,膝下无儿,只有一女乳名英莲,年方三岁。

一日炎夏永昼,士隐于书房闲坐,手倦抛书,伏几盹睡,不觉朦胧中走至一处,
不辨是何地方。忽见那厢来了一僧一道,且行且谈。只听道人问道:“你携了此物,
意欲何往?”那僧笑道:“你放心,如今现有一段风流公案正该了结,这一干风流
冤家尚未投胎入世。趁此机会,就将此物夹带于中,使他去经历经历。”那道人道:
“原来近日风流冤家又将造劫历世,但不知起于何处,落于何方?”那僧道:“此
事说来好笑。只因当年这个石头,娲皇未用,自己却也落得逍遥自在,各处去游玩。
一日来到警幻仙子处,那仙子知他有些来历,因留他在赤霞宫中,名他为赤霞宫神
瑛侍者。他却常在西方灵河岸上行走,看见那灵河岸上三生石畔有棵绛珠仙草,十
分娇娜可爱,遂日以甘露灌溉,这绛珠草始得久延岁月。后来既受天地精华,复得
甘露滋养,遂脱了草木之胎,幻化人形,仅仅修成女体,终日游于离恨天外,饥餐
秘情果,渴饮灌愁水。只因尚未酬报灌溉之德,故甚至五内郁结着一段缠绵不尽之
意。常说:‘自己受了他雨露之惠,我并无此水可还。他若下世为人,我也同去走
一遭,但把我一生所有的眼泪还他,也还得过了。’因此一事,就勾出多少风流冤
家都要下凡,造历幻缘,那绛珠仙草也在其中。今日这石正该下世,我来特地将他
仍带到警幻仙子案前,给他挂了号,同这些情鬼下凡,一了此案。”那道人道:“果
是好笑,从来不闻有‘还泪’之说。趁此你我何不也下世度脱几个,岂不是一场功
德?”那僧道:“正合吾意。你且同我到警幻仙子宫中将这蠢物交割清楚,待这一
干风流孽鬼下世,你我再去。如今有一半落尘,然犹未全集。”道人道:“既如此,
便随你去来。”

却说甄士隐俱听得明白,遂不禁上前施礼,笑问道:“二位仙师请了。”那僧道
也忙答礼相问。士隐因说道:“适闻仙师所谈因果,实人世罕闻者,但弟子愚拙,
不能洞悉明白。若蒙大开痴顽,备细一闻,弟子洗耳谛听,稍能警省,亦可免沉沦
之苦了。”二仙笑道:“此乃玄机,不可预泄。到那时只不要忘了我二人,便可跳出
火坑矣。”士隐听了,不便再问,因笑道:“玄机固不可泄露,但适云‘蠢物’,不
知为何,或可得见否?”那僧说:“若问此物,倒有一面之缘。”说着取出递与士隐。
士隐接了看时,原来是块鲜明美玉,上面字迹分明,镌着“通灵宝玉”四字,后面
还有几行小字。正欲细看时,那僧便说“已到幻境”,就强从手中夺了去,和那道
人竟过了一座大石牌坊,上面大书四字,乃是“太虚幻境”。两边又有一副对联道:
假作真时真亦假,
无为有处有还无。

士隐意欲也跟着过去,方举步时,忽听一声霹雳若山崩地陷,士隐大叫一声,
定睛看时,只见烈日炎炎,芭蕉冉冉,梦中之事便忘了一半。又见奶母抱了英莲走
来。士隐见女儿越发生得粉装玉琢,乖觉可喜,便伸手接来抱在怀中斗他玩耍一回;
又带至街前,看那过会的热闹。方欲进来时,只见从那边来了一僧一道。那僧癞头
跣足,那道跛足蓬头,疯疯癫癫,挥霍谈笑而至。及到了他门前,看见士隐抱着英
莲,那僧便大哭起来,又向士隐道:“施主,你把这有命无运、累及爹娘之物抱在
怀内作甚!”士隐听了,知是疯话,也不睬他。那僧还说:“舍我罢!舍我罢!”士隐
不耐烦,便抱着女儿转身。才要进去,那僧乃指着他大笑,口内念了四句言词,道
是:
惯养娇生笑你痴,菱花空对雪澌澌。
好防佳节元宵后,便是烟消火灭时。
士隐听得明白,心下犹豫,意欲问他来历。只听道人说道:“你我不必同行,就此
分手,各干营生去罢。三劫后我在北邙山等你,会齐了同往太虚幻境销号。”那僧
道:“最妙,最妙!”说毕,二人一去,再不见个踪影了。

士隐心中此时自忖:这两个人必有来历,很该问他一问,如今后悔却已晚了。
这士隐正在痴想,忽见隔壁葫芦庙内寄居的一个穷儒,姓贾名化、表字时飞、别号
雨村的走来。这贾雨村原系湖州人氏,也是诗书仕宦之族。因他生于末世,父母祖
宗根基已尽,人口衰丧,只剩得他一身一口。在家乡无益,因进京求取功名,再整
基业。自前岁来此,又淹蹇住了,暂寄庙中安身,每日卖文作字为生,故士隐常与
他交接。当下雨村见了士隐,忙施礼陪笑道:“老先生倚门伫望,敢街市上有甚新
闻么?”士隐笑道:“非也。适因小女啼哭,引他出来作耍,正是无聊的很。贾兄
来得正好,请入小斋,彼此俱可消此永昼。”说着便令人送女儿进去,自携了雨村
来至书房中,小童献茶。方谈得三五句话,忽家人飞报:“严老爷来拜。”士隐慌忙
起身谢道:“恕诓驾之罪,且请略坐,弟即来奉陪。”雨村起身也让道:“老先生请
便。晚生乃常造之客,稍候何妨。”说着士隐已出前厅去了。

这里雨村且翻弄诗籍解闷,忽听得窗外有女子嗽声。雨村遂起身往外一看,原
来是一个丫鬟在那里掐花儿,生的仪容不俗,眉目清秀,虽无十分姿色,却也有动
人之处。雨村不觉看得呆了。那甄家丫鬟掐了花儿方欲走时,猛抬头见窗内有人:
敝巾旧服,虽是贫窘,然生得腰圆背厚,面阔口方,更兼剑眉星眼,直鼻方腮。这
丫鬟忙转身回避,心下自想:“这人生的这样雄壮,却又这样褴褛,我家并无这样
贫窘亲友。想他定是主人常说的什么贾雨村了,怪道又说他‘必非久困之人,每每
有意帮助周济他,只是没什么机会。’”如此一想,不免又回头一两次。雨村见他回
头,便以为这女子心中有意于他,遂狂喜不禁,自谓此女子必是个巨眼英豪、风尘
中之知己。一时小童进来,雨村打听得前面留饭,不可久待,遂从夹道中自便门出
去了。士隐待客既散,知雨村已去,便也不去再邀。

一日到了中秋佳节,士隐家宴已毕,又另具一席于书房,自己步月至庙中来邀
雨村。原来雨村自那日见了甄家丫鬟曾回顾他两次,自谓是个知己,便时刻放在心
上。今又正值中秋,不免对月有怀,因而口占五言一律云:
未卜三生愿,频添一段愁。
闷来时敛额,行去几回眸。
自顾风前影,谁堪月下俦?
蟾光如有意,先上玉人头。
雨村吟罢,因又思及平生抱负,苦未逢时,乃又搔首对天长叹,复高吟一联云:
玉在椟中求善价,钗于奁内待时飞。

恰值士隐走来听见,笑道:“雨村兄真抱负不凡也!”雨村忙笑道:“不敢,不
过偶吟前人之句,何期过誉如此。”因问:“老先生何兴至此?”士隐笑道:“今夜
中秋,俗谓团圆之节,想尊兄旅寄僧房,不无寂寥之感。故特具小酌邀兄到敝斋一
饮,不知可纳芹意否?”雨村听了,并不推辞,便笑道:“既蒙谬爱,何敢拂此盛
情。”说着便同士隐复过这边书院中来了。

须臾茶毕,早已设下杯盘,那美酒佳肴自不必说。二人归坐,先是款酌慢饮,
渐次谈至兴浓,不觉飞觥献起来。当时街坊上家家箫管,户户笙歌,当头一轮明
月,飞彩凝辉。二人愈添豪兴,酒到杯干。雨村此时已有七八分酒意,狂兴不禁,
乃对月寓怀,口占一绝云:
时逢三五便团,满把清光护玉栏。
天上一轮才捧出,人间万姓仰头看。
士隐听了大叫:“妙极!弟每谓兄必非久居人下者,今所吟之句,飞腾之兆已见,不
日可接履于云霄之上了。可贺可贺!”乃亲斟一斗为贺。雨村饮干,忽叹道:“非晚
生酒后狂言,若论时尚之学,晚生也或可去充数挂名。只是如今行李路费一概无措,
神京路远,非赖卖字撰文即能到得。”士隐不待说完,便道:“兄何不早言!弟已久
有此意,但每遇兄时并未谈及,故未敢唐突。今既如此,弟虽不才:‘义利’二字
却还识得;且喜明岁正当大比,兄宜作速入都,春闱一捷,方不负兄之所学。其盘
费馀事弟自代为处置,亦不枉兄之谬识矣。”当下即命小童进去速封五十两白银并
两套冬衣,又云:“十九日乃黄道之期,兄可即买舟西上。待雄飞高举,明冬再晤,
岂非大快之事!”雨村收了银衣,不过略谢一语,并不介意,仍是吃酒谈笑。那天
已交三鼓,二人方散。

士隐送雨村去后,回房一觉,直至红日三竿方醒。因思昨夜之事,意欲写荐书
两封与雨村带至都中去,使雨村投谒个仕宦之家为寄身之地。因使人过去请时,那
家人回来说:“和尚说,贾爷今日五鼓已进京去了,也曾留下话与和尚转达老爷,
说:‘读书人不在黄道黑道,总以事理为要,不及面辞了。’”士隐听了,也只得罢
了。

真是闲处光阴易过,倏忽又是元宵佳节。士隐令家人霍启抱了英莲,去看社火
花灯。半夜中霍启因要小解,便将英莲放在一家门槛上坐着。待他小解完了来抱时,
那有英莲的踪影?急的霍启直寻了半夜。至天明不见,那霍启也不敢回来见主人,
便逃往他乡去了。那士隐夫妇见女儿一夜不归,便知有些不好;再使几人去找寻,
回来皆云影响全无。夫妻二人半世只生此女,一旦失去,何等烦恼,因此昼夜啼哭,
几乎不顾性命。

看看一月,士隐已先得病,夫人封氏也因思女构疾,日日请医问卦。不想这日
三月十五,葫芦庙中炸供,那和尚不小心,油锅火逸,便烧着窗纸。此方人家俱用
竹篱木壁,也是劫数应当如此,于是接二连三牵五挂四,将一条街烧得如火焰山一
般。彼时虽有军民来救,那火已成了势了,如何救得下?直烧了一夜方息,也不知
烧了多少人家。只可怜甄家在隔壁,早成了一堆瓦砾场了,只有他夫妇并几个家人
的性命不曾伤了。急的士隐惟跌足长叹而已。与妻子商议,且到田庄上去住。偏值
近年水旱不收,贼盗蜂起,官兵剿捕,田庄上又难以安身,只得将田地都折变了,
携了妻子与两个丫鬟投他岳丈家去。

他岳丈名唤封肃,本贯大如州人氏,虽是务农,家中却还殷实。今见女婿这等
狼狈而来,心中便有些不乐。幸而士隐还有折变田产的银子在身边,拿出来托他随
便置买些房地,以为后日衣食之计,那封肃便半用半赚的,略与他些薄田破屋。士
隐乃读书之人,不惯生理稼穑等事,勉强支持了一二年,越发穷了。封肃见面时,
便说些现成话儿;且人前人后又怨他不会过,只一味好吃懒做。士隐知道了,心中
未免悔恨,再兼上年惊唬,急忿怨痛,暮年之人,那禁得贫病交攻,竟渐渐的露出
了那下世的光景来。

可巧这日拄了拐扎挣到街前散散心时,忽见那边来了一个跛足道人,疯狂落拓,
麻鞋鹑衣,口内念着几句言词道:
世人都晓神仙好,惟有功名忘不了。
古今将相在何方?荒冢一堆草没了。
世人都晓神仙好,只有金银忘不了。
终朝只恨聚无多,及到多时眼闭了。
世人都晓神仙好,只有娇妻忘不了。
君生日日说恩情,君死又随人去了。
世人都晓神仙好,只有儿孙忘不了。
痴心父母古来多,孝顺子孙谁见了?
士隐听了,便迎上来道:“你满口说些什么?只听见些‘好’‘了’‘好’‘了’。”那
道人笑道:“你若果听见‘好’‘了’二字,还算你明白:可知世上万般,好便是了,
了便是好。若不了,便不好;若要好,须是了。我这歌儿便叫《好了歌》。”士隐本
是有夙慧的,一闻此言,心中早已悟彻,因笑道:“且住,待我将你这《好了歌》
注解出来何如?”道人笑道:“你就请解。”士隐乃说道:

陋室空堂,当年笏满床。衰草枯杨,曾为歌舞场。蛛丝儿结满雕粱,绿纱今又
在蓬窗上。说甚么脂正浓、粉正香,如何两鬓又成霜?昨日黄土陇头埋白骨,今宵
红绡帐底卧鸳鸯。
金满箱,银满箱,转眼乞丐人皆谤。正叹他人命不长,那知自己归来丧?训有方,
保不定日后作强梁。择膏粱,谁承望流落在烟花巷!因嫌纱帽小,致使锁枷扛。昨
怜破袄寒,今嫌紫蟒长:乱烘烘你方唱罢我登场,反认他乡是故乡。甚荒唐,到头
来都是“为他人作嫁衣裳”。

那疯跛道人听了,拍掌大笑道:“解得切!解得切!”士隐便说一声“走罢”,将
道人肩上的搭裢抢过来背上,竟不回家,同着疯道人飘飘而去。当下哄动街坊,众
人当作一件新闻传说。封氏闻知此信,哭个死去活来。只得与父亲商议,遣人各处
访寻,那讨音信?无奈何,只得依靠着他父母度日。幸而身边还有两个旧日的丫鬟
伏侍,主仆三人,日夜作些针线,帮着父亲用度。那封肃虽然每日抱怨,也无可奈
何了。

这日那甄家的大丫鬟在门前买线,忽听得街上喝道之声。众人都说:“新太爷
到任了!”丫鬟隐在门内看时,只见军牢快手一对一对过去,俄而大轿内抬着一个
乌帽猩袍的官府来了。那丫鬟倒发了个怔,自思:“这官儿好面善?倒像在那里见过
的。”于是进入房中,也就丢过不在心上。至晚间正待歇息之时,忽听一片声打的
门响,许多人乱嚷,说:“本县太爷的差人来传人问话!”封肃听了,唬得目瞪口呆。

不知有何祸事,且听下回分解。