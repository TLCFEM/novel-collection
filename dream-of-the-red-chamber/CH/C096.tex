\chapter{瞒消息凤姐设奇谋~泄机关颦儿迷本性}

话说贾琏拿了那块假玉忿忿走出,到了书房。那个人看见贾琏的气色不好,心
里先发了虚了,连忙站起来迎着。刚要说话,只见贾琏冷笑道:“好大胆!我把你
这个混账东西!这里是什么地方儿,你敢来掉鬼!”回头便问:“小厮们呢?”外
头轰雷一般,几个小厮齐声答应。贾琏道:“取绳子去捆起他来!等老爷回来回明
了,把他送到衙门里去。”众小厮又一齐答应:“预备着呢。”嘴里虽如此,却不
动身。那人先自唬的手足无措,见这般势派,知道难逃公道,只得跪下给贾琏碰头,
口口声声只叫:“老太爷别生气!是我一时穷极无奈,才想出这个没脸的营生来。
那玉是我借钱做的,我也不敢要了,只得孝敬府里的哥儿玩罢。”说毕,又连连磕
头。贾琏啐道:“你这个不知死活的东西!这府里希罕你的那扔不了的浪东西!”
正闹着,只见赖大进来,陪着笑向贾琏道:“二爷别生气了。靠他算个什么东西!
饶了他,叫他滚出去罢。”贾琏道:“实在可恶!”赖大贾琏作好作歹,众人在外
头都说道:“糊涂狗攮的,还不给爷和赖大爷磕头呢!快快的滚罢,还等窝心脚呢。”
那人赶忙磕了两个头,抱头鼠窜而去。从此,街上闹动了:“贾宝玉弄出‘假宝玉’
来。”

且说贾政那日拜客回来,众人因为灯节底下,恐怕贾政生气,已过去的事了,
便也都不肯回。只因元妃的事,忙碌了好些时,近日宝玉又病着,虽有旧例家宴,
大家无兴,也无有可记之事。

到了正月十七日,王夫人正盼王子腾来京,只见凤姐进来回说:“今日二爷在
外听得有人传说:我们家大老爷赶着进京,离城只二百多里地,在路上没了!太太
听见了没有?”王夫人吃惊道:“我没有听见,老爷昨晚也没有说起。到底在那里
听见的?”凤姐道:“说是在枢密张老爷家听见的。”王夫人怔了半天,那眼泪早
流下来了,因拭泪说道:“回来再叫琏儿索性打听明白了来告诉我。”凤姐答应去
了。

王夫人不免暗里落泪,悲女哭弟,又为宝玉耽忧。如此连三接二,都是不随意
的事,那里搁得住?便有些心口疼痛起来。又加贾琏打听明白了,来说道:“舅太
爷是赶路劳乏,偶然感冒风寒,到了十里屯地方,延医调治,无奈这个地方没有名
医,误用了药,一剂就死了。但不知家眷可到了那里没有。”王夫人听了,一阵心
酸,便心口疼得坐不住,叫彩云等扶了上炕,还扎挣着叫贾琏去回了贾政:“即速
收拾行装,迎到那里,帮着料理完毕,即刻回来告诉我们,好叫你媳妇儿放心。”
贾琏不敢违拗,只得辞了贾政起身。

贾政早已知道,心里很不受用,又知宝玉失玉以后,神志昏愦,医药无效,又
值王夫人心疼。那年正值京察,工部将贾政保列一等,二月,吏部带领引见。皇上
念贾政勤俭谨慎,即放了江西粮道。即日谢恩,已奏明起程日期。虽有众亲朋贺喜,
贾政也无心应酬。只念家中人口不宁,又不敢耽延在家。正在无计可施,只听见贾
母那边叫:“请老爷。”贾政即忙进去。看见王夫人带着病也在那里,便向贾母请
了安。贾母叫他坐下,便说:“你不日就要赴任,我有多少话与你说,不知你听不
听?”说着掉下泪来。贾政忙站起来,说道:“老太太有话,只管吩咐,儿子怎敢
不遵命呢?”贾母哽咽着说道:“我今年八十一岁的人了,你又要做外任去。偏有
你大哥在家,你又不能告亲老。你这一去了,我所疼的只有宝玉,偏偏的又病得糊
涂,还不知道怎么样呢!我昨日叫赖升媳妇出去叫人给宝玉算算命,这先生算得好
灵,说:‘要娶了金命的人帮扶他,必要冲冲喜才好,不然只怕保不住。’我知道
你不信那些话,所以教你来商量。你的媳妇也在这里,你们两个也商量商量:还是
要宝玉好呢?还是随他去呢?”贾政陪笑说道:“老太太当初疼儿子这么疼的,难
道做儿子的就不疼自己的儿子不成么?只为宝玉不上进,所以时常恨他,也不过是
‘恨铁不成钢’的意思。老太太既要给他成家,这也是该当的,岂有逆着老太太不
疼他的理?如今宝玉病着,儿子也是不放心。因老太太不叫他见我,所以儿子也不
敢言语。我到底瞧瞧宝玉是个什么病?”

王夫人见贾政说着也有些眼圈儿红,知道心里是疼的,便叫袭人扶了宝玉来。
宝玉见了他父亲,袭人叫他请安,他便请了个安。贾政见他脸面很瘦,目光无神,
大有疯傻之状,便叫人扶了进去,便想到:“自己也是望六的人了,如今又放外任,
不知道几年回来。倘或这孩子果然不好,一则年老无嗣,虽说有孙子,到底隔了一
层;二则老太太最疼的是宝玉,若有差错,可不是我的罪名更重了?”瞧瞧王夫人
一包眼泪,又想到他身上,复站起来说:“老太太这么大年纪,想法儿疼孙子,做
儿子的还敢违拗?老太太主意该怎么便怎么就是了。但只姨太太那边不知说明白了
没有。”王夫人便道:“姨太太是早应了的,只为蟠儿的事没有结案,所以这些时
总没提起。”贾政又道:“这就是第一层的难处。他哥哥在监里,妹子怎么出嫁?
况且贵妃的事虽不禁婚嫁,宝玉应照已出嫁的姐姐,有九个月的功服,此时也难娶
亲。再者,我的起身日期已经奏明,不敢耽搁,这几天怎么办呢?”贾母想了一想:
“说的果然不错。若是等这几件事过去,他父亲又走了,倘或这病一天重似一天,
怎么好?只可越些礼办了才好。”想定主意,便说道:“你若给他办呢,我自然有
个道理,包管都碍不着:姨太太那边,我和你媳妇亲自过去求他。蟠儿那里,我央
蝌儿去告诉他,说是要救宝玉的命,诸事将就,自然应的。若说服里娶亲,当真使
不得;况且宝玉病着,也不可叫他成亲:不过是冲冲喜。我们两家愿意,孩子们又
有‘金玉’的道理,婚是不用合的了,即挑了好日子,按着咱们家分儿过了礼。趁
着挑个娶亲日子,一概鼓乐不用,倒按宫里的样子,用十二对提灯,一乘八人轿子
抬了来,照南边规矩拜了堂,一样坐床撒帐,可不是算娶了亲了么?宝丫头心地明
白,是不用虑的。内中又有袭人,也还是个妥妥当当的孩子,再有个明白人常劝他,
更好。他又和宝丫头合的来。再者,姨太太曾说:‘宝丫头的金锁也有个和尚说过,
只等有玉的便是婚姻。’焉知宝丫头过来,不因金锁倒招出他那块玉来,也定不得。
从此一天好似一天,岂不是大家的造化?这会子只要立刻收拾屋子,铺排起来,这
屋子是要你派的。一概亲友不请,也不排筵席。待宝玉好了,过了功服,然后再摆
席请人。这么着,都赶的上,你也看见了他们小两口儿的事,也好放心着去。”

贾政听了,原不愿意,只是贾母做主,不敢违命,勉强陪笑说道:“老太太想
得极是,也很妥当。只是要吩咐家下众人,不许吵嚷得里外皆知,这要耽不是的。
姨太太那边只怕不肯,若是果真应了,也只好按着老太太的主意办去。”贾母道:
“姨太太那里有我呢,你去罢。”贾政答应出来,心中好不自在。因赴任事多,部
里领凭,亲友们荐人,种种应酬不绝,竟把宝玉的事听凭贾母交与王夫人凤姐儿了。
惟将荣禧堂后身王夫人内屋旁边一大跨所二十馀间房屋指与宝玉,馀者一概不管。
贾母定了主意,叫人告诉他去,贾政只说“很好”。——此是后话。

且说宝玉见过贾政,袭人扶回里间炕上。因贾政在外,无人敢与宝玉说话,宝
玉便昏昏沉沉的睡去,贾母与贾政所说的话,宝玉一句也没有听见。袭人等却静静
儿的听得明白。头里虽也听得些风声,到底影响,只不见宝钗过了,却也有些信真。
今日听了这些话,心里方才水落归漕,倒也喜欢。心里想道:“果然上头的眼力不
错,这才配的是,我也造化!若他来了,我可以卸了好些担子。但是这一位的心里
只有一个林姑娘,幸亏他没有听见,若知道了,又不知要闹到什么分儿了。”袭人
想到这里,转喜为悲,心想:“这件事怎么好?老太太、太太那里知道他们心里的
事?一时高兴,说给他知道,原想要他病好。若是他还像头里的心,初见林姑娘,
便要摔玉砸玉;况且那年夏天在园里,把我当作林姑娘,说了好些私心话;后来因
为紫鹃说了句玩话儿,便哭得死去活来。若是如今和他说要娶宝姑娘,竟把林姑娘
撂开,除非是他人事不知还可,倘或明白些,只怕不但不能冲喜,竟是催命了。我
再不把话说明,那不是一害三个人了么?”袭人想定主意,待等贾政出去,叫秋纹
照看着宝玉,便从里间出来,走到王夫人身旁,悄悄的请了王夫人到贾母后身屋里
去说话。贾母只道是宝玉有话,也不理会,还在那里打算怎么过礼,怎么娶亲。

那袭人同了王夫人到了后间,便跪下哭了。王夫人不知何意,把手拉着他说:
“好端端的,这是怎么说?有什么委屈,起来说。”袭人道:“这话奴才是不该说
的,这会子因为没有法儿了!”王夫人道:“你慢慢的说。”袭人道:“宝玉的亲
事,老太太、太太已定了宝姑娘了,自然是极好的一件事。只是奴才想着,太太看
去,宝玉和宝姑娘好,还是和林姑娘好呢?”王夫人道:“他两个因从小儿在一处,
所以宝玉和林姑娘又好些。”袭人道:“不是‘好些’。”便将宝玉素与黛玉这些
光景一一的说了,还说:“这些事都是太太亲眼见的,独是夏天的话,我从没敢和
别人说。”王夫人拉着袭人道:“我看外面儿已瞧出几分来了,你今儿一说,更加
是了。但是刚才老爷说的话,想必都听见了,你看他的神情儿怎么样?”袭人道:
“如今宝玉若有人和他说话他就笑,没人和他说话他就睡,所以头里的话却倒都没
听见。”王夫人道:“倒是这件事叫人怎么样呢?”袭人道:“奴才说是说了,还
得太太告诉老太太,想个万全的主意才好。”王夫人便道:“既这么着,你去干你
的。这时候满屋子的人,暂且不用提起。等我瞅空儿回明老太太再作道理。”

说着,仍到贾母跟前。贾母正在那里和凤姐儿商议,见王夫人进来,便问道:
“袭人丫头说什么,这么鬼鬼祟祟的?”王夫人趁问,便将宝玉的心事细细回明贾
母。贾母听了,半日没言语。王夫人和凤姐也都不再说了。只见贾母叹道:“别的
事都好说。林丫头倒没有什么。若宝玉真是这样,这可叫人作了难了。”只见凤姐
想了一想,因说道:“难倒不难。只是我想了个主意,不知姑妈肯不肯。”王夫人
道:“你有主意,只管说给老太太听,大家娘儿们商量着办罢了。”凤姐道:“依
我想,这件事,只有一个‘掉包儿’的法子。”贾母道:“怎么‘掉包儿’?”凤
姐道:“如今不管宝兄弟明白不明白,大家吵嚷起来,说是老爷做主,将林姑娘配
了他了,瞧他的神情儿怎么样。要是他全不管,这个包儿也就不用掉了。若是他有
些喜欢的意思,这事却要大费周折呢。”王夫人道:“就算他喜欢,你怎么样办法
呢?”凤姐走到王夫人耳边,如此这般的说了一遍。王夫人点了几点头儿,笑了一
笑,说道:“也罢了。”贾母便问道:“你们娘儿两个捣鬼,到底告诉我是怎么着
呀。”凤姐恐贾母不懂,露泄机关,便也向耳边轻轻告诉了一遍。贾母果真一时不
懂。凤姐笑着又说了几句。贾母笑道:“这么着也好,可就只忒苦了宝丫头了。倘
或吵嚷出来,林丫头又怎么样呢?”凤姐道:“这个话,原只说给宝玉听,外头一
概不许提起,有谁知道呢?”

正说间,丫头传进话来,说:“琏二爷回来了。”王夫人恐贾母问及,使个眼
色与凤姐。凤姐便出来迎着贾琏,了个嘴儿,同到王夫人屋里等着去了。一会儿,
王夫人进来,已见凤姐哭的两眼通红。贾琏请了安,将到十里屯料理王子腾的丧事
的话说了一遍,便说:“有恩旨赏了内阁的职衔,谥了文勤公,命本家扶柩回籍,
着沿途地方官员照料。昨日起身,连家眷回南去了。舅太太叫我回来请安问好,说:
‘如今想不到不能进京,有多少话不能说。听见我大舅子要进京,若是路上遇见了,
便叫他来到咱们这里细细的说。’”王夫人听毕,其悲痛自不必言。凤姐劝慰了一
番,“请太太略歇一歇,晚上来,再商量宝玉的事罢”。说毕,同了贾琏回到自己
房中,告诉了贾琏,叫他派人收拾新房不提。

一日,黛玉早饭后,带着紫鹃到贾母这边来,一则请安,二则也为自己散散闷。
出了潇湘馆,走了几步,忽然想起忘了手绢子来,因叫紫鹃回去取来,自己却慢慢
的走着等他。刚走到沁芳桥那边山石背后当日同宝玉葬花之处,忽听一个人呜呜咽
咽在那里哭。黛玉煞住脚听时,又听不出是谁的声音,也听不出哭的叨叨的是些什
么话。心里甚是疑惑,便慢慢的走去。及到了跟前,却见一个浓眉大眼的丫头在那
里哭呢。黛玉未见他时,还只疑府里这些大丫头有什么说不出的心事,所以来这里
发泄发泄;及至见了这个丫头,却又好笑,因想到:“这种蠢货,有什么情种。自
然是那屋里作粗活的丫头,受了大女孩子的气了。”细瞧了一瞧,却不认得。

那丫头见黛玉来了,便也不敢再哭,站起来拭眼泪。黛玉问道:“你好好的为
什么在这里伤心?”那丫头听了这话,又流泪道:“林姑娘,你评评这个理:他们
说话,我又不知道,我就说错了一句话,我姐姐也不犯就打我呀。”黛玉听了,不
懂他说的是什么,因笑问道:“你姐姐是那一个?”那丫头道:“就是珍珠姐姐。”
黛玉听了,才知他是贾母屋里的。因又问:“你叫什么?”那丫头道:“我叫傻大
姐儿。”黛玉笑了一笑,又问:“你姐姐为什么打你?你说错了什么话了?”那丫
头道:“为什么呢,就是为我们宝二爷娶宝姑娘的事情。”黛玉听了这句话,如同
一个疾雷,心头乱跳,略定了定神,便叫这丫头:“你跟了我这里来。”那丫头跟
着黛玉到那畸角儿上葬桃花的去处,那里背静,黛玉因问道:“宝二爷娶宝姑娘,
他为什么打你呢?”傻大姐道:“我们老太太和太太、二奶奶商量了,因为我们老
爷要起身,说:就赶着往姨太太商量,把宝姑娘娶过来罢。头一宗,给宝二爷冲什
么喜;第二宗——”这到这里,又瞅着黛玉笑了一笑,才说道:“赶着办了,还要
给林姑娘说婆婆家呢。”

黛玉已经听呆了。这丫头只管说道:“我又不知道他们怎么商量的,不叫人吵
嚷,怕宝姑娘听见害臊。我白和宝二爷屋里的袭人姐姐说了一句:‘咱们明儿更热
闹了,又是宝姑娘,又是宝二奶奶,这可怎么叫呢?’林姑娘,你说我这话害着珍
珠姐姐什么了吗?他走过来就打了我一个嘴巴,说我混说,不遵上头的话,要撵出
我去。——我知道上头为什么不叫言语呢?你们又没告诉我,就打我。”说着,又
哭起来。

那黛玉此时心里,竟是油儿、酱儿、糖儿、醋儿倒在一处的一般,甜、苦、酸、
咸,竟说不上什么味儿来了。停了一会儿,颤巍巍的说道:“你别混说了。你再混
说,叫人听见,又要打你了。你去罢。”说着,自己转身要回潇湘馆去。那身子竟
有千百斤重的,两只脚却像踩着棉花一般,早已软了。只得一步一步慢慢的走将来。
走了半天,还没到沁芳桥畔。原来脚下软了,走的慢,且又迷迷痴痴,信着脚儿从
那边绕过来,更添了两箭地的路。这时刚到沁芳桥畔,却又不知不觉的顺着堤往回
里走起来。紫鹃取了绢子来,不见黛玉。正在那里看时,只见黛玉颜色雪白,身子
恍恍荡荡的,眼睛也直直的,在那里东转西转。又见一个丫头往前头走了,离的远
也看不出是那一个来,心中惊疑不定,只得赶过来,轻轻的问道:“姑娘,怎么又
回去?是要往那里去?”黛玉也只模糊听见,随口应道:“我问问宝玉去。”紫鹃
听了,摸不着头脑,只得搀着他到贾母这边来。

黛玉走到贾母门口,心里似觉明晰,回头看见紫鹃搀着自己,便站住了,问道:
“你作什么来的?”紫鹃陪笑道:“我找了绢子来了。头里见姑娘在桥那边呢,我
赶着过去问姑娘,姑娘没理会。”黛玉笑道:“我打量你来瞧宝二爷来了呢,不然,
怎么往这里走呢?”紫鹃见他心里迷惑,便知黛玉必是听见那丫头什么话来,惟有
点头微笑而已。只是心里怕他见了宝玉,那一个已经是疯疯傻傻,这一个又这样恍
恍惚惚,一时说出些不大体统的话来,那时如何是好?心里虽如此想,却也不敢违
拗,只得搀他进去。

那黛玉却又奇怪,这时不是先前那样软了,也不用紫鹃打帘子,自己掀起帘子
进来。却是寂然无声,因贾母在屋里歇中觉,丫头们也有脱滑儿玩去的,也有打盹
的,也有在那里伺候老太太的。倒是袭人听见帘子响,从屋里出来一看,见是黛玉,
便让道:“姑娘,屋里坐罢。”黛玉笑着道:“宝二爷在家么?”袭人不知底里,
刚要答言,只见紫鹃在黛玉身后和他嘴儿,指着黛玉,又摇摇手儿。袭人不解何
意,也不敢言语。黛玉却也不理会,自己走进房来。看见宝玉在那里坐着,也不起
来让坐,只瞅着嘻嘻的傻笑。黛玉自己坐下,却也瞅着宝玉笑。两个人也不问好,
也不说话,也无推让,只管对着脸傻笑起来。袭人看见这番光景,心里大不得主意,
只是没法儿。忽然听着黛玉说道:“宝玉,你为什么病了?”宝玉笑道:“我为林
姑娘病了。”袭人紫鹃两个吓得面目改色,连忙用言语来岔。两个却又不答言,仍
旧傻笑起来。袭人见了这样,知道黛玉此时心中迷惑,和宝玉一样,因悄和紫鹃说
道:“姑娘才好了,我叫秋纹妹妹同着你搀回姑娘,歇歇去罢。”因回头向秋纹道:
“你和紫鹃姐姐送林姑娘去罢。你可别混说话。”秋纹笑着也不言语,便来同着紫
鹃搀起黛玉。那黛玉也就站起来,瞅着宝玉只管笑,只管点头儿。紫鹃又催道:“姑
娘,回家去歇歇罢。”黛玉道:“可不是,我这就是回去的时候儿了。”说着,便
回身笑着出来了,仍旧不用丫头们搀扶,自己却走得比往常飞快。紫鹃秋纹后面赶
忙跟着走。

黛玉出了贾母院门,只管一直走去,紫鹃连忙搀住,叫道:“姑娘,往这么来。”
黛玉仍是笑着,随了往潇湘馆来。离门口不远,紫鹃道:“阿弥陀佛,可到了家了。”
只这一句话没说完,只见黛玉身子往前一栽,“哇”的一声,一口血直吐出来。

未知性命如何,且听下回分解。