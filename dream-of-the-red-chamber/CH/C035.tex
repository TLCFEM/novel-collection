\chapter{白玉钏亲尝莲叶羹~黄金莺巧结梅花络}

话说宝钗分明听见黛玉克薄他,因惦记着母亲哥哥,并不回头,一径去了。这
里黛玉仍旧立于花阴之下,远远的却向怡红院内望着。只见李纨、迎春、探春、惜
春并丫鬟人等,都向怡红院内去过之后,一起一起的散尽了;只不见凤姐儿来。心
里自己盘算说道:“他怎么不来瞧瞧宝玉呢?便是有事缠住了,他必定也是要来打
个花胡哨,讨老太太、太太的好儿才是呢。今儿这早晚不来,必有原故。”一面猜
疑,一面抬头再看时,只见花花簇簇一群人,又向怡红院内来了。定睛看时,却是
贾母搭着凤姐的手,后头邢夫人、王夫人,跟着周姨娘并丫头媳妇等人,都进院去
了。黛玉看了,不觉点头,想起有父母的好处来,早又泪珠满面。少顷,只见薛姨
妈、宝钗等也进去了。

忽见紫鹃从背后走来,说道:“姑娘吃药去罢,开水又冷了。”黛玉道:“你
到底要怎么样?只是催。我吃不吃,与你什么相干?”紫鹃笑道:“咳嗽的才好了
些,又不吃药了?如今虽是五月里,天气热,到底也还该小心些。大清早起,在这
个潮地上站了半日,也该回去歇歇了。”一句话提醒了黛玉,方觉得有点儿腿酸,
呆了半日,方慢慢的扶着紫鹃,回到潇湘馆来。一进院门,只见满地下竹影参差,
苔痕浓淡,不觉又想起《西厢记》中所云“幽僻处可有人行?点苍苔白露泠泠”二
句来,因暗暗的叹道:“双文虽然命薄,尚有孀母弱弟;今日我黛玉之薄命,一并
连孀母弱弟俱无。”想到这里,又欲滴下泪来。不防廊下的鹦哥见黛玉来了,“嘎”
的一声扑了下来,倒吓了一跳。因说道:“你作死呢,又了我一头灰。”那鹦哥
又飞上架去,便叫:“雪雁,快掀帘子,姑娘来了!”黛玉便止住步,以手扣架,
道:“添了食水不曾?”那鹦哥便长叹一声,竟大似黛玉素日吁嗟音韵,接着念道:
“侬今葬花人笑痴,他年葬侬知是谁!”黛玉紫鹃听了,都笑起来。紫鹃笑道:“这
都是素日姑娘念的,难为他怎么记了。”黛玉便命将架摘下来另挂在月洞窗外的钩
上。于是进了屋子,在月洞窗内坐了,吃毕药。只见窗外竹影映入纱窗,满屋内阴
阴翠润,几簟生凉。黛玉无可释闷,便隔着纱窗,调逗鹦哥做戏,又将素日所喜的
诗词也教与他念。这且不在话下。

且说宝钗来至家中,只见母亲正梳头呢,看见他进来,便笑着说道:“你这么
早就梳上头了。”宝钗道:“我瞧瞧妈妈身上好不好。昨儿我去了,不知他可又过
来闹了没有?”一面说,一面在他母亲身旁坐下,由不得哭将起来。薛姨妈见他一
哭,自己掌不住也就哭了一场,一面又劝他:“我的儿,你别委屈了。你等我处分
那孽障。你要有个好歹,叫我指望那一个呢?”薛蟠在外听见,连忙的跑过来,对
着宝钗左一个揖右一个揖,只说:“好妹妹恕我这次罢!原是我昨儿吃了酒,回来
的晚了,路上撞客着了,来家没醒,不知胡说了些什么,连自己也不知道,怨不得
你生气。”宝钗原是掩面而哭,听如此说由不得也笑了,遂抬头向地下啐了一口,
说道:“你不用做这些像生儿了。我知道你的心里多嫌我们娘儿们,你是变着法儿
叫我们离了你就心净了。”

薛蟠听说,连忙笑道:“妹妹这从那里说起?妹妹从来不是这么多心说歪话的
人哪。”薛姨妈忙又接着道:“你只会听你妹妹的‘歪话’,难道昨儿晚上你说的
那些话,就使得吗?当真是你发昏了?”薛蟠道:“妈妈也不必生气,妹妹也不用
烦恼,从今以后,我再不和他们一块儿喝酒了。好不好?”宝钗笑道:“这才明白
过来了。”薛姨妈道:“你要有个横劲,那龙也下蛋了。”薛蟠道:“我要再和他
们一处喝,妹妹听见了,只管啐我,再叫我畜生、不是人如何?何苦来为我一个人,
娘儿两个天天儿操心。妈妈为我生气还犹可,要只管叫妹妹为我操心,我更不是人
了。如今父亲没了,我不能多孝顺妈妈,多疼妹妹,反叫娘母子生气、妹妹烦恼,
连个畜生不如了!”口里说着,眼睛里掌不住掉下泪来。薛姨妈本不哭了,听他一
说又伤起心来。宝钗勉强笑道:“你闹够了,这会子又来招着妈妈哭了。”薛蟠听
说,忙收泪笑道:“我何曾招妈妈哭来着?罢罢罢,扔下这个别提了,叫香菱来倒
茶妹妹喝。”宝钗道:“我也不喝茶,等妈妈洗了手,我们就进去了。”薛蟠道:
“妹妹的项圈我瞧瞧,只怕该炸一炸去了。”宝钗道:“黄澄澄的,又炸他做什么?”
薛蟠又道:“妹妹如今也该添补些衣裳了,要什么颜色花样,告诉我。”宝钗道:
“连那些衣裳我还没穿遍了,又做什么?”一时薛姨妈换了衣裳,拉着宝钗进去,
薛蟠方出去了。

这里薛姨妈和宝钗进园来看宝玉。到了怡红院中,只见抱厦里外回廊上许多丫
头老婆站着,便知贾母等都在这里。母女两个进来,大家见过了。只见宝玉躺在榻
上,薛姨妈问他:“可好些?”宝玉忙欲欠身,口里答应着:“好些。”又说:“只
管惊动姨娘姐姐,我当不起。”薛姨妈忙扶他睡下,又问他:“想什么,只管告诉
我。”宝玉笑道:“我想起来,自然和姨娘要去。”王夫人又问:“你想什么吃?
回来好给你送来。”宝玉笑道:“也倒不想什么吃。倒是那一回做的那小荷叶儿小
莲蓬儿的汤还好些。”凤姐一旁笑道:“都听听!口味倒不算高贵,只是太磨牙了。
巴巴儿的想这个吃!”贾母便一叠连声的叫做去。凤姐笑道:“老祖宗别急,我想
想这模子是谁收着呢?”因回头吩咐个老婆问管厨房的去要。那老婆去了半天,来
回话:“管厨房的说:‘四副汤模子都缴上来了。’”凤姐听说,又想了一想道:
“我也记得交上来了,就只不记得交给谁了。多半是在茶房里。”又遣人去问管茶
房的,也不曾收。次后还是管金银器的送了来了。

薛姨妈先接过来瞧时,原来是个小匣子,里面装着四副银模子,都有一尺多长,
一寸见方。上面凿着豆子大小,也有菊花的,也有梅花的,也有莲蓬的,也有菱角
的:共有三四十样,打的十分精巧。因笑向贾母王夫人道:“你们府上也都想绝了,
吃碗汤还有这些样子。要不说出来,我见了这个,也不认得是做什么用的。”凤姐
儿也不等人说话,便笑道:“姑妈不知道:这是旧年备膳的时候儿,他们想的法儿。
不知弄什么面印出来,借点新荷叶的清香,全仗着好汤,我吃着究竟也没什么意思。
谁家长吃他?那一回呈样做了一回,他今儿怎么想起来了!”说着,接过来递与个
妇人,吩咐厨房里立刻拿几只鸡,另外添了东西,做十碗汤来。王夫人道:“要这
些做什么?”凤姐笑道:“有个原故:这一宗东西家常不大做,今儿宝兄弟提起来
了,单做给他吃,老太太、姑妈、太太都不吃,似乎不大好。不如就势儿弄些大家
吃吃,托赖着连我也尝个新儿。”贾母听了,笑道:“猴儿,把你乖的!拿着官中
的钱做人情。”说的大家笑了。凤姐忙笑道:“这不相干。这个小东道儿我还孝敬
的起。”便回头吩咐妇人:“说给厨房里,只管好生添补着做了,在我帐上领银子。”
婆子答应着去了。

宝钗一旁笑道:“我来了这么几年,留神看起来,二嫂子凭他怎么巧,再巧不
过老太太。”贾母听说,便答道:“我的儿!我如今老了,那里还巧什么?当日我像
凤丫头这么大年纪,比他还来得呢。他如今虽说不如我,也就算好了,比你姨娘强
远了!你姨娘可怜见的,不大说话,和木头似的,公婆跟前就不献好儿。凤儿嘴乖,
怎么怨得人疼他。”宝玉笑道:“要这么说,不大说话的就不疼了?”贾母道:“不
大说话的,又有不大说话的可疼之处。嘴乖的也有一宗可嫌的,倒不如不说的好。”
宝玉笑道:“这就是了。我说大嫂子倒不大说话呢,老太太也是和凤姐姐一样的疼。
要说单是会说话的可疼,这些姐妹里头也只凤姐姐和林妹妹可疼了。”贾母道:“提
起姐妹,不是我当着姨太太的面奉承:千真万真,从我们家里四个女孩儿算起,都
不如宝丫头。”薛姨妈听了,忙笑道:“这话是老太太说偏了。”王夫人忙又笑道:
“老太太时常背地里和我说宝丫头好,这倒不是假说。”宝玉勾着贾母,原为要赞
黛玉,不想反赞起宝钗来,倒也意出望外,便看着宝钗一笑。宝钗早扭过头去和袭
人说话去了。

忽有人来请吃饭,贾母方立起身来,命宝玉:“好生养着罢。”把丫头们又嘱
咐了一回,方扶着凤姐儿,让着薛姨妈,大家出房去了。犹问:“汤好了不曾?”
又问薛姨妈等:“想什么吃,只管告诉我,我有本事叫凤丫头弄了来咱们吃。”薛
姨妈笑道:“老太太也会怄他,时常他弄了东西来孝敬,究竟又吃不多儿。”凤姐
儿笑道:“姑妈倒别这么说。我们老祖宗只是嫌人肉酸,要不嫌人肉酸,早已把我
还吃了呢!”一句话没说了,引的贾母众人都哈哈的大笑起来。宝玉在屋里也掌不
住笑了。袭人笑道:“真真的二奶奶的嘴,怕死人。”

宝玉伸手拉着袭人笑道:“你站了这半日,可乏了。”一面说,一面拉他身旁
坐下。袭人笑道:“可是又忘了:趁宝姑娘在院子里,你和他说,烦他们莺儿来打
上几根绦子。”宝玉笑道:“亏了你提起来。”说着,便仰头向窗外道:“宝姐姐,
吃过饭叫莺儿来,烦他打几根绦子,可得闲儿?”宝钗听见,回头道:“是了,一
会儿就叫他来。”贾母等尚未听真,都止步问宝钗何事。宝钗说明了,贾母便说道:
“好孩子,你叫他来替你兄弟打几根罢。你要人使,我那里闲的丫头多着的呢。你
喜欢谁,只管叫来使唤。”薛姨妈宝钗等都笑道:“只管叫他来做就是了。有什么
使唤的去处!他天天也是闲着淘气。”大家说着,往前正走,忽见湘云、平儿、香
菱等在山石边掐凤仙花呢,见了他们走来,都迎上来了。

少顷出至园外,王夫人恐贾母乏了,便欲让至上房内坐,贾母也觉脚酸,便点
头依允。王夫人便命丫头忙先去铺设坐位。那时赵姨娘推病,只有周姨娘与那老婆
丫头们忙着打帘子,立靠背,铺褥子。贾母扶着凤姐儿进来,与薛姨妈分宾主坐了,
宝钗湘云坐在下面。王夫人亲自捧了茶来,奉与贾母;李宫裁捧与薛姨妈。贾母向
王夫人道:“让他们小妯娌们伏侍罢,你在那里坐下,好说话儿。”王夫人方向一
张小杌子上坐下,便吩咐凤姐儿道:“老太太的饭放在这里,添了东西来。”凤姐
儿答应出去,便命人去贾母那边告诉。那边的老婆们忙往外传了,丫头们忙都赶过
来。王夫人便命:“请姑娘们去。”请了半天,只有探春惜春两个来了;迎春身上
不耐烦,不吃饭;那黛玉是不消说,十顿饭只好吃五顿,众人也不着意了。

少顷饭至,众人调放了桌子。凤姐儿用手巾裹了一把牙箸,站在地下,笑道:
“老祖宗和姨妈不用让,还听我说就是了。”贾母笑向薛姨妈道:“我们就是这样。”
薛姨妈笑着应了。于是凤姐放下四双箸:上面两双是贾母薛姨妈,两边是宝钗湘云
的。王夫人李宫裁等都站在地下,看着放菜。凤姐先忙着要干净家伙来,替宝玉拣
菜。少顷,莲叶汤来了,贾母看过了,王夫人回头见玉钏儿在那里,便命玉钏儿与
宝玉送去。凤姐道:“他一个人难拿。”可巧莺儿和同喜都来了,宝钗知道他们已
吃了饭,便向莺儿道:“宝二爷正叫你去打绦子,你们两个同去罢。”莺儿答应着,
和玉钏儿出来。莺儿道:“这么远,怪热的,那可怎么端呢?”玉钏儿笑道:“你
放心,我自有道理。”说着,便命一个婆子来,将汤饭等类放在一个捧盒里,命他
端了跟着,他两个却空着手走。一直到了怡红院门口,玉钏儿方接过来了,同着莺
儿进入房中。

袭人、麝月、秋纹三个人正和宝玉玩笑呢,见他两个来了,都忙起来笑道:“你
们两个来的?怎么碰巧一齐来了。”一面说,一面接过来。玉钏儿便向一张杌子上
坐下;莺儿不敢坐,袭人便忙端了个脚踏来,莺儿还不敢坐。宝玉见莺儿来了,却
倒十分欢喜;见了玉钏儿,便想起他姐姐金钏儿来了,又是伤心,又是惭愧,便把
莺儿丢下,且和玉钏儿说话。袭人见把莺儿不理,恐莺儿没好意思的,又见莺儿不
肯坐,便拉了莺儿出来,到那边屋里去吃茶说话儿去了。

这里麝月等预备了碗箸来伺候吃饭。宝玉只是不吃,问玉钏儿道:“你母亲身
上好?”玉钏儿满脸娇嗔,正眼也不看宝玉,半日方说了一个“好”字。宝玉便觉
没趣,半日,只得又陪笑问道:“谁叫你替我送来的?”玉钏儿道:“不过是奶奶
太太们!”宝玉见他还是哭丧着脸,便知他是为金钏儿的原故。待要虚心下气哄他,
又见人多,不好下气的,因而便寻方法将人都支出去,然后又陪笑问长问短。那玉
钏儿先虽不欲理他,只管见宝玉一些性气也没有,凭他怎么丧谤,还是温存和气,
自己倒不好意思的了,脸上方有三分喜色。宝玉便笑央道:“好姐姐,你把那汤端
了来,我尝尝。”玉钏儿道:“我从不会喂人东西,等他们来了再喝。”宝玉笑道:
“我不是要你喂我,我因为走不动,你递给我喝了,你好赶早回去交代了,好吃饭
去。我只管耽误了时候,岂不饿坏了你。你要懒怠动,我少不得忍着疼下去取去。”
说着,便要下床,扎挣起来,禁不住“嗳哟”之声。玉钏儿见他这般,也忍不过,
起身说道:“躺下去罢!那世里造的孽,这会子现世现报,叫我那一个眼睛瞧的上!”
一面说,一面哧的一声又笑了,端过汤来。宝玉笑道:“好姐姐你要生气,只管在
这里生罢,见了老太太、太太,可和气着些。若还这样,你就要挨骂了。”玉钏儿
道:“吃罢,吃罢!你不用和我甜嘴蜜舌的了,我都知道啊!”说着,催宝玉喝了
两口汤。宝玉故意说不好吃。玉钏儿撇嘴道:“阿弥陀佛!这个还不好吃,也不知
什么好吃呢!”宝玉道:“一点味儿也没有,你不信尝一尝,就知道了。”玉钏儿
果真赌气尝了一尝。宝玉笑道:“这可好吃了!”玉钏儿听说,方解过他的意思来,
原是宝玉哄他喝一口,便说道:“你既说不喝,这会子说好吃,也不给你喝了。”
宝玉只管陪笑央求要喝,玉钏儿又不给他,一面又叫人打发吃饭。

丫头方进来时,忽有人来回话,说:“傅二爷家的两个嬷嬷来请安,来见二爷。”
宝玉听说,便知是通判傅试家的嬷嬷来了。那傅试原是贾政的门生,原来都赖贾家
的名声得意,贾政也着实看待,与别的门生不同;他那里常遣人来走动。宝玉素昔
最厌勇男蠢妇的,今日却如何又命这两个婆子进来?其中原来有个原故。只因那宝
玉闻得傅试有个妹子,名唤傅秋芳,也是个琼闺秀玉,常听人说才貌俱全,虽自未
亲睹,然遐思遥爱之心十分诚敬。不命他们进来,恐薄了傅秋芳,因此连忙命让进
来。那傅试原是暴发的,因傅秋芳有几分姿色,聪明过人,那傅试安心仗着妹子,
要与豪门贵族结亲,不肯轻意许人,所以耽误到如今。目今傅秋芳已二十三岁,尚
未许人。怎奈那些豪门贵族又嫌他本是穷酸,根基浅薄,不肯求配。那傅试与贾家
亲密,也自有一段心事。

今日遣来的两个婆子,偏偏是极无知识的,闻得宝玉要见,进来只刚问了好,
说了没两句话。那玉钏儿见生人来,也不和宝玉厮闹了,手里端着汤,却只顾听。
宝玉又只顾和婆子说话,一面吃饭,伸手去要汤,两个人的眼睛都看着人,不想伸
猛了手,便将碗撞翻,将汤泼了宝玉手上。玉钏儿倒不曾烫着,吓了一跳,忙笑着:
“这是怎么了?”慌的丫头们忙上来接碗。宝玉自己烫了手,倒不觉的,只管问玉
钏儿:“烫了那里了?疼不疼?”玉钏儿和众人都笑了。玉钏儿道:“你自己烫了,
只管问我。”宝玉听了,方觉自己烫了。众人上来,连忙收拾。宝玉也不吃饭了,
洗手吃茶,又和那两个婆子说了两句话,然后两个婆子告辞出去。晴雯等送至桥边
方回。

那两个婆子见没人了,一行走一行谈论。这一个笑道:“怪道有人说他们家的
宝玉是相貌好里头糊涂,中看不中吃,果然竟有些呆气。他自己烫了手,倒问别人
疼不疼,这可不是呆了吗!”那个又笑道:“我前一回来,还听见他家里许多人说,
千真万真有些呆气:大雨淋的水鸡儿似的,他反告诉别人:‘下雨了,快避雨去罢。’
你说可笑不可笑?时常没人在跟前,就自哭自笑的,看见燕子就和燕子说话,河里
看见了鱼就和鱼儿说话,见了星星月亮,他不是长吁短叹的,就是咕咕哝哝的。且
一点刚性儿也没有,连那些毛丫头的气都受到了。爱惜起东西来,连个线头儿都是
好的;遭塌起来,那怕值千值万都不管了。”两个人一面说,一面走出园来回去,
不在话下。

且说袭人见人去了,便携了莺儿过来问宝玉:“打什么绦子?”宝玉笑向莺儿
道:“才只顾说话,就忘了你了。烦你来不为别的,替我打几根络子。”莺儿道:
“装什么的络子?”宝玉见问,便笑道:“不管装什么的,你都每样打几个罢。”
莺儿拍手笑道:“这还了得,要这样,十年也打不完了。”宝玉笑道:“好姑娘,
你闲着也没事,都替我打了罢。”袭人笑道:“那里一时都打的完?如今先拣要紧
的打几个罢。”莺儿道:“什么要紧,不过是扇子,香坠儿,汗巾子。”宝玉道:
“汗巾子就好。”莺儿道:“汗巾子是什么颜色?”宝玉道:“大红的。”莺儿道:
“大红的须是黑络子才好看,或是石青的,才压得住颜色。”宝玉道:“松花色配
什么?”莺儿道:“松花配桃红。”宝玉笑道:“这才娇艳。再要雅淡之中带些娇
艳。”莺儿道:“葱绿柳黄可倒还雅致。”宝玉道:“也罢了。也打一条桃红,再
打一条葱绿。”莺儿道:“什么花样呢?”宝玉道:“也有几样花样?”莺儿道:
“‘一炷香’,‘朝天凳’,‘象眼块’,‘方胜’,‘连环’,‘梅花’,‘柳
叶’。”宝玉道:“前儿你替三姑娘打的那花样是什么?”莺儿道:“是‘攒心梅
花’。”宝玉道:“就是那样好。”一面说,一面袭人刚拿了线来。窗外婆子说:
“姑娘们的饭都有了。”宝玉道:“你们吃饭去,快吃了来罢。”袭人笑道:“有
客在这里。我们怎么好意思去呢?”莺儿一面理线,一面笑道:“这打那里说起?
正经快吃去罢。”袭人等听说,方去了,只留下两个小丫头呼唤。

宝玉一面看莺儿打络子,一面说闲话。因问他:“十几岁了?”莺儿手里打着,
一面答话:“十五岁了。”宝玉道:“你本姓什么?”莺儿道:“姓黄。”宝玉笑
道:“这个姓名倒对了,果然是个‘黄莺儿’。”莺儿笑道:“我的名字本来是两
个字,叫做金莺,姑娘嫌拗口,只单叫莺儿,如今就叫开了。”宝玉道:“宝姐姐
也就算疼你了。明儿宝姐姐出嫁,少不得是你跟了去了。”莺儿抿嘴一笑。宝玉笑
道:“我常常和你花大姐姐说,明儿也不知那一个有造化的消受你们主儿两个呢。”
莺儿笑道:“你还不知我们姑娘,有几样世上的人没有的好处呢,模样儿还在其次。”
宝玉见莺儿娇腔婉转,语笑如痴,早不胜其情了,那堪更提起宝钗来?便问道:“什
么好处?你细细儿的告诉我听。”莺儿道:“我告诉你,你可不许告诉他。”宝玉
笑道:“这个自然。”

正说着,只听见外头说道:“怎么这么静悄悄的?”二人回头看时,不是别人,
正是宝钗来了。宝玉忙让坐。宝钗坐下,因问莺儿:“打什么呢?”一面问,一面
向他手里去瞧,才打了半截儿。宝钗笑道:“这有什么趣儿,倒不如打个络子把玉
络上呢。”一句话提醒了宝玉,便拍手笑道:“倒是姐姐说的是,我就忘了。只是
配个什么颜色才好?”宝钗道:“用鸦色断然使不得,大红又犯了色。黄的又不起
眼,黑的太暗。依我说,竟把你的金线拿来配着黑珠儿线,一根一根的拈上,打成
络子,那才好看。”宝玉听说,喜之不尽,一叠连声就叫袭人来取金线。

正值袭人端了两碗菜走进来,告诉宝玉道:“今儿奇怪,刚才太太打发人给我
送了两碗菜来。”宝玉笑道:“必定是今儿菜多,送给你们大家吃的。”袭人道:
“不是,说指名给我的,还不叫过去磕头,这可是奇了。”宝钗笑道:“给你的你
就吃去,这有什么猜疑的。”袭人道:“从来没有的事,倒叫我不好意思的。”宝
钗抿嘴一笑,说道:“这就不好意思了?明儿还有比这个更叫你不好意思的呢!”
袭人听了话内有因,素知宝钗不是轻嘴薄舌奚落人的,自己想起上日王夫人的意思
来,便不再提了。将菜给宝玉看了,说:“洗了手来拿线。”说毕,便一直出去了。
吃过饭洗了手进来,拿金线给莺儿打络子。此时宝钗早被薛蟠遣人来请出去了。

这里宝玉正看着打络子,忽见邢夫人那边遣了两个丫头送了两样果子来给他
吃,问他:“可走得了么?要走的动,叫哥儿明儿过去散散心,太太着实惦记着呢。”
宝玉忙道:“要走得了,必定过来请太太的安去。疼的比先好些,请太太放心罢。”
一面叫他两个坐下,一面又叫:“秋纹来,把才那果子拿一半送给林姑娘去。”秋
纹答应了,刚欲去时,只听黛玉在院内说话。宝玉忙叫快请。

要知端底,且看下回分解。