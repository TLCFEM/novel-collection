\chapter{纵淫心宝蟾工设计~布疑阵宝玉妄谈禅}

话说薛蝌正在狐疑,忽听窗外一笑,唬了一跳,心中想道:“不是宝蟾,定是
金桂。只不理他们,看他们有什么法儿。”听了半日,却又寂然无声。自己也不敢
吃那酒果,掩上房门。刚要脱衣时,只听见窗纸上微微一响。薛蝌此时被宝蟾鬼混
了一阵,心中七上八下,竟不知如何是好。听见窗纸微响,细看时又无动静,自己
反倒疑心起来,掩了怀坐在灯前呆呆的细想,又把那果子拿了一块,翻来覆去的细
看。猛回头,看见窗上的纸湿了一块。走过来觑着眼看时,冷不防外面往里一吹,
把薛蝌唬了一大跳,听得“吱吱”的笑声。薛蝌连忙把灯吹灭了,屏息而卧。只听
外面一个人说道:“二爷为什么不喝酒吃果子就睡了?”这句话仍是宝蟾的话音。
薛蝌只不作声装睡。又隔了两句话时,听得外面似有恨声道:“天下那里有这样没
造化的人!”薛蝌听了似是宝蟾,又似是金桂的语音,这才知道他们原来是这一番
意思。翻来覆去,直到五更后才睡着了。

刚到天明,早有人来扣门。薛蝌忙问:“是谁?”外面也不答应。薛蝌只得起
来,开了门看时,却是宝蟾,拢着头发,掩着怀,穿了件片金边琵琶襟小紧身,上
面系一条松花绿半新的汗巾,下面并无穿裙,正露着石榴红洒花夹裤,一双新绣红
鞋。原来宝蟾尚未梳洗,恐怕人见,赶早来取家伙。薛蝌见他这样打扮便走进来,
心中又是一动,只得陪笑问道:“怎么这么早就起来了?”宝蟾把脸红着,并不答
言,只管把果子折在一个碟子里,端着就走。薛蝌见他这般,知是昨晚的原故,心
里想道:“这也罢了。倒是他们恼了,索性死了心,也省了来缠。”于是把心放下,
叫人舀水洗脸。自己打算在家里静坐两天,一则养养神,二则出去怕人找他。

原来和薛蟠好的那些人,因见薛家无人,只有薛蝌办事,年纪又轻,便生出许
多觊觎之心。也有想插在里头做跑腿儿的;也有能做状子、认得一两个书办、要给
他上下打点的;甚至有叫他在内趁钱的;也有造作谣言恐吓的:种种不一。薛蝌见
了这些人,远远的躲避,又不敢面辞,恐怕激出意外之变,只好藏在家中听候转详
不提。

且说金桂昨夜打发宝蟾,送了些酒果去探探薛蝌的消息,宝蟾回来,将薛蝌的
光景一一的说了。金桂见事有些不大投机,便怕白闹一场,反被宝蟾瞧不起;要把
两三句话遮饰,改过口来,又撂不开这个人。心里倒没了主意,只是怔怔的坐着。
那知宝蟾也想薛蟠难以回家,正要寻个路头儿,因怕金桂拿他,所以不敢透漏。今
见金桂所为先已开了端了,他便乐得借风使船,先弄薛蝌到手,不怕金桂不依,所
以用言挑拨。见薛蝌似非无情,又不甚兜揽,一时也不敢造次。后来见薛蝌吹灯自
睡,大觉扫兴,回来告诉金桂,看金桂有甚方法儿,再作道理。及见金桂怔怔的,
似乎无技可施,他也只得陪金桂收拾睡了。夜里那里睡的着?翻来覆去,想出一个
法子来:不如明儿一早起来,先去取了家伙,却自己换上一两件颜色娇嫩的衣服,
也不梳洗,越显出一番慵妆媚态来,只看薛蝌的神情,自己反倒装出恼意,索性不
理他。那薛蝌若有悔心,自然移船就岸,不愁不先到手:是这个主意。及至见了薛
蝌,仍是昨晚光景,并无邪僻,自己只得以假为真,端了碟子回来,却故意留下酒
壶,以为再来搭转之地。

只见金桂问道:“你拿东西去,有人碰见么?”宝蟾道:“没有。”金桂道:“二
爷也没问你什么?”宝蟾道:“也没有。”金桂因一夜不曾睡,也想不出个法子来,
只得回思道:“若作此事,别人可瞒,宝蟾如何能瞒?不如分惠于他,他自然没的说
了。况我又不能自去,少不得要他作脚,索性和他商量个稳便主意。”因带笑说道:
“你看二爷到底是怎么样的个人?”宝蟾道:“倒像是个糊涂人。”金桂听了笑道:
“你怎么遭塌起爷们来了!”宝蟾也笑道:“他辜负奶奶的心,我就说得他。”金桂
道:“他怎么辜负我的心?你倒得说说。”宝蟾道:“奶奶给他好东西吃,他倒不吃,
这不是辜负奶奶的心么?”说着,把眼溜着金桂一笑。金桂道:“你别胡想。我给
他送东西,为大爷的事不辞劳苦,我所以敬他;又怕人说瞎话,所以问你。你这些
话和我说,我不懂是什么意思。”宝蟾笑道:“奶奶别多心。我是跟奶奶的,还有两
个心么?但是事情要密些,倘或声张起来,不是玩的。”金桂也觉得脸飞红了,因说
道:“你这个丫头,就不是个好货。想来你心里看上了,却拿我作筏子是不是呢?”
宝蟾道:“只是奶奶那么想罢咧,我倒是替奶奶难受。奶奶要真瞧二爷好,我倒有
个主意。奶奶想,‘那个耗子不偷油’呢?他也不过怕事情不密,大家闹出乱子来不
好看。依我想:奶奶且别性急,时常在他身上不周不备的去处张罗张罗。他是个小
叔子,又没娶媳妇儿,奶奶就多尽点心儿,和他贴个好儿,别人也说不出什么来。
过几天他感奶奶的情,他自然要谢候奶奶。那时奶奶再备点东西儿在咱们屋里,我
帮着奶奶灌醉了他,还怕他跑了吗?他要不应,咱们索性闹起来,就说他调戏奶奶。
他害怕,自然得顺着咱们的手儿。他再不应,他也不是人,咱们也不至白丢了脸:
奶奶想怎么样?”金桂听了这话,两颧早已红晕了,笑骂道:“小蹄子,你倒像偷
过多少汉子似的!怪不得大爷在家时离不开你。”宝蟾把嘴一撇,笑说道:“罢哟,
人家倒替奶奶拉纤,奶奶倒和我们说这个话咧。”从此,金桂一心笼络薛蝌,倒无
心混闹了,家中也少觉安静。

当日宝蟾自去取了酒壶,仍是稳稳重重,一脸的正气。薛蝌偷眼看了,反倒后
悔,疑心或者是自己错想了他们,也未可知:“果然如此,倒辜负了他这一番美意,
保不住日后倒要和自己也闹起来,岂非自惹的呢?”过了两天,甚觉安静。薛蝌遇
见宝蟾,宝蟾便低头走了,连眼皮儿也不抬;遇见金桂,金桂却一盆火儿的赶着。
薛蝌见这般光景,反倒过意不去。这且不表。

且说宝钗母女觉得金桂几天安静,待人忽然亲热起来,一家子都为罕事。薛姨
妈十分欢喜,想到:“必是薛蟠娶这媳妇时冲犯了什么,才败坏了这几年。目今闹
出这样事来,亏得家里有钱,贾府出力,方才有了指望。媳妇忽然安静起来,或者
是蟠儿转过运气来也未可知。”于是自己心里倒以为希有之奇。这日饭后,扶了同
贵过来,到金桂房里瞧瞧。走到院中,只听一个男人和金桂说话。同贵知机,便说
道:“大奶奶,老太太过来了。”说着,已到门口,只见一个人影儿在房门后一躲。
薛姨妈一吓,倒退了出来。金桂道:“太太请里头坐,没有外人。他就是我的过继
兄弟,本住在屯里,不惯见人。因没有见过太太,今儿才来,还没去请太太的安。”
薛姨妈道:“既是舅爷,不妨见见。”

金桂叫兄弟出来,见了薛姨妈,作了个揖,问了好。薛姨妈也问了好,坐下叙
起话来。薛姨妈道:“舅爷上京几时了?”那夏三道:“前月我妈没有人管家,把我
过继来的。前日才进京,今日来瞧姐姐。”薛姨妈看那人不尴尬,于是略坐坐儿,
便起身道:“舅爷坐着罢。”回头向金桂道:“舅爷头上末下的来,留在咱们这里吃
了饭再去罢。”金桂答应着,薛姨妈自去了。金桂见婆婆去了,便向夏三道:“你坐
着罢。今日可是过了明路的了,省了我们二爷查考。我今日还要叫你买些东西,只
别叫别人看见。”夏三道:“这个交给我就完了。你要什么,只要有钱,我就买的了
来。”金桂道:“且别说嘴。等你买上了当,我可不收。”说着,二人又嘲谑了一回,
然后金桂陪着夏三吃了晚饭,又告诉他买的东西,又嘱咐一回,夏三自去。从此夏
三往来不绝。虽有个年老的门上人,知是舅爷,也不常回。从此生出无限风波来,
这是后话,不表。

一日,薛蟠有信寄回,薛姨妈打开叫宝钗看时,上写:

男在县里也不受苦,母亲放心。但昨日县里书办说,府里已经准详,想是我们
的情到了。岂知府里详上去,道里反驳下来了。亏得县里主文相公好,即刻做了回
文顶上去了,那道里却把知县申饬。现在道里要亲提,若一上去,又要吃苦。必是
道里没有托到。母亲见字,快快托人求道爷去。还叫兄弟快来,不然就要解道。银
子短不得,火速,火速!
薛姨妈听了,又哭了一场。宝钗和薛蝌一面劝慰,一面说道:“事不宜迟。”薛姨妈
没法,只得叫薛蝌到那里去照料,命人即忙收拾行李,兑了银子,同着当铺中一个
伙计连夜起程。那时手忙脚乱,虽有下人办理,宝钗怕他们思想不到,亲来帮着收
拾,直闹至四更才歇。到底富家女子娇养惯了的,心上又急,又劳苦了一夜,到了
次日就发起烧来,汤水都吃不下去。莺儿忙回了薛姨妈。薛姨妈急来看时,只见宝
钗满面通红,身如燔灼,话都不说。薛姨妈慌了手脚,便哭得死去活来。宝琴扶着
劝解。秋菱见了,也泪如泉涌,只管在旁哭叫。宝钗不能说话,连手也不能摇动,
眼干鼻塞。叫人请医调治,渐渐苏醒回来,薛姨妈等大家略略放心。早惊动荣宁两
府的人,先是凤姐打发人送十香返魂丹来,随后王夫人又送至宝丹来。贾母邢王二
夫人以及尤氏等都打发丫头来问候,却都不叫宝玉知道。一连治了七八天,终不见
效。还是他自己想起“冷香丸”,吃了三丸,才得病好。后来宝玉也知道了,因病
好了,没有瞧去。

那时薛蝌又有信回来。薛姨妈看了,怕宝钗耽忧,也不叫他知道,自己来求王
夫人,并述了一会子宝钗的病。薛姨妈去后,王夫人又求贾政。贾政道:“此事上
头可托,底下难托,必须打点才好。”王夫人又提起宝钗的事来,因说道:“这孩子
也苦了。既是我家的人了,也该早些娶了过来才是,别叫他遭塌坏了身子。”贾政
道:“我也是这么想。但是他家忙乱,况且如今到了冬底,已经年近岁逼,无不各
自要料理些家务。今冬且放了定,明春再过礼。过了老太太的生日,就定日子娶。
你把这番话先告诉薛姨太太。”王夫人答应了。

到了次日,王夫人将贾政的话向薛姨妈说了,薛姨妈想着也是。到了饭后,王
夫人陪着来到贾母房中,大家让了坐。贾母道:“姨太太才过来?”薛姨妈道:“还
是昨儿过来的,因为晚了,没得过来给老太太请安。”王夫人便把贾政昨夜所说的
话向贾母述了一遍,贾母甚喜。说着,宝玉进来了,贾母便问道:“吃了饭了没有?”
宝玉道:“才打学房里回来,吃了,要往学房里去,先见见老太太。又听见说姨妈
来了,过来给姨妈请请安。”因问:“宝姐姐大好了?”薛姨妈笑道:“好了。”原来
方才大家正说着,见宝玉进来都掩住了。宝玉坐了坐,见薛姨妈神情不似从前亲热,
“虽是此刻没有心情,也不犯大家都不言语……”满腹猜疑,自往学中去了。

晚上回来,都见过了,便往潇湘馆来。掀帘进去,紫鹃接着。见里间屋内无人,
宝玉道:“姑娘那里去了?”紫鹃道:“上屋里去了。听见说姨太太过来,姑娘请安
去了。二爷没有到上屋里去么?”宝玉道:“我去了来的,没有见你们姑娘。”紫鹃
道:“没在那里吗?”宝玉道:“没有。到底那里去了?”紫鹃道:“这就不定了。”
宝玉刚要出来,只见黛玉带着雪雁,冉冉而来。宝玉道:“妹妹回来了。”缩身退步,
仍跟黛玉回来。黛玉进来,走入里间屋内,便请宝玉里头坐,——紫鹃拿了一件外
罩换上,——然后坐下,问道:“你上去,看见姨妈了没有?”宝玉道:“见过了。”
黛玉道:“姨妈说起我来没有?”宝玉道:“不但没说你,连见了我也不像先时亲热。
我问起宝姐姐的病来,他不过笑了一笑,并不答言。难道怪我这两天没去瞧他么?”
黛玉笑了一笑,道:“你去瞧过没有?”宝玉道:“头几天不知道;这两天知道了,
也没去。”黛玉道:“可不是呢。”宝玉道:“当真的,老太太不叫我去,太太也不叫
去,老爷又不叫去,我如何敢去?要像从前这小门儿通的时候儿,我一天瞧他十趟
也不难,如今把门堵了,要打前头过去,自然不便了。”黛玉道:“他那里知道这个
原故?”宝玉道:“宝姐姐为人是最体谅我的。”黛玉道:“你不要自己打错了主意。
若论宝姐姐,更不体谅,又不是姨妈病,是宝姐姐病:向来在园中做诗,赏花,饮
酒,何等热闹。如今隔开了,你看见他家里有事了,他病到那步田地,你像没事人
一般,他怎么不恼呢?”宝玉道:“这样,难道宝姐姐便不和我好了不成?”黛玉
道:“他和你好不好,我却不知,我也不过是照理而论。”

宝玉听了,瞪着眼呆了半晌。黛玉看见宝玉这样光景,也不睬他,只是自己叫
人添了香,又翻出书来,看了一会。只见宝玉把眉一皱,把脚一跺,道:“我想这
个人生他做什么!天地间没有了我,倒也干净。”黛玉道:“原是有了我便有了人,
有了人便有无数的烦恼生出来:恐怖,颠倒,梦想,更有许多缠碍。才刚我说的,
都是玩话。你不过是看见姨妈没精打彩,如何便疑到宝姐姐身上去?姨妈过来原为
他的官司事情,心绪不宁,那里还来应酬你?都是你自己心上胡思乱想,钻入魔道
里去了。”宝玉豁然开朗,笑道:“很是,很是。你的性灵,比我竟强远了。怨不得
前年我生气的时候,你和我说过几句禅话,我实在对不上来。我虽丈六金身,还借
你一茎所化。”

黛玉乘此机会,说道:“我便问你一句话,你如何回答?”宝玉盘着腿,合着
手,闭着眼,撅着嘴,道:“讲来。”黛玉道:“宝姐姐和你好,你怎么样?宝姐姐不
和你好,你怎么样?宝姐姐前儿和你好,如今不和你好,你怎么样?今儿和你好,后
来不和你好,你怎么样?你和他好,他偏不和你好,你怎么样?你不和他好,他偏要
和你好,你怎么样?”宝玉呆了半晌,忽然大笑道:“任凭弱水三千,我只取一瓢
饮。”黛玉道:“瓢之漂水,奈何?”宝玉道:“非瓢漂水:水自流,瓢自漂耳。”黛
玉道:“水止珠沉,奈何?”宝玉道:“禅心已作沾泥絮,莫向春风舞鹧鸪。”黛玉
道:“禅门第一戒是不打诳语的。”宝玉道:“有如三宝。”黛玉低头不语。只听见檐
外老鸦呱呱的叫了几声,便飞向东南上去。宝玉道:“不知主何吉凶?”黛玉道:“‘人
有吉凶事,不在鸟音中’。”

忽见秋纹走来说道:“请二爷回去。老爷叫人到园里来问过,说:二爷打学里
回来了没有?袭人姐姐只说‘已经回来了’。快去罢。”吓的宝玉站起身来往外忙走,
黛玉也不敢相留。

未知何事,下回分解。