\chapter{王熙凤历幻返金陵~甄应嘉蒙恩还玉阙}

却说宝玉宝钗听说凤姐病的危急,赶忙起来,丫头秉烛伺候。正要出院,只见
王夫人那边打发人来说:“琏二奶奶不好了,还没有咽气,二爷二奶奶且慢些过去
罢。琏二奶奶的病有些古怪,从三更天起,到四更时候,没有住嘴,说了好些胡话,
要船要轿,只说赶到金陵归入什么册子去。众人不懂,他只是哭哭喊喊。琏二爷没
有法儿,只得去糊船轿,还没拿来,琏二奶奶喘着气等着呢。太太叫我们过来说,
等琏二奶奶去了,再过去罢。”宝玉道:“这也奇,他到金陵做什么去?”袭人轻
轻的说道:“你不是那年做梦,我还记得说有多少册子?莫不琏二奶奶是到那里去
罢?”宝玉听了点头道:“是呀,可惜我都不记得那上头的话了。这么说起来,人
都有个定数的了。但不知林妹妹又到那里去了?我如今被你一说,我有些懂的了。
若再做这个梦时,我必细细的瞧一瞧,便有未卜先知的分儿了。”袭人道:“你这
样的人,可是不可合你说话,我偶然提了一句,你就认起真来了吗?就算你能先知
了,又有什么法儿?”宝玉道:“只怕不能先知;若是能了,我也犯不着为你们瞎
操心了。”两人正说着,宝钗走来,问道:“你们说什么?”宝玉恐他盘诘,只说:
“我们谈论凤姐姐。”宝钗道:“人要死了,你们还只管议论他。旧年你还说我咒
人,那个签不是应了么?”宝玉又想了一想,拍手道:“是的是的,这么说起来,
你倒能先知了。我索性问问你,你知道我将来怎么样?”宝钗笑道:“这是又胡闹
起来了。我是就他求的签上的话混解的,你就认了真了。你和我们二嫂子成了一样
的了。你失了玉,他去求妙玉扶乩,批出来众人不解。他背地里合我说,妙玉怎么
前知,怎么参禅悟道,如今他遭此大难,如何自己都不知道?这可是算得前知吗?就
是我偶然说着了二奶奶的事情,其实知道他是怎么样了?只怕我连我自己也不知道
呢。这些事情,原都是虚诞的,可是信得的么?”

宝玉道:“别提他了。你只说邢妹妹罢,自从我们这里连连的有事,把他这件
事竟忘记了。你们家这么一件大事,怎么就草草的完了?也没请亲唤友的。”宝钗
道:“你这话又是迂了。我们家的亲戚,只有咱们这里和王家最近。王家没了什么
正经人了,咱们家遭了老太太的大事,所以也没请,就是琏二哥张罗了张罗。别的
亲戚虽也有一两门子,你没过去,如何知道?算起来,我们这二嫂子的命和我差不
多。好好的许了我二哥哥,我妈妈原想要体体面面的给二哥哥娶这房亲事的。一则
为我哥哥在监里,二哥哥也不肯大办;二则为咱们家的事;三则为我二嫂子在大太
太那边忒苦,又加着抄了家,大太太是一味的苛刻,他也实在难受。所以我和妈妈
说了,便将将就就的娶了过去。我看二嫂子如今倒是安心乐意的孝敬我妈妈,比亲
媳妇还强十倍呢。待二哥哥也是极尽妇道的,和香菱又甚好。二哥哥不在家,他两
个和和气气的过日子,虽说是穷些,我妈妈近来倒安逸好些。就是想起我哥哥来,
不免伤心。况且常打发人家里来要使用,多亏二哥哥在外头账头儿上讨来应付他。
我听见说:城里的几处房子已经也典了,还剩了一所,如今打算着搬了去住。”宝
玉道:“为什么要搬?住在这里,你来去也便宜些;若搬远了,你去就要一天了。”
宝钗道:“虽说是亲戚,到底各自的稳便些。那里有个一辈子住在亲戚家的呢?”

宝玉还要讲出不搬去的理,王夫人打发人来说:“琏二奶奶咽了气了!所有的
人都过去了,请二爷二奶奶就过去。”宝玉听了,也掌不住跺脚要哭。宝钗虽也悲
戚,恐宝玉伤心,便说:“有在这里哭的,不如到那边哭去。”于是两人一直到凤
姐那里,只见好些人围着哭呢。宝钗走到跟前,见凤姐已经停床,便大放悲声。宝
玉也拉着贾琏的手,大哭起来,贾琏也重新哭泣。平儿等因见无人劝解,只得含悲
上来劝止了。众人都悲哀不止。贾琏此时手足无措,叫人传了赖大来,叫他办理丧
事。自己回明了贾政,然后去行事。但是手头不济,诸事拮据。又想起凤姐素日的
好处来,更加悲哭不已。又见巧姐哭的死去活来,越发伤心。哭到天明,即刻打发
人去请他大舅子王仁过来。

那王仁自从王子腾死后,王子胜又是无能的人,任他胡为,已闹的六亲不和。
今知妹子死了,只得赶着过来哭了一场。见这里诸事将就,心下便不舒服,说:“我
妹妹在你家辛辛苦苦当了好几年家,也没有什么错处,你们家该认真的发送发送才
是,怎么这时候诸事还没有齐备?”贾琏本与王仁不睦,见他说些混帐话,知他不
懂的什么,也不大理他。王仁便叫了他外甥女儿巧姐过来,说:“你娘在时,本来
办事不周到:只知道一味的奉承老太太,把我们的人都不大看在眼里。外甥女儿!
你也大了,看见我从来沾染过你们没有?如今你娘死了,诸事要听着舅舅的话。你
母亲娘家的亲戚就是我和你二舅舅了。你父亲的为人,我也早知道了,只有敬重别
人的。那年什么尤姨娘死了,我虽不在京,听见说花了好些银子。如今你娘死了,
你父亲倒是这样的将就办去,你也不知道劝劝你父亲吗?”巧姐道:“我父亲巴不
得要好看,只是如今比不得从前了。现在手里没钱,所以诸事省些是有的。”王仁
道:“你的东西还少么?”巧姐儿道:“旧年抄去,何尝还有呢?”王仁道:“你
也这样说?我听见老太太又给了好些东西,你该拿出来。”巧姐又不好说父亲用去,
只推不知道。王仁便道:“哦,我知道了,不过是你要留着做嫁妆罢咧。”巧姐听
了,不敢回言,只气得哽噎难鸣的哭起来了。平儿生气说道:“舅老爷,有话等我
们二爷进来再说。姑娘这么点年纪,他懂的什么?”王仁道:“你们是巴不得二奶
奶死了,你们就好为王了。我并不要什么,好看些,也是你们的脸面!”说着赌气
坐着。巧姐满心的不舒服,心想:“我父亲并不是没情。我妈妈在时,舅舅不知拿
了多少东西去,如今说得这样干净!”于是便不大瞧得起他舅舅了。岂知王仁心里
想来,他妹妹不知积攒了多少。“虽说抄了家,那屋里的银子还怕少吗?必是怕我
来缠他们,所以也帮着这么说。这小东西儿也是不中用的!”从此王仁也嫌了巧姐
儿了。

贾琏并不知道,只忙着弄银钱使用。外头的大事叫赖大办了,里头也要用好些
钱,一时实在不能张罗。平儿知他着急,便叫贾琏道:“二爷也别过于伤了自己的
身子。”贾琏道:“什么身子!现在日用的钱都没有,这件事怎么办?偏有个糊涂行
子又在这里蛮缠,你想有什么法儿?”平儿道:“二爷也不用着急。若说没钱使唤,
我还有些东西,旧年幸亏没有抄在里头去,二爷要,就拿去当着使唤罢。”贾琏听
了,心想:“难得这样。”便笑道:“这样更好,省得我各处张罗。等我银子弄到
手了还你。”平儿道:“我的也是奶奶给的,什么还不还。只要这件事办的好看些
就是了。”贾琏心里倒着实感激他,便将平儿的东西拿了去,当钱使用。诸凡事情,
便与平儿商量。秋桐看着,心里就有些不甘,每每口角里头便说:“平儿没有了奶
奶,他要上去了。我是老爷的人,他怎么就越过我去了呢?”平儿也看出来了,只
不理他。倒是贾琏一时明白,越发把秋桐嫌了,碰着有些烦恼,便拿着秋桐出气。
邢夫人知道,反说贾琏不好。贾琏忍气不提。

再说凤姐停了十馀天,送了殡。贾政守着老太太的孝,总在外书房。那时清客
相公,渐渐的都辞去了,只有个程日兴还在那里,时常陪着说说话儿。提起:“家
运不好,一连人口死了好些,大老爷合珍大爷又在外头。家计一天难似一天,外头
东庄地亩也不知道怎么样,总不得了!”那程日兴道:“我在这里好些年,也知道
府上的人,那一个不是肥己的?一年一年都往他家里拿,那自然府上是一年不够一
年了。又添了大老爷珍大爷那边两处的费用,外头又有些债务。前儿又破了好些财,
要想衙门里缉贼追赃,那是难事。老世翁若要安顿家事,除非传那些管事的来,派
一个心腹人各处去清查清查:该去的去,该留的留;有了亏空,着在经手的身上赔
补,这就有了数儿了。那一座大园子,人家是不敢买的,这里头的出息也不少,又
不派人管了。几年老世翁不在家,这些人就弄神弄鬼儿的,闹的一个人不敢到园里,
这都是家人的弊。此时把下人查一查,好的使着,不好的便撵了,这才是道理。”
贾政点头道:“先生你有所不知!不必说下人,就是自己的侄儿,也靠不住!若要我
查起来,那能一一亲见亲知?况我又在服中,不能照管这些个。我素来又兼不大理
家,有的没的,我还摸不着呢。”程日兴道:“老世翁最是仁德的人。若在别人家
这样的家计,就穷起来,十年五载还不怕,便向这些管家的要,也就够了。我听见
世翁的家人还有做知县的呢。”贾政道:“一个人若要使起家人们的钱来,便了不
得了,只好自己俭省些。但是册子上的产业,若是实有还好,生怕有名无实了。”
程日兴道:“老世翁所见极是。晚生为什么说要查查呢!”贾政道:“先生必有所
闻?”程日兴道:“我虽知道些那些管事的神通,晚生也不敢言语的。”贾政听了,
便知话里有因,便叹道:“我家祖父以来,都是仁厚的,从没有刻薄过下人。我看
如今这些人一日不似一日了。在我手里行出主子样儿来,又叫人笑话。”

两人正说着,门上的进来回道:“江南甄老爷来了。”贾政便问道:“甄老爷
进京为什么?”那人道:“奴才也打听过了,说是蒙圣恩起复了。”贾政道:“不
用说了,快请罢。”那人出去,请了进来。那甄老爷即是甄宝玉之父,名叫甄应嘉,
表字友忠,也是金陵人氏,功勋之后。原与贾府有亲,素来走动的。因前年挂误革
了职,动了家产,今遇主上眷念功臣,赐还世职,行取来京陛见。知道贾母新丧,
特备祭礼,择日到寄灵的地方拜奠,所以先来拜望。

贾政有服,不能远接,在外书房门口等着。那位甄老爷一见,便悲喜交集。因
在制中,不便行礼,遂拉着手叙了些阔别思念的话。然后分宾主坐下,献了茶,彼
此又将别后事情的话说了。贾政问道:“老亲翁几时陛见的?”甄应嘉道:“前日。”
贾政道:“主上隆恩,必有温谕。”甄应嘉道:“主上的恩典,真是比天还高,下
了好些旨意。”贾政道:“什么好旨意?”甄应嘉道:“近来越寇猖獗,海疆一带,
小民不安,派了安国公征剿贼寇。主上因我熟悉土疆,命我前往安抚,但是即日就
要起身。昨日知老太太仙逝,谨备瓣香至灵前拜奠,稍尽微忱。”贾政即忙叩首拜
谢,便说:“老亲翁即此一行,必是上慰圣心,下安黎庶。诚哉莫大之功,正在此
行。但弟不克亲睹奇才,只好遥聆捷报。现在镇海统制是弟舍亲,会时务望青照。”
甄应嘉道:“老亲翁与统制是什么亲戚?”贾政道:“弟那年在江西粮道任时,将
小女许配与统制少君,结已经三载。因海口案内未清,继以海寇聚奸,所以音信
不通。弟深念小女,俟老亲翁安抚事竣后,拜恳便中一视。弟即修字数行,烦尊纪
带去,便感激不尽了。”甄应嘉道:“儿女之情,人所不免。我正在有奉托老亲翁
的事。昨蒙圣恩召取来京,因小儿年幼,家下乏人,将贱眷全带来京。我因钦限迅
速,昼夜先行,贱眷在后缓行,到京尚需时日。弟奉旨出京,不敢久留。将来贱眷
到京,少不得要到尊府,定叫小犬叩见。如可进教,遇有姻事可图之处,望乞留意
为感。”贾政一一答应。那甄应嘉又说了几句话,就要起身,说:“明日在城外再
见。”贾政见他事忙,谅难再坐,只得送出书房。

贾琏宝玉早已伺候在那里代送,因贾政未叫,不敢擅入。甄应嘉出来,两人上
去请安。应嘉一见宝玉,呆了一呆,心想:“这个怎么甚像我家宝玉!只是浑身缟
素。”问道:“至亲久阔,爷们都不认得了。”贾政忙指贾琏道:“这是家兄名赦
之子琏二侄儿。”又指着宝玉道:“这是第二小犬,名叫宝玉。”应嘉拍手道:“奇!
我在家听见说老亲翁有个衔玉生的爱子,名叫宝玉,因与小儿同名,心中甚为罕异。
后来想着这个也是常有的事,不在意了。岂知今日一见,不但面貌相同,且举止一
般,这更奇了。”问起年纪,“比这里的哥儿略小一岁。”贾政便又提起承荐包勇,
问及“令郎哥儿与小儿同名”的话述了一遍。应嘉因属意宝玉,也不暇问及那包勇
的好歹,只连连的称道:“真真罕异!”因又拉着宝玉的手,极致殷勤。又恐安国
公起身甚速,急须预备长行,勉强分手徐行。贾琏宝玉送出,一路又问了宝玉好些,
然后才登车而去。那贾琏宝玉回来见了贾政,便将应嘉问的话回了一遍。贾政命他
二人散去。贾琏又去张罗,算明凤姐丧事的账目。

宝玉回到自己房中,告诉了宝钗,说是:“常提的甄宝玉,我想一见不能,今
日倒先见了他父亲了。我还听得说,宝玉也不日要到京了,要求拜望我们老爷呢。
他也说和我一模一样的,我只不信。若是他后儿到了咱们这里来,你们都去瞧瞧,
看他果然和我像不像?”宝钗听了道:“嗳,你说话怎么越发没前后了?什么男人
同你一样都说出来了,还叫我们瞧去呢。”宝玉听了,知是失言,脸上一红,连忙
的还要解说。

不知何话,下回分解。