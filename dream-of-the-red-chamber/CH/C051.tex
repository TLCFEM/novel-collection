\chapter{薛小妹新编怀古诗~胡庸医乱用虎狼药}

话说众人闻得宝琴将素昔所经过各省内古迹为题,做了十首怀古绝句,内隐十
物,皆说:“这自然新巧。”都争着看时,只见写道是:

赤壁怀古
赤壁沉埋水不流,徒留名姓载空舟。
喧阗一炬悲风冷,无限英魂在内游。

交趾怀古
铜柱金城振纪纲,声传海外播戎羌。
马援自是功劳大,铁笛无烦说子房。

钟山怀古
名利何曾伴女身,无端被诏出凡尘。
牵连大抵难休绝,莫怨他人嘲笑频。

淮阴怀古
壮士须防恶犬欺,三齐位定盖棺时。
寄言世俗休轻鄙,一饭之恩死也知。

广陵怀古
蝉噪鸦栖转眼过,隋堤风景近如何?
只缘占尽风流号,惹得纷纷口舌多。

桃叶渡怀古
衰草闲花映浅池,桃枝桃叶总分离。
六朝梁栋多如许,小照空悬壁上题。

青冢怀古
黑水茫茫咽不流,冰弦拨尽曲中愁。
汉家制度诚堪笑,樗栎应惭万古羞。

马嵬怀古
寂寞脂痕积汗光,温柔一旦付东洋。
只因遗得风流迹,此日衣裳尚有香。

蒲东寺怀古
小红骨贱一身轻,私掖偷携强撮成。
虽被夫人时吊起,已经勾引彼同行。

梅花观怀古
不在梅边在柳边,个中谁拾画婵娟?
团圆莫忆春香到,一别西风又一年。

众人看了,都称奇妙。宝钗先说道:“前八首都是史鉴上有据的,后二首却无
考。我们也不大懂得,不如另做两首为是。”黛玉忙拦着:“这宝姐姐也忒胶柱鼓
瑟、矫揉造作了。两首虽于史鉴上无考,咱们虽不曾看这些外传,不知底里,难道
咱们连两本戏也没见过不成?那三岁的孩子也知道,何况咱们?”探春便道:“这
话正是了。”李纨又道:“况且他原走到这个地方的。这两件事虽无考,古往今来,
以讹传讹,好事者竟故意的弄出这古迹来以愚人。比如那年上京的时节,便是关夫
子的坟,倒见了三四处。关夫子一身事业皆是有据的,如何又有许多的坟?自然是
后来人敬爱他生前为人,只怕从这敬爱上穿凿出来也是有的。及至看《广舆记》上,
不止关夫子的坟多有,古来有名望的人,那坟就不少。无考的古迹更多。如今这两
首诗虽无考,凡说书唱戏,甚至于求的签上都有。老少男女俗语口头,人人皆知皆
说的。况且又并不是看了《西厢记》、《牡丹亭》的词曲,怕看了邪书了。这也无
妨,只管留着。”宝钗听说,方罢了。大家猜了一回,皆不是的。

冬日天短,觉得又是吃晚饭时候,一齐往前头来吃晚饭。因有人回王夫人说:
“袭人的哥哥花自芳,在外头回进来说,他母亲病重了,想他女儿。他来求恩典,
接袭人家去走走。”王夫人听了,便说:“人家母女一场,岂有不许他去的呢。”
一面就叫了凤姐来告诉了,命他酌量办理。凤姐儿答应了,回至屋里,便命周瑞家
的去告诉袭人原故。吩咐周瑞家的:“再将跟着出门的媳妇传一个,你们两个人,
再带两个小丫头子,跟了袭人去。分头派四个有年纪的跟车。要一辆大车,你们带
着坐,一辆小车,给丫头们坐。”周瑞家的答应了,才要去,凤姐又道:“那袭人
是个省事的,你告诉说我的话:叫他穿几件颜色好衣裳,大大的包一包袱衣裳拿着,
包袱要好好的,拿手炉也拿好的。临走时,叫他先到这里来我瞧。”周瑞家的答应
去了。

半日,果见袭人穿戴了,两个丫头和周瑞家的拿着手炉和衣包,凤姐看袭人头
上戴着几枝金钗珠钏,倒也华丽,又看身上穿着桃红百花刻丝银鼠袄,葱绿盘金彩
绣绵裙,外面穿着青缎灰鼠褂。凤姐笑道:“这三件衣裳都是太太的,赏了你倒是
好的。但这褂子太素了些,如今穿着也冷,你该穿一件大毛的。”袭人笑道:“太
太就给了这件灰鼠的,还有件银鼠的。说赶年下再给大毛的呢。”凤姐笑道:“我
倒有一件大毛的,我嫌风毛出的不好了正要改去,也罢,先给你穿去罢。等年下太
太给你做的时节,我再改罢。只当你还我的一样。”众人都笑道:“奶奶惯会说这
话。成年家大手大脚的,替太太不知背地里赔垫了多少东西,真真赔的是说不出来
的,那里又和太太算去?偏这会子又说这小气话取笑来了。”凤姐儿笑道:“太太
那里想的到这些?究竟这又不是正经事。再不照管,也是大家的体面;说不得我自
己吃些亏,把众人打扮体统了,宁可我得个好名儿也罢了。一个一个‘烧糊了的
子’似的,人先笑话我,说我当家倒把人弄出个花子来了。”众人听了,都叹说:
“谁似奶奶这么着圣明,在上体贴太太,在下又疼顾下人。”一面说,一面只见凤
姐命平儿将昨日那件石青刻丝八团天马皮褂子拿出来,给了袭人。又看包袱,只得
一个弹墨花绫水红绸里的夹包袱,里面只见包着两件半旧绵袄合皮褂子。凤姐又命
平儿把一个玉色绸里的哆罗呢包袱拿出来,又命包上一件雪褂子。

平儿走去拿了出来,一件是件旧大红猩猩毡的,一件是半旧大红羽缎的。袭人
道:“一件就当不起了。”平儿笑道:“你拿这猩猩毡的。把这件顺手带出来,叫
人给邢大姑娘送去,昨儿那么大雪,人人都穿着不是猩猩毡、就是羽缎的,十来件
大红衣裳,映着大雪,好不齐整。只有他穿着那几件旧衣裳,越发显的拱肩缩背,
好不可怜见的,如今把这件给他罢。”凤姐笑道:“我的东西,他私自就要给人。
我一个还花不够,再添上你提着,更好了!”众人笑道:“这都是奶奶素日孝敬太
太,疼爱下人。要是奶奶素日是小气的,收着东西为事的,不顾下人的,姑娘那里
敢这么着?”凤姐笑道:“所以知道我的,也就是他还知三分罢了。”说着,又嘱
咐袭人道:“你妈要好了就罢,要不中用了,只得住下,打发人来回我,我再另打
发人给你送铺盖去。可别使他们的铺盖和梳头的家伙。”又吩咐周瑞家的道:“你
们自然是知道这里的规矩的,也不用我吩咐了。”周瑞家的答应:“都知道:我们
这去到那里,总叫他们的人回避。要住下,必是另要一两间内房的。”说着,跟了
袭人出去,又吩咐小厮预备灯笼,遂坐车往花自芳家来,不在话下。

这里凤姐又将怡红院的嬷嬷唤了两个来,吩咐道:“袭人只怕不来家了。你们
素日知道那个大丫头知好歹,派出来在宝玉屋里上夜。你们也好生照管着,别由着
宝玉胡闹。”两个嬷嬷答应着去了,一时来回说:“派了晴雯和麝月在屋里,我们
四个人原是轮流着带管上夜的。”凤姐听了点头,又说道:“晚上催他早睡,早上
催他早起。”老嬷嬷们答应了,自回园去。一时果有周瑞家的带了信回凤姐说:“袭
人之母业已停床,不能回来。”凤姐回明了王夫人,一面着人往大观园去取他的铺
盖妆奁。宝玉看着晴雯麝月二人打点妥当。

送去之后,晴雯麝月皆卸罢残妆,脱换过裙袄。晴雯只在熏笼上围坐,麝月笑
道:“你今儿别装小姐了,我劝你也动一动儿。”晴雯道:“等你们都去净了,我
再动不迟。有你们一日,我且受用一日。”麝月笑道:“好姐姐,我铺床,你把那
穿衣镜的套子放下来,上头的划子划上。你的身量比我高些。”说着,便去给宝玉
铺床。晴雯了一声,笑道:“人家才坐暖和了,你就来闹。”此时宝玉正坐着纳
闷,想袭人之母不知是死是活,忽听见晴雯如此说,便自己起身出去,放下镜套,
划上消息。进来笑道:“你们暖和罢,我都弄完了。”晴雯笑道:“终久暖和不成,
我又想起来,汤婆子还没拿来呢。”麝月道:“这难为你想着!他素日又不要汤壶,
咱们那熏笼上又暖和,比不得那屋里炕凉,今儿可以不用。”宝玉笑道:“你们两
个都在那上头睡了,我这外边没个人,我怪怕的,一夜也睡不着。”晴雯道:“我
是在这里睡的,麝月,你叫他往外边睡去。”说话之间,天已一更,麝月早已放下
帘幔,移灯炷香,伏侍宝玉卧下,二人方睡。晴雯自在熏笼上,麝月便在暖阁外边。

至三更以后,宝玉睡梦之中,便叫袭人。叫了两声,无人答应,自己醒了,方
想起袭人不在家,自己也好笑起来。晴雯已醒,因唤麝月道:“连我都醒了,他守
在旁边还不知道,真是挺死尸呢!”麝月翻身打个哈什,笑道:“他叫袭人,与我
什么相干!”因问:“做什么?”宝玉说要吃茶。麝月忙起来,单穿着红绸小绵袄
儿。宝玉道:“披了我的皮袄再去,仔细冷着。”麝月听说,回手便把宝玉披着起
来的一件貂颏满襟暖袄披上,下去向盆内洗洗手,先倒了一钟温水,拿了大漱盂,
宝玉漱了口。然后才向茶桶上取了茶碗,先用温水过了,向暖壶中倒了半碗茶,递
给宝玉吃了,自己也漱了一漱,吃了半碗。晴雯笑道:“好妹妹,也赏我一口儿呢。”
麝月笑道:“越发上脸儿了!”晴雯道:“好妹妹,明儿晚上你别动,我伏侍你一
夜,如何?”麝月听说,只得也伏侍他漱了口,倒了半碗茶给他吃了。麝月笑道:
“你们两个别睡,说着话儿,我出去走走回来。”晴雯笑道:“外头有个鬼等着呢。”
宝玉道:“外头自然有大月亮的。我们说着话,你只管去。”一面说,一面便嗽了
两声。麝月便开了后房门,揭起毡帘一看,果然好月色。晴雯等他出去,便欲唬他
玩耍,仗着素日比别人气壮,不畏寒冷,也不披衣,只穿着小袄便蹑手蹑脚的下了
熏笼,随后出来。宝玉劝道:“罢呀,冻着不是玩的!”晴雯只摆手,随后出了屋
门,只见月光如水。忽听一阵微风,只觉侵肌透骨,不禁毛骨悚然。心下自思道:
“怪道人说热身子不可被风吹,这一冷果然利害。”一面正要唬他,只听宝玉在内
高声说道:“晴雯出来了!”

晴雯忙回身进来,笑道:“那里就唬死了他了?偏惯会这么蝎蝎螫螫老婆子的
样儿。”宝玉笑道:“倒不是怕唬坏了他。头一件你冻着也不好,二则他不防,不
免一喊,倘或惊醒了别人,不说咱们是玩意儿,倒反说袭人才去了一夜,你们就见
神见鬼的。你来把我这边的被掖掖罢。”晴雯听说,就上来掖了一掖,伸手进去就
渥一渥。宝玉笑道:“好冷手,我说看冻着。”一面又见晴雯两腮如胭脂一般,用
手摸一摸,也觉冰冷。宝玉道:“快进被来渥渥罢。”一语未了,只听咯噔的一声
门响,麝月慌慌张张的笑着进来,说着笑道:“唬我一跳好的!黑影子里,山子石
后头,只见一个人蹲着。我才要叫喊,原来是那个大锦鸡,见了人,一飞飞到亮处
来,我才见了。要冒冒失失一嚷,倒闹起人来。”一面说,一面洗手,又笑道:“说
晴雯出去了?我怎么没见。一定是要唬我去了。”宝玉笑道:“这不是他?在这里渥
着呢。我若不嚷的快,可是倒唬一跳。”晴雯笑道:“也不用我唬去,这小蹄子已
经自惊自怪的了。”一面说,一面仍回自己被中去。麝月道:“你就这么‘跑解马’
的打扮儿,伶伶俐俐的出去了不成?”宝玉笑道:“可不就是这么出去了。”麝月
道:“你死不拣好日子!你出去自站一站,瞧把皮不冻破了你的。”说着又将火盆
上的铜罩揭起,拿灰锹重将熟炭埋了一埋,拈了两块速香放上,仍旧罩了。至屏后,
重剔亮了灯,方才睡下。

晴雯因方才一冷,如今又一暖,不觉打了两个嚏喷。宝玉叹道:“如何?到底
伤了风了。”麝月笑道:“他早起就嚷不受用,一日也没吃碗正经饭。他这会子不
说保养着些,还要捉弄人,明儿病了,叫他自作自受。”宝玉问道:“头上热不热?”
晴雯嗽了两声,说道:“不相干,那里这么娇嫩起来了。”说着,只听外间屋里
上的自鸣钟“当当”的两声,外间值宿的老嬷嬷嗽了两声,因说道:“姑娘们睡罢,
明儿再说笑罢。”宝玉方悄悄的笑道:“咱们别说话了,看又惹他们说话。”说着,
方大家睡了。

至次日起来,晴雯果觉有些鼻塞声重,懒怠动弹。宝玉道:“快别声张。太太
知道了,又要叫你搬回家去养着。家里纵好,到底冷些,不如在这里。你就在里间
屋里躺着,我叫人请了大夫,悄悄的从后门进来瞧瞧就是了。”晴雯道:“虽这么
说,你到底要告诉大奶奶一声儿。不然一时大夫来了,人问起来怎么说呢?”宝玉
听了有理,便唤一个老嬷嬷来吩咐道:“你回大奶奶去,就说晴雯白冷着了些,不
是什么大病。袭人又不在家,他若家去养病,这里更没有人了。传一个大夫,从后
门悄悄的进来瞧瞧,别回太太了。”老嬷嬷去了,半日回来说:“大奶奶知道了。
说两剂药好了便罢,若不好时,还是出去为是。如今的时气不好,沾染了别人事小,
姑娘们的身子要紧。”晴雯睡在暖阁里,只管咳嗽,听了这话,气的嚷道:“我那
里就害瘟病了?生怕招了人。我离了这里,看你们这一辈子都别头疼脑热的!”说
着,便真要起来。宝玉忙按他,笑道:“别生气,这原是他的责任,生恐太太知道
了说他。不过白说一句。你素昔又爱生气,如今肝火自然又盛了。”

正说时,人回大夫来了。宝玉便走过来,避在书架后面。只见两三个后门口的
老婆子带了一个太医进来。这里的丫头都回避了,有三四个老嬷嬷放下暖阁上的大
红绣幔,晴雯从幔中单伸出手来。那大夫见这只手上有两根指甲,足有二三寸长,
尚有金凤仙花染的通红的痕迹,便回过头来。有一个老嬷嬷忙拿了一块绢子掩上
了。那大夫方诊了一回脉,起身到外间,向嬷嬷们说道:“小姐的症是外感内滞。
近日时气不好,竟算是个小伤寒。幸亏是小姐,素日饮食有限,风寒也不大,不过
是气血原弱,偶然沾染了些,吃两剂药疏散疏散就好了。”说着,便又随婆子们出
去。彼时李纨已遣人知会过后门上的人及各处丫鬟回避。大夫只见了园中景致,并
不曾见一个女子。一时出了园门,就在守园门的小厮们的班房内坐了,开了药方。
老嬷嬷道:“老爷且别去,我们小爷罗嗦,恐怕还有话问。”那太医忙道:“方才
不是小姐,是位爷不成?那屋子竟是绣房,又是放下幔子来瞧的,如何是位爷呢?”
老嬷嬷笑道:“我的老爷,怪道小子才说:‘今儿请了一位新太医来了。’真不知
我们家的事。那屋子是我们小哥儿的,那人是屋里的丫头,倒是个‘大姐’,那里
的小姐的绣房?小姐病了,你那么容易就进去了?”说着,拿了药方进去。

宝玉看时,上面有紫苏、桔梗、防风、荆芥等药,后面又有枳实、麻黄。宝玉
道:“该死该死,他拿着女孩儿们也像我们一样的治法,如何使得?凭他有什么内
滞,这枳实、麻黄如何禁得?谁请了来的?快打发他去罢,再请一个熟的来罢。”老
嬷嬷道:“用药好不好,我们不知道。如今再叫小厮去请王大夫去倒容易,只是这
个大夫又不是告诉总管房请的,这马钱是要给他的。”宝玉道:“给他多少?”婆
子道:“少了不好,看来得一两银子,才是我们这样门户的礼。”宝玉道:“王大
夫来了,给他多少?”婆子笑道:“王大夫和张大夫每常来了,也并没个给钱的,
不过每年四节一个趸儿送礼,那是一定的年例。这个人新来了一次,须得给他一两
银子。”宝玉听说,就命麝月去取银子。麝月道:“花大姐姐还不知搁在那里呢?”
宝玉道:“我常见着在那小螺甸柜子里拿银子,我和你找去。”说着二人来至袭人
堆东西的屋内,开了螺甸柜子。上一 都是些笔墨、扇子、香饼、各色荷包、汗
巾等类的东西,下一却有几串钱。于是开了抽屉,才看见一个小笸箩内放着几块
银子,倒也有戥子。麝月便拿了一块银,提起戥子来问宝玉:“那是一两的星儿?”
宝玉笑道:“你问的我有趣儿,你倒成了是才来的了。”麝月也笑了,又要去问人。
宝玉道:“拣那大的给他一块就是了。又不做买卖,算这些做什么。”麝月听了,
便放下戥子,拣了一块掂了一掂,笑道:“这一块只怕是一两了。宁可多些好,别
少了叫那穷小子笑话:不说咱们不认得戥子,倒说咱们有心小气似的。”那婆子站
在门口笑道:“那是五两的锭子夹了半个,这一块至少还有二两呢。这会子又没夹
剪,姑娘收了这块,拣一块小些的。”麝月早关了柜子出来,笑道:“谁又找去呢,
多少你拿了去就完了!”宝玉道:“你快叫焙茗再请个大夫来罢。”婆子接了银子,
自去料理。

一时焙茗果请了王大夫来,先诊了脉,后说病症,也与前头不同。方子上果然
没有枳实、麻黄等药,倒有当归、陈皮、白芍等药,那分两较先也减了些。宝玉喜
道:“这才是女孩儿们的药。虽疏散,也不可太过。旧年我病了,却是伤寒,内里
饮食停滞,他瞧了还说我禁不起麻黄、石膏、枳实等狼虎药。我和你们就如秋天芸
儿进我的那才开的白海棠似的;我禁不起的药,你们那里经得起?比如人家坟里的
大杨树,看着枝叶茂盛,都是空心子的。”麝月笑道:“野坟里只有杨树,难道就
没有松柏不成?最讨人嫌的是杨树,那么大树只一点子叶子,没一点风儿他也是乱
响。你偏要比他,你也太下流了。”宝玉笑道:“松柏不敢比。连孔夫子都说‘岁
寒然后知松柏之后雕’呢,可知这两件东西高雅。不害臊的才拿他混比呢。”

说着,只见老婆子取了药来。宝玉命把煎药的银铞子找了出来,就命在火盆上
煎。晴雯因说:“正经给他们茶房里煎去罢咧,弄的这屋里药气,如何使得?”宝
玉道:“药气比一切的花香还香呢。神仙采药烧药,再者高人逸士采药治药,最妙
的一件东西。这屋里我正想各色都齐了,就只少药香,如今恰全了。”一面说,一
面早命人煨上。又嘱咐麝月打点些东西,叫个老嬷嬷去看袭人,劝他少哭。一一妥
当,方过前边来贾母王夫人处请安吃饭。

正值凤姐儿和贾母王夫人商议说:“天又短,又冷,不如以后大嫂子带着姑娘
们在园子里吃饭。等天暖和了,再来回的跑,也不妨。”王夫人笑道:“这也是好
主意。刮风下雪倒便宜。吃东西受了冷气也不好,空心走来,一肚子冷气,压上些
东西也不好。不如园子后门里头的五间大屋子,横竖有女人们上夜的,挑两个女厨
子在那里单给他姐妹弄饭。新鲜菜蔬是有分例的,在总管账房里支了去,或要钱要
东西。那些野鸡獐狍各样野味,分些给他们就是了。”贾母道:“我也正想着呢,
就怕又添厨房事多些。”凤姐道:“并不事多:一样的分例,这里添了,那里减了。
就便多费些事,小姑娘们受了冷气,别人还可,第一,林妹妹如何禁得住?就连宝
玉兄弟也禁不住。况兼众位姑娘都不是结实身子。”

凤姐儿说毕,未知贾母何言,且听下回分解。