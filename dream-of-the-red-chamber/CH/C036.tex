\chapter{绣鸳鸯梦兆绛芸轩~识分定情悟梨香院}

话说贾母自王夫人处回来,见宝玉一日好似一日,心中自是欢喜。因怕将来贾
政又叫他,遂命人将贾政的亲随小厮头儿唤来,吩咐:“以后倘有会人待客诸样的
事,你老爷要叫宝玉,你不用上来传话,就回他说我说的:一则打重了,得着实将
养几个月才走得;二则他的星宿不利,祭了星,不见外人,过了八月,才许出二门。”
那小厮头儿听了领命而去。贾母又命李嬷嬷袭人等来将此话说与宝玉,使他放心。
那宝玉素日本就懒与士大夫诸男人接谈,又最厌峨冠礼服贺吊往还等事,今日得了
这句话,越发得意了,不但将亲戚朋友一概杜绝了,而且连家庭中晨昏定省一发都
随他的便了。日日只在园中游玩坐卧,不过每日一清早到贾母王夫人处走走就回来
了,却每日甘心为诸丫头充役,倒也得十分消闲日月。或如宝钗辈有时见机劝导,
反生起气来,只说:“好好的一个清净洁白女子,也学的钓名沽誉,入了国贼禄鬼
之流。这总是前人无故生事,立意造言,原为引导后世的须眉浊物。不想我生不幸,
亦且琼闺绣阁中亦染此风,真真有负天地钟灵毓秀之德了!”众人见他如此,也都
不向他说正经话了。独有黛玉自幼儿不曾劝他去立身扬名,所以深敬黛玉。

闲言少述。如今且说凤姐自见金钏儿死后,忽见几家仆人常来孝敬他些东西,
又不时的来请安奉承,自己倒生了疑惑,不知何意。这日又见人来孝敬他东西,因
晚间无人时笑问平儿。平儿冷笑道:“奶奶连这个都想不起来了?我猜他们的女孩
儿都必是太太屋里的丫头,如今太太屋里有四个大的,一个月一两银子的分例,下
剩的都是一个月只几百钱。如今金钏儿死了,必定他们要弄这一两银子的窝儿呢。”
凤姐听了,笑道:“是了,是了,倒是你想的不错。只是这起人也太不知足。钱也
赚够了,苦事情又摊不着他们,弄个丫头搪塞身子儿也就罢了,又要想这个巧宗儿!
他们几家的钱也不是容易花到我跟前的,这可是他们自寻。送什么我就收什么,横
竖我有主意。”凤姐儿安下这个心,所以只管耽延着,等那些人把东西送足了,然
后乘空方回王夫人。

这日午间,薛姨妈、宝钗、黛玉等正在王夫人屋里,大家吃西瓜。凤姐儿得便
回王夫人道:“自从玉钏儿的姐姐死了,太太跟前少着一个人,太太或看准了那个
丫头,就吩咐了,下月好发放月钱。”王夫人听了,想了一想道:“依我说,什么
是例,必定四个五个的?够使就罢了。竟可以免了罢。”凤姐笑道:“论理,太太
说的也是;只是原是旧例。别人屋里还有两个呢,太太倒不按例了。况且省下一两
银子,也有限的。”王夫人听了,又想了想道:“也罢,这个分例只管关了来,不
用补人,就把这一两银子给他妹妹玉钏儿罢。他姐姐伏侍了我一场,没个好结果,
剩下他妹妹跟着我,吃个双分儿也不为过。”凤姐答应着,回头望着玉钏儿笑道:
“大喜,大喜!”玉钏儿过来磕了头。

王夫人又问道:“正要问你:如今赵姨娘周姨娘的月例多少?”凤姐道:“那
是定例,每人二两。赵姨娘有环兄弟的二两,共是四两,另外四串钱。”王夫人道:
“月月可都按数给他们?”凤姐见问得奇,忙道:“怎么不按数给呢!”王夫人道:
“前儿恍惚听见有人抱怨,说短了一串钱,什么原故?”凤姐忙笑道:“姨娘们的
丫头月例,原是人各一吊钱,从旧年他们外头商量的,姨娘们每位丫头,分例减半,
人各五百钱。每位两个丫头,所以短了一吊钱。这事其实不在我手里,我倒乐得给
他们呢,只是外头扣着,这里我不过是接手儿,怎么来怎么去,由不得我做主。我
倒说了两三回,仍旧添上这两分儿为是,他们说了‘只有这个数儿’,叫我也难再
说了。如今我手里给他们,每月连日子都不错。先时候儿在外头关,那个月不打饥
荒,何曾顺顺溜溜的得过一遭儿呢。”王夫人听说,就停了半晌,又问:“老太太
屋里几个一两的?”凤姐道:“八个。如今只有七个,那一个是袭人。”王夫人说:
“这就是了。你宝兄弟也并没有一两的丫头,袭人还算老太太房里的人。”凤姐笑
道:“袭人还是老太太的人,不过给了宝兄弟使,他这一两银子还在老太太的丫头
分例上领。如今说因为袭人是宝玉的人,裁了这一两银子,断乎使不得。若说再添
一个人给老太太,这个还可以裁他。若不裁他,须得环兄弟屋里也添上一个,才公
道均匀了。就是晴雯、麝月他们七个大丫头,每月人各月钱一吊,佳蕙他们八个小
丫头们,每月人各月钱五百,还是老太太的话,别人也恼不得气不得呀。”

薛姨妈笑道:“你们只听凤丫头的嘴,倒像倒了核桃车子似的。帐也清楚,理
也公道。”凤姐笑道:“姑妈,难道我说错了吗?”薛姨妈笑道:“说的何尝错,
只是你慢着些儿说不省力些?”凤姐才要笑,忙又忍住了,听王夫人示下。王夫人
想了半日,向凤姐道:“明儿挑一个丫头送给老太太使唤,补袭人,把袭人的一分
裁了。把我每月的月例,二十两银子里拿出二两银子一吊钱来,给袭人去。以后凡
事有赵姨娘周姨娘的,也有袭人的,只是袭人的这一分,都从我的分例上匀出来,
不必动官中的就是了。”凤姐一一的答应了,笑推薛姨妈道:“姑妈听见了?我素
日说的话如何?今儿果然应了。”薛姨妈道:“早就该这么着。那孩子模样儿不用
说,只是他那行事儿的大方,见人说话儿的和气,里头带着刚硬要强,倒实在难得
的。”王夫人含泪说道:“你们那里知道袭人那孩子的好处?比我的宝玉还强十倍
呢!宝玉果然有造化,能够得他长长远远的伏侍一辈子,也就罢了。”凤姐道:“既
这么样,就开了脸,明放他在屋里不好?”王夫人道:“这不好:一则年轻;二则
老爷也不许;三则宝玉见袭人是他的丫头,纵有放纵的事,倒能听他的劝,如今做
了跟前人,那袭人该劝的也不敢十分劝了。如今且浑着,等再过二三年再说。”

说毕,凤姐见无话,便转身出来。刚至廊檐下,只见有几个执事的媳妇子正等
他回事呢,见他出来,都笑道:“奶奶今儿回什么事,说了这半天?可别热着罢。”
凤姐把袖子挽了几挽,着那角门的门槛子,笑道:“这里过堂风,倒凉快,吹一
吹再走。”又告诉众人道:“你们说我回了这半日的话,太太把二百年的事都想起
来问我,难道我不说罢?”又冷笑道:“我从今以后,倒要干几件刻薄事了。抱怨
给太太听,我也不怕!糊涂油蒙了心、烂了舌头、不得好死的下作娼妇们,别做娘
的春梦了!明儿一裹脑子扣的日子还有呢。如今裁了丫头的钱就抱怨了咱们,也不
想想自己也配使三个丫头!”一面骂,一面方走了,自去挑人回贾母话去,不在话
下。

却说薛姨妈等这里吃毕西瓜,又说了一回闲话儿,各自散去。宝钗与黛玉回至
园中,宝钗要约着黛玉往藕香榭去,黛玉因说还要洗澡,便各自散了。宝钗独自行
来,顺路进了怡红院,意欲寻宝玉去说话儿,以解午倦。不想步入院中,鸦雀无闻,
一并连两只仙鹤在芭蕉下都睡着了。宝钗便顺着游廊,来至房中。只见外间床上横
三竖四,都是丫头们睡觉。转过十锦子,来至宝玉的房内,宝玉在床上睡着了,
袭人坐在身旁,手里做针线,傍边放着一柄白犀麈。

宝钗走近前来,悄悄的笑道:“你也过于小心了。这个屋里还有苍蝇蚊子?还
拿蝇刷子赶什么?”袭人不防,猛抬头见是宝钗,忙放针线起身,悄悄笑道:“姑
娘来了,我倒不防,唬了一跳。姑娘不知道:虽然没有苍蝇蚊子,谁知有一种小虫
子,从这纱眼里钻进来,人也看不见。只睡着了咬一口,就像蚂蚁叮的。”宝钗道:
“怨不得,这屋子后头又近水,又都是香花儿,这屋子里头又香,这种虫子都是花
心里长的,闻香就扑。”说着,一面就瞧他手里的针线。原来是个白绫红里的兜肚,
上面扎着鸳鸯戏莲的花样,红莲绿叶,五色鸳鸯。宝钗道:“嗳哟,好鲜亮活计。
这是谁的,也值的费这么大工夫?”袭人向床上嘴儿。宝钗笑道:“这么大了,
还带这个?”袭人笑道:“他原是不带,所以特特的做的好了,叫他看见,由不得
不带。如今天热,睡觉都不留神,哄他带上了,就是夜里纵盖不严些儿,也就罢了。
你说这一个就用了工夫,还没看见他身上带的那一个呢!”宝钗笑道:“也亏你耐
烦。”袭人道:“今儿做的工夫大了,脖子低的怪酸的。”又笑道:“好姑娘,你
略坐一坐,我出去走走就来。”说着就走了。宝钗只顾看着活计便不留心,一蹲身,
刚刚的也坐在袭人方才坐的那个所在。因又见那个活计实在可爱,不由的拿起针
来,就替他作。

不想黛玉因遇见湘云,约他来与袭人道喜,二人来至院中。见静悄悄的,湘云
便转身先到厢房里去找袭人去了。那黛玉却来至窗外,隔着窗纱往里一看,只见宝
玉穿着银红纱衫子,随便睡着在床上,宝钗坐在身旁做针线,傍边放着蝇刷子。黛
玉见了这个景况,早已呆了,连忙把身子一躲,半日又握着嘴笑,却不敢笑出来,
便招手儿叫湘云。湘云见他这般,只当有什么新闻,忙也来看,才要笑,忽然想起
宝钗素日待他厚道,便忙掩住口。知道黛玉口里不让人,怕他取笑,便忙拉过他来,
道:“走罢。我想起袭人来,他说晌午要到池子里去洗衣裳,想必去了,咱们找他
去罢。”黛玉心下明白,冷笑了两声,只得随他走了。

这里宝钗只刚做了两三个花瓣,忽见宝玉在梦中喊骂,说:“和尚道士的话如
何信得?什么‘金玉姻缘’?我偏说‘木石姻缘’!”宝钗听了这话,不觉怔了。忽
见袭人走进来,笑道:“还没醒呢吗?”宝钗摇头。袭人又笑道:“我才碰见林姑
娘史大姑娘,他们进来了么?”宝钗道:“没见他们进来。”因向袭人笑道:“他
们没告诉你什么?”袭人红了脸,笑道:“总不过是他们那些玩话,有什么正经说
的。”宝钗笑道:“今儿他们说的可不是玩话,我正要告诉你呢,你又忙忙的出去
了。”一句话未完,只见凤姐打发人来叫袭人。宝钗笑道:“就是为那话了。”袭
人只得叫起两个丫头来,同着宝钗出怡红院,自往凤姐这里来。果然是告诉他这话,
又教他给王夫人磕头,且不必去见贾母。倒把袭人说的甚觉不好意思。

及见过王夫人回来,宝玉已醒,问起原故,袭人且含糊答应。至夜间人静,袭
人方告诉了。宝玉喜不自禁,又向他笑道:“我可看你回家去不去了!那一回往家
里走了一趟,回来就说你哥哥要赎你,又说在这里没着落,终久算什么,说那些无
情无义的生分话唬我。从今我可看谁来敢叫你去?”袭人听了,冷笑道:“你倒别
这么说。从此以后,我是太太的人了,我要走,连你也不必告诉,只回了太太就走。”
宝玉笑道:“就算我不好,你回了太太去了,叫别人听见说我不好,你去了,你有
什么意思呢?”袭人笑道:“有什么没意思的?难道下流人我也跟着罢?再不然还有
个死呢!人活百岁,横竖要死,这口气没了,听不见看不见就罢了。”宝玉听见这
话,便忙握他的嘴,说道:“罢罢,你别说这些话了。”袭人深知宝玉性情古怪,
听见奉承吉利话,又厌虚而不实,听了这些近情的实话,又生悲感。也后悔自己冒
撞,连忙笑着,用话截开,只拣宝玉那素日喜欢的,说些春风秋月,粉淡脂红,然
后又说到女儿如何好。不觉又说到女儿死的上头,袭人忙掩住口。

宝玉听至浓快处,见他不说了,便笑道:“人谁不死?只要死的好。那些须眉
浊物只听见‘文死谏’‘武死战’这二死是大丈夫的名节,便只管胡闹起来。那里
知道有昏君,方有死谏之臣,只顾他邀名,猛拚一死,将来置君父于何地?必定有
刀兵,方有死战,他只顾图汗马之功,猛拚一死,将来弃国于何地?”袭人不等说
完,便道:“古时候儿这些人,也因出于不得已他才死啊。”宝玉道:“那武将要
是疏谋少略的,他自己无能,白送了性命,这难道也是不得已么?那文官更不比武
官了:他念两句书,记在心里,若朝廷少有瑕疵,他就胡弹乱谏,邀忠烈之名;倘
有不合,浊气一涌,即时拚死,这难道也是不得已?要知道那朝廷是受命于天,若
非圣人,那天也断断不把这万几重任交代。可知那些死的,都是沽名钓誉,并不知
君臣的大义。比如我此时若果有造化,趁着你们都在眼前,我就死了,再能够你们
哭我的眼泪,流成大河,把我的尸首漂起来,送到那鸦雀不到的幽僻去处,随风化
了,自此再不托生为人,这就是我死的得时了。”袭人忽见说出这些疯话来,忙说:
“困了。”不再答言。那宝玉方合眼睡着。次日也就丢开。

一日,宝玉因各处游的腻烦,便想起《牡丹亭》曲子来,自己看了两遍,犹不
惬怀,因闻得梨香院的十二个女孩儿中,有个小旦龄官,唱的最妙。因出了角门来
找时,只见葵官药官都在院内,见宝玉来了,都笑迎让坐。宝玉因问:“龄官在那
里?”都告诉他说:“在他屋里呢。”宝玉忙至他屋内,只见龄官独自躺在枕上,
见他进来,动也不动。宝玉身旁坐下,因素昔与别的女孩子玩惯了的,只当龄官也
和别人一样,遂近前陪笑,央他起来唱一套“袅晴丝”。不想龄官见他坐下,忙抬
起身来躲避,正色说道:“嗓子哑了,前儿娘娘传进我们去,我还没有唱呢。”宝
玉见他坐正了,再一细看,原来就是那日蔷薇花下画“蔷”字的那一个。又见如此
景况,从来未经过这样被人弃厌,自己便讪讪的,红了脸,只得出来了。

药官等不解何故,因问其所以,宝玉便告诉了他。宝官笑说道:“只略等一等,
蔷二爷来了,他叫唱是必唱的。”宝玉听了,心下纳闷,因问:“蔷哥儿那里去了?”
宝官道:“才出去了,一定就是龄官儿要什么,他去变弄去了。”宝玉听了以为奇
特。少站片时,果见贾蔷从外头来了,手里提着个雀儿笼子,上面扎着小戏台,并
一个雀儿,兴兴头头往里来找龄官。见了宝玉,只得站住。宝玉问他:“是个什么
雀儿?”贾蔷笑道:“是个玉顶儿,还会衔旗串戏。”宝玉道:“多少钱买的?”
贾蔷道:“一两八钱银子。”一面说,一面让宝玉坐,自己往龄官屋里来。

宝玉此刻把听曲子的心都没了,且要看他和龄官是怎么样。只见贾蔷进去,笑
道:“你来瞧这个玩意儿。”龄官起身问:“是什么?”贾蔷道:“买了个雀儿给
你玩,省了你天天儿发闷。我先玩个你瞧瞧。”说着,便拿些谷子,哄的那个雀儿
果然在那戏台上衔着鬼脸儿和旗帜乱串。众女孩子都笑了,独龄官冷笑两声,赌气
仍睡着去了。贾蔷还只管陪笑问他:“好不好?”龄官道:“你们家把好好儿的人
弄了来,关在这牢坑里,学这个还不算,你这会子又弄个雀儿来,也干这个浪事!
你分明弄了来打趣形容我们,还问‘好不好’!”贾蔷听了,不觉站起来,连忙赌
神起誓,又道:“今儿我那里的糊涂油蒙了心,费一二两银子买他,原说解闷儿,
就没想到这上头。罢了,放了生,倒也免你的灾。”说着,果然将那雀儿放了,一
顿把那笼子拆了。龄官还说:“那雀儿虽不如人,他也有个老雀儿在窝里,你拿了
他来,弄这个劳什子,也忍得?今儿我咳嗽出两口血来,太太打发人来找你,叫你
请大夫来细问问,你且弄这个来取笑儿。偏是我这没人管没人理的,又偏爱害病!”
贾蔷听说,连忙说道:“昨儿晚上我问了大夫,他说,‘不相干,吃两剂药,后儿
再瞧。’谁知今儿又吐了?这会子就请他去。”说着便要请去。龄官又叫:“站住,
这会子大毒日头地下,你赌气去请了来,我也不瞧。”贾蔷听如此说,只得又站住。

宝玉见了这般景况,不觉痴了。这才领会过画“蔷”深意。自己站不住,便抽
身走了。贾蔷一心都在龄官身上,竟不曾理会,倒是别的女孩子送出来了。那宝玉
一心裁夺盘算,痴痴的回至怡红院中,正值黛玉和袭人坐着说话儿呢。宝玉一进来,
就和袭人长叹,说道:“我昨儿晚上的话,竟说错了,怪不得老爷说我是‘管窥蠡
测’!昨夜说你们的眼泪单葬我,这就错了。看来我竟不能全得。从此后,只好各
人得各人的眼泪罢了。”袭人只道昨夜不过是些玩话,已经忘了,不想宝玉又提起
来,便笑道:“你可真真有些个疯了!”宝玉默默不对。自此深悟人生情缘,各有
分定,只是每每暗伤:“不知将来葬我洒泪者为谁?”

且说黛玉当下见宝玉如此形象,便知是又从那里着了魔来,也不便多问,因说
道:“我才在舅母跟前,听见说明儿是薛姨妈的生日,叫我顺便来问你出去不出去。
你打发人前头说一声去。”宝玉道:“上回连大老爷的生日我也没去,这会子我又
去,倘或碰见了人呢?我一概都不去。这么怪热的,又穿衣裳!我不去,姨妈也未必
恼。”袭人忙道:“这是什么话?他比不得大老爷。这里又住的近,又是亲戚,你
不去,岂不叫他思量?你怕热,就清早起来,到那里磕个头、吃钟茶再来,岂不好
看?”宝玉尚未说话,黛玉便先笑道:“你看着人家赶蚊子的分上,也该去走走。”
宝玉不解,忙问:“怎么赶蚊子?”袭人便将昨日睡觉无人作伴,宝姑娘坐了一坐
的话,告诉宝玉。宝玉听了,忙说:“不该!我怎么睡着了?就亵渎了他!”一面又
说:“明日必去。”

正说着,忽见湘云穿得齐齐整整的走来,辞说家里打发人来接他。宝玉黛玉听
说,忙站起来让坐,湘云也不坐,宝黛两个只得送他至前面。那湘云只是眼泪汪汪
的,见有他家的人在跟前,又不敢十分委屈。少时宝钗赶来,愈觉缱绻难舍。还是
宝钗心内明白,他家里人若回去告诉了他婶娘,待他家去了,又恐怕他受气,因此
倒催着他走了。众人送至二门前,宝玉还要往外送他,倒是湘云拦住了。一时,回
身又叫宝玉到跟前,悄悄的嘱咐道:“就是老太太想不起我来,你时常提着,好等
老太太打发人接我去。”宝玉连连答应了。眼看着他上车去了,大家方才进来。

要知端底,且看下回分解。