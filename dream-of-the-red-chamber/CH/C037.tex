\chapter{秋爽斋偶结海棠社~蘅芜院夜拟菊花题}

话说史湘云回家后,宝玉等仍不过在园中嬉游吟咏不提。

且说贾政自元妃归省之后,居官更加勤慎,以期仰答皇恩。皇上见他人品端方,
风声清肃,虽非科第出身,却是书香世代,因特将他点了学差,也无非是选拔真才
之意。这贾政只得奉了旨,择于八月二十日起身。是日拜别过宗祠及贾母,便起身
而去。宝玉等如何送行,以及贾政出差外面诸事,不及细述。

单表宝玉自贾政起身之后,每日在园中任意纵性游荡,真把光阴虚度,岁月空
添。这日甚觉无聊,便往贾母王夫人处来混了一混,仍旧进园来了。刚换了衣裳,
只见翠墨进来,手里拿着一幅花笺,送与他看。宝玉因道:“可是我忘了,才要瞧
瞧三妹妹去。你来的正好。可好些了?”翠墨道:“姑娘好了,今儿也不吃药了,
不过是冷着一点儿。”宝玉听说,便展开花笺看时,上面写道:

妹探谨启二兄文几:前夕新霁,月色如洗,因惜清景难逢,未忍就卧,漏已三
转,犹徘徊桐槛之下,竟为风露所欺,致获采薪之患。昨亲劳抚嘱已,复遣侍儿问
切,兼以鲜荔并真卿墨迹见赐,抑何惠爱之深耶!今因伏几处默,忽思历来古人,
处名攻利夺之场,犹置些山滴水之区,远招近揖,投辖攀辕,务结二三同志,盘桓
其中,或竖词坛,或开吟社:虽因一时之偶兴,每成千古之佳谈。妹虽不才,幸叨
陪泉石之间,兼慕薛林雅调。风庭月榭,惜未宴集诗人;帘杏溪桃,或可醉飞吟盏。
孰谓雄才莲社,独许须眉;不教雅会东山,让余脂粉耶?若蒙造雪而来,敢请扫花
以俟。谨启。
宝玉看了,不觉喜的拍手笑道:“倒是三妹妹高雅,我如今就去商议。”一面说,
一面就走。翠墨跟在后面。

刚到了沁芳亭,只见园中后门上值日的婆子手里拿着一个字帖儿走来,见了宝
玉,便迎上去,口内说道:“芸哥儿请安,在后门等着呢。这是叫我送来的。”宝
玉打开看时,写道:

不肖男芸恭请父亲大人万福金安:男思自蒙天恩,认于膝下,日夜思一孝顺,
竟无可孝顺之处。前因买办花草,上托大人洪福,竟认得许多花儿匠,并认得许多
名园。前因忽见有白海棠一种,不可多得,故变尽方法,只弄得两盆。大人若视男
是亲男一般,便留下赏玩。因天气暑热,恐园中姑娘们妨碍不便,故不敢面见。谨
奉书恭启,并叩台安。男芸跪书。
宝玉看了,笑问道:“他独来了,还有什么人?”婆子道:“还有两盆花儿。”宝
玉道:“你出去说:我知道了,难为他想着。你就把花儿送到我屋里去就是了。”

一面说,一面同翠墨往秋爽斋来,只见宝钗、黛玉、迎春、惜春已都在那里了。
众人见他进来,都大笑说:“又来了一个。”探春笑道:“我不算俗,偶然起了个
念头,写了几个帖儿试一试,谁知一招皆到。”宝玉笑道:“可惜迟了!早该起个
社的。”黛玉说道:“此时还不算迟,也没什么可惜;但只你们只管起社,可别算
我,我是不敢的。”迎春笑道:“你不敢,谁还敢呢?”宝玉道:“这是一件正经
大事,大家鼓舞起来,别你谦我让的。各有主意只管说出来,大家评论。宝姐姐也
出个主意,林妹妹也说句话儿。”宝钗道:“你忙什么!人还不全呢。”一语未了,
李纨也来了,进门笑道:“雅的很哪!要起诗社,我自举我掌坛。前儿春天,我原
有这个意思的,我想了一想,我又不会做诗,瞎闹什么,因而也忘了,就没有说。
即是三妹妹高兴,我就帮着你作兴起来。”

黛玉道:“既然定要起诗社,咱们就是诗翁了,先把这些‘姐妹叔嫂’的字样
改了才不俗。”李纨道:“极是。何不起个别号,彼此称呼倒雅?我是定了‘稻香
老农’,再无人占的。”探春笑道:“我就是‘秋爽居士’罢。”宝玉道:“‘居
士’‘主人’,到底不雅,又累赘。这里梧桐芭蕉尽有,或指桐蕉起个倒好。”探
春笑道:“有了,我却爱这芭蕉,就称‘蕉下客’罢。”众人都道别致有趣。黛玉
笑道:“你们快牵了他来,炖了肉脯子来吃酒。”众人不解,黛玉笑道:“庄子说
的‘蕉叶覆鹿’,他自称‘蕉下客’,可不是一只鹿么?快做了鹿脯来。”众人听
了都笑起来。探春因笑道:“你又使巧话来骂人!你别忙,我已替你想了个极当的
美号了。”又向众人道:“当日娥皇女英洒泪竹上成斑,故今斑竹又名湘妃竹。如
今他住的是潇湘馆,他又爱哭,将来他那竹子想来也是要变成斑竹的,以后都叫他
做‘潇湘妃子’就完了。”大家听说都拍手叫妙,黛玉低了头也不言语。李纨笑道:
“我替薛大妹妹也早已想了个好的,也只三个字。”众人忙问是什么,李纨道:“我
是封他为‘蘅芜君’,不知你们以为如何?”探春道:“这个封号极好。”

宝玉道:“我呢?你们也替我想一个。”宝钗笑道:“你的号早有了:‘无事
忙。’三字恰当得很!”李纨道:“你还是你的旧号‘绛洞花主’就是了。”宝玉
笑道:“小时候干的营生,还提他做什么。”宝钗道:“还是我送你个号罢,有最
俗的一个号,却于你最当:天下难得的是富贵,又难得的是闲散,这两样再不能兼,
不想你兼有了,就叫你‘富贵闲人’也罢了。”宝玉笑道:“当不起,当不起!倒
是随你们混叫去罢。”黛玉道:“混叫如何使得!你既住怡红院,索性叫‘怡红公
子’不好?”众人道:“也好。”李纨道:“二姑娘、四姑娘起个什么?”迎春道:
“我们又不大会诗,白起个号做什么!”探春道:“虽如此,也起个才是。”宝钗
道:“他住的是紫菱洲,就叫他‘菱洲’;四丫头住藕香榭,就叫他‘藕榭’就完
了。”

李纨道:“就是这样好。但序齿我大,你们都要依我的主意,管教说了大家合
意。我们七个人起社,我和二姑娘四姑娘都不会做诗,须得让出我们三个人去。我
们三个人各分一件事。”探春笑道:“已有了号,还只管这样称呼,不如不有了。
以后错了,也要立个罚约才好。”李纨道:“立定了社,再定罚约。我那里地方儿
大,竟在我那里作社,我虽不能做诗,这些诗人竟不厌俗,容我做个东道主人,我
自然也清雅起来了;还要推我做社长。我一个社长自然不够,必要再请两位副社长,
就请菱洲藕榭二位学究来,一位出题限韵,一位誊录监场。亦不可拘定了我们三个
不做,若遇见容易些的题目韵脚,我们也随便做一首,你们四个却是要限定的。是
这么着就起,若不依我,我也不敢附骥了。”迎春惜春本性懒于诗词,又有薛林在
前,听了这话,深合己意,二人皆说:“是极。”探春等也知此意,见他二人悦服,
也不好相强,只得依了。因笑道:“这话罢了。只是自想好笑,好好儿的我起了个
主意,反叫你们三个管起我来了。”

宝玉道:“既这样,咱们就往稻香村去。”李纨道:“都是你忙。今日不过商
议了,等我再请。”宝钗道:“也要议定几日一会才好。”探春道:“若只管会多
了,又没趣儿了。一月之中,只可两三次。”宝钗说道:“一月只要两次就够了。
拟定日期,风雨无阻。除这两日外,倘有高兴的,他情愿加一社,或请到他那里去,
或附就了来,也使得。岂不活泼有趣?”众人都道:“这个主意更好。”探春道:
“这原是我起的意,我须得先做个东道,方不负我这番高兴。”李纨道:“既这样
说,明日你就先开一社不好吗?”探春道:“明日不如今日,就是此刻好。你就出
题,菱洲限韵,藕榭监场。”迎春道:“依我说,也不必随一人出题限韵,竟是拈
阄儿公道。”李纨道:“方才我来时,看见他们抬进两盆白海棠来,倒很好,你们
何不就咏起他来呢?”迎春道:“都还未赏,先倒做诗?”宝钗道:“不过是白海
棠,又何必定要见了才做。古人的诗赋也不过都是寄兴寓情,要等见了做,如今也
没这些诗了。”迎春道:“这么着,我就限韵了。”说着,走到书架前,抽出一本
诗来随手一揭。这首诗竟是一首七言律,递与众人看了,都该做七言律。迎春掩了
诗,又向一个小丫头道:“你随口说个字来。”那丫头正倚门站着,便说了个“门”
字,迎春笑道:“就是‘门’字韵,‘十三元’了。起头一个韵定要‘门’字。”
说着又要了韵牌匣子过来,抽出“十三元”一屉,又命那丫头随手拿四块。那丫头
便拿了“盆”“魂”“痕”“昏”四块来。宝玉道:“这‘盆’‘门’两个字不大
好做呢!”

侍书一样预备下四分纸笔,便都悄然各自思索起来。独黛玉或抚弄梧桐,或看
秋色,或又和丫鬟们嘲笑。迎春又命丫鬟点了一枝梦甜香。原来这梦甜香只有三寸
来长,有灯草粗细,以其易烬,故以此为限,如香烬未成便要受罚。一时探春便先
有了,自己提笔写出,又改抹了一回,递与迎春。因问宝钗:“蘅芜君,你可有了?”
宝钗道:“有却有了,只是不好。”宝玉背着手在回廊上踱来踱去,因向黛玉说道:
“你听他们都有了。”黛玉道:“你别管我。”宝玉又见宝钗已誊写出来,因说道:
“了不得,香只剩下一寸了!我才有了四句。”又向黛玉道:“香要完了,只管蹲
在那潮地下做什么?”黛玉也不理。宝玉道:“我可顾不得你了,管他好歹,写出
来罢。”说着,走到案前写了。

李纨道:“我们要看诗了。若看完了还不交卷,是必罚的。”宝玉道:“稻香
老农虽不善作,却善看,又最公道,你的评阅,我们是都服的。”众人点头。于是
先看探春的稿上写道:

咏白海棠
斜阳寒草带重门,苔翠盈铺雨后盆。
玉是精神难比洁,雪为肌骨易销魂。
芳心一点娇无力,倩影三更月有痕。
莫道缟仙能羽化,多情伴我咏黄昏。
大家看了,称赏一回,又看宝钗的道:
珍重芳姿昼掩门,自携手瓮灌苔盆。
胭脂洗出秋阶影,冰雪招来露砌魂。
淡极始知花更艳,愁多焉得玉无痕?
欲偿白帝宜清洁,不语婷婷日又昏。
李纨笑道:“到底是蘅芜君!”说着,又看宝玉的道:
秋容浅淡映重门,七节攒成雪满盆。
出浴太真冰作影,捧心西子玉为魂。
晓风不散愁千点,宿雨还添泪一痕。
独倚画栏如有意,清砧怨笛送黄昏。

大家看了,宝玉说探春的好。李纨终要推宝钗:“这诗有身分。”因又催黛玉。
黛玉道:“你们都有了?”说着,提笔一挥而就,掷与众人。李纨等看他写的道:
半卷湘帘半掩门,碾冰为土玉为盆。
看了这句,宝玉先喝起彩来,说:“从何处想来!”又看下面道:
偷来梨蕊三分白,借得梅花一缕魂。
众人看了,也都不禁叫好,说:“果然比别人又是一样心肠。”又看下面道:
月窟仙人缝缟袂,秋闺怨女拭啼痕。
娇羞默默同谁诉?倦倚西风夜已昏。

众人看了,都道:“是这首为上。”李纨道:“若论风流别致,自是这首;若
论含蓄浑厚,终让蘅稿。”探春道:“这评的有理。潇湘妃子当居第二。”李纨道:
“怡红公子是压尾,你服不服?”宝玉道:“我的那首原不好,这评的最公。”又
笑道:“只是蘅潇二首,还要斟酌。”李纨道:“原是依我评论,不与你们相干,
再有多说者必罚。”宝玉听说,只得罢了。李纨道:“从此后,我定于每月初二、
十六这两日开社,出题限韵都要依我。这其间你们有高兴的,只管另择日子补开,
那怕一个月每天都开社我也不管。只是到了初二、十六这两日,是必往我那里去。”
宝玉道:“到底要起个社名才是。”探春道:“俗了又不好,忒新了刁钻古怪也不
好。可巧才是海棠诗开端,就叫个‘海棠诗社’罢,虽然俗些,因真有此事,也就
不碍了。”说毕,大家又商议了一回。略用些酒果,方各自散去,也有回家的,也
有往贾母王夫人处去的。当下无话。

且说袭人因见宝玉看了字帖儿,便慌慌张张同翠墨去了,也不知何事;后来又
见后门上婆子送了两盆海棠花来。袭人问那里来的,婆子们便将前番原故说了。袭
人听说,便命他们摆好,让他们在下房里坐了。自己走到屋里,称了六钱银子封好,
又拿了三百钱走来,都递给那两个婆子道:“这银子赏那抬花儿的小子们。这钱你
们打酒喝罢。”那婆子们站起来,眉开眼笑,千恩万谢的不肯受,见袭人执意不收,
方领了。袭人又道:“后门上外头可有该班的小子们?”婆子忙应道:“天天有四
个,原预备里头差使的。姑娘有什么差使?我们吩咐去。”袭人笑道:“我有什么
差使。今儿宝二爷要打发人到小侯爷家给史大姑娘送东西去,可巧你们来了,顺便
出去叫后门上小子们雇辆车来,回来你们就往这里拿钱,不用叫他们往前头混碰
去。”婆子答应着去了。

袭人回至房中,拿碟子盛东西与湘云送去。却见子上碟子槽儿空着,因回头
见晴雯、秋纹、麝月等都在一处做针黹,袭人问道:“那个缠丝白玛瑙碟子那里去
了?”众人见问,你看我,我看你,都想不起来。半日晴雯笑道:“给三姑娘送荔
枝去了,还没送来呢。”袭人道:“家常送东西的家伙多着呢,巴巴儿的拿这个。”
晴雯道:“我也这么说,但只那碟子配上鲜荔枝才好看。我送去,三姑娘也见了,
说好看,连碟子放着,就没带来。你再瞧那子尽上头的一对联珠瓶还没收来呢。”
秋纹笑道:“提起这瓶来,我又想起笑话儿来了。我们宝二爷说声孝心一动,也孝
敬到二十分:那日见园里桂花,折了两枝,原是自己要插瓶的,忽然想起来,说:
‘这是自己园里才开的新鲜花儿,不敢自己先玩。’巴巴儿的把那对瓶拿下来,亲
自灌水插好了,叫个人拿着,亲自送一瓶进老太太,又进一瓶给太太。谁知他孝心
一动,连跟的人都得了福了。可巧那日是我拿去的,老太太见了喜的无可不可,见
人就说:‘到底是宝玉孝顺我,连一枝花儿也想的到。别人还只抱怨我疼他!’你
们知道老太太素日不大和我说话,有些不入他老人家的眼;那日竟叫人拿几百钱给
我,说我‘可怜见儿的,生的单弱’。这可是再想不到的福气。几百钱是小事,难
得这个脸面。及至到了太太那里,太太正和二奶奶赵姨奶奶好些人翻箱子,找太太
当日年轻的颜色衣裳,不知要给那一个;一见了,连衣裳也不找了,且看花儿。又
有二奶奶在傍边凑趣儿,夸宝二爷又是怎么孝顺,又是怎么知好歹,有的没的说了
两车话。当着众人,太太脸上又增了光,堵了众人的嘴,太太越发喜欢了,现成的
衣裳,就赏了我两件。衣裳也是小事,年年横竖也得,却不像这个彩头。”

晴雯笑道:“呸!好没见世面的小蹄子!那是把好的给了人,挑剩下的才给你,
你还充有脸呢!”秋纹道:“凭他给谁剩的,到底是太太的恩典。”晴雯道:“要
是我,我就不要。若是给别人剩的给我也罢了,一样这屋里的人,难道谁又比谁高
贵些?把好的给他,剩的才给我,我宁可不要,冲撞了太太,我也不受这口气!”
秋纹忙问道:“给这屋里谁的?我因为前日病了几天,家去了,不知是给谁的,好
姐姐,你告诉我知道。”晴雯道:“我告诉了你,难道你这会子退还太太去不成?”
秋纹笑道:“胡说!我白听了喜欢喜欢,那怕给这屋里的狗剩下的,我只领太太的
恩典,也不管别的事。”众人听了都笑道:“骂的巧,可不是给了那西洋花点子哈
巴儿了!”袭人笑道:“你们这起烂了嘴的!得空儿就拿我取笑打牙儿,一个个不
知怎么死呢!”秋纹笑道:“原来姐姐得了!我实在不知道,我陪个不是罢。”袭
人笑道:“少轻狂罢!你们谁取了碟子来是正经。”麝月道:“那瓶也该得空儿收
来了。老太太屋里还罢了,太太屋里人多手杂,别人还可已,那个主儿的一伙子人
见是这屋里的东西,又该使黑心弄坏了才罢。太太又不大管这些,不如早收来是正
经。”晴雯听说,便放下针线道:“这是等我取去呢。”秋纹道:“还是我取去罢,
你取你的碟子去。”晴雯道:“我偏取一遭儿。是巧宗儿,你们都得了,难道不许
我得一遭儿吗?”麝月笑道:“统共秋丫头得了一遭儿衣裳,那里今儿又巧,你也
遇见找衣裳不成?”晴雯冷笑道:“虽然碰不见衣裳,或者太太看见我勤谨,也把
太太的公费里一个月分出二两银子来给我,也定不得。”说着,又笑道:“你们别
和我装神弄鬼的,什么事我不知道!”一面说,一面往外跑了。秋纹也同他出来,
自去探春那里取了碟子来。

袭人打点齐备东西,叫过本处的一个老宋妈妈来,向他说道:“你去好生梳洗
了,换了出门的衣裳来,回来打发你给史大姑娘送东西去。”宋妈妈道:“姑娘只
管交给我,有话说与我,我收拾了就好一顺去。”袭人听说,便端过两个小摄丝盒
子来。先揭开一个,里面装的是红菱、鸡头两样鲜果;又揭开那个,是一碟子桂花
糖蒸的新栗粉糕。又说道:“这都是今年咱们这里园里新结的果子,宝二爷送来给
姑娘尝尝。再前日姑娘说这玛瑙碟子好,姑娘就留下玩罢。这绢包儿里头是姑娘前
日叫我做的活计,姑娘别嫌粗糙,将就着用罢。替二爷问好,替我们请安,就是了。”
宋妈妈道:“宝二爷不知还有什么说的?姑娘再问问去,回来别又说忘了。”袭人
因问秋纹:“方才可是在三姑娘那里么?”秋纹道:“他们都在那里商议起什么诗
社呢,又是做诗。想来没话,你只管去罢。”宋妈妈听了,便拿了东西出去,穿戴
了,袭人又嘱咐他:“你打后门去,有小子和车等着呢。”宋妈妈去了,不在话下。

一时宝玉回来,先忙着看了一回海棠,至屋里告诉袭人起诗社的事,袭人也把
打发宋妈妈给史湘云送东西去的话告诉了宝玉。宝玉听了,拍手道:“偏忘了他!
我只觉心里有件事,只是想不起来,亏你提起来,正要请他去。这诗社里要少了他,
还有个什么意思!”袭人劝道:“什么要紧,不过玩意儿。他比不得你们自在,家
里又作不得主儿。告诉他,他要来又由不得他,要不来他又牵肠挂肚的,没的叫他
不受用。”宝玉道:“不妨事,我回老太太,打发人接他去。”正说着,宋妈妈已
经回来道生受,给袭人道乏,又说:“问二爷做什么呢,我说:‘和姑娘们起什么
诗社做诗呢。’史姑娘道,他们做诗,也不告诉他去。急的了不得!”宝玉听了,
转身便往贾母处来,立逼着叫人接去。贾母因说:“今儿天晚了,明日一早去。”
宝玉只得罢了。回来闷闷的,次日一早,便又往贾母处来催逼人接去。

直到午后,湘云才来了,宝玉方放了心。见面时,就把始末原由告诉他,又要
与他诗看。李纨等因说道:“且别给他看,先说给他韵脚;他后来的,先罚他和了
诗。要好,就请入社;要不好,还要罚他一个东道儿再说。”湘云笑道:“你们忘
了请我,我还要罚你们呢。就拿韵来,我虽不能,只得勉强出丑。容我入社,扫地
焚香,我也情愿。”众人见他这般有趣,越发喜欢,都埋怨:“昨日怎么忘了他呢!”
遂忙告诉他诗韵。

湘云一心兴头,等不得推敲删改,一面只管和人说着话,心内早已和成,即用
随便的纸笔录出,先笑说道:“我却依韵和了两首,好歹我都不知,不过应命而已。”
说着,递与众人。众人道:“我们四首也算想绝了,再一首也不能了,你倒弄了两
首!那里有许多话说?必要重了我们的。”一面说,一面看时,只见那两首诗写道:

白海棠和韵
神仙昨日降都门,种得蓝田玉一盆。
自是霜娥偏爱冷,非关倩女欲离魂。
秋阴捧出何方雪?雨渍添来隔宿痕。
却喜诗人吟不倦,肯令寂寞度朝昏?

其
二
蘅芷阶通萝薜门,也宜墙角也宜盆。
花因喜洁难寻偶,人为悲秋易断魂。
玉烛滴干风里泪,晶帘隔破月中痕。
幽情欲向嫦娥诉,无那虚廊月色昏。
众人看一句惊讶一句,看到了赞到了都说:“这个不枉做了海棠诗!真该要起‘海
棠社’了。”湘云道:“明日先罚我个东道儿,就让我先邀一社,可使得?”众人
道:“这更妙了。”因又将昨日的诗与他评论了一回。

至晚,宝钗将湘云邀往蘅芜院去安歇。湘云灯下计议如何设东拟题。宝钗听他
说了半日,皆不妥当,因向他说道:“既开社,就要作东。虽然是个玩意儿,也要
瞻前顾后;又要自己便宜,又要不得罪了人,然后方大家有趣。你家里你又做不得
主,一个月统共那几吊钱,你还不够使。这会子又干这没要紧的事,你婶娘听见了
越发抱怨你了。况且你就都拿出来,做这个东也不够,难道为这个家去要不成?还
是和这里要呢?”一席话提醒了湘云,倒踌蹰起来。宝钗道:“这个我已经有个主
意了。我们当铺里有个伙计,他们地里出的好螃蟹,前儿送了几个来。现在这里的
人,从老太太起,连上屋里的人,有多一半都是爱吃螃蟹的,前日姨娘还说要请老
太太在园里赏桂花、吃螃蟹,因为有事,还没有请。你如今且把诗社别提起,只普
同一请,等他们散了,咱们有多少诗做不得的?我和我哥哥说,要他几篓极肥极大
的螃蟹来,再往铺子里取上几坛好酒来,再备四五桌果碟子,岂不又省事,又大家
热闹呢?”湘云听了,心中自是感服,极赞想的周到。宝钗又笑道:“我是一片真
心为你的话,你可别多心,想着我小看了你,咱们两个就白好了。你要不多心,我
就好叫他们办去。”湘云忙笑道:“好姐姐!你这么说,倒不是真心待我了。我凭
怎么胡涂,连个好歹也不知,还是个人吗!我要不把姐姐当亲姐姐待,上回那些家
常烦难事,我也不肯尽情告诉你了。”宝钗听说,便唤一个婆子来:“出去和大爷
说,照前日的大螃蟹要几篓来,明日饭后请老太太、姨娘赏桂花。你说与大爷:好
歹别忘了,我今儿已经请下人了。”那婆子出去说明,回来无话。

这里宝钗又向湘云道:“诗题也别过于新巧了,你看古人中那里有那些刁钻古
怪的题目和那极险的韵呢?若题目过于新巧,韵过于险,再不得好诗,倒小家子气。
诗固然怕说熟话,然也不可过于求生;头一件,只要主意清新,措词就不俗了。究
竟这也算不得什么,还是纺绩针黹是你我的本等。一时闲了,倒是把那于身心有益
的书看几章,却还是正经。”湘云只答应着,因笑道:“我心里想着,昨日做了海
棠诗,我如今要做个菊花诗如何?”宝钗道:“菊花倒也合景,只是前人太多了。”
湘云道:“我也是这么想着,恐怕落套。”宝钗想了一想,说道:“有了。如今以
菊花为宾,以人为主,竟拟出几个题目来,都要两个字,一个虚字一个实字。实字
就用‘菊’字,虚字便用通用门的。如此,又是咏菊,又是赋事,前人虽有这么做
的,还不很落套。赋景咏物两关着,也倒新鲜大方。”湘云笑道:“很好,只是不
知用什么虚字才好?你先想一个我听听。”

宝钗想了一想,笑道:“‘菊梦’就好。”湘云笑道:“果然好。我也有一个:
‘菊影’可使得?”宝钗道:“也罢了,只是也有人做过。若题目多,这个也搭的
上。我又有了一个。”湘云道:“快说出来。”宝钗道:“‘问菊’如何?”湘云
拍案叫妙,因接说道:“我也有了:‘访菊’好不好?”宝钗也赞有趣。因说道:
“索性拟出十个来,写上再来。”说着,二人研墨蘸笔,湘云便写,宝钗便念,一
时凑了十个。湘云看了一遍,又笑道:“十个还不成幅,索性凑成十二个,就全了,
也和人家的字画册页一样。”宝钗听说,又想了两个,一共凑成十二个,说道:“既
这么着,一发编出个次序来。”湘云道:“更妙,竟弄成个菊谱了。”

宝钗道:“起首是《忆菊》;忆之不得,故访,第二是《访菊》。访之既得,
便种,第三是《种菊》。种既盛开,故相对而赏,第四是《对菊》。相对而兴有馀,
故折来供瓶为玩,第五是《供菊》。既供而不吟,亦觉菊无彩色,第六便是《咏菊》。
既入词章,不可以不供笔墨,第七便是《画菊》。既然画菊,若是默默无言,究竟
不知菊有何妙处,不禁有所问,第八便是《问菊》。菊若能解语,使人狂喜不禁,
便越要亲近他,第九竟是《簪菊》。如此人事虽尽,犹有菊之可咏者,《菊影》《菊
梦》二首,续在第十、第十一。末卷便以《残菊》总收前题之感。这便是三秋的妙
景妙事都有了。”湘云依言将题录出,又看了一回,又问:“该限何韵?”宝钗道:
“我平生最不喜限韵,分明有好诗,何苦为韵所缚?咱们别学那小家派。只出题,
不拘韵:原为大家偶得了好句取乐,并不为以此难人。”湘云道:“这话很是。既
这样,自然大家的诗还进一层。但只咱们五个人,这十二个题目,难道每人作十二
首不成?”宝钗道:“那也太难人了。将这题目誊好,都要七言律诗,明日贴在墙
上,他们看了,谁能那一个就做那一个。有力量者十二首都做也可,不能的作一首
也可,高才捷足者为尊。若十二首已全,便不许他赶着又做,罚他便完了。”湘云
道:“这也罢了。”二人商议妥贴,方才息灯安寝。

要知端底,下回分解。