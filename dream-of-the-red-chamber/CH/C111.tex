\chapter{鸳鸯女殉主登太虚~狗彘奴欺天招伙盗}

话说凤姐听了小丫头的话,又气又急又伤心,不觉吐了一口血,便昏晕过去,
坐在地下。平儿急来扶住,忙叫了人来搀扶着,慢慢的送到自己房中,将凤姐轻轻
的安放在炕上,立刻叫小红斟上一杯开水送到凤姐唇边。凤姐呷了一口,昏迷仍睡。
秋桐过来略瞧了一瞧,便走开了,平儿也不叫他。只见丰儿在旁站着,平儿便说:
“快去回明二位太太。”于是丰儿将凤姐吐血不能照应的话回了邢王二夫人。邢夫
人打量凤姐推病藏躲,因这时女亲都在内里,也不好说别的,心里却不全信,只说:
“叫他歇着去罢。”众人也并无言语。自然这晚亲友来往不绝,幸得几个内亲照应。
家下人等见凤姐不在,也有偷闲歇力的,乱乱吵吵,已闹得七颠八倒,不成事体了。

到二更多天,远客去后,便预备辞灵,孝幕内的女眷,大家都哭了一阵。只见
鸳鸯已哭的昏晕过去了,大家扶住,捶闹了一阵,才醒过来,便说“老太太疼了一
场,要跟了去”的话。众人都打量人到悲哭,俱有这些言语,也不理会。及至辞灵
的时候,上上下下也有百十馀人,只不见鸳鸯,众人因为忙乱,却也不曾检点。到
琥珀等一干人哭奠之时,才要找鸳鸯,又恐是他哭乏了,暂在别处歇着,也不言语。

辞灵以后,外头贾政叫了贾琏问明送殡的事,便商量着派人看家。贾琏回说:
“上人里头,派了芸儿在家照应,不必送殡;下人里头,派了林之孝的一家子照应
拆棚等事。但不知里头派谁看家?”贾政道:“听见你母亲说是你媳妇病了,不能
去,就叫他在家的。你珍大嫂子又说你媳妇病得利害,还叫四丫头陪着,带领了几
个丫头婆子,照看上屋里才好。”贾琏听了,心想:“珍大嫂子与四丫头两个不合,
所以撺掇着不叫他去。若是上头就是他照应,也是不中用的,我们那一个又病着,
也难照应。”想了一回,回贾政道:“老爷且歇歇儿,等进去商量定了再回。”贾
政点了点头,贾琏便进去了。

谁知此时鸳鸯哭了一场,想到:“自己跟着老太太一辈子,身子也没有着落。
如今大老爷虽不在家,大太太的这样行为,我也瞧不上。老爷是不管事的人,以后
便‘乱世为王’起来了,我们这些人不是要叫他们掇弄了么?谁收在屋子里,谁配
小子,我是受不得这样折磨的,倒不如死了干净。但是一时怎么样的个死法呢?”
一面想,一面走到老太太的套间屋内。刚跨进门,只见灯光惨淡,隐隐有个女人拿
着汗巾子,好似要上吊的样子。鸳鸯也不惊怕,心里想道:“这一个是谁?和我的
心事一样,倒比我走在头里了。”便问道:“你是谁?咱们两个人是一样的心,要
死一块儿死。”那个人也不答言。鸳鸯走到跟前一看,并不是这屋子的丫头。仔细
一看,觉得冷气侵人,一时就不见了。鸳鸯呆了一呆,退出在炕沿上坐下,细细一
想,道:“哦!是了,这是东府里的小蓉大奶奶啊!他早死了的了,怎么到这里来?
必是来叫我来了。他怎么又上吊呢?”想了一想,道:“是了,必是教给我死的法
儿。”鸳鸯这么一想,邪侵入骨,便站起来,一面哭,一面开了妆匣,取出那年铰
的一绺头发揣在怀里,就在身上解下一条汗巾,按着秦氏方才比的地方拴上。自己
又哭了一回,听见外头人客散去,恐有人进来,急忙关上屋门。然后端了一个脚凳,
自己站上,把汗巾拴上扣儿,套在咽喉,便把脚凳蹬开。可怜咽喉气绝,香魂出窍!
正无投奔,只见秦氏隐隐在前,鸳鸯的魂魄疾忙赶上,说道:“蓉大奶奶,你等等
我。”那个人道:“我并不是什么蓉大奶奶,乃警幻之妹可卿是也。”鸳鸯道:“你
明明是蓉大奶奶,怎么说不是呢?”那人道:“这也有个缘故,待我告诉你,你自
然明白了:我在警幻宫中,原是个钟情的首坐,管的是风情月债;降临尘世,自当
为第一情人,引这些痴情怨女,早早归入情司,所以我该悬梁自尽的。因我看破凡
情,超出情海,归入情天,所以太虚幻境‘痴情’一司,竟自无人掌管。今警幻仙
子已经将你补入,替我掌管此司,所以命我来引你前去的。”鸳鸯的魂道:“我是
个最无情的,怎么算我是个有情的人呢?”那人道:“你还不知道呢。世人都把那
淫欲之事当作‘情’字,所以作出伤风败化的事来,还自谓风月多情,无关紧要。
不知情之一字,喜怒哀乐未发之时,便是个‘性’;喜怒哀乐已发,便是‘情’了。
至于你我这个情,正是未发之情,就如那花的含苞一样。若待发泄出来,这情就不
为真情了。”鸳鸯的魂听了,点头会意,便跟了秦氏可卿而去。

这里琥珀辞了灵,听邢王二夫人分派看家的人,想着去问鸳鸯明日怎样坐车,
便在贾母的那间屋里找了一遍。不见,又找到套间里头。刚到门口,见门儿掩着;
从门缝里望里看时,只见灯光半明半灭的,影影绰绰。心里害怕,又不听见屋里有
什么动静,便走回来说道:“这蹄子跑到那里去了?”劈头见了珍珠,说:“你见
鸳鸯姐姐来着没有?”珍珠道:“我也找他,太太们等他说话呢。必在套间里睡着
了罢?”琥珀道:“我瞧了,屋里没有。那灯也没人夹蜡花儿,漆黑怪怕的,我没
进去。如今咱们一块儿进去,瞧看有没有。”琥珀等进去,正夹蜡花,珍珠说:“谁
把脚凳撂在这里,几乎绊我一跤!”说着,往上一瞧,唬的“嗳哟”一声,身子往
后一仰,“咕咚”的栽在琥珀身上。琥珀也看见了,便大嚷起来,只是两只脚挪不
动。外头的人也都听见了,跑进来一瞧,大家嚷着,报与邢王二夫人知道。

王夫人宝钗等听了,都哭着去瞧。邢夫人道:“我不料鸳鸯倒有这样志气!快
叫人去告诉老爷。”只有宝玉听见此信,便唬的双眼直竖。袭人等慌忙扶着说道:
“你要哭就哭,别着气。”宝玉死命的才哭出来了。心想:“鸳鸯这样一个人,
偏又这样死法!”又想:“实在天地间的灵气,独钟在这些女子身上了。他算得了
死所。我们究竟是一件浊物,还是老太太的儿孙,谁能赶得上他?”复又喜欢起来。
那时,宝钗听见宝玉大哭了出来了,及到跟前,见他又笑。袭人等忙说:“不好了,
又要疯了。”宝钗道:“不妨事,他有他的意思。”宝玉听了,更喜欢宝钗的话,
“到底他还知道我的心,别人那里知道。”正在胡思乱想,贾政等进来,着实的嗟
叹着说道:“好孩子,不枉老太太疼他一场!”即命贾琏:“出去吩咐人连夜买棺
盛殓,明日便跟着老太太的殡送出,也停在老太太棺后,全了他的心志。”贾琏答
应出去,这里命人将鸳鸯放下,停放里间屋内。

平儿也知道了,过来同袭人莺儿等一干人都哭的哀哀欲绝。内中紫鹃也想起自
己终身,一无着落,恨不跟了林姑娘去,又全了主仆的恩义,又得了死所。如今空
悬在宝玉屋内,虽说宝玉仍是柔情密意,究竟算不得什么,于是更哭得哀切。

王夫人即传了鸳鸯的嫂子进来,叫他看着入殓,遂与邢夫人商量了,在老太太
项内赏了他嫂子一百两银子,还说等闲了将鸳鸯所有的东西俱赏他们。他嫂子磕了
头出去,反喜欢说:“真真的我们姑娘是个有志气的有造化的!又得了好名声,又
得了好发送。”傍边一个婆子说道:“罢呀嫂子,这会子你把一个活姑娘卖了一百
银便这么喜欢了,那时候儿给了大老爷,你还不知得多少银钱呢,你该更得意了。”
一句话戳了他嫂子的心,便红了脸走开了。刚走到二门上,见林之孝带了人抬进棺
材来了,他只得也跟进去,帮着盛殓,假意哭嚎了几声。

贾政因他为贾母而死,要了香来,上了三炷,作了个揖,说:“他是殉葬的人,
不可作丫头论,你们小一辈的都该行个礼儿。”宝玉听了,喜不自胜,走来恭恭敬
敬磕了几个头。贾琏想他素日的好处,也要上来行礼,被邢夫人说道:“有了一个
爷们就是了,别折受的他不得超生。”贾琏就不便过来了。宝钗听着这话,好不自
在,便说道:“我原不该给他行礼,但只老太太去世,咱们都有未了之事,不敢胡
为。他肯替咱们尽孝,咱们也该托托他,好好的替咱们伏侍老太太西去,也少尽一
点子心哪。”说着,扶了莺儿走到灵前,一面奠酒,那眼泪早扑簌簌流下来了。奠
毕,拜了几拜,狠狠的哭了他一场。众人也有说宝玉的两口子都是傻子,也有说他
两个心肠儿好的,也有说他知礼的,贾政反倒合了意。一面商量定了看家的,仍是
凤姐惜春,馀者都遣去伴灵。一夜谁敢安眠。一到五更,听见外面齐人。到了辰初
发引,贾政居长,衰麻哭泣,极尽孝子之礼。灵柩出了门,便有各家的路祭,一路
上的风光,不必细述。走了半日,来至铁槛寺安灵,所有孝男等俱应在庙伴宿,不
提。

且说家中林之孝带领拆了棚,将门窗上好,打扫净了院子,派了巡更的人,到
晚打更上夜。只是荣府规例:一交二更,三门掩上,男人就进不去了,里头只有女
人们查夜。凤姐虽隔了一夜,渐渐的神气清爽了些,只是那里动得。只有平儿同着
惜春各处走了一走,吩咐了上夜的人,也便各自归房。

却说周瑞的干儿子何三,去年贾珍管事之时,因他和鲍二打架,被贾珍打了一
顿,撵在外头,终日在赌场过日。近知贾母死了,必有些事情领办,岂知探了几天
的信,一些也没有想头,便嗳声叹气的回到赌场中,闷闷的坐下。那些人便说道:
“老三,你怎么不下来捞本儿了吗?”何三道:“倒想要捞一捞呢,就只没有钱么。”
那些人道:“你到你们周大太爷那里去了几日,府里的钱,你也不知弄了多少来,
又来和我们装穷儿了。”何三道:“你们还说呢。他们的金银不知有几百万,只藏
着不用。明儿留着,不是火烧了,就是贼偷了,他们才死心呢。”那些人道:“你
又撒谎。他家抄了家,还有多少金银?”何三道:“你们还不知道呢。抄的是撂不
了的。如今老太太死后,还留了好些金银,他们一个也不使,都在老太太屋里搁着,
等送了殡回来才分呢。”内中有一个人听在心里,掷了几骰,便说:“我输了几个
钱也不翻本儿了,睡去了。”说着,便走出来,拉了何三道:“老三,我和你说句
话。”何三跟他出来。那人道:“你这么个伶俐人,这么穷,我替你不服这口气。”
何三道:“我命里穷,可有什么法儿呢?”那人道:“你才说荣府的银子这么多,
为什么不去拿些使唤使唤?”何三道:“我的哥哥!他家的金银虽多,你我去白要
一二钱,他们给咱们吗?”那人笑道:“他不给咱们,咱们就不会拿吗?”

何三听了这话里有话,忙问道:“依你说,怎么样拿呢?”那人道:“我说你
没有本事,若是我,早拿了来了。”何三道:“你有什么本事?”那人便轻轻的说
道:“你若要发财,你就引个头儿。我有好些朋友,都是通天的本事。别说他们送
殡去了,家里只剩下几个女人,就让有多少男人也不怕。只怕你没这么大胆子罢
咧。”何三道:“什么敢不敢,你打量我怕那个干老子吗!我是瞧着干妈的情儿上
头,才认他做干老子罢咧,他又算了人了?你刚才的话,就只怕弄不来,倒招了饥
荒。他们那个衙门不熟?别说拿不来,倘或拿了来,也要闹出来的。”那人道:“这
么说,你的运气来了。我的朋友还有海边上的呢,现今都在这里。看个风头,等个
门路,若到了手,你我在这里也无益,不如大家下海去受用,不好么?你若撂不下
你干妈,咱们索性把你干妈也带了去,大家伙儿乐一乐,好不好?”何三道:“老
大,你别是醉了罢?这些话混说的是什么。”说着,拉了那人走到个僻静地方,两
个人商量了一回,各人分头而去,暂且不提。

且说包勇自被贾政吆喝,派去看园,贾母的事出来,也忙了,不曾派他差使。
他也不理会,总是自做自吃,闷来睡一觉,醒时便在园里耍刀弄棍,倒也无拘无束。
那日贾母一早出殡,他虽知道,因没有派他差使,他任意闲游。只见一个女尼带了
一个道婆,来到园内腰门那里扣门。包勇走来,说道:“女师父那里去?”道婆道:
“今日听得老太太的事完了,不见四姑娘送殡,想必是在家看家。恐他寂寞,我们
师父来瞧他一瞧。”包勇道:“主子都不在家,园门是我看的,请你们回去罢。要
来呢,等主子们回来了再来。”婆子道:“你是那里来的个黑炭头,也要管起我们
的走动来了。”包勇道:“我嫌你们这些人,我不叫你们来,你们有什么法儿?”
婆子生了气,嚷道:“这都是反了天的事了,连老太太在日还不能拦我们的来往走
动呢。你是那里的这么个横强盗,这样没法没天的?我偏要打这里走!”说着,便
把手在门环上狠狠的打了几下。妙玉已气的不言语,正要回身便走,不料里头看二
门的婆子听见有人拌嘴,连忙开门一看,见是妙玉,已经回身走去,明知必是包勇
得罪了走了。近日婆子们都知道上头太太们四姑娘都和他亲近,恐他日后说出门上
不放进他来,那时如何耽得住,赶忙走来,说:“不知师父来,我们开门迟了。我
们四姑娘在家里,还正想师父呢。快请回来。看园的小子是个新来的,他不知咱们
的事。回来回了太太,打他一顿,撵出去就完了。”妙玉虽是听见,总不理他。那
禁得看腰门的婆子赶上,再四央求,后来才说出怕自己担不是,几乎急的跪下。妙
玉无奈,只得随着那婆子过来。包勇见这般光景,自然不好再拦,气得瞪眼叹气而
回。

这里妙玉带了道婆走到惜春那里,道了恼,叙些闲话。惜春说起:“在家看家,
只好熬个几夜,但是二奶奶病着,一个人又闷又害怕,能有一个人在这里我就放心,
如今里头一个男人也没有。今儿你既光降,肯伴我一宵,咱们下棋说话儿,可使得
么?”妙玉本来不肯,见惜春可怜,又提起下棋,一时高兴应了。打发道婆回去取
了他的茶具衣褥,命侍儿送了过来,大家坐谈一夜。惜春欣幸异常,便命彩屏去开
上年蠲的雨水,预备好茶。那妙玉自有茶具。道婆去了不多一时,又来了一个侍者,
送下妙玉日用之物。惜春亲自烹茶。两人言语投机,说了半天。那时天有初更时候,
彩屏放下棋枰,两人对弈。惜春连输两盘,妙玉又让了四个子儿,惜春方赢了半子。
不觉已到四更,正是天空地阔,万籁无声。妙玉道:“我到五更须得打坐,我自有
人伏侍,你自去歇息。”惜春犹是不舍,见妙玉要自己养神,不便扭他。

刚要歇去,猛听得东边上屋内上夜的人一片声喊起。惜春那里的老婆子们也接
着声嚷道:“了不得了!有了人了!”唬得惜春彩屏等心胆俱裂,听见外头上夜的
男人便声喊起来。妙玉道:“不好了,必是这里有了贼了。”说着赶忙的关上屋门。
便掩了灯光,在窗户眼内往外一瞧,只见几个男人站在院内。唬得不敢作声,回身
摆着手,轻轻的爬下来,说:“了不得!外头有几个大汉站着。”说犹未了,又听
得房上响声不绝,便有外头上夜的人进来吆喝拿贼。一个人说道:“上屋里的东西
都丢了,并不见人。东边有人去了,咱们到西边去。”惜春的老婆子听见有自己的
人,便在外间屋里说道:“这里有好些人上了房了。”上夜的都道:“你瞧,这可
不是吗!”大家一齐嚷起来。只听房上飞下好些瓦来,众人都不敢上前。

正在没法,只听园里腰门一声大响,打进门来。见一个梢长大汉,手执木棍,
众人唬得藏躲不及。听得那人喊说道:“不要跑了他们一个!你们都跟我来!”这
些家人听了这话,越发唬得骨软筋酥,连跑也跑不动了。只见这人站在当地,只管
乱喊。家人中有一个眼尖些的看出来了,你道是谁,正是甄家荐来的包勇。这些家
人不觉胆壮起来,便颤巍巍的说道:“有一个走了,有的在房上呢。”包勇便向地
下一扑,耸身上房,追赶那贼。这些贼人明知贾家无人,先在院内偷看惜春房内,
见有个绝色尼姑,便顿起淫心。又欺上屋俱是女人,且又畏惧,正要踹进门去,因
听外面有人进来追赶,所以贼众上房。见人不多,还想抵挡,猛见一人上房赶来,
那些贼见是一人,越发不理论了,便用短兵抵住。那经得包勇用力一棍打去,将贼
打下房来。那些贼飞奔而逃,从园墙过去。包勇也在房上追捕。岂知园内早藏下了
几个在那里接赃,已经接过好些。见贼伙跑回,大家举械保护。见追的只有一人,
明欺寡不敌众,反倒迎上来。包勇一见生气,道:“这些毛贼,敢来和我斗斗!”
那伙贼便说:“我们有一个伙计被他们打倒了,不知死活,咱们索性抢了他出来。”
这里包勇闻声即打。那伙贼便轮起器械,四五个人围住包勇,乱打起来。外头上夜
的人也都仗着胆子只顾赶了来。众贼见斗他不过,只得跑了。包勇还要赶时,被一
个箱子一绊,立定看时,心想东西未丢,众贼远逃,也不追赶,便叫众人将灯照看。
地下只有几个空箱,叫人收拾,他便欲跑回上房。因路径不熟,走到凤姐那边,见
里面灯烛辉煌,便问:“这里有贼没有?”里头的平儿战兢兢的说道:“这里也没
开门,只听上屋叫喊,说有贼呢,你到那里去罢。”包勇正摸不着路头,遥见上夜
的人过来,才跟着一齐寻到上屋。见是门开户启,那些上夜的在那里啼哭。

一时贾芸林之孝都进来了,见是失盗,大家着急。进内查点,老太太的房门大
开,将灯一照,锁头拧折。进内一瞧,箱柜已开。便骂那些上夜女人道:“你们都
是死人么?贼人进来,你们都不知道么?”那些上夜的人啼哭着说道:“我们几个
人轮更上夜,是管二三更的。我们都没有住脚,前后走的。他们是四更五更。我们
才下班儿,只听见他们喊起来,并不见一个人。赶着照看,不知什么时候把东西早
已丢了。求爷们问管四更五更的。”林之孝道:“你们个个要死!回来再说,咱们
先到各处看去。”上夜的男人领着走到尤氏那边,门儿关紧。有几个接音说:“唬
死我们了!”林之孝问道:“这里没有丢东西呀?”里头的人方开了门,道:“这
里没丢东西。”林之孝带着人走到惜春院内,只听得里面说道:“了不得,唬死了
姑娘了。醒醒儿罢!”林之孝便叫人开门,问是怎么了。里头婆子开门,说:“贼
在这里打仗,把姑娘都唬坏了。亏得妙师父和彩屏才将姑娘救醒。东西是没失。”
林之孝道:“贼人怎么打仗?”上夜的男人说:“幸亏包大爷上了房把贼打跑了去
了,还听见打倒了一个人呢。”包勇道:“在园门那里呢,你们快瞧去罢。”贾芸
等走到那边,果然看见一个人躺在地下死了,细细的一瞧,好像是周瑞的干儿子。
众人见了诧异,派了一个人看守着,又派了两个人照看前后门。走到门前看时,那
门俱仍旧关锁着。林之孝便叫人开了门,报了营官。立刻到来查勘贼踪,是从后夹
道子上了房的,到了西院房上,见那瓦片破碎不堪,一直过了后园去了。众上夜的
人齐声说道:“这不是贼,是强盗。”营官着急道:“并非明火执仗,怎么便算是
强盗呢?”上夜的道:“我们赶贼,他在房上撇瓦,我们不能到他跟前,幸亏我们
家的姓包的上房打退。赶到园里,还有好几个贼竟和姓包的打起仗来,打不过姓包
的,才都跑了。”营官道:“可又来,若是强盗,难道倒打不过你们的人么?不用
说了,你们快查清了东西,递了失单,我们报就是了。”

贾芸等又到了上屋里,已见凤姐扶病过来,惜春也来了。贾芸请了凤姐的安,
问了惜春的好,大家查看失物。因鸳鸯已死,琥珀等又送灵去了,那些东西都是老
太太的,并没见过数儿,只用封锁,如今打从那里查起?众人都说:“箱柜东西不
少,如今一空,偷的时候儿自然不小了。那些上夜的人管做什么的?况且打死的贼
是周瑞的干儿子,必是他们通同一气的。”凤姐听了,气的眼睛直瞪瞪的,便说:
“把那些上夜的女人都拴起来,交给营里去审问!”众人叫苦连天,跪地哀求。

不知怎生发放,并失去的物件有无着落,下回分解。