\chapter{情中情因情感妹妹~错里错以错劝哥哥}

话说袭人见贾母王夫人等去后,便走来宝玉身边坐下,含泪问他:“怎么就打
到这步田地?”宝玉叹气说道:“不过为那些事,问他做什么!只是下半截疼的很,
你瞧瞧,打坏了那里?”袭人听说,便轻轻的伸手进去,将中衣脱下,略动一动,
宝玉便咬着牙叫嗳哟,袭人连忙停住手:如此三四次,才褪下来了。袭人看时,只
见腿上半段青紫,都有四指阔的僵痕高起来。袭人咬着牙说道:“我的娘,怎么下
这般的狠手!你但凡听我一句话,也不到这个分儿。幸而没动筋骨,倘或打出个残
疾来,可叫人怎么样呢?”

正说着,只听丫鬟们说:“宝姑娘来了。”袭人听见,知道穿不及中衣,便拿了
一床夹纱被替宝玉盖了。只见宝钗手里托着一丸药走进来,向袭人说道:“晚上把
这药用酒研开,替他敷上,把那淤血的热毒散开,就好了。”说毕,递与袭人。又
问:“这会子可好些?”宝玉一面道谢,说:“好些了。”又让坐。宝钗见他睁开眼
说话,不像先时,心中也宽慰了些,便点头叹道:“早听人一句话,也不至有今日。
别说老太太、太太心疼,就是我们看着,心里也——”刚说了半句,又忙咽住,不
觉眼圈微红,双腮带赤,低头不语了。宝玉听得这话如此亲切,大有深意,忽见他
又咽住不往下说,红了脸低下头含着泪只管弄衣带,那一种软怯娇羞、轻怜痛惜之
情,竟难以言语形容,越觉心中感动,将疼痛早已丢在九霄云外去了。想道:“我
不过挨了几下打,他们一个个就有这些怜惜之态,令人可亲可敬。假若我一时竟别
有大故,他们还不知何等悲感呢。既是他们这样,我便一时死了,得他们如此,一
生事业纵然尽付东流,也无足叹惜了。”

正想着,只听宝钗问袭人道:“怎么好好的动了气,就打起来了?”袭人便把
焙茗的话悄悄说了。宝玉原来还不知贾环的话,见袭人说出,方才知道;因又拉上
薛蟠,惟恐宝钗沉心,忙又止住袭人道:“薛大哥从来不是这样,你们别混猜度。”
宝钗听说,便知宝玉是怕他多心,用话拦袭人。因心中暗暗想道:“打得这个形象,
疼还顾不过来,还这样细心,怕得罪了人。你既这样用心,何不在外头大事上做工
夫,老爷也欢喜了,也不能吃这样亏。你虽然怕我沉心所以拦袭人的话,难道我就
不知我哥哥素日恣心纵欲、毫无防范的那种心性吗?当日为个秦种还闹的天翻地覆,
自然如今比先又加利害了。”想毕,因笑道:“你们也不必怨这个怨那个,据我想,
到底宝兄弟素日肯和那些人来往,老爷才生气。就是我哥哥说话不防头,一时说出
宝兄弟来,也不是有心挑唆:一则也是本来的实话,二则他原不理论这些防嫌小事。
袭姑娘从小儿只见过宝兄弟这样细心的人,何曾见过我哥哥那天不怕地不怕、心里
有什么口里说什么的人呢?”袭人因说出薛蟠来,见宝玉拦他的话,早已明白自己
说造次了,恐宝钗没意思;听宝钗如此说,更觉羞愧无言。宝玉又听宝钗这一番话,
半是堂皇正大,半是体贴自己的私心,更觉比先心动神移。方欲说话时,只见宝钗
起身道:“明日再来看你,好生养着罢。方才我拿了药来,交给袭人,晚上敷上管
就好了。”说着便走出门去。袭人赶着送出院外,说:“姑娘倒费心了。改日宝二爷
好了,亲自来谢。”宝钗回头笑道:“这有什么的?只劝他好生养着,别胡思乱想就
好了。要想什么吃的玩的,悄悄的往我那里只管取去,不必惊动老太太、太太众人。
倘或吹到老爷耳朵里,虽然彼时不怎么样,将来对景,终是要吃亏的。”说着去了。

袭人抽身回来,心内着实感激宝钗。进来见宝玉沉思默默,似睡非睡的模样,
因而退出房外栉沐。宝玉默默的躺在床上,无奈臀上作痛,如针挑刀挖一般,更热
如火炙,略展转时,禁不住“嗳哟”之声。那时天色将晚,因见袭人去了,却有两
三个丫鬟伺候,此时并无呼唤之事,因说道:“你们且去梳洗,等我叫时再来。”众
人听了,也都退出。

这里宝玉昏昏沉沉,只见蒋玉函走进来了,诉说忠顺府拿他之事;一时又见金
钏儿进来,哭说为他投井之情。宝玉半梦半醒,刚要诉说前情,忽又觉有人推他,
恍恍惚惚听得悲切之声。宝玉从梦中惊醒,睁眼一看,不是别人,却是黛玉。犹恐
是梦,忙又将身子欠起来,向脸上细细一认,只见他两个眼睛肿得桃儿一般,满面
泪光,不是黛玉却是那个?宝玉还欲看时,怎奈下半截疼痛难禁,支持不住,便“嗳
哟”一声仍旧倒下,叹了口气说道:“你又做什么来了?太阳才落,那地上还是怪热
的,倘或又受了暑,怎么好呢?我虽然捱了打,却也不很觉疼痛。这个样儿是装出
来哄他们,好在外头布散给老爷听。其实是假的,你别信真了。”

此时黛玉虽不是嚎啕大哭,然越是这等无声之泣,气噎喉堵,更觉利害。听了
宝玉这些话,心中提起万句言词,要说时却不能说得半句。半天,方抽抽噎噎的道:
“你可都改了罢!”宝玉听说,便长叹一声道:“你放心。别说这样话。我便为这些
人死了,也是情愿的。”

一句话未了,只见院外人说:“二奶奶来了。”黛玉便知是凤姐来了,连忙立起
身,说道:“我从后院子里去罢,回来再来。”宝玉一把拉住道:“这又奇了,好好
的怎么怕起他来了?”黛玉急得跺脚,悄悄的说道:“你瞧瞧我的眼睛!又该他们拿
咱们取笑儿了。”宝玉听说,赶忙的放了手。黛玉三步两步转过床后,刚出了后院,
凤姐从前头已进来了。问宝玉:“可好些了?想什么吃?叫人往我那里取去。”接着薛
姨妈又来了。一时贾母又打发了人来。

至掌灯时分,宝玉只喝了两口汤,便昏昏沉沉的睡去。接着周瑞媳妇、吴新登
媳妇、郑好时媳妇这几个有年纪长来往的,听见宝玉捱了打,也都进来。袭人忙迎
出来,悄悄的笑道:“婶娘们略来迟了一步,二爷睡着了。”说着,一面陪他们到那
边屋里坐着,倒茶给他们吃。那几个媳妇子都悄悄的坐了一回,向袭人说:“等二
爷醒了,你替我们说罢。”袭人答应了,送他们出去。刚要回来,只见王夫人使个
老婆子来说:“太太叫一个跟二爷的人呢。”袭人见说,想了一想,便回身悄悄的告
诉晴雯、麝月、秋纹等人说:“太太叫人,你们好生在屋里,我去了就来。”说毕,
同那老婆子一径出了园子,来至上房。

王夫人正坐在凉榻上,摇着芭蕉扇子。见他来了,说道:“你不管叫谁来也罢
了,又撂下他来了,谁伏侍他呢?”袭人见说,连忙陪笑回道:“二爷才睡了,那
四五个丫头,如今也好了,会伏侍了。太太请放心。恐怕太太有什么话吩咐,打发
他们来,一时听不明白倒耽误了事。”王夫人道:“也没什么话,白问问他这会子疼
的怎么样了?”袭人道:“宝姑娘送来的药,我给二爷敷上了,比先好些了。先疼
的躺不住,这会子都睡沉了,可见好些。”王夫人又问:“吃了什么没有?”袭人道:
“老太太给的一碗汤,喝了两口,只嚷干渴,要吃酸梅汤。我想酸梅是个收敛东西,
刚才捱打,又不许叫喊,自然急的热毒热血未免存在心里。倘或吃下这个去激在心
里,再弄出病来,那可怎么样呢。因此我劝了半天,才没吃。只拿那糖腌的玫瑰卤
子和了,吃了小半碗,嫌吃絮了,不香甜。”王夫人道:“嗳哟,你何不早来和我说?
前日倒有人送了几瓶子香露来。原要给他一点子,我怕胡遭塌了,就没给。既是他
嫌那玫瑰膏子吃絮了,把这个拿两瓶子去,一碗水里只用挑上一茶匙,就香的了不
得呢。”说着,就唤彩云来:“把前日的那几瓶香露拿了来。”袭人道:“只拿两瓶来
罢,多也白遭塌。等不够再来取也是一样。”彩云听了,去了半日,果然拿了两瓶
来付与袭人。袭人看时,只见两个玻璃小瓶却有三寸大小,上面螺丝银盖,鹅黄笺
上写着“木樨清露”,那一个写着“玫瑰清露”。袭人笑道:“好尊贵东西!这么个小
瓶儿,能有多少?”王夫人道:“那是进上的,你没看见鹅黄笺子?你好生替他收着,
别遭塌了。”

袭人答应着,方要走时,王夫人又叫:“站着,我想起一句话来问你。”袭人忙
又回来。王夫人见房内无人,便问道:“我恍惚听见宝玉今日捱打,是环儿在老爷
跟前说了什么话,你可听见这个话没有?”袭人道:“我倒没听见这个话,只听见
说为二爷认得什么王府的戏子,人家来和老爷说了,为这个打的。”王夫人摇头说
道:“也为这个。只是还有别的原故呢。”袭人道:“别的原故,实在不知道。”又低
头迟疑了一会,说道:“今日大胆在太太跟前说句冒撞话,论理——”说了半截,
却又咽住。王夫人道:“你只管说。”袭人道:“太太别生气,我才敢说。”王夫人道:
“你说就是了。”袭人道:“论理宝二爷也得老爷教训教训才好呢!要老爷再不管,
不知将来还要做出什么事来呢。”

王夫人听见了这话,便点头叹息,由不得赶着袭人叫了一声:“我的儿!你这话
说的很明白,和我的心里想的一样。其实,我何曾不知道宝玉该管?比如先时你珠
大爷在,我是怎么样管他,难道我如今倒不知管儿子了?只是有个原故:如今我想
我已经五十岁的人了,通共剩了他一个,他又长的单弱,况且老太太宝贝似的,要
管紧了他,倘或再有个好歹儿,或是老太太气着,那时上下不安,倒不好,所以就
纵坏了他了。我时常掰着嘴儿说一阵,劝一阵,哭一阵。彼时也好,过后来还是不
相干,到底吃了亏才罢!设若打坏了,将来我靠谁呢!”说着,由不得又滴下泪来。

袭人见王夫人这般悲感,自己也不觉伤了心,陪着落泪。又道:“二爷是太太
养的,太太岂不心疼;就是我们做下人的,伏侍一场,大家落个平安,也算造化了。
要这样起来,连平安都不能了。那一日那一时我不劝二爷?只是再劝不醒。偏偏那
些人又肯亲近他,也怨不得他这样。如今我们劝的倒不好了。今日太太提起这话来,
我还惦记着一件事,要来回太太,讨太太个主意。只是我怕太太疑心,不但我的话
白说了,且连葬身之地都没有了!”王夫人听了这话内中有因,忙问道:“我的儿!
你只管说。近来我因听见众人背前面后都夸你,我只说你不过在宝玉身上留心,或
是诸人跟前和气这些小意思。谁知你方才和我说的话,全是大道理,正合我的心事。
你有什么只管说什么,只别叫别人知道就是了。”袭人道:“我也没什么别的说,我
只想着讨太太一个示下,怎么变个法儿,以后竟还叫二爷搬出园外来住就好了。”

王夫人听了,吃一大惊,忙拉了袭人的手,问道:“宝玉难道和谁作怪了不成?”
袭人连忙回道:“太太别多心,并没有这话,这不过是我的小见识:如今二爷也大
了,里头姑娘们也大了,况且林姑娘宝姑娘又是两姨姑表姐妹,虽说是姐妹们,到
底是男女之分,日夜一处,起坐不方便,由不得叫人悬心。既蒙老太太和太太的恩
典,把我派在二爷屋里,如今跟在园中住,都是我的干系。太太想:多有无心中做
出,有心人看见,当做有心事,反说坏了的,倒不如预先防着点儿。况且二爷素日
的性格,太太是知道的,他又偏好在我们队里闹。倘或不防,前后错了一点半点,
不论真假,人多嘴杂——那起坏人的嘴,太太还不知道呢:心顺了,说的比菩萨还
好;心不顺,就没有忌讳了。二爷将来倘或有人说好,不过大家落个直过儿;设若
叫人哼出一声不是来,我们不用说,粉身碎骨,还是平常,后来二爷一生的声名品
行,岂不完了呢?那时老爷太太也白疼了,白操了心了。不如这会子防避些,似乎
妥当。太太事情又多,一时固然想不到;我们想不到便罢了,既想到了,要不回明
了太太,罪越重了。近来我为这件事,日夜悬心,又恐怕太太听着生气,所以总没
敢言语。”

王夫人听了这话,正触了金钏儿之事,直呆了半晌,思前想后,心下越发感爱
袭人。笑道:“我的儿!你竟有这个心胸,想得这样周全。我何曾又不想到这里?只
是这几次有事就混忘了。你今日这话提醒了我,难为你这样细心,真真好孩子!也
罢了,你且去罢,我自有道理。只是还有一句话,你如今既说了这样的话,我索性
就把他交给你了。好歹留点心儿,别叫他遭塌了身子才好。自然不辜负你。”袭人
低了一回头,方道:“太太吩咐,敢不尽心吗?”说着,慢慢的退出。

回到院中,宝玉方醒。袭人回明香露之事,宝玉甚喜,即命调来吃,果然香妙
非常。因心下惦着黛玉,要打发人去,只是怕袭人拦阻,便设法先使袭人往宝钗那
里去借书。袭人去了,宝玉便命晴雯来,吩咐道:“你到林姑娘那里,看他做什么
呢。他要问我,只说我好了。”晴雯道:“白眉赤眼儿的,作什么去呢!到底说句话
儿,也像件事啊。”宝玉道:“没有什么可说的么。”晴雯道:“或是送件东西,或是
取件东西,不然我去了怎么搭讪呢?”宝玉想了一想,便伸手拿了两条旧绢子,撂
与晴雯,笑道:“也罢,就说我叫你送这个给他去了。”晴雯道:“这又奇了,他要
这半新不旧的两条绢子?他又要恼了,说你打趣他。”宝玉笑道:“你放心,他自然
知道。”

晴雯听了,只得拿了绢子,往潇湘馆来。只见春纤正在栏杆上晾手巾,见他进
来,忙摇手儿说:“睡下了。”晴雯走进来,满屋漆黑,并未点灯,黛玉已睡在床上,
问:“是谁?”晴雯忙答道:“晴雯。”黛玉道:“做什么?”晴雯道:“二爷叫给姑
娘送绢子来了。”黛玉听了,心中发闷,暗想:“做什么送绢子来给我?”因问:“这
绢子是谁送他的?必定是好的,叫他留着送别人罢,我这会子不用这个。”晴雯笑道:
“不是新的,就是家常旧的。”黛玉听了,越发闷住了。细心揣度,一时方大悟过
来,连忙说:“放下,去罢。”晴雯只得放下,抽身回去。一路盘算,不解何意。

这黛玉体贴出绢子的意思来,不觉神痴心醉,想到:宝玉能领会我这一番苦意,
又令我可喜。我这番苦意,不知将来可能如意不能,又令我可悲。要不是这个意思,
忽然好好的送两块帕子来,竟又令我可笑了。再想到私相传递,又觉可惧。他既如
此,我却每每烦恼伤心,反觉可愧。如此左思右想,一时五内沸然。由不得馀意缠
绵,便命掌灯,也想不起嫌疑避讳等事,研墨蘸笔,便向那两块旧帕上写道:
眼空蓄泪泪空垂,暗洒闲抛更向谁?
尺幅鲛绡劳惠赠,为君那得不伤悲!

其
二
抛珠滚玉只偷潸,镇日无心镇日闲。
枕上袖边难拂拭,任他点点与斑斑。

其
三
彩线难收面上珠,湘江旧迹已模糊。
窗前亦有千竿竹,不识香痕渍也无?
那黛玉还要往下写时,觉得浑身火热,面上作烧,走至镜台揭起锦袱一照,只见腮
上通红,真合压倒桃花,却不知病由此起。一时方上床睡去,犹拿着绢子思索,不
在话下。

却说袭人来见宝钗,谁知宝钗不在园内,往他母亲那里去了。袭人不便空手回
来,等至起更,宝钗方回。

原来宝钗素知薛蟠情性,心中已有一半疑是薛蟠挑唆了人来告宝玉了,谁知又
听袭人说出来,越发信了。究竟袭人是焙茗说的,那焙茗也是私心窥度,并未据实,
大家都是一半猜度,竟认作十分真切了。可笑那薛蟠因素日有这个名声,其实这一
次却不是他干的,竟被人生生的把个罪名坐定。这日正从外头吃了酒回来,见过了
母亲,只见宝钗在这里坐着,说了几句闲话儿,忽然想起,因问道:“听见宝玉挨
打,是为什么?”薛姨妈正为这个不自在,见他问时,便咬着牙道:“不知好歹的
冤家,都是你闹的,你还有脸来问!”薛蟠见说便怔了,忙问道:“我闹什么?”薛
姨妈道:“你还装腔呢!人人都知道是你说的。”薛蟠道:“人人说我杀了人,也就信
了罢?”薛姨妈道:“连你妹妹都知道是你说,难道他也赖你不成?”宝钗忙劝道:
“妈妈和哥哥且别叫喊,消消停停的,就有个青红皂白了。”又向薛蟠道:“是你说
的也罢,不是你说的也罢,事情也过去了,不必较正,把小事倒弄大了。我只劝你
从此以后少在外头胡闹,少管别人的事。天天一处大家胡逛,你是个不防头的人,
过后没事就罢了,倘或有事,不是你干的,人人都也疑惑说是你干的。不用别人,
我先就疑惑你。”

薛蟠本是个心直口快的人,见不得这样藏头露尾的事;又是宝钗劝他别再胡逛
去;他母亲又说他犯舌,宝玉之打,是他治的:早已急得乱跳,赌神发誓的分辩。
又骂众人:“谁这么编派我?我把那囚攮的牙敲了!分明是为打了宝玉,没的献勤儿,
拿我来做幌子。难道宝玉是天王?他父亲打他一顿,一家子定要闹几天。那一回为
他不好,姨父打了他两下子,过后儿老太太不知怎么知道了,说是珍大哥治的,好
好儿的叫了去骂了一顿。今日越发拉上我了!既拉上我也不怕,索性进去把宝玉打
死了,我替他偿命!”一面嚷,一面找起一根门闩来就跑。慌的薛姨妈拉住骂道:“作
死的孽障,你打谁去?你先打我来!”薛蟠的眼急的铜铃一般,嚷道:“何苦来!又不
叫我去,为什么好好的赖我?将来宝玉活一日,我耽一日的口舌,不如大家死了清
净!”宝钗忙也上前劝道:“你忍耐些儿罢。妈妈急的这个样儿,你不说来劝,你倒
反闹的这样。别说是妈妈,就是旁人来劝你,也是为好,倒把你的性子劝上来!”
薛蟠道:“你这会子又说这话,都是你说的。”宝钗道:“你只怨我说,再不怨你那
顾前不顾后的形景!”薛蟠道:“你只会怨我顾前不顾后,你怎么不怨宝玉外头招风
惹草的呢?别说别的,就拿前日琪官儿的事比给你们听:那琪官儿我们见了十来次,
他并没和我说一句亲热话,怎么前儿他见了,连姓名还不知道,就把汗巾子给他?
难道这也是我说的不成?”薛姨妈和宝钗急的说道:“还提这个!可不是为这个打他
呢。可见是你说的了。”薛蟠道:“真真的气死人了!赖我说的我不恼,我只气一个
宝玉闹的这么天翻地覆的!”宝钗道:“谁闹来着?你先持刀动杖的闹起来,倒说别
人闹。”

薛蟠见宝钗说的话句句有理,难以驳正,比母亲的话反难回答,因此便要设法
拿话堵回他去,就无人敢拦自己的话了。也因正在气头儿上,未曾想话之轻重,便
道:“好妹妹,你不用和我闹,我早知道你的心了。从先妈妈和我说:你这金锁要
拣有玉的才可配,你留了心,见宝玉有那劳什子,你自然如今行动护着他。”话未
说了,把个宝钗气怔了,拉着薛姨妈哭道:“妈妈,你听哥哥说的是什么话!”薛蟠
见妹子哭了,便知自己冒撞,便赌气走到自己屋里安歇不提。

宝钗满心委屈气忿,待要怎样,又怕他母亲不安,少不得含泪别了母亲,各自
回来。到屋里整哭了一夜。次日一早起来,也无心梳洗,胡乱整理了衣裳,便出来
瞧母亲。可巧遇见黛玉独立在花阴之下,问他那里去,宝钗因说:“家去。”口里说
着,便只管走。黛玉见他无精打彩的去了,又见眼上好似有哭泣之状,大非往日可
比,便在后面笑道:“姐姐也自己保重些儿。就是哭出两缸泪来,也医不好棒疮!”

不知宝钗如何答对,且听下回分解。