\chapter{贾宝玉神游太虚境~警幻仙曲演红楼梦}

第四回中既将薛家母子在荣府中寄居等事略已表明,此回暂可不写了。如今且
说林黛玉自在荣府,一来贾母万般怜爱,寝食起居一如宝玉,把那迎春、探春、惜
春三个孙女儿倒且靠后了;就是宝玉黛玉二人的亲密友爱,也较别人不同,日则同
行同坐,夜则同止同息,真是言和意顺,似漆如胶。不想如今忽然来了一个薛宝钗,
年纪虽大不多,然品格端方,容貌美丽,人人都说黛玉不及。那宝钗却又行为豁达,
随分从时,不比黛玉孤高自许,目无下尘,故深得下人之心,就是小丫头们亦多和
宝钗亲近。因此黛玉心中便有些不忿,宝钗却是浑然不觉。那宝玉也在孩提之间,
况他天性所,一片愚拙偏僻,视姊妹兄弟皆如一体,并无亲疏远近之别。如今与
黛玉同处贾母房中,故略比别的姊妹熟惯些,既熟惯便更觉亲密,既亲密便不免有
些不虞之隙、求全之毁。这日不知为何,二人言语有些不和起来,黛玉又在房中独
自垂泪。宝玉也自悔言语冒撞,前去俯就,那黛玉方渐渐的回转过来。

因东边宁府花园内梅花盛开,贾珍之妻尤氏乃治酒具,请贾母、邢夫人、王夫
人等赏花,是日先带了贾蓉夫妻二人来面请。贾母等于早饭后过来,就在会芳园游
玩,先茶后酒。不过是宁荣二府眷属家宴,并无别样新文趣事可记。

一时宝玉倦怠,欲睡中觉。贾母命人:“好生哄着,歇息一回再来。”贾蓉媳妇
秦氏便忙笑道:“我们这里有给宝二叔收拾下的屋子,老祖宗放心,只管交给我就
是了。”因向宝玉的奶娘丫鬟等道:“嬷嬷、姐姐们,请宝二叔跟我这里来。”贾母
素知秦氏是极妥当的人,因他生得袅娜纤巧,行事又温柔和平,乃重孙媳中第一个
得意之人。见他去安置宝玉,自然是放心的了。

当下秦氏引一簇人来至上房内间,宝玉抬头看见是一幅画挂在上面,人物固好,
其故事乃是“燃藜图”也,心中便有些不快。又有一副对联,写的是:“世事洞明
皆学问,人情练达即文章。”及看了这两句,纵然室宇精美,铺陈华丽,亦断断不
肯在这里了,忙说:“快出去,快出去!”秦氏听了笑道:“这里还不好,往那里去
呢?要不就往我屋里去罢。”宝玉点头微笑。一个嬷嬷说道:“那里有个叔叔往侄儿
媳妇房里睡觉的礼呢?”秦氏笑道:“不怕他恼,他能多大了,就忌讳这些个?上月
你没有看见我那个兄弟来了,虽然和宝二叔同年,两个人要站在一处,只怕那一个
还高些呢。”宝玉道:“我怎么没有见过他?你带他来我瞧瞧。”众人笑道:“隔着二
三十里,那里带去?见的日子有呢。”

说着大家来至秦氏卧房。刚至房中,便有一股细细的甜香。宝玉此时便觉眼饧
骨软,连说:“好香!”入房向壁上看时,有唐伯虎画的《海棠春睡图》,两边有宋
学士秦太虚写的一副对联云:
嫩寒锁梦因春冷,
芳气袭人是酒香。
案上设着武则天当日镜室中设的宝镜,一边摆着赵飞燕立着舞的金盘,盘内盛着安
禄山掷过伤了太真乳的木瓜。上面设着寿昌公主于含章殿下卧的宝榻,悬的是同昌
公主制的连珠帐。宝玉含笑道:“这里好,这里好!”秦氏笑道:“我这屋子,大约
神仙也可以住得了。”说着,亲自展开了西施浣过的纱衾,移了红娘抱过的鸳枕。
于是众奶姆伏侍宝玉卧好了,款款散去,只留下袭人、晴雯、麝月、秋纹四个丫鬟
为伴。秦氏便叫小丫鬟们好生在檐下看着猫儿打架。

那宝玉才合上眼,便恍恍惚惚的睡去,犹似秦氏在前,悠悠荡荡,跟着秦氏到
了一处。但见朱栏玉砌,绿树清溪,真是人迹不逢,飞尘罕到。宝玉在梦中欢喜,
想道:“这个地方儿有趣!我若能在这里过一生,强如天天被父母师傅管束呢。”正
在胡思乱想,听见山后有人作歌曰:
春梦随云散,飞花逐水流。
寄言众儿女,何必觅闲愁。
宝玉听了,是个女孩儿的声气。歌音未息,早见那边走出一个美人来,蹁跹袅娜,
与凡人大不相同。有赋为证:

方离柳坞,乍出花房。但行处鸟惊庭树,将到时影度回廊。仙袂乍飘兮,闻麝
兰之馥郁;荷衣欲动兮,听环之铿锵。靥笑春桃兮,云髻堆翠;唇绽樱颗兮,榴
齿含香。纤腰之楚楚兮,风回雪舞;耀珠翠之的的兮,鸭绿鹅黄。出没花间兮,
宜嗔宜喜;徘徊池上兮,若飞若扬。蛾眉欲颦兮,将言而未语;莲步乍移兮,欲止
而仍行。羡美人之良质兮,冰清玉润;慕美人之华服兮,闪烁文章。爱美人之容貌
兮,香培玉篆;比美人之态度兮,凤翥龙翔。其素若何,春梅绽雪;其洁若何,秋
蕙披霜。其静若何,松生空谷;其艳若何,霞映澄塘。其文若何,龙游曲沼;其神
若何,月射寒江。远惭西子,近愧王嫱。生于孰地?降自何方?若非宴罢归来,瑶池
不二;定应吹箫引去,紫府无双者也。

宝玉见是一个仙姑,喜的忙来作揖,笑问道:“神仙姐姐,不知从那里来,如
今要往那里去?我也不知这里是何处,望乞携带携带。”那仙姑道:“吾居离恨天之
上灌愁海之中,乃放春山遣香洞太虚幻境警幻仙姑是也。司人间之风情月债,掌尘
世之女怨男痴。因近来风流冤孽缠绵于此,是以前来访察机会,布散相思。今日与
尔相逢,亦非偶然。此离吾境不远,别无他物,仅有自采仙茗一盏,亲酿美酒几瓮,
素练魔舞歌姬数人,新填《红楼梦》仙曲十二支。可试随我一游否?”宝玉听了,
喜跃非常,便忘了秦氏在何处了,竟随着这仙姑到了一个所在,忽见前面有一座石
牌横建,上书“太虚幻境”四大字,两边一副对联,乃是:
假作真时真亦假,
无为有处有还无。
转过牌坊便是一座宫门,上面横书着四个大字,道是“孽海情天”。也有一副对联,
大书云:
厚地高天,堪叹古今情不尽;
痴男怨女,可怜风月债难酬。
宝玉看了,心下自思道:“原来如此。但不知何为‘古今之情’,又何为‘风月之债’?
从今倒要领略领略。”宝玉只顾如此一想,不料早把些邪魔招入膏肓了。当下随了
仙姑进入二层门内,只见两边配殿皆有匾额对联,一时看不尽许多,惟见几处写着
的是“痴情司”、“结怨司”、“朝啼司”、“暮哭司”、“春感司”、“秋悲司”。看了,
因向仙姑道:“敢烦仙姑引我到那各司中游玩游玩,不知可使得么?”仙姑道:“此
中各司存的是普天下所有的女子过去未来的簿册,尔乃凡眼尘躯,未便先知的。”
宝玉听了,那里肯舍,又再四的恳求。那警幻便说:“也罢,就在此司内略随喜随
喜罢。”

宝玉喜不自胜,抬头看这司的匾上,乃是“薄命司”三字,两边写着对联道:
春恨秋悲皆自惹,
花容月貌为谁妍。
宝玉看了,便知感叹。进入门中,只见有十数个大橱,皆用封条封着,看那封条上
皆有各省字样。宝玉一心只拣自己家乡的封条看,只见那边橱上封条大书“金陵十
二钗正册”,宝玉因问:“何为‘金陵十二钗正册’?”警幻道:“即尔省中十二冠
首女子之册,故为正册。”宝玉道:“常听人说金陵极大,怎么只十二个女子?如今
单我们家里上上下下就有几百个女孩儿。”警幻微笑道:“一省女子固多,不过择其
紧要者录之,两边二橱则又次之。馀者庸常之辈便无册可录了。”宝玉再看下首一
橱,上写着“金陵十二钗副册”,又一橱上写着“金陵十二钗又副册”。宝玉便伸手
先将“又副册”橱门开了,拿出一本册来,揭开看时,只见这首页上画的既非人物
亦非山水,不过是水墨染,满纸乌云浊雾而已。后有几行字迹,写道是:

霁月难逢,彩云易散。心比天高,身为下贱。风流灵巧招人怨。寿夭多因诽谤
生,多情公子空牵念。
宝玉看了不甚明白。又见后面画着一簇鲜花,一床破席,也有几句言词写道是:
枉自温柔和顺,空云似桂如兰。
堪羡优伶有福,谁知公子无缘。
宝玉看了,益发解说不出是何意思,遂将这一本册子搁起来,又去开了“副册”橱
门。拿起一本册来打开看时,只见首页也是画,却画着一枝桂花,下面有一方池沼,
其中水涸泥干,莲枯藕败,后面书云:
根并荷花一茎香,平生遭际实堪伤。
自从两地生孤木,致使香魂返故乡。
宝玉看了又不解。又去取那“正册”看时,只见头一页上画着是两株枯木,木上悬
着一围玉带;地下又有一堆雪,雪中一股金簪。也有四句诗道:
可叹停机德,堪怜咏絮才。
玉带林中挂,金簪雪里埋。
宝玉看了仍不解,待要问时,知他必不肯泄漏天机,待要丢下又不舍。遂往后看,
只见画着一张弓,弓上挂着一个香橼。也有一首歌词云:
二十年来辨是非,榴花开处照宫闱。
三春争及初春景?虎兔相逢大梦归。
后面又画着两个人放风筝,一片大海,一只大船,船中有一女子掩面泣涕之状。画
后也有四句写着道:
才自清明志自高,生于末世运偏消。
清明涕泣江边望,千里东风一梦遥。
后面又画着几缕飞云,一湾逝水。其词曰:
富贵又何为?襁褓之间父母违。
展眼吊斜辉,湘江水逝楚云飞。
后面又画着一块美玉落在泥污之中。其断语云:
欲洁何曾洁?云空未必空。
可怜金玉质,终陷淖泥中。
后面忽画一恶狼,追扑一美女,欲啖之意。其下书云:
子系中山狼,得志便猖狂。
金闺花柳质,一载赴黄粱。
后面便是一所古庙,里面有一美人在内看经独坐。其判云:
勘破三春景不长,缁衣顿改昔年妆。
可怜绣户侯门女,独卧青灯古佛旁。
后面便是一片冰山,上有一只雌凤。其判云:
凡鸟偏从末世来,都知爱慕此生才。
一从二令三人木,哭向金陵事更哀。
后面又是一座荒村野店,有一美人在那里纺绩。其判曰:
势败休云贵,家亡莫论亲。
偶因济村妇,巧得遇恩人。
诗后又画一盆茂兰,旁有一位凤冠霞帔的美人。也有判云:
桃李春风结子完,到头谁似一盆兰。
如冰水好空相妒,枉与他人作笑谈。
诗后又画一座高楼,上有一美人悬梁自尽。其判云:
情天情海幻情深,情既相逢必主淫。
漫言不肖皆荣出,造衅开端实在宁。
宝玉还欲看时,那仙姑知他天分高明、性情颖慧,恐泄漏天机,便掩了卷册笑向宝
玉道:“且随我去游玩奇景,何必在此打这闷葫芦?”

宝玉恍恍惚惚,不觉弃了卷册,又随警幻来至后面。但见画栋雕檐,珠帘绣幕,
仙花馥郁,异草芬芳,真好所在也。正是:
光摇朱户金铺地,雪照琼窗玉作宫。
又听警幻笑道:“你们快出来迎接贵客。”一言未了,只见房中走出几个仙子来,荷
袂蹁跹,羽衣飘舞,娇若春花,媚如秋月。见了宝玉,都怨谤警幻道:“我们不知
系何贵客,忙的接出来!姐姐曾说今日今时必有绛珠妹子的生魂前来游玩,故我等
久待,何故反引这浊物来污染清净女儿之境?”宝玉听如此说,便吓的欲退不能,
果觉自形污秽不堪。警幻忙携住宝玉的手向众仙姬笑道:“你等不知原委。今日原
欲往荣府去接绛珠,适从宁府经过,偶遇宁荣二公之灵,嘱吾云:‘吾家自国朝定
鼎以来,功名奕世,富贵流传,已历百年。奈运终数尽不可挽回,我等之子孙虽多,
竟无可以继业者。惟嫡孙宝玉一人,禀性乖张,用情怪谲,虽聪明灵慧,略可望成,
无奈吾家运数合终,恐无人规引入正。幸仙姑偶来,望先以情欲声色等事警其痴顽,
或能使他跳出迷人圈子,入于正路,便是吾兄弟之幸了。’如此嘱吾,故发慈心,
引彼至此。先以他家上中下三等女子的终身册籍令其熟玩,尚未觉悟;故引了再到
此处,遍历那饮馔声色之幻,或冀将来一悟,未可知也。”

说毕,携了宝玉入室。但闻一缕幽香,不知所闻何物。宝玉不禁相问,警幻冷
笑道:“此香乃尘世所无,尔如何能知!此系诸名山胜境初生异卉之精,合各种宝林
珠树之油所制,名为‘群芳髓’。”宝玉听了,自是羡慕。于是大家入座,小鬟捧上
茶来,宝玉觉得香清味美,迥非常品,因又问何名。警幻道:“此茶出在放春山遣
香洞,又以仙花灵叶上所带的宿露烹了,名曰‘千红一窟’。”宝玉听了,点头称赏。
因看房内瑶琴、宝鼎、古画、新诗,无所不有;更喜窗下亦有唾绒,奁间时渍粉污。
壁上也挂着一副对联,书云:
幽微灵秀地,
无可奈何天。
宝玉看毕,因又请问众仙姑姓名:一名痴梦仙姑,一名钟情大士,一名引愁金女,
一名度恨菩提,各各道号不一。少刻,有小鬟来调桌安椅,摆设酒馔。正是:
琼浆满泛玻璃盏,玉液浓斟琥珀杯。
宝玉因此酒香冽异常,又不禁相问,警幻道:“此酒乃以百花之蕤,万木之汁,加
以麟髓凤乳酿成,因名为‘万艳同杯’。”宝玉称赏不迭。

饮酒间,又有十二个舞女上来,请问演何调曲。警幻道:“就将新制《红楼梦》
十二支演上来。”舞女们答应了,便轻敲檀板,款按银筝,听他歌道是:
开辟鸿蒙,
方歌了一句,警幻道:“此曲不比尘世中所填传奇之曲,必有生旦净末之则,又有
南北九宫之调。此或咏叹一人,或感怀一事,偶成一曲,即可谱入管弦。若非个中
人,不知其中之妙。料尔亦未必深明此调,若不先阅其稿,后听其曲,反成嚼蜡矣。”
说毕,回头命小鬟取了《红楼梦》原稿来,递与宝玉。宝玉接过来,一面目视其文,
耳聆其歌曰:

[红楼梦引子]
开辟鸿蒙,谁为情种?都只为风月情浓。奈何天,伤怀日,寂
寥时,试遣愚衷。因此上演出这悲金悼玉的“红楼梦”。

[终身误]
都道是金玉良缘,俺只念木石前盟。空对着山中高士晶莹雪,终不
忘世外仙姝寂寞林。叹人间美中不足今方信。纵然是齐眉举案,到底意难平。

[枉凝眉]
一个是阆苑仙葩,一个是美玉无瑕。若说没奇缘,今生偏又遇着他;
若说有奇缘,如何心事终虚话?一个枉自嗟呀,一个空劳牵挂。一个是水中月,一
个是镜中花。想眼中能有多少泪珠儿,怎禁得秋流到冬,春流到夏!

却说宝玉听了此曲,散漫无稽,未见得好处;但其声韵凄婉,竟能销魂醉魄。
因此也不问其原委,也不究其来历,就暂以此释闷而已。因又看下面道:

[恨无常]
喜荣华正好,恨无常又到,眼睁睁把万事全抛,荡悠悠芳魂销耗。
望家乡路远山高。故向爹娘梦里相寻告:儿命已入黄泉,天伦呵须要退步抽身早!

[分骨肉]
一帆风雨路三千,把骨肉家园,齐来抛闪,恐哭损残年,告爹娘休
把儿悬念,自古穷通皆有定,离合岂无缘?从今分两地,各自保平安。奴去也,莫
牵连。

[乐中悲]
襁褓中,父母叹双亡。纵居那绮罗丛谁知娇养?幸生来英豪阔大宽
宏量,从未将儿女私情,略萦心上。好一似霁月光风耀玉堂。厮配得才貌仙郎,博
得个地久天长,准折得幼年时坎坷形状。终久是云散高唐,水涸湘江。这是尘寰中
消长数应当,何必枉悲伤?

[世难容]
气质美如兰,才华馥比仙。天生成孤癖人皆罕。你道是啖肉食腥膻,
视绮罗俗厌;却不知好高人愈妒,过洁世同嫌。可叹这青灯古殿人将老,孤负了红
粉朱楼春色阑,到头来依旧是风尘肮脏违心愿。好一似无瑕白玉遭泥陷,又何须王
孙公子叹无缘?

[喜冤家]
中山狼,无情兽,全不念当日根由。一味的骄奢淫荡贪欢媾。觑着
那侯门艳质同蒲柳,作践的公府千金似下流。叹芳魂艳魄,一载荡悠悠。

[虚花悟]
将那三春勘破,桃红柳绿待如何?把这韶华打灭,觅那清淡天和。
说什么天上夭桃盛,云中杏蕊多,到头来谁见把秋捱过?则看那白杨村里人呜咽,
青枫林下鬼吟哦,更兼着连天衰草遮坟墓。这的是昨贫今富人劳碌,春荣秋谢花折
磨。似这般生关死劫谁能躲?闻说道西方宝树唤婆娑,上结着长生果。

[聪明累]
机关算尽太聪明,反算了卿卿性命。生前心已碎,死后性空灵。家
富人宁,终有个家亡人散各奔腾。枉费
了意悬悬半世心,好一似荡悠悠三更梦。急喇喇似大厦倾,昏惨惨似灯将尽。呀!
一场欢喜忽悲辛,叹人世终难定!

[留馀庆]
留馀庆,留馀庆,忽遇恩人;幸娘亲,幸娘亲,积得阴功。劝人生
济困扶穷,休似俺那爱银钱忘骨肉的狠舅奸兄。正是乘除加减,上有苍穹。

[晚韶华]
镜里恩情,更那堪梦里功名!那美韶华去之何迅,再休提绣帐鸳衾。
只这戴珠冠披凤袄也抵不了无常性命。虽说是人生莫受老来贫,也须要阴骘积儿孙。
气昂昂头戴簪缨,光灿灿胸悬金印,威赫赫爵禄高登,昏惨惨黄泉路近!问古来将
相可还存?也只是虚名儿后人钦敬。

[好事终]
画梁春尽落香尘。擅风情,秉月貌,便是败家的根本。箕裘颓堕皆
从敬,家事消亡首罪宁,宿孽总因情!

[飞鸟各投林]
为官的家业雕零,富贵的金银散尽。有恩的死里逃生,无情的
分明报应。欠命的命已还,欠泪的泪已尽:冤冤相报自非轻,分离聚合皆前定。欲
知命短问前生,老来富贵也真侥幸,看破的遁入空门,痴迷的枉送了性命。好一似
食尽鸟投林,落了片白茫茫大地真干净!
歌毕,还又歌副歌。警幻见宝玉甚无趣味,因叹:“痴儿竟尚未悟!”那宝玉忙止歌
姬不必再唱,自觉朦胧恍惚,告醉求卧。

警幻便命撤去残席,送宝玉至一香闺绣阁中,其间铺陈之盛,乃素所未见之物。
更可骇者,早有一位仙姬在内,其鲜艳妩媚大似宝钗,袅娜风流又如黛玉。正不知
是何意,忽见警幻说道:“尘世中多少富贵之家,那些绿窗风月,绣阁烟霞,皆被
那些淫污纨与流荡女子玷辱了。更可恨者,自古来多少轻薄浪子,皆以‘好色不
淫’为解,又以‘情而不淫’作案,此皆饰非掩丑之语耳。好色即淫,知情更淫。
是以巫山之会,云雨之欢,皆由既悦其色、复恋其情所致。吾所爱汝者,乃天下古
今第一淫人也!”宝玉听了,唬的慌忙答道:“仙姑差了:我因懒于读书,家父母尚
每垂训饬,岂敢再冒‘淫’字?况且年纪尚幼,不知‘淫’为何事。”警幻道:“非
也。淫虽一理,意则有别。如世之好淫者,不过悦容貌,喜歌舞,调笑无厌,云雨
无时,恨不能天下之美女供我片时之趣兴:此皆皮肤滥淫之蠢物耳。如尔则天分中
生成一段痴情,吾辈推之为‘意淫’。惟‘意淫’二字,可心会而不可口传,可神
通而不能语达。汝今独得此二字,在闺阁中虽可为良友,却于世道中未免迂阔怪诡,
百口嘲谤,万目睚眦。今既遇尔祖宁荣二公剖腹深嘱,吾不忍子独为我闺阁增光而
见弃于世道。故引子前来,醉以美酒,沁以仙茗,警以妙曲。再将吾妹一人,乳名
兼美表字可卿者许配与汝,今夕良时即可成姻。不过令汝领略此仙闺幻境之风光尚
然如此,何况尘世之情景呢。从今后万万解释,改悟前情,留意于孔孟之间,委身
于经济之道。”说毕,便秘授以云雨之事,推宝玉入房中,将门掩上自去。

那宝玉恍恍惚惚,依着警幻所嘱,未免作起儿女的事来,也难以尽述。至次日,
便柔情缱绻,软语温存,与可卿难解难分。因二人携手出去游玩之时,忽然至一个
所在,但见荆榛遍地,狼虎同行,迎面一道黑溪阻路,并无桥梁可通。正在犹豫之
间,忽见警幻从后追来,说道:“快休前进,作速回头要紧!”宝玉忙止步问道:“此
系何处?”警幻道:“此乃迷津,深有万丈,遥亘千里。中无舟楫可通,只有一个
木筏,乃木居士掌柁,灰侍者撑篙,不受金银之谢,但遇有缘者渡之。尔今偶游至
此,设如坠落其中,便深负我从前谆谆警戒之语了。”话犹未了,只听迷津内响如
雷声,有许多夜叉海鬼将宝玉拖将下去。吓得宝玉汗下如雨,一面失声喊叫:“可
卿救我!”吓得袭人辈众丫鬟忙上来搂住,叫:“宝玉不怕,我们在这里呢!”

却说秦氏正在房外嘱咐小丫头们好生看着猫儿狗儿打架,忽闻宝玉在梦中唤他
的小名儿,因纳闷道:“我的小名儿这里从无人知道,他如何得知,在梦中叫出来?”

未知何因,下回分解。