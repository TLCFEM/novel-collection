\chapter{情切切良宵花解语~意绵绵静日玉生香}

话说贾妃回宫,次日见驾谢恩,并回奏归省之事。龙颜甚悦,又发内帑彩缎金
银等物以赐贾政及各椒房等员,不必细说。

且说荣宁二府中连日用尽心力,真是人人力倦,各各神疲,又将园中一应陈设
动用之物,收拾了两三天方完。第一个凤姐事多任重,别人或可偷闲躲静,独他是
不能脱得的;二则本性要强,不肯落人褒贬,只扎挣着与无事的人一样。第一个宝
玉是极无事最闲暇的。偏这一早,袭人的母亲又亲来回过贾母,接袭人家去吃年茶,
晚上才得回来。因此,宝玉只和众丫头们掷骰子赶围棋作戏。正在房内玩得没兴头,
忽见丫头们来回说:“东府里珍大爷来请过去看戏,放花灯。”宝玉听了,便命换
衣裳。才要去时,忽又有贾妃赐出糖蒸酥酪来。宝玉想上次袭人喜吃此物,便命留
与袭人了,自己回过贾母,过去看戏。

谁想贾珍这边唱的是《丁郎认父》、《黄伯央大摆阴魂阵》,更有《孙行者大
闹天宫》、《姜太公斩将封神》等类的戏文。倏尔神鬼乱出,忽又妖魔毕露。内中
扬幡过会、号佛行香、锣鼓喊叫之声,闻于巷外。弟兄子侄,互为献酬;姊妹婢妾,
共相笑语。独有宝玉见那繁华热闹到如此不堪的田地,只略坐了一坐,便走往各处
闲耍。先是进内去和尤氏并丫头姬妾鬼混了一回,便出二门来。尤氏等仍料他出来
看戏,遂也不曾照管。贾珍、贾琏、薛蟠等只顾猜谜行令,百般作乐,纵一时不见
他在座,只道在里边去了,也不理论。至于跟宝玉的小厮们,那年纪大些的,知宝
玉这一来了必是晚上才散,因此偷空儿也有会赌钱的,也有往亲友家去的,或赌或
饮,都私自散了,待晚上再来;那些小些的,都钻进戏房里瞧热闹儿去了。

宝玉见一个人没有,因想:“素日这里有个小书房内曾挂着一轴美人,画的很
得神。今日这般热闹,想那里自然无人,那美人也自然是寂寞的,须得我去望慰他
一回。”想着,便往那里来。刚到窗前,听见屋里一片喘息之声。宝玉倒唬了一跳,
心想:“美人活了不成?”乃大着胆子,舐破窗纸。向内一看,那轴美人却不曾活,
却是茗烟按着个女孩子,也干那警幻所训之事,正在得趣,故此呻吟。

宝玉禁不住,大叫“了不得”,一脚踹进门去。将两个唬的抖衣而颤。茗烟见
是宝玉,忙跪下哀求。宝玉道:“青天白日,这是怎么说!珍大爷要知道了,你是
死是活?”一面看那丫头,倒也白白净净儿的有些动人心处,在那里羞的脸红耳赤,
低首无言。宝玉跺脚道:“还不快跑!”一语提醒,那丫头飞跑去了。宝玉又赶出
去叫道:“你别怕,我不告诉人!”急的茗烟在后叫:“祖宗,这是分明告诉人了!”
宝玉因问:“那丫头十几岁了?”茗烟道:“不过十六七了。”宝玉道:“连他的
岁数也不问问,就作这个事,可见他白认得你了。可怜,可怜!”又问:“名字叫
什么?”茗烟笑道:“若说出名字来话长,真正新鲜奇文。他说他母亲养他的时节,
做了一个梦,梦得了一匹锦,上面是五色富贵不断头的‘’字花样,所以他的名
字就叫做万儿。”宝玉听了笑道:“想必他将来有些造化。等我明儿说了给你作媳
妇,好不好?”茗烟也笑了。因问:“二爷为何不看这样的好戏?”宝玉道:“看
了半日,怪烦的,出来逛逛,就遇见你们了。这会子作什么呢?”茗烟微微笑道:
“这会子没人知道,我悄悄的引二爷城外逛去,一会儿再回这里来。”宝玉道:“不
好,看仔细花子拐了去。况且他们知道了,又闹大了。不如往近些的地方去,还可
就来。”茗烟道:“就近地方谁家可去?这却难了。”宝玉笑道:“依我的主意,
咱们竟找花大姐姐去,瞧他在家作什么呢。”茗烟笑道:“好!好!倒忘了他家。”
又道:“他们知道了,说我引着二爷胡走,要打我呢。”宝玉道:“有我呢!”茗
烟听说,拉了马,二人从后门就走了。

幸而袭人家不远,不过一半里路程,转眼已到门前。茗烟先进去叫袭人之兄花
自芳。此时袭人之母接了袭人与几个外甥女儿几个侄女儿来家,正吃果茶,听见外
面有人叫“花大哥”,花自芳忙出去看时,见是他主仆两个,唬的惊疑不定,连忙
抱下宝玉来,至院内嚷道:“宝二爷来了!”别人听见还可,袭人听了,也不知为
何,忙跑出来迎着宝玉,一把拉着问:“你怎么来了?”宝玉笑道:“我怪闷的,
来瞧瞧你作什么呢。”袭人听了,才把心放下来,说道:“你也胡闹了!可作什么
来呢?”一面又问茗烟:“还有谁跟了来了?”茗烟笑道:“别人都不知道。”袭
人听了,复又惊慌道:“这还了得!倘或碰见人,或是遇见老爷,街上人挤马碰,
有个失闪,这也是玩得的吗?你们的胆子比斗还大呢!都是茗烟调唆的,等我回去告
诉嬷嬷们,一定打你个贼死。”茗烟撅了嘴道:“爷骂着打着叫我带了来的,这会
子推到我身上。我说别来罢!——要不,我们回去罢。”花自芳忙劝道:“罢了,
已经来了,也不用多说了。只是茅檐草舍,又窄又不干净,爷怎么坐呢?”

袭人的母亲也早迎出来了。袭人拉着宝玉进去。宝玉见房中三五个女孩儿,见
他进来,都低了头,羞的脸上通红。花自芳母子两个恐怕宝玉冷,又让他上炕,又
忙另摆果子,又忙倒好茶。袭人笑道:“你们不用白忙,我自然知道,不敢乱给他
东西吃的。”一面说,一面将自己的坐褥拿了来,铺在一个杌子上,扶着宝玉坐下,
又用自己的脚炉垫了脚,向荷包内取出两个梅花香饼儿来,又将自己的手炉掀开焚
上,仍盖好,放在宝玉怀里,然后将自己的茶杯斟了茶,送与宝玉。彼时他母兄已
是忙着齐齐整整的摆上一桌子果品来,袭人见总无可吃之物,因笑道:“既来了,
没有空回去的理,好歹尝一点儿,也是来我家一趟。”说着,捻了几个松瓤,吹去
细皮,用手帕托着给他。

宝玉看见袭人两眼微红,粉光融滑,因悄问袭人道:“好好的哭什么?”袭人
笑道:“谁哭来着?才迷了眼揉的。”因此便遮掩过了。因见宝玉穿着大红金蟒狐
腋箭袖,外罩石青貂裘排穗褂,说道:“你特为往这里来,又换新衣裳,他们就不
问你往那里去吗?”宝玉道:“原是珍大爷请过去看戏换的。”袭人点头,又道:
“坐一坐就回去罢,这个地方儿不是你来得的。”宝玉笑道:“你就家去才好呢,
我还替你留着好东西呢。”袭人笑道:“悄悄儿的罢!叫他们听着作什么?”一面
又伸手从宝玉项上将通灵玉摘下来,向他姊妹们笑道:“你们见识见识。时常说起
来都当稀罕,恨不能一见,今儿可尽力儿瞧瞧。再瞧什么稀罕物儿,也不过是这么
着了。”说毕递与他们,传看了一遍,仍与宝玉挂好。又命他哥哥去雇一辆干干净
净、严严紧紧的车,送宝玉回去。花自芳道:“有我送去,骑马也不妨了。”袭人
道:“不为不妨,为的是碰见人。”花自芳忙去雇了一辆车来,众人也不好相留,
只得送宝玉出去。袭人又抓些果子给茗烟,又把些钱给他买花爆放,叫他:“别告
诉人,连你也有不是。”一面说着,一直送宝玉至门前,看着上车,放下车帘。茗
烟二人牵马跟随。来至宁府街,茗烟命住车,向花自芳道:“须得我和二爷还到东
府里混一混,才过去得呢,看大家疑惑。”花自芳听说有理,忙将宝玉抱下车来,
送上马去。宝玉笑说:“倒难为你了。”于是仍进了后门来,俱不在话下。

却说宝玉自出了门,他房中这些丫鬟们都索性恣意的玩笑,也有赶围棋的,也
有掷骰抹牌的,磕了一地的瓜子皮儿,偏奶母李嬷嬷拄拐进来请安,瞧瞧宝玉;见
宝玉不在家,丫鬟们只顾玩闹,十分看不过。因叹道:“只从我出去了不大进来,
你们越发没了样儿了,别的嬷嬷越不敢说你们了。那宝玉是个‘丈八的灯台——照
见人家,照不见自己’的,只知嫌人家腌。这是他的房子,由着你们遭塌,越不
成体统了。”这些丫头们明知宝玉不讲究这些,二则李嬷嬷已是告老解事出去的了,
如今管不着他们。因此,只顾玩笑,并不理他。那李嬷嬷还只管问:“宝玉如今一
顿吃多少饭?什么时候睡觉?”丫头们总胡乱答应,有的说:“好个讨厌的老货!”

李嬷嬷又问道:“这盖碗里是酪,怎么不送给我吃?”说毕,拿起就吃。一个
丫头道:“快别动!那是说了给袭人留着的,回来又惹气了。你老人家自己承认,
别带累我们受气。”李嬷嬷听了,又气又愧,便说道:“我不信他这么坏了肠子!
别说我吃了一碗牛奶,就是再比这个值钱的,也是应该的。难道待袭人比我还重?
难道他不想想怎么长大了?我的血变了奶,吃的长这么大,如今我吃他碗牛奶,他
就生气了?我偏吃了,看他怎么着!你们看袭人不知怎么样,那是我手里调理出来的
毛丫头,什么阿物儿!”一面说,一面赌气把酪全吃了。又一个丫头笑道:“他们
不会说话,怨不得你老人家生气。宝玉还送东西给你老人家去,岂有为这个不自在
的?”李嬷嬷道:“你也不必妆狐媚子哄我,打量上次为茶撵茜雪的事我不知道呢!
明儿有了不是,我再来领。”说着,赌气去了。

少时,宝玉回来,命人去接袭人,只见晴雯躺在床上不动,宝玉因问:“可是
病了?还是输了呢?”秋纹道:“他倒是赢的;谁知李老太太来了混输了,他气的
睡去了。”宝玉笑道:“你们别和他一般见识,由他去就是了。”

说着,袭人已来,彼此相见。袭人又问宝玉何处吃饭,多早晚回来;又代母妹
问诸同伴姊妹好。一时换衣卸妆。宝玉命取酥酪来,丫鬟们回说:“李奶奶吃了。”
宝玉才要说话,袭人便忙笑说道:“原来留的是这个,多谢费心。前儿我因为好吃,
吃多了,好肚子疼,闹的吐了才好了。他吃了倒好,搁在这里白遭塌了。我只想风
干栗子吃,你替我剥栗子,我去铺炕。”宝玉听了,信以为真,方把酥酪丢开,取
了栗子来,自向灯下检剥。一面见众人不在房中,乃笑问袭人道:“今儿那个穿红
的是你什么人?”袭人道:“那是我两姨姐姐。”宝玉听了,赞叹了两声。袭人道:
“叹什么?我知道你心里的缘故。想是说:他那里配穿红的?”宝玉笑道:“不是
不是。那样的人不配穿红的,谁还敢穿?我因为见他实在好的很,怎么也得他在咱
们家就好了。”袭人冷笑道:“我一个人是奴才命罢了,难道连我的亲戚都是奴才
命不成?定还要拣实在好的丫头才往你们家来?”宝玉听了,忙笑道:“你又多心
了!我说往咱们家来,必定是奴才不成,说亲戚就使不得?”袭人道:“那也搬配
不上。”

宝玉便不肯再说,只是剥栗子。袭人笑道:“怎么不言语了?想是我才冒撞冲
犯了你?明儿赌气花几两银子买进他们来就是了。”宝玉笑道:“你说的话怎么叫
人答言呢?我不过是赞他好,正配生在这深宅大院里,没的我们这宗浊物倒生在这
里!”袭人道:“他虽没这样造化,倒也是娇生惯养的,我姨父姨娘的宝贝儿似的,
如今十七岁,各样的嫁妆都齐备了,明年就出嫁。”宝玉听了“出嫁”二字,不禁
又了两声。正不自在,又听袭人叹道:“我这几年,姊妹们都不大见。如今我要
回去了,他们又都去了!”宝玉听这话里有文章,不觉吃了一惊,忙扔下栗子,问
道:“怎么着,你如今要回去?”袭人道:“我今儿听见我妈和哥哥商量,教我再
耐一年,明年他们上来就赎出我去呢。”宝玉听了这话,越发忙了,因问:“为什
么赎你呢?”袭人道:“这话奇了!我又比不得是这里的家生子儿,我们一家子都
在别处,独我一个人在这里,怎么是个了手呢?”宝玉道:“我不叫你去也难哪!”
袭人道:“从来没这个理。就是朝廷宫里,也有定例,几年一挑,几年一放,没有
长远留下人的理,别说你们家!”

宝玉想一想,果然有理,又道:“老太太要不放你呢?”袭人道:“为什么不
放呢?我果然是个难得的,或者感动了老太太、太太不肯放我出去,再多给我们家
几两银子留下,也还有的;其实我又不过是个最平常的人,比我强的多而且多。我
从小儿跟着老太太,先伏侍了史大姑娘几年,这会子又伏侍了你几年,我们家要来
赎我,正是该叫去的,只怕连身价不要就开恩放我去呢。要说为伏侍的你好不叫我
去,断然没有的事。那伏侍的好,是分内应当的,不是什么奇功;我去了仍旧又有
好的了,不是没了我就使不得的。”宝玉听了这些话,竟是有去的理无留的理,心
里越发急了,因又道:“虽然如此说,我的一心要留下你,不怕老太太不和你母亲
说,多多给你母亲些银子,他也不好意思接你了。”袭人道:“我妈自然不敢强。
且慢说和他好说,又多给银子;就便不好和他说,一个钱也不给,安心要强留下我,
他也不敢不依。但只是咱们家从没干过这倚势仗贵霸道的事。这比不得别的东西,
因为喜欢,加十倍利弄了来给你,那卖的人不吃亏,就可以行得的;如今无故平空
留下我,于你又无益,反教我们骨肉分离,这件事,老太太、太太肯行吗?”宝玉
听了,思忖半晌,乃说道:“依你说来说去,是去定了?”袭人道:“去定了。”
宝玉听了自思道:“谁知这样一个人,这样薄情无义呢!”乃叹道:“早知道都是
要去的,我就不该弄了来。临了剩我一个孤鬼儿!”说着便赌气上床睡了。

原来袭人在家,听见他母兄要赎他回去,他就说:“至死也不回去。”又说:
“当日原是你们没饭吃,就剩了我还值几两银子,要不叫你们卖,没有个看着老子
娘饿死的理;如今幸而卖到这个地方儿,吃穿和主子一样,又不朝打暮骂。况如今
爹虽没了,你们却又整理的家成业就,复了元气。若果然还艰难,把我赎出来再多
掏摸几个钱,也还罢了,其实又不难了。这会子又赎我做什么?权当我死了,再不
必起赎我的念头了!”因此哭了一阵。他母兄见他这般坚执,自然必不出来的了。
况且原是卖倒的死契,明仗着贾宅是慈善宽厚人家儿,不过求求,只怕连身价银一
并赏了还是有的事呢;二则贾府中从不曾作践下人,只有恩多威少的,且凡老少房
中所有亲侍的女孩子们,更比待家下众人不同,平常寒薄人家的女孩儿也不能那么
尊重:因此他母子两个就死心不赎了。次后忽然宝玉去了,他两个又是那个光景儿,
母子二人心中更明白了,越发一块石头落了地,而且是意外之想,彼此放心,再无
别意了。

且说袭人自幼儿见宝玉性格异常,其淘气憨顽出于众小儿之外,更有几件千奇
百怪口不能言的毛病儿。近来仗着祖母溺爱,父母亦不能十分严紧拘管,更觉放纵
弛荡,任情恣性,最不喜务正。每欲劝时,谅不能听。今日可巧有赎身之论,故先
用骗词以探其情,以压其气,然后好下箴规。今见宝玉默默睡去,知其情有不忍,
气已馁堕。自己原不想栗子吃,只因怕为酥酪生事,又像那茜雪之茶,是以假要栗
子为由,混过宝玉不提就完了。于是命小丫头子们将栗子拿去吃了,自己来推宝玉。
只见宝玉泪痕满面,袭人便笑道:“这有什么伤心的?你果然留我,我自然不肯出
去。”宝玉见这话头儿活动了,便道:“你说说我还要怎么留你?我自己也难说了!”
袭人笑道:“咱们两个的好,是不用说了。但你要安心留我,不在这上头。我另说
出三件事来,你果然依了,那就是真心留我了,刀搁在脖子上我也不出去了。”

宝玉忙笑道:“你说那几件?我都依你。好姐姐,好亲姐姐!别说两三件,就是
两三百件我也依的。只求你们看守着我,等我有一日化成了飞灰,——飞灰还不好,
灰还有形有迹,还有知识的。——等我化成一股轻烟,风一吹就散了的时候儿,你
们也管不得我,我也顾不得你们了,凭你们爱那里去那里去就完了。”急的袭人忙
握他的嘴,道:“好爷!我正为劝你这些个。更说的狠了!”宝玉忙说道:“再不
说这话了。”袭人道:“这是头一件要改的。”宝玉道:“改了,再说你就拧嘴!
还有什么?”

袭人道:“第二件,你真爱念书也罢,假爱也罢,只在老爷跟前,或在别人跟
前,你别只管嘴里混批,只作出个爱念书的样儿来,也叫老爷少生点儿气,在人跟
前也好说嘴。老爷心里想着:我家代代念书,只从有了你,不承望不但不爱念书,
已经他心里又气又恼了,而且背前面后混批评。凡读书上进的人,你就起个外号儿,
叫人家‘禄蠹’;又说只除了什么‘明明德’外就没书了,都是前人自己混编纂出
来的。这些话你怎么怨得老爷不气,不时时刻刻的要打你呢?”宝玉笑道:“再不
说了。那是我小时候儿不知天多高地多厚信口胡说的,如今再不敢说了。还有什么
呢?”袭人道:“再不许谤僧毁道的了。还有更要紧的一件事,再不许弄花儿,弄
粉儿,偷着吃人嘴上擦的胭脂,和那个爱红的毛病儿了。”宝玉道:“都改!都改!
再有什么快说罢。”袭人道:“也没有了,只是百事检点些,不任意任性的就是了。
你要果然都依了,就拿八人轿也抬不出我去了。”宝玉笑道:“你这里长远了,不
怕没八人轿你坐。”袭人冷笑道:“这我可不希罕的。有那个福气,没有那个道理,
纵坐了也没趣儿。”

二人正说着,只见秋纹走进来,说:“三更天了,该睡了。方才老太太打发嬷
嬷来问,我答应睡了。”宝玉命取表来看时,果然针已指到子初二刻了,方从新盥
漱,宽衣安歇,不在话下。

至次日清晨,袭人起来,便觉身体发重,头疼目胀,四肢火热。先时还扎挣的
住,次后捱不住,只要睡,因而和衣躺在炕上。宝玉忙回了贾母,传医诊视,说道:
“不过偶感风寒,吃一两剂药疏散疏散就好了。”开方去后,令人取药来煎好,刚
服下去,命他盖上被窝渥汗,宝玉自去黛玉房中来看视。

彼时黛玉自在床上歇午,丫鬟们皆出去自便,满屋内静悄悄的。宝玉揭起绣线
软帘,进入里间,只见黛玉睡在那里,忙上来推他道:“好妹妹,才吃了饭,又睡
觉!”将黛玉唤醒。黛玉见是宝玉,因说道:“你且出去逛逛,我前儿闹了一夜,
今儿还没歇过来,浑身酸疼。”宝玉道:“酸疼事小,睡出来的病大,我替你解闷
儿,混过困去就好了。”黛玉只合着眼,说道:“我不困,只略歇歇儿,你且别处
去闹会子再来。”宝玉推他道:“我往那里去呢,见了别人就怪腻的。”黛玉听了,
“嗤”的一笑道:“你既要在这里,那边去老老实实的坐着,咱们说话儿。”宝玉
道:“我也歪着。”黛玉道:“你就歪着。”宝玉道:“没有枕头。咱们在一个枕
头上罢。”黛玉道:“放屁!外头不是枕头?拿一个来枕着。”宝玉出至外间,看了
一看,回来笑道:“那个我不要,也不知是那个腌老婆子的。”黛玉听了,睁开
眼,起身笑道:“真真你就是我命中的‘魔星’。请枕这一个!”说着,将自己枕
的推给宝玉,又起身将自己的再拿了一个来枕上,二人对着脸儿躺下。

黛玉一回眼,看见宝玉左边腮上有钮扣大小的一块血迹,便欠身凑近前来,以
手抚之细看道:“这又是谁的指甲划破了?”宝玉倒身,一面躲,一面笑道:“不
是划的,只怕是才刚替他们淘澄胭脂膏子溅上了一点儿。”说着,便找绢子要擦。
黛玉便用自己的绢子替他擦了,咂着嘴儿说道:“你又干这些事了。干也罢了,必
定还要带出幌子来。就是舅舅看不见,别人看见了,又当作奇怪事新鲜话儿去学舌
讨好儿,吹到舅舅耳朵里,大家又该不得心净了。”宝玉总没听见这些话,只闻见
一股幽香,却是从黛玉袖中发出,闻之令人醉魂酥骨。宝玉一把便将黛玉的衣袖拉
住,要瞧瞧笼着何物。黛玉笑道:“这时候谁带什么香呢?”宝玉笑道:“那么着,
这香是那里来的?”黛玉道:“连我也不知道,想必是柜子里头的香气熏染的,也
未可知。”宝玉摇头道:“未必。这香的气味奇怪,不是那些香饼子、香球子、香
袋儿的香。”黛玉冷笑道:“难道我也有什么‘罗汉’‘真人’给我些奇香不成?
就是得了奇香,也没有亲哥哥亲兄弟弄了花儿、朵儿、霜儿、雪儿替我炮制。我有
的是那些俗香罢了!”宝玉笑道:“凡我说一句,你就拉上这些。不给你个利害也
不知道,从今儿可不饶你了!”说着翻身起来,将两只手呵了两口,便伸向黛玉膈
肢窝内两胁下乱挠。黛玉素性触痒不禁,见宝玉两手伸来乱挠,便笑的喘不过气来。
口里说:“宝玉!你再闹,我就恼了!”

宝玉方住了手,笑问道:“你还说这些不说了?”黛玉笑道:“再不敢了。”
一面理鬓笑道:“我有奇香,你有‘暖香’没有?”宝玉见问,一时解不来,因问:
“什么‘暖香’?”黛玉点头笑叹道:“蠢才,蠢才!你有玉,人家就有金来配你;
人家有‘冷香’,你就没有‘暖香’去配他?”宝玉方听出来,因笑道:“方才告
饶,如今更说狠了!”说着又要伸手。黛玉忙笑道:“好哥哥,我可不敢了。”宝
玉笑道:“饶你不难,只把袖子我闻一闻。”说着便拉了袖子笼在面上,闻个不住。
黛玉夺了手道:“这可该去了。”宝玉笑道:“要去不能。咱们斯斯文文的躺着说
话儿。”说着复又躺下,黛玉也躺下,用绢子盖上脸。

宝玉有一搭没一搭的说些鬼话,黛玉总不理。宝玉问他几岁上京,路上见何景
致,扬州有何古迹,土俗民风如何,黛玉不答。宝玉只怕他睡出病来,便哄他道:
“嗳哟!你们扬州衙门里有一件大故事,你可知道么?”黛玉见他说的郑重,又且
正言厉色,只当是真事,因问:“什么事?”宝玉见问,便忍着笑顺口诌道:“扬
州有一座黛山,山上有个林子洞。”黛玉笑道:“这就扯谎,自来也没听见这山。”
宝玉道:“天下山水多着呢,你那里都知道?等我说完了你再批评。”黛玉道:“你
说。”

宝玉又诌道:“林子洞里原来有一群耗子精。那一年腊月初七老耗子升座议事,
说:‘明儿是腊八儿了,世上的人都熬腊八粥,如今我们洞里果品短少,须得趁此
打劫些个来才好。’乃拔令箭一枝,遣了个能干小耗子去打听。小耗子回报:‘各
处都打听了,惟有山下庙里果米最多。’老耗子便问:‘米有几样?果有几品?’
小耗子道:‘米豆成仓。果品却只有五样:一是红枣,二是栗子,三是落花生,四
是菱角,五是香芋。’老耗子听了大喜,即时拔了一枝令箭,问:‘谁去偷米?’
一个耗子便接令去偷米。又拔令箭问:‘谁去偷豆?’又一个耗子接令去偷豆。然
后一一的都各领令去了。只剩下香芋。因又拔令箭问:‘谁去偷香芋?’只见一个
极小极弱的小耗子应道:‘我愿去偷香芋。”老耗子和众耗见他这样,恐他不谙练,
又怯懦无力,不准他去。小耗子道:‘我虽年小身弱,却是法术无边,口齿伶俐,
机谋深远。这一去,管比他们偷的还巧呢!’众耗子忙问:‘怎么比他们巧呢?’
小耗子道:‘我不学他们直偷,我只摇身一变,也变成个香芋,滚在香芋堆里,叫
人瞧不出来,却暗暗儿的搬运,渐渐的就搬运尽了:这不比直偷硬取的巧吗?’众
耗子听了,都说:‘妙却妙,只是不知怎么变?你先变个我们瞧瞧。’小耗子听了,
笑道:‘这个不难,等我变来。’说毕,摇身说:‘变。’竟变了一个最标致美貌
的一位小姐。众耗子忙笑说:‘错了,错了!原说变果子,怎么变出个小姐来了呢?’
小耗子现了形笑道:‘我说你们没见世面,只认得这果子是香芋,却不知盐课林老
爷的小姐才是真正的“香玉”呢!’”

黛玉听了,翻身爬起来,按着宝玉笑道:“我把你这个烂了嘴的!我就知道你
是编派我呢。”说着便拧。宝玉连连央告:“好妹妹,饶了我罢,再不敢了。我因
为闻见你的香气,忽然想起这个故典来。”黛玉笑道:“饶骂了人,你还说是故典
呢。”

一语未了,只见宝钗走来,笑问:“谁说故典呢?我也听听。”黛玉忙让坐,
笑道:“你瞧瞧,还有谁?他饶骂了,还说是故典。”宝钗笑道:“哦!是宝兄弟哟!
怪不得他。他肚子里的故典本来多么!就只是可惜一件,该用故典的时候儿他就偏
忘了。有今儿记得的,前儿夜里的芭蕉诗就该记得呀,眼面前儿的倒想不起来。别
人冷的了不得,他只是出汗。这会子偏又有了记性了!”黛玉听了笑道:“阿弥陀
佛!到底是我的好姐姐。你一般也遇见对子了。可知一还一报,不爽不错的。”刚
说到这里,只听宝玉房中一片声吵嚷起来。

未知何事,下回分解。