\chapter{嫌隙人有心生嫌隙~鸳鸯女无意遇鸳鸯}

话说贾母处两个丫头,匆匆忙忙来找宝玉,口里说道:“二爷快跟着我们走罢,
老爷家来了。”宝玉听了,又喜又愁,只得忙忙换了衣服,前来请安。贾政正在贾
母房中,连衣服未换,看见宝玉进来请安,心中自是喜欢,却又有些伤感之意。又
叙了些任上的事情,贾母便说:“你也乏了,歇歇去罢。”贾政忙站起来,笑着答
应了个“是”,又略站着说了几句话,才退出来。宝玉等也都跟过来。贾政自然问
问他的工课,也就散了。

原来贾政回京复命,因是学差,故不敢先到家中。珍、琏、宝玉头一天便迎出
一站去;接见了,贾政先请了贾母的安,便命都回家伺候。次日面圣,诸事完毕,
才回家来。又蒙恩赐假一月,在家歇息。因年景渐老,事重身衰,又近因在外几年,
骨肉离异,今得宴然复聚,自觉喜幸不尽。一应大小事务,一概亦付之度外,只是
看书;闷了便与清客们下棋吃酒,或日间在里边,母子夫妻,共叙天伦之乐。

因今岁八月初三日乃贾母八旬大庆,又因亲友全来,恐筵宴排设不开,便早同
贾赦及贾琏等商议,议定于七月二十八日起至八月初五日止,宁荣两处齐开筵宴。
宁国府中单请官客,荣国府中单请堂客。大观园中收拾出缀锦阁并嘉荫堂等几处大
地方来做退居。二十八日,请皇亲、驸马、王公、诸王、郡主、王妃、公主、国君、
太君、夫人等;二十九日,便是阁府督镇及诰命等;三十日,便是诸官长及诰命并
远近亲友及堂客。初一日,是贾赦的家宴;初二日,是贾政;初三日,是贾珍贾琏;
初四日,是贾府中合族长幼大小共凑家宴;初五日,是赖大林之孝等家下管事人等
共凑一日。

自七月上旬,送寿礼者便络绎不绝。礼部奉旨:钦赐金玉如意一柄,彩缎四端,
金玉杯各四件,帑银五百两。元春又命太监送出金寿星一尊,沉香拐一支,伽楠珠
一串,福寿香一盒,金锭一对,银锭四对,彩缎十二匹,玉杯四只。馀者自亲王驸
马以及大小文武官员家,凡所来往者,莫不有礼,不能胜记。堂屋内设下大桌案,
铺了红毡,将凡有精细之物都摆上,请贾母过目。先一二日,还高兴过来瞧瞧,后
来烦了,也不过目,只说:“叫凤丫头收了,改日闷了再瞧。”

至二十八日,两府中俱悬灯结彩,屏开鸾凤,褥设芙蓉,笙箫鼓乐之音,通衢
越巷。宁府中,本日只有北静王、南安郡王、永昌驸马、乐善郡王并几位世交公侯
荫袭;荣府中,南安王太妃、北静王妃并世交公侯诰命。贾母等皆是按品大妆迎接。
大家厮见,先请至大观园内嘉荫堂,茶毕更衣,方出至荣庆堂上拜寿入席。大家谦
逊半日,方才入座。上面两席是南北王妃,下面依序便是众公侯命妇。左边下手一
席,陪客是锦乡侯诰命与临昌伯诰命;右边下手方是贾母主位。邢夫人王夫人带领
尤氏凤姐并族中几个媳妇,两溜雁翅站在贾母身后侍立。林之孝赖大家的带领众媳
妇,都在竹帘外面,伺候上菜上酒。周瑞家的带领几个丫鬟,在围屏后伺候呼唤。
凡跟来的人,早又有人款待,别处去了。

一时参了场,台下一色十二个未留发的小丫头,都是小厮打扮,垂手伺候。须
臾,一个捧了戏单至阶下,先递给回事的媳妇,这媳妇接了,才递给林之孝家的。
林之孝家的用小茶盘托上,挨身入帘来,递给尤氏的侍妾佩凤,佩凤接了才奉与尤
氏,尤氏托着走至上席。南安太妃谦让了一回,点了一出吉庆戏文,然后又让北静
王妃,也点了一出。众人又让了一回,命随便拣好的唱罢了。

少时,菜已四献,汤始一道,跟来各家的放了赏,大家便更衣服入园来,另献
好茶。南安太妃因问宝玉。贾母笑道:“今日几处庙里念保安延寿经,他跪经去了。”
又问众小姐们。贾母笑道:“他们姊妹们病的病,弱的弱,见人腼,所以叫他们
给我看屋子去了。有的是小戏子传了一班在那边厅上,陪着他姨娘家姊妹们也看戏
呢。”南安太妃笑道:“既这样,叫人请来。”贾母回头命了凤姐儿,“去把史、
薛、林四位姑娘带来。再只叫你三妹妹陪着来罢。”凤姐答应了,来至贾母这边,
只见他姊妹们正吃果子看戏,宝玉也才从庙里跪经回来。凤姐说了,宝钗姊妹与黛
玉湘云五人来至园中,见了大众,俱请安问好。内中也有见过的,还有一两家不曾
见过的,都齐声夸赞不绝。其中湘云最熟,南安太妃因笑道:“你在这里,听见我
来了还不出来,还等请去!我明儿和你叔叔算帐。”因一手拉着探春,一手拉着宝
钗,问:“十几岁了?”又连声夸赞,因又松了他两个,又拉着黛玉宝琴,也着实
细着,极夸一回,又笑道:“都是好的!不知叫我夸那一个的是。”早有人将备用
礼物打点出几分来:金玉戒指各五个,腕香珠五串。南安太妃笑道:“你姊妹们别
笑话,留着赏丫头们罢。”五人忙拜谢过。北静王妃也有五样礼物。馀者不必细说。

吃了茶,园中略逛了一逛,贾母等因又让入席。南安太妃便告辞,说:“身上
不快。今日若不来,实在使不得。因此,恕我竟先要告别了。”贾母等听说,也不
便强留,大家又让了一回,送至园门,坐轿而去。接着北静王妃略坐了一坐,也就
告辞了。馀者也有终席的,也有不终席的。贾母劳乏了一日,次日便不见人,一应
都是邢夫人款待。有那些世家子弟拜寿的,只到厅上行礼,贾赦、贾政、贾珍还礼,
看待至宁府坐席,不在话下。

这几日尤氏晚间也不回那府去,白日间待客,晚上陪贾母玩笑,又帮着凤姐料
理出入大小器皿以及收放礼物。晚上往园内李氏房中歇宿。这日伏侍过贾母晚饭
后,贾母因说:“你们乏了,我也乏了,早些找点子什么吃了,歇歇去罢。明儿还
要起早呢。”尤氏答应着,退出去,到凤姐儿屋里来吃饭。凤姐儿正在楼上看着人
收送来的围屏呢,只有平儿在屋里,给凤姐叠衣服。尤氏想起二姐儿在时多承平儿
照应,便点着头儿,说道:“好丫头,你这么个好心人,难为在这里熬。”平儿把
眼圈儿一红,忙拿话岔过去了。尤氏因笑问道:“你们奶奶吃了饭了没有?”平儿
笑道:“吃饭么还不请奶奶去?”尤氏笑道:“既这么着,我别处找吃的去罢,饿
的我受不得了。”说着就走。平儿忙笑道:“奶奶请回来,这里有饽饽,且点补些
儿,回来再吃饭。”尤氏笑道:“你们忙忙的,我园里和他姐儿们闹去。”一面说
一面走,平儿留不住,只得罢了。

且说尤氏一径来至园中,只见园中正门和各处角门仍未关好,犹吊着各色彩
灯,因回头命小丫头叫该班的女人。那丫鬟走入班房中,竟没一个人影,回来回了
尤氏。尤氏便命传管家的女人。这丫头应了便出去,到二门外鹿顶内,乃是管事的
女人议事取齐之所。到了这里,只有两个婆子分果菜吃。因问:“那一位管事的奶
奶在这里?东府里的奶奶立等一位奶奶,有话吩咐。”这两个婆子只顾分菜果,又
听见是东府里的奶奶,不大在心上,因就回说:“管家奶奶们才散了。”小丫头道:
“既散了,你们家里传他去。”婆子道:“我们只管看屋子,不管传人,姑娘要传
人,再派传人的去。”小丫头听了道:“嗳哟!这可反了!怎么你们不传去?你哄新
来的,怎么哄起我来了。素日你们不传,谁传去?这会子打听了体己信儿,或是赏
了那位管家奶奶的东西,你们争着狗颠屁股儿的传去,不知谁是谁呢!琏二奶奶要
传,你们也敢这么回吗?”这婆子一则吃了酒,二则被这丫头揭着弊病,便羞恼成
怒了,因回口道:“扯你的臊!我们的事传不传,不与你相干。你未从揭挑我们,
你想想你那老子娘,在那边管家爷们跟前,比我们还更会溜呢。各门各户的,你有
本事排揎你们那边的人去!我们这边,你离着还远些呢。”丫头听了,气白了脸,
因说道:“好好,这话说的好!”一面转身进来回话。

尤氏已早进园来,因遇见了袭人、宝琴、湘云三人,同着地藏庵的两个姑子正
说故事玩笑。尤氏因说饿了,先到怡红院,袭人装了几样荤素点心出来给尤氏吃。
那小丫头子一径找了来,气狠狠的把方才的话都说了。尤氏听了,半晌冷笑道:“这
是两个什么人?”两个姑子笑推这丫头道:“你这姑娘好气性大,那糊涂老妈妈们
的话,你也不该来回才是。咱们奶奶万金之体,劳乏了几日,黄汤辣水没吃,咱们
只有哄他欢喜的,说这些话做什么?”袭人也忙笑拉他出去,说:“好妹子,你且
出去歇歇,我打发人叫他们去。”尤氏道:“你不用叫人,你去就叫这两个老婆来,
到那边把他们家的凤姐叫来。”袭人笑道:“我请去。”尤氏笑道:“偏不用你。”
两个姑子忙立起身来笑说:“奶奶素日宽洪大量,今日老祖宗千秋,奶奶生气,岂
不惹人议论?”宝琴湘云二人也都笑劝。尤氏道:“不为老太太的千秋,我一定不
依。且放着就是了。”

说话之间,袭人早又遣了一个丫头去到园门外找人。可巧遇见周瑞家的,这小
丫头子就把这话告诉他了。周瑞家的虽不管事,因他素日仗着王夫人的陪房,原有
些体面,心性乖滑,专惯各处献勤讨好,所以各房主子都喜欢他。他今日听了这话,
忙跑入怡红院,一面飞走,一面说:“可了不得,气坏了奶奶了。偏我不在跟前。
且打他们几个耳刮子,再等过了这几天算帐!”尤氏见了他,也便笑道:“周姐姐
你来,有个理你说说:这早晚园门还大开着,明灯蜡烛,出入的人又杂,倘有不防
的事,如何使得。因此,叫该班的人吹灯关门。谁知一个人牙儿也没有!”周瑞家
的道:“这还了得!前儿二奶奶还吩咐过的,今儿就没了人。过了这几日,必要打
几个才好。”尤氏又说小丫头子的话。周瑞家的说:“奶奶不用生气。等过了事,
我告诉管事的,打他个贼死,只问他们谁说‘各门各户’的话。我已经叫他们吹灯
关门呢。奶奶也别生气了。”正乱着,只见凤姐儿打发人来请吃饭。尤氏道:“我
也不饿了,才吃了几个饽饽,请你奶奶自己吃罢。”

一时,周瑞家的出去,便把方才之事回了凤姐。凤姐便命:“将那两个的名字
记上,等过了这几日,捆了送到那府里,凭大奶奶开发。或是打,或是开恩,随他
就完了。什么大事!”周瑞家的听了,巴不得一声,素日因与这几个人不睦,出来
了便命一个小厮到林之孝家去传凤姐的话,立刻叫林之孝家的进来见大奶奶;一面
又传人立刻捆起这两个婆子来,交到马圈里,派人看守。林之孝家的不知甚么事,
忙坐车进来,先见凤姐。至二门上,传进话去,丫头们出来说:“奶奶才歇下了。
大奶奶在园内,叫大娘见见大奶奶就是了。”林之孝家的只得进园来,到稻香村。
丫鬟们回进去。尤氏听了,反过不去,忙唤进他来,因笑向他道:“我不过为找人
找不着,因问你;你既去了,也不是什么大事,谁又把你叫进来?倒叫你白跑一趟。
不大的事,已经撂过手了。”林之孝家的也笑回道:“二奶奶打发人传我,说奶奶
有话吩咐。”尤氏道:“大约周姐姐说的。你家去歇着罢,没有什么大事。”李纨
又要说原故,尤氏反拦住了。林之孝家的见如此,只得便回身出园去。可巧遇见赵
姨娘,因笑说:“嗳哟哟!我的嫂子!这会子还不家去歇歇,跑什么?”林之孝家的
便笑说:“何曾没家去?”如此这般,“进来了。”赵姨娘便说:“这事也值一个
屁!开恩呢,就不理论;心窄些儿,也不过打几下就完了,也值的叫你进来!你快歇
歇去,我也不留你喝茶了。”

说毕,林之孝家的出来。到了侧门前,就有才两个婆子的女儿上来哭着求情。
林之孝家的笑道:“你这孩子好糊涂!谁叫他好喝酒、混说话?惹出事来,连我也不
知道。二奶奶打发人捆他,连我还有不是呢,我替谁讨情去?”这两个小丫头子才
十来岁,原不识事,只管啼哭求告。缠的林之孝家的没法,因说道:“糊涂东西,
你放着门路不去求,尽着缠我。你姐姐现给了那边大太太的陪房费大娘的儿子,你
过去告诉你姐姐,叫亲家娘和太太一说,什么完不了的?”一语提醒了这一个,那
一个还求。林之孝家的啐道:“糊涂攮的!他过去一说,自然都完了。没有单放他
妈、又打你妈的理。”说毕上车去了。

这一个小丫头子,果然过来告诉了他姐姐,和费婆子说了。这费婆子原是个大
不安静的,便隔墙大骂一阵,走了来求邢夫人,说他亲家“与大奶奶的小丫头白斗
了两句话,周瑞家的挑唆了二奶奶,现捆在马圈里,等过两日还要打呢。求太太和
二奶奶说声,饶他一次罢。”邢夫人自为要鸳鸯讨了没意思,贾母冷淡了他;且前
日南安太妃来,贾母又单令探春出来,自己心内早已怨忿。又有在侧一干小人,心
内嫉妒,挟怨凤姐,便调唆的邢夫人着实憎恶凤姐。如今又听了如此一篇话,也不
说长短。

至次日一早,见过贾母。众族人到齐,开戏。贾母高兴,又今日都是自己族中
子侄辈,只便妆出来堂上受礼。当中独设一榻,引枕、靠背、脚踏俱全,自己歪在
榻上。榻之前后左右,皆是一色的矮凳。宝钗、宝琴、黛玉、湘云、迎春、探春、
惜春姊妹等围绕。因贾之母也带了女儿喜鸾,贾琼之母也带了女儿四姐儿,还有
几房的孙女儿,大小共有二十来个,贾母独见喜鸾四姐儿生得又好,说话行事与众
不同,心中欢喜,便叫他两个也坐在榻前。宝玉却在榻上,与贾母捶腿。首席便是
薛姨妈,下边两溜顺着房头辈数下去。帘外两廊,都是族中男客,也依次而坐。先
是那女客一起一起行礼,后是男客行礼。贾母歪在榻上,只命人说:“免了罢。”
然后赖大等带领众家人,从仪门直跪至大厅上磕头。礼毕,又是众家下媳妇。然后
各房丫鬟。足闹了两三顿饭时。然后又抬了许多雀笼来,在当院中放了生。贾赦等
焚过天地寿星纸,方开戏饮酒。直到歇了中台,贾母方进来歇息,命他们取便,因
命凤姐儿留下喜鸾四姐儿玩两日再去。凤姐儿出来,便和他母亲说。他两个母亲素
日承凤姐的照顾,愿意在园内玩笑,至晚便不回去了。

邢夫人直至晚间散时,当着众人,陪笑和凤姐求情说:“我昨日晚上听见二奶
奶生气,打发周管家的奶奶儿捆了两个老婆,可也不知犯了什么罪?论理我不该讨
情,我想老太太好日子,发狠的还要舍钱舍米,周贫济老,咱们先倒挫磨起老奴才
来了?就不看我的脸,权且看老太太,暂且竟放了他们罢。”说毕,上车去了。凤
姐听了这话,又当着众人,又羞又气,一时找寻不着头脑,别的脸紫胀,回头向赖
大家的等冷笑道:“这是那里的话?昨儿因为这里的人得罪了那府里大奶奶,我怕
大奶奶多心,所以尽让他发放,并不为得罪了我。这又是谁的耳报神这么快?”王
夫人因问:“为什么事?”凤姐儿笑将昨日的事说了。尤氏也笑道:“连我并不知
道,你原也太多事了。”凤姐儿道:“我为你脸上过不去,所以等你开发,不过是
个礼。就如我在你那里,有人得罪了我,你自然送了来尽我。凭他是什么好奴才,
到底错不过这个礼去。这又不知谁过去,没的献勤儿,这也当作一件事情来说。”
王夫人道:“你太太说的是。就是你珍大嫂子也不是外人,也不用这些虚礼。老太
太的千秋要紧,放了他们为是。”说着,回头便命人去放了那两个婆子。凤姐由不
得越想越气越愧,不觉的一阵心灰,落下泪来。因赌气回房哭泣,又不使人知觉;
偏是贾母打发了琥珀来叫,立等说话。琥珀见了,诧异道:“好好的,这是什么原
故?那里立等你呢。”凤姐听了,忙擦干了泪,洗面另施了脂粉,方同琥珀过来。

贾母因问道:“前儿这些人家送礼来的,共有几家有围屏?”凤姐儿道:“共
有十六家。有十二架大的,四架小的炕屏。内中只有甄家一架大屏,十二扇大红缎
子刻丝‘满床笏’、一面泥金‘百寿图’的是头等。还有海将军邬家的一架玻璃
的还罢了。”贾母道:“既这么样,这两架别动,好生搁着,我要送人的。”凤姐
答应了。鸳鸯忽过来,向凤姐脸上细瞧。引的贾母问说:“你不认得他?只管瞧什
么?”鸳鸯笑道:“我看他的眼肿肿的,所以我诧异。”贾母便叫“过来”,也细
细的看。凤姐笑道:“才觉的发痒,揉肿了些。”鸳鸯笑道:“别又是受了谁的气
了罢。”凤姐笑道:“谁敢给我气受?就受了气,老太太好日子,我也不敢哭啊。”
贾母道:“正是呢。我正要吃饭,你在这里打发我吃,剩下的,你和珍儿媳妇吃了。
你们两个在这里帮着师父们替我拣佛头儿,你们也积积寿。前儿你妹妹们和宝玉都
拣了,如今也叫你们拣拣,别说我偏心。”说话时先摆上一桌素馔来,两个姑子吃。
然后摆上荤的,贾母吃毕,抬出外间。尤氏凤姐二人正吃着,贾母又叫把喜鸾四姐
儿二人叫来,跟他二人吃毕,洗了手,点上香,捧上一升豆子来,两个姑子先念了
佛偈,然后一个一个的拣在一个笸箩内,明日煮熟了,令人在十字街结寿缘。贾母
歪着,听两个姑子说些因果。

鸳鸯早已听见琥珀说凤姐哭之一事,又和平儿前打听得原故,晚间人散时,便
回说:“二奶奶还是哭的,那边大太太当着人给二奶奶没脸。”贾母因问:“为什
么原故?”鸳鸯便将原故说了。贾母道:“这才是凤丫头知礼处。难道为我的生日,
由着奴才们把一族中的主子都得罪了,也不管罢?这是大太太素日没好气,不敢发
作,所以今儿拿着这个作法,明是当着众人给凤姐儿没脸罢了。”正说着,只见宝
琴来了,也就不说了。

贾母忽想起留下的喜姐儿四姐儿,叫人吩咐园中婆子们:“要和家里的姑娘一
样照应。倘有人小看了他们,我听见可不饶。”婆子答应了,方要走时,鸳鸯道:
“我说去罢。他们那里听他的话?”说着,便一径往园里来。先到稻香村中,李纨
与尤氏都不在这里。问丫鬟们,都说:“在三姑娘那里呢。”鸳鸯回身,又来至晓
翠堂,果见那园中人都在那里说笑。见他来了,都笑说:“你这会子又跑到这里做
什么?”又让他坐。鸳鸯笑道:“不许我逛逛么?”于是把方才的话说了一遍。李
纨忙起身听了,即刻就叫人把各处的头儿唤了一个来,令他们传与诸人知道,不在
话下。这里尤氏笑道:“老太太也太想的到。实在我们年轻力壮的人,捆上十个也
赶不上。”李纨道:“凤丫头仗着鬼聪明,还离脚踪儿不远,咱们是不能的了。”
鸳鸯道:“罢哟,还提‘凤丫头’‘虎丫头’呢。他的为人,也可怜见儿的。虽然
这几年没有在老太太、太太跟前有个错缝儿,暗里也不知得罪了多少人。总而言之,
为人是难做的:若太老实了,没有个机变,公婆又嫌太老实了,家里人也不怕;若
有些机变,未免又‘治一经损一经’。如今咱们家更好,新出来的这些底下字号的
奶奶们,一个个心满意足,都不知道要怎么样才好,少不得意,不是背地里嚼舌根,
就是调三窝四的。我怕老太太生气,一点儿也不肯说,不然我告诉出来,大家别过
太平日子。这不是我当着三姑娘说:老太太偏疼宝玉,有人背地怨言还罢了,算是
偏心;如今老太太偏疼你,我听着也是不好。这可笑不可笑?”探春笑道:“糊涂
人多,那里较量得许多?我说倒不如小户人家,虽然寒素些,倒是天天娘儿们欢天
喜地,大家快乐。我们这样人家,人都看着我们不知千金万金、何等快乐,殊不知
这里说不出来的烦难,更利害!”

宝玉道:“谁都像三妹妹多心多事?我常劝你总别听那些俗语、想那些俗事,
只管安富尊荣才是,比不得我们,没这清福,应该混闹的。”尤氏道:“谁都像你
是一心无挂碍,只知道和姊妹们玩笑,饿了吃,困了睡,再过几年,不过是这样,
一点后事也不虑。”宝玉笑道:“我能够和姊妹们过一日,是一日,死了就完了,
什么后事不后事。”李纨等都笑道:“这可又是胡说了。就算你是个没出息的,终
老在这里,难道他姐儿们都不出门子罢?”尤氏笑道:“怨不得都说你空长了个好
胎子,真真是个傻东西。”宝玉笑道:“人事难定,谁死谁活?倘或我在今日明日、
今年明年死了,也算是随心一辈子了。”众人不等说完,便说:“越发胡说了!别
和他说话才好。要和他说话,不是呆话,就是疯话。”喜鸾因笑道:“二哥哥,你
别这么说,等这里姐姐们果然都出了门,横竖老太太、太太也闷的慌,我来和你作
伴儿。”李纨尤氏都笑道:“姑娘也别说呆话。难道你是不出门子的吗?”一句说
的喜鸾也臊了,低了头。当下已起更时分,大家各自归房安歇,不提。

且说鸳鸯一径回来,刚至园门前,只见角门虚掩,犹未上闩。此时园内无人来
往,只有班儿房子里灯光掩映,微月半天。鸳鸯又不曾有伴,也不曾提灯,独自一
个,脚步又轻,所以该班的人皆不理会。偏要小解,因下了甬路,找微草处走动,
行至一块湘山石后大桂树底下来。刚转至石边,只听一阵衣衫响,吓了一惊不小。
定睛看时,只见是两个人在那里,见他来了,便想往树丛石后藏躲。鸳鸯眼尖,趁
着半明的月色,早看见一个穿红袄儿、梳头、高大丰壮身材的,是迎春房里司棋。
鸳鸯只当他和别的女孩子也在此方便,见自己来了,故意藏躲,吓着玩耍,因便笑
叫道:“司棋!你不快出来,吓着我,我就喊起来,当贼拿了。这么大丫头,也没
个黑家白日,只是玩不够。”这本是鸳鸯戏语,叫他出来。谁知他贼人胆虚,只当
鸳鸯已看见他的首尾了,生恐叫喊出来,使众人知觉,更不好;且素日鸳鸯又和自
己亲厚,不比别人:便从树后跑出来,一把拉住鸳鸯,便双膝跪下,只说:“好姐
姐!千万别嚷!”

鸳鸯反不知他为什么,忙拉他起来,问道:“这是怎么说?”司棋只不言语,
浑身乱颤。鸳鸯越发不解。再瞧了一瞧,又有一个人影儿,恍惚像是个小厮,心下
便猜着了八九分,自己反羞的心跳耳热,又怕起来。因定了一会,忙悄问:“那一
个是谁?”司棋又跪下道:“是我姑舅哥哥。”鸳鸯啐了一口,却羞的一句话也说
不出来。司棋又回头悄叫道:“你不用藏着,姐姐已经看见了。快出来磕头。”那
小厮听了,只得也从树后跑出来,磕头如捣蒜。鸳鸯忙要回身,司棋拉住苦求,哭
道:“我们的性命都在姐姐身上,只求姐姐超生我们罢了!”鸳鸯道:“你不用多
说了,快叫他去罢。横竖我不告诉人就是了。你这是怎么说呢!”一语未了,只听
角门上有人说道:“金姑娘已经出去了,角门上锁罢。”鸳鸯正被司棋拉住,不得
脱身,听见如此说,便忙着接声道:“我在这里有事,且略等等儿我出来了。”司
棋听了,只得松手,让他去了。

要知端底,下回分解。