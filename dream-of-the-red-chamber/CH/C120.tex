\chapter{甄士隐详说太虚情~贾雨村归结红楼梦}

话说宝钗听秋纹说袭人不好,连忙进去瞧看,巧姐儿同平儿也随着。走到袭人
炕前,只见袭人心痛难禁,一时气厥。宝钗等用开水灌了过来,仍旧扶他睡下,一
面传请大夫。巧姐儿因问宝钗道:“袭人姐姐怎么病到这个样儿?”宝钗道:“大
前儿晚上哭伤了心了,一时发晕栽倒了。太太叫人扶他回来,他就睡倒了。因外头
有事,没有请大夫瞧他,所以致此。”说着,大夫来了,宝钗等略避。大夫看了脉,
说是急怒所致,开了方子去了。

原来袭人模糊听见说宝玉若不回来,便要打发屋里的人都出去,一急越发不好
了。到大夫瞧后,秋纹给他煎药,他各自一人躺着,神魂未定。好像宝玉在他面前,
恍惚又像是见个和尚,手里拿着一本册子揭着看,还说道:“你不是我的人,日后
自然有人家儿的。”袭人似要和他说话,秋纹走来说:“药好了,姐姐吃罢。”袭
人睁眼一瞧,知是个梦,也不告诉人。吃了药,便自己细细的想:“宝玉必是跟了
和尚去。上回他要拿玉出去,便是要脱身的样子。被我揪住,看他竟不像往常,把
我混推混搡的,一点情意都没有。后来待二奶奶更生厌烦,在别的姊妹跟前,也是
没有一点情意:这就是悟道的样子。但是你悟了道,抛了二奶奶怎么好?我是太太
派我服侍你,虽是月钱照着那样的分例,其实我究竟没有在老爷太太跟前回明,就
算了你的屋里人。若是老爷太太打发我出去,我若死守着,又叫人笑话;若是我出
去,心想宝玉待我的情分,实在不忍。”左思右想,万分难处。想到刚才的梦,“说
我是别人的人,那倒不如死了干净。”岂知吃药以后,心痛减了好些,也难躺着,
只好勉强支持。过了几日,起来服侍宝钗。宝钗想念宝玉,暗中垂泪,自叹命苦。
又知他母亲打算给哥哥赎罪,很费张罗,不能不帮着打算。暂且不表。

且说贾政扶贾母灵柩,贾蓉送了秦氏、凤姐、鸳鸯的棺木到了金陵,先安了葬。
贾蓉自送黛玉的灵,也去安葬。贾政料理坟墓的事。一日,接到家书,一行一行的
看到宝玉贾兰得中,心里自是喜欢;后来看到宝玉走失,复又烦恼。只得赶忙回来。
在道儿上又闻得有恩赦的旨意,又接着家书,果然赦罪复职,更是喜欢,便日夜趱
行。

一日,行到陵驿地方,那天乍寒,下雪,泊在一个清静去处。贾政打发众人
上岸投帖辞谢朋友,总说即刻开船,都不敢劳动。船上只留一个小厮伺候,自己在
船中写家书,先要打发人起早到家。写到宝玉的事,便停笔。抬头忽见船头上微微
的雪影里面一个人,光着头,赤着脚,身上披着一领大红猩猩毡的斗篷,向贾政倒
身下拜。贾政尚未认清,急忙出船,欲待扶住问他是谁。那人已拜了四拜,站起来
打了个问讯。贾政才要还揖,迎面一看,不是别人,却是宝玉。贾政吃一大惊,忙
问道:“可是宝玉么?”那人只不言语,似喜似悲。贾政又问道:“你若是宝玉,
如何这样打扮,跑到这里来?”宝玉未及回言,只见船头上来了两人,一僧一道,
夹住宝玉道:“俗缘已毕,还不快走。”说着,三个人飘然登岸而去。贾政不顾地
滑,疾忙来赶,见那三人在前,那里赶得上?只听得他们三人口中不知是那个作歌
曰:
我所居兮青埂之峰,我所游兮鸿蒙太空。
谁与我逝兮吾谁与从?渺渺茫茫兮归彼大荒!
贾政一面听着,一面赶去,转过一小坡,倏然不见。贾政已赶得心虚气喘,惊疑不
定。回过头来,见自己的小厮也随后赶来,贾政问道:“你看见方才那三个人么?”
小厮道:“看见的。奴才为老爷追赶,故也赶来。后来只见老爷,不见那三个人了。”
贾政还欲前走,只见白茫茫一片旷野,并无一人。贾政知是古怪,只得回来。

众家人回船,见贾政不在舱中,问了船夫,说是老爷上岸追赶两个和尚一个道
士去了。众人也从雪地里寻踪迎去,远远见贾政来了,迎上去接着,一同回船。贾
政坐下,喘息方定,将见宝玉的话说了一遍。众人回禀,便要在这地方寻觅。贾政
叹道:“你们不知道,这是我亲眼见的,并非鬼怪。况听得歌声,大有玄妙。宝玉
生下时,衔了玉来,便也古怪,我早知是不祥之兆,为的是老太太疼爱,所以养育
到今。便是那和尚道士,我也见了三次:头一次,是那僧道来说玉的好处;第二次,
便是宝玉病重,他来了,将那玉持诵了一番,宝玉便好了;第三次,送那玉来,坐
在前厅,我一转眼就不见了。我心里便有些诧异,只道宝玉果真有造化,高僧仙道
来护佑他的。岂知宝玉是下凡历劫的,竟哄了老太太十九年!如今叫我才明白。”
说到那里,掉下泪来。众人道:“宝二爷果然是下凡的和尚,就不该中举人了。怎
么中了才去?”贾政道:“你们那里知道?大凡天上星宿,山中老僧,洞里的精灵,
他自具一种性情。你看宝玉何尝肯念书?他若略一经心,无有不能的。他那一种脾
气,也是各别另样。”说着又叹了几声。众人便拿兰哥得中、家道复兴的话解了一
番。贾政仍旧写家书,便把这事写上,劝谕合家不必想念了。写完封好,即着家人
回去,贾政随后赶回。暂且不提。

且说薛姨妈得了赦罪的信,便命薛蝌去各处借贷,并自己凑齐了赎罪银两。刑
部准了,收兑了银子,一角文书,将薛蟠放出。他们母子姊妹弟兄见面,不必细述,
自然是悲喜交集了。薛蟠自己立誓说道:“若是再犯前病,必定犯杀犯剐!”薛姨
妈见他这样,便握他的嘴,说:“只要自己拿定主意,必定还要妄口巴舌血淋淋的
起这样恶誓么?只是香菱跟你受了多少苦处,你媳妇儿已经自己治死自己了,如今
虽说穷了,这碗饭还有得吃:据我的主意,我便算他是媳妇了。你心里怎么样?”
薛蟠点头愿意。宝钗等也说:“很该这样。”倒把香菱急得脸胀通红,说是:“伏
侍大爷一样的,何必如此?”众人便称起“大奶奶”来,无人不服。

薛蟠便要去拜谢贾家。薛姨妈宝钗也都过来。见了众人,彼此聚首,又说了一
番的话。正说着,恰好那日贾政的家人回家,呈上书子,说:“老爷不日到了。”
王夫人叫贾兰将书子念给听。贾兰念到贾政亲见宝玉的一段,众人听了,都痛哭起
来,王夫人、宝钗、袭人等更甚。大家又将贾政书内叫家内不必悲伤,原是借胎的
话解说了一番:“与其作了官,倘或命运不好,犯了事,坏家败产,那时倒不好了。
宁可咱们家出一位佛爷,倒是老爷太太的积德,所以才投到咱们家来。不是说句不
顾前后的话:当初东府里太爷,倒是修炼了十几年,也没有成了仙,这佛是更难成
的。太太这么一想,心里便开豁了。”王夫人哭着和薛姨妈道:“宝玉抛了我,我
还恨他呢。我叹的是媳妇的命苦,才成了一二年的亲,怎么他就硬着肠子,都撂下
了走了呢!”薛姨妈听了,也甚伤心。

宝钗哭得人事不知。所有爷们都在外头。王夫人便说道:“我为他担了一辈子
的惊,刚刚儿的娶了亲,中了举人,又知道媳妇作了胎,我才喜欢些,不想弄到这
样结局!早知这样,就不该娶亲,害了人家的姑娘。”薛姨妈道:“这是自己一定
的。咱们这样人家,还有什么别的说的吗?幸喜有了胎,将来生个外孙子,必定是
有成立的,后来就有了结果了。你看大奶奶,如今兰哥儿中了举人,明年成了进士,
可不是就做了官了么?他头里的苦也算吃尽的了,如今的甜来,也是他为人的好处。
我们姑娘的心肠儿姐姐是知道的,并不是刻薄轻佻的人,姐姐倒不必耽忧。”王夫
人被薛姨妈一番言语说得极有理,心想:“宝钗小时候便是廉静寡欲极爱素淡的,
他所以才有这个事。想人生在世,真有个定数的。看着宝钗虽是痛哭,他那端庄样
儿一点不走,却倒来劝我,这是真真难得。不想宝玉这样一个人,红尘中福分竟没
有一点儿!”想了一回,也觉解了好些。又想到袭人身上:“若说别的丫头呢,没
有什么难处的:大的配了出去,小的伏侍二奶奶就是了。独有袭人可怎么处呢?”
此时人多也不好说,且等晚上和薛姨妈商量。

那日薛姨妈并未回家,因恐宝钗痛哭,住在宝钗房中解劝。那宝钗却是极明理,
思前想后:“宝玉原是一种奇异的人,夙世前因,自有一定,原无可怨天尤人。”
更将大道理的话告诉他母亲了。薛姨妈心里反倒安慰,便到王夫人那里,先把宝钗
的话说了。王夫人点头叹道:“若说我无德,不该有这样好媳妇了。”说着更又伤
心起来。薛姨妈倒又劝了一会子。因又提起袭人来,说:“我见袭人近来瘦的了不
得,他是一心想着宝哥儿。但是正配呢理应守的,屋里人愿守也是有的。惟有这袭
人,虽说是算个屋里人,到底他和宝哥儿并没有过明路儿的。”王夫人道:“我才
刚想着,正要等妹妹商量商量。若说放他出去,恐怕他不愿意,又要寻死觅活的;
若要留着他也罢,又恐老爷不依:所以难处。”薛姨妈道:“我看姨老爷是再不肯
叫守着的。再者,姨老爷并不知道袭人的事,想来不过是个丫头,那有留的理呢?
只要姐姐叫他本家的人来,狠狠的吩咐他,叫他配一门正经亲事,再多多的陪送他
些东西。那孩子心肠儿也好,年纪儿又轻,也不枉跟了姐姐会子,也算姐姐待他不
薄了。袭人那里,还得我细细劝他。就是叫他家的人来,也不用告诉他;只等他家
里果然说定了好人家儿,我们还打听打听,若果然足衣足食、女婿长的像个人儿,
然后叫他出去。”王夫人听了,道:“这个主意很是。不然叫老爷冒冒失失的一办,
我可不是又害了一个人了么?”薛姨妈听了,点头道:“可不是么?”又说了几句,
便辞了王夫人仍到宝钗房中去了。看见袭人泪痕满面,薛姨妈便劝解譬喻了一会。
袭人本来老实,不是伶牙俐齿的人,薛姨妈说一句,他应一句,回来说道:“我是
做下人的人,姨太太瞧得起我,才和我说这些话。我是从不敢违拗太太的。”薛姨
妈听他的话,“好一个柔顺的孩子!”心里更加喜欢。宝钗又将大义的话说了一遍,
大家各自相安。

过了几日,贾政回家,众人迎接。贾政见贾赦贾珍已都回家,弟兄叔侄相见,
大家历叙别来的景况。然后内眷们见了,不免想起宝玉来,又大家伤了一会子心。
贾政喝住道:“这是一定的道理!如今只要我们在外把持家事,你们在内相助,断
不可仍是从前这样的散漫。别房的事,各有各家料理,也不用承总。我们本房的事,
里头全归于你,都要按理而行。”王夫人便将宝钗有孕的话也告诉了,“将来丫头
们都放出去。”贾政听了,点头无语。

次日,贾政进内请示大臣们,说是:“蒙恩感激。但未服阕,应该怎么谢恩之
处,望乞大人们指教。”众朝臣说是代奏请旨。于是圣恩浩荡,即命陛见。贾政进
内谢了恩。圣上又降了好些旨意,又问起宝玉的事来。贾政据实回奏。圣上称奇,
旨意说:宝玉的文章固是清奇,想他必是过来人,所以如此。若在朝中,可以进用;
他既不敢受圣朝的爵位,便赏了一个“文妙真人”的道号。贾政又叩头谢恩而出。
回到家中,贾琏贾珍接着,贾政将朝内的话述了一遍,众人喜欢。贾珍便回说:“宁
国府第,收拾齐全,回明了要搬过去。栊翠庵圈在园内,给四妹妹养静。”贾政并
不言语,隔了半日,却吩咐了一番仰报天恩的话。

贾琏也趁便回说:“巧姐亲事,父亲太太都愿意给周家为媳。”贾政昨晚也知
巧姐的始末,便说:“大老爷大太太作主就是了。莫说村居不好,只要人家清白,
孩子肯念书,能够上进。朝里那些官,难道都是城里的人么?”贾琏答应了“是”,
又说:“父亲有了年纪,况且又有痰症的根子,静养几年,诸事原仗二老爷为主。”
贾政道:“提起村居养静,甚合我意,只是我受恩深重,尚未酬报耳。”贾政说毕
进内,贾琏打发请了刘老老来,应了这件事。刘老老见了王夫人等,便说些将来怎
样升官,怎样起家,怎样子孙昌盛。

正说着,丫头回道:“花自芳的女人进来请安。”王夫人问几句话,花自芳的
女人将亲戚作媒,说的是城南蒋家的,现在有房有地,又有铺面。姑爷年纪略大几
岁,并没有娶过的,况且人物儿长的是百里挑一的。王夫人听了愿意,说道:“你
去应了,隔几日进来,再接你妹子罢。”王夫人又命人打听,都说是好。王夫人便
告诉了宝钗,仍请了薛姨妈细细的告诉了袭人。袭人悲伤不已,又不敢违命的,心
里想起宝玉那年到他家去,回来说的死也不回去的话,“如今太太硬作主张,若说
我守着,又叫人说我不害臊;若是去了,实不是我的心愿。”便哭得咽哽难鸣。又
被薛姨妈宝钗等苦劝,回过念头想道:“我若是死在这里,倒把太太的好心弄坏了,
我该死在家里才是。”于是袭人含悲叩辞了众人。那姐妹分手时,自然更有一番不
忍说。

袭人怀着必死的心肠,上车回去,见了哥哥嫂子,也是哭泣,但只说不出来。
那花自芳悉把蒋家的聘礼送给他看,又把自己所办妆奁一一指给他瞧,说:“那是
太太赏的,那是置办的。”袭人此时更难开口。住了两天,细想起来:“哥哥办事
不错。若是死在哥哥家里,岂不又害了哥哥呢?”千思万想,左右为难,真是一缕
柔肠,几乎牵断,只得忍住。

那日已是迎娶吉期,袭人本不是那一种泼辣人,委委屈屈的上轿而去,心里另
想到那里再作打算。岂知过了门,见那蒋家办事,极其认真,全都按着正配的规矩。
一进了门,丫头仆妇,都称“奶奶”。袭人此时欲要死在这里,又恐害了人家,辜
负了一番好意。那夜原是哭着不肯俯就的,那姑爷却极柔情曲意的承顺。到了第二
天开箱,这姑爷看见一条猩红汗巾,方知是宝玉的丫头。原来当初只知是贾母的侍
儿,益想不到是袭人。此时蒋玉函念着宝玉待他的旧情,倒觉满心惶愧,更加周旋;
又故意将宝玉所换那条松花绿的汗巾拿出来。袭人看了,方知这姓蒋的原来就是蒋
玉函,始信姻缘前定。袭人才将心事说出。蒋玉函也深为叹息敬服,不敢勉强,并
越发温柔体贴,弄得个袭人真无死所了。看官听说:虽然事有前定,无可奈何,但
孽子孤臣,义夫节妇,这“不得已”三字也不是一概推委得的。此袭人所以在“又
副册”也。正是前人过那桃花庙的诗上说道:
千古艰难惟一死,伤心岂独息夫人!

不言袭人从此又是一番天地。且说那贾雨村犯了婪索的案件,审明定罪,今遇
大赦,递籍为民。雨村因叫家眷先行,自己带了一个小厮,一车行李,来到急流津
觉迷渡口。只见一个道者,从那渡头草棚里出来,执手相迎。雨村认得是甄士隐,
也连忙打恭。士隐道:“贾老先生,别来无恙?”雨村道:“老仙长到底是甄老先
生!何前次相逢,觌面不认?后知火焚草亭,鄙下深为惶恐。今日幸得相逢,益叹老
仙翁道德高深。奈鄙人下愚不移,致有今日。”甄士隐道:“前者老大人高官显爵,
贫道怎敢相认?原因故交,敢赠片言,不意老大人相弃之深。然而富贵穷通,亦非
偶然,今日复得相逢,也是一桩奇事。这里离草庵不远,暂请膝谈,未知可否?”
雨村欣然领命。

两人携手而行,小厮驱车随后,到了一座茅庵。士隐让进,雨村坐下,小童献
茶上来。雨村便请教仙长超尘始末。士隐笑道:“一念之间,尘凡顿易。老先生从
繁华境中来,岂不知温柔富贵乡中有一宝玉乎?”雨村道:“怎么不知。近闻纷纷
传述,说他也遁入空门。下愚当时也曾与他往来过数次,再不想此人竟有如是之决
绝。”士隐道:“非也。这一段奇缘,我先知之。昔年我与先生在仁清巷旧宅门口
叙话之前,我已会过他一面。”雨村惊讶道:“京城离贵乡甚远,何以能见?”士
隐道:“神交久矣。”雨村道:“既然如此,现今宝玉的下落,仙长定能知之?”
士隐道:“宝玉,即‘宝玉’也。那年荣宁查抄之前,钗黛分离之日,此玉早已离
世:一为避祸,二为撮合。从此夙缘一了,形质归一。又复稍示神灵,高魁贵子,
方显得此玉乃天奇地灵锻炼之宝,非凡间可比。前经茫茫大士渺渺真人携带下凡,
如今尘缘已满,仍是此二人携归本处:便是宝玉的下落。”雨村听了,虽不能全然
明白,却也十知四五,便点头叹道:“原来如此,下愚不知。但那宝玉既有如此的
来历,又何以情迷至此,复又豁悟如此?还要请教。”士隐笑道:“此事说来,先
生未必尽解。太虚幻境,即是真如福地。两番阅册,原始要终之道,历历生平,如
何不悟?仙草归真,焉有通灵不复原之理呢?”

雨村听着,却不明白,知是仙机,也不便更问。因又说道:“宝玉之事,既得
闻命。但敝族闺秀如是之多,何元妃以下,算来结局俱属平常呢?”士隐叹道:“老
先生莫怪拙言!贵族之女,俱属从情天孽海而来。大凡古今女子,那‘淫’字固不
可犯,只这‘情’字也是沾染不得的。所以崔莺苏小,无非仙子尘心;宋玉相如,
大是文人口孽。但凡情思缠绵,那结局就不可问了。”

雨村听到这里,不觉拈须长叹。因又问道:“请教仙翁:那荣宁两府,尚可如
前否?”士隐道:“福善祸淫,古今定理。现今荣宁两府,善者修缘,恶者悔祸,
将来兰桂齐芳,家道复初,也是自然的道理。”雨村低了半日头,忽然笑道:“是
了,是了。现在他府中有一个名兰的,已中乡榜,恰好应着‘兰’字。适间老仙翁
说‘兰桂齐芳’,又道‘宝玉高魁贵子’,莫非他有遗腹之子,可以飞黄腾达的么?”
士隐微微笑道:“此系后事,未便预说。”

雨村还要再问,士隐不答,便命人设具盘飧,邀雨村共食。食毕,雨村还要问
自己的终身。士隐便道:“老先生草庵暂歇。我还有一段俗缘未了,正当今日完结。”
雨村惊讶道:“仙长纯修若此,不知尚有何俗缘?”士隐道:“也不过是儿女私情
罢了。”雨村听了,益发惊异:“请问仙长何出此言?”士隐道:“老先生有所不
知:小女英莲,幼遭尘劫,老先生初任之时,曾经判断。今归薛姓,产难完劫,遗
一子于薛家,以承宗祧。此时正是尘缘脱尽之时,只好接引接引。”士隐说着,拂
袖而起。雨村心中恍恍惚惚,就在这急流津觉迷渡口草庵中睡着了。

这士隐自去度脱了香菱,送到太虚幻境,交那警幻仙子对册。刚过牌坊,见那
一僧一道缥缈而来,士隐接着说道:“大士、真人,恭喜贺喜!情缘完结,都交割
清楚了么?”那僧道说:“情缘尚未全结,倒是那蠢物已经回来了。还得把他送还
原所,将他的后事叙明,不枉他下世一回。”士隐听了,便拱手而别。那僧道仍携
了玉到青埂峰下,将“宝玉”安放在女娲炼石补天之处,各自云游而去。从此后:
天外书传天外事,两番人作一番人。

这一日,空空道人又从青埂峰前经过,见那补天未用之石仍在那里,上面字迹
依然如旧,又从头的细细看了一遍。见后面偈文后又历叙了多少收缘结果的话头,
便点头叹道:“我从前见石兄这段奇文,原说可以闻世传奇,所以曾经抄录,但未
见返本还原。不知何时,复有此段佳话?方知石兄下凡一次,磨出光明,修成圆觉,
也可谓无复遗憾了。只怕年深日久,字迹模糊,反有舛错,不如我再抄录一番,寻
个世上清闲无事的人,托他传遍,知道奇而不奇,俗而不俗,真而不真,假而不假。
或者尘梦劳人,聊倩鸟呼归去;山灵好客,更从石化飞来:亦未可知。”想毕,便
又抄了,仍袖至那繁华昌盛地方。遍寻了一番,不是建功立业之人,即系糊口谋衣
之辈,那有闲情去和石头饶舌?直寻到急流津觉迷渡口草庵中,睡着一个人,因想
他必是闲人,便要将这抄录的《石头记》给他看看。那知那人再叫不醒。空空道人
复又使劲拉他,才慢慢的开眼坐起。便接来草草一看,仍旧掷下道:“这事我已亲
见尽知,你这抄录的尚无舛错。我只指与你一个人,托他传去,便可归结这段新鲜
公案了。”空空道人忙问何人,那人道:“你须待某年某月某日某时,到一个悼红
轩中,有个曹雪芹先生。只说贾雨村言,托他如此如此。”说毕,仍旧睡下了。

那空空道人牢牢记着此言,又不知过了几世几劫,果然有个悼红轩,见那曹雪
芹先生正在那里翻阅历来的古史。空空道人便将贾雨村言了,方把这《石头记》示
看。那雪芹先生笑道:“果然是‘贾雨村言’了!”空空道人便问:“先生何以认
得此人,便肯替他传述?”那雪芹先生笑道:“说你‘空空’,原来肚里果然空空。
既是‘假语村言’,但无鲁鱼亥豕以及背谬矛盾之处,乐得与二三同志,酒馀饭饱,
雨夕灯窗,同消寂寞,又不必大人先生品题传世。似你这样寻根究底,便是刻舟求
剑、胶柱鼓瑟了。”那空空道人听了,仰天大笑,掷下抄本,飘然而去。一面走着,
口中说道:“原来是敷衍荒唐!不但作者不知,抄者不知,并阅者也不知。不过游
戏笔墨,陶情适性而已!”

后人见了这本传奇,亦曾题过四句偈语,为作者缘起之言更进一竿。云:
说到辛酸处,荒唐愈可悲。
由来同一梦,休笑世人痴!