\chapter{惑偏私惜春矢素志~证同类宝玉失相知}

话说宝玉为自己失言,被宝钗问住,想要掩饰过去,只见秋纹进来说:“外头
老爷叫二爷呢。”宝玉巴不得一声儿,便走了。到贾政那里,贾政道:“我叫你来不
为别的。现在你穿着孝,不便到学里去,你在家里,必要将你念过的文章温习温习。
我这几天倒也闲着。隔两三日要做几篇文章我瞧瞧,看你这些时进益了没有。”宝
玉只得答应着。贾政又道:“你环兄弟兰侄儿我也叫他们温习去了。倘若你做的文
章不好,反倒不及他们,那可就不成事了。”宝玉不敢言语,答应了个“是”,站着
不动。贾政道:“去罢。”宝玉退了出来,正遇见赖大诸人拿着些册子进来,宝玉一
溜烟回到自己房中。宝钗问了,知道叫他作文章,倒也喜欢。惟有宝玉不愿意,也
不敢怠慢。

正要坐下静静心,只见两个姑子进来,是地藏庵的。见了宝钗,说道:“请二
奶奶安。”宝钗待理不理的说:“你们好。”因叫人来:“倒茶给师父们喝。”宝玉原
要和那姑子说话,见宝钗似乎厌恶这些,也不好兜搭。那姑子知道宝钗是个冷人,
也不久坐,辞了要去。宝钗道:“再坐坐去罢。”那姑子道:“我们因在铁槛寺做了
功德,好些时没来请太太奶奶们的安。今日来了,见过了奶奶太太们,还要看看四
姑娘呢。”宝钗点头,由他去了。那姑子到了惜春那里,看见彩屏,便问:“姑娘在
那里呢?”彩屏道:“不用提了。姑娘这几天饭都没吃,只是歪着。”那姑子道:“为
什么?”彩屏道:“说也话长。你见了姑娘,只怕他就和你说了。”惜春早已听见,
急忙坐起,说:“你们两个人好啊,见我们家事差了,就不来了。”那姑子道:“阿
弥陀佛!有也是施主,没也是施主,别说我们是本家庵里,受过老太太多少恩惠的。
如今老太太的事,太太奶奶们都见过了,只没有见姑娘,心里惦记,今儿是特特的
来瞧姑娘来了。”

惜春便问起水月庵的姑子来。那姑子道:“他们庵里闹了些事,如今门上也不
肯常放进来了。”便问惜春道:“前儿听见说,栊翠庵的妙师父怎么跟了人走了?”
惜春道:“那里的话?说这个话的人提防着割舌头!人家遭了强盗抢去,怎么还说这
样的坏话。”那姑子道:“妙师父的为人古怪,只怕是假惺惺罢?在姑娘面前,我们
也不好说的。那里像我们这些粗夯人,只知道讽经念佛,给人家忏悔,也为着自己
修个善果。”惜春道:“怎么样就是善果呢?”那姑子道:“除了咱们家这样善德人
家儿不怕,若是别人家那些诰命夫人小姐,也保不住一辈子的荣华。到了苦难来了,
可就救不得了。只有个观世音菩萨大慈大悲,遇见人家有苦难事,就慈心发动,设
法儿救济。为什么如今都说‘大慈大悲救苦救难的观世音菩萨’呢。我们修了行的
人,虽说比夫人小姐们苦多着呢,只是没有险难的了。虽不能成佛作祖,修修来世
或者转个男身,自己也就好了。不像如今脱生了个女人胎子,什么委屈烦难都说不
出来。姑娘你还不知道呢,要是姑娘们到了出了门子,这一辈子跟着人,是更没法
儿的。若说修行,也只要修得真。那妙师父自为才情比我们强,他就嫌我们这些人
俗。岂知俗的才能得善缘呢,他如今到底是遭了大劫了。”

惜春被那姑子一番话说的合在机上,也顾不得丫头们在这里,便将尤氏待他怎
样,前儿看家的事说了一遍,并将头发指给他瞧,道:“你打量我是什么没主意恋
火坑的人么?早有这样的心,只是想不出道儿来。”那姑子听了,假作惊慌道:“姑
娘再别说这个话!珍大奶奶听见,还要骂杀我们,撵出庵去呢。姑娘这样人品,这
样人家,将来配个好姑爷,享一辈子的荣华富贵——”惜春不等说完,便红了脸,
说:“珍大奶奶撵得你,我就撵不得么?”那姑子知是真心,便索性激他一激,说
道:“姑娘别怪我们说错了话。太太奶奶们那里就依得姑娘的性子呢?那时闹出没意
思来倒不好。我们倒是为姑娘的话。”惜春道:“这也瞧罢咧。”彩屏等听这话头不
好,便使个眼色儿给姑子,叫他走。那姑子会意,本来心里也害怕,不敢挑逗,便
告辞出去。惜春也不留他,便冷笑道:“打量天下就是你们一个地藏庵么?”那姑
子也不敢答言,去了。

彩屏见事不妥,恐耽不是,悄悄的去告诉了尤氏说:“四姑娘铰头发的念头还
没有息呢。他这几天不是病,竟是怨命。奶奶提防些,别闹出事来,那会子归罪我
们身上。”尤氏道:“他那里是为要出家?他为的是大爷不在家,安心和我过不去。
也只好由他罢了!”彩屏等没法,也只好常常劝解。岂知惜春一天一天的不吃饭,
只想铰头发。彩屏等吃不住,只得到各处告诉。邢王二夫人等也都劝了好几次,怎
奈惜春执迷不解。

邢王二夫人正要告诉贾政,只听外头传进来说:“甄家的太太带了他们家的宝
玉来了。”众人急忙接出,便在王夫人处坐下。众人行礼,叙些寒温,不必细述。
只言王夫人提起甄宝玉与自己的宝玉无二,要请甄宝玉进来一见。传话出去,回来
说道:“甄少爷在外书房同老爷说话,说的投了机了,打发人来请我们二爷三爷,
还叫兰哥儿在外头吃饭,吃了饭进来。”说毕,里头也便摆饭。

原来此时贾政见甄宝玉相貌果与宝玉一样,试探他的文才,竟应对如流,甚是
心敬,故叫宝玉等三人出来警励他们,再者到底叫宝玉来比一比。宝玉听命,穿了
素服,带了兄弟侄儿出来,见了甄宝玉,竟是旧相识一般。那甄宝玉也像那里见过
的。两人行了礼,然后贾环贾兰相见。本来贾政席地而坐,要让甄宝玉在椅子上坐,
甄宝玉因是晚辈,不敢上坐,就在地下铺了褥子坐下。如今宝玉等出来,又不能同
贾政一处坐着,为甄宝玉是晚一辈,又不好竟叫宝玉等站着。贾政知是不便,站起
来又说了几句话,叫人摆饭,说:“我失陪,叫小儿辈陪着,大家说话儿,好叫他
们领领大教。”甄宝玉逊谢道:“老伯大人请便,小侄正欲领世兄们的教呢。”贾政
回复了几句,便自往内书房去。那甄宝玉却要送出来,贾政拦住。宝玉等先抢了一
步,出了书房门槛站立着,看贾政进去,然后进来让甄宝玉坐下。彼此套叙了一回,
诸如久慕渴想的话,也不必细述。

且说贾宝玉见了甄宝玉,想到梦中之景,并且素知甄宝玉为人,必是和他同心,
以为得了知己。因初次见面,不便造次,且又贾环贾兰在坐,只有极力夸赞说:“久
仰芳名,无由亲炙,今日见面,真是谪仙一流的人物。”那甄宝玉素来也知贾宝玉
的为人,今日一见,果然不差,“只是可与我共学,不可与我适道。他既和我同名
同貌,也是三生石上的旧精魂了。我如今略知些道理,何不和他讲讲?但只是初见,
尚不知他的心与我同不同,只好缓缓的来。”便道:“世兄的才名,弟所素知的。在
世兄是数万人里头选出来最清最雅的。至于弟乃庸庸碌碌一等愚人,忝附同名,殊
觉玷辱了这两个字。”贾宝玉听了,心想:“这个人果然同我的心一样的,但是你我
都是男人,不比那女孩儿们清洁,怎么他拿我当作女孩儿看待起来?”便道:“世
兄谬赞,实不敢当。弟至浊至愚,只不过一块顽石耳,何敢比世兄品望清高,实称
此两字呢?”甄宝玉道:“弟少时不知分量,自谓尚可琢磨;岂知家遭消索,数年
来更比瓦砾犹贱。虽不敢说历尽甘苦,然世道人情,略略的领悟了些须。世兄是锦
衣玉食,无不遂心的,必是文章经济高出人上,所以老伯钟爱,将为席上之珍。弟
所以才说尊名方称。”贾宝玉听这话头又近了禄蠹的旧套,想话回答。贾环见未与
他说话,心中早不自在。倒是贾兰听了这话,甚觉合意,便说道:“世叔所言,固
是太谦,若论到文章经济,实在从历练中出来的,方为真才实学。在小侄年幼,虽
不知文章为何物,然将读过的细味起来,那膏粱文绣,比着令闻广誉,真是不啻百
倍的了!”甄宝玉未及答言。

贾宝玉听了兰儿的话,心里越发不合,想道:“这孩子从几时也学了这一派酸
论!”便说道:“弟闻得世兄也诋尽流俗,性情中另有一番见解。今日弟幸会芝范,
想欲领教一番超凡入圣的道理,从此可以洗净俗肠,重开眼界。不意视弟为蠢物,
所以将世路的话来酬应。”甄宝玉听说,心里晓得:“他知我少年的性情,所以疑我
为假。我索性把话说明,或者与我作个知心朋友,也是好的。”便说:“世兄高论,
固是真切。但弟少时也曾深恶那些旧套陈言,只是一年长似一年,家君致仕在家,
懒于酬应,委弟接待。后来见过那些大人先生,尽都是显亲扬名的人;便是著书立
说,无非言忠言孝,自有一番立德立言的事业,方不枉生在圣明之时,也不致负了
父亲师长养育教诲之恩。所以把少时那些迂想痴情,渐渐的淘汰了些。如今尚欲访
师觅友,教导愚蒙。幸会世兄,定当有以教我。适才所言,并非虚意。”贾宝玉愈
听愈不耐烦,又不好冷淡,只得将言语支吾。幸喜里头传出话来,说:“若是外头
爷们吃了饭,请甄少爷里头去坐呢。”宝玉听了,趁势便邀甄宝玉进去。那甄宝玉
依命前行,贾宝玉等陪着来见王夫人。贾宝玉见是甄太太上坐,便先请过了安。贾
环贾兰也见了。甄宝玉也请了王夫人的安。两母两子,互相厮认。虽是贾宝玉是娶
过亲的,那甄夫人年纪已老,又是老亲,因见贾宝玉的相貌身材与他儿子一般,不
禁亲热起来。王夫人更不用说,拉着甄宝玉问长问短,觉得比自己家的宝玉老成些。
回看贾兰,也是清秀超群的,虽不能像两个宝玉的形象,也还随得上,只有贾环粗
夯,未免有偏爱之色。

众人一见两个宝玉在这里,都来瞧看,说道:“真真奇事!名字同了也罢,怎么
相貌身材都是一样的。亏得是我们宝玉穿孝,若是一样的衣服穿着,一时也认不出
来。”内中紫鹃一时痴意发作,因想起黛玉来,心里说道:“可惜林姑娘死了,若不
死时,就将那甄宝玉配了他,只怕也是愿意的。”正想着,只听得甄夫人道:“前日
听得我们老爷回来说:我们宝玉年纪也大了,求这里老爷留心一门亲事。”王夫人
正爱甄宝玉,顺口便说道:“我也想要与令郎作伐。我家有四个姑娘:那三个都不
用说,死的死,嫁的嫁了。还有我们珍大侄儿的妹子,只是年纪过小几岁,恐怕难
配。倒是我们大媳妇的两个堂妹子,生得人材齐正。二姑娘呢,已经许了人家;三
姑娘正好与令郎为配。过一天,我给令郎作媒。但是他家的家计如今差些。”甄夫
人道:“太太这话又客套了。如今我们家还有什么?只怕人家嫌我们穷罢咧。”王夫
人道:“现今府上复又出了差,将来不但复旧,必是比先前更要鼎盛起来。”甄夫人
笑着道:“但愿依着太太的话更好。这么着,就求太太作个保山。”甄宝玉听见他们
说起亲事,便告辞出来,贾宝玉等只得陪着来到书房。见贾政已在那里,复又立谈
几句。听见甄家的人来回甄宝玉道:“太太要走了,请爷回去罢。”于是甄宝玉告辞
出来。贾政命宝玉、环、兰相送,不提。

且说宝玉自那日见了甄宝玉之父,知道甄宝玉来京,朝夕盼望。今儿见面,原
想得一知己,岂知谈了半天,竟有些冰炭不投。闷闷的回到自己房中,也不言,也
不笑,只管发怔。宝钗便问:“那甄宝玉果然像你么?”宝玉道:“相貌倒还是一样
的,只是言谈间看起来,并不知道什么,不过也是个禄蠹。”宝钗道:“你又编派人
家了。怎么就见得也是个禄蠹呢?”宝玉道:“他说了半天,并没个明心见性之谈,
不过说些什么‘文章经济’,又说什么‘为忠为孝’。这样人可不是个禄蠹么?只可
惜他也生了这样一个相貌。我想来,有了他,我竟要连我这个相貌都不要了。”宝
钗见他又说呆话,便说道:“你真真说出句话来叫人发笑,这相貌怎么能不要呢!况
且人家这话是正理,做了一个男人,原该要立身扬名的,谁像你一味的柔情私意?
不说自己没有刚烈,倒说人家是禄蠹。”宝玉本听了甄宝玉的话,甚不耐烦,又被
宝钗抢白了一场,心中更加不乐,闷闷昏昏,不觉将旧病又勾起来了,并不言语,
只是傻笑。宝钗不知,只道自己的话错了,他所以冷笑,也不理他。岂知那日便有
些发呆,袭人等怄他,也不言语。过了一夜,次日起来,只是呆呆的,竟有前番病
的样子。

一日,王夫人因为惜春定要铰发出家,尤氏不能拦阻,看着惜春的样子是若不
依他必要自尽的,虽然昼夜着人看守终非常事,便告诉了贾政。贾政叹气跺脚,只
说:“东府里不知干了什么,闹到如此地位!”叫了贾蓉来说了一顿,叫他去和他母
亲说:“认真劝解劝解。若是必要这样,就不是我们家的姑娘了。”岂知尤氏不劝还
好,一劝了,更要寻死,说:“做了女孩儿,终不能在家一辈子的。若像二姐姐一
样,老爷太太们倒要操心,况且死了。如今譬如我死了似的,放我出了家,干干净
净的一辈子,就是疼我了。况且我又不出门,就是栊翠庵原是咱们家的基址,我就
在那里修行。我有什么,你们也照应得着。现在妙玉的当家的在那里。你们依我呢,
我就算得了命了;若不依我呢,我也没法,只有死就完了!我如若遂了自己的心愿,
那时哥哥回来,我和他说并不是你们逼着我的;若说我死了,未免哥哥回来,倒说
你们不容我。”尤氏本与惜春不合,听他的话,也似乎有理,只得去回王夫人。

王夫人已到宝钗那里,见宝玉神魂失所,心下着忙,便说袭人道:“你们忒不
留神!二爷犯了病,也不来回我。”袭人道:“二爷的病原来是常有的,一时好,一
时不好。天天到太太那里,仍旧请安去,原是好好儿的,今日才发糊涂些。二奶奶
正要来回太太,恐怕太太说我们大惊小怪。”宝玉听见王夫人说他们,心里一时明
白,怕他们受委屈,便说道:“太太放心,我没什么病,只是心里觉着有些闷闷的。”
王夫人道:“你是有这病根子,早说了,好请大夫瞧瞧,吃两剂药好了不好?若再闹
到头里丢了玉的样子,那可就费事了。”宝玉道:“太太不放心,便叫个人瞧瞧,我
就吃药。”王夫人便叫丫头传话出来请大夫。这一个心思都在宝玉身上,便将惜春
的事忘了。迟了一回,大夫看了服药,王夫人回去。

过了几天,宝玉更糊涂了,甚至于饮食不进,大家着急起来。恰又忙着脱孝,
家中无人,又叫了贾芸来照应大夫。贾琏家下无人,请了王仁来在外帮着料理。那
巧姐儿是日夜哭母,也是病了。所以荣府中又闹得马仰人翻。

一日,又当脱孝来家,王夫人亲身又看宝玉,见宝玉人事不醒,急得众人手足
无措,一面哭着,一面告诉贾政说:“大夫说了,不肯下药,只好预备后事!”贾政
叹气连连,只得亲自看视,见其光景果然不好,便又叫贾琏办去。贾琏不敢违拗,
只得叫人料理;手头又短,正在为难。只见一个人跳进来说:“二爷不好了,又有
饥荒来了!”贾琏不知何事,这一吓非同小可,瞪着眼说道:“什么事?”那小厮道:
“门上来了一个和尚,手里拿着二爷的这块丢的玉,说要一万赏银。”贾琏照脸啐
道:“我打量什么事,这样慌张!前番那假的你不知道么?就是真的,现在人要死了,
要这玉做什么?”小厮道:“奴才也说了。那和尚说,给他银子就好了。”正说着,
外头嚷进来说:“这和尚撒野,各自跑进来了,众人拦他拦不住!”贾琏道:“那里
有这样怪事?你们还不快打出去呢。”又闹着,贾政听见了,也没了主意了。里头又
哭出来,说:“宝二爷不好了!”贾政益发着急。只见那和尚说道:“要命拿银子来。”
贾政忽然想起:“头里宝玉的病是和尚治好的;这会子和尚来,或者有救星。但是
这玉倘或是真,他要起银子来,怎么样呢?”想一想:“如今且不管他,果真人好
了再说。”

贾政叫人去请,那和尚已进来了,也不施礼,也不答话,便往里就跑。贾琏拉
着道:“里头都是内眷,你这野东西混跑什么?”那和尚道:“迟了就不能救了。”
贾琏急得一面走,一面乱嚷道:“里头的人不要哭了,和尚进来了!”王夫人等只顾
着哭,那里理会。贾琏走进来又嚷。王夫人等回过头来,见一个长大的和尚,吓了
一跳,躲避不及。那和尚直走到宝玉炕前。宝钗避过一边,袭人见王夫人站着,不
敢走开。只见那和尚道:“施主们,我是送玉来的。”说着,把那块玉擎着道:“快
把银子拿出来,我好救他。”王夫人等惊惶无措,也不择真假,便说道:“若是救活
了人,银子是有的。”那和尚笑道:“拿来!”王夫人道:“你放心,横竖折变的出来。”
和尚哈哈大笑,手拿着玉,在宝玉耳边叫道:“宝玉,宝玉!你的‘宝玉’回来了。”
说了这一句,王夫人等见宝玉把眼一睁。袭人说道:“好了!”只见宝玉便问道:“在
那里呢?”那和尚把玉递给他手里。宝玉先前紧紧的攥着,后来慢慢的回过手来,
放在自己眼前,细细的一看,说:“嗳呀!久违了。”里外众人都喜欢的念佛,连宝
钗也顾不得有和尚了。

贾琏也走过来一看,果见宝玉回过来了,心里一喜,疾忙躲出去了。那和尚也
不言语,赶来拉着贾琏就跑。贾琏只得跟着,到了前头,赶着告诉贾政。贾政听了
喜欢,即找和尚施礼叩谢。和尚还了礼坐下。贾琏心下狐疑:“必是要了银子才走。”
贾政细看那和尚,又非前次见的,便问:“宝刹何方?法师大号?这玉是那里得的?怎
么小儿一见便会活过来呢?”那和尚微微笑道:“我也不知道,只要拿一万银子来
就完了。”贾政见这和尚粗鲁,也不敢得罪,便说:“有。”和尚道:“有便快拿来罢,
我要走了。”贾政道:“略请少坐,待我进内瞧瞧。”和尚道:“你去,快出来才好。”

贾政果然进去,也不及告诉,便走到宝玉炕前。宝玉见是父亲来,欲要爬起,
因身子虚弱,起不来。王夫人按着说道:“不要动。”宝玉笑着,拿这玉给贾政瞧,
道:“宝玉来了。”贾政略略一看,知道此玉有些根源,也不细看,便和王夫人道:
“宝玉好过来了,这赏银怎么样?”王夫人道:“尽着我所有的折变了给他就是了。”
宝玉道:“只怕这和尚不是要银子的罢?”贾政点头道:“我也看来古怪,但是他口
口声声的要银子。”王夫人道:“老爷出去先款留着他再说。”贾政出来。宝玉便嚷
饿了,喝了一碗粥,还说要饭。婆子们果然取了饭来。王夫人还不敢给他吃。宝玉
说:“不妨的,我已经好了。”便爬着吃了一碗,渐渐的神气果然好过来了,便要坐
起来。麝月上去轻轻的扶起,因心里喜欢忘了情,说道:“真是宝贝,才看见了一
会儿,就好了。亏的当初没有砸破!”宝玉听了这话,神色一变,把玉一撂,身子
往后一仰。

未知死活,下回分解。