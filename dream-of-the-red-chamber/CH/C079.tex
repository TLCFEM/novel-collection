\chapter{薛文起悔娶河东吼~贾迎春误嫁中山狼}

话说宝玉才祭完了晴雯,只听花阴中有个人声,倒吓了一跳。细看不是别人,
却是黛玉,满面含笑,口内说道:“好新奇的祭文!可与《曹娥碑》并传了。”宝
玉听了,不觉红了脸,笑答道:“我想着世上这些祭文,都过于熟烂了,所以改个
新样。原不过是我一时的玩意儿,谁知被你听见了。有什么大使不得的,何不改削
改削?”黛玉道:“原稿在那里?倒要细细的看看。长篇大论,不知说的是什么。
只听见中间两句,什么‘红绡帐里,公子情深;黄土陇中,女儿命薄’,这一联意
思却好。只是‘红绡帐里’未免俗滥些。放着现成的真事,为什么不用?”宝玉忙
问:“什么现成的真事?”黛玉笑道:“咱们如今都系霞彩纱糊的窗,何不说‘茜
纱窗下,公子多情’呢?”宝玉听了,不禁跌脚笑道:“好极,好极!到底是你想
得出,说得出。可知天下古今现成的好景好事尽多,只是我们愚人想不出来罢了。
但只一件:虽然这一改新妙之极,却是你在这里住着还可以,我实不敢当。”说着,
又连说“不敢”。黛玉笑道:“何妨?我的窗即可为你之窗,何必如此分晰,也太
生疏了。古人异姓陌路,尚然‘肥马轻裘,敝之无憾’,何况咱们?”宝玉笑道:
“论交道,不在‘肥马轻裘’,即黄金白璧亦不当锱铢较量。倒是这唐突闺阁上头,
却万万使不得的。如今我索性将‘公子’‘女儿’改去,竟算是你诔他的倒妙。况
且素日你又待他甚厚,所以宁可弃了这一篇文,万不可弃这‘茜纱’新句。莫若改
作‘茜纱窗下,小姐多情;黄土陇中,丫鬟薄命’。如此一改,虽与我不涉,我也
惬怀。”黛玉笑道:“他又不是我的丫头,何用此话?况且‘小姐’‘丫鬟’,亦
不典雅。等得紫鹃死了,我再如此说,还不算迟呢。”宝玉听了笑道:“这是何苦,
又咒他。”黛玉笑道:“是你要咒的,并不是我说的。”宝玉说:“我又有了,这
一改恰就妥当了:莫若说‘茜纱窗下,我本无缘;黄土陇中,卿何薄命!’”

黛玉听了,陡然变色。虽有无限狐疑,外面却不肯露出,反连忙含笑点头称妙,
说:“果然改得好。再不必乱改了,快去干正经事罢。刚才太太打发人叫你,说明
儿一早过大舅母那边去呢。你二姐姐已有人家求准了,所以叫你们过去呢。”宝玉
忙道:“何必如此忙?我身上也不大好,明儿还未必能去呢。”黛玉道:“又来了。
我劝你把脾气改改罢。一年大,二年小,……”一面说话一面咳嗽起来。宝玉忙道:
“这里风冷,咱们只顾站着,凉着呢可不是玩的,快回去罢。”黛玉道:“我也家
去歇息了,明儿再见罢。”说着,便自取路去了。宝玉只得闷闷的转步,忽想起黛
玉无人随伴,忙命小丫头子跟送回去。自己到了怡红院中,果有王夫人打发嬷嬷们
来,吩咐他明日一早过贾赦这边来,与方才黛玉之言相对。

原来贾赦已将迎春许与孙家了。这孙家乃是大同府人氏,祖上系军官出身,乃
当日宁荣府中之门生,算来亦系至交。如今孙家只有一人在京,现袭指挥之职。此
人名唤孙绍祖,生得相貌魁梧,体格健壮,弓马娴熟,应酬权变,年纪未满三十,
且又家资饶富,现在兵部候缺题升。因未曾娶妻,贾赦见是世交子侄,且人品家当
都相称合,遂择为东床娇婿。亦曾回明贾母,贾母心中却不大愿意,但想儿女之事,
自有天意,况且他亲父主张,何必出头多事?因此只说“知道了”三字,馀不多及。
贾政又深恶孙家,虽是世交,不过是他祖父当日希慕宁荣之势,有不能了结之事挽
拜在门下的,并非诗礼名族之裔。因此,倒劝谏过两次,无奈贾赦不听,也只得罢
了。

宝玉却未曾会过这孙绍祖一面的,次日只得过去,聊以塞责。只听见那娶亲的
日子甚近,不过今年就要过门的,又见邢夫人等回了贾母,将迎春接出大观园去,
越发扫兴。每每痴痴呆呆的,不知作何消遣。又听说要陪四个丫头过去,更又跌足
道:“从今后这世上又少了五个清净人了!”因此天天到紫菱洲一带地方徘徊瞻顾。
见其轩窗寂寞,屏帐然,不过只有几个该班上夜的老妪。再看那岸上的蓼花苇叶,
也都觉摇摇落落,似有追忆故人之态,迥非素常逞妍斗色可比。所以情不自禁,乃
信口吟成一歌曰:
池塘一夜秋风冷,吹散芰荷红玉影。
蓼花菱叶不胜悲,重露繁霜压纤梗。
不闻永昼敲棋声,燕泥点点污棋枰。
古人惜别怜朋友,况我今当手足情!

宝玉方才吟罢,忽闻背后有人笑道:“你又发什么呆呢?”宝玉回头忙看是谁,
原来是香菱。宝玉忙转身笑问道:“我的姐姐,你这会子跑到这里来做什么?许多
日子也不进来逛逛。”香菱拍手笑嘻嘻的说道:“我何曾不要来。如今你哥哥回来
了,那里比先时自由自在的了?才刚我们太太使人找你凤姐姐去,竟没有找着,说
往园子里来了。我听见这个话,我就讨了这个差进来找他。遇见他的丫头,说在稻
香村呢。如今我往稻香村去,谁知又遇见了你。我还要问你:袭人姐姐这几日可好?
怎么忽然把个晴雯姐姐也没了?到底是什么病?二姑娘搬出去的好快!你瞧瞧,这地
方一时间就空落落的了。”宝玉只有一味答应,又让他同到怡红院去吃茶。香菱道:
“此刻竟不能,等找着琏二奶奶,说完了正经话再来。”宝玉道:“什么正经话,
这般忙?”香菱道:“为你哥哥娶嫂子的话,所以要紧。”宝玉道:“正是说的是
那一家的好?只听见吵嚷了这半年,今儿又说张家的好,明儿又要李家的,后儿又
议论王家的好。这些人家的女儿,他也不知造了什么罪,叫人家好端端的议论。”
香菱道:“如今定了,可以不用拉扯别人家了。”宝玉问道:“定了谁家的?”香
菱道:“因你哥哥上次出门时,顺路到了个亲戚家去。这门亲原是老亲,且又和我
们是同在户部挂名行商,也是数一数二的大门户。前日说起来时,你们两府都也知
道的:合京城里,上至王侯,下至买卖人,都称他家是‘桂花夏家’。”宝玉忙笑
道:“如何又称为‘桂花夏家’?”香菱道:“本姓夏,非常的富贵。其馀田地不
用说,单有几十顷地种着桂花,凡这长安那城里城外桂花局,俱是他家的,连宫里
一应陈设盆景,亦是他家供奉。因此才有这个混号。如今太爷也没了,只有老奶奶
带着一个亲生的姑娘过活,也并没有哥儿弟兄。可惜他竟一门尽绝了后。”宝玉忙
道:“咱们也别管他绝后不绝后,只是这姑娘可好?你们大爷怎么就中意了?”香
菱笑道:“一则是天缘,二来是‘情人眼里出西施’。当年时又通家来往,从小儿
都在一处玩过。叙亲是姑舅兄妹,又没嫌疑。虽离了这几年,前儿一到他家,夏奶
奶又是没儿子的,一见了你哥哥出落的这么,又是哭,又是笑,竟比见了儿子的还
胜。又令他兄妹相见。谁知这姑娘出落的花朵似的了,在家里也读书写字,所以你
哥哥当时就一心看准了。连当铺里老伙计们一群人,遭扰了人家三四日。他们还留
多住几天,好容易苦辞,才放回家。你哥哥一进门,就咕咕唧唧求我们太太去求亲。
我们太太原是见过的,又且门当户对,也依了。和这里姨太太凤姑娘商议了打发人
去一说,就成了。只是娶的日子太急,所以我们忙乱的很。我也巴不得早些过来,
又添了一个做诗的人了。”宝玉冷笑道:“虽如此说,但只我倒替你担心虑后呢。”
香菱道:“这是什么话?我倒不懂了。”宝玉笑道:“这有什么不懂的?只怕再有个
人来,薛大哥就不肯疼你了。”香菱听了,不觉红了脸,正色道:“这是怎么说?
素日咱们都是厮抬厮敬,今日忽然提起这些事来。怪不得人人都说你是个亲近不得
的人。”一面说,一面转身走了。

宝玉见他这样,便怅然如有所失,呆呆的站了半日,只得没精打彩,还入怡红
院来。一夜不曾安睡,种种不宁。次日便懒进饮食,身体发热。也因近日抄检大观
园、逐司棋、别迎春、悲晴雯等羞辱、惊恐、悲凄所致,兼以风寒外感,遂致成疾,
卧床不起。贾母听得如此,天天亲来看视。王夫人心中自悔,不合因晴雯过于逼责
了他。心中虽如此,脸上却不露出,只吩咐众奶娘等好生伏侍看守。一日两次带进
医生来诊脉下药。一月之后,方才渐渐的痊愈。好生保养过百日,方许动荤腥油面,
方可出门行走。这百日内,院门前皆不许到,只在屋里玩笑。四五十天后,就把他
拘的火星乱迸,那里忍耐的住?虽百般设法,无奈贾母王夫人执意不从,也只得罢
了。因此,和些丫鬟们无所不至,恣意耍笑。又听得薛蟠那里摆酒唱戏,热闹非常,
已娶亲入门。闻得这夏家小姐十分俊俏,也略通文翰,宝玉恨不得就过去一见才好。
再过些时,又闻得迎春出了阁。宝玉思及当时姊妹耳鬓厮磨,从今一别,纵得相逢,
必不得似先前这等亲热了。眼前又不能去一望,真令人凄惶不尽。少不得潜心忍耐,
暂同这些丫鬟们厮闹释闷,幸免贾政责备逼迫读书之难。这百日内,只不曾拆毁了
怡红院,和这些丫头们无法无天,凡世上所无之事,都玩耍出来,如今且不消细说。

且说香菱自那日抢白了宝玉之后,自为宝玉有意唐突,“从此倒要远避他些才
好。”因此,以后连大观园也不轻易进来了。日日忙乱着薛蟠娶过亲,因为得了护
身符,自己身上分去责任,到底比这样安静些;二则又知是个有才有貌的佳人,自
然是典雅和平的:因此,心里盼过门的日子比薛蟠还急十倍呢。好容易盼得一日娶
过来,他便十分殷勤小心伏侍。

原来这夏家小姐今年方十七岁,生得亦颇有姿色,亦颇识得几个字。若论心里
的丘壑泾渭,颇步熙凤的后尘。只吃亏了一件:从小时父亲去世的早,又无同胞兄
弟,寡母独守此女,娇养溺爱,不啻珍宝,凡女儿一举一动,他母亲皆百依百顺,
因此未免酿成个盗跖的情性:自己尊若菩萨,他人秽如粪土;外具花柳之姿,内秉
风雷之性。在家里和丫鬟们使性赌气、轻骂重打的。今儿出了阁,自为要作当家的
奶奶,比不得做女儿时腼腆温柔,须要拿出威风来才钤压得住人。况且见薛蟠气质
刚硬,举止骄奢,若不趁热灶一气炮制,将来必不能自竖旗帜矣。又见有香菱这等
一个才貌俱全的爱妾在室,越发添了“宋太祖灭南唐”之意。因他家多桂花,他小
名就叫做金桂。他在家时,不许人口中带出“金”“桂”二字来,凡有不留心误道
一字者,他便定要苦打重罚才罢。他因想“桂花”二字是禁止不住的,须得另换一
名,想桂花曾有广寒嫦娥之说,便将桂花改为“嫦娥花”,又寓自己身分。如今薛
蟠本是个怜新弃旧的人,且是有酒胆、无饭力的,如今得了这一个妻子,正在新鲜
兴头上,凡事未免尽让他些。那夏金桂见是这般形景,便也试着一步紧似一步。一
月之中,二人气概都还相平;至两月之后,便觉薛蟠的气概渐次的低矮了下去。

一日,薛蟠酒后,不知要行何事,先和金桂商议。金桂执意不从,薛蟠便忍不
住,便发了几句话,赌气自行了。金桂便哭的如醉人一般,茶汤不进,装起病来,
请医疗治。医生又说:“气血相逆,当进宽胸顺气之剂。”薛姨妈恨得骂了薛蟠一
顿,说:“如今娶了亲,眼前抱儿子了,还是这么胡闹!人家凤凰似的,好容易养
了一个女儿,比花朵儿还轻巧,原看的你是个人物,才给你做媳妇。你不说收了心,
安分守己,一心一计,和和气气的过日子,还是这么胡闹,喝了黄汤折磨人家。这
会子花钱吃药白遭心。”一席话说的薛蟠后悔不迭,反来安慰金桂。金桂见婆婆如
此说,越发得了意,更装出些张致来,不理薛蟠。薛蟠没了主意,惟有自软而已。
好容易十天半月之后,才渐渐的哄转过金桂的心来。

自此,便加一倍小心,气概不免又矮了半截下来。那金桂见丈夫旗纛渐倒,婆
婆良善,也就渐渐的持戈试马。先时不过挟制薛蟠;后来倚娇作媚,将及薛姨妈;
后将至宝钗。宝钗久察其不轨之心,每每随机应变,暗以言语弹压其志。金桂知其
不可犯,便欲寻隙,苦得无隙可乘,倒只好曲意俯就。一日,金桂无事,因和香菱
闲谈,问香菱家乡父母。香菱皆答“忘记”,金桂便不悦,说有意欺瞒了他。因问:
“‘香菱’二字是谁起的?”香菱便答道:“姑娘起的。”金桂冷笑道:“人人都
说姑娘通,只这一个名字就不通。”香菱忙笑道:“奶奶若说姑娘不通,奶奶没合
姑娘讲究过。说起来,他的学问,连咱们姨老爷常时还夸的呢。”

欲知香菱说出何话,且听下回分解。