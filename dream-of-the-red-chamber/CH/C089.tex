\chapter{人亡物在公子填词~蛇影杯弓颦卿绝粒}

却说凤姐正自起来纳闷,忽听见小丫头这话,又唬了一跳,连忙又问:“什么
官事?”小丫头道:“也不知道。刚才二门上小厮回进来,回老爷有要紧的官事,
所以太太叫我请二爷来了。”凤姐听了工部里的事,才把心略略的放下。因说道:
“你回去回太太,就说二爷昨日晚上出城有事没有回来,打发人先回珍大爷去罢。”
那丫头答应着去了。一时贾珍过来见了部里的人,问明了。进来见了王夫人回道:
“部中来报:昨日总河奏到,河南一带决了河口,湮没了几府州县。又要开销国帑,
修理城工。工部司官又有一番照料。所以部里特来报知老爷的。”说完退出。及贾
政回家来,回明。从此,直到冬间,贾政天天有事,常在衙门里。宝玉的工课也渐
渐松了,只是怕贾政觉察出来,不敢不常在学房里去念书,连黛玉处也不敢常去。

那时已到十月中旬,宝玉起来,要往学房中去。这日天气陡寒,只见袭人早已
打点出一包衣裳,向宝玉道:“今日天气很凉,早晚宁可暖些。”说着,把衣裳拿
出来,给宝玉挑了一件穿。又包了一件,叫小丫头拿出,交给焙茗,嘱咐道:“天
气冷,二爷要换时,好生预备着。”焙茗答应了,抱着毡包,跟着宝玉自去。宝玉
到了学房中,做了自己的工课,忽听得纸窗呼喇喇一派风声。代儒道:“天气又变
了。”把风门推开一看,只见西北上一层层的黑云,渐渐往东南扑上来。焙茗走进
来回宝玉道:“二爷,天气冷了,再添些衣裳罢。”宝玉点点头儿。只见焙茗拿进
一件衣裳来。宝玉不看则已,看了时神已痴了,那些小学生都巴着眼瞧。却原是晴
雯所补的那件雀金裘。宝玉道:“怎么拿这一件来?是谁给你的?”焙茗道:“是
里头姑娘们包出来的。”宝玉道:“我身上不大冷,且不穿呢,包上罢。”代儒只
当宝玉可惜这件衣裳,却也心里喜他知道俭省。焙茗道:“二爷穿上罢。着了冷,
又是奴才的不是了,二爷只当疼奴才罢。”宝玉无奈,只得穿上,呆呆的对着书坐
着。代儒也只当他看书,不甚理会。

晚间放学时,宝玉便往代儒托病告假一天。代儒本来上年纪的人,也不过伴着
几个孩子解闷儿,时常也八病九痛的,乐得去一个少操一日心。况且明知贾政事忙,
贾母溺爱,便点点头儿。宝玉一径回来,见过贾母王夫人,也是这么说,自然没有
不信的。略坐一坐,便回园中去了。见了袭人等,也不似往日有说有笑的,便和衣
躺在炕上。袭人道:“晚饭预备下了,这会儿吃,还是等一等儿?”宝玉道:“我
不吃了,心里不舒服。你们吃去罢。”袭人道:“那么着,你也该把这件衣裳换下
来了。那个东西那里禁得住揉搓?”宝玉道:“不用换。”袭人道:“倒也不但是
娇嫩物儿,你瞧瞧那上头的针线,也不该这么遭塌他呀。”宝玉听了这话,正碰在
他心坎儿上,叹了一口气道:“那么着,你就收起来,给我包好了。我也总不穿他
了!”说着,站起来脱下。袭人才过来接时,宝玉已经自己叠起。袭人道:“二爷
怎么今日这样勤谨起来了?”宝玉也不答言,叠好了,便问:“包这个的包袱呢?”
麝月连忙递过来,让他自己包好,回头和袭人挤着眼儿笑。宝玉也不理会,自己坐
着,无精打彩。猛听架上钟响,自己低头看了看表针,已指到酉初二刻了。一时小
丫头点上灯来,袭人道:“你不吃饭,喝半碗热粥儿罢,别净饿着。看仔细饿上虚
火来,那又是我们的累赘了。”宝玉摇摇头儿,说:“这不大饿,强吃了倒不受用。”
袭人道:“既这么着,就索性早些歇着罢。”于是袭人麝月铺设好了,宝玉也就歇
下,翻来覆去只睡不着。将及黎明,反朦胧睡去,有一顿饭时,早又醒了。

此时袭人麝月也都起来。袭人道:“昨夜听着你翻腾到五更天,我也不敢问你。
后来我就睡着了,不知到底你睡着了没有?”宝玉道:“也睡了一睡,不知怎么就
醒了。”袭人道:“你没有什么不受用?”宝玉道:“没有,只是心上发烦。”袭
人道:“今日学房里去不去?”宝玉道:“我昨儿已经告了一天假了,今儿我要想
园里逛一天,散散心,只是怕冷。你叫他们收拾一间屋子,备了一炉香,搁下纸墨
笔砚,你们只管干你们的,我自己静坐半天才好,别叫他们来搅我。”麝月接着道:
“二爷要静静儿的用工夫,谁敢来搅。”袭人道:“这么着很好,也省得着了凉,
自己坐坐,心神也不搅。”因又问:“你既懒怠吃饭,今日吃什么早说,好传给厨
房里去。”宝玉道:“还是随便罢,不必闹的大惊小怪的。倒是要几个果子搁在那
屋里,借点果子香。”袭人道:“那个屋里好?别的都不大干净,只有晴雯起先住
的那一间,因一向无人,还干净。就是清冷些。”宝玉道:“不妨,把火盆挪过去
就是了。”袭人答应了。正说着,只见一个小丫头端了一个茶盘儿,一个碗,一双
牙箸,递给麝月道:“这是刚才花姑娘要的,厨房里老婆子送了来了。”麝月接了
一看,却是一碗燕窝汤,便问袭人道:“这是姐姐要的么?”袭人笑道:“昨夜二
爷没吃饭,又翻腾了一夜,想来今儿早起心里必是发空的,所以我告诉小丫头们,
叫厨房里做了这个来的。”袭人一面叫小丫头放桌儿。麝月打发宝玉喝了,漱了口,
只见秋纹走来说道:“那屋里已经收拾妥了,但等着一时炭劲过了,二爷再进去罢。”
宝玉点头,只是一腔心事,懒意说话。

一时小丫头来请,说:“笔砚都安放妥当了。”宝玉道:“知道了。”又一个
小丫头回道:“早饭得了,二爷在那里吃?”宝玉道:“就拿了来罢,不必累赘了。”
小丫头答应了自去,一时端上饭来。宝玉笑了一笑,向麝月袭人道:“我心里闷得
很,自己吃只怕又吃不下去,不如你们两个同我一块儿吃,或者吃的香甜,我也多
吃些。”麝月笑道:“这是二爷的高兴,我们可不敢。”袭人道:“其实也使得,
我们一处喝酒,也不止今日。只是偶然替你解闷儿还使得,若认真这样,还有什么
规矩体统呢。”说着,三人坐下。宝玉在上首,袭人麝月两个打横陪着。吃了饭,
小丫头端上漱口茶来,两个看着撤了下去。宝玉因端着茶,默默如有所思,又坐了
一坐,便问道:“那屋里收拾妥了么?”麝月道:“头里就回过了。这会子又问!”

宝玉略坐了一坐,便过这间屋子来。亲自点了一炷香,摆上些果品,便叫人出
去,关上门。外面袭人等都静悄无声。宝玉拿了一幅泥金角花的粉红笺出来,口中
祝了几句,便提起笔来写道:
怡红主人焚付晴姐知之:酌茗清香,庶几来飨。
其词云:

随身伴,独自意绸缪。谁料风波平地起,顿教躯命即时休:孰与话轻柔?

东
逝水,无复向西流。想像更无怀梦草,添衣还见翠云裘。脉脉使人愁!
写毕,就在香上点个火,焚化了。静静儿等着,直待一炷香点尽了,才开门出来。
袭人道:“怎么出来了?想来又闷的慌了?”宝玉笑了一笑,假说道:“我原是心
里烦,才找个清静地方儿坐坐。这会子好了,还要外头走走去呢。”

说着一径出来到了潇湘馆里。在院里问道:“林妹妹在家里呢么?”紫鹃接应
道:“是谁?”掀帘看时,笑道:“原来是宝二爷。姑娘在屋里呢,请二爷到屋里
坐着。”宝玉同着紫鹃走进来。黛玉却在里间呢,说道:“紫鹃,请二爷屋里坐罢。”
宝玉走到里间门口,看见新写的一副紫墨色泥金云龙笺的小对,上写着:“绿窗明
月在,青史古人空。”宝玉看见,笑了一笑,走入门去,笑问道:“妹妹做什么呢?”
黛玉站起来,迎了两步,笑着让道:“请坐。我在这里写经,只剩得两行了。等写
完了再说话儿。”因叫雪雁倒茶。宝玉道:“你别动,只管写。”说着,一面看见
中间挂着一幅单条,上面画着一个嫦娥,带着一个侍者;又一个女仙,也有一个侍
者,捧着一个长长儿的衣囊似的。二人身旁边略有些云护,别无点缀,全仿李龙眠
白描笔意,上有“斗寒图”三字,用八分书写着。宝玉道:“妹妹这幅斗寒图可是
新挂上的?”黛玉道:“可不是昨日他们收拾屋子,我想起来,拿出来叫他们挂上
的。”宝玉道:“是什么出处?”黛玉笑道:“眼前熟的很的,还要问人。”宝玉
笑道:“我一时想不起,妹妹告诉我罢。”黛玉道:“岂不闻‘青女素娥俱耐冷,
月中霜里斗婵娟’?”宝玉道:“是啊,这个实在新奇雅致。却好此时拿出来挂。”
说着,又东瞧瞧,西走走。

雪雁沏了茶来,宝玉吃着。又等了一会子,黛玉经才写完,站起来道:“简慢
了。”宝玉笑道:“妹妹还是这么客气。”但见黛玉身上穿着月白绣花小毛皮袄,
加上银鼠坎肩,头上挽着随常云髻,簪上一枝赤金扁簪,别无花朵。腰下系着杨妃
色绣花绵裙。真比如:
亭亭玉树临风立,冉冉香莲带露开。
宝玉因问道:“妹妹这两日弹琴来着没有?”黛玉道:“两日没弹了。因为写字已
经觉得手冷,那里还去弹琴?”宝玉道:“不弹也罢了。我想琴虽是清高之品,却
不是好东西,从没有弹琴里弹出富贵寿考来的,只有弹出忧思怨乱来的。再者,弹
琴也得心里记谱,未免费心。依我说,妹妹身子又单弱,不操这心也罢了。”黛玉
抿着嘴儿笑。宝玉指着壁上道:“这张琴可就是么?怎么这么短?”黛玉笑道:“这
张琴不是短,因我小时学抚的时候,别的琴都够不着,因此特地做起来的。虽不是
焦尾枯桐,这鹤仙凤尾还配得齐整,龙池雁足高下还相宜。你看这断纹,不是牛旄
似的么?所以音韵也还清越。”宝玉道:“妹妹这几天来做诗没有?”黛玉道:“自
结社以后,没大做。”宝玉笑道:“你别瞒我。我听见你吟的,什么‘不可,素
心如何天上月’,你搁在琴里,觉得音响分外的响亮。有的没的?”黛玉道:“你
怎么听见了?”宝玉道:“我那一天从蓼风轩来听见的,又恐怕打断你的清韵,所
以静听了一会,就走了。我正要问你:前路是平韵,到末了儿忽转了仄韵,是个什
么意思?”黛玉道:“这是人心自然之音,做到那里就到那里,原没有一定的。”
宝玉道:“原来如此。可惜我不知音,枉听了一会子。”黛玉道:“古来知音人能
有几个!”宝玉听了,又觉得出言冒失了,又怕寒了黛玉的心。坐了一坐,心里像
有许多话,却再无可讲的。黛玉因方才的话也是冲口而出,此时回想,觉得太冷淡
些,也就无话。宝玉越发打量黛玉设疑,遂讪讪的站起来说道:“妹妹坐着罢,我
还要到三妹妹那里瞧瞧去呢。”黛玉道:“你若见了三妹妹,替我问候一声罢。”
宝玉答应着,便出来了。

黛玉送至屋门口,自己回来,闷闷的坐着,心里想道:“宝玉近来说话,半吐
半吞,忽冷忽热,也不知他是什么意思。”正想着,紫鹃走来道:“姑娘,经不写
了?我把笔砚都收好了?”黛玉道:“不写了,收起去罢。”说着,自己走到里间
屋里床上歪着,慢慢的细想。紫鹃进来问道:“姑娘喝碗茶罢?”黛玉道:“不吃
呢。我略歪歪罢。你们自己去罢。”

紫鹃答应着出来,只见雪雁一个人在那里发呆。紫鹃走到他跟前,问道:“你
这会子也有了什么心事了么?”雪雁只顾发呆,倒被他吓了一跳,因说道:“你别
嚷,今日我听见了一句话,我告诉你听奇不奇。你可别言语!”说着,往屋里努嘴
儿。因自己先行,点着头儿叫紫鹃同他出来,到门外平台底下,悄悄儿的道:“姐
姐,你听见了么?宝玉定了亲了。”紫鹃听见,吓了一跳,说道:“这是那里来的
话?只怕不真罢?”雪雁道:“怎么不真!别人大概都知道,就只咱们没听见。”紫
鹃道:“你在那里听来的?”雪雁道:“我听见侍书说的,是个什么知府家,家资
也好,人才也好。”紫鹃正听时,只听见黛玉咳嗽了一声,似乎起来的光景。紫鹃
恐怕他出来听见,便拉了雪雁摇摇手儿,往里望望,不见动静,才又悄悄儿的问道:
“他到底怎么说来着?”雪雁道:“前儿不是叫我到三姑娘那里去道谢吗,三姑娘
不在屋里,只有侍书在那里。大家坐着,无意中说起宝二爷淘气来。他说:‘宝二
爷怎么好?只会玩儿,全不像大人的样子,已经说亲了,还是这么呆头呆脑。’我
问他:‘定了没有?’他说是:‘定了,是个什么王大爷做媒的。那王大爷是东府
里的亲戚,所以也不用打听,一说就成了。’”紫鹃侧着头想了一想,“这句话奇!”
又问道:“怎么家里没有人说起?”雪雁道:“侍书也说的,是老太太的意思。若
一说起,恐怕宝玉野了心,所以都不提起。侍书告诉了我,又叮咛千万不可露风说
出来,知道是我多嘴。”把手往里一指,“所以他面前也不提。今日是你问起,我
不犯瞒你。”正说到这里,只听鹦鹉叫唤,学着说:“姑娘回来了,快倒茶来!”
倒把紫鹃雪雁吓了一跳。回头并不见有人,便骂了鹦鹉一声。走进屋内,只见黛玉
喘吁吁的刚坐在椅子上。紫鹃搭讪着问茶问水。黛玉问道:“你们两个那里去了?
再叫不出一个人来。”说着,便走到炕边,将身子一歪,仍旧倒在炕上,往里躺下,
叫把帐儿撩下。紫鹃雪雁答应出去,他两个心里疑惑方才的话只怕被他听了去了,
只好大家不提。

谁知黛玉一腔心事,又窃听了紫鹃雪雁的话,虽不很明白,已听得了七八分,
如同将身撂在大海里一般。思前想后,竟应了前日梦中之谶,千愁万恨,堆上心来。
左右打算,不如早些死了,免得眼见了意外的事情,那时反倒无趣。又想到自己没
了爹娘的苦,自今以后,把身子一天一天的遭塌起来,一年半载,少不得身登清净。
打定了主意,被也不盖,衣也不添,竟是合眼装睡。紫鹃和雪雁来伺候几次,不见
动静,又不好叫唤。晚饭都不吃。点灯以后,紫鹃掀开帐子,见已睡着了,被窝都
蹬在脚后。怕他着了凉,轻轻儿拿来盖上。黛玉也不动,单待他出去,仍然褪下。
那紫鹃只管问雪雁:“今儿的话到底是真的是假的?”雪雁道:“怎么不真!”紫
鹃道:“侍书怎么知道的?”雪雁道:“是小红那里听来的。”紫鹃道:“头里咱
们说话,只怕姑娘听见了。你看刚才的神情,大有原故。今日以后,咱们倒别提这
件事了。”说着,两个人也收拾要睡。紫鹃进来看时,只见黛玉被窝又蹬下来,复
又给他轻轻盖上。一宿晚景不提。

次日,黛玉清早起来,也不叫人,独自一个呆呆的坐着。紫鹃醒来,看见黛玉
已起,便惊问道:“姑娘怎么这样早?”黛玉道:“可不是。睡得早,所以醒得早。”
紫鹃连忙起来,叫醒雪雁,伺候梳洗。那黛玉对着镜子,只管呆呆的自看。看了一
回,那珠泪儿断断连连,早已湿透了罗帕。正是:
瘦影正临春水照,卿须怜我我怜卿!
紫鹃在旁也不敢劝,只怕倒把闲话勾引旧恨来。迟了好一会,黛玉才随便梳洗了,
那眼中泪渍,终是不干。又自坐了一会,叫紫鹃道:“你把藏香点上。”紫鹃道:
“姑娘,你睡也没睡得几时,如何点香?不是要写经?”黛玉点点头儿。紫鹃道:
“姑娘今日醒得太早,这会子又写经,只怕太劳神了罢。”黛玉道:“不怕!早完
了早好!况且我也并不是为经,倒借着写字解解闷儿。以后你们见了我的字迹,就
算见了我的面儿了。”说着,那泪直流下来。紫鹃听了这话,不但不能再劝,连自
己也掌不住滴下泪来。

原来黛玉立定主意,自此以后,有意遭塌身子,茶饭无心,每日渐减下来。宝
玉下学时,也常抽空问候。只是黛玉虽有万千言语,自知年纪已大,又不便似小时
可以柔情挑逗,所以满腔心事,只是说不出来。宝玉欲将实言安慰,又恐黛玉生嗔,
反添病症。两个人见了面,只得用浮言劝慰,真真是“亲极反疏”了。那黛玉虽有
贾母王夫人等怜恤,不过请医调治,只说黛玉常病,那里知他的心病。紫鹃等虽知
其意,也不敢说。从此,一天一天的减。到半月之后,肠胃日薄一日,果然粥都不
能吃了。黛玉日间听见的话,都似宝玉娶亲的话;看见怡红院中的人,无论上下,
也像宝玉娶亲的光景。薛姨妈来看,黛玉不见宝钗,越发起疑心,索性不要人来看
望,也不肯吃药,只要速死。睡梦之中,常听见有人叫“宝二奶奶”的。一片疑心,
竟成蛇影。一日竟是绝粒,粥也不喝,恹恹一息,垂毙殆尽。

未知黛玉性命如何,且看下回分解。