\chapter{美香菱屈受贪夫棒~王道士胡诌妒妇方}

话说金桂听了,将脖项一扭,嘴唇一撇,鼻孔里哧哧两声,冷笑道:“菱角花
开,谁见香来?若是菱角香了,正经那些香花放在那里?可是不通之极!”香菱道:
“不独菱花香,就连荷叶、莲蓬,都是有一般清香的。但他原不是花香可比,若静
日静夜或清早半夜细领略了去,那一股清香比是花都好闻呢。就连菱角、鸡头、苇
叶、芦根得了风露,那一股清香也是令人心神爽快的。”金桂道:“依你说,这兰
花桂花,倒香的不好了?”香菱说到热闹头上,忘了忌讳,便接口道:“兰花桂花
的香,又非别的香可比。”一句未完,金桂的丫鬟名唤宝蟾的,忙指着香菱的脸说
道:“你可要死,你怎么叫起姑娘的名字来?”香菱猛省了,反不好意思,忙陪笑
说:“一时顺了嘴,奶奶别计较。”金桂笑道:“这有什么,你也太小心了。但只
是我想这个‘香’字到底不妥,意思要换一个字,不知你服不服?”香菱笑道:“奶
奶说那里话?此刻连我一身一体俱是奶奶的,何得换一个名字反问我服不服,叫我
如何当得起。奶奶说那一个字好,就用那一个。”金桂冷笑道:“你虽说得是,只
怕姑娘多心。”香菱笑道:“奶奶原来不知:当日买了我时,原是老太太使唤的,
故此姑娘起了这个名字。后来伏侍了爷,就与姑娘无涉了。如今又有了奶奶,越发
不与姑娘相干。且姑娘又是极明白的人,如何恼得这些呢?”金桂道:“既这样说,
‘香’字竟不如‘秋’字妥当。菱角菱花皆盛于秋,岂不比香字有来历些?”香菱
笑道:“就依奶奶这样罢了。”自此后遂改了“秋”字。宝钗亦不在意。

只因薛蟠是天性得陇望蜀的,如今娶了金桂,又见金桂的丫头宝蟾有三分姿
色,举止轻浮可爱,便时常要茶要水的故意撩逗他。宝蟾虽亦解事,只是怕金桂,
不敢造次,且看金桂的眼色。金桂亦觉察其意,想着:“正要摆布香菱,无处寻隙。
如今他既看上宝蟾,我且舍出宝蟾与他,他一定就和香菱疏远了。我再乘他疏远之
时,摆布了香菱,那时宝蟾原是我的人,也就好处了。”打定了主意,俟机而发。
这日薛蟠晚间微醺,又命宝蟾倒茶来吃。薛蟠接碗时故意捏他的手,宝蟾又乔装躲
闪,连忙缩手。两下失误,豁啷一声茶碗落地,泼了一身一地的茶。薛蟠不好意思,
佯说宝蟾不好生拿着,宝蟾说:“姑爷不好生接。”金桂冷笑道:“两个人的腔调
儿都够使的了。别打量谁是傻子!”薛蟠低头微笑不语,宝蟾红了脸出去。一时安
歇之时,金桂便故意的撵薛蟠:“别处去睡,省的得了馋痨似的。”薛蟠只是笑。
金桂道:“要做什么和我说,别偷偷摸摸的不中用。”薛蟠听了,仗着酒盖脸,就
势跪在被上,拉着金桂笑道:“好姐姐,你若把宝蟾赏了我,你要怎样就怎样。你
要活人脑子,也弄来给你。”金桂笑道:“这话好不通!你爱谁,说明了,就收在
房里,省得别人看着不雅。我可要什么呢?”薛蟠得了这话,喜的称谢不尽。是夜
曲尽丈夫之道,竭力奉承金桂。次日也不出门,只在家中厮闹,越发放大了胆了。

至午后,金桂故意出去,让个空儿与他二人,薛蟠便拉拉扯扯的起来。宝蟾心
里也知八九了,也就半推半就。正要入港,谁知金桂是有心等候的,料着在难分之
际,便叫小丫头子舍儿过来。原来这小丫头也是金桂在家从小使唤的,因他自小父
母双亡,无人看管,便大家叫他做小舍儿,专做些粗活。金桂如今有意,独唤他来
吩咐道:“你去告诉秋菱,到我屋里,将我的绢子取来,不必说我说的。”小舍儿
听了,一径去寻着秋菱,说:“菱姑娘,奶奶的绢子忘记在屋里了,你去取了来,
送上去,岂不好?”秋菱正因金桂近日每每的挫折他,不知何意,百般竭力挽回,
听了这话,忙往房里来取。不防正遇见他二人推就之际,一头撞进去了,自己倒羞
的耳面通红,转身回避不及。薛蟠自为是过了明路的,除了金桂,无人可怕,所以
连门也不掩。这会子秋菱撞来,故虽不十分在意,无奈宝蟾素日最是说嘴要强,今
既遇见秋菱,便恨无地可入,忙推开薛蟠一径跑了,口内还怨恨不绝,说他强奸力
逼。薛蟠好容易哄得上手,却被秋菱打散,不免一腔的兴头变做了一腔的恶怒,都
在秋菱身上。不容分说,赶出来啐了两口,骂道:“死娼妇!你这会子做什么来撞
尸游魂?”秋菱料事不好,三步两步,早已跑了。薛蟠再来找宝蟾,已无踪迹了。
于是只恨的骂秋菱。至晚饭后,已吃得醺醺然,洗澡时,不防水略热了些,烫了脚,
便说秋菱有意害他。他赤条精光,赶着秋菱踢打了两下。秋菱虽未受过这气苦,既
到了此时,也说不得了,只好自悲自怨,各自走开。

彼时金桂已暗和宝蟾说明,今夜令薛蟠在秋菱房中去成亲,命秋菱过来陪自己
安睡。先是秋菱不肯,金桂说他嫌腌了,再必是图安逸,怕夜里伏侍劳动。又骂
说:“你没见世面的主子,见一个爱一个,把我的丫头霸占了去,又不叫你来,到
底是什么主意?想必是逼死我就罢了!”薛蟠听了这话,又怕闹黄了宝蟾之事,忙
又赶来骂秋菱:“不识抬举,再不去就要打了!”秋菱无奈,只得抱了铺盖来。金
桂命他在地下铺着睡,秋菱只得依命。刚睡下,便叫倒茶,一时又要捶腿:如是者
一夜七八次,总不使其安逸稳卧片时。那薛蟠得了宝蟾,如获珍宝,一概都置之不
顾。恨得金桂暗暗的发恨道:“且叫你乐几天,等我慢慢的摆弄了他,那时可别怨
我!”一面隐忍,一面设计摆弄秋菱。

半月光景,忽又装起病来,只说心痛难忍,四肢不能转动,疗治不效。众人都
说是秋菱气的。闹了两天,忽又从金桂枕头内抖出个纸人来,上面写着金桂的年庚
八字,有五根针钉在心窝并肋肢骨缝等处。于是,众人当作新闻,先报与薛姨妈。
薛姨妈先忙手忙脚的,薛蟠自然更乱起来,立刻要拷打众人。金桂道:“何必冤枉
众人?大约是宝蟾的镇魔法儿。”薛蟠道:“他这些时并没多空儿在你房里,何苦
赖好人?”金桂冷笑道:“除了他还有谁?莫不是我自己害自己不成?虽有别人,如
何敢进我的房呢?”薛蟠道:“秋菱如今是天天跟着你,他自然知道,先拷问他,
就知道了。”金桂冷笑道:“拷问谁?谁肯认?依我说,竟装个不知道,大家丢开手
罢了。横竖治死我也没什么要紧,乐得再娶好的。若据良心上说,左不是你三个多
嫌我。”一面说着,一面痛哭起来。薛蟠更被这些话激怒,顺手抓起一根门闩来,
一径抢步,找着秋菱,不容分说,便劈头劈脸浑身打起来,一口只咬定是秋菱所施。
秋菱叫屈。薛姨妈跑来禁喝道:“不问明白就打起人来了!这丫头伏侍这几年,那
一时不小心?他岂肯如今做这没良心的事!你且问个清浑皂白,再动粗卤。”金桂听
见他婆婆如此说,怕薛蟠心软意活了,便泼声浪气大哭起来,说:“这半个多月,
把我的宝蟾霸占了去,不容进我的房,惟有秋菱跟着我睡。我要拷问宝蟾,你又护
在头里。你这会子又赌气打他去。治死我,再拣富贵的标致的娶来就是了,何苦做
出这些把戏来?”薛蟠听了这些话,越发着了急。

薛姨妈听见金桂句句挟制着儿子,百般恶赖的样子,十分可恨。无奈儿子偏不
硬气,已是被他挟制软惯了。如今又勾搭上丫头,被他说霸占了去,自己还要占温
柔让夫之礼。这魇魔法究竟不知谁做的?正是俗语说的好,“清官难断家务事”,
此时正是公婆难断床帏的事了。因无法,只得赌气喝薛蟠,说:“不争气的孽障,
狗也比你体面些!谁知你三不知的,把陪房丫头也摸索上了,叫老婆说霸占了丫头,
什么脸出去见人?也不知谁使的法子,也不问清就打人。我知道你是个得新弃旧的
东西,白辜负了当日的心。他既不好,你也不该打。我即刻叫人牙子来卖了他,你
就心净了。”气着,又命:“秋菱,收拾了东西,跟我来。”一面叫人去快叫个人
牙子来:“多少卖几两银子,拔去肉中刺、眼中钉,大家过太平日子!”薛蟠见母
亲动了气,早已低了头。金桂听了这话,便隔着窗子,往外哭道:“你老人家只管
卖人,不必说着一个、拉着一个的。我们很是那吃醋拈酸容不得下人的不成?怎么
‘拔去肉中刺、眼中钉’?是谁的钉?谁的刺?但凡多嫌着他,也不肯把我的丫鬟也
收在房里了。”薛姨妈听说,气得身战气咽,道:“这是谁家的规矩?婆婆在这里
说话,媳妇隔着窗子拌嘴!亏你是旧人家的女儿,满嘴里大呼小喊,说的是什么!”
薛蟠急得跺脚,说:“罢哟,罢哟!看人家听见笑话。”金桂意谓一不做,二不休,
越发喊起来了,说:“我不怕人笑话!你的小老婆治害我,我倒怕人笑话了?再不然,
留下他,卖了我。谁还不知道薛家有钱,行动拿钱垫人,又有好亲戚,挟制着别人!
你不趁早施为,还等什么?嫌我不好,谁叫你们瞎了眼,三求四告的,跑了我们家
做什么去了?”一面哭喊,一面自己拍打。薛蟠急得说又不好,劝又不好,打又不
好,央告又不好,只是出入嗳声叹气,抱怨说运气不好。

当下薛姨妈被宝钗劝进去了,只命人来卖香菱。宝钗笑道:“咱们家只知买人,
并不知卖人之说,妈妈可是气糊涂了。倘或叫人听见,岂不笑话?哥哥嫂子嫌他不
好,留着我使唤,我正也没人呢。”薛姨妈道:“留下他还是惹气,不如打发了他
干净。”宝钗笑道:“他跟着我也是一样,横竖不叫他到前头去。从此,断绝了他
那里,也和卖了的一样。”香菱早已跑到薛姨妈跟前,痛哭哀求,不愿出去,情愿
跟姑娘。薛姨妈只得罢了。自此,后来香菱果跟随宝钗去了,把前面路径竟自断绝。
虽然如此,终不免对月伤悲,挑灯自叹。虽然在薛蟠房中几年,皆因血分中有病,
是以并无胎孕。今复加以气怒伤肝,内外折挫不堪,竟酿成干血之症,日渐羸瘦,
饮食懒进,请医服药不效。

那时金桂又吵闹了数次,薛蟠有时仗着酒胆,挺撞过两次。持棍欲打,那金桂
便递身叫打;这里持刀欲杀时,便伸着脖项。薛蟠也实不能下手,只得乱了一阵罢
了。如今已成习惯自然,反使金桂越长威风。又惭次辱嗔宝蟾。宝蟾比不得香菱,
正是个烈火干柴,既和薛蟠情投意合,便把金桂放在脑后。近见金桂又作践他,他
便不肯低服半点。先是一冲一撞的拌嘴;后来金桂气急,甚至于骂,再至于打。他
虽不敢还手,便也撒泼打滚,寻死觅活,昼则刀剪,夜则绳索,无所不闹。薛蟠一
身难以两顾,惟徘徊观望,十分闹得无法,便出门躲着。金桂不发作性气,有时喜
欢,便纠聚人来斗牌掷骰行乐。又生平最喜啃骨头,每日务要杀鸡鸭,将肉赏人吃,
只单是油炸的焦骨头下酒。吃得不耐烦,便肆行海骂,说:“有别的忘八粉头乐的,
我为什么不乐。”薛家母女总不去理他,惟暗里落泪。薛蟠亦无别法,惟悔恨不该
娶这“搅家精”,都是一时没了主意。于是宁荣二府之人,上上下下,无有不知,
无有不叹者。

此时宝玉已过了百日,出门行走。亦曾过来见过金桂:举止形容也不怪厉,一
般是鲜花嫩柳,与众姊妹不差上下,焉得这等情性?可为奇事。因此,心中纳闷。
这日,与王夫人请安去,又正遇见迎春奶娘来家请安,说起孙绍祖甚属不端,“姑
娘惟有背地里淌眼泪,只要接了家来,散荡两日。”王夫人因说:“我正要这两日
接他去,只是七事八事的都不遂心,所以就忘了。前日宝玉去了,回来也曾说过的。
明日是个好日子,就接他去。”正说时,贾母打发人来找宝玉,说:“明儿一早往
天齐庙还愿去。”宝玉如今巴不得各处去逛逛,听见如此,喜的一夜不曾合眼。

次日一早,梳洗穿戴已毕,随了两三个老嬷嬷,坐车出西城门外天齐庙烧香还
愿。这庙里已于昨日预备停妥的。宝玉天性怯懦,不敢近狰狞神鬼之像,是以忙忙
的焚过纸马钱粮,便退至道院歇息。一时吃饭毕,众嬷嬷和李贵等围随宝玉到各处
玩耍了一回,宝玉困倦,复回至净室安歇。众嬷嬷生恐他睡着了,便请了当家的老
王道士来陪他说话儿。这老道士专在江湖上卖药,弄些海上方治病射利,庙外现挂
着招牌,丸散膏药,色色俱备。亦长在宁荣二府走动惯熟,都给他起了个混号,唤
他做“王一贴”:言他膏药灵验,一贴病除。当下王一贴进来。宝玉正歪在炕上,
看见王一贴进来,便笑道:“来的好。我听见说你极会说笑话儿的,说一个给我们
大家听听。”王一贴笑道:“正是呢,哥儿别睡,仔细肚子里面筋作怪。”说着,
满屋里的都笑了,宝玉也笑着起身整衣。王一贴命徒弟们:“快沏好茶来。”焙茗
道:“我们爷不吃你的茶,坐在这屋里还嫌膏药气息呢。”王一贴笑道:“不当家
花拉的!膏药从不拿进屋里来的。知道二爷今日必来,三五日头里就拿香熏了。”
宝玉道:“可是呢,天天只听见说你的膏药好,到底治什么病?”王一贴道:“若
问我的膏药,说来话长,其中底细,一言难尽:共药一百二十味,君臣相际,温凉
兼用。内则调元补气,养荣卫,开胃口,宁神定魄,去寒去暑,化食化痰;外则和
血脉,舒筋络,去死生新,去风散毒。其效如神,贴过便知。”宝玉道:“我不信
一张膏药就治这些病?我且问你,倒有一种病,也贴得好么?”王一贴道:“百病
千灾,无不立效。若不效,二爷只管揪胡子,打我这老脸,拆我这庙,何如?只说
出病源来。”宝玉道:“你猜。若猜得着,便贴得好了。”王一贴听了,寻思一会,
笑道:“这倒难猜,只怕膏药有些不美了。”宝玉命他坐在身边。王一贴心动,便
笑着悄悄的说道:“我可猜着了。想是二爷如今有了房中的事情,要滋助的药可是
不是?”话犹未完,焙茗先喝道:“该死,打嘴!”宝玉犹未解,忙问:“他说什
么?”焙茗道:“信他胡说!”唬得王一贴不等再问,只说:“二爷明说了罢。”
宝玉道:“我问你,可有贴女人的妒病的方子没有?”王一贴听了,拍手笑道:“这
可罢了,不但说没有方子,就是听也没有听见过。”宝玉笑道:“这样还算不得什
么!”王一贴又忙道:“这贴妒的膏药倒没经过。有一种汤药,或者可医,只是慢
些儿,不能立刻见效的。”宝玉道:“什么汤?怎样吃法?”

王一贴道:“这叫做‘疗妒汤’:用极好的秋梨一个,二钱冰糖,一钱陈皮,
水三碗,梨熟为度。每日清晨吃这一个梨,吃来吃去就好了。”宝玉道:“这也不
值什么。只怕未必见效。”王一贴道:“一剂不效,吃十剂;今日不效,明日再吃;
今年不效,明年再吃。横竖这三味药都是润肺开胃不伤人的,甜丝丝的,又止咳嗽,
又好吃。吃过一百岁,人横竖是要死的,死了还妒什么?那时就见效了。”说着,
宝玉焙茗都大笑不止,骂“油嘴的牛头”。王一贴道:“不过是闲着解午盹罢了,
有什么关系?说笑了你们就值钱。告诉你们说:连膏药也是假的。我有真药,我还
吃了做神仙呢,有真的跑到这里来混?”正说着,吉时已到,请宝玉出去奠酒,焚
化钱粮,散福。功课完毕,宝玉方进城回家。

那时迎春已来家好半日,孙家婆娘媳妇等人已待晚饭,打发回家去了。迎春方
哭哭啼啼,在王夫人房中诉委屈,说:“孙绍祖一味好色,好赌,酗酒,家中所有
的媳妇丫头,将及淫遍。略劝过两三次,便骂我是‘醋汁子老婆拧出来的’。又说
老爷曾收着五千银子,不该使了他的。如今他来要了两三次不得,便指着我的脸说
道:‘你别和我充夫人娘子!你老子使了我五千银子,把你准折卖给我的。好不好,
打你一顿,撵到下房里睡去。当日有你爷爷在时,希冀上我们的富贵,赶着相与的。
论理我和你父亲是一辈,如今压着我的头晚了一辈,不该做了这门亲,倒没的叫人
看着赶势利似的。’”一行说,一行哭的呜呜咽咽,连王夫人并众姊妹无不落泪。
王夫人只得用言解劝,说:“已是遇见不晓事的人,可怎么样呢?想当日你叔叔也
曾劝过大老爷,不叫做这门亲的;大老爷执意不听,一心情愿。到底做不好了。我
的儿,这也是你的命。”迎春哭道:“我不信我的命就这么苦?从小儿没有娘,幸
而过婶娘这边来,过了几年心净日子。如今偏又是这么个结果。”王夫人一面劝,
一面问他随意要在那里安歇。迎春道:“乍乍的离了姊妹们,只是眠思梦想;二则
还惦记着我的屋子。还得在园里住个三五天,死也甘心了。不知下次来还得住不得
住了呢。”王夫人忙劝道:“快休乱说。年轻的夫妻们,斗牙斗齿,也是泛泛人的
常事,何必说这些丧话?”仍命人忙忙的收拾紫菱洲房屋,命姊妹们陪伴着解释。
又吩咐宝玉:“不许在老太太跟前走漏一些风声。倘或老太太知道了这些事,都是
你说的。”宝玉唯唯的听命。

迎春是夕仍在旧馆安歇。众姐妹丫鬟等更加亲热异常。一连住了三日,才往邢
夫人那边去。先辞过贾母及王夫人,然后与众姐妹分别,各皆悲伤不舍。还是王夫
人薛姨妈等安慰劝释,方止住了,过那边去。又在邢夫人处住了两日,就有孙家的
人来接去。迎春虽不愿去,无奈孙绍祖之恶,勉强忍情作辞去了。邢夫人本不在意,
也不问其夫妻和睦、家务烦难,只面情塞责而已。

要知后事,下回分解。