\chapter{林黛玉重建桃花社~史湘云偶填柳絮词}

话说贾琏自在梨香院伴宿七日夜,天天僧道不断做佛事。贾母唤了他去,吩咐
不许送往家庙中,贾琏无法,只得又和时觉说了,就在尤三姐之上,点了一个穴,
破土埋葬。那日送殡,只不过族中人与王姓夫妇、尤氏婆媳而已。

凤姐一应不管,只凭他自去办理。又因年近岁逼,诸事烦杂不算外,又有林之
孝开了一个人单子来回:共有八个二十五岁的单身小厮,应该娶妻成房的,等里面
有该放的丫头,好求指配。凤姐看了,先来问贾母和王夫人。大家商议,虽有几个
应该发配的,奈各人皆有缘故:第一个鸳鸯,发誓不去。自那日之后,一向未与宝
玉说话,也不盛妆浓饰。众人见他志坚,也不好相强。第二个琥珀,现又有病,这
次不能了。彩云因近日和贾环分崩,也染了无医之症。只有凤姐儿和李纨房中粗使
的大丫头发出去了。其馀年纪未足,令他们外头自娶去了。

原来这一向因凤姐儿病了,李纨探春料理家务,不得闲暇。接着过年过节,许
多杂事,竟将诗社搁起。如今仲春天气,虽得了工夫,争奈宝玉因柳湘莲遁迹空门,
又闻得尤三姐自刎,尤二姐被凤姐逼死,又兼柳五儿自那夜监禁之后,病越重了:
连连接接,闲愁胡恨,一重不了一重添,弄的情色若痴,语言常乱,似染怔仲之病。
慌的袭人等又不敢回贾母,只百般逗他玩笑。

这日清晨方醒,只听得外间屋内咭咭呱呱,笑声不断。袭人因笑说:“你快出
去拉拉罢,晴雯和麝月两个人按住芳官那里隔肢呢。”宝玉听了,忙披上灰鼠长袄
出来一瞧,只见他三人被褥尚未叠起,大衣也未穿:那晴雯只穿着葱绿杭绸小袄,
红绸子小衣儿,披着头发,骑在芳官身上。麝月是红绫抹胸,披着一身旧衣,在那
里抓芳官的肋肢,芳官却仰在炕上,穿着撒花紧身儿,红裤绿袜,两脚乱蹬,笑的
喘不过气来。宝玉忙笑说:“两个大的欺负一个小的!等我来挠你们。”说着也上
床来隔肢晴雯。晴雯触痒,笑的忙丢下芳官,来合宝玉对抓,芳官趁势将晴雯按倒。
袭人看他四人滚在一处,倒好笑,因说道:“仔细冻着了可不是玩的,都穿上衣裳
罢。”忽见碧月进来说:“昨儿晚上,奶奶在这里把块绢子忘了去,不知可在这里
没有?”春燕忙应道:“有。我在地下捡起来,不知是那一位的,才洗了,刚晾着,
还没有干呢。”碧月见他四人乱滚,因笑道:“倒是你们这里热闹,大清早起就咭
咭呱呱的玩成一处。”宝玉笑道:“你们那里人也不少,怎么不玩?”碧月道:“我
们奶奶不玩,把两个姨娘和姑娘也都拘住了。如今琴姑娘跟了老太太前头去,更冷
冷清清的了。两个姨娘到明年冬天,也都家去了,更那才冷清呢。你瞧瞧,宝姑娘
那里出去了一个香菱,就像短了多少人似的,把个云姑娘落了单了。”正说着,见
湘云又打发了翠缕来说:“请二爷快出去瞧好诗。”宝玉听了,忙梳洗出去。

果见黛玉、宝钗、湘云、宝琴、探春,都在那里,手里拿着一篇诗看。见他来
时,都笑道:“这会子还不起来!咱们的诗社散了一年,也没有一个人作兴作兴。
如今正是初春时节,万物更新,正该鼓舞另立起来才好。”湘云笑道:“一起诗社
时是秋天,就不发达。如今却好万物逢春,咱们重新整理起这个社来,自然要有生
趣了。况这首‘桃花诗’又好,就把海棠社改作桃花社,岂不大妙呢?”宝玉听着
点头,说:“很好。”且忙着要诗看。众人都又说:“咱们此时就访稻香老农去,
大家议定好起社。”说着,一齐站起来,都往稻香村来。宝玉一壁走,一壁看,写
着是:

桃花行
桃花帘外东风软,桃花帘内晨妆懒。
帘外桃花帘内人,人与桃花隔不远。
东风有意揭帘栊,花欲窥人帘不卷。
桃花帘外开仍旧,帘中人比桃花瘦。
花解怜人花亦愁,隔帘消息风吹透。
风透帘栊花满庭,庭前春色倍伤情。
闲苔院落门空掩,斜日栏杆人自凭。
凭栏人向东风泣,茜裙偷傍桃花立。
桃花桃叶乱纷纷,花绽新红叶凝碧。
树树烟封一万株,烘楼照壁红模糊。
天机烧破鸳鸯锦,春酣欲醒移珊枕。
侍女金盆进水来,香泉饮蘸胭脂冷。
胭脂鲜艳何相类,花之颜色人之泪。
若将人泪比桃花,泪自长流花自媚。
泪眼观花泪易干,泪干春尽花憔悴。
憔悴花遮憔悴人,花飞人倦易黄昏。
一声杜宇春归尽,寂寞帘栊空月痕。
宝玉看了,并不称赞,痴痴呆呆,竟要滚下泪来。又怕众人看见,忙自己拭了。因
问:“你们怎么得来?”宝琴笑道:“你猜是谁做的?”宝玉笑道:“自然是潇湘
子的稿子了。”宝琴笑道:“现在是我做的呢。”宝玉笑道:“我不信。这声调口
气,迥乎不像。”宝琴笑道:“所以你不通。难道杜工部首首都作‘丛菊两开他日
泪’不成?一般的也有‘红绽雨肥梅’、‘水荇牵风翠带长’等语。”宝玉笑道:
“固然如此,但我知道姐姐断不许妹妹有此伤悼之句。妹妹本有此才,却也断不肯
做的。比不得林妹妹曾经离丧,作此哀音。”众人听说,都笑了。

已至稻香村中,将诗与李纨看了,自不必说,称赏不已。说起诗社,大家议定:
明日乃三月初二日,就起社,便改“海棠社”为“桃花社”,黛玉为社主。明日饭
后,齐集潇湘馆。因又大家拟题。黛玉便说:“大家就要《桃花诗》一百韵。”宝
钗道:“使不得。古来桃花诗最多,纵作了必落套,比不得你这一首古风。须得再
拟。”正说着,人回:“舅太太来了,请姑娘们出去请安。”因此大家都往前头来
见王子胜的夫人,陪着说话。饭毕,又陪着入园中来游玩一遍,至晚饭后掌灯方去。

次日乃是探春的寿日,元春早打发了两个小太监,送了几件玩器。合家皆有寿
礼,自不必细说。饭后,探春换了礼服,各处行礼。黛玉笑向众人道:“我这一社
开的又不巧了,偏忘了这两日是他的生日。虽不摆酒唱戏,少不得都要陪他在老太
太、太太跟前玩笑一日,如何能得闲空儿?”因此,改至初五。

这日,众姊妹皆在房中侍早膳毕,便有贾政书信到了。宝玉请安,将请贾母的
安禀拆开,念与贾母听。上面不过是请安的话,说六月准进京等语。其馀家信事物
之帖,自有贾琏和王夫人开读。众人听说六七月回京,都喜之不尽。偏生这日王子
胜将侄女许与保宁侯之子为妻,择于五月间过门,凤姐儿又忙着张罗,常三五日不
在家。这日王子胜的夫人又来接凤姐儿,一并请众甥男甥女乐一日。贾母和王夫人
命宝玉、探春、黛玉、宝钗四人同凤姐儿去,众人不敢违拗,只得回房去另妆饰了
起来。五人去了一日,掌灯方回。

宝玉进入怡红院,歇了半刻,袭人便乘机劝他收一收心,闲时把书理一理,好
预备着。宝玉屈指算了一算,说:“还早呢。”袭人道:“书还是第二件。到那时
纵然你有了书,你的字写的在那里呢?”宝玉笑道:“我时常也有写了的好些,难
道都没收着?”袭人道:“何曾没收着。你昨儿不在家,我就拿出来,统共数了一
数,才有五百六十几篇。这二三年的工夫,难道只有这几张字不成?依我说,明日
起把别的心先都收起来,天天快临几张字补上。虽不能按日都有,也要大概看的过
去。”宝玉听了,忙着自己又亲检了一遍,实在搪塞不过。便说:“明日为始,一
天写一百字才好。”说话时,大家睡下。至次日起来,梳洗了,便在窗下恭楷临帖。

贾母因不见他,只当病了,忙使人来问。宝玉方去请安,便说:“写字之故,
因此出来迟了。”贾母听说,十分喜欢,就吩咐他:“以后只管写字,念书,不用
出来也使得。你去回你太太知道。”宝玉听说,遂到王夫人屋里来说明。王夫人便
道:“临阵磨枪也不中用。有这会子着急,天天写写念念,有多少完不了的?这一
赶,又赶出病来才罢。”宝玉回说:“不妨事。”宝钗探春等都笑说:“太太不用
着急,书虽替不得他,字却替得的。我们每日每人临一篇给他,搪塞过这一步儿去
就完了,一则老爷不生气,二则他也急不出病来。”王夫人听说,点头而笑。

原来黛玉闻得贾政回家,必问宝玉的功课,宝玉一向分心,到临期自然要吃亏
的。因自己只装不耐烦,把诗社更不提起。探春宝钗二人,每日也临一篇楷书字与
宝玉。宝玉自己每日也加功,或写二百三百不拘。至三月下旬,便将字又积了许多。
这日正算着再得几十篇,也就搪的过了。谁知紫鹃走来,送了一卷东西,宝玉拆开
看时,却是一色去油纸上临的钟王蝇头小楷,字迹且与自己十分相类。喜的宝玉和
紫鹃作了一个揖,又亲自来道谢。接着湘云宝琴二人也都临了几篇相送。凑成虽不
足功课,亦可搪塞了。宝玉放了心,于是将应读之书,又温理过几次。正是天天用
功,可巧近海一带海啸,又遭塌了几处生民,地方官题本奏闻,奉旨就着贾政顺路
查看赈济回来。如此算去,至七月底方回。宝玉听了,便把书字又丢过一边,仍是
照旧游荡。

时值暮春之际,湘云无聊,因见柳花飘舞,便偶成一小词,调寄《如梦令》。
其词曰:

岂是绣绒才吐。卷起半帘香雾。纤手自拈来,空使鹃啼燕妒。且住,且住。莫
使春光别去。
自己做了,心中得意,便用一条纸儿写好给宝钗看了。又来找黛玉,黛玉看毕笑道:
“好的很,又新鲜,又有趣儿。”湘云说道:“咱们这几社总没有填词,你明日何
不起社填词,岂不新鲜些?”黛玉听了,偶然兴动,便说:“这话也倒是。”湘云
道:“咱们趁今日天气好,为什么不就是今日?”黛玉道:“也使得。”说着,一
面吩咐预备了几色果点,一面就打发人分头去请。这里二人便拟了“柳絮”为题,
又限出几个调来,写了粘在壁上。众人来看时:“以柳絮为题,限各色小调。”又
都看了湘云的,称赏了一回。宝玉笑道:“这词上我倒平常,少不得也要胡诌了。”
于是大家拈阄。宝钗炷了一支梦甜香,大家思索起来。

一时黛玉有了,写完。接着宝琴也忙写出来。宝钗笑道:“我已有了。瞧了你
们的,再看我的。”探春笑道:“今儿这香怎么这么快?我才有了半首。”因又问
宝玉:“你可有了?”宝玉虽做了些,自己嫌不好,又都抹了,要另做,回头看香
已尽了。李纨等笑道:“宝玉又输了。蕉丫头的呢?”探春听说,便写出来。众人
看时,上面却只半首《南柯子》,写道是:

空挂纤纤缕,徒垂络络丝。也难绾系也难羁,一任东西南北各分离。
李纨笑道:“这却也好。何不再续上?”宝玉见香没了,情愿认输,不肯勉强塞责,
将笔搁下,来瞧这半首。见没完时,反倒动了兴,乃提笔续道:

落去君休惜,飞来我自知。莺愁蝶倦晚芳时,纵是明春再见隔年期。
众人笑道:“正经你分内的又不能,这却偏有了。纵然好,也算不得。”说着,看
黛玉的,是一阕《唐多令》:

粉堕百花洲,香残燕子楼。一团团逐队成球。漂泊亦如人命薄,空缱绻,说风
流。

草木也知愁,韶华竟白头。叹今生谁舍谁收。嫁与东风春不管,凭尔去,
忍淹留?
众人看了,俱点头感叹说:“太作悲了。好是果然好的。”因又看宝琴的《西江月》:

汉苑零星有限,隋堤点缀无穷。三春事业付东风。明月梨花一梦。

几处落
红庭院,谁家香雪帘栊?江南江北一般同。偏是离人恨重。
众人都笑说:“到底是他的声调悲壮。‘几处’、‘谁家’两句最妙。”

宝钗笑道:“总不免过于丧败。我想柳絮原是一件轻薄无根的东西,依我的主
意,偏要把他说好了,才不落套。所以我诌了一首来,未必合你们的意思。”众人
笑道:“别太谦了,自然是好的,我们赏鉴赏鉴。”因看这一阕《临江仙》道:
白玉堂前春解舞,东风卷得均匀。
湘云先笑道:“好一个‘东风卷得均匀’,这一句就出人之上了。”

蜂围蝶阵乱纷纷:几曾随逝水?岂必委芳尘?

万缕千丝终不改,任他随聚随
分。韶华休笑本无根:好风凭借力,送我上青云。
众人拍案叫绝,都说:“果然翻的好。自然这首为尊。缠绵悲戚,让潇湘子;情致
妩媚,却是枕霞;小薛与蕉客今日落第,要受罚的。”宝琴笑道:“我们自然受罚。
但不知交白卷子的,又怎么罚?”李纨道:“不用忙,这定要重重的罚他,下次为
例。”

一语未了,只听窗外竹子上一声响,恰似窗屉子倒了一般,众人吓了一跳。丫
鬟们出去瞧时,帘外丫头子们回道:“一个大蝴蝶风筝,挂在竹梢上了。”众丫鬟
笑道:“好一个齐整风筝。不知是谁家放的,断了线?咱们拿下他来。”宝玉等听
了,也都出来看时,宝玉笑道:“我认得这风筝,这是大老爷那院里嫣红姑娘放的。
拿下来给他送过去罢。”紫鹃笑道:“难道天下没有一样的风筝,单他有这个不成?
二爷也太死心眼儿了。我不管,我且拿起来。”探春笑道:“紫鹃也太小器,你们
一般有的,这会子拾人走了的,也不嫌个忌讳?”黛玉笑道:“可是呢。把咱们的
拿出来,咱们也放放晦气。”

丫头们听见放风筝,巴不得一声儿,七手八脚,都忙着拿出来,也有美人儿的,
也有沙雁儿的。丫头们搬高墩,捆剪子股儿,一面拨起子来。宝钗等立在院门前,
命丫头们在院外敞地下放去。宝琴笑道:“你这个不好看,不如三姐姐的一个软翅
子大凤凰好。”宝钗回头向翠墨笑道:“你去把你们的拿来也放放。”宝玉又兴头
起来,也打发个小丫头子家去,说:“把昨日赖大娘送的那个大鱼取来。”小丫头
去了半天,空手回来,笑道:“晴雯姑娘昨儿放走了。”宝玉道:“我还没放一遭
儿呢。”探春笑道:“横竖是给你放晦气罢了。”宝玉道:“再把大螃蟹拿来罢。”
丫头去了,同了几个人,扛了一个美人并子来,回说:“袭姑娘说:昨儿把螃蟹
给了三爷了,这一个是林大娘才送来的,放这一个罢。”宝玉细看了一回,只见这
美人做的十分精致,心中欢喜,便叫放起来。此时探春的也取了来了,丫头们在那
山坡上已放起来。宝琴叫丫头放起一个大蝙蝠来,宝钗也放起个一连七个大雁来。
独有宝玉的美人儿,再放不起来。宝玉说丫头们不会放,自己放了半天,只起房高,
就落下来,急的头上的汗都出来了。众人都笑他,他便恨的摔在地下,指着风筝说
道:“要不是个美人儿,我一顿脚跺个稀烂!”黛玉笑道:“那是顶线不好。拿去
叫人换好了,就好放了。再取一个来放罢。”

宝玉等大家都仰面,看天上这几个风筝起在空中。一时风紧,众丫鬟都用绢子
垫着手放。黛玉见风力紧了,过去将子一松,只听豁喇喇一阵响,登时线尽,风
筝随风去了。黛玉因让众人来放。众人都说:“林姑娘的病根儿都放了去了,咱们
大家都放了罢。”于是丫头们拿过一把剪子来,铰断了线。那风筝都飘飘摇摇随风
而去,一时只有鸡蛋大,一展眼只剩下一点黑星儿,一会儿就不见了。众人仰面说
道:“有趣,有趣!”说着,有丫头来请吃饭,大家方散。

从此宝玉的工课,也不敢像先竟撂在脖子后头了,有时写写字,有时念念书。
闷了也出来,合姐妹们玩笑半天,或往潇湘馆去闲话一回。众姐妹都知他工课亏欠,
大家自去吟诗取乐,或讲习针黹,也不肯去招他。那黛玉更怕贾政回来宝玉受气,
每每推睡,不大兜揽他。宝玉也只得在自己屋里,随便用些工课。

展眼已是夏末秋初。一日,贾母处两个丫头,匆匆忙忙来叫宝玉。

不知何事,下回分解。