\chapter{薄命女偏逢薄命郎~葫芦僧判断葫芦案}

却说黛玉同姐妹们至王夫人处,见王夫人正和兄嫂处的来使计议家务,又说姨
母家遭人命官司等语。因见王夫人事情冗杂,姐妹们遂出来,至寡嫂李氏房中来了。
原来这李氏即贾珠之妻。珠虽夭亡,幸存一子,取名贾兰,今方五岁,已入学攻书。
这李氏亦系金陵名宦之女,父名李守中,曾为国子祭酒;族中男女无不读诗书者。
至李守中继续以来,便谓“女子无才便是德”,故生了此女不曾叫他十分认真读书,
只不过将些《女四书》、《列女传》读读,认得几个字,记得前朝这几个贤女便了。
却以纺绩女红为要,因取名为李纨,字宫裁。所以这李纨虽青春丧偶,且居处于膏
粱锦绣之中,竟如槁木死灰一般,一概不问不闻,惟知侍亲养子,闲时陪侍小姑等
针黹诵读而已。今黛玉虽客居于此,已有这几个姑嫂相伴,除老父之外,馀者也就
无用虑了。

如今且说贾雨村授了应天府,一到任就有件人命官司详至案下,却是两家争买
一婢,各不相让,以致殴伤人命。彼时雨村即拘原告来审。那原告道:“被打死的
乃是小人的主人。因那日买了个丫头,不想系拐子拐来卖的。这拐子先已得了我家
的银子,我家小主人原说第三日方是好日,再接入门;这拐子又悄悄的卖与了薛家。
被我们知道了,去找拿卖主,夺取丫头。无奈薛家原系金陵一霸,倚财仗势,众豪
奴将我小主人竟打死了。凶身主仆已皆逃走,无有踪迹,只剩了几个局外的人。小
人告了一年的状,竟无人作主。求太老爷拘拿凶犯,以扶善良,存殁感激大恩不尽!”
雨村听了,大怒道:“那有这等事!打死人竟白白的走了拿不来的?”便发签差公
人立刻将凶犯家属拿来拷问。只见案旁站着一个门子,使眼色不叫他发签。雨村心
下狐疑,只得停了手。退堂至密室,令从人退去,只留这门子一人伏侍。门子忙上
前请安,笑问:“老爷一向加官进禄,八九年来,就忘了我了?”雨村道:“我看
你十分眼熟,但一时总想不起来。”门子笑道:“老爷怎么把出身之地竟忘了!老
爷不记得当年葫芦庙里的事么?”雨村大惊,方想起往事。原来这门子本是葫芦庙
里一个小沙弥,因被火之后无处安身,想这件生意倒还轻省,耐不得寺院凄凉,遂
趁年纪轻,蓄了发,充当门子。雨村那里想得是他?便忙携手笑道:“原来还是故
人。”因赏他坐了说话。这门子不敢坐,雨村笑道:“你也算贫贱之交了,此系私
室,但坐不妨。”门子才斜签着坐下。

雨村道:“方才何故不令发签?”门子道:“老爷荣任到此,难道就没抄一张
本省的护官符来不成?”雨村忙问:“何为护官符?”门子道:“如今凡作地方官
的,都有一个私单,上面写的是本省最有权势极富贵的大乡绅名姓,各省皆然。倘
若不知,一时触犯了这样的人家,不但官爵,只怕连性命也难保呢!——所以叫做
护官符。方才所说的这薛家,老爷如何惹得他!他这件官司并无难断之处,从前的
官府都因碍着情分脸面,所以如此。”一面说,一面从顺袋中取出一张抄的护官符
来,递与雨村看时,上面皆是本地大族名宦之家的俗谚口碑,云:

贾不假,白玉为堂金作马。阿房宫,三百里,住不下金陵一个史。东海缺少白
玉床,龙王来请金陵王。丰年好大“雪”,珍珠如土金如铁。
雨村尚未看完,忽闻传点,报“王老爷来拜”。雨村忙具衣冠接迎。有顿饭工夫方
回来,问这门子,门子道:“四家皆连络有亲,一损俱损,一荣俱荣。今告打死人
之薛,就是‘丰年大雪’之薛。不单靠这三家,他的世交亲友在都在外的本也不少,
老爷如今拿谁去?”

雨村听说,便笑问门子道:“这样说来,却怎么了结此案?你大约也深知这凶
犯躲的方向了?”门子笑道:“不瞒老爷说,不但这凶犯躲的方向,并这拐的人我
也知道,死鬼买主也深知道,待我细说与老爷听。这个被打死的是一个小乡宦之子,
名唤冯渊,父母俱亡,又无兄弟,守着些薄产度日,年纪十八九岁,酷爱男风,不
好女色。这也是前生冤孽,可巧遇见这丫头,他便一眼看上了,立意买来作妾,设
誓不近男色,也不再娶第二个了。所以郑重其事,必得三日后方进门。谁知这拐子
又偷卖与薛家,他意欲卷了两家的银子逃去。谁知又走不脱,两家拿住,打了个半
死,都不肯收银,各要领人。那薛公子便喝令下人动手,将冯公子打了个稀烂,抬
回去三日竟死了。这薛公子原择下日子要上京的,既打了人夺了丫头,他便没事人
一般,只管带了家眷走他的路,并非为此而逃:这人命些些小事,自有他弟兄奴仆
在此料理。这且别说,老爷可知这被卖的丫头是谁?”雨村道:“我如何晓得?”
门子冷笑道:“这人还是老爷的大恩人呢!他就是葫芦庙旁住的甄老爷的女儿,小
名英莲的。”雨村骇然道:“原来是他!听见他自五岁被人拐去,怎么如今才卖呢?”

门子道:“这种拐子单拐幼女,养至十二三岁,带至他乡转卖。当日这英莲,
我们天天哄他玩耍,极相熟的,所以隔了七八年,虽模样儿出脱的齐整,然大段未
改,所以认得,且他眉心中原有米粒大的一点胭脂,从胎里带来的。偏这拐子又
租了我的房子居住。那日拐子不在家,我也曾问他,他说是打怕了的,万不敢说,
只说拐子是他的亲爹,因无钱还债才卖的。再四哄他,他又哭了,只说:‘我原不
记得小时的事!’这无可疑了。那日冯公子相见了,兑了银子,因拐子醉了,英莲
自叹说:‘我今日罪孽可满了!’后又听见三日后才过门,他又转有忧愁之态。我
又不忍,等拐子出去,又叫内人去解劝他:‘这冯公子必待好日期来接,可知必不
以丫鬟相看。况他是个绝风流人品,家里颇过得,素性又最厌恶堂客,今竟破价买
你,后事不言可知。只耐得三两日,何必忧闷?’他听如此说方略解些,自谓从此
得所。谁料天下竟有不如意事,第二日,他偏又卖与了薛家!若卖与第二家还好,
这薛公子的混名,人称他‘呆霸王’,最是天下第一个弄性尚气的人,而且使钱如
土。只打了个落花流水,生拖死拽把个英莲拖去,如今也不知死活。这冯公子空喜
一场,一念未遂,反花了钱,送了命,岂不可叹!”

雨村听了,也叹道:“这也是他们的孽障遭遇,亦非偶然,不然这冯渊如何偏
只看上了这英莲?这英莲受了拐子这几年折磨,才得了个路头,且又是个多情的,
若果聚合了倒是件美事,偏又生出这段事来。这薛家纵比冯家富贵,想其为人,自
然姬妾众多,淫佚无度,未必及冯渊定情于一人。这正是梦幻情缘,恰遇见一对薄
命儿女!且不要议论他人,只目今这官司如何剖断才好?”门子笑道:“老爷当年
何其明决,今日何反成个没主意的人了?小的听见老爷补升此任,系贾府王府之力;
此薛蟠即贾府之亲:老爷何不顺水行舟做个人情,将此案了结,日后也好去见贾王
二公?”雨村道:“你说的何尝不是。但事关人命,蒙皇上隆恩起复委用,正竭力
图报之时,岂可因私枉法,是实不忍为的。”门子听了冷笑道:“老爷说的自是正
理,但如今世上是行不去的。岂不闻古人说的:‘大丈夫相时而动。’又说:‘趋
吉避凶者为君子。’依老爷这话,不但不能报效朝廷,亦且自身不保,还要三思为
妥!”

雨村低了头,半日说道:“依你怎么着?”门子道:“小人已想了个很好的主
意在此:老爷明日坐堂,只管虚张声势,动文书发签拿人,凶犯自然是拿不来的,
原告固是不依,只用将薛家族人及奴仆人等拿几个来拷问,小的在暗中调停,令他
们报个‘暴病身亡’,合族中及地方上共递一张保呈,老爷只说善能扶鸾请仙,堂
上设了乩坛,令军民人等只管来看。老爷便说:‘乩仙批了,死者冯渊与薛蟠原系
夙孽,今狭路相遇,原因了结。今薛蟠已得了无名之病,被冯渊的魂魄追索而死。
其祸皆由拐子而起,除将拐子按法处治外,馀不累及……’等语。小人暗中嘱咐拐
子,令其实招,众人见乩仙批语与拐子相符,自然不疑了。薛家有的是钱,老爷断
一千也可,五百也可,与冯家作烧埋之费;那冯家也无甚要紧的人,不过为的是钱,
有了银子也就无话了。老爷细想此计如何?”雨村笑道:“不妥,不妥。等我再斟
酌斟酌,压服得口声才好。”二人计议已定。

至次日坐堂,勾取一干有名人犯。雨村详加审问,果见冯家人口稀少,不过赖
此欲得些烧埋之银;薛家仗势倚情,偏不相让,故致颠倒未决。雨村便徇情枉法,
胡乱判断了此案,冯家得了许多烧埋银子,也就无甚话说了。雨村便疾忙修书二封
与贾政并京营节度使王子腾,不过说“令甥之事已完,不必过虑”之言寄去。此事
皆由葫芦庙内沙弥新门子所为,雨村又恐他对人说出当日贫贱时事来,因此心中大
不乐意。后来到底寻了他一个不是,远远的充发了才罢。

当下言不着雨村。且说那买了英莲、打死冯渊的那薛公子,亦系金陵人氏,本
是书香继世之家。只是如今这薛公子幼年丧父,寡母又怜他是个独根孤种,未免溺
爱纵容些,遂致老大无成;且家中有百万之富,现领着内帑钱粮,采办杂料。这薛
公子学名薛蟠,表字文起,性情奢侈,言语傲慢;虽也上过学,不过略识几个字,
终日惟有斗鸡走马、游山玩景而已。虽是皇商,一应经纪世事全然不知,不过赖祖
父旧日的情分,户部挂个虚名支领钱粮,其馀事体,自有伙计老家人等措办,寡母
王氏乃现任京营节度王子腾之妹,与荣国府贾政的夫人王氏是一母所生的姊妹,今
年方五十上下,只有薛蟠一子。还有一女,比薛蟠小两岁,乳名宝钗,生得肌骨莹
润,举止娴雅。当时他父亲在日极爱此女,令其读书识字,较之乃兄竟高十倍。自
父亲死后,见哥哥不能安慰母心,他便不以书字为念,只留心针黹家计等事,好为
母亲分忧代劳。

近因今上崇尚诗礼,征采才能,降不世之隆恩,除聘选妃嫔外,在世宦名家之
女,皆得亲名达部,以备选择,为宫主郡主入学陪侍,充为才人赞善之职,自薛蟠
父亲死后,各省中所有的卖买承局、总管、伙计人等,见薛蟠年轻不谙世事,便趁
时拐骗起来,京都几处生意渐亦销耗。薛蟠素闻得都中乃第一繁华之地,正思一游,
便趁此机会,一来送妹待选,二来望亲,三来亲自入部销算旧帐,再计新支,——
其实只为游览上国风光之意。因此早已检点下行装细软以及馈送亲友各色土物人情
等类,正择日起身,不想偏遇着那拐子,买了英莲。薛蟠见英莲生的不俗,立意买
了作妾,又遇冯家来夺,因恃强喝令豪奴将冯渊打死,便将家中事务,一一嘱托了
族中人并几个老家人,自己同着母亲妹子竟自起身长行去了。人命官司他却视为儿
戏,自谓花上几个钱没有不了的。

在路不记其日,那日已将入都,又听见母舅王子腾升了九省统制,奉旨出都查
边。薛蟠心中暗喜道:“我正愁进京去有舅舅管辖,不能任意挥霍,如今升出去,
可知天从人愿。”因和母亲商议道:“咱们京中虽有几处房舍,只是这十来年没人
居住,那看守的人未免偷着租赁给人住,须得先着人去打扫收拾才好。”他母亲道:
“何必如此招摇!咱们这进京去,原是先拜望亲友,或是在你舅舅处,或是你姨父
家,他两家的房舍极是宽敞的。咱们且住下,再慢慢儿的着人去收拾,岂不消停
些?”薛蟠道:“如今舅舅正升了外省去,家里自然忙乱起身,咱们这会子反一窝
一拖的奔了去,岂不没眼色呢?”他母亲道:“你舅舅虽升了去,还有你姨父家。
况这几年来你舅舅姨娘两处,每每带信捎书接咱们来。如今既来了,你舅舅虽忙着
起身,你贾家的姨娘未必不苦留我们,咱们且忙忙的收拾房子岂不使人见怪?你的
意思我早知道了:守着舅舅姨母住着,未免拘紧了,不如各自住着,好任意施为。
你既如此,你自去挑所宅子去住,我和你姨娘姊妹们别了这几年,却要住几日。我
带了你妹子去投你姨娘家去,你道好不好?”薛蟠见母亲如此说,情知扭不过,只
得吩咐人夫,一路奔荣国府而来。

那时王夫人已知薛蟠官司一事亏贾雨村就中维持了,才放了心。又见哥哥升了
边缺,正愁少了娘家的亲戚来往,略加寂寞。过了几日,忽家人报:“姨太太带了
哥儿姐儿合家进京在门外下车了。”喜的王夫人忙带了人接到大厅上,将薛姨妈等
接进去了。姊妹们一朝相见,悲喜交集,自不必说。叙了一番契阔,又引着拜见贾
母,将人情土物各种酬献了。合家俱厮见过,又治席接风。薛蟠拜见过贾政贾琏,
又引着见了贾赦贾珍等。贾政便使人进来对王夫人说:“姨太太已有了年纪,外甥
年轻,不知庶务,在外住着恐又要生事:咱们东南角上梨香院,那一所房十来间白
空闲着,叫人请了姨太太和姐儿哥儿住了甚好。”王夫人原要留住,贾母也就遣人
来说:“请姨太太就在这里住下,大家亲密些。”薛姨妈正欲同居一处,方可拘紧
些儿,若另在外边,又恐纵性惹祸,遂忙应允。又私与王夫人说明:“一应日费供
给,一概都免,方是处常之法。”王夫人知他家不难于此,遂亦从其自便,从此后,
薛家母女就在梨香院住了。

原来这梨香院乃当日荣公暮年养静之所,小小巧巧,约有十馀间房舍,前厅后
舍俱全。另有一门通街,薛蟠的家人就走此门出入;西南上又有一个角门,通着夹
道子,出了夹道便是王夫人正房的东院了。每日或饭后或晚间,薛姨妈便过来,或
与贾母闲谈,或与王夫人相叙。宝钗日与黛玉、迎春姊妹等一处,或看书下棋,或
做针黹,倒也十分相安。只是薛蟠起初原不欲在贾府中居住,生恐姨父管束不得自
在;无奈母亲执意在此,且贾宅中又十分殷勤苦留,只得暂且住下,一面使人打扫
出自家的房屋再移居过去。谁知自此间住了不上一月,贾宅族中凡有的子侄俱已认
熟了一半,都是那些纨气习,莫不喜与他来往。今日会酒,明日观花,甚至聚赌
嫖娼,无所不至,引诱的薛蟠比当日更坏了十倍。虽说贾政训子有方,治家有法,
一则族大人多,照管不到;二则现在房长乃是贾珍,彼乃宁府长孙,又现袭职,凡
族中事都是他掌管;三则公私冗杂,且素性潇洒,不以俗事为要,每公暇之时,不
过看书着棋而已。况这梨香院相隔两层房舍,又有街门别开,任意可以出入,这些
子弟们所以只管放意畅怀的。因此薛蟠遂将移居之念渐渐打灭了。

日后如何,下回分解。