\chapter{醉金刚小鳅生大浪~痴公子馀痛触前情}

话说贾雨村刚欲过渡,见有人飞奔而来,跑到跟前,口称:“老爷,方才逛的
那庙火起了。”雨村回首看时,只见烈焰烧天,飞灰蔽日。雨村心想:“这也奇怪。
我才出来,走不多远,这火从何而来?莫非士隐遭劫于此?”欲待回去,又恐误了
过河;若不回去,心下又不安。想了一想,便问道:“你方才见那老道士出来了没
有?”那人道:“小的原随老爷出来,因腹内疼痛,略走了一走。回头看见一片火
光,原来就是那庙中火起,特赶来禀知老爷,并没有见有人出来。”雨村虽则心里
狐疑,究竟是名利关心的人,那肯回去看视,便叫那人:“你在这里等火灭了,进
去瞧那老道在与不在,即来回禀。”那人只得答应了伺候。雨村过河,仍自去查看,
查了几处,遇公馆便自歇下。

明日,又行一程,进了都门,众衙役接着,前呼后拥的走着。雨村坐在轿内,
听见轿前开路的人吵嚷。雨村问是何事,那开路的拉了一个人过来跪在轿前,禀道:
“那人酒醉,不知回避,反冲突过来。小的吆喝他,他倒恃酒撒泼,躺在街心,说
小的打了他了。”雨村便道:“我是管理这里地方的,你们都是我的子民。知道本府
经过,喝了酒不知退避,还敢撒赖!”那人道:“我喝酒是自己的钱,醉了躺的是皇
上的地,就是大人老爷也管不得。”雨村怒道:“这人目无法纪!问他叫什么名字。”
那人回道:“我叫醉金刚倪二。”雨村听了生气,叫人:“打这东西!瞧他是金刚不
是。”手下把倪二按倒,着实的打了几鞭子。倪二负痛,酒醒求饶。雨村在轿内哈
哈笑道:“原来是这么个金刚。我且不打你,叫人带进衙门里慢慢的问你。”众衙役
答应,拴了倪二拉着就走,倪二哀求也不中用。

雨村进内复旨回曹,那里把这件事放在心上。那街上看热闹的,三三两两传说:
“倪二仗着有些力气,恃酒讹人,今儿碰在贾大人手里,只怕不轻饶的。”这话已
传到他妻女耳边。那夜果等倪二不见回家,他女儿便到各处赌场寻觅。那赌博的都
是这么说,他女儿哭了。众人都道:“你不用着急。那贾大人是荣府的一家。荣府
里的一个什么二爷和你父亲相好,你同你母亲去找他说个情,就放出来了。”倪二
的女儿想了一想:“果然我父亲常说间壁贾二爷和他好,为什么不找他去?”赶着
回来就和母亲说了,娘儿两个去找贾芸。那日贾芸恰好在家,见他母女两个过来,
便让坐,贾芸的母亲便命倒茶。倪家母女将倪二被贾大人拿去的话说了一遍:“求
二爷说个情儿放出来。”贾芸一口应承,说:“这算不得什么,我到西府里说一声就
放了。那贾大人全仗着西府里才得做了这么大官,只要打发个人去一说就完了。”
倪家母女欢喜,回来便到府里告诉了倪二,叫他不用忙,已经求了贾二爷,他满口
应承,讨个情便放出来的。倪二听了也喜欢。

不料贾芸自从那日给凤姐送礼不收,不好意思进来,也不常到荣府。那荣府的
门上原看着主子的行事,叫谁走动才有些体面,一时来了他便进去通报;若主子不
大理了,不论本家亲戚,他一概不回,支回去就完事。那日贾芸到府,说:“给琏
二爷请安。”门上的说:“二爷不在家,等回来我们替回罢。”贾芸欲要说“请二奶
奶的安”,又恐门上厌烦,只得回家。又被倪家母女催逼着,说:“二爷常说府上不
论那个衙门,说一声儿谁敢不依。如今还是府里的一家儿,又不为什么大事,这个
情还讨不来,白是我们二爷了。”贾芸脸上下不来,嘴里还说硬话:“昨儿我们家里
有事,没打发人说去,少不得今儿说了就放。什么大不了的事!”倪家母女只得听
信。岂知贾芸近日大门竟不得进去,绕到后头,要进园内找宝玉,不料园门锁着,
只得垂头丧气的回来。想起:“那年倪二借银,买了香料送他,才派我种树,如今
我没钱打点,就把我拒绝。那也不是他的能为。拿着太爷留下的公中银钱在外放加
一钱,我们穷当家儿,要借一两也不能,他打谅保得住一辈子不穷的了?那里知道
外头的名声儿很不好!我不说罢了,若说起来,人命官司不知有多少呢。”一面想着,
来到家中,只见倪家母女正等着呢。贾芸无言可支,便说是:“西府里已经打发人
说了,只言贾大人不依。你还求我们家的奴才周瑞的亲戚冷子兴去才中用。”倪家
母女听了,说:“二爷这样体面爷们还不中用,若是奴才,是更不中用了。”贾芸不
好意思,心里发急道:“你不知道,如今的奴才比主子强多着呢。”倪家母女听来无
法,只得冷笑几声,说:“这倒难为二爷白跑了这几天。等我们那一个出来再道乏
罢。”说毕出来,另托人将倪二弄出来了,只打了几板,也没有什么罪。

倪二回家,他妻女将贾家不肯说情的话说了一遍。倪二正喝着酒,便生气要找
贾芸,说:“这小杂种,没良心的东西!头里他没有饭吃,要到府内钻谋事办,亏我
倪二爷帮了他。如今我有了事,他不管。好罢咧!要是我倪二闹起来,连两府里都
不干净!”他妻女忙劝道:“嗳,你又喝了黄汤,就是这么有天没日头的。前儿可不
是醉了闹的乱子。捱了打还没好呢,你又闹了。”倪二道:“捱了打就怕他不成?只
怕拿不着由头儿!我在监里的时候儿,倒认得了好几个有义气的朋友。听见他们说
起来,不独是城里姓贾的多,外省姓贾的也不少,前儿监里收下了好几个贾家的家
人,我倒说这里的贾家小一辈子连奴才们虽不好,他们老一辈的还好,怎么犯了事
呢?我打听了打听,说是和这里贾家是一家儿,都住在外省,审明白了,解进来问
罪的,我才放心。若说贾二这小子,他忘恩负义,我就和几个朋友说他家怎么欺负
人,怎么放重利,怎么强娶活人妻。吵嚷出去,有了风声到了都老爷耳朵里头,这
一闹起来,叫他们才认得倪二金刚呢。”他女人道:“你喝了酒睡去罢。他又强占谁
家的女人来着?没有的事,你不用混说了。”倪二道:“你们在家里,那里知道外头
的事?前年我在场儿里碰见了小张,说他女人被贾家占了,他还和我商量,我倒劝
着他才压住了。不知道小张如今那里去了,这两年没见。若碰着了他,我倪二太爷
出个主意,叫贾二小子死给我瞧瞧!好好儿的孝敬孝敬我倪二太爷才罢了!”说着,
倒身躺下,嘴里还是咕咕哝哝的说了一回,便睡去了。他妻女只当是醉话,也不理
他。明日早起,倪二又往赌场中去了,不提。

且说雨村回到家中,歇息了一夜,将道上遇见甄士隐的事告诉了他夫人一遍。
他夫人便埋怨他:“为什么不回去瞧一瞧?倘或烧死了,可不是咱们没良心。”说着
掉下泪来。雨村道:“他是方外的人了,不肯和咱们在一处的。”正说着,外头传进
话来禀说:“前日老爷吩咐瞧那庙里失火去的人回来了。”雨村踱了出来。那衙役请
了安,回说:“小的奉老爷的命回去,也没等火灭,冒着火进去瞧那道士,那里知
他坐的地方儿都烧了。小的想着那道士必烧死了。那烧的墙屋往后塌了,道士的影
儿都没有了。只有一个蒲团,一个瓢儿,还是好好的。小的各处找他的尸首,连骨
头都没有一点儿。小的恐怕老爷不信,想要拿这蒲团瓢儿回来做个证见,小的这么
一拿,谁知都成了灰了。”雨村听毕,心下明白,知士隐仙去,便把那衙役打发出
去了。回到房中,并没提起士隐火化之言,恐怕妇女不知,反生悲感,只说并无形
迹,必是他先走了。

雨村出来,独坐书房,正要细想士隐的话,忽有家人传报说:“内廷传旨,交
看事件。”雨村疾忙上轿进内。只听见人说:“今日贾存周江西粮道被参回来,在朝
内谢罪。”雨村忙到了内阁,见了各大臣,将海疆办理不善的旨意看了,出来即忙
找着贾政,先说了些为他抱屈的话,后又道喜,问一路可好。贾政也将违别以后的
话细细的说了一遍。雨村道:“谢罪的本上了去没有?”贾政道:“已上去了。等膳
后下来看旨意罢。”正说着,只听里头传出旨来叫贾政,贾政即忙进去。各大人有
与贾政关切的,都在里头等着。等了好一回,方见贾政出来。看见他带着满头的汗,
众人迎上去接着,问:“有什么旨意?”贾政吐舌道:“吓死人,吓死人!倒蒙各位
大人关切,幸喜没有什么事。”众人道:“旨意问了些什么?”贾政道:“旨意问的
是云南私带神枪一案。本上奏明是原任太师贾化的家人,主上一时记着我们先祖的
名字,便问起来。我忙着磕头奏明先祖的名字是代化,主上便笑了,还降旨意说:
‘前放兵部,后降府尹的,不是也叫贾化么?’”那时雨村也在傍边,倒吓了一跳,
便问贾政道:“老先生怎么奏的?”贾政道:“我便慢慢奏道:‘原任太师贾化是云
南人;现任府尹贾某是浙江人。’主上又问:‘苏州刺史奏的贾范,是你一家子么?’
我又磕头奏道:‘是。’主上便变色道:‘纵使家奴强占良民妻女,还成事么?’我
一句不敢奏。主上又问道:‘贾范是你什么人?’我忙奏道:‘是远族。’主上哼了
一声,降旨叫出来了。可不是诧事!”众人道:“本来也巧。怎么一连有这两件事?”
贾政道:“事倒不奇,倒是都姓贾的不好。算来我们寒族人多,年代久了,各处都
有。现在虽没有事,究竟主上记着一个‘贾’字就不好。”众人说:“真是真,假是
假,怕什么?”贾政道:“我心里巴不得不做官,只是不敢告老,现在我们家里两
个世袭,这也无可奈何的。”雨村道:“如今老先生仍是工部,想来京官是没有事的。”
贾政道:“京官虽然无事,我究竟做过两次外任,也就说不齐了。”众人道:“二老
爷的人品行事,我们都佩服的。就是令兄大老爷,也是个好人。只要在令侄辈身上
严紧些就是了。”贾政道:“我因在家的日子少,舍侄的事情不大查考,我心里也不
甚放心。诸位今日提起,都是至相好,或者听见东宅的侄儿家有什么不奉规矩的事
么?”众人道:“没听见别的,只有几位侍郎心里不大和睦,内监里头也有些。想
来不怕什么,只要嘱咐那边令侄,诸事留神就是了。”

众人说毕,举手而散,贾政然后回家。众子侄等都迎接上来。贾政迎着请贾母
的安,然后众子侄俱请了贾政的安,一同进府。王夫人等已到了荣禧堂迎接。贾政
先到了贾母那里拜见了,陈述些违别的话。贾母问探春消息,贾政将许嫁探春的事
都禀明了,还说:“儿子起身急促,难过重阳,虽没有亲见,听见那边亲家的人来,
说的极好。亲家老爷太太都说请老太太的安。还说今冬明春,大约还可调进京来。
这便好了。如今闻得海疆有事,只怕那时还不能调。”贾母始则因贾政降调回来,
知探春远在他乡,一无亲故,心下伤感;后听贾政将官事说明,探春安好,也便转
悲为喜,便笑着叫贾政出去。然后弟兄相见,众子侄拜见,定了明日清晨拜祠堂。

贾政回到自己屋内,王夫人等见过,宝玉贾琏替另拜见。贾政见了宝玉果然比
起身之时脸面丰满,倒觉安静,独不知他心里糊涂,所以心甚喜欢,不以降调为念,
心想幸亏老太太办理的好。又见宝钗沉厚更胜先时,兰儿文雅俊秀,便喜形于色。
独见环儿仍是先前,究不甚钟爱。歇息了半天,忽然想起:“为何今日短了一人?”
王夫人知是想着黛玉,前因家书未报,今日又刚到家,正是喜欢,不便直告,只说
是病着。岂知宝玉的心里已如刀搅,因父亲到家,只得把持心性伺候。王夫人设筵
接风,子孙敬酒。凤姐虽是侄媳,现办家事,也随了宝钗等递酒。贾政便叫递了一
巡酒,“都歇息去罢。”命众家人不必伺候,待明早拜过宗祠,然后进见。分派已定,
贾政与王夫人说些别后的话,馀者王夫人都不敢言。倒是贾政先提起王子腾的事来,
王夫人也不敢悲戚。贾政又说蟠儿的事,王夫人只说他是自作自受;趁便也将黛玉
已死的话告诉。贾政反吓了一惊,不觉掉下泪来,连声叹息。王夫人也掌不住,也
哭了。傍边彩云等即忙拉衣,王夫人止住,重又说些喜欢的话,便安寝了。

次日一早,至宗祠行礼,众子侄都随往。贾政便在祠旁厢房坐下,叫了贾珍贾
琏过来,问起家中事务。贾珍拣可说的说了。贾政又道:“我初回家,也不便来细
细查问,只是听见外头说起你家里更不比从前,诸事要谨慎才好。你年纪也不小了,
孩子们该管教管教,别叫他们在外头得罪人。琏儿也该听着。不是才回家就说你们,
因我有所闻所以才说的。你们更该小心些。”贾珍等脸涨通红的,也只答应个“是”
字,不敢说什么。贾政也就罢了。回归西府,众家人磕头毕,仍复进内,众女仆行
礼,不必多赘。

只说宝玉因昨日贾政问起黛玉,王夫人答以有病,他便暗里伤心,直待贾政命
他回去,一路上已滴了好些眼泪。回到房中,见宝钗和袭人等说话,他便独坐外间
纳闷。宝钗叫袭人送过茶去,知他必是怕老爷查问工课,所以如此,只得过来安慰。
宝玉便借此过去向宝钗说:“你今夜先睡,我要定定神。这时更不如从前了,三言
倒忘两语,老爷瞧着不好。你先睡,叫袭人陪我略坐坐。”宝钗不便强他,点头应
允。

宝玉出来便轻轻和袭人说,央他:“把紫鹃叫来,有话问他。但是紫鹃见了我,
脸上总是有气,须得你去解劝开了再来才好。”袭人道:“你说要定神,我倒喜欢,
怎么又定到这上头去了?有话你明儿问不得?”宝玉道:“我就是今晚得闲,明日倘
或老爷叫干什么,便没空儿了。好姐姐,你快去叫他来。”袭人道:“他不是二奶奶
叫是不来的。”宝玉道:“所以得你去说明了才好。”袭人道:“叫我说什么?”宝玉
道:“你还不知道我的心和他的心么?都为的是林姑娘。你说我并不是负心,我如今
叫你们弄成了一个负心的人了!”说着这话,便瞧瞧里间屋子,用手指着说:“他是
我本不愿意的,都是老太太他们捉弄的。好端端把个林妹妹弄死了。就是他死,也
该叫我见见,说个明白,他死了也不抱怨我嗄。你到底听见三姑娘他们说过的,临
死恨怨我。那紫鹃为他们姑娘,也是恨的我了不得。你想我是无情的人么?晴雯到
底是个丫头,也没有什么大好处,他死了,我实告诉你罢,我还做个祭文祭他呢。
这是林姑娘亲眼见的。如今林姑娘死了,难道倒不及晴雯么?我连祭都不能祭一祭,
况且林姑娘死了还有灵圣的,他想起来不更要怨我么?”袭人道:“你要祭就祭去,
谁拦着你呢。”宝玉道:“我自从好了起来,就想要做一篇祭文,不知道如今怎么一
点灵机儿都没了。要祭别人呢,胡乱还使得,祭他是断断粗糙不得一点儿的。所以
叫紫鹃来问他姑娘的心,他打那里看出来的。我没病的头里还想的出来,病后都不
记得了。你倒说林姑娘已经好了,怎么忽然死的?他好的时候,我不去,他怎么说
来着?我病的时候,他不来,他又怎么说来着?所有他的东西,我诓过来,你二奶奶
总不叫动,不知什么意思。”袭人道:“二奶奶惟恐你伤心罢了,还有什么呢。”宝
玉道:“我不信。林姑娘既是念我,为什么临死把诗稿烧了,不留给我作个记念?又
听见说天上有音乐响,必是他成了神,或是登了仙去。我虽见过了棺材,到底不知
道棺材里有他没有。”袭人道:“你这话越发糊涂了,怎么一个人没死就搁在一个棺
材里当死了的呢!”宝玉道:“不是嗄!大凡成仙的人,或是肉身去的,或是脱胎去
的。好姐姐,你到底叫了紫鹃来。”袭人道:“如今等我细细的说明了你的心,他要
肯来还好,要不肯来还得费多少话;就是来了,见你也不肯细说。据我的主意:明
日等二奶奶上去了,我慢慢的问他,或者倒可仔细。遇着闲空儿,我再慢慢的告诉
你。”宝玉道:“你说得也是,你不知道我心里的着急。”

正说着,麝月出来说:“二奶奶说:天已四更了,请二爷进去睡罢。袭人姐姐
必是说高了兴了,忘了时候儿了。”袭人听了,道:“可不是该睡了。有话明儿再说
罢。”宝玉无奈,只得进去,又向袭人耳边道:“明儿好歹别忘了。”袭人笑道:“知
道了。”麝月抹着脸笑道:“你们两个又闹鬼儿了。为什么不和二奶奶说明了,就到
袭人那边睡去?由着你们说一夜,我们也不管。”宝玉摆手道:“不用言语。”袭人恨
道:“小蹄子儿,你又嚼舌根,看我明儿撕你的嘴!”回头对宝玉道:“这不是你闹
的?说了四更天的话。”一面说,一面送宝玉进屋,各人散去。

那夜宝玉无眠,到了次日,还想这事。只听得外头传进话来,说:“众亲朋因
老爷回家,都要送戏接风。老爷再四推辞,说‘不必唱戏,竟在家里备了水酒,倒
请亲朋过来大家谈谈’。于是定了后儿摆席请人,所以进来告诉。”

不知所请何人,下回分解。