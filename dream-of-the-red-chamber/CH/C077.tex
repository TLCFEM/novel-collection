\chapter{俏丫鬟抱屈夭风流~美优伶斩情归水月}

话说王夫人见中秋已过,凤姐病也比先减了,虽未大愈,然亦可以出入行走得
了,仍命大夫每日诊脉服药。又开了丸药方来,配“调经养荣丸”。因用上等人参
二两,王夫人取时,翻寻了半日,只向小匣内寻了几枝簪挺粗细的。王夫人看了嫌
不好,命再找去,又找了一大包须沫出来。王夫人焦躁道:“用不着偏有,但用着
了,再找不着!成日家我叫你们查一查,都归拢一处,你们白不听,就随手混撂。”
彩云道:“想是没了,就只有这个。上次那边的太太来寻了去了。”王夫人道:“没
有的话。你再细找找。”彩云只得又去找寻,拿了几包药材来,说:“我们不认的
这个,请太太自看。除了这个没有了。”王夫人打开看时,也都忘了,不知都是什
么,并没有一支人参。因一面遣人去问凤姐有无。凤姐来说:“也只有些参膏。芦
须虽有几根,也不是上好的,每日还要煎药里用呢。”王夫人听了,只得向邢夫人
那里问去。说:“因上次没了,才往这里来寻,早已用完了。”王夫人没法,只得
亲身过来请问贾母。贾母忙命鸳鸯取出当日馀的来,竟还有一大包,皆有手指头粗
细不等,遂秤了二两给王夫人。王夫人出来,交给周瑞家的拿去,令小厮送与医生
家去。又命将那几包不能辨的药也带了去,命医生认了,各包号上。一时周瑞家的
又拿进来,说:“这几样都各包号上名字了。但那一包人参固然是上好的,只是年
代太陈。这东西比别的却不同,凭是怎么好的,只过一百年后,就自己成了灰了。
如今这个虽未成灰,然已成了糟朽烂木,也没有力量的了。请太太收了这个,倒不
拘粗细,多少再换些新的才好。”

王夫人听了,低头不语,半日才说:“这可没法了,只好去买二两来罢。”也
无心看那些,只命:“都收了罢。”因问周瑞家的:“你就去说给外头人们,拣好
的换二两来。倘或一时老太太问你们,只说用的是老太太的,不必多说。”周瑞家
的方才要去时,宝钗因在坐,乃笑道:“姨娘且住,如今外头人参都没有好的。虽
有全枝,他们也必截做两三段,镶嵌上芦泡须枝,搀匀了好卖,看不得粗细。我们
铺子里常和行里交易,如今我去和妈妈说了,哥哥去托个伙计过去和参行里要他二
两原枝来,不妨咱们多使几两银子,到底得了好的。”王夫人笑道:“倒是你明白。
但只还得你亲自走一趟,才能明白。”于是宝钗去了,半日回来说:“已遣人去,
赶晚就有回信。明日一早去配也不迟。”王夫人自是喜悦,因说道:“‘卖油的娘
子水梳头’。自来家里有的给人多少,这会子轮到自己用,反倒各处寻去。”说毕
长叹。宝钗笑道:“这东西虽然值钱,总不过是药,原该济众散人才是。咱们比不
得那没见世面的人家,得了这个,就珍藏密敛的。”王夫人点头道:“你这话也是。”

一时宝钗去后,因见无别人在室,遂唤周瑞家的,问:“前日园中搜检的事情,
可得下落?”周瑞家的是已和凤姐商议停妥,一字不隐,遂回明王夫人。王夫人吃
了一惊。想到司棋系迎春丫头,乃系那边的人,只得令人去回邢氏。周瑞家的回道:
“前日那边太太嗔着王善保家的多事,打了几个嘴巴子,如今他也装病在家,不肯
出头了。况且又是他外孙女儿,自己打了嘴,他只好装个忘了,日久平服了再说。
如今我们过去回时,恐怕又多心,倒像咱们多事是的。不如直把司棋带过去,一并
连赃证与那边太太瞧了,不过打一顿配了人,再指个丫头来,岂不省事?如今白告
诉去,那边太太再推三阻四的,又说:‘既这样,你太太就该料理,又来说什么呢?’
岂不倒耽搁了?倘或那丫头瞅空儿寻了死,反不好了。如今看了两三天,都有些偷
懒,倘一时不到,岂不倒弄出事来?”王夫人想了一想,说:“这也倒是。快办了
这一件,再办咱们家的那些妖精。”

周瑞家的听说,会齐了那边几个媳妇,先到迎春房里,回明迎春。迎春听了,
含泪似有不舍之意,因前夜之事,丫头们悄悄说了原故,虽数年之情难舍,但事关
风化,亦无可如何了。那司棋也曾求了迎春,实指望能救,只是迎春语言迟慢,耳
软心活,是不能作主的。司棋见了这般,知不能免,因跪着哭道:“姑娘好狠心!
哄了我这两日,如今怎么连一句话也没有?”周瑞家的说道:“你还要姑娘留你不
成?便留下,你也难见园里的人了。依我们的好话,快快收了这样子,倒是人不知
鬼不觉的去罢,大家体面些。”迎春手里拿着一本书正看呢,听了这话,书也不看,
话也不答,只管扭着身子呆呆的坐着。周瑞家的又催道:“这么大女孩儿,自己作
的还不知道?把姑娘都带的不好了,你还敢紧着缠磨他!”迎春听了,方发话道:
“你瞧入画也是几年的,怎么说去就去了?自然不止你两个,想这园里凡大的都要
去呢。依我说,将来总有一散,不如各人去罢。”周瑞家的道:“所以到底是姑娘
明白。明儿还有打发的人呢,你放心罢。”司棋无法,只得含泪给迎春磕头,和众
人告别。又向迎春耳边说:“好歹打听我受罪,替我说个情儿,就是主仆一场!”
迎春亦含泪答应:“放心。”

于是周瑞家的等人带了司棋出去,又有两个婆子将司棋所有的东西都与他拿
着。走了没几步,只见后头绣橘赶来,一面也擦着泪,一面递给司棋一个绢包,说:
“这是姑娘给你的。主仆一场,如今一旦分离,这个给你做个念心儿罢。”司棋接
了,不觉更哭起来了,又和绣橘哭了一回。周瑞家的不耐烦,只管催促,二人只得
散了。司棋因又哭告道:“婶子大娘们,好歹略徇个情儿:如今且歇一歇,让我到
相好姊妹跟前辞一辞,也是这几年我们相好一场。”周瑞家的等人皆各有事,做这
些事便是不得已了,况且又深恨他们素日大样,如今那里工夫听他的话?因冷笑道:
“我劝你去罢,别拉拉扯扯的了,我们还有正经事呢。谁是你一个衣胞里爬出来的?
辞他们做什么?你不过挨一会是一会,难道算了不成?依我说,快去罢!”一面说,
一面总不住脚,直带着出后角门去。司棋无奈,又不敢再说,只得跟着出来。

可巧正值宝玉从外头进来,一见带了司棋出去,又见后面抱着许多东西,料着
此去再不能来了。因听见上夜的事,并晴雯的病也因那日加重,细问晴雯,又不说
是为何。今见司棋亦走,不觉如丧魂魄,因忙拦住问道:“那里去?”周瑞家的等
皆知宝玉素昔行为,又恐唠叨误事,因笑道:“不干你事,快念书去罢。”宝玉笑
道:“姐姐们且站一站,我有道理。”周瑞家的便道:“太太吩咐不许少捱时刻。
又有什么道理?我们只知道太太的话,管不得许多。”司棋见了宝玉,因拉住哭道:
“他们做不得主,好歹求求太太去!”宝玉不禁也伤心,含泪说道:“我不知你做
了什么大事!晴雯也气病着,如今你又要去了,这却怎么着好!”周瑞家的发躁向
司棋道:“你如今不是副小姐了,要不听说,我就打得你了。别想往日有姑娘护着,
任你们作耗!越说着,还不好生走。一个小爷见了面,也拉拉扯扯的,什么意思!”
那几个妇人不由分说,拉着司棋,便出去了。

宝玉又恐他们去告舌,恨的只瞪着他们。看走远了,方指着恨道:“奇怪,奇
怪!怎么这些人只一嫁了汉子,染了男人的气味,就这样混账起来,比男人更可杀
了!”守园门的婆子听了,也不禁好笑起来,因问道:“这样说,凡女儿个个是好
的了,女人个个是坏的了?”宝玉发恨道:“不错,不错!”正说着,只见几个老
婆子走来,忙说道:“你们小心传齐了伺候着。此刻太太亲自到园里查人呢。”又
吩咐:“快叫怡红院晴雯姑娘的哥嫂来,在这里等着,领出他妹子去。”因又笑道:
“阿弥陀佛!今日天睁了眼,把这个祸害妖精退送了,大家清净些。”宝玉一闻得
王夫人进来亲查,便料道晴雯也保不住了,早飞也似的赶了去,所以后来趁愿之话,
竟未听见。

宝玉及到了怡红院,只见一群人在那里。王夫人在屋里坐着,一脸怒色,见宝
玉也不理。晴雯四五日水米不曾沾牙,如今现打炕上拉下来,蓬头垢面的,两个女
人搀架起来去了。王夫人吩咐:“把他贴身的衣服撂出去,馀者留下,给好的丫头
们穿。”又命:“把这里所有的丫头们都叫来!”一一过目。

原来王夫人惟怕丫头们教坏了宝玉,乃从袭人起以至于极小的粗活小丫头们,
个个亲自看了一遍。因问:“谁是和宝玉一日的生日?”本人不敢答言。李嬷嬷指
道:“这一个蕙香,又叫做四儿的,是同宝玉一日生日的。”王夫人细看了一看,
虽比不上晴雯一半,却有几分水秀,视其行止,聪明皆露在外面,且也打扮的不同。
王夫人冷笑道:“这也是个没廉耻的货!他背地里说的同日生日就是夫妻,这可是
你说的?打量我隔的远,都不知道呢。可知我身子虽不大来,我的心耳神意时时都
在这里。难道我统共一个宝玉,就白放心凭你们勾引坏了不成?”这个四儿见王夫
人说着他素日和宝玉的私语,不禁红了脸,低头垂泪。王夫人即命:“也快把他家
人叫来,领出去配人。”又问:“那芳官呢?”芳官只得过来。王夫人道:“唱戏
的女孩子,自然更是狐狸精了!上次放你们,你们又不愿去,可就该安分守己才是。
你就成精鼓捣起来,调唆宝玉,无所不为!”芳官等辩道:“并不敢调唆什么了。”
王夫人笑道:“你还强嘴!你连你干娘都压倒了,岂止别人。”因喝命:“唤他干
娘来领去!就赏他外头找个女婿罢。他的东西,一概给他。”吩咐:“上年凡有姑
娘分的唱戏女孩子们,一概不许留在园里,都令其各人干娘带出,自行聘嫁。”一
语传出,这些干娘皆感恩趁愿不尽,都约齐给王夫人磕头领去。

王夫人又满屋里搜检宝玉之物。凡略有眼生之物,一并命收卷起来,拿到自己
房里去了。因说:“这才干净,省得旁人口舌。”又吩咐袭人麝月等人:“你们小
心,往后再有一点分外之事,我一概不饶!因叫人查看了,今年不宜迁挪,暂且挨
过今年,明年一并给我仍旧搬出去,才心净。”说毕,茶也不吃,遂带领众人,又
往别处去阅人。

暂且说不到后文,如今且说宝玉只道王夫人不过来搜检搜检,无甚大事,谁知
竟这样雷嗔电怒的来了。所责之事,皆系平日私语,一字不爽,料必不能挽回的。
虽心下恨不能一死,但王夫人盛怒之际,自不敢多言。一直跟送王夫人到沁芳亭,
王夫人命:“回去好生念念那书!仔细明儿问你。才已发下狠了。”宝玉听如此说,
才回来。一路打算:“谁这样犯舌?况这里事也无人知道,如何就都说着了?”一
面想,一面进来,只见袭人在那里垂泪,且去了第一等的人,岂不伤心?便倒在床
上大哭起来。

袭人知他心里别的犹可,独有晴雯是第一件大事,乃劝道:“哭也不中用。你
起来,我告诉你:晴雯已经好了,他这一家去,倒心净养几天。你果然舍不得他,
等太太气消了,你再求老太太,慢慢的叫进来,也不难。太太不过偶然听了别人的
闲言,在气头上罢了。”宝玉道:“我究竟不知晴雯犯了什么迷天大罪!”袭人道:
“太太只嫌他生的太好了,未免轻狂些。太太是深知这样美人似的人,心里是不能
安静的,所以很嫌他。像我们这粗粗笨笨的倒好。”宝玉道:“美人似的,心里就
不安静么?你那里知道,古来美人安静的多着呢。这也罢了,咱们私自玩话,怎么
也知道了?又没外人走风,这可奇怪了。”袭人道:“你有什么忌讳的?一时高兴,
你就不管有人没人了。我也曾使过眼色,也曾递过暗号,被那人知道了,你还不觉。”
宝玉道:“怎么人人的不是,太太都知道了,单不挑出你和麝月秋纹来?”袭人听
了这话,心内一动,低头半日,无可回答,因便笑道:“正是呢。若论我们,也有
玩笑不留心的去处,怎么太太竟忘了?想是还有别的事,等完了再发放我们也未可
知。”宝玉笑道:“你是头一个出了名的至善至贤的人,他两个又是你陶冶教育的,
焉得有什么该罚之处?只是芳官尚小,过于伶俐些,未免倚强压倒了人,惹人厌。
四儿是我误了他:还是那年我和你拌嘴的那日起,叫上来做细活的。众人见我待他
好,未免夺了地位,也是有的,故有今日。只是晴雯,也是和你们一样从小儿在老
太太屋里过来的,虽生的比人强些,也没什么妨碍着谁的去处。就只是他的性情爽
利,口角锋芒,竟也没见他得罪了那一个。可是你说的,想是他过于生得好了,反
被这个好带累了!”说毕,复又哭起来。

袭人细揣,此话只是宝玉有疑他之意,竟不好再劝,因叹道:“天知道罢了。
此时也查不出人来了。白哭一会子,也无益了。”宝玉冷笑道:“原是想他自幼娇
生惯养的,何尝受过一日委屈?如今是一盆才透出嫩箭的兰花送到猪圈里去一般。
况又是一身重病,里头一肚子闷气。他又没有亲爹热娘,只有一个醉泥鳅姑舅哥哥,
他这一去,那里还等得一月半月?再不能见一面两面的了!”说着,越发心痛起来。
袭人笑道:“可是你‘自许州官放火,不许百姓点灯’。我们偶说一句妨碍的话,
你就说不吉利;你如今好好的咒他,就该的了?”宝玉道:“我不是妄口咒人,今
年春天已有兆头的。”袭人忙问:“何兆?”宝玉道:“这阶下好好的一株海棠花,
竟无故死了半边,我就知道有坏事,果然应在他身上。”袭人听了,又笑起来说:
“我要不说,又掌不住,你也太婆婆妈妈的了。这样的话,怎么是你读书的人说
的?”宝玉叹道:“你们那里知道?不但草木,凡天下有情有理的东西,也和人一
样,得了知己,便极有灵验的。若用大题目比,就像孔子庙前桧树,坟前的蓍草,
诸葛祠前的柏树,岳武穆坟前的松树:这都是堂堂正大之气,千古不磨之物。世乱
他就枯干了,世治他就茂盛了,凡千年枯了又生的几次,这不是应兆么?若是小题
目比,就像杨太真沈香亭的木芍药,端正楼的相思树,王昭君坟上的长青草,难道
不也有灵验?所以这海棠亦是应着人生的。”袭人听了这篇痴话,又可笑,又可叹,
因笑道:“真真的这话越发说上我的气来了。那晴雯是个什么东西?就费这样心思,
比出这些正经人来。还有一说:他纵好,也越不过我的次序去。就是这海棠,也该
先来比我,也还轮不到他。想是我要死的了。”

宝玉听说,忙掩他的嘴,劝道:“这是何苦?一个未是,你又这样起来。罢了,
再别提这事,别弄的去了三个,又饶上一个。”袭人听说,心下暗喜道:“若不如
此,也没个了局。”宝玉又道:“我还有一句话要和你商量,不知你肯不肯:现在
他的东西,是瞒上不瞒下,悄悄的送还他去。再或有咱们常日积攒下的钱,拿几吊
出去,给他养病,也是你姐妹好了一场。”袭人听了,笑道:“你太把我看得忒小
器又没人心了。这话还等你说?我才把他的衣裳各物已打点下了,放在那里。如今
白日里人多眼杂,又恐生事,且等到晚上,悄悄的叫宋妈给他拿去。我还有攒下的
几吊钱,也给他去。”宝玉听了,点点头儿。袭人笑道:“我原是久已‘出名的贤
人’,连这一点子好名还不会买去不成?”宝玉听了他方才说的,又陪笑抚慰他,
怕他寒了心。晚间,果遣宋妈送去。

宝玉将一切人稳住,便独自得便,到园子后角门,央一个老婆子,带他到晴雯
家去。先这婆子百般不肯,只说怕人知道,“回了太太,我还吃饭不吃饭?”无奈
宝玉死活央告,又许他些钱,那个婆子方带了他去。

却说这晴雯当日系赖大买的。还有个姑舅哥哥,叫做吴贵,人都叫他贵儿。那
时晴雯才得十岁,时常赖嬷嬷带进来,贾母见了喜欢,故此赖嬷嬷就孝敬了贾母。
过了几年,赖大又给他姑舅哥哥娶了一房媳妇。谁知贵儿一味胆小老实,那媳妇却
倒伶俐,又兼有几分姿色,看着贵儿无能为,便每日家打扮的妖妖调调,两只眼儿
水汪汪的。招惹的赖大家人如蝇逐臭,渐渐做出些风流勾当来。那时晴雯已在宝玉
屋里,他便央及了晴雯转求凤姐,合赖大家的要过来。目今两口儿就在园子后角门
外居住,伺候园中买办杂差。这晴雯一时被撵出来,住在他家。那媳妇那里有心肠
照管?吃了饭便自去串门子,只剩下晴雯一人,在外间屋内爬着。

宝玉命那婆子在外望,他独掀起布帘进来,一眼就看见晴雯睡在一领芦席
上,幸而被褥还是旧日铺盖的。心内不知自己怎么才好,因上来含泪伸手,轻轻拉
他,悄唤两声。当下晴雯又因着了风,又受了哥嫂的歹话,病上加病,嗽了一日,
才朦胧睡了。忽闻有人唤他,强展双眸,一见是宝玉,又惊又喜,又悲又痛,一把
死攥住他的手,哽咽了半日,方说道:“我只道不得见你了!”接着便嗽个不住。
宝玉也只有哽咽之分。晴雯道:“阿弥陀佛,你来得好,且把那茶倒半碗我喝。渴
了半日,叫半个人也叫不着。”宝玉听说,忙拭泪问:“茶在那里?”晴雯道:“在
炉台上。”宝玉看时,虽有个黑煤乌嘴的吊子,也不像个茶壶。只得桌上去拿一个
碗,未到手内,先闻得油膻之气。宝玉只得拿了来,先拿些水洗了两次,复用自己
的绢子拭了,闻了闻还有些气味,没奈何,提起壶来斟了半碗。看时绛红的也不大
像茶。晴雯扶枕道:“快给我喝一口罢,这就是茶了。那里比得咱们的茶呢。”宝
玉听说,先自己尝了一尝,并无茶味,咸涩不堪,只得递给晴雯。只见晴雯如得了
甘露一般,一气都灌下去了。

宝玉看着,眼中泪直流下来,连自己的身子都不知为何物了,一面问道:“你
有什么说的?趁着没人,告诉我。”晴雯呜咽道:“有什么可说的!不过是挨一刻是
一刻,挨一日是一日。我已知横竖不过三五日的光景,我就好回去了,只是一件,
我死也不甘心:我虽生得比别人好些,并没有私情勾引你,怎么一口死咬定了我是
个‘狐狸精’!我今儿既担了虚名,况且没了远限,不是我说一句后悔的话:早知
如此,我当日——”说到这里,气往上咽,便说不出来,两手已经冰凉。宝玉又痛
又急,又害怕,便歪在席上,一只手攥着他的手,一只手轻轻的给他捶打着。又不
敢大声的叫,真真万箭攒心。两三句话时晴雯才哭出来,宝玉拉着他的手,只觉瘦
如枯柴。腕上犹戴着四个银镯,因哭道:“除下来,等好了再戴上去罢。”又说:
“这一病好了,又伤好些!”晴雯拭泪,把那手用力拳回,搁在口边,狠命一咬,
只听“咯吱”一声,把两根葱管一般的指甲齐根咬下,拉了宝玉的手,将指甲搁在
他手里。又回手扎挣着,连揪带脱,在被窝内将贴身穿着的一件旧红绫小袄儿脱下,
递给宝玉。不想虚弱透了的人,那里禁得这么抖搂,早喘成一处了。宝玉见他这般,
已经会意,连忙解开外衣,将自己的袄儿褪下来,盖在他身上。却把这件穿上,不
及扣钮子,只用外头衣裳掩了。刚系腰时,只见晴雯睁眼道:“你扶起我来坐坐。”
宝玉只得扶他。那里扶得起?好容易欠起半身,晴雯伸手把宝玉的袄儿往自己身上
拉。宝玉连忙给他披上,拖着膊,伸上袖子,轻轻放倒,然后将他的指甲装在荷
包里。晴雯哭道:“你去罢!这里腌,你那里受得?你的身子要紧。今日这一来,
我就死了,也不枉担了虚名!”

一语未完,只见他嫂子笑嘻嘻掀帘进来道:“好呀,你两个的话,我已都听见
了。”又向宝玉道:“你一个做主子的,跑到下人房里来做什么?看着我年轻长的
俊,你敢只是来调戏我么?”宝玉听见,吓得忙陪笑央及道:“好姐姐,快别大声
的。他伏侍我一场,我私自来瞧瞧他。”那媳妇儿点着头儿,笑道:“怨不得人家
都说你有情有义儿的。”便一手拉了宝玉进里间来,笑道:“你要不叫我嚷,这也
容易。你只是依我一件事。”说着,便自己坐在炕沿上,把宝玉拉在怀中,紧紧的
将两条腿夹住。宝玉那里见过这个?心内早突突的跳起来了。急得满面红胀,身上
乱战,又羞又愧又怕又恼,只说:“好姐姐,别闹。”那媳妇乜斜了眼儿,笑道:
“呸,成日家听见你在女孩儿们身上做工夫,怎么今儿个就发起讪来了?”宝玉红
了脸,笑道:“姐姐撒开手,有话咱们慢慢儿的说。外头有老妈妈听见,什么意思
呢?”那媳妇那里肯放,笑道:“我早进来了,已经叫那老婆子去到园门口儿等着
呢。我等什么儿似的,今日才等着你了!你要不依我,我就嚷起来,叫里头太太听
见了,我看你怎么样?你这么个人,只这么大胆子儿。我刚才进来了好一会子,在
窗下细听,屋里只你两个人,我只道有些个体己话儿。这么看起来,你们两个人竟
还是各不相扰儿呢。我可不能像他那么傻。”说着,就要动手。宝玉急的死往外拽。

正闹着,只听窗外有人问:“这晴雯姐姐在这里住呢不是?”那媳妇子也吓了
一跳,连忙放了宝玉。这宝玉已经吓怔了,听不出声音。外边晴雯听见他嫂子缠磨
宝玉,又急又臊又气,一阵虚火上攻,早昏晕过去。那媳妇连忙答应着,出来看,
不是别人,却是柳五儿和他母亲两个,抱着一个包袱。柳家的拿着几吊钱,悄悄的
问那媳妇道:“这是里头袭姑娘叫拿出来给你们姑娘的。他在那屋里呢?”那媳妇
儿笑道:“就是这个屋子,那里还有屋子?”

那柳家的领着五儿刚进门来,只见一个人影儿往屋里一闪。柳家的素知这媳妇
子不妥,只打量是他的私人。看见晴雯睡着了,连忙放下,带着五儿便往外走。谁
知五儿眼尖,早已见是宝玉,便问他母亲道:“头里不是袭人姐姐那里悄悄儿的找
宝二爷呢吗?”柳家的道:“嗳哟,可是忘了。方才老宋妈说:‘见宝二爷出角门
来了。门上还有人等着,要关园门呢。’”因回头问那媳妇儿。那媳妇儿自己心虚,
便道:“宝二爷那里肯到我们这屋里来?”柳家的听说,便要走。这宝玉一则怕关
了门,二则怕那媳妇子进来又缠,也顾不得什么了,连忙掀了帘子出来道:“柳嫂
子,你等等我,一路儿走。”柳家的听了,倒唬了一大跳,说:“我的爷,你怎么
跑了这里来了?”那宝玉也不答言,一直飞走。那五儿道:“妈妈,你快叫住宝二
爷不用忙,留神冒冒失失,被人碰见倒不好。况且才出来时,袭人姐姐已经打发人
留了门了。”说着,赶忙同他妈来赶宝玉。这里晴雯的嫂子干瞅着,把个妙人儿走
了。

却说宝玉跑进角门,才把心放下来,还是突突乱跳。又怕五儿关在外头,眼巴
巴瞅着他母女也进来了。远远听见里边嬷嬷们正查人,若再迟一步,就关了园门了。
宝玉进入园中,且喜无人知道。到了自己房里,告诉袭人,只说在薛姨妈家去的,
也就罢了。一时铺床,袭人不得不问:“今日怎么睡?”宝玉道:“不管怎么睡罢
了。”原来这一二年来,袭人因王夫人看重了他,越发自要尊重,凡背人之处或夜
晚之间,总不与宝玉狎昵,较先小时反倒疏远了。虽无大事办理,然一应针线,并
宝玉及诸小丫头出入银钱衣履什物等事,也甚烦琐,且有吐血之症,故近来夜间总
不与宝玉同房。宝玉夜间胆小,醒了便要唤人,因晴雯睡卧警醒,故夜间一应茶水
起坐呼唤之事,悉皆委他一人,所以宝玉外床只是晴雯睡着。他今去了,袭人只得
将自己铺盖搬来,铺设床外。

宝玉发了一晚上的呆。袭人催他睡下,然后自睡。只听宝玉在枕上长吁短叹,
覆去翻来,直至三更以后,方渐渐安顿了。袭人方放心,也就蒙睡着。没半盏茶
时,只听宝玉叫“晴雯”。袭人忙连声答应,问:“做什么?”宝玉因要茶吃。袭
人倒了茶来,宝玉乃叹道:“我近来叫惯了他,却忘了是你。”袭人笑道:“他乍
来,你也曾睡梦中叫我,以后才改了的。”说着,大家又睡下。宝玉又翻转了一个
更次。至五更方睡去时,只见晴雯从外走来,仍是往日行景,进来向宝玉道:“你
们好生过罢。我从此就别过了!”说毕,翻身就走。宝玉忙叫时,又将袭人叫醒。
袭人还只当他惯了口乱叫,却见宝玉哭了,说道:“晴雯死了!”袭人笑道:“这
是那里的话?叫人听着什么意思。”宝玉那里肯听?恨不得一时亮了就遣人去问信。

及至亮时,就有王夫人房里小丫头叫开前角门,传王夫人的话:“‘即时叫起
宝玉,快洗脸换了衣裳来。因今儿有人请老爷赏秋菊,老爷因喜欢他前儿做的诗好,
故此要带了他们去。’这都是太太的话,你们快告诉去,立逼他快来,老爷在上屋
里等他们吃面茶呢。环哥儿早来了。快快儿的去罢。我去叫兰哥儿去了。”里面的
婆子听一句,应一句,一面扣着钮子,一面开门。袭人听得叩门,便知有事,一面
命人问时,自己已起来了。听得这话,忙催人来舀了洗脸水,催宝玉起来梳洗,他
自去取衣。因思跟贾政出门,便不肯拿出十分出色的新鲜衣服来,只拣那三等成色
的来。宝玉此时已无法,只得忙忙前来。果然贾政在那里吃茶,十分喜悦。宝玉请
了早安,贾环贾兰二人也都见过,贾政命坐吃茶,向环兰二人道:“宝玉读书,不
及你两个;论题联、和诗这种聪明,你们皆不及他。今日此去,未免叫你们做诗,
宝玉须随便助他们两个。”

王夫人自来不曾听见这等考语,真是意外之喜。一时候他父子去了,方欲过贾
母那边来时,就有芳官等三个干娘走来,回说:“芳官自前日蒙太太的恩典赏出来
了,他就疯了似的,茶饭都不吃,勾引上藕官蕊官,三个人寻死觅活,只要铰了头
发做尼姑去。我只当是小孩子家,一时出去不惯,也是有的,不过隔两日就好了,
谁知越闹越凶,打骂着也不怕。实在没法,所以来求太太,或是依他们去做尼姑去,
或教导他们一顿,赏给别人做女孩儿去罢。我们没这福。”王夫人听了,道:“胡
说!那里由得他们起来?佛门也是轻易进去的么?每人打一顿给他们,看还闹不闹!”
当下因八月十五日各庙内上供去,皆有各庙内的尼姑来送供尖,因曾留下水月庵的
智通与地藏庵的圆信住下未回,听得此信,就想拐两个女孩子去做活使唤。都向王
夫人说:“府上到底是善人家。因太太好善,所以感应得这些小姑娘们皆如此。虽
然说‘佛门容易难上’,也要知道‘佛法平等’,我佛立愿,原度一切众生。如今
两三个姑娘既然无父母,家乡又远,他们既经了这富贵,又想从小命苦,入了风流
行次,将来知道终身怎么样?所以‘苦海回头’,立意出家,修修来世,也是他们
的高意。太太倒不要阻了善念。”王夫人原是个善人,起先听见这话,谅系小孩子
不遂心的话,将来熬不得清净,反致获罪。今听了这两个拐子的话,大近情理。且
近日家中多故,又有邢夫人遣人过来知会,明日接迎春家去住两日,以备人家相看;
且又有官媒来求说探春等,心绪正烦,那里着意在这些小事?既听此言,便笑答道:
“你两个既这等说,你们就带了做徒弟去,如何?”二姑子听了,念一声佛,道:
“善哉,善哉!若如此,可是老人家的阴功不小。”说毕便稽首拜谢。王夫人道:
“既这样,你们问他去。若果真心,即上来当着我拜了师父去罢。”

这三个女人听了出去,果然将他三人带来。王夫人问之再三,他三人已立定主
意,遂与两个姑子叩了头,又拜辞了王夫人。王夫人见他们意皆决断,知不可强了,
反倒伤心可怜,忙命人来取了些东西来赏了他们,又送了两个姑子些礼物。从此芳
官跟了水月庵的智通,蕊官藕官二人跟了地藏庵圆信,各自出家去了。

要知后事,下回分解。