\chapter{史太君寿终归地府~王凤姐力诎失人心}

却说贾母坐起说道:“我到你们家已经六十多年了,从年轻的时候到老来,福
也享尽了。自你们老爷起,儿子孙子也都算是好的了。就是宝玉呢,我疼了他一场
——”说到那里,拿眼满地下瞅着,王夫人便推宝玉走到床前。贾母从被窝里伸出
手来拉着宝玉,道:“我的儿,你要争气才好!”宝玉嘴里答应,心里一酸,那眼泪
便要流下来,又不敢哭,只得站着。听贾母说道:“我想再见一个重孙子,我就安
心了。我的兰儿在那里呢?”李纨也推贾兰上去。贾母放了宝玉,拉着贾兰道:“你
母亲是要孝顺的。将来你成了人,也叫你母亲风光风光。凤丫头呢?”凤姐本来站
在贾母旁边,赶忙走到跟前说:“在这里呢。”贾母道:“我的儿,你是太聪明了,
将来修修福罢。我也没有修什么,不过心实吃亏。那些吃斋念佛的事我也不大干,
就是旧年叫人写了些《金刚经》送送人,不知送完了没有?”凤姐道:“没有呢。”
贾母道:“早该施舍完了才好。我们大老爷和珍儿是在外头乐了;最可恶的是史丫
头没良心,怎么总不来瞧我!”鸳鸯等明知其故,都不言语。

贾母又瞧了一瞧宝钗,叹了口气,只见脸上发红。贾政知是回光返照,即忙进
上参汤。贾母的牙关已经紧了,合了一回眼,又睁着满屋里瞧了一瞧。王夫人宝钗
上去,轻轻扶着,邢夫人凤姐等便忙穿衣。地下婆子们已将床安设停当,铺了被褥。
听见贾母喉间略一响动,脸变笑容,竟是去了。享年八十三岁。众婆子疾忙停床。

于是贾政等在外一边跪着,邢夫人等在内一边跪着,一齐举起哀来。外面家人
各样预备齐全,只听里头信儿一传出来,从荣府大门起至内宅门,扇扇大开,一色
净白纸糊了;孝棚高起,大门前的牌楼立时竖起。上下人等登时成服。贾政报了丁
忧,礼部奏闻。主上深仁厚泽,念及世代功勋,又系元妃祖母,赏银一千两,谕礼
部主祭。家人们各处报丧。众亲友虽知贾家势败,今见圣恩隆重,都来探丧。择了
吉时成殓,停灵正寝。

贾赦不在家,贾政为长;宝玉、贾环、贾兰是亲孙,年纪又小,都应守灵。贾
琏虽也是亲孙,带着贾蓉,尚可分派家人办事。虽请了些男女外亲来照应,内里邢
王二夫人、李纨、凤姐、宝钗等是应灵旁哭泣的;尤氏虽可照应,他自贾珍外出,
依住荣府,一向总不上前,且又荣府的事不甚谙练;贾蓉的媳妇更不必说;惜春年
小,虽在这里长的,他于家事全不知道。所以内里竟无一人支持,只有凤姐可以照
管里头的事,况又贾琏在外作主,里外他二人,倒也相宜。

凤姐先前仗着自己的才干,原打量老太太死了,他大有一番作用。邢王二夫人
等本知他曾办过秦氏的事,必是妥当,于是仍叫凤姐总理里头的事。凤姐本不应辞,
自然应了,心想:“这里的事本是我管的。那些家人更是我手下的人。太太和珍大
嫂子的人本来难使唤,如今他们都去了。银项虽没有对牌,这种银子却是现成的。
外头的事又是我们那个办。虽说我现今身子不好,想来也不致落褒贬,必比宁府里
还得办些。”心下已定,且待明日接了三,后日一早分派。便叫周瑞家的传出话去,
将花名册取上来。凤姐一一的瞧了,统共男仆只有二十一人,女仆只有十九人,馀
者俱是些丫头,连各房算上,也不过三十多人,难以派差。心里想道:“这回老太
太的事倒没有东府里的人多。”又将庄上的弄出几个,也不敷差遣。

正在思算,只见一个小丫头过来说:“鸳鸯姐姐请奶奶。”凤姐只得过去。只见
鸳鸯哭得泪人一般,一把拉着凤姐儿,说道:“二奶奶请坐,我给二奶奶磕个头。
虽说服中不行礼,这个头是要磕的。”鸳鸯说着跪下,慌的凤姐赶忙拉住,说道:“这
是什么礼?有话好好的说。”鸳鸯跪着,凤姐便拉起来。鸳鸯说道:“老太太的事,
一应内外,都是二爷和二奶奶办。这种银子是老太太留下的。老太太这一辈子也没
有遭塌过什么银钱,如今临了这件大事,必得求二奶奶体体面面的办一办才好。我
方才听见老爷说什么‘诗云’‘子曰’,我也不懂;又说什么‘丧与其易,宁戚’,
我更不明白。我问宝二奶奶,说是老爷的意思:老太太的丧事,只要悲切才是真孝,
不必糜费,图好看的念头。我想老太太这样一个人,怎么不该体面些?我虽是奴才
丫头,敢说什么?只是老太太疼二奶奶和我这一场,临死了还不叫他风光风光?我想
二奶奶是能办大事的,故此我请二奶奶来,作个主意。我生是跟老太太的人,老太
太死了,我也是跟老太太的!若是瞧不见老太太的事怎么办,将来怎么见老太太
呢?”凤姐听了这话来的古怪,便说:“你放心,要体面是不难的。虽是老爷口说
要省,那势派也错不得。便拿这项银子都花在老太太身上,也是该当的。”鸳鸯道:
“老太太的遗言说,所有剩下的东西是给我们的,二奶奶倘或用着不够,只管拿这
个去折变补上。就是老爷说什么,也不好违了老太太的遗言。况且老太太分派的时
候,不是老爷在这里听见的么?”凤姐道:“你素来最明白的,怎么这会子这样的
着急起来了?”鸳鸯道:“不是我着急,为的是大太太是不管事的,老爷是怕招摇
的。若是二奶奶心里也是老爷的想头,说抄过家的人家,丧事还是这么好,将来又
要抄起来,也就不顾起老太太来,怎么样呢?我呢,是个丫头,好歹碍不着,到底
是这里的声名!”凤姐道:“我知道了。你只管放心,有我呢。”鸳鸯千恩万谢的托
了凤姐。

那凤姐出来,想道:“鸳鸯这东西好古怪!不知打了什么主意。论理,老太太身
上本该体面些。嗳,且别管他,只按着咱们家先前的样子办去。”于是叫旺儿家的
来,把话传出去,请二爷进来。不多时,贾琏进来,说道:“怎么找我?你在里头照
应着些就是了。横竖作主是老爷太太们,他说怎么着,我们就怎么着。”凤姐道:“你
也说起这个话来了,可不是鸳鸯说的话应验了么?”贾琏道:“什么鸳鸯的话?”
凤姐便将鸳鸯请进去的话述了一遍。贾琏道:“他们的话算什么!刚才二老爷叫我
去,说:‘老太太的事固要认真办理,但是知道的呢,说是老太太自己结果自己;
不知道的,只说咱们都隐匿起来了,如今很宽裕。老太太的这种银子用不了,谁还
要么?仍旧该用在老太太身上。老太太是在南边的,虽有坟地,却没有阴宅。老太
太的灵是要归到南边去的。留这银子在祖坟上盖起些房屋来,再馀下的,置买几顷
祭田。咱们回去也好;就是不回去,便叫那些贫穷族中住着,也好按时按节早晚上
香,时常祭扫祭扫。’你想这些话可不是正经主意么?据你的话,难道都花了罢?”
凤姐道:“银子发出来了没有?”贾琏道:“谁见过银子!我听见咱们太太听见了二
老爷的话,极力的撺掇二太太和二老爷说:‘这是好主意。’叫我怎么着?现在外头
棚杠上要支几百银子,这会子还没有发出来。我要去,他们都说有,先叫外头办了,
回来再算。你想,这些奴才,有钱的早溜了。按着册子叫去,有说告病的,有说下
庄子去了的。剩下几个走不动的,只有赚钱的能耐,还有赔钱的本事么?”凤姐听
了,呆了半天,说道:“这还办什么!”

正说着,见来了一个丫头,说:“大太太的话,问二奶奶:今儿第三天了,里
头还很乱,供了饭,还叫亲戚们等着吗?叫了半天,上了菜,短了饭:这是什么办
事的道理?”凤姐急忙进去吆喝人来伺候,将就着把早饭打发了。偏偏那日人来的
多,里头的人都死眉瞪眼的。凤姐只得在那里照料了一会子,又惦记着派人,赶着
出来,叫了旺儿家的传齐了家下女人们,一一分派了。众人都答应着不动。凤姐道:
“什么时候,还不供饭?”众人道:“传饭是容易的,只要将里头的东西发出来,
我们才好照管去。”凤姐道:“糊涂东西!派定了你们,少不得有的。”众人只得勉强
应着。凤姐即往上房取发应用之物,要去请示邢王二夫人。见人多难说,看那时候
已经日渐平西了,只得找了鸳鸯,说要老太太存的那一分家伙。鸳鸯道:“你还问
我呢!那一年二爷当了,赎了来了么?”凤姐道:“不用银的金的,只要那一分平常
使的。”鸳鸯道:“大太太珍大奶奶屋里使的是那里来的?”凤姐一想不差,转身就
走,只得到王夫人那边找了玉钏彩云,才拿了一分出来,急忙叫彩明登帐,发与众
人收管。

鸳鸯见凤姐这样慌张,又不好叫他回来,心想:“他头里作事何等爽利周到,
如今怎么掣肘的这个样儿。我看这两三天连一点头脑都没有,不是老太太白疼了他
了吗!”那里知邢夫人一听贾政的话,正合着将来家计艰难的心,巴不得留一点子
作个收局。况且老太太的事原是长房作主。贾赦虽不在家,贾政又是拘泥的人,有
件事便说:“请大太太的主意。”邢夫人素知凤姐手脚大,贾琏的闹鬼,所以死拿住
不放松。鸳鸯只道已将这项银两交了出去了,故见凤姐掣肘如此,却疑为不肯用心,
便在贾母灵前唠唠叨叨哭个不了。邢夫人等听了话中有话,不想到自己不令凤姐便
宜行事,反说:“凤丫头果然有些不用心。”王夫人到了晚上,叫了凤姐过来,说:
“咱们家虽说不济,外头的体面是要的。这两三天人来人往,我瞧着那些人都照应
不到,想必你没有吩咐,还得你替我们操点心儿才好。”凤姐听了,呆了一会,要
将银两不凑手的话说出来,但只银钱是外头管的,王夫人说的是照应不到,凤姐也
不敢辩,只好不言语。邢夫人在旁说道:“论理,该是我们做媳妇的操心,本不是
孙子媳妇的事,但是我们动不得身,所以托你。你是打不得撒手的。”凤姐紫涨了
脸,正要回说,只听外头鼓乐一奏,是烧黄昏纸的时候了,大家举起哀来,又不得
说。凤姐原想回来再说,王夫人催他出去料理,说道:“这里有我们呢,你快快儿
的去料理明儿的事罢。”

凤姐不敢再言,只得含悲忍泣的出来,又叫人传齐了众人,又吩咐了一会,说:
“大娘婶子们可怜我罢!我上头捱了好些说,为的是你们不齐截,叫人笑话,明儿
你们豁出些辛苦来罢!”那些人回道:“奶奶办事,不是今儿个一遭儿了,我们敢违
拗吗?只是这回的事,上头过于累赘。只说打发这顿饭罢:有在这里吃的,有要在
家里吃的;请了这位太太,又是那位奶奶不来。诸如此类,那里能齐全?还求奶奶
劝劝那些姑娘们少挑饬就好了。”凤姐道:“头一层是老太太的丫头们是难缠的,太
太们的也难说话,叫我说谁去呢?”众人道:“从前奶奶在东府里还是署事,要打
要骂,怎么那样锋利?谁敢不依?如今这些姑娘们都压不住了?”凤姐叹道:“东府
里的事,虽说托办的,太太虽在那里,不好意思说什么。如今是自己的事情,又是
公中的,人人说得话。再者,外头的银钱也叫不灵:即如棚里要一件东西,传出去
了,总不见拿进来,这叫我什么法儿呢?”众人道:“二爷在外头,倒怕不应付么?”
凤姐道:“还提这个!他也是那里为难。第一件,银钱不在他手里,要一件得回一件,
那里凑手?”众人道:“老太太这项银子不在二爷手里吗?”凤姐道:“你们回来问
管事的,就知道了。”众人道:“怨不得我们听见外头男人抱怨说:‘这么件大事,
咱们一点摸不着,净当苦差。’叫人怎么能齐心呢?”凤姐道:“如今不用说了。眼
面前的事,大家留些神罢。倘或闹的上头有了什么说的,我可和你们不依。”众人
道:“奶奶要怎么样,我们敢抱怨吗?只是上头一人一个主意,我们实在难周到。”
凤姐听了也没法,只得央及道:“好大娘们,明儿且帮我一天。等我把姑娘们闹明
白了,再说罢了。”众人听命而去。

凤姐一肚子的委屈,愈想愈气,直到天亮,又得上去。要把各处的人整理整理,
又恐邢夫人生气;要和王夫人说,怎奈邢夫人挑唆。这些丫头们见邢夫人等不助着
凤姐的威风,更加作践起他来。幸得平儿替凤姐排解,说是:“二奶奶巴不得要好,
只是老爷太太们吩咐了外头,不许糜费,所以我们二奶奶不能应付到了。”说过几
次,才得安静些。虽说僧经道忏,吊祭供饭,络绎不绝,终是银钱吝啬,谁肯踊跃,
不过草草了事。连日王妃诰命也来的不少,凤姐也不能上去照应,只好在底下张罗。
叫了那个,走了这个;发一回急,央及一回;支吾过了一起,又打发一起。别说鸳
鸯等看去不像样,连凤姐自己心里也过不去了。

邢夫人虽说是冢妇,仗着“悲戚为孝”四个字,倒也都不理会。王夫人只得跟
着邢夫人行事,馀者更不必说了。独有李纨瞧出凤姐的苦处,却不敢替他说话,只
自叹道:“俗话说的,‘牡丹虽好,全仗绿叶扶持’,太太们不亏了凤丫头,那些人
还帮着吗?若是三姑娘在家还好,如今只有他几个自己的人瞎张罗,背前面后的也
抱怨,说是一个钱摸不着,脸面也不能剩一点儿。老爷是一味的尽孝,庶务上头不
大明白。这样的一件大事,不撒散几个钱就办的开了吗?可怜凤丫头闹了几年,不
想在老太太的事上只怕保不住脸了。”于是抽空儿叫了他的人来,吩咐道:“你们别
看着人家的样儿,也遭塌起琏二奶奶来。别打量什么穿孝守灵就算了大事了,不过
混过几天就是了。看见那些人张罗不开,就插个手儿,也未为不可。这也是公事,
大家都该出力的。”那些素服李纨的人都答应着说:“大奶奶说的很是,我们也不敢
那么着。只听见鸳鸯姐姐们的口话儿,好像怪琏二奶奶的似的。”李纨道:“就是鸳
鸯,我也告诉过他。我说琏二奶奶并不是在老太太的事上不用心,只是银子钱都不
在他手里,叫他巧媳妇还作的上没米的粥来吗?如今鸳鸯也知道了,所以也不怪他
了。只是鸳鸯的样子竟是不像从前了,这也奇怪。那时候有老太太疼他,倒没有作
过什么威福;如今老太太死了,没有了仗腰子的了,我看他倒有些气质不大好了。
我先前替他愁,这会子幸喜大老爷不在家,才躲过去了;不然,他有什么法儿?”

说着,只见贾兰走来说:“妈妈睡罢。一天到晚人来客去的也乏了,歇歇罢。
我这几天总没有摸摸书本儿。今儿爷爷叫我家里睡,我喜欢的很,要理个一两本书
才好,别等脱了孝再都忘了。”李纨道:“好孩子,看书呢,自然是好的,今儿且歇
歇罢,等老太太送了殡再看罢。”贾兰道:“妈妈要睡,我也就睡在被窝里头想想也
罢了。”众人听了,都夸道:“好哥儿!怎么这点年纪,得了空儿就想到书上?不像宝
二爷,娶了亲的人还是那么孩子气。这几日跟着老爷跪着,瞧他很不受用,巴不得
老爷一动身就跑过来找二奶奶,不知唧唧咕咕的说些什么。甚至弄的二奶奶都不理
他了,他又去找琴姑娘。琴姑娘也躲着他,邢姑娘也不很和他说话。倒是咱们本家
儿的什么喜姑娘四姑娘咧,哥哥长哥哥短的和他亲密。我们看那宝二爷除了和奶奶
姑娘们混混,只怕他心里也没有别的事,白过费了老太太的心,疼了他这么大,那
里及兰哥儿一零儿呢?大奶奶将来是不愁的了。”李纨道:“就好也还小呢。只怕到
他大了,咱们家还不知怎么样了呢。环哥儿你们瞧着怎么样?”众人道:“那一个
更不像样儿了。两只眼睛倒像个活猴儿似的,东溜溜,西看看。虽在那里嚎丧,见
了奶奶姑娘们来了,他在孝幔子里头净偷着眼儿瞧人呢。”李纨道:“他的年纪其实
也不小了。前日听见说还要给他说亲呢,如今又得等着了。嗳,还有一件事,——
咱们家这些人,我看来也是说不清的,且不必说闲话儿。——后日送殡,各房的车
是怎么样了?”众人道:“琏二奶奶这几天闹的像失魂落魄的样儿了,也没见传出
去。昨儿听见外头男人们说:二爷派了蔷二爷料理,说是咱们家的车也不够,赶车
的也少,要到亲戚家去借去呢。”李纨笑道:“车也都是借得的么?”众人道:“奶
奶说笑话儿了,车怎么借不得?只是那一日所有的亲戚都用车,只怕难借,想来还
得雇呢。”李纨道:“底下人的只得雇,上头白车也有雇的么?”众人道:“现在大
太太,东府里的大奶奶小蓉奶奶,都没有车了,不雇,那里来的呢?”李纨听了,
叹息道:“先前见有咱们家里的太太奶奶们坐了雇的车来,咱们都笑话,如今轮到
自己头上了。你明儿去告诉你们的男人:我们的车马,早早的预备好了,省了挤。”
众人答应了出去,不提。

且说史湘云因他女婿病着,贾母死后,只来了一次,屈指算是后日送殡,不能
不去。又见他女婿的病已成痨症,暂且不妨,只得坐夜前一日过来。想起贾母素日
疼他;又想到自己命苦,刚配了一个才貌双全的女婿,情性又好,偏偏的得了冤孽
症候,不过捱日子罢了。于是更加悲痛,直哭了半夜。鸳鸯等再三劝慰不止。宝玉
瞅着也不胜悲伤,又不好上前去劝。见他淡妆素服,不敷脂粉,更比未出嫁的时候
犹胜几分。回头又看宝琴等也都是淡素妆饰,丰韵嫣然。独看到宝钗浑身挂孝,那
一种雅致,比寻常穿颜色时更自不同。心里想道:“古人说:千红万紫,终让梅花
为魁。看来不止为梅花开的早,竟是那‘洁白清香’四字真不可及了。但只这时候
若有林妹妹,也是这样打扮,更不知怎样的丰韵呢。”想到这里,不觉的心酸起来,
那泪珠儿便一直的滚下来了,趁着贾母的事,不妨放声大哭。众人正劝湘云,外间
忽又添出一个哭的人来。大家只道是想着贾母疼他的好处,所以悲伤,岂知他们两
个人各自有各自的眼泪。这场大哭,招得满屋的人无不下泪。还是薛姨妈李婶娘等
劝住。

次日乃坐夜之期,更加热闹。凤姐这日竟支撑不住,也无方法,只得用尽心力,
甚至咽喉嚷哑,敷衍过了半日。到了下半天,亲友更多了,事情也更繁了,瞻前不
能顾后。正在着急,只见一个小丫头跑来说:“二奶奶在这里呢。怪不得大太太说:
‘里头人多,照应不过来,二奶奶是躲着受用去了!’”凤姐听了这话,一口气撞上
来,往下一咽,眼泪直流,只觉得眼前一黑,嗓子里一甜,便喷出鲜红的血来,身
子站不住,就蹲倒在地。幸亏平儿急忙过来扶住。只见凤姐的血一口一口的吐个不
住。

未知性命如何,下回分解。