\chapter{手足眈眈小动唇舌~不肖种种大承笞挞}

却说王夫人唤上金钏儿的母亲来,拿了几件簪环当面赏了,又吩咐:“请几众
僧人念经超度他。”金钏儿的母亲磕了头,谢了出去。

原来宝玉会过雨村回来,听见金钏儿含羞自尽,心中早已五内摧伤,进来又被
王夫人数说教训了一番,也无可回说。看见宝钗进来,方得便走出,茫然不知何往,
背着手,低着头,一面感叹,一面慢慢的信步走至厅上。刚转过屏门,不想对面来
了一人正往里走,可巧撞了个满怀。只听那人喝一声:“站住!”宝玉唬了一跳,
抬头看时,不是别人,却是他父亲。早不觉倒抽了一口凉气,只得垂手一旁站着。
贾政道:“好端端的,你垂头丧气的什么?方才雨村来了要见你,那半天才出来!
既出来了,全无一点慷慨挥洒的谈吐,仍是委委琐琐的。我看你脸上一团私欲愁闷
气色!这会子又嗳声叹气,你那些还不足、还不自在?无故这样,是什么原故?”宝
玉素日虽然口角伶俐,此时一心却为金钏儿感伤,恨不得也身亡命殒;如今见他父
亲说这些话,究竟不曾听明白了,只是怔怔的站着。

贾政见他惶悚,应对不似往日,原本无气的,这一来倒生了三分气。方欲说话,
忽有门上人来回:“忠顺亲王府里有人来,要见老爷。”贾政听了,心下疑惑,暗
暗思忖道:“素日并不与忠顺府来往,为什么今日打发人来?”一面想,一面命:
“快请厅上坐。”急忙进内更衣。出来接见时,却是忠顺府长府官,一面彼此见了
礼,归坐献茶。未及叙谈,那长府官先就说道:“下官此来,并非擅造潭府,皆因
奉命而来,有一件事相求。看王爷面上,敢烦老先生做主,不但王爷知情,且连下
官辈亦感谢不尽。”贾政听了这话,摸不着头脑,忙陪笑起身问道:“大人既奉王
命而来,不知有何见谕?望大人宣明,学生好遵谕承办。”那长府官冷笑道:“也
不必承办,只用老先生一句话就完了。我们府里有一个做小旦的琪官,一向好好在
府,如今竟三五日不见回去,各处去找,又摸不着他的道路。因此各处察访,这一
城内十停人倒有八停人都说:他近日和衔玉的那位令郎相与甚厚。下官辈听了,尊
府不比别家,可以擅来索取,因此启明王爷。王爷亦说:‘若是别的戏子呢,一百
个也罢了;只是这琪官,随机应答,谨慎老成,甚合我老人家的心境,断断少不得
此人。’故此求老先生转致令郎,请将琪官放回:一则可慰王爷谆谆奉恳之意,二
则下官辈也可免操劳求觅之苦。”说毕,忙打一躬。

贾政听了这话,又惊又气,即命唤宝玉出来。宝玉也不知是何原故,忙忙赶来,
贾政便问:“该死的奴才!你在家不读书也罢了,怎么又做出这些无法无天的事来!
那琪官现是忠顺王爷驾前承奉的人,你是何等草莽,无故引逗他出来,如今祸及于
我!”宝玉听了,唬了一跳,忙回道:“实在不知此事。究竟‘琪官’两个字,不
知为何物,况更加以‘引逗’二字!”说着便哭。贾政未及开口,只见那长府官冷
笑道:“公子也不必隐饰。或藏在家,或知其下落,早说出来,我们也少受些辛苦,
岂不念公子之德呢!”宝玉连说:“实在不知。恐是讹传,也未见得。”那长府官
冷笑两声道:“现有证据,必定当着老大人说出来,公子岂不吃亏?既说不知,此
人那红汗巾子怎得到了公子腰里?”宝玉听了这话,不觉轰了魂魄,目瞪口呆。心
下自思:“这话他如何知道?他既连这样机密事都知道了,大约别的瞒不过他。不
如打发他去了,免得再说出别的事来。”因说道:“大人既知他的底细,如何连他
置买房舍这样大事倒不晓得了。听得说他如今在东郊离城二十里有个什么紫檀堡,
他在那里置了几亩田地,几间房舍。想是在那里,也未可知。”那长府官听了,笑
道:“这样说,一定是在那里了。我且去找一回,若有了便罢;若没有,还要来请
教。”说着,便忙忙的告辞走了。

贾政此时气得目瞪口歪,一面送那官员,一面回头命宝玉:“不许动!回来有
话问你!”一直送那官去了。才回身时,忽见贾环带着几个小厮一阵乱跑。贾政喝
命小厮:“给我快打!”贾环见了他父亲,吓得骨软筋酥,赶忙低头站住。贾政便
问:“你跑什么?带着你的那些人都不管你,不知往那里去,由你野马一般!”喝
叫:“跟上学的人呢?”贾环见他父亲甚怒,便乘机说道:“方才原不曾跑,只因
从那井边一过,那井里淹死了一个丫头,我看脑袋这么大,身子这么粗,泡的实在
可怕,所以才赶着跑过来了。”贾政听了,惊疑问道:“好端端,谁去跳井?我家
从无这样事情。自祖宗以来,皆是宽柔待下,大约我近年于家务疏懒,自然执事人
操克夺之权,致使弄出这暴殒轻生的祸来。若外人知道,祖宗的颜面何在!”喝命:
“叫贾琏、赖大来!”小厮们答应了一声,方欲去叫,贾环忙上前拉住贾政袍襟,
贴膝跪下道:“老爷不用生气。此事除太太屋里的人,别人一点也不知道。我听见
我母亲说——”说到这句,便回头四顾一看。贾政知其意,将眼色一丢,小厮们明
白,都往两边后面退去。贾环便悄悄说道:“我母亲告诉我说:宝玉哥哥前日在太
太屋里,拉着太太的丫头金钏儿,强奸不遂,打了一顿,金钏儿便赌气投井死了。”

话未说完,把个贾政气得面如金纸,大叫:“拿宝玉来!”一面说,一面便往
书房去,喝命:“今日再有人来劝我,我把这冠带家私,一应就交与他和宝玉过去!
我免不得做个罪人,把这几根烦恼鬓毛剃去,寻个干净去处自了,也免得上辱先人、
下生逆子之罪!”众门客仆从见贾政这个形景,便知又是为宝玉了,一个个咬指吐
舌,连忙退出。贾政喘吁吁直挺挺的坐在椅子上,满面泪痕,一叠连声:“拿宝玉
来!拿大棍拿绳来!把门都关上!有人传信到里头去,立刻打死!”众小厮们只得齐
齐答应着,有几个来找宝玉。

那宝玉听见贾政吩咐他“不许动”,早知凶多吉少,那里知道贾环又添了许多
的话?正在厅上旋转,怎得个人往里头捎信,偏偏的没个人来,连焙茗也不知在那
里。正盼望时,只见一个老妈妈出来。宝玉如得了珍宝,便赶上来拉他,说道:“快
进去告诉:老爷要打我呢!快去,快去!要紧,要紧!”宝玉一则急了说话不明白,
二则老婆子偏偏又耳聋,不曾听见是什么话,把“要紧”二字只听做“跳井”二字,
便笑道:“跳井让他跳去,二爷怕什么?”宝玉见是个聋子,便着急道:“你出去
叫我的小厮来罢!”那婆子道:“有什么不了的事?老早的完了。太太又赏了银子,
怎么不了事呢?”

宝玉急的手脚正没抓寻处,只见贾政的小厮走来,逼着他出去了。贾政一见,
眼都红了,也不暇问他在外流荡优伶,表赠私物,在家荒疏学业,逼淫母婢,只喝
命:“堵起嘴来,着实打死!”小厮们不敢违,只得将宝玉按在凳上,举起大板,
打了十来下。宝玉自知不能讨饶,只是呜呜的哭。贾政还嫌打的轻,一脚踢开掌板
的,自己夺过板子来,狠命的又打了十几下。宝玉生来未经过这样苦楚,起先觉得
打的疼不过还乱嚷乱哭,后来渐渐气弱声嘶,哽咽不出。众门客见打的不祥了,赶
着上来,恳求夺劝。贾政那里肯听?说道:“你们问问他干的勾当,可饶不可饶!素
日皆是你们这些人把他酿坏了,到这步田地,还来劝解!明日酿到他弑父弑君,你
们才不劝不成?”众人听这话不好,知道气急了,忙乱着觅人进去给信。王夫人听
了,不及去回贾母,便忙穿衣出来,也不顾有人没人,忙忙扶了一个丫头赶往书房
中来,慌得众门客小厮等避之不及。

贾政正要再打,一见王夫人进来,更加火上浇油,那板子越下去的又狠又快。
按宝玉的两个小厮忙松手走开,宝玉早已动弹不得了。贾政还欲打时,早被王夫人
抱住板子。贾政道:“罢了,罢了!今日必定要气死我才罢!”王夫人哭道:“宝
玉虽然该打,老爷也要保重。且炎暑天气,老太太身上又不大好,打死宝玉事小,
倘或老太太一时不自在了,岂不事大?”贾政冷笑道:“倒休提这话!我养了这不
肖的孽障,我已不孝;平昔教训他一番,又有众人护持。不如趁今日结果了他的狗
命,以绝将来之患!”说着,便要绳来勒死。王夫人连忙抱住哭道:“老爷虽然应
当管教儿子,也要看夫妻分上。我如今已五十岁的人,只有这个孽障,必定苦苦的
以他为法,我也不敢深劝。今日越发要弄死他,岂不是有意绝我呢?既要勒死他,
索性先勒死我,再勒死他!我们娘儿们不如一同死了,在阴司里也得个倚靠。”说
毕,抱住宝玉,放声大哭起来。贾政听了此话,不觉长叹一声,向椅上坐了,泪如
雨下。王夫人抱着宝玉,只见他面白气弱,底下穿着一条绿纱小衣,一片皆是血渍。
禁不住解下汗巾去,由腿看至臀胫,或青或紫,或整或破,竟无一点好处,不觉失
声大哭起“苦命的儿”来。因哭出“苦命儿”来,又想起贾珠来,便叫着贾珠哭道:
“若有你活着,便死一百个我也不管了!”此时里面的人闻得王夫人出来,李纨、
凤姐及迎、探姊妹两个也都出来了。王夫人哭着贾珠的名字,别人还可,惟有李纨
禁不住也抽抽搭搭的哭起来了。贾政听了,那泪更似走珠一般滚了下来。

正没开交处,忽听丫鬟来说:“老太太来了!”一言未了,只听窗外颤巍巍的
声气说道:“先打死我,再打死他,就干净了!”贾政见母亲来了,又急又痛,连
忙迎出来。只见贾母扶着丫头,摇头喘气的走来。贾政上前躬身陪笑说道:“大暑
热的天,老太太有什么吩咐,何必自己走来,只叫儿子进去吩咐便了。”贾母听了,
便止步喘息,一面厉声道:“你原来和我说话!我倒有话吩咐,只是我一生没养个
好儿子,却叫我和谁说去!”贾政听这话不像,忙跪下含泪说道:“儿子管他,也
为的是光宗耀祖。老太太这话,儿子如何当的起?”贾母听说,便啐了一口,说道:
“我说了一句话,你就禁不起!你那样下死手的板子,难道宝玉儿就禁的起了?你说
教训儿子是光宗耀祖,当日你父亲怎么教训你来着。”说着也不觉泪往下流。贾政
又陪笑道:“老太太也不必伤感,都是儿子一时性急,从此以后再不打他了。”贾
母便冷笑两声道:“你也不必和我赌气,你的儿子,自然你要打就打。想来你也厌
烦我们娘儿们,不如我们早离了你,大家干净。”说着,便令人:“去看轿!我和
你太太、宝玉儿立刻回南京去!”家下人只得答应着。贾母又叫王夫人道:“你也
不必哭了。如今宝玉儿年纪小,你疼他;他将来长大,为官作宦的,也未必想着你
是他母亲了。你如今倒是不疼他,只怕将来还少生一口气呢!”贾政听说,忙叩头
说道:“母亲如此说,儿子无立足之地了。”贾母冷笑道:“你分明使我无立足之
地,你反说起你来!只是我们回去了,你心里干净,看有谁来不许你打!”一面说,
一面只命:“快打点行李车辆轿马回去!”贾政直挺挺跪着,叩头谢罪。

贾母一面说,一面来看宝玉。只见今日这顿打不比往日,又是心疼,又是生气,
也抱着哭个不了。王夫人与凤姐等解劝了一会,方渐渐的止住。早有丫鬟媳妇等上
来要搀宝玉。凤姐便骂:“糊涂东西!也不睁开眼瞧瞧,这个样儿,怎么搀着走的?
还不快进去把那藤屉子春凳抬出来呢!”众人听了,连忙飞跑进去,果然抬出春凳
来,将宝玉放上,随着贾母王夫人等进去,送至贾母屋里。

彼时贾政见贾母怒气未消,不敢自便,也跟着进来。看看宝玉果然打重了,再
看看王夫人一声“肉”一声“儿”的哭道:“你替珠儿早死了,留着珠儿,也免你
父亲生气,我也不白操这半世的心了!这会子你倘或有个好歹,撂下我,叫我靠那
一个?”数落一场,又哭“不争气的儿”。贾政听了,也就灰心自己不该下毒手打
到如此地步。先劝贾母,贾母含泪说道:“儿子不好,原是要管的,不该打到这个
分儿。你不出去,还在这里做什么!难道于心不足,还要眼看着他死了才算吗?”
贾政听说,方诺诺的退出去了。

此时薛姨妈、宝钗、香菱、袭人、湘云等也都在这里。袭人满心委屈,只不好
十分使出来。见众人围着,灌水的灌水,打扇的打扇,自己插不下手去,便索性走
出门,到二门前,命小厮们找了焙茗来细问:“方才好端端的,为什么打起来?你
也不早来透个信儿!”焙茗急的说:“偏我没在跟前,打到半中间,我才听见了。
忙打听原故,却是为琪官儿和金钏儿姐姐的事。”袭人道:“老爷怎么知道了?”
焙茗道:“那琪官儿的事,多半是薛大爷素昔吃醋,没法儿出气,不知在外头挑唆
了谁来,在老爷跟前下的蛆。那金钏儿姐姐的事,大约是三爷说的,我也是听见跟
老爷的人说。”袭人听了这两件事都对景,心中也就信了八九分。然后回来,只见
众人都替宝玉疗治。调停完备,贾母命:“好生抬到他屋里去。”众人一声答应,
七手八脚,忙把宝玉送入怡红院内自己床上卧好。又乱了半日,众人渐渐的散去了。
袭人方才进前来,经心服侍细问。

要知端底,究竟如何,且听下回分解。