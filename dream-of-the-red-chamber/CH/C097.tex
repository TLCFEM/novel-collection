\chapter{林黛玉焚稿断痴情~薛宝钗出闺成大礼}

话说黛玉到潇湘馆门口,紫鹃说了一句话,更动了心,一时吐出血来,几乎晕
倒,亏了紫鹃还同着秋纹,两个人搀扶着黛玉到屋里来。那时秋纹去后,紫鹃雪雁
守着,见他渐渐苏醒过来,问紫鹃道:“你们守着哭什么?”紫鹃见他说话明白,
倒放了心了,因说:“姑娘刚才打老太太那边回来,身上觉着不大好,唬的我们没
了主意,所以哭了。”黛玉笑道:“我那里就能够死呢。”这一句话没完,又喘成
一处。原来黛玉因今日听得宝玉宝钗的事情,这本是他数年的心病,一时急怒,所
以迷惑了本性。及至回来吐了这一口血,心中却渐渐的明白过来,把头里的事一字
也不记得。这会子见紫鹃哭了,方模糊想起傻大姐的话来。此时反不伤心,惟求速
死,以完此债。这里紫鹃雪雁只得守着,想要告诉人去,怕又像上回招的凤姐说他
们失惊打怪。那知秋纹回去神色慌张,正值贾母睡起中觉来,看见这般光景,便问:
“怎么了?”秋纹吓的连忙把刚才的事回了一遍。贾母大惊,说:“这还了得!”
连忙着人叫了王夫人凤姐过来,告诉了他婆媳两个。凤姐道:“我都嘱咐了,这是
什么人走了风了呢?这不更是一件难事了吗!”贾母道:“且别管那些,先瞧瞧去
是怎么样了。”说着,便起身带着王夫人凤姐等过来看视。见黛玉颜色如雪,并无
一点血色,神气昏沉,气息微细,半日又咳嗽了一阵,丫头递了痰盂,吐出都是痰
中带血的,大家都慌了。

只见黛玉微微睁眼,看见贾母在他旁边,便喘吁吁的说道:“老太太!你白疼
了我了。”贾母一闻此言,十分难受,便道:“好孩子,你养着罢!不怕的。”黛
玉微微一笑,把眼又闭上了。外面丫头进来回凤姐道:“大夫来了。”于是大家略
避。王大夫同着贾琏进来,诊了脉,说道:“尚不妨事。这是郁气伤肝,肝不藏血,
所以神气不定。如今要用敛阴止血的药,方可望好。”王大夫说完,同着贾琏出去
开方取药去了。贾母看黛玉神气不好,便出来告诉凤姐等道:“我看这孩子的病,
不是我咒他,只怕难好。你们也该替他预备预备,冲一冲,或者好了,岂不是大家
省心?就是怎么样,也不至临时忙乱。咱们家里这两天正有事呢。”凤姐儿答应了。
贾母又问了紫鹃一回,到底不知是那个说的。贾母心里只是纳闷,因说:“孩子们
从小儿在一处儿玩,好些是有的。如今大了,懂的人事,就该要分别些,才是做女
孩儿的本分,我才心里疼他。若是他心里有别的想头,成了什么人了呢,我可是白
疼了他了。你们说了,我倒有些不放心。”因回到房中,又叫袭人来问,袭人仍将
前日回王夫人的话并方才黛玉的光景述了一遍。贾母道:“我方才看他却还不至糊
涂。这个理我就不明白了!咱们这种人家,别的事自然没有的,这心病也是断断有
不得的。林丫头若不是这个病呢,我凭着花多少钱都使得;就是这个病,不但治不
好,我也没心肠了。”凤姐道:“林妹妹的事,老太太倒不必张罗,横竖有他二哥
哥天天同着大夫瞧:倒是姑妈那边的事要紧。今儿早起,听见说,房子不差什么就
妥当了。竟是老太太、太太到姑妈那边去,我也跟了去商量商量。就只一件:姑妈
家里有宝妹妹在那里,难以说话,不如索性请姑妈晚上过来,咱们一夜都说结了,
就好办了。”贾母王夫人都道:“你说的是。今儿晚了,明儿饭后咱们娘儿们就过
去。”说着,贾母用了晚饭,凤姐同王夫人各自归房不提。

且说次日凤姐吃了早饭过来,便要试试宝玉,走进屋里说道:“宝兄弟大喜!
老爷已择了吉日,要给你娶亲了。你喜欢不喜欢?”宝玉听了,只管瞅着凤姐笑,
微微的点点头儿。凤姐笑道:“给你娶林妹妹过来,好不好?”宝玉却大笑起来。
凤姐看着,也断不透他是明白,是糊涂,因又问道:“老爷说:你好了就给你娶林
妹妹呢。若还是这么傻,就不给你娶了。”宝玉忽然正色道:“我不傻,你才傻呢。”
说着,便站起来说:“我去瞧瞧林妹妹,叫他放心。”凤姐忙扶住了,说:“林妹
妹早知道了。他如今要做新媳妇了,自然害羞,不肯见你的。”宝玉道:“娶过来,
他到底是见我不见?”凤姐又好笑,又着忙,心里想:“袭人的话不差。提到林妹
妹,虽说仍旧说些疯话,却觉得明白些。若真明白了,将来不是林姑娘,打破了这
个灯虎儿,那饥荒才难打呢。”便忍笑说道:“你好好儿的便见你;若是疯疯癫癫
的,他就不见你了。”宝玉说道:“我有一个心,前儿已交给林妹妹了。他要过来,
横竖给我带来,还放在我肚子里头。”凤姐听着竟是疯话,便出来看着贾母笑。贾
母听了又是笑,又是疼,说道:“我早听见了。如今且不用理他,叫袭人好好的安
慰他,咱们走罢。”

说着,王夫人也来。大家到了薛姨妈那里,只说:“惦记着这边的事,来瞧瞧。”
薛姨妈感激不尽,说些薛蟠的话。喝了茶,薛姨妈要叫人告诉宝钗,凤姐连忙拦住,
说:“姑妈不必告诉宝妹妹。”又向薛姨妈陪笑说道:“老太太此来,一则为瞧姑
妈,二则也有句要紧的话,特请姑妈到那边商议。”薛姨妈听了,点点头儿说,“是
了”。于是大家又说些闲话,便回来了。当晚薛姨妈果然过来,见过了贾母,到王
夫人屋里来,不免说起王子腾来,大家落了一回眼。薛姨妈便问道:“刚才我到老
太太那里,宝哥儿出来请安,还好好儿的,不过略瘦些,怎么你们说得很利害?”
凤姐便道:“其实也不怎么,这只是老太太悬心。目今老爷又要起身外任去,不知
几年才来。老太太的意思:头一件叫老爷看着宝兄弟成了家,也放心;二则也给宝
兄弟冲冲喜,借大妹妹的金锁压压邪气,只怕就好了。”薛姨妈心里也愿意,只虑
着宝钗委屈,说道:“也使得,只是大家还要从长计较计较才好。”王夫人便按着
凤姐的话和薛姨妈说,只说:“姨太太这会子家里没人,不如把妆奁一概蠲免,明
日就打发蝌儿告诉蟠儿,一面这里过门,一面给他变法儿撕掳官事。”并不提宝玉
的心事。又说:“姨太太既作了亲,娶过来,早好一天,大家早放一天心。”正说
着,只见贾母差鸳鸯过来候信。薛姨妈虽恐宝钗委屈,然也没法儿,又见这般光景,
只得满口应承。鸳鸯回去回了贾母,贾母也甚喜欢,又叫鸳鸯过来求薛姨妈和宝钗
说明原故,不叫他受委屈。薛姨妈也答应了。便议定凤姐夫妇作媒人。大家散了,
王夫人姊妹不免又叙了半夜的话儿。

次日,薛姨妈回家,将这边的话细细的告诉了宝钗,还说:“我已经应承了。”
宝钗始则低头不语,后来便自垂泪。薛姨妈用好言劝慰,解释了好些说。宝钗自回
房内,宝琴随去解闷。薛姨妈又告诉了薛蝌,叫他:“明日起身,一则打听审详的
事,一则告诉你哥哥一个信儿。你即便回来。”

薛蝌去了四日,便回来回覆薛姨妈道:“哥哥的事,上司已经准了误杀,一过
堂就要题本了,叫咱们预备赎罪的银子。妹妹的事,说:‘妈妈做主很好的。赶着
办又省了好些银子。叫妈妈不用等我。该怎么着就怎么办罢。’”薛姨妈听了,一
则薛蟠可以回家,二则完了宝钗的事,心里安顿了好些。便是看着宝钗心里好像不
愿意似的,“虽是这样,他是女儿家,素来也孝顺守礼的人,知我应了,他也没得
说的”。便叫薛蝌:“办泥金庚帖,填上八字,即叫人送到琏二爷那边去,还问了
过礼的日子来,你好预备。本来咱们不惊动亲友。哥哥的朋友,是你说的,都是混
账人;亲戚呢,就是贾王两家。如今贾家是男家,王家无人在京里。史姑娘放定的
事,他家没有来请咱们,咱们也不用通知。倒是把张德辉请了来,托他照料些,他
上几岁年纪的人,到底懂事。”薛蝌领命,叫人送帖过去。

次日,贾琏过来见了薛姨妈,请了安,便说:“明日就是上好的日子。今日过
来回姨太太,就是明日过礼罢。只求姨太太不要挑饬就是了。”说着,捧过通书来。
薛姨妈也谦逊了几句,点头应允。贾琏赶着回去,回明贾政。贾政便道:“你回老
太太说:既不叫亲友们知道,诸事宁可简便些。若是东西上,请老太太瞧了就是了,
不必告诉我。”贾琏答应,进内将话回明贾母。这里王夫人叫了凤姐命人将过礼的
物件都送与贾母过目,并叫袭人告诉宝玉。那宝玉又嘻嘻的笑道:“这里送到园里,
回来园里又送到这里,咱们的人送,咱们的人收,何苦来呢?”贾母王夫人听了,
都喜欢道:“说他糊涂,他今日怎么这么明白呢。”鸳鸯等忍不住好笑,只得上来
一件一件的点明给贾母瞧,说:“这是金项圈,这是金珠首饰,共八十件。这是妆
蟒四十匹。这是各色绸缎一百二十匹。这是四季的衣服,共一百二十件。外面也没
有预备羊酒,这是折羊酒的银子。”贾母看了都说好,轻轻的与凤姐说道:“你去
告诉姨太太说:不是虚礼,求姨太太等蟠儿出来,慢慢的叫人给他妹妹做来就是了。
那好日子的被褥,还是咱们这里代办了罢。”凤姐答应出来,叫贾琏先过去。又叫
周瑞旺儿等,吩咐他们:“不必走大门,只从园里从前开的便门内送去。我也就过
去。这门离潇湘馆还远,倘别处的人见了,嘱咐他们不用在潇湘馆里提起。”众人
答应着,送礼而去。

宝玉认以为真,心里大乐,精神便觉的好些,只是语言总有些疯傻。那过礼的
回来,都不提名说姓,因此上下人等虽都知道,只因凤姐吩咐,都不敢走漏风声。

且说黛玉虽然服药,这病日重一日。紫鹃等在旁苦劝,说道:“事情到了这个
分儿,不得不说了。姑娘的心事,我们也都知道。至于意外之事,是再没有的。姑
娘不信,只拿宝玉的身子说起,这样大病,怎么做得亲呢?姑娘别听瞎话,自己安
心保重才好。”黛玉微笑一笑,也不答言,又咳嗽数声,吐出好些血来。紫鹃等看
去,只有一息奄奄,明知劝不过来,惟有守着流泪。天天三四趟去告诉贾母,鸳鸯
测度贾母近日比前疼黛玉的心差了些,所以不常去回。况贾母这几日的心都在宝钗
宝玉身上,不见黛玉的信儿,也不大提起,只请太医调治罢了。

黛玉向来病着,自贾母起直到姊妹们的下人常来问候,今见贾府中上下人等都
不过来,连一个问的人都没有,睁开眼只有紫鹃一人。自料万无生理,因扎挣着向
紫鹃说道:“妹妹,你是我最知心的。虽是老太太派你伏侍我,这几年,我拿你就
当作我的亲妹妹。”说到这里,气又接不上来。紫鹃听了,一阵心酸,早哭得说不
出话来。迟了半日,黛玉又一面喘,一面说道:“紫鹃妹妹,我躺着不受用,你扶
起我来靠着坐坐才好。”紫鹃道:“姑娘的身上不大好,起来又要抖搂着了。”黛
玉听了,闭上眼不言语了,一时又要起来。紫鹃没法,只得同雪雁把他扶起,两边
用软枕靠住,自己却倚在旁边。黛玉那里坐得住,下身自觉硌的疼,狠命的掌着。
叫过雪雁来道:“我的诗本子……”说着,又喘。

雪雁料是要他前日所理的诗稿,因找来送到黛玉跟前。黛玉点点头儿,又抬眼
看那箱子。雪雁不解,只是发怔。黛玉气的两眼直瞪,又咳嗽起来,又吐了一口血。
雪雁连忙回身取了水来,黛玉漱了,吐在盂内。紫鹃用绢子给他拭了嘴,黛玉便拿
那绢子指着箱子,又喘成一处,说不上来,闭了眼。紫鹃道:“姑娘歪歪儿罢。”
黛玉又摇摇头儿。紫鹃料是要绢子,便叫雪雁开箱,拿出一块白绫绢子来。黛玉瞧
了,撂在一边,使劲说道:“有字的。”紫鹃这才明白过来要那块题诗的旧帕,只
得叫雪雁拿出来递给黛玉。紫鹃劝道:“姑娘歇歇儿罢,何苦又劳神?等好了再瞧
罢。”只见黛玉接到手里也不瞧,扎挣着伸出那只手来,狠命的撕那绢子。却是只
有打颤的分儿,那里撕得动。紫鹃早已知他是恨宝玉,却也不敢说破,只说:“姑
娘,何苦自己又生气!”黛玉微微的点头,便掖在袖里。说叫:“点灯。”

雪雁答应,连忙点上灯来。黛玉瞧瞧,又闭上眼坐着,喘了一会子,又道:“笼
上火盆。”紫鹃打量他冷,因说道:“姑娘躺下,多盖一件罢。那炭气只怕耽不住。”
黛玉又摇头儿。雪雁只得笼上,搁在地下火盆架上。黛玉点头,意思叫挪到炕上来。
雪雁只得端上来,出去拿那张火盆炕桌。那黛玉却又把身子欠起,紫鹃只得两只手
来扶着他。黛玉这才将方才的绢子拿在手中,瞅着那火,点点头儿,往上一撂。紫
鹃唬了一跳,欲要抢时,两只手却不敢动。雪雁又出去拿火盆桌子,此时那绢子已
经烧着了。紫鹃劝道:“姑娘!这是怎么说呢!”黛玉只作不闻,回手又把那诗稿
拿起来,瞧了瞧,又撂下了。紫鹃怕他也要烧,连忙将身倚住黛玉,腾出手来拿时,
黛玉又早拾起,撂在火上。此时紫鹃却够不着,干急。雪雁正拿进桌子来,看见黛
玉一撂,不知何物,赶忙抢时,那纸沾火就着,如何能够少待,早已烘烘的着了。
雪雁也顾不得烧手,从火里抓起来,撂在地下乱踩,却已烧得所馀无几了。那黛玉
把眼一闭,往后一仰,几乎不曾把紫鹃压倒。紫鹃连忙叫雪雁上来,将黛玉扶着放
倒,心里突突的乱跳。欲要叫人时,天又晚了;欲不叫人时,自己同着雪雁和鹦哥
等几个小丫头,又怕一时有什么原故。好容易熬了一夜。

到了次日早起,觉黛玉又缓过一点儿来。饭后,忽然又嗽又吐,又紧起来。紫
鹃看着不好了,连忙将雪雁等都叫进来看守,自己却来回贾母。那知到了贾母上房,
静悄悄的,只有两三个老妈妈和几个做粗活的丫头在那里看屋子呢。紫鹃因问道:
“老太太呢?”那些人都说:“不知道。”紫鹃听这话诧异,遂到宝玉屋里去看,
竟也无人。遂问屋里的丫头,也说不知。紫鹃已知八九:“但这些人怎么竟这样狠
毒冷淡!”又想到黛玉这几天竟连一个人问的也没有,越想越悲,索性激起一腔闷
气来,一扭身便出来了。自己想了一想:“今日倒要看看宝玉是何形状,看他见了
我怎么样过的去!那一年我说了一句谎话,他就急病了,今日竟公然做出这件事来。
可知天下男子之心真真是冰寒雪冷,令人切齿的!”

一面走一面想,早已来到怡红院。只见院门虚掩,里面却又寂静的很。紫鹃忽
然想到:“他要娶亲,自然是有新屋子的,但不知他这新屋子在何处?”正在那里
徘徊瞻顾,看见墨雨飞跑,紫鹃便叫住他。墨雨过来笑嘻嘻的道:“姐姐到这里做
什么?”紫鹃道:“我听见宝二爷娶亲,我要来看看热闹儿,谁知不在这里。也不
知是几儿?”墨雨悄悄的道:“我这话只告诉姐姐,你可别告诉雪雁。他们上头吩
咐了,连你们都不叫知道呢。就是今日夜里娶。那里是在这里?老爷派琏二爷另收
拾了房子了。”说着,又问:“姐姐有什么事么?”紫鹃道:“没什么事,你去罢。”
墨雨仍旧飞跑去了。紫鹃自己发了一回呆,忽然想起黛玉来,这时候还不知是死是
活,因两泪汪汪,咬着牙,发狠道:“宝玉!我看他明儿死了,你算是躲的过,不
见了!你过了你那如心如意的事儿,拿什么脸来见我!”一面哭一面走,呜呜咽咽
的,自回去了。

还未到潇湘馆,只见两个小丫头在门里往外探头探脑的,一眼看见紫鹃,那一
个便嚷道:“那不是紫鹃姐姐来了吗!”紫鹃知道不好了,连忙摆手儿不叫嚷。赶
忙进来看时,只见黛玉肝火上炎,两颧红赤。紫鹃觉得不妥,叫了黛玉的奶妈王奶
奶来,一看,他便大哭起来。这紫鹃因王奶妈有些年纪,可以仗个胆儿,谁知竟是
个没主意的人,反倒把紫鹃弄的心里七上八下。忽然想起一个人来,便命小丫头急
忙去请。你道是谁?原来紫鹃想起李宫裁是个孀居,今日宝玉结亲,他自然回避;
况且园中诸事,向系李纨料理,所以打发人去请他。李纨正在那里给贾兰改诗,冒
冒失失的见一个丫头进来回说:“大奶奶!只怕林姑娘不好了!那里都哭呢。”李纨
听了,吓了一大跳,也不及问了,连忙站起身来便走,素云碧月跟着。一头走着,
一头落泪,想着:“姐妹在一处一场,更兼他那容貌才情,真是寡二少双,惟有青
女素娥可以仿佛一二。竟这样小小的年纪,就作了北邙乡女。偏偏凤姐想出一条偷
梁换柱之计,自己也不好过潇湘馆来,竟未能少尽姊妹之情,真真可怜可叹!”一
头想着,已走到潇湘馆的门口。里面却又寂然无声,李纨倒着起忙来:“想来必是
已死,都哭过了,那衣衾装裹未知妥当了没有?”连忙三步两步走进屋子来。里间
门口一个小丫头已经看见,便说:“大奶奶来了。”紫鹃忙往外走,和李纨走了个
对面。李纨忙问:“怎么样?”紫鹃欲说话时,惟有喉中哽咽的分儿,却一字说不
出,那眼泪一似断线珍珠一般,只将一只手回过去指着黛玉。

李纨看了紫鹃这般光景,更觉心酸,也不再问,连忙走过来看时,那黛玉已不
能言。李纨轻轻叫了两声。黛玉却还微微的开眼,似有知识之状,但只眼皮嘴唇微
有动意,口内尚有出入之息,却要一句话、一点泪也没有了。李纨回身,见紫鹃不
在眼前,便问雪雁。雪雁道:“他在外头屋里呢。”李纨连忙出来,只见紫鹃在外
间空床上躺着,颜色青黄,闭了眼,只管流泪,那鼻涕眼泪把一个砌花锦边的褥子
已湿了碗大的一片。李纨连忙唤他,那紫鹃才慢慢的睁开眼,欠起身来。李纨道:
“傻丫头,这是什么时候,且只顾哭你的。林姑娘的衣衾,还不拿出来给他换上,
还等多早晚呢?难道他个女孩儿家,你还叫他失身露体,精着来,光着去吗?”紫
鹃听了这句话,一发止不住痛哭起来。李纨一面也哭,一面着急,一面拭泪,一面
拍着紫鹃的肩膀说:“好孩子!你把我的心都哭乱了!快着收拾他的东西罢,再迟一
会子就了不得了。”

正闹着,外边一个人慌慌张张跑进来,倒把李纨唬了一跳。看时,却是平儿,
跑进来看见这样,只是呆磕磕的发怔。李纨道:“你这会子不在那边,做什么来了?”
说着,林之孝家的也进来了。平儿道:“奶奶不放心,叫来瞧瞧。既有大奶奶在这
里,我们奶奶就只顾那一头儿了。”李纨点点头儿。平儿道:“我也见见林姑娘。”
说着,一面往里走,一面早已流下泪来。这里李纨因和林之孝家的道:“你来的正
好,快出去瞧瞧去,告诉管事的预备林姑娘的后事。妥当了,叫他来回我,不用到
那边去。”林之孝家的答应了,还站着。李纨道:“还有什么话呢?”林之孝家的
道:“刚才二奶奶和老太太商量了,那边用紫鹃姑娘使唤使唤呢。”李纨还未答言,
只见紫鹃道:“林奶奶,你先请罢!等着人死了,我们自然是出去的,那里用这么
——”说到这里,却又不好说了,因又改说道:“况且我们在这里守着病人,身上
也不洁净。林姑娘还有气儿呢,不时的叫我。”李纨在旁解说道:“当真的,林姑
娘和这丫头也是前世的缘法儿。倒是雪雁是他南边带来的,他倒不理会;惟有紫鹃,
我看他两个一时也离不开。”林之孝家的头里听了紫鹃的话,未免不受用,被李纨
这一番话,却也没有说的了。又见紫鹃哭的泪人一般,只好瞅着他微微的笑,说道:
“紫鹃姑娘这些闲话倒不要紧,只是你却说得,我可怎么回老太太呢?况且这话是
告诉得二奶奶的吗?”正说着,平儿擦着眼泪出来道:“告诉二奶奶什么事?”林
之孝家的将方才的话说了一遍。平儿低了一回头,说:“这么着罢,就叫雪姑娘去
罢。”李纨道:“他使得吗?”平儿走到李纨耳边说了几句。李纨点点头儿道:“既
是这么着,就叫雪雁过去也是一样的。”林之孝家的因问平儿道:“雪姑娘使得吗?”
平儿道:“使得,都是一样。”林家的道:“那么着,姑娘就快叫雪姑娘跟了我去。
我先回了老太太和二奶奶。这可是大奶奶和姑娘的主意,回来姑娘再各自回二奶奶
去。”李纨道:“是了,你这么大年纪,连这么点子事还不耽呢。”林家的笑道:
“不是不耽:头一宗,这件事,老太太和二奶奶办事,我们都不能很明白;再者,
又有大奶奶和平姑娘呢。”

说着,平儿已叫了雪雁出来。原来雪雁因这几日黛玉嫌他“小孩子家懂得什
么”,便也把心冷淡了,况且听是老太太和二奶奶叫,也不敢不去,连忙收拾了头。
平儿叫他换了新鲜衣服,跟着林家的去了。随后平儿又和李纨说了几句话。李纨又
嘱咐平儿,打那么催着林家的叫他男人快办了来。平儿答应着出来,转了个弯子,
看见林家的带着雪雁在前头走呢,赶忙叫住道:“我带了他去罢。你先告诉林大爷
办林姑娘的东西去罢。奶奶那里我替回就是了。”那林家的答应着去了。这里平儿
带了雪雁到了新房子里回明了,自去办事。

却说雪雁看见这个光景,想起他家姑娘,也未免伤心,只是在贾母凤姐跟前不
敢露出。因又想道:“也不知用我作什么?我且瞧瞧,宝玉一日家和我们姑娘好的
蜜里调油,这时候总不见面了,也不知是真病假病。只怕是怕我们姑娘恼,假说丢
了玉,装出傻子样儿来,叫那一位寒了心,他好娶宝姑娘的意思。我索性看看他,
看他见了我傻不傻。难道今儿还装傻么?”一面想着,已溜到里间屋子门口,偷偷
儿的瞧。这时宝玉虽因失玉昏愦,但只听见娶了黛玉为妻,真乃是从古至今、天上
人间、第一件畅心满意的事了,那身子顿觉健旺起来,只不过不似从前那般灵透,
所以凤姐的妙计,百发百中。巴不得就见黛玉,盼到今日完姻,真乐的手舞足蹈,
虽有几句傻话,却与病时光景大相悬绝了。雪雁看了,又是生气,又是伤心,他那
里晓得宝玉的心事,便各自走开。

这里宝玉便叫袭人快快给他装新,坐在王夫人屋里。看见凤姐尤氏忙忙碌碌,
再盼不到吉时,只管问袭人道:“林妹妹打园里来,为什么这么费事,还不来?”
袭人忍着笑道:“等好时辰呢。”又听见凤姐和王夫人说道:“虽然有服,外头不
用鼓乐,咱们家的规矩要拜堂的,冷清清的使不的。我传了家里学过音乐管过戏的
那些女人来,吹打着热闹些。”王夫人点头说:“使得。”

一时,大轿从大门进来,家里细乐迎出去,十二对宫灯排着进来,倒也新鲜雅
致。傧相请了新人出轿,宝玉见喜娘披着红,扶着新人,着盖头。下首扶新人的
你道是谁,原来就是雪雁。宝玉看见雪雁,犹想:“因何紫鹃不来,倒是他呢?”
又想道:“是了,雪雁原是他南边家里带来的,紫鹃是我们家的,自然不必带来。”
因此,见了雪雁竟如见了黛玉的一般欢喜。傧相喝礼,拜了天地。请出贾母受了四
拜,后请贾政夫妇等登堂,行礼毕,送入洞房。还有坐帐等事,俱是按本府旧例,
不必细说。贾政原为贾母作主,不敢违拗,不信冲喜之说。那知今日宝玉居然像个
好人,贾政见了,倒也喜欢。

那新人坐了帐,就要揭盖头的。凤姐早已防备,请了贾母王夫人等进去照应。
宝玉此时到底有些傻气,便走到新人跟前说道:“妹妹,身上好了?好些天不见了。
盖着这劳什子做什么?”欲待要揭去,反把贾母急出一身冷汗来。宝玉又转念一想
道:“林妹妹是爱生气的,不可造次了。”又歇了一歇,仍是按捺不住,只得上前,
揭了盖头。喜娘接去,雪雁走开,莺儿上来伺候。宝玉睁眼一看,好像是宝钗。心
中不信,自己一手持灯,一手擦眼一看,可不是宝钗么!只见他盛妆艳服,丰肩软
体,鬟低鬓,眼息微,论雅淡似荷粉露垂,看娇羞真是杏花烟润了。

宝玉发了一回怔,又见莺儿立在傍边,不见了雪雁。此时心无主意,自己反以
为是梦中了,呆呆的只管站着。众人接过灯去,扶着坐下,两眼直视,半语全无。
贾母恐他病发,亲自过来招呼着。凤姐尤氏请了宝钗进入里间坐下。宝钗此时自然
是低头不语。宝玉定了一回神,见贾母王夫人坐在那边,便轻轻的叫袭人道:“我
是在那里呢?这不是做梦么?”袭人道:“你今日好日子,什么梦不梦的混说!老爷
可在外头呢。”宝玉悄悄的拿手指着道:“坐在那里的这一位美人儿是谁?”袭人
握了自己的嘴,笑的说不出话来,半日才说道:“那是新娶的二奶奶。”众人也都
回过头去忍不住的笑。宝玉又道:“好糊涂!你说‘二奶奶’,到底是谁?”袭人
道:“宝姑娘。”宝玉道:“林姑娘呢?”袭人道:“老爷作主娶的是宝姑娘,怎
么混说起林姑娘来?”宝玉道:“我才刚看见林姑娘了么,还有雪雁呢。怎么说没
有?你们这都是做什么玩呢?”凤姐便走上来,轻轻的说道:“宝姑娘在屋里坐着
呢,别混说。回来得罪了他,老太太不依的。”宝玉听了,这会子糊涂的更利害了。
本来原有昏愦的病,加以今夜神出鬼没,更叫他不得主意,便也不顾别的,口口声
声只要找林妹妹去。贾母等上前安慰,无奈他只是不懂。又有宝钗在内,又不好明
说。知宝玉旧病复发,也不讲明,只得满屋里点起安息香来,定住他的神魂,扶他
睡下。众人鸦雀无闻。停了片时,宝玉便昏沉睡去,贾母等才得略略放心,只好坐
以待旦,叫凤姐去请宝钗安歇。宝钗置若罔闻,也便和衣在内暂歇。贾政在外,未
知内里原由,只就方才眼见的光景想来,心下倒放宽了。恰是明日就是起程的吉日,
略歇了一歇,众人贺喜送行。贾母见宝玉睡着,也回房去暂歇。

次早,贾政辞了宗祠,过来拜别贾母,称:“不孝远离,惟愿老太太顺时颐
养。儿子一到任所,即修禀请安,不必挂念。宝玉的事,已经依了老太太完结,只
求老太太训诲。”贾母恐贾政在路不放心,并不将宝玉复病的话说起,只说:“我
有一句话:宝玉昨夜完姻,并不是同房,今日你起身,必该叫他远送才是。但他因
病冲喜,如今才好些,又是昨日一天劳乏,出来恐怕着了风。故此问你:你叫他送
呢,即刻去叫他;你若疼他,就叫人带了他来你见见,叫他给你磕个头就算了。”
贾政道:“叫他送什么?只要他从此以后认真念书,比送我还喜欢呢。”贾母听了,
又放了一条心。便叫贾政坐着,叫鸳鸯去,如此如此,带了宝玉,叫袭人跟着来。
鸳鸯去了不多一会,果然宝玉来了,仍是叫他行礼他便行礼。只可喜此时宝玉见了
父亲,神志略敛些,片时清楚,也没什么大差。贾政吩咐了几句,宝玉答应了。贾
政叫人扶他回去了,自己回到王夫人房中,又切实的叫王夫人管教儿子:“断不可
如前骄纵。明年乡试,务必叫他下场。”王夫人一一的听了,也没提起别的,即忙
命人搀扶着宝钗过来,行了新妇送行之礼,也不出房。其馀内眷俱送至二门而回。
贾珍等也受了一番训饬。大家举酒送行,一班子弟及晚辈亲友直送至十里长亭而
别。

不言贾政起程赴任。且说宝玉回来,旧病陡发,更加昏愦,连饮食也不能进了。

未知性命如何,且看下回分解。