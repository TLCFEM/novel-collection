\chapter{诉肺腑心迷活宝玉~含耻辱情烈死金钏}

话说宝玉见那麒麟,心中甚是欢喜,便伸手来拿,笑道:“亏你拣着了!你是
怎么拾着的?”湘云笑道:“幸而是这个。明日倘或把印也丢了,难道也就罢了不
成?”宝玉笑道:“倒是丢了印平常,若丢了这个,我就该死了。”

袭人倒了茶来与湘云吃,一面笑道:“大姑娘,我前日听见你大喜呀。”湘云
红了脸,扭过头去吃茶,一声也不答应。袭人笑道:“这会子又害臊了?你还记得
那几年,咱们在西边暖阁上住着,晚上你和我说的话?那会子不害臊,这会子怎么
又臊了?”湘云的脸越发红了,勉强笑道:“你还说呢!那会子咱们那么好,后来
我们太太没了,我家去住了一程子,怎么就把你配给了他。我来了,你就不那么待
我了。”袭人也红了脸,笑道:“罢呦!先头里,‘姐姐’长,‘姐姐’短,哄着
我替你梳头洗脸,做这个弄那个,如今拿出小姐款儿来了。你既拿款,我敢亲近
吗?”湘云道:“阿弥陀佛,冤枉冤哉!我要这么着,就立刻死了。你瞧瞧,这么
大热天,我来了必定先瞧瞧你。你不信问缕儿:我在家时时刻刻,那一回不想念你
几句?”袭人和宝玉听了,都笑劝道:“说玩话儿,你又认真了。还是这么性儿急。”
湘云道:“你不说你的话咽人,倒说人性急。”

一面说,一面打开绢子,将戒指递与袭人。袭人感谢不尽,因笑道:“你前日
送你姐姐们的,我已经得了。今日你亲自又送来,可见是没忘了我。就为这个试出
你来了。戒指儿能值多少,可见你的心真。”史湘云道:“是谁给你的?”袭人道:
“是宝姑娘给我的。”湘云叹道:“我只当林姐姐送你的,原来是宝姐姐给了你。
我天天在家里想着,这些姐姐们,再没一个比宝姐姐好的。可惜我们不是一个娘养
的。我但凡有这么个亲姐姐,就是没了父母,也没妨碍的!”说道,眼圈儿就红了。
宝玉道:“罢罢罢,不用提起这个话了。”史湘云道:“提这个便怎么?我知道你
的心病:恐怕你的林妹妹听见,又嗔我赞了宝姐姐了。可是为这个不是?”袭人在
旁嗤的一笑,说道:“云姑娘,你如今大了,越发心直嘴快了。”宝玉笑道:“我
说你们这几个人难说话,果然不错。”史湘云道:“好哥哥,你不必说话叫我恶心。
只会在我跟前说话,见了你林妹妹,又不知怎么好了。”

袭人道:“且别说玩话,正有一件事要求你呢。”史湘云便问:“什么事?”
袭人道:“有一双鞋,抠了垫心子,我这两日身上不好,不得做,你可有工夫替我
做做?”史湘云道:“这又奇了。你家放着这些巧人不算,还有什么针线上的、裁
剪上的,怎么叫我做起来?你的活计叫人做,谁好意思不做呢?”袭人笑道:“你
又糊涂了。你难道不知道:我们这屋里的针线,是不要那些针线上的人做的。”史
湘云听了,便知是宝玉的鞋,因笑道:“既这么说,我就替你做做罢。只是一件:
你的我才做,别人的我可不能。”袭人笑道:“又来了。我是个什么儿,就敢烦你
做鞋了!实告诉你:可不是我的。你别管是谁的,横竖我领情就是了。”史湘云道:
“论理,你的东西也不知烦我做了多少。今日我倒不做的原故,你必定也知道。”
袭人道:“我倒也不知道。”史湘云冷笑道:“前日我听见把我做的扇套儿拿着和
人家比,赌气又铰了。我早就听见了,你还瞒我?这会子又叫我做,我成了你们奴
才了。”宝玉忙笑道:“前日的那个本不知是你做的。”袭人也笑道:“他本不知
是你做的,是我哄他的话,说是‘新近外头有个会做活的,扎的绝出奇的好花儿,
叫他们拿了一个扇套儿试试看好不好’,他就信了,拿出去给这个瞧、那个看的。
不知怎么又惹恼了那一位,铰了两段。回来他还叫赶着做去,我才说了是你做的,
他后悔的什么似的!”史湘云道:“这越发奇了。林姑娘也犯不上生气,他既会剪,
就叫他做。”袭人道:“他可不做呢。饶这么着老太太还怕他劳碌着了,大夫又说
好生静养才好,谁还肯烦他做呢?旧年好一年的工夫做了个香袋儿,今年半年还没
见拿针线呢。”

正说着,有人来回说:“兴隆街的大爷来了,老爷叫二爷出去会。”宝玉听了,
便知贾雨村来了,心中好不自在。袭人忙去拿衣服。宝玉一面登着靴子,一面抱怨
道:“有老爷和他坐着就罢了,回回定要见我!”史湘云一边摇着扇子,笑道:“自
然你能迎宾接客,老爷才叫你出去呢。”宝玉道:“那里是老爷?都是他自己要请
我见的。”湘云笑道:“‘主雅客来勤’,自然你有些警动他的好处,他才要会你。”
宝玉道:“罢,罢,我也不过俗中又俗的一个俗人罢了,并不愿和这些人来往。”
湘云笑道:“还是这个性儿,改不了!如今大了,你就不愿意去考举人进士的,也
该常会会这些为官作宦的,谈讲谈讲那些仕途经济,也好将来应酬事务,日后也有
个正经朋友。让你成年家只在我们队里,搅的出些什么来?”

宝玉听了,大觉逆耳,便道:“姑娘请别的屋里坐坐罢,我这里仔细腌了你
这样知经济的人!”袭人连忙解说道:“姑娘快别说他。上回也是宝姑娘说过一回,
他也不管人脸上过不去,了一声,拿起脚来就走了。宝姑娘的话也没说完,见他
走了,登时羞的脸通红,说不是,不说又不是。幸而是宝姑娘,那要是林姑娘,不
知又闹的怎么样、哭的怎么样呢!提起这些话来,宝姑娘叫人敬重。自己过了一会
子去了,我倒过不去,只当他恼了,谁知过后还是照旧一样,真真是有涵养、心地
宽大的。谁知这一位反倒和他生分了。那林姑娘见他赌气不理,他后来不知赔多少
不是呢。”宝玉道:“林姑娘从来说过这些混帐话吗?要是他也说过这些混帐话,
我早和他生分了。”袭人和湘云都点头笑道:“这原是混帐话么?”

原来黛玉知道史湘云在这里,宝玉一定又赶来,说麒麟的原故。因心下忖度着,
近日宝玉弄来的外传野史,多半才子佳人,都因小巧玩物上撮合,或有鸳鸯,或有
凤凰,或玉环金佩,或鲛帕鸾绦,皆由小物而遂终身之愿。今忽见宝玉也有麒麟,
便恐借此生隙,同湘云也做出那些风流佳事来。因而悄悄走来,见机行事,以察二
人之意。不想刚走进来,正听见湘云说“经济”一事,宝玉又说“林妹妹不说这些
混帐话,要说这话,我也和他生分了”。黛玉听了这话,不觉又喜又惊,又悲又叹。
所喜者:果然自己眼力不错,素日认他是个知己,果然是个知己;所惊者:他在人
前一片私心称扬于我,其亲热厚密,竟不避嫌疑;所叹者:你既为我的知己,自然
我亦可为你的知己,既你我为知己,又何必有“金玉”之论呢?既有“金玉”之论,
也该你我有之,又何必来一宝钗呢?所悲者:父母早逝,虽有铭心刻骨之言,无人
为我主张;况近日每觉神思恍惚,病已渐成,医者更云:“气弱血亏,恐致劳怯之
症。”我虽为你的知己,但恐不能久待;你纵为我的知己,奈我薄命何!想到此间,
不禁泪又下来。待要进去相见,自觉无味,便一面拭泪,一面抽身回去了。

这里宝玉忙忙的穿了衣裳出来,忽见黛玉在前面慢慢的走着,似乎有拭泪之
状,便忙赶着上来笑道:“妹妹往那里去?怎么又哭了?又是谁得罪了你了?”黛玉
回头见是宝玉,便勉强笑道:“好好的,我何曾哭来。”宝玉笑道:“你瞧瞧,睛
睛上的泪珠儿没干,还撒谎呢。”一面说,一面禁不住抬起手来,替他拭泪。黛玉
忙向后退了几步,说道:“你又要死了!又这么动手动脚的。”宝玉笑道:“说话
忘了情,不觉的动了手,也就顾不得死活。”黛玉道:“死了倒不值什么,只是丢
下了什么‘金’,又是什么‘麒麟’,可怎么好呢!”一句话又把宝玉说急了,赶
上来问道:“你还说这些话,到底是咒我还是气我呢?”黛玉见问,方想起前日的
事来,遂自悔这话又说造次了,忙笑道:“你别着急,我原说错了。这有什么要紧,
筋都叠暴起来,急的一脸汗!”一面说,一面也近前伸手替他拭面上的汗。

宝玉瞅了半天,方说道:“你放心。”黛玉听了,怔了半天,说道:“我有什
么不放心的?我不明白你这个话。你倒说说,怎么放心不放心?”宝玉叹了一口气,
问道:“你果然不明白这话?难道我素日在你身上的心都用错了?连你的意思若体贴
不着,就难怪你天天为我生气了。”黛玉道:“我真不明白放心不放心的话。”宝
玉点头叹道:“好妹妹,你别哄我。你真不明白这话,不但我素日白用了心,且连
你素日待我的心也都辜负了。你皆因都是不放心的原故,才弄了一身的病了。但凡
宽慰些,这病也不得一日重似一日了!”

黛玉听了这话,如轰雷掣电,细细思之,竟比自己肺腑中掏出来的还觉恳切,
竟有万句言语,满心要说,只是半个字也不能吐出,只管怔怔的瞅着他。此时宝玉
心中也有万句言词,不知一时从那一句说起,却也怔怔的瞅着黛玉。两个人怔了半
天,黛玉只了一声,眼中泪直流下来,回身便走。宝玉忙上前拉住道:“好妹妹,
且略站住,我说一句话再走。”黛玉一面拭泪,一面将手推开,说道:“有什么可
说的?你的话我都知道了。”口里说着,却头也不回,竟去了。

宝玉望着,只管发起呆来。原来方才出来忙了,不曾带得扇子,袭人怕他热,
忙拿了扇子赶来送给他,猛抬头看见黛玉和他站着。一时黛玉走了,他还站着不动,
因而赶上来说道:“你也不带了扇子去,亏了我看见,赶着送来。”宝玉正出了神,
见袭人和他说话,并未看出是谁,只管呆着脸说道:“好妹妹,我的这个心,从来
不敢说,今日胆大说出来,就是死了也是甘心的!我为你也弄了一身的病,又不敢
告诉人,只好捱着。等你的病好了,只怕我的病才得好呢。睡里梦里也忘不了你!”
袭人听了,惊疑不止,又是怕,又是急,又是臊,连忙推他道:“这是那里的话?
你是怎么着了?还不快去吗?”宝玉一时醒过来,方知是袭人。虽然羞的满面紫涨,
却仍是呆呆的,接了扇子,一句话也没有,竟自走去。

这里袭人见他去后,想他方才之言必是因黛玉而起,如此看来,倒怕将来难免
不才之事,令人可惊可畏。却是如何处治,方能免此丑祸?想到此间,也不觉呆呆
的发起怔来。谁知宝钗恰从那边走来,笑道:“大毒日头地下,出什么神呢?”袭
人见问,忙笑说道:“我才见两个雀儿打架,倒很有个玩意儿,就看住了。”宝钗
道:“宝兄弟才穿了衣服,忙忙的那里去了?我要叫住问他呢,只是他慌慌张张的
走过去,竟像没理会我的,所以没问。”袭人道:“老爷叫他出去的。”宝钗听了,
忙说道:“嗳哟,这么大热的天,叫他做什么?别是想起什么来生了气,叫他出去
教训一场罢?”袭人笑道:“不是这个,想必有客要会。”宝钗笑道:“这个客也
没意思,这么热天不在家里凉快,跑什么!”袭人笑道:“你可说么!”

宝钗因问:“云丫头在你们家做什么呢?”袭人笑道:“才说了会子闲话儿,
又瞧了会子我前日粘的鞋帮子,明日还求他做去呢。”宝钗听见这话,便两边回头,
看无人来往,笑道:“你这么个明白人,怎么一时半刻的就不会体谅人?我近来看
着云姑娘的神情儿,风里言风里语的听起来,在家里一点儿做不得主。他们家嫌费
用大,竟不用那些针线上的人,差不多儿的东西都是他们娘儿们动手。为什么这几
次他来了,他和我说话儿,见没人在跟前,他就说家里累的慌?我再问他两句家常
过日子的话,他就连眼圈儿都红了,嘴里含含糊糊待说不说的。看他的形景儿,自
然从小儿没了父母是苦的。我看见他也不觉的伤起心来。”袭人见说这话,将手一
拍道:“是了。怪道上月我求他打十根蝴蝶儿结子,过了那些日子才打发人送来,
还说:‘这是粗打的,且在别处将就使罢;要匀净的,等明日来住着再好生打。’
如今听姑娘这话,想来我们求他,他不好推辞,不知他在家里怎么三更半夜的做呢!
可是我也糊涂了,早知道是这么着,我也不该求他!”宝钗道:“上次他告诉我,
说在家里做活做到三更天,要是替别人做一点半点儿,那些奶奶太太们还不受用
呢。”袭人道:“偏我们那个牛心的小爷,凭着小的大的活计,一概不要家里这些
活计上的人做,我又弄不开这些。”宝钗笑道:“你理他呢!只管叫人做去就是了。”
袭人道:“那里哄的过他?他才是认得出来呢。说不得我只好慢慢的累去罢了。”
宝钗笑道:“你不必忙,我替你做些就是了。”袭人笑道:“当真的?这可就是我
的造化了!晚上我亲自过来——”

一句话未了,忽见一个老婆子忙忙走来,说道:“这是那里说起!金钏儿姑娘
好好儿的投井死了!”袭人听得,唬了一跳,忙问:“那个金钏儿?”那老婆子道:
“那里还有两个金钏儿呢?就是太太屋里的。前日不知为什么撵出去,在家里哭天
抹泪的,也都不理会他,谁知找不着他,才有打水的人说那东南角上井里打水,见
一个尸首,赶着叫人打捞起来,谁知是他!他们还只管乱着要救,那里中用了呢?”
宝钗道:“这也奇了!”袭人听说,点头赞叹,想素日同气之情,不觉流下泪来。
宝钗听见这话,忙向王夫人处来安慰。这里袭人自回去了。

宝钗来至王夫人房里,只见鸦雀无闻,独有王夫人在里间房内坐着垂泪。宝钗
便不好提这事,只得一旁坐下。王夫人便问:“你打那里来?”宝钗道:“打园里
来。”王夫人道:“你打园里来,可曾见你宝兄弟?”宝钗道:“才倒看见他了:
穿着衣裳出去了,不知那里去。”王夫人点头叹道:“你可知道一件奇事?金钏儿
忽然投井死了!”宝钗见说,道:“怎么好好儿的投井?这也奇了。”王夫人道:
“原是前日他把我一件东西弄坏了,我一时生气,打了他两下子,撵了下去。我只
说气他几天,还叫他上来,谁知他这么气性大,就投井死了。岂不是我的罪过!”
宝钗笑道:“姨娘是慈善人,固然是这么想。据我看来,他并不是赌气投井,多半
他下去住着,或是在井傍边儿玩,失了脚掉下去的。他在上头拘束惯了,这一出去
自然要到各处去玩玩逛逛儿,岂有这样大气的理?纵然有这样大气,也不过是个糊
涂人,也不为可惜。”王夫人点头叹道:“虽然如此,到底我心里不安!”宝钗笑
道:“姨娘也不劳关心。十分过不去,不过多赏他几两银子发送他,也就尽了主仆
之情了。”王夫人道:“才刚我赏了五十两银子给他妈,原要还把你姐妹们的新衣
裳给他两件装裹,谁知可巧都没有什么新做的衣裳,只有你林妹妹做生日的两套。
我想你林妹妹那孩子,素日是个有心的,况且他也三灾八难的,既说了给他作生日,
这会子又给人去装裹,岂不忌讳?因这么着,我才现叫裁缝赶着做一套给他。要是
别的丫头,赏他几两银子,也就完了。金钏儿虽然是个丫头,素日在我跟前,比我
的女孩儿差不多儿!”口里说着,不觉流下泪来。宝钗忙道:“姨娘这会子何用叫
裁缝赶去。我前日倒做了两套,拿来给他,岂不省事?况且他活的时候儿也穿过我
的旧衣裳,身量也相对。”王夫人道:“虽然这样,难道你不忌讳?”宝钗笑道:
“姨娘放心,我从来不计较这些。”一面说,一面起身就走。王夫人忙叫了两个人
跟宝钗去。

一时宝钗取了衣服回来,只见宝玉在王夫人旁边坐着垂泪。王夫人正才说他,
因见宝钗来了,就掩住口不说了。宝钗见此景况,察言观色,早知觉了七八分。于
是将衣服交明王夫人,王夫人便将金钏儿的母亲叫来拿了去了。

后事如何,下回分解。