\chapter{感秋声抚琴悲往事~坐禅寂走火入邪魔}

却说黛玉叫进宝钗家的女人来,问了好,呈上书子,黛玉叫他去喝茶,便将宝
钗来书打开看时,只见上面写着:

妹生辰不偶,家运多艰,姊妹伶仃,萱亲衰迈。兼之声狺语,旦暮无休;更
遭惨祸飞灾,不啻惊风密雨。夜深辗侧,愁绪何堪。属在同心,能不为之愍恻乎?
回忆海棠结社,序属清秋,对菊持螯,同盟欢洽。犹记“孤标傲世偕谁隐,一样开
花为底迟”之句,未尝不叹冷节馀芳,如吾两人也!感怀触绪,聊赋四章。匪曰无
故呻吟,亦长歌当哭之意耳。

悲时序之递嬗兮,又属清秋。感遭家之不造兮,独处离愁。北堂有萱兮,何以
忘忧?无以解忧兮,我心咻咻。

云凭凭兮秋风酸,步中庭兮霜叶干。何去何从兮失我故欢,静言思之兮恻肺肝。

惟鲔有潭兮,惟鹤有梁。鳞甲潜伏兮,羽毛何长!搔首问兮茫茫,高天厚地兮,
谁知余之永伤?

银河耿耿兮寒气侵,月色横斜兮玉漏沉。忧心炳炳兮发我哀吟。吟复吟兮寄我
知音。
黛玉看了,不胜伤感。又想:“宝姐姐不寄与别人,单寄与我,也是‘惺惺惜惺惺’
的意思。”正在沉吟,只听见外面有人说道:“林姐姐在家里呢么?”黛玉一面把
宝钗的书叠起,口内便答应道:“是谁?”正问着,早见几个人进来,却是探春、
湘云、李纹、李绮。彼此问了好,雪雁倒上茶来,大家喝了,说些闲话。因想起前
年的“菊花诗”来,黛玉便道:“宝姐姐自从挪出去,来了两遭,如今索性有事也
不来了,真真奇怪。我看他终久还来我们这里不来!”探春微笑道:“怎么不来?
横竖要来的。如今是他们尊嫂有些脾气,姨妈上了年纪的人,又兼有薛大哥的事,
自然得宝姐姐照料一切。那里还比得先前有工夫呢?”

正说着,忽听得唿喇喇一片风声,吹了好些落叶打在窗纸上。停了一回儿,又
透过一阵清香来。众人闻着,都说道:“这是何处来的香风?这像什么香?”黛玉
道:“好像木樨香。”探春笑道:“林姐姐终不脱南边人的话。这大九月里的,那
里还有桂花呢?”黛玉笑道:“原是啊!不然,怎么不竟说‘是’桂花香,只说似
乎‘像’呢?”湘云道:“三姐姐,你也别说。你可记得‘十里荷花,三秋桂子’?
在南边正是晚桂开的时候了,你只没有见过罢了。等你明日到南边去的时候,你自
然也就知道了。”探春笑道:“我有什么事到南边去?况且这个也是我早知道的,
不用你们说嘴。”李纹、李绮只抿着嘴儿笑。黛玉道:“妹妹,这可说不齐。俗语
说:‘人是地行仙。’今日在这里,明日就不知在那里。譬如我原是南边人,怎么
到了这里呢?”湘云拍着手笑道:“今儿三姐姐可叫林姐姐问住了。不但林姐姐是
南边人到这里,就是我们这几个人就不同:也有本来是北边的;也有根子是南边,
生长在北边的;也有生长在南边,到这北边的。今儿大家都凑在一处,可见人总有
一个定数。大凡地和人,总是各自有缘分的。”众人听了都点头,探春也只是笑。
又说了一会子闲话儿,大家散出。黛玉送至门口,大家都说:“你身上才好些,别
出来了,看着了风。”

于是黛玉一面说着话儿,一面站在门口,又与四人殷勤了几句,便看着他们出
院去了。进来坐着,看看已是林鸟归山,夕阳西坠。因史湘云说起南边的话,便想
着:“父母若在,南边的景致,春花秋月,水秀山明,二十四桥,六朝遗迹。不少
下人伏侍,诸事可以任意,言语亦可不避。香车画舫,红杏青帘,惟我独尊。今日
寄人篱下,纵有许多照应,自己无处不要留心。不知前生作了什么罪孽,今生这样
孤凄!真是李后主说的,‘此间日中只以眼泪洗面’矣!”一面思想,不知不觉神
往那里去了。

紫鹃走来,看见这样光景,想着必是因刚才说起南边北边的话来,一时触着黛
玉的心事了。便问道:“姑娘们来说了半天话,想来姑娘又劳了神了。刚才我叫雪
雁告诉厨房里,给姑娘作了一碗火肉白菜汤,加了一点儿虾米儿,配了点青笋紫菜,
姑娘想着好么?”黛玉道:“也罢了。”紫鹃道:“还熬了一点江米粥。”黛玉点
点头儿,又说道:“那粥得你们两个自己熬了,不用他们厨房里熬才是。”紫鹃道:
“我也怕厨房里弄的不干净,我们自己熬呢。就是那汤,我也告诉雪雁合柳嫂儿说
了,要弄干净着。柳嫂儿说了:他打点妥当,拿到他屋里,叫他们五儿瞅着炖呢。”
黛玉道:“我倒不是嫌人家腌。只是病了好些日子,不周不备,都是人家,这会
子又汤儿粥儿的调度,未免惹人厌烦。”说着,眼圈儿又红了。紫鹃道:“姑娘这
话也是多想。姑娘是老太太的外孙女儿,又是老太太心坎儿上的。别人求其在姑娘
跟前讨好儿还不能呢,那里有抱怨的?”黛玉点点头儿。因又问道:“你才说的五
儿,不是那日合宝二爷那边的芳官在一处的那个女孩儿?”紫鹃道:“就是他。”
黛玉道:“不听见说要进来么?”紫鹃道:“可不是,因为病了一场。后来好了,
才要进来,正是晴雯他们闹出事来的时候,也就耽搁住了。”黛玉道:“我看那丫
头倒也还头脸儿干净。”说着,外头婆子送了汤来。雪雁出来接时,那婆子说道:
“柳嫂儿叫回姑娘:这是他们五儿作的,没敢在大厨房里作,怕姑娘嫌腌。”雪
雁答应着,接了进来。黛玉在屋里,已听见了,吩咐雪雁:“告诉那老婆子回去说,
叫他费心。”雪雁出来说了,老婆子自去。这里雪雁将黛玉的碗箸安放在小几儿上,
因问黛玉道:“还有咱们南来的五香大头菜,拌些麻油、醋,可好么?”黛玉道:
“也使得,只不必累坠了。”一面盛上粥来。黛玉吃了半碗,用羹匙舀了两口汤喝,
就搁下了。两个丫鬟撤下来了,拭净了小几,端下去,又换上一张常放的小几。黛
玉漱了口,盥了手,便道:“紫鹃,添了香了没有?”紫鹃道:“就添去。”黛玉
道:“你们就把那汤合粥吃了罢,味儿还好,且是干净。待我自己添香罢。”两个
人答应了,在外间自吃去了。

这里黛玉添了香,自己坐着,才要拿本书看,只听得园内的风自西边直透到东
边,穿过树枝,都在那里唏哗喇不住的响。一会儿,檐下的铁马也只管叮叮当当
的乱敲起来。一时雪雁先吃完了,进来伺候。黛玉便问道:“天气冷了,我前日叫
你们把那些小毛儿衣裳晾晾,可曾晾过没有?”雪雁道:“都晾过了。”黛玉道:
“你拿一件来我披披。”雪雁走去,将一包小毛衣裳抱来,打开毡包,给黛玉自拣。
只见内中夹着个绢包儿。黛玉伸手拿起,打开看时,却是宝玉病时送来的旧绢子,
自己题的诗,上面泪痕犹在。里头却包着那剪破了的香囊、扇袋并宝玉通灵玉上的
穗子。原来晾衣裳时从箱中检出,紫鹃恐怕遗失了,遂夹在这毡包里的。这黛玉不
看则已,看了时,也不说穿那一件衣裳,手里只拿着那两方手帕,呆呆的看那旧诗。
看了一回,不觉得簌簌泪下。

紫鹃刚从外间进来,只见雪雁正捧着一毡包衣裳,在傍边呆立,小几上却搁着
剪破了的香囊和两三截儿扇袋并那铰拆了的穗子。黛玉手中却拿着两方旧帕子,上
边写着字迹,在那里对着滴泪呢。正是:
失意人逢失意事,新啼痕间旧啼痕。
紫鹃见了这样,知是他触物伤情,感怀旧事,料道劝也无益,只得笑着道:“姑娘,
还看那些东西作什么?那都是那几年宝二爷和姑娘小时,一时好了,一时恼了,闹
出来的笑话儿。要像如今这样厮抬厮敬的,那里能把这些东西白遭塌了呢。”紫鹃
这话原给黛玉开心,不料这几句话更提起黛玉初来时和宝玉的旧事来,一发珠泪连
绵起来。紫鹃又劝道:“雪雁这里等着呢,姑娘披上一件罢。”那黛玉才把手帕撂
下。紫鹃连忙拾起,将香袋等物包起拿开。这黛玉方披了一件皮衣,自己闷闷的走
到外间来坐下。回头看见案上宝钗的诗启尚未收好,又拿出来瞧了两遍,叹道:“境
遇不同,伤心则一。不免也赋四章,翻入琴谱,可弹可歌,明日写出来寄去,以当
和作。”便叫雪雁将外边桌上笔砚拿来,濡墨挥毫,赋成四叠。又将琴谱翻出,借
他《猗兰》《思贤》两操,合成音韵,与自己做的配齐了,然后写出,以备送与宝
钗。又即叫雪雁向箱中将自己带来的短琴拿出,调上弦,又操演了指法。黛玉本是
个绝顶聪明人,又在南边学过几时,虽是手生,到底一理就熟。抚了一番,夜已深
了,便叫紫鹃收拾睡觉,不提。

却说宝玉这日起来,梳洗了,带着焙茗正往书房中来,只见墨雨笑嘻嘻的跑来,
迎头说道:“二爷今日便宜了。太爷不在书房里,都放了学了。”宝玉道:“当真
的么?”墨雨道:“二爷不信,那不是三爷和兰哥来了?”宝玉看时,只见贾环贾
兰跟着小厮们,两个笑嘻嘻的,嘴里咭咭呱呱不知说些什么,迎头来了。见了宝玉,
都垂手站住。宝玉问道:“你们两个怎么就回来了?”贾环道:“今日太爷有事,
说是放一天学,明儿再去呢。”宝玉听了,方回身到贾母贾政处去禀明了,然后回
到怡红院中。袭人问道:“怎么又回来了?”宝玉告诉了他。只坐了一坐儿,便往
外走,袭人道:“往那里去,这样忙法?就放了学,依我说,也该养养神儿了。”
宝玉站住脚,低了头,说道:“你的话也是,但是好容易放一天学,还不散散去。
你也该可怜我些儿了。”袭人见说的可怜,笑道:“由爷去罢。”正说着,端了饭
来,宝玉也没法儿,只得且吃饭。三口两口忙忙的吃完,漱了口,一溜烟往黛玉房
中去了。

走到门口,只见雪雁在院中晾绢子呢。宝玉因问:“姑娘吃了饭了么?”雪雁
道:“早起喝了半碗粥,懒怠吃饭,这时候打盹儿呢。二爷且到别处走走,回来再
来罢。”宝玉只得回来。无处可去,忽然想起惜春有好几天没见,便信步走到蓼风
轩来。刚到窗下,只见静悄悄一无人声,宝玉打量他也睡午觉,不便进去。才要走
时,只听屋里微微一响,不知何声;宝玉站住再听,半日,又“拍”的一响。宝玉
还未听出,只见一个人道:“你在这里下了一个子儿,那里你不应么?”宝玉方知
是下棋呢。但只急切听不出这个人的语音是谁。底下方听见惜春道:“怕什么?你
这么一吃我,我这么一应;你又这么吃,我又这么应:还缓着一着儿呢,终久连的
上。”那一个又道:“我要这么一吃呢?”惜春道:“啊嗄,还有一着反扑在里头
呢,我倒没防备。”宝玉听了听那一个声音很熟,却不是他们姊妹,料着惜春屋里
也没外人,轻轻的掀帘进去。看时不是别人,却是那栊翠庵的槛外人妙玉。这宝玉
见是妙玉,不敢惊动。妙玉和惜春正在凝思之际,也没理会。宝玉却站在旁边,看
他两个的手段。只见妙玉低着头,问惜春道:“你这个畸角儿不要了么?”惜春道:
“怎么不要?你那里头都是死子儿,我怕什么?”妙玉道:“且别说满话,试试看。”
惜春道:“我便打了起来,看你怎么着。”妙玉却微微笑着,把边上子一接,却搭
转一吃,把惜春的一个角儿都打起来了,笑着说道:“这叫做‘倒脱靴势’。”

惜春尚未答言,宝玉在旁情不自禁,哈哈一笑,把两个人都唬了一大跳。惜春
道:“你这是怎么说?进来也不言语。这么使促狭唬人!你多早晚进来的?”宝玉道:
“我头里就进来了,看着你们两个争这个畸角儿。”说着,一面与妙玉施礼,一面
又笑问道:“妙公轻易不出禅关,今日何缘下凡一走?”妙玉听了,忽然把脸一红,
也不答言,低了头自看那棋。宝玉自觉造次,连忙陪笑道:“倒是出家人比不得我
们在家的俗人。头一件,心是静的。静则灵,灵则慧。”宝玉尚未说完,只见妙玉
微微的把眼一抬,看了宝玉一眼,复又低下头去,那脸上的颜色渐渐的红晕起来。
宝玉见他不理,只得讪讪的旁边坐了。

惜春还要下子,妙玉半日说道:“再下罢。”便起身理理衣裳,重新坐下,痴
痴的问着宝玉道:“你从何处来?”宝玉巴不得这一声,好解释前头的话,忽又想
道:“或是妙玉的机锋?”转红了脸,答应不出来。妙玉微微一笑,自合惜春说话。
惜春也笑道:“二哥哥,这什么难答的?你没有听见人家常说的,‘从来处来’么?
这也值得把脸红了,见了生人的似的。”妙玉听了这话,想起自家,心上一动,脸
上一热,必然也是红的,倒觉不好意思起来。因站起来说道:“我来得久了,要回
庵里去了。”惜春知妙玉为人,也不深留,送出门口。妙玉笑道:“久已不来,这
里弯弯曲曲的,回去的路头都要迷住了。”宝玉道:“这倒要我来指引指引,何如?”
妙玉道:“不敢,二爷前请。”

于是二人别了惜春,离了蓼风轩,弯弯曲曲,走近潇湘馆,忽听得叮咚之声。
妙玉道:“那里的琴声?”宝玉道:“想必是林妹妹那里抚琴呢。”妙玉道:“原
来他也会这个吗?怎么素日不听见提起?”宝玉悉把黛玉的事说了一遍,因说:“咱
们去看他。”妙玉道:“从古只有听琴,再没有看琴的。”宝玉笑道:“我原说我
是个俗人。”说着,二人走至潇湘馆外,在山子石上坐着静听,甚觉音调清切。只
听得低吟道:
风萧萧兮秋气深,美人千里兮独沉吟。
望故乡兮何处?倚栏杆兮涕沾襟。
歇了一回,听得又吟道:
山迢迢兮水长,照轩窗兮明月光。
耿耿不寐兮银河渺茫,罗衫怯怯兮风露凉。
又歇了一歇。妙玉道:“刚才‘侵’字韵是第一叠,如今‘阳’字韵是第二叠了。
咱们再听。”里边又吟道:
子之遭兮不自由,予之遇兮多烦忧。
之子与我兮心焉相投,思古人兮俾无尤。
妙玉道:“这又是一拍。何忧思之深也!”宝玉道:“我虽不懂得,但听他声音,
也觉得过悲了。”里头又调了一回弦。妙玉道:“君弦太高了,与无射律只怕不配
呢。”里边又吟道:
人生斯世兮如轻尘,天上人间兮感夙因。
感夙因兮不可,素心如何天上月!
妙玉听了,呀然失色道:“如何忽作变徵之声?音韵可裂金石矣!只是太过。”宝玉
道:“太过便怎么?”妙玉道:“恐不能持久。”正议论时,听得君弦“蹦”的一
声断了。妙玉站起来,连忙就走。宝玉道:“怎么样?”妙玉道:“日后自知,你
也不必多说。”竟自走了。弄得宝玉满肚疑团,没精打采的,归至怡红院中,不表。

且说妙玉归去,早有道婆接着,掩了庵门,坐了一回,把《禅门日诵》念了一
遍。吃了晚饭,点上香,拜了菩萨,命道婆子自去歇着。自己的禅床靠背俱已整齐,
屏息垂帘,跏趺坐下,断除妄想,趁向真如。坐到三更以后,听得房上一片
响声,妙玉恐有贼来,下了禅床,出到前轩,但见云影横空,月华如水。那时天气
尚不很凉,独自一个凭栏站了一回,忽听房上两个猫儿一递一声厮叫。那妙玉忽想
起日间宝玉之言,不觉一阵心跳耳热,自己连忙收摄心神,走进禅房,仍到禅床上
坐了。怎奈神不守舍,一时如万马奔驰,觉得禅床便恍荡起来,身子已不在庵中。
便有许多王孙公子,要来娶他;又有些媒婆扯扯拽拽扶他上车,自己不肯去。一回
儿,又有盗贼劫他,持刀执棍的逼勒,只得哭喊求救。

早惊醒了庵中女尼道婆等众,都拿火来照看,只见妙玉两手撒开,口中流沫。
急叫醒时,只见眼睛直竖,两颧鲜红,骂道:“我是有菩萨保佑,你们这些强徒敢
要怎么样?”众人都唬的没了主意,都说道:“我们在这里呢,快醒转来罢!”妙
玉道:“我要回家去!你们有什么好人,送我回去罢。”道婆道:“这里就是你住
的房子。”说着,又叫别的女尼忙向观音前祷告。求了签,翻开签书看时,是触犯
了西南角上的阴人。就有一个说:“是了,大观园中西南角上本来没有人住,阴气
是有的。”一面弄汤弄水的在那里忙乱。那女尼原是自南边带来的,伏侍妙玉自然
比别人尽心,围着妙玉坐在禅床上。妙玉回头道:“你是谁?”女尼道:“是我。”
妙玉仔细瞧了一瞧道:“原来是你!”便抱住那女尼,呜呜咽咽的哭起来,说道:
“你是我的妈呀,你不救我,我不得活了!”那女尼一面唤醒他,一面给他揉着。
道婆倒上茶来喝了,直到天明才睡了。

女尼便打发人去请大夫来看脉。也有说是思虑伤脾的,也有说是热入血室的,
也有说是邪祟触犯的,也有说是内外感冒的:终无定论。后请得一个大夫来看了,
问:“曾打坐过没有?”道婆说道:“向来打坐的。”大夫道:“这病可是昨夜忽
然来的么?”道婆道:“是。”大夫道:“这是走魔入火的原故。”众人问:“有
碍没有?”大夫道:“幸亏打坐不久,魔还入得浅,可以有救。”写了降伏心火的
药,吃了一剂,稍稍平复些。外面那些游头浪子听见了,便造作许多谣言,说:“这
么年纪,那里忍得住?况且又是很风流的人品,很乖觉的性灵!以后不知飞在谁手
里,便宜谁去呢。”过了几日,妙玉病虽略好了些,神思未复,终有些恍惚。

一日,惜春正坐着,彩屏忽然进来,回道:“姑娘知道妙玉师父的事吗?”惜
春道:“他有什么事?”彩屏道:“我昨日听见邢姑娘和大奶奶在那里说呢:他自
从那日合姑娘下棋回去,夜间忽然中了邪,嘴里乱嚷,说强盗来抢他来了。到如今
还没好呢。姑娘,你说这不是奇事吗?”惜春听了,默默无语。因想:“妙玉虽然
洁净,毕竟尘缘未断。可惜我生在这种人家,不便出家,我若出了家时,那有邪魔
缠扰?一念不生,万缘俱寂。”想到这里,蓦与神会,若有所得,便口占一偈云:
大造本无方,云何是应住?
既从空中来,应向空中去。

占毕,即命丫头焚香。自己静坐了一回,又翻开那棋谱来,把孔融、王积薪等
所著看了几篇。内中“茂叶包蟹势”、“黄莺搏兔势”,都不出奇;“三十六局杀
角势”,一时也难会难记;独看到“十龙走马”,觉得甚有意思。正在那里作想,
只听见外面一个人走进院来,连叫彩屏。

未知是谁,下回分解。