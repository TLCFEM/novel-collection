\chapter{老学士闲征姽婳词~痴公子杜撰芙蓉}

话说两个尼姑领了芳官等去后,王夫人便往贾母处来。见贾母喜欢,便趁便回
道:“宝玉屋里有个晴雯,那个丫头也大了,而且一年之间病不离身。我常见他比
别人分外淘气,也懒;前日又病倒了十几天,叫大夫瞧,说是女儿痨,所以我就赶
着叫他下去了。若养好了,也不用叫他进来,就赏他家配人去也罢了。再那几个学
戏的女孩子,我也做主放了:一则他们都会戏,口里没轻没重,只会混说,女孩儿
们听了,如何使得?二则他们唱会子戏,白放了他们,也是应该的。况丫头们也太
多,若说不够使,再挑上几个来,也是一样。”贾母听了点头道:“这是正理,我
也正想着如此。但晴雯这丫头,我看他甚好,言谈针线都不及他,将来还可以给宝
玉使唤的,谁知变了。”王夫人笑道:“老太太挑中的人原不错,只是他命里没造
化,所以得了这个病。俗语又说:‘女大十八变。’况且有本事的人,未免就有些
调歪,老太太还有什么不曾经历过的?三年前我也就留心这件事,先只取中了他。
我留心看了去,他色色比人强,只是不大沉重。知大体,莫若袭人第一。虽说贤妻
美妾,也要性情和顺,举止重的更好些。袭人的模样虽比晴雯次一等,然放在房
里也算是一二等的。况且行事大方,心地老实,这几年从未同着宝玉淘气。凡宝玉
十分胡闹的事,他只有死劝的。因此,品择了二年,一点不错了,我悄悄的把他丫
头的月钱止住,我的月分银子里批出二两银子来给他,不过使他自己知道,越发小
心效好之意。且没有明说,一则宝玉年纪尚小,老爷知道了,又恐就耽误了书;二
则宝玉自以为自己跟前的人,不敢劝他说他,反倒纵性起来。所以直到今日,才回
明老太太。”贾母听了,笑道:“原来这样,如此更好了。袭人本来从小儿不言不
语,我只说是‘没嘴的葫芦’。既是你深知,岂有大错误的?”王夫人又回今日贾
政如何夸奖,如何带他们逛去。贾母听了,更加喜悦。

一时,只见迎春妆扮了前来告辞过去。凤姐也来请早安,伺候早饭。又说笑一
回,贾母歇晌,王夫人便唤了凤姐,问他丸药可曾配来。凤姐道:“还不曾呢,如
今还是吃汤药。太太只管放心,我已大好了。”王夫人见他精神复初,也就信了,
因告诉撵逐晴雯等事。又说:“宝丫头怎么私自回家去了?你们都不知道?我前儿顺
路都查了一查。谁知兰小子的这一个新进来的奶子,也十分的妖调,也不喜欢他。
我说给你大嫂子了:好不好,叫他各自去罢。我因问你大嫂子:‘宝丫头出去,难
道你们不知道吗?’他说是告诉了他了,不两三日,等姨妈病好了就进来。姨妈究
竟没什么大病,不过咳嗽腰疼,年年是如此的。他这去的必有原故,不是有人得罪
了他了?那孩子心重,亲戚们住一场,别得罪了人,反不好了。”凤姐笑道:“谁
可好好的得罪着他?”王夫人道:“别是宝玉有嘴无心,从来没个忌讳,高了兴信
嘴胡说也是有的。”凤姐笑道:“这可是太太过于操心了。若说他出去干正经事,
说正经话去,却像傻子;若只叫他进来,在这些姊妹跟前,以至于大小的丫头跟前,
最有尽让,又恐怕得罪了人,那是再不得有人恼他的。我想薛妹妹此去必是为前夜
搜检众丫头的原故,他自然为信不及园里的人,他又是亲戚,现也有丫头老婆在内,
我们又不好去搜检。他恐我们疑他,所以多了这个心,自己回避了。也是应该避嫌
疑的。”王夫人听了这话不错,自己遂低头一想,便命人去请了宝钗来,分晰前日
的事,以解他的疑心,又仍命他进来照旧居住。宝钗陪笑道:“我原要早出去的,
因姨妈有许多大事,所以不便来说。可巧前日妈妈又不好了,家里两个靠得的女人
又病,所以我趁便去了。姨妈今日既已知道了,我正好回明,就从今日辞了,好搬
东西。”王夫人凤姐都笑道:“你太固执了。正经再搬进来为是,休为没要紧的事
反疏远了亲戚。”宝钗笑道:“这话说的太重了,并没为什么事要出去。我为的是
妈妈近来神思比先大减,而且夜晚没有得靠的人,统共只我一个人;二则如今我哥
哥眼看娶嫂子,多少针线活计,并家里一切动用器皿,尚有未齐备的,我也须得帮
着妈妈去料理料理。姨妈和凤姐姐都知道我们家的事,不是我撒谎。再者,自我在
园里,东南上小角门子就常开着,原是为我走的,保不住出入的人图省走路,也从
那里走。又没个人盘查,设若从那里弄出事来,岂不两碍?而且我进园里来睡,原
不是什么大事。因前几年年纪都小,且家里没事,在外头不如进来,姊妹们在一处
玩笑作针线,都比在外头一人闷坐好些。如今彼此都大了,况姨娘这边历年皆遇不
遂心之事,所以那园子里,倘有一时照顾不到的,皆有关系。惟有少几个人,就可
以少操些心了。所以今日不但我决意辞去,此外还要劝姨娘:如今该减省的就减省
些,也不为失了大家的体统。据我看,园里的这一项费用也竟可以免的,说不得当
日的话。姨娘深知我家的,难道我家当日也是这样零落不成?”凤姐听了这篇话,
便向王夫人笑道:“这话依我竟不必强他。”王夫人点头道:“我也无可回答,只
好随你的便罢了。”

说话之间,只见宝玉已回来了,因说:“老爷还未散,恐天黑了,所以先叫我
们回来了。”王夫人忙问:“今日可丢了丑了没有?”宝玉笑道:“不但不丢丑,
拐了许多东西来。”接着就有老婆子们从二门上小厮手内接进东西来。王夫人一看
时,只见扇子三把,扇坠三个,笔墨共六匣,香珠三串,玉绦环三个。宝玉说道:
“这是梅翰林送的,那是杨侍郎送的,这是李员外送的:每人一分。”说着,又向
怀中取出一个檀香小护身佛来,说:“这是庆国公单给我的。”王夫人又问在席何
人,做何诗词。说毕,只将宝玉一分令人拿着,同宝玉、环、兰前来见贾母。贾母
看了,喜欢不尽,不免又问些话。无奈宝玉一心记着晴雯,答应完了,便说:“骑
马颠了,骨头疼。”贾母便说:“快回房去,换了衣服,疏散疏散就好了,不许睡。”
宝玉听了,便忙进园来。

当下麝月秋纹已带了两个丫头来等候。见宝玉辞了贾母出来,秋纹便将墨笔等
物拿着,随宝玉进园来。宝玉满口里说:“好热。”一壁走一面便摘冠解带,将外
面的大衣服都脱下来麝月拿着,只穿着一件松花绫子夹袄,襟内露出血点般大红裤
子来。秋纹见这条红裤是晴雯针线,因叹道:“真是‘物在人亡’了!”麝月将秋
纹拉了一把,笑道:“这裤子配着松花色袄儿、石青靴子,越显出靛青的头,雪白
的脸来了。”宝玉在前,只装没听见,又走了两步便止步道:“我要走一走,这怎
么好?”麝月道:“大白日里还怕什么,还怕丢了你不成?”因命两个小丫头跟着,
“我们送了这些东西去再来。”宝玉道:“好姐姐,等一等我再去。”麝月道:“我
们去了就来。两个人手里都有东西,倒像摆执事的,一个捧着文房四宝,一个捧着
冠袍带履,成个什么样子。”

宝玉听了,正中心怀,便让他二人去了。他便带了两个小丫头到一块山子石后
头,悄问他二人道:“自我去了,你袭人姐姐打发人去瞧晴雯姐姐没有?”这一个
答道:“打发宋妈瞧去了。”宝玉道:“回来说什么?”小丫头道:“回来说:晴
雯姐姐直着脖子叫了一夜,今日早起,就闭了眼住了口,世事不知,只有倒气的分
儿了。”宝玉忙道:“一夜叫的是谁?”小丫头道:“一夜叫的是娘。”宝玉拭泪
道:“还叫谁?”小丫头说:“没有听见叫别人了。”宝玉道:“你糊涂。想必没
有听真。”旁边那一个小丫头最伶俐,听宝玉如此说,便上来说:“真个他糊涂!”
又向宝玉说:“不但我听的真切,我还亲自偷着看去来着。”宝玉听说,忙问:“怎
么又亲自看去?”小丫头道:“我想,晴雯姐姐素日和别人不同,待我们极好。如
今他虽受了委屈出去,我们不能别的法子救他,只亲去瞧瞧,也不枉素日疼我们一
场。就是人知道了,回了太太,打我们一顿,也是愿受的。所以我拚着一顿打,偷
着出去瞧了一瞧。谁知他平生为人聪明,至死不变,见我去了,便睁开眼拉我的手
问:‘宝玉那里去了?’我告诉他了。他叹了一口气,说:‘不能见了!’我就说:
‘姐姐何不等一等他回来见一面?’他就笑道:‘你们不知道,我不是死:如今天
上少了一个花神,玉皇爷叫我去管花儿。我如今在未正二刻就上任去了,宝玉须得
未正三刻才到家,只少一刻儿的工夫,不能见面。世上凡有该死的人,阎王勾取了
去,是差些个小鬼来拿他的魂儿。要迟延一时半刻,不过烧些纸浇些浆饭,那鬼只
顾抢钱去了,该死的人就可挨磨些工夫。我这如今是天上的神仙来请,那里捱得时
刻呢?’我听了这话,竟不大信。及进来到屋里,留神看时辰表,果然是未正二刻,
他咽了气;正三刻上,就有人来叫我们说你来了。”宝玉忙道:“你不认得字,所
以不知道,这原是有的。不但花有一花神,还有总花神。但他不知做总花神去了,
还是单管一样花神?”这丫头听了,一时诌不来。恰好这是八月时节,园中池上芙
蓉正开,这丫头便见景生情,忙答道:“我已曾问他:‘是管什么花的神?告诉我
们,日后也好供养的。’他说:‘你只可告诉宝玉一人,除他之外,不可泄了天机。’
就告诉我说,他就是专管芙蓉花的。”

宝玉听了这话,不但不为怪,亦且去悲生喜,便回过头来,看着那芙蓉笑道:
“此花也须得这样一个人去主管。我就料定他那样的人必有一番事业!虽然超生苦
海,从此再不能相见了。”免不得伤感思念;因又想:“虽然临终未见,如今且去
灵前一拜,也算尽这五六年的情意。”想毕,忙至屋里,正值麝月秋纹找来。宝玉
又自穿戴了,只说去看黛玉,遂一人出园,往前次看望之处来。意为停柩在内,谁
知他哥嫂见他一咽气,便回了进去,希图早早些得几两发送例银。王夫人闻知,便
命赏了十两银子,又命:“即刻送到外头焚化了罢。女子痨死的,断不可留!”他
哥嫂听了这话,一面得银,一面催人立刻入殓,抬往城外化人厂上去了。剩的衣裳
簪环,约有三四百金之数,他哥嫂自收了,为后日之计。二人将门锁上,一同送殡
去了。

宝玉走来扑了一个空,站了半天,并无别法,只得复身进入园中。及回至房中,
甚觉无味,因顺路来找黛玉,不在房里。问其何往,丫鬟们回说:“往宝姑娘那里
去了。”宝玉又至蘅芜院中,只见寂静无人,房内搬出,空空落落,不觉吃一大惊,
才想起前日仿佛听见宝钗要搬出去,只因这两日工课忙就混忘了,这时看见如此,
才知道果然搬出。怔了半天,因转念一想:“不如还是和袭人厮混,再与黛玉相伴。
只这两三个人,只怕还是同死同归。”想毕,仍往潇湘馆来。偏黛玉还未回来。正
在不知所之,忽见王夫人的丫头进来找他,说:“老爷回来了,找你呢。又得了好
题目了。快走,快走。”宝玉听了,只得跟了出来。到王夫人屋里,他父亲已出去
了,王夫人命人送宝玉至书房里。

彼时贾政正与众幕友们谈论寻书之胜。又说:“临散时,忽谈及一事,最是千
古佳谈,‘风流隽逸,忠义感慨’,八字皆备。倒是个好题目,大家要做一首挽词。”
众幕宾听了,都请教:“系何等妙事?”贾政乃道:“当日曾有一位王爵,封曰恒
王,出镇青州。这恒王最喜女色,且公馀好武,因选了许多美女,日习武事,令众
美女学习战攻斗伐之事。内中有个姓林行四的,姿色既佳,且武艺更精,皆呼为林
四娘。恒王最得意,遂超拔林四娘统辖诸姬,又呼为姽婳将军。”众清客都称:“妙
极神奇。竟以‘姽婳’下加‘将军’二字,反更觉妩媚风流,真绝世奇文也。想这
恒王也是千古第一风流人物了。”贾政笑道:“这话自然如此。但更有可奇可叹之
事。”众清客都惊问道:“不知底下有何等奇事?”贾政道:“谁知次年,便有‘黄
巾’‘赤眉’一干流贼馀党复又乌合,抢掠山左一带。恒王意为犬羊之辈,不足大
举,因轻骑进剿。不意贼众诡谲,两战不胜,恒王遂被众贼所戮。于是青州城内文
武官员,各各皆谓:‘王尚不胜,你我何为?’遂将有献城之举。林四娘得闻凶信,
遂聚集众女将,发令说道:‘你我皆向蒙王恩,戴天履地,不能报其万一。今王既
殒身国患,我意亦当殒身于下。尔等有愿随着,即同我前往,不愿者亦早自散去。’
众女将听他这样,都一齐说:‘愿意!’于是林四娘带领众人,连夜出城,直杀至
贼营。里头众贼不防,也被斩杀了几个首贼。后来大家见是不过几个女人,料不能
济事,遂回戈倒兵,奋力一阵,把林四娘等一个不曾留下,倒作成了这林四娘的一
片忠心之志。后来报至都中,天子百官,无不叹息。想其朝中自然又有人去剿灭,
天兵一到,化为乌有,不必深论。只就林四娘一节,众位听了,可羡不可羡?”众
幕友都叹道:“实在可羡可奇!实是个妙题,原该大家挽一挽才是。”说着,早有
人取了笔砚,按贾政口中之言,稍加改易了几个字,便成了一篇短序,递给贾政看
了,贾政道:“不过如此。他们那里已有原序。昨日内又奉恩旨:着察核前代以来
应加褒奖而遗落未经奏请各项人等,无论僧、尼、乞丐、女妇人等,有一事可嘉,
即行汇送履历至礼部,备请恩奖。所以他这原序也送往礼部去了。大家听了这新闻,
所以都要做一首《姽婳词》,以志其忠义。”众人听了,都又笑道:“这原该如此。
只是更可羡者,本朝皆系千古未有之旷典,可谓‘圣朝无阙事’了。”贾政点头道:
“正是。”

说话间,宝玉、贾环、贾兰俱起身来看了题目。贾政命他三人各吊一首,谁先
做成者赏,佳者额外加赏。贾环贾兰二人近日当着许多人皆做过几首了,胆量愈壮。
今看了题目,遂自去思索。一时贾兰先有了,贾环生恐落后,也就有了。二人皆已
录出,宝玉尚自出神。

贾政与众人且看他二人的二首。贾兰的是一首七言绝句,写道是:
姽婳将军林四娘,玉为肌骨铁为肠。
捐躯自报恒王后,此日青州土尚香。

众幕宾看了,便皆大赞:“小哥儿十三岁的人就如此,可知家学渊深真不诬矣。”
贾政笑道:“稚子口角,也还难为他。”又看贾环的,是首五言律,写道是:
红粉不知愁,将军意未休。
掩啼离绣幕,抱恨出青州。
自谓酬王德,谁能复寇仇?
好题忠义墓,千古独风流。
众人道:“更佳。到底大几岁年纪,立意又自不同。”贾政道:“倒还不甚大错,
终不恳切。”众人道:“这就罢了。三爷才大不多几岁,俱在未冠之时。如此用心
做去,再过几年,怕不是大阮小阮了么?”贾政笑道:“过奖了。只是不肯读书的
过失。”

因问宝玉。众人道:“二爷细心镂刻,定又是风流悲感,不同此等的了。”宝
玉笑道:“这个题目似不称近体,须得古体或歌或行长篇一首,方能恳切。”众人
听了,都站起身来,点头拍手道:“我说他立意不同!每一题到手,必先度其体格
宜与不宜,这便是老手妙法。这题目名曰《姽婳词》,且既有了序,此必是长篇歌
行,方合体式。或拟温八叉《击瓯歌》,或拟李长吉《会稽歌》,或拟白乐天《长
恨歌》,或拟咏古词,半叙半咏,流利飘逸,始能尽妙。”贾政听说,也合了主意,
遂自提笔向纸上要写。又向宝玉笑道:“如此甚好。你念,我写。若不好了,我捶
你的肉,谁许你先大言不惭的!”宝玉只得念了一句道:
恒王好武兼好色,
贾政写了看时,摇头道:“粗鄙!”一幕友道:“要这样方古,究竟不粗。且看他
底下的。”贾政道:“姑存之。”宝玉又道:
遂教美女习骑射。歌艳舞不成欢,列阵挽戈为自得。
贾政写出,众人都道:“只这第三句便古朴老健,极妙。这第四句平叙,也最得体。”
贾政道:“休谬加奖誉,且看转的如何。”宝玉念道:
眼前不见尘沙起,将军俏影红灯里。
众人听了这两句,便都叫妙:“好个‘不见尘沙起’!又承了一句‘俏影红灯里’,
用字用句皆入神化了。”宝玉道:
叱咤时闻口舌香,霜矛雪剑娇难举。
众人听了更拍手笑道:“越发画出来了。当日敢是宝公也在坐,见其娇而且闻其香?
不然何体贴至此。”宝玉笑道:“闺阁习武,任其勇悍,怎似男人?不问而可知娇
怯之形了。”贾政道:“还不快续,这又有你说嘴的了?”宝玉只得又想了一想,
念道:
丁香结子芙蓉绦,
众人都道:“转‘萧’韵更妙,这才流利飘逸。而且这句子也绮靡秀媚得妙。”贾
政写了,道:“这一句不好,已有过了‘口舌香’、‘娇难举’,何必又如此?这
是力量不加,故又弄出这些堆砌货来搪塞。”宝玉笑道:“长歌也须得要些词藻点
缀点缀,不然便觉萧索。”贾政道:“你只顾说那些,这一句底下如何转至武事呢?
若再多说两句,岂不蛇足了?”宝玉道:“如此,底下一句兜转煞住,想也使得。”
贾政冷笑道:“你有多大本领!上头说了一句大开门的散话,如今又要一句连转带
煞,岂不心有馀而力不足呢。”宝玉听了,垂头想了一想,说了一句道:
不系明珠系宝刀。
忙问:“这一句可还使得?”众人拍案叫绝。贾政笑道:“且放着,再续。”宝玉
道:“使得,我便一气连下去了;若使不得,索性涂了,我再想别的意思出来,再
另措词。”贾政听了,便喝道:“多话!不好了再做。便做十篇百篇,还怕辛苦了
不成?”宝玉听了,只得想了一会,便念道:
战罢夜阑心力怯,脂痕粉渍污鲛绡。
贾政道:“这又是一段了。底下怎么样?”宝玉道:
明年流寇走山东,强吞虎豹势如蜂。
众人道:“好个‘走’字,便见得高低了。且通句转的也不板。”宝玉又念道:
王率天兵思剿灭,一战再战不成功。
腥风吹折陇中麦,日照旌旗虎帐空。
青山寂寂水澌澌,正是恒王战死时。
雨淋白骨血染草,月冷黄昏鬼守尸。
众人都道:“妙极,妙极!布置叙事词藻,无不尽美。且看如何至四娘,必另有妙
转奇句。”宝玉又念道:
纷纷将士只保身,青州眼见皆灰尘。
不期忠义明闺阁,愤起恒王得意人。
众人都道:“铺叙得委婉!”贾政道:“太多了,底下只怕累赘呢。”宝玉又道:
恒王得意数谁行:姽婳将军林四娘。
号令秦姬驱赵女,桃艳李临疆场。
绣鞍有泪春愁重,铁甲无声夜气凉。
胜负自难先预定,誓盟生死报前王。
贼势猖獗不可敌,柳折花残血凝碧。
马践胭脂骨髓香,魂依城郭家乡隔。
星驰时报入京师,谁家儿女不伤悲!
天子惊慌愁失守,此时文武皆垂首。
何事文武立朝纲,不及闺中林四娘?
我为四娘长叹息,歌成馀意尚彷徨!
念毕,众人都大赞不止。又从头看了一遍。贾政笑道:“虽说了几句,到底不大恳
切。”因说:“去罢。”三人如放了赦的一般,一齐出来,各自回房。众人皆无别
话,不过至晚安歇而已。

独有宝玉,一心凄楚。回至园中,猛见池上芙蓉,想起小丫鬟说晴雯做了芙蓉
之神,不觉又喜欢起来,乃看着芙蓉嗟叹了一会。忽又想起:“死后并未至灵前一
祭,如今何不在芙蓉前一祭,岂不尽了礼?”想毕,便欲行礼。忽又止道:“虽如
此,亦不可太草率了,须得衣冠整齐,奠仪周备,方为诚敬。”想了一想:“古人
云,‘潢污行潦,荇藻苹蘩之贱,可以羞王公,荐鬼神’,原不在物之贵贱,只在
心之诚敬而已。然非自作一篇诔文,这一段凄惨酸楚,竟无处可以发泄了。”因用
晴雯素日所喜之冰鲛縠一幅,楷字写成,名曰《芙蓉女儿诔》,前序后歌;又备了
晴雯素喜的四样吃食。于是黄昏人静之时,命那小丫头捧至芙蓉前,先行礼毕,将
那诔文即挂于芙蓉枝上,乃泣涕念曰:

维太平不易之元,蓉桂竞芳之月,无可奈何之日,怡红院浊玉谨以群花之蕊、
冰鲛之縠、沁芳之泉、枫露之茗:四者虽微,聊以达诚申信,乃致祭于白帝宫中抚
司秋艳芙蓉女儿之前曰:

窃思女儿自临人世,迄今凡十有六载。其先之乡籍姓氏,湮沦而莫能考者久矣。
而玉得于衾枕栉沐之间,栖息宴游之夕,亲昵狎亵,相与共处者,仅五年八月有奇。
忆女曩生之昔,其为质则金玉不足喻其贵,其为体则冰雪不足喻其洁。其为神则星
日不足喻其精,其为貌则花月不足喻其色。姊娣悉慕媖娴,妪媪咸仰慧德。孰料鸠
鸩恶其高,鹰鸷翻遭罦罬;薋箷妒其臭,茝兰竟被芟鉏。花原自怯,岂奈狂飚?柳本
多愁,何禁骤雨!偶遭蛊虿之谗,遂抱膏肓之疾。故樱唇红褪,韵吐呻吟;杏脸香枯,
色陈顑颔。诼谣謑诟,出自屏帷;荆棘蓬榛,蔓延窗户。既怀幽沉于不尽,复含罔屈
于无穷。高标见嫉,闺闱恨比长沙;贞烈遭危,巾帼惨于雁塞。自蓄辛酸,谁怜夭折?
仙云既散,芳趾难寻。洲迷聚窟,何来却死之香?海失灵槎,不获回生之药。眉黛烟
青,昨犹我画;指环玉冷,今倩谁温?鼎炉之剩药犹存,襟泪之馀痕尚渍。镜分鸾影,
愁开麝月之奁;梳化龙飞,哀折檀云之齿。委金钿于草莽,拾翠盒于尘埃。楼空鳷
鹊,从悬七夕之针;带断鸳鸯,谁续五丝之缕?况乃金天属节,白帝司时;孤衾有
梦,空室无人。桐阶月暗,芳魂与倩影同消;蓉帐香残,娇喘共细腰俱绝。连天衰
草,岂独蒹葭;匝地悲声,无非蟋蟀。露阶晚砌,穿帘不度寒砧;雨荔秋垣,隔院
希闻怨笛。芳名未泯,檐前鹦鹉犹呼;艳质将亡,槛外海棠预萎。捉迷屏后,莲瓣
无声;斗草庭前,兰芳枉待。抛残绣线,银笺彩袖谁裁?折断冰丝,金斗御香未熨。
昨承严命,既趋车而远陟芳园;今犯慈威,复拄杖而遣抛孤柩。及闻蕙棺被燹,顿
违共穴之情;石椁成灾,愧逮同灰之诮。尔乃西风古寺,淹滞青磷;落日荒丘,零
星白骨。楸榆飒飒,蓬艾萧萧。隔雾圹以啼猿,绕烟塍而泣鬼。岂道红绡帐里,公
子情深;始信黄土陇中,女儿命薄!汝南斑斑泪血,洒向西风;梓泽默默馀衷,诉
凭冷月。呜呼!固鬼蜮之为灾,岂神灵之有妒!毁詖奴之口,讨岂从宽?剖悍妇之心,
忿犹未释。在卿之尘缘虽浅,而玉之鄙意尤深。因蓄惓惓之思,不禁谆谆之问。始
知上帝垂旌,花宫待诏。生侪兰蕙,死辖芙蓉。听小婢之言,似涉无稽;据浊玉之思,
深为有据。何也:昔叶法善摄魂以撰碑,李长吉被诏而为记:事虽殊,其理则一也。
故相物以配才,苟非其人,恶乃滥乎?始信上帝委托权衡,可谓至洽至协,庶不负其
所秉赋也。因希其不昧之灵,或陟降于兹,特不揣鄙俗之词,有污慧听。乃歌而招
之曰:

天何如是之苍苍兮,乘玉虬以游乎穹窿耶?地何如是之茫茫兮,驾瑶象以降乎
泉壤耶?望伞盖之陆离兮,抑箕尾之光耶?列羽葆而为前导兮,卫危虚于傍耶?驱丰
隆以为庇从兮,望舒月以临耶?听车轨而伊轧兮:御鸾鹥以征耶?闻馥郁而飘然兮,
纫蘅杜以为佩耶?斓裙裾之烁烁兮,镂明月以为珰耶?借葳蕤而成坛畤兮,檠莲焰以
烛兰膏耶?文瓠匏以为觯斝兮,洒醽醁以浮桂醑耶?瞻云气而凝眸兮,仿佛有所觇耶?
俯波痕而属耳兮,恍惚有所闻耶?期汗漫而无际兮,捐弃予于尘埃耶?倩风廉之为余
驱车兮,冀联辔而携归耶?余中心为之慨然兮,徒噭噭而何为耶?卿偃然而长寝兮,
岂天运之变于斯耶?既窀穸且安稳兮,反其真而又奚化耶?余犹桎梏而悬附兮,灵格
余以嗟来耶?来兮止兮,卿其来耶?

若夫鸿蒙而居,寂静以处,虽临于兹,余亦莫睹。搴烟萝而为步障,列苍蒲而
森行伍。警柳眼之贪眠,释莲心之味苦,素女约于桂岩,宓妃迎于兰渚。弄玉吹笙,
寒簧击敔。征嵩岳之妃,启骊山之姥。龟呈洛浦之灵,兽作咸池之舞。潜赤水兮龙
吟,集珠林兮凤翥。爰格爰诚,匪笤匪簠。发轫乎霞城,还旌乎玄圃。既显微而若
逋,复氤氲而倏阻。离合兮烟云,空蒙兮雾雨。尘霾敛兮星高,溪山丽兮月午。何
心意之怦怦,若寤寐之栩栩?余乃欷歔怅怏,泣涕彷徨。人语兮寂历,天籁兮筼筜。
鸟惊散而飞,鱼唼喋以响。志哀兮是祷,成礼兮期祥。呜呼哀哉!尚飨!
读毕,遂焚帛奠茗,依依不舍。小丫鬟催至再四,方才回身。

忽听山石之后有一人笑道:“且请留步。”二人听了,不觉大惊。那小丫鬟回
头一看,却是个人影儿从芙蓉花里走出来,他便大叫:“不好,有鬼!晴雯真来显
魂了!”唬得宝玉也忙看时,——

究竟是人是鬼,下回分解。