\chapter{秦可卿死封龙禁尉~王熙凤协理宁国府}

话说凤姐儿自贾琏送黛玉往扬州去后,心中实在无趣,每到晚间不过同平儿说
笑一回,就胡乱睡了。这日夜间和平儿灯下拥炉,早命浓熏绣被,二人睡下,屈指
计算行程该到何处。不知不觉已交三鼓,平儿已睡熟了。凤姐方觉睡眼微蒙,恍惚
只见秦氏从外走进来,含笑说道:“婶娘好睡!我今日回去,你也不送我一程。因
娘儿们素日相好,我舍不得婶娘,故来别你一别。还有一件心愿未了,非告诉婶娘,
别人未必中用。”凤姐听了,恍惚问道:“有何心愿?只管托我就是了。”秦氏道:
“婶娘,你是个脂粉队里的英雄,连那些束带顶冠的男子也不能过你。你如何连两
句俗语也不晓得?常言:‘月满则亏,水满则溢。’又道是:‘登高必跌重。’如
今我们家赫赫扬扬,已将百载,一日倘或乐极生悲,若应了那句‘树倒猢狲散’的
俗语,岂不虚称了一世诗书旧族了?”凤姐听了此话,心胸不快,十分敬畏,忙问
道:“这话虑的极是,但有何法可以永保无虞?”秦氏冷笑道:“婶娘好痴也!‘否
极泰来’,荣辱自古周而复始,岂人力所能常保的?但如今能于荣时筹画下将来衰
时的世业,亦可以常远保全了。即如今日诸事俱妥,只有两件未妥,若把此事如此
一行,则后日可保无患了。”

凤姐便问道:“什么事?”秦氏道:“目今祖茔虽四时祭祀,只是无一定的钱
粮;第二,家塾虽立,无一定的供给。依我想来,如今盛时固不缺祭祀供给,但将
来败落之时,此二项有何出处?莫若依我定见,趁今日富贵,将祖茔附近多置田庄、
房舍、地亩,以备祭祀、供给之费皆出自此处;将家塾亦设于此。合同族中长幼,
大家定了则例,日后按房掌管这一年的地亩钱粮、祭祀供给之事。如此周流,又无
争竞,也没有典卖诸弊。便是有罪,己物可以入官,这祭祀产业连官也不入的。便
败落下来,子孙回家读书务农也有个退步,祭祀又可永继。若目今以为荣华不绝,
不思后日,终非长策。眼见不日又有一件非常的喜事,真是烈火烹油、鲜花着锦之
盛。要知道也不过是瞬息的繁华,一时的欢乐,万不可忘了那‘盛筵必散’的俗语。
若不早为后虑,只恐后悔无益了!”凤姐忙问:“有何喜事?”秦氏道:“天机不
可泄漏。只是我与婶娘好了一场,临别赠你两句话,须要记着!”因念道:
三春去后诸芳尽,各自须寻各自门。
凤姐还欲问时,只听二门上传出云板,连叩四下,正是丧音,将凤姐惊醒。人回:
“东府蓉大奶奶没了。”凤姐吓了一身冷汗,出了一回神,只得忙穿衣服往王夫人
处来。彼时合家皆知,无不纳闷,都有些伤心。那长一辈的想他素日孝顺,平辈的
想他素日和睦亲密,下一辈的想他素日慈爱,以及家中仆从老小想他素日怜贫惜
贱、爱老慈幼之恩,莫不悲号痛哭。

闲言少叙,却说宝玉因近日林黛玉回去,剩得自己落单,也不和人玩耍,每到
晚间,便索然睡了。如今从梦中听见说秦氏死了,连忙翻身爬起来,只觉心中似戳
了一刀的,不觉的“哇”的一声,直喷出一口血来。袭人等慌慌忙忙上来,扶着问:
“是怎么样的?”又要回贾母去请大夫。宝玉道:“不用忙,不相干。这是急火攻
心,血不归经。”说着便爬起来,要衣服换了,来见贾母,即时要过去。袭人见他
如此,心中虽放不下,又不敢拦阻,只得由他罢了。贾母见他要去,因说:“才咽
气的人,那里不干净。二则夜里风大,等明早再去不迟。”宝玉那里肯依。贾母命
人备车多派跟从人役,拥护前来。

一直到了宁国府前,只见府门大开,两边灯火,照如白昼。乱烘烘人来人往,
里面哭声摇山振岳。宝玉下了车,忙忙奔至停灵之室,痛哭一番。然后见过尤氏,
谁知尤氏正犯了胃气疼的旧症,睡在床上。然后又出来见贾珍。彼时贾代儒、代修、
贾敕、贾效、贾敦、贾赦、贾政、贾琮、贾、贾珩、贾、贾琛、贾琼、贾、
贾蔷、贾菖、贾菱、贾芸、贾芹、贾蓁、贾萍、贾藻、贾蘅、贾芬、贾芳、贾蓝、
贾菌、贾芝等都来了。贾珍哭的泪人一般,正和贾代儒等说道:“合家大小,远近
亲友,谁不知我这媳妇比儿子还强十倍。如今伸腿去了,可见这长房内绝灭无人
了!”说着又哭起来。众人劝道:“人已辞世,哭也无益,且商议如何料理要紧。”
贾珍拍手道:“如何料理!不过尽我所有罢了!”正说着,只见秦邦业、秦钟、尤
氏几个眷属尤氏姊妹也都来了,贾珍便命贾琼、贾琛、贾、贾蔷四个人去陪客,
一面吩咐去请钦天监阴阳司来择日。择准停灵七七四十九日,三日后开丧送讣闻。
这四十九日,单请一百零八众僧人在大厅上拜“大悲忏”,超度前亡后死鬼魂;另
设一坛于天香楼,是九十九位全真道士,打十九日解冤洗业醮。然后停灵于会芳园
中,灵前另外五十众高僧、五十位高道对坛,按七作好事。那贾敬闻得长孙媳妇死
了,因自为早晚就要飞升,如何肯又回家染了红尘将前功尽弃呢。故此并不在意,
只凭贾珍料理。

且说贾珍恣意奢华,看板时,几副杉木板皆不中意。可巧薛蟠来吊,因见贾珍
寻好板,便说:“我们木店里有一副板,说是铁网山上出的,作了棺材,万年不坏
的。这还是当年先父带来的,原系忠义亲王老千岁要的,因他坏了事,就不曾用。
现在还封在店里,也没有人买得起。你若要,就抬来看看。”贾珍听说甚喜,即命
抬来。大家看时,只见帮底皆厚八寸,纹若槟榔,味若檀麝,以手扣之,声如玉石。
大家称奇。贾珍笑问道:“价值几何?”薛蟠笑道:“拿着一千两银子只怕没处买;
什么价不价,赏他们几两银子作工钱就是了。”贾珍听说,连忙道谢不尽,即命解
锯造成。贾政因劝道:“此物恐非常人可享。殓以上等杉木也罢了。”贾珍如何肯
听。

忽又听见秦氏之丫鬟,名唤瑞珠,见秦氏死了,也触柱而亡。此事更为可罕,
合族都称叹。贾珍遂以孙女之礼殡殓之,一并停灵于会芳园之登仙阁。又有小丫鬟
名宝珠的,因秦氏无出,乃愿为义女,请任摔丧驾灵之任。贾珍甚喜,即时传命,
从此皆呼宝珠为“小姑娘”。那宝珠按未嫁女之礼在灵前哀哀欲绝。于是合族人并
家下诸人都各遵旧制行事,自不得错乱。

贾珍因想道:“贾蓉不过是黉门监生,灵幡上写时不好看;便是执事也不多。”
因此心下甚不自在。可巧这日正是首七第四日,早有大明宫掌宫内监戴权,先备了
祭礼遣人来,次后坐了大轿,打道鸣锣,亲来上祭。贾珍忙接待,让坐至逗蜂轩献
茶。贾珍心中早打定主意,因而趁便就说要与贾蓉捐个前程的话。戴权会意,因笑
道:“想是为丧礼上风光些?”贾珍忙道:“老内相所见不差。”戴权道:“事倒
凑巧,正有个美缺:如今三百员龙禁尉缺了两员,昨儿襄阳侯的兄弟老三来求我,
现拿了一千五百两银子送到我家里。你知道,咱们都是老相好,不拘怎么样,看着
他爷爷的分上,胡乱应了。还剩了一个缺。谁知永兴节度使冯胖子要求与他孩子捐,
我就没工夫应他。既是咱们的孩子要捐,快写个履历来。”贾珍忙命人写了一张红
纸履历来。戴权看了,上写着:

江南应天府江宁县监生贾蓉,年二十岁。曾祖,原任京营节度使世袭一等神威
将军贾代化。祖,丙辰科进士贾敬。父,世袭三品爵威烈将军贾珍。

戴权看了,回手递与一个贴身的小厮收了,道:“回去送与户部堂官老赵,说
我拜上他起一张五品龙禁尉的票,再给个执照,就把这履历填上。明日我来兑银子
送过去。”小厮答应了。戴权告辞,贾珍款留不住,只得送出府门。临上轿,贾珍
问:“银子还是我到部去兑,还是送入内相府中?”戴权道:“若到部里兑,你又
吃亏了。不如平准一千两银子送到我家就完了。”贾珍感谢不尽,说:“待服满,
亲带小犬到府叩谢。”于是作别。

接着又听喝道之声,原来是忠靖侯史鼎的夫人,带着侄女史湘云来了。王夫人、
邢夫人、凤姐等刚迎入正房,又见锦乡侯、川宁侯、寿山伯三家祭礼也摆在灵前;
少时,三人下轿,贾珍接上大厅。如此亲朋你来我去,也不能计数。只这四十九日,
宁国府街上一条白漫漫人来人往,花簇簇官去官来。

贾珍令贾蓉次日换了吉服,领凭回来。灵前供用执事等物俱按五品职例,灵牌
疏上皆写“诰授贾门秦氏宜人之灵位”。会芳园临街大门洞开,两边起了鼓乐厅,
两班青衣按时奏乐,一对对执事摆的刀斩斧截。更有两面朱红销金大牌竖在门外,
上面大书道:“防护内廷紫禁道御前侍卫龙禁尉。”对面高起着宣坛,僧道对坛;
榜上大书“世袭宁国公冢孙妇防护内廷御前侍卫龙禁尉贾门秦氏宜人之丧。四大部
洲至中之地,奉天永建太平之国,总理虚无寂静沙门僧录司正堂万、总理元始正一
教门道纪司正堂叶等,敬谨修斋,朝天叩佛”以及“恭请诸伽蓝、揭谛、功曹等神,
圣恩普锡,神威远振,四十九日销灾洗业平安水陆道场”等语,亦不及繁记。

只是贾珍虽然心意满足,但里面尤氏又犯了旧疾,不能料理事务,惟恐各诰命
来往,亏了礼数,怕人笑话,因此心中不自在。当下正忧虑时,因宝玉在侧,便问
道:“事事都算安贴了,大哥哥还愁什么?”贾珍便将里面无人的话告诉了他。宝
玉听说,笑道:“这有何难,我荐一个人与你,权理这一个月的事,管保妥当。”
贾珍忙问:“是谁?”宝玉见坐间还有许多亲友,不便明言,走向贾珍耳边说了两
句。贾珍听了,喜不自胜,笑道:“这果然妥贴。如今就去。”说着拉了宝玉,辞
了众人,便往上房里来。

可巧这日非正经日期,亲友来的少,里面不过几位近亲堂客,邢夫人、王夫人、
凤姐并合族中的内眷陪坐。闻人报:“大爷进来了。”唬的众婆娘“唿”的一声,
往后藏之不迭。独凤姐款款站了起来。贾珍此时也有些病症在身,二则过于悲痛,
因拄个拐踱了进来。邢夫人等因说道:“你身上不好,又连日多事,该歇歇才是,
又进来做什么?”贾珍一面拄拐,扎挣着要蹲身跪下请安道乏,邢夫人等忙叫宝玉
搀住,命人椅子与他坐。贾珍不肯坐,因勉强陪笑道:“侄儿进来有一件事要求
二位婶娘、大妹妹。”邢夫人等忙问:“什么事?”贾珍忙说道:“婶娘自然知道:
如今孙子媳妇没了,侄儿媳妇又病倒。我看里头着实不成体统,要屈尊大妹妹一个
月,在这里料理料理,我就放心了。”邢夫人笑道:“原来为这个。你大妹妹现在
你二婶娘家,只和你二婶娘说就是了。”王夫人忙道:“他一个小孩子,何曾经过
这些事,倘或料理不清,反叫人笑话,倒是再烦别人好。”贾珍笑道:“婶娘的意
思侄儿猜着了,是怕大妹妹劳苦了。若说料理不开,从小儿大妹妹玩笑时就有杀伐
决断,如今出了阁,在那府里办事,越发历练老成了。我想了这几日,除了大妹妹
再无人可求了。婶娘不看侄儿和侄儿媳妇面上,只看死的分上罢!”说着流下泪来。

王夫人心中为的是凤姐未经过丧事,怕他料理不起,被人见笑;今见贾珍苦苦
的说,心中已活了几分,却又眼看着凤姐出神。那凤姐素日最喜揽事,好卖弄能干,
今见贾珍如此央他,心中早已允了。又见王夫人有活动之意,便向王夫人道:“大
哥说得如此恳切,太太就依了罢。”王夫人悄悄的问道:“你可能么?”凤姐道:
“有什么不能的。外面的大事已经大哥哥料理清了,不过是里面照管照管。便是我
有不知的,问太太就是了。”王夫人见说得有理,便不出声。贾珍见凤姐允了,又
陪笑道:“也管不得许多了,横竖要求大妹妹辛苦辛苦。我这里先与大妹妹行礼,
等完了事,我再到那府里去谢。”说着就作揖,凤姐连忙还礼不迭。

贾珍便命人取了宁国府的对牌来,命宝玉送与凤姐,说道:“妹妹爱怎么就怎
么样办,要什么,只管拿这个取去,也不必问我。只求别存心替我省钱,要好看为
上;二则也同那府里一样待人才好,不要存心怕人抱怨。只这两件外,我再没不放
心的了。”凤姐不敢就接牌,只看着王夫人,王夫人道:“你大哥既这么说,你就
照看照看罢了。只是别自作主意,有了事打发人问你哥哥嫂子一声儿要紧。”宝玉
早向贾珍手里接过对牌来,强递与凤姐了。贾珍又问:“妹妹还是住在这里,还是
天天来呢?若是天天来,越发辛苦了。我这里赶着收拾出一个院落来,妹妹住过这
几日,倒安稳。”凤姐笑说:“不用,那边也离不得我,倒是天天来的好。”贾珍
说:“也罢了。”然后又说了一回闲话,方才出去。

一时女眷散后,王夫人因问凤姐:“你今儿怎么样?”凤姐道:“太太只管请
回去;我须得先理出一个头绪来才回得去呢。”王夫人听说,便先同邢夫人回去,
不在话下。这里凤姐来至三间一所抱厦中坐了。因想:头一件是人口混杂,遗失东
西;二件,事无专管,临期推委;三件,需用过费,滥支冒领;四件,任无大小,
苦乐不均;五件,家人豪纵,有脸者不能服钤束,无脸者不能上进。此五件实是宁
府中风俗。

不知凤姐如何处治,且听下回分解。