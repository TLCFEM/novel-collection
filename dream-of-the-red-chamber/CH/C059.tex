\chapter{柳叶渚边嗔莺叱燕~绛芸轩里召将飞符}

话说宝玉闻听贾母等回来,随多添了一件衣裳,拄了杖前边来,都见过了。贾
母等因每日辛苦,都要早些歇息,一宿无话。次日五鼓,又往朝中去。

离送灵日不远,鸳鸯、琥珀、翡翠、玻璃四人都忙着打点贾母之物,玉钏、彩
云、彩霞皆打点王夫人之物,当面查点与跟随的管事媳妇们。跟随的一共大小六个
丫鬟,十个老婆媳妇子,男人不算。连日收拾驮轿器械。鸳鸯和玉钏儿皆不随去,
只看屋子。一面先几日预备帐幔铺陈之物,先有四五个媳妇并几个男子领出来,坐
了几辆车绕过去,先至下处,铺陈安插等候。临日贾母带着贾蓉媳妇坐一乘驮轿,
王夫人在后,亦坐一乘驮轿,贾珍骑马率领众家丁围护。又有几辆大车与婆子丫鬟
等坐,并放些随换的衣包等件。是日薛姨妈尤氏率领诸人直送至大门外方回。贾琏
恐路上不便,一面打发他父母起身,赶上了贾母王夫人驮轿,自己也随后带领家丁
押后跟来。

荣府内,赖大添派人丁上夜,将两处厅院都关了,一应出入人等皆走西边小角
门,日落时便命关了仪门,不放人出入。园中前后东西角门亦皆关锁,只留王夫人
大房之后常系他姐妹出入之门,东边通薛姨妈的角门,这两门因在里院,不必关锁。
里面鸳鸯和玉钏儿也将上房关了,自领丫鬟婆子下房去歇。每日林之孝家的带领十
来个老婆子上夜,穿堂内又添了许多小厮打更,已安插得十分妥当。

一日清晓,宝钗春困已醒,搴帷下榻,微觉轻寒。及启户视之,见院中土润苔
青,原来五更时落了几点微雨。于是唤起湘云等人来,一面梳洗。湘云因说两腮作
痒,恐又犯了桃花癣,因问宝钗要些蔷薇硝擦。宝钗道:“前日剩的都给了琴妹妹
了。”因说:“颦儿配了许多,我正要要他些来,因今年竟没发痒就忘了。”因命
莺儿去取些来。莺儿应了才去时,蕊官便说:“我和你去,顺便瞧瞧藕官。”说着
径同莺儿出了蘅芜院。

二人你言我语,一面行走一面说笑,不觉到了柳叶渚。顺着柳堤走来,因见叶
才点碧,丝若垂金,莺儿便笑道:“你会拿这柳条子编东西不会?”蕊官笑道:“编
什么东西?”莺儿道:“什么编不得?玩的使的都可。等我摘些下来,带着这叶子
编一个花篮,掐了各色花儿放在里头,才是好玩呢。”说着且不去取硝,只伸手采
了许多嫩条命蕊官拿着,他却一行走一行编花篮。随路见花便采一二枝,编出一个
玲珑过梁的篮子。枝上自有本来翠叶满布,将花放上,却也别致有趣。喜得蕊官笑
说:“好姐姐,给了我罢。”莺儿道:“这一个送咱们林姑娘,回来咱们再多采些,
编几个大家玩。”说着来至潇湘馆中。黛玉也正晨妆,见了这篮子,便笑说:“这
个新鲜花篮是谁编的?”莺儿说:“我编的,送给姑娘玩的。”黛玉接了,笑道:
“怪道人人赞你的手巧,这玩意儿却也别致。”一面瞧了,一面便叫紫鹃挂在那里。
莺儿又问候薛姨妈,方和黛玉要硝。黛玉忙命紫鹃去包了一包,递给莺儿。黛玉又
说道:“我好了,今日要出去逛逛。你回去说给姐姐,不用过来问候妈妈,也不敢
劳他过来。我梳了头,和妈妈都往那里去吃饭,大家热闹些。”

莺儿答应了出来,便到紫鹃房中找蕊官。只见蕊官却与藕官二人正说得高兴,
不能相舍,莺儿便笑说:“姑娘也去呢,藕官先同去等着不好吗?”紫鹃听见如此
说,便也说道:“这话倒很是。他这里淘气的可厌。”一面说,一面便将黛玉的匙
箸用了一块洋巾包了交给藕官,道:“你先带了这个去,也算一趟差了。”藕官接
了,笑嘻嘻同他二人出来,一径顺着柳堤走来。莺儿便又采些柳条,索性坐在山石
上编起来,又命蕊官先送了硝去再来。他二人只顾爱看他编,那里舍得去?莺儿只
管催,说:“你们再不去,我就不编了。”藕官便说:“同你去了,再快回来。”
二人方去了。

这里莺儿正编,只见何妈的女儿春燕走来,笑问:“姐姐编什么呢?”正说着,
蕊官藕官也到了,春燕便向藕官道:“前日你到底烧了什么纸?叫我姨妈看见了,
要告你没告成,倒被宝玉赖了他好些不是,气得他一五一十告诉我妈。你们在外头
二三年了,积了些什么仇恨,如今还不解开?”藕官冷笑道:“有什么仇恨?他们
不知足,反怨我们。在外头这两年,不知赚了我们多少东西,你说说可有的没的?”
春燕也笑道:“他是我的姨妈,也不好向着外人反说他的。怨不得宝玉说:‘女孩
儿未出嫁是颗无价宝珠,出了嫁不知怎么就变出许多不好的毛病儿来,再老了,更
不是珠子,竟是鱼眼睛了。分明一个人,怎么变出三样来。’这话虽是混帐话,想
起来真不错。别人不知道,只说我妈和姨妈他老姐儿两个,如今越老了越把钱看的
真了。先是老姐儿两个在家抱怨没个差使进益,幸亏有了这园子,把我挑进来。可
巧把我分到怡红院,家里省了我一个人的费用不算外,每月还有四五百钱的馀剩,
这也还说不够。后来老姐儿两个都派到梨香院去照看他们,藕官认了我姨妈,芳官
认了我妈,这几年着实宽绰了。如今挪进来,也算撂开手了,还只无厌,你说可笑
不可笑?接着我妈和芳官又吵了一场,又要给宝玉吹汤,讨个没趣儿。幸亏园里的
人多,没人记的清楚谁是谁的亲故,要有人记得,我们一家子叫人家看着什么意思
呢。你这会子又跑了来弄这个,这一带地方上的东西都是我姑妈管着。他一得了这
地,每日起早睡晚自己辛苦了还不算,每日逼着我们来照看,生怕有人遭塌,——
我又怕误了我的差使。如今我们进来了,老姑嫂两个照看得谨谨慎慎,一根草也不
许人乱动。你还掐这些好花儿,又折他的嫩树枝子,他们即刻就来,你看他们抱怨。”
莺儿道:“别人折掐使不得,独我使得。自从分了地基之后,各房里每日皆有分例
的不用算,单算花草玩意儿:谁管什么,每日谁就把各房里姑娘丫头戴的,必要各
色送些折枝去,另有插瓶的。惟有我们姑娘说了:‘一概不用送,等要什么再和你
要。’究竟总没要过一次。我今便掐些,他们也不好意思说的。”

一言未了,他姑妈果然拄了拐杖走来,莺儿春燕等忙让坐。那婆子见采了许多
嫩柳,又见藕官等采了许多鲜花,心里便不受用,看着莺儿编弄,又不好说什么。
便说春燕道:“我叫你来照看照看,你就贪着玩不去了。倘或叫起你来,你又说我
使你了,拿我作隐身草儿,你来乐!”春燕道:“你老人家又使我,又怕,这会子
反说我,难道把我劈八瓣子不成?”莺儿笑道:“姑妈,你别信小燕儿的话。这都
是他摘下来,烦我给他编,我撵他,他不去。”春燕笑道:“你可少玩儿!你只顾
玩,他老人家就认真的。”那婆子本是愚夯之辈,兼之年迈昏,惟利是命,一概
情面不管。正心疼肝断,无计可施,听莺儿如此说,便倚老卖老,拿起拄杖向春燕
身上击了几下,骂道:“小蹄子!我说着你,你还和我强嘴儿呢。你妈恨的牙痒痒,
要撕你的肉吃呢,你还和我梆子似的!”打得春燕又愧又急,因哭道:“莺儿姐姐
玩话,你就认真打我!我妈为什么恨我?又没烧糊了洗脸水,有什么不是?”莺儿本
是玩话,忽见婆子认真动了气,忙上前拉住,笑道:“我才是玩话,你老人家打他,
这不是臊我了吗?”那婆子道:“姑娘你别管我们的事。难道为姑娘在这里,不许
我们管孩子不成?”莺儿听这般蠢话,便赌气红了脸,撒了手,冷笑道:“你要管,
那一刻管不得?偏我说了一句玩话,就管他了?我看你管去!”说着便坐下,仍编柳
篮子。

偏又春燕的娘出来找他,喊道:“你不来舀水,在那里做什么?”那婆子便接
声儿道:“你来瞧瞧!你女孩儿连我也不服了,在这里排揎我呢。”那婆子一面走
过来,说:“姑奶奶又怎么了?我们丫头眼里没娘罢了,连姑妈也没了不成?”莺
儿见他娘来了,只得又说原故。他姑娘那里容人说话?便将石上的花柳与他娘瞧,
道:“你瞧瞧,你女孩儿这么大孩子顽的。他领着人遭塌我,我怎么说人?”他娘
也正为芳官之气未平,又恨春燕不遂他的心,便走上来打了个耳刮子,骂道:“小
娼妇,你能上了几年台盘,你也跟着那起轻薄浪小妇学!怎么就管不得你们了?干的
我管不得,你是我自己生出来的,难道也不敢管你不成?既是你们这起蹄子到得去
的地方我到不去,你就死在那里伺候,又跑出来浪汉子!”一面又抓起那柳条子来,
直送到他脸上,问道:“这叫做什么?这编的是你娘的什么?”莺儿忙道:“那是
我编的,你别指桑骂槐的。”那婆子深妒袭人晴雯一干人,早知道凡房中大些的丫
鬟,都比他们有些体统权势。凡见了这一干人,心中又畏又让,未免又气又恨,亦
且迁怒于众;复又看见了藕官,又是他姐姐的冤家:四处凑成一股怒气。

那春燕啼哭着往怡红院去了。他娘又恐问他为何哭,怕他又说出来,又要受晴
雯等的气,不免赶着来喊道:“你回来!我告诉你再去。”春燕那里肯回来。急的
他娘跑了去要拉他,春燕回头看见,便也往前飞跑。他娘只顾赶他,不防脚下被青
苔滑倒。招的莺儿三个人反都笑了。莺儿赌气将花柳皆掷于河中,自回房去。这里
把个婆子心疼的只念佛,又骂:“促狭小蹄子!遭塌了花儿,雷也是要劈的。”自
己且掐花与各房送去。

却说春燕一直跑进院中,顶头遇见袭人往黛玉处问安去,春燕便一把抱住袭人
说:“姑娘救我,我妈又打我呢!”袭人见他娘来了,不免生气,便说道:“三日
两头儿,打了干的打亲的。还是卖弄你女孩儿多,还是认真不知王法?”这婆子来
了几日,见袭人不言不语,是好性儿的,便说道:“姑娘,你不知道,别管我们的
闲事。都是你们纵的,还管什么?”说着,便又赶着打。袭人气的转身进来,见麝
月正在海棠下晾手巾,——听如此喊闹,便说:“姐姐别管,看他怎么着。”一面
使眼色给春燕。春燕会意,直奔了宝玉去。众人都笑说:“这可是!从来没有的事,
今儿都闹出来了。”麝月向婆子道:“你再略煞一煞气儿,难道这些人的脸面,和
你讨一个情还讨不出来不成?”

那婆子见他女儿奔到宝玉身边去,又见宝玉拉了春燕的手,说:“你别怕,有
我呢。”春燕一行哭,一行将方才莺儿等事都说出来。宝玉越发急起来,说:“你
只在这里闹倒罢了,怎么把你妈也都得罪起来?”麝月又向婆子及众人道:“怨不
得这嫂子说我们管不着他们的事。我们原无知,错管了,如今请出一个管得着的人
来管一管,嫂子就心服口服,也知道规矩了。”便回头命小丫头子:“去把平儿给
我叫来,平儿不得闲,就把林大娘叫了来。”那小丫头子应了便走。众媳妇上来笑
说:“嫂子快求姑娘们叫回那孩子来罢。平姑娘来了,可就不好了。”那婆子说道:
“凭是那个姑娘来了,也要评个理。没有见个娘管女孩儿,大家管着娘的!”众人
笑道:“你当是那个平姑娘?是二奶奶屋里的平姑娘啊。他有情么,说你两句;他
一翻脸,嫂子你吃不了兜着走。”说着只见那个小丫头回来说:“平姑娘正有事呢,
问我做什么,我告诉了他。他说,叫先撵出他去,告诉林大娘,在角门子上打四十
板子就是了。”那婆子听见如此说了,吓得泪流满面,央告袭人等说:“好容易我
进来了,况且我是寡妇家,没有坏心,一心在里头伏侍姑娘们。我这一去,不知苦
到什么田地!”袭人见他如此说,又心软了,便说:“你既要在这里,又不守规矩,
又不听话,又乱打人。那里弄你这个不晓事的人来!天天斗口齿,也叫人笑话。”
晴雯道:“理他呢,打发他去了正经。那里那么大工夫和他对嘴对舌的?”那婆子
又央众人道:“我虽错了,姑娘们吩咐了,以后改过。姑娘们那不是行好积德?”
一面又央告春燕:“原是为打你起的,饶没打成你,我如今反受了罪。好孩子,你
好歹替我求求罢!”宝玉见如此可怜,便命留下:“不许再闹!再闹,一定打了撵
出去。”

那婆子一一谢过下去。只见平儿走来,问系何事,袭人等忙说:“已完了,不
必再提了。”平儿笑道:“‘得饶人处且饶人’,得将就的就省些事罢。但只听见
各屋里大小人等都作起反来了,一处不了又一处,叫我不知管那一处是。”袭人笑
道:“我只说我们这里反了,原来还有几处。”平儿笑道:“这算什么事!这三四
日的工夫,一共大小出了八九件呢,比这里的还大,可气可笑。”袭人等听了诧异。

不知何事,下回分解。