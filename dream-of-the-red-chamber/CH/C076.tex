\chapter{凸碧堂品笛感凄清~凹晶馆联诗悲寂寞}

话说贾赦贾政带领贾珍等散去不提。且说贾母这里命将围屏撤去,两席并作一
席。众媳妇另行擦桌整果,更杯洗箸,陈设一番。贾母等都添了衣,盥漱吃茶,方
又坐下,团团围绕。贾母看时,宝钗姊妹二人不在坐内,知他家去圆月,且李纨凤
姐二人又病,少了这四个人,便觉冷清了好些。贾母因笑道:“往年你老爷们不在
家,咱们都是请过姨太太来大家赏月,却十分热闹。忽一时想起你老爷来,又不免
想到母子夫妻儿女不能一处,也都没兴。及至今年你老爷来了,正该大家团圆取乐,
又不便请他们娘儿们来说笑说笑,况且他们今年又添了两口人,也难撂下他们跑到
这里来。偏又把凤丫头病了,有他一个人说说笑笑,还抵得十个人的空儿:可见天
下事总难十全!”说毕,不觉长叹一声,随命拿大杯来斟热酒。王夫人笑道:“今
日得母子团圆,自比往年有趣。往年娘儿们虽多,终不似今年骨肉齐全的好。”贾
母笑道:“正是为此,所以我才高兴,拿大杯来吃酒。你们也换大杯才是。”邢夫
人等只得换上大杯来。因夜深体乏,且不能胜酒,未免都有些倦意。无奈贾母兴犹
未阑,只得陪饮。贾母又命将毡毯铺在阶上,命将月饼、西瓜、果品等类都叫搬下
去,命丫头媳妇们也都团团围坐赏月。

贾母因见月至天中,比先越发精彩可爱,因说:“如此好月,不可不闻笛。”
因命又将十番上女子传来。贾母道:“音乐多了,反失雅致,只用吹笛的远远的吹
起来,就够了。”说毕,刚才去吹时,只见跟邢夫人的媳妇走来向邢夫人说了两句
话。贾母便问:“什么事?”邢夫人便回说:“方才大老爷出去,被石头绊了一下,
歪了腿。”贾母听说,忙命两个婆子快看去,又命邢夫人快去。邢夫人遂告辞起身。
贾母便又说:“珍哥媳妇也趁便儿就家去罢,我也就睡了。”尤氏笑道:“我今日
不回去了,定要和老祖宗吃一夜。”贾母笑道:“使不得。你们小两口儿今夜要团
团圆圆的,如何为我耽搁了?”尤氏红了脸,笑道:“老祖宗说的我们太不堪了。
虽是我们年轻,已经是二十来年的夫妻,也奔四十岁的人了,况且孝服未满。陪着
老太太玩一夜是正理。”贾母听说,笑道:“这话很是。我倒也忘了孝未满。可怜
你公公已死了二年多了!可是我倒忘了,该罚我一大杯。既这样,你就别送,竟陪
着我罢。叫蓉儿媳妇送去,就顺便回去罢。”尤氏说给贾蓉媳妇答应着,送出邢夫
人,一同至大门,各自上车回去,不在话下。

这里众人赏了一回桂花,又入席换暖酒来。正说着闲话,猛不防那壁里桂花树
下,呜咽悠扬,吹出笛声来。趁着这明月清风,天空地静,真令人烦心顿释,万虑
齐除,肃然危坐,默然相赏。听约两盏茶时,方才止住。大家称赞不已,于是遂又
斟上暖酒来,贾母笑道:“果然好听么?”众人笑道:“实在好听。我们也想不到
这样,须得老太太带领着,我们也得开些心儿。”贾母道:“这还不大好,须得拣
那曲谱越慢的吹来越好听。”便命斟一大杯酒送给吹笛之人,慢慢的吃了再细细的
吹一套来。媳妇们答应了。方送去,只见方才看贾赦的两个婆子回来说:“瞧了。
右脚面上白肿了些。如今调服了药,疼的好些了,也没大关系。”贾母点头叹道:
“我也太操心!打紧说我偏心,我反这样。”

说着,鸳鸯拿巾兜与大斗篷来,说:“夜深了,恐露水下了,风吹了头,坐坐
也该歇了。”贾母道:“偏今儿高兴,你又来催。难道我醉了不成?偏要坐到天亮。”
因命再斟来,一面戴上兜巾,披了斗篷,大家陪着又饮,说些笑话。只听桂花阴里
又发出一缕笛音来,果然比先越发凄凉,大家都寂然而坐。夜静月明,众人不禁伤
感,忙转身陪笑说语解释,又命换酒止笛。尤氏笑说道:“我也就学了一个笑话,
说给老太太解闷儿。”贾母勉强笑道:“这样更好,快说来我听。”尤氏乃说道:
“一家子养了四个儿子:大儿子只一个眼睛;二儿子只一个耳朵;三儿子只一个鼻
子眼;四儿子倒都齐全,偏又是个哑吧。”正说到这里,只见席上贾母已朦胧双眼,
似有睡去之态。尤氏方住了,忙和王夫人轻轻叫请。贾母睁眼笑道:“我不困,白
闭闭眼养神。你们只管说,我听着呢。”王夫人等道:“夜已深了,风露也大,请
老太太安歇罢了,明日再赏:十六月色也好。”贾母道:“什么时候?”王夫人笑
道:“已交四更。他们姊妹们熬不过,都去睡了。”贾母听说,细看了一看,果然
都散了,只有探春一人在此。贾母笑道:“也罢。你们也熬不惯,况且弱的弱,病
的病,去了倒省心。只是三丫头可怜,尚还等着。你也去罢,我们散了。”说着便
起身,吃了一口清茶,便坐竹椅小轿,两个婆子搭起,众人围随出园去了,不在话
下。

这里众媳妇收拾杯盘,却少了个细茶杯,各处寻觅不见。又问众人:“必是失
手打了。撂在那里?告诉我,拿了磁瓦去交,好作证见;不然,又说偷起来了。”
众人都说:“没有打碎。只怕跟姑娘的人打了,也未可知。你细想想,或问问他们
去?”一语提醒了那媳妇,笑道:“是了。那一会记得是翠缕拿着的,我去问他。”
说着便找时,刚到了甬道,就遇见紫鹃和翠缕来了。翠缕便问道:“老太太散了?
可知我们姑娘那里去了?”这媳妇道:“我来问你一个茶钟那里去了,你倒问我要
姑娘。”翠缕笑道:“我因倒茶给姑娘喝来着,展眼回头连姑娘也没了。”那媳妇
道:“太太才说,都睡觉去了。你不知那里玩去了,还不知道呢。”翠缕和紫鹃道:
“断乎没有悄悄儿睡去的,只怕在那里走了一走?如今老太太走了,赶过前边送去,
也未可知,我们且往前边找去。有了姑娘,自然你的茶钟也有了。你明日一早再找
罢,有什么忙的。”媳妇笑道:“有了下落就不必忙了,明儿和你要罢。”说毕回
去查收家伙。这里紫鹃和翠缕便往贾母处来,不在话下。

原来黛玉和湘云二人并未去睡。只因黛玉见贾府中许多人赏月,贾母犹叹人
少,又想宝钗姐妹家去,母女弟兄自去赏月,不觉对景感怀,自去倚栏垂泪。宝玉
近因晴雯病势甚重,诸务无心,王夫人再四遣他去睡,他从此去了。探春又因近日
家事恼着,无心游玩。虽有迎春惜春二人,偏又素日不大甚合,所以只剩湘云一人
宽慰他。因说:“你是个明白人,还不自己保养。可恨宝姐姐琴妹妹天天说亲道热,
早已说今年中秋要大家一处赏月,必要起诗社,大家联句。到今日,便扔下咱们自
己赏月去了,社也散了,诗也不做了。倒是他们父子叔侄纵横起来!你可知宋太祖
说的好:‘卧榻之侧,岂容他人酣睡?’他们不来,咱们两个竟联起句来,明日羞
他们一羞。”黛玉见他这般劝慰,也不肯负他的豪兴,因笑道:“你看这里这等人
声嘈杂,有何诗兴!”湘云笑道:“这山上赏月虽好,总不及近水赏月更妙。你知
道这山坡底下就是池沿。山凹里近水一个所在,就是凹晶馆。可知当日盖这园子,
就有学问。这山之高处,就叫凸碧;山之低洼近水处,就叫凹晶。这‘凸’‘凹’
二字,历来用的人最少,如今直用作轩馆之名,更觉新鲜,不落窠臼。可知这两处,
一上一下,一明一暗,一高一矮,一山一水,竟是特因玩月而设此处。有爱那山高
月小的,便往这里来;有爱那皓月清波的,便往那里去。只是这两个字俗念作‘洼’
‘拱’二音,便说俗了,不大见用。只陆放翁用了一个‘凹’字,‘古砚微凹聚墨
多’,还有人批他俗,岂不可笑?”黛玉道:“也不只放翁才用,古人中用者太多。
如《青苔赋》,东方朔《神异经》,以至《画记》上云‘张僧繇画一乘寺’的故事,
不可胜举。只是今日不知,误作俗字用了。实和你说罢:这两个字,还是我拟的呢。
因那年试宝玉,宝玉拟了未妥,我们拟写出来,送给大姐姐瞧了。他又带出来,命
给舅舅瞧过,所以都用了。如今咱们就往凹晶馆去。”

说着,二人同下山坡,只一转弯就是。池沿上一带竹栏相接,直通着那边藕香
榭的路径。只有两个婆子上夜,因知在凸碧山庄赏月,与他们无干,早已息灯睡了。
黛玉湘云见息了灯,都笑道:“倒是他们睡了好,咱们就在卷篷底下赏这水月,何
如?”二人遂在两个竹墩上坐下。只见天上一轮皓月,池中一个月影,上下争辉,
如置身于晶宫鲛室之内。微风一过,粼粼然池面皱碧叠纹,真令人神清气爽。湘云
笑道:“怎么得这会子上船吃酒才好!要是在我家里,我就立刻坐船了。”黛玉道:
“正是古人常说的:‘事若求全何所乐?’据我说,这也罢了,何必偏要坐船。”
湘云笑道:“得陇望蜀,人之常情。”

正说间,只听笛韵悠扬起来。黛玉笑道:“今日老太太、太太高兴,这笛子吹
的有趣,倒是助咱们的兴趣了。咱们两个都爱五言,就还是五言排律罢。”湘云道:
“什么韵?”黛玉笑道:“咱们数这个栏杆上的直棍,这头到那头为止,他是第几
根,就是第几韵。”湘云笑道:“这倒别致。”于是二人起身,便从头数至尽头,
止得十三根。湘云道:“偏又是‘十三元’了,这个韵可用的少,作排律只怕牵强
不能压韵呢。少不得你先起一句罢了。”黛玉笑道:“倒要试试咱们谁强谁弱。只
是没有纸笔记。”湘云道:“明儿再写,只怕这一点聪明儿还有。”黛玉道:“我
先起一句现成的俗语罢。”因念道:
三五中秋夕,
湘云想了一想,道:
清游拟上元。撒天箕斗灿,
黛玉笑道:
匝地管弦繁。几处狂飞盏?
湘云笑道:“这一句‘几处狂飞盏’有些意思。这倒要对得好呢。”想了一想,笑
道:
谁家不启轩?轻寒风剪剪,
黛玉道:“好对!比我的却好。只是这句又说俗话了,就该加劲说了去才是。”湘
云笑道:“诗多韵险,也要铺陈些才是。纵有好的,且留在后头。”黛玉笑道:“到
后头没有好的,我看你羞不羞。”因联道:
良夜景暄暄。争饼嘲黄发,
湘云笑道:“这句不好,杜撰。用俗事来难我了。”黛玉笑道:“我说你不曾见过
书呢,‘吃饼’是旧典。《唐书》《唐志》,你看了来再说。”湘云笑道:“这也
难不倒,我也有了。”因联道:
分瓜笑绿媛。香新荣玉桂,
黛玉道:“这实是你的杜撰了。”湘云笑道:“明日咱们对查了出来,大家看看,
这会子别耽搁工夫。”黛玉笑道:“虽如此,下句也不好。不犯又用‘玉桂’‘金
兰’等字样来塞责。”因联道:
色健茂金萱。蜡烛辉琼宴,
湘云笑道:“‘金萱’二字,便宜了你,省了多少力!这样现成的韵,被你得了。
只不犯着替他们颂圣去。况且下句你也是塞责了。”黛玉笑道:“你不说‘玉桂’,
我难道强对个‘金萱’罢?再也要铺陈些富丽,方是即景之实事。”湘云只得又联
道:
觥筹乱绮园。分曹尊一令,
黛玉笑道:“下句好。只难对些。”因想了一想,联道:
射覆听三宣。骰彩红成点,
湘云笑道:“‘三宣’有趣,竟化俗成雅了。只是下句又说上骰子!”少不得联道:
传花鼓滥喧。晴光摇院宇,
黛玉笑道:“对得却好。下句又溜了,只管拿些风月来塞责吗?”湘云道:“究竟
没说到月上,也要点缀点缀,方不落题。”黛玉道:“且姑存之,明日再斟酌。”
因联道:
素彩接乾坤。赏罚无宾主,
湘云道:“又倒说他们做什么?不如说咱们。”因联道:
吟诗序仲昆。构思时倚槛,
黛玉道:“这可以入上你我了。”因联道:
拟句或依门。酒尽情犹在,
湘云说道:“这时候了!”乃联道:
更残乐已谖。渐闻语笑寂,
黛玉说道:“这时候,可知一步难似一步了。”因联道:
空剩雪霜痕。阶露团朝菌,
湘云道:“这一句怎么叶韵?让我想想。”因起身负手想了一想,笑道:“够了,
幸而想出一个字来,不然几乎败了。”因联道:
庭烟敛夕。秋湍泻石髓,
黛玉听了,不禁也起身叫妙,说:“这促狭鬼!果然留下好的。这会子方说‘’
字,亏你想得出。”湘云道:“幸而昨日看《历朝文选》,见了这个字。我不知是
何树,因要查一查,宝姐姐说:‘不用查,这就是如今俗叫做“朝开夜合”的。’
我信不及,到底查了一查,果然不错。看来宝姐姐知道的竟多。”黛玉笑道:“‘’
字用在此时更恰,也还罢了。只是‘秋湍’一句,亏你好想。只这一句,别的都要
抹倒,我少不得打起精神来对这一句,只是再不能似这一句了。”因想了又想,方
对道:
风叶聚云根。宝婺情孤洁,
湘云道:“这对得也还好。只是这一句,你也溜了。幸而是景中情,不单用‘宝婺’
来塞责。”因联道:
银蟾气吐吞。药催灵兔捣,
黛玉不语点头,半日遂念道:
人向广寒奔。犯斗邀牛女,
湘云也望月点首,联道:
乘槎访帝孙。盈虚轮莫定,
黛玉道:“对句不好,合掌。下句推开一步,倒还是‘急脉缓灸法’。”因又联道:
晦朔魄空存。壶漏声将涸,
湘云方欲联时,黛玉指池中黑影与湘云看道:“你看那河里,怎么像个人到黑影里
去了?敢是个鬼?”湘云笑道:“可是又见鬼了!我是不怕鬼的,等我打他一下。”
因弯腰拾了一块小石片,向那池中打去。只听打得水响,一个大圆圈将月影激荡,
散而复聚者几次。只听那黑影里“嘎”的一声,却飞起一个白鹤来,直往藕香榭去
了。黛玉笑道:“原是他,猛然想不到,反吓了一跳。”湘云笑道:“正是这个鹤
有趣,倒助了我了。”因联道:
窗灯焰已昏。寒塘渡鹤影,
黛玉听了,又叫好,又跺足,说:“了不得了,这鹤真是助他的了!这一句更比‘秋
湍’不同,叫我对什么才好?‘影’字只有一个‘魂’字可对。况且‘寒塘渡鹤’,
何等自然,何等现成,何等有景,且又新鲜,我竟要搁笔了。”湘云笑道:“大家
细想就有了;不然,就放着明日再联也可。”黛玉只看天,不理他,半日,猛然笑
道:“你不必捞嘴,我也有了,你听听。”因对道:
冷月葬诗魂。
湘云拍手赞道:“果然好极,非此不能对。好个‘葬诗魂’!”因又叹道:“诗固
新奇,只是太颓丧了些。你现病着,不该作此过于凄清奇谲之语。”黛玉笑道:“不
如此,如何压倒你?只为用工在这一句了。”

一语未了,只见栏外山石后转出一个人来,笑道:“好诗,好诗,果然太悲凉
了,不必再往下做。若底下只这样去,反不显这两句了,倒弄的堆砌牵强。”二人
不防,倒吓了一跳。细看时不是别人,却是妙玉。二人皆诧异,因问:“你如何到
了这里?”妙玉笑道:“我听见你们大家赏月,又吹得好笛,我也出来玩赏这清池
皓月。顺脚走到这里,忽听见你们两个吟诗,更觉清雅异常,故此就听住了。只是
方才我听见这一首中,有几句虽好,只是过于颓败凄楚。此亦关人之气数,所以我
出来止住你们。如今老太太都早已散了,满园的人想俱已睡熟了,你两个的丫头还
不知在那里找你们呢,你们也不怕冷了?快同我来,到我那里去吃杯茶,只怕就天
亮了。”黛玉笑道:“谁知道就这个时候了。”

三人遂一同来至栊翠庵中,只见龛焰犹青,炉香未烬。几个老道婆也都睡了,
只有小丫头在蒲团上垂头打盹,妙玉唤起来现烹茶。忽听扣门之声,小丫鬟忙开门
看时,却是紫鹃翠缕和几个老嬷嬷,来找他姊妹两个。进来见他们正吃茶,因都笑
道:“叫我们好找。一个园子里走遍了,连姨太太那里都找到了。那小亭里找时,
可巧那里上夜的正睡醒了,我们问他们,他们说:‘方才亭外头棚下两个人说话,
后来又添了一个人,听见说大家往庵里去。’我们就知道这里来了。”妙玉忙命丫
鬟,引他们到那边去坐着歇息吃茶。自却取了笔砚纸墨出来,将方才的诗命他二人
念着,遂从头写出来。黛玉见他今日十分高兴,便笑道:“从来没见你这样高兴,
我也不敢唐突请教。这还可以见教否?若不堪时,便就烧了;若或可改,即请改正
改正。”妙玉笑道:“也不敢妄评。只是这才有二十二韵。我意思想着你二位警句
已出,再续时,倒恐后力不加。我竟要续貂,又恐有玷。”黛玉从没见妙玉做过诗,
今见他高兴如此,忙说:“果然如此,我们虽不好,亦可以带好了。”妙玉道:“如
今收结,到底还归到本来面目上去。若只管丢了真情真事,且去搜奇检怪,一则失
了咱们的闺阁面目,二则也与题目无涉了。”林史二人皆道:“极是。”妙玉提笔
微吟,一挥而就,递与他二人道:“休要见笑。依我必须如此,方翻转过来。虽前
头有凄楚之句,亦无甚碍了。”二人接了看时,只见他续道:
香篆销金鼎,冰脂腻玉盆。
箫憎嫠妇泣,衾倩侍儿温。
空帐悲文凤,闲屏设彩鸳。
露浓苔更滑,霜重竹难扪。
犹步萦纡沼,还登寂历原。
石奇神鬼缚,木怪虎狼蹲。
朝光透,罘晓露屯。
振林千树鸟,啼谷一声猿。
歧熟焉忘径?泉知不问源。
钟鸣栊翠寺,鸡唱稻香村。
有兴悲何极?无愁意岂烦?
芳情只自遣,雅趣向谁言!
彻旦休云倦,烹茶更细论。
后书“右中秋夜大观园即景联句三十五韵”。

黛玉湘云二人称赞不已,说:“可见咱们天天是舍近求远。现有这样诗人在此,
却天天去纸上谈兵。”妙玉笑道:“明日再润色。此时已天明了,到底也歇息歇息
才是。”林史二人听说,便起身告辞,带领了丫鬟出来。妙玉送至门外,看他们去
远方掩门进来,不在话下。

这里翠缕向湘云道:“大奶奶那里还有人等着咱们睡去呢。如今还是那里去
好。”湘云笑道:“你顺路告诉他们,叫他们睡罢。我这一去,未免惊动病人,不
如闹林姑娘去罢。”说着,大家走至潇湘馆中。有一半人已睡去。二人进去了,卸
妆宽衣,盥洗已毕,方上床安歇。紫鹃放下绡帐,移灯掩门出去。谁知湘云有择席
之病,虽在枕上,只是睡不着。黛玉又是个心血不足,常常不眠的,今日又错过困
头,自然也是睡不着。二人在枕上翻来覆去。黛玉因问道:“怎么还睡不着?”湘
云微笑道:“我有个择席的病,况且走了困,只好躺躺儿罢。你怎么也睡不着?”
黛玉叹道:“我这睡不着也并非一日了。大约一年之中,通共也只好睡十夜满足的
觉。”湘云道:“你这病就怪不得了。”

要知端底,下回分解。