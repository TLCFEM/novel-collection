\chapter{蜂腰桥设言传心事~潇湘馆春困发幽情}

话说宝玉养过了三十三天之后,不但身体强壮,亦且连脸上疮痕平复,仍回大
观园去。这也不在话下。

且说近日宝玉病的时节,贾芸带着家下小厮坐更看守,昼夜在这里,那小红同
众丫鬟也在这里守着宝玉。彼此相见日多,渐渐的混熟了。小红见贾芸手里拿着块
绢子,倒像是自己从前掉的,待要问他,又不好问。不料那和尚道士来过,用不着
一切男人,贾芸仍种树去了;这件事待放下又放不下,待要问去又怕人猜疑。正是
犹豫不决、神魂不定之际,忽听窗外问道:“姐姐在屋里没有?”小红闻听,在窗
眼内望外一看,原来是本院的个小丫头佳蕙,因答说:“在家里呢,你进来罢。”
佳蕙听了跑进来,就坐在床上,笑道:“我好造化!才在院子里洗东西,宝玉叫往
林姑娘那里送茶叶,花大姐姐交给我送去。可巧老太太给林姑娘送钱来,正分给他
们的丫头们呢,见我去了,林姑娘就抓了两把给我。也不知是多少,你替我收着。”
便把手绢子打开,把钱倒出来交给小红。小红就替他一五一十的数了收起。

佳蕙道:“你这两日心里到底觉着怎么样?依我说,你竟家去住两日,请一个
大夫来瞧瞧,吃两剂药,就好了。”小红道:“那里的话?好好儿的,家去做什么?”
佳蕙道:“我想起来了。林姑娘生的弱,时常他吃药,你就和他要些来吃,也是一
样。”小红道:“胡说,药也是混吃的?”佳蕙道:“你这也不是个长法儿,又懒
吃懒喝的,终久怎么样?”小红道:“怕什么?还不如早些死了倒干净。”佳蕙道:
“好好儿的,怎么说这些话?”小红道:“你那里知道我心里的事!”佳蕙点头,
想了一会道:“可也怨不得你。这个地方,本也难站。就像昨儿老太太因宝玉病了
这些日子,说伏侍的人都辛苦了,如今身上好了,各处还香了愿,叫把跟着的人都
按着等儿赏他们。我们算年纪小,上不去,我也不抱怨;像你怎么也不算在里头?
我心里就不服。袭人那怕他得十分儿,也不恼他,原该的。说句良心话,谁还能比
他呢?别说他素日殷勤小心,就是不殷勤小心,也拼不得。只可气晴雯绮霞他们这
几个都算在上等里去,伏着宝玉疼他们,众人就都捧着他们。你说可气不可气?”
小红道:“也犯不着气他们。俗语说的:‘千里搭长棚——没有个不散的筵席。’
谁守一辈子呢?不过三年五载,各人干各人的去了,那时谁还管谁呢?”这两句话
不觉感动了佳蕙心肠,由不得眼圈儿红了,又不好意思无端的哭,只得勉强笑道:
“你这话说的是。昨儿宝玉还说:明儿怎么收拾房子,怎么做衣裳。倒像有几百年
熬煎似的。”

小红听了,冷笑两声,方要说话,只见一个未留头的小丫头走进来,手里拿着
些花样子并两张纸,说道:“这两个花样子叫你描出来呢。”说着,向小红撂下,
回转身就跑了。小红向外问道:“到底是谁的?也等不的说完就跑。‘谁蒸下馒头
等着你——怕冷了不成?’”那小丫头在窗外只说得一声:“是绮大姐姐的。”抬
起脚来,咕咚咕咚又跑了。小红便赌气把那样子撂在一边,向抽屉内找笔。找了半
天,都是秃的,因说道:“前儿一枝新笔放在那里了?怎么想不起来?”一面说,
一面出神,想了一回,方笑道:“是了,前儿晚上莺儿拿了去了。”因向佳蕙道:
“你替我取了来。”佳蕙道:“花大姐姐还等着我替他拿箱子,你自己取去罢。”
小红道:“他等着你,你还坐着闲磕牙儿?我不叫你取去,他也不‘等’你了。坏
透了的小蹄子!”

说着自己便出房来。出了怡红院,一径往宝钗院内来,刚至沁芳亭畔,只见宝
玉的奶娘李嬷嬷从那边来。小红立住,笑问道:“李奶奶,你老人家那里去了?怎
么打这里来?”李嬷嬷站住,将手一拍,道:“你说,好好儿的,又看上了那个什
么‘云哥儿’‘雨哥儿’的,这会子逼着我叫了他来。明儿叫上屋里听见,可又是
不好。”小红笑道:“你老人家当真的就信着他去叫么?”李嬷嬷道:“可怎么样
呢?”小红笑道:“那一个要是知好歹,就不进来才是。”李嬷嬷道:“他又不傻,
为什么不进来?”小红道:“既是进来,你老人家该别和他一块儿来;回来叫他一
个人混碰,看他怎么样!”李嬷嬷道:“我有那样大工夫和他走!不过告诉了他,
回来打发个小丫头子,或是老婆子,带进他来就完了。”说着拄着拐一径去了。

小红听说,便站着出神,且不去取笔。不多时,只见一个小丫头跑来,见小红
站在那里,便问道:“红姐姐,你在这里作什么呢?”小红抬头见是小丫头子坠儿,
小红道:“那里去?”坠儿道:“叫我带进芸二爷来。”说着,一径跑了。这里小
红刚走至蜂腰桥门前,只见那边坠儿引着贾芸来了。那贾芸一面走,一面拿眼把小
红一溜;那小红只装着和坠儿说话,也把眼去一溜贾芸:四目恰好相对。小红不觉
把脸一红,一扭身往蘅芜院去了。不在话下。

这里贾芸随着坠儿逶迤来至怡红院中,坠儿先进去回明了,然后方领贾芸进
去。贾芸看时,只见院内略略有几点山石,种着芭蕉,那边有两只仙鹤,在松树下
剔翎。一溜回廊上吊着各色笼子,笼着仙禽异鸟。上面小小五间抱厦,一色雕镂新
鲜花样扇,上面悬着一个匾,四个大字,题道是:“怡红快绿。”贾芸想道:“怪
道叫‘怡红院’,原来匾上是这四个字。”正想着,只听里面隔着纱窗子笑说道:
“快进来罢,我怎么就忘了你两三个月!”贾芸听见是宝玉的声音,连忙进入房内,
抬头一看,只见金碧辉煌,文章烁,却看不见宝玉在那里。一回头,只见左边立
着一架大穿衣镜,从镜后转出两个一对儿十五六岁的丫头来,说:“请二爷里头屋
里坐。”贾芸连正眼也不敢看,连忙答应了。

又进一道碧纱厨,只见小小一张填漆床上,悬着大红销金撒花帐子,宝玉穿着
家常衣服,着鞋,倚在床上,拿着本书;看见他进来,将书掷下,早带笑立起身
来。贾芸忙上前请了安,宝玉让坐,便在下面一张椅子上坐了。宝玉笑道:“只从
那个月见了你,我叫你往书房里来,谁知接接连连许多事情,就把你忘了。”贾芸
笑道:“总是我没造化,偏又遇着叔叔欠安。叔叔如今可大安了?”宝玉道:“大
好了。我倒听见说你辛苦了好几天。”贾芸道:“辛苦也是该当的。叔叔大安了,
也是我们一家子的造化。”说着,只见有个丫鬟端了茶来与他。那贾芸嘴里和宝玉
说话,眼睛却瞅那丫鬟:细挑身子,容长脸儿,穿着银红袄儿,青缎子坎肩,白绫
细褶儿裙子。那贾芸自从宝玉病了,他在里头混了两天,都把有名人口记了一半,
他看见这丫鬟,知道是袭人。他在宝玉房中比别人不同,如今端了茶来,宝玉又在
旁边坐着,便忙站起来笑道:“姐姐怎么给我倒起茶来?我来到叔叔这里,又不是
客,等我自己倒罢了。”宝玉道:“你只管坐着罢。丫头们跟前也是这么着。”贾
芸笑道:“虽那么说,叔叔屋里的姐姐们,我怎么敢放肆呢。”一面说,一面坐下
吃茶。

那宝玉便和他说些没要紧的散话:又说道谁家的戏子好,谁家的花园好,又告
诉他谁家的丫头标致,谁家的酒席丰盛,又是谁家有奇货,又是谁家有异物。那贾
芸口里只得顺着他说。说了一回,见宝玉有些懒懒的了,便起身告辞。宝玉也不甚
留,只说:“你明儿闲了只管来。”仍命小丫头子坠儿送出去了。

贾芸出了怡红院,见四顾无人,便慢慢的停着些走,口里一长一短和坠儿说话。
先问他:“几岁了?名字叫什么?你父母在那行上?在宝叔屋里几年了?一个月多少钱?
共总宝叔屋内有几个女孩子?”那坠儿见问,便一桩桩的都告诉他了。贾芸又道:
“刚才那个和你说话的,他可是叫小红?”坠儿笑道:“他就叫小红。你问他作什
么?”贾芸道:“方才他问你什么绢子,我倒拣了一块。”坠儿听了笑道:“他问
了我好几遍:可有看见他的绢子的。我那里那么大工夫管这些事?今儿他又问我,
他说我替他找着了他还谢我呢。才在蘅芜院门口儿说的,二爷也听见了,不是我撒
谎。好二爷,你既拣了,给我罢,我看他拿什么谢我。”原来上月贾芸进来种树之
时,便拣了一块罗帕,知是这园内的人失落的,但不知是那一个人的,故不敢造次。
今听见小红问坠儿,知是他的,心内不胜喜幸。又见坠儿追索,心中早得了主意,
便向袖内将自己的一块取出来,向坠儿笑道:“我给是给你,你要得了他的谢礼,
可不许瞒着我。”坠儿满口里答应了,接了绢子,送出贾芸,回来找小红,不在话
下。

如今且说宝玉打发贾芸去后,意思懒懒的,歪在床上,似有朦胧之态。袭人便
走上来,坐在床沿上推他,说道:“怎么又要睡觉?你闷的很,出去逛逛不好?”
宝玉见说,携着他的手笑道:“我要去,只是舍不得你。”袭人笑道:“你没别的
说了!”一面说,一面拉起他来。宝玉道:“可往那里去呢?怪腻腻烦烦的。”袭
人道:“你出去了就好了。只管这么委琐,越发心里腻烦了。”宝玉无精打彩,只
得依他。晃出了房门,在回廊上调弄了一回雀儿,出至院外,顺着沁芳溪,看了一
回金鱼。只见那边山坡上两只小鹿儿箭也似的跑来。宝玉不解何意,正自纳闷,只
见贾兰在后面,拿着一张小弓儿赶来。一见宝玉在前,便站住了,笑道:“二叔叔
在家里呢,我只当出门去了呢。”宝玉道:“你又淘气了。好好儿的,射他做什么?”
贾兰笑道:“这会子不念书,闲着做什么?所以演习演习骑射。”宝玉道:“磕了
牙,那时候儿才不演呢。”

说着,便顺脚一径来至一个院门前,看那凤尾森森,龙吟细细:正是潇湘馆。
宝玉信步走入,只见湘帘垂地,悄无人声。走至窗前,觉得一缕幽香从碧纱窗中暗
暗透出,宝玉便将脸贴在纱窗上。看时,耳内忽听得细细的长叹了一声,道:“‘每
日家情思睡昏昏!’”宝玉听了,不觉心内痒将起来。再看时,只见黛玉在床上伸
懒腰。宝玉在窗外笑道:“为什么‘每日家情思睡昏昏’的?”一面说,一面掀帘
子进来了。黛玉自觉忘情,不觉红了脸,拿袖子遮了脸,翻身向里装睡着了。宝玉
才走上来,要扳他的身子,只见黛玉的奶娘并两个婆子却跟进来了,说:“妹妹睡
觉呢,等醒来再请罢。”刚说着,黛玉便翻身坐起来,笑道:“谁睡觉呢?”那两
三个婆子见黛玉起来,便笑道:“我们只当姑娘睡着了。”说着,便叫紫鹃说:“姑
娘醒了,进来伺候。”一面说,一面都去了。

黛玉坐在床上,一面抬手整理鬓发,一面笑向宝玉道:“人家睡觉,你进来做
什么?”宝玉见他星眼微饧,香腮带赤,不觉神魂早荡,一歪身坐在椅子上,笑道:
“你才说什么?”黛玉道:“我没说什么。”宝玉笑道:“给你个榧子吃呢!我都
听见了。”二人正说话,只见紫鹃进来,宝玉笑道:“紫鹃,把你们的好茶沏碗我
喝。”紫鹃道:“我们那里有好的?要好的只好等袭人来。”黛玉道:“别理他。
你先给我舀水去罢。”紫鹃道:“他是客,自然先沏了茶来再舀水去。”说着,倒
茶去了。宝玉笑道:“好丫头!‘若共你多情小姐同鸳帐,怎舍得叫你叠被铺床?’”
黛玉登时急了,撂下脸来说道:“你说什么?”宝玉笑道:“我何尝说什么?”黛
玉便哭道:“如今新兴的,外头听了村话来,也说给我听;看了混帐书,也拿我取
笑儿。我成了替爷们解闷儿的了。”一面哭,一面下床来,往外就走。宝玉心下慌
了,忙赶上来说:“好妹妹,我一时该死,你好歹别告诉去!我再敢说这些话,嘴
上就长个疔,烂了舌头。”

正说着,只见袭人走来,说道:“快回去穿衣裳去罢,老爷叫你呢。”宝玉听
了,不觉打了个焦雷一般,也顾不得别的,疾忙回来穿衣服。出园来,只见焙茗在
二门前等着。宝玉问道:“你可知道老爷叫我是为什么?”焙茗道:“爷快出来罢,
横竖是见去的,到那里就知道了。”一面说,一面催着宝玉。转过大厅,宝玉心里
还自狐疑,只听墙角边一阵呵呵大笑,回头见薛蟠拍着手跳出来,笑道:“要不说
姨夫叫你,你那里肯出来的这么快!”焙茗也笑着跪下了。宝玉怔了半天,方想过
来,是薛蟠哄出他来。薛蟠连忙打恭作揖赔不是,又求:“别难为了小子,都是我
央及他去的。”宝玉也无法了,只好笑问道:“你哄我也罢了,怎么说是老爷呢?
我告诉姨娘去,评评这个理,可使得么?”薛蟠忙道:“好兄弟,我原为求你快些
出来,就忘了忌讳这句话,改日你要哄我,也说我父亲,就完了。”宝玉道:“嗳
哟,越发的该死了。”又向焙茗道:“反叛杂种,还跪着做什么?”焙茗连忙叩头
起来。

薛蟠道:“要不是,我也不敢惊动:只因明儿五月初三日,是我的生日,谁知
老胡和老程他们,不知那里寻了来的:这么粗这么长粉脆的鲜藕,这么大的西瓜,
这么长这么大的暹罗国进贡的灵柏香熏的暹罗猪、鱼。你说这四样礼物,可难得不
难得?那鱼、猪不过贵而难得,这藕和瓜亏他怎么种出来的!我先孝敬了母亲,赶着
就给你们老太太、姨母送了些去。如今留了些,我要自己吃恐怕折福,左思右想除
我之外惟你还配吃。所以特请你来。可巧唱曲儿的一个小子又来了,我和你乐一天
何如?”

一面说,一面来到他书房里,只见詹光、程日兴、胡斯来、单聘仁等并唱曲儿
的小子都在这里。见他进来,请安的,问好的,都彼此见过了。吃了茶,薛蟠即命
人:“摆酒来。”话犹未了,众小厮七手八脚摆了半天,方才停当归坐。宝玉果见
瓜藕新异,因笑道:“我的寿礼还没送来,倒先扰了。”薛蟠道:“可是呢,你明
儿来拜寿,打算送什么新鲜物儿?”宝玉道:“我没有什么送的。若论银钱吃穿等
类的东西,究竟还不是我的;惟有写一张字,或画一张画,这才是我的。”薛蟠笑
道:“你提画儿,我才想起来了:昨儿我看见人家一本春宫儿,画的很好。上头还
有许多的字,我也没细看,只看落的款,原来是什么‘庚黄’的。真好的了不得。”
宝玉听说,心下猜疑道:“古今字画也都见过些,那里有个‘庚黄’?”想了半天,
不觉笑将起来,命人取过笔来,在手心里写了两个字,又问薛蟠道:“你看真了是
‘庚黄’么?”薛蟠道:“怎么没看真?”宝玉将手一撒给他看道:“可是这两个
字罢?其实和‘庚黄’相去不远。”众人都看时,原来是“唐寅”两个字,都笑道:
“想必是这两个字,大爷一时眼花了,也未可知。”薛蟠自觉没趣,笑道:“谁知
他是‘糖银’是‘果银’的!”

正说着,小厮来回:“冯大爷来了。”宝玉便知是神武将军冯唐之子冯紫英来
了。薛蟠等一齐都叫“快请”。说犹未了,只见冯紫英一路说笑已进来了,众人忙
起席让坐。冯紫英笑道:“好啊!也不出门了,在家里高乐罢。”宝玉薛蟠都笑道:
“一向少会。老世伯身上安好?”紫英答道:“家父倒也托庇康健。但近来家母偶
着了些风寒,不好了两天。”薛蟠见他面上有些青伤,便笑道:“这脸上又和谁挥
拳来,挂了幌子了?”冯紫英笑道:“从那一遭把仇都尉的儿子打伤了,我记了,
再不怄气,如何又挥拳?这脸上是前日打围,在铁网山叫兔鹘梢了一翅膀。”宝玉
道:“几时的话?”紫英道:“三月二十八日去的,前儿也就回来了。”宝玉道:
“怪道前儿初三四儿我在沈世兄家赴席不见你呢!我要问,不知怎么忘了。单你去
了,还是老世伯也去了?”紫英道:“可不是家父去!我没法儿,去罢了。难道我
闲疯了,咱们几个人吃酒听唱的不乐,寻那个苦恼去?这一次,大不幸之中却有大
幸。”

薛蟠众人见他吃完了茶,都说道:“且入席,有话慢慢的说。”冯紫英听说,
便立起身来说道:“论理,我该陪饮几杯才是,只是今儿有一件很要紧的事,回去
还要见家父面回,实不敢领。”薛蟠宝玉众人那里肯依,死拉着不放。冯紫英笑道:
“这又奇了。你我这些年,那一回有这个道理的?实在不能遵命。若必定叫我喝,
拿大杯来,我领两杯就是了。”众人听说,只得罢了,薛蟠执壶,宝玉把盏,斟了
两大海。那冯紫英站着,一气而尽。宝玉道:“你到底把这个‘不幸之幸’说完了
再走。”冯紫英笑道:“今儿说的也不尽兴,我为这个,还要特治一个东儿,请你
们去细谈一谈;二则还有奉恳之处。”说着撒手就走。薛蟠道:“越发说的人热剌
剌的扔不下,多早晚才请我们?告诉了也省了人打闷雷。”冯紫英道:“多则十日,
少则八天。”一面说,一面出门上马去了。众人回来,依席又饮了一回方散。

宝玉回至园中,袭人正惦记他去见贾政,不知是祸是福,只见宝玉醉醺醺回来,
因问其原故,宝玉一一向他说了。袭人道:“人家牵肠挂肚的等着,你且高乐去,
也到底打发个人来给个信儿!”宝玉道:“我何尝不要送信儿,因冯世兄来了,就
混忘了。”正说着,只见宝钗走进来,笑道:“偏了我们新鲜东西了。”宝玉笑道:
“姐姐家的东西,自然先偏了我们了。”宝钗摇头笑道:“昨儿哥哥倒特特的请我
吃,我不吃,我叫他留着送给别人罢。我知道我的命小福薄,不配吃那个。”说着,
丫鬟倒了茶来,吃茶说闲话儿,不在话下。

却说那黛玉听见贾政叫了宝玉去了,一日不回来,心中也替他忧虑。至晚饭后,
闻得宝玉来了,心里要找他问问是怎么样了,一步步行来。见宝钗进宝玉的园内去
了,自己也随后走了来。刚到了沁芳桥,只见各色水禽尽都在池中浴水,也认不出
名色来,但见一个个文彩灼,好看异常,因而站住,看了一回。再往怡红院来,
门已关了,黛玉即便叩门。谁知晴雯和碧痕二人正拌了嘴,没好气,忽见宝钗来了,
那晴雯正把气移在宝钗身上,偷着在院内抱怨说:“有事没事跑了来坐着,叫我们
三更半夜的不得睡觉!”忽听又有人叫门,晴雯越发动了气,也并不问是谁,便说
道:“都睡下了,明儿再来罢!”

黛玉素知丫头们的性情,他们彼此玩耍惯了,恐怕院内的丫头没听见是他的声
音,只当别的丫头们了,所以不开门;因而又高声说道:“是我,还不开门么?”
晴雯偏偏还没听见,便使性子说道:“凭你是谁,二爷吩咐的,一概不许放进人来
呢!”黛玉听了这话,不觉气怔在门外。待要高声问他,逗起气来,自己又回思一
番:“虽说是舅母家如同自己家一样,到底是客边。如今父母双亡,无依无靠,现
在他家依栖,若是认真怄气,也觉没趣。”一面想,一面又滚下泪珠来了。真是回
去不是,站着不是。正没主意,只听里面一阵笑语之声,细听一听,竟是宝玉宝钗
二人。黛玉心中越发动了气,左思右想,忽然想起早起的事来:“必竟是宝玉恼我
告他的原故。但只我何尝告你去了?你也不打听打听,就恼我到这步田地!你今儿不
叫我进来,难道明儿就不见面了?”越想越觉伤感,便也不顾苍苔露冷,花径风寒,
独立墙角边花阴之下,悲悲切切,呜咽起来。原来这黛玉秉绝代之姿容,具稀世之
俊美,不期这一哭,把那些附近的柳枝花朵上宿鸟栖鸦,一闻此声,俱忒楞楞飞起
远避,不忍再听。正是。
花魂点点无情绪,鸟梦痴痴何处惊。
因又有一首诗道:
颦儿才貌世应稀,独抱幽芳出绣闺。
呜咽一声犹未了,落花满地鸟惊飞。
那黛玉正自啼哭,忽听吱娄娄一声,院门开处,不知是那一个出来。

要知端的,下回分解。