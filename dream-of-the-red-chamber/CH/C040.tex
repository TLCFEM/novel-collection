\chapter{史太君两宴大观园~金鸳鸯三宣牙牌令}

话说宝玉听了,忙进来看时,只见琥珀站在屏风跟前,说:“快去罢,立等你
说话呢。”宝玉来至上房,只见贾母正和王夫人众姐妹商议给史湘云还席。宝玉因
说:“我有个主意:既没有外客,吃的东西也别定了样数,谁素日爱吃的,拣样儿
做几样。也不必按桌席,每人跟前摆一张高几,各人爱吃的东西一两样,再一个十
锦攒心盒子、自斟壶,岂不别致?”贾母听了,说:“很是。”即命人传与厨房:
“明日就拣我们爱吃的东西做了,按着人数,再装了盒子来,早饭也摆在园里吃。”
商议之间,早又掌灯,一夕无话。

次日清早起来,可喜这日天气清朗。李纨清晨起来,看着老婆子丫头们扫那些
落叶,并擦抹桌椅,预备茶酒器皿。只见丰儿带了刘老老板儿进来,说:“大奶奶
倒忙的很。”李纨笑道:“我说你昨儿去不成,只忙着要去。”刘老老笑道:“老
太太留下我,叫我也热闹一天去。”丰儿拿了几把大小钥匙,说道:“我们奶奶说
了,外头的高几儿怕不够使,不如开了楼,把那收的拿下来使一天罢。奶奶原该亲
自来,因和太太说话呢,请大奶奶开了,带着人搬罢。”李氏便命素云接了钥匙。
又命婆子出去,把二门上小厮叫几个来。李氏站在大观楼下往上看着,命人上去开
了缀锦阁,一张一张的往下抬。小厮、老婆子、丫头一齐动手,抬了二十多张下来。
李纨道:“好生着,别慌慌张张鬼赶着似的,仔细碰了牙子!”又回头向刘老老笑
道:“老老也上去瞧瞧。”刘老老听说巴不得一声儿,拉了板儿登梯上去。进里面
只见乌压压的堆着些围屏桌椅、大小花灯之类,虽不大认得,只见五彩灼,各有
奇妙,念了几声佛便下来了。然后锁上门,一齐下来。李纨道:“恐怕老太太高兴,
越发把船上划子、篙、桨、遮阳幔子,都搬下来预备着。”众人答应,又复开了门,
色色的搬下来。命小厮传驾娘们,到船坞里撑出两只船来。

正乱着,只见贾母已带了一群人进来了,李纨忙迎上去,笑道:“老太太高兴,
倒进来了;我只当还没梳头呢,才掐了菊花要送去。”一面说,一面碧月早已捧过
一个大荷叶式的翡翠盘子来,里面养着各色折枝菊花。贾母便拣了一朵大红的簪在
鬓上,因回头看见了刘老老,忙笑道:“过来带花儿。”一语未完,凤姐儿便拉过
刘老老来,笑道:“让我打扮你。”说着,把一盘子花,横三竖四的插了一头。贾
母和众人笑的了不得。刘老老也笑道:“我这头也不知修了什么福,今儿这样体面
起来。”众人笑道:“你还不拔下来摔到他脸上呢,把你打扮的成了老妖精了。”
刘老老笑道:“我虽老了,年轻时也风流,爱个花儿粉儿的,今儿索性作个老风流!”

说话间,已来至沁芳亭上,丫鬟们抱了个大锦褥子来,铺在栏杆榻板上。贾母
倚栏坐下,命刘老老也坐在旁边,因问他:“这园子好不好?”刘老老念佛说道:
“我们乡下人,到了年下,都上城来买画儿贴。闲了的时候儿大家都说:‘怎么得
到画儿上逛逛!’想着画儿也不过是假的,那里有这个真地方儿?谁知今儿进这园
里一瞧,竟比画儿还强十倍!怎么得有人也照着这个园子画一张,我带了家去给他
们见见,死了也得好处。”贾母听说,指着惜春笑道:“你瞧我这个小孙女儿,他
就会画,等明儿叫他画一张如何?”刘老老听了,喜的忙跑过来拉着惜春,说道:
“我的姑娘!你这么大年纪儿,又这么个好模样儿,还有这个能干,别是个神仙托
生的罢?”贾母众人都笑了。

歇了歇,又领着刘老老都见识见识。先到了潇湘馆。一进门,只见两边翠竹夹
路,土地下苍苔布满,中间羊肠一条石子漫的甬路。刘老老让出来与贾母众人走,
自己却走土地。琥珀拉他道:“老老你上来走,看青苔滑倒了。”刘老老道:“不
相干,我们走熟了,姑娘们只管走罢。可惜你们的那鞋,别沾了泥。”他只顾上头
和人说话,不防脚底下果踩滑了,“咕咚”一交跌倒,众人都拍手呵呵的大笑。贾
母笑骂道:“小蹄子们,还不搀起来,只站着笑!”说话时,刘老老已爬起来了,
自己也笑了,说道:“才说嘴,就打了嘴了。”贾母问他:“可扭了腰了没有?叫
丫头们捶捶。”刘老老道:“那里说的我这么娇嫩了?那一天不跌两下子?都要捶起
来,还了得呢。”

紫鹃早打起湘帘,贾母等进来坐下。黛玉亲自用小茶盘儿捧了一盖碗茶来奉与
贾母。王夫人道:“我们不吃茶,姑娘不用倒了。”黛玉听说,便命丫头把自己窗
下常坐的一张椅子挪到下手,请王夫人坐了。刘老老因见窗下案上设着笔砚,又见
书架上放着满满的书,刘老老道:“这必定是那一位哥儿的书房了?”贾母笑指黛
玉道:“这是我这外孙女儿的屋子。”刘老老留神打量了黛玉一番,方笑道:“这
那里像个小姐的绣房?竟比那上等的书房还好呢。”贾母因问:“宝玉怎么不见?”
众丫头们答说:“在池子里船上呢。”贾母道:“谁又预备下船了?”李纨忙回说:
“才开楼拿的。我恐怕老太太高兴,就预备下了。”

贾母听了,方欲说话时,有人回说:“姨太太来了。”贾母等刚站起来,只见
薛姨妈早进来了,一面归坐,笑道:“今儿老太太高兴,这早晚就来了。”贾母笑
道:“我才说,来迟了的要罚他,不想姨太太就来迟了。”说笑一回。贾母因见窗
上纱颜色旧了,便和王夫人说道:“这个纱新糊上好看,过了后儿就不翠了。这院
子里头又没有个桃杏树,这竹子已是绿的,再拿绿纱糊上,反倒不配。我记得咱们
先有四五样颜色糊窗的纱呢。明儿给他把这窗上的换了。”凤姐儿忙道:“昨儿我
开库房,看见大板箱里还有好几匹银红蝉翼纱,也有各样折枝花样的,也有‘流云
蝙蝠’花样的,也有‘百蝶穿花’花样的,颜色又鲜,纱又轻软,我竟没见这个样
的,拿了两匹出来,做两床绵纱被,想来一定是好的。”贾母听了笑道:“呸,人
人都说你没有没经过没见过的,连这个纱还不能认得,明儿还说嘴。”薛姨妈等都
笑说:“凭他怎么经过见过,怎么敢比老太太呢!老太太何不教导了他,连我们也
听听。”凤姐儿也笑说:“好祖宗,教给我罢。”贾母笑向薛姨妈众人道:“那个
纱,比你们的年纪还大呢,怪不得他认做蝉翼纱,原也有些像。不知道的都认做蝉
翼纱。正经名字叫‘软烟罗’。”凤姐儿道:“这个名儿也好听,只是我这么大了,
纱罗也见过几百样,从没听见过这个名色。”贾母笑道:“你能活了多大?见过几
样东西?就说嘴来了。那个软烟罗只有四样颜色:一样雨过天青,一样秋香色,一
样松绿的,一样就是银红的。要是做了帐子,糊了窗屉,远远的看着就和烟雾一样,
所以叫做‘软烟罗’。那银红的又叫做‘霞影纱’。如今上用的府纱也没有这样软
厚轻密的了。”薛姨妈笑道:“别说凤丫头没见,连我也没听见过。”凤姐儿一面
说话,早命人取了一匹来了,贾母说:“可不是这个!先时原不过是糊窗屉,后来
我们拿这个做被做帐子试试,也竟好。明日就找出几匹来,拿银红的替他糊窗户。”
凤姐答应着。众人看了,都称赞不已。刘老老也觑着眼看,口里不住的念佛,说道:
“我们想做衣裳也不能,拿着糊窗子岂不可惜?”贾母道:“倒是做衣裳不好看。”
凤姐忙把自己身上穿的一件大红棉纱袄的襟子拉出来,向贾母薛姨妈道:“看我的
这袄儿。”贾母薛姨妈都说:“这也是上好的了,这是如今上用内造的,竟比不上
这个。”凤姐儿道:“这个薄片子还说是内造上用呢,竟连这个官用的也比不上啊。”
贾母道:“再找一找,只怕还有,要有就都拿出来,送这刘亲家两匹。有雨过天青
的,我做一个帐子挂上。剩的配上里子,做些个夹坎肩儿给丫头们穿,白收着霉坏
了。”凤姐儿忙答应了,仍命人送去。

贾母便笑道:“这屋里窄,再往别处逛去罢。”刘老老笑道:“人人都说:‘大
家子住大房。’昨儿见了老太太正房,配上大箱、大柜、大桌子、大床,果然威武。
那柜子比我们一间房子还大还高。怪道后院子里有个梯子,我想又不上房晒东西,
预备这梯子做什么?后来我想起来,一定是为开顶柜取东西,离了那梯子怎么上得
去呢?如今又见了这小屋子,更比大的越发齐整了。满屋里东西都只好看,可不知
叫什么。我越看越舍不得离了这里了!”凤姐道:“还有好的呢,我都带你去瞧瞧。”

说着,一径离了潇湘馆,远远望见池中一群人在那里撑船。贾母道:“他们既
备下船,咱们就坐一回。”说着,向紫菱洲蓼溆一带走来。未至池前,只见几个婆
子手里都捧着一色摄丝戗金五彩大盒子走来,凤姐忙问王夫人:“早饭在那里摆?”
王夫人道:“问老太太在那里就在那里罢了。”贾母听说,便回头说:“你三妹妹
那里好,你就带了人摆去,我们从这里坐了船去。”凤姐儿听说,便回身和李纨、
探春、鸳鸯、琥珀带着端饭的人等,抄着近路到了秋爽斋,就在晓翠堂上调开桌案。
鸳鸯笑道:“天天咱们说外头老爷们吃酒吃饭,都有个凑趣儿的,拿他取笑儿。咱
们今儿也得了个女清客了。”李纨是个厚道人,倒不理会;凤姐儿却听着是说刘老
老,便笑道:“咱们今儿就拿他取个笑儿。”二人便如此这般商议。李纨笑劝道:
“你们一点好事儿不做。又不是个小孩儿,还这么淘气,仔细老太太说!”鸳鸯笑
道:“很不与大奶奶相干,有我呢。”

正说着,只见贾母等来了,各自随便坐下。先有丫鬟挨人递了茶。大家吃毕,
凤姐手里拿着西洋布手巾,裹着一把乌木三镶银箸,按席罢下。贾母因说:“把那
一张小楠木桌子抬过来,让刘亲家挨着我这边坐。”众人听说,忙抬过来。凤姐一
面递眼色与鸳鸯,鸳鸯便忙拉刘老老出去,悄悄的嘱咐了刘老老一席话,又说:“这
是我们家的规矩,要错了,我们就笑话呢。”调停已毕,然后归坐。薛姨妈是吃过
饭来的,不吃了,只坐在一边吃茶。贾母带着宝玉、湘云、黛玉、宝钗一桌,王夫
人带着迎春姐妹三人一桌,刘老老挨着贾母一桌。贾母素日吃饭,皆有小丫鬟在旁
边拿着漱盂、麈尾、巾帕之物,如今鸳鸯是不当这差的了,今日偏接过麈尾来拂着。
丫鬟们知他要捉弄刘老老,便躲开让他。鸳鸯一面侍立,一面递眼色。刘老老道:
“姑娘放心。”

那刘老老入了坐,拿起箸来,甸甸的不伏手,原是凤姐和鸳鸯商议定了,单
拿了一双老年四楞象牙镶金的筷子给刘老老。刘老老见了,说道:“这个叉巴子,
比我们那里的铁锨还,那里拿的动他?”说的众人都笑起来。只见一个媳妇端了
一个盒子站在当地,一个丫鬟上来揭去盒盖,里面盛着两碗菜,李纨端了一碗放在
贾母桌上,凤姐偏拣了一碗鸽子蛋放在刘老老桌上。贾母这边说声“请”,刘老老
便站起身来,高声说道:“老刘,老刘,食量大如牛。吃个老母猪,不抬头!”说
完,却鼓着腮帮子,两眼直视,一声不语。众人先还发怔,后来一想,上上下下都
一齐哈哈大笑起来。湘云掌不住,一口茶都喷出来。黛玉笑岔了气,伏着桌子只叫
“嗳哟”。宝玉滚到贾母怀里,贾母笑的搂着叫“心肝”。王夫人笑的用手指着凤
姐儿,却说不出话来。薛姨妈也掌不住,口里的茶喷了探春一裙子。探春的茶碗都
合在迎春身上。惜春离了坐位,拉着他奶母,叫“揉揉肠子”。地下无一个不弯腰
屈背,也有躲出去蹲着笑去的,也有忍着笑上来替他姐妹换衣裳的。独有凤姐鸳鸯
二人掌着,还只管让刘老老。

刘老老拿起箸来,只觉不听使,又道:“这里的鸡儿也俊,下的这蛋也小巧,
怪俊的。我且得一个儿!”众人方住了笑,听见这话,又笑起来。贾母笑的眼泪出
来只忍不住,琥珀在后捶着。贾母笑道:“这定是凤丫头促狭鬼儿闹的!快别信他
的话了。”那刘老老正夸鸡蛋小巧,凤姐儿笑道:“一两银子一个呢!你快尝尝罢,
冷了就不好吃了。”刘老老便伸筷子要夹,那里夹的起来?满碗里闹了一阵,好容
易撮起一个来,才伸着脖子要吃,偏又滑下来,滚在地下。忙放下筷子要亲自去拣,
早有地下的人拣出去了。刘老老叹道:“一两银子,也没听见个响声儿就没了!”

众人已没心吃饭,都看着他取笑。贾母又说:“谁这会子又把那个筷子拿出来
了,又不请客摆大筵席!都是凤丫头支使的,还不换了呢。”地下的人原不曾预备
这牙箸,本是凤姐和鸳鸯拿了来的,听如此说,忙收过去了,也照样换上一双乌木
镶银的。刘老老道:“去了金的,又是银的,到底不及俺们那个伏手。”凤姐儿道:
“菜里要有毒,这银子下去了就试的出来。”刘老老道:“这个菜里有毒,我们那
些都成了砒霜了!那怕毒死了,也要吃尽了。”贾母见他如此有趣,吃的又香甜,
把自己的菜也都端过来给他吃。又命一个老嬷嬷来,将各样的菜给板儿夹在碗上。

一时吃毕,贾母等都往探春卧室中去闲话,这里收拾残桌,又放了一桌。刘老
老看着李纨与凤姐儿对坐着吃饭,叹道:“别的罢了,我只爱你们家这行事!怪道
说,‘礼出大家’。”凤姐儿忙笑道:“你可别多心,才刚不过大家取乐儿。”一
言未了,鸳鸯也进来笑道:“老老别恼,我给你老人家赔个不是儿罢。”刘老老忙
笑道:“姑娘说那里的话?咱们哄着老太太开个心儿,有什么恼的!你先嘱咐我,我
就明白了,不过大家取笑儿。我要恼,也就不说了。”鸳鸯便骂人:“为什么不倒
茶给老老吃!”刘老老忙道:“才刚那个嫂子倒了茶来,我吃过了,姑娘也该用饭
了。”凤姐儿便拉鸳鸯坐下道:“你和我们吃罢,省了回来又闹。”鸳鸯便坐下了,
婆子们添上碗箸来,三人吃毕。刘老老笑道:“我看你们这些人,都只吃这一点儿
就完了,亏你们也不饿。怪道风儿都吹的倒!”鸳鸯便问:“今儿剩的不少,都那
里去了?”婆子们道:“都还没散呢,在这里等着,一齐散给他们吃。”鸳鸯道:
“他们吃不了这些,挑两碗给二奶奶屋里平丫头送去。”凤姐道:“他早吃了饭了,
不用给他。”鸳鸯道:“他吃不了,喂你的猫。”婆子听了,忙拣了两样,拿盒子
送去。鸳鸯道:“素云那里去了?”李纨道:“他们都在这里一处吃,又找他做什
么?”鸳鸯道:“这就罢了。”凤姐道:“袭人不在这里,你倒是叫人送两样给他
去。”鸳鸯听说,便命人也送两样去。鸳鸯又问婆子们:“回来吃酒的攒盒,可装
上了?”婆子道:“想必还得一会子。”鸳鸯道:“催着些儿。”婆子答应了。

凤姐等来至探春房中,只见他娘儿们正说笑。探春素喜阔朗,这三间屋子并不
曾隔断,当地放着一张花梨大理石大案,案上堆着各种名人法帖,并数十方宝砚,
各色笔筒,笔海内插的笔如树林一般。那一边设着斗大的一个汝窑花囊,插着满满
的一囊水晶球的白菊。西墙上当中挂着一大幅米襄阳《烟雨图》。左右挂着一副对
联,乃是颜鲁公墨迹。其联云:
烟霞闲骨格,
泉石野生涯。
案上设着大鼎,左边紫檀架上放着一个大官窑的大盘,盘内盛着数十个娇黄玲珑大
佛手。右边洋漆架上悬着一个白玉比目磬,傍边挂着小槌。那板儿略熟了些,便要
摘那槌子去击,丫鬟们忙拦住他。他又要那佛手吃,探春拣了一个给他,说:“玩
罢,吃不得的。”东边便设着卧榻拔步床,上悬着葱绿双绣花卉草虫的纱帐。板儿
又跑来看,说:“这是蝈蝈,这是蚂蚱。”刘老老忙打了他一巴掌,道:“下作黄
子!没干没净的乱闹。倒叫你进来瞧瞧,就上脸了!”打的板儿哭起来,众人忙劝
解方罢。

贾母隔着纱窗后往院内看了一回,因说道:“后廊檐下的梧桐也好了,只是细
些。”正说话,忽一阵风过,隐隐听得鼓乐之声。贾母问:“是谁家娶亲呢?这里
临街倒近。”王夫人等笑回道:“街上的那里听的见?这是咱们的那十来个女孩子
们演习吹打呢。”贾母便笑道:“既他们演,何不叫他们进来演习,他们也逛一逛,
咱们也乐了,不好吗?”凤姐听说,忙命人出去叫来,赶着吩咐摆下条桌,铺上红
毡子。贾母道:“就铺排在藕香榭的水亭子上,借着水音更好听。回来咱们就在缀
锦阁底下吃酒,又宽阔,又听的近。”众人都说好。贾母向薛姨妈笑道:“咱们走
罢,他们姐妹们都不大喜欢人来,生怕腌了屋子。咱们别没眼色儿,正经坐会子
船,喝酒去罢。”说着,大家起身便走。探春笑道:“这是那里的话?求着老太太、
姨妈、太太来坐坐还不能呢!”贾母笑道:“我的这三丫头倒好,只有两个玉儿可
恶。回来喝醉了,咱们偏往他们屋里闹去!”说着众人都笑了。

一齐出来走不多远,已到了荇叶渚,那姑苏选来的几个驾娘早把两只棠木舫撑
来。众人扶了贾母,王夫人、薛姨妈、刘老老、鸳鸯、玉钏儿上了这一只船,次后
李纨也跟上去。凤姐也上去,立在船头上,也要撑船。贾母在舱内道:“那不是玩
的!虽不是河里,也有好深的,你快给我进来。”凤姐笑道:“怕什么!老祖宗只管
放心。”说着,便一篙点开,到了池当中。船小人多,凤姐只觉乱晃,忙把篙子递
与驾娘,方蹲下去。然后迎春姐妹等并宝玉上了那只,随后跟来。其馀老嬷嬷众丫
鬟俱沿河随行。宝玉道:“这些破荷叶可恨,怎么还不叫人来拔去?”宝钗笑道:
“今年这几日,何曾饶了这园子闲了一闲,天天逛,那里还有叫人来收拾的工夫
呢?”黛玉道:“我最不喜欢李义山的诗,只喜他这一句:‘留得残荷听雨声。’
偏你们又不留着残荷了。”宝玉道:“果然好句,以后咱们别叫拔去了。”

说着已到了花溆的萝港之下,觉得阴森透骨,两滩上衰草残菱,更助秋兴。贾
母因见岸上的清厦旷朗,便问:“这是薛姑娘的屋子不是?”众人道:“是。”贾
母忙命拢岸,顺着云步石梯上去,一同进了蘅芜院。只觉异香扑鼻,那些奇草仙藤,
愈冷愈苍翠,都结了实,似珊瑚豆子一般,累垂可爱。及进了房屋,雪洞一般,一
色的玩器全无。案上止有一个土定瓶,瓶中供着数枝菊,并两部书,茶奁、茶杯而
已。床上只吊着青纱帐幔,衾褥也十分朴素。贾母叹道:“这孩子太老实了!你没
有陈设,何妨和你姨娘要些?我也没理论,也没想到。你们的东西,自然在家里没
带了来。”说着,命鸳鸯去取些古董来,又嗔着凤姐儿:“不送些玩器来给你妹妹,
这样小器!”王夫人凤姐等都笑回说:“他自己不要么,我们原送了来,都退回去
了。”薛姨妈也笑说道:“他在家里也不大弄这些东西。”贾母摇头道:“那使不
得。虽然他省事,倘或来个亲戚,看着不像,二则年轻的姑娘们,屋里这么素净,
也忌讳。我们这老婆子,越发该住马圈去了。你们听那些书上戏上说的小姐们的绣
房,精致的还了得呢!他们姐妹们虽不敢比那些小姐们,也别很离了格儿。有现成
的东西,为什么不摆呢?要很爱素净,少几样倒使得。我最会收拾屋子,如今老了,
没这个闲心了。他们姐妹们也还学着收拾的好。只怕俗气,有好东西也摆坏了。我
看他们还不俗。如今等我替你收拾,包管又大方又素净。我的两件体己,收到如今,
没给宝玉看见过,若经了他的眼也没了。”说着,叫过鸳鸯来,吩咐道:“你把那
石头盆景儿和那架纱照屏,还有个墨烟冻石鼎拿来:这三样摆在这案上就够了。再
把那水墨字画白绫帐子拿来,把这帐子也换了。”鸳鸯答应着,笑道:“这些东西
都搁在东楼上不知那个箱子里,还得慢慢找去,明儿再拿去也罢了。”贾母道:“明
日后日都使得,只别忘了。”

说着,坐了一回,方出来,一径来至缀锦阁下。文官等上来请过安,因问:“演
习何曲?”贾母道:“只拣你们熟的演习几套罢。”文官等下来,往藕香榭去不提。
这里凤姐已带着人摆设齐整,上面左右两张榻,榻上都铺着锦蓉簟,每一榻前两
张雕漆几,也有海棠式的,也有梅花式的,也有荷叶式的,也有葵花式的,也有方
的,有圆的,其式不一。一个上头放着一分炉瓶,一个攒盒。上面二榻四几,是贾
母薛姨妈;下面一椅两几,是王夫人的。馀者都是一椅一几。东边刘老老,刘老老
之下便是王夫人。西边便是湘云,第二便是宝钗,第三便是黛玉,第四迎春,探春
惜春挨次排下去,宝玉在末。李纨凤姐二人之几设于三层槛内、二层纱厨之外。攒
盒式样,亦随几之式样。每人一把乌银洋錾自斟壶,一个十锦珐琅杯。

大家坐定,贾母先笑道:“咱们先吃两杯,今日也行一个令,才有意思。”薛
姨妈笑说道:“老太太自然有好酒令,我们如何会呢!安心叫我们醉了。我们都多
吃两杯就有了。”贾母笑道:“姨太太今儿也过谦起来,想是厌我老了。”薛姨妈
笑道:“不是谦,只怕行不上来,倒是笑话了。”王夫人忙笑道:“便说不上来,
只多吃了一杯酒,醉了睡觉去,还有谁笑话咱们不成。”薛姨妈点头笑道:“依令。
老太太到底吃一杯令酒才是。”贾母笑道:“这个自然。”说着便吃了一杯。凤姐
儿忙走至当地,笑道:“既行令,还叫鸳鸯姐姐来行才好。”众人都知贾母所行之
令,必得鸳鸯提着,故听了这话都说很是。凤姐便拉着鸳鸯过来。王夫人笑道:“既
在令内,没有站着的理。”回头命小丫头子:“端一张椅子,放在你二位奶奶的席
上。”鸳鸯也半推半就,谢了坐便坐下,也吃了一钟酒,笑道:“酒令大如军令。
不论尊卑,惟我是主,违了我的话,是要受罚的。”王夫人等都笑道:“一定如此,
快些说。”鸳鸯未开口,刘老老便下席,摆手道:“别这样捉弄人!我家去了。”
众人都笑道:“这却使不得。”鸳鸯喝令小丫头子们:“拉上席去!”小丫头子们
也笑着,果然拉入席中。刘老老只叫:“饶了我罢!”鸳鸯道:“再多言的罚一壶。”
刘老老方住了。

鸳鸯道:“如今我说骨牌副儿,从老太太起,顺领下去,至刘老老止。比如我
说一副儿,将这三张牌拆开,先说头一张,再说第二张,说完了,合成这一副儿的
名字,无论诗词歌赋,成语俗话,比上一句,都要合韵。错了的罚一杯。”众人笑
道:“这个令好,就说出来。”

鸳鸯道:“有了一副了。左边是张天。”贾母道:“头上有青天。”众人道好。
鸳鸯道:“当中是个五合六。”贾母道:“六桥梅花香彻骨。”鸳鸯道:“剩了一
张六合么。”贾母道:“一轮红日出云霄。”鸳鸯道:“凑成却是个‘蓬头鬼’。”
贾母道:“这鬼抱住钟馗腿。”说完,大家笑着喝彩。贾母饮了一杯。

鸳鸯又道:“又有一副了。左边是个大长五。”薛姨妈道:“梅花朵朵风前舞。”
鸳鸯道:“右边是个大五长。”薛姨妈道:“十月梅花岭上香。”鸳鸯道:“当中
二五是杂七。”薛姨妈道:“织女牛郎会七夕。”鸳鸯道:“凑成‘二郎游五岳’。”
薛姨妈道:“世人不及神仙乐。”说完,大家称赏,饮了酒。

鸳鸯又道:“有了一副了。左边长么两点明。”湘云道:“双悬日月照乾坤。”
鸳鸯道:“右边长么两点明。”湘云道:“闲花落地听无声。”鸳鸯道:“中间还
得么四来。”湘云道:“日边红杏倚云栽。”鸳鸯道:“凑成一个‘樱桃九熟’。”
湘云道:“御园却被鸟衔出。”说完,饮了一杯。

鸳鸯道:“有了一副了。左边是长三。”宝钗道:“双双燕子语梁间。”鸳鸯
道:“右边是三长。”宝钗道:“水荇牵风翠带长。”鸳鸯道:“当中三六九点在。”
宝钗道:“三山半落青天外。”鸳鸯道:“凑成‘铁练锁孤舟’。”宝钗道:“处
处风波处处愁。”说完饮毕。

鸳鸯又道:“左边一个天。”黛玉道:“良辰美景奈何天。”宝钗听了,回头
看着他,黛玉只顾怕罚,也不理论。鸳鸯道:“中间锦屏颜色俏。”黛玉道:“纱
窗也没有红娘报。”鸳鸯道:“剩了二六八点齐。”黛玉道:“双瞻玉座引朝仪。”
鸳鸯道:“凑成‘篮子’好采花。”黛玉道:“仙杖香挑芍药花。”说完,饮了一
口。

鸳鸯道:“左边四五成花九。”迎春道:“桃花带雨浓。”众人笑道:“该罚!
错了韵,而且又不像。”迎春笑着,饮了一口。

原是凤姐和鸳鸯都要听刘老老的笑话儿,故意都叫说错了。至王夫人,鸳鸯便
代说了一个,下便该刘老老。刘老老道:“我们庄家闲了,也常会几个人弄这个儿,
可不像这么好听就是了。少不得我也试试。”众人都笑道:“容易的,你只管说,
不相干。”鸳鸯笑道:“左边大四是个人。”刘老老听了,想了半日,说道:“是
个庄家人罢!”众人哄堂笑了。贾母笑道:“说的好,就是这么说。”刘老老也笑
道:“我们庄家人不过是现成的本色儿,姑娘姐姐别笑。”鸳鸯道:“中间三四绿
配红。”刘老老道:“大火烧了毛毛虫。”众人笑道:“这是有的,还说你的本色。”
鸳鸯笑道:“右边么四真好看。”刘老老道:“一个萝卜一头蒜。”众人又笑了。
鸳鸯笑道:“凑成便是‘一枝花’。”刘老老两只手比着,也要笑,却又掌住了,
说道:“花儿落了结个大倭瓜。”众人听了,由不的大笑起来。

只听外面乱嚷嚷的,不知何事,且听下回分解。