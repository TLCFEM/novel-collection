\chapter{惑奸谗抄检大观园~避嫌隙杜绝宁国府}

话说平儿听迎春说了,正自好笑,忽见宝玉也来了。原来管厨房柳家媳妇的妹
子也因放头开赌,得了不是,因这园中有素和柳家的不好的,便又告出柳家的来,
说和他妹子是伙计,赚了平分。因此凤姐要治柳家之罪。那柳家的听得此言,便慌
了手脚,因思素与怡红院的人最为深厚,故走来悄悄的央求晴雯芳官等人,转告诉
了宝玉。宝玉因思内中迎春的嬷嬷也现有此罪,不若来约同迎春去讨情,比自己独
去单为柳家的说情又更妥当,故此前来。忽见许多人在此,见他来时,都问道:“你
的病可好了?跑来做什么?”宝玉不便说出讨情一事,只说:“来看二姐姐。”当
下众人也不在意,且说些闲话。

平儿便出去办累金凤一事。那玉柱儿媳妇紧跟在后,口内百般央求,只说:“姑
娘好歹口内超生,我横竖去赎了来。”平儿笑道:“你迟也赎,早也赎,‘既有今
日,何必当初’。你的意思得过就过,既这么样,我也不好意思告诉人。趁早儿取
了来,交给我,一字不提。”玉柱儿媳妇听说,方放下心来,就拜谢,又说:“姑
娘自去贵干。赶晚赎了来,先回了姑娘再送去如何?”平儿道:“赶晚不来,可别
怨我!”说毕,二人方分路各自散了。平儿到房,凤姐问他:“三姑娘叫你做什么?”
平儿笑道:“三姑娘怕奶奶生气,叫我劝着奶奶些,问奶奶这两天可吃些什么?”
凤姐笑道:“倒是他还惦记我。刚才又出来了一件事:有人来告柳二媳妇和他妹子
通同开局,凡妹子所为都是他作主。我想你素日肯劝我多一事不如少一事,自己保
养保养也是好的。我因听不进去,果然应了,先把太太得罪了,而且反赚了一场病。
如今我也看破了,随他们闹去罢,横竖还有许多人呢。我白操一会子心,倒惹的万
人咒骂,不如且自家养养病。就是病好了,我也会做好好先生,得乐且乐,得笑且
笑,一概是非都凭他们去罢,所以我只答应着‘知道了’。”平儿笑道:“奶奶果
然如此,那就是我们的造化了。”

一语未了,只见贾琏进来,拍手叹气道:“好好的又生事!前儿我和鸳鸯借当,
那边太太怎么知道了?刚才太太叫过我去,叫我不管那里先借二百银子,做八月十
五节下使用。我回没处借,太太就说:‘你没有钱就有地方挪移,我白和你商量,
你就搪塞我!你就没地方儿!前儿一千银子的当是那里的?连老太太的东西你都有神
通弄出来,这会二百银子你就这样难。亏我没和别人说去!’我想太太分明不短,
何苦来又寻事奈何人!”凤姐儿道:“那日并没个外人,谁走了这个消息?”平儿
听了,也细想那日有谁在此,想了半日,笑道:“是了。那日说话时没人,就只晚
上送东西来的时候儿,老太太那边傻大姐的娘可巧来送浆洗衣裳,他在下房里坐了
一会子,看见一大箱子东西,自然要问。必是丫头们不知道,说出来了,也未可知。”
因此便唤了几个小丫头来问:“那日谁告诉傻大姐的娘了?”众小丫头慌了,都跪
下赌神发誓说:“自来也没敢多说一句话。有人凡问什么,都答应不知道,这事如
何敢说!”凤姐详情度理,说:“他们必不敢多说一句话,倒别委屈了他们。如今
把这事靠后,且把太太打发了去要紧。宁可咱们短些,别又讨没意思。”因叫平儿:
“把我的金首饰再去押二百银子来,送去完事。”贾琏道:“索性多押二百,咱们
也要使呢。”凤姐道:“很不必,我没处使。这不知还指那一项赎呢。”平儿拿了
去,吩咐旺儿媳妇领去。不一时拿了银子来,贾琏亲自送去,不在话下。

这里凤姐和平儿猜疑走风的人:“反叫鸳鸯受累,岂不是咱们之过!”正在胡
想,人报:“太太来了。”凤姐听了诧异,不知何事,遂与平儿等忙迎出来。只见
王夫人气色更变,只带一个贴己小丫头走来,一语不发,走至里间坐下。凤姐忙捧
茶,因陪笑问道:“太太今日高兴,到这里逛逛?”王夫人喝命:“平儿出去!”
平儿见了这般,不知怎么了,忙应了一声,带着众小丫头一齐出去,在房门外站住。
一面将房门掩了,自己坐在台阶上,所有的人一个不许进去。凤姐也着了慌,不知
有何事。只见王夫人含着泪,从袖里扔出一个香袋来,说:“你瞧!”凤姐忙拾起
一看,见是十锦春意香袋,也吓了一跳,忙问:“太太从那里得来?”王夫人见问,
越发泪如雨下,颤声说道:“我从那里得来?我天天坐在井里!想你是个细心人,所
以我才偷空儿,谁知你也和我一样!这样东西,大天白日,明摆在园里山石上,被
老太太的丫头拾着。不亏你婆婆看见,早已送到老太太跟前去了。我且问你:这个
东西如何丢在那里?”凤姐听得,也更了颜色,忙问:“太太怎么知道是我的?”
王夫人又哭又叹道:“你反问我?你想,一家子除了你们小夫小妻,馀者老婆子们,
要这个何用?女孩子们是从那里得来?自然是那琏儿不长进下流种子那里弄来的。你
们又和气,当作一件玩意儿。年轻的人,儿女闺房私意是有的,你还和我赖!幸而
园内上下人还不解事,尚未拣得,倘或丫头们拣着,你姊妹看见,这还了得?不然,
有那小丫头们拣着出去,说是园内拣的,外人知道,这性命脸面要也不要?”

凤姐听说,又急又愧,登时紫胀了面皮,便挨着炕沿双膝跪下,也含泪诉道:
“太太说的固然有理,我也不敢辩。但我并无这样东西,其中还要求太太细想:这
香袋儿是外头仿着内工绣的,连穗子一概都是市卖的东西。我虽年轻不尊重,也不
肯要这样东西。再者,这也不是常带着的,我纵然有,也只好在私处搁着,焉肯在
身上常带,各处逛去?况且又在园里去,个个姊妹,我们都肯拉拉扯扯,倘或露出
来,不但在姊妹前看见,就是奴才看见,我有什么意思?三则论主子内我是年轻媳
妇,算起来,奴才比我更年轻的又不止一个了,况且他们也常在园走动,焉知不是
他们掉的?再者,除我常在园里,还有那边太太常带过几个小姨娘来,嫣红翠云那
几个人也都是年轻的人,他们更该有这个了。还有那边珍大嫂子,他也不算很老,
也常带过佩凤他们来,又焉知又不是他们的?况且园内丫头也多,保不住都是正经
的。或者年纪大些的知道了人事,一刻查问不到,偷出去了,或借着因由合二门上
小么儿们打牙撂嘴儿,外头得了来的,也未可知。不但我没此事,就连平儿,我也
可以下保的:太太请细想。”

王夫人听了这一席话,很近情理,因叹道:“你起来。我也知道你是大家子的
姑娘出身,不至这样轻薄,不过我气激你的话。但只如今且怎么处?你婆婆才打发
人封了这个给我瞧,把我气了个死。”凤姐道:“太太快别生气。若被众人觉察了,
保不定老太太不知道。且平心静气,暗暗访察,才能得这个实在;纵然访不着,外
人也不能知道。如今惟有趁着赌钱的因由革了许多人这空儿,把周瑞媳妇、旺儿媳
妇等四五个贴近不能走话的人,安插在园里,以查赌为由。再如今他们的丫头也太
多了,保不住人大心大,生事作耗,等闹出来,反悔之不及。如今若无故裁革,不
但姑娘们委屈,就连太太和我也过不去。不如趁着这个机会,以后凡年纪大些的,
或有些磨牙难缠的,拿个错儿撵出去,配了人:一则保的住没有别事,二则也可省
些用度。太太想我这话如何?”王夫人叹道:“你说的何尝不是。但从公细想,你
这几个姊妹,每人只有两三个丫头像人,馀者竟是小鬼儿似的。如今再去了,不但
我心里不忍,只怕老太太未必就依。虽然艰难,也还穷不至此。我虽没受过大荣华,
比你们是强些,如今宁可省我些,别委屈了他们。你如今且叫人传周瑞家的等人进
来,就吩咐他们快快暗访这事要紧!”

凤姐即唤平儿进来,吩咐出去。一时,周瑞家的与吴兴家的、郑华家的、来旺
家的、来喜家的现在五家陪房进来。王夫人正嫌人少,不能勘察,忽见邢夫人的陪
房王善保家的走来,正是方才是他送香袋来的。王夫人向来看视邢夫人之得力心腹
人等原无二意,今见他来打听此事,便向他说:“你去回了太太,也进园来照管照
管,比别人强些。”王善保家的因素日进园去,那些丫鬟们不大趋奉他,他心里不
自在,要寻他们的故事又寻不着,恰好生出这件事来,以为得了把柄;又听王夫人
委托他,正碰在心坎上,道:“这个容易。不是奴才多话,论理这事该早严紧些的。
太太也不大往园里去,这些女孩子们,一个个倒像受了诰封似的,他们就成了千金
小姐了。闹下天来,谁敢哼一声儿。不然,就调唆姑娘们,说欺负了姑娘们了,谁
还耽得起!”王夫人点头道:“跟姑娘们的丫头比别的娇贵些,这也是常情。”王
善保家的道:“别的还罢了,太太不知,头一个是宝玉屋里的晴雯那丫头,仗着他
的模样儿比别人标致些,又长了一张巧嘴,天天打扮的像个西施样子,在人跟前能
说惯道,抓尖要强。一句话不投机,他就立起两只眼睛来骂人。妖妖调调,大不成
个体统。”王夫人听了这话,猛然触动往事,便问凤姐道:“上次我们跟了老太太
进园逛去,有一个水蛇腰,削肩膀儿,眉眼又有些像你林妹妹的,正在那里骂小丫
头,我心里很看不上那狂样子。因同老太太走,我不曾说他;后来要问是谁,偏又
忘了。今日对了槛儿,这丫头想必就是他了?”凤姐道:“若论这些丫头们,共总
比起来,都没晴雯长得好。论举止言语,他原轻薄些。方才太太说的倒很像他,我
也忘了那日的事,不敢混说。”王善保家的便道:“不用这样,此刻不难叫了他来,
太太瞧瞧。”王夫人道:“宝玉屋里常见我的,只有袭人麝月,这两个笨笨的倒好。
要有这个,他自然不敢来见我呀。我一生最嫌这样的人,且又出来这个事。好好的
宝玉倘或叫这蹄子勾引坏了,那还了得。”因叫自己的丫头来,吩咐他道:“你去,
只说我有话问他,留下袭人麝月伏侍宝玉,不必来;有一个晴雯最伶俐,叫他即刻
快来。你不许和他说什么!”

小丫头答应了,走入怡红院,正值晴雯身上不好,睡中觉才起来,发闷呢,听
如此说,只得跟了他来。素日晴雯不敢出头,因连日不自在,并没十分妆饰,自为
无碍。及到了凤姐房中,王夫人一见他钗鬓松,衫垂带褪,大有春睡捧心之态,
而且形容面貌恰是上月的那人,不觉勾起方才的火来。王夫人便冷笑道:“好个美
人儿,真像个‘病西施’了。你天天作这轻狂样儿给谁看!你干的事,打量我不知
道呢。我且放着你,自然明儿揭你的皮!——宝玉今日可好些?”晴雯一听如此说,
心内大异,便知有人暗算了他,虽然着恼,只不敢作声。他本是个聪明过顶的人,
见问宝玉可好些,他便不肯以实话答应,忙跪下回道:“我不大到宝玉房里去,又
不常和宝玉在一处,好歹我不能知,那都是袭人合麝月两个人的事,太太问他们。”
王夫人道:“这就该打嘴。你难道是死人?要你们做什么?”晴雯道:“我原是跟
老太太的人,因老太太说园里空大,人少,宝玉害怕,所以拨了我去外间屋里上夜,
不过看屋子。我原回过我笨,不能伏侍,老太太骂了我,‘又不叫你管他的事,要
伶俐的做什么?’我听了不敢不去,才去的。不过十天半月之内,宝玉叫着了,答
应几句话,就散了。至于宝玉的饮食起居,上一层有老奶奶老妈妈们,下一层有袭
人、麝月、秋纹几个人。我闲着还要做老太太屋里的针线,所以宝玉的事竟不曾留
心。太太既怪,从此后我留心就是了。”王夫人信以为实了,忙说:“阿弥陀佛!
你不近宝玉,是我的造化。竟不劳你费心!既是老太太给宝玉的,我明儿回了老太
太再撵你!”因向王善保家的道:“你们进去,好生防他几日,不许他在宝玉屋里
睡觉,等我回过老太太,再处治他。”喝声:“出去!站在这里,我看不上这浪样
儿!谁许你这么花红柳绿的妆扮!”晴雯只得出来。这气非同小可,一出门,便拿
绢子握着脸,一头走,一头哭,直哭到园内去。

这里王夫人向凤姐等自怨道:“这几年我越发精神短了,照顾不到,这样妖精
似的东西竟没看见!只怕这样的还有,明日倒得查查。”凤姐见王夫人盛怒之际,
又因王善保家的是邢夫人的耳目,常时调唆的邢夫人生事,纵有千百样言语,此刻
也不敢说,只低头答应着。王善保家的道:“太太且请息怒。这些事小。只交与奴
才。如今要查这个是极容易的。等到晚上园门关了的时节,内外不通风,我们竟给
他们个冷不防,带着人到各处丫头们房里搜寻。想来谁有这个,断不单有这个,自
然还有别的。那时翻出别的来,自然这个也是他的了。”王夫人道:“这话倒是。
若不如此,断乎不能明白。”因问凤姐:“如何?”凤姐只得答应说:“太太说是,
就行罢了。”王夫人道:“这主意很是,不然一年也查不出来。”于是大家商议已
定。

至晚饭后,待贾母安寝了,宝钗等入园时,王家的便请了凤姐一并进园,喝命
将角门皆上锁,便从上夜的婆子处来抄检起。不过抄检些多馀攒下蜡烛灯油等物。
王善保家的道:“这也是赃,不许动的,等明日回过太太再动。”于是先就到怡红
院中,喝命关门。当下宝玉正因晴雯不自在,忽见这一干人来,不知为何直扑了丫
头们的房门去。因迎出凤姐来,问是何故。凤姐道:“丢了一件要紧的东西,因大
家混赖,恐怕有丫头们偷了,所以大家都查一查,去疑儿。”一面说,一面坐下吃
茶。王家的等搜了一回,又细问:“这几个箱子是谁的?”都叫本人来亲自打开。
袭人因见晴雯这样,必有异事,又见这番抄检,只得自己先出来打开了箱子并匣子,
任其搜检一番,不过平常通用之物。随放下又搜别人的,挨次都一一搜过。到晴雯
的箱子,因问:“是谁的?怎么不打开叫搜?”袭人方欲替晴雯开时,只见晴雯挽
着头发闯进来,啷一声将箱子掀开,两手提着底子往地下一倒,将所有之物尽都
倒出来。王善保家的也觉没趣儿,便紫胀了脸,说道:“姑娘你别生气。我们并非
私自就来的,原是奉太太的命来搜察,你们叫翻呢,我们就翻一翻,不叫翻,我们
还许回太太去呢。那用急的这个样子!”晴雯听了这话,越发火上浇油,便指着他
的脸说道:“你说你是太太打发来的,我还是老太太打发来的呢!太太那边的人我
也都见过,就只没看见你这么个有头有脸大管事的奶奶!”凤姐见晴雯说话锋利尖
酸,心中甚喜,却碍着邢夫人的脸,忙喝住晴雯。那王善保家的又羞又气,刚要还
言,凤姐道:“妈妈,你也不必和他们一般见识,你且细细搜你的,咱们还到各处
走走呢。再迟了走了风,我可担不起。”王善保家的只得咬咬牙,且忍了这口气,
细细的看了一看,也无甚私弊之物。回了凤姐,要别处去,凤姐道:“你可细细的
查,若这一番查不出来,难回话的。”众人都道:“尽都细翻了,没有什么差错东
西。虽有几样男人物件,都是小孩子的东西,想是宝玉的旧物,没甚关系的。”凤
姐听了,笑道:“既如此,咱们就走,再瞧别处去。”

说着,一径出来,向王善保家的道:“我有一句话,不知是不是:要抄检只抄
检咱们家的人,薛大姑娘屋里,断乎抄检不得的。”王善保家的笑道:“这个自然,
岂有抄起亲戚家来的。”凤姐点头道:“我也这样说呢。”一头说,一头到了潇湘
馆内。黛玉已睡了,忽报这些人来,不知为甚事。才要起来,只见凤姐已走进来,
忙按住他不叫起来,只说:“睡着罢,我们就走的。”这边且说些闲话。那王善保
家的带了众人到了丫鬟房中,也一一开箱倒笼抄检了一番,因从紫鹃房中搜出两副
宝玉往常换下来的寄名符儿,一副束带上的帔带,两个荷包并扇套,套内有扇子,
打开看时,皆是宝玉往日手内曾拿过的。王善保家的自为得了意,遂忙请凤姐过来
验视,又说:“这些东西从那里来的?”凤姐笑道:“宝玉和他们从小儿在一处混
了几年,这自然是宝玉的旧东西。况且这符儿合扇子,都是老太太和太太常见的。
妈妈不信,咱们只管拿了去。”王家的忙笑道:“二奶奶既知道就是了。”凤姐道:
“这也不是什么稀罕事,撂下再往别处去是正经。”紫鹃笑道:“直到如今,我们
两下里的账也算不清,要问这一个,连我也忘了是那年月日有的了。”

这里凤姐合王善保家的又到探春院内。谁知早有人报与探春了。探春也就猜着
必有原故,所以引出这等丑态来,遂命众丫鬟秉烛开门而待。一时众人来了,探春
故问:“何事?”凤姐笑道:“因丢了一件东西,连日访察不出人来,恐怕旁人赖
这些女孩子们。所以大家搜一搜,使人去疑儿,倒是洗净他们的好法子。”探春笑
道:“我们的丫头自然都是些贼,我就是头一个窝主。既如此,先来搜我的箱柜,
他们所偷了来的,都交给我藏着呢。”说着,便命丫鬟们把箱一齐打开,将镜奁、
妆盒、衾袱、衣包若大若小之物,一齐打开,请凤姐去抄阅。凤姐陪笑道:“我不
过是奉太太的命来,妹妹别错怪了我。”因命丫鬟们:“快快给姑娘关上。”平儿
丰儿等先忙着替侍书等关的关,收的收。探春道:“我的东西倒许你们搜阅,要想
搜我的丫头这可不能。我原比众人歹毒,凡丫头所有的东西,我都知道,都在我这
里间收着:一针一线,他们也没得收藏。要搜,所以只来搜我。你们不依,只管去
回太太,只说我违背了太太,该怎么处治,我去自领。——你们别忙,自然你们抄
的日子有呢!你们今日早起不是议论甄家,自己盼着好好的抄家,果然今日真抄了!
咱们也渐渐的来了!可知这样大族人家,若从外头杀来,一时是杀不死的。这可是
古人说的,‘百足之虫,死而不僵’,必须先从家里自杀自灭起来,才能一败涂地
呢!”说着,不觉流下泪来。凤姐只看着众媳妇们。周瑞家的便道:“既是女孩子
的东西全在这里,奶奶且请到别处去罢,也让姑娘好安寝。”凤姐便起身告辞。探
春道:“可细细搜明白了!若明日再来,我就不依了。”凤姐笑道:“既然丫头们
的东西都在这里,就不必搜了。”探春冷笑道:“你果然倒乖!连我的包袱都打开
了,还说没翻,明日敢说我护着丫头们,不许你们翻了。你趁早说明,若还要翻,
不妨再翻一遍。”凤姐知道探春素日与众不同的,只得陪笑道:“已经连你的东西
都搜察明白了。”探春又问众人:“你们也都搜明白了没有?”周瑞家的等都陪笑
说:“都明白了。”

那王善保家的本是个心内没成算的人,素日虽闻探春的名,他想众人没眼色、
没胆量罢了,那里一个姑娘就这样利害起来?况且又是庶出,他敢怎么着?自己又仗
着是邢夫人的陪房,连王夫人尚另眼相待,何况别人?只当是探春认真单恼凤姐,
与他们无干。他便要趁势作脸,因越众向前,拉起探春的衣襟,故意一掀,嘻嘻的
笑道:“连姑娘身上我都翻了,果然没有什么。”凤姐见他这样,忙说:“妈妈走
罢,别疯疯癫癫的——”一语未了,只听“拍”的一声,王家的脸上早着了探春一
巴掌。探春登时大怒,指着王家的问道:“你是什么东西,敢来拉扯我的衣裳!我
不过看着太太的面上,你又有几岁年纪,叫你一声‘妈妈’,你就狗仗人势,天天
作耗,在我们跟前逞脸。如今越发了不得了,你索性望我动手动脚的了!你打量我
是和你们姑娘那么好性儿,由着你们欺负?你就错了主意了!你来搜检东西我不恼,
你不该拿我取笑儿!”说着,便亲自要解钮子,拉着凤姐儿细细的翻,“省得叫你
们奴才来翻我!”

凤姐平儿等都忙与探春理裙整袂,口内喝着王善保家的说:“妈妈吃两口酒,
就疯疯癫癫起来,前儿把太太也冲撞了。快出去,别再讨脸了!”又忙劝探春:“好
姑娘,别生气。他算什么,姑娘气着倒值多了。”探春冷笑道:“我但凡有气,早
一头碰死了。不然,怎么许奴才来我身上搜贼赃呢!明儿一早,先回过老太太、太
太,再过去给大娘赔礼。该怎么着,我去领!”那王善保家的讨了个没脸,赶忙躲
出窗外,只说:“罢了,罢了!这也是头一遭挨打!我明儿回了太太,仍回老娘家去
罢,这个老命还要他做什么。”探春喝命丫鬟:“你们听着他说话,还等我和他拌
嘴去不成?”侍书听说,便出去说道:“妈妈,你知点道理儿,省一句儿罢。你果
然回老娘家去,倒是我们的造化了,只怕你舍不得去。你去了,叫谁讨主子的好儿,
调唆着察考姑娘、折磨我们呢?”凤姐笑道:“好丫头,真是有其主必有其仆。”
探春冷笑道:“我们做贼的人,嘴里都有三言两语的,就只不会背地里调唆主子!”
平儿忙也陪笑解劝,一面又拉了侍书进来。周瑞家的等人劝了一番,凤姐直待伏侍
探春睡下,方带着人往对过暖香坞来。

彼时李纨犹病在床上,他与惜春是紧邻,又和探春相近,故顺路先到这两处。
因李纨才吃了药睡着,不好惊动,只到丫鬟们房中,一一的搜了一遍,也没有什么
东西,遂到惜春房中来。因惜春年少,尚未识事,吓的不知当有什么事故,凤姐少
不得安慰他。谁知竟在入画箱中寻出一大包银锞子来,约共三四十个,为察奸情,
反得贼赃。又有一副玉带版子,并一包男人的靴袜等物。凤姐也黄了脸,因问:“是
那里来的?”入画只得跪下哭诉真情,说:“这是珍大爷赏我哥哥的。因我们老子
娘都在南方,如今只跟着叔叔过日子;我叔叔婶子只要喝酒赌钱,我哥哥怕交给他
们又花了,所以每常得了,悄悄的烦老妈妈带进来,叫我收着的。”惜春胆小,见
了这个,也害怕说:“我竟不知道,这还了得。二嫂子要打他,好歹带出他去打罢,
我听不惯的。”凤姐笑道:“若果真呢,也倒可恕,只是不该私自传送进来。这个
可以传递,怕什么不可传递?这倒是传递人的不是了。若这话不真,倘是偷来的,
你可就别想活了。”入画跪哭道:“我不敢撒谎,奶奶只管明日问我们奶奶和大爷
去,若说不是赏的,就拿我和我哥哥一同打死无怨。”凤姐道:“这个自然要问的。
只是真赏的,也有不是,谁许你私自传送东西呢?你且说是谁接的,我就饶你。下
次万万不可。”惜春道:“嫂子别饶他,这里人多,要不管了他,那些大的听见了
又不知怎么样呢。嫂子要依他,我也不依。”凤姐道:“素日我看他还使得。谁没
一个错,只这一次。二次再犯,两罪俱罚。但不知传递是谁?”惜春道:“若说传
递,再无别人,必是后门上的老张。他常和这些丫头们鬼鬼祟祟的,这些丫头们也
都肯照顾他。”凤姐听说,便命人记下,将东西且交给周瑞家的暂且拿着,等明日
对明再议。谁知那老张妈原和王善保家有亲,近因王善保家的在邢夫人跟前作了心
腹人,便把亲戚和伴儿们都看不到眼里了。后来张家的气不平,斗了两次口,彼此
都不说话了。如今王家的听见是他传递,碰在他心坎儿上,更兼刚才挨了探春的打,
受了侍书的气,没处发泄,听见张家的这事,因撺掇凤姐道:“这传东西的事关系
更大。想来那些东西,自然也是传递进来的。奶奶倒不可不问。”凤姐儿道:“我
知道,不用你说。”

于是别了惜春,方往迎春房内去。迎春已经睡着了,丫鬟们也才要睡,众人扣
门,半日才开。凤姐吩咐:“不必惊动姑娘。”遂往丫鬟们房里来。因司棋是王善
保家的外孙女儿,凤姐要看王家的可藏私不藏,遂留神看他搜检。先从别人箱子搜
起,皆无别物。及到了司棋箱中,随意掏了一回,王善保家的说:“也没有什么东
西。”才要关箱时,周瑞家的道:“这是什么话?有没有,总要一样看看才公道。”
说着,便伸手掣出一双男子的绵袜并一双缎鞋,又有一个小包袱。打开看时,里面
是一个同心如意,并一个字帖儿。一总递给凤姐。凤姐因理家久了,每每看帖看帐,
也颇识得几个字了。那帖是大红双喜笺,便看上面写道:

上月你来家后,父母已觉察了。但姑娘未出阁,尚不能完你我心愿。若园内可
以相见,你可托张妈给一信。若得在园内一见,倒比来家好说话。千万千万!再所
赐香珠二串,今已查收。外特寄香袋一个,略表我心。千万收好。表弟潘又安具。
凤姐看了,不由的笑将起来。那王善保家的素日并不知道他姑表兄妹有这一节风流
故事,见了这鞋袜,心内已有些毛病,又见有一红帖,凤姐看着笑,他便说道:“必
是他们写的帐不成字,所以奶奶见笑。”凤姐笑道:“正是这个帐竟算不过来!你
是司棋的老娘,你表弟也该姓王,怎么又姓潘呢?”王善保家的见问的奇怪,只得
勉强告道:“司棋的姑妈给了潘家,所以他姑表弟兄姓潘。上次逃走了的潘又安,
就是他。”凤姐笑道:“这就是了。”因说:“我念给你听听。”说着,从头念了
一遍,大家都吓一跳。这王家的一心只要拿人的错儿,不想反拿住了他外孙女儿,
又气又臊。周瑞家的四人听见凤姐儿念了,都吐舌头,摇头儿。周瑞家的道:“王
大妈听见了!这是明明白白,再没得话说了。这如今怎么样呢?”王家的只恨无地
缝儿可钻。凤姐只瞅着他,抿着嘴儿嘻嘻的笑,向周瑞家的道:“这倒也好。不用
他老娘操一点心儿,鸦雀不闻,就给他们弄了个好女婿来了。”周瑞家的也笑着凑
趣儿。王家的无处煞气,只好打着自己的脸骂道:“老不死的娼妇,怎么造下孽了?
说嘴打嘴,现世现报!”众人见他如此,要笑又不敢笑,也有趁愿的,也有心中感
动报应不爽的。

凤姐见司棋低头不语,也并无畏惧惭愧之意,倒觉可异。料此时夜深,且不必
盘问,只怕他夜间自寻短志,遂唤两个婆子监守,且带了人,拿了赃证,回来歇息,
等待明日料理。谁知夜里下面淋血不止,次日便觉身体十分软弱起来,遂掌不住,
请医诊视;开方立案,说要保重而去。老嬷嬷们拿了方子,回过王夫人,不免又添
一番愁闷,遂将司棋之事暂且搁起。

可巧这日尤氏来看凤姐,坐了一回,又看李纨等。忽见惜春遣人来请,尤氏到
他房中,惜春便将昨夜之事细细告诉了,又命人将入画的东西一概要来与尤氏过
目。尤氏道:“实是你哥哥赏他哥哥的。只不该私自传送,如今官盐反成了私盐了。”
因骂入画:“糊涂东西!”惜春道:“你们管教不严,反骂丫头。这些姊妹,独我
的丫头没脸,我如何去见人!昨儿叫凤姐姐带了他去,又不肯。今日嫂子来的恰好,
快带了他去,或打或杀或卖,我一概不管。”入画听说,跪地哀求,百般苦告。尤
氏和奶妈等人也都十分解说:“他不过一时糊涂,下次再不敢的。看他从小儿伏侍
一场。”谁知惜春年幼,天性孤僻,任人怎说,只是咬定牙,断乎不肯留着。更又
说道:“不但不要入画,如今我也大了,连我也不便往你们那边去了。况且近日闻
得多少议论,我若再去,连我也编派。”尤氏道:“谁敢议论什么?又有什么可议
论的?姑娘是谁?我们是谁?姑娘既听见人议论我们,就该问着他才是。”惜春冷笑
道:“你这话问着我倒好!我一个姑娘家,只好躲是非的,我反寻是非,成个什么
人了。况且古人说的,‘善恶生死,父子不能有所勖助’,何况你我二人之间。我
只能保住自己就够了,以后你们有事好歹别累我。”尤氏听了,又气又好笑,因向
地下众人道:“怪道人人都说四姑娘年轻糊涂,我只不信。你们听这些话,无原无
故,又没轻重,真真的叫人寒心。”众人都劝说道:“姑娘年轻,奶奶自然该吃些
亏的。”惜春冷笑道:“我虽年轻,这话却不年轻。你们不看书,不识字,所以都
是呆子,倒说我糊涂。”尤氏道:“你是状元,第一个才子!我们糊涂人,不如你
明白。”惜春道:“据你这话就不明白。状元难道没有糊涂的?可知你们这些人都
是世俗之见,那里眼里识的出真假、心里分的出好歹来?你们要看真人,总在最初
一步的心上看起,才能明白呢。”尤氏笑道:“好,好,才是才子,这会子又做大
和尚,讲起参悟来了。”惜春道:“我也不是什么参悟。我看如今人一概也都是入
画一般,没有什么大说头儿。”尤氏道:“可知你真是个心冷嘴冷的人。”惜春道:
“怎么我不冷!我清清白白的一个人,为什么叫你们带累坏了?”

尤氏心内原有病,怕说这些话,听说有人议论,已是心中羞恼,只是今日惜春
分中不好发作,忍耐了大半天。今见惜春又说这话,因按捺不住,便问道:“怎么
就带累了你?你的丫头的不是,无故说我;我倒忍了这半日,你倒越发得了意,只
管说这些话。你是千金小姐,我们以后就不亲近你,仔细带累了小姐的美名儿!即
刻就叫人将入画带了过去。”说着,便赌气起身去了。惜春道:“你这一去了,若
果然不来,倒也省了口舌是非,大家倒还干净。”尤氏听了,越发生气,但终久他
是姑娘,任凭怎么样也不好和他认真的拌起嘴来,只得索性忍了这口气。便也不答
言,一径往前边去了。

未知后事如何,且听下回分解。