\chapter{苦尤娘赚入大观园~酸凤姐大闹宁国府}

话说贾琏起身去后,偏值平安节度巡边在外,约一个月方回,贾琏未得确信,
只得住在下处等候。及至回来相见,将事办妥,回程已是将近两个月的限了。

谁知凤姐早已心下算定,只得贾琏前脚走了,回来便传各色匠役,收拾东厢房
三间,照依自己正室一样,装饰陈设。至十四日,便回明贾母王夫人,说十五日一
早要到姑子庙进香去。只带了平儿、丰儿、周瑞媳妇、旺儿媳妇四人。未曾上车,
便将原故告诉了众人,又吩咐众男人,素衣素盖,一径前来。兴儿引路,一直到了
门前扣门。鲍二家的开了,兴儿笑道:“快回二奶奶去:大奶奶来了。”鲍二家的
听了这句,顶梁骨走了真魂,忙飞跑进去报与尤二姐。尤二姐虽也一惊,但已来了,
只得以礼相见,于是忙整理衣裳,迎了出来。至门前,凤姐方下了车进来,二姐一
看,只见头上都是素白银器,身上月白缎子袄,青缎子掐银线的褂子,白绫素裙;
眉弯柳叶,高吊两梢,目横丹凤,神凝三角:俏丽若三春之桃,清素若九秋之菊。
周瑞旺儿的二女人搀进院来。二姐陪笑,忙迎上来拜见,张口便叫“姐姐”,说:
“今儿实在不知姐姐下降,不曾远接,求姐姐宽恕!”说着便拜下去。凤姐忙陪笑
还礼不迭,赶着拉了二姐儿的手,同入房中。

凤姐在上坐,二姐忙命丫头拿褥子,便行礼,说:“妹子年轻,一从到了这里,
诸事都是家母和家姐商议主张。今儿有幸相会,若姐姐不弃寒微,凡事求姐姐的指
教,情愿倾心吐胆,只伏侍姐姐。”说着便行下礼去。凤姐忙下坐还礼,口内忙说:
“皆因我也年轻,向来总是妇人的见识,一味的只劝二爷保重,别在外边眠花宿柳,
恐怕叫太爷太太耽心:这都是你我的痴心,谁知二爷倒错会了我的意。若是外头包
占人家姐妹,瞒着家里也罢了;如今娶了妹妹作二房,这样正经大事,也是人家大
礼,却不曾合我说。我也劝过二爷,早办这件事,果然生个一男半女,连我后来都
有靠。不想二爷反以我为那等妒忌不堪的人,私自办了,真真叫我有冤没处诉!我
的这个心,惟有天地可表。头十天头里,我就风闻着知道了,只怕二爷又错想了,
遂不敢先说,目今可巧二爷走了,所以我亲自过来拜见。还求妹妹体谅我的苦心,
起动大驾,挪到家中。你我姐妹同居同处,彼此合心合意的谏劝二爷,谨慎世务,
保养身子,这才是大礼呢。要是妹妹在外头,我在里头,妹妹白想想,我心里怎么
过的去呢?再者叫外人听着,不但我的名声不好听,就是妹妹的名儿也不雅。况且
二爷的名声更是要紧的,倒是谈论咱们姐儿们还是小事。至于那起下人小人之言,
未免见我素昔持家太严,背地里加减些话,也是常情。妹妹想:自古说的:‘当家
人,恶水缸。’我要真有不容人的地方儿,上头三层公婆,当中有好几位姐姐、妹
妹、妯娌们,怎么容的我到今儿?就是今儿二爷私娶妹妹,在外头住着,我自然不
愿意见妹妹,我如何还肯来呢?拿着我们平儿说起,我还劝着二爷收他呢。这都是
天地神佛不忍的叫这些小人们遭塌我,所以才叫我知道了。我如今来求妹妹,进去
和我一块儿,住的、使的、穿的、带的,总是一样儿的。妹妹这样伶透人,要肯真
心帮我,我也得个膀臂。不但那起小人堵了他们的嘴,就是二爷回来一见,他也从
今后悔,我并不是那种吃醋调歪的人,你我三人,更加和气。所以妹妹还是我的大
恩人呢。要是妹妹不合我去,我也愿意搬出来陪着妹妹住,只求妹妹在二爷跟前替
我好言方便方便,留我个站脚的地方儿,就叫我伏侍妹妹梳头洗脸,我也是愿意
的!”说着,便呜呜咽咽,哭将起来了。

二姐见了这般,也不免滴下泪来。二人对见了礼,分序坐下。平儿忙也上来要
见礼。二姐见他打扮不凡,举止品貌不俗,料定必是平儿,连忙亲身搀住,只叫:
“妹子快别这么着,你我是一样的人。”凤姐忙也起身笑说:“折死了他!妹妹只
管受礼,他原是咱们的丫头。以后快别这么着。”说着,又命周瑞家的从包袱里取
出四匹上色尺头,四对金珠簪环,为拜见的礼。二姐忙拜受了。二人吃茶,对诉已
往之事。凤姐口内全是自怨自错:“怨不得别人。如今只求妹妹疼我。”二姐是个
实心人,便认做他是个好人,想道:“小人不遂心,诽谤主子,也是常理。”故倾
心吐胆,叙了一回,竟把凤姐认为知己。又见周瑞家等媳妇在傍边称扬凤姐素日许
多善政,“只是吃亏心太痴了,反惹人怨。”又说:“已经预备了房屋,奶奶进去,
一看便知。”尤氏心中早已要进去同住方好,今又见如此,岂有不充之理?便说:
“原该跟了姐姐去,只是这里怎么着呢?”凤姐道:“这有何难?妹妹的箱笼细软,
只管着小厮搬了进去。这些粗夯货,要他无用,还叫人看着。妹妹说谁妥当,就叫
谁在这里。”二姐忙说:“今儿既遇见姐姐,这一进去,凡事只凭姐姐料理。我也
来的日子浅,也不曾当过家事,不明白,如何敢作主呢?这几件箱柜拿进去罢。我
也没有什么东西,那也不过是二爷的。”凤姐听了,便命周瑞家的记清,好生看管
着,抬到东厢房去。于是催着尤二姐急忙穿戴了,二人携手上车,又同坐一处,又
悄悄的告诉他:“我们家的规矩大。这事老太太、太太一概不知;倘或知道,二爷
孝中娶你,管把他打死了。如今且别见老太太、太太。我们有一个花园子极大,姐
妹们住着,容易没人去的。你这一去,且在园子里住两天,等我设个法子,回明白
了,那时再见方妥。”二姐道:“任凭姐姐裁处。”那些跟车的小厮们皆是预先说
明的,如今不进大门,只奔后门来。下了车,赶散众人,凤姐便带了尤氏,进了大
观园的后门,来到李纨处相见了。

彼时大观园里的十停人已有九停人知道了。今忽见凤姐带了进来,引动众人来
看问。二姐一一见过。众人见了他标致和悦,无不称扬。凤姐一一的吩咐了众人:
“都不许在外走了风声。若老太太、太太知道,我先叫你们死!”园里的婆子丫头
都素惧凤姐的,又系贾琏国孝家孝中所行之事,知道关系非常,都不管这事。凤姐
悄悄的求李纨收养几天:“等回明了,我们自然过去。”李纨见凤姐那边已收拾房
屋,况在服中不好倡扬,自是正理,只得收下权住。凤姐又便去将他的丫头一概退
出,又将自己的一个丫头送他使唤,暗暗吩咐他园里的媳妇们:“好生照看着他。
若是走失逃亡,一概和你们算帐。”自己又去暗中行事不提。

且说合家之人都暗暗的纳罕,说:“看他如何这等贤惠起来了?”那二姐得了
这个所在,又见园里姐妹个个相好,倒也安心乐业的,自为得所。谁知三日之后,
丫头善姐便有些不服使唤起来。二姐因说:“没了头油了,你去回一声大奶奶,拿
些个来。”善姐儿便道:“二奶奶:你怎么不知好歹,没眼色?我们奶奶天天承应
了老太太,又要承应这边太太、那边太太。这些姑娘妯娌们,上下几百男女人,天
天起来都等他的话,一日少说大事也有一二十件,小事还有三五十件。外头从娘娘
算起,以及王公侯伯家,多少人情;家里又有这些亲友的调度;银子上千钱上万,
一天都从他一个人手里出入,一个嘴里调度:那里为这点子小事去烦琐他?我劝你
能着些儿罢!咱们又不是明媒正娶来的。这是他亘古少有一个贤良人,才这样待你。
若差些儿的人,听见了这话,吵嚷起来,把你丢在外头,死不死活不活,你敢怎么
着呢?”一席话说的尤氏垂了头。自为有这一说,少不得将就些罢了。那善姐渐渐
的连饭也怕端来给他吃了,或早一顿,晚一顿,所拿来的东西皆是剩的。二姐说过
两次,他反瞪着眼叫唤起来了。二姐又怕人笑他不安本分,少不得忍着。隔上五日
八日见凤姐一面,那凤姐却是和容悦色,满嘴里“好妹妹”不离口。又说:“倘有
下人不到之处,你降不住他们,只管告诉我,我打他们。”又骂丫头媳妇说:“我
深知你们软的欺,硬的怕,背着我的眼,还怕谁?倘或二奶奶告诉我一个‘不’字,
我要你们的命。”二姐见他这般好心,“既有他,我又何必多事?下人不知好歹是
常情。我要告了他们,受了委屈,反叫人说我不贤良。”因此,反替他们遮掩。

凤姐一面使旺儿在外打听这二姐的底细,皆已深知:果然已有了婆家的,女婿
现在才十九岁,成日在外赌博,不理世业,家私花尽了,父母撵他出来,现在赌钱
场存身。父亲得了尤婆子二十两银子,退了亲的,这女婿尚不知道。原来这小伙子
名叫张华。凤姐都一一尽知原委,便封了二十两银子给旺儿,悄悄命他将张华勾来
养活,“着他写一张状子,只要往有司衙门里告去,就告琏二爷国孝家孝的里头,
背旨瞒亲,仗财依势,强逼退亲,停妻再娶。”这张华也深知利害,先不敢造次。
旺儿回了凤姐。凤姐气的骂道:“真是他娘的话!怨不得俗语说,‘癞狗扶不上墙
的。’你细细说给他:‘就告我们家谋反也没要紧!’不过是借他一闹,大家没脸;
要闹大了,我这里自然能够平服的。”旺儿领命,只得细说与张华。凤姐又吩咐旺
儿:“他若告了你,你就和他对词去”,如此如此,“我自有道理。”旺儿听了有
他作主,便又命张华状子上添上自己,说:“你只告我来旺的过付,一应调唆二爷
做的。”张华便得了主意,和旺儿商议定了。写一张状子,次日便往都察院处喊了
冤。

察院坐堂,看状子是告贾琏的事,上面有“家人来旺一人”,只得遣人去贾府
传来旺儿来对词。青衣不敢擅入,只命人带信。那旺儿正等着此事,不用人带信,
早在这条街上等候,见了青衣,反迎上去,笑道:“起动众位弟兄,必是兄弟的事
犯了。说不得,快来套上。”众青衣不敢,只说:“好哥哥你去罢,别闹了。”于
是来至堂前跪了。察院命将状子给他看。旺儿故意看了一遍,碰头说道:“这事小
的尽知的,主人实有此事。但这张华素与小的有仇,故意拉小的在内,其中还有人,
求老爷再问。”张华碰头道:“虽还有人,小的不敢告他,所以只告他下人。”旺
儿故意的说:“糊涂东西,还不快说出来!这是朝廷公堂上,凭是主子,也要说出
来。”张华便说出贾蓉来。察院听了无法,只得去传贾蓉。凤姐又差了庆儿暗中打
听告下来了,便忙将王信唤来,告诉他此事,命他托察院,只要虚张声势,惊唬而
已。又拿了三百银子给他去打点。是夜,王信到了察院私宅,安了根子。那察院深
知原委,收了赃银,次日回堂,只说张华无赖,因拖欠了贾府银两,妄捏虚词,诬
赖良人。都察院素与王子腾相好,王信也只到家说了一声,况是贾府之人,巴不得
了事,便也不提此事,且都收下,只传贾蓉对词。

且说贾蓉等正忙着贾琏之事,忽有人来报信,说:“有人告你们如此如此,这
般这般,快作道理!”贾蓉慌忙来回贾珍。贾珍说:“我却早防着这一着。倒难为
他这么大胆子。”即刻封了二百银子,着人去打点察院,又命家人去对词。正商议
间,又报:“西府二奶奶来了。”贾珍听了这话,倒吃了一惊,忙要和贾蓉藏躲,
不想凤姐已经进来了,说:“好大哥哥,带着兄弟们干的好事!”贾蓉忙请安。凤
姐拉了他就进来。贾珍还笑说:“好生伺候你婶娘,吩咐他们杀牲口备饭。”说着,
便命备马,躲往别处去了。

这里凤姐带着贾蓉,走进上屋。尤氏也迎出来了,见凤姐气色不善,忙说:“什
么事情,这么忙?”凤姐照脸一口唾沫,啐道:“你尤家的丫头没人要了,偷着只
往贾家送!难道贾家的人都是好的,普天下死绝了男人了?你就愿意给,也要三媒六
证,大家说明,成个体统才是。你痰迷了心,脂油蒙了窍,国孝家孝两层在身,就
把个人送了来。这会子叫人告我们,连官场中都知道我利害,吃醋。如今指名提我,
要休我。我到了这里,干错了什么不是,你这么利害?或是老太太、太太有了话在
你心里,叫你们做这个圈套挤出我去?如今咱们两个一同去见官,分证明白,回来
咱们公同请了合族中人,大家觌面说个明白,给我休书,我就走!”一面说,一面
大哭,拉着尤氏只要去见官。急的贾蓉跪在地下碰头,只求:“婶娘息怒!”凤姐
一面又骂贾蓉:“天打雷劈、五鬼分尸的没良心的东西!不知天有多高,地有多厚,
成日家调三窝四,干出这些没脸面、没王法、败家破业的营生。你死了的娘,阴灵
儿也不容你,祖宗也不容你!还敢来劝我!”一面骂着,扬手就打。唬的贾蓉忙碰
头说道:“婶娘别动气。只求婶娘别看这一时,侄儿千日的不好,还有一日的好。
实在婶娘气不平,何用婶娘打,等我自己打,婶娘只别生气。”说着,就自己举手,
左右开弓,自己打了一顿嘴巴子。又自己问着自己说:“以后可还再顾三不顾四的
不了?以后还单听叔叔的话、不听婶娘的话不了?婶娘是怎么样待你?你这么没天理
没良心的!”众人又要劝,又要笑,又不敢笑。

凤姐儿滚到尤氏怀里,嚎天动地,大放悲声,只说:“给你兄弟娶亲,我不恼,
为什么使他违旨背亲,把混帐名儿给我背着?咱们只去见官,省了捕快皂隶来拿。
再者,咱们过去,只见了老太太、太太和众族人等,大家公议了,我既不贤良,又
不容男人买妾,只给我一纸休书,我即刻就去!你妹妹,我也亲身接了来家,生怕
老太太、太太生气,也不敢回,现在三茶六饭、金奴银婢的住在园里。我这里赶着
收拾房子,和我一样的,只等老太太知道了。原说下接过来大家安分守己的,我也
不提旧事了,谁知又是有了人家的!不知你们干的什么事!我一概又不知道。如今告
我,我昨日急了,纵然我出去见官,也丢的是你贾家的脸,少不得偷把太太的五百
两银子去打点。如今把我的人还锁在那里!”说了又哭,哭了又骂。后来又放声大
哭起“祖宗爷娘”来,又要寻死撞头。把个尤氏揉搓成一个面团儿,衣服上全是眼
泪鼻涕,并无别话,只骂贾蓉:“混帐种子!和你老子做的好事!我当初就说使不得。”
凤姐儿听说这话,哭着搬着尤氏的脸,问道:“你发昏了?你的嘴里难道有茄子
着?不就是他们给你嚼子衔上了?为什么你不来告诉我去?你要告诉了我,这会子不
平安了?怎么得惊官动府,闹到这步田地?你这会子还怨他们!自古说‘妻贤夫祸
少’,‘表壮不如里壮’,你但凡是个好的,他们怎敢闹出这些事来?你又没才干,
又没口齿,锯了嘴子的葫芦,就只会一味瞎小心,应贤良的名儿。”说着,啐了几
口。尤氏也哭道:“何曾不是这样?你不信,问问跟的人,我何曾不劝的?也要他们
听。叫我怎么样呢?怨不得妹妹生气,我只好听着罢了。”

众姬妾丫头媳妇等已是黑压压跪了一地,陪笑求说:“二奶奶最圣明的。虽是
我们奶奶的不是,奶奶也作践够了,当着奴才们。奶奶们素日何等的好来?如今还
求奶奶给留点脸儿。”说着,捧上茶来,凤姐也摔了。一回止了哭,挽头发,又喝
骂贾蓉:“出去请你父亲来,我对面问他!问亲大爷的孝才五七,侄儿娶亲,这个
礼,我竟不知道,我问问也好学着,日后教导你们!”贾蓉只跪着磕头,说:“这
事原不与父母相干,都是侄儿一时吃了屎,调唆着叔叔做的。我父亲也并不知道。
婶娘要闹起来了,侄儿也是个死;只求婶娘责罚侄儿,侄儿谨领。这官司还求婶娘
料理,侄儿竟不能干这大事。婶娘是何等样人,岂不知俗语说的‘膊折了,在袖
子里’?侄儿糊涂死了,既做了不肖的事,就和那猫儿狗儿一般,少不得还要婶娘
费心费力,将外头的事压住了才好。只当婶娘有这个不孝的儿子,就惹了祸,少不
得委屈还要疼他呢。”说着,又磕头不绝。凤姐儿见了贾蓉这般,心里早软了,只
是碍着众人面前,又难改过口来,因叹了一口气,一面拉起来,一面拭泪向尤氏道:
“嫂子也别恼我,我是年轻不知事的人,一听见有人告诉了,把我吓昏了,才这么
着急的顾前不顾后了。可是蓉儿说的,‘膊折了在袖子里。’刚才的话,嫂子可
别恼,还得嫂子在哥哥跟前替说,先把这官司按下去才好。”尤氏贾蓉一齐都说:
“婶娘放心。横竖一点儿连累不着叔叔。婶娘方才说用过了五百两银子,少不得我
们娘儿们打点五百两银子,给婶娘送过去,好补上,那有叫婶娘又添上亏空的理?
那越发我们该死了。但还有一件:老太太、太太们跟前,婶娘还要周全方便,别提
这些话才好。”

凤姐又冷笑道:“你们饶压着我的头干了事,这会子反哄着我替你们周全!我
就是个傻子,也傻不到如此:嫂子的兄弟,是我的什么人?嫂子既怕他绝了后,我
难道不更比嫂子更怕绝后?嫂子的妹子,就合我的妹子一样,我一听见这话,连夜
喜欢的连觉也睡不成,赶着传人收拾了屋子,就要接进来同住。倒是奴才小人的见
识,他们倒说:‘奶奶太性急,若是我们的主意,先回了老太太、太太,看是怎么
样,再收拾房子去接也不迟。’我听了这话,叫我要打要骂的,才不言语了。谁知
偏不称我的意,偏偏儿的打嘴,半空里跑出一个张华来告了一状。我听见了,吓的
两夜没合眼儿,又不敢声张,只得求人去打听这张华是什么人,这样大胆。打听了
两日,谁知是个无赖的花子。小子们说:‘原是二奶奶许了他的。他如今急了,冻
死饿死也是个死,现在有这个理他抓住,纵然死了,死的倒比冻死饿死还值些,怎
么怨的他告呢?这事原是爷做的太急了:国孝一层罪,家孝一层罪,背着父母私娶
一层罪,停妻再娶一层罪。俗语说,“拚着一身剐,敢把皇帝拉下马”,他穷疯了
的人,什么事做不出来?况且他又拿着这满理,不告等请不成?’嫂子说,我就是
个韩信、张良,听了这话,也把智谋吓回去了。你兄弟又不在家,又没个人商量,
少不得拿钱去垫补。谁知越使钱越叫人拿住刀靶儿,越发来讹。我是‘耗子尾巴上
长疮,——多少脓血儿’。所以又急又气,少不得来找嫂子。”尤氏贾蓉不等说完,
都说:“不必操心,自然要料理的。”贾蓉又道:“那张华不过是穷急,故舍了命
才告咱们。如今想了一个法儿:竟许他些银子,只叫他应个妄告不实之罪,咱们替
他打点完了官司,他出来时,再给他些银子就完了。”凤姐儿咂着嘴儿,笑道:“难
为你想,怨不得你顾一不顾二的做出这些事来:原来你竟是这么个有心胸的,我往
日错看了你了。若你说的这话,他暂且依了,且打出官司来,又得了银子,眼前自
然了事。这些人既是无赖的小人,银子到手,三天五天一光了,他又来找事讹诈,
再要叨蹬起来,咱们虽不怕,终久耽心。搁不住他说:既没毛病,为什么反给他银
子?”贾蓉原是个明白人,听如此一说,便笑道:“我还有个主意:‘来是是非人,
去是是非者’,这事还得我了才好。如今我竟问张华个主意,或是他定要人?或是
他愿意了事,得钱再娶?他若说一定要人,少不得我去劝我二姨娘,叫他出来,还
嫁他去;若说要钱,我们少不得给他些个。”凤姐儿忙道:“虽如此说,我断舍不
得你姨娘出去,我也断不肯使他出去。他要出去了,咱们家的脸在那里呢?依我说,
只宁可多给钱为是。”贾蓉深知凤姐儿口虽如此,心却是巴不得只要本人出来,他
却做贤良人。如今怎么说,且只好怎么依着。

凤姐儿又说:“外头好处了,家里终久怎么样呢?你也和我过去回明了老太太、
太太才是。”尤氏又慌了,拉凤姐儿讨主意,怎么撒谎才好。凤姐冷笑道:“既没
这本事,谁叫你干这样事?这会子这个腔儿,我又看不上。待要不出个主意,我又
是个心慈面软的人,凭人撮弄我,我还是一片傻心肠儿,说不得等我应起来。如今
你们只别露面,我只领了你妹妹去给老太太、太太们磕头。只说:原系你妹妹我看
上了很好,正因我不大生长,原说买两个人放在屋里的;今既见了你妹妹很好,而
且又是亲上做亲的,我愿意娶来做二房。皆因家中父母姊妹亲近一概死了,日子又
难,不能度日,若等百日之后,无奈无家无业,实在难等。就算我的主意,接进来
了,已经厢房收拾出来了,暂且住着,等满了孝再圆房儿。仗着我这不害臊的脸,
死活赖去,有了不是,也寻不着你们了。你们娘儿两个想想,可使得?”

尤氏贾蓉一齐笑说:“到底是婶娘宽洪大量,足智多谋!等事妥了,少不得我
们娘儿们过去拜谢。”凤姐儿道:“罢呀,还说什么拜谢不拜谢。”又指着贾蓉道:
“今日我才知道你了。”说着,把脸却一红,眼圈儿也红了,似有多少委屈的光景。
贾蓉忙陪笑道:“罢了,少不得担待我这一次罢。”说着,忙又跪下了。凤姐儿扭
过脸去不理他,贾蓉才笑着起来了。这里尤氏忙命丫头们舀水,取妆奁,伏侍凤姐
儿梳洗了,赶忙又命预备晚饭。凤姐儿执意要回去,尤氏拦着道:“今日二婶子要
这么走了,我们什么脸还过那边去呢?”贾蓉旁边笑着劝道:“好婶娘!亲婶娘!以
后蓉儿要不真心孝顺你老人家,天打雷劈。”凤姐瞅了他一眼,啐道:“谁信你这
——”说到这里,又咽住了。一面老婆丫头们摆上酒菜来,尤氏亲自递酒布菜。贾
蓉又跪着敬了一钟酒。凤姐便合尤氏吃了饭。丫头们递了漱口茶,又捧上茶来。凤
姐喝了两口,便起身回去。贾蓉亲身送过来,进门时,又悄悄的央告了几句私心话,
凤姐也不理他,只得怏怏的回去了。

且说凤姐进园中,将此事告诉尤二姐,又说,我怎么操心,又怎么打听,须得
如此如此,方保得众人无罪,“少不得咱们按着这个法儿来才好。”

不知凤姐又想出什么计策,且听下回分解。