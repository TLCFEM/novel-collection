\chapter{敏探春兴利除宿弊~贤宝钗小惠全大体}

话说平儿陪着凤姐吃了饭,伏侍盥漱毕,方往探春处来,只见院中寂静,只有
丫鬟婆子一个个都站在窗外听候。平儿进入厅中,他姐妹姑嫂三人正商议些家务,
说的便是年内赖大家请吃酒,他家花园中事故。见他来了,探春便命他脚踏上坐了,
因说道:“我想的事,不为别的,只想着我们一月所用的头油脂粉又是二两的事。
我想咱们一月已有了二两月银,丫头们又另有月钱,可不是又同刚才学里的八两一
样重重叠叠?这事虽小,钱有限,看起来也不妥当,你奶奶怎么就没想到这个呢?”
平儿笑道:“这有个原故:姑娘们所用的这些东西,自然该有分例,每月每处买办
买了,令女人们交送我们收管,不过预备姑娘们使用就罢了,没有个我们天天各人
拿着钱,找人买这些去的。所以外头买办总领了去,按月使女人按房交给我们。至
于姑娘们每月的这二两,原不是为买这些的,为的是一时当家的奶奶太太,或不在
家,或不得闲,姑娘们偶然要个钱使,省得找人去:这不过是恐怕姑娘们受委屈意
思。如今我冷眼看着,各屋里我们的姐妹都是现拿钱买这些东西的,竟有了一半子。
我就疑惑不是买办脱了空,就是买的不是正经货。”探春李纨都笑道:“你也留心
看出来了。脱空是没有的,只是迟些日子,催急了,不知那里弄些来,不过是个名
儿。其实使不得,依然还得现买,就用二两银子,另叫别人的奶妈子的弟兄儿子买
来方才使得。要使官中的人去,依然是那一样的,不知他们是什么法子?”平儿便
笑道:“买办买的是那东西,别人买了好的来,买办的也不依他,又说他使坏心,
要夺他的买办。所以他们宁可得罪了里头,不肯得罪了外头办事的。要是姑娘们使
了奶妈子们,他们也就不敢说闲话了。”

探春道:“因此我心里不自在,饶费了两起钱,东西又白丢一半。不如竟把买
办的这一项每月蠲了为是。此是第一件事。第二件,年里往赖大家去。你也去的:
你看他那小园子比咱们这个如何?”平儿笑道:“还没有咱们这一半大,树木花草
也少多着呢。”探春道:“我因和他们家的女孩儿说闲话儿,他说这园子除他们带
的花儿,吃的笋菜鱼虾,一年还有人包了去,年终足有二百两银子剩。从那日,我
才知道一个破荷叶、一根枯草根子,都是值钱的。”宝钗笑道:“真真膏粱纨之
谈!你们虽是千金,原不知道这些事,但只你们也都念过书,识过字的,竟没看见
过朱夫子有一篇‘不自弃’的文么?”探春笑道:“虽也看过,不过是勉人自励,
虚比浮词,那里真是有的?”宝钗道:“朱子都行了虚比浮词了?那句句都是有的。
你才办了两天事,就利欲熏心,把朱子都看虚浮了。你再出去,见了那些利弊大事,
越发连孔子也都看虚了呢!”探春笑道:“你这样一个通人,竟没看见姬子书?当
日姬子有云:‘登利禄之场,处运筹之界者,穷尧舜之词,背孔孟之道。’”宝钗
笑道:“底下一句呢?”探春笑道:“如今断章取意;念出底下一句,我自己骂我
自己不成?”宝钗道:“天下没有不可用的东西,既可用,便值钱。难为你是个聪
明人,这大节目正事竟没经历。”李纨笑道:“叫人家来了,又不说正事,你们且
对讲学问!”宝钗道:“学问中便是正事。若不拿学问提着,便都流入市俗去了。”

三人取笑了一回,便仍谈正事。探春又接说道:“咱们这个园子,只算比他们
的多一半,加一倍算起来,一年就有四百银子的利息。若此时也出脱生发银子,自
然小器,不是咱们这样人家的事。若派出两个一定的人来,既有许多值钱的东西,
任人作践了,也似乎暴殄天物。不如在园子里所有的老妈妈中,拣出几个老成本分、
能知园圃的,派他们收拾料理。也不必要他们交租纳税,只问他们一年可以孝敬些
什么。一则园子有专定之人修理花木,自然一年好似一年了,也不用临时忙乱;二
则也不致作践,白辜负了东西;三则老妈妈们也可借此小补,不枉成年家在园中辛
苦;四则也可省了这些花儿匠、山子匠并打扫人等的工费。将此有馀,以补不足,
未为不可。”宝钗正在地下看壁上的字画,听如此说,便点头笑道:“善哉!‘三
年之内,无饥馑矣。’”李纨道:“好主意!果然这么行,太太必喜欢。省钱事小,
园子有人打扫,专司其职,又许他去卖钱,使之以权,动之以利,再无不尽职的了。”

平儿道:“这件事须得姑娘说出来。我们奶奶虽有此心,未必好出口。此刻姑
娘们在园里住着,不能多弄些玩意儿陪衬,反叫人去监管修理,图省钱,这话断不
好出口。”宝钗忙走过来,摸着他的脸笑道:“你张开嘴,我瞧瞧你的牙齿舌头是
什么做的?从早起来到这会子,你说了这些话,一套一个样子:也不奉承三姑娘,
也不说你们奶奶才短想不到;三姑娘说一套话出来,你就有一套话回奉,总是三姑
娘想得到的,你们奶奶也想到了,只是必有个不可办的原故。这会子又是因姑娘们
住的园子,不好因省钱令人去监管。你们想想这话,要果真交给人弄钱去的,那人
自然是一枝花也不许掐,一个果子也不许动了,姑娘们分中自然是不敢讲究,天天
和小姑娘们就吵不清。他这远愁近虑,不亢不卑,他们奶奶就不是和咱们好,听他
这一番话,也必要自愧的变好了。”探春笑道:“我早起一肚子气,听他来了,忽
然想起他主子来:素日当家,使出来的好撒野的人!我见了他更生气了。谁知他来
了,避猫鼠儿似的,站了半日,怪可怜的。接着又说了那些话,不说他主子待我好,
倒说‘不枉姑娘待我们奶奶素日的情意了’,这一句话,不但没了气,我倒愧了,
又伤起心来。我细想:我一个女孩儿家,自己还闹得没人疼没人顾的,我那里还有
好处去待人?”口内说到这里,不免又流下泪来。李纨等见他说得恳切,又想他素
日赵姨娘每生诽谤,在王夫人跟前,亦为赵姨娘所累,也都不免流下泪来,都忙劝
他:“趁今日清净,大家商议两件兴利剔弊的事情,也不枉太太委托一场。又提这
没要紧的事做什么。”平儿忙道:“我已明白了。姑娘说谁好,竟一派人就完了。”
探春道:“虽如此说,也须得回你奶奶一声儿。我们这里搜剔小利,已经不当。皆
因你奶奶是个明白人,我才这样行;若是糊涂多歪多妒的,我也不肯,倒像抓他的
乖的似的。岂可不商议了行呢?”平儿笑道:“这么着,我去告诉一声儿。”说着
去了;半日方回来,笑道:“我说是白走一趟。这样好事,奶奶岂有不依的!”

探春听了,便和李纨命人将园中所有婆子的名单要来,大家参度,大概定了几
个人。又将他们一齐传来,李纨大概告诉给他们。众人听了,无不愿意。也有说:
“那片竹子单交给我,一年工夫,明年又是一片。除了家里吃的笋,一年还可交些
钱粮。”这一个说:“那一片稻地交给我,一年这些玩的大小雀鸟的粮食,不必动
官中钱粮,我还可以交钱粮。”探春才要说话,人回:“大夫来了,进园瞧史姑娘
去。”众婆子只得去领大夫。平儿忙说:“单你们,有一百也不成个体统。难道没
有两个管事的头脑儿带进大夫来?”回事的那人说:“有吴大娘和单大娘,他两个
在西南角上聚锦门等着呢。”平儿听说,方罢了。

众婆子去后,探春问宝钗:“如何?”宝钗笑答道:“幸于始者怠于终,善其
辞者嗜其利。”探春听了,点头称赞,便向册上指出几个来与他三人看。平儿忙去
取笔砚来。他三人说道:“这一个老祝妈,是个妥当的,况他老头子和他儿子,代
代都是管打扫竹子,如今竟把这所有的竹子交与他。这一个老田妈本是种庄稼的,
稻香村一带,凡有菜蔬稻稗之类,虽是玩意儿,不必认真大治大耕,也须得他去再
细细按时加些植养,岂不更好?”探春又笑道:“可惜蘅芜院和怡红院这两处大地
方,竟没有出息之物。”李纨忙笑道:“蘅芜院里更利害,如今香料铺并大市大庙
卖的各处香料香草儿,都不是这些东西?算起来,比别的利息更大。怡红院别说别
的,单只说春夏两季的玫瑰花,共下多少花朵儿?还有一带篱笆上的蔷薇、月季、
宝相、金银花、藤花,这几色草花,干了卖到茶叶铺药铺去,也值好些钱。”探春
笑着点头儿,又道:“只是弄香草没有在行的人。”平儿忙笑道:“跟宝姑娘的莺
儿他妈,就是会弄这个的。上回他还采了些晒干了,编成花篮葫芦给我玩呢。姑娘
倒忘了么?”宝钗笑道:“我才赞你,你倒来捉弄我了。”三人都诧异问道:“这
是为何?”宝钗道:“断断使不得。你们这里多少得用的人,一个个闲着没事办,
这会子我又弄个人来,叫那起人连我也看小了。我倒替你们想出一个人来:怡红院
有个老叶妈,他就是焙茗的娘。那是个诚实老人家,他又合我们莺儿妈极好。不如
把这事交与叶妈,他有不知的,不必咱们说给他,就找莺儿的娘去商量了。那怕叶
妈全不管,竟交与那一个,这是他们私情儿,有人说闲话也就怨不到咱们身上。如
此一行,你们办的又公道,于事又妥当。”李纨平儿都道:“很是。”探春笑道:
“虽如此,只怕他们见利忘义呢。”平儿笑道:“不相干。前日莺儿还认了叶妈做
干娘,请吃饭吃酒,两家和厚的很呢。”探春听了,方罢了。又共斟酌出几个人来,
俱是他四人素昔冷眼取中的,用笔圈出。

一时婆子们来回:“大夫已去。”将药方送上去,三人看了。一面遣人送出外
边去取药,监派调服,一面探春与李纨明示诸人:某人管某处,“按四季,除家中
定例用多少外,馀者任凭你们采取去取利,年终算账。”探春笑道:“我又想起一
件事:若年终算账,归钱时自然归到账房,仍是上头又添一层管主,还在他们手心
里又剥一层皮。这如今我们兴出这件事,派了你们,已是跨过他们的头去了,心里
有气只说不出来,你们年终去归账,他还不捉弄你们等什么?再者这一年间管什么
的,主子有一全分,他们就得半分,这是每常的旧规,人所共知的。如今这园子是
我的新创,竟别入他们的手,每年归账,竟归到里头来才好。”宝钗笑道:“依我
说,里头也不用归账,这个多了,那个少了,倒多了事。不如问他们谁领这一分的,
他就揽一宗事去。不过是园里的人动用。我替你们算出来了,有限的几宗事,不过
是头油、胭粉、香、纸,每一位姑娘,几个丫头,都是有定例的;再者各处苕帚、
簸箕、掸子,并大小禽鸟鹿兔吃的粮食。不过这几样。都是他们包了去,不用账房
去领钱。你算算,就省下多少来?”平儿笑道:“这几宗虽小,一年通共算了,也
省的下四百多银子。”宝钗笑道:“却又来。一年四百,二年八百两,打租的房子
也能多买几间,薄沙地也可以添几亩了。虽然还有敷馀,但他们既辛苦了一年,也
要叫他们剩些,粘补自家。虽是兴利节用为纲,然也不可太过,要再省上二三百银
子,失了大体统,也不像。所以这么一行,外头帐房里一年少出四五百银子,也不
觉的很艰啬了;他们里头却也得些小补;这些没营生的妈妈们,也宽裕了;园子里
花木,也可以每年滋长繁盛;就是你们,也得了可使之物:这庶几不失大体。若一
味要省时,那里搜寻不出几个钱来?凡有些馀利的,一概入了官中,那时里外怨声
载道,岂不失了你们这样人家的大体?如今这园里几十个老妈妈们,若只给了这个,
那剩的也必抱怨不公;我才说的他们只供给这个几样,也未免太宽裕了。一年竟除
这个之外,他每人不论有馀无馀,只叫他拿出若干吊钱来,大家凑齐,单散与这些
园中的妈妈们。他们虽不料理这些,却日夜也都在园中照料;当差之人,关门闭户,
起早睡晚,大雨大雪,姑娘们出入,抬轿子、撑船、拉冰床一应粗重活计,都是他
们的差使:一年在园里辛苦到头,这园内既有出息,也是分内该沾带些的。还有一
句至小的话,越发说破了:你们只顾了自己宽裕,不分与他们些,他们虽不敢明怨,
心里却都不服,只用假公济私的,多摘你们几个果子,多掐几枝花儿,你们有冤还
没处诉呢。他们也沾带些利息,你们有照顾不到的,他们就替你们照顾了。”

众婆子听了这个议论,又去了帐房受辖制,又不与凤姐儿去算帐,一年不过多
拿出若干吊钱来,各各欢喜异常,都齐声说:“愿意!强如出去被他们揉搓着,还
得拿出钱来呢。”那不得管地的,听了每年终无故得钱,更都喜欢起来,口内说:
“他们辛苦收拾,是该剩些钱粘补的;我们怎么好‘稳吃三注’呢?”宝钗笑道:
“妈妈们也别推辞了,这原是分内应当的。你们只要日夜辛苦些,别躲懒纵放人吃
酒赌钱就是了。不然,我也不该管这事。你们也知道,我姨娘亲口嘱托我三五回,
说大奶奶如今又不得闲,别的姑娘又小,托我照看照看。我若不依,分明是叫姨娘
操心。我们太太又多病,家务也忙,我原是个闲人,就是街坊邻舍,也要帮个忙儿,
何况是姨娘托我?讲不起众人嫌我。倘或我只顾沽名钓誉的,那时酒醉赌输,再生
出事来,我怎么见姨娘?你们那时后悔也迟了,就连你们素昔的老脸也都丢了。这
些姑娘们,这么一所大花园子,都是你们照管着,皆因看的你们是三四代的老妈妈,
最是循规蹈矩,原该大家齐心顾些体统。你们反纵放别人,任意吃酒赌博。姨娘听
见了,教训一场犹可,倘若被那几个管家娘子听见了,他们也不用回姨娘,竟教导
你们一场,你们这年老的反受了小的教训。虽是他们是管家管的着你们,何如自己
存些体面,他们如何得来作践呢!所以我如今替你们想出这个额外的进益来,也为
的是大家齐心,把这园里周全得谨谨慎慎的,使那些有权执事的看见这般严肃谨
慎,且不用他们操心,他们心里岂不敬服?也不枉替你们筹画些进益了。你们去细
细想想这话。”众人都欢喜说:“姑娘说的很是。从此姑娘奶奶只管放心。姑娘奶
奶这么疼顾我们,我们再要不体上情,天地也不容了。”

刚说着,只见林之孝家的进来,说:“江南甄府里家眷昨日到京,今日进宫朝
贺,此刻先遣人来送礼请安。”说着便将礼单送上去。探春接了,看道是:“上用
的妆缎蟒缎十二匹。上用杂色缎十二匹。上用各色纱十二匹。上用宫绸十二匹。宫
用各色缎纱绸绫二十四匹。”李纨探春看过,说:“用上等封儿赏他。”因又命人
去回了贾母。贾母命人叫李纨、探春、宝钗等都过来,将礼物看了。李纨收过一边,
吩咐内库上人说:“等太太回来看了再收。”贾母因说:“这甄家又不与别家相同。
上等封儿赏男人。只怕转眼又打发女人来请安,预备下尺头。”

一语未了,果然人回:“甄府四个女人来请安。”贾母听了,忙命人带进来。
那四个人都是四十往上年纪,穿带之物皆比主子不大差别。请安问好毕,贾母便命
拿了四个脚踏来。他四人谢了坐,等着宝钗等坐了,方都坐下。贾母便问:“多早
晚进京的?”四人忙起身回说:“昨儿进的京,今儿太太带了姑娘进宫请安去了,
所以叫女人们来请安,问候姑娘们。”贾母笑问道:“这些年没进京,也不想到就
来。”四人也都笑回道:“正是。今年是奉旨唤进京的。”贾母问道:“家眷都来
了?”四人回说:“老太太和哥儿、两位小姐,并别位太太,都没来;就只太太带
了三姑娘来了。”贾母道:“有人家没有?”四人道:“还没有呢。”贾母笑道:
“你们大姑娘和二姑娘,这两家,都和我们家甚好。”四人笑道:“正是。每年姑
娘们有信回来说,全亏府上照看。”贾母笑道:“什么‘照看’?原是世交,又是
老亲,原应当的。你们二姑娘更好,不自尊大,所以我们才走的亲密。”四人笑道:
“这是老太太过谦了。”贾母又问:“你这哥儿也跟着你们老太太?”四人回说:
“也跟着老太太呢。”贾母道:“几岁了?”又问:“上学不曾?”四人笑说:“今
年十三岁。因长的齐整,老太太很疼,自幼淘气异常,天天逃学,老爷太太也不便
十分管教。”贾母笑道:“也不成了我们家的了?你这哥儿叫什么名字?”四人道:
“因老太太当作宝贝一样,他又生的白,老太太便叫作‘宝玉’。”贾母笑向李纨
道:“偏也叫个‘宝玉’!”李纨等忙欠身笑道:“从古至今,同时隔代,重名的
很多。”四人也笑道:“起了这小名儿之后,我们上下都疑惑,不知那位亲友家也
倒像曾有一个的。只是这十来年没进京来,却记不真了。”贾母笑道:“那就是我
的孙子。——人来。”众媳妇丫头答应了一声,走近几步,贾母笑道:“园里把咱
们的宝玉叫了来,给这四个管家娘子瞧瞧,比他们的宝玉如何。”

众媳妇听了,忙去了,半刻,围了宝玉进来。四人一见,忙起身笑道:“唬了
我们一跳!要是我们不进府来,倘若别处遇见,还只当我们的宝玉后赶着也进了京
呢。”一面说,一面都上来拉他的手,问长问短。宝玉也笑问个好。贾母笑道:“比
你们的长的如何?”李纨等笑道:“四位妈妈才一说,可知是模样儿相仿了。”贾
母笑道:“那有这样巧事。大家子孩子们,再养的娇嫩,除了脸上有残疾十分丑的,
大概看去都是一样齐整,这也没有什么怪处。”四人笑道:“如今看来,模样是一
样!据老太太说,淘气也一样,我们看来,这位哥儿性情却比我们的好些。”贾母
忙笑问怎么。四人笑道:“方才我们拉哥儿的手说话,便知道了。若是我们那一位,
只说我们糊涂。慢说拉手,他的东西我们略动一动也不依。所使唤的人都是女孩子
们。”四人未说完,李纨姊妹等禁不住都失声笑出来。贾母也笑道:“我们这会子
也打发人去见了你们宝玉,若拉他的手,他也自然勉强忍耐着。不知你我这样人家
的孩子,凭他们有什么刁钻古怪的毛病,见了外人,必是要还出正经礼数来的。若
他不还正经礼数,也断不容他刁钻去了。就是大人溺爱的,也因为他一则生的得人
意儿;二则见人礼数,竟比大人行出来的还周到,使人见了可爱可怜,背地里所以
才纵他一点子。若一味他只管没里没外,不给大人争光,凭他生的怎样,也是该打
死的。”四人听了,都笑道:“老太太这话正是。虽然我们宝玉淘气古怪,有时见
了客,规矩礼数,比大人还有趣,所以无人见了不爱,只说:‘为什么还打他?’
殊不知他在家里无法无天,大人想不到的话偏会说,想不到的事偏会行,所以老爷
太太恨的无法。就是任性,也是小孩子的常情;胡乱花费,也是公子哥儿的常情;
怕上学,也是小孩子的常情:都还治的过来。第一,天生下来这一种刁钻古怪的脾
气,如何使得?”一语未了,人回:“太太回来了。”王夫人进来,问过安,他四
人请了安,大概说了两句,贾母便命:“歇歇去罢。”王夫人亲捧过茶,方退出去。
四人告辞了贾母,便往王夫人处来,说了一会子家务,打发他们回去,不必细说。

这里贾母喜得逢人便告诉:也有一个宝玉,也都一般行景。众人都想着天下的
世宦人家,同名的这也很多,祖母溺爱孙子也是常事,不是什么罕事,皆不介意。
独宝玉是个迂阔呆公子的心性,自为是那四人承悦贾母之词。后至园中去看湘云病
去,湘云因说他:“你放心闹罢,先还‘单丝不成线,独树不成林’,如今有了个
对子了。闹利害了,再打急了,你好逃到南京找那个去。”宝玉道:“那里的谎话,
你也信了?偏又有个宝玉了?”湘云道:“怎么列国有个蔺相如,汉朝又有个司马
相如呢?”宝玉笑道:“这也罢了,偏又模样儿也一样,这也是有的事吗?”湘云
道:“怎么匡人看见孔子,只当是阳货呢?”宝玉笑道:“孔子阳货虽同貌,却不
同名;蔺与司马虽同名,而又不同貌。偏我和他就两样俱同不成?”湘云没了话答
对,因笑道:“你只会胡搅,我也不和你分证。有也罢,没也罢,与我无干!”说
着,便睡下了。

宝玉心中便又疑惑起来:若说必无,也似必有;若说必有,又并无目睹。心中
闷闷,回至房中榻上,默默盘算,不觉昏昏睡去,竟到一座花园之内。宝玉诧异道:
“除了我们大观园,竟又有这一个园子?”正疑惑间,忽然那边来了几个女孩儿,
都是丫鬟,宝玉又诧异道:“除了鸳鸯、袭人、平儿之外,也竟还有这一干人?”
只见那些丫鬟笑道:“宝玉怎么跑到这里来?”宝玉只当是说他,忙来陪笑说道:
“因我偶步到此,不知是那位世交的花园?姐姐们带我逛逛。”众丫鬟都笑道:“原
来不是咱们家的宝玉。他生的也还干净,嘴儿也倒乖觉。”宝玉听了,忙道:“姐
姐们这里,也竟还有个宝玉?”丫鬟们忙道:“‘宝玉’二字,我们家是奉老太太、
太太之命,为保佑他延年消灾,我们叫他,他听见喜欢;你是那里远方来的小厮,
也乱叫起来!仔细你的臭肉,不打烂了你的。”又一个丫鬟笑道:“咱们快走罢,
别叫宝玉看见。”又说:“同这臭小子说了话,把咱们熏臭了。”说着一径去了。
宝玉纳闷道:“从来没有人如此荼毒我,他们如何竟这样的?莫不真也有我这样一
个人不成?”

一面想,一面顺步早到了一所院内。宝玉诧异道:“除了怡红院,也竟还有这
么一个院落?”忽上了台阶,进入屋内,只见榻上有一个人卧着,那边有几个女儿
做针线,或有嬉笑玩耍的。只见榻上那个少年叹了一声,一个丫鬟笑问道:“宝玉,
你不睡,又叹什么?想必为你妹妹病了,你又胡愁乱恨呢。”宝玉听说,心下也便
吃惊,只见榻上少年说道:“我听见老太太说,长安都中也有个宝玉,和我一样的
性情,我只不信。我才做了一个梦,竟梦中到了都中一个大花园子里头,遇见几个
姐姐,都叫我臭小厮,不理我。好容易找到他房里,偏他睡觉,空有皮囊,真性不
知往那里去了。”宝玉听说,忙说道:“我因找宝玉来到这里,原来你就是宝玉?”
榻上的忙下来拉住,笑道:“原来你就是宝玉!这可不是梦里了?”宝玉道:“这
如何是梦?真而又真的!”一语未了,只见人来说:“老爷叫宝玉。”吓得二人皆
慌了,一个宝玉就走。一个便忙叫:“宝玉快回来!宝玉快回来!”

袭人在旁听他梦中自唤,忙推醒他,笑问道:“宝玉在那里?”此时宝玉虽醒,
神意尚自恍惚,因向门外指说:“才去不远。”袭人笑道:“那是你梦迷了。你揉
眼细瞧,是镜子里照的你的影儿。”宝玉向前瞧了一瞧,原是那嵌的大镜对面相照,
自己也笑了。早有丫鬟捧过漱盂茶卤来漱了口。麝月道:“怪道老太太常嘱咐说:
‘小人儿屋里不可多有镜子,人小魂不全,有镜子照多了,睡觉惊恐做胡梦。’如
今倒在大镜子那里安了一张床!有时放下镜套还好,往前去天热困倦,那里想的到
放他?比如方才就忘了,自然先躺下照着影儿玩来着,一时合上眼自然是胡梦颠倒
的。不然,如何叫起自己的名字来呢?不如明日挪进床来是正经。”一语未了,只
见王夫人遣人来叫宝玉。

不知有何话说,且听下回分解。