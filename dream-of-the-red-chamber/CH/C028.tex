\chapter{蒋玉函情赠茜香罗~薛宝钗羞笼红麝串}

话说林黛玉只因昨夜晴雯不开门一事,错疑在宝玉身上。次日又可巧遇见饯花
之期,正在一腔无明未曾发泄,又勾起伤春愁思,因把些残花落瓣去掩埋,由不得
感花伤己,哭了几声,便随口念了几句。不想宝玉在山坡上听见,先不过点头感叹;
次又听到“侬今葬花人笑痴,他年葬侬知是谁”、“一朝春尽红颜老,花落人亡两
不知”等句,不觉恸倒山坡上,怀里兜的落花撒了一地。试想林黛玉的花颜月貌,
将来亦到无可寻觅之时,宁不心碎肠断?既黛玉终归无可寻觅之时,推之于他人,
如宝钗、香菱、袭人等,亦可以到无可寻觅之时矣。宝钗等终归无可寻觅之时,则
自己又安在呢?且自身尚不知何在何往,将来斯处、斯园、斯花、斯柳,又不知当
属谁姓?因此一而二二而三反复推求了去,真不知此时此际如何解释这段悲伤!正
是:
花影不离身左右,鸟声只在耳东西。

那黛玉正自伤感,忽听山坡上也有悲声,心下想道:“人人都笑我有痴病,难
道还有一个痴的不成?”抬头一看,见是宝玉,黛玉便啐道:“呸!我打量是谁,
原来是这个狠心短命的——”刚说到“短命”二字,又把口掩住,长叹一声,自己
抽身便走。

这里宝玉悲恸了一回,见黛玉去了,便知黛玉看见他躲开了,自己也觉无味。
抖抖土起来,下山寻归旧路,往怡红院来。可巧看见黛玉在前头走,连忙赶上去,
说道:“你且站着。我知道你不理我;我只说一句话,从今以后撩开手。”黛玉回
头见是宝玉,待要不理他,听他说只说一句话,便道:“请说。”宝玉笑道:“两
句话,说了你听不听呢?”黛玉听说,回头就走。宝玉在身后面叹道:“既有今日,
何必当初?”黛玉听见这话,由不得站住,回头道:“当初怎么样?今日怎么样?”
宝玉道:“嗳!当初姑娘来了,那不是我陪着玩笑?凭我心爱的,姑娘要就拿去;我
爱吃的,听见姑娘也爱吃,连忙收拾的干干净净收着,等着姑娘回来。一个桌子上
吃饭,一个床儿上睡觉。丫头们想不到的,我怕姑娘生气,替丫头们都想到了。我
想着姊妹们从小儿长大,亲也罢,热也罢,和气到了儿,才见得比别人好。如今谁
承望姑娘人大心大,不把我放在眼里,三日不理、四日不见的,倒把外四路儿的什
么‘宝姐姐’‘凤姐姐’的放在心坎儿上。我又没个亲兄弟、亲妹妹,虽然有两个,
你难道不知道是我隔母的?我也和你是独出,只怕你和我的心一样。谁知我是白操
了这一番心,有冤无处诉!”说着,不觉哭起来。

那时黛玉耳内听了这话,眼内见了这光景,心内不觉灰了大半,也不觉滴下泪
来,低头不语。宝玉见这般形象,遂又说道:“我也知道我如今不好了,但只任凭
我怎么不好,万不敢在妹妹跟前有错处。就有一二分错处,你或是教导我,戒我下
次,或骂我几句,打我几下,我都不灰心。谁知你总不理我,叫我摸不着头脑儿,
少魂失魄,不知怎么样才好。就是死了也是个屈死鬼,任凭高僧高道忏悔,也不能
超生,还得你说明了原故,我才得托生呢!”

黛玉听了这话,不觉将昨晚的事都忘在九霄云外了,便说道:“你既这么说,
为什么我去了,你不叫丫头开门呢!”宝玉诧异道:“这话从那里说起?我要是这
么着,立刻就死了!”黛玉啐道:“大清早起‘死’呀‘活’的,也不忌讳!你说
有呢就有,没有就没有,起什么誓呢!”宝玉道:“实在没有见你去,就是宝姐姐
坐了一坐,就出来了。”黛玉想了一想,笑道:“是了:必是丫头们懒怠动,丧声
歪气的,也是有的。”宝玉道:“想必是这个原故。等我回去问了是谁,教训教训
他们就好了。”黛玉道:“你的那些姑娘们,也该教训教训。只是论理我不该说。
今儿得罪了我的事小,倘或明儿‘宝姑娘’来,什么‘贝姑娘’来,也得罪了,事
情可就大了。”说着抿着嘴儿笑。宝玉听了,又是咬牙,又是笑。

二人正说话,见丫头来请吃饭,遂都往前头来了。王夫人见了黛玉,因问道:
“大姑娘,你吃那鲍太医的药可好些?”黛玉道:“也不过这么着。老太太还叫我
吃王大夫的药呢。”宝玉道:“太太不知道:林妹妹是内症,先天生的弱,所以禁
不住一点儿风寒;不过吃两剂煎药,疏散了风寒,还是吃丸药的好。”王夫人道:
“前儿大夫说了个丸药的名字,我也忘了。”宝玉道:“我知道那些丸药,不过叫
他吃什么人参养荣丸。”王夫人道:“不是。”宝玉又道:“八珍益母丸?左归,
右归?再不就是八味地黄丸?”王夫人道:“都不是。我只记得有个‘金刚’两个
字的。”宝玉拍手笑道:“从来没听见有个什么‘金刚丸’!若有了‘金刚丸’,
自然有‘菩萨散’了!”说的满屋里人都笑了。宝钗抿嘴笑道:“想是天王补心丹。”
王夫人笑道:“是这个名儿。如今我也糊涂了。”宝玉道:“太太倒不糊涂,都是
叫‘金刚’‘菩萨’支使糊涂了。”王夫人道:“扯你娘的臊!又欠你老子捶你了。”
宝玉笑道:“我老子再不为这个捶我。”

王夫人又道:“既有这个名儿,明儿就叫人买些来吃。”宝玉道:“这些药都
是不中用的。太太给我三百六十两银子,我替妹妹配一料丸药,包管一料不完就好
了。”王夫人道:“放屁!什么药就这么贵?”宝玉笑道:“当真的呢。我这个方
子比别的不同,那个药名儿也古怪,一时也说不清,只讲那头胎紫河车,人形带叶
参,三百六十两不足。龟大何首乌,千年松根茯苓胆,诸如此类的药不算为奇,只
在群药里算。那为君的药,说起来,唬人一跳!前年薛大哥哥求了我一二年,我才
给了他这方子。他拿了方子去,又寻了二三年,花了有上千的银子,才配成了。太
太不信,只问宝姐姐。”宝钗听说,笑着摇手儿说道:“我不知道,也没听见。你
别叫姨娘问我。”王夫人笑道:“到底是宝丫头好孩子,不撒谎。”宝玉站在当地,
听见如此说,一回身把手一拍,说道:“我说的倒是真话呢,倒说撒谎!”口里说
着,忽一回身,只见林黛玉坐在宝钗身后抿着嘴笑,用手指头在脸上画着羞他。

凤姐因在里间屋里看着人放桌子,听如此说,便走来笑道:“宝兄弟不是撒谎,
这倒是有的。前日薛大爷亲自和我来寻珍珠,我问他做什么,他说配药。他还抱怨
说:‘不配也罢了,如今那里知道这么费事!’我问:‘什么药?’他说是宝兄弟
说的方子,说了多少药,我也不记得。他又说:‘不是我就买几颗珍珠了,只是必
要头上戴过的,所以才来寻几颗。要没有散的花儿,就是头上戴过的拆下来也使得。
过后儿我拣好的再给穿了来。’我没法儿,只得把两枝珠子花儿现拆了给他。还要
一块三尺长、上用的大红纱,拿乳钵研了面子呢。”凤姐说一句,宝玉念一句佛。
凤姐说完了,宝玉又道:“太太打量怎么着?这不过也是将就罢咧。正经按方子,
这珍珠宝石是要在古坟里找,有那古时富贵人家儿装裹的头面拿了来才好。如今那
里为这个去刨坟掘墓?所以只是活人带过的也使得。”王夫人听了道:“阿弥陀佛,
不当家花拉的!就是坟里有,人家死了几百年,这会子翻尸倒骨的,作了药也不灵
啊。”

宝玉因向黛玉道:“你听见了没有?难道二姐姐也跟着我撒谎不成?”脸望着
黛玉说,却拿眼睛瞟着宝钗。黛玉便拉王夫人道:“舅母听听,宝姐姐不替他圆谎,
他只问着我!”王夫人也道:“宝玉很会欺负你妹妹。”宝玉笑道:“太太不知道
这个原故。宝姐姐先在家里住着,薛大哥的事他也不知道,何况如今在里头住着呢?
自然是越发不知道了。林妹妹才在背后,以为是我撒谎,就羞我。”

正说着,见贾母房里的丫头找宝玉和黛玉去吃饭。黛玉也不叫宝玉,便起身带
着那丫头走。那丫头说:“等着宝二爷一块儿走啊。”黛玉道:“他不吃饭,不和
咱们走,我先走了。”说着,便出去了。宝玉道:“我今儿还跟着太太吃罢。”王
夫人道:“罢罢,我今儿吃斋,你正经吃你的去罢。”宝玉道:“我也跟着吃斋。”
说着,便叫那丫头:“去罢。”自己跑到桌子上坐了。王夫人向宝钗等笑道:“你
们只管吃你们的,由他去罢。”宝钗因笑道:“你正经去罢。吃不吃,陪着林妹妹
走一趟,他心里正不自在呢。何苦来?”宝玉道:“理他呢,过一会子就好了。”

一时吃过饭,宝玉一则怕贾母惦记,二则也想着黛玉,忙忙的要茶漱口。探春
惜春都笑道:“二哥哥,你成日家忙的是什么?吃饭吃茶也是这么忙碌碌的。”宝
钗笑道:“你叫他快吃了瞧黛玉妹妹去罢。叫他在这里胡闹什么呢?”宝玉吃了茶
便出来,一直往西院来。可巧走到凤姐儿院前,只见凤姐儿在门前站着,蹬着门槛
子,拿耳挖子剔牙,看着十来个小厮们挪花盆呢。见宝玉来了,笑道:“你来的好,
进来,进来,替我写几个字儿。”宝玉只得跟了进来。到了房里,凤姐命人取过笔
砚纸来,向宝玉道:“大红妆缎四十匹,蟒缎四十匹,各色上用纱一百匹,金项圈
四个。”宝玉道:“这算什么?又不是帐,又不是礼物,怎么个写法儿?”凤姐儿
道:“你只管写上,横竖我自己明白就罢了。”宝玉听说,只得写了。凤姐一面收
起来,一面笑道:“还有句话告诉你,不知依不依?你屋里有个丫头叫小红的,我
要叫了来使唤,明儿我再替你挑一个,可使得么?”宝玉道:“我屋里的人也多的
很,姐姐喜欢谁,只管叫了来,何必问我?”凤姐笑道:“既这么着,我就叫人带
他去了。”宝玉道:“只管带去罢。”说着要走。凤姐道:“你回来,我还有一句
话呢。”宝玉道:“老太太叫我呢,有话等回来罢。”

说着,便至贾母这边。只见都已吃完了饭了。贾母因问道:“跟着你娘吃了什
么好的了?”宝玉笑道:“也没什么好的,我倒多吃了一碗饭。”因问:“林姑娘
在那里?”贾母道:“里头屋里呢。”宝玉进来,只见地下一个丫头吹熨斗,炕上
两个丫头打粉线,黛玉弯着腰拿剪子裁什么呢。宝玉走进来,笑道:“哦!这是做
什么呢?才吃了饭,这么控着头,一会子又头疼了。”黛玉并不理,只管裁他的。
有一个丫头说道:“那块绸子角儿还不好呢,再熨熨罢。”黛玉便把剪子一撂,说
道:“理他呢,过一会子就好了。”宝玉听了,自是纳闷。只见宝钗、探春等也来
了,和贾母说了一回话,宝钗也进来问:“妹妹做什么呢?”因见林黛玉裁剪,笑
道:“越发能干了,连裁铰都会了。”黛玉笑道:“这也不过是撒谎哄人罢了。”
宝钗笑道:“我告诉你个笑话儿,才刚为那个药,我说了个不知道,宝兄弟心里就
不受用了。”黛玉道:“理他呢,过会子就好了。”宝玉向宝钗道:“老太太要抹
骨牌,正没人,你抹骨牌去罢。”宝钗听说,便笑道:“我是为抹骨牌才来么?”
说着便走了。黛玉道:“你倒是去罢,这里有老虎,看吃了你!”说着又裁。宝玉
见他不理,只得还陪笑说道:“你也去逛逛,再裁不迟。”黛玉总不理。宝玉便问
丫头们:“这是谁叫他裁的?”黛玉见问丫头们,便说道:“凭他谁叫我裁,也不
管二爷的事。”宝玉方欲说话,只见有人进来,回说“外头有人请呢”。宝玉听了,
忙撤身出来。黛玉向外头说道:“阿弥陀佛,赶你回来,我死了也罢了!”

宝玉来到外面,只见焙茗说:“冯大爷家请。”宝玉听了,知道是昨日的话,
便说:“要衣裳去。”就自己往书房里来。焙茗一直到了二门前等人,只见出来了
一个老婆子,焙茗上去说道:“宝二爷在书房里等出门的衣裳,你老人家进去带个
信儿。”那婆子啐道:“呸!放你娘的屁!宝玉如今在园里住着,跟他的人都在园里,
你又跑了这里来带信儿了!”焙茗听了笑道:“骂的是,我也糊涂了!”说着,一
径往东边二门前来。可巧门上小厮在甬路底下踢球,焙茗将原故说了,有个小厮跑
了进去,半日才抱了一个包袱出来,递给焙茗。回到书房里,宝玉换上,叫人备马,
只带着焙茗、锄药、双瑞、寿儿四个小厮去了。

一径到了冯紫英门口,有人报与冯紫英,出来迎接进去。只见薛蟠早已在那里
久候了,还有许多唱曲儿的小厮们,并唱小旦的蒋玉函,锦香院的妓女云儿。大家
都见过了,然后吃茶。宝玉擎茶笑道:“前儿说的‘幸与不幸’之事,我昼夜悬想,
今日一闻呼唤即至。”冯紫英笑道:“你们令姑表弟兄倒都心实。前日不过是我的
设辞,诚心请你们喝一杯酒,恐怕推托,才说下这句话。谁知都信了真了。”说毕,
大家一笑。然后摆上酒来,依次坐定。冯紫英先叫唱曲儿的小厮过来递酒,然后叫
云儿也过来敬三钟。那薛蟠三杯落肚,不觉忘了情,拉着云儿的手笑道:“你把那
体己新鲜曲儿唱个我听,我喝一坛子,好不好?”云儿听说,只得拿起琵琶来,唱
道:

两个冤家,都难丢下,想着你来又惦记着他。两个人形容俊俏都难描画,想昨
宵幽期私订在荼架。一个偷情,一个寻拿:拿住了三曹对案我也无回话。
唱毕,笑道:“你喝一坛子罢了。”薛蟠听说,笑道:“不值一坛,再唱好的来。”

宝玉笑道:“听我说罢:这么滥饮,易醉而无味。我先喝一大海,发一个新令,
有不遵者,连罚十大海,逐出席外,给人斟酒。”冯紫英蒋玉函等都道:“有理,
有理。”宝玉拿起海来,一气饮尽,说道:“如今要说‘悲’‘愁’‘喜’‘乐’
四个字,却要说出‘女儿’来,还要注明这四个字的原故。说完了,喝门杯,酒面
要唱一个新鲜曲子,酒底要席上生风一样东西——或古诗、旧对、《四书》《五经》
成语。”薛蟠不等说完,先站起来拦道:“我不来,别算我。这竟是玩我呢!”云
儿也站起来,推他坐下,笑道:“怕什么?这还亏你天天喝酒呢,难道连我也不及?
我回来还说呢。说是了罢,不是了不过罚上几杯,那里就醉死了你?如今一乱令,
倒喝十大海,下去斟酒不成?”众人都拍手道:“妙!”薛蟠听说无法,只得坐了。

听宝玉说道:“女儿悲,青春已大守空闺。女儿愁,悔教夫婿觅封侯。女儿喜,
对镜晨妆颜色美。女儿乐,秋千架上春衫薄。”众人听了,都说道:“好!”薛蟠
独扬着脸,摇头说:“不好,该罚。”众人问:“如何该罚?”薛蟠道:“他说的
我全不懂,怎么不该罚?”云儿便拧他一把,笑道:“你悄悄儿的想你的罢。回来
说不出来,又该罚了。”于是拿琵琶听宝玉唱道:

滴不尽相思血泪抛红豆,开不完春柳春花满画楼。睡不稳纱窗风雨黄昏后,忘
不了新愁与旧愁。咽不下玉粒金波噎满喉,照不尽菱花镜里形容瘦。展不开的眉头,
捱不明的更漏:呀!恰便似遮不住的青山隐隐,流不断的绿水悠悠。
唱完,大家齐声喝彩,独薛蟠说:“没板儿。”宝玉饮了门杯,便拈起一片梨来,
说道:“‘雨打梨花深闭门’。”完了令。

下该冯紫英,说道:“女儿喜,头胎养了双生子。女儿乐,私向花园掏蟋蟀。
女儿悲,儿夫染病在垂危。女儿愁,大风吹倒梳妆楼。”说毕,端起酒来,唱道:

你是个可人,你是个多情,你是个刁钻古怪鬼灵精,你是个神仙也不灵。我说
的话儿你全不信,只叫你去背地里细打听,才知道我疼你不疼!
唱完,饮了门杯,说道:“‘鸡声茅店月’。”令完。

下该云儿,云儿便说道:“女儿悲,将来终身倚靠谁?”薛蟠笑道:“我的儿,
有你薛大爷在,你怕什么?”众人都道:“别混他,别混他!”云儿又道:“女儿
愁,妈妈打骂何时休?”薛蟠道:“前儿我见了你妈,还嘱咐他,不叫他打你呢。”
众人都道:“再多说的,罚酒十杯!”薛蟠连忙自己打了一个嘴巴子,说道:“没
耳性,再不许说了。”云儿又说:“女儿喜,情郎不舍还家里。女儿乐,住了箫管
弄弦索。”说完,便唱道:

豆蔻花开三月三,一个虫儿往里钻。钻了半日钻不进去,爬到花儿上打秋千。
肉儿小心肝,我不开了你怎么钻?
唱毕,饮了门杯,说道:“‘桃之夭夭’。”令完,下该薛蟠。

薛蟠道:“我可要说了:女儿悲——”说了,半日不见说底下的。冯紫英笑道:
“悲什么?快说。”薛蟠登时急的眼睛铃铛一般,便说道:“女儿悲——”又咳嗽
了两声,方说道:“女儿悲,嫁了个男人是乌龟。”众人听了都大笑起来。薛蟠道:
“笑什么?难道我说的不是?一个女儿嫁了汉子,要做忘八,怎么不伤心呢?”众人
笑的弯着腰说道:“你说的是!快说底下的罢。”薛蟠瞪了瞪眼,又说道:“女儿
愁——”说了这句,又不言语了。众人道:“怎么愁?”薛蟠道:“绣房钻出个大
马猴。”众人哈哈笑道:“该罚,该罚!先还可恕,这句更不通了。”说着,便要
斟酒。宝玉道:“押韵就好。”薛蟠道:“令官都准了,你们闹什么!”众人听说
方罢了。云儿笑道:“下两句越发难说了,我替你说罢。”薛蟠道:“胡说!当真
我就没好的了?听我说罢:女儿喜,洞房花烛朝慵起。”众人听了,都诧异道:“这
句何其太雅?”薛蟠道:“女儿乐,一根往里戳。”众人听了,都回头说道:
“该死,该死!快唱了罢。”薛蟠便唱道:“一个蚊子哼哼哼。”众人都怔了,说
道:“这是个什么曲儿?”薛蟠还唱道:“两个苍蝇嗡嗡嗡。”众人都道:“罢,
罢,罢!”薛蟠道:“爱听不听,这是新鲜曲儿,叫做‘哼哼韵’儿,你们要懒怠
听,连酒底儿都免了,我就不唱。”众人都道:“免了罢,倒别耽误了别人家。”

于是蒋玉函说道:“女儿悲,丈夫一去不回归。女儿愁,无钱去打桂花油。女
儿喜,灯花并头结双蕊。女儿乐,夫唱妇随真和合。”说毕,唱道:

可喜你天生成百媚娇,恰便似活神仙离碧霄。度青春,年正小;配鸾凤,真也
巧。呀!看天河正高,听谯楼鼓敲,剔银灯同入鸳帏悄。
唱毕,饮了门杯,笑道:“这诗词上我倒有限,幸而昨日见了一副对子,只记得这
句,可巧席上还有这件东西。”说毕,便干了酒,拿起一朵木樨来,念道:“‘花
气袭人知昼暖’。”众人都倒依了完令,薛蟠又跳起来喧嚷道:“了不得,了不得,
该罚,该罚!这席上并没有宝贝,你怎么说起宝贝来了?”蒋玉函忙说道:“何曾
有宝贝?”薛蟠道:“你还赖呢!你再说。”蒋玉函只得又念了一遍。薛蟠道:“这
‘袭人’可不是宝贝是什么?你们不信只问他!”说毕,指着宝玉。宝玉没好意思
起来,说:“薛大哥,你该罚多少?”薛蟠道:“该罚,该罚!”说着,拿起酒来,
一饮而尽。冯紫英和蒋玉函等还问他原故,云儿便告诉了出来,蒋玉函忙起身陪罪。
众人都道:“不知者不作罪。”

少刻,宝玉出席解手,蒋玉函随着出来,二人站在廊檐下,蒋玉函又赔不是。
宝玉见他妩媚温柔,心中十分留恋,便紧紧的攥着他的手,叫他:“闲了往我们那
里去。还有一句话问你,也是你们贵班中,有一个叫琪官儿的,他如今名驰天下,
可惜我独无缘一见。”蒋玉函笑道:“就是我的小名儿。”宝玉听说,不觉欣然跌
足笑道:“有幸,有幸!果然名不虚传。今儿初会,却怎么样呢?”想了一想,向
袖中取出扇子,将一个玉扇坠解下来,递给琪官,道:“微物不堪,略表今日之
谊。”琪官接了,笑道:“无功受禄,何以克当?也罢,我这里也得了一件奇物,
今日早起才系上,还是簇新,聊可表我一点亲热之意。”说毕撩衣,将系小衣儿的
一条大红汗巾子解下来递给宝玉道:“这汗巾子是茜香国女国王所贡之物,夏天系
着肌肤生香,不生汗渍。昨日北静王给的,今日才上身。若是别人,我断不肯相赠。
二爷请把自己系的解下来给我系着。”宝玉听说,喜不自禁,连忙接了,将自己一
条松花汗巾解下来递给琪官。二人方束好,只听一声大叫:“我可拿住了!”只见
薛蟠跳出来,拉着二人道:“放着酒不喝,两个人逃席出来,干什么?快拿出来我
瞧瞧。”二人都道:“没有什么。”薛蟠那里肯依,还是冯紫英出来才解开了。复
又归坐饮酒,至晚方散。

宝玉回至园中,宽衣吃茶,袭人见扇上的坠儿没了,便问他:“往那里去了?”
宝玉道:“马上丢了。”袭人也不理论。及睡时,见他腰里一条血点似的大红汗巾
子,便猜着了八九分,因说道:“你有了好的系裤子了,把我的那条还我罢。”宝
玉听说,方想起那汗巾子原是袭人的,不该给人。心里后悔,口里说不出来,只得
笑道:“我赔你一条罢。”袭人听了,点头叹道:“我就知道你又干这些事了,也
不该拿我的东西给那些混帐人哪。也难为你心里没个算计儿!”还要说几句,又恐
怄上他的酒来,少不得也睡了。一宿无话。

次日天明方醒,只见宝玉笑道:“夜里失了盗也不知道,你瞧瞧裤子上。”袭
人低头一看,只见昨日宝玉系的那条汗巾子,系在自己腰里了,便知是宝玉夜里换
的,忙一顿就解下来,说道:“我不希罕这行子,趁早儿拿了去。”宝玉见他如此,
只得委婉解劝了一回。袭人无法,暂且系上。过后宝玉出去,终久解下来,扔在个
空箱子里了,自己又换了一条系着。

宝玉并未理论。因问起:“昨日可有什么事情?”袭人便回说:“二奶奶打发
人叫了小红去了。他原要等你来着,我想什么要紧,我就做了主,打发他去了。”
宝玉道:“很是。我已经知道了,不必等我罢了。”袭人又道:“昨儿贵妃打发夏
太监出来送了一百二十两银子,叫在清虚观初一到初三打三天平安醮,唱戏献供,
叫珍大爷领着众位爷们跪香拜佛呢。还有端午儿的节礼也赏了。”说着,命小丫头
来,将昨日的所赐之物取出来,却是上等宫扇两柄,红麝香珠二串,凤尾罗二端,
芙蓉簟一领。宝玉见了,喜不自胜,问:“别人的也都是这个吗?”袭人道:“老
太太多着一个香玉如意,一个玛瑙枕。老爷、太太、姨太太的,只多着一个香玉如
意。你的和宝姑娘的一样。林姑娘和二姑娘、三姑娘、四姑娘只单有扇子和数珠儿,
别的都没有。大奶奶、二奶奶他两个是每人两匹纱、两匹罗,两个香袋儿,两个锭
子药。”

宝玉听了,笑道:“这是怎么个原故,怎么林姑娘的倒不和我的一样,倒是宝
姐姐的和我一样?别是传错了罢?”袭人道:“昨儿拿出来,都是一分一分的写着
签子,怎么会错了呢。你的是在老太太屋里,我去拿了来了的。老太太说了:明儿
叫你一个五更天进去谢恩呢。”宝玉道:“自然要走一趟。”说着,便叫了紫鹃来:
“拿了这个到你们姑娘那里去,就说是昨儿我得的,爱什么留下什么。”紫鹃答应
了,拿了去。不一时回来,说:“姑娘说了,昨儿也得了,二爷留着罢。”宝玉听
说,便命人收了。

刚洗了脸出来,要往贾母那里请安去,只见黛玉顶头来了,宝玉赶上去笑道:
“我的东西叫你拣,你怎么不拣?”黛玉昨日所恼宝玉的心事,早又丢开,只顾今
日的事了,因说道:“我没这么大福气禁受,比不得宝姑娘,什么‘金’哪‘玉’
的,我们不过是个草木人儿罢了!”宝玉听他提出“金玉”二字来,不觉心里疑猜,
便说道:“除了别人说什么金什么玉,我心里要有这个想头,天诛地灭,万世不得
人身!”黛玉听他这话,便知他心里动了疑了,忙又笑道:“好没意思,白白的起
什么誓呢?谁管你什么金什么玉的!”宝玉道:“我心里的事也难对你说,日后自
然明白。除了老太太、老爷、太太这三个人,第四个就是妹妹了。有第五个人,我
也起个誓。”黛玉道:“你也不用起誓,我很知道你心里有‘妹妹’。但只是见了
‘姐姐’,就把‘妹妹’忘了。”宝玉道:“那是你多心,我再不是这么样的。”
黛玉道:“昨儿宝丫头他不替你圆谎,你为什么问着我呢?那要是我,你又不知怎
么样了!”正说着,只见宝钗从那边来了,二人便走开了。

宝钗分明看见,只装没看见,低头过去了。到了王夫人那里,坐了一回,然后
到了贾母这边,只见宝玉也在这里呢。宝钗因往日母亲对王夫人曾提过“金锁是个
和尚给的,等日后有玉的方可结为婚姻”等语,所以总远着宝玉。昨日见元春所赐
的东西,独他和宝玉一样,心里越发没意思起来。幸亏宝玉被一个黛玉缠绵住了,
心心念念只惦记着黛玉,并不理论这事。此刻忽见宝玉笑道:“宝姐姐,我瞧瞧你
的那香串子呢?”可巧宝钗左腕上笼着一串,见宝玉问他,少不得褪了下来。

宝钗原生的肌肤丰泽,一时褪不下来,宝玉在傍边看着雪白的胳膊,不觉动了
羡慕之心。暗暗想道:“这个膀子若长在林姑娘身上,或者还得摸一摸;偏长在他
身上,正是恨我没福。”忽然想起“金玉”一事来,再看看宝钗形容,只见脸若银
盆,眼同水杏,唇不点而含丹,眉不画而横翠,比黛玉另具一种妩媚风流,不觉又
呆了。宝钗褪下串子来给他,他也忘了接。宝钗见他呆呆的,自己倒不好意思的,
起来扔下串子。回身才要走,只见黛玉蹬着门槛子,嘴里咬着绢子笑呢。宝钗道:
“你又禁不得风吹,怎么又站在那风口里?”黛玉笑道:“何曾不是在房里来着。
只因听见天上一声叫,出来瞧了瞧,原来是个呆雁。”宝钗道:“呆雁在那里呢?
我也瞧瞧。”黛玉道:“我才出来,他就‘忒儿’的一声飞了。”口里说着,将手
里的绢子一甩,向宝玉脸上甩来,宝玉不知,正打在眼上,“嗳哟”了一声。

要知端的,下回分解。