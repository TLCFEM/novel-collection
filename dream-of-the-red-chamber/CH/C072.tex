\chapter{王熙凤恃强羞说病~来旺妇倚势霸成亲}

且说鸳鸯出了角门,脸上犹热,心内突突的乱跳,真是意外之事。因想这事非
常,若说出来奸盗相连,关系人命,还保不住带累旁人。横竖与自己无干,且藏在
心内,不说给人知道。回房复了贾母的命,大家安息不提。

却说司棋因从小儿和他姑表兄弟一处玩笑,起初时小儿戏言,便都订下将来不
娶不嫁;近年大了,彼此又出落得品貌风流。常时司棋回家时,二人眉来眼去,旧
情不断,只不能入手。又彼此生怕父母不从,二人便设法,彼此里外买嘱园内老婆
子们,留门看道。今日赶乱,方从外进来,初次入港。虽未成双,却也海誓山盟,
私传表记,已有无限风情。忽被鸳鸯惊散,那小厮早穿花度柳,从角门出去了。司
棋一夜不曾睡着,又后悔不来。至次日见了鸳鸯,自是脸上一红一白,百般过不去,
心内怀着鬼胎,茶饭无心,起坐恍惚。挨了两日,竟不听见有动静,方略放下了心。
这日晚间,忽有个婆子来悄悄告诉道:“你表兄竟逃走了,三四天没上家。如今打
发人四处找他呢。”司棋听了,又急又气又伤心,因想道:“纵然闹出来,也该死在
一处。真真男人没情意,先就走了。”因此,又添了一层气,次日便觉心内不快,
支持不住,一头躺倒,恹恹的成了病了。

鸳鸯闻知那边无故走了一个小厮,园内司棋病重,要往外挪,心下料定是二人
惧罪之故,“生怕我说出来。”因此,自己反过意不去,指着来望候司棋,支出人去,
反自己赌咒发誓,与司棋说:“我若告诉一个人,立刻现死现报!你只管放心养病,
别白遭塌了小命儿。”司棋一把拉住,哭道:“我的姐姐!咱们从小儿耳鬓厮磨,你
不曾拿我当外人待,我也不敢怠慢了你,如今我虽一着走错了,你若果然不告诉一
个人,你就是我的亲娘一样。从此后,我活一日,是你给我一日。我的病要好了,
把你立个长生牌位,我天天烧香磕头,保佑你一辈子福寿双全的,我若死了时,变
驴变狗报答你。倘或咱们散了,以后遇见,我自有报答的去处。”一面说,一面哭。
这一席话,反把鸳鸯说的酸心,也哭起来了。因点头道:“你也是自家要作死哟!我
作什么管你这些事坏你的名儿,我白去献勤儿?况且这事我也不便开口和人说。你
只放心。从此养好了,可要安分守己的,再别胡行乱闹了。”司棋在枕上点首不绝。

鸳鸯又安慰了他一番,方出来。因知贾琏不在家中,又因这两日凤姐儿声色怠
惰了些,不似往日一样,便顺路来问候。刚进入凤姐院中,二门上的人见是他来,
便站立待他进去。鸳鸯来至堂屋,只见平儿从里头出来,见了他来,便忙上来悄声
笑道:“才吃了一口饭,歇了中觉了。你且这屋里略坐坐。”鸳鸯听了,只得同平儿
到东边房里来。小丫头倒了茶来。鸳鸯悄问道:“你奶奶这两日是怎么了?我近来看
着他懒懒的。”平儿见问,因房内无人,便叹道:“他这懒懒的,也不止今日了。这
有一月前头,就是这么着。这几日忙乱了几天,又受了些闲气,从新又勾起来。这
两日比先又添了些病,所以支不住,就露出马脚来了。”鸳鸯道:“既这样,怎么不
早请大夫治?”平儿叹道:“我的姐姐,你还不知道他那脾气的?别说请大夫来吃药,
我看不过,白问一声‘身上觉怎么样’,他就动了气,反说我咒他病了。饶这样,
天天还是察三访四。自己再不看破些,且养身子!”鸳鸯道:“虽然如此,到底该请
大夫来瞧瞧是什么病,也都好放心。”平儿叹道:“说起病来,据我看也不是什么小
症候。”鸳鸯忙道:“是什么病呢?”平儿见问,又往前凑了一凑,向耳边说道:“只
从上月行了经之后,这一个月,竟沥沥淅淅的没有止住。这可是大病不是?”鸳鸯
听了忙答应道:“嗳哟,依这么说,可不成了‘血山崩’了吗?”平儿忙啐了一口,
又悄笑道:“你个女孩儿家,这是怎么说?你倒会咒人。”鸳鸯见说,不禁红了脸,
又悄笑道:“究竟我也不懂什么是崩不崩的。你倒忘了不成:先我姐姐不是害这病
死了?我也不知是什么病,因无心中听见妈和亲家妈说,我还纳闷,后来听见原故,
才明白了一二分。”二人正说着,只见小丫头向平儿道:“方才朱大娘又来了。我们
回了他:‘奶奶才歇中觉。’他往太太上头去了。”平儿听了点头。鸳鸯问:“那一个
朱大娘?”平儿道:“就是官媒婆朱嫂子。因有个什么孙大人来和咱们求亲,所以
他这两日天天弄个帖子来,闹得人怪烦的。”

一语未了,小丫头跑来说:“二爷进来了。”说话之间,贾琏已走至堂屋门口,
平儿忙迎出来。贾琏见平儿在东屋里,便也过这间房内来,走至门前,忽见鸳鸯坐
在炕上,便煞住脚,笑道:“鸳鸯姐姐,今儿贵步幸临贱地!”鸳鸯只坐着,笑道:
“来请爷奶奶的安,偏又不在家的不在家,睡觉的睡觉。”贾琏笑道:“姐姐一年到
头辛苦,伏侍老太太,我还没看你去,那里还敢劳动来看我们。”又说:“巧的很。
我才要找姐姐去,因为穿着这袍子热,先来换了夹袍子,再过去找姐姐去,不想老
天爷可怜,省我走这一趟。”一面说,一面在椅子上坐下。鸳鸯因问:“又有什么说
的?”贾琏未语先笑,道:“因有一件事竟忘了,只怕姐姐还记得:上年老太太生
日,曾有一个外路和尚来孝敬一个腊油冻的佛手,因老太太爱,就即刻拿过来摆着。
因前日老太太的生日,我看古董帐,还有一笔在这帐上,却不知此时这件着落在何
处。古董房里的人也回过了我两次,等我问准了,好注了一笔。所以我问姐姐:如
今还是老太太摆着呢,还是交到谁手里去了呢?”鸳鸯听说,便说道:“老太太摆
了几日,厌烦了,就给你们奶奶了,你这会子又问我来了。我连日子还记得,还是
我打发了老王家的送来。你忘了,或是问你们奶奶和平儿。”平儿正拿衣裳,听见
如此说,忙出来回说:“交过来了,现在楼上放着呢。奶奶已经打发人去说过,他
们发昏没记上,又来叨蹬这些没要紧的事。”贾琏听说,笑道:“既然给了你奶奶,
我怎么不知道,你们就昧下了?”平儿道:“奶奶告诉二爷,二爷还要送人,奶奶
不肯,好容易留下的。这会子自己忘了,倒说我们昧下!那是什么好东西?比那强十
倍的也没昧下一遭儿,这会子就爱上那不值钱的咧?”贾琏垂头含笑想了想,拍手
道:“我如今竟糊涂了!丢三忘四,惹人抱怨,竟大不像先了。”鸳鸯笑道:“也怨不
得:事情又多,口舌又杂,你再喝上两钟酒,那里记得许多?”一面说,一面起身
要走。

贾琏忙也立起身来,说道:“好姐姐,略坐一坐儿,兄弟还有一事相求。”说着,
便骂小丫头:“怎么不沏好茶来?快拿干净盖碗,把昨日进上的新茶沏一碗来!”说
着,向鸳鸯道:“这两日,因老太太千秋,所有的几千两都使了。几处房租、地租,
统在九月才得,这会子竟接不上。明儿又要送南安府里的礼,又要预备娘娘的重阳
节,还有几家红白大礼,至少还得三二千两银子用,一时难去支借。俗语说的好:
‘求人不如求己。’说不得姐姐担个不是,暂且把老太太查不着的金银家伙,偷着
运出一箱子来,暂押千数两银子,支腾过去。不上半月的光景银子来了,我就赎了
交还,断不能叫姐姐落不是。”鸳鸯听了,笑道:“你倒会变法儿!亏你怎么想了。”
贾琏笑道:“不是我撒谎:若论除了姐姐,也还有人手里管得起千数两银子;只是
他们为人都不如你明白有胆量,我和他们一说,反吓住了他们。所以我‘宁撞金钟
一下,不打铙钹三千’。”一语未了,贾母那边小丫头子忙忙走来找鸳鸯,说:“老
太太找姐姐呢。这半日,我那里没找到?却在这里。”鸳鸯听说,忙着去见贾母。

贾琏见他去了,只得回来瞧凤姐。谁知凤姐已醒了,听他和鸳鸯借当,自己不
便答话,只躺在榻上。听见鸳鸯去了,贾琏进来,凤姐因问道:“他可应准了?”
贾琏笑道:“虽未应准,却有几分成了。须得你再去和他说一说,就十分成了。”凤
姐笑道:“我不管这些事。倘或说准了,这会子说着好听,到了有钱的时节,你就
撂在脖子后头了,谁和你打饥荒去?倘或老太太知道了,倒把我这几年的脸面都丢
了。”贾琏笑道:“好人,你要说定了,我谢你。”凤姐笑道:“你说谢我什么?”贾
琏笑道:“你说要什么就有什么。”平儿一旁笑道:“奶奶不用要别的。刚才正说要
做一件什么事,恰少一二百银子使,不如借了来,奶奶拿这么一二百银子,岂不两
全其美?”凤姐笑道:“幸亏提起我来。就是这么也罢了。”贾琏笑道:“你们太也
狠了。你们这会子别说一千两的当头,就是现银子,要三五千,只怕也难不倒。我
不和你们借就罢了!这会子烦你说一句话,还要个利钱,难为你们和我——”凤姐
不等说完,翻身起来说道:“我三千五千,不是赚的你的!如今里外上下,背着嚼说
我的不少了,就短了你来说我了!可知‘没家亲引不出外鬼来’。我们看着你家什么
石崇邓通?把我王家的缝子扫一扫,就够你们一辈子过的了。说出来的话也不害臊!
现有对证:把太太和我的嫁妆细看看,比一比,我们那一样是配不上你们的?”贾
琏笑道:“说句玩话儿就急了。这有什么的呢。你要使一二百两银子值什么?多的没
有,这还能够。先拿进来,你使了再说去,如何?”凤姐道:“我又不等着‘衔口
垫背’,忙什么呢。”贾琏道:“何苦来?犯不着这么肚火盛。”凤姐听了,又笑起来,
道:“不是我着急,你说的话戳人的心。我因为想着后日是二姐的周年,我们好了
一场,虽不能别的,到底给他上个坟,烧张纸,也是姊妹一场。他虽没个儿女留下,
也别‘前人洒土,迷了后人的眼睛’才是。”贾琏半晌方道:“难为你想的周全。”
凤姐一语倒把贾琏说没了话,低头打算,说:“既是后日才用,若明日得了这个,
你随便使多少就是了。”

一语未了,只见旺儿媳妇走进来。凤姐便问:“可成了没有?”旺儿媳妇道:“竟
不中用。我说须得奶奶作主就成了。”贾琏便问:“又是什么事?”凤姐儿见问,便
说道:“不是什么大事。旺儿有个小子,今年十七岁了还没娶媳妇儿,因要求太太
房里的彩霞,不知太太心里怎么样。前日太太见彩霞大了,二则又多病多灾的,因
此开恩打发他出去了,给他老子随便自己择女婿去罢。因此旺儿媳妇来求我。我想
他两家也就算门当户对了,一说去自然成的,谁知他这会子来了说不中用。”贾琏
道:“这是什么大事?比彩霞好的多着呢。”旺儿家的便笑道:“爷虽如此说,连他家
还看不起我们,别人越发看不起我们了。好容易相看准一个媳妇儿,我只说求爷奶
奶的恩典,替作成了,奶奶又说他必是肯的,我就烦了人过去试一试,谁知白讨了
个没趣儿。若论那孩子倒好,据我素日合意儿试他,心里没有什么说的,只是他老
子娘两个老东西太心高了些。”

一语戳动了凤姐和贾琏。凤姐因见贾琏在此,且不做一声,只看贾琏的光景。
贾琏心中有事,那里把这点事放在心里?待要不管,只是看着凤姐儿的陪房,且素
日出过力的,脸上实在过不去,因说:“什么大事?只管咕咕唧唧的!你放心且去,
我明日作媒,打发两个有体面的人,一面说一面带着定礼去,就说是我的主意。他
十分不依,叫他来见我。”旺儿家的看着凤姐,凤姐便努嘴儿。旺儿家的会意,忙
爬下就给贾琏磕头谢恩。这贾琏忙道:“你只管给你们姑奶奶磕头。我虽说了,到
底也得你们姑奶奶打发人叫他女人上来,和他好说更好些,不然太霸道了,日后你
们两亲家也难走动。”凤姐忙道:“连你还这么开恩操心呢,我反倒袖手旁观不成?
旺儿家的你听见了:这事说了,你也忙忙的给我完了事来。说给你男人,外头所有
的账目,一概赶今年年底都收进来,少一个钱也不依。我的名声不好,再放一年,
都要生吃了我呢。”旺儿媳妇笑道:“奶奶也太胆小了。谁敢议论奶奶?若收了时,
我也是一场痴心白使了。”凤姐道:“我真个还等钱做什么?不过为的是日用,出的
多,进的少。这屋里有的没的,我和你姑爷一月的月钱,再连上四个丫头的月钱,
通共一二十两银子,还不够三五天使用的呢。若不是我千凑万挪的,早不知过到什
么破窑里去了!如今倒落了一个放账的名儿。既这样,我就收了回来。我比谁不会
花钱?咱们以后就坐着花,到多早晚就是多早晚。这不是样儿?前儿老太太生日,太
太急了两个月,想不出法儿来,还是我提了一句,后楼上现有些没要紧的大铜锡家
伙,四五箱子拿出去弄了三百银子,才把太太遮羞礼儿搪过去了。我是你们知道的:
那一个金自鸣钟卖了五百六十两银子,没有半个月,大事小事没十件,白填在里头。
今儿外头也短住了,不知是谁的主意,搜寻上老太太了。明儿再过一年,便搜寻到
头面衣裳,可就好了!”旺儿媳妇笑道:“那一位太太奶奶的头面衣裳,折变了不够
过一辈子的?只是不肯罢咧。”凤姐道:“不是我说没能耐的话,要像这么着我竟不
能了。昨儿晚上,忽然做了个梦,说来可笑:梦见一个人,虽然面善,却又不知名
姓,找我说娘娘打发他来,要一百匹锦。我问他是那一位娘娘,他说的又不是咱们
的娘娘。我就不肯给他,他就来夺。正夺着,就醒了。”旺儿家的笑道:“这是奶奶
日间操心,惦记应候宫里的事。”

一语未了,人回:“夏太监打发了一个小内家来说话。”贾琏听了,忙皱眉道:
“又是什么话?一年他们也搬够了。”凤姐道:“你藏起来,等我见他。若是小事罢
了,若是大事,我自有回话。”贾琏便躲入内套间去。这里凤姐命人带进小太监来,
让他椅上坐了吃茶,因问何事。那小太监便说:“夏爷爷因今儿偶见一所房子,如
今竟短二百两银子,打发我来问舅奶奶家里,有现成的银子暂借一二百,这一两日
就送来。”凤姐儿听了,笑道:“什么是送来?有的是银子,只管先兑了去。改日等
我们短住,再借去也是一样。”小太监道:“夏爷爷还说:上两回还有一千二百两银
子没送来,等今年年底下自然一齐都送过来的。”凤姐笑道:“你夏爷爷好小气。这
也值的放在心里?我说一句话,不怕他多心:要都这么记清了还我们,不知要还多
少了。只怕我们没有,要有只管拿去。”因叫旺儿媳妇来,“出去,不管那里先支二
百银来。”旺儿媳妇会意,因笑道:“我才因别处支不动,才来和奶奶支的。”凤姐
道:“你们只会里头来要钱,叫你们外头弄去,就不能了。”说着,叫平儿:“把我
那两个金项圈拿出去,暂且押四百两银子。”平儿答应去了,果然拿了一个锦盒子
来,里面两个锦袱包着。打开时,一个金累丝攒珠的,那珍珠都有莲子大小;一个
点翠嵌宝石的:两个都与宫中之物不离上下。一时拿去,果然拿了四百两银子来。
凤姐命给小太监打叠一半,那一半与了旺儿媳妇,命他拿去办八月中秋的节。那小
太监便告辞了,凤姐命人替他拿着银子,送出大门去了。这里贾琏出来笑道:“这
一起外祟,何日是了!”凤姐笑道:“刚说着,就来了一股子。”贾琏道:“昨儿周太
监来,张口一千两,我略应慢了些,他就不自在。将来得罪人的地方儿多着呢。这
会子再发个三五万的财就好了!”一面说,一面平儿伏侍凤姐另洗了脸、更衣,往
贾母处伺候晚饭。

这里贾琏出来,刚至外书房,忽见林之孝走来。贾琏因问何事。林之孝说道:
“才听见雨村降了,却不知何事。只怕未必真。”贾琏道:“真不真,他那官儿未必
保的长。只怕将来有事,咱们宁可疏远着他好。”林之孝道:“何从不是?只是一时
难以疏远。如今东府大爷和他更好,老爷又喜欢他,时常来往,那个不知?”贾琏
道:“横竖不和他谋事,也不相干。你去再打听真了是为什么。”林之孝答应了,却
不动身,坐在椅子上再说闲话。因又说起家道艰难,便趁势说:“人口太众了。不
如拣个空日回明老太太老爷,把这些出过力的老家人,用不着的,开恩放几家出去:
一则他们各有营运,二则家里一年也省口粮月钱。再者,里头的姑娘也太多。俗语
说,‘一时比不得一时’如今说不得先时的例了,少不的大家委屈些,该使八个的
使六个,使四个的使两个。若各房算起来,一年也可以省许多月米月钱。况且里头
的女孩子们,一半都大了,也该配人的配人,成了房,岂不又滋生出些人来?”贾
琏道:“我也这么想,只是老爷才回家来,多少大事未回,那里议到这个上头?前儿
官媒拿了个庚帖来求亲,太太还说老爷才来家,每日欢天喜地的说骨肉完聚,忽然
提起这事,恐老爷又伤心,所以且不叫提起。”林之孝道:“这也是正理,太太想的
周到。”贾琏道:“正是,提起这话,我想起一件事来:我们旺儿的小子,要说太太
屋里的彩霞,他昨儿求我。我想什么大事,不管谁去说一声去,就说我的话。”林
之孝答应了,半晌笑道:“依我说,二爷竟别管这件事。旺儿的那小子虽然年轻,
在外吃酒赌钱,无所不至。虽说都是奴才,到底是一辈子的事。彩霞这孩子这几年
我虽没看见,听见说越发出跳的好了,何苦来白遭塌一个人呢?”贾琏道:“哦!他
小子竟会喝酒不成人吗?这么着,那里还给他老婆?且给他一顿棍,锁起来,再问他
老子娘。”林之孝笑道:“何必在这一时?等他再生事,我们自然回爷处治,如今且
也不用究办。”贾琏不语。一时林之孝出去。

晚间凤姐已命人唤了彩霞之母来说媒。那彩霞之母满心纵不愿意,见凤姐自和
他说,何等体面,便心不由己的满口应了出去。凤姐又问贾琏:“可说了没有?”
贾琏因说:“我原要说来着,听见他这小子大不成人,所以还没说。若果然不成人,
且管教他两日,再给他老婆不迟。”凤姐笑道:“我们王家的人,连我还不中你们的
意,何况奴才呢。我已经和他娘说了,他娘倒欢天喜地,难道又叫进他来不要了不
成?”贾琏道:“你既说了,又何必退呢?明日说给他老子,好生管他就是了。”这
里说话不提。

且说彩霞因前日出去等父母择人,心中虽与贾环有旧,尚未作准。今日又见旺
儿每每来求亲,早闻得旺儿之子酗酒赌博,而且容颜丑陋,不能如意。自此,心中
越发懊恼,惟恐旺儿仗势作成,终身不遂,未免心中急躁。至晚间,悄命他妹子小
霞进二门来找赵姨娘,问个端底。赵姨娘素日深与彩霞好,巴不得给了贾环,方有
个膀臂,不承望王夫人又放出去了。每每调唆贾环去讨,一则贾环羞口难开,二则
贾环也不在意,不过是个丫头,他去了将来自然还有好的,遂迁延住不肯说去,意
思便丢开了手。无奈赵姨娘又不舍,又见他妹子来问,是晚得空,便先求了贾政。
贾政说道:“且忙什么。等他们再念一二年书,再放人不迟。我已经看中了两个丫
头,一个给宝玉,一个给环儿。只是年纪还小,又怕他们误了念书,再等一二年再
提。”赵姨娘还要说话,只听外面一声响,不知何物,大家吃了一惊。

未知如何,下回分解。