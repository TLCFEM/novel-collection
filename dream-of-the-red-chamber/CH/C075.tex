\chapter{开夜宴异兆发悲音~赏中秋新词得佳谶}

话说尤氏从惜春处赌气出来,正欲往王夫人处去,跟从的老嬷嬷们因悄悄的
道:“回奶奶:且别往上屋里去。才有甄家的几个人来,还有些东西,不知是什么
机密事。奶奶这一去恐怕不便。”尤氏听了道:“昨日听见你老爷说看见抄报上甄
家犯了罪,现今抄没家私,调取进京治罪。怎么又有人来?”老嬷嬷道:“正是呢。
才来了几个女人,气色不成气色,慌慌张张的,想必有什么瞒人的事。”尤氏听了,
便不往前去,仍往李纨这边来了。

恰好太医才诊了脉去,李纨近日也觉清爽了些,拥衾倚枕坐在床上,正欲人来
说些闲话。因见尤氏进来,不似方才和蔼,只呆呆的坐着,李纨因问道:“你过来
了,可吃些东西?只怕饿了?”命素云:“瞧有什么新鲜点心拿来。”尤氏忙止道:
“不必不必。你这一向病着,那里有什么新鲜东西?况且我也不饿。”李纨道:“昨
日人家送来的好茶面子,倒是对碗来你喝罢。”说毕,便吩咐去对茶。尤氏出神无
语。跟来的丫头媳妇们因问:“奶奶今日晌午尚未洗脸,这会子趁便可净一净好?”
尤氏点头。李纨忙命素云来取自己妆奁。素云又将自己脂粉拿来,笑道:“我们奶
奶就少这个。奶奶不嫌腌,能着用些。”李纨道:“我虽没有,你就该往姑娘们
那里取去,怎么公然拿出你的来?幸而是他,要是别人,岂不恼呢?”尤氏笑道:
“这有何妨?”说着,一面洗脸。丫头只弯腰捧着脸盆。李纨道:“怎么这样没规
矩?”那丫头赶着跪下。尤氏笑道:“我们家下大小的人,只会讲外面,假礼假体
面,究竟做出来的事都够使的了。”李纨听如此说,便已知道昨夜的事,因笑道:
“你这话有因。是谁做的事够使的了?”尤氏道:“你倒问我,你敢是病着过阴去
了?”

一语未了,只见人报:“宝姑娘来了。”二人忙说快请,宝钗已走进来。尤氏
忙擦脸起身让坐,因问:“怎么一个人忽然走进来,别的姊妹都不见?”宝钗道:
“正是,我也没有见他们。只因今日我们奶奶身上不自在,家里两个女人也都因时
症未起炕,别的靠不得,我今儿要出去陪着老人家夜里作伴。要去回老太太、太太,
我想又不是什么大事,且不用提,等好了,我横竖进来呢。所以来告诉大嫂子一声。”
李纨听说,只看着尤氏笑,尤氏也看着李纨笑。一时尤氏盥洗已毕,大家吃面茶。
李纨因笑着向宝钗道:“既这样,且打发人去请姨娘的安,问是何病。我也病着,
不能亲自来瞧。好妹妹,你去只管去,我且打发人去到你那里去看屋子。你好歹住
一两天,还进来,别叫我落不是。”宝钗笑道:“落什么不是呢?也是人之常情。
你又不曾卖放了贼。依我的主意,也不必添人过去,竟把云丫头请了来,你和他住
一两日,岂不省事?”尤氏道:“可是,史大妹妹往那里去了?”宝钗道:“我才
打发他们找你们探丫头去了,叫他同到这里来,我也明白告诉他。”

正说着,果然报:“云姑娘和三姑娘来了。”大家让坐已毕,宝钗便说要出去
一事。探春道:“很好。不但姨妈好了还来,就便好了不来也使得。”尤氏笑道:
“这话又奇了,怎么撵起亲戚来了?”探春冷笑道:“正是呢,有别人撵的,不如
我先撵。亲戚们好,也不必要死住着才好。咱们倒是一家子亲骨肉呢,一个个不像
乌眼鸡似的?恨不得你吃了我,我吃了你!”尤氏忙笑道:“我今儿是那里来的晦
气?偏都碰着你姐儿们气头儿上了。”探春道:“谁叫你趁热灶火来了?”因问:
“谁又得罪了你呢?”因又寻思,道:“凤丫头也不犯合你怄气。是谁呢?”尤氏
只含糊答应。探春知他怕事,不肯多言,因笑道:“你别装老实了。除了朝廷治罪,
没有砍头的,你不必唬的这个样儿。告诉你罢:我昨日把王善保的老婆打了,我还
顶着徒罪呢。也不过背地里说些闲话罢咧,难道也还打我一顿不成?”宝钗忙问:
“因何又打他?”探春悉把昨夜的事一一都说了。尤氏见探春已经说出来了,便把
惜春方才的事也说了一遍。探春道:“这是他向来的脾气,孤介太过,我们再扭不
过他的。”又告诉他们说:“今日一早不见动静,打听凤丫头病着,就打发人四下
里打听王善保家的是怎么样。回来告诉我说:‘王善保家的挨了一顿打,嗔着他多
事。’”尤氏李纨道:“这倒也是正理。”探春冷笑道:“这种遮人眼目儿的事,
谁不会做?且再瞧就是了。”尤氏李纨皆默无所答。一时,丫头们来请用饭,湘云
宝钗回房打点衣衫,不在话下。

尤氏辞了李纨,往贾母这边来。贾母歪在榻上,王夫人正说甄家因何获罪,如
今抄没了家产,来京治罪等话。贾母听了,心中甚不自在。恰好见他姊妹来了,因
问:“从那里来的?可知凤姐儿妯娌两个病着,今日怎么样?”尤氏等忙回道:“今
日都好些。”贾母点头叹道:“咱们别管人家的事,且商量咱们八月十五赏月是正
经。”王夫人笑道:“已预备下了,不知老太太拣那里好?只是园里恐夜晚风凉。”
贾母笑道:“多穿两件衣服何妨?那里正是赏月的地方,岂可倒不去的?”说话之
间,媳妇们抬过饭桌,王夫人尤氏等忙上来放箸捧饭。贾母见自己几色菜已摆完,
另有两大捧盒内盛了几色菜,便是各房孝敬的旧规矩。贾母说:“我吩咐过几次,
蠲了罢,你们都不听。”王夫人笑道:“不过都是家常东西。今日我吃斋,没有别
的孝顺。那些面筋豆腐,老太太又不甚爱吃,只拣了一样椒油莼酱来。”贾母笑
道:“我倒也想这个吃。”鸳鸯听说,便将碟子挪在跟前。宝琴一一的让了,方归
坐。贾母便命探春来同吃。探春也都让过了,便和宝琴对面坐下,侍书忙去取了碗
箸。鸳鸯又指那几样菜道:“这两样看不出是什么东西来,是大老爷孝敬的。这一
碗是鸡髓笋,是外头老爷送上来的。”一面说,一面就将这碗笋送至桌上。贾母略
尝了两点,便命:“将那几样着人都送回去,就说我吃了,以后不必天天送。我想
吃什么自然着人来要。”媳妇们答应着仍送过去,不在话下。

贾母因问:“拿稀饭来吃些罢。”尤氏早捧过一碗来,说是红稻米粥。贾母接
来吃了半碗,便吩咐:“将这粥送给凤姐儿吃去。”又指着这一盘果子:“独给平
儿吃去。”又向尤氏道:“我吃了,你就来吃了罢。”尤氏答应着,待贾母漱口洗
手毕。贾母便下地,和王夫人说闲话行食,尤氏告坐吃饭。贾母又命鸳鸯等来陪吃。
贾母见尤氏吃的仍是白米饭,因问说:“怎么不盛我的饭?”丫头们回道:“老太
太的饭完了。今日添了一位姑娘,所以短了些。”鸳鸯道:“如今都是‘可着头做
帽子’了,要一点儿富馀也不能的。”王夫人忙回道:“这一二年旱涝不定,庄上
的米都不能按数交的。这几样细米更艰难,所以都是可着吃的做。”贾母笑道:“正
是:‘巧媳妇做不出没米儿粥来。’”众人都笑起来。鸳鸯一面回头向门外伺候媳
妇们道:“既这样,你们就去把三姑娘的饭拿来添上,也是一样。”尤氏笑道:“我
这个就够了,也不用去取。”鸳鸯道:“你够了,我不会吃的?”媳妇们听说,方
忙着取去了。

一时王夫人也去用饭。这里尤氏直陪贾母说话取笑到起更的时候,贾母说:“你
也过去罢。”尤氏方告辞出来。走至二门外,上了车,众媳妇放下帘子来,四个小
厮拉出来,套上牲口,几个媳妇带着小丫头子们先走,到那边大门口等着去了。这
里送的丫鬟们也回来了。

尤氏在车内,因见自己门首两边狮子下,放着四五辆大车,便知系来赴赌之人,
向小丫头银蝶儿道:“你看,坐车的是这些,骑马的又不知有几个呢。”说着进府,
已到了厅上,贾蓉媳妇带了丫鬟媳妇也都秉着羊角手罩接出来了。尤氏笑道:“成
日家我要偷着瞧瞧他们赌钱也没得便,今儿倒巧,顺便打他们窗户跟前走过去。”
众媳妇答应着,提灯引路。又有一个先去悄悄的知会伏侍的小厮们,不许失惊打怪。
于是尤氏一行人悄悄的来至窗下,只听里面称三赞四,耍笑之音虽多,又兼有恨五
骂六,忿怨之声亦不少。

原来贾珍近因居丧,不得游玩,无聊之极,便生了个破闷的法子,日间以习射
为由,请了几位世家弟兄及诸富贵亲友来较射。因说:“白白的只管乱射终是无益,
不但不能长进,且坏了式样;必须立了罚约,赌个利物,大家才有勉力之心。”因
此,天香楼下箭道内立了鹄子,皆约定每日早饭后时射鹄子。贾珍不好出名,便命
贾蓉做局家。这些都是少年,正是斗鸡走狗、问柳评花的一干游侠纨。因此大家
议定,每日轮流做晚饭之主。天天宰猪割羊,屠鹅杀鸭,好似临潼斗宝的一般,都
要卖弄自己家里的好厨役好烹调。不到半月工夫,贾政等听见这般,不知就里,反
说:“这才是正理。文既误了,武也当习,况在武荫之属。”遂也令宝玉、贾环、
贾琮、贾兰等四人,于饭后过来跟着贾珍,习射一回方许回去。贾珍志不在此,再
过几日,便渐次以歇肩养力为由,晚间或抹骨牌,赌个酒东儿,至后渐次至钱。如
今三四个月的光景,竟一日一日赌胜于射了,公然斗叶掷骰,放头开局,大赌起来。
家下人借此各有些利益,巴不得如此,所以竟成了局势。外人皆不知一字。

近日邢夫人的胞弟邢德全也酷好如此,所以也在其中。又有薛蟠头一个惯喜送
钱与人的,见此岂不快乐?这邢德全虽系邢夫人的胞弟,却居心行事,大不相同。
他只知吃酒赌钱、眠花宿柳为乐,手中滥漫使钱,待人无心,因此都叫他“傻大舅”。
薛蟠早已出名的“呆大爷”。今日二人凑在一处,都爱抢快,便又会了两家,在外
间炕上抢快。又有几个,在当地下大桌子上赶羊。里间又有一起斯文些的抹骨牌,
打天九。此间伏侍的小厮都是十五岁以下的孩子。——此是前话。

且说尤氏潜至窗外偷看。其中有两个陪酒的小么儿,都打扮的粉妆锦饰。今日
薛蟠又掷输了,正没好气,幸而后手里渐渐翻过来了,除了冲账的反赢了好些,心
中自是兴头起来。贾珍道:“且打住,吃了东西再来。”因问:“那两处怎么样?”
此时打天九赶老羊的未清,先摆下一桌,贾珍陪着吃。薛蟠兴头了,便搂着一个小
么儿喝酒,又命将酒去敬傻大舅。傻大舅输家没心肠,喝了两碗便有些醉意,嗔着
陪酒的小么儿只赶赢家不理输家了,因骂道:“你们这起兔子,真是些没良心的忘
八羔子!天天在一处,谁的恩你们不沾?只不过这会子输了几两银子,你们就这么三
六九等儿的了。难道从此以后再没有求着我的事了?”众人见他带酒,那些输家不
便言语,只抿着嘴儿笑。那些赢家忙说:“大舅骂的很是。这小狗攮的们都是这个
风俗儿。”因笑道:“还不给舅太爷斟酒呢。”两个小孩子都是演就的圈套,忙都
跪下奉酒,扶着傻大舅的腿,一面撒娇儿说道:“你老人家别生气,看着我们两个
小孩子罢。我们师父教的:不论远近厚薄,只看一时有钱的就亲近。你老人家不信,
回来大大的下一注,赢了,白瞧瞧我们两个是什么光景儿。”说的众人都笑了。这
傻大舅掌不住也笑了,一面伸手接过酒来,一面说道:“我要不看着你们两个素日
怪可怜见儿的,我这一脚把你们的小蛋黄子踢出来。”说着,把腿一抬。两个孩子
趁势儿爬起来,越发撒娇撒痴,拿着洒花绢子托了傻大舅的手,把那钟酒灌在傻大
舅嘴里。傻大舅哈哈的笑着,一扬脖儿把一钟酒都干了。因拧了那孩子的脸一下儿,
笑说道:“我这会子看着,又怪心疼的了。”说着,忽然想起旧事来,乃拍案对贾
珍说道:“昨日我和你令伯母怄气,你可知道么?”贾珍道:“没有听见。”傻大
舅叹道:“就为钱这件东西!老贤甥,你不知我们邢家的底里。我们老太太去世时,
我还小呢,世事不知。他姐妹三个人,只有你令伯母居长。他出阁时,把家私都带
过来了。如今你二姨儿也出了门子了,他家里也很艰窘。你三姨儿尚在家里。一应
用度,都是这里陪房王善保家的掌管。我就是来要几个钱,也并不是要贾府里的家
私,我邢家的家私也就够我花了。无奈竟不得到手,你们就欺负我没钱!”贾珍见
他酒醉,外人听见不雅,忙用话解劝。

外面尤氏等听得十分真切,乃悄向银蝶儿等笑说:“你听见了,这是北院里大
太太的兄弟抱怨他呢。可见他亲兄弟还是这样,就怨不得这些人了。”因还要听时,
正值赶老羊的那些人也歇住了,要酒。有一个人问道:“方才是谁得罪了舅太爷?
我们竟没听明白。且告诉我们,评评理。”邢德全便把两个陪酒的孩子不理的话说
了一遍。那人接过来就说:“可恼,怨不得舅太爷生气。我问你:舅太爷不过输了
几个钱罢咧,并没有输掉了,怎么你们就不理了?”说着,大家都笑起来。邢
德全也喷了一地饭,说:“你这个东西,行不动儿就撒村捣怪的。”尤氏在外面听
了这话,悄悄的啐了一口,骂道:“你听听这一起没廉耻的小挨刀的!再灌丧了黄
汤,还不知出些什么新样儿的来呢。”一面便进去卸妆安歇。至四更时,贾珍方
散,往佩凤房里去了。

次日起来,就有人回:“西瓜月饼都全了,只待分派送人。”贾珍吩咐佩凤道:
“你请奶奶看着送罢,我还有别的事呢。”佩凤答应去了,回了尤氏,一一分派,
遣人送去。一时佩凤来说:“爷问奶奶今儿出门不出门?说咱们是孝家,十五过不
得节,今儿晚上倒好,可以大家应个景儿。”尤氏道:“我倒不愿意出门呢。那边
珠大奶奶又病了,琏二奶奶也躺下了,我再不去,越发没个人了。”佩凤道:“爷
说,奶奶出门,好歹早些回来,叫我跟了奶奶去呢。”尤氏道:“既这么样,快些
吃了,我好走。”佩凤道:“爷说早饭在外头吃,请奶奶自己吃罢。”尤氏问道:
“今日外头有谁?”佩凤道:“听见外头有两个南京新来的,倒不知是谁。”说毕,
吃饭更衣,尤氏等仍过荣府来,至晚方回去。

果然贾珍煮了一口猪,烧了一腔羊,备了一桌菜蔬果品。在汇芳园丛绿堂中,
带领妻子姬妾先吃过晚饭,然后摆上酒,开怀作乐赏月。将一更时分,真是风清月
朗,银河微隐。贾珍因命佩凤等四个人也都入席,下面一溜坐下,猜枚拳。饮了
一回,贾珍有了几分酒,高兴起来,便命取了一支紫竹箫来,命佩凤吹箫,文花唱
曲。喉清韵雅,甚令人心动神移。唱罢,复又行令。那天将有三更时分,贾珍酒已
八分。大家正添衣喝茶、换盏更酌之际,忽听那边墙下有人长叹之声。大家明明听
见,都毛发竦然。贾珍忙厉声叱问:“谁在那边?”连问几声,无人答应。尤氏道:
“必是墙外边家里人,也未可知。”贾珍道:“胡说,这墙四面皆无下人的房子,
况且那边又紧靠着祠堂,焉得有人?”一语未了,只听得一阵风声,竟过墙去了。
恍惚闻得祠堂内扇开阖之声,只觉得风气森森,比先更觉凄惨起来。看那月色时,
也淡淡的,不似先前明朗。众人都觉毛发倒竖。贾珍酒已吓醒了一半,只比别人拿
得住些,心里也十分警畏,便大没兴头,勉强又坐了一会,也就归房安歇去了。

次日一早起来,乃是十五日,带领众子侄开祠行朔望之礼。细察祠内,都仍是
照旧好好的,并无怪异之迹。贾珍自为醉后自怪,也不提此事。礼毕仍旧闭上门,
看着锁禁起来。

贾珍夫妻至晚饭后方过荣府来。只见贾赦、贾政都在贾母房里坐着说闲话儿,
与贾母取笑呢。贾琏、宝玉、贾环、贾兰皆在地下侍立。贾珍来了,都一一见过,
说了两句话,贾珍方在挨门小杌子上告了坐,侧着身子坐下。贾母笑问道:“这两
日你宝兄弟的箭如何了?”贾珍忙起身笑道:“大长进了,不但式样好,而且弓也
长了一个劲。”贾母道:“这也够了,且别贪力,仔细努伤着。”贾珍忙答应了几
个“是”。贾母又道:“你昨日送来的月饼好。西瓜看着倒好,打开却也不怎么样。”
贾珍陪笑道:“月饼是新来的一个饽饽厨子,我试了试果然好,才敢做了孝敬来的。
西瓜往年都还可以,不知今年怎么就不好了。”贾政道:“大约今年雨水太勤之过。”
贾母笑道:“此时月亮已上来了,咱们且去上香。”说着,便起身扶着宝玉的肩,
带领众人齐往园中来。

当下园子正门俱已大开,挂着羊角灯。嘉荫堂月台上,焚着斗香,秉着烛,陈
设着瓜果月饼等物。邢夫人等皆在里面久候。真是月明灯彩,人气香烟,晶艳氤氲,
不可名状。地下铺着拜毡锦褥,贾母盥手上香拜毕,于是大家皆拜过。贾母便说:
“赏月在山上最好。”因命在那山上的大花厅上去,众人听说,就忙着在那里铺设。
贾母且在嘉荫堂中吃茶少歇,说些闲话。一时人回:“都齐备了。”贾母方扶着人
上山来。王夫人等因回说:“恐石上苔滑,还是坐竹椅上去。”贾母道:“天天打
扫,况且极平稳的宽路,何不疏散疏散筋骨也好?”于是贾赦贾政等在前引导,又
是两个老婆秉着两把羊角手罩,鸳鸯、琥珀、尤氏等贴身搀扶,邢夫人等在后围随。
从下逶迤不过百余步,到了主山峰脊上,便是一座敞厅。因在山之高脊,故名曰凸
碧山庄。厅前平台上列下桌椅,又用一架大围屏隔做两间,凡桌椅形式皆是圆的,
特取团圆之意。上面居中贾母坐下,左边贾赦、贾珍、贾琏、贾蓉,右边贾政、宝
玉、贾环、贾兰。团团围坐,只坐了半桌,下面还有半桌馀空。贾母笑道:“往常
倒还不觉人少,今日看来,究竟咱们的人也甚少,算不得什么。想当年过的日子,
今夜男女三四十个,何等热闹,今日那有那些人?如今叫女孩儿们来坐那边罢。”
于是令人向围屏后邢夫人等席上将迎春、探春、惜春三个叫过来。贾琏宝玉等一齐
出坐,先尽他姊妹坐下,然后在下依次坐定。贾母便命折一枝桂花来,叫个媳妇在
屏后击鼓传花:“若花在手中,饮酒一杯,罚说笑话一个。”

于是先从贾母起,次贾赦,一一接过。鼓声两转,恰恰在贾政手中住了,只得
饮了酒。众姊妹弟兄都你悄悄的扯我一下,我暗暗的又捏你一把,都含笑心里想着,
倒要听是何笑话儿。贾政见贾母欢喜,只得承欢。方欲说时,贾母又笑道:“要说
的不笑了,还要罚。”贾政笑道:“只得一个,若不说笑了,也只好愿罚。”贾母
道:“你就说这一个。”贾政因说道:“一家子一个人最怕老婆,”只说了这一句,
大家都笑了,因从没听见贾政说过所以才笑。贾母笑道:“这必是好的。”贾政笑
道:“若好,老太太先多吃一杯。”贾母笑道:“使得。”贾赦连忙捧杯,贾政执
壶,斟了一杯。贾赦仍旧递给贾政,贾赦旁边侍立。贾政捧上,安放在贾母面前,
贾母饮了一口。贾赦贾政退回本位。

于是贾政又说道:“这个怕老婆的人,从不敢多走一步。偏偏那日是八月十五,
到街上买东西,便见了几个朋友,死活拉到家里去吃酒。不想吃醉了,便在朋友家
睡着了。第二日醒了,后悔不及,只得来家赔罪。他老婆正洗脚,说:‘既是这样,
你替我舔舔就饶你。’这男人只得给他舔,未免恶心要吐。他老婆便恼了,要打,
说:‘你这样轻狂!’吓得他男人忙跪下求说:‘并不是奶奶的脚腌,只因昨儿
喝多了黄酒,又吃了月饼馅子,所以今日有些作酸呢。’”说得贾母和众人都笑了。
贾政忙又斟了一杯送与贾母。贾母笑道:“既这样,快叫人取烧酒来,别叫你们有
媳妇的人受累。”众人又都笑起来。只贾琏宝玉不敢大笑。

于是又击鼓,便从贾政起,可巧到宝玉鼓止。宝玉因贾政在坐,早已不安,
偏又在他手中,因想:“说笑话,倘或说不好了,又说没口才;说好了,又说正经
的不会,只惯贫嘴,更有不是。不如不说。”乃起身辞道:“我不能说,求限别的
罢。”贾政道:“既这样,限个‘秋’字,就即景做一首诗。好便赏你;若不好,
明日仔细!”贾母忙道:“好好的行令,怎么又做诗?”贾政陪笑道:“他能的。”
贾母听说:“既这样,就做。快命人取纸笔来。”贾政道:“只不许用这些‘水’
‘晶’‘冰’‘玉’‘银’‘彩’‘光’‘明’‘素’等堆砌字样。要另出主见,
试试你这几年情思。”宝玉听了,碰在心坎儿上,遂立想了四句,向纸上写了,呈
与贾政看。贾政看了,点头不语。贾母见这般,知无甚不好,便问:“怎么样?”
贾政因欲贾母喜欢,便说:“难为他。只是不肯念书,到底词句不雅。”贾母道:
“这就罢了。就该奖励,以后越发上心了。”贾政道:“正是。”因回头命个老嬷
嬷出去,“吩咐小厮们,把我海南带来的扇子取来给两把与宝玉。”宝玉磕了一个
头,仍复归坐行令。

当下贾兰见奖励宝玉,他便出席,也做一首,呈与贾政看。贾政看了,更觉欣
喜。遂并讲与贾母听时,贾母也十分欢喜,也忙令贾政赏他。于是大家归坐,复行
起令来。

这次贾赦手内住了,只得吃了酒,说笑话。因说道:“一家子一个儿子最孝顺,
偏生母亲病了。各处求医不得,便请了一个针灸的婆子来。这婆子原不知道脉理,
只说是心火,一针就好了。这儿子慌了,便问:‘心见铁就死,如何针得?’婆子
道:‘不用针心,只针肋条就是了。’儿子道:‘肋条离心远着呢,怎么就好了呢?’
婆子道:‘不妨事。你不知天下作父母的,偏心的多着呢!’”众人听说,也都笑
了。贾母也只得吃半杯酒,半日笑道:“我也得这婆子针一针就好了。”贾赦听说,
自知出言冒撞,贾母疑心,忙起身笑与贾母把盏,以别言解释。

贾母亦不好再提,且行令。不料这花却在贾环手里。贾环近日读书稍进,亦好
外务。今见宝玉做诗受奖,他便技痒,只当着贾政,不敢造次。如今可巧花在手中,
便也索纸笔来,立就一绝,呈与贾政。贾政看了,亦觉罕异,只见词句中终带着不
乐读书之意,遂不悦道:“可见是弟兄了:发言吐意,总属邪派。古人中有‘二难’,
你两个也可以称‘二难’了。就只不是那一个‘难’字,却是做‘难以教训’‘难’
字讲才好。哥哥是公然温飞卿自居,如今兄弟又自为曹唐再世了。”说得众人都笑
了。

贾赦道:“拿诗来我瞧。”便连声赞好,道:“这诗据我看,甚是有气骨。想
来咱们这样人家,原不必寒窗萤火,只要读些书,比人略明白些,可以做得官时,
就跑不了一个官儿的。何必多费了工夫,反弄出书呆子来?所以我爱他这诗,竟不
失咱们侯门的气概。”因回头吩咐人去取自己的许多玩物来赏赐与他,因又拍着贾
环的脑袋笑道:“以后就这样做去,这世袭的前程就跑不了你袭了。”贾政听说,
忙劝说:“不过他胡诌如此,那里就论到后事了?”说着,便斟了酒,又行了一回
令。贾母便说:“你们去罢。自然外头还有相公们候着,也不可轻忽了他们。况且
二更多了,你们散了,再让姑娘们多乐一会子,好歇着了。”贾政等听了方止令起
身,大家公进了一杯酒,才带着子侄们出去了。

要知端底,下回分解。