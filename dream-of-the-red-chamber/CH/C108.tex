\chapter{强欢笑蘅芜庆生辰~死缠绵潇湘闻鬼哭}

却说贾政先前曾将房产并大观园奏请入官,内廷不收,又无人居住,只好封锁。
因园子接连尤氏惜春住宅,太觉旷阔无人,遂将包勇罚看荒园。此时贾政理家,奉
了贾母之命,将人口渐次减少,诸凡省俭,尚且不能支持。幸喜凤姐是贾母心爱的
人,王夫人等虽不大喜欢,若说治家办事,尚能出力,所以内事仍交凤姐办理。但
近来因被抄以后,诸事运用不来,也是每形拮据。那些房头上下人等原是宽裕惯了
的,如今较往日十去其七,怎能周到?不免怨言不绝。凤姐也不敢推辞,在贾母前
扶病承欢。过了些时,贾赦贾珍各到当差地方,恃有用度,暂且自安。写书回家,
都言安逸,家中不必挂念。于是贾母放心,邢夫人尤氏也略略宽怀。

一日,史湘云出嫁回门,来贾母这边请安。贾母提起他女婿甚好,史湘云也将
那里家中平安的话说了,请老太太放心。又提起黛玉去世,不免大家落泪。贾母又
想起迎春苦楚,越觉悲伤起来。史湘云解劝一回,又到各家请安问好毕,仍到贾母
房中安歇。言及薛家这样人家,“被薛大哥闹的家破人亡,今年虽是缓决人犯,明
年不知可能减等?”贾母道:“你还不知道呢:昨儿蟠儿媳妇死的不明白,几乎又
闹出一场事来。还幸亏老佛爷有眼,叫他带来的丫头自己供出来了,那夏奶奶没的
闹了,自家拦住相验,你姨妈这里才将皮裹肉的打发出去了。如今守着蝌儿过日子。
这孩子却有良心,他说哥哥在监里尚没完事,不肯娶亲。你邢妹妹在大太太那边,
也就很苦。琴姑娘为他公公死了还没满服,梅家尚未娶去。你说说,真真是‘六亲
同运’:薛家是这么着;二太太的娘家大舅太爷一死,凤丫头的哥哥也不成人;那
二舅太爷是个小气的,又是官项不清,也是打饥荒;甄家自从抄家以后,别无信息。”
湘云道:“三姐姐去了,曾有书字回来么?”贾母道:“自从出了嫁,二老爷回来说,
你三姐姐在海疆很好。只是没有书信,我也是日夜惦记。为我们家连连的出些不好
事,所以我也顾不来。如今四丫头也没有给他提亲。环儿呢,谁有功夫提起他来?
如今我们家的日子比你从前在这里的时候更苦了。只可怜你宝姐姐,自过了门,没
过一天舒服日子。你二哥哥还是那么疯疯颠颠,这怎么好呢!”

湘云道:“我从小儿在这里长大的,这里那些人的脾气,我都知道的。这一回
来了,竟都改了样子了。我打量我隔了好些时没来,他们生疏我;我细想起来,竟
不是的。就是见了我,瞧他们的意思,原要像先一样的热闹,不知道怎么说说就伤
起心来了,所以我坐了坐儿就到老太太这里来了。”贾母道:“如今的日子在我也罢
了,他们年轻轻儿的人,还了得。我正要想个法儿,叫他们还热闹一天才好,只是
打不起这个精神来。”湘云道:“我想起来了:宝姐姐不是后儿的生日吗?我多住一
天,给他拜个寿,大家热闹一天。不知老太太怎么样?”贾母道:“我真正气糊涂
了。你不提,我竟忘了。后日可不是他的生日吗!我明日拿出钱来,给他办个生日。
他没有定亲的时候,倒做过好几次,如今过了门倒没有做。宝玉这孩子,头里很伶
俐,很淘气;如今因为家里的事不好,把这孩子越发弄的话都没有了。倒是珠儿婚
妇还好。他有的时候是这么着,没的时候他也是这么着,带着兰儿静静儿的过日子,
倒难为他。”湘云道:“别人还不离,独有琏二嫂子,连模样儿都改了,说话也不伶
俐了。明日等我来引逗他们,看他们怎么样。但只他们嘴里不说,心里要抱怨我,
说我有了——”刚说到这里,却把个脸飞红了。贾母会意道:“这怕什么?当初姊妹
们都是在一处乐惯了的,说说笑笑,再别留这些心。大凡一个人有也罢没也罢,总
要受得富贵、耐得贫贱才好呢。你宝姐姐生来是个大方的人。头里他家这样好,他
也一点儿不骄傲;后来他家坏了事,他也是舒舒坦坦的。如今在我家里,宝玉待他
好,他也是那样安顿;一时待他不好,也不见他有什么烦恼。我看这孩子倒是个有
福的。你林姐姐他就最小性儿,又多心,所以到底儿不长命的。凤丫头也见过些事,
很不该略见些风波就改了样子。他若这样没见识,也就是小器了。后儿宝丫头的生
日,我另拿出银子来,热热闹闹的给他做个生日,也叫他喜欢这么一天。”湘云答
应道:“老太太说的很是。索性把那些姐妹们都请了来,大家叙一叙。”贾母道:“自
然要请的。”一时高兴,遂叫鸳鸯拿出一百银子来,交给外头:“叫他明日起,预备
两天的酒饭。”鸳鸯领命,叫婆子交了出去。一宿无话。

次日传话出去,打发人去接迎春,又请了薛姨妈宝琴,叫带了香菱来,又请李
婶娘,不多半日,李纹李绮都来了。宝钗本不知道,听见老太太的丫头来请,说:
“薛姨太太来了,请二奶奶过去呢。”宝钗心里喜欢,便是随身衣服过去,要见他
母亲。只见他妹子宝琴并香菱都在这里,又见李婶娘等人也都来了,心想:“那些
人必是知道我们家的事情完了,所以来问候的。”便去问了李婶娘好,见了贾母,
然后与他母亲说了几句话,和李家姐妹们问好。

湘云在旁说道:“太太们请都坐下,让我们姐妹们给姐姐拜寿。”宝钗听了,倒
呆了一呆,回来一想,“可不是明日是我的生日吗?”便说:“姐妹们过来瞧老太太
是该的,若说为我的生日,是断断不敢的。”正推让着,宝玉也来请薛姨妈李婶娘
的安。听见宝钗自己推让,他心里本早打算过宝钗生日,因家中闹得七颠八倒,也
不敢在贾母处提起。今儿湘云等众人要拜寿,便喜欢道:“明日才是生日,我正要
告诉老太太来。”湘云笑道:“扯臊,老太太还等你告诉?你打量这些人为什么来?是
老太太请的。”宝钗听了,心下未信,只听贾母合他母亲道:“可怜宝丫头做了一年
新媳妇,家里接二连三的有事,总没有给他做过生日。今日我给他做个生日,请姨
太太、太太们来,大家说说话儿。”薛姨妈道:“老太太这些时心里才安,他小人儿
家还没有孝敬老太太,倒要老太太操心。”湘云道:“老太太最疼的孙子是二哥哥,
难道二嫂子就不疼了么?况且宝姐姐也配老太太给他做生日。”宝钗低头不语。宝玉
心里想道:“我只说史妹妹出了阁必换了一个人了,我所以不敢亲近他,他也不来
理我;如今听他的话,竟和先前是一样的。为什么我们那个过了门,更觉的腼腆了,
话都说不出来了呢?”正想着,小丫头进来说:“二姑奶奶回来了。”随后李纨凤姐
都进来,大家厮见一番。迎春提起他父亲出门,说:“本要赶来见见,只是他拦着
不许来,说是咱们家正是晦气时候,不要沾染在身上。我扭不过,没有来,直哭了
两三天。”凤姐道:“今儿为什么肯放你回来?”迎春道:“他又说咱们家二老爷又
袭了职,还可以走走,不妨事的,所以才放我来。”说着又哭起来。贾母道:“我原
为闷的慌,今日接你们来给孙子媳妇过生日,说说笑笑,解个闷儿,你们又提起这
些烦事来,又招起我的烦恼来了。”迎春等都不敢作声了。

凤姐虽勉强说了几句有兴的话,终不似先前爽利、招人发笑。贾母心里要宝钗
喜欢,故意的怄凤姐儿说话。凤姐也知贾母之意,便竭力张罗,说道:“今儿老太
太喜欢些了。你看这些人好几时没有聚在一处,今儿齐全。”说着,回过头去,看
见婆婆、尤氏不在这里,又缩住了口。贾母为着“齐全”两字,也想邢夫人等,叫
人请去。邢夫人、尤氏、惜春等听见老太太叫,不敢不来,心内也十分不愿意,想
着家业零败,偏又高兴给宝钗做生日,到底老太太偏心,便来了也是无精打彩的。
贾母问起岫烟来,邢夫人假说病着不来。贾母会意,知薛姨妈在这里有些不便,也
不提了。

一时摆下果酒。贾母说:“也不送到外头,今日只许咱们娘儿们乐一乐。”宝玉
虽然娶过亲的人,因贾母疼爱,仍在里头打混,但不与湘云宝琴等同席,便在贾母
身旁设着一个坐儿,他替宝钗轮流敬酒。贾母道:“如今且坐下,大家喝酒。到挨
晚儿再到各处行礼去。若如今行起礼来,大家又闹规矩,把我的兴头打回去,就没
趣了。”宝钗便依言坐下。贾母又向众人道:“咱们今儿索性洒脱些,各留一两个人
伺候。我叫鸳鸯带了彩云、莺儿、袭人、平儿等在后间去也喝一钟酒。”鸳鸯等说:
“我们还没有给二奶奶磕头,怎么就好喝酒去呢?”贾母道:“我说了,你们只管
去。用的着你们再来。”鸳鸯等去了。这里贾母才让薛姨妈等喝酒。见他们都不是
往常的样子,贾母着急道:“你们到底是怎么着?大家高兴些才好。”湘云道:“我们
又吃又喝,还要怎么着呢?”凤姐道:“他们小的时候都高兴,如今碍着脸不敢混
说,所以老太太瞧着冷净了。”宝玉轻轻的告诉贾母道:“话是没有什么说的,再说
就说到不好的上头去了。不如老太太出个主意,叫他们行个令儿罢。”贾母侧着耳
朵听了,笑道:“若是行令,又得叫鸳鸯去。”

宝玉听了,不待再说,就出席到后间去找鸳鸯,说:“老太太要行令,叫姐姐
去呢。”鸳鸯道:“小爷,让我们舒舒服服的喝一钟罢。何苦来,又来搅什么?”宝
玉道:“当真老太太说的,叫你去呢。与我什么相干?”鸳鸯没法,说道:“你们只
管喝,我去了就来。”便到贾母那边。老太太道:“你来了么?这里要行令呢。”鸳鸯
道:“听见宝二爷说老太太叫我,才来的。不知老太太要行什么令儿?”贾母道:“那
文的怪闷的慌,武的又不好,你倒是想个新鲜玩意儿才好。”鸳鸯想了想道:“如今
姨太太有了年纪,不肯费心,倒不如拿出令盆骰子来,大家掷个曲牌名儿赌输赢酒
罢。”贾母道:“这也使得。”便命人取骰盆放在案上。鸳鸯说:“如今用四个骰子掷
去,掷不出名儿来的罚一杯;掷出名儿来,每人喝酒的杯数儿,掷出来再定。”众
人听了道:“这是容易的,我们都随着。”鸳鸯便打点儿。众人叫鸳鸯喝了一杯,就
在他身上数起,恰是薛姨妈先掷。薛姨妈便掷了一下,却是四个么。鸳鸯道:“这
是有名的,叫做‘商山四皓’。有年纪的喝一杯。”于是贾母、李婶娘、邢、王两夫
人都该喝。贾母举酒要喝,鸳鸯道:“这是姨太太掷的,还该姨太太说个曲牌名儿,
下家接一句‘千家诗’。说不出来的罚一杯。”薛姨妈道:“你又来算计我了,我那
里说的上来?”贾母道:“不说到底寂寞,还是说一句的好。下家儿就是我了,若
说不出来,我陪姨太太喝一钟就是了。”薛姨妈便道:“我说个‘临老入花丛’。”贾
母点点头儿道:“将谓偷闲学少年。”

说完,骰盆过到李纹,便掷了两个四,两个二。鸳鸯说:“也有名儿了,这叫
‘刘阮入天台’。”李纹便接着说了个“二士入桃源”。下手儿便是李纨,说道:“寻
得桃花好避秦。”大家又喝了一口。

骰盆又过到贾母跟前,便掷了两个二,两个三。贾母道:“这要喝酒了。”鸳鸯
道:“有名儿的,这是‘江燕引雏’。众人都该喝一杯。”凤姐道:“雏是雏,倒飞了
好些了。”众人瞅了他一眼,凤姐便不言语。贾母道:“我说什么呢?‘公领孙’罢。”
下手是李绮,便说道:“闲看儿童捉柳花。”众人都说好。

宝玉巴不得要说,只是令盆轮不到,正想着,恰好到了跟前,便掷了一个二,
两个三,一个么,便说道:“这是什么?”鸳鸯笑道:“这是个‘臭’!先喝一钟再
掷罢。”宝玉只得喝了又掷。这一掷掷了两个三,两个四。鸳鸯道:“有了,这叫做
‘张敞画眉’。”宝玉知是打趣他。宝钗的脸也飞红了。凤姐不大懂得,还说:“二
兄弟快说了,再找下家儿是谁。”宝玉难说,自认:“罚了罢。我也没下家儿。”

过了令盆,轮到李纨,便掷了一下。鸳鸯道:“大奶奶掷的是‘十二金钗’。”
宝玉听了,赶到李纨身旁看时,只见红绿对开,便说:“这一个好看的很。”忽然想
起“十二钗”的梦来,便呆呆的退到自己座上,心里想:“这‘十二钗’说是金陵
的,怎么我家这些人,如今七大八小的就剩了这几个?”复又看看湘云宝钗,虽说
都在,只是不见了黛玉。一时按捺不住,眼泪便要下来,恐人看见,便说身上燥的
很,脱脱衣裳去,挂了筹出席去了。史湘云看见宝玉这般光景,打量宝玉掷不出好
的来,被别人掷了去,心里不喜欢才去的;又嫌那个令儿没趣,便有些烦。只见李
纨道:“我不说了。席间的人也不齐,不如罚我一杯。”

贾母道:“这个令儿也不热闹,不如蠲了罢。让鸳鸯掷一下,看掷出个什么来。”
小丫头便把令盆放在鸳鸯跟前。鸳鸯依命,便掷了两个二,一个五,那一个骰子在
盆里只管转。鸳鸯叫道:“不要五!”那骰子单单转出一个五来。鸳鸯道:“了不得!
我输了。”贾母道:“这是不算什么的吗?”鸳鸯道:“名儿倒有,只是我说不上曲
牌名来。”贾母道:“你说名儿,我给你诌。”鸳鸯道:“这是‘浪扫浮萍’。”贾母道:
“这也不难,我替你说个‘秋鱼入菱窠’。”鸳鸯下手的就是湘云,便道:“白萍吟
尽楚江秋。”众人都道:“这句很确。”

贾母道:“这令完了,咱们喝两杯,吃饭罢。”回头一看,见宝玉还没进来,便
问道:“宝玉那里去了,还不来?”鸳鸯道:“换衣裳去了。”贾母道:“谁跟了去
的?”那莺儿便上来回道:“我看见二爷出去,我叫袭人姐姐跟了去了。”贾母王夫
人才放心。等了一回,王夫人叫人去找。小丫头到了新房子里,只见五儿在那里插
蜡。小丫头便问:“宝二爷那里去了?”五儿道:“在老太太那边喝酒呢。”小丫头
道:“我打老太太那里来,太太叫我来找,岂有在那里倒叫我来找的呢。”五儿道:
“这就不知道了,你到别处找去罢。”小丫头没法,只得回来,遇见秋纹,问道:“你
见二爷那里去了?”秋纹道:“我也找他,太太们等他吃饭。这会子那里去了呢?你
快去回老太太去,不必说不在家,只说喝了酒不大受用,不吃饭了,略躺一躺再来,
请老太太、太太们吃饭罢。”小丫头依言回去,告诉珍珠,珍珠回了贾母。贾母道:
“他本来吃不多,不吃也罢了,叫他歇歇罢。告诉他今儿不必过来,有他媳妇在这
里就是了。”珍珠便向小丫头道:“你听见了?”小丫头答应着,不便说明,只得在
别处转了一转,说“告诉了”。众人也不理会,吃毕饭,大家散坐闲话,不提。

且说宝玉一时伤心,走出来,正无主意。只见袭人赶来,问是怎么了。宝玉道:
“不怎么,只是心里怪烦的。要不趁他们喝酒,咱们两个到珍大奶奶那里逛逛去。”
袭人道:“珍大奶奶在这里,去找谁?”宝玉道:“不找谁,瞧瞧他,既在这里,住
的房屋怎么样。”袭人只得跟着,一面走,一面说。走到尤氏那边,又一个小门儿
半开半掩,宝玉也不进去。只见看园门的两个婆子坐在门槛上说话儿。宝玉问道:
“这小门儿开着么?”婆子道:“天天不开。今儿有人出来说,今日预备老太太要
用园里的果子,才开着门等着呢。”宝玉便慢慢的走到那边,果见腰门半开。宝玉
才要进去,袭人忙拉住道:“不用去。园里不干净,常没有人去,别再撞见什么。”
宝玉仗着酒气,说道:“我不怕那些。”袭人苦苦的拉住,不容他去。婆子们上来说
道:“如今这园子安静的了。自从那日道士拿了妖去,我们摘花儿,打果子,一个
人常走的。二爷要去,咱们都跟着,有这些人怕什么。”宝玉喜欢。袭人也不便相
强,只得跟着。

宝玉进得园来,只见满目凄凉。那些花木枯萎,更有几处亭馆,彩色久经剥落。
远远望见一丛翠竹,倒还茂盛。宝玉一想,说:“我自病时出园,住在后边,一连
几个月不准我到这里,瞬息荒凉。你看独有那几竿翠竹菁葱,这不是潇湘馆么?”
袭人道:“你几个月没来,连方向儿都忘了。咱们只管说话儿,不觉将怡红院走过
了。”回头用手指着道:“这才是潇湘馆呢。”宝玉顺着袭人的手一瞧,道:“可不是
过了吗?咱们回去瞧瞧。”袭人道:“天晚了,老太太必是等着吃饭,该回去了。”宝
玉不言,找着旧路,竟往前走。你道宝玉虽离了大观园将及一载,岂遂忘了路径?
只因袭人怕他见了潇湘馆,想起黛玉,又要伤心,所以要用言混过。后来见宝玉只
望里走,只怕他招了邪气,所以哄着他,只说已经走过了。那里知道宝玉的心全在
潇湘馆上。此时宝玉往前急走,袭人只得赶上。见他站着,似有所见,如有所闻,
便道:“你听什么?”宝玉道:“潇湘馆倒有人住么?”袭人道:“大约没有人罢。”
宝玉道:“我明明听见有人在内啼哭,怎么没有人?”袭人道:“是你疑心。素常你
到这里,常听见林姑娘伤心,所以如今还是那样。”宝玉不信,还要听去。婆子们
赶上说道:“二爷快回去罢,天已晚了。别处我们还敢走走;这里的路儿隐僻,又
听见人说,这里打林姑娘死后,常听见有哭声,所以人都不敢走的。”宝玉袭人听
说,都吃了一惊。宝玉道:“可不是?”说着,便滴下泪来,说:“林妹妹,林妹妹!
好好儿的,是我害了你了!你别怨我,只是父母作主,并不是我负心!”愈说愈痛,
便大哭起来。袭人正在没法,只见秋纹带着些人赶来,对袭人道:“你好大胆子!怎
么和二爷到这里来?老太太、太太急的打发人各处都找到了。刚才腰门上有人说是
你和二爷到这里来了,唬的老太太、太太们了不得,骂着我叫我带人赶来。还不快
回去呢。”宝玉犹自痛哭,袭人也不顾他哭,两个人拉着就走,一面替他拭眼泪,
告诉他老太太着急。宝玉没法,只得回来。

袭人知老太太不放心,将宝玉仍送到贾母那边,众人都等着未散。贾母便说:
“袭人!我素常因你明白,才把宝玉交给你,怎么今儿带他园里去?他的病才好,倘
或撞着什么,又闹起来,那可怎么好?”袭人也不敢分辨,只得低头不语。宝钗看
宝玉颜色不好,心里着实的吃惊。倒还是宝玉恐袭人受委屈,说道:“青天白日怕
什么?我因为好些时没到园里逛逛,今儿趁着酒兴走走,那里就撞着什么了呢?”
凤姐在园里吃过大亏的,听到那里,寒毛直竖,说:“宝兄弟胆子忒大了。”湘云道:
“不是胆大,倒是心实。不知是会芙蓉神去了,还是寻什么仙去了。”宝玉听着,
也不答言。独有王夫人急的一言不发。贾母问道:“你到园里没有唬着呀?不用说了。
以后要逛,到底多带几个人才好。不是你闹的,大家早散了。去罢,好好的睡一夜,
明儿一早过来,我要找补,叫你们再乐一天呢。别为他又闹出什么原故来。”众人
听说遂辞了贾母出来。薛姨妈便到王夫人那里住下,史湘云仍在贾母房中,迎春便
往惜春那里去了。馀者各自回去不提。

独有宝玉回到房中,嗳声叹气。宝钗明知其故,也不理他。只是怕他忧闷勾出
旧病来,便进里间,叫袭人来,细问他宝玉到园怎么样的光景。

未知袭人怎生回说,下回分解。