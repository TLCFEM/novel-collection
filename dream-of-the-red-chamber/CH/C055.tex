\chapter{辱亲女愚妾争闲气~欺幼主刁奴蓄险心}

且说荣府中刚将年事忙过,凤姐儿因年内年外操劳太过,一时不及检点,便小
月了,不能理事,天天两三个大夫用药。凤姐儿自恃强壮,虽不出门,然筹画计算,
想起什么事来,就叫平儿去回王夫人,任人谏劝,他只不听。王夫人便觉失了膀臂,
一人能有多少精神?凡有了大事就自己主张,将家中琐碎之事,一应都暂令李纨协
理。李纨本是个尚德不尚才的,未免逞纵了下人,王夫人便命探春合同李纨裁处,
只说过了一月,凤姐将养好了,仍交给他。谁知凤姐赋气血不足,兼年幼不知保
养,平生争强斗智,心力更亏,故虽系小月,竟着实亏虚下来。一月之后,又添了
下红之症。他虽不肯说出来,众人看他面目黄瘦,便知失于调养。王夫人只令他好
生服药调养,不令他操心。他自己也怕成了大症,遗笑于人,便想偷空调养,恨不
得一时复旧如常。谁知服药调养,直到三月间,才渐渐的起复过来,下红也渐渐止
了。此是后话。

如今且说目今王夫人见他如此,探春和李纨暂难谢事,园中人多,又恐失于照
管,特请了宝钗来,托他各处小心。因嘱咐他:“老婆子们不中用,得空儿吃酒斗
牌,白日里睡觉,夜里斗牌,我都知道的。凤丫头在外头,他们还有个怕惧,如今
他们又该取便了。好孩子,你还是个妥当人,你兄弟妹妹们又小,我又没工夫,你
替我辛苦两天照应照应。凡有想不到的事你来告诉我,别等老太太问出来我没话
回。那些人不好你只管说,他们不听你来回我。别弄出大事来才好。”宝钗听说,
只得答应了。

时届季春,黛玉又犯了咳嗽;湘云又因时气所感,也病卧在蘅芜院,一天医药
不断。探春和李纨相住间壁,二人近日同事,不比往年,往来回话人等亦甚不便,
故二人议定,每日早晨,皆到园门口南边的三间小花厅上去会齐办事,吃过早饭,
于午错方回。这三间厅原系预备省亲之时众执事太监起坐之处,故省亲以后也用不
着了,每日只有婆子们上夜。如今天已和暖,不用十分修理,只不过略略的陈设些,
便可他二人起坐。这厅上也有一处匾,题着“辅仁谕德”四字,家下俗语皆只叫“议
事厅儿”。如今他二人每日卯正至此,午正方散,凡一应执事的媳妇等来往回话的,
络绎不绝。

众人先听见李纨独办,各各心中暗喜,因为李纨素日是个厚道多恩无罚的人,
自然比凤姐儿好搪塞些;便添了一个探春,都想着不过是个未出闺阁的年轻小姐,
且素日也最平和恬淡,因此都不在意,比凤姐儿前便懈怠了许多。只三四天后,几
件事过手,渐觉探春精细处不让凤姐,只不过是言语安静、性情和顺而已。可巧连
日有王公侯伯世袭官员十几处,皆系荣宁非亲即世交之家,或有升迁,或有黜降,
或有婚丧红白等事,王夫人贺吊迎送,应酬不暇,前边更无人照管。他二人便一日
皆在厅上起坐,宝钗便一日在上房监察,至王夫人回方散。每于夜间针线暇时,临
寝之先,坐了轿,带领园中上夜人等,各处巡察一次。他三人如此一理,便觉比凤
姐儿当权时倒更谨慎了些。因而里外下人都暗中抱怨,说:“刚刚的倒了一个‘巡
海夜叉’,又添了三个‘镇山太岁’,越发连夜里偷着吃酒玩的工夫都没了!”

这日王夫人正是往锦乡侯府去赴席,李纨与探春早已梳洗,伺候出门去后,回
至厅上坐了。刚吃茶时,只见吴新登的媳妇进来回说:“赵姨娘的兄弟赵国基昨儿
出了事,已回过老太太、太太,说知道了,叫回姑娘来。”说毕,便垂手旁侍,再
不言语。彼时来回话者不少,都打听他二人办事如何。若办得妥当,大家则安个畏
惧之心,若少有嫌隙不当之处,不但不畏服,一出二门,还说出许多笑话来取笑。
吴新登的媳妇心中已有主意,若是凤姐前,他便早已献勤,说出许多主意、又查出
许多旧例来,任凤姐拣择施行;如今他藐视李纨老实,探春是年轻的姑娘,所以只
说出这一句话来,试他二人有何主见。探春便问李纨,李纨想了一想,便道:“前
日袭人的妈死了,听见说赏银四十两,这也赏他四十两罢了。”吴新登的媳妇听了,
忙答应了个“是”,接了对牌就走。探春道:“你且回来。”吴新登家的只得回来。
探春道:“你且别支银子。我且问你:那几年老太太屋里的几位老姨奶奶,也有家
里的,也有外头的,有两个分别。家里的若死了人是赏多少?外头的死了人是赏多
少?你且说两个我们听听。”一问,吴新登家的便都忘了,忙陪笑回说道:“这也
不是什么大事,赏多赏少,谁还敢争不成?”探春笑道:“这话胡闹。依我说,赏
一百倒好!若不按理,别说你们笑话,明儿也难见你二奶奶。”吴新登家的笑道:
“既这么说,我查旧账去;此时却不记得。”探春笑道:“你办事办老了的,还不
记得,倒来难我们!你素日回你二奶奶,也现查去?若有这道理,凤姐姐还不算利害,
也就算是宽厚了。还不快找了来我瞧!再迟一日,不说你们粗心,倒像我们没主意
了。”吴新登家的满面通红,忙转身出来。众媳妇们都伸舌头。

这里又回别的事;一时吴家的取了旧账来。探春看时,两个家里的赏过皆二十
四两,两个外头的皆赏过四十两。外还有两个外头的,一个赏过一百两,一个赏过
六十两。这两笔底下皆有原故:一个是隔省迁父母之柩,外赏六十两;一个是现买
葬地,外赏二十两。探春便递给李纨看了,探春便说:“给他二十两银子,把这账
留下我们细看。”吴新登家的去了。

忽见赵姨娘进来,李纨探春忙让坐。赵姨娘开口便说道:“这屋里的人,都踹
下我的头去还罢了,姑娘你也想一想,该替我出气才是!”一面说,一面便眼泪鼻
涕哭起来。探春忙道:“姨娘这话说谁?我竟不懂。谁踹姨娘的头?说出来,我替姨
娘出气。”赵姨娘道:“姑娘现踹我,我告诉谁去?”探春听说,忙站起来说道:
“我并不敢。”李纨也忙站起来劝。赵姨娘道:“你们请坐下,听我说。我这屋里
熬油似的熬了这么大年纪,又有你兄弟,这会子连袭人都不如了,我还有什么脸?
连你也没脸面,别说是我呀。”探春笑道:“原来为这个,我说我并不敢犯法违礼。”
一面便坐了,拿账翻给赵姨娘瞧,又念给他听,又说道:“这是祖宗手里旧规矩,
人人都依着,偏我改了不成?这也不但袭人,将来环儿收了外头的,自然也是和袭
人一样。这原不是什么争大争小的事,讲不到有脸没脸的话上。他是太太的奴才,
我是按着旧规矩办。说办的好,领祖宗的恩典、太太的恩典;若说办的不公,那是
他糊涂不知福,也只好凭他抱怨去。太太连房子赏了人,我有什么有脸的地方儿?
一文不赏,我也没什么没脸的。依我说,太太不在家,姨娘安静些,养神罢,何苦
只要操心?太太满心疼我,因姨娘每每生事,几次寒心。我但凡是个男人,可以出
得去,我早走了,立出一番事业来,那时自有一番道理,偏我是女孩儿家,一句多
话也没我乱说的。太太满心里都知道,如今因看重我,才叫我管家务。还没有做一
件好事,姨娘倒先来作践我。倘或太太知道了,怕我为难,不叫我管,那才正经没
脸呢!连姨娘真也没脸了!”一面说,一面抽抽搭搭的哭起来。

赵姨娘没话答对,便说道:“太太疼你,你该越发拉扯拉扯我们。你只顾讨太
太的疼,就把我们忘了!”探春道:“我怎么忘了?叫我怎么拉扯?这也问他们各人。
那一个主子不疼出力得用的人?那一个好人用人拉扯呢?”李纨在旁只管劝说:“姨
娘别生气,也怨不得姑娘。他满心里要拉扯,口里怎么说的出来?”探春忙道:“这
大嫂子也糊涂了!我拉扯谁?谁家姑娘们拉扯奴才了?他们的好歹,你们该知道,与
我什么相干?”赵姨娘气的问道:“谁叫你拉扯别人去了?你不当家,我也不来问
你。你如今现在说一是一,说二是二!如今你舅舅死了,你多给了二三十两银子,
难道太太就不依你?分明太太是好太太,都是你们尖酸克薄!可惜太太有恩无处使!
姑娘放心:这也使不着你的银子,明日等出了阁,我还想你额外照看赵家呢!如今
没有长翎毛儿就忘了根本,只‘拣高枝儿飞’去了。”探春没听完,气的脸白气噎,
越发呜呜咽咽的哭起来。因问道:“谁是我舅舅?我舅舅早升了九省的检点了!那里
又跑出一个舅舅来?我倒素昔按礼尊敬,怎么敬出这些亲戚来了!既这么说,每日环
儿出去,为什么赵国基又站起来?又跟他上学?为什么不拿出舅舅的款来?何苦来!谁
不知道我是姨娘养的,必要过两三个月寻出由头来,彻底来翻腾一阵,怕人不知道,
故意表白表白!也不知道是谁给谁没脸!幸亏我还明白,但凡糊涂不知礼的,早急
了!”

李纨急得只管劝,赵姨娘只管还唠叨。忽听有人说:“二奶奶打发平姑娘说话
来了。”赵姨娘听说,方把嘴止住。只见平儿走来,赵姨娘忙陪笑让坐,又忙问:
“你奶奶好些?我正要瞧去,就只没得空儿。”李纨见平儿进来,因问他:“来作
什么?”平儿笑道:“奶奶说:赵姨奶奶的兄弟没了,恐怕奶奶和姑娘不知有旧例。
若照常例,只得二十两;如今请姑娘裁度着,再添些也使得。”探春早已拭去泪痕,
忙说道:“又好好的添什么?谁又是二十四个月养的?不然,也是出兵放马、背着主
子逃出命来过的人不成?你主子真个倒巧,叫我开了例,他做好人,拿着太太不心
疼的钱,乐得做人情!你告诉他:我不敢添减混出主意。他添他施恩,等他好了出
来,爱怎么添怎么添!”平儿一来时已明白了对半,今听这话越发会意。见探春有
怒色,便不敢以往日喜乐之时相待,只一边垂手默侍。

时值宝钗也从上房中来,探春等忙起身让坐。未及开言,又有一个媳妇进来回
事,因探春才哭了,便有三四个小丫鬟捧了脸盆、巾帕、靶镜等物来。此时探春因
盘膝坐在矮板榻上,那捧盆丫鬟走至跟前,便双膝跪下,高捧脸盆,那两个丫鬟也
都在旁屈膝捧着巾帕并靶镜脂粉之饰。平儿见侍书不在这里,便忙上来与探春挽袖
卸镯,又接过一条大手巾来,将探春面前衣襟掩了,探春方伸手向脸盆中盥沐。媳
妇便回道:“奶奶,姑娘:家学里支环爷和兰哥儿一年的公费。”平儿先道:“你
忙什么?你睁着眼看见姑娘洗脸,你不出去伺候着,倒先说话来!二奶奶跟前,你也
这样没眼色来着?姑娘虽恩宽,我去回了二奶奶,只说你们眼里都没姑娘,你们都
吃了亏可别怨我。”唬得那个媳妇忙陪笑说:“我粗心了!”一面说,一面忙退出
去。

探春一面匀脸,一面向平儿冷笑道:“你迟了一步,没见还有可笑的。连吴姐
姐这么个办老了事的,也不查清楚了就来混我们。幸亏我们问他,他竟有脸说‘忘
了’,我说他回二奶奶事也忘了再找去?我料着你主子未必有耐性儿等他去找!”
平儿笑道:“他有这么一次,包管腿上的筋早折了两根。姑娘别信他们。那是他们
瞅着大奶奶是个菩萨,姑娘又是腼腆小姐,固然是托懒来混。”说着,又向门外说
道:“你们只管撒野,等奶奶大安了,咱们再说。”门外的众媳妇都笑道:“姑娘,
你是个最明白的人,俗语说:‘一人作罪一人当。’我们并不敢欺蔽主子。如今主
子是娇客,若认真惹恼了,死无葬身之地!”平儿冷笑道:“你们明白就好了。”
又陪笑向探春道:“姑娘知道,二奶奶本来事多,那里照看得这些?保不住不忽略。
俗语说:‘旁观者清。’这几年姑娘冷眼看着,或有该添该减的去处,二奶奶没行
到,姑娘竟一添减:头一件,与太太有益;第二件,也不枉姑娘待我们奶奶的情义
了。”话未说完,宝钗李纨皆笑道:“好丫头,真怨不得凤丫头偏疼他!本来无可
添减之事,如今听你一说,倒要找出两件来斟酌斟酌,不辜负你这话。”

探春笑道:“我一肚子气,正要拿他奶奶出气去,偏他碰了来,说了这些话,
叫我也没了主意了。”一面说,一面叫进方才那媳妇来问:“环爷和兰哥家学里这
一年的银子,是做那一项用的?”那媳妇便回说:“一年学里吃点心或者买纸笔,
每位有八两银子的使用。”探春道:“凡爷们的使用,都是各屋里月钱之内:环哥
的是姨娘领二两;宝玉的,老太太屋里袭人领二两;兰哥儿是大奶奶屋里领:怎么
学里每人多这八两?原来上学去的是为这八两银子!从今日起,把这一项蠲了。平儿
回去,告诉你奶奶,说我的话,把这一条务必免了。”平儿笑道:“早就该免。旧
年奶奶原说要免来着,因年下忙,就忘了。”那媳妇只得答应着去了。

就有大观园中媳妇捧了饭盒子来,侍书素云早已抬过一张小饭桌来,平儿也忙
着上菜。探春笑道:“你说完了话,干你的去罢,在这里又忙什么?”平儿笑道:
“我原没事,二奶奶打发了我来,一则说话,二则怕这里的人不方便,叫我帮着妹
妹们伏侍奶奶姑娘来了。”探春因问:“宝姑娘的怎么不端来一处吃?”丫鬟们听
说,忙出至檐外,命媳妇们去说:“宝姑娘如今在厅上一处吃,叫他们把饭送了这
里来。”探春听说,便高声说道:“你别混支使人!那都是办大事的管家娘子们,
你们支使他要饭要茶的?连个高低都不知道!平儿这里站着,叫他叫去。”平儿忙答
应了一声出来,那些媳妇们都悄悄的拉住笑道:“那里用姑娘去叫?我们已有人叫
去了。”一面说,一面用绢子掸台阶的土,说:“姑娘站了半天,乏了,这太阳地
里歇歇儿罢。”平儿便坐下。又有茶房里的两个婆子拿了个坐褥铺下,说:“石头
冷,这是极干净的,姑娘将就坐一坐儿罢。”平儿点头笑道:“多谢。”一个又捧
了一碗精致新茶出来,也悄悄笑说:“这不是我们常用的茶,原是伺候姑娘们的,
姑娘且润一润罢。”平儿遂欠身接了,因指众媳妇悄悄说道:“你们太闹的不像了。
他是个姑娘家,不肯发威动怒,这是他尊重,你们就藐视欺负他。果然招他动了大
气,不过说他一个粗糙就完了,你们就现吃不了的亏!他撒个娇儿,太太也得让他
一二分,二奶奶也不敢怎么。你们就这么大胆子小看他,可是鸡蛋往石头上碰。”
众人都忙道:“我们何尝敢大胆了?都是赵姨娘闹的。”平儿也悄悄的道:“罢了!
好奶奶们,‘墙倒众人推’。那赵姨娘原有些颠倒,着三不着两,有了事就都赖他。
你们素日那眼里没人、心术利害,我这几年难道还不知道!二奶奶要是略差一点儿
的,早叫你们这些奶奶们治倒了。饶这么着,得一点空儿,还要难他一难,好几次
没落了你们的口声。众人都说他利害,你们都怕他,惟我知道他心里也就不算不怕
你们的。前儿我们还议论到这里:再不能依头顺尾,必有两场气生。那三姑娘虽是
个姑娘,你们都横看了他!二奶奶在这些大姑子小姑子里头,也就只单怕他五分儿。
你们这会子倒不把他放在眼里了。”

正说着,只见秋纹走来,众媳妇忙赶着问好,又说:“姑娘也且歇歇,里头摆
饭呢。等撤下桌子来,再回话去罢。”秋纹笑道:“我比不得你们,我那里等得?”
说着,便直要上厅去。平儿忙叫:“快回来!”秋纹回头,见了平儿,笑道:“你
又在这里充什么‘外围子的防护’?”一面回身便坐在平儿褥上。平儿悄问:“回
什么?”秋纹道:“问一问宝玉的月钱、我们的月钱,多早晚才领?”平儿道:“这
什么大事!你快回去告诉袭人,说我的话:凭有什么事,今日都别回。若回一件管
驳一件,回一百件管驳一百件。”秋纹听了,忙问:“这是为什么?”平儿与众媳
妇等都忙告诉他原故,又说:“正要找几处利害事与有体面的人来开例,作法子镇
压,与众人作榜样呢。何苦你们先来碰在这钉子上?你这一去说了,他们若拿你们
也作一二件榜样,又碍着老太太、太太;若不拿着你们做一二件,人家又说:‘偏
一个向一个,仗着老太太、太太威势的就怕,不敢惹,只拿着软的做鼻子头。’你
听听罢,二奶奶的事他还要驳两件,才压得众人口声呢。”秋纹听了,伸了伸舌头
笑道:“幸而平姐姐在这里,没得臊一鼻子灰,趁早知会他们去。”说着便起身走
了。

接着宝钗的饭至,平儿忙进来伏侍。那时赵姨娘已去,三人在板床上吃饭,宝
钗面南,探春面西,李纨面东。众媳妇皆在廊下静候,里头只有他们紧跟常侍的丫
鬟伺候,别人一概不敢擅入。这些媳妇们都悄悄的议论说:“大家省事罢,别安着
没良心的主意。连吴大娘才都讨了没意思,咱们又是什么有脸的?”都一边悄议,
等饭完回事。此时里面惟闻微嗽之声,不闻碗箸之响。一时,只见一个丫头将帘栊
高揭,又有两个将桌抬出。茶房内有三个丫鬟,捧着三个沐盆儿,见饭桌已出,三
人便进去了。一回又捧出沐盆并漱盂来,方有侍书、素云、莺儿三个人,每人用茶
盘捧了三盖碗茶进去。一时等他三人出来,侍书命小丫头子:“好生伺候着,我们
吃饭来换你们,可又别偷坐着去。”众媳妇们方慢慢的安分回事,不敢如先前轻慢
疏忽了。

探春气方渐平,因向平儿道:“我有一件大事,早要和你奶奶商议,如今可巧
想起来。你吃了饭快来。宝姑娘也在这里,咱们四个人商议了,再细细的问你奶奶
可行可止。”平儿答应回去。凤姐因问:“为何去这半日?”平儿便笑着将方才的
原故细细说与他听了。凤姐儿笑道:“好,好,好!好个三姑娘,我说不错,只可
惜他命薄,没托生在太太肚里。”平儿笑道:“奶奶也说糊涂话了。他就不是太太
养的,难道谁敢小看他,不和别的一样看待么?”凤姐叹道:“你那里知道?虽然
正出庶出是一样,但只女孩儿却比不得儿子。将来作亲时,如今有一种轻狂人,先
要打听姑娘是正出是庶出,多有为庶出不要的。殊不知庶出只要人好,比正出的强
百倍呢。将来不知那个没造化的,为挑正庶误了事呢,也不知那个有造化的,不挑
正庶的得了去。”说着,又向平儿笑道:“你知道我这几年生了多少省俭的法子,
一家子大约也没个背地里不恨我的。我如今也是骑上老虎了,虽然看破些,无奈一
时也难宽放。二则家里出去的多,进来的少,凡有大小事儿,仍是照着老祖宗手里
的规矩。却一年进的产业又不及先时多,省俭了外人又笑话,老太太、太太也受委
屈,家下也抱怨克薄。若不趁早儿料理省俭之计,再几年就都赔尽了。”平儿道:
“可不是这话!将来还有三四位姑娘,还有两三个小爷们,一位老太太,这几件大
事未完呢。”凤姐儿笑道:“我也虑到这里,倒也够了,宝玉和林妹妹,他两个一
娶一嫁,可以使不着官中钱,老太太自有体己拿出来。二姑娘是大老爷那边的,也
不算。剩了三四个,满破着每人花上七八千银子。环哥娶亲有限,花上三千银子,
若不够,那里省一抿子也就够了。老太太的事出来,一应都是全了的,不过零星杂
项使费些,满破三五千两。如今再俭省些,陆续就够了。只怕如今平空再生出一两
件事来,可就了不得了。咱们且别虑后事,你且吃了饭,快听他们商议什么。这正
碰了我的机会,我正愁没个膀臂。虽有个宝玉,他又不是这里头的货,纵收伏了他
也不中用。大奶奶是个佛爷,也不中用。二姑娘更不中用,亦且不是这屋里的人。
四姑娘小呢。兰小子和环儿更是个燎毛的小冻猫子,只等有热灶火炕让他钻去罢,
——真真一个娘肚子里跑出这样天悬地隔的两个人来,我想到那里就不服!再者林
丫头和宝姑娘他两个人倒好,偏又都是亲戚,又不好管咱们家务事。况且一个是美
人灯儿,风吹吹就坏了;一个是拿定了主意,不干己事不张口,一问摇头三不知,
也难十分去问他。倒只剩了三姑娘一个,心里嘴里都也来得,又是咱家的正人,太
太又疼他,虽然脸上淡淡的,皆因是赵姨娘那老东西闹的,心里却是和宝玉一样呢。
比不得环儿,实在令人难疼,要依我的性子,早撵出去了!如今他既有这主意,正
该和他协同,大家做个膀臂,我也不孤不独了。按正礼天理良心上论,咱们有他这
一个人帮着,咱们也省些心,与太太的事也有益。若按私心藏奸上论,我也太行毒
了,也该抽回退步,回头看看;再要穷追苦克,人恨极了,他们笑里藏刀,咱们两
个才四个眼睛两个心,一时不防,倒弄坏了。趁着紧溜之中,他出头一料理,众人
就把往日咱们的恨暂可解了。还有一件,我虽知你极明白,恐怕你心里挽不过来,
如今嘱咐你:他虽是姑娘家,心里却事事明白,不过是言语谨慎。他又比我知书识
字,更利害一层了。如今俗语说:‘擒贼必先擒王。’他如今要作法开端,一定是
先拿我开端,倘或他要驳我的事,你可别分辩,你只越恭敬越说驳的是才好。千万
别想着怕我没脸,和他一强,就不好了。”

平儿不等说完,便笑道:“你太把人看糊涂了!我才已经行在先了,这会子才
嘱咐我。”凤姐儿笑道:“我是恐怕你心里眼里只有了我、一概没有他人之故,不
得不嘱咐,既已行在先,更比我明白了。这不是你又急了,满嘴里‘你’呀‘我’
的起来了!”平儿道:“偏说‘你’!你不依,这不是嘴巴子?再打一顿。难道这脸
上还没尝过的不成?”凤姐儿笑道:“你这小蹄子儿,要掂多少过儿才罢?你看我
病的这个样儿,还来怄我呢。过来坐下,横竖没人来,咱们一处吃饭是正经。”说
着,丰儿等三四个小丫头子进来,放小炕桌。凤姐只吃燕窝粥,两碟子精致小菜,
每日分例菜已暂减去。丰儿便将平儿的四样分例菜端至桌上,与平儿盛了饭来。平
儿屈一膝于炕沿之上,半身犹立于炕下,陪着凤姐儿吃了饭,伏侍漱口毕,吩咐了
丰儿些话,方往探春处来。只见院中寂静,人已散出。

要知后事何如,且听下回分解。