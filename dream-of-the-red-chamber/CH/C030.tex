\chapter{宝钗借扇机带双敲~椿龄画蔷痴及局外}

话说林黛玉自与宝玉口角后也觉后悔,但又无去就他之理,因此日夜闷闷如有
所失。紫鹃也看出八九,便劝道:“论前儿的事,竟是姑娘太浮躁了些。别人不知
宝玉的脾气,难道咱们也不知道?为那玉也不是闹了一遭两遭了。”黛玉啐道:“呸!
你倒来替人派我的不是。我怎么浮躁了?”紫鹃笑道:“好好儿的,为什么铰了那
穗子?不是宝玉只有三分不是,姑娘倒有七分不是?我看他素日在姑娘身上就好,皆
因姑娘小性儿,常要歪派他,才这么样。”黛玉欲答话,只听院外叫门。紫鹃听了
听,笑道:“这是宝玉的声音,想必是来赔不是来了。”黛玉听了,说:“不许开
门!”紫鹃道:“姑娘又不是了,这么热天,毒日头地下,晒坏了他,如何使得呢。”
口里说着,便出去开门,果然是宝玉。一面让他进来,一面笑着说道:“我只当宝
二爷再不上我们的门了,谁知道这会子又来了。”宝玉笑道:“你们把极小的事倒
说大了,好好的为什么不来?我就死了,魂也要一日来一百遭。妹妹可大好了?”
紫鹃道:“身上病好了,只是心里气还不大好。”宝玉笑道:“我知道了,有什么
气呢。”一面说着,一面进来。只见黛玉又在床上哭。

那黛玉本不曾哭,听见宝玉来,由不得伤心,止不住滚下泪来。宝玉笑着走近
床来道:“妹妹身上可大好了?”黛玉只顾拭泪,并不答应。宝玉因便挨在床沿上
坐了,一面笑道:“我知道你不恼我,但只是我不来,叫旁人看见,倒像是咱们又
拌了嘴的似的。要等他们来劝咱们,那时候儿岂不咱们倒觉生分了?不如这会子你
要打要骂,凭你怎么样,千万别不理我!”说着,又把“好妹妹”叫了几十声。黛
玉心里原是再不理宝玉的,这会子听见宝玉说“别叫人知道咱们拌了嘴就生分了似
的”这一句话,又可见得比别人原亲近,因又掌不住,便哭道:“你也不用来哄我!
从今以后,我也不敢亲近二爷,权当我去了。”宝玉听了笑道:“你往那里去呢?”
黛玉道:“我回家去。”宝玉笑道:“我跟了去。”黛玉道:“我死了呢?”宝玉
道:“你死了,我做和尚。”黛玉一闻此言,登时把脸放下来,问道:“想是你要
死了!胡说的是什么?你们家倒有几个亲姐姐亲妹妹呢!明儿都死了,你几个身子做
和尚去呢?等我把这个话告诉别人评评理。”宝玉自知说的造次了,后悔不来,登
时脸上红涨,低了头不敢作声。幸而屋里没人。

黛玉两眼直瞪瞪的瞅了他半天,气的“嗳”了一声,说不出话来。见宝玉别的
脸上紫涨,便咬着牙,用指头狠命的在他额上戳了一下子,“哼”了一声,说道:
“你这个——”刚说了三个字,便又叹了一口气,仍拿起绢子来擦眼泪。宝玉心里
原有无限的心事,又兼说错了话,正自后悔;又见黛玉戳他一下子,要说也说不出
来,自叹自泣:因此自己也有所感,不觉掉下泪来。要用绢子揩拭,不想又忘了带
来,便用衫袖去擦。黛玉虽然哭着,却一眼看见他穿着簇新藕合纱衫,竟去拭泪,
便一面自己拭泪,一面回身将枕上搭的一方绡帕拿起来向宝玉怀里一摔,一语不
发,仍掩面而泣。宝玉见他摔了帕子来,忙接住拭了泪,又挨近前些,伸手拉了他
一只手,笑道:“我的五脏都揉碎了,你还只是哭。走罢,我和你到老太太那里去
罢。”黛玉将手一摔道:“谁和你拉拉扯扯的!一天大似一天,还这么涎皮赖脸的,
连个理也不知道。”

一句话没说完,只听嚷道:“好了!”宝黛两个不防,都唬了一跳。回头看时,
只见凤姐儿跑进来,笑道:“老太太在那里抱怨天,抱怨地,只叫我来瞧瞧你们好
了没有,我说:‘不用瞧,过不了三天,他们自己就好了。’老太太骂我,说我懒;
我来了,果然应了我的话了。——也没见你们两个!有些什么可拌的,三日好了,
两日恼了,越大越成了孩子了。有这会子拉着手哭的,昨儿为什么又成了‘乌眼鸡’
似的呢?还不跟着我到老太太跟前,叫老人家也放点儿心呢。”说着,拉了黛玉就
走。黛玉回头叫丫头们,一个也没有。凤姐道:“又叫他们做什么,有我伏侍呢。”
一面说,一面拉着就走,宝玉在后头跟着。出了园门,到了贾母跟前,凤姐笑道:
“我说他们不用人费心,自己就会好的,老祖宗不信,一定叫我去说和。赶我到那
里说和,谁知两个人在一块儿对赔不是呢,倒像‘黄鹰抓住鹞子的脚’——两个人
都‘扣了环’了!那里还要人去说呢?”说的满屋里都笑起来。

此时宝钗正在这里,那黛玉只一言不发,挨着贾母坐下。宝玉没什么说的,便
向宝钗笑道:“大哥哥好日子,偏我又不好,没有别的礼送,连个头也不磕去。大
哥哥不知道我病,倒像我推故不去似的。倘或明儿姐姐闲了,替我分辩分辩。”宝
钗笑道:“这也多事。你就要去,也不敢惊动,何况身上不好。弟兄们常在一处,
要存这个心倒生分了。”宝玉又笑道:“姐姐知道体谅我就好了。”又道:“姐姐
怎么不听戏去?”宝钗道:“我怕热。听了两出,热的很,要走呢,客又不散;我
少不得推身上不好,就躲了。”宝玉听说,自己由不得脸上没意思,只得又搭讪笑
道:“怪不得他们拿姐姐比杨妃,原也富胎些。”宝钗听说,登时红了脸,待要发
作,又不好怎么样;回思了一回,脸上越下不来,便冷笑了两声,说道:“我倒像
杨妃,只是没个好哥哥好兄弟可以做得杨国忠的!”正说着,可巧小丫头靓儿因不
见了扇子,和宝钗笑道:“必是宝姑娘藏了我的。好姑娘,赏我罢。”宝钗指着他
厉声说道:“你要仔细!你见我和谁玩过!有和你素日嘻皮笑脸的那些姑娘们,你该
问他们去!”说的靓儿跑了。宝玉自知又把话说造次了,当着许多人,比才在黛玉
跟前更不好意思,便急回身,又向别人搭讪去了。

黛玉听见宝玉奚落宝钗,心中着实得意,才要搭言,也趁势取个笑儿,不想靓
儿因找扇子,宝钗又发了两句话,他便改口说道:“宝姐姐,你听了两出什么戏?”
宝钗因见黛玉面上有得意之态,一定是听了宝玉方才奚落之言,遂了他的心愿。忽
又见他问这话,便笑道:“我看的是李逵骂了宋江,后来又赔不是。”宝玉便笑道:
“姐姐通今博古,色色都知道,怎么连这一出戏的名儿也不知道,就说了这么一套。
这叫做《负荆请罪》。”宝钗笑道:“原来这叫‘负荆请罪’!你们通今博古,才
知道‘负荆请罪’,我不知什么叫‘负荆请罪’。”一句话未说了,宝玉黛玉二人
心里有病,听了这话,早把脸羞红了。凤姐这些上虽不通,但只看他三人的形景,
便知其意,也笑问道:“这们大热的天,谁还吃生姜呢?”众人不解,便道:“没
有吃生姜的。”凤姐故意用手摸着腮,诧异道:“既没人吃生姜,怎么这么辣辣的
呢?”宝玉黛玉二人听见这话,越发不好意思了。宝钗再欲说话,见宝玉十分羞愧,
形景改变,也就不好再说,只得一笑收住。别人总没解过他们四个人的话来,因此
付之一笑。

一时宝钗凤姐去了,黛玉向宝玉道:“你也试着比我利害的人了。谁都像我心
拙口夯的,由着人说呢!”宝玉正因宝钗多心,自己没趣儿,又见黛玉问着他,越
发没好气起来。欲待要说两句,又怕黛玉多心,说不得忍气,无精打彩,一直出来。

谁知目今盛暑之际,又当早饭已过,各处主仆人等多半都因日长神倦,宝玉背
着手,到一处,一处鸦雀无声。从贾母这里出来往西,走过了穿堂便是凤姐的院落。
到他院门前,只见院门掩着,知道凤姐素日的规矩,每到天热,午间要歇一个时辰
的,进去不便。遂进角门,来到王夫人上房里。只见几个丫头手里拿着针线,却打
盹儿。王夫人在里间凉床上睡着,金钏儿坐在傍边捶腿,也乜斜着眼乱恍。宝玉轻
轻的走到跟前,把他耳朵上的坠子一摘。金钏儿睁眼,见是宝玉,宝玉便悄悄的笑
道:“就困的这么着?”金钏抿嘴儿一笑,摆手叫他出去,仍合上眼。宝玉见了他,
就有些恋恋不舍的,悄悄的探头瞧瞧王夫人合着眼,便自己向身边荷包里带的香雪
润津丹掏了一丸出来,向金钏儿嘴里一送,金钏儿也不睁眼,只管噙了。宝玉上来,
便拉着手,悄悄的笑道:“我和太太讨了你,咱们在一处吧?”金钏儿不答。宝玉
又道:“等太太醒了,我就说。”金钏儿睁开眼,将宝玉一推,笑道:“你忙什么?
‘金簪儿掉在井里头——有你的只是有你的。’连这句俗语难道也不明白?我告诉
你个巧方儿:你往东小院儿里头拿环哥儿和彩云去。”宝玉笑道:“谁管他的事呢!
咱们只说咱们的。”

只见王夫人翻身起来,照金钏儿脸上就打了个嘴巴,指着骂道:“下作小娼妇
儿!好好儿的爷们,都叫你们教坏了!”宝玉见王夫人起来,早一溜烟跑了。这里
金钏儿半边脸火热,一声不敢言语。登时众丫头听见王夫人醒了,都忙进来。王夫
人便叫:“玉钏儿把你妈叫来!带出你姐姐去。”金钏儿听见,忙跪下哭道:“我
再不敢了!太太要打要骂,只管发落,别叫我出去,就是天恩了。我跟了太太十来
年,这会子撵出去,我还见人不见人呢!”王夫人固然是个宽仁慈厚的人,从来不
曾打过丫头们一下子,今忽见金钏儿行此无耻之事,这是平生最恨的,所以气忿不
过,打了一下子,骂了几句。虽金钏儿苦求也不肯收留,到底叫了金钏儿的母亲白
老媳妇儿领出去了。那金钏儿含羞忍辱的出去,不在话下。

且说宝玉见王夫人醒了,自己没趣,忙进大观园来。只见赤日当天,树阴匝地,
满耳蝉声,静无人语。刚到了蔷薇架,只听见有人哽噎之声。宝玉心中疑惑,便站
住细听,果然那边架下有人。此时正是五月,那蔷薇花叶茂盛之际,宝玉悄悄的隔
着药栏一看,只见一个女孩子蹲在花下,手里拿着根别头的簪子在地下抠土,一面
悄悄的流泪。宝玉心中想道:“难道这也是个痴丫头,又像颦儿来葬花不成?”因
又自笑道:“若真也葬花,可谓‘东施效颦’了,不但不为新奇,而且更是可厌。”
想毕,便要叫那女子说:“你不用跟着林姑娘学了。”话未出口,幸而再看时,这
女孩子面生,不是个侍儿,倒像是那十二个学戏的女孩子里头的一个,却辨不出他
是生、旦、净、丑那一个脚色来。宝玉把舌头一伸,将口掩住,自己想道:“幸而
不曾造次。上两回皆因造次了,颦儿也生气,宝儿也多心。如今再得罪了他们,越
发没意思了。”一面想,一面又恨不认得这个是谁。再留神细看,见这女孩子眉蹙
春山,眼颦秋水,面薄腰纤,袅袅婷婷,大有黛玉之态。宝玉早又不忍弃他而去,
只管痴看。

只见他虽然用金簪画地,并不是掘土埋花,竟是向土上画字。宝玉拿眼随着簪
子的起落,一直到底,一画、一点、一勾的看了去,数一数,十八笔。自己又在手
心里拿指头按着他方才下笔的规矩写了,猜是个什么字。写成一想,原来就是个蔷
薇花的“蔷”字。宝玉想道:“必定是他也要做诗填词,这会子见了这花,因有所
感。或者偶成了两句,一时兴至,怕忘了,在地下画着推敲,也未可知。且看他底
下再写什么。”一面想,一面又看,只见那女孩子还在那里画呢。画来画去,还是
个“蔷”字;再看,还是个“蔷”字。里面的原是早已痴了,画完一个“蔷”又画
一个“蔷”,已经画了有几十个。外面的不觉也看痴了,两个眼睛珠儿只管随着簪
子动,心里却想:“这女孩子一定有什么说不出的心事,才这么个样儿。外面他既
是这个样儿,心里还不知怎么熬煎呢?看他的模样儿这么单薄,心里那里还搁的住
熬煎呢?可恨我不能替你分些过来。”

却说伏中阴晴不定,片云可以致雨,忽然凉风过处,飒飒的落下一阵雨来。宝
玉看那女孩子头上往下滴水,把衣裳登时湿了。宝玉想道:“这是下雨了,他这个
身子,如何禁得骤雨一激。”因此禁不住便说道:“不用写了,你看身上都湿了。”
那女孩子听说,倒唬了一跳,抬头一看,只见花外一个人叫他“不用写了”。一则
宝玉脸面俊秀,二则花叶繁茂,上下俱被枝叶隐住,刚露着半边脸儿:那女孩子只
当也是个丫头,再不想是宝玉,因笑道:“多谢姐姐提醒了我。难道姐姐在外头有
什么遮雨的?”一句提醒了宝玉,“嗳哟”了一声,才觉得浑身冰凉。低头看看自
己身上,也都湿了。说:“不好!”只得一气跑回怡红院去了。心里却还记挂着那
女孩子没处避雨。

原来明日是端阳节,那文官等十二个女孩子都放了学,进园来各处玩耍。可巧
小生宝官正旦玉官两个女孩子,正在怡红院和袭人玩笑,被雨阻住,大家堵了沟,
把水积在院内,拿些绿头鸭、花、彩鸳鸯,捉的捉,赶的赶,缝了翅膀,放在
院内玩耍,将院门关了。袭人等都在游廊上嘻笑。宝玉见关着门,便用手扣门,里
面诸人只顾笑,那里听见。叫了半日,拍得门山响,里面方听见了。料着宝玉这会
子再不回来的,袭人笑道:“谁这会子叫门?没人开去。”宝玉道:“是我。”麝
月道:“是宝姑娘的声音。”晴雯道:“胡说,宝姑娘这会子做什么来?”袭人道:
“等我隔着门缝儿瞧瞧,可开就开,别叫他淋着回去。”说着,便顺着游廊到门前
往外一瞧,只见宝玉淋得雨打鸡一般。袭人见了,又是着忙,又是好笑,忙开了门,
笑着弯腰拍手道:“那里知道是爷回来了!你怎么大雨里跑了来?”

宝玉一肚子没好气,满心里要把开门的踢几脚。方开了门,并不看真是谁,还
只当是那些小丫头们,便一脚踢在肋上。袭人“嗳哟”了一声。宝玉还骂道:“下
流东西们,我素日担待你们得了意,一点儿也不怕,越发拿着我取笑儿了!”口里
说着,一低头见是袭人哭了,方知踢错了。忙笑道:“嗳哟!是你来了!踢在那里了?”
袭人从来不曾受过一句大话儿的,今忽见宝玉生气踢了他一下子,又当着许多人,
又是羞又是气又是疼,真一时置身无地。待要怎么样,料着宝玉未必是安心踢他,
少不得忍着说道:“没有踢着,还不换衣裳去呢!”宝玉一面进房解衣,一面笑道:
“我长了这么大,头一遭儿生气打人,不想偏偏儿就碰见你了。”袭人一面忍痛换
衣裳,一面笑道:“我是个起头儿的人,也不论事大事小,是好是歹,自然也该从
我起。但只是别说打了我,明日顺了手,只管打起别人来。”宝玉道:“我才也不
是安心。”袭人道:“谁说是安心呢!素日开门关门的都是小丫头们的事,他们是
憨皮惯了的,早已恨的人牙痒痒。他们也没个怕惧,要是他们,踢一下子唬唬也好。
刚才是我淘气,不叫开门的。”

说着,那雨已住了,宝官玉官也早去了。袭人只觉肋下疼的心里发闹,晚饭也
不曾吃。到晚间脱了衣服,只见肋上青了碗大的一块,自己倒唬了一跳,又不好声
张。一时睡下,梦中作痛,由不得“嗳哟”之声从睡中哼出。宝玉虽说不是安心,
因见袭人懒懒的,心里也不安稳。半夜里听见袭人“嗳哟”,便知踢重了,自己下
床来,悄悄的秉灯来照。刚到床前,只见袭人嗽了两声,吐出一口痰来,嗳哟一声,
睁眼见了宝玉,倒唬了一跳,道:“作什么?”宝玉道:“你梦里‘嗳哟’,必是
踢重了。我瞧瞧。”袭人道:“我头上发晕,嗓子里又腥又甜,你倒照一照地下罢。”
宝玉听说,果然持灯向地下一照,只见一口鲜血在地。宝玉慌了,只说:“了不得
了!”袭人见了,也就心冷了半截。

要知端的,下回分解。