\chapter{阻超凡佳人双护玉~欣聚党恶子独承家}

话说王夫人打发人来叫宝钗过去商量,宝玉听见说是和尚在外头,赶忙的独自
一人走到前头,嘴里乱嚷道:“我的师父在那里?”叫了半天,并不见有和尚,只
得走到外面。见李贵将和尚拦住,不放他进来。宝玉便说道:“太太叫我请师父进
去。”李贵听了,松了手,那和尚便摇摇摆摆的进来。宝玉看见那僧的形状与他死
去时所见的一般,心里早有些明白了,便上前施礼,连叫:“师父,弟子迎候来迟。”
那僧说:“我不要你们接待,只要银子拿了来,我就走。”宝玉听来,又不像有道
行的话。看他满头癞疮,浑身腌破烂,心里想道:“自古说‘真人不露相,露相
不真人’,也不可当面错过。我且应了他谢银,并探探他的口气。”便说道:“师
父不必性急。现在家母料理,请师父坐下,略等片刻。弟子请问师父:可是从太虚
幻境而来?”那和尚道:“什么‘幻境’,不过是来处来、去处去罢了。我是送还
你的玉来的。我且问你,那玉是从那里来的?”宝玉一时对答不来,那僧笑道:“你
自己的来路还不知,便来问我!”宝玉本来颖悟,又经点化,早把红尘看破,只是
自己的底里未知。一闻那僧问起玉来,好像当头一棒,便说道:“你也不用银子的,
我把那玉还你罢。”那僧笑道:“也该还我了。”

宝玉也不答言,往里就跑。走到自己院内,见宝钗袭人等都到王夫人那里去了,
忙向自己床边取了那玉,便走出来。迎面碰见了袭人,撞了一个满怀,把袭人唬了
一跳,说道:“太太说你陪着和尚坐着很好。太太在那里打算送他些银两,你又回
来做什么?”宝玉道:“你快去回太太说:不用张罗银子了,我把这玉还了他就是
了。”袭人听说,即忙拉住宝玉,道:“这断使不得的!那玉就是你的命,若是他
拿了去,你又要病着了。”宝玉道:“如今再不病的了。我已经有了心了,要那玉
何用?”摔脱袭人,便想要走。袭人急的赶着嚷道:“你回来,我告诉你一句话。”
宝玉回过头来道:“没有什么说的了。”袭人顾不得什么,一面赶着跑,一面嚷道:
“上回丢了玉,几乎没有把我的命要了。刚刚儿的有了,他拿了去,你也活不成,
我也活不成了!你要还他,除非是叫我死了!”说着,赶上一把拉住。宝玉急了,
道:“你死也要还,你不死也要还。”狠命的把袭人一推,抽身要走。怎奈袭人两
只手绕着宝玉的带子不放,哭着喊着坐在地下。

里面的丫头听见,连忙赶来,瞧见他两个人的神情不好。只听见袭人哭道:“快
告诉太太去!宝二爷要把那玉去还和尚呢!”丫头赶忙飞报王夫人。那宝玉更加生
气,用手来掰开了袭人的手。幸亏袭人忍痛不放。紫鹃在屋里听见宝玉要把玉给人,
这一急比别人更甚,把素日冷淡宝玉的主意都忘在九霄云外了,连忙跑出来,帮着
抱住宝玉。那宝玉虽是个男人,用力摔打,怎奈两个人死命的抱住不放,也难脱身,
叹口气道:“为一块玉,这样死命的不放!若是我一个人走了,你们又怎么样?”
袭人紫鹃听了这话,不禁嚎啕大哭起来。

正在难分难解,王夫人宝钗急忙赶来。见是这样形景,王夫人便哭着喝道:“宝
玉!你又疯了!”宝玉见王夫人来了,明知不能脱身,只得陪笑道:“这当什么,
又叫太太着急!他们总是这样大惊小怪。我说那和尚不近人情,他必要一万银子,
少一个不能。我生气进来,拿了这玉还他,就说是假的,要这玉干什么?他见我们
不希罕那玉,便随意给他些,就过去了。”王夫人道:“我打量真要还他!这也罢
了。为什么不告诉明白了他们?叫他们哭哭喊喊的像什么?”宝钗道:“这么说呢,
倒还使得。要是真拿那玉给他,那和尚有些古怪,倘或一给了他又闹到家口不宁,
岂不是不成事了么?至于银钱呢,就把我的头面折变了,也还够了呢。”王夫人听
了,道:“也罢了,且就这么办罢。”宝玉也不回答。只见宝钗走上来,在宝玉手
里拿了这玉,说道:“你也不用出去,我合太太给他钱就是了。”宝玉道:“玉不
还他也使得,只是我还得当面见他一见才好。”袭人等仍不肯放手。到底宝钗明决,
说:“放了手,由他去就是了。”袭人只得放手。宝玉笑道:“你们这些人,原来
重玉不重人哪。你们既放了我,我便跟着他走了,看你们就守着那块玉怎么样?”
袭人心里又着急起来,仍要拉他,只碍着王夫人和宝钗的面前,又不好太露轻薄,
恰好宝玉一撒手就走了。袭人忙叫小丫头在三门口传了焙茗等:“告诉外头照应着
二爷,他有些疯了。”小丫头答应了出去。

王夫人宝钗等进来坐下,问起袭人来由。袭人便将宝玉的话细细说了。王夫人
宝钗甚是不放心,又叫人出去,吩咐众人伺候,听着和尚说些什么。回来,小丫头
传话进来回王夫人道:“二爷真有些疯了。外头小厮们说:里头不给他玉,他也没
法儿;如今身子出来了,求那和尚带了他去。”王夫人听了,说道:“这还了得!
那和尚说什么来着?”小丫头回道:“和尚说,要玉不要人。”宝钗道:“不要银
子了么?”小丫头道:“没听见说。后来和尚合二爷两个人说着笑着,有好些话,
外头小厮们都不大懂。”王夫人道:“糊涂东西,听不出来,学是自然学得来的!”
便叫小丫头:“你把那小厮叫进来。”小丫头连忙出去叫进那小厮,站在廊下,隔
着窗户请了安。王夫人便问道:“和尚和二爷的话,你们不懂,难道学也学不来吗?”
那小厮回道:“我们只听见说什么‘大荒山’,什么‘青埂峰’,又说什么‘太虚
境’‘斩断尘缘’这些话。”王夫人听着也不懂。宝钗听了,唬得两眼直瞪,半句
话都没有了。

正要叫人出去拉宝玉进来,只见宝玉笑嘻嘻的进来,说:“好了,好了。”宝
钗仍是发怔。王夫人道:“你疯疯癫癫的说的是什么?”宝玉道:“正经话,又说
我疯癫!那和尚与我原认得的,他不过也是要来见我一见。他何尝是真要银子呢?也
只当化个善缘就是了。所以说明了,他自己就飘然而去了。这可不是好了么?”王
夫人不信,又隔着窗户问那小厮。那小厮连忙出去问了门上的人,进来回说:“果
然和尚走了,说:‘请太太们放心,我原不要银子,’只要宝二爷时常到他那里去
去就是了,‘诸事只要随缘,自有一定的道理。’”王夫人道:“原来是个好和尚!
你们曾问他住在那里?”小厮道:“门上的说,他说来着,我们二爷知道的。”王
夫人便问宝玉:“他到底住在那里?”宝玉笑道:“这个地方儿,说远就远,说近
就近。”宝钗不待说完,便道:“你醒醒儿罢!别尽着迷在里头!现在老爷太太就疼
你一个人,老爷还吩咐叫你干功名上进呢。”宝玉道:“我说的不是功名么?你们
不知道‘一子出家,七祖升天’?”王夫人听到那里,不觉伤起心来,说:“我们
的家运怎么好?一个四丫头口口声声要出家,如今又添出一个来了。我这样的日子
过他做什么!”说着,放声大哭。宝钗见王夫人伤心,只得上前苦劝。宝玉笑道:
“我说了一句玩话儿,太太又认起真来了。”王夫人止住哭声道:“这些话也是混
说的么?”

正闹着,只见丫头来回话:“琏二爷回来了,颜色大变,说请太太回去说话。”
王夫人又吃了一惊,说道:“将就些叫他进来罢。小婶子也是旧亲,不用回避了。”
贾琏进来见了王夫人,请了安。宝钗迎着,也问了贾琏的安。贾琏回道:“刚才接
了我父亲的书信,说是病重的很,叫我就去,迟了恐怕不能见面!说到那里,眼泪
便掉下来了。王夫人道:“书上写的是什么病?”贾琏道:“写的是感冒风寒起的,
如今竟成了痨病了。现在危急,专差一个人连日连夜赶来的,说:‘如若再耽搁一
两天,就不能见面了。’故来回太太,侄儿必得就去才好。只是家里没人照管。蔷
儿芸儿虽说糊涂,到底是个男人,外头有了事来,还可传个话。侄儿家里倒没有什
么事。秋桐是天天哭着喊着,不愿意在这里,侄儿叫了他娘家的人来领了去了,倒
省了平儿好些气。虽是巧姐没人照应,还亏平儿的心不很坏。姐儿心里也明白,只
是性气比他娘还刚硬些,求太太时常管教管教他。”说着,眼圈儿一红,连忙把腰
里拴槟榔荷包的小绢子拉下来擦眼。王夫人道:“放着他亲祖母在那里,托我做什
么?”贾琏轻轻的说道:“太太要说这个话,侄儿就该活活儿的打死了。没什么说
的,总求太太始终疼侄儿就是了!”说着,就跪下来了。

王夫人也眼圈儿红了,说:“你快起来!娘儿们说话儿,这是怎么说?只是一件:
孩子也大了,倘或你父亲有个一差二错,又耽搁住了,或者有个门当户对的来说亲,
还是等你回来,还是你太太作主?”贾琏道:“现在太太们在家,自然是太太们做
主,不必等我。”王夫人道:“你要去,就写了禀帖给二老爷送个信,说家下无人,
你父亲不知怎样,快请二老爷将老太太的大事早早的完结,快快回来。”贾琏答应
了“是”,正要走出去,复转回来,回说道:“咱们家的家下人,家里还够使唤,
只是园里没有人,太空了。包勇又跟了他们老爷去了。姨太太住的房子,薛二爷已
搬到自己的房子内住了。园里一带屋子都空着,忒没照应,还得太太叫人常查看查
看。那栊翠庵原是咱们家的地基,如今妙玉不知那里去了,所有的根基,他的当家
女尼不敢自己作主,要求府里一个人管理管理。”王夫人道:“自己的事还闹不清,
还搁得住外头的事么?这句话好歹别叫四丫头知道,若是他知道了,又要吵着出家
的念头出来了。你想咱们家什么样的人家?好好的姑娘出家,还了得。”贾琏道:
“太太不提起,侄儿也不敢说。四妹妹到底是东府里的,又没有父母,他亲哥哥又
在外头,他亲嫂子又不大说的上话。侄儿听见要寻死觅活了好几次。他既是心里这
么着的了,若是牛着他,将来倘或认真寻了死,比出家更不好了。”王夫人听了点
头,道:“这件事真真叫我也难担。我也做不得主,由他大嫂子去就是了。”

贾琏又说了几句,才出来,叫了众家人来,交代清楚。写了书,收拾了行装,
平儿等不免叮咛了好些话。只有巧姐儿惨伤的了不得。贾琏又欲托王仁照应,巧姐
到底不愿意;听见外头托了芸蔷二人,心里更不受用,嘴里却说不出来。只得送了
他父亲,谨谨慎慎的随着平儿过日子。丰儿小红因凤姐去世,告假的告假,告病的
告病。平儿意欲接了家中一个姑娘来,一则给巧姐作伴,二则可以带量他。遍想无
人。只有喜鸾四姐儿是贾母旧日钟爱的,偏偏四姐儿新近出了嫁了,喜鸾也有了人
家儿,不日就要出阁,也只得罢了。

且说贾芸贾蔷送了贾琏,便进来见了邢王二夫人。他两个倒替着在外书房住
下,日间便与家人厮闹,有时找了几个朋友吃个“车箍辘会”,甚至聚赌,里头那
里知道。一日邢大舅王仁来,瞧见了贾芸贾蔷住在这里,知他热闹,也就借着照看
的名儿时常在外书房设局赌钱喝酒。所有几个正经的家人,贾政带了几个去,贾琏
又跟去了几个,只有那赖林诸家的儿子侄儿。那些少年,托着老子娘的福吃喝惯了
的,那知当家立计的道理?况且他们长辈都不在家,便是“没笼头的马”了。又有
两个旁主人怂恿,无不乐为。这一闹,把个荣国府闹得没上没下,没里没外。

那贾蔷还想勾引宝玉。贾芸拦住道:“宝二爷那个人没运气的,不用惹他。那
一年我给他说了一门子绝好的亲:父亲在外头做税官,家里开几个当铺,姑娘长的
比仙女儿还好看。我巴巴儿的细细的写了一封书子给他,谁知他没造化。”说到这
里,瞧了瞧左右无人,又说:“他心里早和咱们这个二婶娘好上了。你没听见说:
还有一个林姑娘呢,弄的害了相思病死的,谁不知道!这也罢了,各自的姻缘罢咧。
谁知他为这件事倒恼了我了,总不大理。他打量谁必是借谁的光儿呢!”贾蔷听了,
点点头,才把这个心歇了。

他两个还不知道宝玉自会那和尚以后,他是欲断尘缘,一则在王夫人跟前不敢
任性,已与宝钗袭人等皆不大款洽了。那些丫头不知道,还要逗他,宝玉那里看得
到眼里。他也并不将家事放在心里。时常王夫人宝钗劝他念书,他便假作攻书,一
心想着那个和尚引他到那仙境的机关,心目中触处皆为俗人。却在家难受,闲来倒
与惜春闲讲。他们两个人讲得上了,那种心更加准了几分,那里还管贾环贾兰等。
那贾环为他父亲不在家,赵姨娘已死,王夫人不大理会,他便入了贾蔷一路。倒是
彩云时常规劝,反被贾环辱骂。玉钏儿见宝玉疯癫更甚,早和他娘说了,要求着出
去。如今宝玉贾环他哥儿两个,各有一种脾气,闹得人人不理。独有贾兰跟着他母
亲上紧攻书,作了文字,送到学里请教代儒。因近来代儒老病在床,只得自己刻苦。
李纨是素来沉静的,除请王夫人的安,会会宝钗,馀者一步不走,只有看着贾兰攻
书。所以荣府住的人虽不少,竟是各自过各自的,谁也不肯做谁的主。贾环贾蔷等
愈闹的不像事了,甚至偷典偷卖,不一而足。贾环更加宿娼滥赌,无所不为。

一日,邢大舅王仁都在贾家外书房喝酒,一时高兴,叫了几个陪酒的来唱着喝
着劝酒。贾蔷便说:“你们闹的太俗,我要行个令儿。”众人道:“使得。”贾蔷
道:“咱们‘月字流觞’罢。我先说起,‘月’字数到那个,便是那个喝酒。还要
酒面酒底;须得依着令官,不依者罚三大杯。”众人都依了。贾蔷喝了一杯令酒,
便说:“飞羽觞而醉月。”顺饮数到贾环。贾蔷道:“酒面要个‘桂’字。”贾环
便说道:“冷露无声湿桂花。——酒底呢?”贾蔷道:“说个‘香’字。”贾环道:
“天香云外飘。”邢大舅说道:“没趣没趣,你又懂得什么字了,也假斯文起来。
这不是取乐,竟是怄人了。咱们都蠲了,倒是拳,输家喝输家唱,叫作‘苦中苦’。
若是不会唱的,说个笑话儿也使得,只要有趣。”众人都道:“使得。”于是乱
起来。王仁输了,喝了一杯,唱了一个,众人道好。又起来了,是个陪酒的输了,
唱了一个什么“小姐小姐多丰彩”。以后邢大舅输了,众人要他唱曲儿。他道:“我
唱不上来,我说个笑话儿罢。”贾蔷道:“若说不笑人,仍要罚的。”

邢大舅就喝了一杯,说道:“诸位听着:村庄上有一座玄帝庙,旁边有个土地
祠。那玄帝老爷常叫土地来说闲话儿。一日,玄帝庙里被了盗,便叫土地去查访。
土地禀道:‘这地方没有贼的,必是神将不小心,被外贼偷了东西去。”玄帝道:
‘胡说!你是土地,失了盗,不问你问谁去呢?你倒不去拿贼,反说我的神将不小心
吗?’土地禀道:‘虽说是不小心,到底是庙里的风水不好。’玄帝道:‘你倒会
看风水么?’土地道:‘待小神看看。’那土地向各处瞧了一会,便来回禀道:‘老
爷坐的身子背后,两扇红门,就不谨慎。小神坐的背后,是砌的墙,自然东西丢不
了。以后老爷的背后也改了墙就好了。’玄帝老爷听来有理,便叫神将派人打墙。
众神将叹口气道:‘如今香火一炷也没有,那里有砖灰人工来打墙呢?’玄帝老爷
没法,叫神将作法,却都没有主意。那玄帝老爷脚下的龟将军站起来道:‘你们不
中用,我有主意:你们将红门拆下来,到了夜里,拿我的肚子堵住这门口,难道当
不得一堵墙么?’众神将都说道:‘好,又不花钱,又便当结实。’于是龟将军便
当这个差使,竟安静了。岂知过了几天,那庙里又丢了东西。众神将叫了土地来,
说道:‘你说砌了墙就不丢东西,怎么如今有了墙还要丢?’那土地道:‘这墙砌
的不结实。’众神将道:‘你瞧去。’土地一看,果然是一堵好墙,怎么还有失事?
把手摸了一摸,道:‘我打量是真墙,那里知道是个假墙!’”

众人听了,大笑起来。贾蔷也忍不住的笑,说道:“傻大舅你好!我没有骂你,
你为什么骂我?快拿杯来,罚一大杯。”邢大舅喝了,已有醉意。众人又喝了几杯,
都醉起来。邢大舅说他姐姐不好,王仁说他妹妹不好,都说的狠狠毒毒的。贾环听
了,趁着酒兴,也说凤姐不好,怎样苛刻我们,怎么样踏我们的头。众人道:“大
凡做个人,原要厚道些。看凤姑娘仗着老太太这样的利害,如今‘焦了尾巴梢子’
了,只剩了一个姐儿,只怕也要现世现报呢。”贾芸想着凤姐待他不好,又想起巧
姐儿见他就哭,也信着嘴儿混说。还是贾蔷道:“喝酒罢,说人家做什么。”那两
个陪酒的道:“这位姑娘多大年纪了?长得怎么样?”贾蔷道:“模样儿是好的很
的,年纪也有十三四岁了。”那陪酒的说道:“可惜这样人生在府里这样人家。若
生在小户人家,父母兄弟都做了官,还发了财呢。”众人道:“怎么样?”那陪酒
的说:“现今有个外藩王爷,最是有情的,要选一个妃子。若合了式,父母兄弟都
跟了去,可不是好事儿吗?”众人都不大理会,只有王仁心里略动了一动,仍旧喝
酒。

只见外头走进赖林两家的子弟来,说:“爷们好乐呀!”众人站起来说道:“老
大,老三,怎么这时候才来?叫我们好等。”那两个人说道:“今早听见一个谣言,
说是咱们家又闹出事来了。心里着急,赶到里头打听去,并不是咱们。”众人道:
“不是咱们就完了,为什么不就来?”那两个说道:“虽不是咱们,也有些干系。
你们知道是谁?就是贾雨村老爷。我们今儿进去,看见带着锁子,说要解到三法司
衙门里审问去呢。我们见他常在咱们家里来往,恐有什么事,便跟了去打听。”贾
芸道:“到底老大用心,原该打听打听。你且坐下喝一杯再说。”两人让了一回,
便坐下喝着酒,道:“这位雨村老爷,人也能干,也会钻营,官也不小了,只是贪
财。被人家参了个‘婪索属员’的几款。如今的万岁爷是最圣明最仁慈的,独听了
一个‘贪’字,或因遭塌了百姓,或因恃势欺良,是极生气的,所以旨意便叫拿问。
若问出来了,只怕搁不住;若是没有的事,那参的人也不便。如今真真是好时候!
只要有造化,做个官儿就好。”众人道:“你的哥哥就是有造化的,现做知县,还
不好么?”赖家的说道:“我哥哥虽是做了知县,他的行为只怕也保不住怎么样呢。”
众人道:“手也长么?”赖家的点点头儿,便举起杯来喝酒。

众人又道:“里头还听见什么新闻?”两人道:“别的事没有,只听见海疆的
贼寇拿住了好些,也解到法司衙门里审问。还审出好些贼寇,也有藏在城里的,打
听消息,抽空儿就劫抢人家。如今知道朝里那些老爷们都是能文能武,出力报效,
所到之处,早就消灭了。”众人道:“你听见有在城里的,不知审出咱们家失盗的
一案来没有?”两人道:“倒没有听见,恍惚有人说是有个内地里的人,城里犯了
事,抢了一个女人下海去了,那女人不依,被这贼寇杀了。那贼寇正要逃出关去,
被官兵拿住了,就在拿获的地方正了法了。”众人道:“咱们栊翠庵的什么妙玉,
不是叫人抢去?不要就是他罢?”贾环道:“必是他。”众人道:“你怎么知道?”
贾环道:“妙玉这个东西是最讨人嫌的,他一日家捏酸,见了宝玉就眉开眼笑了。
我若见了他,他从不拿正眼瞧我一瞧,真要是他,我才趁愿呢!”众人道:“抢的
人也不少,那里就是他?”贾芸说:“有点信儿。前日有个人说他庵里的道婆做梦,
说看见是妙玉叫人杀了。”众人笑道:“梦话算不得。”邢大舅道:“管他梦不梦,
咱们快吃饭罢,今夜做个大输赢。”众人愿意,便吃毕了饭,大赌起来。

赌到三更多天,只听见里头乱嚷,说是:“四姑娘合珍大奶奶拌嘴,把头发都
铰了。赶到邢夫人王夫人那里去磕了头,说是要求容他做尼姑呢,送他一个地方儿;
若不容他,他就死在眼前。那邢王两位太太没主意,叫请蔷大爷芸二爷进去。”贾
芸听了,便知是那回看家的时候起的念头,想来是劝不过来的了,便合贾蔷商议道:
“太太叫我们进去,我们是做不得主的,况且也不好做主。只好劝去,若劝不住,
只好由他们罢。咱们商量了写封书给琏二叔,便卸了我们的干系了。”两人商量定
了主意,进去见了邢王两位太太,便假意的劝了一回。无奈惜春立意必要出家,就
不放他出去,只求一两间净屋子,给他诵经拜佛。尤氏见他两个不肯作主,又怕惜
春寻死,自己便硬做主张,说是:“这个不是索性我耽了罢:说我做嫂子的容不下
小姑子,逼的他出了家了,就完了!若说到外头去呢,断断使不得;若在家里呢,
太太们都在这里,算我的主意罢。叫蔷哥儿写封书子给你珍大爷琏二叔就是了。”
贾蔷等答应了。

不知邢王二夫人依与不依,下回分解。