\chapter{宴海棠贾母赏花妖~失宝玉通灵知奇祸}

话说赖大带了贾芹出来,一宿无话,静候贾政回来。单是那些女尼女道重进园
来,都喜欢的了不得,欲要到各处逛逛,明日预备进宫。不料赖大便吩咐了看园的
婆子并小厮看守,惟给了些饭食,却是一步不准走开。那些女孩子摸不着头脑,只
得坐着,等到天亮。园里各处的丫头虽都知道拉进女尼们来,预备宫里使唤,却也
不能深知原委。

到了明日早起,贾政正要下班,因堂上发下两省城工估销册子,立刻要查核,
一时不能回家,便叫人回来告诉贾琏,说:“赖大回来,你务必查问明白。该如何
办就如何办了,不必等我。”贾琏奉命,先替芹儿喜欢,又想道:若是办得一点影
儿都没有,又恐贾政生疑,“不如回明二太太,讨个主意办去,便是不合老爷的心,
我也不至甚担干系。”主意定了,进内去见王夫人,陈说:“昨日老爷见了揭帖生
气,把芹儿和女尼女道等都叫进府来查办。今日老爷没空问这件不成体统的事,叫
我来回太太,该怎么便怎么样。我所以来请示太太,这件事如何办理?”

王夫人听了诧异道:“这是怎么说!若是芹儿这么样起来,这还成咱们家的人
了么?但只这个贴帖儿的也可恶,这些话可是混嚼说得的么?你到底问了芹儿有这件
事没有呢?”贾琏道:“刚才也问过了。太太想,别说他干了没有,就是干了,一
个人干了混账事也肯应承么?但只我想芹儿也不敢行此事:知道那些女孩子都是娘
娘一时要叫的,倘或闹出事来,怎么样呢?依侄儿的主见,要问也不难,若问出来,
太太怎么个办法呢?”王夫人道:“如今那些女孩子在那里?”贾琏道:“都在园
里锁着呢。”王夫人道:“姑娘们知道不知道?”贾琏道:“大约姑娘们也都知道
是预备宫里头的话,外头并没提起别的来。”王夫人道:“很是。这些东西一刻也
是留不得的。头里我原要打发他们去来着,都是你们说留着好,如今不是弄出事来
了么?你竟叫赖大带了去细细儿的问他的本家儿有人没有,将文书查出,花上几十
两银子,雇只船,派个妥当人,送到本地,一概连文书发还了,也落得无事。若是
为着一两个不好,个个都押着他们还俗,那又太造孽了。若在这里发给官媒,虽然
我们不要身价,他们弄去卖钱,那里顾人的死活呢?芹儿呢,你便狠狠的说他一顿,
除了祭祀喜庆,无事叫他不用到这里来。看仔细碰在老爷气头儿上,那可就吃不了
兜着走了。也说给账房儿里,把这一项钱粮档子销了。还打发个人到水月庵,说老
爷的谕,除了上坟烧纸,要有本家爷们到他那里去,不许接待。若再有一点不好风
声,连老姑子一块儿撵出去。”

贾琏一一答应了。出去将王夫人的话告诉赖大,说:“太太的主意,叫你这么
办。办完了,告诉我去回太太。你快办去罢。回来老爷来,你也按着太太的话回去。”
赖大听说,便道:“我们太太真正是个佛心。这班东西还着人送回去,既是太太好
心,不得不挑个好人。芹哥儿竟交给二爷开发了罢。那贴帖儿的,奴才想法儿查出
来,重重的收拾他才好。”贾琏点头说:“是了。”即刻将贾芹发落。赖大也赶着
把女尼等领出,按着主意办去了。晚上贾政回来,贾琏赖大回明贾政,贾政本是省
事的人,听了也便撂开手了。独有那些无赖之徒,听得贾府发出二十四个女孩子来,
那个不想?究竟那些人能够回家不能,未知着落,亦难虚拟。

且说紫鹃因黛玉渐好,园中无事,听见女尼等预备宫内使唤,不知何事便到贾
母那边打听打听。恰遇着鸳鸯下来闲着,坐下说闲话儿,提起女尼的事,鸳鸯诧异
道:“我并没有听见。回来问问二奶奶就知道了。”正说着,只见傅试家两个女人
过来请贾母的安,鸳鸯要陪了上去。那两个女人因贾母正睡晌觉,就与鸳鸯说了一
声儿,回去了。紫鹃问:“这是谁家差来的?”鸳鸯道:“好讨人嫌!家里有了一
个女孩儿,长的好些儿,就献宝的似的,常在老太太跟前夸他们姑娘怎么长的好,
心地儿怎么好,‘礼貌上又好,说话儿又简绝,做活计儿手儿又巧,会写会算,尊
长上头最孝敬的,就是待下人也是极和平的。’来了就编这么一大套,常说给老太
太听。我听着很烦。这几个老婆子真讨人嫌,我们老太太偏爱听那些个话。老太太
也罢了,还有宝玉,素常见了老婆子便很厌烦的,偏见了他们家的老婆子就不厌烦,
你说奇不奇?前儿还来说:他们姑娘现有多少人家儿来求亲,他们老爷总不肯应,
心里只要和咱们这样人家作亲才肯。夸奖一回,奉承一回,把老太太的心都说活
了。”

紫鹃听了一呆,便假意道:“若老太太喜欢,为什么不就给宝玉定了呢?”鸳
鸯正要说出原故,听见上头说:“老太太醒了。”鸳鸯赶着上去,紫鹃只得起身出
来。回到园里,一头走,一头想道:“天下莫非只有一个宝玉?你也想他,我也想
他。我们家的那一位,越发痴心起来了!看他的那个神情儿,是一定在宝玉身上的
了,三番两次的病,可不是为着这个是什么?这家里‘金’的‘银’的还闹不清,
再添上一个什么傅姑娘,更了不得了。我看宝玉的心也在我们那一位的身上啊,听
着鸳鸯的话,竟是见一个爱一个的。这不是我们姑娘白操了心了吗?”紫鹃本是想
着黛玉,往下一想,连自己也不得主意了,不免神都痴了。要想叫黛玉不用瞎操心
呢,又恐怕他烦恼;要是看着他这样,又可怜见儿的。左思右想,一时烦躁起来,
自己啐自己道:“你替人耽什么忧!就是林姑娘真配了宝玉,他的那性情儿也是难
伏侍的。宝玉性情虽好,又是贪多嚼不烂的。我倒劝人不必瞎操心,我自己才是瞎
操心呢,从今以后,我尽我的心伏侍姑娘,其馀的事全不管。”这么一想,心里倒
觉清净。回到潇湘馆来,见黛玉独自一人坐在炕上,理从前做过的诗文词稿。抬头
见紫鹃进来,便问:“你到那里去了?”紫鹃道:“今儿瞧了瞧姐妹们去。”黛玉
道:“可是找袭人姐姐去么?”紫鹃道:“我找他做什么?”黛玉一想:“这话怎
么顺嘴说出来了呢?”反觉不好意思,便啐道:“你找不找与我什么相干!倒茶去
罢。”

紫鹃也心里暗笑,出来倒茶。只听园里一叠声乱嚷,不知何故。一面倒茶,一
面叫人去打听。回来说道:“怡红院里的海棠本来萎了几棵,也没人去浇灌他。昨
日宝玉走去瞧,见枝头上好像有了朵儿似的。人都不信,没有理他。忽然今日开
的很好的海棠花,众人诧异,都争着去看,连老太太、太太都哄动了,来瞧花儿呢。
所以大奶奶叫人收拾园里的树叶子,这些人在那里传唤。”黛玉也听见了,知道老
太太来,便更了衣,叫雪雁去打听:“若是老太太来了,即来告诉我。”雪雁去不
多时,便跑来说:“老太太、太太好些人都来了,请姑娘就去罢。”黛玉略自照了
一照镜子,掠了一掠鬓发,便扶着紫鹃到怡红院来,已见老太太坐在宝玉常卧的榻
上。黛玉便说道:“请老太太安。”退后便见了邢王二夫人,回来与李纨、探春、
惜春、邢岫烟彼此问了好。只有凤姐因病未来;史湘云因他叔叔调任回京,接了家
去;薛宝琴跟他姐姐家去住了;李家姐妹因见园内多事,李婶娘带了在外居住:所
以黛玉今日见的只有数人。

大家说笑了一回,讲究这花开得古怪。贾母道:“这花儿应在三月里开的,如
今虽是十一月,因节气迟,还算十月,应着小阳春的天气,因为和暖,开花也是有
的。”王夫人道:“老太太见的多,说得是,也不为奇。”邢夫人道:“我听见这
花已经萎了一年,怎么这回不应时候儿开了?必有个原故。”李纨笑道:“老太太
和太太说的都是。据我的糊涂想头,必是宝玉有喜事来了,此花先来报信。”探春
虽不言语,心里想道:“必非好兆。大凡顺者昌,逆者亡;草木知运,不时而发,
必是妖孽。”但只不好说出来。独有黛玉听说是喜事,心里触动,便高兴说道:“当
初田家有荆树一棵,弟兄三个因分了家,那荆树便枯了。后来感动了他弟兄们,仍
旧归在一处,那荆树也就荣了。可知草木也随人的。如今二哥哥认真念书,舅舅喜
欢,那棵树也就发了。”贾母王夫人听了喜欢,便说:“林姑娘比方得有理,很有
意思。”

正说着,贾赦、贾政、贾环、贾兰都进来看花。贾赦便说:“据我的主意,把
他砍去。必是花妖作怪。”贾政道:“见怪不怪,其怪自败。不用砍他,随他去就
是了。”贾母听见,便说:“谁在这里混说?人家有喜事好处,什么怪不怪的。若
有好事,你们享去;若是不好,我一个人当去。你们不许混说!”贾政听了,不敢
言语,讪讪的同贾赦等走了出来。

那贾母高兴,叫人传话到厨房里:“快快预备酒席,大家赏花。”叫宝玉、环
儿、兰儿:“各人做一首诗志喜。林姑娘的病才好,别叫他费心,若高兴,给你们
改改。”对着李纨道:“你们都陪我喝酒。”李纨答应了是,便笑对探春笑道:“都
是你闹的。”探春道:“饶不叫我们做诗,怎么我们闹的?”李纨道:“海棠社不
是你起的么?如今那棵海棠也要来入社了。”大家听着都笑了。

一时摆上酒菜,一面喝着,彼此都要讨老太太的喜欢,大家说些兴头话。宝玉
上来斟了酒,便立成了四句诗,写出来念与贾母听,道:
海棠何事忽摧?今日繁花为底开?
应是北堂增寿考,一阳旋复占先梅。
贾环也写了来,念道:
草木逢春当茁芽,海棠未发候偏差。
人间奇事知多少,冬月开花独我家。
贾兰恭楷誊正,呈与贾母。贾母命李纨念道:
烟凝媚色春前萎,霜微红雪后开。
莫道此花知识浅,欣荣预佐合欢杯。
贾母听毕,便说:“我不大懂诗,听去倒是兰儿的好,环儿做的不好。都上来吃饭
罢。”宝玉看见贾母喜欢,更是兴头,因想起:“晴雯死的那年,海棠死的;今日
海棠复荣,我们院内这些人,自然都好,但是晴雯不能像花的死而复生了。”顿觉
转喜为悲。忽又想起前日巧姐提凤姐要把五儿补入,“或此花为他而开,也未可知。”
却又转悲为喜,依旧说笑。

贾母还坐了半天,然后扶了珍珠回去了,王夫人等跟着过来。只见平儿笑嘻嘻
的迎上来,说:“我们奶奶知道老太太在这里赏花,自己不得来,叫奴才来伏侍老
太太、太太们。还有两匹红送给宝二爷包裹这花,当作贺礼。”袭人过来接了,呈
与贾母看。贾母笑道:“偏是凤丫头行出点事儿来,叫人看着又体面,又新鲜,很
有趣儿。”袭人笑着向平儿道:“回去替宝二爷给二奶奶道谢:要有喜,大家喜。”
贾母听了,笑道:“嗳哟!我还忘了呢。凤丫头虽病着,还是他想的到,送的也巧。”
一面说着,众人就随着去了。平儿私与袭人道:“奶奶说,这花儿开的怪,叫你铰
块红绸子挂挂,就应在喜事上去了。以后也不必只管当作奇事混说。”袭人点头答
应,送了平儿出去不提。

且说那日宝玉本来穿着一裹圆的皮袄在家歇息,因见花开,只管出来看一回、
赏一回、叹一回、爱一回的,心中无数悲喜离合,都弄到这株花上去了。忽然听说
贾母要来,便去换了一件狐腋箭袖,罩一件玄狐腿外褂,出来迎接贾母。匆匆穿换,
未将“通灵宝玉”挂上。及至后来贾母去了,仍旧换衣,袭人见宝玉脖子上没有挂
着,便问:“那块玉呢?”宝玉道:“刚才忙乱换衣,摘下来放在炕桌上,我没有
带。”袭人回看桌上,并没有玉,便向各处找寻,踪影全无,吓得袭人满身冷汗。
宝玉道:“不用着急,少不得在屋里的。问他们就知道了。”袭人当作麝月等藏起
吓他玩,便向麝月等笑着说道:“小蹄子们,玩呢,到底有个玩法。把这件东西藏
在那里了?别真弄丢了,那可就大家活不成了!”麝月等都正色道:“这是那里的
话?玩是玩,笑是笑,这个事非同儿戏,你可别混说。你自己昏了心了,想想罢,
想想搁在那里了?这会子又混赖人了!”袭人见他这般光景不像是玩话,便着急道:
“皇天菩萨!小祖宗!你到底撂在那里了?”宝玉道:“我记的明明儿放在炕桌上,
你们到底找啊。”

袭人麝月等也不敢叫人知道,大家偷偷儿的各处搜寻。闹了大半天,毫无影响,
甚至翻箱倒笼,实在没处去找,便疑到方才这些人进来,不知谁检了去了。袭人说
道:“进来的,谁不知道这玉是性命似的东西呢?谁敢检了去!你们好歹先别声张,
快到各处问去。若有姐妹们检着和我们玩呢,你们给他磕个头,要了来;要是小丫
头们偷了去,问出来,也不回上头,不论做些什么送他换了来,都使得的。这可不
是小事,真要丢了这个,比丢了宝二爷的还利害呢!”麝月秋纹刚要往外走,袭人
又赶出来嘱咐道:“头里在这里吃饭的倒别先问去。找不成,再惹出些风波来,更
不好了。”麝月等依言,分头各处追问。人人不晓,个个惊疑。二人连忙回来,俱
目瞪口呆,面面相窥。宝玉也吓怔了,袭人急的只是干哭。找是没处找,回又不敢
回,怡红院里的人吓的一个个像木雕泥塑一般。

大家正在发呆,只见各处知道的都来了。探春叫把园门关上,先叫个老婆子带
着两个丫头,再往各处去寻去;一面又叫告诉众人:“若谁找出来,重重的赏他。”
大家头宗要脱干系,二宗听见重赏,不顾命的混找了一遍,甚至于茅厕里都找到了。
谁知那块玉竟像绣花针儿一般,找了一天,总无影响。李纨急了,说:“这件事不
是玩的,我要说句无礼的话了。”众人道:“什么话?”李纨道:“事情到了这里
也顾不得了。现在园里除了宝玉,都是女人。要求各位姐姐、妹妹、姑娘都要叫跟
来的丫头脱了衣服,大家搜一搜。若没有,再叫丫头们去搜那些老婆子并粗使的丫
头,不知使得使不得?”大家说道:“这话也说的有理。现在人多手乱,鱼龙混杂,
倒是这么着,他们也洗洗清。”探春独不言语。那些丫头们也都愿意洗净自己。先
是平儿起,平儿说道:“打我先搜起。”于是各人自己解怀。李纨一气儿混搜。探
春嗔着李纨道:“大嫂子,你也学那起不成材料的样子来了!那个人既偷了去还肯
藏在身上?况且这件东西,在家里是宝,到了外头不知道的是废物,偷他做什么?我
想来必是有人使促狭。”

众人听说,又见环儿不在这里,昨儿是他满屋里乱跑,都疑到他身上,只是不
肯说出来。探春又道:“使促狭的只有环儿。你们叫个人去悄悄的叫了他来,背地
里哄着他,叫他拿出来,然后吓着他叫他别声张就完了。”大家点头。李纨便向平
儿道:“这件事还得你去才弄的明白。”平儿答应,就赶着去了。不多时,同着贾
环来了。众人假意装出没事的样子,叫人沏了茶,搁在里间屋里。众人故意搭讪走
开,原叫平儿哄他。平儿便笑着向贾环道:“你二哥哥的玉丢了,你瞧见了没有?”
贾环便急的紫涨了脸,瞪着眼,说道:“人家丢了东西,你怎么又叫我来查问疑我!
我是犯过案的贼么?”平儿见这样子,倒不敢再问,便又陪笑道:“不是这么说。
怕三爷要拿了去吓他们,所以白问问瞧见了没有,好叫他们找。”贾环道:“他的
玉在他身上,看见没看见该问他,怎么问我呢?你们都捧着他,得了什么不问我,
丢了东西就来问我!”说着,起身就走。众人不好拦他。这里宝玉倒急了,说道:
“都是这劳什子闹事!我也不要他了,你们也不用闹了。环儿一去,必是嚷的满院
里都知道了,这可不是闹事了么?”袭人等急的又哭道:“小祖宗儿,你看这玉丢
了没要紧;要是上头知道了,我们这些人就要粉身碎骨了。”说着,便嚎啕大哭起
来。

众人更加着急,明知此事掩饰不来,只得要商议定了话,回来好回贾母诸人。
宝玉道:“你们竟也不用商量,硬说我砸了就完了。”平儿道:“我的爷,好轻巧
话儿!上头要问为什么砸的呢?他们也是个死啊。倘或要起砸破的碴儿来,那又怎么
样呢?”宝玉道:“不然,就说我出门丢了。”众人一想:“这句话倒还混的过去,
但只这两天又没上学,又没往别处去。”宝玉道:“怎么没有?大前儿还到临安伯
府里听戏去了呢。就说那日丢的就完了。”探春道:“那也不妥。既是前儿丢的,
为什么当日不来回?”众人正在胡思乱想要装点撒谎,只听见赵姨娘的声儿哭着喊
着走来,说:“你们丢了东西,自己不找,怎么叫人背地里拷问环儿!我把环儿带
了来,索性交给你们这一起上水的,该杀该剐随你们罢!”说着将环儿一推,说:
“你是个贼,快快的招罢!”气的环儿也哭喊起来。

李纨正要劝解,丫头来说:“太太来了。”袭人等此时无地可容。宝玉等赶忙
出来迎接。赵姨娘暂且也不敢作声,跟了出来。王夫人见众人都有惊惶之色,才信
方才听见的话,便道:“那块玉真丢了么?”众人都不敢作声。王夫人走进屋里坐
下,便叫袭人,慌的袭人连忙跪下,含泪要禀。王夫人道:“你起来,快快叫人细
细的找去,一忙乱倒不好了。”袭人哽咽难言。宝玉恐袭人直告诉出来,便说道:
“太太,这事不与袭人相干,是我前日到临安伯府里听戏在路上丢了。”王夫人道:
“为什么那日不找呢?”宝玉道:“我怕他们知道,没有告诉他们。我叫焙茗等在
外头各处找过的。”王夫人道:“胡说,如今脱换衣服,不是袭人他们伏侍的么?
大凡哥儿出门回来,手巾荷包短了,还要个明白,何况这块玉不见了,难道不问
么?”宝玉无言可答。赵姨娘听见,便得意了,忙接口道:“外头丢了东西,也赖
环儿——”话未说完,被王夫人喝道:“这里说这个,你且说那些没要紧的话!”
赵姨娘便也不敢言语了。还是李纨探春从实的告诉了王夫人一遍。王夫人也急的眼
中落泪,索性要回明了贾母,去问邢夫人那边来的这些人去。

凤姐病中也听见宝玉失玉,知道王夫人过来,料躲不住,便扶了丰儿来到园里。
正值王夫人起身要走,凤姐娇怯怯的说:“请太太安。”宝玉等过来问了凤姐好。
王夫人因说道:“你也听见了么?这可不是奇事吗?刚才眼错不见就丢了,再找不着。
你去想想:打老太太那边的丫头起,至你们平儿,谁的手不稳,谁的心促狭,我要
回了老太太,认真的查出来才好。不然,是断了宝玉的命根子了!”凤姐回道:“咱
们家人多手杂,自古说的,‘知人知面不知心’,那里保的住谁是好的?但只一吵
嚷,已经都知道了,偷玉的人要叫太太查出来,明知是死无葬身之地,他着了急,
反要毁坏了灭口,那时可怎么处呢。据我的糊涂想头,只说宝玉本不爱他,撂丢了,
也没有什么要紧,只要大家严密些,别叫老太太老爷知道。这么说了,暗暗的派人
去各处察访,哄骗出来,那时玉也可得,罪名也可定:不知太太心里怎么样?”王
夫人迟了半日,才说道:“你这话虽也有理,但只是老爷跟前怎么瞒的过呢?”便
叫环儿来说道:“你二哥哥的玉丢了,白问了你一句,怎么你就乱嚷?要是嚷破了,
人家把那个毁坏了,我看你活得活不得!”贾环吓得哭道:“我再不敢嚷了。”赵
姨娘听了,那里还敢言语。王夫人便吩咐众人道:“想来自然有没找到的地方儿。
好端端的在家里的,还怕他飞到那里去不成?只是不许声张。限袭人三天内给我找
出来。要是三天找不着,只怕也瞒不住,大家那就不用过安静日子了!”说着,便
叫凤姐儿跟到邢夫人那边,商议踩缉不提。

这里李纨等纷纷议论,便传唤看园子的一干人来,叫把园门锁上,快传林之孝
家的来,悄悄儿的告诉了他,叫他:“吩咐前后门上:三天之内,不论男女下人,
从里头可以走动,要出去时,一概不许放出。只说里头丢了东西,等这件东西有了
着落,然后放人出来。”林之孝家的答应了“是”,因说:“前儿奴才家里也丢了
一件不要紧的东西,林之孝必要明白,上街去找了一个测字的。那人叫做什么刘铁
嘴,测了一个字,说的很明白,回来按着一找,就找着了。”袭人听见,便央及林
家的道:“好林奶奶,出去快求林大爷替我们问问去。”那林之孝家的答应着出去
了。邢岫烟道:“若说那外头测字打卦的,是不中用的。我在南边闻妙玉能扶乩,
何不烦他问一问?况且我听见说,这块玉原有仙机,想来问的出来。”众人都诧异
道:“咱们常见的,从没有听他说起。”麝月便忙问岫烟道:“想来别人求他是不
肯的,好姑娘,我给姑娘磕个头,求姑娘就去!若问出来了,我一辈子总不忘你的
恩。”说着,赶忙就要磕下头去,岫烟连忙拦住。黛玉等也都怂恿着岫烟速往栊翠
庵去。

一面林之孝家的进来说道:“姑娘们大喜!林之孝测了字回来,说这玉是丢不
了的,将来横竖有人送还来的。”众人听了,也都半信半疑,惟有袭人麝月喜欢的
了不得。探春便问:“测的是什么字?”林之孝家的道:“他的话多,奴才也学不
上来。记得是拈了个赏人东西的‘赏’字。那刘铁嘴也不问,便说:‘丢了东西不
是?’”李纨道:“这就算好。”林之孝家的道:“他还说:‘“赏”字上头一个
“小”字,底下一个“口”字,这件东西,很可嘴里放得,必是个珠子宝石。’”
众人听了,夸赞道:“真是神仙!往下怎么说?”林之孝家的道:“他说:‘底下
“贝”字拆开,不成一个“见”字,可不是“不见”了?’因上头拆了‘当’字,
叫快到当铺里找去。‘赏’字加一‘人’字,可不是‘’字?只要找着当铺就有
人,有了人便赎了来,可不是偿还了吗?”众人道:“既这么着,就先往左近找起。
横竖几个当铺都找遍了,少不得就有了。咱们有了东西,再问人就容易了。”李纨
道:“只要东西,那怕不问人都使得。林嫂子你去,就把测字的话快告诉了二奶奶,
回了太太,也叫太太放心。就叫二奶奶快派人查去。”林家的答应了便走。

众人略安了一点儿神,呆呆的等岫烟回来。正呆等时,只见跟宝玉的焙茗在门
外招手儿,叫小丫头子快出来。那小丫头赶忙的出去了。焙茗便说道:“你快进去
告诉我们二爷和里头太太、奶奶、姑娘们,天大的喜事!”那小丫头子道:“你快
说罢,怎么这么累赘?”焙茗笑着拍手道:“我告诉姑娘,姑娘进去回了,咱们两
个人都得赏钱呢。你打量是什么事情?宝二爷的那块玉呀,我得了准信儿来了。”

未知如何,下回分解。