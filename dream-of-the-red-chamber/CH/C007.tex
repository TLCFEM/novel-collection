\chapter{送宫花贾琏戏熙凤~宴宁府宝玉会秦钟}

话说周瑞家的送了刘老老去后,便上来回王夫人话,谁知王夫人不在上房,问
丫鬟们,方知往薛姨妈那边说话儿去了。周瑞家的听说,便出东角门过东院往梨香
院来。刚至院门前,只见王夫人的丫鬟金钏儿和那一个才留头的小女孩儿站在台阶
儿上玩呢。看见周瑞家的进来,便知有话来回,因往里努嘴儿。

周瑞家的轻轻掀帘进去,见王夫人正和薛姨妈长篇大套的说些家务人情话。周
瑞家的不敢惊动,遂进里间来。只见薛宝钗家常打扮,头上只挽着儿,坐在炕里
边,伏在几上和丫鬟莺儿正在那里描花样子呢。见他进来,便放下笔,转过身,满
面堆笑让:“周姐姐坐。”周瑞家的也忙陪笑问道:“姑娘好?”一面炕沿边坐了,
因说:“这有两三天也没见姑娘到那边逛逛去,只怕是你宝兄弟冲撞了你不成?”
宝钗笑道:“那里的话。只因我那宗病又发了,所以且静养两天。”周瑞家的道:
“正是呢。姑娘到底有什么病根儿?也该趁早请个大夫认真医治医治。小小的年纪
儿倒作下个病根儿,也不是玩的呢。”宝钗听说笑道:“再别提起这个病!也不知
请了多少大夫,吃了多少药,花了多少钱,总不见一点效验儿。后来还亏了一个和
尚,专治无名的病症,因请他看了。他说我这是从胎里带来的一股热毒,幸而我先
天壮还不相干,要是吃凡药是不中用的。他就说了个海上仙方儿,又给了一包末药
作引子,异香异气的。他说犯了时吃一丸就好了。倒也奇怪,这倒效验些。”周瑞
家的因问道:“不知是什么方儿?姑娘说了,我们也好记着说给人知道。要遇见这
样病,也是行好的事”宝钗笑道:“不问这方儿还好,若问这方儿,真把人琐碎死
了!东西药料一概却都有限,最难得是‘可巧’二字:要春天开的白牡丹花蕊十二
两,夏天开的白荷花蕊十二两,秋天的白芙蓉蕊十二两,冬天的白梅花蕊十二两。
将这四样花蕊于次年春分这一天晒干,和在末药一处,一齐研好;又要雨水这日的
天落水十二钱……”周瑞家的笑道:“嗳呀,这么说就得三年的工夫呢。倘或雨水
这日不下雨,可又怎么着呢?”宝钗笑道:“所以了!那里有这么可巧的雨?也只好
再等罢了。还要白露这日的露水十二钱,霜降这日的霜十二钱,小雪这日的雪十二
钱。把这四样水调匀了,丸了龙眼大的丸子,盛在旧磁坛里,埋在花根底下。若发
了病的时候儿,拿出来吃一丸,用一钱二分黄柏煎汤送下。”

周瑞家的听了,笑道:“阿弥陀佛!真巧死了人。等十年还未必碰的全呢!”
宝钗道:“竟好。自他去后,一二年间,可巧都得了,好容易配成一料。如今从家
里带了来,现埋在梨花树底下。”周瑞家的又道:“这药有名字没有呢?”宝钗道:
“有。也是那和尚说的,叫作‘冷香丸’。”周瑞家的听了点头儿,因又说:“这
病发了时,到底怎么着?”宝钗道:“也不觉什么,不过只喘嗽些,吃一丸也就罢
了。”

周瑞家的还要说话时,忽听王夫人问道:“谁在里头?”周瑞家的忙出来答应
了,便回了刘老老之事。略待半刻,见王夫人无话,方欲退出去,薛姨妈忽又笑道:
“你且站住。我有一件东西,你带了去罢。”说着便叫:“香菱!”帘栊响处,才
和金钏儿玩的那个小丫头进来,问:“太太叫我做什么?”薛姨妈道:“把那匣子
里的花儿拿来。”香菱答应了,向那边捧了个小锦匣儿来。薛姨妈道:“这是宫里
头作的新鲜花样儿堆纱花,十二枝。昨儿我想起来,白放着可惜旧了,何不给他们
姐妹们戴去。昨儿要送去,偏又忘了;你今儿来得巧,就带了去罢。你家的三位姑
娘每位两枝,下剩六枝送林姑娘两枝,那四枝给凤姐儿罢。”王夫人道:“留着给
宝丫头戴也罢了,又想着他们。”薛姨妈道:“姨太太不知,宝丫头怪着呢,他从
来不爱这些花儿粉儿的。”

说着,周瑞家的拿了匣子,走出房门。见金钏儿仍在那里晒日阳儿,周瑞家的
问道:“那香菱小丫头子可就是时常说的,临上京时买的、为他打人命官司的那个
小丫头吗?”金钏儿道:“可不就是他。”正说着,只见香菱笑嘻嘻的走来,周瑞
家的便拉了他的手细细的看了一回,因向金钏儿笑道:“这个模样儿,竟有些像咱
们东府里的小蓉奶奶的品格儿。”金钏儿道:“我也这么说呢。”周瑞家的又问香
菱:“你几岁投身到这里?”又问:“你父母在那里呢?今年十几了?本处是那里的
人?”香菱听问,摇头说:“不记得了。”周瑞家的和金钏儿听了,倒反为叹息了
一回。

一时周瑞家的携花至王夫人正房后。原来近日贾母说孙女们太多,一处挤着倒
不便,只留宝玉黛玉二人在这边解闷,却将迎春、探春、惜春三人移到王夫人这边
房后三间抱厦内居住,令李纨陪伴照管。如今周瑞家的故顺路先往这里来,只见几
个小丫头都在抱厦内默坐,听着呼唤。迎春的丫鬟司棋和探春的丫鬟侍书二人,正
掀帘子出来,手里都捧着茶盘茶钟,周瑞家的便知他姐妹在一处坐着,也进入房内。
只见迎春、探春二人正在窗下围棋。周瑞家的将花送上,说明原故,二人忙住了棋,
都欠身道谢,命丫鬟们收了。

周瑞家的答应了,因说:“四姑娘不在房里,只怕在老太太那边呢?”丫鬟们
道:“在那屋里不是?”周瑞家的听了,便往这边屋里来。只见惜春正同水月庵的
小姑子智能儿两个一处玩耍呢,见周瑞家的进来,便问他何事。周瑞家的将花匣打
开,说明原故,惜春笑道:“我这里正和智能儿说,我明儿也要剃了头跟他作姑子
去呢。可巧又送了花来,要剃了头,可把花儿戴在那里呢?”说着,大家取笑一回,
惜春命丫鬟收了。周瑞家的因问智能儿:“你是什么时候来的?你师父那秃歪剌那
里去了?”智能儿道:“我们一早就来了。我师父见过太太,就往于老爷府里去了,
叫我在这里等他呢。”周瑞家的又道:“十五的月例香供银子可得了没有?”智能
儿道:“不知道。”惜春便问周瑞家的:“如今各庙月例银子是谁管着?”周瑞家
的道:“余信管着。”惜春听了笑道:“这就是了。他师父一来了,余信家的就赶
上来,和他师父咕唧了半日,想必就是为这个事了。”

那周瑞家的又和智能儿唠叨了一回,便往凤姐处来。穿过了夹道子,从李纨后
窗下越过西花墙,出西角门,进凤姐院中。走至堂屋,只见小丫头丰儿坐在房门槛
儿上,见周瑞家的来了,连忙的摆手儿,叫他往东屋里去。周瑞家的会意,忙着蹑
手蹑脚儿的往东边屋里来,只见奶子拍着大姐儿睡觉呢。周瑞家的悄悄儿问道:“二
奶奶睡中觉呢吗?也该清醒了。”奶子笑着,撇着嘴摇头儿。正问着,只听那边微
有笑声儿,却是贾琏的声音。接着房门响,平儿拿着大铜盆出来,叫人舀水。平儿
便进这边来,见了周瑞家的,便问:“你老人家又来作什么?”周瑞家的忙起身拿
匣子给他看道:“送花儿来了。”平儿听了,便打开匣子,拿了四枝,抽身去了。
半刻工夫,手里拿出两枝来,先叫彩明来,吩咐:“送到那边府里,给小蓉大奶奶
戴的。”次后方命周瑞家的回去道谢。

周瑞家的这才往贾母这边来,过了穿堂,顶头忽见他的女孩儿打扮着才从他婆
家来。周瑞家的忙问:“你这会子跑来作什么?”他女孩儿说:“妈,一向身上好?
我在家里等了这半日,妈竟不去,什么事情这么忙的不回家?我等烦了,自己先到
了老太太跟前请了安了,这会子请太太的安去。妈还有什么不了的差事?手里是什
么东西?”周瑞家的笑道:“嗳!今儿偏偏来了个刘老老,我自己多事,为他跑了
半日。这会子叫姨太太看见了,叫送这几枝花儿给姑娘奶奶们去,这还没有送完呢。
你今儿来,一定有什么事情。”他女孩儿笑道:“你老人家倒会猜,一猜就猜着了。
实对你老人家说:你女婿因前儿多喝了点子酒,和人分争起来,不知怎么叫人放了
把邪火,说他来历不明,告到衙门里,要递解还乡。所以我来和你老人家商量商量,
讨个情分。不知求那个可以了事?”周瑞家的听了道:“我就知道。这算什么大事,
忙的这么着!你先家去,等我送下林姑娘的花儿就回去。这会儿太太二奶奶都不得
闲儿呢!”他女孩儿听说,便回去了,还说:“妈,好歹快来。”周瑞家的道:“是
了罢!小人儿家没经过什么事,就急的这么个样儿。”说着,便到黛玉房中去了。

谁知此时黛玉不在自己房里,却在宝玉房中,大家解九连环作戏。周瑞家的进
来,笑道:“林姑娘,姨太太叫我送花儿来了。”宝玉听说,便说:“什么花儿?
拿来我瞧瞧。”一面便伸手接过匣子来看时,原来是两枝宫制堆纱新巧的假花。黛
玉只就宝玉手中看了一看,便问道:“还是单送我一个人的,还是别的姑娘们都有
呢?”周瑞家的道:“各位都有了,这两枝是姑娘的。”黛玉冷笑道:“我就知道
么!别人不挑剩下的也不给我呀。”周瑞家的听了,一声儿也不敢言语。宝玉问道:
“周姐姐,你作什么到那边去了?”周瑞家的因说:“太太在那里,我回话去了,
姨太太就顺便叫我带来的。”宝玉道:“宝姐姐在家里作什么呢?怎么这几日也不
过来?”周瑞家的道:“身上不大好呢。”宝玉听了,便和丫头们说:“谁去瞧瞧,
就说我和林姑娘打发来问姨娘姐姐安,问姐姐是什么病,吃什么药。论理,我该亲
自来的,就说才从学里回来,也着了些凉,改日再亲自来看。”说着,茜雪便答应
去了。周瑞家的自去无话。

原来周瑞家的女婿便是雨村的好友冷子兴,近日因卖古董,和人打官司,故叫
女人来讨情。周瑞家的仗着主子的势,把这些事也不放在心上,晚上只求求凤姐便
完了。

至掌灯时,凤姐卸了妆,来见王夫人,回说:“今儿甄家送了来的东西,我已
收了。咱们送他的,趁着他家有年下送鲜的船,交给他带了去了。”王夫人点点头
儿。凤姐又道:“临安伯老太太生日的礼已经打点了,太太派谁送去?”王夫人道:
“你瞧谁闲着,叫四个女人去就完了,又来问我。”凤姐道:“今日珍大嫂子来请
我明日去逛逛,明日有什么事没有?”王夫人道:“有事没事都碍不着什么。每常
他来请,有我们,你自然不便;他不请我们单请你,可知是他的诚心叫你散荡散荡。
别辜负了他的心,倒该过去走走才是。”凤姐答应了。当下李纨探春等姊妹们也都
定省毕,各归房无话。

次日凤姐梳洗了,先回王夫人毕,方来辞贾母。宝玉听了,也要逛去,凤姐只
得答应着。立等换了衣裳,姐儿两个坐了车。一时进入宁府,早有贾珍之妻尤氏与
贾蓉媳妇秦氏,婆媳两个带着多少侍妾丫鬟等接出仪门。那尤氏一见凤姐,必先嘲
笑一阵,一手拉了宝玉,同入上房里坐下。秦氏献了茶。凤姐便说:“你们请我来
作什么?拿什么孝敬我?有东西就献上来罢,我还有事呢!”尤氏未及答应,几个媳
妇们先笑道:“二奶奶今日不来就罢,既来了,就依不得你老人家了。”正说着,
只见贾蓉进来请安。宝玉因道:“大哥哥今儿不在家么?”尤氏道:“今儿出城请
老爷的安去了。”又道:“可是你怪闷的,坐在这里作什么?何不出去逛逛呢?”
秦氏笑道:“今日可巧:上回宝二叔要见我兄弟,今儿他在这里书房里坐着呢,为
什么不瞧瞧去?”宝玉便去要见,尤氏忙吩咐人小心伺候着跟了去。凤姐道:“既
这么着,为什么不请进来我也见见呢?”尤氏笑道:“罢,罢,可以不必见。比不
得咱们家的孩子,胡打海摔的惯了的。人家的孩子都是斯斯文文的,没见过你这样
泼辣货。还叫人家笑话死呢!”凤姐笑道:“我不笑话他就罢了,他敢笑话我?”
贾蓉道:“他生的腼腆,没见过大阵仗儿,婶子见了,没的生气。”凤姐啐道:“呸!
扯臊!他是哪吒我也要见见。别放你娘的屁了!再不带来,打你顿好嘴巴子。”贾蓉
溜湫着眼儿笑道:“何苦婶子又使利害!我们带了来就是了。”凤姐也笑了。

说着出去一会儿,果然带了个后生来:比宝玉略瘦些,眉清目秀,粉面朱唇,
身材俊俏,举止风流,似更在宝玉之上,只是怯怯羞羞有些女儿之态,腼腆含糊的
向凤姐请安问好。凤姐喜的先推宝玉笑道:“比下去了!”便探身一把攥了这孩子
的手,叫他身旁坐下,慢慢问他年纪读书等事,方知他学名叫秦钟。早有凤姐跟的
丫鬟媳妇们,看见凤姐初见秦钟并未备得表礼来,遂忙过那边去告诉平儿。平儿素
知凤姐和秦氏厚密,遂自作主意,拿了一匹尺头,两个“状元及第”的小金锞子,
交付来人送过去。凤姐还说太简薄些。秦氏等谢毕,一时吃过了饭,尤氏、凤姐、
秦氏等抹骨牌,不在话下。

宝玉、秦钟二人随便起坐说话儿。那宝玉自一见秦钟,心中便如有所失,痴了
半日,自己心中又起了个呆想,乃自思道:“天下竟有这等的人物!如今看了,我
竟成了泥猪癞狗了,可恨我为什么生在这侯门公府之家?要也生在寒儒薄宦的家
里,早得和他交接,也不枉生了一世。我虽比他尊贵,但绫锦纱罗,也不过裹了我
这枯株朽木;羊羔美酒,也不过填了我这粪窟泥沟。‘富贵’二字,真真把人荼毒
了。”那秦钟见了宝玉形容出众,举止不凡,更兼金冠绣服,艳婢娇童,——“果
然怨不得姐姐素日提起来就夸不绝口。我偏偏生于清寒之家,怎能和他交接亲厚一
番,也是缘法”。二人一样胡思乱想。宝玉又问他读什么书,秦钟见问,便依实而
答。二人你言我语,十来句话,越觉亲密起来了。一时捧上茶果吃茶,宝玉便说:
“我们两个又不吃酒,把果子摆在里间小炕上,我们那里去,省了闹的你们不安。”
于是二人进里间来吃茶。秦氏一面张罗凤姐吃果酒,一面忙进来嘱咐宝玉道:“宝
二叔:你侄儿年轻,倘或说话不防头,你千万看着我,别理他。他虽腼腆,却脾气
拐孤,不大随和儿。”宝玉笑道:“你去罢,我知道了。”秦氏又嘱咐了他兄弟一
回,方去陪凤姐儿去了。

一时凤姐尤氏又打发人来问宝玉:“要吃什么,只管要去。”宝玉只答应着,
也无心在饮食上,只问秦钟近日家务等事。秦钟因言:“业师于去岁辞馆,家父年
纪老了,残疾在身,公务繁冗,因此尚未议及延师,目下不过在家温习旧课而已。
再读书一事也必须有一二知己为伴,时常大家讨论才能有些进益——”宝玉不待说
完,便道:“正是呢!我们家却有个家塾,合族中有不能延师的便可入塾读书,亲
戚子弟可以附读。我因上年业师回家去了,也现荒废着。家父之意亦欲暂送我去,
且温习着旧书,待明年业师上来,再各自在家读书。家祖母因说:一则家学里子弟
太多,恐怕大家淘气,反不好;二则也因我病了几天,遂暂且耽搁着。如此说来,
尊翁如今也为此事悬心,今日回去,何不禀明,就在我们这敝塾中来?我也相伴,
彼此有益,岂不是好事?”秦钟笑道:“家父前日在家提起延师一事,也曾提起这
里的义学倒好,原要来和这里的老爷商议引荐;因这里又有事忙,不便为这点子小
事来絮聒。二叔果然度量侄儿或可磨墨洗砚,何不速速作成,彼此不致荒废,既可
以常相聚谈,又可以慰父母之心,又可以得朋友之乐,岂不是美事?”宝玉道:“放
心,放心!咱们回来告诉你姐夫姐姐和琏二嫂子,今日你就回家禀明令尊,我回去
禀明了祖母,再无不速成之理。”

二人计议已定,那天气已是掌灯时分,出来又看他们玩了一回牌。算帐时,却
又是秦氏尤氏二人输了戏酒的东道,言定后日吃这东道,一面又吃了晚饭。因天黑
了,尤氏说:“派两个小子送了秦哥儿家去。”媳妇们传出去半日。秦钟告辞起身,
尤氏问:“派谁送去?”媳妇们回说:“外头派了焦大,谁知焦大醉了,又骂呢。”
尤氏秦氏都道:“偏又派他作什么?那个小子派不得?偏又惹他!”凤姐道:“成日
家说你太软弱了,纵的家里人这样,还了得吗?”尤氏道:“你难道不知这焦大的?
连老爷都不理他,你珍大哥哥也不理他。因他从小儿跟着太爷出过三四回兵,从死
人堆里把太爷背出来了,才得了命;自己挨着饿,却偷了东西给主子吃;两日没水,
得了半碗水,给主子喝,他自己喝马溺:不过仗着这些功劳情分,有祖宗时,都另
眼相待,如今谁肯难为他?他自己又老了,又不顾体面,一味的好酒,喝醉了无人
不骂。我常说给管事的,以后不用派他差使,只当他是个死的就完了。今儿又派了
他!”凤姐道:“我何曾不知这焦大?到底是你们没主意,何不远远的打发他到庄
子上去就完了!”说着,因问:“我们的车可齐备了?”众媳妇们说:“伺候齐了。”

凤姐也起身告辞,和宝玉携手同行。尤氏等送至大厅前,见灯火辉煌,众小厮
都在丹墀侍立。那焦大又恃贾珍不在家,因趁着酒兴,先骂大总管赖二,说他:“不
公道,欺软怕硬!有好差使派了别人,这样黑更半夜送人就派我,没良心的忘八羔
子!瞎充管家!你也不想想焦大太爷跷起一只腿,比你的头还高些。二十年头里的焦
大太爷眼里有谁?别说你们这一把子的杂种们!”正骂得兴头上,贾蓉送凤姐的车
出来。众人喝他不住,贾蓉忍不住便骂了几句,叫人:“捆起来!等明日酒醒了,
再问他还寻死不寻死!”那焦大那里有贾蓉在眼里?反大叫起来,赶着贾蓉叫:“蓉
哥儿,你别在焦大跟前使主子性儿!别说你这样儿的,就是你爹、你爷爷,也不敢
和焦大挺腰子呢。不是焦大一个人,你们作官儿,享荣华,受富贵!你祖宗九死一
生挣下这个家业,到如今不报我的恩,反和我充起主子来了。不和我说别的还可;
再说别的,咱们‘白刀子进去,红刀子出来’!”凤姐在车上和贾蓉说:“还不早
些打发了没王法的东西!留在家里,岂不是害?亲友知道,岂不笑话咱们这样的人
家,连个规矩都没有?”贾蓉答应了“是”。

众人见他太撒野,只得上来了几个,揪翻捆倒,拖往马圈里去。焦大益发连贾
珍都说出来,乱嚷乱叫,说:“要往祠堂里哭太爷去,那里承望到如今生下这些畜
生来!每日偷狗戏鸡,爬灰的爬灰,养小叔子的养小叔子,我什么不知道?咱们‘胳
膊折了往袖子里藏’!”众小厮见说出来的话有天没日的,唬得魂飞魄丧,把他捆
起来,用土和马粪满满的填了他一嘴。

凤姐和贾蓉也遥遥的听见了,都装作没听见。宝玉在车上听见,因问凤姐道:
“姐姐,你听他说‘爬灰的爬灰’,这是什么话?”凤姐连忙喝道:“少胡说!那
是醉汉嘴里胡,你是什么样的人,不说没听见,还倒细问!等我回了太太,看是
捶你不捶你!”吓得宝玉连忙央告:“好姐姐,我再不敢说这些话了。”凤姐哄他
道:“好兄弟,这才是呢。等回去咱们回了老太太,打发人到家学里去说明了,请
了秦钟学里念书去要紧。”说着自回荣府而来。

要知端的,下回分解。