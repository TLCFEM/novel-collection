\chapter{幽淑女悲题五美吟~浪荡子情遗九龙佩}

话说贾蓉见家中诸事已妥,连忙赶至寺中,回明贾珍。于是连夜分派各项执事
人役,并预备一切应用幡杠等物,择于初四日卯时请灵柩进城,一面使人知会诸位
亲友。是日丧仪耀,宾客如云,自铁槛寺至宁府,夹路看的何止数万人。内中有
嗟叹的,也有羡慕的,又有一等半瓶醋的读书人,说是丧礼与其奢易莫若俭戚的:
一路纷纷议论不一。至未申时方到,将灵柩停放正堂之内,供奠举哀已毕,亲友渐
次散回,只剩族中人分理迎宾送客等事。近亲只有邢舅太爷相伴未去。贾珍贾蓉此
时为礼法所拘,不免在灵旁籍草枕块,恨苦居丧;人散后,仍乘空在内亲女眷中厮
混。宝玉亦每日在宁府穿孝,至晚人散,方回园里。凤姐身体未愈,虽不能时常在
此,或遇着开坛诵经、亲友上祭之日,亦扎挣过来相帮尤氏料理。

一日供毕早饭,因天气尚长,贾珍等连日劳倦,不免在灵旁假寐。宝玉见无客
至,遂欲回家看视黛玉,因先回至怡红院中。进入门来,只见院中寂静无人,有几
个老婆子和那小丫头们在回廊下取便乘凉,也有睡卧的,也有坐着打盹的。宝玉也
不去惊动。只有四儿看见,连忙上前来打帘子。将掀起时,只见芳官自内带笑跑出,
几乎和宝玉撞个满怀。一见宝玉,方含笑站着,说道:“你怎么来了?你快给我拦
住晴雯,他要打我呢。”一语未了,只听见屋里唏哗喇的乱响,不知是何物撒了
一地。随后晴雯赶来骂道:“我看你这小蹄子儿往那里去?输了不叫打。宝玉不在
家,我看有谁来救你!”宝玉连忙带笑拦住,道:“你妹子小,不知怎么得罪了你,
看我的分上饶他罢。”晴雯也不想宝玉此时回来,乍一见不觉好笑,遂笑说道:“芳
官竟是个狐狸精变的?就是会拘神遣将的符咒也没有这么快。”又笑道:“就是你
真请了神来,我也不怕。”遂夺手仍要捉拿。芳官早已藏在身后,搂着宝玉不放。
宝玉遂一手拉了晴雯,一手携了芳官,进来看时,只见西边炕上麝月、秋纹、碧痕、
春燕等正在那里抓子儿赢瓜子儿呢。却是芳官输给晴雯,芳官不肯叫打,跑出去了,
晴雯因赶芳官,将怀内的子儿撒了一地。宝玉笑道:“如此长天,我不在家里,正
怕你们寂寞,吃了饭睡觉,睡出病来;大家寻件事玩笑消遣甚好。”因不见袭人,
又问道:“你袭人姐姐呢?”晴雯道:“袭人么?越发道学了,独自个在屋里面壁
呢。这好一会我们没进去,不知他做什么呢,一点声儿也听不见。你快瞧瞧去罢,
或者此时参悟了,也不可知。”

宝玉听说,一面笑,一面走至里间。只见袭人坐在近窗床上,手中拿着一根灰
色绦子,正在那里打结子呢,见宝玉进来,连忙站起,笑道:“晴雯这东西编派我
什么呢!我因要赶着打完了这结子,没工夫和他们瞎闹,因哄他说:‘你们玩去罢。
趁着二爷不在家,我要在这里静坐一坐,养一养神。’他就编派了我这些个话,什
么‘面壁了’、‘参禅了’的。等一会我不撕他那嘴!”宝玉笑着挨近袭人坐下,
瞧他打结子,问道:“这么长天,你也该歇息歇息,或和他们玩笑,要不瞧瞧林妹
妹去也好。怪热的打这个,那里使?”袭人道:“我见你带的扇套,还是那年东府
里蓉大奶奶的事情上做的。那个青东西,除族中或亲友家夏天有白事才带的着,一
年遇着带一两遭,平常又不犯做。如今那府里有事,这是要过去天天带的,所以我
赶着另作一个,等打完了结子给你换下那旧的来。你虽然不讲究这个,要叫老太太
回来看见,又该说我们躲懒,连你穿带的东西都不经心了。”宝玉笑道:“这真难
为你想的到。只是也不可过于赶,热着了,倒是大事。”说着,芳官早托了一杯凉
水内新湃的茶来。因宝玉素昔秉赋柔脆,虽暑月不敢用冰,只以新汲井水,将茶连
壶浸在盆内,不时更换,取其凉意而已。宝玉就芳官手内吃了半盏,遂向袭人道:
“我来时,已吩咐了焙茗,要珍大哥那边有要紧的客来时,叫他即刻送信。要没要
紧的事,我就不过去了。”说毕,遂出了房门,又回头向碧痕等道:“要有事,到
林姑娘那里找我。”

于是一径往潇湘馆来看黛玉。将过了沁芳桥,只见雪雁领了两个老婆子,手中
都拿着菱藕瓜果之类。宝玉忙问雪雁道:“你们姑娘从来不吃这些凉东西,拿这些
瓜果作什么?不是要请那位姑娘奶奶么?”雪雁笑道:“我告诉你,可不许你对姑
娘说去。”宝玉点头应允。雪雁便命两个婆子:“先将瓜果送去,交与紫鹃姐姐。
他要问我,你就说我做什么呢,就来。”那婆子答应着去了。雪雁方说道:“我们
姑娘这两日方觉身上好些了。今日饭后,三姑娘来会着要瞧二奶奶去,姑娘也没去,
又不知想起什么来了,自己哭了一回,提笔写了好些不知是诗是词。叫我传瓜果去
时,又听叫紫鹃将屋内摆着的小琴桌上的陈设搬下来,将桌子挪在外间当地,又叫
将那龙文鼎放在桌上,等瓜果来时听用。要说是请人呢,不犯先忙着把个炉摆出来;
要说点香呢,我们姑娘素日屋内除摆新鲜花果木瓜之类,又不大喜熏衣服。就是点
香,也当点在常坐卧的地方儿,难道是老婆子们把屋子熏臭了,要拿香熏熏不成?
究竟连我也不知为什么。二爷白瞧瞧去。”宝玉听了,不由的低头心内细想道:“据
雪雁说,必有原故。要是同那一位姐妹们闲坐,亦不必如此先设馔具。或者是姑爷
姑妈的忌辰?但我记得每年到此日期,老太太都吩咐另外整理肴馔送去林妹妹私
祭,此时已过。大约必是七月,因为瓜果之节,家家都上秋季的坟,林妹妹有感于
心,所以在私室自己奠祭,取《礼记》‘春秋荐其时食’之意,也未可定。但我此
刻走去,见他伤感,必极力劝解,又怕他烦恼郁结于心;若竟不去,又恐他过于伤
感,无人劝止:两件皆足致疾。莫若先到凤姐姐处一看,到彼稍坐即回。如若见林
妹妹伤感,再设法开解。既不至使其过悲,哀痛稍申,亦不至抑郁致病。”

想毕,遂别了雪雁,出了园门,一径到凤姐处来。正有许多婆子们回事毕,纷
纷散出,凤姐倚着门和平儿说话呢。一见了宝玉,笑道:“你回来了么?我才吩咐
了林之孝家的,叫他使人告诉跟你的小厮,若没什么事,趁便请你回来歇息歇息。
再者那里人多,你那里禁的住那些气味?不想恰好你倒来了。”宝玉笑道:“多谢
姐姐惦记。我也因今日没事,又见姐姐这两日没往那府里去,不知身上可大愈了,
所以回来看看。”凤姐道:“左右也不过是这么着,三日好两日不好的。老太太、
太太不在家,这些大娘们,嗳!那一个是安分的?每日不是打架,就是拌嘴,连赌博
偷盗的事情都闹出来了两三件了。虽说有三姑娘帮着办理,他又是个没出阁的姑
娘,也有叫他知道得的,也有往他说不得的事,也只好强扎挣着罢了。总不得心静
一会儿!别说想病好,求其不添,也就罢了。”宝玉道:“姐姐虽如此说,姐姐还
要保重身体,少操些心才是。”说毕,又说了些闲话,别了凤姐,回身往园中走来。

进了潇湘馆院门看时,只见炉袅残烟,奠馀玉醴,紫鹃正看着人往里收桌子,
搬陈设呢。宝玉便知已经奠祭完了。走入屋内,只见黛玉面向里歪着,病体恹恹,
大有不胜之态。紫鹃连忙说道:“宝二爷来了。”黛玉方慢慢的起来。含笑让坐。
宝玉道:“妹妹这两天可大好些了?气色倒觉静些,只是为何又伤心了?”黛玉道:
“可是你没的说了。好好的,我多早晚又伤心了?”宝玉笑道:“妹妹脸上现有泪
痕,如何还哄我呢?只是我想妹妹素日本来多病,凡事当各自宽解,不可过作无益
之悲。若作践坏了身子,使我——”刚说到这里,觉得以下的话有些难说,连忙咽
住。只因他虽和黛玉一处长大,情投意合,又愿同生同死,却只心中领会,从来未
曾当面说出。况兼黛玉心多,每每说话造次,得罪了他。今日原为的是来劝解,不
想把话又说造次了,接不下去。心中一急,又怕黛玉恼他,又想一想自己的心,实
在的是为好,因而转念为悲,反倒掉下泪来。黛玉起先原恼宝玉说话不论轻重,如
今见此光景,心有所感,本来素昔爱哭,此时亦不免无言对泣。

却说紫鹃端了茶来,打量二人又为何事口角,因说道:“姑娘身上才好些,宝
二爷又来怄气了。到底是怎么样?”宝玉一面拭泪,笑道:“谁敢怄妹妹了?”一
面搭讪着起来闲步,只见砚台底下微露一纸角,不禁伸手拿起。黛玉忙要起身来夺,
已被宝玉揣在怀内,笑央道:“好妹妹,赏我看看罢!”黛玉道:“不管什么,来
了就混翻。”一语未了,只见宝钗走来,笑道:“宝兄弟要看什么?”宝玉因未见
上面是何言词,又不知黛玉心中如何,未敢造次回答,却望着黛玉笑。黛玉一面让
宝钗坐,一面笑道:“我曾见古史中有才色的女子,终身遭际,令人可欣可羡、可
悲可叹者甚多,今日饭后无事,因欲择出数人,胡乱凑几首诗,以寄感慨。可巧探
丫头来会我瞧凤姐姐去,我也身上懒懒的,没同他去。将才做了五首,一时困倦起
来,撂在那里,不想二爷来了,就瞧见了。其实给他看也没有什么,但只我嫌他是
不是的写给人看去。”宝玉忙道:“我多早晚给人看来?昨日那把扇子,原是我爱
那几首《白海棠》诗,所以我自己用小楷写了,不过为的是拿在手中看着便易。我
岂不知闺阁中诗词字迹是轻易往外传诵不得的?自从你说了我,总没拿出园子去。”
宝钗道:“林妹妹这虑的也是。你既写在扇子上,偶然忘记了,拿在书房里去,被
相公们看见了,岂有不问是谁做的呢?倘或传扬开了,反为不美。自古道‘女子无
才便是德’,总以贞静为主,女工还是第二件。其馀诗词,不过是闺中游戏,原可
以会可以不会,咱们这样人家的姑娘,倒不要这些才华的名誉。”因又笑向黛玉道:
“拿出来给我看看无妨,只不叫宝兄弟拿出去就是了。”黛玉笑道:“既如此说,
连你也可以不必看了。”又指着宝玉笑道:“他早已抢了去了。”

宝玉听了,方自怀内取出,凑至宝钗身旁,一同细看。只见写道:

西
施
一代倾城逐浪花,吴宫空自忆儿家。
效颦莫笑东村女,头白溪边尚浣纱。

虞
姬
肠断乌啼夜啸风,虞兮幽恨对重瞳。
黥彭甘受他年醢,饮剑何如楚帐中?

明
妃
绝艳惊人出汉宫,红颜命薄古今同。
君王纵使轻颜色,予夺权何畀画工?

绿
珠
瓦砾明珠一例抛,何曾石尉重娇娆?
都缘顽福前生造,更有同归慰寂寥。

红
拂
长剑雄谈态自殊,美人巨眼识穷途。
尸居馀气杨公幕,岂得羁縻女丈夫?
宝玉看了,赞不绝口,又说道:“妹妹这诗,恰好只做了五首,何不就命曰《五美
吟》?”于是不容分说,便提笔写在后面。宝钗亦说道:“做诗不论何题,只要善
翻古人之意。若要随人脚踪走去,纵使字句精工,已落第二义,究竟算不得好诗。
即如前人所咏昭君之诗甚多,有悲挽昭君的,有怨恨延寿的,又有讥汉帝不能使画
工图貌贤臣而画美人的,纷纷不一。后来王荆公复有‘意态由来画不成,当时枉杀
毛延寿’,永叔有‘耳目所见尚如此,万里安能制夷狄’:二诗俱能各出己见,不
与人同。今日林妹妹这五首诗,亦可谓命意新奇,别开生面了。”

仍欲往下说时,只见有人回道:“琏二爷回来了。适才外头传说,往东府里去
了,好一会了,想必就回来的。”宝玉听了,连忙起身,迎至大门以内等待,恰好
贾琏自外下马进来。于是宝玉先迎着贾琏打千儿,口中给贾母王夫人等请了安,又
给贾琏请了安。二人携手走进来。只见李纨、凤姐、宝钗、黛玉、迎、探、惜等早
在中堂等候,一一相见已毕。因听贾琏说道:“老太太明日一早到家,一路身体甚
好。今日先打发了我来,回家看视,明日五更,仍要出城迎接。”说毕,众人又问
了些路途的景况。因贾琏是远归,遂大家别过,让贾琏回房歇息。一宿晚景,不必
细述。

至次日饭时前后,果见贾母王夫人等到来。众人接见已毕,略坐了一坐,吃了
一杯茶,便领了王夫人等人过宁府中来。只听见里面哭声震天,却是贾赦贾琏送贾
母到家,即过这边来了。当下贾母进入里面,早有贾赦贾琏率领族中人哭着迎出来
了。他父子一边一个,挽了贾母,走至灵前,又有贾珍贾蓉跪着,扑入贾母怀中痛
哭。贾母暮年人,见此光景,亦搂了珍蓉等痛哭不已。贾赦贾琏在旁苦劝,方略略
止住。又转至灵右,见了尤氏婆媳,不免又相持大痛一场。哭毕,众人方上前,一
一请安问好。贾琏因贾母才回家来,未得歇息,坐在此间看着未免要伤心,遂再三
的劝。贾母不得已,方回来了。果然年迈的人,禁不住风霜伤感,至夜间便觉头闷
心酸,鼻塞声重,连忙请了医生来诊脉下药,足足的忙乱了半夜一日。幸而发散的
快,未曾传经,至三更天,些须发了点汗,脉静身凉,大家方放了心。至次日,仍
服药调理。

又过了数日,乃贾敬送殡之期,贾母犹未大愈,遂留宝玉在家侍奉。凤姐因未
曾甚好,亦未去。其余贾赦、贾琏、邢夫人、王夫人等,率领家人仆妇,都送至铁
槛寺,至晚方回。贾珍尤氏并贾蓉仍在寺中守灵,等过百日后,方扶柩回籍。家中
仍托尤老娘并二姐儿三姐儿照管。

却说贾琏素日既闻尤氏姐妹之名,恨无缘得见,近因贾敬停灵在家,每日与二
姐儿三姐儿相认已熟,不禁动了垂涎之意。况知与贾珍贾容素日有聚之诮,因而
乘机百般撩拨,眉目传情。那三姐儿却只是淡淡相对,只有二姐儿也十分有意,但
只是眼目众多,无从下手。贾琏又怕贾珍吃醋,不敢轻动,只好二人心领神会而已。
此时出殡以后,贾珍家下人少,除尤老娘带领二姐儿三姐儿并几个粗使的丫鬟老婆
子在正室居住外,其余婢妾都随在寺中。外面仆妇,不过晚间巡更,日间看守门户,
白日无事,亦不进里面去。所以贾琏便欲趁此时下手,遂托相伴贾珍为名,亦在寺
中住宿。又时常借着替贾珍料理家务,不时至宁府中来勾搭二姐儿。

一日有小管家俞禄来回贾珍道:“前者所用棚杠孝布并请杠人青衣,共使银一
千一百十两,除给银五百两外,仍欠六百零十两。昨日两处买卖人俱来催讨,奴才
特来讨爷的示下。”贾珍道:“你先往库上领去就是了,这又何必来回我。”俞禄
道:“昨日已曾上库上去领,但只是老爷宾天以后,各处支领甚多,所剩还要预备
百日道场及庙中用度,此时竟不能发给。所以奴才今日特来回爷,或者爷内库里暂
且发给,或者挪借何项,吩咐了奴才好办。”贾珍笑道:“你还当是先呢,有银子
放着不使。你无论那里借了给他罢。”俞禄笑回道:“若说一二百,奴才还可巴结,
这五六百,奴才一时那里办得来?”贾珍想了一回,向贾蓉道:“你问你娘去,昨
日出殡以后,有江南甄家送来吊祭银五百两,未曾交到库上去。家里再找找,凑齐
了,给他去罢。”贾蓉答应了,连忙过这边来,回了尤氏,复转来回他父亲道:“昨
日那项银子已使了二百两,下剩的三百两,令人送至家中,交给老娘收了。”贾珍
道:“既然如此,你就带了他去,合你老娘要出来,交给他。再者也瞧瞧家中有事
无事,问你两个姨娘好。下剩的,俞禄先借了添上罢。”贾蓉和俞禄答应了。

方欲退出,只见贾琏走进来了。俞禄忙上前请了安。贾琏便问何事,贾珍一一
告诉了。贾琏心中想道:“趁此机会,正可至宁府寻二姐儿。”一面遂说道:“这
有多大事,何必向人借去?昨日我方得了一项银子,还没有使呢,莫若给他添上,
岂不省事?”贾珍道:“如此甚好,你就吩咐蓉儿,一并叫他取去。”贾琏忙道:
“这个必得我亲身取去。再我这几日没回家了,还要给老太太、老爷、太太们请请
安去;到大哥那边查查家人们有无生事,再也给亲家太太请请安。”贾珍笑道:“只
是又劳动你,我心里倒不安。”贾琏也笑道:“自家兄弟,这有何妨呢。”贾珍又
吩咐贾蓉道:“你跟了你叔叔去,也到那边给老太太、老爷、太太们请安,说我和
你娘都请安。打听打听老太太身上可大安了,还服药呢没有。”贾蓉一一答应了,
跟随贾琏出来,带了几个小厮,骑上马,一同进城。在路叔侄闲话,贾琏有心,便
提到尤二姐,因夸说如何标致,如何做人好,“举止大方,言语温柔,无一处不令
人可敬可爱。人人都说你婶子好,据我看,那里及你二姨儿一零儿呢?”贾蓉揣知
其意,便笑道:“叔叔既这么爱他,我给叔叔作媒,说了做二房何如?”贾琏笑道:
“你这是玩话,这是正经话?”贾蓉道:“我说的是当真的话。”贾琏又笑道:“敢
自好,只是怕你婶子不依;再也怕你老娘不愿意。况且我听见说你二姨儿已有了人
家了。”贾蓉道:“这都无妨。我二姨儿三姨儿,都不是我老爷养的,原是我老娘
带了来的。听见说,我老娘在那一家时,就把我二姨儿许给皇粮庄头张家,指腹为
婚。后来张家遭了官司败落了,我老娘又自那家嫁了出来。如今这十数年两家音信
不通,我老娘时常报怨,要给他家退婚。我父亲也要将姨儿转聘,只等有了好人家,
不过令人找着张家,给他十几两银子,写上一张退婚的字儿。想张家穷极了的人,
见了银子,有什么不依的?再他也知道咱们这样的人家,也不怕他不依。又是叔叔
这样人说了做二房,我管保我老娘和我父亲都愿意。倒只是婶子那里却难。”

贾琏听到这里,心花都开了,那里还有什么话说?只是一味呆笑而已。贾蓉又
想了一想,笑道:“叔叔要有胆量,依我的主意,管保无妨,不过多花几个钱。”
贾琏忙道:“好孩子,你有什么主意,只管说给我听听。”贾蓉道:“叔叔回家,
一点声色也别露。等我回明了我父亲,向我老娘说妥,然后在咱们府后方近左右,
买上一所房子及应用家伙,再拨两拨子家人过去服侍,择了日子,人不知鬼不觉娶
了过去。嘱咐家人不许走漏风声,婶子在里面住着,深宅大院,那里就得知道了?
叔叔两下里住着,过个一年半载,即或闹出来,不过挨上老爷一顿骂。叔叔只说婶
子总不生育,原是为子嗣起见,所以私自在外面作成此事。就是婶子,见生米做成
熟饭,也只得罢了。再求一求老太太,没有不完的事。”自古道欲令智昏,贾琏只
顾贪图二姐美色,听了贾蓉一篇话,遂为计出万全,将现今身上有服,并停妻再娶,
严父妒妻,种种不妥之处,皆置之度外了。却不知贾蓉亦非好意:素日因同他姨娘
有情,只因贾珍在内,不能畅意,如今要是贾琏娶了,少不得在外居住,趁贾琏不
在时好去鬼混之意。贾琏那里思想及此?遂向贾蓉致谢道:“好侄儿!你果然能够说
成了,我买两个绝色的丫头谢你。”

说着,已至宁府门首,贾蓉说道:“叔叔进去向我老娘要出银子来,就交给俞
禄罢。我先给老太太请安去。”贾琏含笑点头道:“老太太跟前,别说我和你一同
来的。”贾蓉说:“知道。”又附耳向贾琏道:“今儿要遇见二姨儿,可别性急了,
闹出事来,往后倒难办了。”贾琏笑道:“少胡说。你快去罢。我在这里等你。”
于是贾蓉自去给贾母请安。

贾琏进入宁府,早有家人头儿率领家人等请安,一路围随至厅上。贾琏一一的
问了些话,不过塞责而已,便命家人散去,独自往里面走来。原来贾琏贾珍素日亲
密,又是兄弟,本无可避忌之人,自来是不等通报的。于是走至上屋,早有廊下伺
候的老婆子打起帘子让贾琏进去。贾琏进入房中一看,只见南边炕上只有尤二姐带
着两个丫鬟一处做活,却不见尤老娘与三姐儿。贾琏忙上前问好相见。尤二姐含笑
让坐,便靠东边排插儿坐下。贾琏仍将上首让与二姐儿,说了几句见面情儿,便笑
问道:“亲家太太合三妹妹那里去了?怎么不见?”二姐笑道:“才有事往后头去
了,也就来的。”此时伺候的丫鬟因倒茶去,无人在跟前,贾琏不住的拿眼瞟看二
姐儿。二姐儿低了头,只含笑不理。贾琏又不敢造次动手动脚的,因见二姐儿手里
拿着一条拴着荷包的绢子摆弄,便搭讪着,往腰里摸了摸,说道:“槟榔荷包也忘
记带了来,妹妹有槟榔,赏我一口吃。”二姐道:“槟榔倒有,就只是我的槟榔从
来不给人吃。”贾琏便笑着欲近身来拿。二姐儿怕有人来看见不雅,便连忙一笑,
撂了过来。贾琏接在手里,都倒了出来,拣了半块吃剩下的撂在口里吃了,又将剩
下的都揣了起来。刚要把荷包亲身送过去,只见两个丫鬟倒了茶来。贾琏一面接了
茶吃茶,一面暗将自己带的一个汉玉九龙佩解了下来,拴在手绢上,趁丫鬟回头时,
仍撂了过去。二姐儿亦不去拿,只装看不见,坐着吃茶。

只听后面一阵帘子响,却是尤老娘三姐儿带着两个小丫鬟自后面走来。贾琏送
目与二姐儿,令其拾取,这二姐亦只是不理。贾琏不知二姐儿何意思,甚实着急,
只得迎上来与尤老娘三姐儿相见。一面又回头看二姐儿时,只见二姐儿笑着,没事
人似的;再又看一看,绢子已不知那里去了。贾琏方放了心。于是大家归坐后叙了
些闲话。贾琏说道:“大嫂子说,前儿有了包银子交给亲家太太收起来了,今儿因
要还人,大哥令我来取,再也看看家里有事无事。”尤老娘听了,连忙使二姐儿拿
钥匙去取银子。这里贾琏又说道:“我也要给亲家太太请请安,瞧瞧二位妹妹。亲
家太太脸面倒好,只是二位妹妹在我们家里受委屈。”尤老娘笑道:“咱们都是至
亲骨肉,说那里的话?在家里也是住着,在这里也是住着。不瞒二爷说:我们家里,
自从先夫去世,家计也着实艰难了,全亏了这里姑爷帮助着。如今姑爷家里有了这
样大事,我们不能别的出力,白看一看家,还有什么委屈了的呢?”正说着,二姐
儿已取了银子来,交给尤老娘,老娘便递给贾琏。贾琏叫一个小丫头叫了一个老婆
子来,吩咐他道:“你把这个交给俞禄,叫他拿过那边去等我。”老婆子答应了出
去。

只听得院内是贾蓉的声音说话。须臾进来,给他老娘姨娘请了安,又向贾琏笑
道:“才刚老爷还问叔叔呢,说是有什么事情要使唤,原要使人到庙里去叫。我回
老爷说,‘叔叔就来’。老爷还吩咐我,路上遇着叔叔,叫快去呢。”贾琏听了,
忙要起身。又听贾蓉和他老娘说道:“那一次我和老太太说的,我父亲要给二姨儿
说的姨父,就和我这叔叔的面貌身量差不多儿。老太太说好不好?”一面说着,又
悄悄的用手指着贾琏,和他二姨儿努嘴。二姐儿倒不好意思说什么,只见三姐儿似
笑非笑、似恼非恼的骂道:“坏透了的小猴儿崽子,没了你娘的说了!多早晚我才
撕他那嘴呢!”贾蓉早笑着跑了出去,贾琏也笑着辞了出来。走至厅上,又吩咐了
家人们,不可耍钱吃酒等话。又悄悄的央贾蓉,回去急速和他父亲说。一面便带了
俞禄过来,将银子添足,交给他拿去。一面给贾赦请安,又给贾母去请安,不提。

却说贾蓉见俞禄跟了贾琏去取银子,自己无事,便仍回至里面,和他两个姨娘
嘲戏一回,方起身。至晚到寺,见了贾珍,回道:“银子已竟交给俞禄了。老太太
已大愈了,如今已经不服药了。”说毕,又趁便将路上贾琏要娶尤二姐做二房之意
说了,又说如何在外面置房子住,不给凤姐知道,“此时总不过为的是子嗣艰难起
见,为的是二姨儿是见过的,亲上做亲,比别处不知道的人家说了来的好。所以二
叔再三央我对父亲说。”只不说是他自己的主意。贾珍想一想,笑道:“其实倒也
罢了,只不知你二姨娘心里愿意不愿意。明儿你先去和你老娘商量,叫你老娘问准
了你二姨娘,再作定夺。”于是又教了贾蓉一篇话,便走过来将此事告诉了尤氏。
尤氏却知此事不妥,因而极力劝止。无奈贾珍主意已定,素日又是顺从惯了的,况
且他与二姐儿本非一母,不便深管,因而也只得由他们闹去了。

至次日一早,果然贾蓉复进城来见他老娘,将他父亲之意说了。又添上许多话,
说贾琏做人如何好,目今凤姐身子有病,已是不能好的了,暂且买了房子,在外面
住着,过个一年半载,只等凤姐一死,便接了二姨儿进去做正室。又说他父亲此时
如何聘,贾琏那边如何娶,如何“接了你老人家养老,往后三姨儿也是那边应了替
聘”,说得天花乱坠,不由的尤老娘不肯。况且素日全亏贾珍周济,此时又是贾珍
作主替聘,而且妆奁不用自己置买,贾琏又是青年公子,强胜张家,遂忙过来与二
姐儿商议。二姐儿又是水性人儿,在先已和姐夫不妥,又常怨恨当时错许张华,致
使后来终身失所。今见贾琏有情,况是姐夫将他聘嫁,有何不肯?也便点头依允。
当下回复了。

贾蓉回了他父亲,次日命人请了贾琏到寺中来,贾珍当面告诉了他尤老娘应允
之事。贾琏自是喜出望外,感谢贾珍贾蓉父子不尽。于是二人商量着,使人看房子,
打首饰,给二姐儿置买妆奁及新房中应用床帐等物。不过几日,早将诸事办妥,已
于宁荣街后二里远近小花枝巷内买定一所房子,共二十余间,又买了两个小丫鬟。
只是府里家人不敢擅动,外头买人又怕不知心腹,走漏了风声。忽然想起家人鲍二
来,当初因和他女人偷情,被凤姐儿打闹了一阵,含羞吊死了,贾琏给了一百银子,
叫他另娶一个。那鲍二向来却就合厨子多浑虫的媳妇多姑娘有一手儿,后来多浑虫
酒痨死了,这多姑娘儿见鲍二手里从容了,便嫁了鲍二。况且这多姑娘儿原也和贾
琏好的,此时都搬出外头住着。贾琏一时想起来,便叫了他两口儿到新房子里来,
预备二姐儿过来时伏侍。那鲍二两口子听见这个巧宗儿,如何不来呢。

再说张华之祖,原当皇粮庄头,后来死去,至张华父亲时,仍充此役。因与尤
老娘前夫相好,所以将张华与尤二姐指腹为婚。后来不料遭了官司,败落了家产,
弄得衣食不周,那里还娶的起媳妇呢?尤老娘又自那家嫁了出来,两家有十数年音
信不通。今被贾府家人唤至,逼他与二姐儿退婚,心中虽不愿意,无奈惧怕贾珍等
势焰,不敢不依,只得写了一张退婚文约。尤老娘给了二十两银子,两家退亲不提。
这里贾琏等见诸事已妥,遂择了初三黄道吉日,以便迎娶二姐儿过门。

下回分解。