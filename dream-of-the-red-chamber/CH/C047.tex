\chapter{呆霸王调情遭苦打~冷郎君惧祸走他乡}

话说王夫人听见邢夫人来了,连忙迎着出去。邢夫人犹不知贾母已知鸳鸯之
事,正还又来打听信息,进了院门,早有几个婆子悄悄的回了他,他才知道。待要
回去,里面已知;又见王夫人接出来了,少不得进来。先与贾母请安,贾母一声儿
不言语,自己也觉得愧悔。凤姐儿早指一事回避了。鸳鸯也自回房去生气。薛姨妈
王夫人等恐碍着邢夫人的脸面,也都渐渐退了。邢夫人且不敢出去。贾母见无人,
方说道:“我听见你替你老爷说媒来了。你倒也‘三从四德’的,只是这贤惠也太
过了!你们如今也是孙子儿子满眼了,你还怕他使性子。我听见你还由着你老爷的
那性子闹。”邢夫人满面通红,回道:“我劝过几次不依。老太太还有什么不知道
的呢?我也是不得已儿。”贾母道:“他逼着你杀人,你也杀去?如今你也想想:你
兄弟媳妇,本来老实,又生的多病多痛,上上下下,那不是他操心?你一个媳妇,
虽然帮着,也是天天‘丢下耙儿弄扫帚’。凡百事情,我如今自己减了。他们两个
就有些不到的去处,有鸳鸯那孩子还心细些,我的事情,他还想着一点子:该要的,
他就要了来;该添什么,他就趁空儿告诉他们添了。鸳鸯再不这么着,娘儿两个,
里头外头大的小的,那里不忽略一件半件?我如今反倒自己操心去不成?还是天天盘
算和他们要东要西去?我这屋里有的没有的剩了他一个,年纪也大些,我凡做事的
脾气性格儿,他还知道些。他二则也还投主子的缘法,他也并不指着我和那位太太
要衣裳去,又和那位奶奶要银子去。所以这几年,一应事情,他说什么,从你小婶
和你媳妇起,至家下大大小小,没有不信的。所以不单我得靠,连你小婶、媳妇也
都省心。我有了这么个人,就是媳妇、孙子媳妇想不到的,我也不得缺了,也没气
可生了。这会子他去了,你们又弄什么人来我使?你们就弄他那么个真珠儿似的人
来,不会说话也无用。我正要打发人和你老爷说去,他要什么人,我这里有钱,叫
他只管一万八千的买去就是,要这个丫头,不能!留下他伏侍我几年,就和他日夜
伏侍我尽了孝的一样。你来的也巧,就去说,更妥当了。”

说毕,命人来:“请了姨太太你姑娘们来。才高兴说个话儿,怎么又都散了!”
丫头忙答应找去了。众人赶忙的又来。只有薛姨妈向那丫鬟道:“我才来了,又做
什么去?你就说我睡了。”那丫头道:“好亲亲的姨太太,姨祖宗!我们老太太生气
呢。你老人家不去,没个开交了。只当疼我们罢!你老人家怕走,我背了你老人家
去。”薛姨妈笑道:“小鬼头儿!你怕什么?不过骂几句就完了。”说着,只得和这
小丫头子走来。贾母忙让坐,又笑道:“咱们斗牌罢?姨太太的牌也生了,咱们一
处坐着,别叫凤丫头混了我们去。”薛姨妈笑道:“正是呢,老太太替我看着些儿。
就是咱们娘儿四个斗呢,还是添一两个人呢?”王夫人笑道:“可不只四个人?”
凤姐儿道:“再添一个人,热闹些。”贾母道:“叫鸳鸯来,叫他在这下手里坐着。
姨太太的眼花了,咱们两个的牌,都叫他看着些儿。”凤姐笑了一声,向探春道:
“你们知书识字的,倒不学算命?”探春道:“这又奇了,这会子你不打点精神赢
老太太几个钱,又想算命?”凤姐儿道:“我正要算算今儿该输多少。我还想赢呢?
你瞧瞧,场儿没上,左右都埋伏下了。”说的贾母薛姨妈都笑起来。

一时鸳鸯来了,便坐在贾母下首。鸳鸯之下,便是凤姐儿。铺下红毡,洗牌告
么,五人起牌,斗了一回。鸳鸯见贾母的牌已十成,只等一张二饼,便递了暗号儿
与凤姐儿。凤姐儿正该发牌,便故意踌躇了半晌,笑道:“我这一张牌定在姨妈手
里扣着呢,我若不发这一张牌,再顶不下来的。”薛姨妈道:“我手里并没有你的
牌。”凤姐儿道:“我回来是要查的。”薛姨妈道:“你只管查。你且发下来,我
瞧瞧是张什么。”凤姐儿便送在薛姨妈跟前,薛姨妈一看,是个二饼,便笑道:“我
倒不稀罕他,只怕老太太满了。”凤姐听了,忙笑道:“我发错了!”贾母笑的已
掷下牌来,说:“你敢拿回去!谁叫你错的不成?”凤姐儿道:“可是我要算一算
命呢。这是自己发的,也怨不得人了。”贾母笑道:“可是你自己打着你那嘴,问
着你自己才是。”又向薛姨妈笑道:“我不是小气爱赢钱,原是个彩头儿。”薛姨
妈笑道:“我们可不是这样想?那里有那样糊涂人,说老太太爱钱呢?”凤姐儿正
数着钱,听了这话,忙又把钱穿上了,向众人笑道:“够了我的了!竟不为赢钱,
单为赢彩头儿。我到底小气,输了就数钱,快收起来罢。”贾母规矩是鸳鸯代洗牌
的,便和薛姨妈说笑。不见鸳鸯动手。贾母道:“你怎么恼了,连牌也不替我洗?”
鸳鸯拿起牌来笑道:“奶奶不给钱么!”贾母道:“他不给钱,那是他交运了!”
便命小丫头子:“把他那一吊钱都拿过来!”小丫头子真就拿了,搁在贾母傍边。
凤姐儿笑道:“赏我罢,照数儿给就是了。”薛姨妈笑道:“果然凤姐儿小气,不
过玩儿罢了。”凤姐儿听说便站起来拉住薛姨妈,回头指着贾母素日放钱的一个木
箱子笑道:“姑妈瞧瞧,那个里头不知玩了我多少去了。这一吊钱玩不了半个时辰,
那里头的钱就招手儿叫他了。只等把这一吊也叫进去了,牌也不用斗了,老祖宗气
也平了,又有正经事差我办去了。”话未说完,引的贾母众人笑个不住。正说着,
偏平儿怕钱不够,又送了一吊来。凤姐儿道:“不用放在我跟前,也放在老太太的
那一处罢。一齐叫进去倒省事,不用做两次,叫箱子里的钱费事。”贾母笑的手里
的牌撒了一桌子,推着鸳鸯,叫:“快撕他的嘴!”

平儿依言放下钱,也笑了一回,方回来。至院门前,遇见贾琏,问他:“太太
在那里呢?老爷叫我请过去呢。”平儿忙笑道:“在老太太跟前站了这半日,还没
动呢。趁早儿丢开手罢。老太太生了半日气,这会子亏二奶奶凑了半日的趣儿,才
略好了些。”贾琏道:“我过去,只说讨老太太示下,十四往赖大家去不去,好预
备轿子。又请了太太,又凑了趣儿,岂不好呢。”平儿笑道:“依我说,你竟别过
去罢。合家子连太太宝玉都有了不是,这会子你又填限去了。”贾琏道:“已经完
了,难道还找补不成?况且与我又无干。二则老爷亲自吩咐我请太太去,这会子我
打发了人去,倘或知道了,正没好气呢,指着这个拿我出气罢。”说着就走。平儿
见他说的有理,也就跟了贾琏过来。到了堂屋里,便把脚步放轻了,往里间探头,
只见邢夫人站在那里。凤姐儿眼尖,先瞧见了,便使眼色儿,不命他进来,又使眼
色与邢夫人。邢夫人不便就走,只得倒了一碗茶来,放在贾母跟前。贾母一回身,
贾琏不防,便没躲过。贾母便问:“外头是谁?倒像个小子一伸头的似的。”凤姐
儿忙起身说:“我也恍惚看见有一个人影儿。”一面说,一面起身出来。贾琏忙进
去,陪笑道:“打听老太太十四可出门?好预备轿子。”贾母道:“既这么样,怎
么不进来,又做神做鬼的?”贾琏陪笑道:“见老太太玩牌,不敢惊动,不过叫媳
妇出来问问。”贾母道:“就忙到这一时!等他家去,你问他多少问不得?那一遭儿
你这么小心来?这又不知是来做耳报神的,也不知是来做探子的,鬼鬼祟祟,倒吓
我一跳。什么好下流种子!你媳妇和我玩牌呢,还有半日的空儿,你家去再和那赵
二家的商量治你媳妇去罢!”说着众人都笑了。鸳鸯笑道:“鲍二家的,老祖宗又
拉上赵二家的去。”贾母也笑道:“可不?我那里记得什么‘抱’着‘背’着的。
提起这些事来,不由我不生气。我进了这门子做重孙媳妇起,到如今我也有个重孙
子媳妇了,连头带尾五十四年,凭着大惊大险、千奇百怪的事也经了些,从没经过
这些事。还不离了我这里呢!”

贾琏一声儿不敢说,忙退出来。平儿在窗外站着,悄悄的笑道:“我说你不听,
到底碰在网里了。”正说着,只见邢夫人也出来。贾琏道:“都是老爷闹的,如今
都搁在我和太太身上。”邢夫人道:“我把你这没孝心的种子!人家还替老子死呢。
白说了几句,你就抱怨天、抱怨地了。你还不好好的呢!这几日生气,仔细他捶你。”
贾琏道:“太太快过去罢,叫我来请了好半日了。”说着,送他母亲出来过那边去。

邢夫人将方才的话只略说了几句,贾赦无法,又且含愧,自此便告了病,且不
敢见贾母,只打发邢夫人及贾琏每日过去请安。只得又各处遣人购求寻觅,终久费
了五百两银子买了一个十七岁女孩子来,名唤嫣红,收在屋里,不在话下。这里斗
了半日牌,吃晚饭才罢。此一二日间无话。

转眼到了十四,黑早,赖大的媳妇又进来请。贾母高兴,便带了王夫人薛姨妈
及宝玉姐妹等至赖大花园中,坐了半日。那花园虽不及大观园,却也十分齐整宽阔,
泉石林木,楼台亭轩,也有好几处动人的。外面大厅上,薛蟠、贾珍、贾琏、贾蓉
并几个近族的都来了。那赖大家内,也请了几个现任的官长并几个大家子弟作陪。
因其中有个柳湘莲,薛蟠自上次会过一次,已念念不忘。又打听他最喜串戏,且都
串的是生旦风月戏文,不免错会了意,误认他做了风月子弟,正要与他相交,恨没
有个引进,这一天可巧遇见,乐得无可不可。且贾珍等也慕他的名,酒盖住了脸,
就求他串了两出戏。下来,移席和他一处坐着,问长问短,说东说西。那柳湘莲原
系世家子弟,读书不成,父母早丧,素性爽侠,不拘细事,酷好耍枪舞剑,赌博吃
酒,以至眠花卧柳,吹笛弹筝,无所不为。因他年纪又轻,生得又美,不知他身分
的人,都误认作优伶一类。那赖大之子赖尚荣与他素昔交好,故今儿请来做陪。不
想酒后别人犹可,独薛蟠又犯了旧病。心中早已不快,得便意欲走开完事。无奈赖
尚荣又说:“方才宝二爷又嘱咐我:才一进门,虽见了,只是人多不好说话,叫我
嘱咐你散的时候别走,他还有话说呢。你既一定要去,等我叫出他来,你两个见了
再走,与我无干。”说着,便命小厮们:“到里头,找一个老婆子,悄悄告诉,请
出宝二爷来。”那小厮去了。

没一杯茶时候,果见宝玉出来了。赖尚荣向宝玉笑道:“好叔叔,把他交给你,
我张罗人去了。”说着已经去了。宝玉便拉了柳湘莲到厅侧书房坐下,问他:“这
几日可到秦钟的坟上去了?”湘莲道:“怎么不去?前儿我们几个放鹰去,离他坟
上还有二里,我想今年夏天雨水勤,恐怕他坟上站不住。我背着众人走到那里去瞧
了一瞧,略又动了一点子,回家来就便弄了几百钱,第三日一早出去雇了两个人收
拾好了。”宝玉说:“怪道呢。上月我们大观园的池子里头结了莲蓬,我摘了十个,
叫焙茗出去到坟上供他去。回来我也问他:‘可被雨冲坏了没有?’他说:‘不但
没冲,更比上回新了些。’我想着必是这几个朋友新收拾了。我只恨我天天圈在家
里,一点儿做不得主,行动就有人知道,不是这个拦就是那个劝的,能说不能行。
虽然有钱,又不由我使。”柳湘莲道:“这个事也用不着你操心,外头有我,你只
心里有了就是了。眼前十月初一日,我已经打点下上坟的花销。你知道,我一贫如
洗,家里是没的积聚的;纵有几个钱来,随手就光的。不如趁空儿留下这一分,省
的到了跟前扎煞手。”宝玉道:“我也正为这个,要打发焙茗找你。你又不大在家,
知道你天天萍踪浪迹,没个一定的去处。”柳湘莲道:“你也不用找我,这个事也
不过各尽其道。眼前我还要出门去走走,外头游逛三年五载再回来。”宝玉听了,
忙问:“这是为何?”柳湘莲冷笑道:“我的心事,等到跟前,你自然知道。我如
今要别过了。”宝玉道:“好容易会着,晚上同散,岂不好?”湘莲道:“你那令
姨表兄还是那样,再坐着未免有事,不如我回避了倒好。”宝玉想一想,说道:“既
是这么样,倒是回避他为是。只是你要果真远行,必须先告诉我一声,千万别悄悄
的去了。”说着,便滴下泪来。柳湘莲说道:“自然要辞你去,你只别和别人说就
是了。”说着就站起来要走;又道:“你就进去罢,不必送我。”

一面说,一面出了书房。刚至大门前,早遇见薛蟠在那里乱叫:“谁放了小柳
儿走了?”柳湘莲听了,火星乱迸,恨不得一拳打死;复思酒后挥拳,又碍着赖尚
荣的脸面,只得忍了又忍。薛蟠忽见他走出来,如得了珍宝,忙趔趄着,走上去一
把拉住,笑道:“我的兄弟!你往那里去了?”湘莲道:“走走就来。”薛蟠笑道:
“你一去都没了兴头了,好歹坐一坐,就算疼我了!凭你什么要紧的事,交给哥哥,
只别忙。你有这个哥哥,你要做官发财都容易。”湘莲见他如此不堪,心中又恨又
恼,早生一计,拉他到僻净处,笑道:“你真心和我好,还是假心和我好呢?”薛
蟠听见这话,喜得心痒难挠,乜斜着眼,笑道:“好兄弟!你怎么问起我这样话来?
我要是假心,立刻死在眼前。”湘莲道:“既如此,这里不便。等坐一坐,我先走,
你随后出来,跟到我下处,咱们索性喝一夜酒。我那里还有两个绝好的孩子,从没
出门的。你可连一个跟的人也不用带,到了那里,伏侍人都是现成的。”薛蟠听如
此说,喜的酒醒了一半,说:“果然如此?”湘莲笑道:“如何!人拿真心待你,
你倒不信了。”薛蟠忙笑道:“我又不是呆子,怎么有个不信的呢?既如此,我又
不认得,你先去了,我在那里找你?”湘莲道:“我这下处在北门外头,你可舍得
家,城外住一夜去?”薛蟠道:“有了你,我还要家做什么!”湘莲道:“既如此,
我在北门外头桥上等你。咱们席上且吃酒去。你看我走了之后你再走,他们就不留
神了。”薛蟠听了,连忙答应道是。

二人复又入席,饮了一回。那薛蟠难熬,只拿眼看湘莲,心内越想越乐,左一
壶,右一壶,并不用人让,自己就吃了又吃,不觉酒有八九分了。湘莲就起身出来,
瞅人不防出至门外,命小厮杏奴:“先家去罢,我到城外就来。”说毕,已跨马直
出北门,桥上等候薛蟠。一顿饭的工夫,只见薛蟠骑着一匹马,远远的赶了来,张
着嘴,瞪着眼,头似拨浪鼓一般,不住左右乱瞧。及至从湘莲马前过去,只顾往远
处瞧,不曾留心近处。湘莲又笑又恨,他便也撒马随后跟来。薛蟠往前看时,渐渐
人烟稀少,便又圈马回来,再不想一回头见了湘莲,如获奇珍,忙笑道:“我说你
是个再不失信的。”湘莲笑道:“快往前走,仔细人看见跟了来,就不好了。”说
着,先就撒马前去。薛蟠也就紧紧跟来。

湘莲见前面人烟已稀,且有一带苇塘,便下马,将马拴在树上,向薛蟠笑道:
“你下来,咱们先设个誓。日后要变了心,告诉别人的,就应誓。”薛蟠笑道:“这
话有理。”连忙下马,也拴在树上,便跪下说道:“我要日久变心,告诉人去的,
天诛地灭。”一言未了,只听“镗”的一声,背后好似铁锤砸下来,只觉得一阵黑,
满眼金星乱迸,身不由己,就倒在地下了。湘莲走上来瞧瞧,知道他是个不惯捱打
的,只使了三分气力,向他脸上拍了几下,登时便开了果子铺。薛蟠先还要扎挣起
身,又被湘莲用脚尖点了一点,仍旧跌倒。口内说道:“原来是两家情愿,你不依,
只管好说,为什么哄出我来打我?”一面说,一面乱骂。湘莲道:“我把你这瞎了
眼的,你认认柳大爷是谁!你不说哀求,你还伤我!我打死你也无益,只给你个利害
罢。”说着,便取了马鞭过来,从背后至胫,打了三四十下。薛蟠的酒早已醒了大
半,不觉得疼痛难禁,由不的“嗳哟”一声。湘莲冷笑道:“也只如此,我只当你
是不怕打的。”一面说,一面又把薛蟠的左腿拉起来,向苇中泞泥处拉了几步,滚
的满身泥水,又问道:“你可认得我了?”薛蟠不应,只伏着哼哼。湘莲又掷下鞭
子,用拳头向他身上擂了几下,薛蟠便乱滚乱叫,说:“肋条折了!我知道你是正
经人,因为我错听了旁人的话了!”湘莲道:“不用拉旁人,你只说现在的。”薛
蟠道:“现在也没什么说的,不过你是个正经人,我错了!”湘莲道:“还要说软
些,才饶你。”薛蟠哼哼的道:“好兄弟——”湘莲便又一拳。薛蟠“嗳”了一声
道:“好哥哥——”湘莲又连两拳。薛蟠忙嗳哟叫道:“好老爷!饶了我这没眼睛
的瞎子罢!从今以后,我敬你怕你了!”湘莲道:“你把那水喝两口。”薛蟠一面
听了,一面皱眉道:“这水实在腌,怎么喝的下去!”湘莲举拳就打,薛蟠忙道:
“我喝我喝!”说着,只得俯头向苇根下喝了一口,犹未咽下去,只听“哇”的一
声,把方才吃的东西都吐了出来。湘莲道:“好腌东西,你快吃完了,饶你。”
薛蟠听了,叩头不迭,说:“好歹积阴功饶我罢!这至死不能吃的。”湘莲道:“这
么气息,倒熏坏了我!”说着,丢下了薛蟠,便牵马认镫去了。这里薛蟠见他已去,
方放下心来,后悔自己不该误认了人。待要扎挣起来,无奈遍体疼痛难禁。

谁知贾珍等席上忽不见了他两个,各处寻找不见。有人说:“恍惚出北门去了。”
薛蟠的小厮素日是惧他的,他吩咐了不许跟去,谁敢找去。后来还是贾珍不放心,
命贾蓉带着小厮们寻踪问迹的,直找出北门,下桥二里多路,忽见苇坑旁边薛蟠的
马拴在那里。众人都道:“好了,有马必有人。”一齐来至马前,只听苇中有人呻
吟。大家忙走来一看,只见薛蟠的衣衫零碎,面目肿破,没头没脸,遍身内外滚的
似个泥母猪一般。贾蓉心内已猜着八九了,忙下马命人搀了起来,笑道:“薛大叔
天天调情,今日调到苇子坑里。必定是龙王爷也爱上你风流,要你招驸马去,你就
碰到龙犄角上了!”薛蟠羞的没地缝儿钻进去,那里爬的上马去?贾蓉命人赶到关
厢里雇了一乘小轿子,薛蟠坐了,一齐进城。贾蓉还要抬往赖家去赴席,薛蟠百般
苦告,央及他不用告诉人,贾蓉方依允了,让他各自回家。贾蓉仍往赖家回复贾珍
并方才的形景。贾珍也知湘莲所打,也笑道:“他须得吃个亏才好。”至晚散了,
便来问候。薛蟠自在卧房将养,推病不见。

贾母等回来各自归家时,薛姨妈与宝钗见香菱哭的眼睛肿了,问起原故,忙来
瞧薛蟠时,脸上身上虽见伤痕,并未伤筋动骨。薛姨妈又是心疼,又是发恨,骂一
回薛蟠,又骂一回湘莲,意欲告诉王夫人,遣人寻拿湘莲。宝钗忙劝道:“这不是
什么大事,不过他们一处吃酒,酒后反脸常情。谁醉了,多挨几下子打,也是有的。
况且咱们家的无法无天的人,也是人所共知的。妈妈不过是心疼的原故,要出气也
容易。等三五天哥哥好了出得去的时候,那边珍大爷琏二爷这干人也未必白丢开
手,自然备个东道,叫了那个人来,当着众人替哥哥赔不是认罪就是了。如今妈妈
先当件大事告诉众人,倒显的妈妈偏心溺爱,纵容他生事招人,今儿偶然吃了一次
亏,妈妈就这样兴师动众,倚着亲戚之势欺压常人。”薛姨妈听了道:“我的儿!
到底是你想的到,我一时气糊涂了。”宝钗笑道:“这才好呢。他又不怕妈妈,又
不听人劝,一天纵似一天。吃过两三个亏,他也罢了。”

薛蟠睡在炕上,痛骂湘莲,又命小厮:“去拆他的房子,打死他,和他打官司!”
薛姨妈喝住小厮们,只说:“湘莲一时酒后放肆,如今酒醒,后悔不及,惧罪逃走
了。”

薛蟠听见如此说了,要知端底,且看下回分解。