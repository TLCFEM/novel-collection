\chapter{宁国府除夕祭宗祠~荣国府元宵开夜宴}

话说宝玉见晴雯将雀裘补完,已使得力尽神危,忙命小丫头子来替他捶着,彼
此捶打了一会。歇下没一顿饭的工夫,天已大亮,且不出门,只叫快请大夫。一时
王大夫来了,诊了脉,疑惑说道:“昨日已好了些,今日如何反虚浮微缩起来?敢
是吃多了饮食?不然就是劳了神思。外感却倒轻了,这汗后失调养,非同小可。”
一面说,一面出去开了药方进来。宝玉看时,已将疏散驱邪诸药减去,倒添茯苓、
地黄、当归等益神养血之剂。宝玉一面忙命人煎去,一面叹说:“这怎么处?倘或
有个好歹,都是我的罪孽!”晴雯睡在枕上,道:“好二爷!你干你的去罢。那
里就得了痨病了呢!”宝玉无奈,只得去了。至下半天,说身上不好,就回来了。
晴雯此症虽重,幸亏他素昔是个使力不使心的人,再者素昔饮食清淡,饥饱无伤的。
这贾宅中的秘法,无论上下,只略有些伤风咳嗽,总以净饿为主,次则服药调养。
故于前一日病时,就饿了两三天,又谨慎服药调养。如今虽劳碌了些,又加倍培养
了几日,便渐渐的好了。近日园中姐妹皆各在房中吃饭,炊爨饮食甚便,宝玉自能
要汤要羹调停,不必细说。

袭人送母殡后,业已回来,麝月便将坠儿一事,并“晴雯撵逐出去,也曾回过
宝玉”等语,一一的告诉袭人。袭人也没说别的,只说:“太性急了。”

只因李纨亦因时气感冒;邢夫人正害火眼,迎春岫烟皆过去朝夕侍药;李婶之
弟又接了李婶娘、李纹、李绮家去住几天;宝玉又见袭人常常思母含悲,晴雯又未
大愈:因此诗社一事,皆未有人作兴,便空了几社。

当下已是腊月,离年日近,王夫人和凤姐儿治办年事。王子腾升了九省都检点,
贾雨村补授了大司马,协理军机,参赞朝政,不提。

且说贾珍那边开了宗祠,着人打扫,收拾供器,请神主,又打扫上屋以备悬供
遗真影像。此时荣宁二府内外上下,皆是忙忙碌碌。这日宁府中尤氏正起来,同贾
蓉之妻打点送贾母这边的针线礼物,正值丫头捧了一茶盘押岁锞子进来,回说:“兴
儿回奶奶:前儿那一包碎金子,共是一百五十三两六钱七分,里头成色不等,总倾
了二百二十个锞子。”说着递上去。尤氏看了一看,只见也有梅花式的,也有海棠
式的,也有“笔锭如意”的,也有“八宝联春”的。尤氏命:“收拾起来,就叫兴
儿将银锞子快快交了进来。”丫鬟答应去了。

一时贾珍进来吃饭,贾蓉之妻回避了。贾珍因问尤氏:“咱们春祭的恩赏可领
了不曾?”尤氏道:“今儿我打发蓉儿关去了。”贾珍道:“咱们家虽不等这几两
银子使,多少是皇上天恩。早关了来,给那边老太太送过去,置办祖宗的供,上领
皇上的恩,下则是托祖宗的福。咱们那怕用一万银子供祖宗,到底不如这个有体面,
又是沾恩锡福。除咱们这么一二家之外,那些世袭穷官儿家,要不仗着这银子,拿
什么上供过年?真正皇恩浩荡,想得周到。”尤氏道:“正是这话。”二人正说着,
只见人回:“哥儿来了。”贾珍便命:“叫他进来。”只见贾蓉捧了一个小黄布口
袋进来。贾珍道:“怎么去了这一日?”贾蓉陪笑回说:“今儿不在礼部关领了,
又在光禄寺库上。因又到了光禄寺,才领下来了。光禄寺老爷们都说,问父亲好,
多日不见,都着实想念。”贾珍笑道:“他们那里是想我?这又到了年下了,不是
想我的东西,就是想我的戏酒了。”一面说,一面瞧那黄布口袋,上有封条,就是
“皇恩永锡”四个大字;那一边又有礼部祠祭司的印记。一行小字,道是:“宁国
公贾演,荣国公贾法,恩赐永远春祭赏共二分,净折银若干两,某年月日,龙禁尉
候补侍卫贾蓉当堂领讫。值年寺丞某人。”下面一个朱笔花押。

贾珍看了,吃过饭,盥漱毕,换了靴帽,命贾蓉捧着银子跟了来,回过贾母王
夫人,又至这边回过贾赦邢夫人,方回家去,取出银子,命将口袋向宗祠大炉内焚
了。又命贾蓉道:“你去问问你那边二婶娘,正月里请吃年酒的日子拟了没有?若
拟定了,叫书房里明白开了单子来,咱们再请时,就不能重复了。旧年不留神重了
几家,人家不说咱们不留心,倒像两家商议定了,送虚情怕费事的一样。”贾蓉忙
答应去了。一时,拿了请人吃年酒的日期单子来了,贾珍看了,命:“交给赖升去
看了,请人别重了这上头的日子。”因在厅上看着小厮们抬围屏,擦抹几案金银供
器。只见小厮手里拿着一个禀帖,并一篇账目,回说:“黑山村乌庄头来了。”贾
珍道:“这个老砍头的,今儿才来!”

贾蓉接过禀帖和账目,忙展开捧着,贾珍倒背着两手,向贾蓉手内看去。那红
禀上写着:“门下庄头乌进孝叩请爷奶奶万福金安,并公子小姐金安。新春大喜大
福,荣贵平安,加官进禄,万事如意。”贾珍笑道:“庄家人有些意思。”贾蓉也
忙笑道:“别看文法,只取个吉利儿罢。”一面忙展开单子看时,只见上面写着:

大鹿三十只,獐子五十只,子五十只,暹猪二十个,汤猪二十个,龙猪二十
个,野猪二十个,家腊猪二十个,野羊二十个,青羊二十个,家汤羊二十个,家风
羊二十个,鲟鳇鱼二百个,各色杂鱼二百斤,活鸡、鸭、鹅各二百只,风鸡、鸭、
鹅二百只,野鸡野猫各二百对,熊掌二十对,鹿筋二十斤,海参五十斤,鹿舌五十
条,牛舌五十条,蛏干二十斤,榛、松、桃、杏瓤各二口袋,大对虾五十对,干虾
二百斤,银霜炭上等选用一千斤,中等二千斤,柴炭三万斤,御田胭脂米二担,碧
糯五十斛,白糯五十斛,粉粳五十斛,杂色粱谷各五十斛,下用常米一千担,各
色干菜一车,外卖粱谷牲口各项折银二千五百两。外门下孝敬哥儿玩意儿:活鹿两
对,白兔四对,黑兔四对,活锦鸡两对,西洋鸭两对。

贾珍看完,说:“带进他来。”一时只见乌进孝进来,只在院内磕头请安。贾
珍命人拉起他来,笑说:“你还硬朗?”乌进孝笑道:“不瞒爷说,小的们走惯了,
不来也闷的慌。他们可都不是愿意来见见天子脚下世面?他们到底年轻,怕路上有
闪失,再过几年就可以放心了。”贾珍道:“你走了几日?”乌进孝道:“回爷的
话:今年雪大,外头都是四五尺深的雪,前日忽然一暖一化,路上竟难走的很,耽
搁了几日。虽走了一个月零两日,日子有限,怕爷心焦,可不赶着来了!”贾珍道:
“我说呢,怎么今儿才来!我才看那单子上,今年你这老货又来打擂台来了。”乌
进孝忙进前两步回道:“回爷说:今年年成实在不好。从三月下雨,接连着直到八
月,竟没有一连晴过五六日;九月一场碗大的雹子,方近二三百里地方,连人带房
并牲口粮食,打伤了上千上万的:所以才这样。小的并不敢说谎。”贾珍绉眉道:
“我算定你至少也有五千银子来,这够做什么的?如今你们一共只剩了八九个庄
子,今年倒有两处报了旱潦,你们又打擂台,真真是叫别过年了!”乌进孝道:“爷
的这地方还算好呢。我兄弟离我那里只一百多地,竟又大差了。他现管着那府八处
庄地,比爷这边多着几倍,今年也是这些东西,不过二三千两银子,也是有饥荒打
呢!”贾珍道:“正是呢。我这边倒可已,没什么外项大事,不过是一年的费用。
我受用些就费些,我受些委曲就省些。再者年例送人请人,我把脸皮厚些也就完了。
比不得那府里,这几年添了许多花钱的事,一定不可免是要花的,却又不添些银子
产业。这一二年里赔了许多,不和你们要,找谁去?”

乌进孝笑道:“那府里如今虽添了事,有去有来。娘娘和万岁爷岂不赏呢?”
贾珍听了,笑向贾蓉等道:“你们听听,他说的可笑不可笑?”贾蓉等忙笑道:“你
们山坳海沿子上的人,那里知道这道理?娘娘难道把皇上的库给我们不成?他心里纵
有这心,他不能作主。岂有不赏之理,按时按节,不过是些彩缎、古董、玩意儿。
就是赏,也不过一百两金子,才值一千多两银子,够什么?这二年,那一年不赔出
几千两银子来?头一年省亲连盖花园子,你算算那一注花了多少,就知道了。再二
年,再省一回亲,只怕就精穷了!”贾珍笑道:“所以他们庄客老实人:‘外明不
知里暗的事’,‘黄柏木作了磬槌子——外头体面里头苦。’”贾蓉又说又笑向贾
珍道:“果真那府里穷了,前儿我听见二婶娘和鸳鸯悄悄商议,要偷老太太的东西
去当银子呢。”贾珍笑道:“那又是凤姑娘的鬼,那里就穷到如此?他必定是见去
路大了,实在赔得很了,不知又要省那一项的钱,先设出这法子来,使人知道,说
穷到如此了。我心里却有个算盘,还不至此田地。”说着,便命人带了乌进孝出去,
好生待他,不在话下。

这里贾珍吩咐将方才各物留出供祖宗的来,将各样取了些,命贾蓉送过荣府里
来,然后自己留了家中所用的,馀者派出等第,一份一份的堆在月台底下,命人将
族中子侄唤来分给他们。接着荣国府也送了许多供祖之物及给贾珍之物。贾珍看着
收拾完备供器,着鞋,披着一件猞猁狲大皮袄,命人在厅柱下石阶上太阳中,铺
了一个大狼皮褥子负暄,闲看各子弟们来领取年物。因见贾芹亦来领物,贾珍叫他
过来,说道:“你做什么也来了?谁叫你来的?”贾芹垂手回说:“听见大爷这里
叫我们领东西,我没等人去就来了。”贾珍道:“我这东西,原是给你那些闲着无
事没进益的叔叔兄弟们的,那二年你闲着,我也给过你的。你如今在那府里管事,
家庙里管和尚道士们,一月又有你的分例外,这些和尚的分例银钱都从你手里过,
你还来取这个来!太也贪了!你自己瞧瞧,你穿的可像个手里使钱办事的?先前你说
没进益,如今又怎么了?比先倒不像了?”贾芹道:“我家里原人口多,费用大。”
贾珍冷笑道:“你又支吾我!你在家庙里干的事,打量我不知道呢。你到那里,自
然是爷了,没人敢抗违你。你手里又有了钱,离着我们又远,你就为王称霸起来,
夜夜招聚匪类赌钱,养老婆小子。这会子花得这个形象,你还敢领东西来?领不成
东西,领一顿驮水棍去才罢!等过了年,我必和你二叔说,换回你来。”贾芹红了
脸,不敢答言。人回:“北府王爷送了对联荷包来了。”贾珍听说,忙命贾蓉:“出
去款待,只说我不在家。”贾蓉去了。这里贾珍撵走贾芹,看着领完东西,回屋与
尤氏吃毕晚饭,一宿无话。至次日更忙,不必细说。

已到了腊月二十九日了,各色齐备,两府中都换了门神、联对、挂牌,新油了
桃符,焕然一新。宁国府从大门、仪门、大厅、暖阁、内厅、内三门、内仪门并内
垂门,直到正堂,一路正门大开,两边阶下一色朱红大高烛,点的两条金龙一般。
次日由贾母有封诰者,皆按品级着朝服,先坐八人大轿,带领众人进宫朝贺行礼。
领宴毕回来,便到宁府暖阁下轿。诸子弟有未随入朝者,皆在宁府门前排班伺候,
然后引入宗祠。

且说宝琴是初次进贾祠观看,一面细细留神打量这宗祠:原来宁府西边另一个
院子,黑油栅栏内五间大门,上面悬一匾,写着是“贾氏宗祠”四个字,旁书“特
晋爵太傅前翰林掌院事王希献书”。两边有一副长联,写道:
肝脑涂地,兆姓赖保育之恩;
功名贯天,百代仰蒸尝之盛。
也是王太傅所书。进入院中,白石甬路,两边皆是苍松翠柏,月台上设着古铜鼎彝
等器。抱厦前面悬一块九龙金匾,写道“星辉辅弼”,乃先皇御笔。两边一副对联,
写道是:
勋业有光昭日月,
功名无间及儿孙。
也是御笔。五间正殿前,悬一块闹龙填青匾,写道是“慎终追远”。傍边一副对联,
写道是:
已后儿孙承福德,
至今黎庶念宁荣。
俱是御笔。里边灯烛辉煌,锦幛绣幕,虽列着些神主,却看不真。

只见贾府人分了昭穆,排班立定。贾敬主祭,贾赦陪祭,贾珍献爵,贾琏贾琮
献帛,宝玉捧香,贾菖贾菱展拜垫、守焚池。青衣乐奏,三献爵,兴、拜毕,焚帛,
奠酒。礼毕乐止退出,众人围随贾母至正堂上。影前锦帐高挂,彩屏张护,香烛辉
煌,上面正居中悬着荣宁二祖遗像,皆是披蟒腰玉,两边还有几轴列祖遗像。贾荇
贾芷等从内仪门挨次站列,直到正堂廊下,槛外方是贾敬贾赦,槛内是各女眷。众
家人小厮皆在仪门之外。每一道菜至,传至仪门,贾荇贾芷等便接了,按次传至阶
下贾敬手中。贾蓉系长房长孙,独他随女眷在槛里,每贾敬捧菜至,传于贾蓉,贾
蓉便传于他媳妇,又传于凤姐尤氏诸人,直传至供桌前,方传与王夫人。王夫人传
与贾母,贾母方捧放在桌上。邢夫人在供桌之西,东向立,同贾母供放。直至将菜
饭汤点酒茶传完,贾蓉方退出去,归入贾芹阶位之首。当时凡从“文”旁之名者,
贾敬为首;下则从“玉”者,贾珍为首;再下从“草头”者,贾蓉为首:左昭右穆,
男东女西。俟贾母拈香下拜,众人方一齐跪下,将五间大厅,三间抱厦,内外廊檐,
阶上阶下两丹墀内,花团锦簇,塞的无一些空地。鸦雀无闻,只听铿锵叮当,金铃
玉微微摇曳之声,并起跪靴履飒沓之响。

一时礼毕,贾敬贾赦等便忙退出至荣府,专候与贾母行礼。尤氏上房地下,铺
满红毡,当地放着象鼻三足泥鳅流金珐琅大火盆,正面炕上铺着新猩红毡子,设着
大红彩绣云龙捧寿的靠背、引枕、坐褥,外另有黑狐皮的袱子搭在上面。大白狐皮
坐褥,请贾母上去坐了。两边又铺皮褥,请贾母一辈的两三位妯娌坐了。这边横头
排插之后小炕上,也铺了皮褥,让邢夫人等坐下。地下两面相对十二张雕漆椅上,
都是一色灰鼠椅搭小褥,每一张椅下一个大铜脚炉,让宝琴等姐妹坐。尤氏用茶盘
亲捧茶与贾母,贾蓉媳妇捧与众老祖母,然后尤氏又捧与邢夫人等,贾蓉媳妇又捧
与众姐妹。凤姐李纨等只在地下伺候。

茶毕,邢夫人等便先起身来侍贾母吃茶。贾母与年老妯娌们闲话了两三句,便
命看轿,凤姐儿忙上去搀起来。尤氏笑回说:“已经预备下老太太的晚饭。每年都
不肯赏些体面,用过晚饭再过去。果然我们就不济凤丫头了?”凤姐儿搀着贾母笑
道:“老祖宗走罢。咱们家去吃去,别理他。”贾母笑道:“你这里供着祖宗,忙
得什么儿似的,那里还搁的住我闹?况且我每年不吃,你们也要送去的;不如还送
了来,我吃不了,留着明儿再吃,岂不多吃些?”说的众人都笑了。又吩咐他:“好
生派妥当人夜里坐着看香火,不是大意得的。”尤氏答应了。一面走出来,至暖阁
前,尤氏等闪过屏风,小厮们才领轿夫请了轿出大门。尤氏亦随邢夫人等回至荣府。
这里轿出大门,这一条街上东一边设立着宁国公的仪仗执事乐器,西一边设立着荣
国公的仪仗执事乐器,来往行人皆屏退不从此过。

一时来至荣府,也是大门正门一直开到里头。如今便不在暖阁下轿了,过了大
厅,转弯向西,至贾母这边正厅上下轿。众人围随同至贾母正堂中间,亦是锦绣
屏,焕然一新。当地火盆内焚着松柏香、百合草。贾母归了坐,老嬷嬷来回:“老
太太们来行礼。”贾母忙起身要迎,只见两三个老妯娌已进来了。大家挽手笑了一
回,让了一回。吃茶去后,贾母只送至内仪门就回来,归了正坐。贾敬贾赦等领了
诸子弟进来。贾母笑道:“一年家难为你们,不行礼罢。”一面男一起,女一起,
一起一起俱行过了礼。左右设下交椅,然后又按长幼挨次归坐受礼。两府男女、小
厮、丫鬟,亦按差役上、中、下行礼毕。然后散了押岁钱并荷包金银锞等物。摆上
合欢宴来,男东女西归坐,献屠苏酒、合欢汤、吉祥果、如意糕毕。贾母起身,进
内间更衣,众人方各散出。那晚各处佛堂灶王前焚香上供。王夫人正房院内设着天
地纸马香供。大观园正门上挑着角灯,两旁高照,各处皆有路灯。上下人等,打扮
的花团锦簇。一夜人声杂沓,语笑喧阗,爆竹起火,络绎不绝。

至次日五鼓,贾母等人按品上妆,摆全副执事进宫朝贺,兼祝元春千秋。领宴
回来,又至宁府祭过列祖,方回来。受礼毕,便换衣歇息。所有贺节来的亲友,一
概不会,只和薛姨妈李婶娘二人说话随便,或和宝玉宝钗等姐妹赶围棋摸牌作戏。
王夫人和凤姐天天忙着请人吃年酒,那边厅上和院内皆是戏酒,亲友络绎不绝。

一连忙了七八天,才完了,早又元宵将近。宁荣二府皆张灯结彩。十一日是贾
赦请贾母等,次日贾珍又请贾母。王夫人和凤姐儿也连日被人请去吃年酒,不能胜
记。至十五这一晚上,贾母便在大花厅上命摆几席酒,定一班小戏,满挂各色花灯,
带领荣宁二府各子侄孙男孙媳等家宴。贾敬素不饮酒茹荤,因此不去请他。十七日
祀祖已完,他就出城修养。就是这几天在家,也只静室默处,一概无闻,不在话下。
贾赦领了贾母之赏,告辞而去。贾母知他在此不便,也随他去了。贾赦到家中,和
众门客赏灯吃酒,笙歌聒耳,锦绣盈眸,其取乐与这里不同。

这里贾母花厅上摆了十来席酒,每席傍边设一几,几上设炉瓶三事,焚着御赐
百合宫香;又有八寸来长、四五寸宽、二三寸高、点缀着山石的小盆景,俱是新鲜
花卉。又有小洋漆茶盘放着旧窑十锦小茶杯,又有紫檀雕嵌的大纱透绣花草诗字的
缨络。各色旧窑小瓶中,都点缀着“岁寒三友”、“玉堂富贵”等鲜花。上面两席
是李婶娘薛姨妈坐,东边单设一席,乃是雕夔龙护屏矮足短榻,靠背、引枕、皮褥
俱全。榻上设一个轻巧洋漆描金小几,几上放着茶碗、漱盂、洋巾之类,又有一个
眼镜匣子。贾母歪在榻上,和众人说笑一回,又取眼镜向戏台上照一回,又说:“恕
我老了骨头疼,容我放肆些,歪着相陪罢。”又命琥珀坐在榻上,拿着美人拳捶腿。
榻下并不摆席面,只一张高几,设着高架缨络、花瓶、香炉等物,外另设一小高桌,
摆着杯箸。在傍边一席,命宝琴、湘云、黛玉、宝玉四人坐着,每馔果菜来,先捧
给贾母看,喜则留在小桌上尝尝,仍撤了放在席上。只算他四人跟着贾母坐。下面
方是邢夫人王夫人之位。下边便是尤氏、李纨、凤姐、贾蓉的媳妇,西边便是宝钗、
李纹、李绮、岫烟、迎春姐妹等。两边大梁上挂着联三聚五玻璃彩穗灯,每席前竖
着倒垂荷叶一柄,柄上有彩烛插着。这荷叶乃是洋錾珐琅活信,可以扭转向外,将
灯影逼住,照着看戏,分外真切。窗门户,一齐摘下,全挂彩穗各种宫灯。廊檐
内外及两边游廊罩棚,将羊角、玻璃、戳纱、料丝,或绣、或画、或绢、或纸诸灯
挂满。廊上几席,就是贾珍、贾琏、贾环、贾琮、贾蓉、贾芹、贾芸、贾菖、贾菱
等。

贾母也曾差人去请众族中男女,奈他们有年老的,懒于热闹;有家内没有人,
又有疾病淹留,要来竟不能来;有一等妒富愧贫,不肯来的;更有憎畏凤姐之为人,
赌气不来的;更有羞手羞脚,不惯见人,不敢来的:因此族中虽多,女眷来者不过
贾蓝之母娄氏带了贾蓝来,男人只有贾芹、贾芸、贾菖、贾菱四个现在凤姐麾下办
事的来了。当下人虽不全,在家庭小宴,也算热闹的。

当下又有林之孝的媳妇,带了六个媳妇,抬了三张炕桌,每一张上搭着一条红
毡,放着选净一般大新出局的铜钱,用大红绳串穿着,每二人搭一张,共三张。林
之孝家的叫将那两张摆至薛姨妈李婶娘的席下,将一张送至贾母榻下。贾母便说:
“放在当地罢。”这媳妇素知规矩,放下桌子,一并将钱都打开,将红绳抽去,堆
在桌上。此时唱的《西楼会》,正是这出将完,于叔夜赌气去了。那文豹便发科诨
道:“你赌气去了。恰好今日正月十五,荣国府里老祖宗家宴,待我骑了这马,赶
进去讨些果子吃,是要紧的。”说毕,引得贾母等都笑了。薛姨妈等都说:“好个
鬼头孩子,可怜见的。”凤姐便说:“这孩子才九岁了。”贾母笑说:“难为他说
得巧。”说了一个“赏”字。早有三个媳妇已经手下预备下小笸箩,听见一个“赏”
字,走上去将桌上散堆钱每人撮了一笸箩,走出来向戏台说:“老祖宗、姨太太、
亲家太太赏文豹买果子吃的。”说毕,向台一撒,只听“豁啷啷”,满台的钱响。
贾珍贾琏已命小厮们抬大笸箩的钱预备。

未知怎生赏去,且听下回分解。