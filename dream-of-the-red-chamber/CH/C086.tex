\chapter{受私贿老官翻案牍~寄闲情淑女解琴书}

话说薛姨妈听了薛蝌的来书,因叫进小厮,问道:“你听见你大爷说,到底是
怎么就把人打死了呢?”小厮道:“小的也没听真切。那一日,大爷告诉二爷说—
—”说着回头看了一看,见无人,才说道:“大爷说:自从家里闹的特利害,大爷
也没心肠了,所以要到南边置货去。这日想着约一个人同行,这人在咱们这城南二
百多地住。大爷找他去了,遇见在先和大爷好的那个蒋玉函,带着些小戏子进城,
大爷同他在个铺子里吃饭喝酒。因为这当槽儿的尽着拿眼瞟蒋玉函,大爷就有了气
了。后来蒋玉函走了。第二天,大爷就请找的那个人喝酒。酒后想起头一天的事来,
叫那当槽儿的换酒,那当槽儿的来迟了,大爷就骂起来了。那个人不依,大爷就拿
起酒碗照他打去。谁知那个人也是个泼皮,便把头伸过来叫大爷打。大爷拿碗就砸
他的脑袋,一下子就冒了血了,躺在地下。头里还骂,后头就不言语了。”薛姨妈
道:“怎么也没人劝劝吗?”那小厮道:“这个没听见大爷说,小的不敢妄言。”薛
姨妈道:“你先去歇歇罢。”小厮答应出来。

这里薛姨妈自来见王夫人,托王夫人转求贾政。贾政问了前后,也只好含糊应
了,只说等薛蝌递了呈子,看他本县怎么批了,再作道理。这里薛姨妈又在当铺里
兑了银子,叫小厮赶着去了。三日后果有回信,薛姨妈接着了,即叫小丫头告诉宝
钗,连忙过来看了。只见书上写道:

带去银两做了衙门上下使费。哥哥在监,也不大吃苦,请太太放心。独是这里
的人很刁,尸亲见证都不依,连哥哥请的那个朋友也帮着他们。我与李祥两个俱系
生地生人,幸找着一个好先生,许他银子,才讨个主意,说是须得拉扯着同哥哥喝
酒的吴良,弄人保出他来,许他银两,叫他撕掳。他若不依,便说张三是他打死,
明推在异乡人身上。他吃不住,就好办了。我依着他,果然吴良出来。现在买嘱尸
亲见证,又做了一张呈子,前日递的,今日批来,请看呈底便知。
因又念呈底道:

具呈人某,呈为兄遭飞祸、代伸冤抑事:窃生胞兄薛蟠,本籍南京,寄寓西京,
于某年月日,备本往南贸易。去未数日,家奴送信回家,说遭人命,生即奔宪治,
知兄误伤张姓。及至囹圄,据兄泣告,实与张姓素不相认,并无仇隙。偶因换酒角
口,先兄将酒泼地,恰值张三低头拾物,一时失手,酒碗误碰囟门身死。蒙恩拘讯,
兄惧受刑,承认斗殴致死。仰蒙宪天仁慈,知有冤抑,尚未定案。生兄在禁,具呈
诉辩,有干例禁;生念手足,冒死代呈。伏乞宪慈恩准提证质讯,开恩莫大,生等
举家仰戴鸿仁,永永无既矣!激切上呈。
批的是:

尸场检验,证据确凿。且并未用刑,尔兄自认斗杀,招供在案。今尔远来,并
非目睹,何得捏词妄控?理应治罪,姑念为兄情切,且恕。不准。

薛姨妈听到那里,说道:“这不是救不过来了么?这怎么好呢?”宝钗道:“二
哥的书还没看完,后面还有呢。”因又念道:“有要紧的问来使便知。”

薛姨妈便问来人。因说道:“县里早知我们的家当充足。须得在京里谋干得大
情,再送一分大礼,还可以复审,从轻定案。太太此时必得快办,再迟了就怕大爷
要受苦了。”薛姨妈听了,叫小厮自去,即刻又到贾府与王夫人说明原委,恳求贾
政。贾政只肯托人与知县说情,不肯提及银物。薛姨妈恐不中用,求凤姐与贾琏说
了,花上几千银子,才把知县买通。

薛蝌那里也便弄通了,然后知县挂牌坐堂,传齐了一干邻保、证见、尸亲人等,
监里提出薛蟠,刑房书吏俱一一点名。知县便叫地保对明初供,又叫尸亲张王氏并
尸叔张二问话。张王氏哭禀:“小的的男人是张大,南乡里住,十八年头里死了。
大儿子、二儿子,也都死了。光留下这个死的儿子,叫张三,今年二十三岁,还没
有娶女人呢。为小人家里穷,没得养活,在李家店里做当槽儿的。那一天晌午,李
家店里打发人来叫俺,说:‘你儿子叫人打死了。’我的青天老爷!小的就唬死了!跑
到那里,看见我儿子头破血出的躺在地下喘气儿,问他话也说不出来,不多一会儿
就死了。小人就要揪住这个小杂种拚命!”众衙役吆喝一声,张王氏便磕头道:“求
青天老爷伸冤!小人就只这一个儿子了。”

知县便叫:“下去。”又叫李家店的人问道:“那张三是在你店内佣工的么?”
那李二回道:“不是佣工,是做当槽儿的。”知县道:“那日尸场上,你说张三是薛
蟠将碗砸死的,你亲眼见的么?”李二说道:“小的在柜上,听见说客房里要酒,
不多一回,便听见说,‘不好了,打伤了!’小的跑进去,只见张三躺在地下,也不
能言语。小的便喊禀地保,一面报他母亲去了。他们到底怎样打的,实在不知道,
求太爷问那喝酒的便知道了。”知县喝道:“初审口供你是亲见的,怎么如今说没有
见!”李二道:“小的前日唬昏了乱说。”衙役又吆喝了一声。知县便叫吴良问道:“你
是同在一处喝酒的么?薛蟠怎么打的?据实供来!”吴良说:“小的那日在家,这个薛
大爷叫我喝酒。他嫌酒不好,要换,张三不肯。薛大爷生气,把酒向他脸上泼去,
不晓得怎么样就碰在那脑袋上了。这是亲眼见的。”知县道:“胡说,前日尸场上薛
蟠自己认拿碗砸死的,你说你亲眼见的,怎么今日的供不对?掌嘴!”衙役答应着要
打。吴良求着说:“薛蟠实没有和张三打架,酒碗失手,碰在脑袋上的。求老爷问
薛蟠,便是恩典了!”

知县叫上薛蟠,问道:“你与张三到底有什么仇隙?毕竟是如何死的?实供上
来。”薛蟠道:“求太老爷开恩:小的实没有打他,为他不肯换酒,故拿酒泼地。不
想一时失手,酒碗误碰在他的脑袋上。小的即忙掩他的血,那里知道再掩不住,血
淌多了,过一回就死了。前日尸场上,怕太老爷要打,所以说是拿碗砸他的。只求
太老爷开恩!”知县便喝道:“好个糊涂东西!本县问你怎么砸他的,你便供说恼他
不换酒,才砸的,今日又供是失手碰的!”知县假作声势,要打要夹。薛蟠一口咬
定。知县叫仵作:“将前日尸场填写伤痕,据实报来。”仵作禀报说:“前日验得张
三尸身无伤,惟囟门有磁器伤,长一寸七分,深五分,皮开,囟门骨脆,裂破三分。
实系磕碰伤。”

知县查对尸格相符,早知书吏改轻,也不驳诘,胡乱便叫画供。张王氏哭喊道:
“青天老爷!前日听见还有多少伤,怎么今日都没有了?”知县道:“这妇人胡说!
现有尸格,你不知道么?”叫尸叔张二,便问道:“你侄儿身死,你知道有几处伤?”
张二忙供道:“脑袋上一伤。”知县道:“可又来。”叫书吏将尸格给张王氏瞧去,并
叫地保、尸叔指明与他瞧:现有尸场亲押、证见,俱供并未打架,不为斗殴,只依
误伤吩咐画供,将薛蟠监禁候详,馀令原保领出,退堂。张王氏哭着乱嚷,知县叫
众衙役撵他出去。张二也劝张王氏道:“实在误伤,怎么赖人?现在太老爷断明,别
再胡闹了。”

薛蝌在外打听明白,心内喜欢,便差人回家送信,等批详回来,便好打点赎罪,
且住着等信。只听路上三三两两传说:“有个贵妃薨了,皇上辍朝三日。”这里离陵
寝不远,知县办差垫道,一时料着不得闲,住在这里无益,不如到监,告诉哥哥:
“安心等着,我回家去,过几日再来。”薛蟠也怕母亲痛苦,带信说:“我无事,必
须衙门再使费几次便可回家了。只是别心疼银子钱。”薛蝌留下李祥在此照料,一
径回家,见了薛姨妈,陈说知县怎样徇情,怎样审断,终定了误伤:“将来尸亲那
里再花些银子,一准赎罪便没事了。”薛姨妈听说暂且放心,说:“正盼你来家中照
应。贾府里本该谢去,况且周贵妃薨了,他们天天进去,家里空落落的。我想着要
去替姨太太那边照应照应,作伴儿,只是咱们家又没人,你这来的正好。”薛蝌道:
“我在外头,原听见说是贾妃薨了,这么才赶回来的。我们娘娘好好儿的,怎么就
死了?”薛姨妈道:“上年原病过一次,也就好了。这回又没听见娘娘有什么病,
只闻那府里头几天老太太不大受用,合上眼便看见元妃娘娘,众人都不放心。直至
打听起来,又没有什么事。到了大前儿晚上,老太太亲口说是‘怎么元妃独自一个
人到我这里?’众人只道是病中想的话,总不信。老太太又说:‘你们不信,元妃
还和我说是:“荣华易尽,须要退步抽身。”’众人都说:‘谁不想到?这是有年纪的
人思前想后的心事。’所以也不当件事。恰好第二天早起,里头吵嚷出来,说娘娘
病重,宣各诰命进去请安。他们就惊疑的了不得,赶着进去。他们还没有出来,我
们家里已听见周贵妃薨逝了。你想外头的讹言,家里的疑心,恰碰在一处,可奇不
奇?”宝钗道:“不但是外头的讹言舛错,便在家里的,一听见‘娘娘’两个字,
也就都忙了,过后才明白。这两天那府里这些丫头婆子来说,他们早知道不是咱们
家的娘娘。我说:‘你们那里拿得定呢?’他说道:‘前几年正月,外省荐了一个算
命的,说是很准的。老太太叫人将元妃八字夹在丫头们八字里头,送出去叫他推算,
他独说:“这正月初一日生日的那位姑娘,只怕时辰错了;不然,真是个贵人,也
不能在这府中。”老爷和众人说:“不管他错不错,照八字算去。”那先生便说:“甲
申年,正月丙寅,这四个字内,有‘伤官’‘败财’。惟‘申’字内有‘正官’禄马,
这就是家里养不住的,也不见什么好。这日子是乙卯,初春木旺,虽是‘比肩’,
那里知道愈‘比’愈好,就像那个好木料,愈经斫削,才成大器。”独喜得时上什
么辛金为贵,什么巳中“正官”禄马独旺:这叫作“飞天禄马格”。又说什么“日
逢‘专禄’,贵重的很。‘天月二德’坐本命,贵受椒房之宠。这位姑娘,若是时辰
准了,定是一位主子娘娘。”这不是算准了么?我们还记得说:“可惜荣华不久;只
怕遇着寅年卯月,这就是‘比’而又‘比’,‘劫’而又‘劫’,譬如好木,太要做
玲珑剔透,本质就不坚了。”他们把这些话都忘记了,只管瞎忙。我才想起来,告
诉我们大奶奶,今年那里是寅年卯月呢?’”宝钗尚未述完这话,薛蝌急道:“且别
管人家的事。既有这个神仙算命的,我想哥哥今年什么恶星照命,遭这么横祸?快
开八字儿,我给他算去,看有妨碍么。”宝钗道:“他是外省来的,不知今年在京不
在了。”说着,便打点薛姨妈往贾府去。

到了那里,只有李纨探春等在家接着,便问道:“大爷的事怎么样了?”薛姨
妈道:“等详了上司才定,看来也到不了死罪。”这才大家放心。探春便道:“昨晚
太太想着说:‘上回家里有事,全仗姨太太照应,如今自己有事,也难提了。’心里
只是不放心。”薛姨妈道:“我在家里,也是难过。只是你大哥遭了这事,你二兄弟
又办事去了,家里你姐姐一个人,中什么用?况且我们媳妇儿又是个不大晓事的,
所以不能脱身过来。目今那里知县也正为预备周贵妃的差使,不得了结案件,所以
你二兄弟回来了,我才得过来看看。”李纨便道:“请姨太太这里住几天更好。”薛
姨妈点头道:“我也要在这边给你们姐妹们作作伴儿,就只你宝妹妹冷静些。”惜春
道:“姨妈要惦着,为什么不把宝姐姐也请过来?”薛姨妈笑着说道:“使不得。”
惜春道:“怎么使不得?他先怎么住着来呢?”李纨道:“你不懂的。人家家里如今
有事,怎么来呢?”惜春也信以为实,不便再问。

正说着,贾母等回来,见了薛姨妈,也顾不得问好,便问薛蟠的事。薛姨妈细
述了一遍。宝玉在旁听见什么蒋玉函一段,当着人不问,心里打量是:“他既回了
京,怎么不来瞧我?”又见宝钗也不过来,不知是怎么个原故。心内正自呆呆的想
呢,恰好黛玉也来请安。宝玉稍觉心里喜欢,便把想宝钗来的念头打断,同着姊妹
们在老太太那里吃了晚饭。大家散了,薛姨妈将就住在老太太的套间屋里。

宝玉回到自己房中,换了衣裳,忽然想起蒋玉函给的汗巾,便向袭人道:“你
那一年没有系的那条红汗巾子,还有没有?”袭人道:“我搁着呢,问他做什么?”
宝玉道:“我白问问。”袭人道:“你没有听见薛大爷相与这些混帐人,所以闹到人
命关天,你还提那些做什么?有这样白操心,倒不如静静儿的念念书,把这些个没
要紧的事撂开了也好。”宝玉道:“我并不闹什么,偶然想起,有也罢没也罢,我白
问一声,你们就有这些话。”袭人笑道:“并不是我多话。一个人知书达礼,就该往
上巴结才是。就是心爱的人来了,也叫他瞧着喜欢尊敬啊。”宝玉被袭人一提,便
说:“了不得!方才我在老太太那边,看见人多,没有和林妹妹说话,他也不曾理我。
散的时候他先走了,此时必在屋里,我去就来。”说着就走。袭人道:“快些回来罢。
这都是我提头儿,倒招起你的高兴来了。”

宝玉也不答言,低着头,一径走到潇湘馆来。只见黛玉靠在桌上看书。宝玉走
到跟前,笑说道:“妹妹早回来了?”黛玉也笑道:“你不理我,我还在那里做什
么?”宝玉一面笑说:“他们人多说话,我插不下嘴去,所以没有和你说话。”一面
瞧着黛玉看的那本书,书上的字一个也不认得。有的像“芍”字;有的像“茫”字;
也有一个“大”字旁边“九”字加上一勾,中间又添个“五”字;也有上头“五”
字“六”字又添一个“木”字,底下又是一个“五”字。看着又奇怪,又纳闷,便
说:“妹妹近日越发进了,看起天书来了。”黛玉“嗤”的一声笑道:“好个念书的
人,连个琴谱都没有见过?”宝玉道:“琴谱怎么不知道?为什么上头的字一个也不
认得?妹妹你认得么?”黛玉道:“不认得瞧他做什么?”宝玉道:“我不信,从没
有听见你会抚琴。我们书房里挂着好几张,前年来了一个清客先生,叫做什么嵇好
古,老爷烦他抚了一曲。他取下琴来,说都使不得,还说:‘老先生若高兴,改日
携琴来请教。’想是我们老爷也不懂,他便不来了。怎么你有本事藏着?”黛玉道:
“我何尝真会呢。前日身上略觉舒服,在大书架上翻书,看有一套琴谱,甚有雅趣,
上头讲的琴理甚通,手法说的也明白,真是古人静心养性的工夫。我在扬州,也听
得讲究过,也曾学过,只是不弄了,就没有了。这果真是‘三日不弹,手生荆棘’。
前日看这几篇,没有曲文,只有操名,我又到别处找了一本有曲文的来看着,才有
意思。究竟怎么弹的好,实在也难。书上说的:师旷鼓琴,能来风雷龙凤。孔圣人
尚学琴于师襄,一操便知其为文王。高山流水,得遇知音。”说到这里,眼皮儿微
微一动,慢慢的低下头去。

宝玉正听得高兴,便道:“好妹妹,你才说的实在有趣。只是我才见上头的字
都不认得,你教我几个呢。”黛玉道:“不用教的,一说便可以知道的。”宝玉道:“我
是个糊涂人,得教我那个‘大’字加一勾,中间一个‘五’字的。”黛玉笑道:“这
‘大’字‘九’字是用左手大拇指按琴上的‘九徽’,这一勾加‘五’字是右手钩
‘五弦’,并不是一个字,乃是一声:是极容易的。还有吟、揉、绰、注、撞、走、
飞、推等法,是讲究手法的。”宝玉乐得手舞足蹈的说:“好妹妹你既明琴理,我们
何不学起来?”黛玉道:“琴者禁也。古人制下,原以治身,涵养性情,抑其淫荡,
去其奢侈。若要抚琴,必择静室高斋,或在层楼的上头,在林石的里面,或是山巅
上,或是水涯上。再遇着那天地清和的时候,风清月朗,焚香静坐,心不外想,气
血和平,才能与神合灵,与道合妙。所以古人说:‘知音难遇。’若无知音,宁可独
对着那清风明月苍松怪石野猿老鹤抚弄一番,以寄兴趣,方为不负了这琴。还有一
层,又要指法好,取音好。若必要抚琴,先须衣冠整齐,或鹤氅或深衣,要如古人
的象表,那才能称圣人之器。然后盥了手,焚上香,方才将身就在榻边,把琴放在
案上,坐在第五徽的地方儿,对着自己的当心,两手方从容抬起:这才心身俱正。
还要知道轻重疾徐、卷舒自若、体态尊重方好。”宝玉道:“我们学着玩,若这么讲
究起来,那就难了。”

两个人正说着,只见紫鹃进来,看见宝玉,笑说道:“宝二爷今日这样高兴!”
宝玉笑道:“听见妹妹讲究的,叫人顿开茅塞,所以越听越爱听。”紫鹃道:“不是
这个高兴,说的是二爷到我们这边来的话。”宝玉道:“先时妹妹身上不舒服,我怕
闹的他烦。再者我又上学,因此显着就疏远了似的。”紫鹃不等说完,便道:“姑娘
也是才好。二爷既这么说,坐坐也该让姑娘歇歇儿了,别叫姑娘只是讲究劳神了。”
宝玉笑道:“可是我只顾爱听,也就忘了妹妹劳神了。”黛玉笑道:“说这些倒也开
心,也没有什么劳神的。只是怕我只管说,你只管不懂呢。”宝玉道:“横竖慢慢的
自然明白了。”说着,便站起来,道:“当真的妹妹歇歇儿罢。明儿我告诉三妹妹和
四妹妹去,叫他们都学起来,让我听。”黛玉笑道:“你也太受用了。即如大家学会
了抚起来,你不懂,可不是对——”黛玉说到那里,想起心上的事,便缩住口,不
肯往下说了。宝玉便笑着道:“只要你们能弹,我便爱听,也不管‘牛’不‘牛’
的了。”黛玉红了脸一笑,紫鹃雪雁也都笑了。

于是走出门来。只见秋纹带着小丫头,捧着一小盆兰花来,说:“太太那边有
人送了四盆兰花来。因里头有事,没有空儿玩他,叫给二爷一盆,林姑娘一盆。”
黛玉看时,却有几枝双朵儿的,心中忽然一动,也不知是喜是悲,便呆呆的呆看。
那宝玉此时却一心只在琴上,便说:“妹妹有了兰花,就可以做《猗兰操》了。”黛
玉听了,心里反不舒服。回到房中,看着花,想到:“草木当春,花鲜叶茂,想我
年纪尚小,便像三秋蒲柳。若是果能随愿,或者渐渐的好来。不然只恐似那花柳残
春,怎禁得风催雨送!”想到那里,不禁又滴下泪来。紫鹃在旁看见这般光景,却
想不出原故来:“方才宝玉在这里那么高兴,如今好好的看花,怎么又伤起心来?”
正愁着没法儿劝解,只见宝钗那边打发人来。

未知何事,下回分解。