\chapter{散馀资贾母明大义~复世职政老沐天恩}

话说贾政进内,见了枢密院各位大臣,又见了各位王爷。北静王道:“今日我
们传你来,有遵旨问你的事。”贾政急忙跪下。众大臣便问道:“你哥哥交通外官、
恃强凌弱、纵儿聚赌、强占良民妻女不遂逼死的事,你都知道么?”贾政回道:“犯
官自从主恩钦点学政任满后,查看赈恤,于上年冬底回家,又蒙堂派工程,后又任
江西粮道,题参回都,仍在工部行走,日夜不敢怠惰。一应家务,并未留心伺察,
实在糊涂。不能管教子侄,这就是辜负圣恩。只求主上重重治罪。”北静王据说转
奏。不多时传出旨来,北静王便述道:“主上因御史参奏贾赦交通外官,恃强凌弱
——据该御史指出平安州互相往来,贾赦包揽词讼——严鞫贾赦,据供平安州原系
姻亲来往,并未干涉官事,该御史亦不能指实。惟有倚势强索石呆子古扇一款是实
的,然系玩物,究非强索良民之物可比。虽石呆子自尽,亦系疯傻所致,与逼勒致
死者有间。今从宽将贾赦发往台站效力赎罪。所参贾珍强占良民妻女为妾不从逼死
一款,提取都察院原案,看得尤二姐实系张华指腹为婚未娶之妻,因伊贫苦自愿退
婚,尤二姐之母愿结贾珍之弟为妾,并非强占。再尤三姐自刎掩埋、并未报官一款,
查尤三姐原系贾珍妻妹,本意为伊择配,因被逼索定礼,众人扬言秽乱,以致羞忿
自尽,并非贾珍逼勒致死。但身系世袭职员,罔知法纪,私埋人命,本应重治,念
伊究属功臣后裔,不忍加罪,亦从宽革去世职,派往海疆效力赎罪。贾蓉年幼无干,
省释。贾政实系在外任多年,居官尚属勤慎,免治伊治家不正之罪。”

贾政听了,感激涕零,叩首不及,又叩求王爷代奏下忱。北静王道:“你该叩
谢天恩,更有何奏?”贾政道:“犯官仰蒙圣恩,不加大罪,又蒙将家产给还,实
在扪心惶愧。愿将祖宗遗受重禄,积馀置产,一并交官。”北静王道:“主上仁慈待
下,明慎用刑,赏罚无差。如今既蒙莫大深恩,给还财产,你又何必多此一奏?”
众官也说不必。贾政便谢了恩,叩谢了王爷出来,恐贾母不放心,急忙赶回。上下
男女人等不知传进贾政是何吉凶,都在外头打听,一见贾政回家,都略略的放心,
也不敢问。

只见贾政忙忙的走到贾母跟前,将蒙圣恩宽免的事细细告诉了一遍。贾母虽则
放心,只是两个世职革去,贾赦又往台站效力,贾珍又往海疆,不免又悲伤起来。
邢夫人尤氏听见这话,更哭起来。贾政便道:“老太太放心。大哥虽则台站效力,
也是为国家办事,不致受苦,只要办得妥当,就可复职。珍儿正是年轻,很该出力。
若不是这样,便是祖父的馀德亦不能久享。”说了些宽慰的话。贾母素来本不大喜
欢贾赦,那边东府贾珍究竟隔了一层,只有邢夫人尤氏痛哭不止。邢夫人想着:“家
产一空,丈夫年老远出,膝下虽有琏儿,又是素来顺他二叔的,如今都靠着二叔,
他两口子自然更顺着那边去了。独我一人孤苦伶仃,怎么好?”那尤氏本来独掌宁
府的家计,除了贾珍,也算是惟他为尊,又与贾珍夫妻相和;如今犯事远出,家财
抄尽,依住荣府,虽则老太太疼爱,终是依人门下。又兼带着佩凤偕鸾,那蓉儿夫
妇也还不能兴家立业。又想起:“二妹妹三妹妹都是琏二爷闹的,如今他们倒安然
无事,依旧夫妻完聚,只剩我们几个,怎么度日?”想到这里,痛哭起来。贾母不
忍,便问贾政道:“你大哥和珍儿现已定案,可能回家?蓉儿既没他的事,也该放出
来了。”贾政道:“若在定例呢,大哥是不能回家的。我已托人徇个私情,叫我大哥
同着侄儿回家,好置办行装,衙门内业已应了。想来蓉儿同着他爷爷父亲一起出来。
只请老太太放心,儿子办去。”

贾母又道:“我这几年老的不成人了,总没有问过家事。如今东府里是抄了去
了,房子入官不用说;你大哥那边,琏儿那里,也都抄了。咱们西府里的银库和东
省地土,你知道还剩了多少?他两个起身,也得给他们几千银子才好。”贾政正是没
法,听见贾母一问,心想着:“若是说明,又恐老太太着急;若不说明,不用说将
来,只现在怎样办法呢?”想毕,便回道:“若老太太不问,儿子也不敢说。如今
老太太既问到这里,现在琏儿也在这里,昨日儿子已查了:旧库的银子早已虚空,
不但用尽,外头还有亏空。现今大哥这件事,若不花银托人,虽说主上宽恩,只怕
他们爷儿两个也不大好,就是这项银子尚无打算。东省的地亩,早已寅年吃了卯年
的租儿了,一时也弄不过来,只好尽所有蒙圣恩没有动的衣服首饰折变了,给大哥
和珍儿作盘费罢了。过日的事只可再打算。”贾母听了,又急的眼泪直淌。说道:“怎
样着?咱们家到了这个田地了么?我虽没有经过,我想起我家向日比这里还强十倍,
也是摆了几年虚架子,没有出这样事,已经塌下来了,不消一二年就完了!据你说
起来,咱们竟一两年就不能支了?”贾政道:“若是这两个世俸不动,外头还有些
挪移。如今无可指称,谁肯接济?”说着,也泪流满面,“想起亲戚来,用过我们
的,如今都穷了;没有用过我们的,又不肯照应。昨日儿子也没有细查,只看了家
下的人丁册子,别说上头的钱一无所出,那底下的人也养不起许多。”

贾母正在忧虑,只见贾赦、贾珍、贾蓉一齐进来给贾母请安。贾母看这般光景,
一只手拉着贾赦,一只手拉着贾珍,便大哭起来。他两人脸上羞惭,又见贾母哭泣,
都跪在地下哭着说道:“儿孙们不长进,将祖上功勋丢了,又累老太太伤心,儿孙
们是死无葬身之地的了!”满屋中人看这光景,又一齐大哭起来。贾政只得劝解:“倒
先要打算他两个的使用。大约在家只可住得一两日,迟则人家就不依了。”老太太
含悲忍泪的说道:“你两个且各自同你们媳妇们说说话儿去罢。”又吩咐贾政道:“这
件事是不能久待的。想来外面挪移,恐不中用,那时误了钦限,怎么好?只好我替
你们打算罢了。就是家中如此乱糟糟的,也不是常法儿。”一面说着,便叫鸳鸯吩
咐去了。这里贾赦等出来,又与贾政哭泣了一会,都不免将从前任性、过后恼悔、
如今分离的话说了一会,各自夫妻们那边悲伤去了。贾赦年老,倒还撂的下;独有
贾珍与尤氏怎忍分离?贾琏贾蓉两个也只有拉着父亲啼哭。虽说是比军流减等,究
竟生离死别。这也是事到如此,只得大家硬着心肠过去。

却说贾母叫邢王二夫人同着鸳鸯等开箱倒笼,将做媳妇到如今积攒的东西都拿
出来,又叫贾赦、贾政、贾珍等一一的分派。给贾赦三千两,说:“这里现有的银
子你拿二千两去做你的盘费使用,留一千给大太太零用。这三千给珍儿:你只许拿
一千去,留下二千给你媳妇收着。仍旧各自过日子。房子还是一处住,饭食各自吃
罢。四丫头将来的亲事,还是我的事。只可怜凤丫头操了一辈子心,如今弄的精光,
也给他三千两,叫他自己收着,不许叫琏儿用。如今他还病的神昏气短,叫平儿来
拿去。这是你祖父留下的衣裳,还有我少年穿的衣服首饰,如今我也用不着了。男
的呢,叫大老爷、珍儿、琏儿、蓉儿拿去分了。女的呢,叫大太太、珍儿媳妇、凤
丫头拿了分去。这五百两银子交给琏儿,明年将林丫头的棺材送回南去。”分派定
了,又叫贾政道:“你说外头还该着账呢,这是少不得的,你叫拿这金子变卖偿还。
这是他们闹掉了我的。你也是我的儿子,我并不偏向。宝玉已经成了家,我下剩的
这些金银东西,大约还值几千银子,这是都给宝玉的了。珠儿媳妇向来孝顺我,兰
儿也好,我也分给他们些。这就是我的事情完了。”贾政等见母亲如此明断分晰,
俱跪下哭着说:“老太太这么大年纪,儿孙们没点孝顺,承受老祖宗这样恩典,叫
儿孙们更无地自容了。”贾母道:“别瞎说了。要不闹出这个乱儿来,我还收着呢。
只是现在家人太多,只有二老爷当差,留几个人就够了。你就吩咐管事的,将人叫
齐了,分派妥当。各家有人就罢了。譬如那时都抄了,怎么样呢?我们里头的,也
要叫人分派,该配人的配人,赏去的赏去。如今虽说这房子不入官,你到底把这园
子交了才是呢。那些地亩还交琏儿清理,该卖的卖,留的留,再不可支架子,做空
头。我索性说了罢:江南甄家还有几两银子,二太太那里收着,该叫人就送去罢。
倘或再有点事儿出来,可不是他们‘躲过了风暴又遭了雨’了么?”贾政本是不知
当家立计的人,一听贾母的话,一一领命,心想:“老太太实在真真是理家的人。
都是我们这些不长进的闹坏了。”

贾政见贾母劳乏,求着老太太歇歇养神。贾母又道:“我所剩的东西也有限,
等我死了,做结果我的使用。下剩的都给伏侍我的丫头。”贾政等听到这里,更加
伤感,大家跪下:“请老太太宽怀。只愿儿子们托老太太的福,过了些时,都邀了
恩眷,那时兢兢业业的治起家来,以赎前愆,奉养老太太到一百岁。”贾母道:“但
愿这样才好,我死了也好见祖宗。你们别打量我是享得富贵受不得贫穷的人哪!不
过这几年看着你们轰轰烈烈,我乐得都不管,说说笑笑,养身子罢了。那知道家运
一败,直到这样!若说外头好看,里头空虚,是我早知道的了,只是‘居移气,养
移体’,一时下不了台就是了。如今借此正好收敛,守住这个门头儿,不然,叫人
笑话。你还不知,只打量我知道穷了,就着急的要死。我心里是想着祖宗莫大的功
勋,无一日不指望你们比祖宗还强,能够守住也罢了。谁知他们爷儿两个做些什么
勾当!”

贾母正自长篇大论的说,只见丰儿慌慌张张的跑来回王夫人道:“今早我们奶
奶听见外头的事,哭了一场,如今气都接不上了,平儿叫我来回太太。”丰儿没有
说完,贾母听见,便问:“到底怎么样?”王夫人便代回道:“如今说是不大好。”
贾母起身道:“嗳!这些冤家,竟要磨死我了。”说着,叫人扶着,要亲自看去。贾
政急忙拦住劝道:“老太太伤了好一会子心,又分派了好些事,这会子该歇歇儿了。
就是孙子媳妇有什么事,叫媳妇瞧去就是了,何必老太太亲身过去呢?倘或再伤感
起来,老太太身上要有一点儿不好,叫做儿子的怎么处呢?”贾母道:“你们各自
出去,等一会子再进来,我还有话说。”贾政不敢多言,只得出来料理兄侄起身的
事,又叫贾琏挑人跟去。这里贾母才叫鸳鸯等派人拿了给凤姐的东西,跟着过来。

凤姐正在气厥。平儿哭的眼肿腮红,听见贾母带着王夫人等过来,疾忙出来迎
接。贾母便问:“这会子怎么样了?”平儿恐惊了贾母,便说:“这会子好些儿。”
说着,跟了贾母等进来,赶忙先走过来,轻轻的揭开帐子。凤姐开眼瞧着,只见贾
母进来,满心惭愧。先前原打量贾母等恼他,不疼他了,是死活由他的,不料贾母
亲自来瞧,心里一宽,觉那拥塞的气略松动些,便要扎挣坐起。贾母叫平儿按着:
“不用动。你好些么?”凤姐含泪道:“我好些了。只是从小儿过来,老太太、太
太怎么样疼我!那知我福气薄,叫神鬼支使的失魂落魄,不能够在老太太、太太跟
前尽点儿孝心,讨个好儿,还这样把我当人,叫我帮着料理家务,被我闹的七颠八
倒,我还有什么脸见老太太、太太呢?今日老太太、太太亲自过来,我更担不起了,
恐怕该活三天的又折了两天去了。”说着悲咽。贾母道:“那些事原是外头闹起来的,
与你什么相干?就是你的东西被人拿去,这也算不了什么呀。我带了好些东西给你,
你瞧瞧。”说着,叫人拿上来给他瞧。凤姐本是贪得无厌的人,如今被抄净尽,自
然愁苦,又恐人埋怨,正是几不欲生的时候。今见贾母仍旧疼他,王夫人也不嗔怪,
过来安慰他,又想贾琏无事,心下安放好些。便在枕上与贾母磕头,说道:“请老
太太放心。若是我的病托着老太太的福好了,我情愿自己当个粗使的丫头,尽心竭
力的伏侍老太太、太太罢!”贾母听他说的伤心,不免掉下泪来。

宝玉是从来没有经过这大风浪的,心下只知安乐、不知忧患的人,如今碰来碰
去,都是哭泣的事,所以他竟比傻子尤甚,见人哭他就哭。凤姐看见众人忧闷,反
倒勉强说几句宽慰贾母的话,求着:“请老太太、太太回去,我略好些过来磕头。”
说着,将头仰起。贾母叫平儿:“好生服侍。短什么,到我那里要去。”说着,带了
王夫人将要回到自己房中,只听见两三处哭声。贾母听着,实在不忍,便叫王夫人
散去,叫宝玉:“去见你大爷大哥,送一送就回来。”自己躺在榻上下泪。幸喜鸳鸯
等能用百样言语劝解,贾母暂且安歇。

不言贾赦等分离悲痛。那些跟去的人,谁是愿意的?不免心中抱怨,叫苦连天。
正是生离果胜死别,看者比受者更加伤心。好好的一个荣国府,闹到人嚎鬼哭。贾
政最循规矩,在伦常上也讲究的,执手分别后,自己先骑马赶至城外,举酒送行,
又叮咛了好些“国家轸恤勋臣,力图报称”的话。贾赦等挥泪分头而别。

贾政带了宝玉回家,未及进门,只见门上有好些人在那里乱嚷,说:“今日旨
意:将荣国公世职着贾政承袭。”那些人在那里要喜钱,门上人和他们分争,说:“是
本来的世职,我们本家袭了。有什么喜报?”那些人说道:“那世职的荣耀,比任
什么还难得,你们大老爷闹掉了,想要这个,再不能的了。如今圣人的恩典比天还
大,又赏给二老爷了,这是千载难逢的,怎么不给喜钱?”正闹着,贾政回家,门
上回了。虽则喜欢,究竟是哥哥犯事所致,反觉感极涕零,赶着进内告诉贾母。贾
母自然喜欢,拉着说了些勤黾报恩的话。王夫人正恐贾母伤心,过来安慰,听得世
职复还,也是欢喜。独有邢夫人尤氏心下悲苦,只不好露出来。

且说外面这些趋炎奉势的亲戚朋友,先前贾宅有事,都远避不来;今儿贾政袭
职,知圣眷尚好,大家都来贺喜。那知贾政纯厚性成,因他袭哥哥的职,心内反生
烦恼,只知感激天恩。于第二日进内谢恩,到底将赏还府第园子备折奏请入官。内
廷降旨不必,贾政才得放心回家,以后循分供职。

但是家计萧条,入不敷出。贾政又不能在外应酬。家人们见贾政忠厚,凤姐抱
病不能理家,贾琏的亏空一日重似一日,难免典房卖地。府内家人几个有钱的,怕
贾琏缠扰,都装穷躲事,甚至告假不来,各自另寻门路。独有一个包勇,虽是新投
到此,恰遇荣府坏事,他倒有些真心办事,见那些人欺瞒主子,便时常不忿。奈他
是个新来乍到的人,一句话也插不上,他便生气,每日吃了就睡。众人嫌他不肯随
和,便在贾政前说他终日贪杯生事,并不当差。贾政道:“随他去罢。原是甄府荐
来,不好意思。横竖家内添这一个人吃饭,虽说穷,也不在他一人身上。”并不叫
驱逐。众人又在贾琏跟前说他怎么样不好,贾琏此时也不敢自作威福,只得由他。

忽一日,包勇耐不过,吃了几杯酒,在荣府街上闲逛,见有两个人说话。那人
说道:“你瞧,这么个大府,前儿抄了家,不知如今怎么样了?”那人道:“他家怎
么能败?听见说,里头有位娘娘是他家的姑娘,虽是死了,到底有根基的。况且我
常见他们来往的都是王公侯伯,那里没有照应?就是现在的府尹,前任的兵部,是
他们的一家儿。难道有这些人还护庇不来么?”那人道:“你白住在这里!别人犹可,
独是那个贾大人更了不得。我常见他在两府来往,前儿御史虽参了,主子还叫府尹
查明实迹再办。你说他怎么样?他本沾过两府的好处,怕人说他回护一家儿,他倒
狠狠的踢了一脚,所以两府里才到底抄了。你说如今的世情还了得吗!”两人无心
说闲话,岂知旁边有人跟着听的明白。包勇心下暗想:“天下有这样人!但不知是我
们老爷的什么人?我若见了他,便打他一个死,闹出事来,我承当去。”那包勇正在
酒后胡思乱想,忽听那边喝道而来。包勇远远站着,只见那两人轻轻的说道:“这
来的就是那个贾大人了。”包勇听了,心里怀恨,趁着酒兴,便大声说道:“没良心
的男女!怎么忘了我们贾家的恩了?”雨村在轿内听得一个“贾”字,便留神观看,
见是一个醉汉,也不理会,过去了。

那包勇醉着,不知好歹,便得意洋洋回到府中,问起同伴,知是方才见的那位
大人是这府里提拔起来的,“他不念旧恩,反来踢弄咱们家里,见了他骂他几句,
他竟不敢答言。”那荣府的人本嫌包勇,只是主人不计较他,如今他又在外头惹祸,
正好趁着贾政无事,便将包勇喝酒闹事的话回了贾政。贾政此时正怕风波,听见家
人回禀,便一时生气,叫进包勇来数骂了几句,也不好深沉责罚他,便派去看园,
不许他在外行走。那包勇本是个直爽的脾气,投了主子,他便赤心护主,那知贾政
反倒听了别人的话骂他。他也不敢再辩,只得收拾行李往园中看守浇灌去了。

未知后事如何,且听下回分解。