\chapter{施毒计金桂自焚身~昧真禅雨村空遇旧}

话说贾琏到了王夫人那边,一一的说了。次日,到了部里,打点停妥,回来又
到王夫人那边将打点吏部之事告知王夫人。王夫人便道:“打听准了么?果然这样,
老爷也愿意,合家也放心。那外任何尝是做得的?不是这样回来,只怕叫那些混帐
东西把老爷的性命都坑了呢。”贾琏道:“太太怎么知道?”王夫人道:“自从你二
叔放了外任,并没有一个钱拿回来,把家里的倒掏摸了好些去了。你瞧那些跟老爷
去的人,他男人在外头不多几时,那些小老婆子们都金头银面的妆扮起来了,可不
是在外头瞒着老爷弄钱?你叔叔就由着他们闹去。要弄出事来,不但自己的官做不
成,只怕连祖上的官也要抹掉了呢。”贾琏道:“太太说的很是。方才我听见参了,
吓的了不得,直等打听明白才放心。也愿意老爷做个京官,安安逸逸的做几年,才
保得住一辈子的声名。就是老太太知道了,倒也是放心的。只要太太说的宽缓些。”
王夫人道:“我知道,你到底再去打听打听。”

贾琏答应了,才要出来,只见薛姨妈家的老婆子慌慌张张的走来,到王夫人里
间屋内,也没说请安,便道:“我们太太叫我来告诉这里的姨太太说:我们家了不
得了,又闹出事来了!”王夫人听了,便问:“闹出什么事来?”那婆子又说:“了
不得,了不得!”王夫人哼道:“糊涂东西!有紧要事你到底说呀。”婆子便说:“我
们家二爷不在家,一个男人也没有,这件事情出来,怎么办!要求太太打发几位爷
们去料理料理。”王夫人听着不懂,便着急道:“到底要爷们去干什么?”婆子道:
“我们大奶奶死了!”王夫人听了,啐道:“呸,那行子女人死就死了罢咧,也值的
大惊小怪的。”婆子道:“不是好好儿死的,是混闹死的。快求太太打发人去办办!”
说着就要走。王夫人又生气,又好笑,说:“这老婆子好混账。琏哥儿,倒不如你
去瞧瞧,别理那糊涂东西。”那婆子没听见打发人去,只听见说“别理他”,他便赌
气跑回去了。这里薛姨妈正在着急,再不见来。好容易那婆子来了,便问:“姨太
太打发谁来?”婆子叹说道:“人再别有急难事。什么好亲好眷,看来也不中用。
姨太太不但不肯照应我们,倒骂我糊涂。”薛姨妈听了,又气又急道:“姨太太不管,
你姑奶奶怎么说来着?”婆子道:“姨太太既不管,我们家的姑奶奶自然更不管了,
没有去告诉。”薛姨妈啐道:“姨太太是外人,姑娘是我养的,怎么不管?”婆子一
时省悟道:“是啊,这么着我还去。”

正说着,只见贾琏来了,给薛姨妈请了安,道了恼,回说:“我婶子知道弟妇
死了,问老婆子再说不明。着急的很,打发我来问个明白,还叫我在这里料理。该
怎么样,姨太太只管说了办去。”薛姨妈本来气的干哭,听见贾琏的话,便赶忙说:
“倒叫二爷费心。我说姨太太是待我最好的,都是这老货说不清,几乎误了事。请
二爷坐下,等我慢慢的告诉你。”便道:“不为别的事,为的是媳妇不是好死的。”
贾琏道:“想是为兄弟犯事,怨命死的?”薛姨妈道:“若这样倒好了。前几个月头
里,他天天赤脚蓬头的疯闹。后来听见你兄弟问了死罪,他虽哭了一场,以后倒擦
胭抹粉的起来。我要说他,又要吵个了不得,我总不理他。有一天,不知为什么来
要香菱去作伴儿。我说:‘你放着宝蟾,要香菱做什么?况且香菱是你不爱的,何苦
惹气呢?’他必不依。我没法儿,只得叫香菱到他屋里去。可怜香菱不敢违我的话,
带着病就去了。谁知道他待香菱很好。我倒喜欢,你大妹妹知道了说:‘只怕不是
好心罢?’我也不理会。头几天香菱病着,他倒亲手去做汤给他喝。谁知香菱没福,
刚端到跟前,他自己烫了手,连碗都砸了。我只说必要迁怒在香菱身上,他倒没生
气,自己还拿笤帚扫了,拿水泼净了地,仍旧两个人很好。昨儿晚上,又叫宝蟾去
做了两碗汤来,自己说和香菱一块儿喝。隔了一会子,听见他屋里闹起来,宝蟾急
的乱嚷,以后香菱也嚷着,扶着墙出来叫人。我忙着看去,只见媳妇鼻子眼睛里都
流出血来,在地下乱滚,两只手在心口里乱抓,两只脚乱蹬,把我就吓死了。问他
也说不出来,闹了一会子就死了。我瞧那个光景儿是服了毒的。宝蟾就哭着来揪香
菱,说他拿药药死奶奶了。我看香菱也不是这么样的人,再者他病的起还起不来,
怎么能药人呢?无奈宝蟾一口咬定,我的二爷,这叫我怎么办?只得硬着心肠叫老婆
子们把香菱捆了,交给宝蟾,便把房门反扣了。我和你二妹妹守了一夜,等府里的
门开了才告诉去的。二爷你是明白人,这件事怎么好?”贾琏道:“夏家知道了没
有?”薛姨妈道:“也得撕掳明白了,才好报啊。”贾琏道:“据我看起来,必要经
官才了的下来。我们自然疑在宝蟾身上,别人却说宝蟾为什么药死他们姑娘呢?若
说在香菱身上,倒还装得上。”

正说着,只见荣府的女人们进来说:“我们二奶奶来了。”贾琏虽是大伯子,因
从小儿见的,也不回避。宝钗进来见了母亲,又见了贾琏,便往里间屋里和宝琴坐
下。薛姨妈进来也将前事告诉了一遍。宝钗便说:“若把香菱捆了,可不是我们也
说是香菱药死的了么?妈妈说这汤是宝蟾做的,就该捆起宝蟾来问他呀。一面就该
打发人报夏家去,一面报官才是。”薛姨妈听见有理,便问贾琏。贾琏道:“二妹子
说的很是。报官还得我去托了刑部里的人,相验问口供的时候,方有照应。只是要
捆宝蟾放香菱,倒怕难些。”薛姨妈道:“并不是我要捆香菱,我恐怕香菱病中受冤
着急,一时寻死,又添了一条人命,才捆了交给宝蟾,也是个主意。”贾琏道:“虽
是这么说,我们倒帮了宝蟾了。若要放都放,要捆都捆,他们三个人是一处的。只
要叫人安慰香菱就是了。”薛姨妈便叫人开门进去。宝钗就派了带来的几个女人帮
着捆宝蟾。只见香菱已哭的死去活来。宝蟾反得意洋洋,以后见人要捆他,便乱嚷
起来,那禁得荣府的人吆喝着,也就捆了,竟开着门,好叫人看着。这里报夏家的
人已经去了。

那夏家先前不住在京里,因近年消索,又惦记女孩儿,新近搬进京来。父亲已
没,只有母亲,又过继了一个混账儿子,把家业都花完了,不时的常到薛家。那金
桂原是个水性人儿,那里守得住空房,况兼天天心里想念薛蝌,便有些饥不择食的
光景。无奈他这个干兄弟又是个蠢货,虽也有些知觉,只是尚未入港,所以金桂时
常回去,也帮贴他些银钱。这些时正盼金桂回家,只见薛家的人来,心里想着:“又
拿什么东西来了。”不料说这里的姑娘服毒死了,他就气的乱嚷乱叫。金桂的母亲
听见了,更哭喊起来,说:“好端端的女孩儿在他家,为什么服了毒呢!”哭着喊着
的,带了儿子,也等不得雇车,便要走来。那夏家本是买卖人家,如今没了钱,那
顾什么脸面,儿子头里走,他就跟了个破老婆子出了门,在街上哭哭啼啼的雇了一
辆车,一直跑到薛家。进门也不搭话,就“儿”一声“肉”一声的闹起。那时贾琏
到刑部去托人,家里只有薛姨妈、宝钗、宝琴,何曾见过这个阵仗儿,都吓的不敢
则声。要和他讲理,他也不听,只说:“我女孩儿在你家,得过什么好处?两口子朝
打暮骂,闹了几时,还不容他两口子在一处。你们商量着把我女婿弄在监里,永不
见面。你们娘儿们仗着好亲戚受用也罢了,还嫌他碍眼,叫人药死他,倒说是服毒!
他为什么服毒?”说着,直奔薛姨妈来。薛姨妈只得退后,说:“亲家太太!且瞧瞧
你女孩儿,问问宝蟾,再说歪话还不迟呢!”宝钗宝琴因外面有夏家的儿子,难以
出来拦护,只在里边着急。

恰好王夫人打发周瑞家的照看,一进门来,见一个老婆子指着薛姨妈的脸哭骂。
周瑞家的知道必是金桂的母亲,便走上来说:“这位是亲家太太么?大奶奶自己服毒
死的,与我们姨太太什么相干?也不犯这么遭塌呀。”那金桂的母亲问:“你是谁?”
薛姨妈见有了人,胆子略壮了些,便说:“这就是我们亲戚贾府里的。”金桂的母亲
便道:“谁不知道你们有仗腰子的亲戚,才能够叫姑爷坐在监里!如今我的女孩儿倒
白死了不成?”说着,便拉薛姨妈说:“你到底把我女孩儿怎么弄杀了?给我瞧瞧!”
周瑞家的一面劝说:“只管瞧去,不用拉拉扯扯。”把手只一推。夏家的儿子便跑进
来不依,道:“你仗着府里的势头儿来打我母亲么?”说着,便将椅子打去,却没
有打着。里头跟宝钗的人听见外头闹起来,赶着来瞧,恐怕周瑞家的吃亏,齐打伙
儿上去,半劝半喝。那夏家的母子,索性撒起泼来,说:“知道你们荣府的势头儿!
我们家的姑娘已经死了,如今也都不要命了!”说着,仍奔薛姨妈拚命。地下的人
虽多,那里挡得住,自古说的:“一人拚命,万夫莫当。”

正闹到危急之际,贾琏带了七八个家人进来,见是如此,便叫人先把夏家的儿
子拉出去,便说:“你们不许闹,有话好好儿的说。快将家里收拾收拾,刑部里头
的老爷们就来相验了。”金桂的母亲正在撒泼,只见来了一位老爷,几个在头里吆
喝,那些人都垂手侍立。金桂的母亲见这个光景,也不知是贾府何人。又见他儿子
已被众人揪住,又听见说刑部来验,他心里原想看见女孩儿的尸首,先闹个稀烂,
再去喊冤,不承望这里先报了官,也便软了些。薛姨妈已吓糊涂了,还是周瑞家的
回说:“他们来了也没去瞧瞧他们姑娘,便作践起姨太太来了。我们为好劝他,那
里跑进一个野男人,在奶奶们里头混撒村混打,这可不是没有王法了!”贾琏道:“这
会子不用和他讲理,等回来打着问他,说:男人有男人的地方儿,里头都是些姑娘
奶奶们。况且有他母亲还瞧不见他们姑娘么?他跑进来不是要打抢来了么!”家人们
做好做歹,压伏住了。周瑞家的仗着人多,便说:“夏太太,你不懂事!既来了,该
问个青红皂白。你们姑娘是自己服毒死了,不然就是宝蟾药死他主子了。怎么不问
明白,又不看尸首,就想讹人来了呢?我们就肯叫一个媳妇儿白死了不成?现在把宝
蟾捆着,因为你们姑娘必要点病儿,所以叫香菱陪着他,也在一个屋里住,故此两
个人都看守在那里。原等你们来眼看着刑部相验,问出道理来才是啊。”金桂的母
亲此时势孤,也只得跟着周瑞家的到他女孩儿屋里,只见满脸黑血,直挺挺的躺在
炕上,便叫哭起来。宝蟾见是他家的人来,便哭喊说:“我们姑娘好意待香菱,叫
他在一块儿住,他倒抽空儿药死我们姑娘!”那时薛家上下人等俱在,便齐声吆喝
道:“胡说!昨日奶奶喝了汤才药死的,这汤可不是你做的?”宝蟾道:“汤是我做
的,端了来,我有事走了。不知香菱起来放了些什么在里头,药死的。”金桂的母
亲没听完,就奔香菱,众人拦住。薛姨妈便道:“这样子是砒霜药的,家里决无此
物。不管香菱宝蟾,终有替他买的,回来刑部少不得问出来,才赖不去。如今把媳
妇权放平正,好等官来相验。”众婆子上来抬放。宝钗道:“都是男人进来,你们将
女人动用的东西检点检点。”只见炕褥底下有一个揉成团的纸包儿。金桂的母亲瞧
见,便拾起打开看时,并没有什么,便撩开了。宝蟾看见道:“可不是有了凭据了!
这个纸包儿我认得:头几天耗子闹的慌,奶奶家去找舅爷要的,拿回来搁在首饰匣
内。必是香菱看见了,拿来药死奶奶的。若不信,你们看看首饰匣里有没有了。”

金桂的母亲便依着宝蟾的话,取出匣子来,只有几支银簪子。薛姨妈便说:“怎
么好些首饰都没有了?”宝钗叫人打开箱柜,俱是空的,便道:“嫂子这些东西被
谁拿去?这可要问宝蟾。”金桂的母亲心里也虚了好些,见薛姨妈查问宝蟾,便说:
“姑娘的东西,他那里知道?”周瑞家的道:“亲家太太别这么说么。我知道宝姑
娘是天天跟着大奶奶的,怎么说不知道?”宝蟾见问得紧,又不好胡赖,只得说道:
“奶奶自己每每带回家去,我管得么?”众人便说:“好个亲家太太!哄着拿姑娘的
东西,哄完了叫他寻死来讹我们。好罢咧,回来相验,就是这么说。”宝钗叫人:“到
外头告诉琏二爷说:别放了夏家的人。”里头金桂的母亲忙了手脚,便骂宝蟾道:“小
蹄子,别嚼舌头了!姑娘几时拿东西到我家去?”宝蟾道:“如今东西是小,给姑娘
偿命是大。”宝琴道:“有了东西,就有偿命的人了。快请琏二哥哥问准了夏家的儿
子买砒霜的话,回来好回刑部里的话。”金桂的母亲着了急道:“这宝蟾必是撞见鬼
了,混说起来。我们姑娘何尝买过砒霜?要这么说,必是宝蟾药死了的!”宝蟾急的
乱嚷,说:“别人赖我也罢了,怎么你们也赖起我来呢?你们不是常和姑娘说,叫他
别受委屈,闹得他们家破人亡,那时将东西卷包儿一走,再配一个好姑爷。这个话
是有的没有?”金桂的母亲还未及答言,周瑞家的便接口说道:“这是你们家的人
说的,还赖什么呢?”金桂的母亲恨的咬牙切齿的骂宝蟾,说:“我待你不错呀,
为什么你倒拿话来葬送我呢?回来见了官,我就说是你药死姑娘的!”

宝蟾气的瞪着眼说:“请太太放了香菱罢,不犯着白害别人,我见官自有我的
话。”宝钗听出这个话头儿来了,便叫人反倒放开了宝蟾,说:“你原是个爽快人,
何苦白冤在里头?你有话,索性说了大家明白,岂不完了事了呢?”宝蟾也怕见官
受苦,便说:“我们奶奶天天抱怨说:‘我这样人,为什么碰着这个瞎眼的娘,不配
给二爷,偏给了这么个混账糊涂行子。要是能够和二爷过一天,死了也是愿意的。’
说到那里,便恨香菱。我起初不理会,后来看见和香菱好了,我只道是香菱怎么哄
转了。不承望昨儿的汤不是好意。”金桂的母亲接说道:“越发胡说了!若是要药香
菱,为什么倒药了自己呢?”宝钗便问道:“香菱,昨日你喝汤来着没有?”香菱
道:“头几天我病的抬不起头来,奶奶叫我喝汤,我不敢说不喝。刚要扎挣起来,
那碗汤已经洒了,倒叫奶奶收拾了个难,我心里很过不去。昨儿听见叫我喝汤,我
喝不下去,没有法儿,正要喝的时候儿,偏又头晕起来。见宝蟾姐姐端了去。我正
喜欢,刚合上眼,奶奶自己喝着汤,叫我尝尝,我便勉强也喝了两口。”宝蟾不待
说完便道:“是了!我老实说罢。昨儿奶奶叫我做两碗汤,说是和香菱同喝。我气不
过,心里想着:香菱那里配我做汤给他喝呢?我故意的一碗里头多抓了一把盐,记
了暗记儿,原想给香菱喝的。刚端进来,奶奶却拦着我叫外头叫小子们雇车,说今
日回家去。我出去说了回来,见盐多的这碗汤在奶奶跟前呢。我恐怕奶奶喝着咸,
又要骂我。正没法的时候,奶奶往后头走动,我眼错不见,就把香菱这碗汤换过来
了。也是合该如此。奶奶回来就拿了汤去到香菱床边,喝着说:‘你到底尝尝。’那
香菱也不觉咸,两个人都喝完了。我正笑香菱没嘴道儿,那里知道这死鬼奶奶要药
香菱,必定趁我不在,将砒霜撒上了,也不知道我换碗。这可就是天理昭彰,自害
自身了。”于是众人往前后一想,真正一丝不错,便将香菱也放了,扶着他仍旧睡
在床上。

不说香菱得放,且说金桂的母亲心虚事实,还想辩赖。薛姨妈等你言我语,反
要他儿子偿还金桂之命。正然吵嚷,贾琏在外嚷说:“不用多说了,快收拾停当。
刑部的老爷就到了。”此时惟有夏家母子着忙,想来总要吃亏的,不得已反求薛姨
妈道:“千不是,万不是,总是我死的女孩儿不长进。这也是他自作自受。要是刑
部相验,到底府上脸面不好看,求亲家太太息了这件事罢。”宝钗道:“那可使不得。
已经报了,怎么能息呢?”周瑞家的等人大家做好做歹的劝说:“若要息事,除非
夏亲家太太自己出去拦验,我们不提长短罢了。”贾琏在外也将他儿子吓住。他情
愿迎到刑部具结拦验,众人依允。薛姨妈命人买棺成殓,不提。

且说贾雨村升了京兆府尹,兼管税务。一日,出都查勘开垦地亩,路过知机县,
到了急流津,正要渡过彼岸,因待人夫,暂且停轿。只见村旁有一座小庙,墙壁坍
颓,露出几株古松,倒也苍老。雨村下轿,闲步进庙,但见庙内神像,金身脱落,
殿宇歪斜,旁有断碣,字迹模糊,也看不明白。意欲行至后殿,只见一株翠柏下荫
着一间茅庐,庐中有一个道士,合眼打坐。雨村走近看时,面貌甚熟,想着倒像在
那里见过的,一时再想不起来。从人便欲吆喝,雨村止住,徐步向前,叫一声“老
道”。那道士双眼略启,微微的笑道:“贵官何事?”雨村便道:“本府出都查勘事
件,路过此地,见老道静修自得,想来道行深通,意欲冒昧请教。”那道人说:“来
自有地,去自有方。”雨村知是有些来历的,便长揖请问:“老道从何处焚修,在此
结庐?此庙何名?庙中共有几人?或欲真修,岂无名山?或欲结缘,何不通衢?”那道
人道:“‘葫芦’尚可安身,何必名山结舍?庙名久隐,断碣犹存,形影相随,何须
修募?岂似那‘玉在椟中求善价,钗于匣内待时飞’之辈耶!”雨村原是个颖悟人,
初听见“葫芦”两字,后闻“钗玉”一对,忽然想起甄士隐的事来,重复将那道士
端详一回,见他容貌依然,便屏退从人,问道:“君家莫非甄老先生么?”那道人
微微笑道:“什么‘真’?什么‘假’?要知道‘真’即是‘假’,‘假’即是‘真’。”
雨村听说出“贾”字来,益发无疑,便从新施礼,道:“学生自蒙慨赠到都,托庇
获隽公车,受任贵乡,始知老先生超悟尘凡,飘举仙境。学生虽溯洄思切,自念风
尘俗吏,末由再睹仙颜,今何幸于此处相遇!求老仙翁指示愚蒙。倘荷不弃,京寓
甚近,学生当得供奉,得以朝夕聆教。”那道人也站起来回礼,道:“我于蒲团之外,
不知天地间尚有何物。适才尊官所言,贫道一概不解。”说毕依旧坐下。雨村复又
心疑:“想去若非士隐,何貌言相似若此?离别来十九载,面色如旧,必是修炼有成,
未肯将前身说破。但我既遇恩公,又不可当面错过。看来不能以富贵动之,那妻女
之私更不必说了。”想罢,又道:“仙师既不肯说破前因,弟子于心何忍!”正要下
礼,只见从人进来禀说:“天色将晚,快请渡河。”雨村正无主意,那道人道:“请
尊官速登彼岸,见面有期,迟则风浪顿起。果蒙不弃,贫道他日尚在渡头候教。”
说毕,仍合眼打坐。雨村无奈,只得辞了道人出庙。正要过渡,只见一人飞奔而来。

未知何人,下回分解。