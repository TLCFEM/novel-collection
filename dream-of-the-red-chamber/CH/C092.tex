\chapter{评女传巧姐慕贤良~玩母珠贾政参聚散}

话说宝玉从潇湘馆出来,连忙问秋纹道:“老爷叫我作什么?”秋纹笑道:“没
有叫。袭人姐姐叫我请二爷,我怕你不来,才哄你的。”宝玉听了,才把心放下,
因说:“你们请我也罢了,何苦来唬我?”说着,回到怡红院内。袭人便问道:“你
这好半天到那里去了?”宝玉道:“在林姑娘那边,说起姨妈家宝姐姐的事来,就
坐住了。”袭人又问道:“说些什么?”宝玉将打禅语的话述了一遍。袭人道:“你
们再没个计较。正经说些家常闲话儿,或讲究些诗句,也是好的,怎么又说到禅语
上了?又不是和尚。”宝玉道:“你不知道,我们有我们的禅机,别人是插不下嘴
去的。”袭人笑道:“你们参禅参翻了,又叫我们跟着打闷葫芦了。”宝玉道:“头
里我也年纪小,他也孩子气,所以我说了不留神的话,他就恼了。如今我也留神,
他也没有恼的了。只是他近来不常过来,我又念书,偶然到一处,好像生疏了似的。”
袭人道:“原该这么着才是。都长了几岁年纪了,怎么好意思还像小孩子时候的样
子?”

宝玉点头道:“我也知道。如今且不用说那个。我问你:老太太那里打发人来
说什么来着没有?”袭人道:“没有说什么。”宝玉道:“必是老太太忘了。明儿
不是十一月初一日么?年年老太太那里必是个老规矩,要办消寒会,齐打伙儿坐下
喝酒说笑。我今日已经在学房里告了假了。这会子没有信儿,明儿可是去不去呢?
若去了呢,白白的告了假;若不去,老爷知道了,又说我偷懒。”袭人道:“据我
说,你竟是去的是。才念的好些儿了,又想歇着。我劝你也该上点紧儿了。昨儿听
见太太说,兰哥儿念书真好,他打学房里回来,还各自念书作文章,天天晚上弄到
四更多天才睡。你比他大多了,又是叔叔,倘或赶不上他,又叫老太太生气。倒不
如明儿早起去罢。”麝月道:“这么冷天,已经告了假,又去,叫学房里说既这么
着就不该告假呀,显见的是告谎假脱滑儿。依我说,乐得歇一天。就是老太太忘记
了,咱们这里就不消寒了么?咱们也闹个会儿,不好么?”袭人道:“都是你起头
儿,二爷更不肯去了。”麝月道:“我也是乐一天是一天,比不得你要好名儿,使
唤一个月,再多得二两银子。”袭人啐道:“小蹄子儿,人家说正经话,你又来胡
拉混扯的了。”麝月道:“我倒不是混拉扯,我是为你。”袭人道:“为我什么?”
麝月道:“二爷上学去了,你又该咕嘟着嘴想着,巴不得二爷早些儿回来,就有说
有笑的了。这会子又假撇清,何苦呢!我都看见了。”

袭人正要骂他,只见老太太那里打发人来,说道:“老太太说了,叫二爷明儿
不用上学去呢。明儿请了姨太太来给他解闷,只怕姑娘们都来家里的。史姑娘、邢
姑娘、李姑娘们都请了,明儿来赴什么消寒会呢。”宝玉没有听完,便喜欢道:“可
不是?老太太最高兴的。明日不上学,是过了明路的了。”袭人也不便言语了。那
丫头回去。宝玉认真念了几天书,巴不得玩这一天,又听见薛姨妈过来,想着宝姐
姐自然也来,心里喜欢。便说:“快睡罢,明日早些起来。”于是一夜无话。

到了次日,果然一早到老太太那里请了安。又到贾政王夫人那里请了安,回明
了老太太今儿不叫上学,贾政也没言语,便慢慢退出来。走了几步,便一溜烟跑到
贾母房中。见众人都没来,只有凤姐那边的奶妈子,带了巧姐儿,跟着几个小丫头
过来,给老太太请了安,说:“我妈妈先叫我来请安,陪着老太太说说话儿。妈妈
回来就来。”贾母笑着道:“好孩子,我一早就起来了,等他们总不来。只有你二
叔叔来了。”那奶妈子便说:“姑娘,给叔叔请安。”巧姐便请了安。宝玉也问了
一声“妞妞好?”巧姐道:“昨夜听见我妈妈说,要请二叔叔去说话。”宝玉道:
“说什么?”巧姐道:“我妈妈说,跟着李妈认了几年字,不知道我认得不认得。
我说都认得。我认给妈妈瞧,妈妈说我瞎认,不信,说我一天尽子玩,那里认得。
我瞧着那些字也不要紧,就是那《女孝经》也是容易念的。妈妈说我哄他,要请二
叔叔得空儿的时候给我理理。”贾母听了,笑道:“好孩子,你妈妈是不认得字的,
所以说你哄他。明儿叫你二叔叔理给他瞧瞧他就信了。”宝玉道:“你认了多少字
了?”巧姐儿道:“认了二三千多字,念了一本《女孝经》,半个月头里又上了《列
女传》。”宝玉道:“你念了懂的吗?你要不懂,我倒是讲讲这个你听罢。”贾母
道:“做叔叔的也该讲给侄女儿听听。”

宝玉便道:“那文王后妃不必说了。那姜后脱簪待罪和齐国的无盐安邦定国,
是后妃里头的贤能的。”巧姐听了,答应个“是”。宝玉又道:“若说有才的,是
曹大姑、班婕妤、蔡文姬、谢道韫诸人。”巧姐问道:“那贤德的呢?”宝玉道:
“孟光的荆钗布裙,鲍宣妻的提瓮出汲,陶侃母的截发留宾:这些不厌贫的,就是
贤德了。”巧姐欣然点头。宝玉道:“还有苦的,像那乐昌破镜,苏蕙回文;那孝
的,木兰代父从军,曹娥投水寻尸等类,也难尽说。”巧姐听到这些,却默默如有
所思。宝玉又讲那曹氏的引刀割鼻及那些守节的,巧姐听着更觉肃敬起来。宝玉恐
他不自在,又说:“那些艳的,如王嫱、西子、樊素、小蛮、绛仙、文君、红拂,
都是女中的——”尚未说出,贾母见巧姐默然,便说:“够了,不用说了。讲的太
多,他那里记得。”巧姐道:“二叔叔才说的,也有念过的,也有没念过的。念过
的一讲我更知道好处了。”宝玉道:“那字是自然认得的,不用再理了。”

巧姐道:“我还听见我妈妈说:我们家的小红,头里是二叔叔那里的,我妈妈
要了来,还没有补上人呢。我妈妈想着要把什么柳家的五儿补上,不知二叔叔要不
要。”宝玉听了更喜欢,笑着道:“你听你妈妈的话!要补谁就补谁罢咧,又问什
么要不要呢。”因又向贾母笑道:“我瞧大妞妞这个小模样儿,又有这个聪明儿,
只怕将来比凤姐姐还强呢,又比他认的字。”贾母道:“女孩儿家认得字也好,只
是女工针黹倒是要紧的。”巧姐儿道:“我也跟着刘妈妈学着做呢。什么扎花儿咧,
拉锁子咧,我虽弄不好,却也学着会做几针儿。”贾母道:“咱们这样人家,固然
不仗着自己做,但只到底知道些,日后才不受人家的拿捏。”巧姐答应着“是”,
还要宝玉解说《列女传》,见宝玉呆呆的,也不好再问。你道宝玉呆的是什么?只
因柳五儿要进怡红院,头一次是他病了,不能进来,第二次王夫人撵了晴雯,大凡
有些姿色的,都不敢挑。后来又在吴贵家看晴雯去,五儿跟着他妈给晴雯送东西去,
见了一面,更觉娇娜妩媚。今日亏得凤姐想着,叫他补入小红的窝儿,竟是喜出望
外了,所以呆呆的呆想。

贾母等着那些人,见这时候还不来,又叫丫头去请。回来李纨同着他妹子、探
春、惜春、史湘云、黛玉都来了。大家请了贾母的安,众人厮见。独有薛姨妈未到,
贾母又叫请去。果然薛姨妈带着宝琴过来。宝玉请了安,问了好,只不见宝钗邢岫
烟二人。黛玉便问起:“宝姐姐为何不来?”薛姨妈假说身上不好。邢岫烟知道薛
姨妈在坐,所以不来。宝玉虽见宝钗不来,心中纳闷,因黛玉来了,便把想宝钗的
心暂且搁开。不多时,邢王二夫人也来了。凤姐听见婆婆们先到了,自己不好落后,
只得打发平儿先来告假,说是:“正要过来,因身上发热,过一回儿就来。”贾母
道:“既是身上不好,不来也罢。咱们这时候很该吃饭了。”丫头们把火盆往后挪
了一挪,就在贾母榻前一溜摆下两桌,大家序次坐下。吃了饭,依旧围炉闲谈,不
须多赘。

且说凤姐因何不来?头里为着倒比邢王二夫人迟了不好意思,后来旺儿家的来
回说:“迎姑娘那里打发人来请奶奶安,还说并没有到上头,只到奶奶这里来。”
凤姐听了纳闷,不知又是什么事,便叫那人进来,问:“姑娘在家好?”那人道:
“有什么好的。奴才并不是姑娘打发来的,实在是司棋的母亲要我来求奶奶的。”
凤姐道:“司棋已经出去了,为什么来求我?”那人道:“自从司棋出去,终日啼
哭。忽然那一日,他表兄来了。他母亲见了,恨的什么儿似的,说他害了司棋,一
把拉住要打。那小子不敢言语。谁知司棋听见了,急忙出来,老着脸,和他母亲说:
‘我是为他出来的,我也恨他没良心。如今他来了,妈要打他,不如勒死了我罢。’
他妈骂他:‘不害臊的东西,你心里要怎么样?’司棋说道:‘一个女人嫁一个男
人。我一时失脚,上了他的当,我就是他的人了,决不肯再跟着别人的。我只恨他
为什么这么胆小,一身作事一身当,为什么逃了呢?就是他一辈子不来,我也一辈
子不嫁人的。妈要给我配人,我原拚着一死。今儿他来了,妈问他怎么样。要是他
不改心,我在妈跟前磕了头,只当是我死了,他到那里,我跟到那里,就是讨饭吃
也是愿意的。’他妈气的了不得,便哭着骂着说:‘你是我的女儿,我偏不给他,
你敢怎么着?’那知道司棋这东西糊涂,便一头撞在墙上,把脑袋撞破,鲜血流出,
竟碰死了。他妈哭着,救不过来,便要叫那小子偿命。他表兄也奇,说道:‘你们
不用着急。我在外头原发了财,因想着他才回来的,心也算是真了。你们要不信,
只管瞧。’说着,打怀里掏出一匣子金珠首饰来。他妈妈看见了,心软了,说:‘你
既有心,为什么总不言语?’他外甥道:‘大凡女人都是水性杨花,我要说有钱,
他就是贪图银钱了。如今他这为人就是难得的。我把首饰给你们,我去买棺盛殓
他。’那司棋的母亲接了东西,也不顾女孩儿了,由着外甥去。那里知道他外甥叫
人抬了两口棺材来。司棋的母亲看见诧异,说怎么棺材要两口,他外甥笑道:‘一
口装不下,得两口才好。’司棋的母亲见他外甥又不哭,只当是他心疼的傻了。岂
知他忙着把司棋收拾了,也不啼哭,眼错不见,把带的小刀子往脖子里一抹,也就
抹死了。司棋的母亲懊悔起来,倒哭的了不得。如今坊里知道了,要报官。他急了,
央我来求奶奶说个人情,他再过来给奶奶磕头。”

凤姐听了,诧异道:“那有这样傻丫头,偏偏的就碰见这个傻小子!怪不得那
一天翻出那些东西来,他心里没事人似的,敢只是这么个烈性孩子。论起来我也没
这么大工夫管他这些闲事,但只你才说的,叫人听着怪可怜见儿的。也罢了,你回
去告诉他,我和你二爷说,打发旺儿给他撕掳就是了。”凤姐打发那人去了,才过
贾母这边来,不提。

且说贾政这日正与詹光下大棋,通局的输赢也差不多,单为着一只角儿死活未
分,在那里打结。门上的小厮进来回道:“外面冯大爷要见老爷。”贾政说:“请
进来。”小厮出去请了,冯紫英走进门来,贾政即忙迎着。冯紫英进来,在书房中
坐下,见是下棋,便道:“只管下棋,我来观局。”詹光笑道:“晚生的棋是不堪
瞧的。”冯紫英道:“好说,请下罢。”贾政道:“有什么事么?”冯紫英道:“没
有什么话。老伯只管下棋,我也学几着儿。”贾政向詹光道:“冯大爷是我们相好
的,既没事,我们索性下完了这一局再说话儿。冯大爷在旁边瞧着。”冯紫英道:
“下采不下采?”詹光道:“下采的。”冯紫英道:“下采的是不好多嘴的。”贾
政道:“多嘴也不妨,横竖他输了十来两银子,终久是不拿出来的。往后只好罚他
做东便了。”詹光笑道:“这倒使得。”冯紫英道:“老伯和詹公对下么?”贾政
笑道:“从前对下,他输了;如今让他两个子儿,他又输了。时常还要悔几着,不
叫他悔他就急了。”詹光也笑道:“没有的事。”贾政道:“你试试瞧。”大家一
面说笑,一面下完了。做起棋来,詹光还了棋头,输了七个子儿。冯紫英道:“这
盘总吃亏在打结里头。老伯结少,就便宜了。”

贾政对冯紫英道:“有罪,有罪,咱们说话儿罢。”冯紫英道:“小侄与老伯
久不见面。一来会会,二来因广西的同知进来引见,带了四种洋货,可以做得贡的。
一件是围屏,有二十四扇子,都是紫檀雕刻的。中间虽说不是玉,却是绝好的硝
子石,石上镂出山水、人物、楼台、花鸟儿来。一扇上有五六十个人,都是宫妆的
女子,名为‘汉宫春晓’。人的眉、目、口、鼻以及出手、衣褶,刻得又清楚,又
细腻。点缀布置,都是好的。我想尊府大观园中正厅上恰好用的着。还有一架钟表,
有三尺多高,也是一个童儿拿着时辰牌,到什么时候儿就报什么时辰。里头还有消
息人儿打十番儿。这是两件重笨的,却还没有拿来。现在我带在这里的两件,却倒
有些意思儿。”就在身边拿出一个锦匣子来,用几重白绫裹着。揭开了绵子,第一
层是一个玻璃盒子,里头金托子大红绉绸托底,上放着一颗桂圆大的珠子,光华耀
目。冯紫英道:“据说这就叫做‘母珠’。”因叫:“拿一个盘儿来。”詹光即忙
端过一个黑漆茶盘,道:“使得么?”冯紫英道:“使得。”便又向怀里掏出一个
白绢包儿,将包儿里的珠子都倒在盘里散着,把那颗母珠搁在中间,将盘放于桌上。
看见那些小珠子儿滴溜滴溜的都滚到大珠子身边,回来把这颗大珠子抬高了,别处
的小珠子一颗也不剩,都粘在大珠上。詹光道:“这也奇!”贾政道:“这是有的,
所以叫做‘母珠’,原是珠之母。”

那冯紫英又回头看着他跟来的小厮道:“那个匣子呢?”小厮赶忙捧过一个花
梨木匣子来。大家打开看时,原来匣内衬着虎纹锦,锦上叠着一束蓝纱。詹光道:
“这是什么东西?”冯紫英道:“这叫做‘鲛绡帐’。”在匣子里拿出来时,叠得
长不满五寸,厚不上半寸。冯紫英一层一层的打开,打到十来层,已经桌上铺不下
了。冯紫英道:“你看,里头还有两褶,必得高屋里去才张得下。这就是鲛丝所织。
暑热天气张在堂屋里头,苍蝇蚊子一个不能进来,又轻又亮。”贾政道:“不用全
打开,怕叠起来倒费事。”詹光便与冯紫英一层一层折好收拾了。

冯紫英道:“这四件东西,价儿也不贵,两万银他就卖。母珠一万,鲛绡帐五
千,‘汉宫春晓’与自鸣钟五千。”贾政道:“那里买的起!”冯紫英道:“你们
是个国戚,难道宫里头用不着么?”贾政道:“用得着的很多,只是那里有这些银
子?等我叫人拿进去给老太太瞧瞧。”冯紫英道:“很是。”

贾政便着人叫贾琏把这两件东西送到老太太那边去,并叫人请了邢王二夫人、
凤姐儿都来瞧着,又把两件东西一一试过。贾琏道:“他还有两件:一件是围屏,
一件是乐钟。共总要卖二万银子呢。”凤姐儿接着道:“东西自然是好的,但是那
里有这些闲钱?咱们又不比外任督抚要办贡。我已经想了好些年了,像咱们这种人
家,必得置些不动摇的根基才好:或是祭地,或是义庄,再置些坟屋。往后子孙遇
见不得意的事,还是点儿底子,不到一败涂地。我的意思是这样,不知老太太、老
爷、太太们怎么样?若是外头老爷们要买只管买。”贾母与众人都说:“这话说的
倒也是。”贾琏道:“还了他罢。原是老爷叫我送给老太太瞧,为的是宫里好进,
谁说买来搁在家里?老太太还没开口,你便说了一大堆丧气话。”说着,便把两件
东西拿出去了,告诉贾政,只说:“老太太不要。”便与冯紫英道:“这两件东西
好可好,就只没银子。我替你留心,有要买的人我便送信给你去。”冯紫英只得收
拾好了,坐下说些闲话,没有兴头,就要起身。贾政道:“你在这里吃了晚饭去罢。”
冯紫英道:“罢了,来了就叨搅老伯吗?”贾政道:“说那里的话。”

正说着,人回:“大老爷来了。”贾赦早已进来。彼此相见,叙些寒温。不一
时摆上酒来,肴馔罗列,大家喝着酒。至四五巡后,说起洋货的话。冯紫英道:“这
种货本是难消的。除非要像尊府这样人家还可消得,其馀就难了。”贾政道:“这
也不见得。”贾赦道:“我们家里也比不得从前了,这回儿也不过是个空门面。”
冯紫英又问:“东府珍大爷可好么?我前儿见他,说起家常话儿来,提到他令郎续
娶的媳妇远不及头里那位秦氏奶奶了。如今后娶的到底是那一家的?我也没有问
起。”贾政道:“我们这个侄孙媳妇儿也是这里大家,从前做过京畿道的胡老爷的
女孩儿。”冯紫英道:“胡道长我是知道的,但是他家教上也不怎么样。也罢了,
只要姑娘好就好。”

贾琏道:“听得内阁里人说起,雨村又要升了。”贾政道:“这也好。不知准
不准?”贾琏道:“大约有意思的了。”冯紫英道:“我今儿从吏部里来,也听见
这样说。雨村老先生是贵本家不是?”贾政道:“是。”冯紫英道:“是有服的,
还是无服的?”贾政道:“说也话长。他原籍是浙江湖州府人,流寓到苏州,甚不
得意。有个甄士隐和他相好,时常周济他。以后中了进士,得了榜下知县,便娶了
甄家的丫头。如今的太太不是正配。岂知甄士隐弄到零落不堪,没有找处。雨村革
了职以后,那时还与我家并未相识,只因舍妹丈林如海林公在扬州巡盐的时候,请
他在家做西席,外甥女儿是他的学生。因他有起复的信,要进京来,恰好外甥女儿
要上来探亲,林姑老爷便托他照应上来的,还有一封荐书托我吹嘘吹嘘。那时看他
不错,大家常会。岂知雨村也奇:我家世袭起,从‘代’字辈下来,宁荣两宅,人
口房舍,以及起居事宜,一概都明白。因此,遂觉得亲热了。”因又笑说道:“几
年间,门子也会钻了,由知府推升转了御史,不过几年,升了吏部侍郎,兵部尚书。
为着一件事降了三级,如今又要升了。”

冯紫英道:“人世的荣枯,仕途的得失,终属难定。”贾政道:“天下事都是
一个样的理哟。比如方才那珠子,那颗大的就像有福气的人似的,那些小的都托赖
着他的灵气护庇着。要是那大的没有了,那些小的也就没有收揽了。就像人家儿当
头人有了事,骨肉也都分离了,亲戚也都零落了,就是好朋友也都散了。转瞬荣枯,
真似春云秋叶一般。你想做官有什么趣儿呢?像雨村算便宜的了。还有我们差不多
的人家儿,就是甄家,从前一样功勋,一样世袭,一样起居,我们也是时常来往。
不多几年他们进京来,差人到我这里请安,还很热闹。一会儿抄了原籍的家财,至
今杳无音信。不知他近况若何,心下也着实惦记着。”贾赦道:“什么珠子?”贾
政同冯紫英又说了一遍给贾赦听。贾赦道:“咱们家是再没有事的。”冯紫英道:
“果然尊府是不怕的。一则里头有贵妃照应;二则故旧好,亲戚多;三则你们家自
老太太起,至于少爷们,没有一个刁钻刻薄的。”贾政道:“虽无刁钻刻薄的,却
没有德行才情。白白的衣租食税,那里当得起?”贾赦道:“咱们不用说这些话,
大家吃酒罢。”

大家又喝了几杯,摆上饭来。吃毕喝茶,冯家的小厮走来,轻轻的向紫英说了
一句。冯紫英便要告辞。贾赦问那小厮道:“你说什么?”小厮道:“外面下雪,
早已下了梆子了。”贾政叫人看时,已是雪深一寸多了。贾政道:“那两件东西,
你收拾好了么?”冯紫英道:“收好了。若尊府要用,价钱还自然让些。”贾政道:
“我留神就是了。”紫英道:“我再听信罢。天气冷,请罢,别送了。”贾赦贾政
便命贾琏送了出去。

未知后事如何,下回分解。