\chapter{中乡魁宝玉却尘缘~沐皇恩贾家延世泽}

话说莺儿见宝玉说话,摸不着头脑,正自要走,只听宝玉又说道:“傻丫头,
我告诉你罢。你姑娘既是有造化的,你跟着他,自然也是有造化的了。你袭人姐姐
是靠不住的。只要往后你尽心伏侍他就是了,日后或有好处,也不枉你跟着他熬了
一场。”莺儿听着前头像话,后头说的又有些不像了,便道:“我知道了。姑娘还等
我呢。二爷要吃果子时,打发小丫头叫我就是了。”宝玉点头,莺儿才去了。一时,
宝钗袭人回来,各自房中去了,不提。

且说过了几天,便是场期。别人只知盼望他爷儿两个作了好文章;便可以高中
的了,只有宝钗见宝玉的工课虽好,只是那有意无意之间,却别有一种冷静的光景。
知他要进场了,头一件,叔侄两个都是初次赴考,恐人马拥挤,有什么失闪;第二
件,宝玉自和尚去后,总不出门,虽然见他用功喜欢,只是改的太速太好了,反倒
有些信不及,只怕又有什么变故。所以进场的头一天,一面派了袭人带了小丫头们
同着素云等给他爷儿两个收拾妥当,自己又都过了目,好好的搁起,预备着;一面
过来同李纨回了王夫人,拣家里老成的管事的多派了几个,只说怕人马拥挤碰了。

次日,宝玉贾兰换了半新不旧的衣服,欣然过来见了王夫人。王夫人嘱咐道:
“你们爷儿两个都是初次下场,但是你们活了这么大,并不曾离开我一天。就是不
在我跟前,也是丫头媳妇们围着,何曾自己孤身睡过一夜?今日各自进去,孤孤凄
凄,举目无亲,须要自己保重。早些作完了文章出来,找着家人,早些回来,也叫
你母亲、媳妇们放心。”王夫人说着,不免伤起心来。贾兰听一句答应一句。只见
宝玉一声不哼,待王夫人说完了,走过来给王夫人跪下,满眼流泪,磕了三个头,
说道:“母亲生我一世,我也无可答报。只有这一入场,用心作了文章,好好的中
个举人出来,那时太太喜欢喜欢,便是儿子一辈子的事也完了,一辈子的不好也都
遮过去了。”王夫人听了,更觉伤心,便道:“你有这个心,自然是好的,可惜你老
太太不能见你的面了!”一面说,一面哭着拉他。那宝玉只管跪着,不肯起来,便
说道:“老太太见与不见,总是知道的,喜欢的。既能知道了喜欢了,便是不见也
和见了的一样。只不过隔了形质,并非隔了神气啊。”

李纨见王夫人和他如此,一则怕勾起宝玉的病来,二则也觉得光景不大吉祥,
连忙过来说道:“太太,这是大喜的事,为什么这样伤心?况且宝兄弟近来很知好歹,
很孝顺,又肯用功。只要带了侄儿进去,好好的作文章,早早的回来,写出来请咱
们的世交老先生们看了,等着爷儿两个都报了喜,就完了。”一面叫人搀起宝玉来。
宝玉却转过身来给李纨作了个揖,说:“嫂子放心,我们爷儿两个都是必中的。日
后兰哥还有大出息,大嫂子还要带凤冠穿霞帔呢。”李纨笑道:“但愿应了叔叔的话,
也不枉——”说到这里,恐怕又惹起王夫人的伤心来,连忙咽住了。宝玉笑道:“只
要有了个好儿子,能够接续祖基,就是大哥哥不能见,也算他的后事完了。”李纨
见天气不早了,也不肯尽着和他说话,只好点点头儿。

此时宝钗听得,早已呆了。这些话不但宝玉说的不好,便是王夫人李纨所说,
句句都是不祥之兆,却又不敢认真,只得忍泪无言。那宝玉走到跟前,深深的作了
一个揖。众人见他行事古怪,也摸不着是怎么样,又不敢笑他。只见宝钗的眼泪直
流下来,众人更是纳罕。又听宝玉说道:“姐姐,我要走了。你好生跟着太太,听
我的喜信儿罢!”宝钗道:“是时候了,你不必说这些唠叨话了。”宝玉道:“你倒催
的我紧,我自己也知道该走了!”回头见众人都在这里,只没惜春紫鹃,便说道:“四
妹妹和紫鹃姐姐跟前,替我说罢。他们两个横竖是再见的。”

众人见他的话,又像有理,又像疯话。大家只说他从来没出过门,都是太太的
一套话招出来的,不如早早催他去了就完了事了,便说道:“外面有人等你呢,你
再闹就误了时辰了。”宝玉仰面大笑道:“走了,走了!不用胡闹了,完了事了!”众
人也都笑道:“快走罢!”独有王夫人和宝钗娘儿两个倒像生离死别的一般,那眼泪
也不知从那里来的,直流下来,几乎失声哭出。但见宝玉嘻天哈地,大有疯傻之状,
遂从此出门而去。正是:
走来名利无双地,打出樊笼第一关。

不言宝玉贾兰出门赴考,且说贾环见他们考去,自己又气又恨,便自大为王,
说:“我可要给母亲报仇了。家里一个男人没有,上头大太太依了我,还怕谁!”想
定了主意,跑到邢夫人那边请了安,说了些奉承的话。那邢夫人自然喜欢,便说道:
“你这才是明理的孩子呢。像那巧姐儿的事,原该我作主的。你琏二哥糊涂,放着
亲奶奶倒托别人去。”贾环道:“人家那头儿也说了:只认得这一门子,现在定了,
还要备一分大礼来送太太呢。如今太太有了这样的藩王孙女女婿,还怕大老爷没大
官做么?不是我说自己的太太,他们有了元妃姐姐,便欺压的人难受!将来巧姐儿别
也是这样没良心,等我去问问他。”邢夫人道:“你也该告诉他,他才知道你的好处。
只怕他父亲在家也找不出这么门子好亲事来。但只平儿那个糊涂东西,他倒说这件
事不好,说是你太太也不愿意。想来恐怕我们得了意。若迟了,你二哥回来,又听
人家的话,就办不成了。”贾环道:“那边都定了,只等太太出了八字。王府的规矩,
三天就要来娶的。但是一件,只怕太太不愿意:那边说是不该娶犯官的孙女,只好
悄悄的抬了去;等大老爷免了罪,做了官,再大家热闹起来。”邢夫人道:“这有什
么不愿意?也是礼上应该的。”贾环道:“既这么着,这帖子太太出了就是了。”邢夫
人道:“这孩子又糊涂了!里头都是女人,你叫蔷哥儿写了一个就是了。”贾环听说,
喜欢的了不得,连忙答应了出来。赶着和贾芸说了,邀着王仁到那外藩公馆立文书、
兑银子去了。

那知刚才所说的话早被跟邢夫人的丫头听见。那丫头是求了平儿才挑上的,便
抽空儿赶到平儿那里,一五一十的都告诉了。平儿早知此事不好,已和巧姐细细的
说明。巧姐哭了一夜,必要等他父亲回来作主,大太太的话不能遵;今儿又听见这
话,便大哭起来,要和太太讲去。平儿急忙拦住道:“姑娘且慢着。大太太是你的
亲祖母,他说二爷不在家,大太太做得主的,况且还有舅舅做保山。他们都是一气,
姑娘一个人,那里说得过呢?我到底是下人,说不上话去。如今只可想法儿,断不
可冒失的。”邢夫人那边的丫头道:“你们快快的想主意,不然可就要抬走了!”说
着各自去了。

平儿回过头来,见巧姐哭作一团,连忙扶着道:“姑娘,哭是不中用的。如今
是二爷彀不着。听见他们的话头——”这句话还没说完,只见邢夫人那边打发人来
告诉:“姑娘大喜的事来了!叫平儿将姑娘所有应用的东西料理出来。若是赔送呢,
原说明了等二爷回来再办。”平儿只得答应了回来。又见王夫人过来。巧姐儿一把
抱住,哭得倒在怀里。王夫人也哭道:“妞儿不用着急。我为你吃了大太太好些话,
看来是扭不过来的。我们只好应着缓下去,即刻差个家人赶到你父亲那里去告诉。”
平儿道:“太太还不知道么?早起三爷在大太太跟前说了:什么外藩规矩,三日就要
过去的。如今大太太已叫芸哥儿写了名字年庚去了,还等得二爷么?”王夫人听说
是三爷,便气得话也说不出来,呆了半天,一叠声叫找贾环。找了半天,人回:“今
早同蔷哥儿王舅爷出去了。”王夫人问:“芸哥呢?”众人回说:“不知道。”巧姐屋
内人人瞪眼,都无方法。王夫人也难和邢夫人争论,只有大家抱头大哭。

正闹着,一个婆子进来回说:“后门上的人说,那个刘老老又来了。”王夫人道:
“咱们家遭了这样事,那有工夫接待人,不拘怎么回了他去罢。”平儿道:“太太该
叫他进来,他是姐儿的干妈,也得告诉告诉他。”王夫人不言语。那婆子便带了刘
老老进来。各人见了问好。刘老老见众人的眼圈儿通红,也摸不着头脑,迟了一会
子,问道:“怎么了?太太姑娘们必是想二姑奶奶了。”巧姐儿听见提起他母亲,越
发大哭起来。平儿道:“老老别说闲话。你既是姑娘的干妈,也该知道的。”便一五
一十的告诉了。把个刘老老也唬怔了,等了半天,忽然笑道:“你这样一个伶俐姑
娘,没听见过鼓儿词么?这上头的法儿多着呢,这有什么难的?”平儿赶忙问道:“老
老,你有什么法儿快说罢!”刘老老道:“这有什么难的呢,一个人也不叫他们知道,
扔崩一走就完了事了。”平儿道:“这可是混说了。我们这样人家的人,走到那里
去?”刘老老道:“只怕你们不走,你们要走,就到我屯里去。我就把姑娘藏起来,
即刻叫我女婿弄了人,叫姑娘亲笔写个字儿,赶到姑老爷那里,少不得他就来了,
可不好么?”平儿道:“大太太知道呢?”刘老老道:“我来他们知道么?”平儿道:
“大太太住在前头,他待人刻薄,有什么信,没人送给他的。你若前门走来,就知
道了;如今是后门来的,不妨事。”刘老老道:“咱们说定了几时,我叫女婿打了车
来接了去。”平儿道:“这还等得几时吗?你坐着罢。”急忙进去,将刘老老的话,避
了旁人告诉了。

王夫人想了半天不妥当。平儿道:“只好这样。为的是太太,才敢说明。太太
就装不知道,回来倒问大太太。我们那里就有人去,想二爷回来也快。”王夫人不
言语,叹了一口气。巧姐儿听见,便和王夫人道:“求太太救我!横竖父亲回来只有
感激的。”平儿道:“不用说了,太太回去罢。只要太太派人看屋子。”王夫人道:“掩
密些!你们两个人的衣服铺盖是要的啊。”平儿道:“要快走才中用呢,若是他们定
了回来,就有饥荒了。”一句话提醒了王夫人,便道:“是了,你们快办去罢,有我
呢。”

于是王夫人回去,倒过去找邢夫人说闲话儿,把邢夫人先绊住了。平儿这里便
遣人料理去了,嘱咐道:“倒别避人。有人进来看见,就说是大太太吩咐的,要一
辆车子送刘老老去。”这里又买嘱了看后门的人雇了车来。平儿便将巧姐装做青儿
模样,急急的去了。后来平儿只当送人,眼错不见,也跨上车去了。原来近日贾府
后门虽开,只有一两个人看着,馀外虽有几个家下人,因房大人少,空落落的,谁
能照应?且邢夫人又是个不怜下人的。家人明知此事不好,又都感念平儿的好处,
所以通同一气,放走了巧姐。邢夫人还自和王夫人说话,那里理会。只有王夫人甚
不放心,说了一回话,悄悄的走到宝钗那里坐下,心里还是惦记着。宝钗见王夫人
神色恍惚,便问:“太太的心里有什么事?”王夫人将这事背地里和宝钗说了。宝
钗道:“险得很!如今得快快儿的叫芸哥儿止住那里才妥当。”王夫人道:“我找不着
环儿呢。”宝钗道:“太太总要装作不知,等我想个人去叫大太太知道才好。”王夫
人点头,一任宝钗想人,暂且不言。

且说外藩原是要买几个使唤的女人,据媒人一面之辞,所以派人相看。相看的
人回去,禀明了藩王,藩王问起人家,众人不敢隐瞒,只得实说。那外藩听了,知
是世代勋戚,便说:“了不得,这是有干例禁的,几乎误了大事!况我朝觐已过,便
要择日起程。倘有人来再说,快快打发出去!”这日恰好贾芸王仁等递送年庚,只
见府门里头的人便说:“奉王爷的命说:敢拿贾府的人来冒充民女者,要拿住究治
的!如今太平时候,谁敢这样大胆?”这一嚷,唬得王仁等抱头鼠窜的出来,埋怨
那说事的人,大家扫兴而散。

贾环在家候信,又闻王夫人传唤,急得烦躁起来,见贾芸一人回来,赶着问道:
“定了么?”贾芸慌忙跺足道:“了不得,了不得,不知谁露了风了。”还把吃亏的
话说了一遍。贾环气得发怔,说:“我早起在大太太跟前说的这样好,如今怎么样
处呢?这都是你们众人坑了我了!”

正没主意,听见里头乱嚷,叫着贾环等的名字,说:“大太太二太太叫呢!”两
个人只得蹭进去。只见王夫人怒容满面,说:“你们干的好事!如今逼死了巧姐和平
儿了。快快的给我找还尸首来完事!”两个人跪下。贾环不敢言语。贾芸低头说道:
“孙子不敢干什么。为的是邢舅太爷和王舅爷说给巧妹妹作媒,我们才回太太们的。
大太太愿意,才叫孙子写帖儿去的。人家还不要呢,怎么我们逼死了妹妹呢?”王
夫人道:“环儿在大太太那里说的,三日内便要抬了走。说亲作媒,有这样的么?我
也不问,你们快把巧姐儿还了我们,等老爷回来再说!”邢夫人如今也是一句话儿
说不出了,只有落泪。王夫人便骂贾环说:“赵姨娘这样混帐东西,留的种子也是
这混帐的!”说着,叫丫头扶了,回到自己房中。

那贾环、贾芸、邢夫人三个人互相埋怨,说道:“如今且不用埋怨。想来死是
不死的,必是平儿带了他到那什么亲戚家躲着去了。”邢夫人叫了前后的门上人来
骂着,问:“巧姐儿和平儿,知道那里去了?”岂知下人一口同音,说是:“大太太
不必问我们,问当家的爷们就知道了。在大太太也不用闹,等我们太太问起来,我
们有话说。要打大家打,要发大家都发。自从琏二爷出了门,外头闹的还得了!我
们的月钱月米是不给了,赌钱喝酒,闹小旦,还接了外头的媳妇儿到宅里来,这不
是爷吗?”说得贾芸等顿口无言。王夫人那边又打发人来催说:“叫爷们快找来!”
那贾环等急得恨无地缝可钻,又不敢盘问巧姐那边的人。明知众人深恨,是必藏起
来了,但是这句话怎敢在王夫人面前说,只得各处亲戚家打听,毫无踪迹。里头一
个邢夫人,外头环儿等,这几天闹的昼夜不宁。

看看到了出场日期,王夫人只盼着宝玉贾兰回来。等到晌午,不见回来,王夫
人、李纨、宝钗着忙,打发人去到下处打听。去了一起,又无消息,连去的人也不
来了。回来又打发一起人去,又不见回来。三个人心里如热油熬煎。

等到傍晚,有人进来,见是贾兰。众人喜欢,问道:“宝二叔呢?”贾兰也不
及请安,便哭道:“二叔丢了!”王夫人听了这话,便怔了半天,也不言语,便直挺
挺的躺倒床上,亏得彩云等在后面扶着,下死的叫醒转来。哭着见宝钗,也是白瞪
两眼,袭人等已哭得泪人一般。只有哭着骂贾兰道:“糊涂东西!你同二叔在一处,
怎么他就丢了?”贾兰道:“我和二叔在下处是一处吃,一处睡,进了场相离也不
远,刻刻在一处的。今儿一早,二叔的卷子早完了,还等我呢。我们两个人一起去
交了卷子,一同出来,在龙门口一挤,回头就不见了。我们家接场的人都问我。李
贵还说:‘看见的,相离不过数步,怎么一挤就不见了?’现叫李贵等分头的找去。
我也带了人,各处号里都找遍了,没有,我所以这时候才回来。”

王夫人是哭的一句话也说不出来;宝钗心里已知八九;袭人痛哭不已;贾蔷等
不等吩咐,也是分头而去。可怜荣府的人,个个死多活少,空备了接场的酒饭。贾
兰也都忘了辛苦,还要自己找去。倒是王夫人拦住道:“我的儿,你叔叔丢了,还
禁得再丢了你么?好孩子你歇歇去罢。”贾兰那里肯走,尤氏等苦劝不止。众人中只
有惜春心里却明白了,只不好说出来,便问宝钗道:“二哥哥带了玉去了没有?”
宝钗道:“这是随身的东西,怎么不带?”惜春听了,便不言语。袭人想起那日抢
玉的事来,也是料着那和尚作怪,柔肠几断,珠泪交流,呜呜咽咽哭个不住,追想
当年宝玉相待的情分:“有时怄他,他便恼了,也有一种令人回心的好处,那温存
体贴,是不用说了。若怄急了他,便赌誓说做和尚。谁知今日却应了这句话了!”

不言袭人苦想,却说那天已是四更,并没个信儿。李纨怕王夫人苦坏了,极力
劝着回房。众人都跟着伺候,只有邢夫人回去。贾环躲着不敢出来。王夫人叫贾兰
去了,一夜无眠。次日天明,虽有家人回来,都说:“没有一处不寻到,实在没有
影儿。”于是薛姨妈、薛蝌、史湘云、宝琴、李婶娘等接二连三的过来请安问信。

如此一连数日,王夫人哭得饮食不进,命在垂危。忽有家人回道:“海疆来了
一人,口称统制大人那里来的,说我们家的三姑奶奶明日到京了。”王夫人听说探
春回京,虽不能解宝玉之愁,那个心略放了些。到了明日,果然探春回来。众人远
远接着,见探春出挑得比先前更好了,服采鲜明。看见王夫人形容枯槁,众人眼肿
腮红,便也大哭起来,哭了一会,然后行礼。看见惜春道姑打扮,心里很不舒服。
又听见宝玉心迷走失,家中多少不顺的事,大家又哭起来。还亏得探春能言,见解
亦高,把话来慢慢儿的劝解了好些时,王夫人等略觉好些。至次日,三姑爷也来了,
知有这样事,留探春住下劝解。跟探春的丫头老婆也与众姐妹们相聚,各诉别后情
事。从此,上上下下的人,竟是无昼无夜,专等宝玉的信。

那一夜五更多天,外头几个家人进来,到二门口报喜。几个小丫头乱跑进来,
也不及告诉大丫头了,进了屋子,便说:“太太奶奶们大喜!”王夫人打量宝玉找着
了,便喜欢的站起身来说:“在那里找着的?快叫他进来!”那人道:“中了第七名举
人。”王夫人道:“宝玉呢?”家人不言语。王夫人仍旧坐下。探春便问:“第七名
中的是谁?”家人回说:“是宝二爷。”正说着,外头又嚷道:“兰哥儿中了!”那家
人赶忙出去,接了报单回禀,见贾兰中了一百三十名。李纨心下自然喜欢,但因不
见了宝玉,不敢喜形于色。王夫人见贾兰中了,心下也是喜欢,只想:“若是宝玉
一回来,咱们这些人,不知怎样乐呢。”独有宝钗心下悲苦,又不好掉泪。众人道
喜,说是:“宝玉既有中的命,自然再不会丢的,不过再过两天,必然找的着。”王
夫人等想来不错,略有笑容,众人便趁势劝王夫人等多进了些饮食。只见三门外头
焙茗乱嚷说:“我们二爷中了举人,是丢不了的了。”众人问道:“怎么见得?”焙
茗道:“‘一举成名天下闻’,如今二爷走到那里,那里就知道的,谁敢不送来!”里
头的众人都说:“这小子虽是没规矩,这句话是不错的。”惜春道:“这样大人了,
那里有走失的?只怕他勘破世情,入了空门,这就难找着他了。”这句话又招的王夫
人等都大哭起来。李纨道:“古来成佛作祖成神仙的,果然把爵位富贵都抛了,也
多得很。”王夫人哭道:“他若抛了父母,这就是不孝,怎能成佛作祖?”探春道:
“大凡一个人,不可有奇处。二哥哥生来带块玉来,都道是好事,这么说起来,都
是有了这块玉的不好。若是再有几天不见,我不是叫太太生气:就有些原故了,只
好譬如没有生这位哥哥罢了。果然有来头成了正果,也是太太几辈子的修积。”宝
钗听了不言语。袭人那里忍得住,心里一疼,头上一晕,便栽倒了。王夫人看着可
怜,命人扶他回去。

贾环见哥哥侄儿中了,又为巧姐的事,大不好意思,只抱怨蔷芸两个。知道探
春回来,此事不肯干休,又不敢躲开,这几天竟是如在荆棘之中。

次日,贾兰只得先去谢恩,知道甄宝玉也中了,大家序了同年。提起贾宝玉心
迷走失,甄宝玉叹息劝慰。知贡举的将考中的卷子奏闻,皇上一一的披阅,看取中
的文章,俱是平正通达的。见第七名贾宝玉是金陵籍贯,第一百三十名又是金陵贾
兰,皇上传旨询问:“两个姓贾的是金陵人氏,是否贾妃一族?”大臣领命出来,
传贾宝玉贾兰问话。贾兰将宝玉场后迷失的话,并将三代陈明,大臣代为转奏。皇
上最是圣明仁德,想起贾氏功勋,命大臣查复。大臣便细细的奏明。皇上甚是悯恤,
命有司将贾赦犯罪情由,查案呈奏。皇上又看到“海疆靖寇班师善后事宜”一本,
奏的是“海宴河清,万民乐业”的事。皇上圣心大悦,命九卿叙功议赏,并大赦天
下。贾兰等朝臣散后,拜了座师,并听见朝内有大赦的信,便回了王夫人等。合家
略有喜色,只盼宝玉回来。薛姨妈更加喜欢,便要打算赎罪。

一日,人报甄老爷同三姑爷来道喜,王夫人便命贾兰出去接待。不多一时,贾
兰进来,笑嘻嘻的回王夫人道:“太太们大喜了。甄老爷在朝内听见有旨意,说是
大爷爷的罪名免了;珍大爷不但免了罪,仍袭了宁国三等世职。荣国世职,仍是爷
爷袭了,俟丁忧服满,仍升工部郎中。所抄家产,全行赏还。二叔的文章,皇上看
了甚喜。问知元妃兄弟,北静王还奏说人品亦好,皇上传旨召见。众大臣奏称:‘据
伊侄贾兰回称出场时迷失,现在各处寻访。’皇上降旨,着五营各衙门用心寻访。
这旨意一下,请太太们放心,皇上这样圣恩,再没有找不着的。”王夫人等这才大
家称贺,喜欢起来。

只有贾环等心下着急,四处找寻巧姐。那知巧姐随了刘老老,带着平儿出了城,
到了庄上,刘老老也不敢轻亵巧姐,便打扫上房,让给巧姐平儿住下。每日供给,
虽是乡村风味,倒也洁净;又有青儿陪着,暂且宽心。那庄上也有几家富户,知道
刘老老家来了贾府姑娘,谁不来瞧,都道是天上神仙。也有送菜果的,也有送野味
的,倒也热闹。内中有个极富的人家姓周,家财巨万,良田千顷,只有一子,生得
文雅清秀,年纪十四岁。他父母延师读书,新近科试,中了秀才。那日他母亲看见
巧姐,心里羡慕,自想:“我是庄家人家,那里配得起这样世家小姐?”只顾呆想。
刘老老早看出他的心事来,便说:“你的心事我知道了,我给你们做个媒罢。”周妈
妈笑道:“你别哄我。他们什么人家,肯给我们庄家人?”刘老老道:“说着瞧罢。”
于是两人各自走开。

刘老老惦记着贾府,叫板儿进城打听。那日恰好到宁荣街,只见有好些车轿在
那里,板儿便在邻近打听。说是:“宁荣两府复了官,赏还抄的家产,如今府里又
要起来了。只是他们的宝玉中了官,不知走到那里去了。”板儿心里喜欢,便要回
去。又见好几匹马到来,在门前下马,只见门上打千儿请安,说:“二爷回来了,
大喜!大老爷身上安了么?”那位爷笑着道:“好了,又遇恩旨,就要回来了。”还
问:“那些人做什么的?”门上回说:“是皇上派官在这里下旨意,叫人领家产。”
那位爷便喜喜欢欢的进去。板儿料是贾琏,也不再打听,赶忙回去告诉他外祖母。
刘老老听说,喜的眉开眼笑,去给巧姐儿道喜,将板儿的话说了一遍。平儿笑说道:
“可是亏了老老这样一办,不然,姑娘也摸不着这好时候儿了。”巧姐更自喜欢。
正说着,那送贾琏信的人也回来了,说是:“姑老爷感激得很,叫我一到家,快把
姑娘送回去。又赏了我好几两银子。”刘老老听了得意,便叫人赶了两辆车,请巧
姐平儿上车。巧姐等在刘老老家住熟了,反是依依不舍,更有青儿哭着,恨不能留
下。刘老老见他不忍相别,便叫青儿跟了进城,一径直奔荣府而来。

且说贾琏先前知道贾赦病重,赶到配所,父子相见,痛哭一场,渐渐的好起来。
贾琏接着家书,知道家中的事,禀明贾赦回来。走到中途,听得大赦,又赶了两天,
今日到家,恰遇颁赏恩旨。里面邢夫人等正愁无人接旨,虽有贾兰,终是年轻。人
报琏二爷回来,大家相见,悲喜交集。此时也不及叙话,即到前厅,叩见了。钦命
大人问了他父亲好,说:“明日到内府领赏。宁国府第,发交居住。”众人起身辞别。

贾琏送出门去,见有几辆屯车,家人们不许停歇,正在吵闹。贾琏早知道是巧
姐来的车,便骂家人道:“你们这一起糊涂忘八崽子!我不在家,就欺心害主,将姐
儿都逼走了。如今人家送来,还要拦阻,必是你们和我有什么仇么?”众家人原怕
贾琏回来不依,想来少时才破,岂知贾琏说得更明,心下不懂,只得站着回道:“二
爷出门,奴才们有病的,有告假的,都是三爷、蔷大爷、芸二爷作主,不与奴才们
相干。”贾琏道:“什么混帐东西!我完了事,再和你们说。快把车赶进来!”

贾琏进去,见邢夫人也不言语,转身到了王夫人那里,跪下磕了个头,回道:
“姐儿回来了,全亏太太周全。环兄弟也不用说他了。只是芸儿这东西,他上回看
家就闹乱儿,如今我去了几个月,便闹到这样。回太太的话:这种人撵了他不往来
也使得的!”王夫人道:“王仁这下流种子为什么也是这样坏!”贾琏道:“太太不用
说了,我自有道理。”正说着,彩云等回道:“姐儿进来了!”于是巧姐儿见了王夫
人。虽然别不多时,想起那样逃难的景况,不免落下泪来。巧姐儿也便大哭。贾琏
忙过来道谢了刘老老。王夫人便拉他坐下,说起那日的话来。贾琏见了平儿,外面
不好说别的,心里十分感激,眼中不觉流泪。自此,益发敬重平儿,打算等贾赦回
来,要扶平儿为正。此是后话,暂且不提。

只说邢夫人正恐贾琏不见了巧姐,必有一番的周折;又听见贾琏在王夫人那里,
心下更是着急,便叫丫头去打听。回来说是巧姐儿同着刘老老在那里说话儿呢,邢
夫人才如梦初觉,知是他们弄鬼,还抱怨王夫人:“调唆的我母子不和!到底不知是
那个送信给平儿的?”正问着,只见巧姐同着刘老老,带了平儿,王夫人在后头跟
着进来。先把头里的话都说在贾芸王仁身上,说:“大太太原是听见人说,为的是
好事。那里知道外头的鬼?”邢夫人听了,自觉羞惭,想起王夫人主意不差,心里
也服。于是邢王二夫人,彼此倒心下相安了。

平儿回了王夫人,带了巧姐到宝钗那里来请安,各自提各自的苦处。又说到:
“皇上隆恩,咱们家该兴旺起来了。想来宝二爷必回来的。”正说到这句话,只见
秋纹慌慌张张的跑来说道:“袭人不好了!”

不知何事,且听下回分解。