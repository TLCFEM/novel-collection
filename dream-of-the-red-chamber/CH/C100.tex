\chapter{破好事香菱结深恨~悲远嫁宝玉感离情}

话说贾政去见节度,进去了半日,不见出来,外头议论不一。李十儿在外也打
听不出什么事来,便想到报上的饥荒,实在也着急。好容易听见贾政出来了,便迎
上来跟着,等不得回去,在无人处便问:“老爷进去这半天,有什么要紧的事?”
贾政笑道:“并没有事。只为镇海总制是这位大人的亲戚,有书来嘱托照应我,所
以说了些好话。又说:‘我们如今也是亲戚了。’”李十儿听得,心内喜欢,不免
又壮了些胆子,便竭力怂恿贾政许这亲事。

贾政心想薛蟠的事,到底有什么挂碍,在外头信息不通,难以打点。故回到本
任来便打发家人进京打听,顺便将总制求亲之事回明贾母,如若愿意,即将三姑娘
接到任所。家人奉命,赶到京中回明了王夫人,便在吏部打听得贾政并无处分,惟
将署太平县的这位老爷革职。即写了帖,安慰了贾政,然后住着等信。

且说薛姨妈为着薛蟠这件人命官司,各衙门内不知花了多少银钱,才定了误杀
具题。原打量将当铺折变给人,备银赎罪,不想刑部驳审,又托人花了好些钱,总
不中用,依旧定了个死罪,监着守候秋天大审。薛姨妈又气又疼,日夜啼哭。宝钗
虽时常过来劝解,说是:“哥哥本来没造化。承受了祖父这些家业,就该安安顿顿
的守着过日子。在南边已经闹的不像样,便是香菱那件事情就了不得,因为仗着亲
戚们的势力,花了些银钱,这算白打死了一个公子。哥哥就该改过,做起正经人来,
也该奉养母亲才是,不想进了京仍是这样。妈妈为他不知受了多少气,哭掉了多少
眼泪。给他娶了亲,原想大家安安逸逸的过日子,不想命该如此,偏偏娶的嫂子又
是一个不安静的,所以哥哥躲出门去。真正俗语说的,‘冤家路儿狭’,不多几天
就闹出人命来了!妈妈和二哥哥也算不得不尽心的了:花了银钱不算,自己还求三
拜四的谋干。无奈命里应该,也算自作自受。大凡养儿女是为着老来有靠,便是小
户人家,还要挣一碗饭养活母亲,那里有将现成的闹光了,反害的老人家哭的死去
活来的?不是我说,哥哥的这样行为,不是儿子,竟是个冤家对头。妈妈再不明白,
明哭到夜,夜哭到明,又受嫂子的气。我呢,又不能常在这里劝解。我看见妈妈这
样,那里放得下心!他虽说是傻,也不肯叫我回去。前儿老爷打发人回来说,看见
京报,唬的了不得,所以才叫人来打点的。我想哥哥闹了事,担心的人也不少。幸
亏我还是在跟前的一样,若是离乡调远,听见了这个信,只怕我想妈妈也就想杀了。
我求妈妈暂且养养神,趁哥哥的活口现在,问问各处的帐目。人家该咱们的,咱们
该人家的,亦该请个旧伙计来算一算,看看还有几个钱没有。”薛姨妈哭着说道:
“这几天为闹你哥哥的事,你来了,不是你劝我,就是我告诉你衙门的事。你还不
知道:京里官商的名字已经退了,两个当铺已经给了人家,银子早拿来使完了。还
有一个当铺,管事的逃了,亏空了好几千两银子,也夹在里头打官司。你二哥哥天
天在外头要账,料着京里的账已经去了几万银子,只好拿南边公分里银子和住房折
变才够。前两天还听见一个荒信,说是南边的公分当铺也因为折了本儿收了。要是
这么着,你娘的命可就活不成了!”说着,又大哭起来。宝钗也哭着劝道:“银钱
的事,妈妈操心也不中用,还有二哥哥给我们料理。单可恨这些伙计们,见咱们的
势头儿败了,各自奔各自的去也罢了,我还听见说带着人家来挤我们的讹头。可见
我哥哥活了这么大,交的人总不过是些个酒肉弟兄,急难中是一个没有的。妈妈要
是疼我,听我的话:有年纪的人自己保重些。妈妈这一辈子,想来还不至挨冻受饿。
家里这点子衣裳家伙,只好任凭嫂子去,那是没法儿的了。所有的家人老婆们,瞧
他们也没心在这里了,该去的叫他们去。只可怜香菱苦了一辈子,只好跟着妈妈。
实在短什么,我要是有的,还可以拿些个来,料我们那个也没有不依的。就是袭姑
娘也是心术正道的,他听见咱们家的事,他倒提起妈妈来就哭。我们那一个还打量
没事的,所以不大着急,要听见了,也是要唬个半死儿的。”薛姨妈不等说完,便
说:“好姑娘,你可别告诉他。他为一个林姑娘几乎没要了命,如今才好了些。要
是他急出个原故来,不但你添一层烦恼,我越发没了依靠了。”宝钗道:“我也是
这么想,所以总没告诉他。”

正说着,只听见金桂跑来外间屋里哭喊道:“我的命是不要的了!男人呢,已
经是没有活的分儿了。咱们如今索性闹一闹,大伙儿到法场上去拚一拚!”说着,
便将头往隔断板上乱撞,撞的披头散发。气的薛姨妈白瞪着两只眼,一句话也说不
出来。还亏了宝钗嫂子长嫂子短,好一句歹一句的劝他。金桂道:“姑奶奶,如今
你是比不得头里的了。你两口儿好好的过日子,我是个单身人儿,要脸做什么!”
说着,就要跑到街上回娘家去。亏了人还多,拉住了,又劝了半天方住。把个宝琴
唬的再不敢见他。若是薛蝌在家,他便抹粉施脂,描眉画鬓,奇情异致的打扮收拾
起来,不时打从薛蝌住房前过,或故意咳嗽一声,明知薛蝌在屋里,特问房里是谁。
有时遇见薛蝌,他便妖妖调调、娇娇痴痴的问寒问暖,忽喜忽嗔。丫头们看见都连
忙躲开,他自己也不觉得,只是一心一意要弄的薛蝌感情时,好行宝蟾之计。那薛
蝌却只躲着,有时遇见也不敢不周旋他,倒是怕他撒泼放刁的意思。更加金桂一则
为色迷心,越瞧越爱,越想越幻,那里还看的出薛蝌的真假来?只有一宗,他见薛
蝌有什么东西都是托香菱收着,衣服缝洗也是香菱,两个人偶然说话,他来了,急
忙散开:一发动了一个“醋”字。欲待发作薛蝌,却是舍不得,只得将一腔隐恨都
搁在香菱身上。却又恐怕闹了香菱得罪了薛蝌,倒弄的隐忍不发。

一日,宝蟾走来,笑嘻嘻的向金桂道:“奶奶,看见了二爷没有?”金桂道:
“没有。”宝蟾笑道:“我说二爷的那种假正经是信不得的。咱们前儿送了酒去,
他说不会喝,刚才我见他到太太那屋里去,脸上红扑扑儿的一脸酒气。奶奶不信,
回来只在咱们院子门口儿等他。他打那边过来,奶奶叫住他问问,看他说什么。”
金桂听了,一心的恼意,便道:“他那里就出来了呢。他既无情义,问他作什么?”
宝蟾道:“奶奶又迂了。他好说,咱们也好说;他不好说,咱们再另打主意。”金
桂听着有理,因叫宝蟾:“瞧着他,看他出去了。”宝蟾答应着出来,金桂却去打
开镜奁,又照了一照,把嘴唇儿又抹了一抹。然后拿一条洒花绢子,才要出来,又
像忘了什么的,心里倒不知怎么是好了。只听宝蟾外面说道:“二爷今日高兴啊。
那里喝了酒来了?”金桂听了,明知是叫他出来的意思,连忙掀起帘子出来。只见
薛蝌和宝蟾说道:“今日是张大爷的好日子,所以被他们强不过,吃了半钟。到这
时候脸还发烧呢。”一句话没说完,金桂早接口道:“自然人家外人的酒,比咱们
自己家里的酒是有趣儿的。”薛蝌被他拿话一激,脸越红了,连忙走过来陪笑道:
“嫂子说那里的话?”宝蟾见他二人交谈,便躲到屋里去了。这金桂初时原要假意
发作薛蝌两句,无奈一见他两颊微红,双眸带涩,别有一种谨愿可怜之意,早把自
己那骄悍之气,感化到爪洼国去了,因笑说道:“这么说,你的酒是硬强着才肯喝
的呢。”薛蝌道:“我那里喝得来?”金桂道:“不喝也好,强如像你哥哥喝出乱
子来,明儿娶了你们奶奶儿,像我这样守活寡受孤单呢!”说到这里,两个眼已经
乜斜了,两腮上也觉红晕了。薛蝌见这话越发邪僻了,打算着要走。金桂也看出来
了,那里容得,早已走过来一把拉住。薛蝌急了道:“嫂子放尊重些。”说着浑身
乱颤。金桂索性老着脸道:“你只管进来,我和你说一句要紧的话。”

正闹着,忽听背后一个人叫道:“奶奶!香菱来了。”把金桂唬了一跳。回头
瞧时,却是宝蟾掀着帘子看他二人的光景,一抬头见香菱从那边来了,赶忙知会金
桂。金桂这一惊不小,手已松了。薛蝌得便脱身跑了。那香菱正走着,原不理会,
忽听宝蟾一嚷,才瞧见金桂在那里拉住薛蝌,往里死拽。香菱却唬的心头乱跳,自
己连忙转身回去。这里金桂早已连吓带气,呆呆的瞅着薛蝌去了,怔了半天,恨了
一声,自己扫兴归房。从此把香菱恨入骨髓。那香菱本是要到宝琴那里,刚走出腰
门,看见这般,吓回去了。

是日,宝钗在贾母屋里,听得王夫人告诉老太太要聘探春一事。贾母说道:“既
是同乡的人,很好。只是听见说那孩子到过我们家里,怎么你老爷没有提起?”王
夫人道:“连我们也不知道。”贾母道:“好是好,但只道儿太远。虽然老爷在那
里,倘或将来老爷调任,可不是我们孩子太单了吗?”王夫人道:“两家都是做官
的,也是拿不定。或者那边还调进来,即不然,终有个叶落归根。况且老爷既在那
里做官,上司已经说了,好意思不给么?想来老爷的主意定了,只是不敢做主,故
遣人来回老太太的。”贾母道:“你们愿意更好,但是三丫头这一去了,不知三年
两年那边可能回家?若再迟了,恐怕我赶不上再见他一面了。”说着掉下泪来。王
夫人道:“孩子们大了,少不得总要给人家的。就是本乡本土的人,除非不做官还
使得,要是做官的,谁保的住总在一处?只要孩子们有造化就好。譬如迎姑娘倒配
的近呢,偏时常听见他和女婿打闹,甚至于不给饭吃。就是我们送了东西去,他也
摸不着。近来听见益发不好了,也不放他回来。两口子拌起来,就说咱们使了他家
的银钱,可怜这孩子总不得个出头的日子。前儿我掂记他,打发人去瞧他,迎丫头
藏在耳房里,不肯出来。老婆们必要进去,看见我们姑娘这样冷天还穿着几件旧衣
裳。他一包眼泪的告诉老婆们说:‘回去别说我这么苦,这也是我命里所招!也不
用送什么衣裳东西来,不但摸不着,反要添一顿打,说是我告诉的。’老太太想想,
这倒是近处眼见的,若不好,更难受。倒亏了大太太也不理会他,大老爷也不出个
头。如今迎姑娘实在比我们三等使唤的丫头还不及。我想探丫头虽不是我养的,老
爷既看见过女婿,定然是好才许的。只请老太太示下,择个好日子,多派几个人送
到他老爷任上,该怎么着,老爷也不肯将就。”贾母道:“有他老子作主,你就料
理妥当,拣个长行的日子送去,也就定了一件事。”王夫人答应着“是”。宝钗听
的明白,也不敢则声,只是心里叫苦:“我们家的姑娘们就算他是个尖儿。如今又
要远嫁,眼看着这里的人一天少似一天了。”见王夫人起身告辞出去,他也送出来
了。一径回到自己房中,并不与宝玉说知,见袭人独自一个做活,便将听见的话说
了。袭人也很不受用。

却说赵姨娘听见探春这事,反喜欢起来,心里说道:“我这个丫头在家忒瞧不
起我,我何从还是个娘?比他的丫头还不济。况且上水,护着别人。他挡在头里,
连环儿也不得出头。如今老爷接了去,我倒干净。想要他孝敬我不能够了,只愿意
他像迎丫头似的,我也称称愿。”一面想着,一面跑到探春那边与他道喜,说:“姑
娘,你是要高飞的人了。到了姑爷那边自然比家里还好,想来你也是愿意的。就是
养了你一场,并没有借你的光儿。就是我有七分不好,也有三分的好,也别说一去
了把我搁在脑杓子后头。”探春听着毫无道理,只低头作活,一句也不言语。赵姨
娘见他不理,气忿忿的自己去了。

这里探春又气又笑又伤心,也不过自己掉泪而已。坐了一回,闷闷的走到宝玉
这边来。宝玉因问道:“三妹妹,我听见林妹妹死的时候,你在那里来着。我还听
见说:林妹妹死的时候,远远的有音乐之声。或者他是有来历的,也未可知。”探
春笑道:“那是你心里想着罢了。但只那夜却怪,不像人家鼓乐的声儿,你的话或
者也是。”宝玉听了,更以为实。又想前日自己神魂飘荡之时,曾见一人,说是黛
玉生不同人,死不同鬼,必是那里的仙子临凡。又想起那年唱戏做的嫦娥,飘飘艳
艳,何等风致。过了一回探春去了,因必要紫鹃过来,立刻回了贾母去叫他。无奈
紫鹃心里不愿意,虽经贾母王夫人派了过来,自己没法,却是在宝玉跟前,不是嗳
声就是叹气的。宝玉背地里拉着他,低声下气要问黛玉的话,紫鹃从没好话回答。
宝钗倒背地里夸他有忠心,并不嗔怪他。那雪雁虽是宝玉娶亲这夜出过力的,宝玉
见他心地不甚明白,便回了贾母王夫人,将他配了一个小厮,各自过活去了。王奶
妈,养着他将来好送黛玉的灵柩回南。鹦哥等小丫头,仍旧伏侍老太太。

宝玉本想念黛玉,因此及彼,又想跟黛玉的人已经云散,更加纳闷。闷到无可
如何,忽又想黛玉死的这样清楚,必是离凡返仙去了,反又欢喜。忽然听见袭人和
宝钗那里讲究探春出嫁之事,宝玉听了,“啊呀”的一声,哭倒在炕上。唬得宝钗
袭人都来扶起,说:“怎么了?”宝玉早哭的说不出来。定了一回子神,说道:“这
日子过不得了,我姊妹们都一个一个的散了!林妹妹是成了仙去了。大姐姐呢,已
经死了,这也罢了,没天天在一块儿。二姐姐碰着了一个混账不堪的东西。三妹妹
又要远嫁,总不得见的了。史妹妹又不知要到那里去。薛妹妹是有了人家儿的。这
些姐姐妹妹,难道一个都不留在家里,单留我做什么?”袭人忙又拿话解劝。宝钗
摆着手说:“你不用劝他,等我问他。”因问着宝玉道:“据你的心里,要这些姐
妹都在家里陪到你老了,都不为终身的事吗?要说别人,或者还有别的想头。你自
己的姐姐妹妹,不用说没有远嫁的;就是有,老爷作主,你有什么法儿?打量天下
就是你一个人爱姐姐妹妹呢?要是都像你,就连我也不能陪着你了。大凡人念书原
为的是明理,怎么你越念越糊涂了呢。这么说起来,我和袭姑娘各自一边儿去,让
你把姐姐妹妹们都邀了来守着你。”宝玉听了,两只手拉住宝钗袭人道:“我也知
道。为什么散的这么早呢?等我化了灰的时候再散也不迟。”袭人掩着他的嘴道:
“又胡说了。才这两天身上好些,二奶奶才吃些饭。你要是又闹翻了,我也不管了。”
宝玉听他两个人说话都有道理,只是心上不知道怎么着才好,只得说道:“我却明
白,但只是心里闹得慌。”宝钗也不理他,暗叫袭人快把定心丸给他吃了,慢慢的
开导他。袭人便欲告诉探春,说临行不必来辞。宝钗道:“这怕什么?等消停几日,
他心里明白了,还要叫他们多说句话儿呢。况且三姑娘是极明白的人,不像那些假
惺惺的人,少不得有一番箴谏,他以后就不是这样了。”正说着,贾母那边打发过
鸳鸯来说:“知道宝玉旧病又发,叫袭人劝说安慰,叫他不用胡思乱想。”袭人等
应了。鸳鸯坐了一会子去了。

那贾母又想起探春远行,虽不全备妆奁,其一应动用之物俱该预备,便把凤姐
叫来,将老爷的主意告诉了一遍,叫他料理去。凤姐答应。

不知怎么办理,下回分解。