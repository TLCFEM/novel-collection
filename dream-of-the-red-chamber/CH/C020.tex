\chapter{王熙凤正言弹妒意~林黛玉俏语谑娇音}

话说宝玉在黛玉房中说“耗子精”,宝钗撞来,讽刺宝玉元宵不知“绿蜡”之
典,三人正在房中互相取笑。那宝玉恐黛玉饭后贪眠,一时存了食,或夜间走了困,
身体不好;幸而宝钗走来,大家谈笑,那黛玉方不欲睡,自己才放了心。忽听他房
中嚷起来,大家侧耳听了一听,黛玉先笑道:“这是你妈妈和袭人叫唤呢。那袭人
待他也罢了,你妈妈再要认真排揎他,可见老背晦了。”宝玉忙欲赶过去,宝钗一
把拉住道:“你别和你妈妈吵才是呢!他是老糊涂了,倒要让他一步儿的是。”宝
玉道:“我知道了。”说毕走来。

只见李嬷嬷拄着拐杖,在当地骂袭人:“忘了本的小娼妇儿!我抬举起你来,
这会子我来了,你大模厮样儿的躺在炕上,见了我也不理一理儿。一心只想妆狐媚
子哄宝玉,哄的宝玉不理我,只听你的话。你不过是几两银子买了来的小丫头子罢
咧,这屋里你就作起耗来了!好不好的,拉出去配一个小子,看你还妖精似的哄人
不哄!”袭人先只道李嬷嬷不过因他躺着生气,少不得分辩说:“病了,才出汗,
蒙着头,原没看见你老人家。”后来听见他说“哄宝玉”,又说“配小子”,由不
得又羞又委屈,禁不住哭起来了。宝玉虽听了这些话,也不好怎样,少不得替他分
辩,说“病了,吃药”,又说:“你不信,只问别的丫头。”李嬷嬷听了这话,越
发气起来了,说道:“你只护着那起狐狸,那里还认得我了呢?叫我问谁去?谁不帮
着你呢?谁不是袭人拿下马来的?我都知道那些事!我只和你到老太太、太太跟前去
讲讲:把你奶了这么大,到如今吃不着奶了,把我扔在一边儿,逞着丫头们要我的
强!”一面说,一面哭。彼时黛玉宝钗等也过来劝道:“妈妈,你老人家担待他们
些就完了。”李嬷嬷见他二人来了,便诉委屈,将当日吃茶,茜雪出去,和昨日酥
酪等事,唠唠叨叨说个不了。

可巧凤姐正在上房算了输赢帐,听见后面一片声嚷,便知是李嬷嬷老病发了,
又值他今儿输了钱,迁怒于人,排揎宝玉的丫头。便连忙赶过来拉了李嬷嬷,笑道:
“妈妈别生气。大节下,老太太刚喜欢了一日。你是个老人家,别人吵,你还要管
他们才是;难道你倒不知规矩,在这里嚷起来,叫老太太生气不成?你说谁不好,
我替你打他。我屋里烧的滚热的野鸡,快跟了我喝酒去罢。”一面说,一面拉着走,
又叫:“丰儿,替你李奶奶拿着拐棍子、擦眼泪的绢子。”那李嬷嬷脚不沾地跟了
凤姐儿走了,一面还说:“我也不要这老命了,索性今儿没了规矩,闹一场子,讨
了没脸,强似受那些娼妇的气!”后面宝钗黛玉见凤姐儿这般,都拍手笑道:“亏
他这一阵风来,把个老婆子撮了去了。”

宝玉点头叹道:“这又不知是那里的帐,只拣软的欺负!又不知是那个姑娘得
罪了,上在他帐上了。”一句未完,晴雯在旁说道:“谁又没疯了,得罪他做什么?
既得罪了他,就有本事承任,犯不着带累别人!”袭人一面哭,一面拉着宝玉道:
“为我得罪了一个老奶奶,你这会子又为我得罪这些人,这还不够我受的,还只是
拉扯人!”宝玉见他这般病势,又添了这些烦恼,连忙忍气吞声,安慰他仍旧睡下
出汗。又见他汤烧火热,自己守着他,歪在旁边,劝他只养病,别想那些没要紧的
事。袭人冷笑道:“要为这些事生气,这屋里一刻还住得了?但只是天长日久,尽
着这么闹,可叫人怎么过呢!你只顾一时为我得罪了人,他们都记在心里,遇着坎
儿,说的好说不好听的,大家什么意思呢?”一面说,一面禁不住流泪,又怕宝玉
烦恼,只得又勉强忍着。一时杂使的老婆子端了二和药来,宝玉见他才有点汗儿,
便不叫他起来,自己端着给他就枕上吃了,即令小丫鬟们铺炕。袭人道:“你吃饭
不吃饭,到底老太太、太太跟前坐一会子,和姑娘们玩一会子,再回来。我就静静
的躺一躺也好啊。”宝玉听说,只得依他,看着他去了簪环躺下,才去上屋里跟着
贾母吃饭。

饭毕,贾母犹欲和那几个老管家的嬷嬷斗牌。宝玉惦记袭人,便回至房中。见
袭人朦胧睡去,自己要睡,天气尚早。彼时晴雯、绮霞、秋纹、碧痕都寻热闹,找
鸳鸯、琥珀等耍戏去了。见麝月一人在外间屋里灯下抹骨牌。宝玉笑道:“你怎么
不和他们去?”麝月道:“没有钱。”宝玉道:“床底下堆着钱,还不够你输的?”
麝月道:“都乐去了,这屋子交给谁呢?那一个又病了,满屋里上头是灯,下头是
火,那些老婆子们都老天拔地伏侍了一天,也该叫他们歇歇儿了。小丫头们也伏侍
了一天,这会子还不叫玩玩儿去吗?所以我在这里看着。”宝玉听了这话,公然又
是一个袭人了。因笑道:“我在这里坐着,你放心去罢。”麝月道:“你既在这里,
越发不用去了。咱们两个说话儿不好?”宝玉道:“咱们两个做什么呢?怪没意思
的。也罢了,早起你说头上痒痒,这会子没什么事,我替你篦头罢。”麝月听了道:
“使得。”说着,将文具镜匣搬来,卸去钗,打开头发,宝玉拿了篦子替他篦。

只篦了三五下儿,见晴雯忙忙走进来取钱,一见他两个,便冷笑道:“哦!交
杯盏儿还没吃,就上了头了!”宝玉笑道:“你来,我也替你篦篦。”晴雯道:“我
没这么大造化。”说着,拿了钱,摔了帘子,就出去了。宝玉在麝月身后,麝月对
镜,二人在镜内相视而笑。宝玉笑着道:“满屋里就只是他磨牙。”麝月听说,忙
向镜中摆手儿。宝玉会意,忽听“唿”一声帘子响,晴雯又跑进来问道:“我怎么
磨牙了?咱们倒得说说!”麝月笑道:“你去你的罢,又来拌嘴儿了。”晴雯也笑
道:“你又护着他了!你们瞒神弄鬼的,打量我都不知道呢!等我捞回本儿来再说。”
说着,一径去了。这里宝玉通了头,命麝月悄悄的伏侍他睡下,不肯惊动袭人。一
宿无话。

次日清晨,袭人已是夜间出了汗,觉得轻松了些,只吃些米汤静养。宝玉才放
了心,因饭后走到薛姨妈这边来闲逛。

彼时正月内学房中放年学,闺阁中忌针黹,都是闲时,因贾环也过来玩。正遇
见宝钗、香菱、莺儿三个赶围棋作耍,贾环见了也要玩。宝钗素日看他也如宝玉,
并没他意,今儿听他要玩,让他上来,坐在一处玩。一注十个钱。头一回,自己赢
了,心中十分喜欢。谁知后来接连输了几盘,就有些着急。赶着这盘正该自己掷骰
子,若掷个七点便赢了,若掷个六点也该赢,掷个三点就输了。因拿起骰子来狠命
一掷,一个坐定了二,那一个乱转。莺儿拍着手儿叫“么!”贾环便瞪着眼,“六!”
“七!”“八!”混叫。那骰子偏生转出么来。贾环急了,伸手便抓起骰子来,就
要拿钱,说是个四点。莺儿便说:“明明是个么!”宝钗见贾环急了,便瞅了莺儿
一眼,说道:“越大越没规矩!难道爷们还赖你?还不放下钱来呢。”莺儿满心委屈,
见姑娘说,不敢出声,只得放下钱来,口内嘟囔说:“一个做爷的,还赖我们这几
个钱,连我也瞧不起!前儿和宝二爷玩,他输了那些也没着急,下剩的钱还是几个
小丫头子们一抢,他一笑就罢了。”

宝钗不等说完,连忙喝住了。贾环道:“我拿什么比宝玉?你们怕他,都和他
好,都欺负我不是太太养的!”说着便哭。宝钗忙劝他:“好兄弟,快别说这话,
人家笑话。”又骂莺儿。正值宝玉走来,见了这般景况,问:“是怎么了?”贾环
不敢则声。宝钗素知他家规矩,凡做兄弟的怕哥哥。却不知那宝玉是不要人怕他的,
他想着:“兄弟们一并都有父母教训,何必我多事,反生疏了。况且我是正出,他
是庶出,饶这样看待,还有人背后谈论,还禁得辖治了他?”更有个呆意思存在心
里。你道是何呆意?因他自幼姐妹丛中长大,亲姊妹有元春探春,叔伯的有迎春惜
春,亲戚中又有湘云黛玉宝钗等人,他便料定天地间灵淑之气只钟于女子,男儿们
不过是些渣滓浊沫而已。因此把一切男子都看成浊物,可有可无。只是父亲、伯叔、
兄弟之伦,因是圣人遗训,不敢违忤,所以弟兄间亦不过尽其大概就罢了,并不想
自己是男子,须要为子弟之表率。是以贾环等都不甚怕他,只因怕贾母不依,才只
得让他三分。现今宝钗生怕宝玉教训他,倒没意思,便连忙替贾环掩饰。宝玉道:
“大正月里,哭什么?这里不好,到别处玩去。你天天念书,倒念糊涂了。譬如这
件东西不好,横竖那一件好,就舍了这件取那件,难道你守着这件东西哭会子就好
了不成?你原是要取乐儿,倒招的自己烦恼。还不快去呢!”

贾环听了,只得回来。赵姨娘见他这般,因问:“是那里垫了踹窝来了?”贾
环便说:“同宝姐姐玩来着。莺儿欺负我,赖我的钱;宝玉哥哥撵了我来了。”赵
姨娘啐道:“谁叫你上高台盘了?下流没脸的东西!那里玩不得?谁叫你跑了去讨这
没意思?”正说着,可巧凤姐在窗外过,都听到耳内,便隔着窗户说道:“大正月
里,怎么了?兄弟们小孩子家,一半点儿错了,你只教导他,说这样话做什么?凭他
怎么着,还有老爷太太管他呢,就大口家啐他?他现是主子,不好,横竖有教导他
的人,与你什么相干?环兄弟,出来!跟我玩去。”贾环素日怕凤姐比怕王夫人更甚,
听见叫他,便赶忙出来。赵姨娘也不敢出声。凤姐向贾环道:“你也是个没性气的
东西呦!时常说给你:要吃,要喝,要玩,你爱和那个姐姐妹妹哥哥嫂子玩,就和
那个玩。你总不听我的话,倒叫这些人教的你歪心邪意、狐媚魇道的。自己又不尊
重,要往下流里走,安着坏心,还只怨人家偏心呢。输了几个钱,就这么个样儿!”
因问贾环:“你输了多少钱?”贾环见问,只得诺诺的说道:“输了一二百钱。”
凤姐啐道:“亏了你还是个爷,输了一二百钱就这么着!”回头叫:“丰儿,去取
一吊钱来;姑娘们都在后头玩呢,把他送了去。你明儿再这么狐媚子,我先打了你,
再叫人告诉学里,皮不揭了你的!为你这不尊贵,你哥哥恨得牙痒痒,不是我拦着,
窝心脚把你的肠子还窝出来呢!”喝令:“去罢!”贾环诺诺的,跟了丰儿得了钱,
自去和迎春等玩去,不在话下。

且说宝玉正和宝钗玩笑,忽见人说:“史大姑娘来了。”宝玉听了,连忙就走。
宝钗笑道:“等着,咱们两个一齐儿走,瞧瞧他去。”说着,下了炕,和宝玉来至
贾母这边。只见史湘云大说大笑的,见了他两个,忙站起来问好。正值黛玉在旁,
因问宝玉:“打那里来?”宝玉便说:“打宝姐姐那里来。”黛玉冷笑道:“我说
呢!亏了绊住,不然,早就飞了来了。”宝玉道:“只许和你玩,替你解闷儿;不
过偶然到他那里,就说这些闲话。”黛玉道:“好没意思的话!去不去,管我什么
事?又没叫你替我解闷儿!还许你从此不理我呢!”说着,便赌气回房去了。

宝玉忙跟了来,问道:“好好儿的又生气了!就是我说错了,你到底也还坐坐
儿,合别人说笑一会子啊?”黛玉道:“你管我呢!”宝玉笑道:“我自然不敢管
你,只是你自己遭塌坏了身子呢。”黛玉道:“我作践了我的身子,我死我的,与
你何干?”宝玉道:“何苦来?大正月里,‘死’了‘活’了的。”黛玉道:“偏
说‘死’!我这会子就死!你怕死,你长命百岁的活着,好不好?”宝玉笑道:“要
像只管这么闹,我还怕死吗?倒不如死了干净。”黛玉忙道:“正是了,要是这样
闹,不如死了干净!”宝玉道:“我说自家死了干净,别错听了话,又赖人。”正
说着,宝钗走来,说:“史大妹妹等你呢。”说着,便拉宝玉走了。这黛玉越发气
闷,只向窗前流泪。

没两盏茶时,宝玉仍来了。黛玉见了,越发抽抽搭搭的哭个不住。宝玉见了这
样,知难挽回,打叠起百样的款语温言来劝慰。不料自己没张口,只听黛玉先说道:
“你又来作什么?死活凭我去罢了!横竖如今有人和你玩,比我又会念,又会作,又
会写,又会说会笑,又怕你生气,拉了你去哄着你。你又来作什么呢?”宝玉听了,
忙上前悄悄的说道:“你这么个明白人,难道连‘亲不隔疏,后不僭先’也不知道?
我虽糊涂,却明白这两句话。头一件,咱们是姑舅姐妹,宝姐姐是两姨姐妹,论亲
戚也比你远。第二件,你先来,咱们两个一桌吃,一床睡,从小儿一处长大的,他
是才来的,岂有个为他远你的呢?”黛玉啐道:“我难道叫你远他?我成了什么人
了呢?我为的是我的心!”宝玉道:“我也为的是我的心。你难道就知道你的心,
不知道我的心不成?”黛玉听了,低头不语,半日说道:“你只怨人行动嗔怪你,
你再不知道你怄的人难受。就拿今日天气比,分明冷些,怎么你倒脱了青肷披风
呢?”宝玉笑道:“何尝没穿?见你一恼,我一暴燥,就脱了。”黛玉叹道:“回
来伤了风,又该讹着吵吃的了。”

二人正说着,只见湘云走来,笑道:“爱哥哥,林姐姐,你们天天一处玩,我
好容易来了,也不理我理儿。”黛玉笑道:“偏是咬舌子爱说话,连个‘二’哥哥
也叫不上来,只是‘爱’哥哥‘爱’哥哥的。回来赶围棋儿,又该你闹‘么爱三’
了。”宝玉笑道:“你学惯了,明儿连你还咬起来呢。”湘云道:“他再不放人一
点儿,专会挑人。就算你比世人好,也不犯见一个打趣一个。我指出个人来,你敢
挑他,我就服你。”黛玉便问:“是谁?”湘云道:“你敢挑宝姐姐的短处,就算
你是个好的。”黛玉听了冷笑道:“我当是谁,原来是他。我可那里敢挑他呢?”
宝玉不等说完,忙用话分开。湘云笑道:“这一辈子我自然比不上你。我只保佑着
明儿得一个咬舌儿林姐夫,时时刻刻你可听‘爱’呀‘厄’的去!阿弥陀佛,那时
才现在我眼里呢!”说的宝玉一笑,湘云忙回身跑了。

要知端详,且听下回分解。