\chapter{训劣子李贵承申饬~嗔顽童茗烟闹书房}

话说秦邦业父子专候贾家人来送上学之信。原来宝玉急于要和秦钟相遇,遂择
了后日一定上学,打发人送了信。到了这天,宝玉起来时,袭人早已把书笔文物收
拾停妥,坐在床沿上发闷,见宝玉起来,只得伏侍他梳洗。宝玉见他闷闷的,问道:
“好姐姐,你怎么又不喜欢了?难道怕我上学去,撂的你们冷清了不成?”袭人笑
道:“这是那里的话?念书是很好的事,不然就潦倒一辈子了,终久怎么样呢?但只
一件:只是念书的时候儿想着书,不念的时候儿想着家。总别和他们玩闹,碰见老
爷不是玩的。虽说是奋志要强,那工课宁可少些,一则贪多嚼不烂,二则身子也要
保重。这就是我的意思,你好歹体谅些。”袭人说一句,宝玉答应一句。袭人又道:
“大毛儿衣服我也包好了,交给小子们去了。学里冷,好歹想着添换,比不得家里
有人照顾。脚炉手炉也交出去了,你可逼着他们给你笼上。那一起懒贼,你不说他
们乐得不动,白冻坏了你。”宝玉道:“你放心,我自己都会调停的。你们也可别
闷死在这屋里,长和林妹妹一处玩玩儿去才好。”说着俱已穿戴齐备,袭人催他去
见贾母、贾政、王夫人。宝玉又嘱咐了晴雯麝月几句,方出来见贾母。贾母也不免
有几句嘱咐的话。然后去见王夫人,又出来到书房中见贾政。

这日贾政正在书房中和清客相公们说闲话儿,忽见宝玉进来请安,回说上学
去。贾政冷笑道:“你要再提‘上学’两个字,连我也羞死了。依我的话,你竟玩
你的去是正经。看仔细站腌了我这个地,靠腌了我这个门!”众清客都起身笑
道:“老世翁何必如此。今日世兄一去,二三年就可显身成名的,断不似往年仍作
小儿之态了。天也将饭时了,世兄竟快请罢。”说着便有两个年老的携了宝玉出去。
贾政因问:“跟宝玉的是谁?”只听见外面答应了一声,早进来三四个大汉,打千
儿请安。贾政看时,是宝玉奶姆的儿子名唤李贵的,因向他道:“你们成日家跟他
上学,他到底念了些什么书!倒念了些流言混话在肚子里,学了些精致的淘气。等
我闲一闲,先揭了你的皮,再和那不长进的东西算帐!”吓的李贵忙双膝跪下,摘
了帽子碰头,连连答应“是”,又回说:“哥儿已经念到第三本《诗经》,什么‘攸
攸鹿鸣,荷叶浮萍’,小的不敢撒谎。”说的满坐哄然大笑起来,贾政也掌不住笑
了。因说道:“那怕再念三十本《诗经》,也是‘掩耳盗铃’,哄人而已。你去请
学里太爷的安,就说我说的:什么《诗经》、古文,一概不用虚应故事,只是先把
《四书》一齐讲明背熟是最要紧的。”李贵忙答应“是”,见贾政无话,方起来退
出去。

此时宝玉独站在院外,屏声静候,等他们出来同走。李贵等一面掸衣裳,一面
说道:“哥儿可听见了?先要揭我们的皮呢。人家的奴才跟主子赚些个体面,我们
这些奴才白陪着挨打受骂的。从此也可怜见些才好!”宝玉笑道:“好哥哥,你别
委屈,我明儿请你。”李贵道:“小祖宗,谁敢望‘请’,只求听一两句话就有了。”

说着又至贾母这边,秦钟早已来了,贾母正和他说话儿呢。于是二人见过,辞
了贾母。宝玉忽想起未辞黛玉,又忙至黛玉房中来作辞。彼时黛玉在窗下对镜理妆,
听宝玉说上学去,因笑道:“好!这一去,可是要‘蟾宫折桂’了!我不能送你了。”
宝玉道:“好妹妹,等我下学再吃晚饭。那胭脂膏子也等我来再制。”唠叨了半日,
方抽身去了。黛玉忙又叫住,问道:“你怎么不去辞你宝姐姐来呢?”宝玉笑而不
答,一径同秦钟上学去了。

原来这义学也离家不远,原系当日始祖所立,恐族中子弟有力不能延师者,即
入此中读书。凡族中为官者皆有帮助银两,以为学中膏火之费;举年高有德之人为
塾师。如今秦宝二人来了,一一的都互相拜见过,读起书来。自此后二人同来同往
同起同坐,愈加亲密。兼贾母爱惜,也常留下秦钟一住三五天,和自己重孙一般看
待。因见秦钟家中不甚宽裕,又助些衣服等物。不上一两月工夫,秦钟在荣府里便
惯熟了。宝玉终是个不能安分守理的人,一味的随心所欲,因此发了癖性,又向秦
钟悄说:“咱们两个人,一样的年纪,况又同窗,以后不必论叔侄,只论弟兄朋友
就是了。”先是秦钟不敢,宝玉不从,只叫他“兄弟”,叫他表字“鲸卿”,秦钟
也只得混着乱叫起来。

原来这学中虽都是本族子弟与些亲戚家的子侄,俗语说的好:“一龙九种,种
种各别。”未免人多了就有龙蛇混杂、下流人物在内。自秦宝二人来了,都生的花
朵儿一般的模样,又见秦钟腼腆温柔,未语先红,怯怯羞羞有女儿之风;宝玉又是
天生成惯能作小服低,赔身下气,性情体贴,话语缠绵。因他二人又这般亲厚,也
怨不得那起同窗人起了嫌疑之念,背地里你言我语,诟谇谣诼,布满书房内外。

原来薛蟠自来王夫人处住后,便知有一家学,学中广有青年子弟。偶动了龙阳
之兴,因此也假说来上学,不过是“三日打鱼,两日晒网”,白送些束礼物与贾
代儒,却不曾有一点儿进益,只图结交些契弟。谁想这学内的小学生,图了薛蟠的
银钱穿吃,被他哄上手了,也不消多记。又有两个多情的小学生,亦不知是那一房
的亲眷,亦未考真姓名,只因生得妩媚风流,满学中都送了两个外号,一个叫“香
怜”,一个叫“玉爱”。别人虽都有羡慕之意、“不利于孺子”之心,只是惧怕薛
蟠的威势,不敢来沾惹。如今秦宝二人一来了,见了他两个,也不免缱绻羡爱,亦
知系薛蟠相知,未敢轻举妄动。香玉二人心中,一般的留情与秦宝:因此四人心中
虽有情意,只未发出。每日一入学中,四处各坐,却八目勾留,或设言托意,或咏
桑寓柳,遥以心照,却外面自为避人眼目。不料偏又有几个滑贼看出形景来,都背
后挤眉弄眼,或咳嗽扬声,这也非止一日。

可巧这日代儒有事回家,只留下一句七言对联,令学生对了明日再来上书,将
学中之事又命长孙贾瑞管理。妙在薛蟠如今不大上学应卯了,因此秦钟趁此和香怜
弄眉挤眼,二人假出小恭,走至后院说话。秦钟先问他:“家里的大人可管你交朋
友不管?”一语未了,只听见背后咳嗽了一声。二人吓的忙回顾时,原来是窗友名
金荣的。香怜本有些性急,便羞怒相激,问他道:“你咳嗽什么?难道不许我们说
话不成?”金荣笑道:“许你们说话,难道不许我咳嗽不成?我只问你们:有话不
分明说,许你们这样鬼鬼祟祟的干什么故事?我可也拿住了!还赖什么?先让我抽个
头儿,咱们一声儿不言语。不然大家就翻起来!”秦香二人就急得飞红的脸,便问
道:“你拿住什么了?”金荣笑道:“我现拿住了是真的。”说着又拍着手笑嚷道:
“贴的好烧饼!你们都不买一个吃去?”秦钟香怜二人又气又急,忙进来向贾瑞前
告金荣,说金荣无故欺负他两个。

原来这贾瑞最是个图便宜没行止的人,每在学中以公报私,勒索子弟们请他;
后又助着薛蟠图些银钱酒肉,一任薛蟠横行霸道,他不但不去管约,反助纣为虐讨
好儿。偏那薛蟠本是浮萍心性,今日爱东,明日爱西,近来有了新朋友,把香玉二
人丢开一边;就连金荣也是当日的好友,自有了香玉二人,便见弃了金荣,近日连
香玉亦已见弃。故贾瑞也无了提携帮衬之人,不怨薛蟠得新厌故,只怨香玉二人不
在薛蟠跟前提携了:因此贾瑞金荣等一干人,也正醋妒他两个。今见秦香二人来告
金荣,贾瑞心中便不自在起来,虽不敢呵叱秦钟,却拿着香怜作法,反说他多事,
着实抢白了几句。香怜反讨了没趣,连秦钟也讪讪的各归坐位去了。

金荣越发得了意,摇头咂嘴的,口内还说许多闲话。玉爱偏又听见,两个人隔
坐咕咕唧唧的角起口来。金荣只一口咬定说:“方才明明的撞见他两个在后院里亲
嘴摸屁股,两个商议,定了一对儿。”论长道短,那时只顾得志乱说,却不防还有
别人。谁知早又触怒了一个人。你道这一个人是谁?原来这人名唤贾蔷,亦系宁府
中之正派玄孙,父母早亡,从小儿跟着贾珍过活,如今长了十六岁,比贾蓉生得还
风流俊俏。他兄弟二人最相亲厚,常共起居,宁府中人多口杂,那些不得志的奴仆,
专能造言诽谤主人,因此不知又有什么小人诟谇谣诼之辞。贾珍想亦风闻得些口声
不好,自己也要避些嫌疑,如今竟分与房舍,命贾蔷搬出宁府,自己立门户过活去
了。这贾蔷外相既美,内性又聪敏,虽然应名来上学,亦不过虚掩眼目而已,仍是
斗鸡走狗、赏花阅柳为事。上有贾珍溺爱,下有贾蓉匡助,因此族中人谁敢触逆于
他。他既和贾蓉最好,今见有人欺负秦钟,如何肯依?如今自己要挺身出来报不平,
心中且忖度一番:“金荣贾瑞一等人,都是薛大叔的相知,我又与薛大叔相好,倘
或我一出头,他们告诉了老薛,我们岂不伤和气呢。欲要不管,这谣言说的大家没
趣。如今何不用计制伏,又止息了口声,又不伤脸面。”想毕,也装出小恭去,走
至后面瞧瞧,把跟宝玉书童茗烟叫至身边,如此这般,调拨他几句。

这茗烟乃是宝玉第一个得用且又年轻不谙事的,今听贾蔷说:“金荣如此欺负
秦钟,连你们的爷宝玉都干连在内,不给他个知道,下次越发狂纵。”这茗烟无故
就要欺压人的,如今得了这信,又有贾蔷助着,便一头进来找金荣。也不叫“金相
公”了,只说:“姓金的,你什么东西!”贾蔷遂跺一跺靴子,故意整整衣服,看
看日影儿说:“正时候了。”遂先向贾瑞说有事要早走一步。贾瑞不敢止他,只得
随他去了。

这里茗烟走进来,便一把揪住金荣问道:“我们屁股不,管你相干?
横竖没你的爹罢了!说你是好小子,出来动一动你茗大爷!”吓的满屋中子弟都
忙忙的痴望。贾瑞忙喝:“茗烟不得撒野!”金荣气黄了脸,说:“反了!奴才小
子都敢如此,我只和你主子说。”便夺手要去抓打宝玉。秦钟刚转出身来,听得脑
后飕的一声,早见一方砚瓦飞来,并不知系何人打来,却打了贾蓝贾菌的座上。这
贾蓝贾菌亦系荣府近派的重孙。这贾菌少孤,其母疼爱非常,书房中与贾蓝最好,
所以二人同坐。谁知这贾菌年纪虽小,志气最大,极是淘气不怕人的。他在位上,
冷眼看见金荣的朋友暗助金荣,飞砚来打茗烟,偏打错了落在自己面前,将个磁砚
水壶儿打粉碎,溅了一书墨水。贾菌如何依得,便骂:“好囚攮的们!这不都动了
手了么!”骂着,也便抓起砚台来要飞。贾蓝是个省事的,忙按住砚台,忙劝道:
“好兄弟,不与咱们相干。”贾菌如何忍得住,见按住砚台,他便两手抱起书箧子
来照这边扔去。终是身小力薄,却扔不到,反扔到宝玉秦钟案上就落下来了。只听
豁啷一响,砸在桌上,书本、纸片、笔、砚等物撒了一桌,又把宝玉的一碗茶也砸
得碗碎茶流。那贾菌即便跳出来,要揪打那飞砚的人。金荣此时随手抓了一根毛竹
大板在手,地狭人多,那里经得舞动长板。茗烟早吃了一下,乱嚷:“你们还不来
动手?”宝玉还有几个小厮,一名扫红,一名锄药,一名墨雨,这三个岂有不淘气
的,一齐乱嚷:“小妇养的!动了兵器了!”墨雨遂掇起一根门闩,扫红锄药手中
都是马鞭子,蜂拥而上。贾瑞急得拦一回这个,劝一回那个,谁听他的话?肆行大
乱。众顽童也有帮着打太平拳助乐的,也有胆小藏过一边的,也有立在桌上拍着手
乱笑、喝着声儿叫打的:登时鼎沸起来。

外边几个大仆人李贵等听见里边作反起来,忙都进来一齐喝住,问是何故,众
声不一,这一个如此说,那一个又如彼说。李贵且喝骂了茗烟等四个一顿,撵了出
去。秦钟的头早撞在金荣的板上,打去一层油皮,宝玉正拿褂襟子替他揉,见喝住
了众人,便命:“李贵,收书,拉马来!我去回太爷去!我们被人欺负了,不敢说别
的,守礼来告诉瑞大爷,瑞大爷反派我们的不是,听着人家骂我们,还调唆人家打
我们。茗烟见人欺负我,他岂有不为我的;他们反打伙儿打了茗烟,连秦钟的头也
打破了。还在这里念书么?”李贵劝道:“哥儿不要性急,太爷既有事回家去了,
这会子为这点子事去聒噪他老人家,倒显的咱们没礼似的。依我的主意,那里的事
情那里了结,何必惊动老人家。这都是瑞大爷的不是,太爷不在家里,你老人家就
是这学里的头脑了,众人看你行事。众人有了不是,该打的打,该罚的罚,如何等
闹到这步田地还不管呢?”贾瑞道:“我吆喝着都不听。”李贵道:“不怕你老人
家恼我:素日你老人家到底有些不是,所以这些兄弟不听。就闹到太爷跟前去,连
你老人家也脱不了的。还不快作主意撕掳开了罢!”宝玉道:“撕掳什么?我必要
回去的!”秦钟哭道:“有金荣在这里,我是要回去的了。”宝玉道:“这是为什
么?难道别人家来得,咱们倒来不得的?我必回明白众人,撵了金荣去!”又问李贵:
“这金荣是那一房的亲戚?”李贵想一想,道:“也不用问了。若说起那一房亲戚,
更伤了兄弟们的和气了。”

茗烟在窗外道:“他是东府里璜大奶奶的侄儿,什么硬挣仗腰子的,也来吓我
们!璜大奶奶是他姑妈。你那姑妈只会打旋磨儿,给我们琏二奶奶跪着借当头,我
眼里就看不起他那样主子奶奶么。”李贵忙喝道:“偏这小狗攮知道,有这些蛆嚼!”
宝玉冷笑道:“我只当是谁亲戚,原来是璜嫂子侄儿。我就去向他问问。”说着便
要走,叫茗烟进来包书。茗烟进来包书,又得意洋洋的道:“爷也不用自己去见他,
等我去找他,就说老太太有话问他呢。雇上一辆车子拉进去,当着老太太问他,岂
不省事?”李贵忙喝道:“你要死啊!仔细回去我好不好先捶了你,然后回老爷、
太太,就说宝哥儿全是你调唆。我这里好容易劝哄的好了一半,你又来生了新法儿!
你闹了学堂,不说变个法儿压息了才是,还往火里奔!”茗烟听了,方不敢做声。

此时贾瑞也生恐闹不清,自己也不干净,只得委曲着来央告秦钟,又央告宝玉。
先是他二人不肯,后来宝玉说:“不回去也罢了,只叫金荣赔不是便罢。”金荣先
是不肯,后来经不得贾瑞也来逼他权赔个不是,李贵等只得好劝金荣,说:“原来
是你起的头儿,你不这样,怎么了局呢?”金荣强不过,只得与秦钟作了个揖。宝
玉还不依,定要磕头。贾瑞只要暂息此事,又悄悄的劝金荣说:“俗语说的:‘忍
得一时忿,终身无恼闷。’”

未知金荣从也不从,下回分解。