\chapter{滥情人情误思游艺~慕雅女雅集苦吟诗}

话说薛蟠听见如此说了,气方渐平。三五日后,疼痛虽愈,伤痕未平,只装病
在家,愧见亲友。

展眼已到十月,因有各铺面伙计内有算年账要回家的,少不得家里治酒饯行。
内有一个张德辉,自幼在薛蟠当铺内揽总,家内也有了二三千金的过活,今岁也要
回家,明春方来。因说起:“今年纸札香料短少,明年必是贵的。明年先打发大小
儿上来,当铺里照管,赶端阳前,我顺路就贩些纸札香扇来卖。除去关税花销,稍
亦可以剩得几倍利息。”薛蟠听了,心下忖度:“如今我捱了打正难见人,想着要
躲避一年半载又没处去躲。天天装病,也不是常法儿。况且我长了这么大,文不文
武不武,虽说做买卖,究竟戥子、算盘从没拿过,地土风俗、远近道路又不知道。
不如也打点几个本钱和张德辉逛一年来,赚钱也罢不赚钱也罢,且躲躲羞去。二则
逛逛山水也是好的。”心内主意已定,至酒席散后,便和气平心与张德辉说知,命
他等一二日,一同前往。晚间薛蟠告诉他母亲,薛姨妈听了,虽是喜欢,但又恐他
在外生事,花了本钱倒是末事。因此不叫他去,只说:“你好歹跟着我,我还放心
些。况且也不用这个买卖,等不着这几百银子使。”薛蟠主意已定,那里肯依?只
说:“天天又说我不知世务,这个也不知,那个也不学;如今我发狠把那些没要紧
的都断了,如今要成人立事,学习买卖,又不准我了。叫我怎么样呢?我又不是个
丫头,把我关在家里,何日是个了手?况且那张德辉又是个有年纪的,咱们和他是
世家,我同他怎么得有错?我就有一时半刻不好的去处,他自然说我劝我,就是东
西贵贱行情,他是知道的,自然色色问他,何等顺利,倒不叫我去!过两日,我不
告诉家里,私自打点了走,明年发了财回来,才知道我呢!”说毕,赌气睡觉去了。

薛姨妈听他如此说,因和宝钗商议。宝钗笑道:“哥哥果然要经历正事,倒也
罢了。只是他在家里说着好听,到了外头,旧病复发,难拘束他了。但也愁不得许
多。他若是真改了,是他一生的福;若不改,妈妈也不能又有别的法子:一半尽人
力,一半听天罢了。这么大人了,若只管怕他不知世路,出不得门,干不得事,今
年关在家里,明年还是这个样儿。他既说的名正言顺,妈妈就打量着丢了一千、八
百银子,竟交与他试一试。横竖有伙计帮着他,也未必好意思哄骗他的。二则他出
去了,左右没了助兴的人,又没有倚仗的人,到了外头,谁还怕谁?有了的吃,没
了的饿着,举眼无靠,他见了这样,只怕比在家里省了事也未可知。”薛姨妈听了,
思忖半晌道:“倒是你说的是。花两个钱叫他学些乖来也值。”商议已定,一宿无
话。至次日,薛姨妈命人请了张德辉来在书房中,命薛蟠款待酒饭。自己在后廊下
隔着窗子,千言万语嘱托张德辉照管照管。张德辉满口应承,吃过饭告辞,又回说:
“十四日是上好出行日期,大世兄即刻打点行李,雇下骡子,十四日一早就长行
了。”薛蟠喜之不尽,将此话告诉了薛姨妈。

薛姨妈和宝钗香菱并两个年老的嬷嬷,连日打点行装,派下薛蟠之奶公老苍头
一名,当年谙事旧仆二名,外有薛蟠随身常使小厮二名:主仆一共六人。雇了三辆
大车,单拉行李使物,又雇了四个长行骡子。薛蟠自骑一匹家内养的铁青大走骡,
外备一匹坐马。诸事完毕,薛姨妈宝钗等连夜劝戒之言,自不必备说。至十三日,
薛蟠先去辞了他母舅,然后过来辞了贾宅诸人,贾珍等未免又有饯行之说,也不必
细述。至十四日一早,薛姨妈宝钗等直同薛蟠出了仪门,母女两个四只眼看他去了
方回来。

薛姨妈上京带来的家人不过四五房,并两三个老嬷嬷小丫头,今跟了薛蟠一
去,外面只剩了一两个男子。因此薛姨妈即日到书房,将一应陈设玩器并帘帐等物
尽行搬进来收贮,命两个跟去的男子之妻,一并也进来睡觉。又命香菱将他屋里也
收拾严紧,“将门锁了,晚上和我去睡。”宝钗道:“妈妈既有这些人作伴,不如
叫菱姐姐和我作伴去。我们园里又空,夜长了,我每夜做活,越多一个人,岂不越
好?”薛姨妈笑道:“正是我忘了,原该叫他和你去才是。我前日还和你哥哥说:
文杏又小,到三不着两的;莺儿一个人,不够伏侍的。还要买一个丫头来你使。”
宝钗道:“买的不知底里,倘或走了眼,花了钱事小,没的淘气。倒是慢慢打听着,
有知道来历的,买个还罢了。”一面说,一面命香菱收拾了衾褥妆奁,命一个老嬷
嬷并臻儿送至蘅芜院去,然后宝钗和香菱才同回园中来。

香菱向宝钗道:“我原要和太太说的,等大爷去了,我和姑娘做伴去。我又恐
怕太太多心,说我贪着园里来玩,谁知你竟说了。”宝钗笑道:“我知道你心里羡
慕这园子不是一日两日的了,只是没有个空儿。每日来一趟,慌慌张张的,也没趣
儿。所以趁着机会,越发住上一年,我也多个做伴的,你也遂了你的心。”香菱笑
道:“好姑娘!趁着这个功夫,你教给我做诗罢!”宝钗笑道:“我说你‘得陇望
蜀’呢。我劝你且缓一缓,今儿头一日进来,先出园东角门,从老太太起,各处各
人,你都瞧瞧,问候一声儿,也不必特意告诉他们搬进园来。若有提起因由儿的,
你只带口说我带了你进来做伴儿就完了。回来进了园,再到各姑娘房里走走。”香
菱应着,才要走时,只见平儿忙忙的走来。香菱忙问了好,平儿只得陪笑相问。宝
钗因向平儿笑道:“我今儿把他带了来做伴儿,正要回你奶奶一声儿。”平儿笑道:
“姑娘说的是那里的话?我竟没话答言了。”宝钗道:“这才是正理。‘店房有个
主人,庙里有个住持。’虽不是大事,到底告诉一声,就是园里坐更上夜的人,知
道添了他两个,也好关门候户的了。你回去就告诉一声罢,我不打发人说去了。”
平儿答应着,因又向香菱道:“你既来了,也不拜拜街坊去吗?”宝钗笑道:“我
正叫他去呢。”平儿道:“你且不必往我们家去,二爷病了在家里呢。”香菱答应
着去了,先从贾母处来,不在话下。

且说平儿见香菱去了,就拉宝钗悄悄说道:“姑娘可听见我们的新文没有?”
宝钗道:“我没听见新文。因连日打发我哥哥出门,所以你们这里的事,一概不知
道;连姐妹们这两天没见。”平儿笑道:“老爷把二爷打的动不得,难道姑娘就没
听见吗?”宝钗道:“早起恍惚听见了一句,也信不真。我也正要瞧你奶奶去呢,
不想你来。又是为了什么打他?”平儿咬牙骂道:“都是那什么贾雨村,半路途中
那里来的饿不死的野杂种!认了不到十年,生了多少事出来。今年春天,老爷不知
在那个地方看见几把旧扇子,回家来,看家里所有收着的这些好扇子,都不中用了,
立刻叫人各处搜求。谁知就有个不知死的冤家,混号儿叫做石头呆子,穷的连饭也
没的吃,偏偏他家就有二十把旧扇子,死也不肯拿出大门来。二爷好容易烦了多少
情,见了这个人,说之再三,他把二爷请了到他家里坐着,拿出这扇子来略瞧了一
瞧。据二爷说,原是不能再得的,全是湘妃、棕竹、麋鹿、玉竹的,皆是古人写画
真迹。回来告诉了老爷,便叫买他的,要多少银子给他多少。偏那石呆子说:‘我
饿死冻死,一千两银子一把,我也不卖。’老爷没法了,天天骂二爷没能为。已经
许他五百银子,先兑银子,后拿扇子,他只是不卖,只说:‘要扇子先要我的命!’
姑娘想想,这有什么法子?谁知那雨村没天理的听见了,便设了法子,讹他拖欠官
银,拿他到了衙门里去,说:‘所欠官银,变卖家产赔补。’把这扇子抄了来,做
了官价,送了来。那石呆子如今不知是死是活。老爷问着二爷说:‘人家怎么弄了
来了?’二爷只说了一句:‘为这点子小事弄的人家倾家败产,也不算什么能为。’
老爷听了就生了气,说二爷拿话堵老爷呢。这是第一件大的。过了几日,还有几件
小的,我也记不清,所以都凑在一处,就打起来了。也没拉倒用板子棍子,就站着,
不知他拿什么东西打了一顿,脸上打破了两处。我们听见姨太太这里有一种药上棒
疮的,姑娘寻一丸给我呢。”宝钗听了,忙命莺儿去找了两丸来与平儿。宝钗道:
“既这样,你去替我问候罢,我就不去了。”平儿向宝钗答应着去了,不在话下。

且说香菱见了众人之后,吃过晚饭,宝钗等都往贾母处去了,自己便往潇湘馆
中来。此时黛玉已好了大半了,见香菱也进园来住,自是喜欢。香菱因笑道:“我
这一进来了,也得空儿,好歹教给我做诗,就是我的造化了。”黛玉笑道:“既要
学做诗,你就拜我为师。我虽不通,大略也还教的起你。”香菱笑道:“果然这样,
我就拜你为师,你可不许腻烦的。”黛玉道:“什么难事,也值得去学?不过是起、
承、转、合,当中承、转是两副对子,平声的对仄声,虚的对实的,实的对虚的。
若是果有了奇句,连平仄虚实不对都使得的。”香菱笑道:“怪道我常弄本旧诗,
偷空儿看一两首,又有对的极工的,又有不对的。又听见说,‘一三五不论,二四
六分明。’看古人的诗上,亦有顺的,亦有二四六上错了的。所以天天疑惑。如今
听你一说,原来这些规矩,竟是没事的,只要词句新奇为上。”黛玉道:“正是这
个道理。词句究竟还是末事,第一是立意要紧。若意趣真了,连词句不用修饰自是
好的,这叫做‘不以词害意’。”香菱道:“我只爱陆放翁的‘重帘不卷留香久,
古砚微凹聚墨多’,说的真切有趣。”黛玉道:“断不可看这样的诗。你们因不知
诗,所以见了这浅近的就爱,一入了这个格局,再学不出来的。你只听我说,你若
真心要学,我这里有《王摩诘全集》,你且把他的五言律一百首细心揣摩透熟了,
然后再读一百二十首老杜的七言律,次之再李青莲的七言绝句读一二百首。肚子里
先有了这三个人做了底子,然后再把陶渊明、应、刘、谢、阮、庾、鲍等人的一看,
你又是这样一个极聪明伶俐的人,不用一年工夫,不愁不是诗翁了。”香菱听了,
笑道:“既这样,好姑娘,你就把这书给我拿出来,我带回去夜里念几首也是好的。”
黛玉听说,便命紫鹃将王右丞的五言律拿来,递与香菱道:“你只看有红圈的,都
是我选的,有一首念一首。不明白的问你姑娘,或者遇见我,我讲与你就是了。”
香菱拿了诗,回至蘅芜院中,诸事不管,只向灯下一首一首的读起来。宝钗连催他
数次睡觉,他也不睡。宝钗见他这般苦心,只得随他去了。

一日,黛玉方梳洗完了,只见香菱笑吟吟的送了书来,又要换杜律。黛玉笑道:
“共记得多少首?”香菱笑道:“凡红圈选的,我尽读了。”黛玉道:“可领略了
些没有?”香菱笑道:“我倒领略了些,只不知是不是,说给你听听。”黛玉笑道:
“正要讲究讨论,方能长进。你且说来我听听。”香菱笑道:“据我看来,诗的好
处,有口里说不出来的意思,想去却是逼真的;又似乎无理的,想去竟是有理有情
的。”黛玉笑道:“这话有了些意思!但不知你从何处见得?”香菱笑道:“我看
他《塞上》一首,内一联云:‘大漠孤烟直,长河落日圆。’想来烟如何直?日自
然是圆的。这‘直’字似无理,‘圆’字似太俗。合上书一想,倒像是见了这景的。
要说再找两个字换这两个,竟再找不出两个字来。再还有:‘日落江湖白,潮来天
地青。’这‘白’‘青’两个字,也似无理,想来必得这两个字才形容的尽,念在
嘴里,倒像有几千斤重的一个橄榄似的。还有‘渡头馀落日,墟里上孤烟’,这‘馀’
字合‘上’字,难为他怎么想来!我们那年上京来,那日下晚便挽住船,岸上又没
有人,只有几棵树。远远的几家人家作晚饭,那个烟竟是青碧连云。谁知我昨儿晚
上看了这两句,倒像我又到了那个地方去了。”

正说着,宝玉和探春来了,都入座听他讲诗。宝玉笑道:“既是这样,也不用
看诗,‘会心处不在远’,听你说了这两句,可知三昧你已得了。”黛玉笑道:“你
说他这‘上孤烟’好,你还不知他这一句还是套了前人的来。我给你这一句瞧瞧,
更比这个淡而现成。”说着,便把陶渊明的“暧暧远人村,依依墟里烟”翻了出来,
递给香菱。香菱瞧了,点头叹赏,笑道:“原来‘上’字是从‘依依’两个字上化
出来的。”宝玉大笑道:“你已得了。不用再讲,要再讲,倒学离了。你就做起来
了,必是好的。”探春笑道:“明儿我补一个柬来,请你入社。”香菱道:“姑娘
何苦打趣我!我不过是心里羡慕,才学这个玩罢了。”探春黛玉都笑道:“谁不是
玩?难道我们是认真做诗呢!要说我们真成了诗,出了这园子,把人的牙还笑掉了
呢。”宝玉道:“这也算自暴自弃了。前儿我在外头和相公们商画儿,他们听见咱
们起诗社,求我把稿子给他们瞧瞧,我就写了几首给他们看看。谁不是真心叹服?
他们抄了刻去了。”探春黛玉忙问道:“这是真话么?”宝玉笑道:“说谎的是那
架上鹦哥。”黛玉探春听说,都道:“你真真胡闹!且别说那不成诗,便成诗,我
们的笔墨,也不该传到外头去。”宝玉道:“这怕什么?古来闺阁中笔墨不要传出
去,如今也没人知道呢。”说着,只见惜春打发了入画来请宝玉,宝玉方去了。

香菱又逼着换出杜律,又央黛玉探春二人:“出个题目让我诌去,诌了来替我
改正。”黛玉道:“昨夜的月最好,我正要诌一首未诌成。你就做一首来。‘十四
寒’的韵,由你爱用那几个字去。”香菱听了,喜的拿着诗回来,又苦思一回,做
两句诗;又舍不得杜诗,又读两首:如此茶饭无心,坐卧不定。宝钗道:“何苦自
寻烦恼?都是颦儿引的你,我和他算帐去!你本来呆头呆脑的,再添上这个,越发弄
成个呆子了。”香菱笑道:“好姑娘,别混我。”一面说,一面做了一首。先给宝
钗看了,笑道:“这个不好,不是这个做法。你别害臊,只管拿了给他瞧去,看是
他怎么说。”香菱听了,便拿了诗找黛玉。黛玉看时,只见写道是:
月桂中天夜色寒,清光皎皎影团团。
诗人助兴常思玩,野客添愁不忍观。
翡翠楼边悬玉镜,珍珠帘外挂冰盘。
良宵何用烧银烛,晴彩辉煌映画栏。
黛玉笑道:“意思却有,只是措词不雅。皆因你看的诗少,被他缚住了。把这首诗
丢开,再做一首。只管放开胆子去做。”

香菱听了,默默的回来,越发连房也不进去,只在池边树下。或坐在山石上出
神,或蹲在地下抠地,来往的人都诧异。李纨、宝钗、探春、宝玉等听得此言,都
远远的站在山坡上瞧着他笑。只见他皱一回眉,又自己含笑一回。宝钗笑道:“这
个人定是疯了。昨夜嘟嘟哝哝,直闹到五更才睡下。没一顿饭的工夫,天就亮了,
我就听见他起来了,忙忙碌碌梳了头,就找颦儿去。一回来了,呆了一天,做了一
首又不好,自然这会子另做呢。”宝玉笑道:“这正是‘地灵人杰’,老天生人,
再不虚赋情性的。我们成日叹说:可惜他这么个人,竟俗了。谁知到底有今日!可
见天地至公。”宝钗听了,笑道:“你能够像他这苦心就好了,学什么有个不成的
吗?”宝玉不答。

只见香菱兴兴头头的,又往黛玉那边来了。探春笑道:“咱们跟了去,看他有
些意思没有。”说着,一齐都往潇湘馆来。只见黛玉正拿着诗和他讲究呢。众人因
问黛玉:“做的如何?”黛玉道:“自然算难为他了,只是还不好。这一首过于穿
凿了,还得另做。”众人因要诗看时,只见做道是:
非银非水映窗寒,试看晴空护玉盘。
淡淡梅花香欲染,丝丝柳带露初干。
只疑残粉涂金砌,恍若轻霜抹玉栏。
梦醒西楼人迹绝,馀容犹可隔帘看。

宝钗笑道:“不像吟月了,月字底下添一个‘色’字,倒还使得。你看句句倒
像是月色。——也罢了,原是‘诗从胡说来’,再迟几天就好了。”香菱自为这首
诗妙绝,听如此说,自己又扫了兴,不肯丢开手,便要思索起来。因见他姐妹们说
笑,便自己走至阶下竹前,挖心搜胆的,耳不旁听,目不别视。一时探春隔窗笑说
道:“菱姑娘,你闲闲罢。”香菱怔怔答道:“‘闲’字是‘十五删’的,错了韵
了。”众人听了,不觉大笑起来。宝钗道:“可真诗魔了!都是颦儿引的他!”黛
玉笑道:“圣人说:‘诲人不倦。’他又来问我,我岂有不说的理!”李纨笑道:
“咱们拉了他往四姑娘屋里去,引他瞧瞧画儿,叫他醒一醒才好。”说着,真个出
来拉他过藕香榭,至暖香坞中。惜春正乏倦,在床上歪着睡午觉,画缯立在壁间,
用纱罩着。众人唤醒了惜春,揭纱看时,十停方有了三停。见画上有几个美人,因
指香菱道:“凡会做诗的,都画在上头,你快学罢。”说着,玩笑了一回,各自散
去。

香菱满心中正是想诗,至晚间,对灯出了一回神,至三更以后,上床躺下,两
眼睁睁直到五更,方才蒙睡着了。一时天亮,宝钗醒了。听了一听,他安稳睡了,
心下想:“他翻腾了一夜,不知可做成了?这会子乏了,且别叫他。”正想着,只
见香菱从梦中笑道:“可是有了!难道这一首还不好吗?”宝钗听了又是可叹又是
可笑,连忙叫醒了他,问他:“得了什么?你这诚心都通了仙了。学不成诗,弄出
病来呢!”一面说,一面梳洗了,和姐妹往贾母处来。

原来香菱苦志学诗,精血诚聚,日间不能做出,忽于梦中得了八句。梳洗已毕,
便忙写出,来到沁芳亭。只见李纨与众姐妹方从王夫人处回来,宝钗正告诉他们,
说他梦中做诗说梦话,众人正笑。抬头见他来了,就都争着要诗看。

要知端底,且看下回分解。