\chapter{贾宝玉初试云雨情~刘老老一进荣国府}

却说秦氏因听见宝玉梦中唤他的乳名,心中纳闷,又不好细问。彼时宝玉迷迷
惑惑,若有所失,遂起身解怀整衣。袭人过来给他系裤带时,刚伸手至大腿处,只
觉冰冷粘湿的一片,吓的忙褪回手来,问:“是怎么了?”宝玉红了脸,把他的手
一捻。袭人本是个聪明女子,年纪又比宝玉大两岁,近来也渐省人事。今见宝玉如
此光景,心中便觉察了一半,不觉把个粉脸羞的飞红,遂不好再问。仍旧理好衣裳,
随至贾母处来,胡乱吃过晚饭,过这边来,趁众奶娘丫鬟不在旁时,另取出一件中
衣与宝玉换上。宝玉含羞央告道:“好姐姐,千万别告诉人。”袭人也含着羞悄悄
的笑问道:“你为什么——”说到这里,把眼又往四下里瞧了瞧,才又问道:“那
是那里流出来的?”宝玉只管红着脸不言语,袭人却只瞅着他笑。迟了一会,宝玉
才把梦中之事细说与袭人听。说到云雨私情,羞的袭人掩面伏身而笑。宝玉亦素喜
袭人柔媚姣俏,遂强拉袭人同领警幻所训之事,袭人自知贾母曾将他给了宝玉,也
无可推托的,扭捏了半日,无奈何,只得和宝玉温存了一番。自此宝玉视袭人更自
不同,袭人待宝玉也越发尽职了。这话暂且不提。

且说荣府中合算起来,从上至下,也有三百馀口人,一天也有一二十件事,竟
如乱麻一般,没个头绪可作纲领。正思从那一件事那一个人写起方妙,却好忽从千
里之外,芥豆之微,小小一个人家,因与荣府略有些瓜葛,这日正往荣府中来,因
此便就这一家说起,倒还是个头绪。

原来这小小之家,姓王,乃本地人氏,祖上也做过一个小小京官,昔年曾与凤
姐之祖王夫人之父认识。因贪王家的势利,便连了宗,认作侄儿。那时只有王夫人
之大兄凤姐之父与王夫人随在京的知有此一门远族,馀者也皆不知。目今其祖早
故,只有一个儿子,名唤王成,因家业萧条,仍搬出城外乡村中住了。王成亦相继
身故,有子小名狗儿,娶妻刘氏,生子小名板儿;又生一女,名唤青儿:一家四口,
以务农为业。因狗儿白日间自作些生计,刘氏又操井臼等事,青板姊弟两个无人照
管,狗儿遂将岳母刘老老接来,一处过活。这刘老老乃是个久经世代的老寡妇,膝
下又无子息,只靠两亩薄田度日。如今女婿接了养活。岂不愿意呢,遂一心一计,
帮着女儿女婿过活。

因这年秋尽冬初,天气冷将上来,家中冬事未办,狗儿未免心中烦躁,吃了几
杯闷酒,在家里闲寻气恼,刘氏不敢顶撞。因此刘老老看不过,便劝道:“姑爷,
你别嗔着我多嘴:咱们村庄人家儿,那一个不是老老实实,守着多大碗儿吃多大的
饭呢!你皆因年小时候,托着老子娘的福,吃喝惯了,如今所以有了钱就顾头不顾
尾,没了钱就瞎生气,成了什么男子汉大丈夫了!如今咱们虽离城住着,终是天子
脚下。这长安城中遍地皆是钱,只可惜没人会去拿罢了。在家跳蹋也没用!”狗儿
听了道:“你老只会在炕头上坐着混说,难道叫我打劫去不成?”刘老老说道:“谁
叫你去打劫呢?也到底大家想个方法儿才好。不然那银子钱会自己跑到咱们家里来
不成?”狗儿冷笑道:“有法儿还等到这会子呢!我又没有收税的亲戚、做官的朋
友,有什么法子可想的?就有,也只怕他们未必来理我们呢。”刘老老道“这倒也
不然。‘谋事在人,成事在天’,咱们谋到了,靠菩萨的保佑,有些机会,也未可
知。我倒替你们想出一个机会来。当日你们原是和金陵王家连过宗的。二十年前,
他们看承你们还好,如今是你们拉硬屎,不肯去就和他,才疏远起来。想当初我和
女儿还去过一遭,他家的二小姐着实爽快会待人的,倒不拿大,如今现是荣国府贾
二老爷的夫人。听见他们说,如今上了年纪,越发怜贫恤老的了,又爱斋僧布施。
如今王府虽升了官儿,只怕二姑太太还认的咱们,你为什么不走动走动?或者他还
念旧,有些好处也未可知。只要他发点好心,拔根寒毛,比咱们的腰还壮呢。”刘
氏接口道:“你老说的好,你我这样嘴脸,怎么好到他门上去?只怕他那门上人也
不肯进去告诉,没的白打嘴现世的!”

谁知狗儿利名心重,听如此说,心下便有些活动;又听他妻子这番话,便笑道:
“老老既这么说,况且当日你又见过这姑太太一次,为什么不你老人家明日就去走
一遭,先试试风头儿去?”刘老老道:“哎哟!可是说的了:‘侯门似海。’我是
个什么东西儿!他家人又不认得我,去了也是白跑。”狗儿道:“不妨,我教给你
个法儿。你竟带了小板儿先去找陪房周大爷,要见了他,就有些意思了。这周大爷
先时和我父亲交过一桩事,我们本极好的。”刘老老道:“我也知道。只是许多时
不走动,知道他如今是怎样?——这也说不得了!你又是个男人,这么个嘴脸,自然
去不得;我们姑娘年轻的媳妇儿,也难卖头卖脚的。倒还是舍着我这副老脸去碰碰,
果然有好处,大家也有益。”当晚计议已定。

次日天未明时,刘老老便起来梳洗了。又将板儿教了几句话。五六岁的孩子,
听见带了他进城逛去,喜欢的无不应承。于是刘老老带了板儿,进城至宁荣街来。
到了荣府大门前石狮子旁边,只见满门口的轿马。刘老老不敢过去,掸掸衣服,又
教了板儿几句话,然后溜到角门前,只见几个挺胸叠肚、指手画脚的人坐在大门上,
说东谈西的。刘老老只得蹭上来问:“太爷们纳福。”众人打量了一会,便问:“是
那里来的?”刘老老陪笑道:“我找太太的陪房周大爷的。烦那位太爷替我请他出
来。”那些人听了,都不理他,半日方说道:“你远远的那墙畸角儿等着,一会子
他们家里就有人出来。”内中有个年老的说道:“何苦误他的事呢?”因向刘老老
道:“周大爷往南边去了。他在后一带住着,他们奶奶儿倒在家呢。你打这边绕到
后街门上找就是了。”刘老老谢了,遂领着板儿绕至后门上,只见门上歇着些生意
担子,也有卖吃的,也有卖玩耍的,闹吵吵三二十个孩子在那里。刘老老便拉住一
个道:“我问哥儿一声:有个周大娘在家么?”那孩子翻眼瞅着道:“那个周大娘?
我们这里周大娘有几个呢,不知那一个行当儿上的?”刘老老道:“他是太太的陪
房。”那孩子道:“这个容易,你跟了我来。”引着刘老老进了后院,到一个院子
墙边,指道:“这就是他家。”又叫道:“周大妈,有个老奶奶子找你呢。”

周瑞家的在内忙迎出来,问:“是那位?”刘老老迎上来笑问道:“好啊?周
嫂子。”周瑞家的认了半日,方笑道:“刘老老,你好?你说么,这几年不见,我
就忘了。请家里坐。”刘老老一面走,一面笑说道:“你老是‘贵人多忘事’了,
那里还记得我们?”说着,来至房中,周瑞家的命雇的小丫头倒上茶来吃着。周瑞
家的又问道:“板儿长了这么大了么!”又问些别后闲话。又问刘老老:“今日还
是路过,还是特来的?”刘老老便说:“原是特来瞧瞧嫂子;二则也请请姑太太的
安。若可以领我见一见更好,若不能,就借重嫂子转致意罢了。”

周瑞家的听了,便已猜着几分来意。只因他丈夫昔年争买田地一事,多得狗儿
他父亲之力,今见刘老老如此,心中难却其意;二则也要显弄自己的体面。便笑说:
“老老你放心。大远的诚心诚意来了,岂有个不叫你见个真佛儿去的呢。论理,人
来客至,却都不与我相干。我们这里都是各一样儿:我们男的只管春秋两季地租子,
闲了时带着小爷们出门就完了;我只管跟太太奶奶们出门的事。皆因你是太太的亲
戚,又拿我当个人,投奔了我来,我竟破个例给你通个信儿去。但只一件,你还不
知道呢:我们这里不比五年前了。如今太太不理事,都是琏二奶奶当家。你打量琏
二奶奶是谁?就是太太的内侄女儿,大舅老爷的女孩儿,小名儿叫凤哥的。”刘老
老听了,忙问道:“原来是他?怪道呢,我当日就说他不错。这么说起来,我今儿
还得见他了?”周瑞家的道:“这个自然。如今有客来,都是凤姑娘周旋接待。今
儿宁可不见太太,倒得见他一面,才不枉走这一遭儿。”刘老老道:“阿弥陀佛!
这全仗嫂子方便了。”周瑞家的说:“老老说那里话。俗语说的好:‘与人方便,
自己方便。’不过用我一句话,又费不着我什么事。”说着,便唤小丫头:“到倒
厅儿上,悄悄的打听老太太屋里摆了饭了没有。”小丫头去了。

这里二人又说了些闲话。刘老老因说:“这位凤姑娘,今年不过十八九岁罢了,
就这等有本事,当这样的家,可是难得的!”周瑞家的听了道:“!我的老老,
告诉不得你了!这凤姑娘年纪儿虽小,行事儿比是人都大呢。如今出挑的美人儿似
的,少说着只怕有一万心眼子;再要赌口齿,十个会说的男人也说不过他呢。回来
你见了就知道了。就只一件,待下人未免太严些儿。”说着,小丫头回来说:“老
太太屋里摆完了饭了,二奶奶在太太屋里呢。”周瑞家的听了连忙起身,催着刘老
老:“快走,这一下来就只吃饭是个空儿,咱们先等着去。若迟了一步,回事的人
多了,就难说了。再歇了中觉,越发没时候了。”说着,一齐下了炕,整顿衣服,
又教了板儿几句话,跟着周瑞家的,逶迤往贾琏的住宅来。

先至倒厅,周瑞家的将刘老老安插住等着,自己却先过影壁,走进了院门,知
凤姐尚未出来,先找着凤姐的一个心腹通房大丫头名唤平儿的。周瑞家的先将刘老
老起初来历说明,又说:“今日大远的来请安,当日太太是常会的,所以我带了他
过来。等着奶奶下来,我细细儿的回明了,想来奶奶也不至嗔着我莽撞的。”平儿
听了,便作了个主意:“叫他们进来,先在这里坐着就是了。”周瑞家的才出去领
了他们进来,上了正房台阶,小丫头打起猩红毡帘,才入堂屋,只闻一阵香扑了脸
来,竟不知是何气味,身子就像在云端里一般。满屋里的东西都是耀眼争光,使人
头晕目眩,刘老老此时只有点头咂嘴念佛而已。于是走到东边这间屋里,乃是贾琏
的女儿睡觉之所。平儿站在炕沿边,打量了刘老老两眼,只得问个好,让了坐。刘
老老见平儿遍身绫罗,插金戴银,花容月貌,便当是凤姐儿了,才要称“姑奶奶”,
只见周瑞家的说:“他是平姑娘。”又见平儿赶着周瑞家的叫他“周大娘”,方知
不过是个有体面的丫头。于是让刘老老和板儿上了炕,平儿和周瑞家的对面坐在炕
沿上,小丫头们倒了茶来吃了。

刘老老只听见咯当咯当的响声,很似打罗筛面的一般,不免东瞧西望的,忽见
堂屋中柱子上挂着一个匣子,底下又坠着一个秤铊似的,却不住的乱晃。刘老老心
中想着:“这是什么东西?有煞用处呢?”正发呆时,陡听得当的一声又若金钟铜
磬一般,倒吓得不住的展眼儿。接着一连又是八九下,欲待问时,只见小丫头们一
齐乱跑,说:“奶奶下来了。”平儿和周瑞家的忙起身说:“老老只管坐着,等是
时候儿我们来请你。”说着迎出去了。刘老老只屏声侧耳默候。只听远远有人笑声,
约有一二十个妇人,衣裙,渐入堂屋,往那边屋内去了。又见三两个妇人,都
捧着大红油漆盒进这边来等候。听得那边说道“摆饭”,渐渐的人才散出去,只有
伺候端菜的几个人。半日鸦雀不闻。忽见两个人抬了一张炕桌来,放在这边炕上,
桌上碗盘摆列,仍是满满的鱼肉,不过略动了几样。板儿一见就吵着要肉吃,刘老
老打了他一巴掌。

忽见周瑞家的笑嘻嘻走过来,点手儿叫他。刘老老会意,于是带着板儿下炕。
至堂屋中间,周瑞家的又和他咕唧了一会子,方蹭到这边屋内,只见门外铜钩上悬
着大红洒花软帘,南窗下是炕,炕上大红条毡,靠东边板壁立着一个锁子锦的靠背
和一个引枕,铺着金线闪的大坐褥,傍边有银唾盒,那凤姐家常带着紫貂昭君套,
围着那攒珠勒子,穿着桃红洒花袄,石青刻丝灰鼠披风,大红洋绉银鼠皮裙,粉光
脂艳,端端正正坐在那里,手内拿着小铜火箸儿拨手炉内的灰。平儿站在炕沿边,
捧着小小的一个填漆茶盘,盘内一个小盖钟儿。凤姐也不接茶,也不抬头,只管拨
那灰,慢慢的道:“怎么还不请进来?”一面说,一面抬身要茶时,只见周瑞家的
已带了两个人立在面前了,这才忙欲起身、犹未起身,满面春风的问好,又嗔着周
瑞家的:“怎么不早说!”刘老老已在地下拜了几拜,问姑奶奶安。凤姐忙说:“周
姐姐,搀着不拜罢。我年轻,不大认得,可也不知是什么辈数儿,不敢称呼。”周
瑞家的忙回道:“这就是我才回的那个老老了。”凤姐点头,刘老老已在炕沿上坐
下了,板儿便躲在他背后,百般的哄他出来作揖,他死也不肯。

凤姐笑道:“亲戚们不大走动,都疏远了。知道的呢,说你们弃嫌我们,不肯
常来,不知道的那起小人,还只当我们眼里没人似的。”刘老老忙念佛道:“我们
家道艰难,走不起。来到这里,没的给姑奶奶打嘴,就是管家爷们瞧着也不像。”
凤姐笑道:“这话没的叫人恶心。不过托赖着祖父的虚名,作个穷官儿罢咧,谁家
有什么?不过也是个空架子,俗语儿说的好,‘朝廷还有三门子穷亲’呢,何况你
我。”说着,又问周瑞家的:“回了太太了没有?”周瑞家的道:“等奶奶的示下。”
凤姐儿道:“你去瞧瞧,要是有人就罢;要得闲呢,就回了,看怎么说。”周瑞家
的答应去了。

这里凤姐叫人抓了些果子给板儿吃,刚问了几句闲话时,就有家下许多媳妇儿
管事的来回话。平儿回了,凤姐道:“我这里陪客呢,晚上再来回。要有紧事,你
就带进来现办。”平儿出去,一会进来说:“我问了,没什么要紧的。我叫他们散
了。”凤姐点头。只见周瑞家的回来,向凤姐道:“太太说:‘今日不得闲儿,二
奶奶陪着也是一样,多谢费心想着。要是白来逛逛呢便罢;有什么说的,只管告诉
二奶奶。’”刘老老道:“也没甚的说,不过来瞧瞧姑太太姑奶奶,也是亲戚们的
情分。”周瑞家的道:“没有什么说的便罢;要有话,只管回二奶奶,和太太是一
样儿的。”一面说一面递了个眼色儿。刘老老会意,未语先红了脸。待要不说,今
日所为何来?只得勉强说道:“论今日初次见,原不该说的,只是大远的奔了你老
这里来,少不得说了……”刚说到这里,只听二门上小厮们回说:“东府里小大爷
进来了。”凤姐忙和刘老老摆手道:“不必说了。”一面便问:“你蓉大爷在那里
呢?”只听一路靴子响,进来了一个十七八岁的少年,面目清秀,身段苗条,美服
华冠,轻裘宝带。刘老老此时坐不是站不是,藏没处藏,躲没处躲。凤姐笑道:“你
只管坐着罢,这是我侄儿。”刘老老才扭扭捏捏的在炕沿儿上侧身坐下。

那贾蓉请了安,笑回道:“我父亲打发来求婶子,上回老舅太太给婶子的那架
玻璃炕屏,明儿请个要紧的客,略摆一摆就送来。”凤姐道:“你来迟了,昨儿已
经给了人了。”贾蓉听说,便笑嘻嘻的在炕沿上下个半跪道:“婶子要不借,我父
亲又说我不会说话了,又要挨一顿好打。好婶子,只当可怜我罢!”凤姐笑道:“也
没见我们王家的东西都是好的?你们那里放着那些好东西,只别看见我的东西才
罢,一见了就想拿了去。”贾蓉笑道:“只求婶娘开恩罢!”凤姐道:“碰坏一点
儿,你可仔细你的皮!”因命平儿拿了楼门上钥匙,叫几个妥当人来抬去。贾蓉喜
的眉开眼笑,忙说:“我亲自带人拿去,别叫他们乱碰。”说着便起身出去了。这
凤姐忽然想起一件事来,便向窗外叫:“蓉儿回来!”外面几个人接声说:“请蓉
大爷回来呢!”贾蓉忙回来,满脸笑容的瞅着凤姐,听何指示。那凤姐只管慢慢吃
茶,出了半日神,忽然把脸一红,笑道:“罢了,你先去罢。晚饭后你来再说罢。
这会子有人,我也没精神了。”贾蓉答应个是,抿着嘴儿一笑,方慢慢退去。

这刘老老方安顿了,便说道:“我今日带了你侄儿,不为别的,因他爹娘连吃
的没有,天气又冷,只得带了你侄儿奔了你老来。”说着,又推板儿道:“你爹在
家里怎么教你的?打发咱们来作煞事的?只顾吃果子!”凤姐早已明白了,听他不会
说话,因笑道:“不必说了,我知道了。”因问周瑞家的道:“这老老不知用了早
饭没有呢?”刘老老忙道:“一早就往这里赶咧,那里还有吃饭的工夫咧?”凤姐
便命快传饭来。一时周瑞家的传了一桌客馔,摆在东屋里,过来带了刘老老和板儿
过去吃饭。凤姐这里道:“周姐姐好生让着些儿,我不能陪了。”一面又叫过周瑞
家的来问道:“方才回了太太,太太怎么说了?”周瑞家的道:“太太说:‘他们
原不是一家子;当年他们的祖和太老爷在一处做官,因连了宗的。这几年不大走动。
当时他们来了,却也从没空过的。如今来瞧我们,也是他的好意,别简慢了他。要
有什么话,叫二奶奶裁夺着就是了。’”凤姐听了说道:“怪道既是一家子,我怎
么连影儿也不知道!”

说话间,刘老老已吃完了饭,拉了板儿过来,舔唇咂嘴的道谢。凤姐笑道:“且
请坐下,听我告诉你:方才你的意思,我已经知道了。论起亲戚来,原该不等上门
就有照应才是;但只如今家里事情太多,太太上了年纪,一时想不到是有的。我如
今接着管事,这些亲戚们又都不大知道,况且外面看着虽是烈烈轰轰,不知大有大
的难处,说给人也未必信。你既大远的来了,又是头一遭儿和我张个口,怎么叫你
空回去呢?可巧昨儿太太给我的丫头们作衣裳的二十两银子还没动呢,你不嫌少,
先拿了去用罢。”那刘老老先听见告艰苦,只当是没想头了;又听见给他二十两银
子,喜的眉开眼笑道:“我们也知道艰难的,但只俗语说的:‘瘦死的骆驼比马还
大’呢。凭他怎样,你老拔一根寒毛比我们的腰还壮哩。”周瑞家的在旁听见他说
的粗鄙,只管使眼色止他。凤姐笑而不睬,叫平儿把昨儿那包银子拿来,再拿一串
钱,都送至刘老老跟前。凤姐道:“这是二十两银子,暂且给这孩子们作件冬衣罢。
改日没事,只管来逛逛,才是亲戚们的意思。天也晚了,不虚留你们了,到家该问
好的都问个好儿罢。”一面说,一面就站起来了。

刘老老只是千恩万谢的,拿了银钱,跟着周瑞家的走到外边。周瑞家的道:“我
的娘!你怎么见了他倒不会说话了呢?开口就是‘你侄儿’。我说句不怕你恼的话:
就是亲侄儿也要说的和软些儿。那蓉大爷才是他的侄儿呢。他怎么又跑出这么个侄
儿来了呢!”刘老老笑道:“我的嫂子!我见了他,心眼儿里爱还爱不过来,那里
还说的上话来?”二人说着,又到周瑞家坐了片刻。刘老老要留下一块银子给周家
的孩子们买果子吃,周瑞家的那里放在眼里,执意不肯。刘老老感谢不尽,仍从后
门去了。

未知去后如何,且听下回分解。