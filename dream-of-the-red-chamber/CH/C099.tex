\chapter{守官箴恶奴同破例~阅邸报老舅自担惊}

话说凤姐见贾母和薛姨妈为黛玉伤心,便说:“有个笑话儿说给老太太和姑妈
听。”未从开口,先自笑了。因说道:“老太太和姑妈打谅是那里的笑话儿?就是
咱们家的那二位新姑爷新媳妇啊。”贾母道:“怎么了?”凤姐拿手比着道:“一
个这么坐着,一个这么站着;一个这么扭过去,一个这么转过来;一个又——”说
到这里,贾母已经大笑起来,说道:“你好生说罢。倒不是他们两口儿,你倒把人
怄的受不得了。”薛姨妈也笑道:“你往下直说罢,不用比了。”凤姐才说道:“刚
才我到宝兄弟屋里,我听见好几个人笑。我只道是谁,巴着窗户眼儿一瞧,原来宝
妹妹坐在炕沿上,宝兄弟站在地下。宝兄弟拉着宝妹妹的袖子,口口声声只叫:‘宝
姐姐,你为什么不会说话了?你这么说一句话,我的病包管全好。’宝妹妹却扭着
头,只管躲。宝兄弟又作了一个揖,上去又拉宝妹妹的衣裳。宝妹妹急的一扯,宝
兄弟自然病后是脚软的,索性一栽,栽在宝妹妹身上了。宝妹妹急的红了脸,说道:
‘你越发比先不尊重了。’”说到这里,贾母和薛姨妈都笑起来。凤姐又道:“宝
兄弟站起来,又笑着说:‘亏了这一栽,好容易才栽出你的话来了。’”薛姨妈笑
道:“这是宝丫头古怪。这有什么?既作了两口儿,说说笑笑的怕什么?他没见他琏
二哥和你。”凤姐儿红了脸笑道:“这是怎么说?我饶说笑话儿给姑妈解闷儿,姑
妈反倒拿我打起卦来了。”贾母也笑道:“要这么着才好。夫妻固然要和气,也得
有个分寸儿。我爱宝丫头就在这尊重上头。只是我愁宝玉还是那么傻头傻脑的,这
么说起来,比头里竟明白多了。你再说说还有什么笑话儿没有?”凤姐道:“明儿
宝玉圆了房儿,亲家太太抱了外孙子,那时候儿不更是笑话儿了么?”贾母笑道:
“猴儿!我在这里和姨太太想你林妹妹,你来怄个笑儿还罢了,怎么臊起皮来了。
你不叫我们想你林妹妹?你不用太高兴了,你林妹妹恨你,将来你别独自一个儿到
园里去,提防他拉着你不依!”凤姐笑道:“他倒不怨我,他临死咬牙切齿,倒恨
宝玉呢。”贾母薛姨妈听着还道是玩话儿,也不理会,便道:“你别胡扯拉了。你
去叫外头挑个很好的日子给你宝兄弟圆了房儿罢。”凤姐答应着,又说了一回话儿,
便出去叫人择了吉日,重新摆酒唱戏请人,不在话下。

却说宝玉虽然病好,宝钗有时高兴,翻书观看,谈论起来,宝玉所有常见的尚
可记忆,若论灵机儿大不似先,连他自己也不解。宝钗明知是“通灵”失去,所以
如此。倒是袭人时常说他:“你为什么把从前的灵机儿都没有了?倒是忘了旧毛病
也好,怎么脾气还照旧,独道理上更糊涂了呢?”宝玉听了,并不生气,反是嘻嘻
的笑。有时宝玉顺性胡闹,亏宝钗劝着,略觉收敛些。袭人倒可少费些唇舌,惟知
悉心伏侍。别的丫头素仰宝钗贞静和平,各人心服,无不安静。只有宝玉到底是爱
动不爱静的,时常要到园里去逛。贾母等一则怕他招受寒暑,二则恐他睹景伤情,
虽黛玉之柩已寄放城外庵中,然而潇湘馆依然人亡屋在,不免勾起旧病来,所以也
不使他去。况且亲戚姊妹们,为宝琴已回到薛姨妈那边去了,史湘云因史侯回京,
也接了家去了,又有了出嫁的日子,所以不大常来,只有宝玉娶亲那一日与吃喜酒
这天来过两次,也只在贾母那边住下,为着宝玉已经娶过亲的人,又想自己就要出
嫁的,也不肯如从前的诙谐谈笑,就是有时过来,也只和宝钗说话,见了宝玉,不
过问好而已。那邢岫烟却是因迎春出嫁之后,便随着邢夫人过去。李家姊妹也另住
在外,即同着李婶娘过来,亦不过到太太们和姐妹们处请安问好,即回到李纨那里
略住一两天就去了。所以园内的只有李纨、探春、惜春了。贾母还要将李纨等挪进
来,为着元妃薨后家中事情接二连三,也无暇及此。现今天气一天热似一天,园里
尚可住得,等到秋天再挪。此是后话,暂且不提。

且说贾政带了几个在京请的幕友,晓行夜宿,一日到了本省,见过上司,即到
任拜印受事,便查盘各属州县米粮仓库。贾政向来作京官,只晓得郎中事务都是一
景儿的事情,就是外任,原是学差,也无关于吏治上。所以外省州县折收粮米、勒
索乡愚这些弊端,虽也听见别人讲究,却未尝身亲其事,只有一心做好官。便与幕
宾商议,出示严禁,并谕以一经查出,必定详参揭报。初到之时,果然胥吏畏惧,
便百计钻营,偏遇贾政这般古执。那些家人跟了这位老爷在都中一无出息,好容易
盼到主人放了外任,便在京指着在外发财的名儿向人借贷做衣裳,装体面,心里想
着到了任,银钱是容易的了。不想这位老爷呆性发作,认真要查办起来,州县馈送
一概不受。门房、签押等人心里盘算道:“我们再挨半个月,衣裳也要当完了,帐
又逼起来,那可怎么样好呢?眼见得白花花的银子,只是不能到手。”那些长随也
道:“你们爷们到底还没花什么本钱来的。我们才冤,花了若干的银子,打了个门
子,来了一个多月,连半个钱也没见过。想来跟这个主儿是不能捞本儿的了。明儿
我们齐打伙儿告假去。”次日果然聚齐都来告假。贾政不知就里,便说:“要来也
是你们,要去也是你们。既嫌这里不好,就都请便。”那些长随怨声载道而去。

只剩下些家人,又商议道:“他们可去的去了,我们去不了的,到底想个法儿
才好。”内中有一个管门的叫李十儿,便说:“你们这些没能耐的东西,着什么急
呢!我见这‘长’字号儿的在这里,不犯给他出头。如今都饿跑了,瞧瞧十太爷的
本领,少不得本主儿依我。只是要你们齐心,打伙儿弄几个钱,回家受用;若不随
我,我也不管了,横竖拚得过你们。”众人都说:“好十爷,你还主儿信得过,若
你不管,我们实在是死症了。”李十儿道:“别等我出了头得了银钱,又说我得了
大分儿了,窝儿里反起来,大家没意思。”众人道:“你万安,没有的事。就没有
多少,也强似我们腰里掏钱。”

正说着,只见粮房书办走来找周二爷。李十儿坐在椅子上,跷着一只腿,挺着
腰,说道:“找他做什么?”书办便垂手陪着笑,说道:“本官到了一个多月的任,
这些州县太爷见得本官的告示利害,知道不好说话,到了这时候,都没有开仓。若
是过了漕,你们太爷们来做什么的?”李十儿说:“你别混说,老爷是有根蒂的,
说到那里是要办到那里。这两天原要行文催兑,因我说了缓几天,才歇的。你到底
找我们周二爷做什么?”书办道:“原为打听催文的事,没有别的。”李十儿道:
“越发胡说。方才我说催文,你就信嘴胡诌。可别鬼鬼祟祟来讲什么帐,我叫本官
打了你,退你!”书办道:“我在这衙门内已经三代了,外头也有些体面,家里还
过得,就规规矩矩伺候本官升了还能够,不像那些等米下锅的。”说着,回了一声:
“二太爷,我走了。”李十儿便站起,堆着笑说:“这么不禁玩,几句话就脸急了?”
书办道:“不是我脸急,若再说什么,岂不带累了二太爷的清名呢?”李十儿过来
拉着书办的手,说:“你贵姓啊?”书办道:“不敢,我姓詹,单名是个会字。从
小儿也在京里混了几年。”李十儿道:“詹先生,我是久闻你的名的。我们弟兄们
是一样的。有什么话,晚上到这里,咱们说一说。”书办也说:“谁不知道李十太
爷是能事的,把我一诈就吓毛了。”大家笑着走开。那晚便与书办咕唧了半夜。

第二天,拿话去探贾政,被贾政痛骂了一顿。隔一天拜客,里头吩咐伺候,外
头答应了。停了一会子,打点已经三下了,大堂上没有人接鼓,好容易叫个人来打
了鼓。贾政踱出暖阁,站班喝道的衙役只有一个。贾政也不查问,在墀下上了轿,
等轿夫,又等了好一回,来齐了,抬出衙门,那个炮只响得一声。吹鼓亭的鼓手,
只有一个打鼓,一个吹号筒。贾政便也生气,说:“往常还好,怎么今儿不齐集至
此?”抬头看那执事,却是搀前落后。勉强拜客回来,便传误班的要打。有的说因
没有帽子误的;有的说是号衣当了误的;又有说是三天没吃饭抬不动的。贾政生气,
打了一两个,也就罢了。隔一天管厨房的上来要钱,贾政将带来银两付了。以后便
觉样样不如意,比在京的时候倒不便了好些。无奈,便唤李十儿问道:“跟我来这
些人,怎么都变了?你也管管。现在带来银两早使没有了,藩库俸银尚早,该打发
京里取去。”李十儿道:“奴才那一天不说他们?不知道怎么样,这些人都是没
精打彩的,叫奴才也没法儿。老爷说家里取银子,取多少?现在打听节度衙门这几
天有生日,别的府道老爷都上千上万的送了,我们到底送多少呢?”贾政道:“为
什么不早说?”李十儿说:“老爷最圣明的。我们新来乍到,又不与别位老爷很来
往,谁肯送信?巴不得老爷不去,好想老爷的美缺呢。”贾政道:“胡说!我这官是
皇上放的,不给节度做生日,便叫我不做不成!”李十儿笑着回道:“老爷说的也
不错。京里离这里很远,凡百的事,都是节度奏闻。他说好便好,说不好便吃不住。
到得明白,已经迟了。就是老太太、太太们,那个不愿意老爷在外头烈烈轰轰的做
官呢?”

贾政听了这话,也自然心里明白,道:“我正要问你,为什么不说起来?”李
十儿回说:“奴才本不敢说,老爷既问到这里,若不说,是奴才没良心;若说了,
少不得老爷又生气。”贾政道:“只要说得在理。”李十儿说道:“那些书吏衙役,
都是花了钱买着粮道的衙门,那个不想发财?俱要养家活口。自从老爷到任,并没
见为国家出力,倒先有了口碑载道。”贾政道:“民间有什么话?”李十儿道:“百
姓说:‘凡有新到任的老爷,告示出的越利害,越是想钱的法儿。州县害怕了,好
多多的送银子。’收粮的时候,衙门里便说,新道爷的法令;明是不敢要钱,这一
留难叨蹬,那些乡民心里愿意花几个钱,早早了事。所以那些人不说老爷好,反说
不谙民情。便是本家大人是老爷最相好的,他不多几年,已巴到极顶的分儿,也只
为识时达务,能够上和下睦罢了。”贾政听到这话,道:“胡说,我就不识时务吗?
若是上和下睦,叫我与他们猫鼠同眠吗!”李十儿回说道:“奴才为着这点心儿不
敢掩住,才这么说。若是老爷就是这样做去,到了功不成、名不就的时候,老爷说
奴才没良心,有什么话不告诉老爷。”贾政道:“依你怎么做才好?”李十儿道:
“也没有别的,趁着老爷的精神年纪,里头的照应,老太太的硬朗,为顾着自己就
是了。不然,到不了一年,老爷家里的钱也都贴补完了,还落了自上至下的人抱怨,
都说老爷是做外任的,自然弄了钱藏着受用。倘遇着一两件为难的事,谁肯帮着老
爷?那时办也办不清,悔也悔不及。”贾政道:“据你一说,是叫我做贪官吗?送了
命还不要紧,必定将祖父的功勋抹了才是?”李十儿回道:“老爷极圣明的人,
没看见旧年犯事的几位老爷吗?这几位都与老爷相好,老爷常说是个做清官的,如
今名在那里?现有几位亲戚,老爷向来说他们不好的,如今升的升,迁的迁。只在
要做的好就是了。老爷要知道:民也要顾,官也要顾。若是依着老爷,不准州县得
一个大钱,外头这些差使谁办?只要老爷外面还是这样清名声原好,里头的委屈,
只要奴才办去,关碍不着老爷的。奴才跟主儿一场,到底也要掏出良心来。”

贾政被李十儿一番言语,说得心无主见,道:“我是要保性命的,你们闹出来
不与我相干。”说着,便踱了进去。李十儿便自己做起威福。钩连内外,一气的哄
着贾政办事,反觉得事事周到,件件随心。所以贾政不但不疑,反都相信。便有几
处揭报,上司见贾政古朴忠厚,也不查察。惟是幕友们耳目最长,见得如此,得便
用言规谏,无奈贾政不信,也有辞去的,也有与贾政相好在内维持的。于是,漕务
事毕,尚无陨越。

一日,贾政无事,在书房中看书。签押上呈进一封书子,外面官封,上开着“镇
守海门等处总制公文一角,飞递江西粮道衙门”。贾政拆封看时,只见上写道:

金陵契好,桑梓情深。昨岁供职来都,窃喜常依座右;仰蒙雅爱,许结朱陈,
至今佩德勿谖。只因调任海疆,未敢造次奉求,衷怀歉仄,自叹无缘。今幸戟遥
临,快慰平生之愿。正申燕贺,先蒙翰教,边帐光生,武夫额手。虽隔重洋,尚叨
樾荫,想蒙不弃卑寒,希望茑萝之附。小儿已承青盼,淑媛素仰芳仪。如蒙践诺,
即遣冰人。途路虽遥,一水可通,不敢云百辆之迎,敬备仙舟以俟。兹修寸幅,恭
贺升祺,并求金允。临颖不胜待命之至。世弟周琼顿首。
贾政看了,心想:“儿女姻缘果然有一定的。旧年因见他就了京职,又是同乡的人,
素来相好,又见那孩子长得好,在席间原提起这件事。因未说定,也没有与他们说
起。后来他调了海疆,大家也不说了。不料我今升任至此,他写书来问。我看起门
户却也相当,与探春倒也相配。但是我并未带家眷,只可写字与他商议。”正在踌
躇,只见门上传进一角文书,是议取到省会议事件,贾政只得收拾上省,候节度派
委。

一日,在公馆闲坐,见桌上堆着许多邸报。贾政一一看去,见刑部一本:“为
报明事,会看得金陵籍行商薛蟠……”贾政便吃惊道:“了不得,已经提本了!”
随用心看下去,是“薛蟠殴伤张三身死,串嘱尸证,捏供误杀一案”。贾政一拍桌
道:“完了!”只得又看底下,是:

据京营节度使咨称:“缘薛蟠籍隶金陵,行过太平县,在李家店歇宿,与店内
当槽之张三素不相认。于某年月日,薛蟠令店主备酒邀请太平县民吴良同饮,令当
槽张三取酒。因酒不甘,薛蟠令换好酒。张三因称酒已沽定,难换。薛蟠因伊倔强,
将酒照脸泼去,不期去势甚猛,恰值张三低头拾箸,一时失手,将酒碗掷在张三囟
门,皮破血出,逾时殒命。李店主趋救
不及,随向张三之母告知。伊母张王氏往看,见已身死,随喊地保,赴县呈报。
前署县诣验,仵作将骨破一寸三分及腰眼一伤,漏报填格,详府审转。看得薛蟠实
系泼酒失手,掷碗误伤张三身死,将薛蟠照过失杀人,准斗杀罪收赎。”等因前来。
臣等细阅各犯证尸亲前后供词不符,且查斗杀律注云:相争为斗,相打为殴。必实
无争斗情形,邂逅身死,方可以过失杀定拟。应令该节度审明实情,妥拟具题。今
据该节度疏称薛蟠因张三不肯换酒,醉后拉着张三右手,先殴腰眼一拳,张三被殴
回骂,薛蟠将碗掷出,致伤囟门深重,骨碎脑破,立时殒命。是张三之死实由薛蟠
以酒碗砸伤深重致死,自应以薛蟠拟抵,将薛蟠依斗杀律拟绞监候。吴良拟以杖徒。
承审不实之府州县,应请……
以下注着“此稿未完”。

贾政因薛姨妈之托,曾托过知县;若请旨革审起来,牵连着自己,好不放心。
即将下一本开看,偏又不是,只好翻来覆去,将报看完,终没有接这一本的。心中
狐疑不定,更加害怕起来。正在纳闷,只见李十儿进来:“请老爷到官厅伺候去,
大人衙门已经打了二鼓了。”贾政只是发怔,没有听见。李十儿又请一遍。贾政道:
“这便怎么处?”李十儿道:“老爷有什么心事?”贾政将看报之事说了一遍。李
十儿道:“老爷放心。若是部里这么办了,还算便宜薛大爷呢。奴才在京的时候,
听见薛大爷在店里叫了好些媳妇儿,都喝醉了生事,直把个当槽儿的活活儿打死
了。奴才听见不但是托了知县,还求琏二爷去花了好些钱,各衙门打通了才提的。
不知道怎么部里没有弄明白。如今就是闹破了,也是官官相护的,不过认个承审不
实,革职处分罢咧,那里还肯认得银子听情的话呢?老爷不用想,等奴才再打听罢,
倒别误了上司的事。”贾政道:“你们那里知道?只可惜那知县听了一个情,把这
个官都丢了,还不知道有罪没有罪。”李十儿道:“如今想他也无益,外头伺候着
好半天了,请老爷就去罢。”

贾政不知节度传办何事,且听下回分解。