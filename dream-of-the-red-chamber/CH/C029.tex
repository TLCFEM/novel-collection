\chapter{享福人福深还祷福~多情女情重愈斟情}

话说宝玉正自发怔,不想黛玉将手帕子扔了来,正碰在眼睛上,倒唬了一跳,
问:“这是谁?”黛玉摇着头儿笑道:“不敢,是我失了手。因为宝姐姐要看呆雁,
我比给他看,不想失了手。”宝玉揉着眼睛,待要说什么,又不好说的。

一时凤姐儿来了。因说起初一日在清虚观打醮的事来,约着宝钗、宝玉、黛玉
等看戏去。宝钗笑道:“罢,罢,怪热的,什么没看过的戏!我不去。”凤姐道:
“他们那里凉快,两边又有楼。咱们要去,我头几天先打发人去,把那些道士都赶
出去,把楼上打扫了,挂起帘子来,一个闲人不许放进庙去,才是好呢。我已经回
了太太了,你们不去,我自家去。这些日子也闷的很了,家里唱动戏,我又不得舒
舒服服的看。”贾母听说,就笑道:“既这么着,我和你去。”凤姐听说,笑道:
“老祖宗也去?敢仔好,可就是我又不得受用了。”贾母道:“到明儿我在正面楼
上,你在傍边楼上,你也不用到我这边来立规矩,可好不好?”凤姐笑道:“这就
是老祖宗疼我了。”贾母因向宝钗道:“你也去,连你母亲也去;长天老日的,在
家里也是睡觉。”宝钗只得答应着。

贾母又打发人去请了薛姨妈,顺路告诉王夫人,要带了他们姊妹去。王夫人因
一则身上不好,二则预备元春有人出来,早已回了不去的;听贾母如此说,笑道:
“还是这么高兴。打发人去到园里告诉,有要逛去的,只管初一跟老太太逛去。”
这个话一传开了,别人还可已,只是那些丫头们,天天不得出门槛儿,听了这话谁
不要去,就是各人的主子懒怠去,他也百般的撺掇了去:因此李纨等都说去。贾母
心中越发喜欢,早已吩咐人去打扫安置,不必细说。

单表到了初一这一日,荣国府门前车辆纷纷,人马簇簇,那底下执事人等,听
见是贵妃做好事,贾母亲去拈香,况是端阳佳节,因此凡动用的物件,一色都是齐
全的,不同往日。少时贾母等出来,贾母坐一乘八人大轿,李氏、凤姐、薛姨妈每
人一乘四人轿,宝钗、黛玉二人共坐一辆翠盖珠缨八宝车,迎春、探春、惜春三人
共坐一辆朱轮华盖车。然后贾母的丫头鸳鸯、鹦鹉、琥珀、珍珠,黛玉的丫头紫鹃、
雪雁、鹦哥,宝钗的丫头莺儿、文杏,迎春的丫头司棋、绣橘,探春的丫头侍书、
翠墨,惜春的丫头入画、彩屏,薛姨妈的丫头同喜、同贵,外带香菱,香菱的丫头
臻儿,李氏的丫头素云、碧月,凤姐儿的丫头平儿、丰儿、小红,并王夫人的两个
丫头金钏、彩云,也跟了凤姐儿来。奶子抱着大姐儿,另在一辆车上。还有几个粗
使的丫头,连上各房的老嬷嬷奶妈子,并跟着出门的媳妇子们,黑压压的站了一街
的车。那街上的人见是贾府去烧香,都站在两边观看。那些小门小户的妇女,也都
开了门在门口站着,七言八语,指手画脚,就像看那过会的一般。只见前头的全副
执事摆开,一位青年公子骑着银鞍白马,彩辔朱缨,在那八人轿前领着那些车轿人
马,浩浩荡荡,一片锦绣香烟,遮天压地而来。却是鸦雀无闻,只有车轮马蹄之声。

不多时,已到了清虚观门口。只听钟鸣鼓响,早有张法官执香披衣,带领众道
士在路旁迎接。宝玉下了马,贾母的轿刚至山门以内,见了本境城隍土地各位泥塑
圣像,便命住轿。贾珍带领各子弟上来迎接。凤姐儿的轿子却赶在头里先到了,带
着鸳鸯等迎接上来,见贾母下了轿,忙要搀扶。可巧有个十二三岁的小道士儿,拿
着个剪筒,照管各处剪蜡花儿,正欲得便且藏出去,不想一头撞在凤姐儿怀里。凤
姐便一扬手照脸打了个嘴巴,把那小孩子打了一个斤斗,骂道:“小野杂种!往那
里跑?”那小道士也不顾拾烛剪,爬起来往外还要跑。正值宝钗等下车,众婆娘媳
妇正围随的风雨不透,但见一个小道士滚了出来,都喝声叫:“拿,拿!打,打!”
贾母听了,忙问:“是怎么了?”贾珍忙过来问。凤姐上去搀住贾母,就回说:“一
个小道士儿剪蜡花的,没躲出去,这会子混钻呢。”贾母听说,忙道:“快带了那
孩子来,别唬着他。小门小户的孩子,都是娇生惯养惯了的,那里见过这个势派?
倘或唬着他,倒怪可怜见儿的。他老子娘岂不疼呢。”说着,便叫贾珍去好生带了
来。贾珍只得去拉了,那孩子一手拿着蜡剪,跪在地下乱颤。贾母命贾珍拉起来,
叫他不用怕,问他几岁了。那孩子总说不出话来。贾母还说:“可怜见儿的!”又
向贾珍道:“珍哥带他去罢。给他几个钱买果子吃,别叫人难为了他。”贾珍答应,
领出去了。

这里贾母带着众人,一层一层的瞻拜观玩。外面小厮们见贾母等进入二层山
门,忽见贾珍领了个小道士出来,叫人:“来带了去,给他几百钱,别难为了他。”
家人听说,忙上来领去。贾珍站在台阶上,因问:“管家在那里?”底下站的小厮
们见问,都一齐喝声说:“叫管家!”登时林之孝一手整理着帽子,跑进来,到了
贾珍跟前。贾珍道:“虽然这里地方儿大,今儿咱们的人多,你使的人,你就带了
在这院里罢,使不着的,打发到那院里去。把小么儿们多挑几个在这二层门上和两
边的角门上,伺候着要东西传话。你可知道不知道?今儿姑娘奶奶们都出来,一个
闲人也不许到这里来。”林之孝忙答应“知道”,又说了几个“是”。贾珍道:“去
罢。”又问:“怎么不见蓉儿?”一声未了,只见贾蓉从钟楼里跑出来了。贾珍道:
“你瞧瞧,我这里没热,他倒凉快去了!”喝命家人啐他。那小厮们都知道贾珍素
日的性子,违拗不得,就有个小厮上来向贾蓉脸上啐了一口。贾珍还瞪着他,那小
厮便问贾蓉:“爷还不怕热,哥儿怎么先凉快去了?”贾蓉垂着手,一声不敢言语。
那贾芸、贾萍、贾芹等听见了,不但他们慌了,并贾琏、贾、贾琼等也都忙了,
一个一个都从墙根儿底下慢慢的溜下来了。贾珍又向贾蓉道:“你站着做什么?还
不骑了马跑到家里告诉你娘母子去!老太太和姑娘们都来了,叫他们快来伺候!”
贾蓉听说,忙跑了出来,一叠连声的要马。一面抱怨道:“早都不知做什么的,这
会子寻趁我。”一面又骂小子:“捆着手呢么?马也拉不来!”要打发小厮去,又
恐怕后来对出来,说不得亲自走一趟,骑马去了。

且说贾珍方要抽身进来,只见张道士站在傍边,陪笑说道:“论理,我不比别
人,应该里头伺候;只因天气炎热,众位千金都出来了,法官不敢擅入,请爷的示
下。恐老太太问,或要随喜那里,我只在这里伺候罢了。”贾珍知道这张道士虽然
是当日荣国公的替身,曾经先皇御口亲呼为“大幻仙人”,如今现掌道录司印,又
是当今封为“终了真人”,现今王公藩镇都称为神仙,所以不敢轻慢。二则他又常
往两个府里去,太太姑娘们都是见的。今见他如此说,便笑道:“咱们自己,你又
说起这话来。再多说,我把你这胡子还揪了你的呢!还不跟我进来呢。”那张道士
呵呵的笑着,跟了贾珍进来。

贾珍到贾母跟前,控身陪笑,说道:“张爷爷进来请安。”贾母听了,忙道:
“请他来。”贾珍忙去搀过来。那张道士先呵呵笑道:“无量寿佛!老祖宗一向福
寿康宁,众位奶奶姑娘纳福!一向没到府里请安,老太太气色越发好了。”贾母笑
道:“老神仙你好?”张道士笑道:“托老太太的万福,小道也还康健。别的倒罢
了,只记挂着哥儿,一向身上好?前日四月二十六,我这里做遮天大王的圣诞,人
也来的少,东西也很干净,我说请哥儿来逛逛,怎么说不在家?”贾母说道:“果
真不在家。”一面回头叫宝玉。

谁知宝玉解手儿去了,才来,忙上前问:“张爷爷好?”张道士也抱住问了好,
又向贾母笑道:“哥儿越发发福了。”贾母道:“他外头好,里头弱。又搭着他老
子逼着他念书,生生儿的把个孩子逼出病来了。”张道士道:“前日我在好几处看
见哥儿写的字,做的诗,都好的了不得。怎么老爷还抱怨哥儿不大喜欢念书呢?依
小道看来,也就罢了。”又叹道:“我看见哥儿的这个形容身段,言谈举动,怎么
就和当日国公爷一个稿子!”说着,两眼酸酸的。贾母听了,也由不得有些戚惨,
说道:“正是呢。我养了这些儿子孙子,也没一个像他爷爷的,就只这玉儿还像他
爷爷。”那张道士又向贾珍道:“当日国公爷的模样儿,爷们一辈儿的不用说了,
自然没赶上;大约连大老爷、二老爷也记不清楚了罢?”说毕,又呵呵大笑道:“前
日在一个人家儿,看见位小姐,今年十五岁了,长的倒也好个模样儿。我想着哥儿
也该提亲了。要论这小姐的模样儿,聪明智慧,根基家当,倒也配的过。但不知老
太太怎么样?小道也不敢造次,等请了示下,才敢提去呢。”贾母道:“上回有个
和尚说了,这孩子命里不该早娶,等再大一大儿再定罢。你如今也讯听着,不管他
根基富贵,只要模样儿配的上,就来告诉我。就是那家子穷,也不过帮他几两银子
就完了。只是模样儿性格儿难得好的。”

说毕,只见凤姐儿笑道:“张爷爷,我们丫头的寄名符儿,你也不换去,前儿
亏你还有那么大脸,打发人和我要鹅黄缎子去!要不给你,又恐怕你那老脸上下不
来。”张道士哈哈大笑道:“你瞧,我眼花了!也没见奶奶在这里,也没道谢。寄
名符早已有了,前日原想送去,不承望娘娘来做好事,也就混忘了。还在佛前镇着
呢。等着我取了来。”说着,跑到大殿上,一时拿了个茶盘,搭着大红蟒缎经袱子,
托出符来。大姐儿的奶子接了符。张道士才要抱过大姐儿来,只见凤姐笑道:“你
就手里拿出来罢了,又拿个盘子托着!”张道士道:“手里不干不净的,怎么拿?
用盘子洁净些。”凤姐笑道:“你只顾拿出盘子,倒唬了我一跳。我不说你是为送
符,倒像和我们化布施来了。”众人听说哄然一笑,连贾珍也掌不住笑了。贾母回
头道:“猴儿,猴儿!你不怕下割舌地狱?”凤姐笑道:“我们爷儿们不相干。他
怎么常常的说我该积阴骘、迟了就短命呢?”张道士也笑道:“我拿出盘子来,一
举两用,倒不为化布施,倒要把哥儿的那块玉请下来,托出去给那些远来的道友和
徒子徒孙们见识见识。”贾母道:“既这么着,你老人家老天拔地的,跑什么呢,
带着他去瞧了叫他进来,就是了。”张道士道:“老太太不知道,看着小道是八十
岁的人,托老太太的福,倒还硬朗;二则外头的人多气味难闻,况且大暑热的天,
哥儿受不惯,倘或哥儿中了腌气味,倒值多了。”贾母听说,便命宝玉摘下通灵
玉来,放在盘内。那张道士兢兢业业的用蟒袱子垫着,捧出去了。

这里贾母带着众人各处游玩一回,方去上楼。只见贾珍回说:“张爷爷送了玉
来。”刚说着,张道士捧着盘子走到跟前,笑道:“众人托小道的福,见了哥儿的
玉,实在稀罕,都没什么敬贺的,这是他们各人传道的法器,都愿意为敬贺之礼。
虽不稀罕,哥儿只留着玩耍赏人罢。”贾母听说,向盘内看时,只见也有金璜,也
有玉,或有“事事如意”,或有“岁岁平安”,皆是珠穿宝嵌、玉琢金镂,共有
三五十件。因说道:“你也胡闹。他们出家人,是那里来的?何必这样?这断不能收。”
张道士笑道:“这是他们一点敬意,小道也不能阻挡。老太太要不留下,倒叫他们
看着小道微薄,不像是门下出身了。”贾母听如此说,方命人接下了。宝玉笑道:
“老太太,张爷爷既这么说,又推辞不得,我要这个也无用,不如叫小子捧了这个,
跟着我出去散给穷人罢。”贾母笑道:“这话说的也是。”张道士忙拦道:“哥儿
虽要行好,但这些东西虽说不甚稀罕,也到底是几件器皿。若给了穷人,一则与他
们也无益,二则反倒遭塌了这些东西。要舍给穷人,何不就散钱给他们呢?”宝玉
听说,便命:“收下,等晚上拿钱施舍罢。”说毕,张道士方才退出。

这里贾母和众人上了楼,在正面楼上归坐。凤姐等上了东楼。众丫头等在西楼
轮流伺候。一时贾珍上来回道:“神前拈了戏,头一本是《白蛇记》。”贾母便问:
“是什么故事?”贾珍道:“汉高祖斩蛇起首的故事。第二本是《满床笏》。”贾
母点头道:“倒是第二本?也还罢了。神佛既这样,也只得如此。”又问:“第三
本?”贾珍道:“第三本是《南柯梦》。”贾母听了,便不言语。贾珍退下来,走
至外边,预备着申表、焚钱粮、开戏,不在话下。

且说宝玉在楼上,坐在贾母傍边,因叫个小丫头子捧着方才那一盘子东西,将
自己的玉带上,用手翻弄寻拨,一件一件的挑与贾母看。贾母因看见有个赤金点翠
的麒麟,便伸手拿起来,笑道:“这件东西,好像是我看见谁家的孩子也带着一个
的。”宝钗笑道:“史大妹妹有一个,比这个小些。”贾母道:“原来是云儿有这
个。”宝玉道:“他这么往我们家去住着,我也没看见?”探春笑道:“宝姐姐有
心,不管什么他都记得。”黛玉冷笑道:“他在别的上头心还有限,惟有这些人带
的东西上,他才是留心呢。”宝钗听说,回头装没听见。宝玉听见史湘云有这件东
西,自己便将那麒麟忙拿起来,揣在怀里。忽又想到怕人看见他听是史湘云有了,
他就留着这件,因此手里揣着,却拿眼睛瞟人。只见众人倒都不理论,惟有黛玉瞅
着他点头儿,似有赞叹之意。宝玉心里不觉没意思起来,又掏出来,瞅着黛玉讪笑
道:“这个东西有趣儿,我替你拿着,到家里穿上个穗子你带,好不好?”黛玉将
头一扭道:“我不稀罕。”宝玉笑道:“你既不稀罕,我可就拿着了。”说着,又
揣起来。

刚要说话,只见贾珍之妻尤氏和贾蓉续娶的媳妇胡氏,婆媳两个来了,见过贾
母。贾母道:“你们又来做什么,我不过没事来逛逛。”一句话说了,只见人报:
“冯将军家有人来了。”原来冯紫英家听见贾府在庙里打醮,连忙预备猪羊、香烛、
茶食之类,赶来送礼。凤姐听了,忙赶过正楼来,拍手笑道:“嗳呀!我却没防着
这个。只说咱们娘儿们来闲逛逛,人家只当咱们大摆斋坛的来送礼。都是老太太闹
的!这又不得预备赏封儿。”刚说了,只见冯家的两个管家女人上楼来了。冯家两
个未去,接着赵侍郎家也有礼来了。于是接二连三,都听见贾府打醮,女眷都在庙
里,凡一应远亲近友,世家相与,都来送礼。贾母才后悔起来,说:“又不是什么
正经斋事,我们不过闲逛逛,没的惊动人。”因此虽看了一天戏,至下午便回来了。
次日便懒怠去。凤姐又说:“‘打墙也是动土’,已经惊动了人,今儿乐得还去逛
逛。”贾母因昨日见张道士提起宝玉说亲的事来,谁知宝玉一日心中不自在,回家
来生气,嗔着张道士与他说了亲,口口声声说“从今以后,再不见张道士了”,别
人也并不知为什么原故。二则黛玉昨日回家,又中了暑。因此二事,贾母便执意不
去了。凤姐见不去,自己带了人去,也不在话下。

且说宝玉因见黛玉病了,心里放不下,饭也懒怠吃,不时来问,只怕他有个好
歹。黛玉因说道:“你只管听你的戏去罢,在家里做什么?”宝玉因昨日张道士提
亲之事,心中大不受用,今听见黛玉如此说,心里因想道:“别人不知道我的心还
可恕,连他也奚落起我来。”因此心中更比往日的烦恼加了百倍。要是别人跟前,
断不能动这肝火,只是黛玉说了这话,倒又比往日别人说这话不同,由不得立刻沉
下脸来,说道:“我白认得你了!罢了,罢了!”黛玉听说,冷笑了两声道:“你
白认得了我吗?我那里能够像人家有什么配的上你的呢!”宝玉听了,便走来,直
问到脸上道:“你这么说,是安心咒我天诛地灭?”黛玉一时解不过这话来。宝玉
又道:“昨儿还为这个起了誓呢,今儿你到底儿又重我一句!我就天诛地灭,你又
有什么益处呢?”黛玉一闻此言,方想起昨日的话来。今日原自己说错了,又是急,
又是愧,便抽抽搭搭的哭起来,说道:“我要安心咒你,我也天诛地灭!何苦来呢!
我知道昨日张道士说亲,你怕拦了你的好姻缘,你心里生气,来拿我煞性子!”

原来宝玉自幼生成来的有一种下流痴病,况从幼时和黛玉耳鬓厮磨,心情相
对,如今稍知些事,又看了些邪书僻传,凡远亲近友之家所见的那些闺英闱秀,皆
未有稍及黛玉者,所以早存一段心事,只不好说出来。故每每或喜或怒,变尽法子
暗中试探。那黛玉偏生也是个有些痴病的,也每用假情试探。因你也将真心真意瞒
起来,我也将真心真意瞒起来,都只用假意试探,如此“两假相逢,终有一真”,
其间琐琐碎碎,难保不有口角之事。即如此刻,宝玉的心内想的是:“别人不知我
的心还可恕,难道你就不想我的心里眼里只有你?你不能为我解烦恼,反来拿这个
话堵噎我,可见我心里时时刻刻白有你,你心里竟没我了。”宝玉是这个意思,只
口里说不出来。那黛玉心里想着:“你心里自然有我,虽有‘金玉相对’之说,你
岂是重这邪说不重人的呢?我就时常提这‘金玉’,你只管了然无闻的,方见的是
待我重,无毫发私心了。怎么我只一提‘金玉’的事,你就着急呢?可知你心里时
时有这个‘金玉’的念头。我一提,你怕我多心,故意儿着急,安心哄我。”那宝
玉心中又想着:“我不管怎么样都好,只要你随意,我就立刻因你死了,也是情愿
的。你知也罢,不知也罢,只由我的心,那才是你和我近,不和我远。”黛玉心里
又想着:“你只管你就是了。你好,我自然好。你要把自己丢开,只管周旋我,是
你不叫我近你,竟叫我远了。”

看官,你道两个人原是一个心,如此看来,却都是多生了枝叶,将那求近之心
反弄成疏远之意了。此皆他二人素昔所存私心,难以备述。如今只说他们外面的形
容。

那宝玉又听见他说“好姻缘”三个字,越发逆了己意。心里干噎,口里说不出
来,便赌气向颈上摘下通灵玉来,咬咬牙,狠命往地下一摔,道:“什么劳什子!
我砸了你,就完了事了!”偏生那玉坚硬非常,摔了一下,竟文风不动。宝玉见不
破,便回身找东西来砸。黛玉见他如此,早已哭起来,说道:“何苦来你砸那哑吧
东西?有砸他的,不如来砸我!”

二人闹着,紫鹃雪雁等忙来解劝。后来见宝玉下死劲的砸那玉,忙上来夺,又
夺不下来。见比往日闹的大了,少不得去叫袭人。袭人忙赶了来,才夺下来。宝玉
冷笑道:“我是砸我的东西,与你们什么相干!”袭人见他脸都气黄了,眉眼都变
了,从来没气的这么样,便拉着他的手,笑道:“你合妹妹拌嘴,不犯着砸他;倘
或砸坏了,叫他心里脸上怎么过的去呢?”黛玉一行哭着,一行听了这话,说到自
己心坎儿上来,可见宝玉连袭人不如,越发伤心大哭起来。心里一急,方才吃的香
薷饮,便承受不住,“哇”的一声,都吐出来了。紫鹃忙上来用绢子接住,登时一
口一口的,把块绢子吐湿。雪雁忙上来捶揉。紫鹃道:“虽然生气,姑娘到底也该
保重些。才吃了药,好些儿,这会子因和宝二爷拌嘴,又吐出来了;倘或犯了病,
宝二爷怎么心里过的去呢?”宝玉听了这话,说到自己心坎儿上来,可见黛玉竟还
不如紫鹃呢。又见黛玉脸红头胀,一行啼哭,一行气凑,一行是泪,一行是汗,不
胜怯弱。宝玉见了这般,又自己后悔:“方才不该和他较证,这会子他这样光景,
我又替不了他。”心里想着,也由不得滴下泪来了。

袭人守着宝玉,见他两个哭的悲痛,也心酸起来。又摸着宝玉的手冰凉,要劝
宝玉不哭罢,一则恐宝玉有什么委屈闷在心里,二则又恐薄了黛玉:两头儿为难。
正是女儿家的心性,不觉也流下泪来。紫鹃一面收拾了吐的药,一面拿扇子替黛玉
轻轻的扇着,见三个人都鸦雀无声,各自哭各自的,索性也伤起心来,也拿着绢子
拭泪。四个人都无言对泣。还是袭人勉强笑向宝玉道:“你不看别的,你看看这玉
上穿的穗子,也不该和林姑娘拌嘴呀。”黛玉听了,也不顾病,赶来夺过去,顺手
抓起一把剪子来就铰。袭人紫鹃刚要夺,已经剪了几段。黛玉哭道:“我也是白效
力,他也不稀罕,自有别人替他再穿好的去呢!”袭人忙接了玉道:“何苦来!这
是我才多嘴的不是了。”宝玉向黛玉道:“你只管铰!我横竖不带他,也没什么。”

只顾里头闹,谁知那些老婆子们见黛玉大哭大吐,宝玉又砸玉,不知道要闹到
什么田地儿,便连忙的一齐往前头去回了贾母王夫人知道,好不至于连累了他们。
那贾母王夫人见他们忙忙的做一件正经事来告诉,也都不知有了什么原故,便一齐
进园来瞧。急的袭人抱怨紫鹃:“为什么惊动了老太太、太太?”紫鹃又只当是袭
人着人去告诉的,也抱怨袭人。那贾母王夫人进来,见宝玉也无言,黛玉也无话,
问起来,又没为什么事,便将这祸移到袭人紫鹃两个人身上,说:“为什么你们不
小心伏侍,这会子闹起来都不管呢?”因此将二人连骂带说教训了一顿。二人都没
的说,只得听着。还是贾母带出宝玉去了,方才平伏。

过了一日,至初三日,乃是薛蟠生日,家里摆酒唱戏,贾府诸人都去了。宝玉
因得罪了黛玉,二人总未见面,心中正自后悔,无精打彩,那里还有心肠去看戏,
因而推病不去。黛玉不过前日中了些暑溽之气,本无甚大病,听见他不去,心里想:
“他是好吃酒听戏的,今日反不去,自然是因为昨儿气着了;再不然他见我不去,
他也没心肠去。只是昨儿千不该万不该铰了那玉上的穗子。管定他再不带了,还得
我穿了他才带。”因而心中十分后悔。那贾母见他两个都生气,只说趁今儿那边去
看戏,他两个见了,也就完了,不想又都不去。老人家急的抱怨说:“我这老冤家,
是那一世里造下的孽障?偏偏儿的遇见了这么两个不懂事的小冤家儿,没有一天不
叫我操心!真真的是俗语儿说的,‘不是冤家不聚头’了。几时我闭了眼,断了这
口气,任凭你们两个冤家闹上天去,我‘眼不见,心不烦’,也就罢了。偏他娘的
又不咽这口气!”自己抱怨着,也哭起来了。谁知这个话传到宝玉黛玉二人耳内,
他二人竟从来没有听见过“不是冤家不聚头”的这句俗话儿,如今忽然得了这句话,
好似参禅的一般,都低着头细嚼这句话的滋味儿,不觉的潸然泪下。虽然不曾会面,
却一个在潇湘馆临风洒泪,一个在怡红院对月长吁,正是“人居两地,情发一心”
了。袭人因劝宝玉道:“千万不是,都是你的不是。往日家里的小厮们和他的姐姐
妹妹拌嘴,或是两口子分争,你要是听见了,还骂那些小厮们蠢,不能体贴女孩儿
们的心肠;今儿怎么你也这么着起来了?明儿初五,大节下的,你们两个再这么仇
人似的,老太太越发要生气了,一定弄的大家不安生。依我劝你,正经下个气儿,
赔个不是,大家还是照常一样儿的,这么着不好吗?”宝玉听了,不知依与不依。

要知端详,下回分解。