\chapter{杏子阴假凤泣虚凰~茜纱窗真情揆痴理}

话说他三人因见探春等进来,忙将此话掩住不提。探春等问候过,大家说笑了
一回方散。谁知上回所表的那位老太妃已薨,凡诰命等皆入朝随班,按爵守制,敕
谕天下,凡有爵之家,一年内不得筵宴音乐,庶民皆三月不得婚姻。贾母婆媳祖孙
等俱每日入朝随祭,至未正以后方回。在大偏宫二十一日后,方请灵入先陵,地名
孝慈县。这陵离都来往得十来日之功,如今请灵至此,还要停放数日,方入地宫,
故得一月光景。宁府贾珍夫妻二人,也少不得是要去的。两府无人,因此大家计议,
家中无主,便报了“尤氏产育”,将他腾挪出来,协理宁荣两处事件。因托了薛姨
妈在园内照管他姊妹丫鬟,只得也挪进园来。此时宝钗处有湘云香菱;李纨处目今
李婶母虽去,然有时来往,三五日不定,贾母又将宝琴送与他去照管;迎春处有岫
烟;探春因家务冗杂,且不时有赵姨娘与贾环嘈聒,甚不方便;惜春处房屋狭小:
因此薛姨妈都难住。况贾母又千叮咛万嘱咐托他照管黛玉,自己素性也最怜爱他,
今既巧遇这事,便挪至潇湘馆和黛玉同房,一应药饵饮食,十分经心。黛玉感戴不
尽,以后便亦如宝钗之称呼。连宝钗前亦直以“姐姐”呼之,宝琴前直以“妹妹”
呼之:俨似同胞共出,较诸人更似亲切。贾母见如此,也十分喜悦放心。薛姨妈只
不过照管他姊妹,禁约的丫鬟辈,一应家中大小事务也不肯多口。尤氏虽天天过来,
也不过应名点卯,不肯乱作威福。且他家内上下,也只剩了他一人料理,再者每日
还要照管贾母王夫人的下处一应所需饮馔铺设之物,所以也甚操劳。

当下荣宁两处主人既如此不暇,并两处执事人等,或有跟随着入朝的,或有朝
外照理下处事务的,又有先踩踏下处的,也都各各忙乱。因此两处下人无了正经头
绪,也都偷安,或乘隙结党,和暂权执事者窃弄威福。荣府只留得赖大并几个管家
照管外务。这赖大手下常用几个人已去,虽另委人,都是些生的,只觉不顺手。且
他们无知,或赚骗无节,或呈告无据,或举荐无因,种种不善,在在生事,也难备
述。

又见各官宦家凡养优伶男女者,一概蠲免遣发,尤氏等便议定,待王夫人回家
回明,也欲遣发十二个女孩子。又说:“这些人原是买的,如今虽不学唱,尽可留
着使唤,只令其教习们自去也罢了。”王夫人因说:“这学戏的倒比不得使唤的,
他们也是好人家的女儿,因无能,卖了做这事,装丑弄鬼的几年。如今有这机会,
不如给他们几两银子盘费,各自去罢。当日祖宗手里都是有这例的。咱们如今损阴
坏德,而且还小器。如今虽有几个老的还在,那是他们各有原故不肯回去的,所以
才留下使唤,大了配了我们家里小厮们了。”尤氏道:“如今我们也去问他十二个,
有愿意回去的,就带了信儿,叫他父母来亲自领回去,给他们几两银子盘缠方妥。
倘若不叫上他的亲人来,只怕有混帐人冒名领出去,又转卖了,岂不辜负了这恩典?
若有不愿意回去的,就留下。”王夫人笑道:“这话妥当。”尤氏等遣人告诉了凤
姐儿,一面说与总理房中,每教习给银八两,令其自便。凡梨香院一应物件,查清
记册收明,派人上夜。将十二个女孩子叫来,当面细问,倒有一多半不愿意回家的。
也有说父母虽有,他只以卖我们姊妹为事,这一去还被他卖了;也有说父母已亡,
或被伯叔兄弟所卖的;也有说无人可投的;也有说恋恩不舍的:所愿去者止四五人。
王夫人听了,只得留下。将去者四五人皆令其干娘领回家去,单等他亲父母来领;
将不愿去者分散在园中使唤。贾母便留下文官自使,将正旦芳官指给了宝玉,小旦
蕊官送了宝钗,小生藕官指给了黛玉,大花面葵官送了湘云,小花面豆官送了宝琴,
老外艾官指给了探春,尤氏便讨了老旦茄官去。当下各得其所,就如那倦鸟出笼,
每日园中游戏。众人皆知他们不能针黹,不惯使用,皆不大责备。其中或有一二个
知事的,愁将来无应时之技,亦将本技丢开,便学起针黹纺绩女工诸务。

一日正是朝中大祭,贾母等五更便去了。下处用些点心小食,然后入朝;早膳
已毕,方退至下处歇息。用过午饭,略歇片刻,复入朝侍中晚二祭,方出至下处歇
息;用过晚饭方回家。可巧这下处乃是一个大官的家庙,是比丘尼焚修,房舍极多
极净。东西二院,荣府便赁了东院,北静王府便赁了西院。太妃少妃每日晏息,见
贾母等在东院,彼此同出同入,都有照应。外面诸事不消细述。

且说大观园内因贾母王夫人天天不在家内,又送灵去一月方回,各丫鬟婆子皆
有空闲,多在园内游玩。更又将梨香院内伏侍的众婆子一概撤回,并散在园内听使,
更觉园内人多了几十个。因文官等一干人,或心性高傲,或倚势凌下,或拣衣挑食,
或口角锋芒,大概不安分守己者多,因此众婆子含怨,只是口中不敢与他们分争。
如今散了学,大家趁了愿,也有丢开手的,也有心地狭窄犹怀旧怨的,因将众人皆
分在各房名下,不敢来厮侵。

可巧这日乃是清明之日,贾琏已备下年例祭祀,带领贾环、贾琮、贾兰三人去
往铁槛寺祭柩烧纸,宁府贾蓉也同族中人各办祭祀前往。因宝玉病未大愈,故不曾
去得。饭后发倦,袭人因说:“天气甚好,你且出去逛逛,省的撂下粥碗就睡,存
在心里。”宝玉听说,只得拄了一支杖,着鞋走出院来。因近日将园中分与众婆
子料理,各司各业,皆在忙时:也有修竹的,也有树的,也有栽花的,也有种豆
的,池中间又有驾娘们行着船夹泥的、种藕的。湘云、香菱、宝琴与些丫鬟等都坐
在山石上瞧他们取乐。宝玉也慢慢行来。湘云见了他来,忙笑说:“快把这船打出
去!他们是接林妹妹的。”众人都笑起来。宝玉红了脸,也笑道:“人家的病,谁
是好意的?你也形容着取笑儿!”湘云笑道:“病也比人家另一样,原招笑儿,反
说起人来。”说着,宝玉便也坐下,看着众人忙乱了一回。湘云因说:“这里有风,
石头上又冷,坐坐去罢。”

宝玉也正要去瞧黛玉,起身拄拐,辞了他们,从沁芳桥一带堤上走来。只见柳
垂金线,桃吐丹霞,山石之后一株大杏树,花已全落,叶稠阴翠,上面已结了豆子
大小的许多小杏。宝玉因想道:“能病了几天,竟把杏花辜负了,不觉到‘绿叶成
阴子满枝’了。”因此仰望杏子不舍。又想起邢岫烟已择了夫婿一事,虽说男女大
事不可不行,但未免又少了一个好女儿,不过二年,便也要“绿叶成阴子满枝”了。
再过几日,这杏树子落枝空;再几年,岫烟也不免乌发如银,红颜似缟。因此,不
免伤心,只管对杏叹息。正想叹时,忽有一个雀儿飞来,落于枝上乱啼。宝玉又发
了呆性,心下想道:“这雀儿必定是杏花正开时他曾来过,今见无花空有叶,故也
乱啼。这声韵必是啼哭之声。可恨公冶长不在眼前,不能问他。但不知明年再发时,
这个雀儿可还记得飞到这里来与杏花一会不能?”

正自胡思间,忽见一股火光从山石那边发出,将雀儿惊飞。宝玉吃了一惊,又
听外边有人喊道:“藕官你要死!怎么弄些纸钱进来烧?我回奶奶们去,仔细你的
肉!”宝玉听了,益发疑惑起来,忙转过山石看时,只见藕官满面泪痕,蹲在那里,
手内还拿着火,守着些纸钱灰作悲。宝玉忙问道:“你给谁烧纸?快别在这里烧!你
或是为父母兄弟,你告诉我名姓儿,外头去叫小厮们打了包袱写上名姓去烧。”

藕官见了宝玉,只不做一声,宝玉数问不答。忽见一个婆子恶狠狠的走来拉藕
官,口内说道:“我已经回了奶奶们,奶奶们气的了不得!”藕官听了,终是孩气,
怕去受辱没脸,便不肯去。婆子道:“我说你们别太兴头过馀了,如今还比得你们
在外头乱闹呢!这是尺寸地方儿。”指着宝玉道:“连我们的爷还守规矩呢,你是
什么阿物儿,跑了这里来胡闹!怕也不中用,跟我快走罢!”宝玉忙道:“他并没
烧纸,原是林姑娘叫他烧那烂字纸,你没看真,反错告了他。”藕官正没了主意,
见了宝玉,更自添了畏惧;忽听他反替遮掩,心内转忧成喜,也便硬着口说道:“很
看真是纸钱子么?我烧的是林姑娘写坏的字纸。”那婆子便弯腰向纸灰中拣出不曾
化尽的遗纸在手内,说道:“你还嘴硬?有证又有凭,只和你厅上讲去。”说着,
拉了袖子,拽着要走。宝玉忙拉藕官,又用拄杖隔开那婆子的手,说道:“你只管
拿了回去。实告诉你,我这夜做了个梦,梦见杏花神和我要一挂白钱,不可叫本房
人烧,另叫生人替烧,我的病就好的快了。所以我请了白钱,巴巴的烦他来替我烧
了,我今日才能起来。偏你又看见了!这会子又不好了,都是你冲了,还要告他去?
藕官,你只管见他们去,就依着这话说!”藕官听了,越得主意,反拉着要走。那
婆子忙丢下纸钱,陪笑央告宝玉说道:“我原不知道,若回太太,我这人岂不完了?”
宝玉道:“你也不许再回,我便不说。”婆子道:“我已经回了,原叫我带他。只
好说他被林姑娘叫去了。”宝玉点头应允,婆子自去。

这里宝玉细问藕官:“为谁烧纸?必非父母兄弟,定有私自的情理。”藕官因
方才护庇之情,心中感激,知他是自己一流人物,况再难隐瞒,便含泪说道:“我
这事,除了你屋里的芳官合宝姑娘的蕊官,并没第三个人知道。今日忽然被你撞见,
这意思少不得也告诉了你,只不许再对一人言讲。”又哭道:“我也不便和你面说,
你只回去,背人悄悄问芳官就知道了。”说毕怏怏而去。

宝玉听了心下纳闷,只得踱到潇湘馆。瞧黛玉越发瘦得可怜,问起来,比往日
大好了些。黛玉见他也比先大瘦了,想起往日之事,不免流下泪来。些微谈了一谈,
便催宝玉去歇息调养。宝玉只得回来。因惦记着要问芳官原委,偏有湘云香菱来了,
正和袭人芳官一处说笑,不好叫他,恐人又盘诘,只得耐着。

一时芳官又跟了他干娘去洗头,他干娘偏又先叫他亲女儿洗过才叫芳官洗。芳
官见了这样,便说他偏心:“把你女儿的剩水给我洗?我一个月的月钱都是你拿着,
沾我的光不算,反倒给我剩东剩西的。”他干娘羞恼变成怒,便骂他:“不识抬举
的东西!怪不得人人都说戏子没一个好缠的,凭你什么好的,入了这一行,都学坏
了!这一点子小崽子也挑么挑六,咸嘴淡舌,咬群的骡子似的。”娘儿两个吵起来。
袭人忙打发人去说:“少乱嚷!瞅着老太太不在家,一个个连句安静话也都不说了!”
晴雯因说:“这是芳官不省事,不知狂的什么,也不过是会两出戏,倒像杀了贼王、
擒过反叛来的。”袭人道:“‘一个巴掌拍不响’,老的也太不公些,小的也太可
恶些。”宝玉道:“怨不得芳官。自古说:‘物不平则鸣。’他失亲少眷的在这里,
没人照看;赚了他的钱,又作践他,如何怪得!”又向袭人说:“他到底一月多少
钱?以后不如你收过来照管他,岂不省事些。”袭人道:“我要照看他,那里不照
看了?又要他那几个钱才照看他?没的招人家骂去。”说着,便起身到那屋里,取了
一瓶花露油、鸡蛋、香皂、头绳之类,叫了一个婆子来:“送给芳官去,叫他另要
水自己洗罢,别吵了。”

他干娘越发羞愧,便说芳官:“没良心!只说我克扣你的钱!”便向他身上拍
了几下,芳官越发哭了。宝玉便走出来,袭人忙劝:“做什么?我去说他。”晴雯
忙先过来,指他干娘说道:“你这么大年纪,太不懂事!你不给他好好的洗,我们
才给他东西,你自己不臊,还有脸打他!他要是还在学里学艺,你也敢打他不成?”
那婆子便说:“‘一日叫娘,终身是母。’他排揎我,我就打得。”袭人唤麝月道:
“我不会和人拌嘴,晴雯性太急,你快过去震吓他两句。”麝月听了,忙过来说道:
“你且别嚷,我问问你:别说我们这一处,你看满园子里谁在主子屋里教导过女儿
的?就是你的亲女儿,既经分了房有了主子,自有主子打骂,再者大些的姑娘姐姐
们也可以打得骂得。谁许你老子娘又半中间管起闲事来了?都这样管,又要叫他们
跟着我们学什么?越老越没了规矩!你见前日坠儿的妈来吵,你如今也跟着他学。你
们放心,因连日这个病那个病,再老太太又不得闲,所以我也没有去回。等两日咱
们去痛回一回,大家把这威风煞一煞儿才好呢!况且宝玉才好了些,连我们也不敢
说话,你反打的人狼号鬼哭的。上头出了几日门,你们就无法无天的,眼珠子里就
没了人了,再两天,你们就该打我们了!他也不要你这干娘,怕粪草埋了他不成?”

宝玉恨的拿拄杖打着门槛子说道:“这些老婆子都是铁心石肠似的,真是大奇
事!不能照看,反倒挫磨他们。地久天长,如何是好?”晴雯道:“什么‘如何是
好’!都撵出去,不要这些中看不中吃的就完了!”那婆子羞愧难当,一言不发。
只见芳官穿着海棠红的小绵袄,底下绿绸洒花夹裤,敞着裤腿,一头乌油油的头发
披在脑后,哭的泪人一般。麝月笑道:“把个莺莺小姐弄成才拷打的红娘了。这会
子又不妆扮了,还是这么着?”晴雯因走过去拉着,替他洗净了发,用手巾拧的干
松松的,挽了一个慵妆髻,命他穿了衣裳,过这边来。

接着内厨房的婆子来问:“晚饭有了,可送不送?”小丫头听了,进来问袭人。
袭人笑道:“方才胡吵了一阵,也没留心听听几下钟了?”晴雯道:“这劳什子又
不知怎么了,又得去收拾。”说着,拿过表来瞧了一瞧,说道:“再略等半钟茶的
工夫就是了。”小丫头去了。麝月笑道:“提起淘气来,芳官也该打两下儿,昨日
是他摆弄了那坠子半日,就坏了。”说话之间,便将食具打点现成。一时小丫头子
捧了盒子进来站住,晴雯麝月揭开看时,还是这四样小菜。晴雯笑道:“已经好了,
还不给两样清淡菜吃,这稀饭咸菜闹到多早晚?”一面摆好,一面又看那盒中,却
有一碗火腿鲜笋汤,忙端了放在宝玉跟前。宝玉便就桌上喝了一口,说道:“好汤!”
众人都笑道:“菩萨!能几日没见荤腥儿,就馋的这个样儿。”一面说,一面端起
来,轻轻用口吹着。因见芳官在侧,便递给芳官道:“你也学些伏侍,别一味傻玩
傻睡。嘴儿轻着些,别吹上唾沫星儿。”芳官依言果吹了几口,甚妥。他干娘也端
饭在门外伺候,向里忙跑进来,笑道:“他不老成,看打了碗,等我吹罢。”一面
说,一面就接。晴雯忙喊道:“快出去!你等他砸了碗,也轮不到你吹!你什么空儿
跑到里儿来了?”一面又骂小丫头们:“瞎了眼的,他不知道,你们也该说给他。”
小丫头们都说:“我们撵他不出去,说他又不信,如今带累我们受气。这是何苦呢!
——你可信了?我们到的地方儿,有你到的一半儿,那一半儿是你到不去的呢。何
况又跑到我们到不去的地方儿,还不算,又去伸手动嘴的了!”一面说,一面推他
出去。阶下几个等空盒家伙的婆子见他出来,都笑道:“嫂子也没有拿镜子照一照,
就进去了。”羞的那婆子又恨又气,只得忍耐下去了。

芳官吹了几口,宝玉笑道:“你尝尝,好了没有?”芳官当是玩话,只是笑着
看袭人等。袭人道:“你就尝一口何妨。”晴雯笑道:“你瞧我尝。”说着便喝一
口。芳官见如此,他便尝了一口,说:“好了。”递给宝玉,喝了半碗,吃了几片
笋,又吃了半碗粥,就算了。众人便收出去。小丫头捧沐盆,漱盥毕,袭人等去吃
饭。宝玉使个眼色给芳官,芳官本来伶俐,又学了几年戏,何事不知?便装肚子疼,
不吃饭了。袭人道:“既不吃,在屋里做伴儿。把粥留下,你饿了再吃。”说着去
了。

宝玉将方才见藕官,如何谎言护庇,如何“藕官叫我问你”,细细的告诉一遍。
又问:“他祭的到底是谁?”芳官听了,眼圈儿一红,又叹一口气,道:“这事说
来,藕官儿也是胡闹。”宝玉忙问:“如何?”芳官道:“他祭的就是死了的药官
儿。”宝玉道:“他们两个也算朋友,也是应当的。”芳官道:“那里又是什么朋
友哩?那都是傻想头:他是小生,药官是小旦,往常时他们扮作两口儿,每日唱戏
的时候都装着那么亲热,一来二去,两个人就装糊涂了,倒像真的一样儿。后来两
个竟是你疼我,我爱你。药官儿一死,他就哭的死去活来的,到如今不忘,所以每
节烧纸。后来补了蕊官,我们见他也是那样:就问他:‘为什么得了新的就把旧的
忘了?’他说:‘不是忘了。比如人家男人死了女人,也有再娶的,只是不把死的
丢过不提就是有情分了。’你说他是傻不是呢?”

宝玉听了这呆话,独合了他的呆性,不觉又喜又悲,又称奇道绝,拉着芳官嘱
咐道:“既如此说,我有一句话嘱咐你,须得你告诉他:以后断不可烧纸,逢时按
节,只备一炉香,一心虔诚就能感应了。我那案上也只设着一个炉,我有心事不论
日期时常焚香,随便新水新茶就供一盏,或有鲜花鲜果,甚至荤腥素菜都可。只在
敬心,不在虚名。以后快叫他不可再烧纸了。”芳官听了,便答应着。一时吃过粥,
有人回说:“老太太回来了。”

要知端底,且看下回分解。