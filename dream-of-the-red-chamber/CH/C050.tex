\chapter{芦雪庭争联即景诗~暖香坞雅制春灯谜}

话说薛宝钗道:“到底分个次序,让我写出来。”说着,便令众人拈阄为序。
起首恰是李氏,然后按次各各开出。凤姐儿道:“既这么说,我也说一句在上头。”
众人都笑起来了,说:“这么更妙了。”宝钗将“稻香老农”之上补了一个“凤”
字,李纨又将题目讲给他听。凤姐儿想了半天,笑道:“你们别笑话我,我只有了
一句粗话,可是五个字的。下剩的我就不知道了。”众人都笑道:“越是粗话越好。
你说了,就只管干正事去罢。”凤姐儿笑道:“想下雪必刮北风,昨夜听见一夜的
北风,我有一句,这一句就是‘一夜北风紧’。使得使不得,我就不管了。”众人
听说,都相视笑道:“这句虽粗,不见底下的,这正是会作诗的起法。不但好,而
且留了写不尽的多少地步与后人。就是这句为首,稻香老农快写上,续下去。”凤
姐儿和李婶娘平儿又吃了两杯酒,自去了。这里李纨就写了:
一夜北风紧,
自己联道:
开门雪尚飘。入泥怜洁白,
香菱道:
匝地惜琼瑶。有意荣枯草,
探春道:
无心饰萎苗。价高村酿熟,
李绮道:
年稔府粱饶。葭动灰飞管,
李纹道:
阳回斗转杓。寒山已失翠,
岫烟道:
冻浦不生潮。易挂疏枝柳,
湘云道:
难堆破叶蕉。麝煤融宝鼎,
宝琴道:
绮袖笼金貂。光夺窗前镜,
黛玉道:
香粘壁上椒。斜风仍故故,
宝玉道:
清梦转聊聊。何处梅花笛?
宝钗道:
谁家碧玉箫?鳌愁坤轴陷,
李纨笑道:“我替你们看热酒去罢。”宝钗命宝琴续联,只见湘云起来道:
龙斗阵云销。野岸回孤棹,
宝琴也联道:
吟鞭指灞桥。赐裘怜抚戍,
湘云那里肯让人?且别人也不如他敏捷,都看他扬眉挺身的说道:
加絮念征徭。坳垤审夷险,
宝钗连声赞好,也便联道:
枝柯怕动摇。皑皑轻趁步,
黛玉忙联道:
剪剪舞随腰。苦茗成新赏,
一面说,一面推宝玉命他联。宝玉正看宝琴、宝钗、黛玉三人共战湘云,十分有趣,
那里还顾得联诗?今见黛玉推他,方联道:
孤松订久要。泥鸿从印迹,
宝琴接着联道:
林斧或闻樵。伏象千峰凸,
湘云忙联道:
盘蛇一径遥。花缘经冷结,
宝钗和众人又都赞好,探春联道:
色岂畏霜凋。深院惊寒雀,
湘云正渴了,忙忙的吃茶,已被岫烟抢着联道:
空山泣老。阶墀随上下,
湘云忙丢了茶杯联道:
池水任浮漂。照耀临清晓,
黛玉忙联道:
缤纷入永宵。诚忘三尺冷,
湘云忙笑联道:
瑞释九重焦。僵卧谁相问,
宝琴也忙笑联道:
狂游客喜招。天机断缟带,
湘云又忙道:
海市失鲛绡。
黛玉不容他道出,接着便道:
寂寞封台榭,
湘云忙联道:
清贫怀箪瓢。
宝琴也不容情,也忙道:
烹茶水渐沸,
湘云见这般,自为得趣,又是笑,又忙联道:
煮酒叶难烧。
黛玉也笑道:
没帚山僧扫,
宝琴也笑道:
埋琴稚子挑。
湘云笑弯了腰,忙念了一句,众人问道:“到底说的是什么?”湘云道:
石楼闲睡鹤,
黛玉笑得握着胸口,高声嚷道:
锦暖亲猫。
宝琴也忙笑道:
月窟翻银浪,
湘云忙联道:
霞城隐赤标。
黛玉忙笑道:
沁梅香可嚼,
宝钗笑称:“好句!”也忙联道:
淋竹醉堪调。
宝琴也忙道:
或湿鸳鸯带,
湘云忙联道:
时凝翡翠翘。
黛玉又忙道:
无风仍脉脉,
宝琴又忙笑联道:
不雨亦潇潇。
湘云伏着,已笑软了。众人看他三人对抢,也都不顾作诗,看着也只是笑。黛玉还
推他往下联,又道:“你也有才尽力穷之时!我听听,还有什么舌头嚼了?”湘云
只伏在宝钗怀里笑个不住。宝钗推他起来,道:“你有本事,把‘二萧’的韵全用
完了,我才服你。”湘云起身笑道:“我也不是作诗,竟是抢命呢!”众人笑道:
“倒是你自己说罢。”探春早已料定没有自己联的了,便早写出来,因说:“还没
收住呢。”李纹听了,接过来,便联了一句道:
欲志今朝乐,
李绮收了一句道:
凭诗祝舜尧。

李纨道:“够了够了。虽没作完了韵,腾挪的字,若生扭了,倒不好了。”说
着大家来细细评论一回,独湘云的多,都笑道:“这都是那块鹿肉的功劳。”李纨
笑道:“逐句评去,却还一气,只是宝玉又落了第了。”宝玉笑道:“我原不会联
句,只好担待我罢。”李纨笑道:“也没有社社担待的:又说‘韵险’了,又整误
了,又‘不会联句’!今日必罚你。我才看见栊翠庵的红梅有趣,我要折一枝插在
瓶。可厌妙玉为人,我不理他,如今罚你取一枝来插着玩儿。”众人都道:“这罚
的又雅又有趣。”宝玉也乐为,答应着就要走。湘云黛玉一起说着:“外头冷得很,
你且吃杯热酒再去。”于是湘云早热起壶酒来了,黛玉递了个大杯,满斟了一杯。
湘云笑道:“你吃了我们这酒,要取不来,加倍罚你!”宝玉忙吃了一杯,冒雪而
去。

李纨命人好好跟着,黛玉忙拦说:“不必,有了人反不得了。”李纨点头道是,
一面命丫鬟将一个美女耸肩瓶拿来,贮了水准备插梅。因又笑道:“回来该吟红梅
了。”湘云忙道:“我先作一首。”宝钗笑道:“今日断不容你再作了,你都抢了
去,别人都闲着也没趣。回来罚宝玉。他说不会联句,如今就叫他自己做去。”黛
玉笑道:“这话很是。我还有主意:方才联句不够,莫若拣那联得少的人作红梅诗。”
宝钗笑道:“这话是极。方才邢李三位屈才,且又是客,琴儿和颦儿云儿抢了他们
许多。我们一概都别作,只他们三人做才是。”李纨因说:“绮儿也不大会做,还
是让琴妹妹罢。”宝钗只得依允。又道:“就用‘红梅花’三个字做韵,每人一首
七言律:邢大妹妹做‘红’字,你们李大妹妹做‘梅’字,琴儿做‘花’字。”李
纨道:“饶过宝玉去,我不服。”湘云忙道:“有个好题目命他做。”众人问:“何
题?”湘云道;“命他就做‘访妙玉乞红梅’,岂不有趣?”众人听了,都说:“有
趣!”

一语未了,只见宝玉笑欣欣擎了一枝红梅进来。众丫鬟忙已接过,插入瓶内。
众人都道:“来赏玩!”宝玉笑道:“你们如今赏罢,也不知费了我多少精神呢。”
说着,探春早又递了一钟暖酒来,众丫鬟上来接了蓑笠掸雪。各人屋里丫鬟都添送
衣裳来,袭人也遣人送了半旧的狐腋褂来。李纨命人将那蒸的大芋头盛了一盘,又
将朱桔、黄橙、橄榄等物盛了两盘,命人带给袭人去。湘云且告诉宝玉方才的诗题,
又催宝玉快做。宝玉道:“好姐姐好妹妹们,让我自己用韵罢,别限韵了。”众人
都说:“随你做去罢。”一面说,一面大家看梅花。原来这一枝梅花只有二尺来高,
旁有一枝纵横而出,约有二三尺长,其间小枝分歧,或如蟠螭,或如僵蚓,或孤削
如笔,或密聚如林,真乃花吐胭脂,香欺兰蕙。各各称赏。

谁知岫烟、李纹、宝琴三人都已吟成,各自写了出来。众人便依“红”“梅”
“花”三字之序看去,写道:

赋得红梅花

邢岫烟
桃未芳菲杏未红,冲寒先喜笑东风。
魂飞庾岭春难辨,霞隔罗浮梦未通。
绿萼添妆融宝炬,缟仙扶醉跨残虹。
看来岂是寻常色,浓淡由他冰雪中。

又

李
纹
白梅懒赋赋红梅,逞艳先迎醉眼开。
冻脸有痕皆是血,酸心无恨亦成灰。
误吞丹药移真骨,偷下瑶池脱旧胎。
江北江南春灿烂,寄言蜂蝶漫疑猜。

又

宝
琴
疏是枝条艳是花,春妆儿女竞奢华。
闲庭曲槛无馀雪,流水空山有落霞。
幽梦冷随红袖笛,游仙香泛绛河槎。
前身定是瑶台种,无复相疑色相差。

众人看了,都笑着称赞了一回,又指末一首更好。宝玉见宝琴年纪最小,才又
敏捷;黛玉湘云二个斟了一小杯酒,都贺宝琴。宝钗笑道:“三首各有好处。你们
两个天天捉弄厌了我,如今又捉弄他来了。”

李纨又问宝玉:“你可有了?”宝玉忙道;“我倒有了,才一看见这三首,又
唬忘了。等我再想。”湘云听了,便拿了一支铜火箸击着手炉,笑道:“我击了,
若鼓绝不成,又要罚的。”宝玉笑道:“我已有了。“黛玉提起笔来,笑道:“你
念我写。”湘云便击了一下,笑道:“一鼓绝。”宝玉笑道:“有了,你写罢。”
众人听他念道:
酒未开樽句未裁,
黛玉写了,摇头笑道:“起的平平。”湘云又道:“快着。”宝玉笑道:
寻春问腊到蓬莱。
黛玉湘云都点头笑道:“有些意思了。”宝玉又道:
不求大士瓶中露,为乞孀娥槛外梅。
黛玉写了,摇头说:“小巧而已。”湘云将手又敲了一下。宝玉笑道:
入世冷挑红雪去,离尘香割紫云来。
槎谁惜诗肩瘦,衣上犹沾佛院苔。

黛玉写毕,湘云大家才评论时,只见几个丫鬟跑进来道:“老太太来了。”众
人忙迎出来,大家又笑道:“怎么这等高兴!”说着,远远见贾母围了大斗篷,带
着灰鼠暖兜,坐着小竹轿,打着青绸油伞,鸳鸯琥珀等五六个丫鬟,每人都是打着
伞,拥轿而来。李纨等忙往上迎。贾母命人止住,说:“只站在那里就是了。”来
至跟前,贾母笑道:“我瞒着你太太和凤丫头来了。大雪地下,我坐着这个无妨,
没的叫他娘儿们踩雪吗。”众人忙上前来接斗篷,搀扶着,一面答应着。

贾母来至室中,先笑道:“好俊梅花!你们也会乐,我也不饶你们!”说着,
李纨早命人拿了一个大狼皮褥子来,铺在当中。贾母坐了,因笑道:“你们只管照
旧玩笑吃喝。我因为天短了,不敢睡中觉,抹了一会牌,想起你们来了,我也来凑
个趣儿。”李纨早又捧过手炉来。探春另拿了一副杯箸来,亲自斟了暖酒奉给贾母。
贾母便饮了一口,问:“那个盘子是什么东西?”众人忙捧了过来回说:“是糟鹌
鹑。”贾母道:“这倒罢了,撕一点子腿儿来。”李纨忙答应了,要水洗手,亲自
来撕。贾母道:“你们仍旧坐下说笑,我听着才喜欢。”又命李纨:“你也只管坐
下,就如同我没来的一样才好,不然我就走了。”众人听了,方才依次坐下,只李
纨挪到尽下边。贾母因问:“你们作什么玩呢?”众人便说:“做诗呢。”贾母道:
“有做诗的,不如做些灯谜儿,大家正月里好玩。”众人答应。说笑了一会,贾母
便说:“这里潮湿,你们别久坐,仔细着了凉。倒是你四妹妹那里暖和,我们到那
里瞧瞧他的画儿,赶年可能有了不能。”众人笑道:“那里能年下就有了?只怕明
年端阳才有呢。”贾母道:“这还了得,他竟比盖这园子还费工夫了。”

说着,仍坐了竹椅轿,大家围随,过了藕香榭,穿入一条夹道,东西两边皆是
过街门,门楼上里外都嵌着石头匾。如今进的是西门,向外的匾上凿着“穿云”二
字,向里的凿着“度月”两字。来至堂中,进了向南的正门,贾母下了轿,惜春已
接出来了。从里面游廊过去,便是惜春卧房,厦檐下挂着“暖香坞”的匾,早有几
个人打起猩红毡帘,已觉暖气拂脸。大家进入屋里,贾母并不归坐,只问惜春:“画
到那里了?”惜春因笑回:“天气寒冷了,胶性都凝涩不润,画了恐不好看,故此
收起来了。”贾母笑道:“我年下就要的,你别脱懒儿,快拿出来给我快画。”一
语未了,忽见凤姐披着紫羯绒褂笑嘻嘻的来了,口内说道:“老祖宗今儿也不告诉
人,私自就来了,叫我好找!”贾母见他来了,心中喜欢,道:“我怕你冻着,所
以不许人告诉你去。你真是个小鬼灵精儿,到底找了我来。论礼,孝敬也不在这上
头。”凤姐儿笑道:“我那里是孝敬的心找了来呢?我因为到了老祖宗那里,鸦没
雀静的,问小丫头子们,他又不肯叫我找到园里来。我正疑惑,忽然又来了两个姑
子。我心里才明白了,那姑子必是来送年疏或要年例香例银子,老祖宗年下的事也
多,一定是躲债来了。我赶忙问了那姑子,果然不错。我才就把年例给了他们去了。
这会子老祖宗的债主儿已去了,不用躲着了。已预备下稀嫩的野鸡,请用晚饭去罢,
再迟一回就老了。”

他一行说,众人一行笑。凤姐儿也不等贾母说话,便命人抬过轿来。贾母笑着
挽了凤姐儿的手,仍上了轿,带着众人,说笑出了夹道东门。一看四面,粉妆银砌,
忽见宝琴披着凫靥裘,站在山坡背后遥等,身后一个丫鬟,抱着一瓶红梅。众人都
笑道:“怪道少了两个,他却在这里等着,——也弄梅花去了!”贾母喜的忙笑道;
“你们瞧,这雪坡儿上,配上他这个人物儿,又是这件衣裳,后头又是这梅花,像
个什么?”众人都笑道:“就像老太太屋里挂的仇十洲画的《艳雪图》。”贾母摇
头笑道:“那画的那里有这件衣裳?人也不能这样好。”一语未了,只见宝琴身后
又转出一个穿大红猩猩毡的人来。贾母道:“那又是那个女孩儿?”众人笑道:“我
们都在这里,那是宝玉。”贾母笑道:“我的眼越发花了。”说话之间,来至跟前,
可不是宝玉和宝琴两个?宝玉笑向宝钗黛玉等道:“我才又到了栊翠庵,妙玉竟每
人送你们一枝梅花,我已经打发人送去了。”众人都笑说:“多谢你费心。”

说话之间,已出了园门,来至贾母房中。吃毕饭大家又说笑了一回,忽见薛姨
妈也来了,说:“好大雪,一日也没过来望候老太太。今日老太太倒不高兴?正该
赏雪才是。”贾母笑道:“何曾不高兴了!我找了他们姐妹去玩了一会子。”薛姨
妈笑道:“昨儿晚上我原想着今日要和我们姨太太借一天园子,摆两桌粗酒,请老
太太赏雪的;又见老太太安歇的早,我听见宝儿说:‘老太太心里不大爽。’因此
如今也不敢惊动。早知如此,我竟该请了才是呢。”贾母笑道:“这才是十月,是
头场雪,往后下雪的日子多着呢,再破费姨太太不迟。”薛姨妈笑道:“果然如此,
算我的孝心虔了。”凤姐儿笑道:“姨妈怎么忘了!如今现秤五十两银子来,交给
我收着,一下雪我就预备下酒。姨妈也不用操心,也不得忘了。”贾母笑道:“既
这么说,姨太太给他五十两银子收着,我和他每人分二十五两,到下雪的日子,我
装心里不爽,混过去了。姨太太更不用操心,我和凤姐倒得实惠呢。”凤姐将手一
拍,笑道:“妙极!这和我的主意一样。”众人都笑了。贾母笑道:“呸!没脸的,
就顺着竿子爬上来了!你不说:姨太太是客,在咱们家受屈,我们该请姨太太才是,
那里有破费姨太太的理?不这么说呢,还有脸先要五十两银子,真不害臊。”凤姐
笑道:“我们老祖宗最是有眼色的,试一试姨妈:要松呢,拿出五十两来,就和我
分;这会子估量着不中用了,翻过来拿我做法子,说出这些大方话来。如今我也不
和姨妈要银子了,我竟替姨妈出银子,治了酒,请老太太吃了,我另外再封五十两
银子孝敬老祖宗,算是罚我个包揽闲事,这可好不好?”话未说完,众人都笑倒在
炕上。

贾母因又说及宝琴雪下折梅,比画儿上还好;又细问他的年庚八字并家内景
况。薛姨妈度其意思,大约是要给他求配。薛姨妈心中因也遂意,只是已许过梅家
了,因贾母尚未说明,自己也不拟定,遂半吐半露告诉贾母道:“可惜了这孩子没
福,前年他父亲就没了。他从小儿见的世面倒多,跟他父亲四山五岳都走遍了。他
父亲好乐的,各处因有买卖,带了家眷这一省逛一年,明年又到那一省逛半年,所
以天下十停走了有五六停了,那年在这里,把他许了梅翰林的儿子,偏第二年他父
亲就辞世了。如今他母亲又是痰症。”凤姐儿也不等说完,便声跺脚的说:“偏
不巧!我正要做个媒呢,又已经许了人家!”贾母笑道:“你要给谁说媒?”凤姐
儿笑道:“老祖宗别管。心里看准了,他们两个是一对。如今有了人家,说也无益,
不如不说罢了。”贾母也知凤姐儿的意思,听见已有人家,也就不提了。大家又闲
话了一会方散。一宿无话。

次日雪晴。饭后,贾母又吩咐惜春:“不管冷暖,你要画去;赶到年下,十分
不能,就罢了。第一要紧把昨儿琴儿和丫头、梅花,照样一笔别错快快添上。”惜
春听了,虽是为难的事,就应了。一时众人都来看他如何画。惜春只是出神。李纨
因笑向众人道:“让他自己想去,咱们且说话儿。昨儿老太太只叫做灯谜儿,回到
家和绮儿纹儿睡不着,我就编了两个《四书》的。他两个每人也编了两个。”众人
听了,都笑道:“这倒该做的。先说了,我们猜猜。”李纨笑道:“‘观音未有世
家传’,打《四书》一句。”湘云接着就说道:“‘在止于至善’。”宝钗笑道:
“你也想一想‘世家传’三个字的意思再猜。”李纨笑道:“再想。”黛玉笑道:
“我猜罢。可是‘虽善无征’?”众人都笑道:“这句是了。”李纨又道:“‘一
池青草草何名’。”湘云又忙道:“这一定是‘蒲芦也’,再不是不成?”李纨笑
道:“这难为你猜。纹儿的是‘水向石边流出冷’,打一古人名。”探春笑着问道:
“可是山涛?”李纨道:“是。”李纨又道:“绮儿是个‘萤’字,打一个字。”
众人猜了半日,宝琴道:“这个意思却深,不知可是花草的‘花’字?”李绮笑道:
“恰是了。”众人道:“萤与花何干?”黛玉笑道:“妙的很,萤可不是草化的?”
众人会意,都笑了,说:“好。”

宝钗道:“这些虽好,不合老太太的意。不如做些浅近的物儿,大家雅俗共赏
才好。”众人都道:“也要做些浅近的俗物才是。”湘云想了一想,笑道:“我编
了一支《点绛唇》,却真是个俗物,你们猜猜。”说着,便念道:

溪壑分离,红尘游戏,真何趣?名利犹虚,后事终难继。
众人都不解,想了半日,也有猜是和尚的,也有猜是道士的,也有猜是偶戏人的。
宝玉笑了半日道:“都不是。我猜着了,必定是耍的猴儿。”湘云笑道:“正是这
个了。”众人道:“前头都好,末后一句怎么样解?”湘云道:“那一个耍的猴儿
不是剁了尾巴去的?”众人听了都笑起来,说:“偏他编个谜儿也是刁钻古怪的。”

李纨道:“昨日姨妈说,琴妹妹见得世面多,走的道路也多,你正该编谜儿。
况且你的诗又好,为什么不编几个儿我们猜一猜?”宝琴听了,点头含笑,自去寻
思。宝钗也有一个,念道:
镂檀镌梓一层层,岂系良工堆砌成?
虽是半天风雨过,何曾闻得梵铃声?
众人猜时,宝玉也有一个,念道:
天上人间两渺茫,琅节过谨提防。
鸾音鹤信须凝睇,好把唏嘘答上苍。
黛玉也有了一个,念道:
何劳缚紫绳?驰城逐堑势狰狞。
主人指示风云动,鳌背三山独立名。

探春也有了一个,方欲念时,宝琴走来,笑道:“从小儿所走的地方的古迹不
少,我也来挑了十个地方古迹,做了十首‘怀古诗’。诗虽粗鄙,却怀往事,又暗
隐俗物十件,姐姐们请猜一猜。”众人听了,都说:“这倒巧,何不写出来大家一
看?”

要知端的,且听下回分解。