\chapter{史太君破陈腐旧套~王熙凤效戏彩斑衣}

却说贾珍贾琏暗暗预备下大笸箩的钱,听见贾母说赏,忙命小厮们快撒钱,只
听满台钱响,贾母大悦。二人遂起身,小厮们忙将一把新暖银壶捧来,递与贾琏手
内,随了贾珍趋至里面。贾珍先到李婶娘席上,躬身取下杯来,回身,贾琏忙斟了
一盏,然后便至薛姨妈席上也斟了。二人忙起来笑说:“二位爷请坐着罢了,何必
多礼。”于是除邢王二夫人,满席都离了席,也俱垂手旁站。贾珍等至贾母榻前,
因榻矮,二人便屈膝跪了,贾珍在前捧杯,贾琏在后捧壶。虽只二人捧酒,那贾琮
弟兄等却都是一溜排班随着他二人进来,见他二人跪下,都一溜跪下。宝玉也忙跪
下。湘云悄推他,笑道:“你这会子又帮着跪下做什么?有这么着的呢,你也去斟
一巡酒,岂不好?”宝玉悄笑道:“再等一会再斟去。”说着,等他二人斟完,起
来,又给邢王二夫人斟过了。贾珍笑说:“妹妹们怎么着呢?”贾母等都说道:“你
们去罢,他们倒便宜些呢。”贾珍等方退出。

当下天有二鼓,戏演的是《八义·观灯》八出,正在热闹之际。宝玉因下席往
外走。贾母问:“往那里去?外头炮仗利害,留神天上吊下火纸来烧着。”宝玉笑
回说:“不往远去,只出去就来。”贾母命婆子们:“好生跟着。”于是宝玉出来,
只有麝月秋纹几个小丫头随着。贾母因说:“袭人怎么不见?他如今也有些拿大了,
单支使小女孩儿出来。”王夫人忙起身笑说道:“他妈前日没了,因有热孝,不便
前头来。”贾母点头,又笑道:“跟主子,却讲不起这孝与不孝。要是他还跟我,
难道这会子也不在这里?这些竟成了例了。”凤姐儿忙过来笑回道:“今晚便没孝,
那园子里头也须得看着灯烛花爆,最是担险的。这里一唱戏,园子里的谁不来偷瞧
瞧,他还细心,各处照看。况且这一散后,宝兄弟回去睡觉,各色都是齐全的。若
他再来了,众人又不经心,散了回去,铺盖也是冷的,茶水也不齐全,便各色都不
便宜,自然我叫他不用来。老祖宗要叫他来,我就叫他就是了。”贾母听了这话,
忙说:“你这话很是,你必想的周到,快别叫他了。但只他妈几时没了?我怎么不
知道?”凤姐儿笑道:“前儿袭人去亲自回老太太的,怎么倒忘了?”贾母想了想,
笑道:“想起来了。我的记性竟平常了。”众人都笑说:“老太太那里记得这些事。”
贾母因又叹道:“我想着他从小儿伏侍我一场,又伏侍了云儿,末后给了个魔王,
给他魔了这好几年。他又不是咱们家根生土长的奴才,没受过咱们什么大恩典,他
娘没了,我想着要给他几两银子发送他娘,也就忘了。”凤姐儿道:“前儿太太赏
了他四十两银子,就是了。”贾母听说,点头道:“这还罢了。正好前儿鸳鸯的娘
也死了,我想他老子娘都在南边,我也没叫他家去守孝。如今他两处全礼,何不叫
他二人一处作伴去?”又命婆子拿些果子菜馔点心之类与他二人吃去。琥珀笑道:
“还等这会子?他早就去了。”说着,大家又吃酒看戏。

且说宝玉一径来至园中,众婆子见他回房,便不跟去,只坐在园门里茶房里烤
火,和管茶的女人偷空饮酒斗牌。宝玉至院中,虽是灯光灿烂,却无人声。麝月道:
“他们都睡了不成?咱们悄悄进去吓他们一跳。”于是大家蹑手蹑脚,潜踪进镜壁
去一看,只见袭人和一个人对歪在地炕上,那一头有两个老嬷嬷打盹。宝玉只当他
两个睡着了,才要进去,忽听鸳鸯嗽了一声,说道:“天下事可知难定。论理你单
身在这里,父母在外头,每年他们东去西来,没个定准,想来你是再不能送终的了;
偏生今年就死在这里,你倒出去送了终。”袭人道:“正是,我也想不到能够看着
父母殡殓。回了太太,又赏了四十两银子,这倒也算养我一场,我也不敢妄想了。”
宝玉听了,忙转身悄向麝月等道:“谁知他也来了。我这一进去,他又赌气走了,
不如咱们回去罢,让他两个清清净净的说话。袭人正在那里闷着,幸他来的好。”
说着,仍悄悄出来。宝玉便走过山石后去,站着撩衣。麝月秋纹皆站住,背过脸去,
口内笑说:“蹲下再解小衣,留神风吹了肚子。”后面两个小丫头知是小解,忙先
出去茶房内预备水去了。

这里宝玉刚过来,只见两个媳妇迎面来了,又问:“是谁?”秋纹道:“宝玉
在这里呢,大呼小叫,留神吓着罢!”那媳妇们忙笑道:“我们不知,大节下来惹
祸了。姑娘们可连日辛苦了!”说着,已到跟前。麝月等问:“手里拿着什么?”
媳妇道:“是老太太赏金、花二位姑娘吃的。”秋纹笑道:“外头唱的是《八义》,
没唱《混元盒》,那里又跑出‘金花娘娘’来了?”宝玉命:“揭起来我瞧瞧。”
秋纹麝月忙上去将两个盒子揭开,两个媳妇忙蹲下身子。宝玉看了两个盒内都是席
上所有的上等果品茶点,点了一点头就走。麝月等忙胡乱掷了盒盖跟上来。宝玉笑
道:“这两个女人倒和气,会说话。他们天天乏了,倒说你们连日辛苦,倒不是那
矜功自伐的。”麝月道:“这两个就好,那不知理的是太不知理。”宝玉道:“你
们是明白人,担待他们是粗夯可怜的人就完了。”一面说,一面就走出了园门。那
几个婆子虽吃酒斗牌,却不住出来打探,见宝玉出来,也都跟上来。到了花厅廊上,
只见那两个小丫头,一个捧着个小盆,又一个搭着手巾,又拿着沤子小壶儿,在那
里久等。秋纹先忙伸手向盆内试了试,说道:“你越大越粗心了,那里弄得这冷水?”
小丫头笑道:“姑娘瞧瞧,这个天,我怕水冷,倒的是滚水,这还冷了。”正说着,
可巧见一个老婆子提着一壶滚水走来,小丫头就说:“好奶奶,过来给我倒上些水。”
那婆子道:“姐姐,这是老太太沏茶的,劝你去舀罢。那里就走大了脚呢?”秋纹
道:“不管你是谁的!你不给我,管把老太太的茶铞子倒了洗手!”那婆子回头见
了秋纹,忙提起壶来倒了些。秋纹道:“够了!你这么大年纪,也没见识。谁不知
是老太太的?要不着的就敢要了?”婆子笑道:“我眼花了,没认出这姑娘来。”
宝玉洗了手,那小丫头子拿小壶儿倒了沤子在他手内,宝玉沤了。秋纹麝月也趁热
水洗了一回,跟进宝玉来。

宝玉便要了一壶暖酒,也从李婶娘斟起。他二人也笑让坐。贾母便说:“他小
人家儿,让他斟去。大家倒要干过这杯。”说着,便自己干了。邢王二夫人也忙干
了,薛姨妈李婶娘也只得干了。贾母又命宝玉道:“你连姐姐妹妹的一齐斟上,不
许乱斟,都要叫他干了。”宝玉听说,答应着,一一按次斟上了。至黛玉前,偏他
不饮,拿起杯来,放在宝玉唇边。宝玉一气饮干,黛玉笑说:“多谢。”宝玉替他
斟上一杯。凤姐儿便笑道:“宝玉别喝冷酒。仔细手颤,明儿写不的字,拉不的弓。”
宝玉道:“没有吃冷酒。”凤姐儿笑道:“我知道没有,不过白嘱咐你。”然后宝
玉将里面斟完,只除贾蓉之妻是命丫鬟们斟的。复出至廊下,又给贾珍等斟了。坐
了一回,方进来,仍归旧坐。

一时上汤之后,又接着献元宵。贾母便命:“将戏暂歇,小孩子们可怜见的,
也给他们些滚汤热菜的吃了再唱。”又命将各样果子元宵等物拿些给他们吃。一时
歇了戏,便有婆子带了两个门下常走的女先儿进来,放了两张杌子在那一边,贾母
命他们坐了,将弦子琵琶递过去。贾母便问李薛二人:“听什么书”他二人都回
说:“不拘什么都好。”贾母便问:“近来可又添些什么新书”两个女先回说:
“倒有一段新书,是残唐五代的故事。”贾母问是何名,女先儿回说:“这叫做《凤
求鸾》。”贾母道:“这个名字倒好,不知因什么起的?你先说大概,若好再说。”
女先儿道:“这书上乃是说残唐之时,那一位乡绅,本是金陵人氏,名唤王忠,曾
做过两朝宰辅,如今告老还家,膝下只有一位公子,名唤王熙凤。”众人听了,笑
将起来。贾母笑道:“这不重了我们凤丫头了!”媳妇忙上去推他说:“是二奶奶
的名字,少混说。”贾母道:“你只管说罢。”女先儿忙笑着站起来说:“我们该
死了!不知是奶奶的讳。”凤姐儿笑道:“怕什么!你说罢。重名重姓的多着呢。”
女先儿又说道:“那年王老爷打发了王公子上京赶考,那日遇了大雨,到了一个庄
子上避雨。谁知这庄上也有位乡绅,姓李,与王老爷是世交,便留下这公子住在书
房里。这李乡绅膝下无儿,只有一位千金小姐。这小姐芳名叫做雏鸾,琴棋书画,
无所不通。”贾母忙道:“怪道叫做《凤求鸾》。不用说了,我已经猜着了:自然
是王熙凤要求这雏鸾小姐为妻了。”女先儿笑道:“老祖宗原来听过这回书?”众
人都道:“老太太什么没听见过!就是没听见,也猜着了。”贾母笑道:“这些书
就是一套子,左不过是些佳人才子,最没趣儿。把人家女儿说的这么坏,还说是‘佳
人’!编的连影儿也没有了。开口都是乡绅门第,父亲不是尚书,就是宰相。一个
小姐,必是爱如珍宝。这小姐必是通文知礼,无所不晓,竟是‘绝代佳人’,只见
了一个清俊男人,不管是亲是友,想起他的终身大事来,父母也忘了,书也忘了,
鬼不成鬼,贼不成贼,那一点儿像个佳人?就是满腹文章,做出这样事来,也算不
得是佳人了。比如一个男人家,满腹的文章,去做贼,难道那王法看他是个才子就
不入贼情一案了不成?可知那编书的是自己堵自己的嘴。再者:既说是世宦书香大
家子的小姐,又知礼读书,连夫人都知书识礼的,就是告老还家,自然奶妈子丫头
伏侍小姐的人也不少,怎么这些书上,凡有这样的事,就只小姐和紧跟的一个丫头
知道?你们想想,那些人都是管做什么的?可是前言不答后语了不是?”

众人听了,都笑说:“老太太这一说,是谎都批出来了。”贾母笑道:“有个
原故:编这样书的人,有一等妒人家富贵的,或者有求不遂心,所以编出来遭塌人
家。再有一等人,他自己看了这些书,看邪了,想着得一个佳人才好,所以编出来
取乐儿。他何尝知道那世宦读书人家儿的道理!别说那书上那些大家子,如今眼下
拿着咱们这中等人家说起,也没那样的事。别叫他诌掉了下巴子罢。所以我们从
不许说这些书,连丫头们也不懂这些话。这几年我老了,他们姐儿们住的远,我偶
然闷了,说几句听听,他们一来,就忙着止住了。”李薛二人都笑说:“这正是大
家子的规矩。连我们家也没有这些杂话叫孩子们听见。”

凤姐儿走上来斟酒,笑道:“罢,罢!酒冷了,老祖宗喝一口润润嗓子再掰谎
罢。这一回就叫做《掰谎记》,就出在本朝,本地,本年,本月,本日,本时。老
祖宗‘一张口难说两家话’,‘花开两朵,各表一枝’,‘是真是谎且不表,再整
观灯看戏的人’。老祖宗且让这二位亲戚吃杯酒、看两出戏着,再从逐朝话言掰起,
如何?”一面说,一面斟酒,一面笑。未说完,众人俱已笑倒了。两个女先儿也笑
个不住,都说:“奶奶好刚口!奶奶要一说书,真连我们吃饭的地方都没了。”薛
姨妈笑道:“你少兴头些!外头有人,比不得往常。”凤姐儿笑道:“外头只有一
位珍大哥哥,我们还是论哥哥妹妹,从小儿一处淘气淘了这么大。这几年因做了亲,
我如今立了多少规矩了!便不是从小儿兄妹,只论大伯子小婶儿,那二十四孝上‘斑
衣戏彩’,他们不能来戏彩引老祖宗笑一笑,我这里好容易引的老祖宗笑一笑,多
吃了一点东西,大家喜欢,都该谢我才是,难道反笑我不成?”贾母笑道:“可是
这两日我竟没有痛痛的笑一场,倒是亏他才一路说,笑的我这里痛快了些。我再吃
钟酒。”吃着酒,又命宝玉:“来敬你姐姐一杯。”凤姐儿笑道:“不用他敬,我
讨老祖宗的寿罢。”说着便将贾母的杯拿起来,将半杯剩酒吃了,将杯递与丫鬟,
另将温水浸的杯换一个上来。于是各席上的都撤去,另将温水浸着的代换,斟了新
酒上来,然后归坐。

女先儿回说:“老祖宗不听这书,或者弹一套曲子听听罢。”贾母道:“你们
两个对一套《将军令》罢。”二人听说,忙合弦按调拨弄起来。贾母因问:“天有
几更了?”众婆子忙回:“三更了。”贾母道:“怪道寒浸浸的起来。”早有众丫
鬟拿了添换的衣裳送来。王夫人起身陪笑说道:“老太太不如挪进暖阁里地炕上,
倒也罢了。这二位亲戚也不是外人,我们陪着就是了。”贾母听说,笑道:“既这
样说,不如大家都挪进去,岂不暖和?”王夫人道:“恐里头坐不下。”贾母道:
“我有道理:如今也不用这些桌子,只用两三张并起来,大家坐在一处挤着,又亲
热又暖和。”众人都道:“这才有趣儿!”说着,便起了席。众媳妇忙撤去残席,
里面直顺并了三张大桌,又添换了果馔摆好。贾母便说:“都别拘礼,听我分派你
们就坐才好。”说着,便让薛李正面上坐,自己西向坐了,叫宝琴、黛玉、湘云三
人皆紧依左右坐下,向宝玉说:“你挨着你太太。”于是邢夫人王夫人之中夹着宝
玉。宝钗等姐妹在西边,挨次下去,便是娄氏带着贾蓝、尤氏李纨夹着贾兰,下面
横头是贾蓉媳妇胡氏。贾母便说:“珍哥带着你兄弟们去罢,我也就睡了。”贾珍
等忙答应,又都进来听吩咐。贾母道:“快去罢,不用进来。才坐好了,又都起来。
你快歇着罢,明儿还有大事呢。”贾珍忙答应了,又笑道:“留下蓉儿斟酒才是。”
贾母笑道:“正是忘了他。”贾珍应了一个“是”,便转身带领贾琏等出来。二人
自是欢喜,便命人将贾琮贾璜各自送回家去,便约了贾琏去追欢买笑,不在话下。

这里贾母笑道:“我正想着,虽然这些人取乐,必得重孙一对双全的在席上才
好。蓉儿这可全了。蓉儿!和你媳妇坐在一处,倒也团圆了。”因有家人媳妇呈上
戏单,贾母笑道:“我们娘儿们正说得兴头,又要吵起来。况且那孩子们熬夜,怪
冷的。也罢,且叫他们歇歇,把咱们的女孩子们叫起来,就在这台上唱两出罢,也
给他们瞧瞧。”媳妇子们听了,答应出来,忙的一面着人往大观园去传人,一面二
门口去传小厮们伺候。小厮们忙至戏房,将班中所有大人一概带出,只留下小孩子
们。

一时,梨香院的教习带了文官等十二人从游廊角门出来,婆子们抱着几个软
包,因不及抬箱,料着贾母爱听的三五出戏的彩衣包了来。婆子们带了文官等进去,
见过,只垂手站着。贾母笑道:“大正月里,你师父也不放你们出来逛逛?你们如
今唱什么?才刚八出《八义》,闹的我头疼,咱们清淡些好。你瞧瞧,薛姨太太,
这李亲家太太,都是有戏的人家,不知听过多少好戏的;这些姑娘们都比咱们家的
姑娘见过好戏,听过好曲子。如今这小戏子又是那有名玩戏的人家的班子,虽是小
孩子,却比大班子还强。咱们好歹别落了褒贬!少不得弄个新样儿的:叫芳官唱一
出《寻梦》,只用箫和笙笛,馀者一概不用。”文官笑道:“老祖宗说的是。我们
的戏,自然不能入姨太太和亲家太太姑娘们的眼;不过听我们一个发脱口齿,再听
个喉咙罢了。”贾母笑道:“正是这话了。”李婶娘薛姨妈喜的笑道:“好个灵透
孩子,你也跟着老太太打趣我们。”贾母笑道:“我们这原是随便的玩意儿,又不
出去做买卖,所以竟不大合时。”说着,又叫葵官:“唱一出《惠明下书》,也不
用抹脸。只用这两出,叫他们二位太太听个助意儿罢了。若省了一点儿力,我可不
依。”文官等听了出来,忙去扮演上台,先是《寻梦》,次是《下书》。众人鸦雀
无闻。薛姨妈笑道:“实在戏也看过几百班,从没见过只用箫管的。”贾母道:“先
有,只是像方才《西楼》《楚江情》一只,多有小生吹箫合的。这合大套的实在少。
这也在人讲究罢了,这算什么出奇。”又指着湘云道:“我像他这么大的时候儿,
他爷爷有一班小戏,偏有一个弹琴的,凑了《西厢记》的《听琴》,《玉簪记》的
《琴挑》,《续琵琶》的《胡茄十八拍》,竟成了真的了。比这个更如何?”众人
都道:“那更难得了。”贾母于是叫过媳妇们来,吩咐文官等叫他们吹弹一套《灯
月圆》。媳妇们领命而去。

当下贾蓉夫妻二人捧酒一巡。凤姐儿因贾母十分高兴,便笑道:“趁着女先儿
们在这里,不如咱们传梅,行一套‘春喜上眉梢’的令,如何?”贾母笑道:“这
是个好令啊!正对时景儿。”忙命人取了黑漆铜钉花腔令鼓来,给女先儿击着。席
上取了一枝红梅,贾母笑道:“到了谁手里住了鼓,吃一杯,也要说些什么才好。”
凤姐儿笑道:“依我说,谁像老祖宗要什么有什么呢?我们这不会的不没意思吗?怎
么能雅俗共赏才好。不如谁住了,谁说个笑话儿罢。”众人听了,都知道他素日善
说笑话儿,肚内有无限的新鲜趣令;今见如此说,不但在席的诸人喜欢,连地下伏
侍的老小人等无不欢喜。那小丫头子们都忙去找姐姐叫妹妹的,告诉他们:“快来
听,二奶奶又说笑话儿了。”众丫头子们便挤了一屋子。

于是戏完乐罢,贾母将些汤细点果给文官等吃去,便命响鼓。那女先儿们都是
惯熟的,或紧或慢,或如残漏之滴,或如迸豆之急,或如惊马之驰,或如疾电之光,
忽然暗其鼓声,那梅方递至贾母手中,鼓声恰住,大家哈哈大笑。贾蓉忙上来斟了
一杯,众人都笑道:“自然老太太先喜了,我们才托赖些喜。”贾母笑道:“这酒
也罢了,只是这笑话儿倒有些难说。”众人都说:“老太太的比凤姑娘说的还好,
赏一个,我们也笑一笑。”贾母笑道:“并没有新鲜招笑儿的,少不得老脸皮厚的
说一个罢。”因说道:

“一家子养了十个儿子,娶了十房媳妇儿。惟有第十房媳妇儿聪明伶俐、心巧
嘴乖,公婆最疼,成日家说那九个不孝顺。这九个媳妇儿委屈,便商议说:‘咱们
九个心里孝顺,只是不像那小蹄子儿嘴巧,所以公公婆婆只说他好。这委屈向谁诉
去?’有主意的说道:‘咱们明儿到阎王庙去烧香,和阎王爷说去,问他一问:叫
我们托生为人,怎么单单给那小蹄子儿一张乖嘴,我们都入了夯嘴里头?’那八个
听了,都喜欢说:‘这个主意不错。’第二日,便都往阎王庙里来烧香。九个都在
供桌底下睡着了。九个魂专等阎王驾到。左等不来,右等也不到。正着急,只见孙
行者驾着斤斗云来了,看见九个魂,便要拿金箍棒打来。吓得九个魂忙跪下央求。
孙行者问起原故来,九个人忙细细的告诉了他。孙行者听了,把脚一跺,叹了一口
气道:‘这原故幸亏遇见我!等着阎王来了,他也不得知道。’九个人听了,就求
说:‘大圣发个慈悲,我们就好了。’孙行者笑道:‘却也不难:那日你们妯娌十
个托生时,可巧我到阎王那里去,因为撒了一泡尿在地下,你那个小婶儿便吃了。
你们如今要伶俐嘴乖,有的是尿,便撒泡你们吃就是了。’”

说毕,大家都笑起来。凤姐儿笑道:“好的呀!幸而我们都是夯嘴夯腮的,不
然,也就吃了猴儿尿了!”尤氏娄氏都笑向李纨道:“咱们这里头谁是吃过猴儿尿
的,别装没事人儿!”薛姨妈笑道:“笑话儿在对景就发笑。”

说着,又击起鼓来。小丫头子们只要听凤姐儿的笑话,便悄悄的和女先儿说明,
以咳嗽为记。须臾传至两遍,刚到了凤姐儿手里,小丫头子们故意咳嗽,女先儿便
住了。众人齐笑道:“这可拿住他了!快吃了酒,说一个好的罢,别太逗人笑的肠
子疼!”

凤姐儿想一想,笑道:“一家子也是过正月节,合家赏灯吃酒,真真的热闹非
常。祖婆婆、太婆婆、媳妇、孙子媳妇、重孙子媳妇、亲孙子媳妇、侄孙子、重孙
子、灰孙子、滴里搭拉的孙子、孙女儿、外孙女儿、姨表孙女儿、姑表孙女儿……
嗳哟哟!真好热闹!”众人听他说着,已经笑了,都说:“听这数贫嘴的!又不知要
编派那一个呢!”尤氏笑道:“你要招我,我可撕你的嘴!”凤姐儿起身拍手笑道:
“人家这里费力,你们紧着混,我就不说了。”贾母笑道:“你说你的,底下怎么
样?”凤姐儿想了一想,笑道:“底下就团团的坐了一屋子,吃了一夜酒,就散了。”

众人见他正言厉色的说了,也都再无有别话,怔怔的还等往下说,只觉他冰冷
无味的就住了。湘云看了他半日。凤姐儿笑道:“再说一个过正月节的:几个人拿
着房子大的炮仗往城外放去,引了上万的人跟着瞧去。有一个性急的人等不得,就
偷着拿香点着了。只见‘噗哧’的一声,众人哄然一笑,都散了。这抬炮仗的人抱
怨卖炮仗的捍的不结实,没等放就散了。”湘云道:“难道本人没听见?”凤姐儿
道:“本人原是个聋子。”众人听说,想了一回,不觉失声都大笑起来。又想着先
前那个没完的,问他道:“先那一个到底怎么样?也该说完了。”凤姐儿将桌子一
拍,道:“好罗唆!到了第二日是十六日,年也完了,节也完了,我看人忙着收东
西还闹不清,那里还知道底下的事了?”众人听说,复又笑起。

凤姐儿笑道:“外头已经四更多了,依我说:老祖宗也乏了,咱们也该‘聋子
放炮仗——散了’罢?”尤氏等用绢子握着嘴,笑的前仰后合,指他说道:“这个
东西真会数贫嘴!”贾母笑道:“真真这凤丫头,越发炼贫了!”一面说,一面吩
咐道:“他提起炮仗来,咱们也把烟火放了,解解酒。”贾蓉听了,忙出去带着小
厮们就在院子内安下屏架,将烟火设吊齐备。这烟火俱系各处进贡之物,虽不甚大,
却极精致,各色故事俱全,夹着各色的花炮。黛玉禀气虚弱,不禁劈拍之声,贾母
便搂他在怀内。薛姨妈便搂湘云,湘云笑道:“我不怕。”宝钗笑道:“他专爱自
己放大炮仗,还怕这个呢!”王夫人便将宝玉搂入怀内。凤姐笑道:“我们是没人
疼的!”尤氏笑道:“有我呢,我搂着你。——你这会子又撒娇儿了,听见放炮仗,
就像‘吃了蜜蜂儿屎’的,今儿又轻狂了。”凤姐儿笑道:“等散了,咱们园子里
放去,我比小厮们还放的好呢。”说话之间,外面一色色的放了又放。又有许多“满
天星”“九龙入云”“平地一声雷”“飞天十响”之类的零星小炮仗。放罢,然后
又命小戏子打了一回“莲花落”,撒得满台的钱,那些孩子们满台的抢钱取乐。

上汤时,贾母说:“夜长,不觉得有些饿了。”凤姐忙回说:“有预备的鸭子
肉粥。”贾母道:“我吃些清淡的罢。”凤姐儿忙道:“也有枣儿熬的粳米粥,预
备太太们吃斋的。”贾母道:“倒是这个还罢了。”说着,已经撤去残席,内外另
设各种精致小菜。大家随意吃了些,用过漱口茶,方散。

十七日一早,又过宁府行礼,伺候掩了祠门,收过影像,方回来。此日便是薛
姨妈家请吃年酒。贾母连日觉得身上乏了,坐了半日,回来了。自十八日以后,亲
友来请或来赴席的,贾母一概不会,有邢夫人、王夫人、凤姐三人料理。连宝玉只
除王子腾家去了,馀者亦皆不去,只说是贾母留下解闷。

当下元宵已过,凤姐忽然小产了,合家惊慌。

要知端底,下回分解。