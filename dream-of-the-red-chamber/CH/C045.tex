\chapter{金兰契互剖金兰语~风雨夕闷制风雨词}

话说凤姐儿正抚恤平儿,忽见众姐妹进来,忙让了坐,平儿斟上茶来。凤姐儿
笑道:“今儿来的这些人,倒像下帖子请了来的。”探春先笑道:“我们有两件事:
一件是我的,一件是四妹妹的,还夹着老太太的话。”凤姐儿笑道:“有什么事这
么要紧?”探春笑道:“我们起了个诗社,头一社就不齐全,众人脸软,所以就乱
了例了。我想必得你去做个‘监社御史’,铁面无私才好。再四妹妹为画园子,用
的东西这般那般不全,回了老太太,老太太说:‘只怕后头楼底下还有先剩下的,
找一找。若有呢,拿出来;若没有,叫人买去。’”凤姐儿笑道:“我又不会做什
么‘湿’咧‘干’的,叫我吃东西去倒会。”探春笑道:“你不会做,也不用你做;
你只监察着我们里头有偷安怠惰的,该怎么罚他就是了。”凤姐儿笑道:“你们别
哄我,我早猜着了,那里是请我做‘监察御史’?分明叫了我去做个进钱的铜商罢
咧。你们弄什么社,必是要轮流着做东道儿。你们的钱不够花,想出这个法子来勾
了我去,好和我要钱。可是这个主意不是?”说的众人都笑道:“你猜着了!”李
纨笑道:“真真你是个水晶心肝玻璃人儿。”凤姐笑道:“亏了你是个大嫂子呢!
姑娘们原是叫你带着念书,学规矩,学针线哪!这会子起诗社!能用几个钱,你就不
管了?老太太、太太罢了,原是老封君。你一个月十两银子的月钱,比我们多两倍
子,老太太、太太还说你‘寡妇失业’的,可怜,不够用,又有个小子,足足的又
添了十两银子,和老太太、太太平等;又给你园子里的地,各人取租子;年终分年
例,你又是上上分儿。你娘儿们主子奴才共总没有十个人,吃的穿的仍旧是大官中
的。通共算起来,也有四五百银子。这会子你就每年拿出一二百两来陪着他们玩玩
儿,有几年呢?他们明儿出了门子,难道你还赔不成?这会子你怕花钱,挑唆他们
来闹我,我乐得去吃个河落海干,我还不知道呢!”

李纨笑道:“你们听听,我说了一句,他就说了两车无赖的话!真真泥腿光棍,
专会打细算盘、分金掰两的。你这个东西,亏了还托生在诗书仕宦人家做小姐,又
是这么出了嫁,还是这么着。要生在贫寒小门小户人家,做了小子丫头,还不知怎
么下作呢!天下人都叫你算计了去!昨儿还打平儿,亏你伸的出手来。那黄汤难道灌
丧了狗肚子里去了?气的我只要替平儿打抱不平儿。忖夺了半日,好容易‘狗长尾
巴尖儿’的好日子,又怕老太太心里不受用,因此没来。究竟气还不平,你今儿倒
招我来了。给平儿拾鞋还不要呢!你们两个,很该换一个过儿才是。”说的众人都
笑了。凤姐忙笑道:“哦,我知道了,竟不是为诗为画来找我,竟是为平儿报仇来
了。我竟不知道平儿有你这么位仗腰子的人。想来就像有鬼拉着我的手似的,从今
我也不敢打他了。平姑娘,过来,我当着你大奶奶、姑娘们替你赔个不是,担待我
‘酒后无德’罢!”说着众人都笑了。李纨笑问平儿道:“如何?我说必要给你争
争气才罢。”平儿笑道:“虽是奶奶们取笑儿,我可禁不起呢。”李纨道:“什么
禁的起禁不起,有我呢。快拿钥匙叫你主子开门找东西去罢。”

凤姐儿笑道:“好嫂子!你且同他们去园子里去。才要把这米帐合他们算一算,
那边大太太又打发人来叫,又不知有什么话说,须得过去走一走。还有你们年下添
补的衣裳,打点给人做去呢。”李纨笑道:“这些事情我都不管,你只把我的事完
了,我好歇着去,省了这些姑娘们闹我。”凤姐儿忙笑道:“好嫂子,赏我一点空
儿。你是最疼我的,怎么今儿为平儿就不疼我了?往常你还劝我说:‘事情虽多,
也该保全身子,检点着偷空儿歇歇。’你今儿倒反逼起我的命来了。况且误了别人
年下的衣裳无碍,他姐儿们的要误了,却是你的责任。老太太岂不怪你不管闲事,
连一句现成的话也不说?我宁可自己落不是,也不敢累你呀。”李纨笑道:“你们
听听,说的好不好?把他会说话的!我且问你:这诗社到底管不管?”凤姐儿笑道:
“这是什么话?我不入社花几个钱,我不成了大观园的反叛了么,还想在这里吃饭
不成?明日一早就到任,下马拜了印,先放下五十两银子给你们慢慢的做会社东道
儿。我又不会作诗作文的,只不过是个大俗人罢了。‘监察’也罢,不‘监察’也
罢,有了钱了,愁着你们还不撵出我来!”说的众人又都笑起来。

凤姐儿道:“过会子我开了楼房,所有这些东西,叫人搬出来你们瞧,要使得,
留着使;要少什么,照你们的单子,我叫人赶着买去就是了。画绢我就裁出来。那
图样没有在老太太那里,那边珍大爷收着呢。说给你们,省了碰钉子去。我去打发
人取了来,一并叫人连绢交给相公们矾去。好不好呢?”李纨点头笑道:“这难为
你。果然这么着还罢了。那么着,咱们家去罢。等着他不送了去,再来闹他。”说
着便带了他姐妹们就走。凤姐儿道:“这些事再没别人,都是宝玉生出来的。”李
纨听了,忙回身笑道:“正为宝玉来,倒忘了他!头一社是他误了。我们脸软,你
说该怎么罚他?”凤姐想了想,说道:“没别的法子,只叫他把你们各人屋子里的
地罚他扫一遍就完了。”众人都笑道:“这话不差。”

说着才要回去,只见一个小丫头扶着赖嬷嬷进来。凤姐等忙站起来,笑道:“大
娘坐下。”又都向他道喜。赖嬷嬷向炕沿上坐了,笑道:“我也喜,主子们也喜。
要不是主子们的恩典,我这喜打那里来呢?昨儿奶奶又打发彩哥赏东西,我孙子在
门上朝上磕了头了。”李纨笑道:“多早晚上任去?”赖嬷嬷叹道:“我那里管他
们?由他们去罢。前儿在家里给我磕头,我没好话。我说:‘小子,别说你是官了,
横行霸道的!你今年活了三十岁,虽然是人家的奴才,一落娘胎胞儿,主子的恩典,
放你出来,上托着主子的洪福,下托着你老子娘,也是公子哥儿似的读书写字,也
是丫头、老婆、奶子捧凤凰似的。长了这么大,你那里知道那奴才两字是怎么写?
只知道享福,也不知你爷爷和你老子受的那苦恼,熬了两三辈子,好容易挣出你这
个东西,从小儿三灾八难,花的银子照样打出你这个银人儿来了。到二十岁上,又
蒙主子的恩典,许你捐了前程在身上。你看那正根正苗,忍饥挨饿的,要多少?你
一个奴才秧子,仔细折了福!如今乐了十年,不知怎么弄神弄鬼,求了主子,又选
出来了。县官虽小,事情却大,作那一处的官,就是那一方的父母。你不安分守己,
尽忠报国,孝敬主子,只怕天也不容你。’”李纨凤姐儿都笑道:“你也多虑。我
们看他也就好。先那几年,还进来了两次,这有好几年没来了。年下生日,只见他
的名字就罢了;前儿给老太太、太太磕头来,在老太太那院里,见他又穿着新官的
服色,倒发的威武了,比先时也胖了。他这一得了官,正该你乐呢,反倒愁起这些
来!他不好,还有他的父母呢,你只受用你的就完了。闲时坐个轿子进来,和老太
太斗斗牌,说说话儿,谁好意思的委屈了你。家去一般也是楼房厦厅,谁不敬你?
自然也是老封君似的了。”

平儿斟上茶来,赖嬷嬷忙站起来道:“姑娘不管叫那孩子倒来罢了,又生受你。”
说着,一面吃茶,一面又道:“奶奶不知道,这小孩子们全要管的严。饶这么严,
他们还偷空儿闹个乱子来,叫大人操心。知道的,说小孩子们淘气;不知道的,人
家就说仗着财势欺人,连主子名声也不好。恨的我没法儿,常把他老子叫了来,骂
一顿才好些。”因又指宝玉道:“不怕你嫌我:如今老爷不过这么管你一管,老太
太就护在头里。当日老爷小时,你爷爷那个打,谁没看见的!老爷小时,何曾像你
这么天不怕地不怕的。还有那边大老爷,虽然淘气,也没像你这扎窝子的样儿,也
是天天打。还有东府里你珍大哥哥的爷爷,那才是火上浇油的性子,说声恼了,什
么儿子,竟是审贼!如今我眼里看着,耳朵里听着,那珍大爷管儿子,倒也像当日
老祖宗的规矩,只是着三不着两的。他自己也不管一管自己,这些兄弟侄儿怎么怨
的不怕他?你心里明白,喜欢我说;不明白,嘴里不好意思,心里不知怎么骂我呢。”

说着,只见赖大家的来了,接着周瑞家的张材家的都进来回事情。凤姐儿笑道:
“媳妇来接婆婆来了。”赖大家的笑道:“不是接他老人家来的,倒是打听打听奶
奶姑娘们赏脸不赏脸?”赖嬷嬷听了,笑道:“可是我糊涂了!正经说的都没说,
且说些陈谷子烂芝麻的。因为我们小子选出来了,众亲友要给他贺喜,少不得家里
摆个酒。我想摆一日酒,请这个不请那个也不是。又想了一想,托主子的洪福,想
不到的这么荣耀光彩,就倾了家我也愿意的。因此吩咐了他老子连摆三日酒:头一
日在我们破花园子里摆几席酒,一台戏,请老太太、太太们、奶奶、姑娘们去散一
日闷,外头大厅上一台戏,几席酒,请老爷们、爷们,增增光;第二日再请亲友;
第三日再把我们两府里的伴儿请一请。热闹三天,也是托着主子的洪福一场,光辉
光辉。”李纨凤姐儿都笑道:“多早晚的日子?我们必去。只怕老太太高兴要去也
定不得。”赖大家的忙道:“择的日子是十四,只看我们奶奶的老脸罢了。”凤姐
儿笑道:“别人我不知道,我是一定去的。先说下:我可没有贺礼,也不知道放赏,
吃了一走儿,可别笑话。”赖大家的笑道:“奶奶说那里话?奶奶一喜欢,赏我们
三二万银子那就有了。”赖嬷嬷笑道:“我才去请老太太,老太太也说去,可算我
这脸还好。”说毕叮咛了一回,方起身要走。因看见周瑞家的,便想起一事来,因
说道:“可是还有一句话问奶奶:这周嫂子的儿子,犯了什么不是,撵了他不用?”
凤姐儿听了,笑道:“正是我要告诉你媳妇儿呢。事情多,也忘了。赖嫂子回去说
给你老头子,两府里不许收留他儿子,叫他各人去罢。”赖大家的只得答应着。

周瑞家的忙跪下央求。赖嬷嬷忙道:“什么事?说给我评评。”凤姐儿道:“前
儿我的生日,里头还没喝酒,他小子先醉了。老娘那边送了礼来,他不在外头张罗,
倒坐着骂人,礼也不送进来。两个女人进来了,他才带领小么儿们往里端。小么儿
们倒好好的,他拿的一盒子倒失了手,撒了一院子馒头。人去了,我打发彩明去说
他,他倒骂了彩明一顿。这样无法无天的忘八羔子,还不撵了做什么!”赖嬷嬷道:
“我当什么事情,原来为这个。奶奶听我说:他有不是,打他骂他,叫他改过就是
了;撵出去断乎使不得。他又比不得是咱们家的家生子儿,他现是太太的陪房,奶
奶只顾撵了他,太太的脸上不好看。我说奶奶教导他几板子,以戒下次,仍旧留着
才是。不看他娘,也看太太。”凤姐儿听了,便向赖大家的说道:“既这么着,明
儿叫了他来,打他四十棍,以后不许他喝酒。”赖大家的答应了。周瑞家的才磕头
起来,又要给赖嬷嬷磕头,赖大家的拉着方罢。然后他三人去了。

李纨等也就回园中来。至晚,果然凤姐命人找了许多旧收的画具出来,送至园
中。宝钗等选了一回。各色东西可用的只有一半,将那一半开了单子,给凤姐去照
样置买,不必细说。

一日外面矾了绢,起了稿子进来。宝玉每日便在惜春那边帮忙,探春、李纨、
迎春、宝钗等也都往那里来闲坐,一则观画,二则便于会面。宝钗因见天气凉爽,
夜复渐长,遂至贾母房中商议,打点些针线来。日间至贾母、王夫人处两次省候,
不免又承色陪坐;闲时园中姐妹处,也要不时闲话一回。故日间不大得闲,每夜灯
下女工,必至三更方寝。黛玉每岁至春分、秋分后必犯旧疾,今秋又遇着贾母高兴,
多游玩了两次,未免过劳了神,近日又复嗽起来。觉得比往常又重,所以总不出门,
只在自己房中将养。有时闷了,又盼个姐妹来说些闲话排遣;及至宝钗等来望候他,
说不得三五句话,又厌烦了。众人都体谅他病中,且素日形体娇弱,禁不得一些委
屈,所以他接待不周,礼数疏忽,也都不责他。

这日宝钗来望他,因说起这病症来。宝钗道:“这里走的几个大夫,虽都还好,
只是你吃他们的药,总不见效,不如再请一个高手的人来瞧一瞧,治好了岂不好?
每年间闹一春一夏,又不老,又不小,成什么,也不是个常法儿。”黛玉道:“不
中用。我知道我的病是不能好的了。且别说病,只论好的时候我是怎么个形景儿,
就可知了。”宝钗点头道:“可正是这话。古人说,‘食谷者生’,你素日吃的竟
不能添养精神气血,也不是好事。”黛玉叹道:“‘死生有命,富贵在天’,也不
是人力可强求的。今年比往年反觉又重了些似的。”说话之间,已咳嗽了两三次。
宝钗道:“昨儿我看你那药方上,人参肉桂觉得太多了。虽说益气补神,也不宜太
热。依我说:先以平肝养胃为要。肝火一平,不能克土,胃气无病,饮食就可以养
人了。每日早起,拿上等燕窝一两、冰糖五钱,用银铞子熬出粥来,要吃惯了,比
药还强,最是滋阴补气的。”

黛玉叹道:“你素日待人,固然是极好的,然我最是个多心的人,只当你有心
藏奸。从前日你说看杂书不好,又劝我那些好话,竟大感激你。往日竟是我错了,
实在误到如今。细细算来,我母亲去世的时候,又无姐妹兄弟,我长了今年十五岁,
竟没一个人像你前日的话教导我。怪不得云丫头说你好。我往日见他赞你,我还不
受用;昨儿我亲自经过,才知道了。比如你说了那个,我再不轻放过你的;你竟不
介意,反劝我那些话:可知我竟自误了。若不是前日看出来,今日这话,再不对你
说。你方才叫我吃燕窝粥的话,虽然燕窝易得,但只我因身子不好了,每年犯了这
病,也没什么要紧的去处;请大夫,熬药,人参,肉桂,已经闹了个天翻地覆了,
这会子我又兴出新文来,熬什么燕窝粥,老太太、太太、凤姐姐这三个人便没话,
那些底下老婆子丫头们,未免嫌我太多事了。你看这里这些人,因见老太太多疼了
宝玉和凤姐姐两个,他们尚虎视眈眈,背地里言三语四的,何况于我?况我又不是
正经主子,原是无依无靠投奔了来的,他们已经多嫌着我呢。如今我还不知进退,
何苦叫他们咒我?”

宝钗道:“这么说,我也是和你一样。”黛玉道:“你如何比我?你又有母亲,
又有哥哥。这里又有买卖地土,家里又仍旧有房有地。你不过亲戚的情分,白住在
这里,一应大小事情又不沾他们一文半个,要走就走了。我是一无所有,吃穿用度,
一草一木,皆是和他们家的姑娘一样,那起小人岂有不多嫌的?”宝钗笑道:“将
来也不过多费得一副嫁妆罢了,如今也愁不到那里。”黛玉听了不觉红了脸,笑道:
“人家把你当个正经人,才把心里烦难告诉你听,你反拿我取笑儿!”宝钗笑道:
“虽是取笑儿,却也是真话。你放心,我在这里一日,我与你消遣一日。你有什么
委屈烦难,只管告诉我,我能解的,自然替你解。我虽有个哥哥,你也是知道的;
只有个母亲,比你略强些。咱们也算同病相怜。你也是个明白人,何必作‘司马牛
之叹’?你才说的也是,多一事不如省一事。我明日家去和妈妈说了,只怕燕窝我
们家里还有,与你送几两。每日叫丫头们就熬了,又便宜,又不惊师动众的。”黛
玉忙笑道:“东西是小,难得你多情如此。”宝钗道:“这有什么放在嘴里的!只
愁我人人跟前失于应候罢了。这会子只怕你烦了,我且去了。”黛玉道:“晚上再
来和我说句话儿。”宝钗答应着便去了,不在话下。

这里黛玉喝了两口稀粥,仍歪在床上。不想日未落时,天就变了,淅淅沥沥下
起雨来。秋霖脉脉,阴晴不定,那天渐渐的黄昏时候了,且阴的沉黑,兼着那雨滴
竹梢,更觉凄凉。知宝钗不能来了,便在灯下随便拿了一本书,却是《乐府杂稿》,
有《秋闺怨》、《别离怨》等词。黛玉不觉心有所感,不禁发于章句,遂成《代别
离》一首,拟《春江花月夜》之格,乃名其词为《秋窗风雨夕》。词曰:
秋花惨淡秋草黄,耿耿秋灯秋夜长。
已觉秋窗秋不尽,那堪风雨助秋凉!
助秋风雨来何速?惊破秋窗秋梦续。
抱得秋情不忍眠,自向秋屏挑泪烛。
泪烛摇摇短檠,牵愁照恨动离情。
谁家秋院无风入?何处秋窗无雨声?
罗衾不奈秋风力,残漏声催秋雨急。
连宵脉脉复飕飕,灯前似伴离人泣。
寒烟小院转萧条,疏竹虚窗时滴沥。
不知风雨几时休,已教泪洒窗纱湿。

吟罢搁笔,方欲安寝,丫鬟报说:“宝二爷来了。”一语未尽,只见宝玉头上
戴着大箬笠,身上披着蓑衣。黛玉不觉笑道:“那里来的这么个渔翁?”宝玉忙问:
“今儿好?吃了药了没有?今儿一日吃了多少饭?”一面说,一面摘了笠,脱了蓑。
一手举起灯来,一手遮着灯儿,向黛玉脸上照了一照,觑着瞧了一瞧,笑道:“今
儿气色好了些。”黛玉看他脱了蓑衣,里面只穿半旧红绫短袄,系着绿汗巾子,膝
上露出绿绸撒花裤子,底下是掐金满绣的绵纱袜子,着蝴蝶落花鞋。黛玉问道:
“上头怕雨,底下这鞋袜子是不怕的?也倒干净些呀。”宝玉笑道:“我这一套是
全的。一双棠木屐,才穿了来,脱在廊檐下了。”黛玉又看那蓑衣斗笠不是寻常市
卖的,十分细致轻巧,因说道:“是什么草编的?怪道穿上不像那刺猬似的。”宝
玉道:“这三样都是北静王送的。他闲常下雨时,在家里也是这样。你喜欢这个,
我也弄一套来送你。别的都罢了,惟有这斗笠有趣:上头这顶儿是活的,冬天下雪
戴上帽子,就把竹信子抽了去,拿下顶子来,只剩了这个圈子,下雪时男女都带得。
我送你一顶,冬天下雪戴。”黛玉笑道:“我不要他。戴上那个,成了画儿上画的
和戏上扮的那渔婆儿了。”及说了出来,方想起来这话恰与方才说宝玉的话相连了,
后悔不迭,羞的脸飞红,伏在桌上,嗽个不住。

宝玉却不留心,因见案上有诗,遂拿起来看了一遍,又不觉叫好。黛玉听了,
忙起来夺在手内,灯上烧了。宝玉笑道:“我已记熟了。”黛玉道:“我要歇了,
你请去罢,明日再来。”宝玉听了,回手向怀内掏出一个核桃大的金表来,瞧了一
瞧,那针已指到戌末亥初之间,忙又揣了,说道:“原该歇了,又搅的你劳了半日
神。”说着,披蓑戴笠出去了,又翻身进来,问道:“你想什么吃?你告诉我,我
明儿一早回老太太,岂不比老婆子们说的明白?”黛玉笑道:“等我夜里想着了,
明日一早告诉你。你听雨越发紧了,快去罢。可有人跟没有?”两个婆子答应:“有,
在外面拿着伞点着灯笼呢。”黛玉笑道:“这个天点灯笼?”宝玉道:“不相干,
是羊角的,不怕雨。”黛玉听说,回手向书架上把个玻璃绣球灯拿下来,命点一枝
小蜡儿来,递与宝玉道:“这个又比那个亮,正是雨里点的。”宝玉道:“我也有
这么一个,怕他们失脚滑倒了打破了,所以没点来。”黛玉道:“跌了灯值钱呢,
是跌了人值钱?你又穿不惯木屐子。那灯笼叫他们前头点着,这个又轻巧又亮,原
是雨里自己拿着的。你自己手里拿着这个,岂不好?明儿再送来。就失了手也有限
的,怎么忽然又变出这‘剖腹藏珠’的脾气来!”宝玉听了,随过来接了。前头两
个婆子打着伞,拿着羊角灯,后头还有两个小丫鬟打着伞。宝玉便将这个灯递给一
个小丫头捧着,宝玉扶着他的肩,一径去了。

就有蘅芜院两个婆子,也打着伞提着灯,送了一大包燕窝来,还有一包子洁粉
梅片雪花洋糖。说:“这比买的强。我们姑娘说:‘姑娘先吃着,完了再送来。’”
黛玉回说:“费心。”命他:“外头坐了吃茶。”婆子笑道:“不喝茶了,我们还
有事呢。”黛玉笑道:“我也知道你们忙。如今天又凉,夜又长,越发该会个夜局,
赌两场了。”一个婆子笑道:“不瞒姑娘说,今年我沾了光了。横竖每夜有几个上
夜的人,误了更又不好,不如会个夜局,又坐了更,又解了闷。今儿又是我的头家,
如今园门关了,就该上场儿了。”黛玉听了,笑道:“难为你们。误了你们的发财,
冒雨送来。”命人:“给他们几百钱打些酒吃,避避雨气。”那两个婆子笑道:“又
破费姑娘赏酒吃。”说着磕了头,出外面接了钱,打伞去了。

紫鹃收起燕窝,然后移灯下帘,伏侍黛玉睡下。黛玉自在枕上感念宝钗,一时
又羡他有母有兄;一回又想宝玉素昔和睦,终有嫌疑。又听见窗外竹梢蕉叶之上,
雨声淅沥,清寒透幕,不觉又滴下泪来。直到四更方渐渐的睡熟了。暂且无话。

要知端底,且看下回分解。