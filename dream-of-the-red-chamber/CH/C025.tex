\chapter{魇魔法叔嫂逢五鬼~通灵玉蒙蔽遇双真}

话说小红心神恍惚,情思缠绵,忽朦胧睡去,遇见贾芸要拉他,却回身一跑,
被门槛绊了一跤,唬醒过来,方知是梦。因此翻来覆去,一夜无眠。至次日天明,
方才起来,有几个丫头来会他去打扫屋子地面,舀洗脸水。这小红也不梳妆,向镜
中胡乱挽了一挽头发,洗了洗手脸,便来打扫房屋。谁知宝玉昨儿见了他,也就留
心,想着指名唤他来使用,一则怕袭人等多心,二则又不知他是怎么个情性,因而
纳闷。早晨起来,也不梳洗,只坐着出神。一时下了纸窗,隔着纱屉子,向外看的
真切,只见几个丫头在那里打扫院子,都擦胭抹粉、插花带柳的,独不见昨儿那一
个。宝玉便拉着鞋,走出房门,只装做看花,东瞧西望。一抬头,只见西南角上
游廊下栏杆旁有一个人倚在那里,却为一株海棠花所遮,看不真切。近前一步仔细
看时,正是昨儿那个丫头,在那里出神。此时宝玉要迎上去,又不好意思。正想着,
忽见碧痕来请洗脸,只得进去了。

却说小红正自出神,忽见袭人招手叫他,只得走上前来。袭人笑道:“咱们的
喷壶坏了,你到林姑娘那边借用一用。”小红便走向潇湘馆去,到了翠烟桥,抬头
一望,只见山坡高处都拦着帷幕,方想起今日有匠役在此种树。原来远远的一簇人
在那里掘土,贾芸正坐在山子石上监工。小红待要过去又不敢过去,只得悄悄向潇
湘馆取了喷壶而回。无精打彩,自向房内躺着。众人只说他是身子不快,也不理论。

过了一日,原来次日是王子腾夫人的寿诞,那里原打发人来请贾母、王夫人,
王夫人见贾母不去,也不便去了。倒是薛姨妈同着凤姐儿并贾家三个姊妹、宝钗、
宝玉,一齐都去了。至晚方回。

王夫人正过薛姨妈院里坐着,见贾环下了学,命他去抄《金刚经咒》唪诵。那
贾环便来到王夫人炕上坐着,命人点了蜡烛,拿腔做势的抄写。一时又叫彩云倒钟
茶来,一时又叫玉钏剪蜡花,又说金钏挡了灯亮儿。众丫鬟们素日厌恶他,都不答
理。只有彩霞还和他合得来,倒了茶给他,因向他悄悄的道:“你安分些罢,何苦
讨人厌。”贾环把眼一瞅道:“我也知道,你别哄我。如今你和宝玉好了,不理我,
我也看出来了。”彩霞咬着牙,向他头上戳了一指头,道:“没良心的,‘狗咬吕
洞宾——不识好歹。’”

两人正说着,只见凤姐跟着王夫人都过来了。王夫人便一长一短问他今日是那
几位堂客,戏文好歹,酒席如何。不多时,宝玉也来了,见了王夫人,也规规矩矩
说了几句话,便命人除去了抹额,脱了袍服,拉了靴子,就一头滚在王夫人怀里。
王夫人便用手摩挲抚弄他,宝玉也扳着王夫人的脖子说长说短的。王夫人道:“我
的儿,又吃多了酒,脸上滚热的。你还只是揉搓,一会子闹上酒来!还不在那里静
静的躺一会子去呢。”说着,便叫人拿枕头。宝玉因就在王夫人身后倒下,又叫彩
霞来替他拍着。宝玉便和彩霞说笑,只见彩霞淡淡的不大答理,两眼只向着贾环。
宝玉便拉他的手,说道:“好姐姐,你也理我理儿。”一面说,一面拉他的手。彩
霞夺手不肯,便说:“再闹就嚷了!”二人正闹着,原来贾环听见了,素日原恨宝
玉,今见他和彩霞玩耍,心上越发按不下这口气。因一沉思,计上心来,故作失手,
将那一盏油汪汪的蜡烛,向宝玉脸上只一推。

只听宝玉“嗳哟”的一声,满屋里人都唬了一跳。连忙将地下的绰灯移过来一
照,只见宝玉满脸是油。王夫人又气又急,忙命人替宝玉擦洗,一面骂贾环。凤姐
三步两步上炕去替宝玉收拾着,一面说:“这老三还是这么‘毛脚鸡’似的。我说
你上不得台盘!赵姨娘平时也该教导教导他!”一句话提醒了王夫人,遂叫过赵姨
娘来,骂道:“养出这样黑心种子来,也不教训教训!几番几次我都不理论,你们
一发得了意了,一发上来了!”那赵姨娘只得忍气吞声,也上去帮着他们替宝玉收
拾。只见宝玉左边脸上起了一溜燎泡,幸而没伤眼睛。王夫人看了,又心疼,又怕
贾母问时难以回答,急的又把赵姨娘骂一顿;又安慰了宝玉,一面取了“败毒散”
来敷上。宝玉说:“有些疼,还不妨事。明日老太太问,只说我自己烫的就是了。”
凤姐道:“就说自己烫的,也要骂人不小心,横竖有一场气生。”王夫人命人好生
送了宝玉回房去。袭人等见了,都慌的了不得。那黛玉见宝玉出了一天的门,便闷
闷的,晚间打发人来问了两三遍,知道烫了,便亲自赶过来。只瞧见宝玉自己拿镜
子照呢,左边脸上满满的敷了一脸药。黛玉只当十分烫的利害,忙近前瞧瞧,宝玉
却把脸遮了,摇手叫他出去:知他素性好洁,故不肯叫他瞧。黛玉也就罢了,但问
他:“疼的怎样?”宝玉道:“也不很疼。养一两日就好了。”黛玉坐了一会回去
了。

次日,宝玉见了贾母,虽自己承认自己烫的,贾母免不得又把跟从的人骂了一
顿。过了一日,有宝玉寄名的干娘马道婆到府里来,见了宝玉,唬了一大跳,问其
缘由,说是烫的,便点头叹息,一面向宝玉脸上用指头画了几画,口内嘟嘟囔囔的,
又咒诵了一回,说道:“包管好了。这不过是一时飞灾。”又向贾母道:“老祖宗,
老菩萨,那里知道那佛经上说的利害!大凡王公卿相人家的子弟,只一生长下来,
暗里就有多少促狭鬼跟着他,得空儿就拧他一下,或掐他一下,或吃饭时打下他的
饭碗来,或走着推他一跤,所以往往的那些大家子孙多有长不大的。”贾母听如此
说,便问:“这有什么法儿解救没有呢?”马道婆便说道:“这个容易,只是替他
多做些因果善事,也就罢了。再那经上还说:西方有位大光明普照菩萨,专管照耀
阴暗邪祟,若有善男信女虔心供奉者,可以永保儿孙康宁,再无撞客邪祟之灾。”
贾母道:“倒不知怎么供奉这位菩萨?”马道婆说:“也不值什么,不过除香烛供
奉以外,一天多添几斤香油,点个大海灯。那海灯就是菩萨现身的法象,昼夜不息
的。”贾母道:“一天一夜也得多少油?我也做个好事。”马道婆说:“这也不拘
多少,随施主愿心。像我家里就有好几处的王妃诰命供奉的:南安郡王府里太妃,
他许的愿心大,一天是四十八斤油,一斤灯草,那海灯也只比缸略小些;锦乡侯的
诰命次一等,一天不过二十斤油;再有几家,或十斤、八斤、三斤、五斤的不等,
也少不得要替他点。”贾母点头思忖。马道婆道:“还有一件,若是为父母尊长的,
多舍些不妨;既是老祖宗为宝玉,若舍多了,怕哥儿担不起,反折了福气了。要舍,
大则七斤,小则五斤,也就是了。”贾母道:“既这么样,就一日五斤,每月打总
儿关了去。”马道婆道:“阿弥陀佛,慈悲大菩萨!”贾母又叫人来吩咐:“以后
宝玉出门,拿几串钱交给他的小子们,一路施舍给僧道贫苦之人。”

说毕,那道婆便往各房问安闲逛去了。一时来到赵姨娘屋里,二人见过,赵姨
娘命小丫头倒茶给他吃。赵姨娘正粘鞋呢,马道婆见炕上堆着些零星绸缎,因说:
“我正没有鞋面子,姨奶奶给我些零碎绸子缎子,不拘颜色,做双鞋穿罢。”赵姨
娘叹口气道:“你瞧,那里头还有块像样儿的么?有好东西也到不了我这里。你不
嫌不好,挑两块去就是了。”马道婆便挑了几块,掖在袖里。赵姨娘又问:“前日
我打发人送了五百钱去,你可在药王面前上了供没有?”马道婆道:“早已替你上
了。”赵姨娘叹气道:“阿弥陀佛!我手里但凡从容些,也时常来上供,只是‘心
有馀而力不足’。”马道婆道:“你只放心,将来熬的环哥大了,得个一官半职,
那时你要做多大功德还怕不能么?”

赵姨娘听了笑道:“罢,罢!再别提起!如今就是榜样。我们娘儿们跟的上这屋
里那一个儿?宝玉儿还是小孩子家,长的得人意儿,大人偏疼他些儿也还罢了;我
只不服这个主儿!”一面说,一面伸了两个指头。马道婆会意,便问道:“可是琏
二奶奶?”赵姨娘唬的忙摇手儿,起身掀帘子一看,见无人,方回身向道婆说:“了
不得,了不得!提起这个主儿,这一分家私要不都叫他搬了娘家去,我也不是个人!”
马道婆见说,便探他的口气道:“我还用你说?难道都看不出来!也亏了你们心里不
理论,只凭他去倒也好。”赵姨娘道:“我的娘!不凭他去,难道谁还敢把他怎么
样吗?”马道婆道:“不是我说句造孽的话:你们没本事,也难怪。明里不敢罢咧,
暗里也算计了,还等到如今!”赵姨娘听这话里有话,心里暗暗的喜欢,便说道:
“怎么暗里算计?我倒有这个心,只是没这样的能干人。你教给我这个法子,我大
大的谢你。”马道婆听了这话拿拢了一处,便又故意说道:“阿弥陀佛!你快别问
我,我那里知道这些事?罪罪过过的。”赵姨娘道:“你又来了!你是最肯济困扶危
的人,难道就眼睁睁的看着人家来摆布死了我们娘儿们不成?难道还怕我不谢你
么?”马道婆听如此,便笑道:“要说我不忍你们娘儿两个受别人的委屈,还犹可,
要说谢我,那我可是不想的呀。”赵姨娘听这话松动了些,便说:“你这么个明白
人,怎么糊涂了?果然法子灵验,把他两人绝了,这家私还怕不是我们的?那时候你
要什么不得呢?”马道婆听了,低了半日头,说:“那时候儿事情妥当了,又无凭
据,你还理我呢!”赵姨娘道:“这有何难?我攒了几两体己,还有些衣裳首饰,
你先拿几样去。我再写个欠契给你,到那时候儿,我照数还你。”马道婆想了一回
道:“也罢了,我少不得先垫上了。”

赵姨娘不及再问,忙将一个小丫头也支开,赶着开了箱子,将首饰拿了些出来,
并体己散碎银子,又写了五十两欠约,递与马道婆道:“你先拿去作供养。”马道
婆见了这些东西,又有欠字,遂满口应承,伸手先将银子拿了,然后收了契。向赵
姨娘要了张纸,拿剪子铰了两个纸人儿,问了他二人年庚,写在上面;又找了一张
蓝纸,铰了五个青面鬼,叫他并在一处,拿针钉了:“回去我再作法,自有效验的。”
忽见王夫人的丫头进来道:“姨奶奶在屋里呢么?太太等你呢。”于是二人散了,
马道婆自去,不在话下。

却说黛玉因宝玉烫了脸不出门,倒常在一处说话儿。这日饭后,看了两篇书,
又和紫鹃作了一会针线,总闷闷不舒,便出来看庭前才迸出的新笋。不觉出了院门,
来到园中,四望无人,惟见花光鸟语,信步便往怡红院来。只见几个丫头舀水,都
在游廊上看画眉洗澡呢。听见房内笑声,原来是李纨、凤姐、宝钗都在这里。一见
他进来,都笑道:“这不又来了两个?”黛玉笑道:“今日齐全,谁下帖子请的?”
凤姐道:“我前日打发人送了两瓶茶叶给姑娘,可还好么?”黛玉道:“我正忘了,
多谢想着。”宝玉道:“我尝了不好,也不知别人说怎么样。”宝钗道:“口头也
还好。”凤姐道:“那是暹罗国进贡的。我尝了不觉怎么好,还不及我们常喝的呢。”
黛玉道:“我吃着却好,不知你们的脾胃是怎样的。”宝玉道:“你说好,把我的
都拿了吃去罢。”凤姐道:“我那里还多着呢。”黛玉道:“我叫丫头取去。”凤
姐道:“不用,我打发人送来。我明日还有一事求你,一同叫人送来罢。”

黛玉听了,笑道:“你们听听:这是吃了他一点子茶叶,就使唤起人来了。”
凤姐笑道:“你既吃了我们家的茶,怎么还不给我们家作媳妇儿?”众人都大笑起
来。黛玉涨红了脸,回过头去,一声儿不言语。宝钗笑道:“二嫂子的诙谐真是好
的。”黛玉道:“什么诙谐!不过是贫嘴贱舌的讨人厌罢了!”说着又啐了一口。
凤姐笑道:“你给我们家做了媳妇,还亏负你么?”指着宝玉道:“你瞧瞧人物儿
配不上?门第儿配不上?根基儿家私儿配不上?那一点儿玷辱你?”黛玉起身便走。
宝钗叫道:“颦儿急了,还不回来呢!走了倒没意思。”说着,站起来拉住。才到
房门,只见赵姨娘和周姨娘两个人都来瞧宝玉。宝玉和众人都起身让坐,独凤姐不
理。宝钗正欲说话,只见王夫人房里的丫头来说:“舅太太来了,请奶奶姑娘们过
去呢。”李纨连忙同着凤姐儿走了。赵周两人也都出去了。宝玉道:“我不能出去,
你们好歹别叫舅母进来。”又说:“林妹妹,你略站站,我和你说话。”凤姐听了,
回头向黛玉道:“有人叫你说话呢,回去罢。”便把黛玉往后一推,和李纨笑着去
了。

这里宝玉拉了黛玉的手,只是笑,又不说话。黛玉不觉又红了脸,挣着要走。
宝玉道:“嗳哟!好头疼!”黛玉道:“该,阿弥陀佛!”宝玉大叫一声,将身一
跳,离地有三四尺高,口内乱嚷,尽是胡话。黛玉并众丫鬟都唬慌了,忙报知王夫
人与贾母。此时王子腾的夫人也在这里,都一齐来看。宝玉一发拿刀弄杖、寻死觅
活的,闹的天翻地覆。贾母王夫人一见,唬的抖衣乱战,儿一声肉一声,放声大哭。
于是惊动了众人,连贾赦、邢夫人、贾珍、贾政并琏、蓉、芸、萍、薛姨妈、薛蟠
并周瑞家的一干家中上下人等并丫鬟媳妇等,都来园内看视,登时乱麻一般。正没
个主意,只见凤姐手持一把明晃晃的刀砍进园来,见鸡杀鸡,见犬杀犬,见了人瞪
着眼就要杀人。众人一发慌了。周瑞家的带着几个力大的女人,上去抱住,夺了刀,
抬回房中。平儿丰儿等哭的哀天叫地。贾政心中也着忙。当下众人七言八语,有说
送祟的,有说跳神的,有荐玉皇阁张道士捉怪的,整闹了半日,祈求祷告,百般医
治,并不见好。日落后,王子腾夫人告辞去了。

次日,王子胜也来问候。接着小史侯家、邢夫人弟兄并各亲戚都来瞧看,也有
送符水的,也有荐僧道的,也有荐医的。他叔嫂二人一发糊涂,不省人事,身热如
火,在床上乱说。到夜里更甚,因此那些婆子丫鬟不敢上前,故将他叔嫂二人都搬
到王夫人的上房内,着人轮班守视。贾母、王夫人、邢夫人并薛姨妈寸步不离,只
围着哭。此时贾赦贾政又恐哭坏了贾母,日夜熬油费火,闹的上下不安。贾赦还各
处去寻觅僧道。贾政见不效验,因阻贾赦道:“儿女之数总由天命,非人力可强。
他二人之病百般医治不效,想是天意该如此,也只好由他去。”贾赦不理,仍是百
般忙乱。

看看三日的光阴,凤姐宝玉躺在床上,连气息都微了。合家都说没了指望了,
忙的将他二人的后事都治备下了。贾母、王夫人、贾琏、平儿、袭人等更哭的死去
活来。只有赵姨娘外面假作忧愁,心中称愿。

至第四日早,宝玉忽睁开眼向贾母说道:“从今已后,我可不在你家了,快打
发我走罢。”贾母听见这话,如同摘了心肝一般。赵姨娘在旁劝道:“老太太也不
必过于悲痛:哥儿已是不中用了,不如把哥儿的衣服穿好,让他早些回去,也省他
受些苦。只管舍不得他,这口气不断,他在那里,也受罪不安——”这些话没说完,
被贾母照脸啐了一口唾沫,骂道:“烂了舌头的混帐老婆!怎么见得不中用了?你愿
意他死了,有什么好处?你别作梦!他死了,我只合你们要命!都是你们素日调唆着,
逼他念书写字,把胆子唬破了,见了他老子就像个避猫鼠儿一样。都不是你们这起
小妇调唆的?这会子逼死了他,你们就随了心了!——我饶那一个?”一面哭,一面
骂。贾政在旁听见这些话,心里越发着急,忙喝退了赵姨娘,委宛劝解了一番。忽
有人来回:“两口棺木都做齐了。”贾母闻之,如刀刺心,一发哭着大骂,问:“是
谁叫做的棺材?快把做棺材的人拿来打死!”闹了个天翻地覆。

忽听见空中隐隐有木鱼声,念了一句“南无解冤解结菩萨!有那人口不利、家
宅不安、中邪祟、逢凶险的,找我们医治。”贾母王夫人都听见了,便命人向街上
找寻去。原来是一个癞和尚同一个跛道士。那和尚是怎的模样?但见:
鼻如悬胆两眉长,目似明星有宝光。
破衲芒鞋无住迹,腌更有一头疮。
那道人是如何模样?看他时:
一足高来一足低,浑身带水又拖泥。
相逢若问家何处,却在蓬莱弱水西。

贾政因命人请进来,问他二人:“在何山修道?”那僧笑道:“长官不消多话,
因知府上人口欠安,特来医治的。”贾政道:“有两个人中了邪,不知有何仙方可
治?”那道人笑道:“你家现有希世之宝,可治此病,何须问方!”贾政心中便动
了,因道:“小儿生时虽带了一块玉来,上面刻着‘能除凶邪’,然亦未见灵效。”
那僧道:“长官有所不知。那宝玉原是灵的,只因为声色货利所迷,故此不灵了。
今将此宝取出来,待我持诵持诵,自然依旧灵了。”贾政便向宝玉项上取下那块玉
来,递与他二人。那和尚擎在掌上,长叹一声,道:“青埂峰下,别来十三载矣。
人世光阴迅速,尘缘未断,奈何奈何!可羡你当日那段好处:
天不拘兮地不羁,心头无喜亦无悲。
只因锻炼通灵后,便向人间惹是非。
可惜今日这番经历呵:
粉渍脂痕污宝光,房栊日夜困鸳鸯。
沉酣一梦终须醒,冤债偿清好散场。”
念毕,又摩弄了一回,说了些疯话,递与贾政道:“此物已灵,不可亵渎,悬于卧
室槛上,除自己亲人外,不可令阴人冲犯。三十三日之后,包管好了。”贾政忙命
人让茶,那二人已经走了,只得依言而行。

凤姐宝玉果一日好似一日的,渐渐醒来,知道饿了,贾母王夫人才放心了。众
姊妹都在外间听消息。黛玉先念了一声佛,宝钗笑而不言。惜春道:“宝姐姐笑什
么?”宝钗道:“我笑如来佛比人还忙:又要度化众生;又要保佑人家病痛,都叫
他速好;又要管人家的婚姻,叫他成就。你说可忙不忙?可好笑不好笑?”一时黛
玉红了脸,啐了一口道:“你们都不是好人!再不跟着好人学,只跟着凤丫头学的
贫嘴贱舌的。”一面说,一面掀帘子出去了。

欲知端详,下回分解。