\chapter{滴翠亭杨妃戏彩蝶~埋香冢飞燕泣残红}

话说黛玉正自悲泣,忽听院门响处,只见宝钗出来了,宝玉袭人一群人都送出
来。待要上去问着宝玉,又恐当着众人问羞了宝玉不便,因而闪过一旁,让宝钗去
了,宝玉等进去关了门,方转过来,尚望着门洒了几点泪。自觉无味,转身回来,
无精打彩的卸了残妆。紫鹃雪雁素日知道黛玉的情性:无事闷坐,不是愁眉,便是
长叹,且好端端的不知为着什么,常常的便自泪不干的。先时还有人解劝,或怕他
思父母,想家乡,受委屈,用话来宽慰。谁知后来一年一月的,竟是常常如此,把
这个样儿看惯了,也都不理论了。所以也没人去理他,由他闷坐,只管外间自便去
了。那黛玉倚着床栏杆,两手抱着膝,眼睛含着泪,好似木雕泥塑的一般,直坐到
二更多天方才睡了。一宿无话。

至次日乃是四月二十六日,原来这日未时交芒种节。尚古风俗:凡交芒种节的
这日,都要设摆各色礼物,祭饯花神,言芒种一过,便是夏日了,众花皆卸,花神
退位,须要饯行。闺中更兴这件风俗,所以大观园中之人都早起来了。那些女孩子
们,或用花瓣柳枝编成轿马的,或用绫锦纱罗叠成干旄旌幢的,都用彩线系了,每
一棵树头每一枝花上,都系了这些物事。满园里绣带飘摇,花枝招展,更兼这些人
打扮的桃羞杏让,燕妒莺惭,一时也道不尽。

且说宝钗、迎春、探春、惜春、李纨、凤姐等并大姐儿、香菱与众丫鬟们,都
在园里玩耍,独不见黛玉,迎春因说道:“林妹妹怎么不见?好个懒丫头,这会子
难道还睡觉不成?”宝钗道:“你们等着,等我去闹了他来。”说着,便撂下众人,
一直往潇湘馆来。正走着,只见文官等十二个女孩子也来了,上来问了好,说了一
回闲话儿,才走开。宝钗回身指道:“他们都在那里呢,你们找他们去,我找林姑
娘去就来。”说着,逶迤往潇湘馆来。忽然抬头见宝玉进去了,宝钗便站住,低头
想了一想:“宝玉和黛玉是从小儿一处长大的,他兄妹间多有不避嫌疑之处,嘲笑
不忌,喜怒无常;况且黛玉素多猜忌,好弄小性儿,此刻自己也跟进去,一则宝玉
不便,二则黛玉嫌疑,倒是回来的妙。”

想毕,抽身回来,刚要寻别的姊妹去。忽见面前一双玉色蝴蝶,大如团扇,一
上一下,迎风翩跹,十分有趣。宝钗意欲扑了来玩耍,遂向袖中取出扇子来,向草
地下来扑。只见那一双蝴蝶忽起忽落,来来往往,将欲过河去了。引的宝钗蹑手蹑
脚的,一直跟到池边滴翠亭上,香汗淋漓,娇喘细细。宝钗也无心扑了,刚欲回来,
只听那亭里边嘁嘁喳喳有人说话。原来这亭子四面俱是游廊曲栏,盖在池中水上,
四面雕镂子,糊着纸。宝钗在亭外听见说话,便煞住脚往里细听。只听说道:“你
瞧这绢子果然是你丢的那一块,你就拿着;要不是,就还芸二爷去。”又有一个说:
“可不是我那块!拿来给我罢。”又听道:“你拿什么谢我呢?难道白找了来不成?”
又答道:“我已经许了谢你,自然是不哄你的。”又听说道:“我找了来给你,自
然谢我;但只是那拣的人,你就不谢他么?”那一个又说道:“你别胡说。他是个
爷们家,拣了我们的东西,自然该还的。叫我拿什么谢他呢?”又听说道:“你不
谢他,我怎么回他呢?况且他再三再四的和我说了,若没谢的,不许我给你呢。”
半晌,又听说道:“也罢,拿我这个给他,算谢他的罢。你要告诉别人呢?须得起
个誓。”又听说道:“我要告诉人,嘴上就长一个疔,日后不得好死!”又听说道:
“嗳哟!咱们只顾说,看仔细有人来悄悄的在外头听见。不如把这子都推开了,
就是人见咱们在这里,他们只当我们说玩话儿呢。走到跟前,咱们也看的见,就别
说了。”

宝钗外面听见这话,心中吃惊,想道:“怪道从古至今那些奸淫狗盗的人,心
机都不错,这一开了,见我在这里,他们岂不臊了?况且说话的语音,大似宝玉房
里的小红。他素昔眼空心大,是个头等刁钻古怪的丫头,今儿我听了他的短儿,‘人
急造反,狗急跳墙’,不但生事,而且我还没趣。如今便赶着躲了料也躲不及,少
不得要使个‘金蝉脱壳’的法子。”犹未想完,只听“咯吱”一声,宝钗便故意放
重了脚步,笑着叫道:“颦儿,我看你往那里藏!”一面说一面故意往前赶。那亭
内的小红坠儿刚一推窗,只听宝钗如此说着往前赶,两个人都唬怔了。宝钗反向他
二人笑道:“你们把林姑娘藏在那里了?”坠儿道:“何曾见林姑娘了?”宝钗道:
“我才在河那边看着林姑娘在这里蹲着弄水儿呢。我要悄悄的唬他一跳,还没有走
到跟前,他倒看见我了,朝东一绕,就不见了。别是藏在里头了?”一面说,一面
故意进去,寻了一寻,抽身就走,口内说道:“一定又钻在山子洞里去了。遇见蛇,
咬一口也罢了!”一面说,一面走,心中又好笑:“这件事算遮过去了。不知他二
人怎么样?”

谁知小红听了宝钗的话,便信以为真,让宝钗去远,便拉坠儿道:“了不得了!
林姑娘蹲在这里,一定听了话去了!”坠儿听了,也半日不言语。小红又道:“这
可怎么样呢?”坠儿道:“听见了,管谁筋疼!各人干各人的就完了。”小红道:
“要是宝姑娘听见还罢了。那林姑娘嘴里又爱克薄人,心里又细,他一听见了,倘
或走露了,怎么样呢?”二人正说着,只见香菱、臻儿、司棋、侍书等上亭子来了。
二人只得掩住这话,且和他们玩笑。只见凤姐儿站在山坡上招手儿,小红便连忙弃
了众人,跑至凤姐前,堆着笑问:“奶奶使唤做什么事?”凤姐打量了一回,见他
生的干净俏丽,说话知趣,因笑道:“我的丫头们今儿没跟进我来。我这会子想起
一件事来,要使唤个人出去,不知你能干不能干?说的齐全不齐全?”小红笑道:
“奶奶有什么话,只管吩咐我说去;要说的不齐全,误了奶奶的事,任凭奶奶责罚
就是了。”凤姐笑道:“你是那位姑娘屋里的?我使你出去,他回来找你,我好替
你说。”小红道:“我是宝二爷屋里的。”凤姐听了笑道:“嗳哟!你原来是宝玉
屋里的,怪道呢。也罢了,等他问,我替你说。你到我们家告诉你平姐姐,外头屋
里桌子上汝窑盘子架儿底下放着一卷银子,那是一百二十两,给绣匠的工价。等张
材家的来,当面秤给他瞧了,再给他拿去。还有一件事:里头床头儿上有个小荷包
儿,拿了来。”小红听说,答应着,撤身去了。

不多时回来,不见凤姐在山坡上了,因见司棋从山洞里出来,站着系带子,便
赶来问道:“姐姐,不知道二奶奶往那里去了?”司棋道:“没理论。”小红听了,
回身又往四下里一看,只见那边探春宝钗在池边看鱼,小红上来陪笑道:“姑娘们
可知道二奶奶刚才那里去了?”探春道:“往你大奶奶院里找去。”小红听了,再
往稻香村来,顶头见晴雯、绮霞、碧痕、秋纹、麝月、侍书、入画、莺儿等一群人
来了。晴雯一见小红,便说道:“你只是疯罢!院子里花儿也不浇,雀儿也不喂,
茶炉子也不弄,就在外头逛!”小红道:“昨儿二爷说了,今儿不用浇花儿,过一
日浇一回。我喂雀儿的时候儿,你还睡觉呢。”碧痕道:“茶炉子呢?”小红道:
“今儿不该我的班儿,有茶没茶,别问我。”绮霞道:“你听听他的嘴!你们别说
了,让他逛罢。”小红道:“你们再问问,我逛了没逛。二奶奶才使唤我说话取东
西去。”说着,将荷包举给他们看,方没言语了,大家走开。晴雯冷笑道:“怪道
呢!原来爬上高枝儿去了,就不服我们说了。不知说了一句话半句话,名儿姓儿知
道了没有,就把他兴头的这个样儿。这一遭半遭儿的也算不得什么:过了后儿,还
得听呵。有本事从今儿出了这园子,长长远远的在高枝儿上才算好的呢!”一面说
着去了。

这里小红听了,不便分证,只得忍气来找凤姐。到了李氏房中,果见凤姐在这
里和李氏说话儿呢。小红上来回道:“平姐姐说:奶奶刚出来了,他就把银子收起
来了;才张材家的来取,当面秤了给他拿了去了。”说着,将荷包递上去。又道:
“平姐姐叫我来回奶奶:才旺儿进来讨奶奶的示下,好往那家子去,平姐姐就把那
话按着奶奶的主意打发他去了。”凤姐笑道:“他怎么按着我的主意打发去了呢?”
小红道:“平姐姐说:‘我们奶奶问这里奶奶好。我们二爷没在家。虽然迟了两天,
只管请奶奶放心。等五奶奶好些,我们奶奶还会了五奶奶来瞧奶奶呢。五奶奶前儿
打发了人来说:舅奶奶带了信来了,问奶奶好,还要和这里的姑奶奶寻几丸延年神
验万金丹;若有了,奶奶打发人来,只管送在我们奶奶这里。明儿有人去,就顺路
给那边舅奶奶带了去。’”小红还未说完,李氏笑道:“嗳哟!这话我就不懂了,
什么‘奶奶’‘爷爷’的一大堆。”凤姐笑道:“怨不得你不懂,这是四五门子的
话呢。”说着,又向小红笑道:“好孩子,难为你说的齐全,不像他们扭扭捏捏蚊
子似的。嫂子不知道,如今除了我随手使的这几个丫头老婆之外,我就怕和别人说
话:他们必定把一句话拉长了,作两三截儿,咬文嚼字,拿着腔儿,哼哼唧唧的。
急的我冒火,他们那里知道?我们平儿先也是这么着,我就问着他:难道必定装蚊
子哼哼就算美人儿了?说了几遭儿才好些儿了。”李纨笑道:“都像你泼辣货才好。”
凤姐道:“这个丫头就好。刚才这两遭说话虽不多,口角儿就很剪断。”说着,又
向小红笑道:“明儿你伏侍我罢,我认你做干女孩儿。我一调理,你就出息了。”

小红听了,“扑哧”一笑。凤姐道:“你怎么笑?你说我年轻,比你能大几岁,
就做你的妈了?你做春梦呢!你打听打听,这些人比你大的赶着我叫妈,我还不理
呢,今儿抬举了你了。”小红笑道:“我不是笑这个,我笑奶奶认错了辈数儿了。
我妈是奶奶的干女孩儿,这会子又认我做干女孩儿!”凤姐道:“谁是你妈?”李
纨笑道:“你原来不认的他?他是林之孝的女孩儿。”凤姐听了,十分诧异,因说
道:“哦,是他的丫头啊。”又笑道:“林子孝两口子,都是锥子扎不出一声儿来
的。我成日家说,他们倒是配就了的一对儿:一个‘天聋’,一个‘地哑’。那里
承望养出这么个伶俐丫头来!你十几了?”小红道:“十七岁了。”又问名字。小
红道:“原叫‘红玉’,因为重了宝二爷,如今只叫小红了。”凤姐听说,将眉一
皱,把头一回,说道:“讨人嫌的很!得了‘玉’的便宜似的,你也‘玉’我也‘玉’。”
因说:“嫂子不知道,我和他妈说:‘赖大家的如今事多,也不知这府里谁是谁,
你替我好好儿的挑两个丫头我使。’他只管答应着;他饶不挑,倒把他的女孩儿送
给别处去。难道跟我必定不好?”李纨笑道:“你可是又多心了。进来在先,你说
在后,怎么怨的他妈?”凤姐也笑道:“既这么着,明儿我和宝玉说,叫他再要人,
叫这丫头跟我去。可不知本人愿意不愿意?”小红笑道:“愿意不愿意,我们也不
敢说。只是跟着奶奶,我们学些眉眼高低,出入上下,大小的事儿,也得见识见识。”
刚说着,只见王夫人的丫头来请,凤姐便辞了李纨去了。小红自回怡红院去,不在
话下。

如今且说黛玉因夜间失寝,次日起来迟了,闻得众姐妹都在园中做饯花会,恐
人笑他痴懒,连忙梳洗了出来。刚到了院中,只见宝玉进门,来了便笑道:“好妹
妹,你昨儿告了我了没有?叫我悬了一夜的心。”黛玉便回头叫紫鹃:“把屋子收
拾了,下一扇纱屉子,看那大燕子回来,把帘子放下来,拿狮子倚住。烧了香,就
把炉罩上。”一面说,一面又往外走。宝玉见他这样,还认作是昨日晌午的事,那
知晚间的这件公案?还打恭作揖的。黛玉正眼儿也不看,各自出了院门,一直找别
的姐妹去了。宝玉心中纳闷,自己猜疑:“看起这样光景来,不像是为昨儿的事。
但只昨日我回来的晚了,又没有见他,再没有冲撞他的去处儿了。”一面想,一面
由不得随后跟了来。

只见宝钗探春正在那边看鹤舞,见黛玉来了,三个一同站着说话儿。又见宝玉
来了,探春便笑道:“宝哥哥身上好?我整整的三天没见你了。”宝玉笑道:“妹
妹身上好?我前儿还在大嫂子跟前问你呢。”探春道:“宝哥哥,你往这里来,我
和你说话。”宝玉听说,便跟了他,离了钗玉两个,到了一棵石榴树下。探春因说
道:“这几天,老爷没叫你吗?”宝玉笑道:“没有叫。”探春道:“昨儿我恍惚
听见说,老爷叫你出去来着。”宝玉笑道:“那想是别人听错了,并没叫我。”探
春又笑道:“这几个月,我又攒下有十来吊钱了。你还拿了去,明儿出门逛去的时
候,或是好字画,好轻巧玩意儿,替我带些来。”宝玉道:“我这么逛去,城里城
外大廊大庙的逛,也没见个新奇精致东西,总不过是那些金、玉、铜、磁器,没处
撂的古董儿,再么就是绸缎、吃食、衣服了。”探春道:“谁要那些作什么!像你
上回买的那柳枝儿编的小篮子儿,竹子根儿挖的香盒儿,胶泥垛的风炉子儿,就好
了,我喜欢的了不的。谁知他们都爱上了,都当宝贝儿似的抢了去了。”宝玉笑道:
“原来要这个。这不值什么,拿几吊钱出去给小子们,管拉两车来。”探春道:“小
厮们知道什么?你拣那有意思儿又不俗气的东西,你多替我带几件来,我还像上回
的鞋做一双你穿,比那双还加工夫,如何呢?”

宝玉笑道:“你提起鞋来,我想起故事来了:一回穿着,可巧遇见了老爷,老
爷就不受用,问:‘是谁做的?’我那里敢提三妹妹,我就回说是前儿我的生日舅
母给的。老爷听了是舅母给的,才不好说什么了。半日还说:‘何苦来!虚耗人力,
作践绫罗,做这样的东西。’我回来告诉了袭人,袭人说:‘这还罢了,赵姨娘气
的抱怨的了不得:正经亲兄弟,鞋塌拉袜塌拉的没人看见,且做这些东西!’”探
春听说,登时沉下脸来,道:“你说,这话糊涂到什么田地!怎么我是该做鞋的人
么?环儿难道没有分例的?衣裳是衣裳,鞋袜是鞋袜,丫头老婆一屋子,怎么抱怨这
些话?给谁听呢!我不过闲着没事作一双半双,爱给那个哥哥兄弟,随我的心,谁敢
管我不成?这也是他瞎气。”宝玉听了,点头笑道:“你不知道,他心里自然又有
个想头了。”

探春听说,一发动了气,将头一扭,说道:“连你也糊涂了!他那想头,自然
是有的。不过是那阴微下贱的见识。他只管这么想,我只管认得老爷太太两个人,
别人我一概不管。就是姐妹弟兄跟前,谁和我好,我就和谁好;什么偏的庶的,我
也不知道。论理我不该说他,但他忒昏愦的不像了!还有笑话儿呢:就是上回我给
你那钱,替我买那些玩的东西,过了两天,他见了我,就说是怎么没钱,怎么难过。
我也不理。谁知后来丫头们出去了,他就抱怨起我来,说我攒的钱为什么给你使,
倒不给环儿使呢!我听见这话,又好笑又好气。我就出来往太太跟前去了。”正说
着,只见宝钗那边笑道:“说完了?来罢。显见的是哥哥妹妹了,撂下别人,且说
体己去。我们听一句儿就使不得了?”说着,探春宝玉二人方笑着来了。

宝玉因不见了黛玉,便知是他躲了别处去了。想了一想:“索性迟两日,等他
的气息一息再去也罢了。”因低头看见许多凤仙石榴等各色落花,锦重重的落了一
地,因叹道:“这是他心里生了气,也不收拾这花儿来了。等我送了去,明儿再问
着他。”说着,只见宝钗约着他们往后头去。宝玉道:“我就来。”等他二人去远,
把那花儿兜起来,登山渡水,过树穿花,一直奔了那日和黛玉葬桃花的去处。

将已到了花冢,犹未转过山坡,只听那边有呜咽之声,一面数落着,哭的好不
伤心。宝玉心下想道:“这不知是那屋里的丫头,受了委屈,跑到这个地方来哭?”
一面想,一面煞住脚步,听他哭道是:
花谢花飞飞满天,红消香断有谁怜?
游丝软系飘春榭,落絮轻沾扑绣帘。
闺中女儿惜春暮,愁绪满怀无着处。
手把花锄出绣帘,忍踏落花来复去?
柳丝榆荚自芳菲,不管桃飘与李飞。
桃李明年能再发,明年闺中知有谁?
三月香巢初垒成,梁间燕子太无情!
明年花发虽可啄,却不道人去梁空巢已倾。
一年三百六十日,风刀霜剑严相逼。
明媚鲜妍能几时,一朝飘泊难寻觅。
花开易见落难寻,阶前愁杀葬花人。
独把花锄偷洒泪,洒上空枝见血痕。
杜鹃无语正黄昏,荷锄归去掩重门。
青灯照壁人初睡,冷雨敲窗被未温。
怪侬底事倍伤神?半为怜春半恼春:
怜春忽至恼忽去,至又无言去不闻。
昨宵庭外悲歌发,知是花魂与鸟魂?
花魂鸟魂总难留,鸟自无言花自羞。
愿侬此日生双翼,随花飞到天尽头。
天尽头,何处有香丘?
未若锦囊收艳骨,一净土掩风流。
质本洁来还洁去,不教污淖陷渠沟。
尔今死去侬收葬,未卜侬身何日丧?
侬今葬花人笑痴,他年葬侬知是谁?
试看春残花渐落,便是红颜老死时。
一朝春尽红颜老,花落人亡两不知!

正是一面低吟,一面哽咽。那边哭的自己伤心,却不道这边听的早已疾倒了。

要知端详,下回分解。