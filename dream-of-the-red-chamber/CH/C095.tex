\chapter{因讹成实元妃薨逝~以假混真宝玉疯癫}

话说焙茗在门口和小丫头子说宝玉的玉有了,那小丫头急忙回来告诉宝玉。众
人听了,都推着宝玉出去问他。众人在廊下听着。宝玉也觉放心,便走到门口,问
道:“你那里得了?快拿来。”焙茗道:“拿是拿不来的,还是托人做保去呢。”
宝玉道:“你快说是怎么得的,我好叫人取去。”焙茗道:“我在外头,知道林爷
爷去测字,我就跟了去。我听见说在当铺里找,我没等他说完,便跑到几个当铺里
去。我比给他们瞧,有一家便说‘有’。我说:‘给我罢。’那铺子里要票子。我
说:‘当多少钱?’他说:‘三百钱的也有,五百钱的也有。前儿有一个人拿这么
一块玉,当了三百钱去;今儿又有人也拿一块玉当了五百钱去。’”宝玉不等说完,
便道:“你快拿三百五百钱去取了来,我们挑着看是不是。”里头袭人便啐道:“二
爷不用理他。我小时候儿听见我哥哥常说,有些人卖那些小玉儿,没钱用便去当,
想来是家家当铺里有的。”众人正在听得诧异,被袭人一说,想了一想,倒大家笑
起来,说:“快叫二爷进来罢,不用理那糊涂东西了。他说的那些玉,想来不是正
经东西。”

宝玉正笑着,只见岫烟来了。原来岫烟走到栊翠庵,见了妙玉,不及闲话,便
求妙玉扶乩。妙玉冷笑几声,说道:“我与姑娘来往,为的是姑娘不是势利场中的
人。今日怎么听了那里的谣言,过来缠我?况且我并不晓得什么叫‘扶乩’。”说
着,将要不理。岫烟懊悔此来,知他脾气是这么着的,“一时我已说出,不好白回
去”。又不好与他质证他会扶乩的话,只得陪着笑将袭人等性命关系的话说了一遍。
见妙玉略有活动,便起身拜了几拜。妙玉叹道:“何必为人作嫁?但是我进京以来,
素无人知,今日你来破例,恐将来缠绕不休。”岫烟道:“我也一时不忍。知你必
是慈悲的。便是将来他人求你,愿不愿在你,谁敢相强?”妙玉笑了一笑,叫道婆
焚香。在箱子里找出沙盘乩架,书了符,命岫烟行礼祝告毕,起来同妙玉扶着乩。
不多时,只见那仙乩疾书道:

噫!来无迹,去无踪,青埂峰下倚古松。欲追寻,山万重,入我门来一笑逢。
书毕,停了乩,岫烟便问:“请的是何仙?”妙玉道:“请的是拐仙。”岫烟录了
出来,请教妙玉识。妙玉道:“这个可不能,连我也不懂。你快拿去,他们的聪明
人多着哩。”岫烟只得回来。

进入院中,各人都问:“怎么样了?”岫烟不及细说,便将所录乩语递与李纨。
众姊妹及宝玉争看,都解的是:“一时要找是找不着的,然而丢是丢不了的,不知
几时不找便出来了。但是青埂峰不知在那里?”李纨道:“这是仙机隐语。咱们家
里那里跑出青埂峰来?必是谁怕查出,撂在有松树的山子石底下,也未可定。独是
‘入我门来’这句,到底是入谁的门呢?”黛玉道:“不知请的是谁?”岫烟道:
“拐仙。”探春道:“若是仙家的门,便难入了。”袭人心里着忙,便捕风捉影的
混找,没一块石底下不找到,只是没有。回到院中,宝玉也不问有无,只管傻笑。
麝月着急道:“小祖宗!你到底是那里丢的?说明了,我们就是受罪,也在明处啊。”
宝玉笑道:“我说外头丢的,你们又不依。你如今问我,我知道么?”李纨探春道:
“今儿从早起闹起,已到三更来的天了。你瞧林妹妹已经掌不住,各自去了。我们
也该歇歇儿了,明儿再闹罢。”说着,大家散去。宝玉即便睡下。可怜袭人等哭一
回,想一回,一夜无眠,暂且不提。

且说黛玉先自回去,想起“金”“石”的旧话来,反自欢喜,心里也道:“和
尚道士的话真个信不得。果真‘金’‘玉’有缘,宝玉如何能把这玉丢了呢?或者
因我之事,拆散他们的‘金玉’,也未可知。”想了半天,更觉安心,把这一天的
劳乏竟不理会,重新倒看起书来。紫鹃倒觉身倦,连催黛玉睡下。黛玉虽躺下,又
想到海棠花上,说:“这块玉原是胎里带来的,非比寻常之物,来去自有关系。若
是这花主好事呢,不该失了这玉呀。看来此花开的不祥,莫非他有不吉之事?”不
觉又伤起心来。又转想到喜事上头,此花又似应开,此玉又似应失:如此一悲一喜,
直想到五更方睡着。

次日,王夫人等早派人到当铺里去查问,凤姐暗中设法找寻。一连闹了几天,
总无下落。还喜贾母贾政未知。袭人等每日提心吊胆。宝玉也好几天不上学,只是
怔怔的,不言不语,没心没绪的。王夫人只知他因失玉而起,也不大着意。那日正
在纳闷,忽见贾琏进来请安,嘻嘻的笑道:“今日听得雨村打发人来告诉咱们二老
爷,说舅太爷升了内阁大学士,奉旨来京,已定于明年正月二十日宣麻,有三百里
的文书去了。想舅太爷昼夜趱行,半个多月就要到了。侄儿特来回太太知道。”王
夫人听说,便欢喜非常。正想娘家人少,薛姨妈家又衰败了,兄弟又在外任照应不
着,今日忽听兄弟拜相回京,王家荣耀,将来宝玉都有倚靠,便把失玉的心又略放
开些了,天天专望兄弟来京。

忽一天,贾政进来,满脸泪痕,喘吁吁的说道:“你快去禀知老太太,即刻进
宫!不用多人的,是你伏侍进去。因娘娘忽得暴病,现在太监在外立等。他说:‘太
医院已经奏明痰厥,不能医治。’”王夫人听说,便大哭起来。贾政道:“这不是
哭的时候,快快去请老太太。说得宽缓些,不要吓坏了老人家。”贾政说着,出来
吩咐家人伺候。王夫人收了泪,去请贾母,只说元妃有病,进去请安。贾母念佛道:
“怎么又病了?前番吓的我了不得,后来又打听错了。这回情愿再错了也罢。”王
夫人一面回答,一面催鸳鸯等开箱取衣饰穿戴起来。王夫人赶着回到自己房中,也
穿戴好了,过来伺候。一时出厅,上轿进宫不提。

且说元春自选了凤藻宫后,圣眷隆重,身体发福,未免举动费力。每日起居劳
乏,时发痰疾。因前日侍宴回宫,偶沾寒气,勾起旧病。不料此回甚属利害,竟至
痰气壅塞,四肢厥冷。一面奏明,即召太医调治。岂知汤药不进,连用通关之剂,
并不见效。内官忧虑,奏请预办后事,所以传旨命贾氏椒房进见。贾母王夫人遵旨
进宫,见元妃痰塞口涎,不能言语。见了贾母,只有悲泣之状,却没眼泪。贾母进
前请安,奏些宽慰的话。少时贾政等职名递进,宫嫔传奏,元妃目不能顾,渐渐脸
色改变。内官太监即要奏闻,恐派各妃看视,椒房姻戚未便久羁,请在外宫伺候。
贾母王夫人怎忍便离,无奈国家制度,只得下来,又不敢啼哭,惟有心内悲感。

朝门内官员有信。不多时,只见太监出来,立传钦天监。贾母便知不好,尚未
敢动。稍刻,小太监传谕出来,说:“贾娘娘薨逝。”是年甲寅年十二月十八日立
春,元妃薨日,是十二月十九日,已交卯年寅月,存年四十三岁。贾母含悲起身,
只得出宫上轿回家。贾政等亦已得信,一路悲戚。到家中,邢夫人、李纨、凤姐、
宝玉等出厅,分东西迎着贾母,请了安,并贾政王夫人请安,大家哭泣不提。

次日早起,凡有品级的,按贵妃丧礼进内请安哭临。贾政又是工部,虽按照仪
注办理,未免堂上又要周旋他些,同事又要请教他,所以两头更忙,非比从前太后
与周妃的丧事了。但元妃并无所出,惟谥曰贤淑贵妃。此是王家制度,不必多赘。
只讲贾府中男女,天天进宫,忙的了不得。幸喜凤姐儿近日身子好些,还得出来照
应家事,又要预备王子腾进京,接风贺喜。凤姐胞兄王仁,知道叔叔入了内阁,仍
带家眷来京。凤姐心里喜欢,便有些心病,有这些娘家的人也便撂开,所以身子倒
觉比先好了些。王夫人看见凤姐照旧办事,又把担子卸了一半,又眼见兄弟来京,
诸事放心,倒觉安静些。

独有宝玉原是无职之人,又不念书,代儒学里知他家里有事,也不来管他;贾
政正忙,自然没有空儿查他。想来宝玉趁此机会,竟可与姊妹们天天畅乐;不料他
自失了玉后,终日懒怠走动,说话也糊涂了。并贾母等出门回来,有人叫他去请安,
便去;没人叫他,他也不动。袭人等怀着鬼胎,又不敢去招惹他,恐他生气。每天
茶饭,端到面前便吃,不来也不要。袭人看这光景,不像是有气,竟像是有病的。
袭人偷着空儿到潇湘馆告诉紫鹃,说是:“二爷这么着,求姑娘给他开导开导。”
紫鹃虽即告诉黛玉,只因黛玉想着亲事上头,一定是自己了,如今见了他,反觉不
好意思:“若是他来呢,原是小时在一处的,也难不理他;若说我去找他,断断使
不得。”所以黛玉不肯过来。袭人又背地里去告诉探春。那知探春心里明明知道海
棠开得怪异,“宝玉”失的更奇,接连着元妃姐姐薨逝,谅家道不祥,日日愁闷,
那有心肠去劝宝玉?况兄妹们男女有别,只好过来一两次,宝玉又终是懒懒的,所
以也不大常来。

宝钗也知失玉。因薛姨妈那日应了宝玉的亲事,回去便告诉了宝钗。薛姨妈还
说:“虽是你姨妈说了,我还没有应准,说等你哥哥回来再定。你愿意不愿意?”
宝钗反正色的对母亲道:“妈妈这话说错了。女孩儿家的事情是父母作主的,如今
我父亲没了,妈妈应该作主的,再不然问哥哥。怎么问起我来?”所以薛姨妈更爱
惜他,说他虽是从小娇养惯的,却也生来的贞静,因此在他面前反不提起宝玉了。
宝钗自从听此一说,把“宝玉”两字自然更不提起了。如今虽然听见失了玉,心里
也甚惊疑,倒不好问,只得听旁人说去,竟像不与自己相干的。只有薛姨妈打发丫
头过来了好几次问信。因他自己的儿子薛蟠的事焦心,只等哥哥进京,便好为他出
脱罪名;又知元妃已薨,虽然贾府忙乱,却得凤姐好了,出来理家,所以也不大过
这边来。这里只苦了袭人,在宝玉跟前低声下气的伏侍劝慰,宝玉竟是不懂。袭人
只有暗暗的着急而已。

过了几日,元妃停灵寝庙,贾母等送殡去了几天。岂知宝玉一日呆似一日,也
不发烧,也不疼痛,只是吃不像吃,睡不像睡,甚至说话都无头绪。那袭人麝月等
一发慌了,回过凤姐几次。凤姐不时过来。起先道是找不着玉生气,如今看他失魂
落魄的样子,只有日日请医调治。煎药吃了好几剂,只有添病的,没有减病的。及
至问他那里不舒服,宝玉也不说出来。直至元妃事毕,贾母惦记宝玉,亲自到园看
视,王夫人也随过来。袭人等叫宝玉接出去请安。宝玉虽说是病,每日原起来行动,
今日叫他接贾母去,他依然仍是请安,惟是袭人在旁扶着指教。贾母见了,便道:
“我的儿,我打量你怎么病着,故此过来瞧你。今你依旧的模样儿,我的心放了好
些。”王夫人也自然是宽心的。但宝玉并不回答,只管嘻嘻的笑。贾母等进屋坐下,
问他的话,袭人教一句,他说一句,大不似往常,直是一个傻子似的。贾母愈看愈
疑,便说:“我才进来看时,不见有什么病;如今细细一瞧,这病果然不轻,竟是
神魂失散的样子。到底因什么起的呢?”王夫人知事难瞒,又瞧瞧袭人怪可怜的样
子,只得便依着宝玉先前的话,将那往临安伯府里去听戏时丢了这块玉的话悄悄的
告诉了一遍,心里也徨的很,生恐贾母着急。并说:“现在着人在四下里找寻。
求签问卦,都说在当铺里找,少不得找着的。”贾母听了,急得站起来,眼泪直流,
说道:“这件玉如何是丢得的!你们忒不懂事了!难道老爷也是撂开手的不成?”王
夫人知贾母生气,叫袭人等跪下,自己敛容低首回说:“媳妇恐老太太着急,老爷
生气,都没敢回。”贾母咳道:“这是宝玉的命根子,因丢了,所以他这么失魂丧
魄的。还了得!这玉是满城里都知道的,谁检了去,肯叫你们找出来么?叫人快快请
老爷,我与他说。”那时吓得王夫人袭人等俱哀告道:“老太太这一生气,回来老
爷更了不得了。现在宝玉病着,交给我们尽命的找来就是了。”贾母道:“你们怕
老爷生气,有我呢。”便叫麝月传人去请。

不一时传话进来,说:“老爷谢客去了。”贾母道:“不用他也使得。你们便
说我说的话,暂且也不用责罚下人。我便叫琏儿来,写出赏格,悬在前日经过的地
方,便说:‘有人检得送来者,情愿送银一万两;如有知人检得,送信找得者,送
银五千两。’如真有了,不可吝惜银子。这么一找,少不得就找出来了。若是靠着
咱们家几个人找,就找一辈子也不能得!”王夫人也不敢直言。贾母传话告诉贾琏,
叫他速办去了。贾母便叫人:“将宝玉动用之物,都搬到我那里去。只派袭人秋纹
跟过来,馀者仍留园内看屋子。”宝玉听了,总不言语,只是傻笑。贾母便携了宝
玉起身,袭人等搀扶出园。

回到自己房中,叫王夫人坐下,看人收拾里间屋内安置,便对王夫人道:“你
知道我的意思么?我为的是园里人少,怡红院的花树忽萎忽开,有些奇怪。头里仗
着那块玉能除邪祟,如今玉丢了,只怕邪气易侵,所以我带过他来一块儿住着。这
几天也不用叫他出去。大夫来,就在这里瞧。”王夫人听说,便接口道:“老太太
想的自然是。如今宝玉同着老太太住了,老太太的福气大,不论什么都压住了。”
贾母道:“什么福气!不过我屋里干净些,经卷也多,都可以念念,定定心神。你
问宝玉好不好?”那宝玉见问只是笑。袭人叫他说好,宝玉也就说好。王夫人见了
这般光景,未免落泪,在贾母这里,不敢出声。贾母知王夫人着急,便说道:“你
回去罢,这里有我调停他。晚上老爷回来,告诉他不必来见我,不许言语就是了。”
王夫人去后,贾母叫鸳鸯找些安神定魄的药,按方吃了,不提。

且说贾政当晚回家,在车内听见道儿上人说道:“人要发财,也容易的很。”
那个问道:“怎么见得?”这个人又道:“今日听见荣府里丢了什么哥儿的玉了,
贴着招帖儿,上头写着玉的大小式样颜色,说有人检了送去,就给一万两银子。送
信的还给五千呢。”贾政虽未听得如此真切,心里诧异,急忙赶回,便叫门上的人,
问起那事来。门上的人禀道:“奴才头里也不知道,今儿晌午琏二爷传出老太太的
话,叫人去贴帖儿,才知道的。”贾政便叹气道:“家道该衰!偏生养这么一个孽
障!才养他的时候,满街的谣言,隔了十几年略好了些。这会子又大张晓谕的找玉,
成何道理!”说着,忙走进里头去问王夫人。王夫人便一五一十的告诉。贾政知是
老太太的主意,又不敢违拗,只抱怨王夫人几句。又走出来,叫瞒着老太太,背地
里揭了这个帖儿下来。岂知早有那些游手好闲的人揭了去了。

过了些时,竟有人到荣府门上,口称送玉来的。家人们听见,喜欢的了不得,
便说:“拿来,我给你回去。”那人便怀内掏出赏格来,指给门上的人瞧,说:“这
不是你们府上的帖子?写明送玉的给银一万两。二太爷,你们这会子瞧我穷,回来
我得了银子,就是财主了,别这么待理不理的。”门上人听他的话头儿硬,便说道:
“你到底略给我瞧瞧,我好给你回。”那人初倒不肯,后来听人说得有理,便掏出
那玉,托在掌中一扬,说:“这是不是?”众家人原是在外服役,只知有玉,也不
常见,今日才看见这玉的模样儿了,急忙跑到里头抢头报的似的。那日贾政贾赦出
门,只有贾琏在家。众人回明,贾琏还问:“真不真?”门上人口称:“亲眼见过,
只是不给奴才,要见主子,一手交银,一手交玉。”贾琏却也喜欢,忙去禀知王夫
人,即便回明贾母,把个袭人乐的合掌念佛。贾母并不改口,一叠连声:“快叫琏
儿请那人到书房里坐着,将玉取来一看,即便给银。”贾琏依言,请那人进来,当
客待他,用好言道谢:“要借这玉送到里头本人见了,谢银分厘不短。”那人只得
将一个红绸子包儿送过去。贾琏打开一看,可不是那一块晶莹美玉吗?贾琏素昔原
不理论,今日倒要看看。看了半日,上面的字也仿佛认得出来,什么“除邪祟”等
字。贾琏看了,喜之不胜,便叫家人伺候,忙忙的送与贾母王夫人认去。

这会子惊动了合家的人,都等着争看。凤姐见贾琏进来,便劈手夺去,不敢先
看,送到贾母手里,贾琏笑道:“你这么一点儿事,还不叫我献功呢。”贾母打开
看时,只见那玉比先前昏暗了好些,一面用手擦摸,鸳鸯拿上眼镜儿来,戴着一瞧,
说:“奇怪。这块玉倒是的,怎么把头里的宝色都没了呢?”王夫人看了一会子,
也认不出,便叫凤姐过来看。凤姐看了道:“像倒像,只是颜色不大对,不如叫宝
兄弟自己一看,就知道了。”袭人在旁,也看着未必是那一块,只是盼得的心盛,
也不敢说出不像来。凤姐于是从贾母手中接过来,同着袭人,拿来给宝玉瞧。这时
宝玉正睡着才醒。凤姐告诉道:“你的玉有了。”宝玉睡眼朦胧,接在手里也没瞧,
便往地下一撂,道:“你们又来哄我了。”说着只是冷笑。凤姐连忙拾起来道:“这
也就奇了,怎么你没瞧就知道呢?”宝玉也不答言,只管笑。王夫人也进屋里来了,
见他这样,便道:“这不用说了。他那玉原是胎里带来的一宗古怪东西,自然他有
道理。想来这个必是人家见了帖儿,照样儿做的。”大家此时恍然大悟。

贾琏在外间屋里听见这个话,便说道:“既不是,快拿来给我问问他去。人家
这样事,他还敢来鬼混!”贾母喝住道:“琏儿,拿了去给他,叫他去罢。那也是
穷极了的人,没法儿了,所以见我们家有这样事,他就想着赚几个钱,也是有的。
如今白白的花了钱弄了这个东西,又叫咱们认出来了。依着我倒别难为他,把这块
玉还他,说不是我们的,赏给他几两银子,外头的人知道了,才肯有信儿就送来呢。
要是难为了这一个人,就有真的人家也不敢拿了来了。”贾琏答应出去。那人还等
着呢,半日不见人来,正在那里心里发虚,只见贾琏气忿忿走出来了。

未知如何,下回分解。