\chapter{蘅芜君兰言解疑癖~潇湘子雅谑补馀音}

话说贾母王夫人去后,姐妹们复进园来吃饭。那刘老老带着板儿,先来见凤姐
儿说:“明日一早定要家去了。虽然住了两三天,日子却不多,把古往今来没见过
的、没吃过的、没听见的都经验过了。难得老太太和姑奶奶并那些小姐们,连各房
里的姑娘们,都这样怜贫惜老照看我。我这一回去没别的报答,惟有请些高香,天
天给你们念佛,保佑你们长命百岁的,就算我的心了。”凤姐儿笑道:“你别喜欢,
都是为你,老太太也叫风吹病了,躺着嚷不舒服;我们大姐儿也着了凉了,在那里
发热呢。”刘老老听了,忙叹道:“老太太有年纪了,不惯十分劳乏的。”凤姐儿道:
“从来不像昨儿高兴。往常也进园子逛去,不过到一两处坐坐就来了。昨儿因为你
在这里,要叫都逛逛,一个园子倒走了多半个。大姐儿因为我找你去,太太递了一
块糕给他,谁知风地里吃了,就发起热来。”刘老老道:“妞妞儿只怕不大进园子。
比不得我们的孩子,一会走,那个坟圈子里不跑去?一则风拍了也是有的,二则只
怕他身上干净,眼睛又净,或是遇见什么神了。依我说,给他瞧瞧祟书本子,仔细
撞客着。”一语提醒了凤姐儿,便叫平儿拿出《玉匣记》来,叫彩明来念。彩明翻
了一会子,念道:“八月二十五日病者,东南方得之,有缢死家亲女鬼作祟,又遇
花神。用五色纸钱四十张,向东南方四十步送之大吉。”凤姐儿笑道:“果然不错,
园子里头可不是花神!只怕老太太也是遇见了。”一面命人请两分纸钱来,着两个人
来,一个与贾母送祟,一个与大姐儿送祟,果见大姐儿安稳睡了。

凤姐儿笑道:“到底是你们有年纪的经历的多。我们大姐儿时常肯病,也不知
是什么原故。”刘老老道:“这也有的。富贵人家养的孩子都娇嫩,自然禁不得一些
儿委屈。再他小人儿家,过于尊贵了也禁不起。以后姑奶奶倒少疼他些就好了。”
凤姐儿道:“也是有的。我想起来,他还没个名字,你就给他起个名字,借借你的
寿;二则你们是庄家人,不怕你恼,到底贫苦些,你们贫苦人起个名字只怕压的住。”
刘老老听说,便想了一想,笑道:“不知他是几时养的?”凤姐儿道:“正是养的日
子不好呢:可巧是七月初七日。”刘老老忙笑道:“这个正好,就叫做巧姐儿好。这
个叫做‘以毒攻毒,以火攻火’的法子。姑奶奶定依我这名字,必然长命百岁。日
后大了,各人成家立业,或一时有不遂心的事,必然遇难成祥,逢凶化吉,都从这
‘巧’字儿来。”凤姐儿听了,自是欢喜,忙谢道:“只保佑他应了你的话就好了。”
说着,叫平儿来吩咐道:“明儿咱们有事,恐怕不得闲儿,你这会子闲着,把送老
老的东西打点了,他明儿一早就好走的便宜了。”刘老老道:“不敢多破费了。已经
遭扰了几天,又拿着走,越发心里不安了。”凤姐儿笑道:“也没有什么,不过随常
的东西。好也罢,歹也罢,带了去,你们街坊邻舍看着也热闹些,也是上城一趟。”
说着只见平儿走来说:“老老过这边瞧瞧。”刘老老忙跟了平儿到那边屋里,只见堆
着半炕东西。平儿一一的拿给他瞧着,又说道:“这是昨日你要的青纱一匹,奶奶
另外送你一个实地月白纱做里子。这是两个茧绸,做袄儿裙子都好。这包袱里是两
匹绸子,年下做件衣裳穿。这是一盒子各样内造小饽饽儿,也有你吃过的,也有没
吃过的,拿去摆碟子请人,比买的强些。这两条口袋是你昨日装果子的,如今这一
个里头装了两斗御田粳米,熬粥是难得的;这一条里头是园子里的果子和各样干果
子。这一包是八两银子。这都是我们奶奶的。这两包每包五十两,共是一百两,是
太太给的,叫你拿去,或者做个小本买卖,或者置几亩地,以后再别求亲靠友的。”
说着又悄悄笑道:“这两件袄儿和两条裙子,还有四块包头,一包绒线,可是我送
老老的。那衣裳虽是旧的,我也没大很穿,你要弃嫌,我就不敢说了。”

平儿说一样,刘老老就念一句佛,已经念了几千佛了;又见平儿也送他这些东
西,又如此谦逊,忙笑道:“姑娘说那里话?这样好东西,我还弃嫌!我就有银子,
没处买这样的去呢。只是我怪臊的,收了不好,不收又辜负了姑娘的心。”平儿笑
道:“别说外话,咱们都是自己,我才这么着。你放心收了罢,我还和你要东西呢。
到年下,你只把你们晒的那个灰条菜和豇豆、扁豆、茄子干子、葫芦条儿,各样干
菜带些来——我们这里上上下下都爱吃这个——就算了。别的一概不要,别罔费了
心。”刘老老千恩万谢的答应了。平儿道:“你只管睡你的去,我替你收拾妥当了,
就放在这里,明儿一早打发小厮们雇辆车装上,不用你费一点心儿。”刘老老越发
感激不尽,过来又千恩万谢的辞了凤姐儿,过贾母这边睡了一夜。次早梳洗了,就
要告辞。

因贾母欠安,众人都过来请安,出去传请大夫。一时婆子回:“大夫来了。”老
嬷嬷请贾母进幔子去坐,贾母道:“我也老了,那里养不出那阿物儿来,还怕他不
成!不用放幔子,就这样瞧罢。”众婆子听了,便拿过一张小桌子来,放下一个小枕
头,便命人请。一时只见贾珍、贾琏、贾蓉三个人,将王太医领来。王太医不敢走
甬路,只走旁阶,跟着贾珍到了台阶上。早有两个婆子在两边打起帘子,两个婆子
在前导引进去,又见宝玉迎接出来。见贾母穿着青绉绸一斗珠儿的羊皮褂子,端坐
在榻上。两边四个未留头的小丫鬟,都拿着蝇刷漱盂等物,又有五六个老嬷嬷雁翅
摆在两旁。碧纱厨后,隐隐约约有许多穿红着绿、戴宝插金的人,王太医也不敢抬
头,忙上来请了安。贾母见他穿着六品服色,便知是御医了,含笑问:“供奉好?”
因问贾珍:“这位供奉贵姓?”贾珍等忙回:“姓王。”贾母笑道:“当日太医院正堂
有个王君效,好脉息。”王太医忙躬身低头含笑,因说:“那是晚生家叔祖。”贾母
听了笑道:“原来这样,也算是世交了。”一面说,一面慢慢的伸手放在小枕头上。
嬷嬷端着一张小杌子放在小桌前面,略偏些。王太医便盘着一条腿儿坐下,歪着头
诊了半日,又诊了那只手,忙欠身低头退出。贾母笑说:“劳动了。珍哥让出去,
好生看茶。”贾珍、贾琏等忙答应了几个“是”,复领王太医到外书房中。王太医说:
“太夫人并无别症,偶感了些风寒,其实不用吃药,不过略清淡些,常暖着点儿,
就好了。如今写个方子在这里,若老人家爱吃,便按方煎一剂吃;若懒怠吃,也就
罢了。”说着,吃茶,写了方子。刚要告辞,只见奶子抱了大姐儿出来,笑说:“王
老爷也瞧瞧我们。”王太医听说,忙起身就奶子怀中,左手托着大姐儿的手,右手
诊了一诊,又摸了一摸头,又叫伸出舌头来瞧瞧,笑道:“我要说了,妞儿该骂我
了:只要清清净净的饿两顿就好了。不必吃煎药,我送点丸药来,临睡用姜汤研开
吃下去就好了。”说毕,告辞而去。贾珍等拿了药方来回明贾母原故,将药方放在
案上出去,不在话下。

这里王夫人和李纨、凤姐儿、宝钗姐妹等,见大夫出去,方从厨后出来。王夫
人略坐一坐,也回房去了。刘老老见无事,方上来和贾母告辞。贾母说:“闲了再
来。”又命鸳鸯来:“好生打发刘老老出去。我身上不好,不能送你。”刘老老道了
谢,又作辞,方同鸳鸯出来。到了下房,鸳鸯指炕上一个包袱说道:“这是老太太
的几件衣裳,都是往年间生日节下众人孝敬的。老太太从不穿人家做的,收着也可
惜,却是一次也没穿过的,昨日叫我拿出两套来送你带了去,或送人,或自己家里
穿罢。这盒子里头是你要的面果子。这包儿里头是你前儿说的药,梅花点舌丹也有,
紫金锭也有,活络丹也有,催生保命丹也有:每一样是一张方子包着,总包在里头
了。这是两个荷包,带着玩罢。”说着,又抽开系子,掏出两个“笔锭如意”的锞
子来给他瞧,又笑道:“荷包你拿去,这个留下给我罢。”刘老老已喜出望外,早又
念了几千佛,听鸳鸯如此说,便忙说道:“姑娘只管留下罢。”鸳鸯见他信以为真,
笑着仍给他装上,说道:“哄你玩呢!我有好些呢。留着年下给小孩子们罢。”说着,
只见一个小丫头拿着个成窑钟子来,递给刘老老,说:“这是宝二爷给你的。”刘老
老道:“这是那里说起?我那一世修来的,今儿这样!”说着便接过来。鸳鸯道:“前
儿我叫你洗澡,换的衣裳是我的,你不弃嫌,我还有几件也送你罢。”刘老老又忙
道谢。鸳鸯果然又拿出几件来,给他包好。刘老老又要到园中辞谢宝玉和众姊妹王
夫人等去,鸳鸯道:“不用去了。他们这会子也不见人,回来我替你说罢。闲了再
来。”又命了一个老婆子,吩咐他:“二门上叫两个小厮来,帮着老老拿了东西送去。”
婆子答应了。又和刘老老到了凤姐儿那边,一并拿了东西,在角门上命小厮们搬出
去,直送刘老老上车去了,不在话下。

且说宝钗等吃过早饭,又往贾母处问安,回园至分路之处,宝钗便叫黛玉道:
“颦儿跟我来!有一句话问你。”黛玉便笑着跟了来。至蘅芜院中,进了房,宝钗便
坐下,笑道:“你还不给我跪下!我要审你呢。”黛玉不解何故,因笑道:“你瞧宝丫
头疯了!审我什么?”宝钗冷笑道:“好个千金小姐!好个不出屋门的女孩儿!满嘴里
说的是什么?你只实说罢。”黛玉不解,只管发笑,心里也不免疑惑,口里只说:“我
何曾说什么?你不过要捏我的错儿罢咧。你倒说出来我听听。”宝钗笑道:“你还装
憨儿呢!昨儿行酒令儿,你说的是什么?我竟不知是那里来的。”黛玉一想,方想起
昨儿失于检点,那《牡丹亭》、《西厢记》说了两句,不觉红了脸,便上来搂着宝钗
笑道:“好姐姐!原是我不知道,随口说的。你教给我,再不说了。”宝钗笑道:“我
也不知道,听你说的怪好的,所以请教你。”黛玉道:“好姐姐!你别说给别人,我
再不说了!”宝钗见他羞的满脸飞红,满口央告,便不肯再往下问。因拉他坐下吃
茶,款款的告诉他道:“你当我是谁?我也是个淘气的,从小儿七八岁上,也够个人
缠的。我们家也算是个读书人家,祖父手里也极爱藏书。先时人口多,姐妹弟兄也
在一处,都怕看正经书。弟兄们也有爱诗的,也有爱词的,诸如这些《西厢》、《琵
琶》以及《元人百种》,无所不有。他们背着我们偷看,我们也背着他们偷看。后
来大人知道了,打的打,骂的骂,烧的烧,丢开了。所以咱们女孩儿家不认字的倒
好:男人们读书不明理,尚且不如不读书的好,何况你我?连做诗写字等事,这也
不是你我分内之事,究竟也不是男人分内之事。男人们读书明理,辅国治民,这才
是好。只是如今并听不见有这样的人,读了书,倒更坏了。这并不是书误了他,可
惜他把书遭塌了,所以竟不如耕种买卖,倒没有什么大害处。至于你我,只该做些
针线纺绩的事才是;偏又认得几个字。既认得了字,不过拣那正经书看也罢了,最
怕见些杂书,移了性情,就不可救了。”一席话,说的黛玉垂头吃茶,心下暗服,
只有答应“是”的一字。

忽见素云进来说:“我们奶奶请二位姑娘商议要紧的事呢。二姑娘、三姑娘、
四姑娘、史姑娘、宝二爷,都等着呢。”宝钗说:“又是什么事?”黛玉道:“咱们
到了那里就知道了。”说着,便和宝钗往稻香村来,果见众人都在那里。李纨见了
他两个,笑道:“社还没起,就有脱滑儿的了,四丫头要告一年的假呢。”黛玉笑道:
“都是老太太昨儿一句话,又叫他画什么园子图儿,惹的他乐得告假了。”探春笑
道:“也别怪老太太,都是刘老老一句话。”黛玉忙笑接道:“可是呢,都是他一句
话。他是那一门子的老老?直叫他是个‘母蝗虫’就是了。”说着,大家都笑起来。
宝钗笑道:“世上的话,到了二嫂子嘴里也就尽了,幸而二嫂子不认得字,不大通,
不过一概是市俗取笑儿。更有颦儿这促狭嘴,他用《春秋》的法子,把市俗粗话撮
其要,删其繁,再加润色,比方出来,一句是一句。这‘母蝗虫’三字,把昨儿那
些形景都画出来了。亏他想的倒也快!”众人听了,都笑道:“你这一注解,也就不
在他两个以下了。”

李纨道:“我请你们大家商议,给他多少日子的假?我给了他一个月的假,他嫌
少,你们怎么说?”黛玉道:“论理,一年也不多,这园子盖就盖了一年,如今要
画,自然得二年的工夫呢:又要研墨,又要蘸笔,又要铺纸,又要着颜色,又要—
—”刚说到这里,黛玉也自己掌不住,笑道:“又要照着样儿慢慢的画,可不得二
年的工夫?”众人听了,都拍手笑个不住。宝钗笑道:“有趣!最妙落后一句是‘慢
慢的画’。他可不画去,怎么就有了呢?所以昨儿那些笑话儿虽然可笑,回想是没趣
的。你们细想,颦儿这几句话,虽没什么,回想却有滋味。我倒笑的动不得了。”
惜春道:“都是宝姐姐赞的他越发逞强,这会子又拿我取笑儿。”黛玉忙拉他笑道:
“我且问你,还是单画这园子呢,还是连我们众人都画在上头呢?”惜春道:“原
是只画这园子。昨儿老太太又说:‘单画园子,成了房样子了。’叫连人都画上,就
像行乐图儿才好。我又不会这工细楼台,又不会画人物,又不好驳回,正为这个为
难呢。”黛玉道:“人物还容易,你草虫儿上不能。”李纨道:“你又说不通的话了。
这上头那里又用草虫儿呢?或者翎毛倒要点缀一两样。”黛玉笑道:“别的草虫儿罢
了,昨儿的‘母蝗虫’不画上,岂不缺了典呢?”众人听了,都笑起来。黛玉一面
笑的两只手捧着胸口,一面说道:“你快画罢,我连题跋都有了:起了名字,就叫
做《携蝗大嚼图》。”众人听了,越发哄然大笑的前仰后合。只听咕咚一声响,不知
什么倒了,急忙看时,原来是湘云伏在椅子背儿上,那椅子原不曾放稳,被他全身
伏着背子大笑,他又不防,两下里错了笋,向东一歪,连人带椅子都歪倒了。幸有
板壁挡住,不曾落地。众人一见,越发笑个不住。宝玉忙赶上去扶住了起来,方渐
渐止了笑。

宝玉和黛玉使个眼色儿,黛玉会意,便走至里间,将镜袱揭起。照了照,只见
两鬓略松了些,忙开了李纨的妆奁,拿出抿子来,对镜抿了两抿,仍旧收拾好了,
方出来指着李纨道:“这是叫你带着我们做针线、教道理呢,你反招了我们来大玩
大笑的!”李纨笑道:“你们听他这刁话。他领着头儿闹,引着人笑了,倒赖我的不
是!真真恨的我!只保佑你明儿得一个利害婆婆,再得几个千刁万恶的大姑子、小姑
子,试试你那会子还这么刁不刁了!”

黛玉早红了脸,拉着宝钗说:“咱们放他一年的假罢。”宝钗道:“我有一句公
道话,你们听听:藕丫头虽会画,不过是几笔写意;如今画这园子,非离了肚子里
头有些丘壑的,如何成画?这园子却是像画儿一般,山石树木,楼阁房屋,远近疏
密,也不多,也不少,恰恰的是这样。你若照样儿往纸上一画,是必不能讨好的。
这要看纸的地步远近,该多该少,分主分宾,该添的要添,该藏该减的要藏要减,
该露的要露,这一起了稿子,再端详斟酌,方成一幅图样。第二件:这些楼台房舍,
是必要界划的。一点儿不留神,栏杆也歪了,柱子也塌了,门窗也倒竖过来,阶砌
也离了缝,甚至桌子挤到墙里头去,花盆放在帘子上来,岂不倒成了一张笑话儿了!
第三:要安插人物,也要有疏密,有高低。衣褶裙带,指手足步,最是要紧;一笔
不细,不是肿了手,就是瘸了脚,染脸撕发倒是小事。依我看来,竟难的很。如今
一年的假也太多,一月的假也太少,竟给他半年的假;再派了宝兄弟帮着他。并不
是为宝兄弟知道教着他画,——那就更误了事;为的是有不知道的,或难安插的,
宝兄弟拿出去问问那会画的先生们,就容易了。”宝玉听了,先喜的说:“这话极是。
詹子亮的工细楼台就极好,程日兴的美人是绝技,如今就问他们去。”

宝钗道:“我说你是‘无事忙’,说了一声,你就问他去!也等着商议定了再去。
如今且说拿什么画?”宝玉道:“家里有雪浪纸,又大,又托墨。”宝钗冷笑道:“我
说你不中用。那雪浪纸写字、画写意画儿,或是会山水的画南宗山水,托墨,禁得
皴染;拿了画这个,又不托色,又难烘,画也不好,纸也可惜。我教给你一个法子:
原先盖这园子就有一张细致图样,虽是画工描的,那地步方向是不错的。你和太太
要出来,也比着那纸的大小,和凤姐姐要一块重绢,交给外边相公们,叫他照着这
图样删补着立了稿子,添了人物,就是了。就是配这些青绿颜色,并泥金泥银,也
得他们配去。你们也得另拢上风炉子,预备化胶、出胶、洗笔。还得一个粉油大案,
铺上毡子。你们那些碟子也不全,笔也不全,都从新再弄一分儿才好。”惜春道:“我
何曾有这些画器?不过随手的笔画画罢了。就是颜色,只有赭石、广花、藤黄、胭
脂这四样。再有不过是两支着色的笔就完了。”宝钗道:“你何不早说?这些东西我
却还有,只是你用不着,给你也白放着。如今我且替你收着,等你用着这个的时候
我送你些。也只可留着画扇子,若画这大幅的,也就可惜了。今儿替你开个单子,
照着单子和老太太要去。你们也未必知道的全,我说着,宝兄弟写。”

宝玉早已预备下笔砚了,原怕记不清白,要写了记着,听宝钗如此说,喜的提
起笔来静听。宝钗说道:“头号排笔四支,二号排笔四支,三号排笔四支,大染四
支,中染四支,小染四支,大南蟹爪十支,小蟹爪十支,须眉十支,大着色二十支,
小着色二十支,开面十支,柳条二十支,箭头朱四两,南赭四面,石黄四两,石青
四两,石绿四两,管黄四两,广花八两,铅粉十四匣,胭脂十二帖,大赤二百帖,
青金二百帖,广匀胶四两,净矾四两。矾绢的胶矾在外,别管他们,只把绢交出去,
叫他们矾去。这些颜色,咱们淘澄飞跌着,又玩了,又使了,包你一辈子都够使了。
再要顶细绢箩四个,粗箩二个,担笔四支,大小乳钵四个,大粗碗二十个,五寸碟
子十个,三寸粗白碟子二十个,风炉两个,沙锅大小四个,新磁缸二口,新水桶二
只,一尺长白布口袋四个,浮炭二十斤,柳木炭一二斤,三屉木箱一个,实地纱一
丈,生姜二两,酱半斤——”黛玉忙笑道:“铁锅一口,铁铲一个。”宝钗道:“这
做什么?”黛玉道:“你要生姜和酱这些作料,我替你要铁锅来,好炒颜色吃啊。”
众人都笑起来。宝钗笑道:“颦儿你知道什么!那粗磁碟子保不住不上火烤,不拿姜
汁子和酱预先抹在底子上烤过,一经了火,是要炸的。”众人听说,都道:“这就是
了。”

黛玉又看了一回单子,笑着拉探春悄悄的道:“你瞧瞧,画个画儿,又要起这
些水缸箱子来。想必糊涂了,把他的嫁妆单子也写上了。”探春听了,笑个不住,
说道“宝姐姐,你还不拧他的嘴?你问问他编派你的话!”宝钗笑道:“不用问,‘狗
嘴里还有象牙不成’!”一面说,一面走上来,把黛玉按在炕上,便要拧他的脸。黛
玉笑着,忙央告道:“好姐姐!饶了我罢!颦儿年纪小,只知说,不知道轻重,做姐
姐的教导我。姐姐不饶我,我还求谁去呢?”众人不知话内有因,都笑道:“说的
好可怜见儿的!连我们也软了,饶了他罢。”宝钗原是和他玩,忽听他又拉扯上前番
说他胡看杂书的话,便不好再和他闹了,放起他来。黛玉笑道:“到底是姐姐,要
是我,再不饶人的。”宝钗笑指他道:“怪不得老太太疼你,众人爱你,今儿我也怪
疼你的了。过来,我替你把头发笼笼罢。”黛玉果然转过身来,宝钗用手笼上去。
宝玉在旁看着,只觉更好,不觉后悔:“不该令他抿上鬓去,也该留着,此时叫他
替他抿上去。”正自胡想,只见宝钗说道:“写完了,明儿回老太太去。若家里有的
就罢,若没有的,就拿些钱去买了来,我帮着你们配。”宝玉忙收了单子。

大家又说了一回闲话儿。至晚饭后,又往贾母处来请安。贾母原没有大病,不
过是劳乏了,兼着了些凉,温存了一日,又吃了一两剂药,发散了发散,至晚也就
好了。

不知次日又有何话,下回分解。