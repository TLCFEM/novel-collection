\chapter{试文字宝玉始提亲~探惊风贾环重结怨}

却说薛姨妈一时因被金桂这场气怄得肝气上逆,左胁作痛。宝钗明知是这个原
故,也等不及医生来看,先叫人去买了几钱钩藤来,浓浓的煎了一碗,给他母亲吃
了。又和秋菱给薛姨妈捶腿揉胸。停了一会儿,略觉安顿些。薛姨妈只是又悲又气:
气的是金桂撒泼;悲的是宝钗见涵养,倒觉可怜。宝钗又劝了一回,不知不觉的睡
了一觉,肝气也渐渐平复了。宝钗便说道:“妈妈,你这种闲气不要放在心上才好。
过几天走的动了,乐得往那边老太太姨妈处去说说话儿,散散闷也好。家里横竖有
我和秋菱照看着,谅他也不敢怎么着。”薛姨妈点点头道:“过两日看罢了。”

且说元妃疾愈之后,家中俱各喜欢。过了几日,有几个老公走来,带着东西银
两,宣贵妃娘娘之命,因家中省问勤劳,俱有赏赐,把物件银两一一交代清楚。贾
赦贾政等禀明了贾母,一齐谢恩毕,太监吃了茶去了。大家回到贾母房中,说笑了
一回,外面老婆子传进来说:“小厮们来回道:‘那边有人请大老爷说要紧的话呢。’”
贾母便向贾赦道:“你去罢。”贾赦答应着,退出来自去了。

这里贾母忽然想起,合贾政笑道:“娘娘心里却甚实惦记着宝玉,前儿还特特
的问他来着呢。”贾政陪笑道:“只是宝玉不大肯念书,辜负了娘娘的美意。”贾
母道:“我倒给他上了个好儿,说他近日文章都做上来了。”贾政笑道:“那里能
像老太太的话呢。”贾母道:“你们时常叫他出去作诗作文,难道他都没作上来么?
小孩子家,慢慢的教导他。可是人家说的:‘胖子也不是一口儿吃的。’”贾政听
了这话,忙陪笑道:“老太太说的是。”贾母又道:“提起宝玉,我还有一件事和
你商量:如今他也大了,你们也该留神,看一个好孩子,给他定下。这也是他终身
的大事。也别论远近亲戚,什么穷啊富的,只要深知那姑娘的脾性儿好、模样儿周
正的,就好。”贾政道:“老太太吩咐的很是。但只一件:姑娘也要好,第一要他
自己学好才好。不然,不稂不莠的,反倒耽误了人家的女孩儿,岂不可惜?”贾母
听了这话,心里却有些不喜欢,便说道:“论起来,现放着你们作父母的,那里用
我去操心?但只我想宝玉这孩子从小儿跟着我,未免多疼他一点儿,耽误了他成人
的正事,也是有的;只是我看他那生来的模样儿也还齐整,心性儿也还实在,未必
一定是那种没出息的,必至遭塌了人家的女孩儿。也不知是我偏心?我看着横竖比
环儿略好些。不知你们看着怎么样?”

几句话说得贾政心中甚实不安,连忙陪笑道:“老太太看的人也多了,既说他
好,有造化,想来是不错的。只是儿子望他成人的性儿太急了一点,或者竟合古人
的话相反,倒是‘莫知其子之美’了。”一句话把贾母也怄笑了,众人也都陪着笑
了。贾母因说道:“你这会子也有了几岁年纪,又居着官,自然越历练越老成。”
说到这里,回头瞅着邢夫人合王夫人,笑道:“想他那年轻的时候,那一种古怪脾
气,比宝玉还加一倍呢。直等娶了媳妇,才略略的懂了些人事儿。如今只抱怨宝玉。
这会子,我看宝玉比他还略体些人情儿呢!”说的邢夫人王夫人都笑了,因说道:
“老太太又说起逗笑儿的话儿来了。”说着,小丫头子们进来告诉鸳鸯:“请示老
太太,晚饭伺候下了。”贾母便问:“你们又咕咕唧唧的说什么?”鸳鸯笑着回明
了。贾母道:“那么着,你们也都吃饭去罢,单留凤姐儿和珍哥媳妇跟着我吃罢。”
贾政及邢王二夫人都答应着,伺候摆上饭来,贾母又催了一遍,才都退出各散。

却说邢夫人自去了。贾政同王夫人进入房中。贾政因提起贾母方才的话来,说
道:“老太太这么疼宝玉。毕竟要他有些实学,日后可以混得功名才好:不枉老太
太疼他一场,也不至遭塌了人家的女儿。”王夫人道:“老爷这话自然是该当的。”
贾政因派个屋里的丫头传出去告诉李贵:“宝玉放学回来,索性吃饭后再叫他过来,
说我还要问他话呢。”李贵答应了“是”。至宝玉放了学,刚要过来请安,只见李
贵道:“二爷先不用过去。老爷吩咐了,今日叫二爷吃了饭就过去呢。听见还有话
问二爷呢。”宝玉听了这话,又是一个闷雷,只得见过贾母,便回园吃饭。三口两
口吃完,忙漱了口,便往贾政这边来。贾政此时在内书房坐着。宝玉进来请了安,
一旁侍立。贾政问道:“这几日我心上有事,也忘了问你。那一日你说你师父叫你
讲一个月的书,就要给你开笔。如今算来将两个月了,你到底开了笔了没有?”宝
玉道:“才做过三次。师父说:‘且不必回老爷知道;等好些,再回老爷知道罢。’
因此,这两天总没敢回。”贾政道:“是什么题目?”宝玉道:“一个是‘吾十有
五而志于学’,一个是‘人不知而不愠’,一个是‘则归墨’三字。”贾政道:“都
有稿儿么?”宝玉道:“都是作了抄出来,师父又改的。”贾政道:“你带了家来
了,还是在学房里呢?”宝玉道:“在学房里呢。”贾政道:“叫人取了来我瞧。”
宝玉连忙叫人传话与焙茗,叫他:“往学房中去,我书桌子抽屉里有一本薄薄儿竹
纸本子,上面写着‘窗课’两字的就是,快拿来。”

一会儿,焙茗拿了来,递给宝玉,宝玉呈与贾政。贾政翻开看时,见头一篇写
着题目是“吾十有五而志于学”。他原本破的是“圣人有志于学,幼而已然矣”。
代儒却将“幼”字抹去,明用“十五”。贾政道:“你原本‘幼’字,便扣不清题
目了。幼字是从小起,至十六以前都是‘幼’。这章书是圣人自言学问工夫与年俱
进的话,所以十五、三十、四十、五十、六十、七十,俱要明点出来,才见得到了
几时有这么个光景,到了几时又有那么个光景。师父把你幼字改了十五,便明白了
好些。”看到承题,那抹去的原本云:“夫不志于学,人之常也。”贾政摇头道:
“不但是孩子气,可见你本性不是个学者的志气。”又看后句:“圣人十五而志之,
不亦难乎?”说道:“这更不成话了!”然后看代儒的改本云:“夫人孰不学?而
志于学者卒鲜。此圣人所为自信于十五时欤?”便问:“改的懂得么?”宝玉答应
道:“懂得。”

又看第二艺,题目是“人不知而不愠”。便先看代儒的改本云:“不以不知而
愠者,终无改其说乐矣。”方觑着眼看那抹去的底本,说道:“你是什么?——‘能
无愠人之心,纯乎学者也。’上一句似单做了‘而不愠’三个字的题目,下一句又
犯了下文君子的分界;必如改笔,才合题位呢。且下句找清上文,方是书理。须要
细心领略。”宝玉答应着。贾政又往下看:“夫不知,未有不愠者也;而竟不然。
是非由说而乐者,曷克臻此?”原本末句“非纯学者乎”。贾政道:“这也与破题
同病的。这改的也罢了,不过清苦,还说得去。”

第三艺是“则归墨”。贾政看了题目,自己扬着头想了一想,因问宝玉道:“你
的书讲到这里了么?”宝玉道:“师父说,《孟子》好懂些,所以倒先讲《孟子》,
大前日才讲完了。如今讲上《论语》呢。”贾政因看这个破承,倒没大改。破题云:
“言于舍杨之外,若别无所归者焉。”贾政道:“第二句倒难为你。”“夫墨,非
欲归者也,而墨之言已半天下矣,则舍杨之外,欲不归于墨,得乎?”贾政道:“这
是你做的么?”宝玉答应道:“是。”贾政点点头儿,因说道:“这也并没有什么
出色处,但初试笔能如此,还算不离。前年我在任上时,还出过‘惟士为能’这个
题目。那些童生都读过前人这篇,不能自出心裁,每多抄袭。你念过没有?”宝玉
道:“也念过。”贾政道:“我要你另换个主意,不许雷同了前人,只做个破题也
使得。”宝玉只得答应着,低头搜索枯肠。

贾政背着手,也在门口站着作想。只见一个小小厮往外飞走。看见贾政,连忙
侧身垂手站住。贾政便问道:“作什么?”小厮回道:“老太太那边姨太太来了,
二奶奶传出话来,叫预备饭呢。”贾政听了,也没言语,那小厮自去了。谁知宝玉
自从宝钗搬回家去,十分想念,听见薛姨妈来了,只当宝钗同来,心中早已忙了,
便乍着胆子回道:“破题倒作了一个,但不知是不是?”贾政道:“你念来我听。”
宝玉念道:“天下不皆士也,能无产者亦仅矣。”贾政听了,点着头道:“也还使
得。以后作文,总要把界限分清,把神理想明白了再去动笔。你来的时候,老太太
知道不知道?”宝玉道:“知道的。”贾政道:“既如此,你还到老太太处去罢。”

宝玉答应了个“是”,只得拿捏着慢慢的退出。刚过穿廊月洞门的影屏,便一
溜烟跑到贾母院门口。急得焙茗在后头赶着叫道:“看跌倒了!老爷来了。”宝玉
那里听的见?刚进得门来,便听见王夫人、凤姐,探春等笑语之声。丫鬟们见宝玉
来了,连忙打起帘子,悄悄告诉道:“姨太太在这里呢。”宝玉赶忙进来给薛姨妈
请安,过来才给贾母请了晚安。贾母便问:“你今儿怎么这早晚才散学?”宝玉悉
把贾政看文章并命作破题的话述了一遍。贾母笑容满面。宝玉因问众人道:“宝姐
姐在那里坐着呢?”薛姨妈笑道:“你宝姐姐没过来,家里和香菱作活呢。”宝玉
听了,心中索然,又不好就走。只见就着话儿已摆上饭来,自然是贾母薛姨妈上坐,
探春等陪坐。薛姨妈道:“宝哥儿呢?”贾母笑着说道:“宝玉跟着我这边坐罢。”
宝玉连忙回道:“头里散学时,李贵传老爷的话,叫吃了饭过去,我赶着要了一碟
菜,泡茶吃了一碗饭,就过去了。老太太和姨妈、姐姐们用罢。”贾母道:“既这
么着,凤丫头就过来跟着我。你太太才说他今儿吃斋,叫他们自己吃去罢。”王夫
人也道:“你跟着老太太姨太太吃罢,不用等我,我吃斋呢。”于是凤姐告了坐,
丫头安了杯箸。凤姐执壶斟了一巡才归坐。

大家吃着酒,贾母便问道:“可是才姨太太提香菱;我听见前儿丫头们说‘秋
菱’,不知是谁,问起来才知道是他。怎么那孩子好好的又改了名字呢?”薛姨妈
满脸飞红,叹了口气,道:“老太太再别提起。自从蟠儿娶了这个不知好歹的媳妇,
成日家咕咕唧唧,如今闹的也不成个人家了。我也说过他几次,他牛心不听说,我
也没那么大精神和他们尽着吵去,只好由他们去。可不是他嫌这丫头的名儿不好改
的。”贾母道:“名儿什么要紧的事呢。”薛姨妈道:“说起来,我也怪臊的。其
实老太太这边,有什么不知道的?他那里是为这名儿不好?听见说,他因为是宝丫头
起的,他才有心要改。”贾母道:“这又是什么原故呢?”薛姨妈把手绢子不住的
擦眼泪,未曾说,又叹了一口气,道:“老太太还不知道呢,这如今媳妇子专和宝
丫头怄气。前日老太太打发人看我去,我们家里正闹呢。”贾母连忙接着问道:“可
是前儿听见姨太太肝气疼,要打发人看去;后来听见说好了,所以没着人去。依我
劝,姨太太竟把他们别放在心上。再者他们也是新过门的小夫妻,过些时自然就好
了。我看宝丫头性格儿温厚和平,虽然年轻,比大人还强几倍。前日那小丫头子回
来说,我们这边,还都赞叹了他一会子。都像宝丫头那样心胸儿、脾气儿,真是百
里挑一的!不是我说句冒失话,那给人家作了媳妇儿,怎么叫公婆不疼,家里上上
下下的不宾服呢?”宝玉头里已经听烦了,推故要走,及听见这话,又坐下呆呆的
往下听。薛姨妈道:“不中用。他虽好,到底是女孩儿家。养了蟠儿这个糊涂孩子,
真真叫我不放心。只怕在外头喝点子酒,闹出事来。幸亏老太太这里的大爷二爷常
和他在一块儿,我还放点儿心。”宝玉听到这里,便接口道:“姨妈更不用悬心。
薛大哥相好的都是些正经买卖大客人,都是有体面的,那里就闹出事来?”薛姨妈
笑道:“依你这样说,我敢只不用操心了。”说话间,饭已吃完。宝玉先告辞了:
“晚间还要看书。”便各自去了。

这里丫头们刚捧上茶来,只见琥珀走过来向贾母耳朵旁边说了几句,贾母便向
凤姐儿道:“你快去罢,瞧瞧巧姐儿去罢。”凤姐听了,还不知何故。大家也怔了。
琥珀遂过来向凤姐道:“刚才平儿打发小丫头子来回二奶奶,说:‘巧姐儿身上不
大好,请二奶奶忙着些过来才好呢。’”贾母因说道:“你快去罢,姨太太也不是
外人。”凤姐连忙答应,在薛姨妈跟前告了辞。又见王夫人说道:“你先过去,我
就去。小孩子家魂儿还不全呢,别叫丫头们大惊小怪的。屋里的猫儿狗儿,也叫他
们留点神儿。尽着孩子贵气,偏有这些琐碎。”凤姐答应了,然后带了小丫头回房
去了。这里薛姨妈又问了一回黛玉的病。贾母道:“林丫头那孩子倒罢了,只是心
重些,所以身子就不大很结实了。要赌灵怪儿,也和宝丫头不差什么;要赌宽厚待
人里头,却不济他宝姐姐有耽待、有尽让了。”薛姨妈又说了两句闲话儿,便道:
“老太太歇着罢,我也要到家里去看看,只剩下宝丫头和香菱了。打那么同着姨太
太看看巧姐儿。”贾母道:“正是。姨太太上年纪的人,看看是怎么不好,说给他
们,也得点主意儿。”薛姨妈便告辞,同着王夫人出来,往凤姐院里去了。

却说贾政试了宝玉一番,心里却也喜欢,走向外面和那些门客闲谈,说起方才
的话来。便有新近到来最善大棋的一个王尔调名作梅的,说道:“据我们看来,宝
二爷的学问已是大进了。”贾政道:“那有进益?不过略懂得些罢咧,‘学问’两
个字早得很呢。”詹光道:“这是老世翁过谦的话。不但王大兄这般说,就是我们
看,宝二爷必定要高发的。”贾政笑道:“这也是诸位过爱的意思。”那王尔调又
道:“晚生还有一句话,不揣冒昧,合老世翁商议。”贾政道:“什么事?”王尔
调陪笑道:“也是晚生的相与,做过南韶道的张大老爷家,有一位小姐,说是生的
德容功貌俱全,此时尚未受聘。他又没有儿子,家资巨万,但是要富贵双全的人家,
女婿又要出众,才肯作亲。晚生来了两个月,瞧着宝二爷的人品学业,都是必要大
成的。老世翁这样门楣,还有何说!若晚生过去,包管一说就成。”贾政道:“宝
玉说亲,却也是年纪了,并且老太太常说起。但只张大老爷素来尚未深悉。”詹光
道:“王兄所提张家,晚生却也知道,况合大老爷那边是旧亲,老世翁一问便知。”
贾政想了一回,道:“大老爷那边,不曾听得这门亲戚。”詹光道:“老世翁原来
不知:这张府上原和邢舅太爷那边有亲的。”贾政听了,方知是邢夫人的亲戚。坐
了一回,进来了,便要同王夫人说知,转问邢夫人去。谁知王夫人陪了薛姨妈到凤
姐那边看巧姐儿去了。那天已经掌灯时候,薛姨妈去了,王夫人才过来了。贾政告
诉了王尔调和詹光的话,又问:“巧姐儿怎么了?”王夫人道:“怕是惊风的光景。”
贾政道:“不甚利害呀?”王夫人道:“看着是搐风的来头,只还没搐出来呢。”
贾政听了,了一声,便不言语,各自安歇不提。

却说次日邢夫人过贾母这边来请安,王夫人便提起张家的事,一面回贾母,一
面问邢夫人。邢夫人道:“张家虽系老亲,但近年来久已不通音信,不知他家的姑
娘是怎么样的。倒是前日孙亲家太太打发老婆子来问安,却说起张家的事,说他家
有个姑娘,托孙亲家那边有对劲的提一提。听见说,只这一个女孩儿,十分娇养,
也识得几个字,见不得大阵仗儿,常在屋里不出来的。张大老爷又说:只有这一个
女孩儿,不肯嫁出去,怕人家公婆严,姑娘受不得委屈。必要女婿过门,赘在他家,
给他料理些家事。”贾母听到这里,不等说完,便道:“这断使不得。我们宝玉,
别人伏侍他还不够呢,倒给人家当家去!”邢夫人道:“正是老太太这个话。”贾
母因向王夫人道:“你回来告诉你老爷,就说我的话:这张家的亲事是作不得的。”
王夫人答应了。贾母便问:“你们昨日看巧姐儿怎么样?头里平儿来回我,说很不
大好,我也要过去看看呢。”邢王二夫人道:“老太太虽疼他,他那里耽的住?”
贾母道:“却也不止为他,我也要走动走动,活活筋骨儿。”说着,便吩咐:“你
们吃饭去罢,回来同我过去。”邢王二夫人答应着出来,各自去了。

一时吃了饭,都来陪贾母到凤姐房中。凤姐连忙出来,接了进去。贾母便问:
“巧姐儿到底怎么样?”凤姐儿道:“只怕是搐风的来头。”贾母道:“这么着还
不请人赶着瞧?”凤姐道:“已经请去了。”贾母因同邢王二夫人进房来看。只见
奶子抱着,用桃红绫子小绵被儿裹着,脸皮趣青,眉梢鼻翅微有动意。贾母同邢王
二夫人看了看,便出外间坐下。正说间,只见一个小丫头回凤姐道:“老爷打发人
问姐儿怎么样。”凤姐道:“替我回老爷,就说请大夫去了。一会儿开了方子,就
过去回老爷。”贾母忽然想起张家的事来,向王夫人道:“你该就去告诉你老爷,
省了人家去说了,回来又驳回。”又问邢夫人道:“你们和张家如今为什么不走了?”
邢夫人因又说:“论起那张家行事,也难合咱们作亲,太啬克,没的玷辱了宝玉。”
凤姐听了这话,已知八九,便问道:“太太不是说宝兄弟的亲事?”邢夫人道:“可
不是么。”贾母接着,因把刚才的话,告诉凤姐。凤姐笑道:“不是我当着老祖宗
太太们跟前说句大胆的话:现放着天配的姻缘,何用别处去找?”贾母笑问道:“在
那里?”凤姐道:“一个‘宝玉’,一个‘金锁’,老太太怎么忘了?”贾母笑了
一笑,因说:“昨日你姑妈在这里,你为什么不提?”凤姐道:“老祖宗和太太们
在前头,那里有我们小孩子家说话的地方儿?况且姨妈过来瞧老祖宗,怎么提这些
个?这也得太太们过去求亲才是。”贾母笑了,邢王二夫人也都笑了。贾母因道:
“可是我背晦了。”

说着,人回:“大夫来了。”贾母便坐在外间,邢王二夫人略避。那大夫同贾
琏进来,给贾母请了安,方进房中。看了出来,站在地下,躬身回贾母道:“妞儿
一半是内热,一半是惊风。须先用一剂发散风痰药,还要用四神散才好,因病势来
的不轻。如今的牛黄都是假的,要找真牛黄方用得。”贾母道了乏。那大夫同贾琏
出去,开了方子,去了。凤姐道:“人参家里常有,这牛黄倒怕未必有。外头买去,
只是要真的才好。”王夫人道:“等我打发人到姨太太那边去找找。他家蟠儿向来
和那些西客们做买卖,或者有真的,也未可知。我叫人去问问。”正说话间众姊妹
都来瞧来了,坐了一回,也都跟着贾母等去了。

这里煎了药,给巧姐儿灌下去了,只见喀的一声,连药带痰都吐出来,凤姐才
略放了一点儿心。只见王夫人那边的小丫头,拿着一点儿的小红纸包儿,说道:“二
奶奶,牛黄有了。太太说了,叫二奶奶亲自把分两对准了呢。”凤姐答应着接过来,
便叫平儿配齐了真珠、冰片、朱砂,快熬起来。自己用戥子按方秤了,搀在里面,
等巧姐儿醒了好给他吃。只见贾环掀帘进来,说:“二姐姐,你们巧姐儿怎么了?
妈叫我来瞧瞧他。”凤姐见了他母子便嫌,说:“好些了。你回去说,叫你们姨娘
想着。”那贾环口里答应,只管各处瞧看。看了一回,便问凤姐儿道:“你这里听
见说有牛黄,不知牛黄是怎么个样儿?给我瞧瞧呢。”凤姐道:“你别在这里闹了,
妞儿才好些。那牛黄都煎上了。”贾环听了,便去伸手拿那铞子瞧时,岂知措手不
及,“沸”的一声,铞子倒了,火已泼灭了一半。贾环见不是事,自觉没趣,连忙
跑了。凤姐急的火星直爆,骂道:“真真那一世的对头冤家!你何苦来还来使促狭!
从前你妈要想害我,如今又来害妞儿,我和你几辈子的仇呢?”一面骂平儿不照应。

正骂着,只见丫头来找贾环。凤姐道:“你去告诉赵姨娘,说他操心也太苦了!
巧姐儿死定了,不用他惦着了。”平儿急忙在那里配药再熬。那丫头摸不着头脑,
便悄悄问平儿道:“二奶奶为什么生气?”平儿将环哥弄倒药铞子说了一遍。丫头
道:“怪不得他不敢回来,躲了别处去了。这环哥儿明日还不知怎么样呢。平姐姐
我替你收拾罢。”平儿说:“这倒不消。幸亏牛黄还有一点,如今配好了,你去罢。”
丫头道:“我一准回去告诉赵姨奶奶,也省了他天天说嘴。”

丫头回去,果然告诉了赵姨娘。赵姨娘气的叫快找环儿。环儿在外间屋子里躲
着,被丫头找了来。赵姨娘便骂道:“你这个下作种子!你为什么弄洒了人家的药,
招的人家咒骂?我原叫你去问一声,不用进去。你偏进去,又不就走,还要‘虎头
上捉虱子’!你看我回了老爷打你不打!”这里赵姨娘正说着,只听贾环在外间屋
子里,更说出些惊心动魄的话来。

未知何言,下回分解。