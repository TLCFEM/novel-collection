\chapter{撕扇子作千金一笑~因麒麟伏白首双星}

话说袭人见了自己吐的鲜血在地,也就冷了半截。想着往日常听人说:“少年
吐血,年月不保,纵然命长终是废人了。”想起此言,不觉将素日想着后来争荣夸
耀之心尽皆灰了,眼中不觉的滴下泪来。宝玉见他哭了,也不觉心酸起来,因问道:
“你心里觉着怎么样?”袭人勉强笑道:“好好儿的,觉怎么样呢!”宝玉的意思
即刻便要叫人烫黄酒,要山羊血峒丸来。袭人拉着他的手,笑道:“你这一闹不
大紧,闹起多少人来,倒抱怨我轻狂。分明人不知道,倒闹的人知道了,你也不好,
我也不好。正经明儿你打发小子问问王大夫去,弄点子药吃吃就好了。人不知鬼不
觉的,不好吗?”宝玉听了有理,也只得罢了,向案上斟了茶来给袭人漱口。袭人
知宝玉心内也不安,待要不叫他伏侍,他又必不依,况且定要惊动别人,不如且由
他去罢。因此倚在榻上,由宝玉去伏侍。

那天刚亮,宝玉也顾不得梳洗,忙穿衣出来,将王济仁叫来亲自确问。王济仁
问其原故,不过是伤损,便说了个丸药的名字,怎么吃,怎么敷。宝玉记了,回园
来依方调治,不在话下。

这日正是端阳佳节,蒲艾簪门,虎符系臂。午间王夫人治了酒席,请薛家母女
等过节。宝玉见宝钗淡淡的,也不和他说话,自知是昨日的原故。王夫人见宝玉没
精打彩,也只当是昨日金钏儿之事,他没好意思的,越发不理他。黛玉见宝玉懒懒
的,只当是他因为得罪了宝钗的原故,心中不受用,形容也就懒懒的。凤姐昨日晚
上王夫人就告诉了他宝玉金钏儿的事,知道王夫人不喜欢,自己如何敢说笑,也就
随着王夫人的气色行事,更觉淡淡的。迎春姐妹见众人没意思,也都没意思了。因
此,大家坐了一坐,就散了。

那黛玉天性喜散不喜聚,他想的也有个道理。他说:“人有聚就有散,聚时喜
欢,到散时岂不清冷?既清冷则生感伤,所以不如倒是不聚的好。比如那花儿开的
时候儿叫人爱,到谢的时候儿便增了许多惆怅,所以倒是不开的好。”故此人以为
欢喜时,他反以为悲恸。那宝玉的性情只愿人常聚不散,花常开不谢;及到筵散花
谢,虽有万种悲伤,也就没奈何了。因此今日之筵大家无兴散了,黛玉还不觉怎么
着,倒是宝玉心中闷闷不乐,回至房中,长吁短叹。

偏偏晴雯上来换衣裳,不防又把扇子失了手掉在地下,将骨子跌折。宝玉因叹
道:“蠢才,蠢才!将来怎么样!明日你自己当家立业,难道也是这么顾前不顾后
的?”晴雯冷笑道:“二爷近来气大的很,行动就给脸子瞧。前儿连袭人都打了,
今儿又来寻我的不是。要踢要打凭爷去。就是跌了扇子,也算不的什么大事。先时
候儿什么玻璃缸,玛瑙碗,不知弄坏了多少,也没见个大气儿,这会子一把扇子就
这么着。何苦来呢!嫌我们就打发了我们,再挑好的使。好离好散的倒不好?”

宝玉听了这些话,气的浑身乱战。因说道:“你不用忙,将来横竖有散的日子!”
袭人在那边早已听见,忙赶过来,向宝玉道:“好好儿的,又怎么了?可是我说的,
一时我不到就有事故儿。”晴雯听了冷笑道:“姐姐既会说,就该早来呀,省了我
们惹的生气。自古以来,就只是你一个人会伏侍,我们原不会伏侍。因为你伏侍的
好,为什么昨儿才挨窝心脚啊!我们不会伏侍的,明日还不知犯什么罪呢?”袭人
听了这话,又是恼,又是愧;待要说几句,又见宝玉已经气的黄了脸,少不得自己
忍了性子道:“好妹妹,你出去逛逛儿,原是我们的不是。”晴雯听他说“我们”
两字,自然是他和宝玉了,不觉又添了醋意,冷笑几声道:“我倒不知道,你们是
谁?别叫我替你们害臊了!你们鬼鬼祟祟干的那些事,也瞒不过我去。不是我说,正
经明公正道的,连个姑娘还没挣上去呢,也不过和我似的,那里就称起‘我们’来
了!”

袭人羞得脸紫涨起来,想想原是自己把话说错了。宝玉一面说道:“你们气不
忿,我明日偏抬举他。”袭人忙拉了宝玉的手道:“他一个糊涂人,你和他分证什
么?况且你素日又是有担待的,比这大的过去了多少,今日是怎么了?”晴雯冷笑
道:“我原是糊涂人,那里配和我说话!我不过奴才罢咧!”袭人听说,道:“姑
娘到底是和我拌嘴,是和二爷拌嘴呢?要是心里恼我,你只和我说,不犯着当着二
爷吵;要是恼二爷,不该这么吵的万人知道。我才也不过为了事,进来劝开了,大
家保重,姑娘倒寻上我的晦气。又不像是恼我,又不像是恼二爷,夹枪带棒,终久
是个什么主意?我就不说,让你说去。”说着便往外走。宝玉向晴雯道:“你也不
用生气,我也猜着你的心事了。我回太太去,你也大了,打发你出去,可好不好?”

晴雯听了这话,不觉越伤起心来,含泪说道:“我为什么出去?要嫌我,变着
法儿打发我去,也不能够的。”宝玉道:“我何曾经过这样吵闹?一定是你要出来
了。不如回太太打发你去罢。”说着,站起来就要走。袭人忙回身拦住,笑道:“往
那里去?”宝玉道:“回太太去!”袭人笑道:“好没意思!认真的去回,你也不
怕臊了他!就是他认真要去,也等把这气下去了,等无事中说话儿回了太太也不迟。
这会子急急的当一件正经事去回,岂不叫太太犯疑?”宝玉道:“太太必不犯疑,
我只明说是他闹着要去的。”晴雯哭道:“我多早晚闹着要去了?饶生了气,还拿
话压派我。只管去回!我一头碰死了,也不出这门儿。”宝玉道:“这又奇了。你
又不去,你又只管闹。我经不起这么吵,不如去了倒干净。”说着一定要去回。袭
人见拦不住,只得跪下了。碧痕、秋纹、麝月等众丫鬟见吵闹的利害,都鸦雀无闻
的在外头听消息,这会子听见袭人跪下央求,便一齐进来,都跪下了。宝玉忙把袭
人拉起来,叹了一声,在床上坐下,叫众人起去。向袭人道:“叫我怎么样才好!
这个心使碎了,也没人知道。”说着,不觉滴下泪来。袭人见宝玉流下泪来,自己
也就哭了。

晴雯在旁哭着,方欲说话,只见黛玉进来,晴雯便出去了。黛玉笑道:“大节
下,怎么好好儿的哭起来了?难道是为争粽子吃,争恼了不成?”宝玉和袭人都“扑
嗤”的一笑。黛玉道:“二哥哥,你不告诉我,我不问就知道了。”一面说,一面
拍着袭人的肩膀,笑道:“好嫂子,你告诉我。必定是你们两口儿拌了嘴了。告诉
妹妹,替你们和息和息。”袭人推他道:“姑娘,你闹什么!我们一个丫头,姑娘
只是混说。”黛玉笑道:“你说你是丫头,我只拿你当嫂子待。”宝玉道:“你何
苦来替他招骂呢?饶这么着,还有人说闲话,还搁得住你来说这些个!”袭人笑道:
“姑娘,你不知道我的心,除非一口气不来,死了,倒也罢了。”黛玉笑道:“你
死了,别人不知怎么样,我先就哭死了。”宝玉笑道:“你死了,我做和尚去。”
袭人道:“你老实些儿罢!何苦还混说。”黛玉将两个指头一伸,抿着嘴儿笑道:
“做了两个和尚了!我从今以后,都记着你做和尚的遭数儿。”宝玉听了,知道是
点他前日的话,自己一笑,也就罢了。

一时黛玉去了,就有人来说:“薛大爷请。”宝玉只得去了,原来是吃酒,不
能推辞,只得尽席而散。晚间回来,已带了几分酒,踉跄来至自己院内,只见院中
早把乘凉的枕榻设下,榻上有个人睡着。宝玉只当是袭人,一面在榻沿上坐下,一
面推他,问道:“疼的好些了?”只见那人翻身起来,说:“何苦来?又招我!”
宝玉一看,原来不是袭人,却是晴雯。宝玉将他一拉,拉在身旁坐下,笑道:“你
的性子越发惯娇了。早起就是跌了扇子,我不过说了那么两句,你就说上那些话。
你说我也罢了,袭人好意劝你,又刮拉上他。你自己想想该不该?”晴雯道:“怪
热的,拉拉扯扯的做什么!叫人看见什么样儿呢!我这个身子本不配坐在这里。”宝
玉笑道:“你既知道不配,为什么躺着呢?”

晴雯没的说,“嗤”的又笑了,说道:“你不来使得,你来了就不配了。起来,
让我洗澡去。袭人麝月都洗了,我叫他们来。”宝玉笑道:“我才喝了好些酒,还
得洗洗。你既没洗,拿水来,咱们两个洗。”晴雯摇手笑道:“罢,罢!我不敢惹
爷。还记得碧痕打发你洗澡啊,足有两三个时辰,也不知道做什么呢,我们也不好
进去。后来洗完了,进去瞧瞧,地下的水,淹着床腿子,连席子上都汪着水。也不
知是怎么洗的。笑了几天!我也没工夫收拾水,你也不用和我一块儿洗。今儿也凉
快,我也不洗了,我倒是舀一盆水来你洗洗脸,篦篦头。才鸳鸯送了好些果子来,
都湃在那水晶缸里呢。叫他们打发你吃不好吗?”宝玉笑道:“既这么着,你不洗,
就洗洗手给我拿果子来吃罢。”晴雯笑道:“可是说的,我一个蠢才,连扇子还跌
折了,那里还配打发吃果子呢!倘或再砸了盘子,更了不得了。”宝玉笑道:“你
爱砸就砸。这些东西,原不过是借人所用,你爱这样,我爱那样,各有性情。比如
那扇子,原是的,你要撕着玩儿也可以使得,只是别生气时拿他出气;就如杯盘,
原是盛东西的,你喜欢听那一声响,就故意砸了也是使得的,只别在气头儿上拿他
出气。这就是爱物了。”晴雯听了,笑道:“既这么说,你就拿了扇子来我撕。我
最喜欢听撕的声儿。”宝玉听了,便笑着递给他。晴雯果然接过来,“嗤”的一声,
撕了两半。接着又听“嗤”“嗤”几声。宝玉在旁笑着说:“撕的好!再撕响些!”

正说着,只见麝月走过来,瞪了一眼,啐道:“少作点孽儿罢!”宝玉赶上来,
一把将他手里的扇子也夺了,递给晴雯,晴雯接了,也撕作几半子,二人都大笑起
来。麝月道:“这是怎么说?拿我的东西开心儿!”宝玉笑道:“你打开扇子匣子
拣去,什么好东西!”麝月道:“既这么说,就把扇子搬出来,让他尽力撕不好吗?”
宝玉笑道:“你就搬去。”麝月道:“我可不造这样孽。他没折了手,叫他自己搬
去。”晴雯笑着,便倚在床上,说道:“我也乏了!明儿再撕罢。”宝玉笑道:“古
人云:‘千金难买一笑。’几把扇子,能值几何?”一面说,一面叫袭人。袭人才
换了衣服走出来,小丫头佳蕙过来拾去破扇,大家乘凉,不消细说。

至次日午间,王夫人、宝钗、黛玉众姐妹正在贾母房中坐着,有人回道:“史
大姑娘来了。”一时,果见史湘云带领众多丫鬟媳妇走进院来。宝钗黛玉等忙迎至
阶下相见。青年姊妹经月不见,一旦相逢自然是亲密的,一时进入房中,请安问好,
都见过了。贾母因说:“天热,把外头的衣裳脱脱罢。”湘云忙起身宽衣。王夫人
因笑道:“也没见穿上这些做什么!”湘云笑道:“都是二婶娘叫穿的,谁愿意穿
这些!”宝钗一旁笑道:“姨妈不知道,他穿衣裳,还更爱穿别人的。可记得旧年
三四月里,他在这里住着,把宝兄弟的袍子穿上,靴子也穿上,带子也系上,猛一
瞧,活脱儿就像是宝兄弟,就是多两个坠子。他站在那椅子后头,哄的老太太只是
叫:‘宝玉,你过来,仔细那上头挂的灯穗子招下灰来,迷了眼。’他只是笑,也
不过去。后来大家忍不住笑了,老太太才笑了,还说:‘扮作小子样儿,更好看了。’”
黛玉道:“这算什么!惟有前年正月里接了他来,住了两日,下起雪来。老太太和
舅母那日想是才拜了影回来,老太太的一件新大红猩猩毡的斗篷放在那里。谁知眼
不见他就披上了,又大又长,他就拿了条汗巾子拦腰系上,和丫头们在后院子里扑
雪人儿玩。一跤栽倒了,弄了一身泥!”说着,大家想起来,都笑了。

宝钗笑问那周奶妈道:“周妈,你们姑娘还那么淘气不淘气了?”周奶妈也笑
了。迎春笑道:“淘气也罢了,我就嫌他爱说话:也没见睡在那里还是咭咭呱呱,
笑一阵,说一阵,也不知是那里来的那些谎话。”王夫人道:“只怕如今好了。前
日有人家来相看,眼见有婆婆家了,还是那么着?”贾母因问:“今日还是住着,
还是家去呢?”周奶妈笑道:“老太太没有看见,衣裳都带了来了,可不住两天。”
湘云问宝玉,道:“宝哥哥不在家么?”宝钗笑道:“他再不想别人,只想宝兄弟。
两个人好玩笑,这可见还没改了淘气。”贾母道:“如今你们大了,别提小名儿了。”

刚说着,只见宝玉来了,笑道:“云妹妹来了!怎么前日打发人接你去不来?”
王夫人道:“这里老太太才说这一个,他又来提名道姓的了。”黛玉道:“你哥哥
有好东西等着给你呢。”湘云道:“什么好东西?”宝玉笑道:“你信他!——几
日不见,越发高了。”湘云笑道:“袭人姐姐好?”宝玉道:“好,多谢你想着。”
湘云道:“我给他带了好东西来了。”说着,拿出绢子来,挽着一个搭。宝玉道:
“又是什么好物儿?你倒不如把前日送来的那绛纹石的戒指儿带两个给他。”湘云
笑道:“这是什么?”说着便打开,众人看时,果然是上次送来的那绛纹戒指,一
包四个。黛玉笑道:“你们瞧瞧他这个人,前日一般的打发人给我们送来,你就把
他的也带了来,岂不省事?今日巴巴儿的自己带了来,我打量又是什么新奇东西呢,
原来还是他!真真你是个糊涂人。”湘云笑道:“你才糊涂呢!我把这理说出来,大
家评评谁糊涂:给你们送东西,就是使来的人不用说话,拿进来一看,自然就知道
是送姑娘们的;要带了他们的来,须得我告诉来人,这是那一个女孩儿的,那是那
一个女孩儿的。那使来的人明白还好,再糊涂些,他们的名字多了,记不清楚,混
闹胡说的,反倒连你们的都搅混了。要是打发个女人来还好,偏前日又打发小子来,
可怎么说女孩儿们的名字呢?还是我来给他们带了来,岂不清白。”说着,把戒指
放下,说道:“袭人姐姐一个,鸳鸯姐姐一个,金钏儿姐姐一个,平儿姐姐一个:
这倒是四个人的,难道小子们也记得这么清楚?”众人听了,都笑道:“果然明白。”
宝玉笑道:“还是这么会说话,不让人。”黛玉听了,冷笑道:“他不会说话,就
配带‘金麒麟’了!”一面说着,便起身走了。幸而诸人都不曾听见,只有宝钗抿
着嘴儿一笑。宝玉听见了,倒自己后悔又说错了话,忽见宝钗一笑,由不得也一笑。
宝钗见宝玉笑,忙起身走开,找了黛玉说笑去了。

贾母因向湘云道:“喝了茶,歇歇儿,瞧瞧你嫂子们去罢。园里也凉快,和你
姐姐们去逛逛。”湘云答应了,因将三个戒指儿包上,歇了歇,便起身要瞧凤姐等
去。众奶娘丫头跟着,到了凤姐那里,说笑了一回。出来便往大观园来见过了李纨;
少坐片时,便往怡红院来找袭人。因回头说道:“你们不必跟着,只管瞧你们的亲
戚去。留下缕儿伏侍就是了。”众人应了,自去寻姑觅嫂,单剩下湘云翠缕两个。

翠缕道:“这荷花怎么还不开?”湘云道:“时候儿还没到呢。”翠缕道:“这
也和咱们家池子里的一样,也是楼子花儿。”湘云道:“他们这个还不及咱们的。”
翠缕道:“他们那边有棵石榴,接连四五枝,真是楼子上起楼子,这也难为他长。”
湘云道:“花草也是和人一样,气脉充足,长的就好。”翠缕把脸一扭,说道:“我
不信这话。要说和人一样,我怎么没见过头上又长出一个头来的人呢?”湘云听了,
由不得一笑,说道:“我说你不用说话,你偏爱说。这叫人怎么答言呢?天地间都
赋阴阳二气所生,或正或邪,或奇或怪,千变万化,都是阴阳顺逆;就是一生出来
人人罕见的,究竟道理还是一样。”翠缕道:“这么说起来,从古至今,开天辟地,
都是些阴阳了?”湘云笑道:“糊涂东西,越说越放屁。什么‘都是些阴阳’!况
且‘阴’‘阳’两个字,还只是一个字:阳尽了就是阴,阴尽了就是阳。不是阴尽
了又有一个阳生出来,阳尽了又有个阴生出来。”

翠缕道:“这糊涂死我了。什么是个阴阳,没影没形的?我只问姑娘:这阴阳
是怎么个样儿?”湘云道:“这阴阳不过是个气罢了。器物赋了,才成形质。譬如
天是阳,地就是阴;水是阴,火就是阳;日是阳,月就是阴。”翠缕听了,笑道:
“是了是了!我今儿可明白了。怪道人都管着日头叫‘太阳’呢,算命的管着月亮
叫什么‘太阴星’,就是这个理了。”湘云笑道:“阿弥陀佛,刚刚儿的明白了。”
翠缕道:“这些东西有阴阳也罢了,难道那些蚊子、虼蚤、蠓虫儿、花儿、草儿、
瓦片儿、砖头儿,也有阴阳不成?”湘云道:“怎么没有呢!比如那一个树叶儿,
还分阴阳呢:向上朝阳的就是阳,背阴覆下的就是阴了。”翠缕听了,点头笑道:
“原来这么着,我可明白了。只是咱们这手里的扇子,怎么是阴,怎么是阳呢?”
湘云道:“这边正面就为阳,那反面就为阴。”

翠缕又点头笑了。还要拿几件东西要问,因想不起什么来,猛低头看见湘云宫
绦上的金麒麟,便提起来,笑道:“姑娘,这个难道也有阴阳?”湘云道:“走兽
飞禽,雄为阳,雌为阴;牝为阴,牡为阳:怎么没有呢。”翠缕道:“这是公的,
还是母的呢?”湘云啐道:“什么‘公’的‘母’的!又胡说了。”翠缕道:“这
也罢了,怎么东西都有阴阳,咱们人倒没有阴阳呢?”湘云沉了脸说道:“下流东
西,好生走罢,越问越说出好的来了!”翠缕道:“这有什么不告诉我的呢?我也
知道了,不用难我。”湘云“扑嗤”的笑道:“你知道什么?”翠缕道:“姑娘是
阳,我就是阴。”湘云拿着绢子掩着嘴笑起来。翠缕道:“说的是了,就笑的这么
样?”湘云道:“很是,很是!”翠缕道:“人家说主子为阳,奴才为阴,我连这
个大道理也不懂得?”湘云笑道:“你很懂得。”

正说着,只见蔷薇架下,金晃晃的一件东西。湘云指着问道:“你看那是什么?”
翠缕听了,忙赶去拾起来,看着笑道:“可分出阴阳来了!”说着,先拿湘云的麒
麟瞧。湘云要把拣的瞧瞧,翠缕只管不放手,笑道:“是件宝贝,姑娘瞧不得!这
是从那里来的?好奇怪!我只从来在这里,没见人有这个。”湘云道:“拿来我瞧瞧。”
翠缕将手一撒,笑道:“姑娘请看。”湘云举目一看,却是文彩辉煌的一个金麒麟,
比自己佩的又大,又有文彩。湘云伸手擎在掌上,心里不知怎么一动,似有所感。
忽见宝玉从那边来了,笑道:“你在这日头底下做什么呢?怎么不找袭人去呢?”
湘云连忙将那个麒麟藏起,道:“正要去呢!咱们一处走。”说着,大家进了怡红
院来。

袭人正在阶下倚槛迎风,忽见湘云来了,连忙迎下来,携手笑说一向别情,一
面进来让坐。宝玉因问道:“你该早来,我得了一件好东西,专等你呢。”说着,
一面在身上掏了半天,“嗳呀”了一声,便问袭人:“那个东西你收起来了么?”
袭人道:“什么东西?”宝玉道:“前日得的麒麟。”袭人道:“你天天带在身上
的,怎么问我?”宝玉听了,将手一拍,说道:“这可丢了!往那里找去?”就要
起身自己寻去。湘云听了,方知是宝玉遗落的,便笑问道:“你几时又有个麒麟了?”
宝玉道:“前日好容易得的呢!不知多早晚丢了,我也糊涂了。”湘云笑道:“幸
而是个玩的东西,还是这么慌张。”说着,将手一撒,笑道:“你瞧瞧是这个不是?”
宝玉一见,由不得欢喜非常。

要知后事,下回分解。