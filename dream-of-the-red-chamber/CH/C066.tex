\chapter{情小妹耻情归地府~冷二郎一冷入空门}

话说兴儿说怕吹倒了林姑娘,吹化了薛姑娘,大家都笑了。那鲍二家的打他一
下子,笑道:“原有些真;到了你嘴里,越发没了捆儿了。你倒不像跟二爷的人,
这些话倒像是宝玉的人。”尤二姐才要又问,忽见尤三姐笑问道:“可是,你们家
那宝玉,除了上学他做些什么?”兴儿笑道:“三姨儿别问他。说起来,三姨儿也
未必信:他长了这么大,独他没有上过正经学。我们家从祖宗直到二爷,谁不是学
里的师老爷严严的管着念书?偏他不爱念书,是老太太的宝贝。老爷先还管,如今
也不敢管了。成天家疯疯癫癫的,说话人也不懂,干的事人也不知。外头人人看着
好清俊模样儿,心里自然是聪明的,谁知里头更糊涂。见了人,一句话也没有。所
有的好处,虽没上过学,倒难为他认得几个字。每日又不习文,又不学武,又怕见
人,只爱在丫头群儿里闹。再者,也没个刚气儿。有一遭见了我们,喜欢时没上没
下,大家乱玩一阵;不喜欢各自走了,他也不理人。我们坐着卧着,见了他也不理
他,他也不责备。因此,没人怕他,只管随便,都过的去。”尤三姐笑道:“主子
宽了,你们又这样;严了,又抱怨:可知你们难缠。”尤二姐道:“我们看他倒好,
原来这样。可惜了儿的一个好胎子!”尤三姐道:“姐姐信他胡说?咱们也不是见
过一面两面的,行事言谈吃喝,原有些女儿气的,自然是天天只在里头惯了的。要
说糊涂,那些儿糊涂?姐姐记得穿孝时,咱们同在一处,那日正是和尚们进来绕棺,
咱们都在那里站着,他只站在头里挡着人。人说他不知礼,又没眼色。过后他没悄
悄的告诉咱们说?——‘姐姐们不知道:我并不是没眼色,想和尚们的那样腌,
只恐怕气味熏了姐姐们。’接着他吃茶,姐姐又要茶,那个老婆子就拿了他的碗去
倒,他赶忙说:‘那碗是腌的,另洗了再斟来。’这两件上,我冷眼看去,原来
他在女孩儿跟前,不管什么都过的去,只不大合外人的式,所以他们不知道。”尤
二姐听说,笑道:“依你说,你两个已是情投意合了。竟把你许了他岂不好?”三
姐见有兴儿,不便说话,只低了头磕瓜子儿。兴儿笑道:“若论模样儿行为,倒是
一对儿好人。只是他已经有了人了,只是没有露形儿,将来准是林姑娘定了的。因
林姑娘多病,二则都还小,所以还没办呢。再过三二年,老太太便一开言,那是再
无不准的了。”大家正说话,只见隆儿又来了,说:“老爷有事,是件机密大事,
要遣二爷往平安州去。不过三五日就起身,来回得十五六天的工夫。今儿不能来了,
请老奶奶早和二姨儿定了那件事,明日爷来好做定夺。”说着带了兴儿,也回去了。

这里尤二姐命掩了门,早睡下了,盘问他妹子一夜。至次日午后贾琏方来了,
尤二姐因劝他,说:“既有正事,何必忙忙又来?千万别为我误事。”贾琏道:“也
没什么事,只是偏偏的又出来了一件远差。出了月儿就起身,得半月工夫才来。”
尤二姐道:“既如此,你只管放心前去,这里一应不用你惦记。三妹妹他从不会朝
更暮改的。他已择定了人,你只要依他就是了。”贾琏忙问:“是谁?”二姐笑道:
“这人此刻不在这里,不知多早晚才来呢。也难为他的眼力。他自己说了:这人一
年不来,他等一年;十年不来,等十年。若这人死了,再不来了,他情愿剃了头当
姑子去,吃常斋念佛,再不嫁人。”贾琏问:“到底是谁,这样动他的心?”二姐
儿笑道:“说来话长。五年前,我们老娘家做生日,妈妈和我们到那里给老娘拜寿,
他家请了一起玩戏的人,也都是好人家子弟。里头有个装小生的,叫做柳湘莲。如
今要是他才嫁。旧年闻得这人惹了祸逃走了,不知回来了不曾。”贾琏听了道:“怪
道呢,我说是个什么人,原来是他。果然眼力不错。你不知道那柳老二那样一个标
致人,最是冷面冷心的,差不多的人,他都无情无义。他最和宝玉合的来。去年因
打了薛呆子,他不好意思见我们的,不知那里去了,一向没来。听见有人说来了,
不知是真是假,一问宝玉的小厮们,就知道了。倘或不来时,他是萍踪浪迹,知道
几年才来?岂不白耽搁了大事?”二姐道:“我们这三丫头,说的出来干的出来,
他怎么说,只依他便了。”

二人正说之间,只见三姐走来说道:“姐夫,你也不知道我们是什么人。今日
和你说罢:你只放心,我们不是那心口两样的人,说什么是什么。若有了姓柳的来,
我便嫁他。从今儿起,我吃常斋念佛,伏侍母亲,等来了嫁了他去;若一百年不来,
我自己修行去了。”说着将头上一根玉簪拔下来,磕作两段,说:“一句不真,就
合这簪子一样!”说着,回房去了,真个竟“非礼不动,非礼不言”起来。贾琏无
了法,只得和二姐商议了一回家务,复回家和凤姐商议起身之事。一面着人问焙茗。
焙茗说:“竟不知道。大约没来,若来了,必是我知道的。”一面又问他的街坊,
也说没来。贾琏只得回复了二姐儿。

至起身之日已近,前两天便说起身,却先往二姐儿这边来住两夜,从这里再悄
悄的长行。果见三姐儿竟像又换了一个人的似的。又见二姐儿持家勤慎,自是不消
惦记。是日,一早出城,竟奔平安州大道,晓行夜住,渴饮饥餐。方走了三日,那
日正走之间,顶头来了一群驮子,内中一伙,主仆十来匹马。走的近了,一看时,
不是别人,就是薛蟠和柳湘莲来了。贾琏深为奇怪,忙伸马迎了上来,大家一齐相
见。说些别后寒温,便入一酒店歇下,共叙谈叙谈。贾琏因笑道:“闹过之后,我
们忙着请你两个和解,谁知柳二弟踪迹全无。怎么你们两个今日倒在一处了?”薛
蟠笑道:“天下竟有这样奇事:我和伙计贩了货物,自春天起身,往回里走,一路
平安。谁知前儿到了平安州地面,遇见一伙强盗,已将东西劫去。不想柳二弟从那
边来了,方把贼人赶散,夺回货物,还救了我们的性命。我谢他又不受,所以我们
结拜了生死兄弟,如今一路进京。从此后,我们是亲弟兄一般。到前面岔口上分路,
他就分路往南二百里,有他一个姑妈家,他去望候望候。我先进京去安置了我的事,
然后给他寻一所房子,寻一门好亲事,大家过起来。”贾琏听了道:“原来如此!
倒好,只是我们白悬了几日心。”因又说道:“方才说给柳二弟提亲,我正有一门
好亲事,堪配二弟。”说着,便将自己娶尤氏,如今又要发嫁小姨子一节,说了出
来,只不说尤三姐自择之语。又嘱薛蟠:“且不可告诉家里。等生了儿子,自然是
知道的。”薛蟠听了大喜,说:“早该如此。这都是舍表妹之过。”湘莲忙笑道:
“你又忘情了,还不住口。”薛蟠忙止住不语,便说:“既是这等,这门亲事定要
做的。”湘莲道:“我本有愿,定要一个绝色的女子。如今既是贵昆仲高谊,顾不
得许多了,任凭定夺,我无不从命。”贾琏笑道:“如今口说无凭,等柳二弟一见,
便知我这内娣的品貌,是古今有一无二的了。”湘莲听了大喜,说:“既如此说,
等弟探过姑母,不过一月内,就进京的,那时再定,如何?”贾琏笑道:“你我一
言为定。只是我信不过二弟,你是萍踪浪迹,倘然去了不来,岂不误了人家一辈子
的大事?须得留一个定礼。”湘莲道:“大丈夫岂有失信之理?小弟素系寒贫,况且
在客中,那里能有定礼?”薛蟠道:“我这里现成,就备一分,二哥带去。”贾琏
道:“也不用金银珠宝,须是二弟亲身自有的东西,不论贵贱,不过带去取信耳。”
湘莲道:“既如此说,弟无别物,囊中还有一把‘鸳鸯剑’,乃弟家中传代之宝,
弟也不敢擅用,只是随身收藏着,二哥就请拿去为定。弟纵系水流花落之性,亦断
不舍此剑。”说毕,大家又饮了几杯,方各自上马,作别起程去了。

且说贾琏一日到了平安州,见了节度,完了公事,因又嘱咐他十月前后务要还
来一次。贾琏领命,次日连忙取路回家,先到尤二姐那边。且说二姐儿操持家务,
十分谨肃,每日关门闭户,一点外事不闻。那三姐儿果是个斩钉截铁之人,每日侍
奉母亲之馀,只和姐姐一处做些活计,虽贾珍趁贾琏不在家,也来鬼混了两次,无
奈二姐儿只不兜揽,推故不见。那三姐儿的脾气,贾珍早已领过教的,那里还敢招
惹他去?所以踪迹一发疏阔了。却说这日贾琏进门,看见二姐儿三姐儿这般景况,
喜之不尽,深念二姐儿之德。大家叙些寒温,贾琏便将路遇柳湘莲一事说了一回,
又将鸳鸯剑取出递给三姐儿。三姐儿看时,上面龙吞夔护,珠宝晶莹;及至拿出来
看时,里面却是两把合体的,一把上面錾一“鸳”字,一把上面錾一“鸯”字,冷
飕飕,明亮亮,如两痕秋水一般。三姐儿喜出望外,连忙收了,挂在自己绣房床上,
每日望着剑,自喜终身有靠。贾琏住了两天,回去复了父命,回家合宅相见。那时
凤姐已大愈,出来理事行走了。贾琏又将此事告诉了贾珍。贾珍因近日又搭上了新
相知,二则正恼他姐妹们无情,把这事丢过了,全不在心上,任凭贾琏裁夺;只怕
贾琏独力不能,少不得又给他几十两银子。贾琏拿来,交给二姐儿,预备妆奁。

谁知八月内湘莲方进了京,先来拜见薛姨妈。又遇见薛蟠,方知薛蟠不惯风霜,
不服水土,一进京时,便病倒在家,请医调治。听见湘莲来了,请入卧室相见。薛
姨妈也不念旧事,只感救命之恩。母子们十分称谢。又说起亲事一节:凡一应东西
皆置办妥当,只等择日。湘莲也感激不尽。

次日,又来见宝玉。二人相会,如鱼得水。湘莲因问贾琏偷娶二房之事。宝玉
笑道:“我听见焙茗说,我却未见。我也不敢多管。我又听见焙茗说,琏二哥哥着
实问你。不知有何话说?”湘莲就将路上所有之事,一概告诉了宝玉。宝玉笑道:
“大喜,大喜!难得这个标致人!果然是个古今绝色,堪配你之为人。”湘莲道:“既
是这样,他那少了人物?如何只想到我?况且我又素日不甚和他相厚,也关切不至于
此。路上忙忙的就那样再三要求定下,难道女家反赶着男家不成?我自己疑惑起来,
后悔不该留下这剑作定。所以后来想起你来,可以细细问了底里才好。”宝玉道:
“你原是个精细人,如何既许了定礼又疑惑起来?你原说只要一个绝色的。如今既
得了个绝色的,便罢了,何必再疑?”湘莲道:“你既不知他来历,如何又知是绝
色?”宝玉道:“他是珍大嫂子的继母带来的两位妹子。我在那里和他们混了一个
月,怎么不知?真真一对尤物!
——他又姓尤。”湘莲听了,跌脚道:“这事不好!断乎做不得。你们东府里,除
了那两个石头狮子干净罢了。”宝玉听说,红了脸。湘莲自惭失言,连忙作揖,说:
“我该死,胡说。你好歹告诉我,他品行如何?”宝玉笑道:“你既深知,又来问
我做甚么?连我也未必干净了。”湘莲笑道:“原是我自己一时忘情,好歹别多心。”
宝玉笑道:“何必再提,这倒似有心了。”

湘莲作揖告辞出来,心中想着要找薛蟠,一则他病着,二则他又浮躁,不如去
要回定礼。主意已定,便一径来找贾琏。贾琏正在新房中,闻湘莲来了,喜之不尽,
忙迎出来,让到内堂,和尤老娘相见。湘莲只作揖,称“老伯母”,自称“晚生”,
贾琏听了诧异。吃茶之间,湘莲便说:“客中偶然忙促,谁知家姑母于四月订了弟
妇,使弟无言可回。要从了二哥,背了姑母,似不合理。若系金帛之定,弟不敢索
取;但此剑系祖父所遗,请仍赐回为幸。”贾琏听了,心中自是不自在,便道:“二
弟,这话你说错了。定者,定也,原怕返悔,所以为定。岂有婚姻之事,出入随意
的?这个断乎使不得。”湘莲笑说:“如此说,弟愿领责备罚,然此事断不敢从命。”
贾琏还要绕舌。湘莲便起身说:“请兄外座一叙,此处不便。”

那尤三姐在房明明听见。好容易等了他来,今忽见返悔,便知他在贾府中听了
什么话来,把自己也当做淫奔无耻之流,不屑为妻。今若容他出去和贾琏说退亲,
料那贾琏不但无法可处,就是争辩起来,自己也无趣味。一听贾琏要同他出去,连
忙摘下剑来,将一股雌锋隐在肘后,出来便说:“你们也不必出去再议,还你的定
礼!”一面泪如雨下,左手将剑并鞘送给湘莲,右手回肘,只往项上一横。可怜:
揉碎桃花红满地,玉山倾倒再难扶!

当下唬的众人急救不迭。尤老娘一面嚎哭,一面大骂湘莲。贾琏揪住湘莲,命
人捆了送官。二姐儿忙止泪,反劝贾琏:“人家并没威逼他,是他自寻短见,你便
送他到官,又有何益?反觉生事出丑。不如放他去罢。”贾琏此时也没了主意,便
放了手,命湘莲快去。湘莲反不动身,拉下手绢,拭泪道:“我并不知是这等刚烈
人!真真可敬!是我没福消受。”大哭一场,等买了棺木,眼看着入殓,又抚棺大哭
一场,方告辞而去。

出门正无所之,昏昏默默,自想方才之事:“原来这样标致人才,又这等刚烈!”
自悔不及,信步行来,也不自知了。正走之间,只听得隐隐一阵环佩之声,三姐从
那边来了,一手捧着鸳鸯剑,一手捧着一卷册子,向湘莲哭道:“妾痴情待君五年,
不期君果冷心冷面。妾以死报此痴情。妾今奉警幻仙姑之命,前往太虚幻境,修注
案中所有一干情鬼。妾不忍相别,故来一会,从此再不能相见矣!”说毕,又向湘
莲洒了几点眼泪,便要告辞而行。湘莲不舍,连忙欲上来拉住问时,那三姐一摔手,
便自去了。这里柳湘莲放声大哭,不觉处梦中哭醒,似梦非梦,睁眼看时,竟是一
座破庙,旁边坐着一个瘸腿道士捕虱。湘莲便起身稽首相问:“此系何方?仙师何
号?”道士笑道:“连我也不知道此系何方,我系何人。不过暂来歇脚而已。”湘
莲听了,冷然如寒冰侵骨。掣出那股雄剑来,将万根烦恼丝,一挥而尽,便随那道
士,不知往那里去了。

要知端底,下回分解。