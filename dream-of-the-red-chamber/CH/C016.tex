\chapter{贾元春才选凤藻宫~秦鲸卿夭逝黄泉路}

且说秦钟宝玉二人跟着凤姐自铁槛寺照应一番,坐车进城,到家见过贾母王夫
人等,回到自己房中,一夜无话。至次日,宝玉见收拾了外书房,约定了和秦钟念
夜书。偏偏那秦钟秉赋最弱,因在郊外受了些风霜,又与智能儿几次偷期缱绻,未
免失于检点,回来时便咳嗽伤风,饮食懒进,大有不胜之态,只在家中调养,不能
上学。宝玉便扫了兴,然亦无法,只得候他病痊再议。

那凤姐却已得了云光的回信,俱已妥协,老尼达知张家,那守备无奈何,忍气
吞声受了前聘之物。谁知爱势贪财的父母,却养了一个知义多情的女儿,闻得退了
前夫,另许李门,他便一条汗巾悄悄的寻了自尽。那守备之子谁知也是个情种,闻
知金哥自缢,遂投河而死。可怜张李二家没趣,真是“人财两空”。这里凤姐却安
享了三千两。王夫人连一点消息也不知。自此凤姐胆识愈壮,以后所作所为,诸如
此类,不可胜数。

一日正是贾政的生辰,宁荣二处人丁都齐集庆贺,热闹非常。忽有门吏报道:
“有六宫都太监夏老爷特来降旨。”吓的贾赦贾政一干人不知何事,忙止了戏文,
撤去酒席,摆香案,启中门跪接。早见都太监夏秉忠乘马而至,又有许多跟从的内
监。那夏太监也不曾负诏捧敕,直至正厅下马,满面笑容,走至厅上,南面而立,
口内说:“奉特旨:立刻宣贾政入朝,在临敬殿陛见。”说毕,也不吃茶,便乘马
去了。贾政等也猜不出是何来头,只得即忙更衣入朝。

贾母等合家人心俱惶惶不定,不住的使人飞马来往探信。有两个时辰,忽见赖
大等三四个管家喘吁吁跑进仪门报喜,又说:“奉老爷的命:就请老太太率领太太
等进宫谢恩呢。”那时贾母心神不定,在大堂廊下伫候,邢王二夫人、尤氏、李纨、
凤姐、迎春姊妹以及薛姨妈等,皆聚在一处打听信息。贾母又唤进赖大来细问端底,
赖大禀道:“奴才们只在外朝房伺候着,里头的信息一概不知。后来夏太监出来道
喜,说咱们家的大姑奶奶封为凤藻宫尚书,加封贤德妃。后来老爷出来也这么吩咐。
如今老爷又往东宫里去了。急速请太太们去谢恩。”贾母等听了方放下心来,一时
皆喜见于面。于是都按品大妆起来。贾母率领邢王二夫人并尤氏,一共四乘大轿,
鱼贯入朝。贾赦贾珍亦换了朝服,带领贾蔷贾蓉,奉侍贾母前往。

宁荣两处上下内外人等,莫不欢天喜地,独有宝玉置若罔闻。你道什么缘故?
原来近日水月庵的智能私逃入城来找秦钟,不意被秦邦业知觉,将智能逐出,将秦
钟打了一顿,自己气的老病发了,三五日,便呜呼哀哉了。秦钟本自怯弱,又带病
未痊受了笞杖,今见老父气死,悔痛无及,又添了许多病症。因此,宝玉心中怅怅
不乐。虽有元春晋封之事,那解得他的愁闷?贾母等如何谢恩,如何回家,亲友如
何来庆贺,宁荣两府近日如何热闹,众人如何得意,独他一个皆视有如无,毫不介
意:因此众人嘲他越发呆了。

且喜贾琏与黛玉回来,先遣人来报信:“明日就可到家了。”宝玉听了,方略
有些喜意。细问原由,方知贾雨村也进京引见,——皆由王子腾累上荐本,此来候
补京缺,——与贾琏是同宗弟兄,又与黛玉有师徒之谊,故同路作伴而来。林如海
已葬入祖茔了,诸事停妥。贾琏这番进京,若按站走时本该出月到家,因听见元春
喜信,遂昼夜兼程而进。一路俱各平安。宝玉只问了黛玉好,馀者也就不在意了。

好容易盼到明日午错,果报:“琏二爷和林姑娘进府了。”见面时彼此悲喜交
集,未免大哭一场,又致庆慰之词。宝玉细看那黛玉时,越发出落的超逸了。黛玉
又带了许多书籍来,忙着打扫卧室,安排器具,又将些纸笔等物分送与宝钗、迎春、
宝玉等。宝玉又将北静王所赠苓香串珍重取出来转送黛玉。黛玉说:“什么臭男
人拿过的,我不要这东西。”遂掷还不取。宝玉只得收回,暂且无话。

且说贾琏自回家见过众人,回至房中,正值凤姐事繁,无片刻闲空,见贾琏远
路归来,少不得拨冗接待。因房内别无外人,便笑道:“国舅老爷大喜!国舅老爷
一路风尘辛苦!小的听见昨日的头起报马来说,今日大驾归府,略预备了一杯水酒
掸尘,不知可赐光谬领否?”贾琏笑道:“岂敢!岂敢!多承,多承!”一面平儿与
众丫鬟参见毕,端上茶来。贾琏遂问别后家中诸事,又谢凤姐的辛苦。凤姐道:“我
那里管的上这些事来!见识又浅,嘴又笨,心又直,人家给个棒槌,我就拿着认作
针了。脸又软,搁不住人家给两句好话儿。况且又没经过事,胆子又小,太太略有
点不舒服,就吓的也睡不着了。我苦辞过几回,太太不许,倒说我图受用,不肯学
习,那里知道我是捻着把汗儿呢!一句也不敢多说,一步也不敢妄行。你是知道的,
咱们家所有的这些管家奶奶,那一个是好缠的?错一点儿他们就笑话打趣,偏一点
儿他们就指桑骂槐的抱怨,‘坐山看虎斗’,‘借刀杀人’,‘引风吹火’,‘站
干岸儿’,‘推倒了油瓶儿不扶’,都是全挂子的本事,况且我又年轻,不压人,
怨不得不把我搁在眼里。更可笑那府里蓉儿媳妇死了,珍大哥再三在太太跟前跪着
讨情,只要请我帮他几天。我再四推辞,太太做情应了,只得从命。到底叫我闹了
个马仰人翻,更不成个体统。至今珍大哥还抱怨后悔呢。你明儿见了他,好歹赔释
赔释,就说我年轻,原没见过世面,谁叫大爷错委了他呢。”

说着,只听外间有人说话,凤姐便问:“是谁?”平儿进来回道:“姨太太打
发香菱妹子来问我一句话,我已经说了,打发他回去了。”,贾琏笑道:“正是呢。
我才见姨妈去,和一个年轻的小媳妇子刚走了个对脸儿,长得好齐整模样儿。我想
咱们家没这个人哪,说话时问姨妈,才知道是打官司的那小丫头子,叫什么香菱的,
竟给薛大傻子作了屋里人。开了脸,越发出挑的标致了。那薛大傻子真玷辱了他!”
凤姐把嘴一撇,道:“哎!往苏杭走了一趟回来,也该见点世面了,还是这么眼馋
肚饱的。你要爱他,不值什么,我拿平儿换了他来好不好?那薛老大也是吃着碗里
瞧着锅里的,这一年来的时候,他为香菱儿不能到手,和姑妈打了多少饥荒。姑妈
看着香菱的模样儿好还是小事,因他做人行事,又比别的女孩子不同,温柔安静,
差不多儿的主子姑娘还跟不上他,才摆酒请客的费事,明堂正道给他做了屋里人。
过了没半月,也没事人一大堆了。”一语未了,二门上的小厮传报:“老爷在大书
房里等着二爷呢。”贾琏听了,忙忙整衣出去。

这里凤姐因问平儿:“方才姑妈有什么事,巴巴儿的打发香菱来?”平儿道:
“那里来的香菱!是我借他暂撒个谎儿。奶奶瞧,旺儿嫂子越发连个算计儿也没
了!”说着,又走至凤姐身边,悄悄说道:“那项利银早不送来,晚不送来,这会
子二爷在家,他偏送这个来了。幸亏我在堂屋里碰见了,不然他走了来回奶奶,叫
二爷要是知道了,咱们二爷那脾气,油锅里的还要捞出来花呢,知道奶奶有了体己,
他还不大着胆子花么?所以我赶着接过来,叫我说了他两句,谁知奶奶偏听见了。
为什么当着二爷我才只说是香菱来了呢!”凤姐听了笑道:“我说呢,姑妈知道你
二爷来了,忽剌巴儿的打发个屋里人来。原来是你这蹄子闹鬼!”

说着贾琏已进来了,凤姐命摆上酒馔来。夫妻对坐。凤姐虽善饮,却不敢任兴。
正喝着,见贾琏的乳母赵嬷嬷走来。贾琏凤姐忙让吃酒,叫他上炕去。赵嬷嬷执意
不肯。平儿等早于炕沿设下一几,摆一脚踏,赵嬷嬷在脚踏上坐了,贾琏向桌上拣
两盘肴馔与他,放在几上自吃。凤姐又道:“妈妈很嚼不动那个,没的倒硌了他的
牙。”因问平儿道:“早起我说那一碗火腿炖肘子很烂,正好给妈妈吃,你怎么不
拿了去赶着叫他们热来?”又道:“妈妈,你尝一尝你儿子带来的惠泉酒。”赵嬷
嬷道:“我喝呢。奶奶也喝一钟怕什么,只不要过多了就是了。我这会子跑了来倒
也不为酒饭,倒有一件正经事,奶奶好歹记在心里,疼顾我些罢!我们这爷,只是
嘴里说的好,到了跟前就忘了我们。幸亏我从小儿奶了你这么大。我也老了,有的
是那两个儿子,你就另眼照看他们些,别人也不敢呲牙儿的。我还再三的求了你几
遍,你答应的倒好,如今还是落空。这如今又从天上跑出这样一件大喜事来,那里
用不着人?所以倒是来和奶奶说是正经。靠着我们爷,只怕我还饿死了呢!”凤姐
笑道:“妈妈,你的两个奶哥哥都交给我。你从小儿奶的儿子还有什么不知他那脾
气的?拿着皮肉,倒往那不相干的外人身上贴。可是现放着奶哥哥那一个不比人强?
你疼顾照看他们,谁敢说个‘不’字儿?没的白便宜了外人。我这话也说错了:我
们看着是‘外人’,你却看着是‘内人’一样呢!”说着,满屋里人都笑了。赵嬷
嬷也笑个不住,又念佛道:“可是屋子里跑出青天来了。要说‘内人’‘外人’这
些混帐事,我们爷是没有的;不过是脸软心慈,搁不住人求两句罢了。”凤姐笑道:
“可不是呢,有‘内人’的他才慈软呢!他在咱们娘儿们跟前才是刚硬呢!”赵嬷
嬷道:“奶奶说的太尽情了,我也乐了,再喝一钟好酒。从此我们奶奶做了主,我
就没的愁了。”

贾琏此时不好意思,只是讪笑道:“你们别胡说了,快盛饭来吃,还要到珍大
爷那边去商量事呢。”凤姐道:“可是,别误了正事,才刚老爷叫你说什么?”贾
琏道:“就为省亲的事。”凤姐忙问道:“省亲的事竟准了?”贾琏笑道:“虽不
十分准,也有八九分了。”凤姐笑道:“可是当今的恩典呢!从来听书听戏,古时
候儿也没有的。”赵嬷嬷又接口道:“可是呢,我也老糊涂了!我听见上上下下吵
嚷了这些日子,什么省亲不省亲,我也不理论;如今又说省亲,到底是怎么个缘故
呢?”贾琏道:“如今当今体贴万人之心,世上至大莫如‘孝’字,想来父母儿女
之性,皆是一理,不在贵贱上分的。当今自为日夜侍奉太上皇、皇太后,尚不能略
尽孝意,因见宫里嫔妃才人等皆是入宫多年,抛离父母,岂有不思想之理?且父母
在家,思想女儿,不能一见,倘因此成疾,亦大伤天和之事。所以启奏太上皇、皇
太后,每月逢二六日期,准椒房眷属入宫请候。于是太上皇、皇太后大喜,深赞当
今至孝纯仁,体天格物,因此二位老圣人又下谕旨,说椒房眷属入宫,未免有关国
体仪制,母女尚未能惬怀。竟大开方便之恩,特降谕诸椒房贵戚,除二六日入宫之
恩外,凡有重宇别院之家,可以驻跸关防者,不妨启请内廷銮舆入其私第,庶可尽
骨肉私情,共享天伦之乐事。此旨下了,谁不踊跃感戴!现今周贵妃的父亲已在家
里动了工,修盖省亲的别院呢。又有吴贵妃的父亲吴天佑家,也往城外踏看地方去
了。这岂非有八九分了?”

赵嬷嬷道:“阿弥陀佛!原来如此。这样说起,咱们家也要预备接大姑奶奶了?”
贾琏道:“这何用说?不么这会子忙的是什么?”凤姐笑道:“果然如此,我可也
见个大世面了。可恨我小几岁年纪,若早生二三十年,如今这些老人家也不薄我没
见世面了。说起当年太祖皇帝仿舜巡的故事,比一部书还热闹,我偏偏的没赶上。”
赵嬷嬷道:“嗳哟!那可是千载难逢的!那时候我才记事儿。咱们贾府正在姑苏扬州
一带监造海船,修理海塘,只预备接驾一次,把银子花的像淌海水似的!说起来—
—”凤姐忙接道:“我们王府里也预备过一次。那时我爷爷专管各国进贡朝贺的事,
凡有外国人来,都是我们家养活。粤、闽、滇、浙所有的洋船货物都是我们家的。”
赵嬷嬷道:“那是谁不知道的?如今还有个俗语儿呢,说:‘东海少了白玉床,龙
王来请金陵王。’这说的就是奶奶府上了。如今还有现在江南的甄家,嗳哟好势派!
独他们家接驾四次。要不是我们亲眼看见,告诉谁也不信的:别讲银子成了粪土,
凭是世上有的,没有不是堆山积海的,‘罪过可惜’四个字竟顾不得了!”凤姐道:
“我常听见我们太爷说,也是这样的。岂有不信的?只纳罕他家怎么就这样富贵
呢?”赵嬷嬷道:“告诉奶奶一句话:也不过拿着皇帝家的银子往皇帝身上使罢了!
谁家有那些钱买这个虚热闹去?”

正说着,王夫人又打发人来瞧凤姐吃完了饭不曾。凤姐便知有事等他,赶忙的
吃了饭,漱口要走,又有二门上小厮们回:“东府里蓉蔷二位哥儿来了。”贾琏才
漱了口,平儿捧着盆盥手,见他二人来了,便问:“说什么话?”凤姐因亦止步,
只听贾蓉先回说:“我父亲打发我来回叔叔:老爷们已经议定了,从东边一带,接
着东府里花园起,至西北,丈量了,一共三里半大,可以盖造省亲别院了。已经传
人画图样去了,明日就得。叔叔才回家,未免劳乏,不用过我们那边去,有话明日
一早再请过去面议。”贾琏笑说:“多谢大爷费心,体谅我,就从命不过去了。正
经是这个主意才省事,盖造也容易;若采置别的地方去,那更费事,且不成体统。
你回去说:这样很好,若老爷们再要改时,全仗大爷谏阻,万不可另寻地方。明日
一早,我给大爷请安去,再细商量。”贾蓉忙应几个“是”。贾蔷又近前回说:“下
姑苏请聘教习,采买女孩子,置办乐器行头等事,大爷派了侄儿,带领着赖管家两
个儿子,还有单聘仁、卜固修两个清客相公,一同前去,所以叫我来见叔叔。”贾
琏听了,将贾蔷打量了打量,笑道:“你能够在行么?这个事虽不甚大,里头却有
藏掖的。”贾蔷笑道:“只好学着办罢咧。”

贾蓉在灯影儿后头悄悄的拉凤姐儿的衣裳襟儿,凤姐会意,也悄悄的摆手儿佯
作不知。因笑道:“你也太操心了!难道大爷比咱们还不会用人?偏你又怕他不在行
了。谁都是在行的?孩子们这么大了,‘没吃过猪肉,也见过猪跑’。大爷派他去,
原不过是个坐纛旗儿,难道认真的叫他讲价钱会经纪去呢。依我说,很好。”贾琏
道:“这是自然。不是我驳回,少不得替他筹算筹算。”因问:“这一项银子动那
一处的?”贾蔷道:“刚才也议到这里。赖爷爷说:竟不用从京里带银子去。江南
甄家还收着我们五万银子。明日写一封书信会票我们带去,先支三万两,剩二万存
着,等置办彩灯花烛并各色帘帐的使用。”贾琏点头道:“这个主意好。”凤姐忙
向贾蔷道:“既这么着,我有两个妥当人,你就带了去办。这可便宜你。”贾蔷忙
陪笑道:“正要和婶娘讨两个人呢,这可巧了。”因问名字。凤姐便问赵嬷嬷。彼
时赵嬷嬷已听呆了,平儿笑着推他,才醒悟过来,忙说:“一个叫赵天梁,一个叫
赵天栋。”凤姐道:“可别忘了,我干我的去了。”说着便出去了。贾蓉忙跟出来,
悄悄的笑向凤姐道:“你老人家要什么,开个帐儿带去,按着置办了来。”凤姐笑
着啐道:“别放你娘的屁!你拿东西换我的人情来了吗?我很不希罕你那鬼鬼祟祟
的!”说着,一笑走了。

这里贾蔷也问贾琏:“要什么东西,顺便织来孝敬。”贾琏笑道:“你别兴头。
才学着办事,倒先学会了这把戏。短了什么,少不得写信来告诉你。”说毕,打发
他二人去了。接着回事的人不止三四起,贾琏乏了,便传与二门上,一应不许传报,
俱待明日料理。凤姐至三更时分方下来安歇。一宿无话。

次早贾琏起来,见过贾赦贾政,便往宁国府中来,合同老管事的家人等并几位
世交门下清客相公们,审察两府地方,缮画省亲殿宇,一面参度办理人丁。自此后,
各行匠役齐全,金银铜锡以及土木砖瓦之物,搬运移送不歇。先令匠役拆宁府会芳
园的墙垣楼阁,直接入荣府东大院中。荣府东边所有下人一带群房已尽拆去。当日
宁荣二宅,虽有一条小巷界断不通,然亦系私地,并非官道,故可以联络。会芳园
本是从北墙角下引了来的一股活水,今亦无烦再引。其山树木石虽不敷用,贾赦住
的乃是荣府旧园,其中竹树山石以及亭榭栏杆等物,皆可挪就前来。如此两处又甚
近便,凑成一处,省许多财力,大概算计起来,所添有限。全亏一个胡老名公号山
子野,一一筹画起造。

贾政不惯于俗务,只凭贾赦、贾珍、贾琏、赖大、赖升、林之孝、吴新登、詹
光、程日兴等几人安插摆布。堆山凿池,起楼竖阁,种竹栽花,一应点景,又有山
子野制度,下朝闲暇,不过各处看望看望,最要紧处和贾赦等商议商议便罢了。贾
赦只在家高卧,有芥豆之事,贾珍等或自去回明,或写略节,或有话说,便传呼贾
琏赖大等来领命。贾蓉单管打造金银器皿。贾蔷已起身往姑苏去了。贾珍赖大等又
点人丁,开册籍,监工等事,一笔不能写到,不过是喧阗热闹而已,暂且无话。

且说宝玉近因家中有这等大事,贾政不来问他的书,心中自是畅快;无奈秦钟
之病日重一日,也着实悬心,不能快乐。这日一早起来,才梳洗了,意欲回了贾母
去望候秦钟,忽见茗烟在二门影壁前探头缩脑。宝玉忙出来问他:“做什么?”茗
烟道:“秦大爷不中用了!”宝玉听了,吓了一跳,忙问道:“我昨儿才瞧了他还
明明白白的,怎么就不中用了呢?”茗烟道:“我也不知道,刚才是他家的老头子
来特告诉我的。”宝玉听毕,忙转身回明贾母。贾母吩咐:“派妥当人跟去,到那
里尽一尽同窗之情就回来,不许多耽搁了。”宝玉忙出来更衣。到外边,车犹未备,
急的满厅乱转。一时催促的车到,忙上了车,李贵茗烟等跟随。来至秦家门首,悄
无一人,遂蜂拥至内室,吓的秦钟的两个远房婶娘、嫂子并几个姐妹,都藏之不迭。

此时秦钟已发过两三次昏,易箦多时矣。宝玉一见,便不禁失声的哭起来。李
贵忙劝道:“不可,秦哥儿是弱症,怕炕上硌的不受用,所以暂且挪下来松泛些。
哥儿这一哭,倒添了他的病了。”宝玉听了,方忍住近前,见秦钟面如白蜡,合目
呼吸,展转枕上。宝玉忙叫道:“鲸哥!宝玉来了。”连叫了两三声,秦钟不睬。
宝玉又叫道:“宝玉来了。”

那秦钟早已魂魄离身,只剩得一口悠悠馀气在胸,正见许多鬼判持牌提索来捉
他。那秦钟魂魄那里肯就去?又记念着家中无人管理家务,又惦记着智能儿尚无下
落,因此百般求告鬼判。无奈这些鬼判都不肯徇私,反叱咤秦钟道:“亏你还是读
过书的人,岂不知俗语说的:‘阎王叫你三更死,谁敢留人到五更。’我们阴间上
下都是铁面无私的,不比阳间瞻情顾意,有许多的关碍处。”正闹着,那秦钟的魂
魄忽听见“宝玉来了”四字,便忙又央求道:“列位神差略慈悲慈悲,让我回去和
一个好朋友说一句话,就来了。”众鬼道:“又是什么好朋友?”秦钟道:“不瞒
列位:就是荣国公的孙子,小名儿叫宝玉的。”那判官听了,先就唬的慌张起来,
忙喝骂那些小鬼道:“我说你们放了他回去走走罢,你们不依我的话。如今闹的请
出个运旺时盛的人来了。怎么好?”众鬼见都判如此,也都忙了手脚,一面又抱怨
道:“你老人家先是那么‘雷霆火炮’,原来见不得‘宝玉’二字。依我们想来,
他是阳间,我们是阴间,怕他亦无益。”那都判越发着急,吆喝起来。

毕竟秦钟死活如何,且听下回分解。