\chapter{大观园月夜警幽魂~散花寺神签惊异兆}

却说凤姐回至房中,见贾琏尚未回来,便分派那管办探春行李妆奁事的一干人。
那天有黄昏以后,因忽然想起探春来,要瞧瞧他去,便叫丰儿与两个丫头跟着,头
里一个丫头打着灯笼。走出门来,见月光已上,照耀如水,凤姐便命:“打灯笼的
回去罢。”因而走至茶房窗下,听见里面有人嘁嘁喳喳的,又似哭,又似笑,又似
议论什么的。凤姐知道不过是家下婆子们又不知搬什么是非,心内大不受用,便命
小红:“进去装做无心的样子,细细打听着,用话套出原委来。”小红答应着去了。

凤姐只带着丰儿来至园门前,门尚未关,只虚虚的掩着。于是主仆二人方推门
进去。只见园中月色比外面更觉明朗,满地下重重树影,杳无人声,甚是凄凉寂静。
刚欲往秋爽斋这条路来,只听唿唿的一声风过,吹的那树枝上落叶,满园中唰喇喇
的作响,枝梢上吱娄娄的发哨,将那些寒鸦宿鸟都惊飞起来。凤姐吃了酒,被风一
吹,只觉身上发噤。丰儿后面也把头一缩,说:“好冷!”凤姐也掌不住,便叫丰儿:
“快回去把那件银鼠坎肩儿拿来,我在三姑娘那里等着。”丰儿巴不得一声,也要
回去穿衣裳,连忙答应一声,回头就跑了。

凤姐刚举步走了不远,只觉身后哧哧似有闻嗅之声,不觉头发森然直竖起
来。由不得回头一看,只见黑油油一个东西在后面伸着鼻子闻他呢,那两只眼睛恰
似灯光一般。凤姐吓的魂不附体,不觉失声的了一声,却是一只大狗。那狗抽头
回身,拖着个扫帚尾巴,一气跑上大土山上,方站住了,回身犹向凤姐拱爪儿。凤
姐此时肉跳心惊,急急的向秋爽斋来。将已来至门口,方转过山子,只见迎面有一
个人影儿一恍。凤姐心中疑惑,还想着必是那一房的丫头,便问:“是谁?”问了
两声,并没有人出来,早已神魂飘荡了。恍恍忽忽的似乎背后有人说道:“婶娘连
我也不认得了?”凤姐忙回头一看,只见那人形容俊俏,衣履风流,十分眼熟,只
是想不起是那房那屋里的媳妇来。只听那人又说道:“婶娘只管享荣华、受富贵的
心盛,把我那年说的‘立万年永远之基’,都付于东洋大海了!”凤姐听说,低头寻
思,总想不起。那人冷笑道:“婶娘那时怎样疼我来,如今就忘在九霄云外了?”
凤姐听了,此时方想起来是贾蓉的先妻秦氏,便说道:“嗳呀!你是死了的人哪,怎
么跑到这里来了呢?”啐了一口,方转回身要走时,不防一块石头绊了一跤,犹如
梦醒一般,浑身汗如雨下。虽然毛发悚然,心中却也明白,只见小红丰儿影影绰绰
的来了。凤姐恐怕落人的褒贬,连忙爬起来,说道:“你们做什么呢,去了这半天?
快拿来我穿上罢。”一面丰儿走至跟前,伏侍穿上。小红过来搀扶着要往前走,凤
姐道:“我才到那里,他们都睡了,回去罢。”一面说着,一面带了两个丫头,急急
忙忙回到家中。贾琏已回来了,凤姐见他脸上神色更变,不似往常,待要问他,又
知他素日性格,不敢突然相问,只得睡了。

至次日五更贾琏就起来,要往总理内庭都检点太监裘世安家来打听事务。因太
早了,见桌上有昨日送来的抄报,便拿起来闲看。第一件:“吏部奏请急选郎中,
奉旨照例用事。”第二件是:“刑部题奏云南节度使王忠一本:新获私带神枪火药出
边事,共十八名人犯,头一名鲍音,系太师镇国公贾化家人。”贾琏想了一想,又
往下看。第三件:“苏州刺史李孝一本:参劾纵放家奴,倚势凌辱军民,以致因奸
不遂,杀死节妇事。凶犯姓时,名福,自称系世袭三等职衔贾范家人。”贾琏看见
这一件,心中不自在起来,待要往下看,又恐迟了不能见裘世安的面,便穿了衣服。
也等不得吃东西,恰好平儿端上茶来,喝了两口,便出来骑马走了。平儿收拾了换
下的衣服。

此时凤姐尚未起来,平儿因说道:“今儿夜里我听着奶奶没睡什么觉,我替奶
奶捶着,好生打个盹儿罢。”凤姐也不言语。平儿料着这意思是了,便爬上炕来,
坐在身边,轻轻的捶着。那凤姐刚有要睡之意,只听那边大姐儿哭了,凤姐又将眼
睁开。平儿连向那边叫道:“李妈,你到底是怎么着?姐儿哭了,你到底拍着他些。
你也忒爱睡了。”那边李妈从梦中惊醒,听得平儿如此说,心中没好气,狠命的拍
了几下,口里嘟嘟囔囔的骂道:“真真的小短命鬼儿,放着尸不挺,三更半夜嚎你
娘的丧!”一面说,一面咬牙,便向那孩子身上拧了一把。那孩子“哇”的一声大
哭起来。凤姐听见,说:“了不得!你听听,他该挫磨孩子了!你过去把那黑心的养
汉老婆下死劲的打他几下子,把妞妞抱过来罢。”平儿笑道:“奶奶别生气,他那里
敢挫磨妞儿?只怕是不堤防碰了一下子也是有的。这会子打他几下子没要紧,明儿
叫他们背地里嚼舌根,倒说三更半夜的打人了。”凤姐听了,半日不言语,长叹一
声,说道:“你瞧瞧,这会子不是我十旺八旺的呢!明儿我要是死了,撂下这小孽障,
还不知怎么样呢。”平儿笑道:“奶奶这是怎么说。大五更的,何苦来呢?”凤姐冷
笑道:“你那里知道?我是早已明白了,我也不久了。虽然活了二十五岁,人家没见
的也见了,没吃的也吃了,衣禄食禄也算全了,所有世上有的也都有了,气也赌尽
了,强也算争足了,就是‘寿’字儿上头缺一点儿也罢了。”平儿听说,由不的眼
圈儿红了。凤姐笑道:“你这会子不用假慈悲,我死了,你们只有喜欢的。你们一
心一计和和气气的过日子,省的我是你们眼里的刺。只有一件,你们知好歹,只疼
我那孩子就是了。”平儿听了,越发掉下泪来。凤姐笑道:“别扯你娘的臊!那里就
死了呢?这么早就哭起来!我不死还叫你哭死了呢。”平儿见说,连忙止住哭,道:“奶
奶说的这么叫人伤心。”一面说,一面又捶,凤姐才蒙的睡着。

平儿方下炕来,只听外面脚步响。谁知贾琏去迟了,那裘世安已经上朝去了,
不遇而回,心中正没好气,进来就问平儿道:“他们还没起来呢么?”平儿回说:“没
有呢。”贾琏一路摔帘子进来,冷笑道:“好啊!这会子还都不起来,安心打擂台打
撒手儿!”一叠声又要吃茶。平儿忙倒了一碗茶来。原来那些丫头老婆见贾琏出了
门,又复睡了,不打量这会子回来,原不曾预备,平儿便把温过的拿了来。贾琏生
气,举起碗来,哗啷一声摔了个粉碎。凤姐惊醒,唬了一身冷汗,“嗳哟”一声,
睁开眼,只见贾琏气狠狠的坐在傍边,平儿弯着腰拾碗片子呢。凤姐道:“你怎么
就回来了?”问了一声,半日不答应,只得又问一声。贾琏嚷道:“你不要我回来,
叫我死在外头罢?”凤姐笑道:“这又是何苦来呢。常时我见你不像今儿回来的快,
问你一声儿,也没什么生气的。”贾琏又嚷道:“又没遇见,怎么不快回来呢!”凤
姐笑道:“没有遇见,少不得耐烦些,明儿再去早些儿,自然遇见了。”贾琏嚷道:
“我可不‘吃着自己的饭,替人家赶獐子’呢。我这里一大堆的事,没个动秤儿的,
没来由为人家的事瞎闹了这些日子,当什么呢!正经那有事的人还在家里受用,死
活不知,还听见说要锣鼓喧天的摆酒唱戏做生日呢,我可瞎跑他娘的腿子!”一面
说,一面往地下啐了一口,又骂平儿。

凤姐听了,气的干咽,要和他分证,想了一想,又忍住了,勉强陪笑道:“何
苦来生这么大气?大清早起,和我叫喊什么?谁叫你应了人家的事?你既应了,只得
耐烦些,少不得替人家办办,也没见这个人自己有为难的事,还有心肠唱戏摆酒的
闹。”贾琏道:“你可说么!你明儿倒也问问他。”凤姐诧异道:“问谁?”贾琏道:“问
谁!问你哥哥!”凤姐道:“是他吗?”贾琏道:“可不是他,还有谁呢?”凤姐忙问
道:“他又有什么事,叫你替他跑?”贾琏道:“你还在坛子里呢。”凤姐道:“真真
这就奇了,我连一个字儿也不知道。”贾琏道:“你怎么能知道呢,这个事,连太太
和姨太太还不知道呢。头一件,怕太太和姨太太不放心;二则你身上又常嚷不好:
所以我在外头压住了,不叫里头知道。说起来,真真可人恼!你今儿不问我,我也
不便告诉你。你打量你哥哥行事像个人呢,你知道外头的人都叫他什么?”凤姐道:
“叫他什么?”贾琏道:“叫他什么?叫他‘忘仁’!”凤姐扑哧的一笑:“他可不叫
王仁,叫什么呢?”贾琏道:“你打量那个‘王仁’吗?是忘了仁义礼智信的那个‘忘
仁’哪。”凤姐道:“这是什么人这么刻薄嘴儿遭塌人!”贾琏道:“不是遭塌他呀。
今儿索性告诉你,你也该知道知道你那哥哥的好处,到底知道他给他二叔做生日
呵!”凤姐想了一想道:“嗳哟,可是呵,我还忘了问你:二叔不是冬天的生日吗?
我记得年年都是宝玉去。前者老爷升了,二叔那边送过戏来,我还偷偷儿的说:‘二
叔为人是最啬刻的,比不得大舅太爷。他们各自家里还乌眼鸡似的。不么,昨儿大
舅太爷没了,你瞧他是个兄弟,他还出了个头儿揽了个事儿吗?’所以那一天说赶
他的生日,咱们还他一班子戏,省了亲戚跟前落亏欠。如今这么早就做生日,也不
知是什么意思。”贾琏道:“你还作梦呢。你哥哥一到京,接着舅太爷的首尾就开了
一个吊。他怕咱们知道拦他,所以没告诉咱们,弄了好几千银子。后来二舅嗔着他,
说他不该一网打尽。他吃不住了,变了个法儿,指着你们二叔的生日撒了个网,想
着再弄几个钱,好打点二舅太爷不生气。也不管亲戚朋友冬天夏天的,人家知道不
知道,这么丢脸!你知道我起早为什么?如今因海疆的事情,御史参了一本,说是大
舅太爷的亏空,本员已故,应着落其弟王子胜、侄儿王仁赔补。爷儿两个急了,找
了我给他们托人情。我见他们吓的那个样儿,再者又关系太太和你,我才应了。想
着找找总理内庭都检点老裘替办办,或者前任后任挪移挪移,偏又去晚了,他进里
头去了。我白起来跑了一趟。他们家里还那里定戏摆酒呢,你说说叫人生气不生
气?”

凤姐听了,才知王仁所行如此,但他素性要强护短,听贾琏如此说,便道:“凭
他怎么样,到底是你的亲大舅儿。再者,这件事,死的大爷、活的二叔都感激你。
罢了,没什么说的,我们家的事,少不得我低三儿下四的求你,省了带累别人受气,
背地里骂我。”说着,眼泪便下来了,掀开被窝,一面坐起来,一面挽头发,一面
披衣裳。贾琏道:“你倒不用这么着,是你哥哥不是人,我并没说你什么。况且我
出去了,你身上又不好,我都起来了,他们还睡着:咱们老辈子有这个规矩么?你
如今作好好先生,不管事了。我说了一句你就起来,明儿我要嫌这些人,难道你都
替了他们么?好没意思啊。”凤姐听了这些话,才把泪止住了,说道:“天也不早了,
我也该起来了。你有这么说的,你替他们家在心的办办,那就是你的情分了。再者
也不光为我,就是太太听见也喜欢。”贾琏道:“是了,知道了。‘大萝卜还用屎
浇’?”平儿道:“奶奶这么早起来做什么?那一天奶奶不是起来有一定的时候儿呢?
爷也不知是那里的邪火,拿着我们出气,何苦来呢。奶奶也算替爷挣够了,那一点
儿不是奶奶挡头阵?不是我说,爷把现成儿的也不知吃了多少,这会子替奶奶办了
一点子事,况且关会着好几层儿呢,就这么拿糖作醋的起来,也不怕人家寒心?况
且这也不单是奶奶的事呀。我们起迟了,原该爷生气,左右到底是奴才呀。奶奶跟
前尽着身子累的成了个病包儿了,这是何苦来呢!”说着,自己的眼圈儿也红了。
那贾琏本是一肚子闷气,那里见得这一对娇妻美妾又尖利又柔情的话呢,便笑道:
“够了,算了罢。他一个人就够使的了,不用你帮着。左右我是外人,多早晚我死
了,你们就清净了。”凤姐道:“你也别说那个话,谁知道谁怎么样呢?你不死,我
还死呢,早死一天早心净。”说着,又哭起来。平儿只得又劝了一回。

那时天已大亮,日影横窗,贾琏也不便再说,站起来出去了。这里凤姐自己起
来,正在梳洗,忽见王夫人那边小丫头过来道:“太太说了,叫问二奶奶今日过舅
太爷那边去不去?要去,说叫二奶奶同着宝二奶奶一路去呢。”凤姐因方才一段话已
经灰心丧意,恨娘家不给争气;又兼昨夜园中受了那一惊,也实在没精神,便说道:
“你先回太太去:我还有一两件事没办清,今日不能去,况且他们那又不是什么正
经事。宝二奶奶要去,各自去罢。”小丫头答应着回去回复了,不在话下。

且说凤姐梳了头,换了衣服,想了想:虽然自己不去,也该带个信儿;再者宝
钗还是新媳妇出门子,自然要过去照应照应的。于是见过王夫人,支吾了一件事,
便过来到宝玉房中。只见宝玉穿着衣服,歪在炕上,两个眼睛呆呆的看宝钗梳头。
凤姐站在门口,还是宝钗一回头看见了,连忙起身让坐。宝玉也爬起来,凤姐才笑
嘻嘻的坐下。宝钗因说麝月道:“你们瞧着二奶奶进来,也不言语声儿。”麝月笑着
道:“二奶奶头里进来就摆手儿不叫言语么。”凤姐因向宝玉道:“你还不走,等什
么呢?没见这么大人了,还是这么小孩子气。人家各自梳头,你爬在傍边看什么?成
日家一块子在屋里,还看不够吗?也不怕丫头们笑话。”说着,“哧”的一笑,又瞅
着他咂嘴儿。宝玉虽也有些不好意思,还不理会;把个宝钗直臊的满脸飞红,又不
好听着,又不好说什么。只见袭人端过茶来,只得搭讪着,自己递了一袋烟。凤姐
儿笑着站起来接了,道:“二妹妹,你别管我们的事,你快穿衣服罢。”

宝玉一面也搭讪着,找这个弄那个。凤姐道:“你先去罢,那里有个爷们等着
奶奶们一块儿走的理呢。”宝玉道:“我只是嫌我这衣裳不大好,不如前年穿着老太
太给的那件雀金呢好。”凤姐因怄他道:“你为什么不穿?”宝玉道:“穿着太早些。”
凤姐忽然想起,自悔失言。幸亏宝钗也和王家是内亲,只是那些丫头们跟前,已经
不好意思了。袭人却接着说道:“二奶奶还不知道呢,就是穿得,他也不穿了。”凤
姐儿道:“这是什么原故?”袭人道:“告诉二奶奶,真真的我们这位爷行的事都是
天外飞来的。那一年因二舅太爷的生日,老太太给了他这件衣裳,谁知那一天就烧
了。我妈病重了,我没在家。那时候还有晴雯妹妹呢,听见说,病着整给他缝了一
夜,第二天老太太才没瞧出来呢。去年那一天,上学天冷,我叫焙茗拿了去给他披
披,谁知这位爷见了这件衣裳,想起晴雯来了,说了总不穿了,叫我给他收一辈子
呢。”凤姐不等说完,便道:“你提晴雯,可惜了儿的。那孩子模样儿手儿都好,就
只嘴头子利害些。偏偏儿的太太不知听了那里的谣言,活活儿的把个小命儿要了。
还有一件事:那一天,我瞧见厨房里柳家的女人,他女孩儿叫什么五儿,那丫头长
的和晴雯脱了个影儿。我心里要叫他进来,后来我问他妈,他妈说是很愿意。我想
着宝二爷屋里的小红跟了我去,我还没还他呢,就把五儿补过来罢。平儿说:‘太
太那一天说了,凡像那个样儿的都不叫派到宝二爷屋里呢。’我所以也就搁下了。
这如今宝二爷也成了家了,还怕什么呢?不如我就叫他进来。可不知宝二爷愿意不
愿意?要想着晴雯,只瞧见这五儿就是了。”宝玉本要走,听见这些话又呆了。袭人
道:“为什么不愿意?早就要弄进来的,只是因为太太的话说的结实罢了。”凤姐道:
“那么着,明儿我就叫他进来。太太的跟前有我呢。”宝玉听了,喜不自胜,才走
到贾母那边去了。这里宝钗穿衣服。

凤姐儿看他两口儿这般恩爱缠绵,想起贾琏方才那种光景,甚实伤心,坐不住,
便起身向宝钗笑道:“我和你上太太屋里去罢。”笑着出了房门,一同来见贾母。宝
玉正在那里回贾母往舅舅家去。贾母点头说道:“去罢,只是少吃酒,早些回来,
你身子才好些。”宝玉答应着出来,刚走到院内,又转身回来,向宝钗耳边说了几
句,不知什么。宝钗笑道:“是了,你快去罢。”将宝玉催着去了。这里贾母和凤姐
宝钗说了没三句话,只见秋纹进来传说:“二爷打发焙茗回来说,请二奶奶。”宝钗
道:“他又忘了什么,又叫他回来?”秋纹道:“我叫小丫头问了焙茗,说是‘二爷
忘了一句话,二爷叫我回来告诉二奶奶:若是去呢,快些来罢;若不去呢,别在风
地里站着。’”说的贾母凤姐并地下站着的老婆子丫头都笑了。宝钗的脸上飞红,把
秋纹啐了一口,说道:“好个糊涂东西,这也值的这么慌慌张张跑了来说?”秋纹
也笑着回去叫小丫头去骂焙茗。那焙茗一面跑着,一面回头说道:“二爷把我巴巴
儿的叫下马来,叫回来说;我若不说,回来对出来,又骂我了。这会子说了,他们
又骂我。”那丫头笑着跑回来说了。贾母向宝钗道:“你去罢,省了他这么不放心。”
说的宝钗站不住,又被凤姐怄着玩笑,没好意思,才走了。

只见散花寺的姑子大了来了,给贾母请安,见过了凤姐,坐着吃茶。贾母因问
他:“这一向怎么不来?”大了道:“因这几日庙中作好事,有几位诰命夫人不时在
庙里起坐,所以不得空儿来。今日特来回老祖宗:明儿还有一家作好事,不知老祖
宗高兴不高兴?若高兴,也去随喜随喜。”贾母便问:“做什么好事?”大了道:“前
月为王大人府里不干净,见神见鬼的,偏生那太太夜间又看见去世的老爷。因此,
昨日在我庙里告诉我,要在散花菩萨跟前许愿烧香,做四十九天的水陆道场,保佑
家口安宁,亡者升天,生者获福。所以我不得空儿来请老太太的安。”却说凤姐素
日最是厌恶这些事,自从昨夜见鬼,心中总只是疑疑惑惑的,如今听了大了这些话,
不觉把素日的心性改了一半,已有三分信意,便问大了道:“这散花菩萨是谁?他怎
么就能避邪除鬼呢?”大了见问,便知他有些信意,说道:“奶奶要问这位菩萨,
等我告诉你奶奶知道:这个散花菩萨,根基不浅,道行非常,生在西天大树园中。
父母打柴为生。养下菩萨来,头长三角,眼横四目,身长八尺,两手拖地。父母说
这是妖精,便弃在冰山背后了。谁知这山上有一个得道的老猢狲出来打食,看见菩
萨顶上白气冲天,虎狼远避,知道来历非常,便抱回洞中抚养。谁知菩萨带了来的
聪慧,禅也会谈,与猢狲天天谈道参禅,说的天花散漫。到了一千年后,便飞升了。
至今山上犹见谈经之处,天花散漫,所求必灵,时常显圣,救人苦厄。因此世人才
盖了庙,塑了像供奉着。”凤姐道:“这有什么凭据呢?”大了道:“奶奶又来搬驳
了。一个佛爷可有什么凭据呢?就是撒谎,也不过哄一两个人罢咧,难道古往今来
多少明白人都被他哄了不成?奶奶只想,惟有佛家香火历来不绝,他到底是祝国裕
民,有些灵验,人才信服啊。”凤姐听了,大有道理,因道:“既这么着,我明儿去
试试。你庙里可有签?我去求一签。我心里的事,签上批的出来,我从此就信了。”
大了道:“我们的签最是灵的,明儿奶奶去求一签就知道了。”贾母道:“既这么着,
索性等到后日初一,你再去求。”说着,大了吃了茶,到王夫人各房里去请了安,
回去不提。

这里凤姐勉强扎挣着,到了初一清早,令人预备了车马,带着平儿并许多奴仆
来至散花寺。大了带了众姑子接了进去,献茶后,便洗手至大殿上焚香。那凤姐儿
也无心瞻仰圣像,一秉虔诚,磕了头,举起签筒,默默的将那见鬼之事并身体不安
等故,祝告了一回。才摇了三下,只听“唰”的一声,筒中撺出一支签来。于是叩
头拾起一看,只见写着“第三十三签:上上大吉”。大了忙查签簿看时,只见上面
写着:“王熙凤衣锦还乡。”凤姐一见这几个字,吃一大惊,忙问大了道:“古人也
有叫王熙凤的么?”大了笑道:“奶奶最是通今博古的,难道汉朝的王熙凤求官的
这一段事也不晓得?”周瑞家的在旁笑道:“前年李先儿还说这一回书来着,我们
还告诉他重着奶奶的名字,不许叫呢。”凤姐笑道:“可是呢,我倒忘了。”说着,
又瞧底下的,写的是:

去国离乡二十年,于今衣锦返家园。蜂采百花成蜜后,为谁辛苦为谁甜?

行人至。音信迟。讼宜和。婚再议。
看完也不甚明白。大了道:“奶奶大喜,这一签巧得很。奶奶自幼在这里长大,何
曾回南京去过?如今老爷放了外任,或者接家眷来,顺便回家,奶奶可不是‘衣锦
还乡’了?”一面说,一面抄了个签经交与丫头。凤姐也半疑半信的。大了摆了斋
来,凤姐只动了一动,放下了要走,又给了香银。大了苦留不住,只得让他走了。
凤姐回至家中,见了贾母王夫人等,问起签来,命人一解,都欢喜非常:“或者老
爷果有此心,咱们走一趟也好。”凤姐儿见人人这么说,也就信了,不在话下。

却说宝玉这一日正睡午觉,醒来不见宝钗,正要问时,只见宝钗进来。宝玉问
道:“那里去了,半日不见?”宝钗笑道:“我给凤姐姐瞧一回签。”宝玉听说,便
问是怎么样的。宝钗把签帖念了一回,又道:“家中人人都说好的,据我看,这‘衣
锦还乡’四字里头,还有缘故。后来再瞧罢了。”宝玉道:“你又多疑了,妄解圣意。
‘衣锦还乡’四字,从古至今都知道是好的,今儿你又偏生看出缘故来了。依你说,
这‘衣锦还乡’还有什么别的解说?”宝钗正要解说,只见王夫人那边打发丫头过
来请二奶奶,宝钗立刻过去。

未知何事,下回分解。