\chapter{甄家仆投靠贾家门~水月庵掀翻风月案}

却说冯紫英去后,贾政叫门上的人来吩咐道:“今儿临安伯那里来请吃酒,知
道是什么事?”门上的人道:“奴才曾问过,并没有什么喜庆事,不过南安王府里
到了一班小戏子,都说是个名班,伯爷高兴,唱两天戏请相好的老爷们瞧瞧,热闹
热闹。大约不用送礼的。”说着,贾赦过来问道:“明儿二老爷去不去?”贾政道:
“承他亲热,怎么好不去的。”说着,门上进来回道:“衙门里书办来请老爷明日
上衙门。有堂派的事,必得早些去。”贾政道:“知道了。”说着,只见两个管屯
里地租子的家人走来,请了安磕了头旁边站着。贾政道:“你们是郝家庄的?”两
个答应了一声。贾政也不往下问,竟与贾赦各自说了一回话儿散了。

家人等秉着手灯送过贾赦去,这里贾琏便叫那管租的人道:“说你的。”那人
说道:“十月里的租子,奴才已经赶上来了。原是明儿可到,谁知京外拿车,把车
上的东西不由分说都掀在地下。奴才告诉他,说是府里收租子的车,不是买卖车,
他更不管这些。奴才叫车夫只管拉着走,几个衙役就把车夫混打了一顿,硬扯了两
辆车去了。奴才所以先来回报。求爷打发个人到衙门里去要了来才好。再者,也整
治整治这些无法无天的差役才好。爷还不知道呢:更可怜的是那买卖车,客商的东
西全不顾,掀下来赶着就走。那些赶车的但说句话,打的头破血出的。”贾琏听了,
骂道:“这个还了得!”立刻写了一个帖儿,叫家人:“拿去向拿车的衙门里要车
去,并车上东西,若少了一件是不依的。快叫周瑞。”周瑞不在家,又叫旺儿。旺
儿晌午出去了,还没有回来。贾琏道:“这些忘八日的,一个都不在家!他们成年
家吃粮不管事!”因吩咐小厮们:“快给我找去!”说着,也回到自己屋里睡下,
不提。

且说临安伯第二天又打发人来请。贾政告诉贾赦道:“我是衙门里有事。琏儿
要在家等候拿车的事情,也不能去。倒是大老爷带着宝玉应酬一天也罢了。”贾赦
点头道:“也使得。”贾政遣人去叫宝玉,说:“今儿跟大爷到临安伯那里听戏去。”
宝玉喜欢的了不得,便换上衣服,带了焙茗、扫红、锄药三个小子,出来见了贾赦,
请了安,上了车,来到临安伯府里。门上人回进去,一会子出来说:“老爷请。”
于是贾赦带着宝玉走入院内,只见宾客喧阗。贾赦宝玉见了临安伯,又与众宾客都
见过了礼,大家坐着,说笑了一回。只见一个掌班拿着一本戏单,一个牙笏,向上
打了一个千儿,说道:“求各位老爷赏戏。”先从尊位点起,挨至贾赦,也点了一
出。那人回头见了宝玉,便不向别处去,竟抢步上来,打个千儿道:“求二爷赏两
出。”宝玉一见那人,面如傅粉,唇若涂朱,鲜润如出水芙渠,飘扬似临风玉树:
原来不是别人,就是蒋玉函。前日听得他带了小戏儿进京,也没有到自己那里;此
时见了,又不好站起来,只得笑道:“你多早晚来的?”蒋玉函把眼往左右一溜,
悄悄的笑道:“怎么二爷不知道么?”宝玉因众人在坐,也难说话,只得乱点了一
出。蒋玉函去了,便有几个议论道:“此人是谁?”有的说:“他向来是唱小旦的,
如今不肯唱小旦,年纪也大了,就在府里掌班。头里也改过小生。他也攒了好几个
钱,家里已经有两三个铺子,只是不肯放下本业,原旧领班。”有的说:“想必成
了家了。”有的说:“亲还没有定。他倒拿定一个主意,说是人生婚配关系一生一
世的事,不是混闹得的,不论尊卑贵贱,总要配的上他的才能。所以到如今还并没
娶亲。”宝玉暗忖度道:“不知日后谁家的女孩儿嫁他?要嫁着这么样的人才儿,
也算是不辜负了。”

那时开了戏,也有昆腔,也有高腔,也有弋腔、平腔,热闹非常。到了晌午,
便摆开桌子吃酒。又看了一回,贾赦便欲起身。临安伯过来留道:“天色尚早。听
见说琪官儿还有一出《占花魁》,他们顶好的首戏。”宝玉听了,巴不得贾赦不走。
于是贾赦又坐了一会。果然蒋玉函扮了秦小官,伏侍花魁醉后神情,把那一种怜香
惜玉的意思,做得极情尽致。以后对饮对唱,缠绵缱绻。宝玉这时不看花魁,只把
两支眼睛独射在秦小官身上。更加蒋玉函声音响亮,口齿清楚,按腔落板,宝玉的
神魂都唱的飘荡了。直等这出戏煞场后,更知蒋玉函极是情种,非寻常脚色可比。
因想着:“《乐记》上说的是:‘情动于中,故形于声;声成文,谓之音。’所以
知声,知音,知乐,有许多讲究。声音之原,不可不察。诗词一道,但能传情,不
能入骨,自后想要讲究讲究音律。”宝玉想出了神,忽见贾赦起身,主人不及相留。
宝玉没法,只得跟了回来。

到了家中,贾赦自回那边去了。宝玉来见贾政。贾政才下衙门,正向贾琏问起
拿车之事。贾琏道:“今儿叫人拿帖儿去,知县不在家。他的门上说了:‘这是本
官不知道的,并无牌票出去拿车,都是那些混帐东西在外头撒野挤讹头。既是老爷
府里的,我便立刻叫人去追办,包管明儿连车连东西一并送来。如有半点差迟,再
行禀过本官,重重处治。此刻本官不在家,求这里老爷看破些,可以不用本官知道
更好。’”贾政道:“既无官票,到底是何等样人在那里作怪?”贾琏道:“老爷
不知,外头都是这样。想来明儿必定送来的。”贾琏说完下来,宝玉上去见了。贾
政问了几句,便叫他往老太太那里去。

贾琏因为昨夜叫空了家人,出来传唤,那起人都已伺候齐全。贾琏骂了一顿,
叫大管家赖大:“将各行档的花名册子拿来,你去查点查点,写一张谕帖,叫那些
人知道。若有并未告假,私自出去,传唤不到,贻误公事的,立刻给我打了撵出去!”
赖大连忙答应了几个“是”,出来吩咐了一回,家人各自留意。

过不几时,忽见有一个人,头上戴着毡帽,身上穿着一身青布衣裳,脚下穿着
一双撒鞋,走到门上,向众人作了个揖。众人拿眼上上下下打量了他一番,便问他:
“是那里来的?”那人道:“我自南边甄府中来的。并有家老爷手书一封,求这里
的爷们呈上尊老爷。”众人听见他是甄府来的,才站起来让他坐下,道:“你乏了,
且坐坐。我们给你回就是了。”门上一面进来回明贾政,呈上来书。贾政拆书看时,
上写着:

世交夙好,气谊素敦,遥仰帷,不胜依切。弟因菲材获谴,自分万死难偿,
幸邀宽宥,待罪边隅。迄今门户雕零,家人星散。所有奴子包勇,向曾使用,虽无
奇技,人尚悫实。倘使得备奔走,糊口有资,屋乌之爱,感佩无涯矣!专此奉达,
馀容再叙,不宣。年家眷弟甄应嘉顿首。
贾政看完,笑道:“这里正因人多,甄家倒荐人来。又不好却的。”吩咐门上:“叫
他见我,且留他住下,因材使用便了。”

门上出去,带进人来,见贾政,便磕了三个头,起来道:“家老爷请老爷安。”
自己又打个千儿,说:“包勇请老爷安。”贾政回问了甄老爷的好,便把他上下一
瞧。但见包勇身长五尺有零,肩背宽肥,浓眉爆眼,磕额长髯,气色粗黑,垂着手
站着。便问道:“你是向来在甄家的,还是住过几年的?”包勇道:“小的向在甄
家的。”贾政道:“你如今为什么要出来呢?”包勇道:“小的原不肯出来,只是
家老爷再四叫小的出来,说别处你不肯去,这里老爷家里和在咱们自己家里一样
的,所以小的来的。”贾政道:“你们老爷不该有这样事情,弄到这个田地。”包
勇道:“小的本不敢说:我们老爷只是太好了,一味的真心待人,反倒招出事来。”
贾政道:“真心是最好的了。”包勇道:“因为太真了,人人都不喜欢,讨人厌烦
是有的。”贾政笑了一笑道:“既这样,皇天自然不负他的。”包勇还要说时,贾
政又问道:“我听见说你们家的哥儿不是也叫宝玉么?”包勇道:“是。”贾政道:
“他还肯向上巴结么?”包勇道:“老爷若问我们哥儿,倒是一段奇事。哥儿的脾
气也和我家老爷一个样子,也是一味的诚实,从小儿只爱和那些姐妹们在一处玩。
老爷太太也狠打过几次,他只是不改。那一年太太进京的时候儿,哥儿大病了一场,
已经死了半日,把老爷几乎急死,装裹都预备了。幸喜后来好了,嘴里说道:走到
一座牌楼那里,见了一个姑娘,领着他到了一座庙里,见了好些柜子,里头见了好
些册子。又到屋里,见了无数女子,说是都变了鬼怪似的,也有变做骷髅儿的。他
吓急了,就哭喊起来。老爷知他醒过来了,连忙调治,渐渐的好了。老爷仍叫他在
姐妹们一处玩去,他竟改了脾气了:好着时候的玩意儿一概都不要了,惟有念书为
事。就有什么人来引诱他,他也全不动心。如今渐渐的能够帮着老爷料理些家务
了。”贾政默然想了一回,道:“你去歇歇去罢。等这里用着你时,自然派你一个
行次儿。”包勇答应着,退下来,跟着这里人出去歇息不提。

一日贾政早起,刚要上衙门,看见门上那些人在那里交头接耳,好像要使贾政
知道的似的,又不好明回,只管咕咕唧唧的说话。贾政叫上来问道:“你们有什么
事这么鬼鬼祟祟的?”门上的人回道:“奴才们不敢说。”贾政道:“有什么事不
敢说的?”门上的人道:“奴才今儿起来,开门出去,见门上贴着一张白纸,上写
着许多不成事体的字。”贾政道:“那里有这样的事!写的是什么?”门上的人道:
“是水月庵里的腌话。”贾政道:“拿给我瞧。”门上的人道:“奴才本要揭下
来,谁知他贴的结实,揭不下来,只得一面抄,一面洗。刚才李德揭了一张给奴才
瞧,就是那门上贴的话。奴才们不敢隐瞒。”说着,呈上那帖儿。贾政接来看时,
上面写着:
西贝草斤年纪轻,水月庵里管尼僧。
一个男人多少女,窝娼聚赌是陶情。
不肖子弟来办事,荣国府内好声名。

贾政看了,气的头昏目晕,赶着叫门上的人不许声张,悄悄叫人往宁荣两府靠
近的夹道子墙壁上再去找寻。随即叫人去唤贾琏出来。贾琏即忙赶至。贾政忙问道:
“水月庵中寄居的那些女尼女道,向来你也查考查考过没有?”贾琏道:“没有,
一向都是芹儿在那里照管。”贾政道:“你知道芹儿照管得来照管不来?”贾琏道:
“老爷既这么说,想来芹儿必有不妥当的地方儿。”贾政叹道:“你瞧瞧这个帖儿
写的是什么!”贾琏一看道:“有这样事么!”正说着,只见贾蓉走来,拿着一封
书子,写着“二老爷密启”。打开看时,也是无头榜一张,与门上所贴的话相同。
贾政道:“快叫赖大带了三四辆车到水月庵里去,把那些女尼姑女道士一齐拉回来。
不许泄漏,只说里头传唤。”赖大领命去了。

且说水月庵中小女尼女道士等,初到庵中,沙弥与道士原系老尼收管,日间教
他些经忏。以后元妃不用,也便习学得懒惰了。那些女孩子们年纪渐渐的大了,都
也有些知觉了。更兼贾芹也是风流人物,打量芳官等出家,只是小孩子性儿,便去
招惹他们。那知芳官竟是真心,不能上手,便把这心肠移到女尼女道士身上。因那
小沙弥中有个名叫沁香的,和女道士中有个叫做鹤仙的,长的都甚妖娆,贾芹便和
这两个人勾搭上了,闲时便学些丝弦,唱个曲儿。

那时正当十月中旬,贾芹给庵中那些人领了月例银子,便想起法儿来,告诉众
人道:“我为你们领月钱,不能进城,又只得在这里歇着,怪冷的。怎么样?我今
儿带些果子酒,大家吃着乐一夜好不好?”那些女孩子都高兴,便摆起桌子,连本
庵的女尼也叫了来。惟有芳官不来。贾芹喝了几杯,便说道要行令。沁香等道:“我
们都不会,倒不如拳罢。谁输了喝一钟,岂不爽快?”本庵的女尼道:“这天刚
过晌午,混嚷混喝的不像,且先喝几钟,爱散的先散去。谁爱陪芹大爷的,回来晚
上尽子喝去,我也不管。”正说着,只见道婆急忙进来说:“快散了罢!府里赖大
爷来了。”众女尼忙乱收拾,便叫贾芹躲开。贾芹因多喝了几杯,便道:“我是送
月钱来的,怕什么?”话犹未完,已见赖大进来,见这般样子,心里大怒。为的是
贾政吩咐不许声张,只得含糊装笑道:“芹大爷也在这里呢么?”贾芹连忙站起来
道:“赖大爷,你来作什么?”赖大说:“大爷在这里更好。快快叫沙弥道士收拾
上车进城,宫里传呢。”贾芹等不知原故,还要细问。赖大说:“天已不早了,快
快的好赶进城。”众女孩子只得一齐上车。赖大骑着大走骡,押着赶进城,不提。

却说贾政知道这事,气的衙门也不能上了,独坐在内书房叹气。贾琏也不敢走
开。忽见门上的进来禀道:“衙门里今夜该班是张老爷。因张老爷病了,有知会来
请老爷补一班。”贾政正等赖大回来要办贾芹,此时又要该班,心里纳闷,也不言
语。贾琏走上去说道:“赖大是饭后出去的,水月庵离城二十来里,就赶进城也得
二更天。今日又是老爷的帮班,请老爷只管去。赖大来了,叫他押着,也别声张,
等明儿老爷回来再发落。倘或芹儿来了,也不用说明,看他明儿见了老爷怎么样
说。”贾政听来有理,只得上班去了。贾琏抽空才要回到自己房中,一面走着,心
里抱怨凤姐出的主意,欲要埋怨,因他病着,只得隐忍,慢慢的走着。

且说那些下人,一人传十,传到里头,先是平儿知道,即忙告诉凤姐。凤姐因
那一夜不好,恹恹的总没精神,正是惦记铁槛寺的事情。听见“外头贴了匿名揭帖”
的一句话,吓了一跳,忙问:“贴的是什么?”平儿随口答应,不留神,就错说了,
道:“没要紧,是馒头庵里的事情。”凤姐本是心虚,听见“馒头庵的事情”,这
一唬直唬怔了,一句话没说出来,急火上攻,眼前发晕,咳嗽了一阵便歪倒了,两
只眼却只是发怔。平儿慌了,说道:“水月庵里,不过是女沙弥女道士的事,奶奶
着什么急呢?”凤姐听是水月庵,才定了定神,道:“嗳!糊涂东西!到底是水月庵,
是馒头庵呢?”平儿道:“是我头里错听了馒头庵,后来听见不是馒头庵,是水月
庵。我刚才也就说溜了嘴,说成馒头庵了。”凤姐道:“我就知道是水月庵。那馒
头庵与我什么相干。原是这水月庵是我叫芹儿管的,大约刻扣了月钱。”平儿道:
“我听着不像月钱的事,还有些腌话呢。”凤姐道:“我更不管那个。你二爷那
里去了?”平儿说:“听见老爷生气,他不敢走开。我听见事情不好,我吩咐这些
人不许吵嚷,不知太太们知道了没有。就听见说,老爷叫赖大拿这些女孩子去了。
且叫人前头打听打听。奶奶现在病着,依我竟先别管他们的闲事。”正说着,只见
贾琏进来。凤姐欲待问他,见贾琏一脸怒气,暂且装作不知。贾琏没吃完饭,旺儿
来说:“外头请爷呢,赖大回来了。”贾琏道:“芹儿来了没有?”旺儿道:“也
来了。”贾琏便道:“你去告诉赖大说:老爷上班儿去了,把这些个女孩子暂且收
在园里,明日等老爷回来,送进宫去。只叫芹儿在内书房等着我。”旺儿去了。

贾芹走进书房,只见那些下人指指戳戳不知说什么,看起这个样儿来,不像宫
里要人。想着问人,又问不出来。正在心里疑惑,只见贾琏走出来,贾芹便请了安,
垂手侍立,说道:“不知道娘娘宫里即刻传那些孩子们做什么?叫侄儿好赶。幸喜
侄儿今儿送月钱去,还没有走,便同着赖大来了。二叔想来是知道的。”贾琏道:
“我知道什么?你才是明白的呢!”贾芹摸不着头脑儿,也不敢再问。贾琏道:“你
干的好事啊!把老爷都气坏了!”贾芹道:“侄儿没有干什么。庵里月钱是月月给
的,孩子们经忏是不忘的。”贾琏见他不知,又是平素常在一处玩笑的,便叹口气
道:“打嘴的东西,你各自去瞧瞧罢。”便从靴掖儿里头拿出那个揭帖来,扔与他
瞧。贾芹拾来一看,吓得面如土色,说道:“这是谁干的!我并没得罪人,为什么
这么坑我?我一月送钱去,只走一趟,并没有这些事。若是老爷回来,打着问我,
侄儿就屈死了!我母亲知道,更要打死。”说着,见没人在旁边,便跪下央及道:
“好叔叔,救我一救儿罢!”说着,只管磕头,满眼流泪。贾琏想道:“老爷最恼
这些,要是问准了有这些事,这场气也不小,闹出去也不好听。又长那个贴帖儿的
人的志气了,将来咱们的事多着呢。倒不如趁着老爷上班儿,和赖大商量着,要混
过去,就可以没事了。现在没有对证。”想定主意,便说:“你别瞒我。你干的鬼
儿,你打量我都不知道呢。若要完事,除非是老爷打着问你,你只一口咬定没有才
好。没脸的东西!起去罢!”叫人去叫赖大。

不多时,赖大来了,贾琏便和他商量。赖大说:“这芹大爷本来闹的不像了。
奴才今儿到庵里的时候,他们正在那里喝酒呢。帖儿上的话一定是有的。”贾琏道:
“芹儿,你听!赖大还赖你不成?”贾芹此时红涨了脸,一句也不敢言语。还是贾
琏拉着赖大,央他:“护庇护庇罢,只说芹哥儿是在家里找了来的。你带了他去,
只说没有见我。明日你求老爷,也不用问那些女孩子了,竟是叫了媒人来,领了去,
一卖完事。果然娘娘再要的时候儿,咱们再买。”赖大想来,闹也无益,且名声不
好,也就应了。贾琏叫贾芹:“跟了赖大爷去罢!听着他教你,你就跟着他。”说
罢,贾芹又磕了一个头,跟着赖大出去。到了没人的地方儿,又给赖大磕头。赖大
说:“我的小爷,你太闹的不像了。不知得罪了谁,闹出这个乱儿来,你想想,谁
和你不对罢?”贾芹想了一会子,并无不对的人,只得无精打彩,跟着赖大走回。

未知如何抵赖,且听下回分解。