\chapter{博庭欢宝玉赞孤儿~正家法贾珍鞭悍仆}

却说惜春正在那里揣摩棋谱,忽听院内有人叫彩屏,不是别人,却是鸳鸯的声
儿。彩屏出去,同着鸳鸯进来。那鸳鸯却带着一个小丫头,提了一个小黄绢包儿。
惜春笑问道:“什么事?”鸳鸯道:“老太太因明年八十一岁,是个‘暗九’,许下
一场九昼夜的功德,发心要写三千六百五十零一部《金刚经》。这已发出外面人写
了。但是俗说:《金刚经》就像那道家的符壳,《心经》才算是符胆,故此,《金刚
经》内必要插着《心经》,更有功德。老太太因《心经》是更要紧的,观自在又是
女菩萨,所以要几个亲丁奶奶姑娘们写上三百六十五部,如此又虔诚,又洁净。咱
们家中除了二奶奶,头一宗他当家没有空儿,二宗他也写不上来,其馀会写字的,
不论写得多少,连东府珍大奶奶姨娘们都分了去。本家里头自不用说。”惜春听了,
点头道:“别的我做不来,若要写经,我最信心的。你搁下,喝茶罢。”鸳鸯才将那
小包儿搁在桌上,同惜春坐下。彩屏倒了一钟茶来。惜春笑问道:“你写不写?”
鸳鸯道:“姑娘又说笑话了。那几年还好,这三四年来,姑娘还见我拿了拿笔儿么?”
惜春道:“这却是有功德的。”鸳鸯道:“我也有一件事:向来伏侍老太太安歇后,
自己念上米佛,已经念了三年多了。我把这个米收好,等老太太做功德的时候,我
将他衬在里头供佛施食,也是我一点诚心。”惜春道:“这样说来,老太太做了观音,
你就是龙女了。”鸳鸯道:“那里跟得上这个分儿?却是除了老太太,别的也伏侍不
来,不晓得前世什么缘分儿。”说着要走,叫小丫头把小绢包打开,拿出来道:“这
素纸一扎是写《心经》的。”又拿起一子儿藏香道:“这是叫写经时点着写的。”惜
春都应了。

鸳鸯遂辞了出来,同小丫头来至贾母房中,回了一遍,看见贾母与李纨打双陆,
鸳鸯旁边瞧着。李纨的骰子好,掷下去,把老太太的锤打下了好几个去,鸳鸯抿着
嘴儿笑。忽见宝玉进来,手中提了两个细篾丝的小笼子,笼内有几个蝈蝈儿,说道:
“我听说老太太夜里睡不着,我给老太太留下解解闷。”贾母笑道:“你别瞅着你老
子不在家,你只管淘气。”宝玉笑道:“我没有淘气。”贾母道:“你没淘气,不在学
房里念书,为什么又弄这个东西呢?”宝玉道:“不是我自己弄的。前儿因师父叫
环儿和兰儿对对子,环儿对不来,我悄悄的告诉了他。他说了,师父喜欢,夸了他
两句。他感激我的情,买了来孝敬我的。我才拿了来孝敬老太太的。”贾母道:“他
没有天天念书么?为什么对不上来?对不上来,就叫你儒大爷爷打他的嘴巴子,看他
臊不臊!你也够受了,不记得你老子在家时,一叫做诗做词,唬的倒像个小鬼儿似
的?这会子又说嘴了。那环儿小子更没出息,求人替做了,就变着方法儿打点人。
这么点子孩子就闹鬼闹神的也不害臊,赶大了还不知是个什么东西呢。”说的满屋
子人都笑了。

贾母又问道:“兰小子呢,做上来了没有?这该环儿替他了,他又比他小了。是
不是?”宝玉笑道:“他倒没有,却是自己对的。”贾母道:“我不信,不然就也是
你闹了鬼了。如今你还了得,‘羊群里跑出骆驼来了’,就只你大,你又会做文章了!”
宝玉笑道:“实在是他作的,师父还夸他明儿一定有大出息呢。老太太不信,就打
发人叫了他来亲自试试,老太太就知道了。”贾母道:“果然这么着,我才喜欢。我
不过怕你撒谎。既是他做的,这孩子明儿大概还有一点儿出息。”因看着李纨,又
想起贾珠来,又说:“这也不枉你大哥哥死了,你大嫂子拉扯他一场。日后也替你
大哥哥顶门壮户。”说到这里,不禁泪下。李纨听了这话,却也动心,只是贾母已
经伤心,自己连忙忍住泪,笑劝道:“这是老祖宗的馀德,我们托着老祖宗的福罢
咧。只要他应的了老祖宗的话,就是我们的造化了。老祖宗看着也喜欢,怎么倒伤
起心来呢?”因又回头向宝玉道:“宝叔叔明儿别这么夸他,他多大孩子,知道什
么?你不过是爱惜他的意思,他那里懂得。一来二去,眼大心肥,那里还能够有长
进呢?”贾母道:“你嫂子这也说的是。就只他还太小呢,也别逼紧了他;小孩
子胆儿小,一时逼急了,弄出点子毛病来,书倒念不成,把你的工夫都白遭塌了。”
贾母说到这里,李纨却忍不住扑簌簌掉下泪来,连忙擦了。

只见贾环贾兰也都进来给贾母请了安。贾兰又见过他母亲,然后过来,在贾母
傍边侍立。贾母道:“我刚才听见你叔叔说你对的好对子,师父夸你来着。”贾兰也
不言语,只管抿着嘴儿笑。鸳鸯过来说道:“请示老太太,晚饭伺候下了。”贾母道:
“请你姨太太去罢。”琥珀接着便叫人去王夫人那边请薛姨妈。这里宝玉贾环退出,
素云和小丫头们过来把双陆收起,李纨尚等着伺候贾母的晚饭。贾兰便跟着他母亲
站着。贾母道:“你们娘儿两个跟着我吃罢。”李纨答应了。一时,摆上饭来,丫鬟
回来禀道:“太太叫回老太太:姨太太这几天浮来暂去,不能过来回老太太,今日
饭后家去了。”于是贾母叫贾兰在身傍边坐下,大家吃饭,不必细言。

却说贾母刚吃完了饭,盥漱了,歪在床上说闲话儿。只见小丫头子告诉琥珀,
琥珀过来回贾母道:“东府大爷请晚安来了。”贾母道:“你们告诉他:如今他办理
家务乏乏的,叫他歇着去罢。我知道了。”小丫头告诉老婆子们,老婆子才告诉贾
珍,贾珍然后退出。

到了次日,贾珍过来料理诸事。门上小厮陆续回了几件事。又一个小厮回道:
“庄头送果子来了。”贾珍道:“单子呢?”那小厮连忙呈上。贾珍看时,上面写着
不过是时鲜果品,还夹带菜蔬野味若干在内。贾珍看完,问:“向来经管的是谁?”
门上的回道:“是周瑞。”便叫周瑞:“照账点清,送往里头交代。等我把来账抄下
一个底子,留着好对。”又叫:“告诉厨房,把下菜中添几宗,给送果子的来人,照
常赏饭给钱。”周瑞答应了,一面叫人搬至凤姐儿院子里去,又把庄上的账和果子
交代明白。出去了一回儿,又进来回贾珍道:“才刚来的果子,大爷曾点过数目没
有?”贾珍道:“我那里有工夫点这个呢?给了你账,你照账点就是了。”周瑞道:“小
的曾点过,也没有少,也不能多出来。大爷既留下底子,再叫送果子来的人,问问
他这账是真的假的。”贾珍道:“这是怎么说?不过是几个果子罢咧,有什么要紧?我
又没有疑你。”说着,只见鲍二走来磕了一个头,说道:“求大爷原旧放小的在外头
伺候罢。”贾珍道:“你们这又是怎么着?”鲍二道:“奴才在这里又说不上话来。”
贾珍道:“谁叫你说话?”鲍二道:“何苦来在这里做眼睛珠儿?”周瑞接口道:“奴
才在这里经管地租庄子银钱出入,每年也有三五十万来往,老爷太太奶奶们从没有
说过话的,何况这些零星东西?若照鲍二说起来,爷们家里的田地房产都被奴才们
弄完了。”贾珍想道:“必是鲍二在这里拌嘴,不如叫他出去。”因向鲍二说道:“快
滚罢!”又告诉周瑞说:“你也不用说了,你干你的事罢。”二人各自散了。

贾珍正在书房里歇着,听见门上闹的翻江搅海,叫人去查问,回来说道:“鲍
二和周瑞的干儿子打架。”贾珍道:“周瑞的干儿子是谁?”门上的回道:“他叫何
三,本来是个没味儿的,天天在家里吃酒闹事,常来门上坐着。听见鲍二和周瑞拌
嘴,他就插在里头。”贾珍道:“这却可恶!把鲍二和那个什么何三给我一块儿捆起
来。周瑞呢?”门上的回道:“打架时,他先走了。”贾珍道:“给我拿了来。这还
了得了!”众人答应了。正嚷着,贾琏也回来了,贾珍便告诉了一遍。贾琏道:“这
还了得。”又添了人去拿周瑞。周瑞知道躲不过,也找到了。贾珍便叫:“都捆上!”
贾琏便向周瑞道:“你们前头的话也不要紧,大爷说开了,很是了,为什么外头又
打架?你们打架已经使不得,又弄个野杂种什么何三来闹。你不压伏压伏他们,倒
竟走了!”就把周瑞踢了几脚。贾珍道:“单打周瑞不中用。”喝命人把鲍二和何三
各人打了五十鞭子,撵了出去,方和贾琏两个商量正事。

下人背地里便生出许多议论来:也有说贾珍护短的;也有说不会调停的;也有
说他本不是好人,“前儿尤家姐妹弄出许多丑事来,那鲍二不是他调停着二爷叫了
来的吗?这会子又嫌鲍二不济事,必是鲍二的女人伏侍不到了。”人多嘴杂,纷纷不
一。

却说贾政自从在工部掌印,家人中尽有发财的。那贾芸听见了,也要插手弄一
点事儿,便在外头说了几个工头,讲了成数,便买了些时新绣货,要走凤姐儿的门
子。

凤姐正在屋里,听见丫头们说:“大爷二爷都生了气,在外头打人呢。”凤姐听
了,不知何故。正要叫人去问问,只见贾琏已进来了,把外面的事告诉了一遍。凤
姐道:“事情虽不要紧,但这风俗儿断不可长。此刻还算咱们家里正旺的时候儿,
他们就敢打架,以后小辈儿们当了家,他们越发难制伏了。前年我在东府里亲眼见
过焦大吃的烂醉,躺在台阶子底下骂人,不管上上下下,一混汤子的混骂。他虽是
有过功的人,到底主子奴才的名分,也要存点体统儿才好。珍大奶奶不是我说,是
个老实头,个个人都叫他养得无法无天的。如今又弄出一个什么鲍二!我还听见是
你和珍大爷得用的人,为什么今儿又打他呢?”贾琏听了这话刺心,便觉讪讪的,
拿话来支开,借有事,说着就走了。

小红进来回道:“芸二爷在外头要见奶奶。”凤姐一想:“他又来做什么?”便
道:“叫他进来罢。”小红出来,瞅着贾芸微微一笑。贾芸赶忙凑近一步,问道:“姑
娘替我回了没有?”小红红了脸,说道:“我就是见二爷的事多!”贾芸道:“何曾
有多少事能到里头来劳动姑娘呢?就是那一年姑娘在宝二叔房里,我才和姑娘——”
小红怕人撞见,不等说完,连忙问道:“那年我换给二爷的一块绢子,二爷见了没
有?”那贾芸听了这句话,喜的心花俱开,才要说话,只见一个小丫头从里面出来,
贾芸连忙同着小红往里走。两个人一左一右,相离不远。贾芸悄悄的道:“回来我
出来,还是你送出我来。我告诉你,还有笑话儿呢。”小红听了,把脸飞红,瞅了
贾芸一眼,也不答言。和他到了凤姐门口,自己先进去回了,然后出来,掀起帘子
点手儿,口中却故意说道:“奶奶请芸二爷进来呢。”

贾芸笑了一笑,跟着他走进房来,见了凤姐儿,请了安,并说:“母亲叫问好。”
凤姐也问了他母亲好。凤姐道:“你来有什么事?”贾芸道:“侄儿从前承婶娘疼爱,
心上时刻想着,总过意不去。欲要孝敬婶娘。又怕婶娘多想。如今重阳时候,略备
了一点儿东西。婶娘这里那一件没有呢?不过是侄儿一点孝心。只怕婶娘不赏脸。”
凤姐儿笑道:“有话坐下说。”贾芸才侧身坐了,连忙将东西捧着搁在傍边桌上。凤
姐又道:“你不是什么有馀的人,何苦又去花钱?我又不等着使。你今儿来意,是怎
么个想头儿,你倒是实说。”贾芸道:“并没有别的想头儿,不过感念婶娘的恩惠,
过意不去罢咧。”说着,微微的笑了。凤姐道:“不是这么说。你手里窄,我很知道,
我何苦白白儿使你的?你要我收下这个东西,须先和我说明白了。要是这么‘含着
骨头露着肉’的,我倒不收。”贾芸没法儿,只得站起来,陪着笑儿说道:“并不是
有什么妄想:前几日听见老爷总办陵工,侄儿有几个朋友办过好些工程,极妥当的,
要求婶娘在老爷跟前提一提。办得一两种,侄儿再忘不了婶娘的恩典!若是家里用
得着侄儿,也能给婶娘出力。”凤姐道:“若是别的,我却可以作主。至于衙门里的
事,上头呢,都是堂官司员定的;底下呢,都是那些书班衙役们办的:别人只怕插
不上手。连自己的家人,也不过跟着老爷伏侍伏侍,就是你二叔去,亦只是为的是
各自家里的事,他也并不能搀越公事。论家事,这里是踩一头儿撬一头儿的,连珍
大爷还弹压不住。你的年纪儿又轻,辈数儿又小,那里缠的清这些人呢?况且衙门
里头的事差不多儿也要完了,不过吃饭瞎跑。你在家里什么事作不得,难道没了这
碗饭吃不成?我这是实在话,你自己回去想想就知道了。你的情意,我已经领了,
把东西快拿回去,是那里弄来的,仍旧给人家送了去罢。”

正说着,只见奶妈子一大起带了巧姐儿进来。那巧姐儿身上穿得锦团花簇,手
里拿着好些玩意儿,笑嘻嘻走到凤姐身边学舌。贾芸一见,便站起来,笑盈盈的赶
着说道:“这就是大妹妹么?你要什么好东西不要?”那巧姐儿便“哑”的一声哭了。
贾芸连忙退下。凤姐道:“乖乖不怕。”连忙将巧姐揽在怀里,道:“这是你芸大哥
哥,怎么认起生来了?”贾芸道:“妹妹生得好相貌,将来又是个有大造化的。”那
巧姐儿回头把贾芸一瞧,又哭起来,叠连几次。贾芸看这光景坐不住,便起身告辞
要走。凤姐道:“你把东西带了去罢。”贾芸道:“这一点子,婶娘还不赏脸?”凤
姐道:“你不带去,我便叫人送到你家去。芸哥儿,你不要这么着。你又不是外人。
我这里有机会,少不得打发人去叫你;没有事也没法儿,不在乎这些东东西西上的。”
贾芸看见凤姐执意不受,只得红着脸道:“既这么着,我再找得用的东西来孝敬婶
娘罢。”凤姐儿便叫小红:“拿了东西,跟着送出芸哥去。”

贾芸走着,一面心中想道:“人说二奶奶利害,果然利害。一点儿都不漏缝,
真正斩钉截铁!怪不得没有后世。这巧姐儿更怪,见了我好像前世的冤家似的。真
正晦气,白闹了这么一天。”小红见贾芸没得彩头,也不高兴,拿着东西跟出来。
贾芸接过来,打开包儿,拣了两件,悄悄的递给小红。小红不接,嘴里说道:“二
爷别这么着。看奶奶知道了,大家倒不好看。”贾芸道:“你好生收着罢。怕什么,
那里就知道了呢?你若不要,就是瞧不起我了。”小红微微一笑,才接过来,说道:
“谁要你这些东西?算什么呢?”说了这句话,把脸又飞红了。贾芸也笑道:“我也
不是为东西。况且那东西也算不了什么。”说着话儿,两个已走到二门口。贾芸把
下剩的仍旧揣在怀内。小红催着贾芸道:“你先去罢。有什么事情只管来找我。我
如今在这院里了,又不隔手。”贾芸点点头儿,说道:“二奶奶太利害,我可惜不能
常来!刚才我说的话,你横竖心里明白,得了空儿再告诉你罢。”小红满脸羞红,说
道:“你去罢。明儿也常来走走。谁叫你和他生疏呢?”贾芸道:“知道了。”贾芸
说着,出了院门。这里小红站在门口,怔怔的看他去远了,才回来了。

却说凤姐在屋里吩咐预备晚饭,因又问道:“你们熬了粥了没有?”丫鬟们连
忙去问,回来回道:“预备了。”凤姐道:“你们把那南边来的糟东西弄一两碟来罢。”
秋桐答应了,叫丫头们伺候。平儿走来笑道:“我倒忘了:今儿晌午,奶奶在上头
老太太那边的时候,水月庵的师父打发人来,要向奶奶讨两瓶南小菜,还要支用几
个月的月钱,说是身上不受用。我问那道婆来着:‘师父怎么不受用?’他说:‘四
五天了。前儿夜里,因那些小沙弥小道士里头有几个女孩子,睡觉没有吹灯,他说
了几次不听。那一夜,看见他们三更以后灯还点着呢,他便叫他们吹灯。个个都睡
着了,没有人答应,只得自己亲自起来给他们吹灭了。回到炕上,只见有两个人,
一男一女,坐在炕上。他赶着问是谁,那里把一根绳子往他脖子上一套,他便叫起
人来。众人听见,点上灯火,一齐赶来,已经躺在地下,满口吐白沫子。幸亏救醒
了。此时还不能吃东西,所以叫来寻些小菜儿的。’我因奶奶不在屋里,不便给他。
我说:‘奶奶此时没有空儿,在上头呢,回来告诉。’便打发他回去了。刚才听见说
起南菜,方想起来了,不然就忘了。”凤姐听了,呆了一呆,说道:“南菜不是还有
呢,叫人送些去就是了。那银子,过一天叫芹哥来领就是了。”又见小红进来回道:
“刚才二爷差人来,说是今晚城外有事,不能回来,先通知一声。”凤姐道:“是了。”

说着,只听见小丫头从后面喘吁吁的嚷着,直跑到院子里来。外面平儿接着,
还有几个丫头们,咕咕唧唧的说话。凤姐道:“你们说什么呢?”平儿道:“小丫头
子有些胆怯,说鬼话。”凤姐说:“那一个?”小丫头进来。问道:“什么鬼话?”
那丫头道:“我刚才到后边去叫打杂儿的添煤,只听得三间空屋子里哗喇哗喇的响,
我还道是猫儿耗子;又听得嗳的一声,像个人出气儿的似的。我害怕,就跑回来了。”
凤姐骂道:“胡说,我这里断不兴说神说鬼。我从来不信这些个话,快滚出去罢!”
那小丫头出去了。凤姐便叫彩明将一天零碎日用账对过一遍。时已将近二更,大家
又歇了一回,略说些闲话,遂叫各人安歇去罢。凤姐也睡下了。

将近三更,凤姐似睡不睡,觉得身上寒毛一乍,自己惊醒了,越躺着越发起碜
来,因叫平儿秋桐过来作伴。二人也不解何意。那秋桐本来不顺凤姐,后来贾琏因
尤二姐之事不大爱惜他了,凤姐又笼络他,如今倒也安静,只是心里比平儿差多了,
外面情儿。今见凤姐不受用,只得端上茶来。凤姐喝了一口道:“难为你,睡去罢,
只留平儿在这里就够了。”秋桐却要献勤儿,因说道:“奶奶睡不着,倒是我们两个
轮流坐坐也使得。”凤姐一面说,一面睡着了。平儿秋桐看见凤姐已睡,只听得远
远的鸡声叫了,二人方都穿着衣裳略躺了一躺,就天亮了,连忙起来伏侍凤姐梳洗。
凤姐因夜中之事,心神恍惚不宁,只是一味要强,仍然扎挣起来。正坐着纳闷,忽
听个小丫头子在院里问道:“平姑娘在屋里么?”平儿答应了一声。那小丫头掀起
帘子进来,却是王夫人打发过来来找贾琏,说:“外头有人回要紧的官事。老爷才
出了门,太太叫快请二爷过去呢。”凤姐听见,唬了一跳。

未知何事,下回分解。