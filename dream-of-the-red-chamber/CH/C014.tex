\chapter{林如海灵返苏州郡~贾宝玉路谒北静王}

话说宁国府中都总管赖升闻知里面委请了凤姐,因传齐同事人等,说道:“如
今请了西府里琏二奶奶管理内事,倘或他来支取东西,或是说话,小心伺候才好。
每日大家早来晚散,宁可辛苦这一个月,过后再歇息,别把老脸面扔了。那是个有
名的烈货,脸酸心硬,一时恼了不认人的!”众人都道:“说的是。”又有一个笑
道“论理,我们里头也得他来整治整治,都忒不像了。”正说着,只见来旺媳妇拿
了对牌来领呈文经文榜纸,票上开着数目。众人连忙让坐倒茶,一面命人按数取纸。
来旺抱着同来旺媳妇一路来至仪门,方交与来旺媳妇自己抱进去了。

凤姐即命彩明钉造册簿,即时传了赖升媳妇,要家口花名册查看,又限明日一
早传齐家人媳妇进府听差。大概点了一点数目单册,问了赖升媳妇几句话,便坐车
回家。至次日卯正二刻,便过来了。那宁国府中老婆媳妇早已到齐,只见凤姐和赖
升媳妇分派众人执事,不敢擅入,在窗外打听。听见凤姐和赖升媳妇道:“既托了
我,我就说不得要讨你们嫌了。我可比不得你们奶奶好性儿,诸事由得你们。再别
说你们‘这府里原是这么样’的话,如今可要依着我行。错我一点儿,管不得谁是
有脸的,谁是没脸的,一例清白处治。”说罢,便吩咐彩明念花名册,按名一个一
个叫进来看视。一时看完,又吩咐道:“这二十个分作两班,一班十个,每日在内
单管亲友来往倒茶,别的事不用管。这二十个也分作两班,每日单管本家亲戚茶饭,
也不管别的事。这四十个人也分作两班,单在灵前上香、添油、挂幔、守灵、供饭、
供茶、随起举哀,也不管别的事。这四个人专在内茶房收管杯碟茶器,要少了一件,
四人分赔。这四个人单管酒饭器皿,少一件也是分赔。这八个人单管收祭礼。这八
个单管各处灯油、蜡烛、纸札,我一总支了来,交给你们八个人,然后按我的数儿
往各处分派。这二十个每日轮流各处上夜,照管门户,监察火烛,打扫地方。这下
剩的按房分开,某人守某处,某处所有桌椅古玩起,至于痰盒掸子等物,一草一苗,
或丢或坏,就问这看守的赔补。赖升家的每日揽总查看,或有偷懒的,赌钱吃酒打
架拌嘴的,立刻拿了来回我。你要徇情,叫我查出来,三四辈子的老脸,就顾不成
了。如今都有了定规,以后那一行乱了,只和那一行算帐。素日跟我的人,随身俱
有钟表,不论大小事,都有一定的时刻。横竖你们上房里也有时辰钟:卯正二刻我
来点卯;巳正吃早饭;凡有领牌回事,只在午初二刻;戌初烧过黄昏纸,我亲到各
处查一遍,回来上夜的交明钥匙。第二日还是卯正二刻过来,说不得咱们大家辛苦
这几日罢,事完了你们大爷自然赏你们。”

说毕,又吩咐按数发茶叶、油烛、鸡毛掸子、笤帚等物,一面又搬取家伙:桌
围、椅搭、坐褥、毡席、痰盒、脚踏之类。一面交发,一面提笔登记,某人管某处,
某人领物件,开的十分清楚。众人领了去,也都有了投奔,不似先时只拣便宜的做,
剩下苦差没个招揽,各房中也不能趁乱迷失东西。便是人来客往,也都安静了,不
比先前紊乱无头绪:一切偷安窃取等弊,一概都蠲了。

凤姐自己威重令行,心中十分得意。因见尤氏犯病,贾珍也过于悲哀,不大进
饮食,自己每日从那府中熬了各样细粥,精美小菜,令人送过来。贾珍也另外吩咐
每日送上等菜到抱厦内,单预备凤姐。凤姐不畏勤劳,天天按时刻过来,点卯理事,
独在抱厦内起坐,不与众妯娌合群,便有女眷来往也不迎送。

这日乃五七正五日上,那应佛僧正开方破狱,传灯照亡,参阎君,拘都鬼,延
请地藏王,开金桥,引幢幡;那道士们正伏章申表,朝三清,叩玉帝;神僧们行香,
放焰口,拜水忏;又有十二众青年尼僧,搭绣衣,红鞋,在灵前默诵接引诸咒:
十分热闹。那凤姐知道今日的客不少,寅正便起来梳洗。及收拾完备,更衣盥手,
喝了几口奶子,漱口已毕,正是卯正二刻了。来旺媳妇率领众人伺候已久。凤姐出
至厅前,上了车,前面一对明角灯,上写“荣国府”三个大字。来至宁府大门首,
门灯朗挂,两边一色绰灯,照如白昼,白汪汪穿孝家人两行侍立,请车至正门上,
小厮退去,众媳妇上来揭起车帘。凤姐下了车,一手扶着丰儿,两个媳妇执着手把
灯照着,撮拥凤姐进来。宁府诸媳妇迎着请安。凤姐款步入会芳园中登仙阁灵前,
一见棺材,那眼泪恰似断线之珠,滚将下来。院中多少小厮垂手侍立,伺候烧纸。
凤姐吩咐一声:“供茶烧纸。”只听一棒锣鸣,诸乐齐奏,早有人请过一张大圈椅
来,放在灵前。凤姐坐下,放声大哭,于是里外上下男女接声嚎哭。

贾珍、尤氏忙令人劝止,凤姐才止住了哭。来旺媳妇倒茶漱口毕,方起身,别
了族中诸人,自入抱厦来,按名查点。各项人数,俱已到齐,只有迎送亲友上的一
人未到,即令传来。那人惶恐,凤姐冷笑道:“原来是你误了!你比他们有体面,
所以不听我的话!”那人回道:“奴才天天都来的早,只有今儿来迟了一步,求奶
奶饶过初次。”正说着,只见荣国府中的王兴媳妇来了,往里探头儿。凤姐且不发
放这人,却问:“王兴媳妇来作什么?”王兴家的近前说:“领牌取线,打车轿网
络。”说着将帖儿递上,凤姐令彩明念道:“大轿两顶,小轿四顶,车四辆,共用
大小络子若干根,每根用珠儿线若干斤。”凤姐听了数目相合,便命彩明登记,取
荣国府对牌发下。王兴家的去了。

凤姐方欲说话,只见荣国府的四个执事人进来,都是支取东西领牌的,凤姐命
他们要了帖念过,听了一共四件,因指两件道:“这个开销错了,再算清了来领。”
说着将帖子摔下来。那二人扫兴而去。凤姐因见张材家的在旁,便问:“你有什么
事?”张材家的忙取帖子回道:“就是方才车轿围子做成,领取裁缝工银若干两。”
凤姐听了,收了帖子,命彩明登记;待王兴交过,得了买办的回押相符,然后与张
材家的去领。一面又命念那一件,是为宝玉外书房完竣,支领买纸料糊裱,凤姐听
了,即命收帖儿登记,待张材家的缴清再发。

凤姐便说道:“明儿他也来迟了,后儿我也来迟了,将来都没有人了。本来要
饶你,只是我头一次宽了,下次就难管别人了,不如开发了好。”登时放下脸来,
叫:“带出去打他二十板子!”众人见凤姐动怒,不敢怠慢,拉出去照数打了,进
来回覆。凤姐又掷下宁府对牌:“说与赖升,革他一个月的钱粮。”吩咐:“散了
罢。”众人方各自办事去了。那被打的也含羞饮泣而去。彼时荣宁两处领牌交牌人
往来不绝,凤姐又一一开发了。于是宁府中人才知凤姐利害,自此俱各兢兢业业,
不敢偷安,不在话下。

如今且说宝玉因见人众,恐秦钟受委曲,遂同他往凤姐处坐坐。凤姐正吃饭,
见他们来了,笑道:“好长腿子,快上来罢。”宝玉道:“我们偏了。”凤姐道:
“在这边外头吃的,还是那边吃的?”宝玉道:“同那些浑人吃什么!还是那边跟
着老太太吃了来的。”说着,一面归坐。

凤姐饭毕,就有宁府一个媳妇来领牌,为支取香灯,凤姐笑道:“我算着你今
儿该来支取,想是忘了。要终久忘了,自然是你包出来,都便宜了我。”那媳妇笑
道:“何尝不是忘了,方才想起来,再迟一步也领不成了。”说毕,领牌而去。一
时登记交牌,秦钟因笑道:“你们两府里都是这牌,倘别人私造一个,支了银子去,
怎么好?”凤姐笑道:“依你说,都没王法了!”宝玉因道:“怎么咱们家没人来
领牌子支东西?”凤姐道:“他们来领的时候,你还做梦呢。我且问你,你们多早
晚才念夜书呢?”宝玉道:“巴不得今日就念才好。只是他们不快给收拾书房,也
是没法儿。”凤姐笑道:“你请我请儿,包管就快了。”宝玉道:“你也不中用,
他们该做到那里的时候,自然有了。”凤姐道:“就是他们做也得要东西,搁不住
我不给对牌,是难的。”宝玉听说,便猴向凤姐身上立刻要牌,说:“好姐姐,给
他们牌,好支东西去收拾。”凤姐道:“我乏的身上生疼,还搁的住你这么揉搓?
你放心罢,今儿才领了裱糊纸去了,他们该要的还等叫去呢,可不傻了?”宝玉不
信,凤姐便叫彩明查册子给他看。

正闹着,人来回:“苏州去的昭儿来了。”凤姐急命叫进来。昭儿打千儿请安。
凤姐便问:“回来做什么?”昭儿道:“二爷打发回来的。林姑老爷是九月初三巳
时没的。二爷带了林姑娘同送林姑老爷的灵到苏州,大约赶年底回来。二爷打发奴
才来报个信儿请安,讨老太太的示下。还瞧瞧奶奶家里好,叫把大毛衣裳带几件
去。”凤姐道:“你见过别人了没有?”昭儿道:“都见过了。”说毕,连忙退出。
凤姐向宝玉笑道:“你林妹妹可在咱们家住长了。”宝玉道:“了不得,想来这几
日他不知哭的怎么样呢!”说着蹙眉长叹。

凤姐见昭儿回来,因当着人不及细问贾琏,心中七上八下。待要回去,奈事未
毕,少不得耐到晚上回来,又叫进昭儿来,细问一路平安。连夜打点大毛衣服,和
平儿亲自检点收拾,再细细追想所需何物,一并包裹交给昭儿。又细细儿的吩咐昭
儿:“在外好生小心些伏侍,别惹你二爷生气。时常劝他少喝酒,别勾引他认得混
帐女人,——我知道了,回来打折了你的腿!”昭儿笑着答应出去。那时天已四更,
睡下,不觉早又天明,忙梳洗过宁府来。

那贾珍因见发引日近,亲自坐车,带了阴阳生往铁槛寺来踏看寄灵之所,又一
一嘱咐住持色空好生预备新鲜陈设,多请名僧,以备接灵使用。色空忙备晚斋。贾
珍也无心茶饭,因天晚不及进城,就在净室胡乱歇了一夜。次日一早,赶忙的进城
来料理出殡之事,一面又派人先往铁槛寺,连夜另外修饰停灵之处,并厨茶等项,
接灵人口。

凤姐见发引日期在迩,也预先逐细分派料理,一面又派荣府中车轿人从跟王夫
人送殡,又顾自己送殡去占下处。目今正值缮国公诰命亡故,邢王二夫人又去吊祭
送殡;西安郡妃华诞,送寿礼;又有胞兄王仁连家眷回南,一面写家信并带往之物;
又兼迎春染疾,每日请医服药,看医生的启帖,讲论症源,斟酌药案。各事冗杂,
亦难尽述,因此忙的凤姐茶饭无心,坐卧不宁。到了宁府里,这边荣府的人跟着;
回到荣府里,那边宁府的人又跟着。凤姐虽然如此之忙,只因素性好胜,惟恐落人
褒贬,故费尽精神,筹划的十分整齐,于是合族中上下无不称叹。

这日伴宿之夕,亲朋满座,尤氏犹卧于内室,一切张罗款待,都是凤姐一人周
全承应。合族中虽有许多妯娌,也有言语钝拙的,也有举止轻浮的,也有羞口羞脚
不惯见人的,也有惧贵怯官的,越显得凤姐洒爽风流,典则俊雅,真是“万绿丛中
一点红”了,那里还把众人放在眼里?挥霍指示,任其所为。那一夜中灯明火彩,
客送官迎,百般热闹自不用说。至天明吉时,一般六十四名青衣请灵,前面铭旌上
大书:“诰封一等宁国公冢孙妇防护内廷紫禁道御前侍卫龙禁尉享强寿贾门秦氏宜
人之灵柩。”一应执事陈设,皆系现赶新做出来的,一色光彩夺目。宝珠自行未嫁
女之礼,摔丧驾灵,十分哀苦。

那时官客送殡的,有镇国公牛清之孙现袭一等伯牛继宗,理国公柳彪之孙现袭
一等子柳芳,齐国公陈翼之孙世袭三品威镇将军陈瑞文,治国公马魁之孙世袭三品
威远将军马尚德,修国公侯晓明之孙世袭一等子侯孝康,——缮国公诰命亡故,其
孙石光珠守孝不得来,——这六家与荣宁二家,当日所称“八公”的便是。馀者更
有南安郡王之孙,西宁郡王之孙,忠靖侯史鼎,平原侯之孙世袭二等男蒋子宁,定
城侯之孙世袭二等男兼京营游击谢鲲,襄阳侯之孙世袭二等男戚建辉,景田侯之孙
五城兵马司裘良。馀者锦乡伯公子韩奇、神武将军公子冯紫英、陈也俊、卫若兰等,
诸王孙公子,不可枚数。堂客也共有十来顶大轿,三四十顶小轿,连家下大小轿子
车辆,不下百十余乘。连前面各色执事陈设,接连一带摆了有三四里远。

走不多时,路上彩棚高搭,设席张筵,和音奏乐,俱是各家路祭:第一棚是东
平郡王府的祭,第二棚是南安郡王的祭,第三棚是西宁郡王的祭,第四棚便是北静
郡王的祭。原来这四王,当日惟北静王功最高,及今子孙犹袭王爵。现今北静王世
荣年未弱冠,生得美秀异常,性情谦和。近闻宁国府冢孙妇告殂,因想当日彼此祖
父有相与之情,同难同荣,因此不以王位自居,前日也曾探丧吊祭,如今又设了路
奠,命麾下的各官在此伺候,自己五更入朝,公事一毕,便换了素服,坐着大轿,
鸣锣张伞而来,到了棚前落轿,手下各官两旁拥侍,军民人众不得往还。

一时只见宁府大殡浩浩荡荡,压地银山一般从北而至。早有宁府开路传事人报
与贾珍,贾珍急命前面执事扎住,同贾赦贾政三人连忙迎上来,以国礼相见。北静
王轿内欠身,含笑答礼,仍以世交称呼接待,并不自大。贾珍道:“犬妇之丧,累
蒙郡驾下临,荫生辈何以克当。”北静王笑道:“世交至谊,何出此言。”遂回头
令长府官主祭代奠。贾赦等一旁还礼,复亲身来谢。北静王十分谦逊。因问贾政道:
“那一位是衔玉而诞者?久欲一见为快,今日一定在此,何不请来?”贾政忙退下
来,命宝玉更衣,领他前来谒见。

那宝玉素闻北静王的贤德,且才貌俱全,风流跌宕,不为官俗国体所缚,每思
相会,只是父亲拘束,不克如愿。今见反来叫他,自是喜欢。一面走,一面瞥见那
北静王坐在轿内,好个仪表。

不知近前又是怎样,且听下回分解。