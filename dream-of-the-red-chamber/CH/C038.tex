\chapter{林潇湘魁夺菊花诗~薛蘅芜讽和螃蟹咏}

话说宝钗湘云计议已定,一宿无话。次日湘云便请贾母等赏桂花。贾母等都说
道:“倒是他有兴头,须要扰他这雅兴。”至午,果然贾母带了王夫人、凤姐,兼
请薛姨妈等进园来。贾母因问:“那一处好?”王夫人道:“凭老太太爱在那一处,
就在那一处。”凤姐道:“藕香榭已经摆下了。那山坡下两棵桂花开的又好,河里
的水又碧清,坐在河当中亭子上,不敞亮吗?看看水,眼也清亮。”贾母听了,说:
“很好。”说着,引了众人往藕香榭来。原来这藕香榭盖在池中,四面有窗,左右
有回廊,也是跨水接峰,后面又有曲折桥。众人上了竹桥,凤姐忙上来搀着贾母,
口里说道:“老祖宗只管迈大步走,不相干,这竹子桥规矩是硌吱硌吱的。”

一时进入榭中,只见栏杆外另放着两张竹案,一个上面设着杯箸酒具,一个上
头设着茶筅茶具各色盏碟。那边有两三个丫头煽风炉煮茶,这边另有几个丫头也煽
风炉烫酒呢。贾母忙笑问:“这茶想的很好,且是地方东西都干净。”湘云笑道:
“这是宝姐姐帮着我预备的。”贾母道:“我说那孩子细致,凡事想的妥当。”一
面说,一面又看见柱子上挂的墨漆嵌蚌的对子,命湘云念道:
芙蓉影破归兰桨,
菱藕香深泻竹桥。
贾母听了,又抬头看匾,因回头向薛姨妈道:“我先小时,家里也有这么一个亭子,
叫做什么枕霞阁。我那时也只像他姐妹们这么大年纪,同着几个人,天天玩去。谁
知那日一下子失了脚掉下去,几乎没淹死,好容易救上来了,到底叫那木钉把头碰
破了。如今这鬓角上那指头顶儿大的一个坑儿,就是那碰破的。众人都怕经了水,
冒了风,说了不得了,谁知竟好了。”凤姐不等人说,先笑道:“那时要活不得,
如今这么大福可叫谁享呢?可知老祖宗从小儿福寿就不小,神差鬼使,碰出那个坑
儿来,好盛福寿啊。寿星老儿头上原是个坑儿,因为万福万寿盛满了,所以倒凸出
些来了。”未及说完,贾母和众人都笑软了。贾母笑道:“这猴儿惯的了不得了,
拿着我也取起笑儿来了!恨的我撕你那油嘴。”凤姐道:“回来吃螃蟹,怕存住冷
在心里,怄老祖宗笑笑儿,就是高兴多吃两个也无妨了。”贾母笑道:“明日叫你
黑家白日跟着我,我倒常笑笑儿,也不许你回屋里去。”王夫人笑道:“老太太因
为喜欢他,才惯的这么样,还这么说,他明儿越发没理了。”贾母笑道:“我倒喜
欢他这么着,况且他又不是那真不知高低的孩子。家常没人,娘儿们原该说说笑笑,
横竖大礼不错就罢了。没的倒叫他们神鬼似的做什么!”

说着,一齐进了亭子。献过茶,凤姐忙安放杯箸。上面一桌,贾母、薛姨妈、
宝钗、黛玉、宝玉;东边一桌,湘云、王夫人、迎、探、惜;西边靠门一小桌,李
纨和凤姐,虚设坐位,二人皆不敢坐,只在贾母王夫人两桌上伺候。凤姐吩咐:“螃
蟹不可多拿来,仍旧放在蒸笼里,拿十个来,吃了再拿。”一面又要水洗了手,站
在贾母跟前剥蟹肉。头次让薛姨妈,薛姨妈道:“我自己掰着吃香甜,不用人让。”
凤姐便奉与贾母。二次的便与宝玉。又说:“把酒烫得滚热的拿来。”又命小丫头
们去取菊花叶儿桂花蕊熏的绿豆面子,预备着洗手。湘云陪着吃了一个,便下座来
让人,又出至外头,命人盛两盘子给赵姨娘送去。又见凤姐走来道:“你张罗不惯,
你吃你的去,我先替你张罗,等散了我再吃。”湘云不肯,又命人在那边廊上摆了
两席,让鸳鸯、琥珀、彩霞、彩云、平儿去坐。鸳鸯因向凤姐笑道:“二奶奶在这
里伺候,我可吃去了。”凤姐儿道:“你们只管去,都交给我就是了。”说着,湘
云仍入了席。凤姐和李纨也胡乱应了个景儿。

凤姐仍旧下来张罗。一时出至廊上,鸳鸯等正吃得高兴,见他来了,鸳鸯等站
起来道:“奶奶又出来做什么?让我们也受用一会子!”凤姐笑道:“鸳鸯丫头越
发坏了!我替你当差,倒不领情,还抱怨我,还不快斟一钟酒来我喝呢。”鸳鸯笑
着,忙斟了一杯酒,送至凤姐唇边,凤姐一挺脖子喝了。琥珀、彩霞二人也斟上一
杯送至凤姐唇边,那凤姐也吃了。平儿早剔了一壳黄子送来,凤姐道:“多倒些姜
醋。”一回也吃了,笑道:“你们坐着吃罢,我可去了。”鸳鸯笑道:“好没脸!
吃我们的东西!”凤姐儿笑道:“你少和我作怪。你知道你琏二爷爱上了你,要和
老太太讨了你做小老婆呢。”鸳鸯红了脸,咂着嘴,点着头道:“哎,这也是做奶
奶说出来的话!我不拿腥手抹你一脸算不得!”说着站起来就要抹。凤姐道:“好
姐姐!饶我这遭儿罢!”琥珀笑道:“鸳丫头要去了,平丫头还饶他?你们看看,他
没吃两个螃蟹,倒喝了一碟子醋了!”平儿手里正剥了个满黄螃蟹,听如此奚落他,
便拿着螃蟹照琥珀脸上来抹,口内笑骂:“我把你这嚼舌根的小蹄子儿……”琥珀
也笑着往傍边一躲。平儿使空了,往前一撞,恰恰的抹在凤姐腮上。凤姐正和鸳鸯
嘲笑,不防吓了一跳,“嗳哟”了一声,众人掌不住都哈哈的大笑起来。凤姐也禁
不住笑骂道:“死娼妇!吃离了眼了!混抹你娘的!”平儿忙赶过来替他擦了,亲自
去端水。鸳鸯道:“阿弥陀佛!这才是现报呢。”贾母那边听见,一叠连声问:“见
了什么了,这么乐?告诉我们也笑笑。”鸳鸯等忙高声笑回道:“二奶奶来抢螃蟹
吃,平儿恼了,抹了他主子一脸螃蟹黄子:主子奴才打架呢!”贾母和王夫人等听
了,也笑起来。贾母笑道:“你们看他可怜见儿的,那小腿子、脐子给他点子吃罢。”
鸳鸯等笑着答应了,高声的说道:“这满桌子的腿子,二奶奶只管吃就是了。”凤
姐笑着洗了脸,走来又伏侍贾母等吃了一回。

黛玉弱不敢多吃,只吃了一点夹子肉就下来了。贾母一时也不吃了。大家都洗
了手。也有看花的,也有弄水看鱼的,游玩了一回。王夫人因问贾母:“这里风大,
才又吃了螃蟹,老太太还是回屋里去歇歇罢。若高兴,明日再来逛逛。”贾母听了,
笑道:“正是呢。我怕你们高兴,我走了,又怕扫了你们的兴;既这么说,咱们就
都去罢。”回头嘱咐湘云:“别让你宝哥哥多吃了。”湘云答应着。又嘱咐湘云、
宝钗二人说:“你们两个也别多吃了。那东西虽好吃,不是什么好的,吃多了肚子
疼。”二人忙应着。送出园外,仍旧回来,命将残席收拾了另摆。宝玉道:“也不
用摆,咱们且做诗。把那大团圆桌子放在当中,酒菜都放着。也不必拘定坐位,有
爱吃的去吃,大家散坐,岂不便宜?”宝钗道:“这话极是。”湘云道:“虽这么
说,还有别人。”因又命另摆一桌,拣了热螃蟹来,请袭人、紫鹃、司棋、侍书、
入画、莺儿、翠墨等一处共坐。山坡桂树底下铺下两条花毯,命支应的婆子并小丫
头等也都坐了,只管随意吃喝,等使唤再来。

湘云便取了诗题,用针绾在墙上。众人看了,都说:“新奇!只怕做不出来。”
湘云又把不限韵的缘故说了一番,宝玉道:“这才是正理。我也最不喜限韵。”黛
玉因不大吃酒,又不吃螃蟹,自命人掇了一个绣墩,倚栏坐着,拿着钓杆钓鱼。宝
钗手里拿着一枝桂花,玩了一回,俯在窗槛上,掐了桂蕊,扔在水面,引的那游鱼
上来唼喋。湘云出一回神,又让一回袭人等,又招呼山坡下的众人只管放量吃。
探春和李纨、惜春正立在垂柳阴中看鸥鹭。迎春却独在花阴下,拿着个针儿穿茉莉
花。宝玉又看了一回黛玉钓鱼,一回又俯在宝钗傍边说笑两句,一回又看袭人等吃
螃蟹,自己也陪他喝两口酒,袭人又剥一壳肉给他吃。

黛玉放下钓杆,走至座间,拿起那乌梅银花自斟壶来,拣了一个小小的海棠冻
石蕉叶杯。丫头看见,知他要饮酒,忙着走上来斟。黛玉道:“你们只管吃去,让
我自己斟才有趣儿。”说着便斟了半盏看时,却是黄酒,因说道:“我吃了一点子
螃蟹,觉得心口微微的疼,须得热热的吃口烧酒。”宝玉忙接道:“有烧酒。”便
命将那合欢花浸的酒烫一壶来,黛玉也只吃了一口便放下了。宝钗也走过来,另拿
了一只杯来,也饮了一口放下,便蘸笔至墙上把头一个《忆菊》勾了,底下又赘一
个“蘅”字。宝玉忙道:“好姐姐,第二个我已有了四句了,你让我做罢。”宝钗
笑道:“我好容易有了一首,你就忙的这样。”黛玉也不说话,接过笔来把第八个
《问菊》勾了,接着把第十一个《菊梦》也勾了,也赘上了一个“潇”字。宝玉也
拿起笔来将第二个《访菊》也勾了,也赘上一个“怡”字。探春起来看着道:“竟
没人作《簪菊》?让我作。”又指着宝玉笑道:“才宣过:总不许带出闺阁字样来,
你可要留神。”说着,只见湘云走来,将第四第五《对菊》《供菊》一连两个都勾
了,也赘上一个“湘”字。探春道:“你也该起个号。”湘云笑道:“我们家里如
今虽有几处轩馆,我又不住着,借了来也没趣。”宝钗笑道:“方才老太太说,你
们家里也有一个水亭,叫做枕霞阁,难道不是你的?如今虽没了,你到底是旧主人。”
众人都道:“有理。”宝玉不待湘云动手,便代将“湘”字抹了,改了一个“霞”
字。

没有顿饭工夫,十二题已全,各自誊出来,都交与迎春,另拿了一张雪浪笺过
来,一并誊录出来。某人作的底下赘明某人的号。李纨等从头看道:

忆
菊

蘅芜君
怅望西风抱闷思,蓼红苇白断肠时。
空篱旧圃秋无迹,冷月清霜梦有知。
念念心随归雁远,寥寥坐听晚砧迟。
谁怜我为黄花瘦,慰语重阳会有期。

访
菊

怡红公子
闲趁霜晴试一游,酒杯药盏莫淹留。
霜前月下谁家种?槛外篱边何处秋?
蜡屐远来情得得,冷吟不尽兴悠悠。
黄花若解怜诗客,休负今朝挂杖头。

种
菊

怡红公子
携锄秋圃自移来,篱畔庭前处处栽。
昨夜不期经雨活,今朝犹喜带霜开。
冷吟秋色诗千首,醉酹寒香酒一杯。
泉溉泥封勤护惜,好和井径绝尘埃。

对
菊

枕霞旧友
别圃移来贵比金,一丛浅淡一丛深。
萧疏篱畔科头坐,清冷香中抱膝吟。
数去更无君傲世,看来惟有我知音!
秋光荏苒休孤负,相对原宜惜寸阴。

供
菊

枕霞旧友
弹琴酌酒喜堪俦,几案婷婷点缀幽。
隔坐香分三径露,抛书人对一枝秋。
霜清纸帐来新梦,圃冷斜阳忆旧游。
傲世也因同气味,春风桃李未淹留。

咏
菊

潇湘妃子
无赖诗魔昏晓侵,绕篱欹石自沉音。
毫端蕴秀临霜写,口角噙香对月吟。
满纸自怜题素怨,片言谁解诉秋心?
一从陶令评章后,千古高风说到今。

画
菊

蘅芜君
诗馀戏笔不知狂,岂是丹青费较量?
聚叶泼成千点墨,攒花染出几痕霜。
淡浓神会风前影,跳脱秋生腕底香。
莫认东篱闲采掇,粘屏聊以慰重阳。

问
菊

潇湘妃子
欲讯秋情众莫知,喃喃负手扣东篱。
孤标傲世偕谁隐?一样开花为底迟?
圃露庭霜何寂寞?雁归蛩病可相思?
莫言举世无谈者,解语何妨话片时?

簪
菊

蕉下客
瓶供篱栽日日忙,折来休认镜中妆。
长安公子因花癖,彭泽先生是酒狂。
短鬓冷沾三径露,葛巾香染九秋霜。
高情不入时人眼,拍手凭他笑路旁。

菊
影

枕霞旧友
秋光叠叠复重重,潜度偷移三径中。
窗隔疏灯描远近,篱筛破月锁玲珑。
寒芳留照魂应驻,霜印传神梦也空。
珍重暗香踏碎处,凭谁醉眼认朦胧。

菊
梦

潇湘妃子
篱畔秋酣一觉清,和云伴月不分明。
登仙非慕庄生蝶,忆旧还寻陶令盟。
睡去依依随雁断,惊回故故恼蛩鸣。
醒时幽怨同谁诉:衰草寒烟无限情!

残
菊

蕉下客
露凝霜重渐倾欹,宴赏才过小雪时。
蒂有馀香金淡泊,枝无全叶翠离披。
半床落月蛩声切,万里寒云雁阵迟。
明岁秋分知再会,暂时分手莫相思!

众人看一首,赞一首,彼此称扬不绝。李纨笑道:“等我从公评来。通篇看来,
各人有各人的警句。今日公评:《咏菊》第一,《问菊》第二,《菊梦》第三,题
目新,诗也新,立意更新了,只得要推潇湘妃子为魁了。然后《簪菊》、《对菊》、
《供菊》、《画菊》、《忆菊》次之。”宝玉听说,喜的拍手叫道:“极是!极公!”
黛玉道:“我那个也不好,到底伤于纤巧些。”李纨道:“巧的却好,不露堆砌生
硬。”黛玉道:“据我看来,头一句好的是‘圃冷斜阳忆旧游’,这句背面傅粉;
‘抛书人对一枝秋’,已经妙绝,将供菊说完,没处再说,故翻回来想到未折未供
之先,意思深远!”李纨笑道:“固如此说,你的‘口角噙香’一句也敌得过了。”
探春又道:“到底要算蘅芜君沉着,‘秋无迹’,‘梦有知’,把个‘忆’字竟烘
染出来了。”宝钗笑道:“你的‘短鬓冷沾’,‘葛巾香染’,也就把簪菊形容的
一个缝儿也没有。”湘云笑道:“‘偕谁隐’,‘为底迟’,真真把个菊花问的无
言可对!”李纨笑道:“那么着,像‘科头坐’,‘抱膝吟’,竟一时也舍不得离
了菊花,菊花有知,倒还怕腻烦了呢!”说的大家都笑了。宝玉笑道:“这场我又
落第了。难道‘谁家种’,‘何处秋’,‘蜡屐远来’,‘冷吟不尽’,那都不是
访不成?‘昨夜雨’,‘今朝霜’,都不是种不成?但恨敌不上‘口角噙香对月吟’、
‘清冷香中抱膝吟’、‘短鬓’、‘葛巾’、‘金淡泊’、‘翠离披’、‘秋无迹’、
‘梦有知’这几句罢了。”又道:“明日闲了,我一个人做出十二首来。”李纨道:
“你的也好,只是不及这几句新雅就是了。”

大家又评了一回,复又要了热螃蟹来,就在大圆桌上吃了一回。宝玉笑道:“今
日持螯赏桂,亦不可无诗,我已吟成,谁还敢作?”说着,便忙洗了手,提笔写出,
众人看道:
持螯更喜桂阴凉,泼醋擂姜兴欲狂。
饕餮王孙应有酒,横行公子竟无肠!
脐间积冷馋忘忌,指上沾腥洗尚香。
原为世人美口腹,坡仙曾笑一生忙。
黛玉笑道:“这样的诗,一时要一百首也有。”宝玉笑道:“你这会子才力已尽,
不说不能作了,还褒贬人家。”黛玉听了,也不答言,略一仰首微吟,提起笔来一
挥,已有了一首。众人看道:
铁甲长戈死未忘,堆盘色相喜先尝。
螯封嫩玉双双满,壳凸红脂块块香。
多肉更怜卿八足,助情谁劝我千觞?
对兹佳品酬佳节,桂拂清风菊带霜。
宝玉看了,正喝彩时,黛玉便一把撕了,命人烧去,因笑道:“我做的不及你的,
我烧了罢。你那个很好,比方才的菊花诗还好,你留着他给人看看。”

宝钗笑道:“我也勉强了一首,未必好,写出来取笑儿罢。”说着,也写出来。
大家看时,写道:
桂霭桐阴坐举觞,长安涎口盼重阳。
眼前道路无经纬,皮里春秋空黑黄。
看到这里,众人不禁叫绝。宝玉道:“骂得痛快!我的诗也该烧了。”看底下道:
酒未涤腥还用菊,性防积冷定须姜。
于今落釜成何益?月浦空馀禾黍香。
众人看毕,都说:“这方是食蟹的绝唱!这些小题目,原要寓大意思,才算是大才。
只是讽刺世人太毒了些。”说着,只见平儿复进园来。

不知却做什么,且听下回分解。