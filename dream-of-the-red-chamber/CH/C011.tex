\chapter{庆寿辰宁府排家宴~见熙凤贾瑞起淫心}

话说是日贾敬的寿辰,贾珍先将上等可吃的东西、稀奇的果品,装了十六大捧
盒,着贾蓉带领家下人送与贾敬去,向贾蓉说道:“你留神看太爷喜欢不喜欢,你
就行了礼起来,说:‘父亲遵太爷的话,不敢前来,在家里率领合家都朝上行了礼
了。’”贾蓉听罢,即率领家人去了。

这里渐渐的就有人来。先是贾琏、贾蔷来看了各处的座位,并问:“有什么玩
意儿没有?”家人答道:“我们爷算计,本来请太爷今日来家,所以并未敢预备玩
意儿。前日听见太爷不来了,现叫奴才们找了一班小戏儿并一档子打十番的,都在
园子里戏台上预备着呢。”次后邢夫人、王夫人、凤姐儿、宝玉都来了,贾珍并尤
氏接了进去。尤氏的母亲已先在这里,大家见过了,彼此让了坐。贾珍尤氏二人递
了茶,因笑道:“老太太原是个老祖宗,我父亲又是侄儿,这样年纪,这个日子,
原不敢请他老人家来;但是这时候,天气又凉爽,满园的菊花盛开,请老祖宗过来
散散闷,看看众儿孙热热闹闹的,是这个意思。谁知老祖宗又不赏脸。”凤姐儿未
等王夫人开口,先说道:“老太太昨日还说要来呢,因为晚上看见宝兄弟吃桃儿,
他老人家又嘴馋,吃了有大半个,五更天时候就一连起来两次。今日早晨略觉身子
倦些,因叫我回大爷,今日断不能来了,说有好吃的要几样,还要很烂的呢。”贾
珍听了笑道:“我说老祖宗是爱热闹的,今日不来必定有个缘故,这就是了。”

王夫人说:“前日听见你大妹妹说,蓉哥媳妇身上有些不大好,到底是怎么
样?”尤氏道:“他这个病得的也奇。上月中秋还跟着老太太、太太玩了半夜,回
家来好好的。到了二十日以后,一日比一日觉懒了,又懒怠吃东西:这将近有半个
多月。经期又有两个月没来。”邢夫人接着说道:“不要是喜罢?”正说着,外头人
回道:“大老爷、二老爷并一家的爷们都来了,在厅上呢。”贾珍连忙出去了。这里
尤氏复说:“从前大夫也有说是喜的。昨日冯紫英荐了他幼时从学过的一个先生,
医道很好,瞧了说不是喜,是一个大症候。昨日开了方子,吃了一剂药。今日头晕
的略好些,别的仍不见大效。”凤姐儿道:“我说他不是十分支持不住,今日这样日
子,再也不肯不挣扎着上来。”尤氏道:“你是初三日在这里见他的。他强扎挣了半
天,也是因你们娘儿两个好的上头,还恋恋的舍不得去。”凤姐听了,眼圈儿红了
一会子,方说道:“‘天有不测风云,人有旦夕祸福。’这点年纪,倘或因这病上有
个长短,人生在世,还有什么趣儿呢!”

正说着,贾蓉进来,给邢夫人、王夫人、凤姐儿都请了安,方回尤氏道:“方
才我给太爷送吃食去,并说我父亲在家伺候老爷们,款待一家子爷们,遵太爷话,
并不敢来。太爷听了很喜欢,说:‘这才是。’叫告诉父亲母亲,好生伺候太爷太太
们。叫我好生伺候叔叔婶子并哥哥们。还说:‘那《阴骘文》叫他们急急刻出来,
印一万张散人。’我将这话都回了我父亲了。我这会子还得快出去打发太爷们并合
家爷们吃饭。”凤姐儿说:“蓉哥儿,你且站着。你媳妇今日到底是怎么着?”贾蓉
皱皱眉儿说道:“不好呢。婶子回来瞧瞧去就知道了。”于是贾蓉出去了。这里尤氏
向邢夫人王夫人道:“太太们在这里吃饭,还是在园子里吃去?有小戏儿现在园子里
预备着呢。”王夫人向邢夫人道:“这里很好。”尤氏就吩咐媳妇婆子们快摆饭来。
门外一齐答应了一声,都各人端各人的去了。不多时摆上了饭,尤氏让邢夫人王夫
人并他母亲都上坐了,他与凤姐儿宝玉侧席坐了。邢夫人王夫人道:“我们来原为
给大老爷拜寿,这岂不是我们来过生日来了么?”凤姐儿说:“大老爷原是好养静
的,已修炼成了,也算得是神仙了。太太们这么一说,就叫作‘心到神知’了。”
一句话说得满屋子里笑起来。

尤氏的母亲并邢夫人、王夫人、凤姐儿都吃了饭,漱了口,净了手。才说要往
园子里去,贾蓉进来向尤氏道:“老爷们并各位叔叔哥哥们都吃了饭了。大老爷说
家里有事,二老爷是不爱听戏,又怕人闹的慌,都去了。别的一家子爷们被琏二叔
并蔷大爷都让过去听戏去了。方才南安郡王、东平郡王、西宁郡王、北静郡王四家
王爷,并镇国公牛府等六家、忠靖侯史府等八家,都差人持名帖送寿礼来,俱回了
我父亲,收在帐房里。礼单都上了档子了,领谢名帖都交给各家的来人了,来人也
各照例赏过,都让吃了饭去了。母亲该请二位太太、老娘、婶子都过园子里去坐着
罢。”尤氏道:“这里也是才吃完了饭,就要过去了。”凤姐儿说道:“我回太太:我
先瞧瞧蓉哥媳妇儿去,我再过去罢。”王夫人道:“很是。我们都要去瞧瞧,倒怕他
嫌我们闹的慌。说我们问他好罢。”尤氏道:“好妹妹,媳妇听你的话,你去开导开
导他我也放心。你就快些过园子里来罢。”

宝玉也要跟着凤姐儿去瞧秦氏。王夫人道:“你看看就过来罢,那是侄儿媳妇
呢。”于是尤氏请了王夫人邢夫人并他母亲,都过会芳园去了,凤姐儿宝玉方和贾
蓉到秦氏这边来。进了房门,悄悄的走到里间房内,秦氏见了要站起来。凤姐儿说:
“快别起来,看头晕。”于是凤姐儿紧行了两步,拉住了秦氏的手,说道:“我的奶
奶!怎么几日不见,就瘦的这样了!”于是就坐在秦氏坐的褥子上。宝玉也问了好,
在对面椅子上坐了。贾蓉叫:“快倒茶来,婶子和二叔在上房还未吃茶呢。”

秦氏拉着凤姐儿的手,强笑道:“这都是我没福。这样人家,公公婆婆当自家
的女孩儿似的待。婶娘你侄儿虽说年轻,却是他敬我,我敬他,从来没有红过脸儿。
就是一家子的长辈同辈之中,除了婶子不用说了,别人也从无不疼我的,也从无不
和我好的。如今得了这个病,把我那要强心一分也没有。公婆面前未得孝顺一天;
婶娘这样疼我,我就有十分孝顺的心,如今也不能够了!我自想着,未必熬得过年
去。”

宝玉正把眼瞅着那《海棠春睡图》并那秦太虚写的“嫩寒锁梦因春冷,芳气袭
人是酒香”的对联,不觉想起在这里睡晌觉时梦到“太虚幻境”的事来,正在出神。
听得秦氏说了这些话,如万箭攒心,那眼泪不觉流下来了。凤姐儿见了,心中十分
难过,但恐病人见了这个样子反添心酸,倒不是来开导他的意思了,因说:“宝玉,
你忒婆婆妈妈的了。他病人不过是这样说,那里就到这个田地?况且年纪又不大,
略病病儿就好了。”又回向秦氏道:“你别胡思乱想,岂不是自己添病了么?”贾蓉
道:“他这病也不用别的,只吃得下些饭食就不怕了。”凤姐儿道:“宝兄弟,太太
叫你快些过去呢。你倒别在这里只管这么着,倒招得媳妇也心里不好过,太太那里
又惦着你。”因向贾蓉说道:“你先同你宝叔叔过去罢,我还略坐坐呢。”贾蓉听说,
即同宝玉过会芳园去。

这里凤姐儿又劝解了一番,又低低说许多衷肠话儿。尤氏打发人来两三遍,凤
姐儿才向秦氏说道:“你好生养着,我再来看你罢。合该你这病要好了,所以前日
遇着这个好大夫,再也是不怕的了。”秦氏笑道:“任凭他是神仙,‘治了病治不了
命’。婶子,我知道这病不过是挨日子的。”凤姐说道:“你只管这么想,这那里能
好呢?总要想开了才好。况且听得大夫说:若是不治,怕的是春天不好。咱们若是
不能吃人参的人家,也难说了;你公公婆婆听见治得好,别说一日二钱人参,就是
二斤也吃得起。好生养着罢,我就过园子里去了。”秦氏又道:“婶子,恕我不能跟
过去了。闲了的时候还求过来瞧瞧我呢,咱们娘儿们坐坐,多说几句闲话儿。”凤
姐儿听了,不觉的眼圈儿又红了,道:“我得了闲儿必常来看你。”

于是带着跟来的婆子媳妇们,并宁府的媳妇婆子们,从里头绕进园子的便门来。
只见:

黄花满地,白柳横坡。小桥通若耶之溪,曲径接天台之路。石中清流滴滴,篱
落飘香;树头红叶翩翩,疏林如画。西风乍紧,犹听莺啼;暖日常暄,又添蛩语。
遥望东南,建几处依山之榭;近观西北,结三间临水之轩。笙簧盈座,别有幽情;
罗绮穿林,倍添韵致。
凤姐儿看着园中景致,一步步行来,正赞赏时,猛然从假山石后走出一个人来,向
前对凤姐说道:“请嫂子安。”凤姐猛吃一惊,将身往后一退,说道:“这是瑞大爷
不是?”贾瑞说道:“嫂子连我也不认得了?”凤姐儿道:“不是不认得,猛然一见,
想不到是大爷在这里。”贾瑞道:“也是合该我与嫂子有缘。我方才偷出了席,在这
里清净地方略散一散,不想就遇见嫂子:这不是有缘么?”一面说着,一面拿眼睛
不住的观看凤姐。

凤姐是个聪明人,见他这个光景,如何不猜八九分呢,因向贾瑞假意含笑道:
“怪不得你哥哥常提你,说你好。今日见了,听你这几句话儿,就知道你是个聪明
和气的人了。这会子我要到太太们那边去呢,不得合你说话;等闲了再会罢。”贾
瑞道:“我要到嫂子家里去请安,又怕嫂子年轻,不肯轻易见人。”凤姐又假笑道:
“一家骨肉,说什么年轻不年轻的话。”贾瑞听了这话,心中暗喜,因想道:“再不
想今日得此奇遇!”那情景越发难堪了。凤姐儿说道:“你快去入席去罢。看他们拿
住了,罚你的酒。”贾瑞听了,身上已木了半边,慢慢的走着,一面回过头来看。
凤姐儿故意的把脚放迟了,见他去远了,心里暗忖道:“这才是‘知人知面不知心’
呢。那里有这样禽兽的人?他果如此,几时叫他死在我手里,他才知道我的手段!”

于是凤姐儿方移步前来。将转过了一重山坡儿,见两三个婆子慌慌张张的走来,
见凤姐儿,笑道:“我们奶奶见二奶奶不来,急的了不得,叫奴才们又来请奶奶来
了。”凤姐儿说:“你们奶奶就是这样急脚鬼似的。”凤姐儿慢慢的走着,问:“戏文
唱了几出了?”那婆子回道:“唱了八九出了。”说话之间,已到天香楼后门,见宝
玉和一群丫头小子们那里玩呢。凤姐儿说:“宝兄弟,别忒淘气了。”一个丫头说道:
“太太们都在楼上坐着呢。请奶奶就从这边上去罢。”

凤姐儿听了,款步提衣上了楼。尤氏已在楼梯口等着。尤氏笑道:“你们娘儿
两个忒好了,见了面总舍不得来了。你明日搬来和他同住罢。你坐下,我先敬你一
钟。”于是凤姐儿至邢夫人王夫人前告坐。尤氏拿戏单来让凤姐儿点戏,凤姐儿说:
“太太们在这里,我怎么敢点。”邢夫人王夫人道:“我们和亲家太太点了好几出了。
你点几出好的我们听。”凤姐儿立起身来答应了,接过戏单,从头一看,点了一出
《还魂》,一出《弹词》,递过戏单来,说:“现在唱的这《双官诰》完了,再唱这
两出,也就是时候了。”王夫人道:“可不是呢,也该趁早叫你哥哥嫂子歇歇。他们
心里又不静。”尤氏道:“太太们又不是常来的,娘儿们多坐一会子去,才有趣儿。
天气还早呢。”凤姐儿立起身来望楼下一看,说:“爷们都往那里去了?”傍边一个
婆子道:“爷们才到凝曦轩,带了十番那里吃酒去了。”凤姐儿道:“在这里不便宜,
背地里又不知干什么去了!”尤氏笑道:“那里都像你这么正经人呢!”

于是说说笑笑,点的戏都唱完了,方才撤下酒席,摆上饭来。吃毕,大家才出
园子,来到上房,坐下吃了茶,才叫预备车,向尤氏的母亲告了辞。尤氏率同众姬
妾并家人媳妇们送出来,贾珍率领众子侄在车旁侍立,都等候着。见了邢王二夫人,
说道:“二位婶子明日还过来逛逛。”王夫人道:“罢了,我们今儿整坐了一日,也
乏了,明日也要歇歇。”于是都上车去了。贾瑞犹不住拿眼看着凤姐儿。贾珍进去
后,李贵才拉过马来,宝玉骑上,随了王夫人去了。

这里贾珍同一家子的弟兄子侄吃过饭,方大家散了。次日仍是众族人等闹了一
日,不必细说。此后凤姐不时亲自来看秦氏。秦氏也有几日好些,也有几日歹些。
贾珍、尤氏、贾蓉甚是焦心。

且说贾瑞到荣府来了几次,偏都值凤姐儿往宁府去了。这年正是十一月三十日
冬至。到交节的那几日,贾母、王夫人、凤姐儿日日差人去看秦氏。回来的人都说:
“这几日没见添病,也没见大好。”王夫人向贾母说:“这个症候遇着这样节气,不
添病就有指望了。”贾母说:“可是呢。好个孩子,要有个长短,岂不叫人疼死。”
说着,一阵心酸,向凤姐儿说道:“你们娘儿们好了一场,明日大初一,过了明日,
你再看看他去。你细细的瞧瞧他的光景,倘或好些儿,你回来告诉我。那孩子素日
爱吃什么,你也常叫人送些给他。”

凤姐儿一一答应了。到初二日,吃了早饭,来到宁府里,看见秦氏光景,虽未
添什么病,但那脸上身上的肉都瘦干了。于是和秦氏坐了半日,说了些闲话,又将
这病无妨的话开导了一番。秦氏道:“好不好,春天就知道了。如今现过了冬至,
又没怎么样,或者好的了也未可知。婶子回老太太、太太放心罢。昨日老太太赏的
那枣泥馅的山药糕,我吃了两块,倒像克化的动的似的。”凤姐儿道:“明日再给你
送来。你到你婆婆那里瞧瞧,就要赶着回去回老太太话去。”秦氏道:“婶子替我请
老太太、太太的安罢。”凤姐儿答应着就出来了。到了尤氏上房坐下,尤氏道:“你
冷眼瞧媳妇是怎么样?”凤姐儿低了半日头,说道:“这个就没法儿了。你也该将
一应的后事给他料理料理,——冲一冲也好。”尤氏道:“我也暗暗的叫人预备了。
就是那件东西不得好木头,且慢慢的办着呢。”于是凤姐儿喝了茶,说了一会子话
儿,说道:“我要快些回去回老太太的话去呢。”尤氏道:“你可慢慢儿的说,别吓
着老人家。”凤姐儿道:“我知道。”

于是凤姐儿起身回到家中,见了贾母,说:“蓉哥媳妇请老太太安,给老太太
磕头,说他好些了。求老祖宗放心罢。他再略好些,还给老太太磕头请安来呢。”
贾母道:“你瞧他是怎么样?”凤姐儿说:“暂且无妨,精神还好呢。”贾母听了,
沉吟了半日,因向凤姐说:“你换换衣裳歇歇去罢。”

凤姐儿答应着出来,见过了王夫人,到了家中,平儿将烘的家常衣服给凤姐儿
换上了。凤姐儿坐下,因问:“家中有什么事没有?”平儿方端了茶来递过去,说
道:“没有什么事。就是那三百两银子的利银,旺儿嫂子送进来,我收了。还有瑞
大爷使人来打听奶奶在家没有,他要来请安说话。”凤姐儿听了,哼了一声,说道:
“这畜生合该作死,看他来了怎么样!”平儿回道:“这瑞大爷是为什么,只管来?”
凤姐儿遂将九月里在宁府园子里遇见他的光景、他说的话,都告诉了平儿。平儿说
道:“‘癞蛤蟆想吃天鹅肉’,没人伦的混帐东西,起这样念头,叫他不得好死!”凤
姐儿道:“等他来了,我自有道理。”

不知贾瑞来时作何光景,且听下回分解。