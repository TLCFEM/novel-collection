\chapter{老学究讲义警顽心~病潇湘痴魂惊恶梦}

话说宝玉下学回来,见了贾母。贾母笑道:“好了,如今野马上了笼头了。去
罢,见见你老爷去来,散散儿去罢。”宝玉答应着,去见贾政。贾政道:“这早晚就
下了学了么?师父给你定了工课没有?”宝玉道:“定了:早起理书,饭后写字,晌
午讲书念文章。”贾政听了,点点头儿,因道:“去罢,还到老太太那边陪着坐坐去。
你也该学些人功道理,别一味的贪玩。晚上早些睡,天天上学,早些起来。你听见
了?”宝玉连忙答应几个“是”,退出来,忙忙又去见王夫人,又到贾母那边打了
个照面儿。赶着出来,恨不得一走就走到潇湘馆才好。

刚进门口,便拍着手笑道:“我依旧回来了。”猛可里倒唬了黛玉一跳。紫鹃打
起帘子,宝玉进来坐下。黛玉道:“我恍惚听见你念书去了,这么早就回来了?”
宝玉道:“嗳呀了不得!我今儿不是被老爷叫了念书去了么?心上倒像没有和你们见
面的日子了。好容易熬了一天,这会子瞧见你们,竟如死而复生的一样。真真古人
说‘一日三秋’,这话再不错的。”黛玉道:“你上头去过了没有?”宝玉道:“都去
过了。”黛玉道:“别处呢?”宝玉道:“没有。”黛玉道:“你也该瞧瞧他们去。”宝
玉道:“我这会子懒怠动了,只和妹妹坐着说一会子话儿罢。老爷还叫早睡早起,
只好明儿再瞧他们去了。”黛玉道:“你坐坐儿,可是正该歇歇儿去了。”宝玉道:“我
那里是乏?只是闷得慌。这会子咱们坐着,才把闷散了,你又催起我来!”黛玉微微
的一笑。因叫紫鹃:“把我的龙井茶给二爷沏一碗。二爷如今念书了,比不得头里。”
紫鹃笑着答应,去拿茶叶,叫小丫头子沏茶。宝玉接着说道:“还提什么念书?我最
厌这些道学话。更可笑的,是八股文章,拿他诓功名混饭吃也罢了,还要说‘代圣
贤立言’。好些的,不过拿些经书凑搭凑搭还罢了;更有一种可笑的,肚子里原没
有什么,东拉西扯,弄的牛鬼蛇神,还自以为博奥。这那里是阐发圣贤的道理?目
下老爷口口声声叫我学这个,我又不敢违拗,你这会子还提念书呢!”黛玉道:“我
们女孩儿家虽然不要这个,但小时跟着你们雨村先生念书,也曾看过。内中也有近
情近理的,也有清微淡远的。那时候虽不大懂,也觉得好,不可一概抹倒。况且你
要取功名,这个也清贵些。”宝玉听到这里,觉得不甚入耳,因想:“黛玉从来不是
这样人,怎么也这样势欲熏心起来?”又不敢在他跟前驳回,只在鼻子眼里笑了一
声。

正说着,忽听外面两个人说话,却是秋纹和紫鹃。只听秋纹说:“袭人姐姐叫
我老太太那里接去,谁知却在这里。”紫鹃道:“我们这里才沏了茶,索性让他喝了
再去。”说着,二人一齐进来。宝玉和秋纹笑道:“我就过去。又劳动你来找。”秋
纹未及答言,只见紫鹃道:“你快喝了茶去罢,人家都想了一天了。”秋纹啐道:“呸!
好混帐丫头。”说的大家都笑了。宝玉起身,才辞了出来。黛玉送到屋门口儿,紫
鹃在台阶下站着,宝玉出去,才回房里来。

却说宝玉回到怡红院中,进了屋子,只见袭人从里间迎出来,便问:“回来了
么?”秋纹应道:“二爷早来了。在林姑娘那边来着。”宝玉道:“今日有事没有?”
袭人道:“事却没有。方才太太叫鸳鸯姐姐来吩咐我们:如今老爷发狠叫你念书,
如有丫鬟们再敢和你玩笑,都要照着晴雯司棋的例办。我想伏侍你一场,赚了这些
言语,也没什么趣儿。”说着,便伤起心来。宝玉忙道:“好姐姐,你放心,我只好
生念书,太太再不说你们了。我今儿晚上还要看书,明日师父叫我讲书呢。我要使
唤,横竖有麝月秋纹呢,你歇歇去罢。”袭人道:“你要真肯念书,我们伏侍你也是
欢喜的。”宝玉听了,赶忙的吃了晚饭,就叫点灯,把念过的《四书》翻出来。只
是从何处看起?翻了一本看去,章章里头,似乎明白;细按起来,却不很明白。看
着小注,又看讲章。闹到起更以后了,自己想道:“我在诗词上觉得很容易,在这
个上头竟没头脑。”便坐着呆呆的呆想。袭人道:“歇歇罢。做工夫也不在这一时的。”
宝玉嘴里只管胡乱答应。麝月袭人才伏侍他睡下,两个才也睡了。及至睡醒一觉,
听得宝玉炕上还是翻来覆去。袭人道:“你还醒着呢么?你倒别混想了,养养神明儿
好念书。”宝玉道:“我也是这样想,只是睡不着,你来给我揭去一层被。”袭人道:
“天气不热,别揭罢。”宝玉道:“我心里烦躁的很。”自把被窝褪下来。袭人忙爬
起来按住,把手去他头上一摸,觉得微微有些发烧。袭人道:“你别动了,有些发
烧了。”宝玉道:“可不是?”袭人道:“这是怎么说呢!”宝玉道:“不怕,是我心
烦的原故,你别吵嚷,省得老爷知道了,必说我装病逃学,不然怎么病的这么巧?
明儿好了,原到学里去,就完事了。”袭人也觉得可怜,说道:“我靠着你睡罢。”
便和宝玉捶了一回脊梁。不知不觉,大家都睡着了。

直到红日高升,方才起来。宝玉道:“不好了,晚了。”急忙梳洗毕,问了安,
就往学里来了。代儒已经变着脸,说:“怪不得你老爷生气,说你没出息。第二天
你就懒惰。这是什么时候才来?”宝玉把昨儿发烧的话说了一遍,方过去了,原旧
念书。到了下晚,代儒道:“宝玉,有一章书,你来讲讲。”宝玉过来一看,却是“后
生可畏”章。宝玉心上说:“这还好,幸亏不是《学》《庸》。”问道:“怎么讲呢?”
代儒道:“你把节旨句子细细儿讲来。”宝玉把这章先朗朗的念了一遍,说:“这章
书是圣人勉励后生,教他及时努力,不要弄到——”说到这里,抬头向代儒一看。
代儒觉得了,笑了一笑道:“你只管说,讲书是没有什么避忌的。《礼记》上说:‘临
文不讳。’只管说,‘不要弄到’什么?”宝玉道:“不要弄到老大无成。先将‘可
畏’二字激发后生的志气,后把‘不足畏’三字警惕后生的将来。”说罢,看着代
儒。代儒道:“也还罢了。串讲呢?”宝玉道:“圣人说:人生少时,心思才力,样
样聪明能干,实在是可怕的,那里料的定他后来的日子不像我的今日?若是悠悠忽
忽,到了四十岁,又到五十岁,既不能够发达,这种人,虽是他后生时像个有用的,
到了那个时候,这一辈子就没有人怕他了。”代儒笑道:“你方才节旨讲的倒清楚,
只是句子里有些孩子气。‘无闻’二字,不是不能发达做官的话。‘闻’是实在自己
能够明理见道,做不做官也是有闻了;不然,古圣贤是遁世不见知的,岂不是不做
官的人?难道也是无闻么?‘不足畏’是使人料得定,方与‘焉知’的‘知’字对针,
不是‘怕’的字眼。要从这里看出,方能入细。你懂得不懂得?”宝玉道:“懂得
了。”

代儒道:“还有一章,你也讲一讲。”代儒往前揭了一篇,指给宝玉。宝玉看时:
“吾未见好德如好色者也。”宝玉觉得这一章却有些刺心,便陪笑道:“这句话没有
什么讲头。”代儒道:“胡说。譬如场中出了这个题目,也说没有做头么?”宝玉不
得已,讲道:“是圣人看见人不肯好德,见了色,便好的了不得,殊不想德是性中
本有的东西,人偏都不肯好他。至于那个色呢,虽也是从先天中带来,无人不好的,
但是德乃天理,色是人欲,人那里肯把天理好的像人欲似的?孔子虽是叹息的话,
又是望人回转来的意思。并且见得人就有好德的,好的终是浮浅,直要像色一样的
好起来,那才是真好呢。”代儒道:“这也讲的罢了。我有句话问你:你既懂得圣人
的话,为什么正犯着这两件病?我虽不在家中,你们老爷也不曾告诉我,其实你的
毛病我却尽知的。做一个人,怎么不望长进?你这会儿正是‘后生可畏’的时候。‘有
闻’、‘不足畏’,全在你自己做去了。我如今限你一个月,把念过的旧书全要理清。
再念一个月文章,以后我要出题目叫你作文章了。如若懈怠,我是断乎不依的。自
古道:‘成人不自在,自在不成人。’你好生记着我的话。”宝玉答应了,也只得天
天按着功课干去,不提。

且说宝玉上学之后,怡红院中甚觉清净闲暇,袭人倒可做些活计,拿着针线要
绣个槟榔包儿。想这如今宝玉有了功课,丫头们可也没有饥荒了,早要如此,晴雯
何至弄到没有结果?兔死狐悲,不觉叹起气来。忽又想到自己终身,本不是宝玉的
正配,原是偏房。宝玉的为人却还拿得住,只怕娶了一个利害的,自己便是尤二姐、
香菱的后身。素来看着贾母王夫人光景,及凤姐儿往往露出话来,自然是黛玉无疑
了。那黛玉就是个多心人。想到此际,脸红心热,拿着针不知戳到那里去了。便把
活计放下,走到黛玉处去探探他的口气。

黛玉正在那里看书,见是袭人,欠身让坐。袭人也连忙迎上来问:“姑娘这几
天身子可大好了?”黛玉道:“那里能够?不过略硬朗些。你在家里做什么呢?”袭
人道:“如今宝二爷上了学,屋里一点事儿没有,因此来瞧瞧姑娘,说说话儿。”说
着,紫鹃拿茶来,袭人忙站起来道:“妹妹坐着罢。”因又笑道:“我前儿听见秋纹
说,妹妹背地里说我们什么来着?”紫鹃也笑道:“姐姐信他的话!我说宝二爷上了
学,宝姑娘又隔断了,连香菱也不过来,自然是闷的。”袭人道:“你还提香菱呢!
这才苦呢!撞着这位‘太岁奶奶’,难为他怎么过!”把手伸着两个指头,道:“说起
来,比他还利害,连外头的脸面都不顾了。”黛玉接着道:“他也够受了。尤二姑娘
怎么死了!”袭人道:“可不是。想来都是一个人,不过名分里头差些,何苦这样毒?
外面名声也不好听。”黛玉从不闻袭人背地里说人,今听此话有因,心里一动,便
说道:“这也难说。但凡家庭之事,不是东风压了西风,就是西风压了东风。”袭人
道:“做了旁边人,心里先怯,那里倒敢欺负人呢?”

说着,只见一个婆子在院里问道:“这里是林姑娘的屋子么?那位姐姐在这里
呢?”雪雁出来一看,模糊认的是薛姨妈那边的人,便问道:“作什么?”婆子道:
“我们姑娘打发来给这里林姑娘送东西的。”雪雁道:“略等等儿。”雪雁进来回了
黛玉,黛玉便叫领他进来。那婆子进来请了安,且不说送什么,只是觑着眼瞧黛玉,
看的黛玉脸上倒不好意思起来,因问道:“宝姑娘叫你来送什么?”婆子方笑着回
道:“我们姑娘叫给姑娘送了一瓶儿蜜饯荔枝来。”回头又瞧见袭人,便问道:“这
位姑娘,不是宝二爷屋里的花姑娘么?”袭人笑道:“妈妈怎么认的我?”婆子笑
道:“我们只在太太屋里看屋子,不大跟太太姑娘出门,所以姑娘们都不大认得。
姑娘们碰着到我们那边去,我们都模糊记得。”说着,将一个瓶儿递给雪雁,又回
头看看黛玉,因笑着向袭人说:“怨不得我们太太说:这林姑娘和你们宝二爷是一
对儿。原来真是天仙似的!”袭人见他说话造次,连忙岔道:“妈妈,你乏了,坐坐
吃茶罢。”那婆子笑嘻嘻的道:“我们那里忙呢,都张罗琴姑娘的事呢。姑娘还有两
瓶荔枝,叫给宝二爷送去。”说着,颤颤巍巍告辞出去。黛玉虽恼这婆子方才冒撞,
但因是宝钗使来的,也不好怎么样他,等他出了屋门,才说一声道:“给你们姑娘
道费心。”那老婆子还只管嘴里咕咕哝哝的说:“这样好模样儿,除了宝玉,什么人
擎受的起!”黛玉只装没听见。袭人笑道:“怎么人到了老来,就是混说白道的,叫
人听着又生气,又好笑。”一时雪雁拿过瓶子来给黛玉看,黛玉道:“我懒怠吃,拿
了搁起去罢。”又说了一回话,袭人才去了。

一时晚妆将卸,黛玉进了套间,猛抬头看见了荔枝瓶,不禁想起日间老婆子的
一番混话,甚是刺心。当此黄昏人静,千愁万绪堆上心来,想起:“自己身子不牢,
年纪又大了,看宝玉的光景,心里虽没别人,但是老太太舅母又不见有半点意思,
深恨父母在时,何不早定了这头婚姻。”又转念一想道:“倘若父母在时,别处定了
婚姻,怎能够似宝玉这般人材心地?不如此时尚有可图。”心内一上一下,辗转缠绵,
竟像辘轳一般。叹了一回气,吊了几点泪,无情无绪,和衣倒下。

不知不觉,只见小丫头走来说道:“外面雨村贾老爷请姑娘。”黛玉道:“我虽
跟他读过书,却不比男学生,要见我做什么?况且他和舅舅往来,从未提起,我也
不必见的。”因叫小丫头回复:“身上有病,不能出来,与我请安道谢就是了。”小
丫头道:“只怕要与姑娘道喜,南京还有人来接。”说着,又见凤姐同邢夫人、王夫
人、宝钗等都来笑道:“我们一来道喜,二来送行。”黛玉慌道:“你们说什么话?”
凤姐道:“你还装什么呆?你难道不知道:林姑爷升了湖北的粮道,娶了一位继母,
十分合心合意。如今想着你撂在这里,不成事体,因托了贾雨村作媒,将你许了你
继母的什么亲戚,还说是续弦,所以着人到这里来接你回去。大约一到家中,就要
过去的。都是你继母作主。怕的是道儿上没有照应,还叫你琏二哥哥送去。”说得
黛玉一身冷汗。黛玉又恍惚父亲果在那里做官的样子。心上急着,硬说道:“没有
的事,都是凤姐姐混闹!”只见邢夫人向王夫人使个眼色儿:“他还不信呢,咱们走
罢。”黛玉含着泪道:“二位舅母坐坐去。”众人不言语,都冷笑而去。

黛玉此时心中干急,又说不出来,哽哽咽咽,恍惚又是和贾母在一处的似的,
心中想道:“此事惟求老太太,或还有救。”于是两腿跪下去,抱着贾母的腿说道:
“老太太救我!我南边是死也不去的。况且有了继母,又不是我的亲娘,我是情愿
跟着老太太一块儿的。”但见贾母呆着脸儿笑道:“这个不干我的事。”黛玉哭道:“老
太太,这是什么事呢。”老太太道:“续弦也好,倒多得一副妆奁。”黛玉哭道:“我
在老太太跟前,决不使这里分外的闲钱,只求老太太救我!”贾母道:“不中用了。
做了女人,总是要出嫁的。你孩子家不知道,在此地终非了局。”黛玉道:“我在这
里,情愿自己做个奴婢过活,自做自吃,也是愿意。只求老太太作主。”见贾母总
不言语,黛玉又抱着贾母哭道:“老太太!你向来最是慈悲的,又最疼我的,到了紧
急的时候儿,怎么全不管?你别说我是你的外孙女儿,是隔了一层了;我的娘是你
的亲生女儿,看我娘分上,也该护庇些。”说着,撞在怀里痛哭。听见贾母道:“鸳
鸯,你来送姑娘出去歇歇,我倒被他闹乏了。”

黛玉情知不是路了,求去无用,不如寻个自尽,站起来,往外就走。深痛自己
没有亲娘,便是外祖母与舅母姊妹们,平时何等待的好,可见都是假的。又一想:
“今日怎么独不见宝玉?或见他一面,他还有法儿。”便见宝玉站在面前,笑嘻嘻的
道:“妹妹大喜呀。”黛玉听了这一句话,越发急了,也顾不得什么了,把宝玉紧紧
拉住,说:“好!宝玉,我今日才知道你是个无情无义的人了!”宝玉道:“我怎么无
情无义?你既有了人家儿,咱们各自干各自的了。”黛玉越听越气,越没了主意,只
得拉着宝玉哭道:“好哥哥!你叫我跟了谁去?”宝玉道:“你要不去,就在这里住
着。你原是许了我的,所以你才到我们这里来。我待你是怎么样的?你也想想。”黛
玉恍惚又像果曾许过宝玉的,心内忽又转悲作喜,问宝玉道:“我是死活打定主意
的了,你到底叫我去不去?”宝玉道:“我说叫你住下。你不信我的话,你就瞧瞧
我的心!”说着,就拿着一把小刀子往胸口上一划,只见鲜血直流。黛玉吓得魂飞
魄散,忙用手握着宝玉的心窝,哭道:“你怎么做出这个事来?你先来杀了我罢!”
宝玉道:“不怕,我拿我的心给你瞧。”还把手在划开的地方儿乱抓。黛玉又颤又哭,
又怕人撞破,抱住宝玉痛哭。宝玉道:“不好了。我的心没有了,活不得了!”说着,
眼睛往上一翻,“咕咚”就倒了。

黛玉拚命放声大哭。只听见紫鹃叫道:“姑娘,姑娘!怎么魇住了?快醒醒儿,
脱了衣服睡罢。”黛玉一翻身,却原来是一场恶梦。喉间犹是哽咽,心上还是乱跳,
枕头上已经湿透,肩背身心,但觉冰冷,想了一回,“父母死的久了,和宝玉尚未
放定,这是从那里说起?”又想梦中光景,无倚无靠,再真把宝玉死了,那可怎么
样好?一时痛定思痛,神魂俱乱。又哭了一回,遍身微微的出了一点儿汗。扎挣起
来,把外罩大袄脱了,叫紫鹃盖好了被窝,又躺下去。翻来覆去那里睡得着,只听
得外面淅淅飒飒,又像风声又像雨声。又停了一会子,又听得远远的吆呼声儿,却
是紫鹃已在那里睡着,鼻息出入之声。自己扎挣着爬起来,围着被坐了一会,觉得
窗缝里透进一缕冷风来,吹得寒毛直竖,便又躺下。正要朦胧睡去,听得竹枝上不
知有多少家雀儿的声儿,啾啾唧唧叫个不住。那窗上的纸,隔着屉子,渐渐的透进
清光来。

黛玉此时已醒得双眸炯炯,一会儿咳嗽起来,连紫鹃都咳嗽醒了。紫鹃道:“姑
娘,你还没睡着么?又咳嗽起来了。想是着了风了,这会儿窗户纸发清了,也待好
亮起来了。歇歇儿罢,养养神,别尽着想长想短的了。”黛玉道:“我何尝不要睡?
只是睡不着。你睡你的罢。”说了又嗽起来。紫鹃见黛玉这般光景,心中也自伤感,
睡不着了。听见黛玉又嗽,连忙起来,捧着痰盒。这时天已亮了。黛玉道:“你不
睡了么?”紫鹃笑道:“天都亮了,还睡什么呢。”黛玉道:“既这样,你就把痰盒
儿换了罢。”紫鹃答应着,忙出来换了一个痰盒儿,将手里的这个盒儿放在桌上,
开了套间门出来,仍旧带上门,放下撒花软帘,出来叫醒雪雁。开了屋门去倒那盒
子时,只见满盒子痰,痰中有些血星。唬了紫鹃一跳,不觉失声道:“嗳哟,这还
了得!”黛玉里面接着问:“是什么?”紫鹃自知失言,连忙改说道:“手里一滑,
几乎撂了痰盒子。”黛玉道:“不是盒子里的痰有了什么?”紫鹃道:“没有什么。”
说着这句话时,心中一酸,那眼泪直流下来,声儿早已岔了。

黛玉因为喉间有些甜腥,早自疑惑;方才听见紫鹃在外边诧异,这会子又听见
紫鹃说话声音带着悲惨的光景,心中觉了八九分,便叫紫鹃:“进来罢,外头看冷
着。”紫鹃答应了一声,这一声更比头里凄惨,竟是鼻中酸楚之音。黛玉听了,冷
了半截。看紫鹃推门进来时,尚拿绢子拭眼。黛玉道:“大清早起,好好的为什么
哭?”紫鹃勉强笑道:“谁哭来?这早起起来,眼睛里有些不舒服。姑娘今夜大概比
往常醒的时候更大罢?我听见咳嗽了半夜。”黛玉道:“可不是?越要睡越睡不着。”
紫鹃道:“姑娘身上不大好,依我说,还得自己开解着些。身子是根本,俗语说的:
‘留得青山在,依旧有柴烧。’况这里自老太太、太太起,那个不疼姑娘?”只这
一句话,又勾起黛玉的梦来,觉得心里一撞,眼中一黑,神色俱变。紫鹃连忙端着
痰盒,雪雁捶着脊梁,半日才吐出一口痰来,痰中一缕紫血,簌簌乱跳。紫鹃雪雁
脸都吓黄了。两个旁边守着,黛玉便昏昏躺下。紫鹃看着不好,连忙努嘴叫雪雁叫
人去。

雪雁才出屋门,只见翠缕翠墨两个人笑嘻嘻的走来。翠缕便道:“林姑娘怎么
这早晚还不出门?我们姑娘和三姑娘都在四姑娘屋里,讲究四姑娘画的那张园子景
儿呢。”雪雁连忙摆手儿。翠缕翠墨二人倒都吓了一跳,说:“这是什么原故?”雪
雁将方才的事一一告诉他二人。二人都吐了吐舌头儿,说:“这可不是玩的。你们
怎么不告诉老太太去?这还了得,你们怎么这么糊涂?”雪雁道:“我这里才要去,
你们就来了。”正说着,只听紫鹃叫道:“谁在外头说话?姑娘问呢。”三个人连忙一
齐进来。翠缕翠墨见黛玉盖着被,躺在床上,见了他二人,便说道:“谁告诉你们
了,你们这样大惊小怪的?”翠墨道:“我们姑娘和云姑娘才都在四姑娘屋里,讲
究四姑娘画的那张园子图儿,叫我们来请姑娘。不知道姑娘身上又欠安了。”黛玉
道:“也不是什么大病,不过觉得身子略软些,躺躺儿就起来了。你们回去告诉三
姑娘和云姑娘,饭后若无事,倒是请他们到这里坐坐罢。宝二爷没到你们那边去?”
二人答道:“没有。”翠墨又道:“宝二爷这两天上了学了,老爷天天要查功课,那
里还能像从前那么乱跑呢。”黛玉听了,默然不言。二人又略站了一回,都悄悄的
退出来了。

且说探春湘云正在惜春那边评论惜春所画《大观园图》,说这个多一点,那个
少一点;这个太疏,那个太密。大家又议着题诗,着人去请黛玉商议。正说着,忽
见翠缕翠墨二人回来,神色匆忙。湘云便先问道:“林姑娘怎么不来?”翠缕道:“林
姑娘昨日夜里又犯了病了,咳嗽了一夜。我们听见雪雁说,吐了一盒子痰血。”探
春听了,诧异道:“这话真么?”翠缕道:“怎么不真?”翠墨道:“我们刚才进去
去瞧了瞧,颜色不成颜色,说话儿的气力儿都微了。”湘云道:“不好的这么着,怎
么还能说话呢?”探春道:“怎么你这么糊涂!不能说话,不是已经——”说到这里,
却咽住了。惜春道:“林姐姐那样一个聪明人,我看他总有些瞧不破,一点半点儿
都要认起真来。天下事那里有多少真的呢。”探春道:“既这么着,咱们都过去看看。
倘若病的利害,咱们也过去告诉大嫂子回老太太,传大夫进来瞧瞧,也得个主意。”
湘云道:“正是这样。”惜春道:“姐姐们先去,我回来再过去。”

于是探春湘云扶了小丫头,都到潇湘馆来。进入房中,黛玉见他二人,不免又
伤起心来。因又转念想起梦中,“连老太太尚且如此,何况他们?况且我不请他们,
他们还不来呢!”心里虽是如此,脸上却碍不过去,只得勉强令紫鹃扶起,口中让
坐。探春湘云都坐在床沿上,一头一个,看了黛玉这般光景,也自伤感。探春便道:
“姐姐怎么身上又不舒服了?”黛玉道:“也没什么要紧,只是身子软得很。”紫鹃
在黛玉身后,偷偷的用手指那痰盒儿。湘云到底年轻,性情又兼直爽,伸手便把痰
盒拿起来看。不看则已,看了吓的惊疑不止,说:“这是姐姐吐的?这还了得!”初
时黛玉昏昏沉沉,吐了也没细看,此时见湘云这么说,回头看时,自己早已灰了一
半。探春见湘云冒失,连忙解说道:“这不过是肺火上炎,带出一半点来,也是常
事。偏是云丫头,不拘什么,就这样蝎蝎螫螫的!”湘云红了脸,自悔失言。探春
见黛玉精神短少,似有烦倦之意,连忙起身说道:“姐姐静静的养养神罢。我们回
来再瞧你。”黛玉道:“累你二位惦着。”探春又嘱咐紫鹃:“好生留神伏侍姑娘。”
紫鹃答应着。探春才要走,只听外面一个人嚷起来。

未知是谁,下回分解。