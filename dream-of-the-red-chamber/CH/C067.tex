\chapter{见土仪颦卿思故里~闻秘事凤姐讯家童}

话说尤三姐自尽之后,尤老娘合二姐儿、贾珍、贾琏等俱不胜悲恸,自不必说,
忙命人盛殓,送往城外埋葬。柳湘莲见三姐身亡,痴情眷恋,却被道人数句冷言,
打破迷关,竟自截发出家,跟随这疯道人飘然而去,不知何往。暂且不表。

且说薛姨妈闻知湘莲已说定了尤三姐为妻,心中甚喜,正是高高兴兴,要打算
替他买房子,治家伙,择吉迎娶,以报他救命之恩。忽有家中小厮吵嚷:“三姐儿
自尽了。”被小丫头们听见,告知薛姨妈。薛姨妈不知为何,心甚叹息。正在猜疑,
宝钗从园里过来,薛姨妈便对宝钗说道:“我的儿,你听见了没有?你珍大嫂子的
妹妹三姑娘,他不是已经许定给你哥哥的义弟柳湘莲了么?不知为什么自刎了,那
湘莲也不知往那里去了。真正奇怪的事,叫人意想不到的。”宝钗听了并不在意,
便说道:“俗语说的好:‘天有不测风云,人有旦夕祸福。’这也是他们前生命定。
前儿妈妈为他救了哥哥,商量着替他料理,如今已经死的死了,走的走了,依我说
也只好由他罢了,妈妈也不必为他们伤感了。倒是自从哥哥打江南回来了一二十
日,贩了来的货物想来也该发完了。那同伴去的伙计们辛辛苦苦的回来几个月了,
妈妈合哥哥商议商议,也该请一请,酬谢酬谢才是。别叫人家看着无理似的。”

母女正说话间,见薛蟠自外而入,眼中尚有泪痕。一进门来,便向他母亲拍手
说道:“妈妈可知道柳二哥尤三姐的事么?”薛姨妈说:“我才听见说,正在这里
合你妹妹说这件公案呢”。薛蟠道:“妈妈可听见说湘莲跟着一个道士出了家了
么?”薛姨妈道:“这越发奇了。怎么柳相公那样一个年轻的聪明人,一时糊涂了
就跟着道士去了呢?我想你们好了一场,他又无父母兄弟,单身一人在此,你该各
处找找他才是。靠那道士,能往那里远去?左不过是在这方近左右的庙里寺里罢
了。”薛蟠说:“何尝不是呢。我一听见这个信儿,就连忙带了小厮们在各处寻找。
连一个影儿也没有。又去问人,都说没看见。”薛姨妈说:“你既找寻过,没有,
也算把你做朋友的心尽了。焉知他这一出家,不是得了好处去呢?只是你如今也该
张罗张罗买卖,二则把你自己娶媳妇应办的事情,倒早些料理料理。咱们家没人,
俗语说的,‘夯雀儿先飞’,省的临时丢三落四的不齐全,令人笑话。再者,你妹
妹才说你也回家半个多月了,想货物也该发完了,同你去的伙计们,也该摆桌酒给
他们道道乏才是。人家陪着你走了二三千里的路程,受了四五个月的辛苦,而且在
路上又替你担了多少的惊怕沉重。”薛蟠听说,便道:“妈妈说的很是。倒是妹妹
想的周到。我也这样想着。只因这些日子,为各处发货,闹的脑袋都大了。又为柳
二哥的事忙了这几日,反倒落了一个空,白张罗了一会子,倒把正经事都误了。要
不然,定了明儿后儿,下帖儿请罢。”薛姨妈道:“由你办去罢。”

话犹未了,外面小厮进来回说:“管总的张大爷差人送了两箱子东西来,说:
‘这是爷各自买的,不在货账里面。本要早送来,因货物箱子压着,没得拿;昨儿
货物发完了,所以今日才送来了。’”一面说,一面又见两个小厮搬进了两个夹板
夹的大棕箱。薛蟠一见,说:“嗳哟,可是我怎么就糊涂到这步田地了。特特的给
妈合妹妹带来的东西,都忘了,没拿了家里来,还是伙计送了来了。”宝钗说:“亏
你说还是‘特特的带来’的,才放了一二十天。要不是‘特特的带来’,大约要放
到年底下才送来呢。我看你也诸事太不留心了。”薛蟠笑道:“想是在路上叫人把
魂打掉了,还没归窍呢。”说着,大家笑了一回,便向小丫头说:“出去告诉小厮
们,东西收下,叫他们回去罢。”薛姨妈和宝钗因问:“到底是什么东西,这样捆
着绑着的?”薛蟠便命叫两个小厮进来,解了绳子,去了夹板,开了锁看时,这一
箱都是绸缎绫锦洋货等家常应用之物。薛蟠笑着道:“那一箱是给妹妹带的。”亲
自来开。母女二人看时,却是些笔、墨、纸、砚,各色笺纸、香袋、香珠、扇子、
扇坠、花粉、胭脂等物。外有虎丘带来的自行人,酒令儿,水银灌的打金斗小小子,
沙子灯,一出一出的泥人儿的戏用青纱罩的匣子装着。又有在虎丘山上泥捏的薛蟠
的小像,与薛蟠毫无相差。宝钗见了别的都不理论,倒是薛蟠的小像,拿着细细看
了一看,又看看他哥哥,不禁笑起来了。因叫莺儿带着几个老婆子,将这些东西连
箱子送到园子里去。又和母亲哥哥说了一回闲话,才回园子里去。这里薛姨妈将箱
子里的东西取出,一分一分的打点清楚,叫同喜送给贾母并王夫人等处,不提。

且说宝钗到了自己房中,将那些玩意儿一件一件的过了目,除了自己留用之
外,一分一分配合妥当:也有送笔、墨、纸、砚的,也有送香袋、扇子、香坠的,
也有送脂粉、头油的,有单送玩意儿的。只有黛玉的比别人不同,且又加厚一倍。
一一打点完毕,使莺儿同着一个老婆子,跟着送往各处。这边姐妹诸人都收了东西,
赏赐来使,说:“见面再谢。”惟有黛玉看见他家乡之物,反自触物伤情,想起:
“父母双亡,又无兄弟,寄居亲戚家中,那里有人也给我带些土物来?”想到这里,
不觉的又伤起心来了。紫鹃深知黛玉心肠,但也不敢说破,只在一旁劝道:“姑娘
的身子多病,早晚服药,这两日看着比那些日子略好些,虽说精神长了一点儿,还
算不得十分大好。今儿宝姑娘送来的这些东西,可见宝姑娘素日看着姑娘很重,姑
娘看着该喜欢才是,为什么反倒伤起心来?这不是宝姑娘送东西来,倒叫姑娘烦恼
了不成?就是宝姑娘听见,反觉脸上不好看。再者,这里老太太们为姑娘的病体,
千方百计请好大夫配药诊治,也为是姑娘的病好。这如今才好些,又这样哭哭啼啼,
岂不是自己遭塌了自己身子,叫老太太看着添了愁烦了么?况且姑娘这病,原是素
日忧虑过度,伤了血气。姑娘的千金贵体,也别自己看轻了。”

紫鹃正在这里劝解,只听见小丫头子在院内说:“宝二爷来了。”紫鹃忙说:
“请二爷进来罢。”只见宝玉进房来了。黛玉让坐毕,宝玉见黛玉泪痕满面,便问:
“妹妹,又是谁气着你了?”黛玉勉强笑道:“谁生什么气。”旁边紫鹃将嘴向床
后桌上一努。宝玉会意,往那里一瞧,见堆着许多东西,就知道是宝钗送来的,便
取笑说道:“那里这些东西?不是妹妹要开杂货铺啊?”黛玉也不答言。紫鹃笑着
道:“二爷还提东西呢。因宝姑娘送了些东西来,姑娘一看,就伤起心来了。我正
在这里劝解,恰好二爷来的很巧,替我们劝劝。”宝玉明知黛玉是这个原故,却也
不敢提头儿,只得笑说道:“你们姑娘的原故,想来不为别的,必是宝姑娘送来的
东西少,所以生气伤心。妹妹你放心,等我明年叫人往江南去,给你多多的带两船
来,省得你淌眼抹泪的。”黛玉听了这些话,也知宝玉是为自己开心,也不好推,
也不好任,因说道:“我任凭怎么没见过世面,也到不了这步田地,因送的东西少
就生气伤心。我又不是两三岁的孩子,你也忒把人看得小气了。我有我的原故,你
那里知道?”说着,眼泪又流下来了。

宝玉忙走到床前挨着黛玉坐下,将那些东西一件一件拿起来,摆弄着细瞧,故
意问:“这是什么,叫什么名字?”“那是什么做的,这样齐整?”“这是什么,
要他做什么使用?”又说:“这一件可以摆在面前。”又说:“那一件可以放在条
桌上,当古董儿倒好呢。”一味的将些没要紧的话来厮混。黛玉见宝玉如此,自己
心里倒过不去,便说:“你不用在这里混搅了,咱们到宝姐姐那边去罢。”宝玉巴
不的黛玉出去散散闷解了悲痛,便道:“宝姐姐送咱们东西,咱们原该谢谢去。”
黛玉道:“自家姐妹,这倒不必。只是到他那边,薛大哥回来了,必然告诉他些南
边的古迹儿,我去听听,只当回了家乡一趟的。”说着眼圈儿又红了。宝玉便站着
等他。黛玉只得和他出来,往宝钗那里去了。

且说薛蟠听了母亲之言,急下了请帖,办了酒席。次日,请了四位伙计,俱已
到齐,不免说些贩卖账目发货之事。不一时,上席让坐,薛蟠挨次斟了酒,薛姨妈
又使人出来致意。大家喝着酒说闲话儿,内中一个道:“今儿这席上短两个好朋友。”
众人齐问:“是谁?”那人道:“还有谁,就是贾府上的琏二爷和大爷的盟弟柳二
爷。”大家果然都想起来,问着薛蟠道:“怎么不请琏二爷合柳二爷来?”薛蟠闻
言,把眉一皱,叹口气道:“琏二爷又往平安州去了,头两天就起了身了。那柳二
爷竟别提起,真是天下头一件奇事。什么是‘柳二爷’,如今不知那里作‘柳道爷’
去了。”众人都诧异道:“这是怎么说?”薛蟠便把湘莲前后事体说了一遍。众人
听了,越发骇异,因说道:“怪不的前儿我们在店里,仿仿佛佛也听见人吵嚷说:
‘有一个道士,三言两语,把一个人度了去了。’又说:‘一阵风刮了去了。’只
不知是谁。我们正发货,那里有闲工夫打听这个事去?到如今还是似信不信的,谁
知就是柳二爷呢。早知是他,我们大家也该劝劝他才是。任他怎么着,也不叫他去。”
内中一个道:“别是这么着罢?”众人问:“怎么样?”那人道:“柳二爷那样个
伶俐人,未必是真跟了道士去罢?他原会些武艺,又有力量,或看破那道士的妖术
邪法,特意跟他去,在背地摆布他,也未可知。”薛蟠道:“果然如此,倒也罢了。
世上这些妖言惑众的人,怎么没人治他一下子!”众人道:“那时难道你知道了也
没找寻他去?”薛蟠说:“城里城外,那里没有找到?不怕你们笑话,我找不着他,
还哭了一场呢。”言毕,只是长吁短叹,无精打彩的,不像往日高兴。众伙计见他
这样光景,自然不便久坐,不过随便喝了几杯酒,吃了饭,大家散了。

且说宝玉和着黛玉到宝钗处来,宝玉见了宝钗,便说道:“大哥哥辛辛苦苦的
带了东西来,姐姐留着使罢,又送我们。”宝钗笑道:“原不是什么好东西,不过
是远路带来的土物儿,大家看着新鲜些就是了。”黛玉道:“这些东西,我们小时
候倒不理会,如今看见,真是新鲜物儿了。”宝钗因笑道:“妹妹知道,这就是俗
语说的‘物离乡贵’,其实可算什么呢!”宝玉听了这话,正对了黛玉方才的心事,
连忙拿话岔道:“明年好歹大哥哥再去时,替我们多带些来。”黛玉瞅了他一眼,
便道:“你要你只管说,不必拉扯上人。姐姐你瞧,宝哥哥不是给姐姐来道谢,竟
又要定下明年的东西来了。”说的宝钗宝玉都笑了。

三个人又闲话了一回,因提起黛玉的病来,宝钗劝了一回,因说道:“妹妹若
觉着身上不爽快,倒要自己勉强扎挣着出来,各处走走逛逛,散散心,比在屋里闷
坐着到底好些。我那两日,不是觉着发懒,浑身发热,只是要歪着?也因为时气不
好,怕病,因此寻些事情,自己混着。这两日才觉得好些了。”黛玉道:“姐姐说
的何尝不是?我也是这么想着呢。”大家又坐了一会子方散。宝玉仍把黛玉送至潇
湘馆门首,才各自回去了。

且说赵姨娘,因见宝钗送了贾环些东西,心中甚是喜欢。想道:“怨不得别人
都说那宝丫头好,会做人,很大方。如今看起来果然不错。他哥哥能带了多少东西
来?他挨门儿送到,并不遗漏一处,也不露出谁薄谁厚。连我们这样没时运的,他
都想到了。要是那林丫头,他把我们娘儿们正眼也不瞧,那里还肯送我们东西?”
一面想,一面把那些东西翻来覆去的摆弄,瞧看一回。忽然想到宝钗系王夫人的亲
戚,为何不到王夫人跟前卖个好儿呢?自己便蝎蝎螫螫的,拿着东西,走至王夫人
房中,站在旁边,陪笑说道:“这是宝姑娘才刚给环哥儿的。难为宝姑娘这么年轻
的人,想的这么周到,真是大户人家的姑娘,又展样,又大方。怎么叫人不敬奉呢。
怪不的老太太和太太成日家都夸他疼他。我也不敢自专就收起来,特拿来给太太瞧
瞧,太太也喜欢喜欢。”王夫人听了,早知道来意了。又见他说的不伦不类,也不
便不理他,说道:“你只管收了去给环哥玩罢。”赵姨娘来时兴兴头头,谁知抹了
一鼻子灰,满心生气,又不敢露出来,只得讪讪的出来了。到了自己房中,将东西
丢在一边,嘴里咕咕哝哝,自言自语道:“这个又算了个什么儿呢!”一面坐着各
自生了一回闷气。

却说莺儿带着老婆子们送东西回来,回复了宝钗,将众人道谢的话并赏赐的银
钱都回完了,那老婆子便出去了。莺儿走近前来一步,挨着宝钗,悄悄的说道:“刚
才我到琏二奶奶那边,看见二奶奶一脸的怒气。我送下东西出来时,悄悄的问小红,
说:‘刚才二奶奶从老太太屋里回来,不似往日欢天喜地的,叫了平儿去,唧唧咕
咕的不知说了些什么。’看那个光景,倒像有什么大事的似的。姑娘没听见那边老
太太有什么事?”宝钗听了,也自己纳闷,想不出凤姐是为什么有气。便道:“各
人家有各人的事,咱们那里管得?你去倒茶去来。”莺儿于是出来,自己倒茶不提。

且说宝玉送了黛玉回来,想着黛玉的孤苦,不免也替他伤感起来,因要将这话
告诉袭人。进来时,却只有麝月秋纹在屋里,因问:“你袭人姐姐那里去了?”麝
月道:“左不过在这几个院里,那里就丢了他?一时不见就这样找。”宝玉笑着道:
“不是怕丢了他。因我方才到林姑娘那边,见林姑娘又正伤心呢。问起来,却是为
宝姐姐送了他东西,他看见是他家乡的土物,不免对景伤情。我要告诉你袭人姐姐,
叫他过去劝劝。”正说着,晴雯进来了,因问宝玉道:“你回来了。你又要叫劝谁?”
宝玉将方才的话说了一遍。晴雯道:“袭人姐姐才出去。听见他说要到琏二奶奶那
边去。保不住还到林姑娘那里去呢。”宝玉听了,便不言语。秋纹倒了茶来,宝玉
漱了一口,递给小丫头子,心中着实不自在,就随便歪在床上。

却说袭人因宝玉出门,自己作了回活计。忽想起凤姐身上不好,这几天也没有
过去看看,况闻贾琏出门,正好大家说说话儿,便告诉晴雯:“好生在屋里,别都
出去了,叫二爷回来抓不着人。”晴雯道:“嗳哟!这屋里单你一个人惦记着他,
我们都是白闲着混饭吃的。”袭人笑着,也不答言,就走了。刚来到沁芳桥畔,那
时正是夏末秋初,池中莲藕新残相间,红绿离披。袭人走着,沿堤看玩了一回,猛
抬头,看见那边葡萄架底下,有人拿着掸子在那里掸什么呢。走到跟前,却是老祝
妈。那老婆子见了袭人,便笑嘻嘻的迎上来,说道:“姑娘怎么今儿得工夫出来逛
逛?”袭人道:“可不是吗,我要到琏二奶奶那里瞧瞧去。你这里做什么呢?”那
婆子道:“我在这里赶蜜蜂儿。今年三伏里雨水少,这果子树上都有虫子,把果子
吃的疤流星的,掉了好些了。姑娘还不知道呢,这马蜂最可恶的:一嘟噜上只咬
破两三个儿,那破的水滴到好的上头,连这一嘟噜都是要烂的。姑娘你瞧咱们说话
的空儿没赶,就落上许多了。”袭人道:“你就是不住手的赶,也赶不了多少。你
倒是告诉买办,叫他多多做些小冷布口袋儿,一嘟噜套上一个,又透风,又不遭塌。”
婆子笑道:“倒是姑娘说的是。我今年才管上,那里知道这个巧法儿呢?”因又笑
着说道:“今年果子虽遭塌了些,味儿倒好,不信摘一个姑娘尝尝。”袭人正色道:
“这那里使得。不但没熟吃不得,就是熟了,上头还没有供鲜,咱们倒先吃了?你
是府里使老了的,难道连这个规矩都不懂了?”老祝妈忙笑道:“姑娘说的是。我
见姑娘很喜欢,我才敢这么说,可就把规矩错了。我可是老糊涂了。”袭人道:“这
也没有什么,只是你们有年纪的老奶奶们,别先领着头儿这么着就好了。”

说着,遂一径出了园门,来到凤姐这边。一到院里,只听凤姐说道:“天理良
心!我在这屋里熬的越发成了贼了!”袭人听见这话,知道有原故了,又不好回来,
又不好进去,遂把脚步放重些,隔着窗子问道:“平姐姐在家里呢么?”平儿忙答
应着迎出来。袭人便问:“二奶奶也在家里呢么?身上可大安了?”说着,已走进
来。凤姐装着在床上歪着呢,见袭人进来,也笑着站起来,说:“好些了,叫你惦
着。怎么这几日不过我们这边坐坐?”袭人道:“奶奶身上欠安,本该天天过来请
安才是。但只怕奶奶身上不爽快,倒要静静儿的歇歇儿,我们来了,倒吵的奶奶烦。”
凤姐笑道:“烦是没的话。倒是宝兄弟屋里虽然人多,也就靠着你一个照看他,也
实在的离不开。我常听见平儿告诉我说,你背地里还惦着我,常常问我。这就是你
尽心了。”一面说着,叫平儿挪了张杌子放在床傍边,让袭人坐下。丰儿端进茶来。
袭人欠身道:“妹妹坐着罢。”

一面说闲话儿。只见一个小丫头子在外间屋里,悄悄的和平儿说:“旺儿来了,
在二门上伺候着呢。”又听见平儿也悄悄的道:“知道了。叫他先去,回来再来。
别在门口儿站着。”袭人知他们有事,又说了两句话,便起身要走。凤姐道:“闲
来坐坐,说说话儿,我倒开心。”因命:“平儿,送送你妹妹。”平儿答应着,送
出来。只见两三个小丫头子都在那里,屏声息气,齐齐的伺候着。袭人不知何事,
便自去了。

却说平儿送出袭人,进来回道:“旺儿才来了,因袭人在这里,我叫他先到外
头等等儿。这会子还是立刻叫他呢,还是等着?请奶奶的示下。”凤姐道:“叫他
来!”平儿忙叫小丫头去传旺儿进来。这里凤姐又问平儿:“你到底是怎么听见说
的?”平儿道:“就是头里那小丫头子的话。他说他在二门里头,听见外头两个小
厮说:‘这个新二奶奶比咱们旧二奶奶还俊呢,脾气儿也好。’不知是旺儿是谁,
吆喝了两个一顿,说:‘什么新奶奶旧奶奶的,还不快悄悄儿的呢!叫里头知道了,
把你的舌头还割了呢。’”平儿正说着,只见一个小丫头进来,回说:“旺儿在外
头伺候着呢。”凤姐听了,冷笑了一声,说:“叫他进来!”那小丫头出来说:“奶
奶叫呢。”旺儿连忙答应着进来。

旺儿请了安,在外间门口垂手侍立。凤姐儿道:“你过来!我问你话。”旺儿
才走到里间门旁站着。凤姐儿道:“你二爷在外头弄了人,你知道不知道?”旺儿
又打着千儿,回道:“奴才天天在二门上听差事,如何能知道二爷外头的事呢?”
凤姐冷笑道:“你自然‘不知道’!你要知道,你怎么拦人呢!”旺儿见这话,知
道刚才的话已经走了风了,料着瞒不过,便又跪回道:“奴才实在不知,就是头里
兴儿和喜儿两个人在那里混说,奴才吆喝了他们两句。内中深情底里,奴才不知道,
不敢妄回,求奶奶问兴儿,他是长跟二爷出门的。”凤姐儿听了,下死劲啐了一口,
骂道:“你们这一起没良心的混账忘八崽子,都是一条藤儿!打量我不知道呢。先
去给我把兴儿那个忘八崽子叫了来,你也不许走!问明白了他,回来再问你。好,
好,好,这才是我使出来的好人呢!”那旺儿只得连声答应几个“是”,磕了个头,
爬起来出去,去叫兴儿。

却说兴儿正在帐房儿里和小厮们玩呢,听见说“二奶奶叫”,先唬了一跳。却
也想不到是这件事发作了,连忙跟着旺儿进来。旺儿先进去,回说:“兴儿来了。”
凤姐儿厉声道:“叫他!”那兴儿听见这个声音儿,早已没了主意了,只得乍着胆
子进来。凤姐儿一见便说:“好小子啊,你和你爷办的好事啊。你只实说罢!”兴
儿一闻此言,又看见凤姐儿气色,及两边丫头们的光景,早唬软了,不觉跪下,只
是磕头。凤姐儿道:“论起这事来,我也听见说不与你相干,但只你不早来回我知
道,这就是你的不是了。你要实说了,我还饶你;再有一句虚言,你先摸摸你腔子
上几个脑袋瓜子!”兴儿战兢兢的朝上磕头道:“奶奶问的是什么事,奴才和爷办
坏了?”凤姐听了,一腔火都发作起来,喝命:“打嘴巴!”旺儿过来才要打时,
凤姐儿骂道:“什么糊涂忘八崽子!叫他自己打,用你打吗?一会子你再各人打你的
嘴巴子还不迟呢。”那兴儿真个自己左右开弓,打了自己十几个嘴巴。凤姐儿喝声
“站住”,问道:“你二爷外头娶了什么‘新奶奶’‘旧奶奶’的事,你大概不知
道啊?”兴儿见说出这件事来,越发着了慌,连忙把帽子抓下来,在砖地上咕咚咕
咚碰的头山响,口里说道:“只求奶奶超生!奴才再不敢撒一个字儿的谎。”凤姐
道:“快说!”

兴儿直蹶蹶的跪起来回道:“这事头里奴才也不知道。就是这一天东府里大老
爷送了殡,俞禄往珍大爷庙里去领银子,二爷同着蓉哥儿到了东府里,道儿上,爷
儿两个说起珍大奶奶那边的二位姨奶奶来,二爷夸他好,蓉哥儿哄着二爷,说把二
姨奶奶说给二爷
——”凤姐听到这里,使劲啐道:“呸!没脸的忘八蛋!他是你那一门子的姨奶奶?”
兴儿忙又磕头说:“奴才该死。”往上瞅着,不敢言语。凤姐儿道:“完了吗?怎
么不说了?”兴儿方才又回道:“奶奶恕奴才,奴才才敢回。”凤姐啐道:“放你
妈的屁!这还什么‘恕’不‘恕’了。你好生给我往下说,好多着呢!”兴儿又回
道:“二爷听见这个话,就喜欢了。后来奴才也不知道怎么就弄真了。”凤姐微微
冷笑道:“这个自然么,你可那里知道呢?你知道的,只怕都烦了呢!——是了,说
底下的罢。”兴儿回道:“后来就是蓉哥儿给二爷找了房子。”凤姐忙问道:“如
今房子在那里?”兴儿道:“就在府后头。”凤姐儿道:“哦!”回头瞅着平儿,
道:“咱们都是死人哪,你听听!”平儿也不敢作声。

兴儿又回道:“珍大爷那边给了张家不知多少银子,那张家就不问了。”凤姐
道:“这里头怎么又扯拉上什么张家李家咧呢?”兴儿回道:“奶奶不知道。这二
奶奶——”刚说到这里,又自己打了个嘴巴,把凤姐儿倒怄笑了,两边的丫头也都
抿嘴儿笑。兴儿想了想,说道:“那珍大奶奶的妹子——”凤姐儿接着道:“怎么
样?快说呀!”兴儿道:“那珍大奶奶的妹子原来从小儿有人家的,姓张,叫什么
张华,如今穷的待好讨饭。珍大爷许了他银子,他就退了亲了。”凤姐儿听到这里,
点了点头儿,回头便望丫头们说道:“你们都听见了?小忘八崽子,头里他还说他
不知道呢。”兴儿又回道:“后来二爷才叫人裱糊了房子,娶过来了。”凤姐道:
“打那里娶过来的?”兴儿回道:“就在他老娘家抬过来的。”凤姐道:“好罢咧!”
又问:“没人送亲么?”兴儿道:“就是蓉哥儿,还有几个丫头老婆子们,没别人。”
凤姐道:“你大奶奶没来吗?”兴儿道:“过了两天,大奶奶才拿了些东西来瞧的。”
凤姐儿笑了一笑,回头向平儿道:“怪道那两天二爷称赞大奶奶不离嘴呢。”掉过
脸来,又问兴儿:“谁伏侍呢?自然是你了?”兴儿赶着碰头,不言语。凤姐又问:
“前头那些日子,说给那府里办事,想来办的就是这个了?”兴儿回道:“也有办
事的时候,也有往新房子里去的时候。”凤姐又问道:“谁和他住着呢?”兴儿道:
“他母亲和他妹子。昨儿他妹子自己抹了脖子了。”凤姐道:“这又为什么?”兴
儿随将柳湘莲的事说了一遍。凤姐道:“这个人还算造化高,省了当那出名儿的忘
八。”因又问道:“没了别的事了么?”兴儿道:“别的事奴才不知道。奴才刚才
说的,字字是实话。一字虚假,奶奶问出来,只管打死奴才,奴才也无怨的。”

凤姐低了一回头,便又指着兴儿说道:“你这个猴儿崽子,就该打死!这有什
么瞒着我的?你想着瞒了我,就在你那糊涂爷跟前讨了好儿了,你新奶奶好疼你。
我不看你刚才还有点怕惧儿不敢撒谎,我把你的腿不给你砸折了呢!”说着,喝声
起去,兴儿磕了个头,才爬起来,退到外间门口不敢就走。凤姐道:“过来!我还
有话呢。”兴儿赶忙垂手敬听。凤姐道:“你忙什么?新奶奶等着赏你什么呢?”
兴儿也不敢抬头。凤姐道:“你从今日不许过去!我什么时候叫你,你什么时候到。
迟一步儿,你试试!出去罢!”兴儿忙答应几个“是”,退出门来。凤姐又叫道:
“兴儿!”兴儿赶忙答应回来。凤姐道:“快出去告诉你二爷去,是不是啊?”兴
儿回道:“奴才不敢。”凤姐道:“你出去提一个字儿,提防你的皮。”兴儿连忙
答应着,才出去了。凤姐又叫:“旺儿呢?”旺儿连忙答应着过来。凤姐把眼直瞪
瞪的瞅了两三句话的工夫,才说道:“好,旺儿!很好!去罢!外头有人提一个字儿,
全在你身上!”旺儿答应着,也慢慢的退出去了。凤姐便叫:“倒茶。”小丫头子
们会意,都出去了。

这里凤姐才和平儿说:“你都听见了?这才好呢!”平儿也不敢答言,只好陪
笑儿。凤姐越想越气,歪在枕上,只是出神。忽然眉头一皱,计上心来,便叫平儿
来。平儿连忙答应过来,凤姐道:“我想这件事,竟该这么着才好,也不必等你二
爷回来再商量了。”

未知凤姐如何办理,下回分解。