\chapter{得通灵幻境悟仙缘~送慈柩故乡全孝道}

话说宝玉一听麝月的话,身往后仰,复又死去,急得王夫人等哭叫不止。麝月
自知失言致祸,此时王夫人等也不及说他。那麝月一面哭着,一面打算主意,心想:
“若是宝玉一死,我便自尽,跟了他去。”

不言麝月心里的事。且说王夫人等见叫不回来,赶着叫人出来找和尚救治。岂
知贾政进内出去时,那和尚已不见了。贾政正在诧异,听见里头又闹,急忙进来,
见宝玉又是先前的样子,牙关紧闭,脉息全无。用手在心窝中一摸,尚是温热。贾
政只得急忙请医,灌药救治。那知那宝玉的魂魄早已出了窍了。你道死了不成?却
原来恍恍惚惚赶到前厅,见那送玉的和尚坐着,便施了礼。那和尚忙站起身来,拉
着宝玉就走。宝玉跟了和尚,觉得身轻如叶,飘飘摇摇,也没出大门,不知从那里
走出来了。

行了一程,到了个荒野地方,远远的望见一座牌楼,好像曾到过的。正要问那
和尚,只见恍恍惚惚又来了一个女人。宝玉心里想道:“这样旷野地方,那得有如
此的丽人?必是神仙下界了。”宝玉想着,走近前来,细细一看,竟有些认得的,
只是一时想不起来。见那女人合和尚打了一个照面,就不见了。宝玉一想,竟是尤
三姐的样子,越发纳闷:“怎么他也在这里?”又要问时,那和尚早拉着宝玉过了
牌楼。只见牌上写着“真如福地”四个大字,两边一副对联,乃是:
假去真来真胜假,
无原有是有非无。
转过牌坊,便是一座宫门。门上也横书着四个大字道:“福善祸淫。”又有一副对
联,大书云:
过去未来,莫谓智贤能打破;
前因后果,须知亲近不相逢。
宝玉看了,心下想道:“原来如此,我倒要问问因果来去的事了。”这么一想,只
见鸳鸯站在那里,招手儿叫他。宝玉想道:“我走了半日,原不曾出园子,怎么改
了样儿了呢?”赶着要合鸳鸯说话,岂知一转眼便不见了,心里不免疑惑起来。走
到鸳鸯站的地方儿,乃是一溜配殿,各处都有匾额。宝玉无心去看,只向鸳鸯立的
所在奔去,见那一间配殿的门半掩半开。宝玉也不敢造次进去,心里正要问那和尚
一声,回过头来,和尚早已不见了。宝玉恍惚见那殿宇巍峨,绝非大观园景象,便
立住脚,抬头看那匾额上写道:“引觉情痴。”两边写的对联道:
喜笑悲哀都是假,
贪求思慕总因痴。
宝玉看了,便点头叹息。想要进去找鸳鸯,问他是什么所在。细细想来,甚是熟识,
便仗着胆子推门进去。满屋一瞧,并不见鸳鸯,里头只是黑漆漆的,心下害怕。正
要退出,见有十数个大橱,橱门半掩。宝玉忽然想起:“我少时做梦,曾到过这样
个地方;如今能够亲身到此,也是大幸。”恍惚间,把找鸳鸯的念头忘了,便仗着
胆子把上首大橱开了橱门一瞧,见有好几本册子。心里更觉喜欢,想道:“大凡人
做梦,说是假的,岂知有这梦便有这事!我常说还要做这个梦再不能的,不料今儿
被我找着了。但不知那册子是那个见过的不是。”伸手在上头取了一本,册上写着
“金陵十二钗正册”。宝玉拿着一想道:“我恍惚记得是那个,只恨记得不清楚。”
便打开头一页看去。见上头有画,但是画迹模糊,再瞧不出来。后面有几行字迹,
也不清楚,尚可摹拟,便细细的看去。见有什么玉带上头有个好像“林”字,心里
想道:“莫不是说林妹妹罢?”便认真看去。底下又有“金簪雪里”四字,诧异道:
“怎么又像他的名字呢?”复将前后四句合起来一念道:“也没有什么道理,只是
暗藏着他两个名字,并不为奇。独有那‘怜’字‘叹’字不好。这是怎么解?”想
到那里,又啐道:“我是偷着看,若只管呆想起来,倘有人来,又看不成了。”遂
往后看,也无暇细玩那画图,只从头看去。看到尾上有几句词,什么“虎兔相逢大
梦归”一句,便恍然大悟道:“是了,果然机关不爽。这必是元春姐姐了。若都是
这样明白,我要抄了去细玩起来,那些姊妹们的寿夭穷通,没有不知的了。我回去
自不肯泄漏,只做一个未卜先知的人,也省了多少闲想。”又向各处一瞧,并没有
笔砚。又恐人来,只得忙着看去。只见图上影影有一个放风筝的人儿,也无心去看。
急急的将那十二首诗词都看遍了,也有一看便知的,也有一想便得的,也有不大明
白的,心下牢牢记着。一面叹息,一面又取那“金陵又副册”一看。看到“堪羡优
伶有福,谁知公子无缘”,先前不懂,见上面尚有花席的影子,便大惊痛哭起来。

待要往后再看,听见有人说道:“你又发呆了,林妹妹请你呢。”好似鸳鸯的
声气,回头却不见人。心中正自惊疑,忽鸳鸯在门外招手。宝玉一见,喜得赶出来,
但见鸳鸯在前,影影绰绰的走,只是赶不上。宝玉叫道:“好姐姐等等我!”那鸳
鸯并不理,只顾前走。宝玉无奈,尽力赶去。忽见别有一洞天,楼阁高耸,殿角玲
珑,且有好些宫女隐约其间。宝玉贪看景致,竟将鸳鸯忘了。宝玉顺步走入一座宫
门,内有奇花异卉,都也认不明白,惟有白石花栏围着一颗青草,叶头上略有红色,
“但不知是何名草,这样矜贵?”只见微风动处,那青草已摆摇不休。虽说是一枝
小草,又无花朵,其妩媚之态,不禁心动神怡,魂消魄丧。宝玉只管呆呆的看着,
只听见旁边有一人说道:“你是那里来的蠢物,在此窥探仙草!”宝玉听了,吃了
一惊,回头看时,却是一位仙女,便施礼道:“我找鸳鸯姐姐,误入仙境,恕我冒
昧之罪。请问神仙姐姐:这里是何地方?怎么我鸳鸯姐姐到此?还说是林妹妹叫我?
望乞明示。”那人道:“谁知你的姐姐妹妹?我是看管仙草的,不许凡人在此逗留。”
宝玉欲待要出来,又舍不得,只得央告道:“神仙姐姐既是那管理仙草的,必然是
花神姐姐了,但不知这草有何好处?”那仙女道:“你要知道这草,说起来话长着
呢。那草本在灵河岸上,名曰‘绛珠草’。因那时萎败,幸得一个神瑛侍者日以甘
露灌溉,得以长生。后来降凡历劫,还报了灌溉之恩,今返归真境。所以警幻仙子
命我看管,不令蜂缠蝶恋。”宝玉听了不解,一心疑定必是遇见了花神了,今日断
不可当面错过,便问:“管这草的是神仙姐姐了。还有无数名花,必有专管的,我
也不敢烦问,只有看管芙蓉花的是那位神仙?”那仙女道:“我却不知,除是我主
人方晓。”宝玉便问道:“姐姐的主人是谁?”那仙女道:“我主人是潇湘妃子。”
宝玉听道:“是了,你不知道,这位妃子就是我的表妹林黛玉。”那仙女道:“胡
说!此地乃上界神女之所,虽号为潇湘妃子,并不是娥皇女英之辈,何得与凡人有
亲?你少来混说!瞧着叫力士打你出去。”

宝玉听了发怔,只觉自形秽浊。正要退出,又听见有人赶来,说道:“里面叫
请神瑛侍者。”那人道:“我奉命等了好些时,总不见有神瑛侍者过来,你叫我那
里请去?”那一个笑道:“才退去的不是么?”那侍女慌忙赶出来,说:“请神瑛
侍者回来。”宝玉只道是问别人,又怕被人追赶,只得踉跄而逃。正走时,只见一
人手提宝剑,迎面拦住,说:“那里走!”吓得宝玉惊惶无措。仗着胆抬头一看,
却不是别人,就是尤三姐。宝玉见了,略定些神,央告道:“姐姐,怎么你也来逼
起我来了?”那人道:“你们弟兄没有一个好人:败人名节,破人婚姻,今儿你到
这里,是不饶你的了!”宝玉听了话头不好,正自着急,只听后面有人叫道:“姐
姐快快拦住,不要放他走了。”尤三姐道:“我奉妃子之命,等候已久。今儿见了,
必定要一剑斩断你的尘缘!”宝玉听了,益发着忙,又不懂这些话到底是什么意思,
只得回头要跑。

岂知身后说话的并非别人,却是晴雯。宝玉一见,悲喜交集,便说:“我一个
人走迷了道儿,遇见仇人,我要逃回,却不见你们一人跟着我。如今好了,晴雯姐
姐,快快的带我回家去罢!”晴雯道:“侍者不必多疑。我非晴雯,我是奉妃子之
命,特来请你一会,并不难为你。”宝玉满腹狐疑,只得问道:“姐姐说是妃子叫
我,那妃子究是何人?”晴雯道:“此时不必问,到了那里自然知道。”宝玉没法,
只得跟着走。细看那人背后举动,恰是晴雯,“那面目声音是不错的了,怎么他说
不是?我此时心里模糊,且别管他。到了那边,见了妃子,就有不是,那时再求他。
到底女人的心肠是慈悲的,必定恕我冒失。”正想着,不多时到了一个所在,只见
殿宇精致,彩色辉煌,庭中一丛翠丛,户外数本苍松。廊檐下立着几个侍女都是宫
妆打扮,见了宝玉进来,便悄悄的说道:“这就是神瑛侍者么?”引着宝玉的说道:
“就是,你快进去通报罢。”

有一侍女笑着招手,宝玉便跟着进去。过了几层房舍,见一正房,珠帘高挂。
那侍女说:“站着候旨。”宝玉听了,也不敢则声,只好在外等着。那侍女进去不
多时,出来说:“请侍者参见。”又有一人卷起珠帘。只见一女子头戴花冠,身穿
绣服,端坐在内。宝玉略一抬头,见是黛玉的形容,便不禁的说道:“妹妹在这里,
叫我好想!”那帘外的侍女悄咤道:“这侍者无礼,快快出去!”说犹未了,又见
一个侍儿将珠帘放下。宝玉此时欲待进去又不敢,要走又不舍,待要问明,见那些
侍女并不认得,又被驱逐,无奈出来。心想要问晴雯,回头四顾,并不见有晴雯。
心下狐疑,只得怏怏出来,又无人引着。正欲找原路而去,却又找不出旧路了。

正在为难,见凤姐站在一所房檐下招手儿。宝玉看见,喜欢道:“可好了,原
来回到自己家里了。怎么一时迷乱如此?”急奔前来说:“姐姐在这里么?我被这
些人捉弄到这个分儿,林妹妹又不肯见我,不知是何原故?”说着,走到凤姐站的
地方,细看起来,并不是凤姐,原来却是贾蓉的前妻秦氏。宝玉只得立住脚,要问
凤姐姐在那里。那秦氏也不答言,竟自往屋里去了。宝玉恍恍惚惚的,又不敢跟进
去,只得呆呆的站着,叹道:“我今儿得了什么不是,众人都不理我!”便痛哭起
来。见有几个黄巾力士执鞭赶来,说是:“何处男人,敢闯入我们这天仙福地来!
快走出去!”宝玉听得,不敢言语。正要寻路出来,远远望见一群女子,说笑前来。
宝玉看时,又像是迎春等一干人走来,心里喜欢,叫道:“我迷住在这里,你们快
来救我!”正嚷着,后面力士赶来,宝玉急得往前乱跑。忽见那一群女子都变作鬼
怪形象,也来追扑。

宝玉正在情急,只见那送玉来的和尚,手里拿着一面镜子一照,说道:“我奉
元妃娘娘旨意,特来救你。”登时鬼怪全无,仍是一片荒郊。宝玉拉着和尚说道:
“我记得是你领我到这里,你一时又不见了。看见了好些亲人,只是都不理我,忽
又变作鬼怪。到底是梦是真?望老师明白指示。”那和尚道:“你到这里,曾偷看
什么东西没有?”宝玉一想,道:“他既能带我到天仙福地,自然也是神仙了,如
何瞒得他?况且正要问个明白。”便道:“我倒见了好些册子来着。”那和尚道:
“可又来。你见了册子,还不解么?世上的情缘,都是那些魔障,只要把历过的事
情细细记着,将来我与你说明。”说着,把宝玉狠命的一推,说:“回去罢。”宝
玉站不住脚,一跤跌倒,口里嚷道:“阿哟!”

众人正在哭泣,听见宝玉苏来,连忙叫唤。宝玉睁眼看时,仍躺在炕上,见王
夫人宝钗等哭的眼泡红肿。定神一想,心里说道:“是了,我是死去过来的。”遂
把神魂所历的事呆呆的细想。幸喜多还记得,便哈哈的笑道:“是了,是了。”王
夫人只道旧病复发,便好延医调治,即命丫头婆子快去告诉贾政,说是:“宝玉回
过来了。头里原是心迷住了,如今说出话来,不用备办后事了。”贾政听了,即忙
进来看视,果见宝玉苏来,便道:“没福的痴儿!你要唬死谁么?”说着,眼泪也
不知不觉流下来了。又叹了几口气,仍出去叫人请医生,诊脉服药。

这里麝月正思自尽,见宝玉一过来,也放了心。只见王夫人叫人端了桂圆汤,
叫他喝了几口,渐渐的定了神。王夫人等放心,也没有说麝月,只叫人仍把那玉交
给宝钗给他带上。想起那和尚来,“这玉不知那里找来的?也是古怪:怎么一时要
银,一时又不见了?莫非是神仙不成?”宝钗道:“说起那和尚来的踪迹、去的影
响,那玉并不是找来的。头里丢的时候,必是那和尚取去的。”王夫人道:“玉在
家里,怎么能取的了去?”宝钗道:“既可送来,就可取去。”袭人麝月道:“那
年丢了玉,林大爷测了个字,后来二奶奶过了门,我还告诉过二奶奶,说测的那字
是什么‘赏’字。二奶奶还记得么?”宝钗想道:“是了,你们说测的是当铺里找
去,如今才明白了,竟是个和尚的‘尚’字在上头,可不是和尚取了去的么?”王
夫人道:“那和尚本来古怪!那年宝玉病的时候,那和尚来说是我们家有宝贝可解,
说的就是这块玉了。他既知道,自然这块玉到底有些来历。况且你女婿养下来就嘴
里含着的,古往今来,你们听见过这么第二个么?只是不知终久这块玉到底怎么着,
就连咱们这一个,也还不知是怎么着呢。病也是这块玉,好也是这块玉,生也是这
块玉——”说到这里,忽然住了,不免又流下泪来。宝玉听了,心里却也明白,更
想死去的事,愈加有因,只不言语,心里细细的记忆。

那时惜春便说道:“那年失玉,还请妙玉请过仙,说是‘青埂峰下倚古松’,
还有什么‘入我门来一笑逢’的话。想起来‘入我门’三字,大有讲究。佛教法门
最大,只怕二哥哥不能入得去。”宝玉听了,又冷笑几声。宝钗听着,不觉的把眉
头儿揪着发起怔来。尤氏道:“偏你一说又是佛门了,你出家的念头还没有歇
么?”惜春笑道:“不瞒嫂子说,我早已断了荤了。”王夫人道:“好孩子,阿弥
陀佛,这个念头是起不得的!”惜春听了,也不言语。宝玉想“青灯古佛旁”的诗
句,不禁连叹几声。忽又想起一床席、一枝花的诗句来,拿眼睛看着袭人,不觉又
流下泪来。众人都见他忽笑忽悲,也不解是何意,只道是他的旧病;岂知宝玉触处
机来,竟能把偷看册上的诗句牢牢记住了,只是不说出来,心中早有一家成见在那
里了,暂且不提。

且说众人见宝玉死去复生,神气清爽,又加连日服药,一天好似一天,渐渐的
复原起来。便是贾政见宝玉已好,现在丁忧无事,想起贾赦不知几时遇赦,老太太
的灵柩久停寺内,终不放心,欲要扶柩回南安葬,便叫了贾琏来商议。贾琏便道:
“老爷想的极是。如今趁着丁忧,干了这件大事更好。将来老爷起了服,只怕又不
能遂意了。但是我父亲不在家,侄儿又不敢僭越。老爷的主意很好,只是这件事也
得好几千银子。衙门里缉赃,那是再缉不出来的。”贾政道:“我的主意是定了。
只为大老爷不在家,叫你来商议商议,怎么个办法。你是不能出门的,现在这里没
有人;我想好几口材,都要带回去,我一个人怎么能够照应?想着把蓉哥儿带了去,
况且有他媳妇的棺材,也在里头。还有你林妹妹的,那是老太太的遗言,说跟着老
太太一块儿回去的。我想这一项银子,只好在那里挪借几千,也就够了。”贾琏道:
“如今的人情过于淡薄。老爷呢,又丁忧;我们老爷呢,又在外头。一时借是借不
出来的了,只好拿房地文书出去押去。”贾政道:“住的房子是官盖的,那里动得?”
贾琏道:“住房是不能动的。外头还有几所可以出脱的,等老爷起复后再赎也使得。
将来我父亲回来了,倘能也再起用,也好赎的。只是老爷这么大年纪,辛苦这一场,
侄儿们心里却不安。”贾政道:“老太太的事是应该的。只要你在家谨慎些,把持
定了才好。”贾琏道:“老爷这倒只管放心,侄儿虽糊涂,断不敢不认真办理的。
况且老爷回南,少不得多带些人去,所留下的人也有限了,这点子费用还可以过的
来。就是老爷路上短少些,必经过赖尚荣的地方,可以叫他出点力儿。”贾政道:
“自己老人家的事,叫人家帮什么呢?”贾琏答应了个“是”,便退出来,打算银
钱。

贾政便告诉了王夫人,叫他管了家,自己择了发引长行的日子,就要起身。宝
玉此时身体复元,贾环贾兰倒认真念书:贾政都交付给贾琏,叫他管教:“今年是
大比的年头,环儿是有服的,不能入场;兰儿是孙子,服满了也可以考的,务必叫
宝玉同着侄儿考去。能够中一个举人,也好赎一赎咱们的罪名。”贾琏等唯唯应命。
贾政又吩咐了在家的人,说了好些话,才别了宗祠,便在城外念了几天经,就发引
下船,带了林之孝等而去。也没有惊动亲友,惟有自家男女送了一程回来。

宝玉因贾政命他赴考,王夫人便不时催逼,查考起他的工课来。那宝钗袭人时
常劝勉,自不必说。那知宝玉病后,虽精神日长,他的念头一发更奇僻了,竟换了
一种,不但厌弃功名仕进,竟把那儿女情缘也看淡了好些。只是众人不大理会,宝
玉也并不说出来。

一日,恰遇紫鹃送了林黛玉的灵柩回来,闷坐自己屋里啼哭,想着:“宝玉无
情,见他林妹妹的灵柩回去,并不伤心落泪;见我这样痛哭,也不来劝慰,反瞅着
我笑。这样负心的人,从前都是花言巧语来哄着我们。前夜亏我想得开,不然几乎
又上了他的当!只是一件叫人不解:如今我看他待袭人也是冷冷儿的。二奶奶是本
来不喜欢亲热的,麝月那些人就不抱怨他么?看来女孩儿们多半是痴心的,白操了
那些时的心,不知将来怎样结局!”正想着,只见五儿走来瞧他。见紫鹃满面泪痕,
便说:“姐姐又哭林姑娘了?我想一个人,闻名不如眼见。头里听着,二爷女孩子
跟前是最好的,我母亲再三的把我弄进来;岂知我进来了,尽心竭力的伏侍了几次
病,如今病好了,连一句好话也没有剩出来,这会子索性连正眼儿也不瞧了。”紫
鹃听他说的好笑,便噗嗤的一笑,啐道:“呸!你这小蹄子,你心里要宝玉怎么样
待你才好?女孩儿家也不害臊。人家明公正气的屋里人他瞧着还没事人一大堆呢,
有功夫理你去?”因又笑着拿个指头往脸上抹着问道:“你到底算宝玉的什么人
那?”那五儿听了自知失言,便飞红了脸。待要解说不是要宝玉怎样看待,说他近
来不怜下的话,只听院门外乱嚷,说:“外头和尚又来了,要那一万银子呢!太太
着急,叫琏二爷和他讲去,偏偏琏二爷又不在家。那和尚在外头说些疯话,太太叫
请二奶奶过去商量。”

不知怎样打发那和尚,下回分解。