\chapter{憨湘云醉眠芍药~呆香菱情解石榴裙}

话说平儿出来吩咐林之孝家的道:“‘大事化为小事,小事化为没事’,方是
兴旺之家。要是一点子小事便扬铃打鼓乱折腾起来,不成道理。如今将他母女带回,
照旧去当差,将秦显家的仍旧追回。再不必提此事,只是每日小心巡察要紧。”说
毕起身走了。柳家的母女忙向上磕头。林家的就带回园中,回了李纨探春。二人都
说:“知道了。宁可无事,很好。”

司棋等人空兴头了一阵。那秦显家的好容易等了这个空子钻了来,只兴头了半
天,在厨房内正乱着收家伙、米粮、煤炭等物。又查出许多亏空来,说:“粳米短
了两担,长用米又多支了一个月的,炭也欠着额数。”一面又打点送林之孝的礼,
悄悄的备了一篓炭一担粳米在外边,就遣人送到林家去了。又打点送帐房儿的礼,
又备几样菜蔬请几位同事的人,说:“我来了,全仗你们列位扶持。自今以后,都
是一家人了,我有照顾不到的,好歹大家照顾些。”正乱着,忽有人来说:“你看
完了这一顿早饭就出去罢。柳嫂儿原无事,如今还交给他管了。”秦显家的听了,
轰去了魂魄,垂头丧气,登时掩旗息鼓,卷包而去。送人之物白白去了许多,自己
倒要折变了赔补亏空。连司棋都气了个直眉瞪眼,无计挽回,只得罢了。

赵姨娘正因彩云私赠了许多东西,被玉钏儿吵出,生恐查问出来,每日捏着一
把汗,偷偷的打听信儿。忽见彩云来告诉,说都是宝玉应了,从此无事,赵姨娘方
把心放下来。谁知贾环听如此说,便起了疑心,将彩云凡私赠之物都拿出来了,照
着彩云脸上摔了来,说:“你这两面三刀的东西,我不希罕!你不和宝玉好,他怎
么肯替你应?你既有担当给了我,原该不叫一个人知道,如今你既然告诉了他,我
再要这个也没趣儿!”彩云见如此,急的赌咒起誓,至于哭了。百般解说,贾环执
意不信,说:“不看你素日,我索性去告诉二嫂子,就说你偷来给我,我不敢要。
你细想去罢!”说毕摔手出去了。急的赵姨娘骂:“没造化的种子,这是怎么说!”
气的彩云哭了个泪干肠断。赵姨娘百般的安慰他:“好孩子,他辜负了你的心,我
横竖看的真。我收起来,过两日,他自然回转过来了。”说着,便要收东西。彩云
赌气一顿卷包起来,趁人不见,来至园中,都撇在河内,顺水沉的沉漂的漂了。自
己气的夜里在被内暗哭了一夜。

当下又值宝玉生日已到。原来宝琴也是这日,二人相同。王夫人不在家,也不
曾像往年热闹,只有张道士送了四样礼,换的寄名符儿,还有几处僧尼庙的和尚姑
子送了供尖儿,并寿星、纸马、疏头,并本宫星官、值年太岁、周岁换的锁。家中
常走的男女,先一日来上寿。王子胜那边,仍是一套衣服,一双鞋袜,一百寿桃,
一百束上用银丝挂面。薛姨妈处减一半。其馀家中尤氏仍是一双鞋袜,凤姐儿是一
个宫制四面扣合堆绣荷包装一个金寿星,一件波斯国的玩器。各庙中遣人去放堂舍
钱。又另有宝琴之礼,不能备述。姐妹中皆随便,或有一扇的,或有一字的,或有
一画的,或有一诗的,聊为应景而已。

这日宝玉清晨起来梳洗已毕,便冠带了来至前厅院中,已有李贵等四个人在那
里设下天地香烛。宝玉炷了香,行了礼,奠茶烧纸后,便至宁府中宗祠祖先堂两处
行毕了礼。出至月台上,又朝上遥拜过贾母、贾政、王夫人等。一顺到尤氏上房,
行过礼,坐了一回,方回荣府。先至薛姨妈处,再三拉着,然后又见过薛蝌,让一
回,方进园来。晴雯麝月二人跟随,小丫头夹着毡子,从李氏起,一一挨着,比自
己长的房中到过;复出二门,至四个奶妈家让了一回,方进来。虽众人要行礼,也
不曾受,回至房中,袭人等只都来说一声就是了。王夫人有言,不令年轻人受礼,
恐折了福寿,故此皆不磕头。

一时贾环贾兰来了,袭人连忙拉住,坐了一坐,便去了。宝玉笑道:“走乏了!”
便歪在床上。方吃了半盏茶,只听外头咭咭呱呱,一群丫头笑着进来,原来是翠墨、
小螺、翠缕、入画,邢岫烟的丫头篆儿,并奶子抱着巧姐儿,彩鸾、绣鸾八九个人,
都抱着红毡子来了,笑说道:“拜寿的挤破了门了,快拿面来我们吃。”刚进来时,
探春、湘云、宝琴、岫烟、惜春也都来了。宝玉忙迎出来,笑说:“不敢起动。—
—快预备好茶!”进入房中,不免推让一回,大家归坐。袭人等捧过茶来,才吃了
一口,平儿也打扮的花枝招展的来了。宝玉忙迎出来,笑说:“我方才到凤姐姐门
上,回进去,说不能见我;我又打发进去让姐姐来着。”平儿笑道:“我正打发你
姐姐梳头,不得出来回你。后来听见又说让我,我那里禁当的起?所以特给二爷来
磕头。”宝玉笑道:“我也禁当不起。”袭人早在门旁安了座让他坐。平儿便拜下
去,宝玉作揖不迭;平儿又跪下去,宝玉也忙还跪下,袭人连忙搀起来;又拜了一
拜,宝玉又还了一揖。袭人笑推宝玉:“你再作揖。”宝玉道:“已经完了,怎么
又作揖?”袭人笑道:“这是他来给你拜寿。今日也是他的生日,你也该给他拜寿。”
宝玉喜的忙作揖,笑道:“原来今日也是姐姐的好日子!”平儿赶着也还了礼。湘
云拉宝琴岫烟说:“你们四个人对拜寿,直拜一天才是。”探春忙问:“原来邢妹
妹也是今日?我怎么就忘了。”忙命丫头:“去告诉二奶奶,赶着补了一分礼,和
琴姑娘的一样,送到二姑娘屋里去。”丫头答应着去了。岫烟见湘云直口说出来,
少不得要到各房去让让。

探春笑道:“倒有些意思。一年十二个月,月月有几个生日。人多了就这样巧,
也有三个一日的,两个一日的。大年初一也不白过,大姐姐占了去,怨不得他福大,
生日比别人都占先;又是大祖太爷的生日冥寿。过了灯节,就是大太太和宝姐姐,
他们娘儿两个遇的巧。三月初一是太太的,初九是琏二哥哥。二月没人。”袭人道:
“二月十二是林姑娘,怎么没人?只不是咱们家的。”探春笑道:“你看我这个记
性儿。”宝玉笑指袭人道:“他和林妹妹是一日,他所以记得。”探春笑道:“原
来你两个倒是一日?每年连头也不给我们磕一个!平儿的生日我们也不知道,这也是
才知道的。”平儿笑道:“我们是那牌儿名上的人?生日也没拜寿的福,又没受礼
的职分,可吵嚷什么,可不悄悄儿的就过去了吗。今日他又偏吵出来了。等姑娘回
房,我再行礼去罢。”探春笑道:“也不敢惊动。只是今日倒要替你作个生日,我
心里才过的去。”宝玉湘云等一齐都说很是。探春便吩咐了丫头去告诉他奶奶说:
“我们大家说了,今日一天不放平儿出去,我们也大家凑了分子过生日呢。”丫头
笑着去了,半日回来说:“二奶奶说了,多谢姑娘们给他脸。不知过生日给他些什
么吃?只别忘了二奶奶,就不来絮聒他了。”众人都笑了。探春因说道:“可巧今
日里头厨房不预备饭,一应下面弄菜都是外头收拾。咱们就凑了钱,叫柳家的来领
了去,只在咱们里头收拾倒好。”众人都说:“很好。”

探春一面遣人去请李纨、宝钗、黛玉,一面遣人去传柳家的进来,吩咐他内厨
房中快收拾两桌酒席。柳家的不知何意,因说:“外厨房都预备了。”探春笑道:
“你原来不知道,今日是平姑娘的好日子,外头预备的是上头的,这如今我们私下
又凑了分子,单为平姑娘预备两桌请他。你只管拣新巧的菜蔬预备了来,开了帐我
那里领钱。”柳家的笑道:“今日又是平姑娘的千秋?我们竟不知道。”说着,便
给平儿磕头,慌得平儿拉起他来。柳家的忙去预备酒席。这里探春又邀了宝玉同到
厅上去吃面,等到李纨宝钗一齐来全,又遣人去请薛姨妈和黛玉。因天气和暖,黛
玉之疾渐愈,故也来了。花团锦簇,挤了一厅的人。

谁知薛蝌又送了巾扇香帛四色寿礼给宝玉,宝玉于是过去陪他吃面。两家皆办
了寿酒,互相酬送,彼此同领。至午间,宝玉又陪薛蝌吃了两杯酒。宝钗带了宝琴
过来给薛蝌行礼,把盏毕,宝钗因嘱咐薛蝌:“家里的酒也不用送过那边去,这虚
套竟收了。你只请伙计们吃罢。我们和宝兄弟进去,还要待人去呢,也不能陪你了。”
薛蝌忙说:“姐姐兄弟只管请,只怕伙计们也就好来了。”宝玉忙又告过罪,方同
他姊妹回来。一进角门,宝钗便命婆子将门锁上,把钥匙要了,自己拿着。宝玉忙
说:“这一道门何必关?又没多的人走,况且姨娘、姐姐、妹妹都在里头,倘或要
家去取什么,岂不费事?”宝钗笑道:“小心没过逾的。你们那边这几日七事八事,
竟没有我们那边的人,可知是这门关的有功效了。要是开着,保不住那起人图顺脚
走近路从这里走,拦谁的是?不如锁了,连妈妈和我也禁着些,大家别走。纵有了
事,也就赖不着这边的人了。”宝玉笑道:“原来姐姐也知道我们那边近日丢了东
西?”宝钗笑道:“你只知道玫瑰露和茯苓霜两件,乃因人而及物,要不是里头有
人,你连这两件还不知道呢。殊不知还有几件比这两件大的呢。若以后叨登不出来,
是大家的造化;若叨登出来了,不知里头连累多少人呢。你也是不管事的人,我才
告诉你。平儿是个明白人,我前日也告诉了他,皆因他奶奶不在外头,所以使他明
白了。若不犯出来,大家落得丢开手;若犯出来,他心里已有了稿儿,自有头绪,
就冤屈不着平人了。你只听我说,以后留神小心就是了。这话也不可告诉第二个
人。”

说着,来到沁芳亭边,只见袭人、香菱、侍书、晴雯、麝月、芳官、蕊官、藕
官十来个人,都在那里看鱼玩呢,见他们来了,都说:“芍药栏里预备下了,快去
上席罢。”宝钗等随携了他们,同到芍药栏中红香圃三间小敞厅内,连尤氏已请过
来了。诸人都在那里,只没平儿。原来平儿出去,有赖林诸家送了礼来,连三接四,
上中下三等家人拜寿送礼的不少。平儿忙着打发赏钱道谢,一面又色色的回明了凤
姐儿,不过留下几样,也有不受的,也有受下即刻赏给人的。忙了一回,又直等凤
姐儿吃过面,方换了衣裳往园里来。刚进了园,就有几个丫鬟来找他,一同到了红
香圃中。只见筵开玳瑁,褥设芙蓉,众人都笑说:“寿星全了!”上面四座,定要
让他们四个人坐。四人皆不肯。

薛姨妈说:“我老天拔地,不合你们的群儿,我倒拘的慌,不如我到厅上随便
躺躺去倒好。我又吃不下什么去,又不大吃酒,这里让他们倒便宜。”尤氏等执意
不从。宝钗道:“这也罢了,倒是让妈妈在厅上歪着自如些。有爱吃的送些过去,
倒还自在。且前头没人在那里,又可照看了。”探春笑道:“既这样,恭敬不如从
命。”因大家送到议事厅上,眼看着命小丫头们铺了一个锦褥并靠背引枕之类,又
嘱咐:“好生给姨太太捶腿。要茶要水,别推三拉四的。回来送了东西来,姨太太
吃了,赏你们吃。只别离了这里。”小丫头子们都答应了,探春等方回来。终久让
宝琴岫烟二人在上,平儿面西坐,宝玉面东坐。探春又接了鸳鸯来,二人并肩对面
相陪。西边一桌,宝钗、黛玉、湘云、迎春、惜春依序,一面又拉了香菱玉钏儿二
人打横。三桌上尤氏李纨,又拉了袭人彩云陪坐。四桌上便是紫鹃、莺儿、晴雯、
小螺、司棋等人团坐。当下探春等还要把盏,宝琴等四人都说:“这一闹,一日也
坐不成了!”方才罢了。两个女先儿要弹词上寿,众人都说:“我们这里没人听那
些野话,你厅上去,说给姨太太解闷儿去罢。”一面又将各色吃食,拣了命人送给
薛姨妈去。

宝玉便说:“雅坐无趣,须要行令才好。”众人中有说行这个令好的,又有说
行那个令才好的。黛玉道:“依我说,拿了笔砚将各色令都写了,拈成阄儿,咱们
抓出那个来就是那个。”众人都道:“妙极!”即命拿了一副笔砚花笺。香菱近日
学了诗,又天天学写字,见了笔砚,便巴不得连忙起来,说:“我写。”众人想了
一回,共得十来个,念着,香菱一一写了。搓成阄儿,掷在一个瓶中,探春便命平
儿拈。平儿向内搅了一搅,用箸夹了一个出来,打开一看,上写着“射覆”二字。
宝钗笑道:“把个令祖宗拈出来了。射覆从古有的,如今失了传。这是后纂的,比
一切的令都难。这里头倒有一半是不会的,不如毁了,另拈一个雅俗共赏的。”探
春笑道:“既拈了出来,如何再毁?如今再拈一个,若是雅俗共赏的,便叫他们行
去,咱们行这一个。”说着,又叫袭人拈了一个,却是“拇战”。湘云先笑着说:
“这个简断爽利,合了我的脾气。我不行这个射覆,没的垂头丧气闷人,我只猜拳
去了。”探春道:“惟有他乱令,宝姐姐快罚他一钟!”宝钗不容分说,笑灌了湘
云一杯。

探春道:“我吃一杯,我是令官;也不用宣,只听我分派。取了骰子令盆来,
从琴妹妹掷起,挨着掷下去,对了点的二人射覆。”宝琴一掷,是个三。岫烟宝玉
等皆掷的不对,直到香菱方掷了个三。宝琴笑道:“只好室内生春,若说到外头去,
可太没头绪了。”探春道:“自然。三次不中者罚一杯。你覆他射。”宝琴想了一
想,说了个“老”字。香菱原生于这令,一时想不到,满室满席都不见有与“老”
字相连的成语。湘云先听了,便也乱看,忽见门斗上贴着“红香圃”三个字,便知
宝琴覆的是“吾不如老圃”的“圃”字。见香菱射不着,众人击鼓又催,便悄悄的
拉香菱,教他说“药”字。黛玉偏看见了,说:“快罚他!又在那里传递呢!”闹
得众人都知道了,忙又罚了一杯,恨的湘云拿筷子敲黛玉的手。于是罚了香菱一杯。
下则宝钗和探春对了点子,探春便覆了一“人”字。宝钗笑道:“这个‘人’字泛
得很。”探春笑道:“添一个字,两覆一射,也不泛了。”说着,便又说了一个“窗”
字。宝钗一想,因见席上有鸡,便猜着他是用“鸡窗”“鸡人”二典了,因射了一
个“埘”字。探春知他射着,用了“鸡栖于埘”的典,二人一笑,各饮一口门杯。

湘云等不得,早和宝玉“三”“五”乱叫猜起拳来。那边尤氏和鸳鸯隔着席,
也“七”“八”乱叫,起拳来。平儿袭人也作了一对。叮叮当当,只听得腕上镯
子响。一时,湘云赢了宝玉,袭人赢了平儿,二人限酒底酒面。湘云便说:“酒面
要一句古文,一句旧诗,一句骨牌名,一句曲牌名,还要一句时宪书上有的话,共
总成一句话。酒底要关人事的果菜名。”众人听了,都说:“惟有他的令比人唠叨!
倒也有些意思。”便催宝玉快说。宝玉笑道:“谁说过这个,也等想一想儿。”黛
玉便道:“你多喝一钟,我替你说。”宝玉真个喝了酒,听黛玉说道:

落霞与孤鹜齐飞,风急江天过雁哀,却是一枝折脚雁,叫得人九回肠,这是鸿
雁来宾。
说得大家笑了。众人说:“这一串子倒有些意思。”黛玉又拈了一个榛瓤,说酒底
道:
榛子非关隔院砧,何来万户捣衣声?
令完。鸳鸯袭人等皆说的是一句俗话,都带一个“寿”字,不须多赘。

大家轮流乱了一阵。这上面湘云又和宝琴对了手,李纨和岫烟对了点子。李纨
便覆了一个“瓢”字,岫烟便射了一个“绿”字,二人会意,各饮一口。湘云的拳
却输了,请酒面酒底。宝琴笑道:“请君入瓮。”大家笑起来,说:“这个典用得
当。”湘云便说道:

奔腾澎湃,江间波浪兼天涌,须要铁索缆孤舟,既遇着一江风,不宜出行。
说的众人都笑了,说:“好个诌断了肠子的!怪道他出这个令,故意惹人笑。”又
催他快说酒底儿。湘云吃了酒,夹了一块鸭肉,呷了口酒,忽见碗内有半个鸭头,
遂夹出来吃脑子。众人催他:“别只顾吃,你到底快说呀。”湘云便用箸子举着说
道:
这鸭头不是那丫头:头上那些桂花油。
众人越发笑起来。引得晴雯小螺等一干人都走过来说:“云姑娘会开心儿,拿着我
们取笑儿,快罚一杯才罢!怎么见得我们就该擦桂花油呢?倒得每人给瓶子桂花油擦
擦。”黛玉笑道:“他倒有心给你们一瓶子油,又怕挂误着打窃盗官司。”众人不
理论,宝玉却明白,忙低了头。彩云心里有病,不觉的红了脸。宝钗忙暗暗的瞅了
黛玉一眼。黛玉自悔失言,原是打趣宝玉的,就忘了村了彩云了,自悔不及,忙一
顿的行令猜拳岔开了。

底下宝玉可巧和宝钗对了点子,宝钗便覆了一个“宝”字,宝玉想了一想,便
知是宝钗作戏,指着自己的通灵玉说的,便笑道:“姐姐拿我作雅谑,我却射着了。
说出来姐姐别恼,就是姐姐的讳——‘钗’字就是了。”众人道:“怎么解?”宝
玉道:“他说‘宝’,底下自然是‘玉’字了。我射‘钗’字,旧诗曾有‘敲断玉
钗红烛冷’,岂不射着了?”湘云说道:“这用时事却使不得,两个人都该罚。”
香菱道:“不止时事,这也是有出处的。”湘云道:“‘宝玉’二字并无出处,不
过是春联上或有之,诗书纪载并无,算不得。”香菱道:“前日我读岑嘉州五言律,
现有一句,说:‘此乡多宝玉。’怎么你倒忘了?后来又读李义山七言绝句,又有
一句:‘宝钗无日不生尘。’我还笑说:他两个名字都原来在唐诗上呢。”众人笑
说:“这可问住了,快罚一杯。”湘云无话,只得饮了。

大家又该对点拳,这些人因贾母王夫人不在家,没了管束,便任意取乐,呼
三喝四,喊七叫八。满厅中红飞翠舞,玉动珠摇,真是十分热闹。玩了一回,大家
方起席散了。却忽然不见了湘云。只当他外头自便就来,谁知越等越没了影儿。使
人各处去找,那里找的着。

接着林之孝家的同着几个老婆子来,一则恐有正事呼唤,二则恐丫鬟们年轻,
趁王夫人不在家,不服探春等约束,恣意痛饮,失了体统,故来请问有事无事。探
春见他们来了,便知其意,忙笑道:“你们又不放心,来查我们来了。我们并没有
多吃酒,不过是大家玩笑,将酒作引子。妈妈们别耽心。”李纨尤氏也都笑说:“你
们歇着去罢,我们也不敢叫他们多吃了。”林之孝家的等人笑说:“我们知道。连
老太太让姑娘们吃酒,姑娘们还不肯吃呢,何况太太们不在家,自然玩罢了。我们
怕有事,来打听打听。二则天长了,姑娘们玩一会子,还该点补些小食儿。素日又
不大吃杂项东西,如今吃一两杯酒,若不多吃些东西,怕受伤。”探春笑道:“妈
妈说的是,我们也正要吃呢。”回头命:“取点心来。”两旁丫鬟们齐声答应了,
忙去传点心。探春又笑让:“你们歇着去,或是姨妈那里说话儿去。我们即刻打发
人送酒你们吃去。”林之孝家的等人笑回:“不敢领了。”又站了一回,方退出去
了。平儿摸着脸笑道:“我的脸都热了,也不好意思见他们。依我说,竟收了罢,
别惹他们再来倒没意思了。”探春笑道:“不相干,横竖咱们不认真喝酒就罢了。”

正说着,只见一个小丫头笑嘻嘻的走来,说:“姑娘们快瞧,云姑娘吃醉了,
图凉快,在山子后头一块青石板磴上睡着了。”众人听说,都笑道:“快别吵嚷。”
说着,都走来看时,果见湘云卧于山石僻处一个石磴子上,业经香梦沈酣。四面芍
药花飞了一身,满头脸衣襟上皆是红香散乱。手中的扇子在地下,也半被落花埋了,
一群蜜蜂蝴蝶闹嚷嚷的围着。又用鲛帕包了一包芍药花瓣枕着。众人看了,又是爱,
又是笑,忙上来推唤搀扶。湘云口内犹作睡语说酒令,嘟嘟囔囔说:“泉香酒冽,……
醉扶归,宜会亲友。”众人笑推他说道:“快醒醒儿,吃饭去。这潮磴上还睡出病
来呢!”湘云慢启秋波,见了众人,又低头看了一看自己,方知是醉了。原是纳凉
避静的,不觉因多罚了两杯酒,娇娜不胜,便睡着了,心中反觉自悔。早有小丫头
端了一盆洗脸水,两个捧着镜奁。众人等着,他便在石磴上重新匀了脸,拢了鬓,
连忙起身,同着来至红香圃中。又吃了两杯浓茶,探春忙命将醒酒石拿来给他衔在
口内,一时又命他吃了些酸汤,方才觉得好了些。

当下又选了几样果菜给凤姐儿送去,凤姐儿也送了几样来。宝钗等吃过点心,
大家也有坐的,也有立的,也有在外观花的,也有倚栏看鱼的,各自取便,说笑不
一。探春便和宝琴下棋,宝钗岫烟观局。黛玉和宝玉在一簇花下唧唧哝哝,不知说
些什么。只见林之孝家的和一群女人,带了一个媳妇进来。那媳妇愁眉泪眼,也不
敢进厅来,到阶下便朝上跪下磕头。探春因一块棋受了敌,算来算去,总得了两个
眼,便折了官着儿,两眼只瞅着棋盘,一只手伸在盒内,只管抓棋子作想。林之孝
家的站了半天。因回头要茶时才看见,问什么事。林之孝家的便指那媳妇说:“这
是四姑娘屋里小丫头彩儿的娘,现是园内伺候的人。嘴很不好,才是我听见了,问
着他,他说的话也不敢回姑娘。竟要撵出去才是。”探春道:“怎么不回大奶奶?”
林之孝家的道:“方才大奶奶往厅上姨太太处去,顶头看见,我已回明白了,叫回
姑娘来。”探春道:“怎么不回二奶奶?”平儿道:“不回去也罢,我回去说一声
就是了。既这么着,就撵他出去,等太太回来再回:请姑娘定夺。”探春点头,仍
又下棋。这里林之孝家的带了那人出去不提。黛玉和宝玉二人站在花下,遥遥盼望,
黛玉便说道:“你家三丫头倒是个乖人。虽然叫他管些事,也倒一步不肯多走,差
不多的人,就早作起威福来了。”宝玉道:“你不知道呢:你病着时,他干了几件
事,这园子也分了人管,如今多掐一根草也不能了。又蠲了几件事,单拿我和凤姐
姐做筏子。最是心里有算计的人,岂止乖呢!”黛玉道:“要这样才好。咱们也太
费了。我虽不管事,心里每常闲了,替他们一算,出的多,进的少,如今若不省俭,
必致后手不接。”宝玉笑道:“凭他怎么后手不接,也不短了咱们两个人的。”

黛玉听了,转身就往厅上寻宝钗说笑去了。宝玉正欲走时,只见袭人走来,手
内捧着一个小连环洋漆茶盘,里面可式放着两钟新茶,因问:“他往那里去呢?我
见你两个半日没吃茶,巴巴的倒了两钟来,他又走了。”宝玉道:“那不是他?你
给他送去。”说着,自拿了一钟。袭人便送了那钟去,偏和宝钗在一处,只得一钟
茶,便说:“那位喝时那位先接了,我再倒去。”宝钗笑道:“我倒不喝,只要一
口漱漱就是了。”说着,先拿起来喝了一口,剩下半杯,递在黛玉手内。袭人笑说:
“我再倒去。”黛玉笑道:“你知道我这病,大夫不许多吃茶,这半钟尽够了,难
为你想的到。”说毕饮干,将杯放下。袭人又来接宝玉的。宝玉因问:“这半日不
见芳官,他在那里呢?”袭人四顾一瞧,说:“才在这里的,几个人斗草玩,这会
子不见了。”

宝玉听说便忙回房中,果见芳官面向里睡在床上。宝玉推他说道:“快别睡觉,
咱们外头玩去。一会子好吃饭。”芳官道:“你们吃酒,不理我,叫我闷了半天,
可不来睡觉罢了。”宝玉拉了他起来,笑道:“咱们晚上家里再吃。回来我叫袭人
姐姐带了你桌上吃饭,何如?”芳官道:“藕官蕊官都不上去,单我在那里,也不
好。我也吃不惯那个面条子,早起也没好生吃。才刚饿了,我已告诉了柳婶子,先
给我做一碗汤,盛半碗粳米饭,送到我这里,吃了就完事。若是晚上吃酒,不许叫
人管着我,我要尽力吃够了才罢。我先在家里,吃二三斤好惠泉酒呢。如今学了这
劳什子,他们说怕坏嗓子,这几年也没闻见。趁今儿我可是要开斋了。”宝玉道:
“这个容易。”

说着,只见柳家的果遣人送了一个盒子来。春燕接着揭开看时,里面是一碗虾
丸鸡皮汤,又是一碗酒酿清蒸鸭子,一碟腌的胭脂鹅脯,还有一碟四个奶油松瓤卷
酥,并一大碗热腾腾碧莹莹绿畦香稻粳米饭。春燕放在案上,走来安小菜碗箸,过
来拨了一碗饭。芳官便说:“油腻腻的,谁吃这些东西!”只将汤泡饭,吃了一碗,
拣了两块腌鹅,就不吃了。宝玉闻着,倒觉比往常之味又胜些似的,遂吃了一个卷
酥。又命春燕也拨了半碗饭,泡汤一吃,十分香甜可口。春燕和芳官都笑了。

吃毕,春燕便将剩的要交回。宝玉道:“你吃了罢,若不够,再要些来。”春
燕道:“不用要,这就够了。方才麝月姐姐拿了两盘子点心给我们吃了,我再吃了
这个,尽够了,不用再吃了。”说着,便站在桌旁,一顿吃了。又留下两个卷酥,
说:“这个留着给我妈吃。晚上要吃酒,给我两碗酒吃就是了。”宝玉笑道:“你
也爱吃酒?等着咱们晚上痛喝一回。你袭人姐姐和晴雯姐姐的量也好,也要喝,只
是每日不好意思的:趁今儿大家开斋。还有件事,想着嘱咐你,竟忘了,此刻才想
起来:以后芳官全要你照看他,他或有不到处,你提他。袭人照顾不过这些人来。”
春燕道:“我都知道,不用你操心。但只五儿的事怎么样?”宝玉道:“你和柳家
的说去,明儿真叫他进来罢。等我告诉他们一声就完了。”芳官听了,笑道:“这
倒是正经事。”春燕又叫两个小丫头进来,伏侍洗手倒茶。自己收了家伙,交给婆
子,也洗手,便去找柳家的,不在话下。

宝玉便出来,仍往红香圃寻众姐妹。芳官在后,拿着巾扇。刚出了院门,只见
袭人晴雯二人携手回来。宝玉问:“你们做什么呢?”袭人道:“摆下饭了,等你
吃饭呢。”宝玉笑着将方才吃饭的一节,告诉了他两个。袭人笑道:“我说你是猫
儿食。虽然如此,也该上去陪他们,多少应个景儿。”晴雯用手指戳在芳官额上,
说道:“你就是狐媚子!什么空儿,跑了去吃饭。两个怎么约下了?也不告诉我们一
声儿。”袭人笑道:“不过是误打误撞的遇见,说约下,可是没有的事。”晴雯道:
“既这么着,要我们无用。明儿我们都走了,让芳官一个人,就够使了。”袭人笑
道:“我们都去了使得,你却去不得。”晴雯道:“惟有我是第一个要去:又懒,
又夯,性子又不好,又没用。”袭人笑道:“倘或那孔雀褂子襟再烧了窟窿,你去
了谁可会补呢?你倒别和我拿三搬四的。我烦你做个什么,把你懒的横针不拈,竖
线不动。一般也不是我的私活烦你,横竖都是他的,你就都不肯。做什么我去了几
天,你病的七死八活,一夜连命也不顾,给他做了出来,这又是什么原故?你到底
说话呀。怎么装憨儿,和我笑?那也当不了什么。”晴雯笑着啐了一口。大家说着
来至厅上。薛姨妈也来了,依序坐下吃饭。宝玉只用茶泡了半碗饭,应景而已。

一时吃毕,大家吃茶闲话,又随便玩笑。外面小螺和香菱、芳官、蕊官、藕官、
豆官等四五个人,满园玩了一回,大家采了些花草来兜着,坐在花草堆里斗草。这
一个说:“我有观音柳。”那一个说:“我有罗汉松。”那一个又说:“我有君子
竹。”这一个又说:“我有美人蕉。”这个又说:“我有星星翠。”那个又说:“我
有月月红。”这个又说:“我有《牡丹亭》上的牡丹花。”那个又说:“我有《琵
琶记》里的枇杷果。”豆官便说:“我有姐妹花。”众人没了,香菱便说:“我有
夫妻蕙。”豆官说:“从没听见有个‘夫妻蕙’!”香菱道:“一个剪儿一个花儿
叫做‘兰’,一个剪儿几个花儿叫做‘蕙’。上下结花的为‘兄弟蕙’,并头结花
的为‘夫妻蕙’。我这枝并头的,怎么不是‘夫妻蕙’?”豆官没的说了,便起身
笑道:“依你说,要是这两枝一大一小,就是‘老子儿子蕙’了?若是两枝背面开
的,就是‘仇人蕙’了?你汉子去了大半年,你想他了,便拉扯着蕙上也有了夫妻
了,好不害臊!”香菱听了,红了脸,忙要起身拧他,笑骂道:“我把你这个烂了
嘴的小蹄子!满口里放屁胡说。”豆官见他要站起来,怎肯容他,就连忙伏身将他
压住,回头笑着央告蕊官等:“来帮着我拧他这张嘴。”两个人滚在地下。众人拍
手笑说:“了不得了!那是一洼子水,可惜弄了他的新裙子。”豆官回头看了一看,
果见傍边有一汪积雨,香菱的半条裙子都污湿了,自己不好意思,忙夺手跑了。众
人笑个不住,怕香菱拿他们出气,也都笑着一哄而散。

香菱起身,低头一瞧,见那裙上犹滴滴点点流下绿水来。正恨骂不绝,可巧宝
玉见他们斗草,也寻了些草花来凑戏,忽见众人跑了,只剩了香菱一个,低头弄裙,
因问:“怎么散了?”香菱便说:“我有一枝夫妻蕙,他们不知道,反说我诌,因
此闹起来,把我的新裙子也遭塌了。”宝玉笑道:“你有夫妻蕙,我这里倒有一枝
并蒂菱。”口内说着,手里真个拈着一枝并蒂菱花,又拈了那枝夫妻蕙在手内。香
菱道:“什么夫妻不夫妻、并蒂不并蒂!你瞧瞧这裙子!”宝玉便低头一瞧,“嗳
呀”了一声,说:“怎么就拉在泥里了?可惜!这石榴红绫,最不禁染。”香菱道:
“这是前儿琴姑娘带了来的,姑娘做了一条,我做了一条,今儿才上身。”宝玉跌
脚叹道:“若你们家,一日遭塌这么一件,也不值什么。只是头一件,既系琴姑娘
带来的,你和宝姐姐每人才一件,他的尚好,你的先弄坏了,岂不辜负他的心?二
则姨妈老人家的嘴碎,饶这么着,我还听见常说你们不知过日子,只会遭塌东西,
不知惜福。这叫姨妈看见了,又说个不清。”香菱听了这话,却碰在心坎儿上,反
倒喜欢起来,因笑道:“就是这话。我虽有几条新裙子,都不合这一样;若有一样
的,赶着换了也就好了,过后再说。”宝玉道:“你快休动,只站着方好,不然,
连小衣、膝裤、鞋面都要弄上泥水了。我有主意:袭人上月做了一条和这个一模一
样的,他因有孝,如今也不穿,竟送了你换下这个来何如?”香菱笑着摇头说:“不
好。倘或他们听见了,倒不好。”宝玉道:“这怕什么?等他孝满了,他爱什么,
难道不许你送他别的不成?你若这样,不是你素日为人了。况且不是瞒人的事,只
管告诉宝姐姐也可。只不过怕姨妈老人家生气罢咧。”香菱想了一想有理,点头笑
道:“就是这样罢了,别辜负了你的心。等着你。千万叫他亲自送来才好!”

宝玉听了喜欢非常,答应了,忙忙的回来。一壁低头心下暗想:“可惜这么一
个人,没父母,连自己本姓都忘了,被人拐出来,偏又卖给这个霸王!”因又想起:
“往日平儿也是意外,想不到的。今儿更是意外之意外的事了。”一面胡思乱想,
来至房中,拉了袭人,细细告诉了他原故。香菱之为人,无人不怜爱的;袭人又本
是个手中撒漫的,况与香菱相好,一闻此信,忙就开箱取了出来,折好,随了宝玉
来寻香菱。见他还站在那里等呢。袭人笑道:“我说你太淘气了,总要淘出个故事
来才罢。”香菱红了脸,笑说:“多谢姐姐了,谁知那起促狭鬼使的黑心。”说着
接了裙子,展开一看,果然合自己的一样。又命宝玉背过脸去,自己向内解下来,
将这条系上。袭人道:“把这腌了的交给我拿回去,收拾了给你送来。你要拿回
去,看见了,又是要问的。”香菱道:“好姐姐,你拿去,不拘给那个妹妹罢。我
有了这个,不要他了。”袭人道:“你倒大方的很。”香菱忙又拜了两拜,道谢袭
人。一面袭人拿了那条泥污了的裙子就走。

香菱见宝玉蹲在地下,将方才夫妻蕙与并蒂菱用树枝儿挖了一个坑,先抓些落
花来铺垫了,将这菱蕙安放上,又将些落花来掩了,方撮土掩埋平伏。香菱拉他的
手笑道:“这又叫做什么?怪道人人说你惯会鬼鬼祟祟使人肉麻呢。你瞧瞧,你这
手弄得泥污苔滑的,还不快洗去。”宝玉笑着,方起身走了去洗手。香菱也自走开。
二人已走了数步,香菱复转身回来,叫住宝玉。宝玉不知有何说话,扎煞着两只泥
手,笑嘻嘻的转来,问:“作什么?”香菱红了脸,只管笑,嘴里却要说什么,又
说不出口来。因那边他的小丫头臻儿走来说:“二姑娘等你说话呢。”香菱脸又一
红,方向宝玉道:“裙子的事,可别和你哥哥说,就完了。”说毕,即转身走了。
宝玉笑道:“可不是我疯了?往虎口里探头儿去呢!”说着,也回去了。

不知端详,下回分解。