\chapter{俏平儿情掩虾须镯~勇晴雯病补孔雀裘}

话说贾母道:“正是这个了。上次我要说这话,我见你们大事多,如今又添出
些事来,你们固然不敢抱怨,未免想着我只顾疼这些小孙子孙女儿们,就不体贴你
们这当家人了。你既这么说出来,便好了。”因此时薛姨妈李婶娘都在座,邢夫人
及尤氏等也都过来请安,还未过去,贾母因向王夫人等说道:“今日我才说这话,
素日我不说:一则怕逞了凤丫头的脸,二则众人不服。今日你们都在这里,都是经
过妯娌姑嫂的,还有他这么想得到的没有?”薛姨妈、李婶娘、尤氏齐笑说:“真
个少有!别人不过是礼上的面情儿,实在他是真疼小姑子小叔子。就是老太太跟前,
也是真孝顺。”贾母点头叹道:“我虽疼他,我又怕他太伶俐了,也不是好事。”
凤姐儿忙笑道:“这话老祖宗说差了。世人都说:‘太伶俐聪明怕活不长。’世人
都说,世人都信,独老祖宗不当说,不当信。老祖宗只有伶俐聪明过我十倍的,怎
么如今这么福寿双全的?只怕我明儿还胜老祖宗一倍呢。我活一千岁后,等老祖宗
归了西,我才死呢。”贾母笑道:“众人都死了,单剩咱们两个老妖精,有什么意
思!”说的众人都笑了。

宝玉因惦记着晴雯等事,便先回园里来。到了屋中,药香满室,一人不见,只
有晴雯独卧于炕上,脸上烧的飞红。又摸了一摸,只觉烫手,忙又向炉上将手烘暖,
伸进被去摸了一摸身上,也是火热。因说道:“别人去了也罢,麝月秋纹也这么无
情,各自去了?”晴雯道:“秋纹是我撵了他去吃饭了,麝月是方才平儿来找他出
去了,两个人鬼鬼祟祟的,不知说什么。必是说我病了不出去。”宝玉道:“平儿
不是那样人。况且他并不知你病特来瞧你,想来一定是找麝月来说话,偶然见你病
了,随口说特瞧你的病,这也是人情乖觉取和儿的常事。便不出去,有不是,与他
何干?你们素日又好,断不肯为这无干的事伤和气。”晴雯道:“这话也是,只是
疑他为什么忽然又瞒起我来?”宝玉笑道:“等我从后门出去,到那窗户根下听听
说些什么,来告诉你。”

说着,果从后门出去至窗下,潜听麝月悄悄问道:“你怎么就得了的?”平儿
道:“那日彼时洗手时不见了,二奶奶就不许吵嚷;出了园子,即刻就传给园里各
处的妈妈们,小心访查。我们只疑惑邢姑娘的丫头,本来又穷,只怕小孩子家没见
过,拿起来是有的,再不料定是你们这里的。幸而二奶奶没有在屋里,你们这里的
宋妈去了,拿着这支镯子,说是小丫头坠儿偷起来的,被他看见,来回二奶奶的。
我赶忙接了镯子。想了一想:宝玉是偏在你们身上留心用意、争胜要强的,那一年
有个良儿偷玉,刚冷了这二年,闲时还常有人提起来趁愿;这会子又跑出一个偷金
子的来了,而且更偷到街坊家去了!偏是他这么着,偏是他的人打嘴。所以我倒忙
叮咛宋妈千万别告诉宝玉,只当没有这事,总别和一个人提起。第二件,老太太、
太太听了生气。三则袭人和你们也不好看。所以我回二奶奶只说:‘我往大奶奶那
里去来着,谁知镯子褪了口,丢在草根底下,雪深了没看见。今儿雪化尽了,黄澄
澄的映着日头,还在那里呢,我就拣了起来。’二奶奶也就信了,所以我来告诉你
们。你们以后防着他些,别使唤他到别处去。等袭人回来,你们商议着,变个法子
打发出去就完了。”麝月道:“这小娼妇也见过些东西,怎么这么眼浅?”平儿道:
“究竟这镯子能多重!原是二奶奶的,说这叫做‘虾须镯’,倒是这颗珠子重了。
晴雯那蹄子是块爆炭,要告诉了他,他是忍不住的,一时气上来,或打或骂,依旧
嚷出来,所以单告诉你留心就是了。”说着,便作辞而去。

宝玉听了,又喜又气又叹:喜的是平儿竟能体帖自己的心;气的是坠儿小窃;
叹的是坠儿那样伶俐,做出这丑事来。因而回至房中,把平儿之话一长一短告诉了
晴雯,又说:“他说你是个要强的,如今病了,听了这话,越发要添病的,等好了
再告诉你。”晴雯听了,果然气的蛾眉倒蹙,凤眼圆睁,即时就叫坠儿。宝玉忙劝
道:“这一喊出来,岂不辜负了平儿待你我的心呢?不如领他这个情,过后打发他
出去就完了。”晴雯道:“虽如此说,只是这气如何忍得住?”宝玉道:“这有什
么气的?你只养病就是了。”

晴雯服了药,至晚间又服了二和,夜间虽有些汗,还未见效,仍是发烧头疼鼻
塞声重。次日,王太医又来诊视,另加减汤剂。虽然稍减了烧,仍是头疼。宝玉便
命麝月取鼻烟来:“给他闻些,痛打几个嚏喷就通快了。”麝月果真去取了一个金
镶双金星玻璃小扁盒儿来递给宝玉。宝玉便揭开盒盖,里面是个西洋珐琅的黄发赤
身女子,两肋又有肉翅,里面盛着些真正上等洋烟。晴雯只顾看画儿,宝玉道:“闻
些,走了气就不好了。”睛雯听说,忙用指甲挑了些抽入鼻中。不见怎么,便又多
多挑了些抽入。忽觉鼻中一股酸辣,透入囟门,接连打了五六个嚏喷,眼泪鼻涕登
时齐流。晴雯忙收了盒子,笑道:“了不得,辣!快拿纸来。”早有小丫头子递过
一搭子细纸,晴雯便一张一张的拿来醒鼻子。宝玉笑问:“如何?”晴雯笑道:“果
然通快些。只是太阳还疼。”宝玉笑道:“越发尽用西洋药治一治,只怕就好了。”
说着,便命麝月:“往二奶奶要去,就说我说了,姐姐那里常有那西洋贴头疼的膏
子药,叫做‘依佛哪’,找寻一点儿。”麝月答应去了,半日,果然拿了半节来。
便去找了一块红缎子角儿,铰了两块指顶大的圆式,将那药烤和了,用簪挺摊上。
晴雯自拿着一面靶儿镜子贴在两太阳上。麝月笑道:“病的蓬头鬼一样,如今贴了
这个,倒俏皮了!二奶奶贴惯了,倒不大显。”说毕,又问宝玉道:“二奶奶说了:
明儿是舅老爷的生日,太太说了叫你去呢。明儿穿什么衣裳?今儿晚上好打点齐备
了,省的明儿早起费手。”宝玉道:“什么顺手就是什么罢了。一年闹生日也闹不
清。”说着,便起身出房,往惜春屋里去看画儿。

刚到院门外边,忽见宝琴小丫头名小螺的从那边过去。宝玉忙赶上问:“那里
去?”小螺笑道:“我们二位姑娘都在林姑娘屋里呢,我如今也往那里去。”宝玉
听了,转步也便和他往潇湘馆来。不但宝钗姐妹在此,且连岫烟也在那里。四人团
坐在熏笼上叙家常。紫鹃倒坐在暖阁里,临窗户做针线。一见他来,都笑说:“又
来了一个!没了你的坐处了。”宝玉笑道:“好一幅‘冬闺集艳图’!可惜我迟来了。
横竖这屋子比各屋子暖,这椅子坐着并不冷。”说着,便坐在黛玉常坐的地方,上
搭着灰鼠椅搭一张椅上。因见暖阁之中有一玉石条盆,里面攒三聚五栽着一盆单瓣
水仙,宝玉便极口赞道:“好花!这屋子越暖,这花香的越浓。怎么昨儿没见?”
黛玉笑道:“这是你家的大总管赖大奶奶送薛二姑娘的两盆水仙、两盆腊梅:他送
了我一盆水仙,送了云丫头一盆腊梅。我原不要的,又恐辜负了他的心。你若要,
我转送你如何?”宝玉道:“我屋里却有两盆,只是不及这个。琴妹妹送你的,如
何又转送人,这个断断使不得。”黛玉道:“我一日药铞子不离火,我竟是药培着
呢,哪里还搁的住花香来熏?越发弱了。况且这屋子里一股药香,反把这花香搅坏
了。不如你抬了去,这花儿倒清净了,没什么杂味来搅他。”宝玉笑道:“我屋里
今儿也有个病人煎药呢。你怎么知道的?”黛玉笑道:“这说奇了。我原是无心话:
谁知你屋里的事?你不早来听古记儿,这会子来了,自惊自怪的。”

宝玉笑道:“咱们明儿下一社又有了题目了:就咏水仙、腊梅。”黛玉听了,
笑道:“罢,罢!再不敢做诗了。做一回,罚一回,没的怪羞的。”说着,便两手
握起脸来。宝玉笑道:“何苦来,又打趣我做什么?我还不怕臊呢,你倒握起脸来
了。”宝钗因笑道:“下次我邀一社,四个诗题,四个词题。每人四首诗,四首词。
头一个诗题《咏太极图》,限‘一先’的韵,五言排律;要把‘一先’的韵都用尽
了,一个不许剩。”宝琴笑道:“这一说,可知是姐姐不是真心起社了,这分明是
难人。要论起来,也强扭的出来,不过颠来倒去,弄些《易经》上的话生填,究竟
有何趣味。我八岁的时节,跟我父亲到西海沿上买洋货。谁知有个真真国的女孩子,
才十五岁,那脸面就和那西洋画上的美人一样,也披着黄头发,打着联垂,满头带
着都是玛瑙、珊瑚、猫儿眼、祖母绿,身上穿着金丝织的锁子甲,洋锦袄袖,带着
倭刀也是镶金嵌宝的。实在画儿上也没他那么好看。有人说他通中国的诗书,会讲
‘五经’,能做诗填词。因此我父亲央烦了一位通官,烦他写了一张字,就写他做
的诗。”众人都称道奇异。宝玉忙笑道:“好妹妹,你拿出来我们瞧瞧。”宝琴笑
道:“在南京收着呢,此时那里去取?”宝玉听了,大失所望,便说:“没福得见
这世面!”黛玉笑拉宝琴道:“你别哄我们:我知道你这一来,你的这些东西未必
放在家里,自然都是要带上来的。这会子又扯谎,说没带来。他们虽信,我是不信
的。”宝琴便红了脸,低头微笑不答。宝钗笑道:“偏这颦儿惯说这些话,你就伶
俐的太过了。”黛玉笑道:“带了来,就给我们见识见识也罢了。”宝钗笑道:“箱
子笼子一大堆,还没理清呢,知道在那个里头呢?等过日子收拾清了找出来,大家
再看罢了。”又向宝琴道:“你要记得,何不念念我们听听?”宝琴答道:“记得
他做的五言律一首,要论外国的女子,也就难为他了。”宝钗道:“你且别念,等
我把云儿叫了来,也叫他听听。”说着,便叫小螺来,吩咐道:“你到我那里去,
就说我们这里有一个外国的美人来了,做的好诗,请你这‘诗疯子’来瞧去,再把
我们‘诗呆子’也带来。”小螺笑着去了。

半日,只听湘云笑问:“那一个外国的美人来了?”一头说,一头走,和香菱
来了。众人笑道:“人未见形,先已闻声。”宝琴等让坐,遂把方才的话重告诉了
一遍。湘云笑道:“快念来听听。”宝琴因念道:
昨夜朱楼梦,今宵水国吟。
岛云蒸大海,岚气接丛林。
月本无今古,情缘自浅深。
汉南春历历,焉得不关心?

众人听了,都道:“难为他!竟比我们中国人还强。”一语未了,只见麝月走
来,说:“太太打发了人来告诉二爷,明儿一早往舅舅那里去,就说太太身上不大
好,不得亲身来。”宝玉忙站起来答应道:“是。”因问宝钗宝琴:“你们二位可
去?”宝钗道:“我们不去。昨儿单送了礼去了。”大家说了一回方散。

宝玉因让诸姐妹先行,自己在后面。黛玉便又叫住他,问道:“袭人到底多早
晚回来?”宝玉道:“自然等送了殡才来呢。”黛玉还有话说,又不能出口,出了
一回神,便说道:“你去罢。”宝玉也觉心里有许多话,只是口里不知要说什么,
想了一想,也笑道:“明儿再说罢。”一面下台阶,低头正欲迈步,复又忙回身问
道:“如今夜越发长了,你一夜咳嗽几次?醒几遍?”黛玉道:“昨儿夜里好了,
只咳嗽两遍,却只睡了四更一个更次,就再不能睡了。”宝玉又笑道:“正是有句
要紧的话,这会子才想起来。”一面说,一面便挨近身来,悄悄道:“我想宝姐姐
送你的燕窝——”一语未了,只见赵姨娘走进来瞧黛玉,问:“姑娘这几天可好了?”
黛玉便知他从探春处来,从门前过,顺路的人情,忙陪笑让坐,说:“难得姨娘想
着,怪冷的,亲自走来。”又忙命倒茶,一面又使眼色给宝玉。宝玉会意,便走了
出来。正值吃晚饭时,见了王夫人,又嘱咐他早去。宝玉回来,看晴雯吃了药。此
夕宝玉便不命晴雯挪出暖阁来,自己便在晴雯外边。又命将熏笼抬至暖阁前,麝月
便在熏笼上睡。一宿无话。

至次日天未明,晴雯便叫醒麝月道:“你也该醒了,只是睡不够。你出去叫人
给他预备茶水,我叫醒他就是了。”麝月忙披衣起来道:“咱们叫他起来,穿好衣
裳,抬过这火箱去,再叫他们进来。老妈妈们已经说过,不叫他在这屋里,怕过了
病气;如今他们见咱们挤在一处,又该唠叨了。”晴雯道:“我也是这么说。”二
人才叫时,宝玉已醒了,忙起身披衣。麝月先叫进小丫头子来收拾妥了,才命秋纹
等进来,一同伏侍。宝玉梳洗已毕,麝月道:“天又阴阴的,只怕下雪,穿一套毡
子的罢。”宝玉点头,即时换了衣裳。小丫头便用小茶盘捧了一盖碗建莲红枣汤来,
宝玉喝了两口;麝月又捧过一小碟法制紫姜来,宝玉噙了一块。又嘱咐了晴雯,便
忙往贾母处来。

贾母犹未起来,知道宝玉出门,便开了屋门,命宝玉进去。宝玉见贾母身后宝
琴面向里睡着未醒。贾母见宝玉身上穿着荔支色哆罗呢的箭袖,大红猩猩毡盘金彩
绣石青妆缎沿边的排穗褂。贾母道:“下雪呢么?”宝玉道:“天阴着,还没下呢。”
贾母便命:“鸳鸯来,把昨儿那一件孔雀毛的氅衣给他罢。”鸳鸯答应走去,果取
了一件来。宝玉看时,金翠辉煌,碧彩灼,又不似宝琴所披之凫靥裘。只听贾母
笑道:“这叫做‘雀金呢’,这是俄罗斯国拿孔雀毛拈了线织的。前儿那件野鸭子
的给了你小妹妹,这件给你罢。”宝玉磕了一个头,便披在身上。贾母笑道:“你
先给你娘瞧瞧去再去。”宝玉答应了,便出来,只见鸳鸯站在地下揉眼睛。因自那
日鸳鸯发誓绝婚之后,他总不合宝玉说话,宝玉正自日夜不安,此时见他又要回避,
宝玉便上来笑道:“好姐姐你瞧瞧,我穿着这个好不好?”鸳鸯一摔手,便进贾母
屋里来了。宝玉只得到了王夫人屋里,给王夫人看了,然后又回至园中,给晴雯麝
月看过,来回覆贾母说:“太太看了,只说可惜了的,叫我仔细穿,别遭塌了。”
贾母道:“就剩了这一件,你遭塌了也再没了。这会子特给你做这个,也是没有的
事。”说着又嘱咐:“不许多吃酒,早些回来。”

宝玉应了几个“是”。老嬷嬷跟至厅上,只见宝玉的奶兄李贵、王荣和张若锦、
赵亦华、钱升、周瑞六个人,带着焙茗、伴鹤、锄药、扫红四个小厮,背着衣包,
拿着坐褥,笼着一匹雕鞍彩辔的白马,已伺候多时了。老嬷嬷又嘱咐他们些话,六
个人连应了几个“是”,忙捧鞍坠镫,宝玉慢慢的上了马。李贵王荣笼着嚼环,钱
升周瑞二人在前引导,张若锦赵亦华在两边,紧贴宝玉身后。宝玉在马上笑道:“周
哥,钱哥,咱们打这角门走罢,省了到老爷的书房门口,又下来。”周瑞侧身笑道:
“老爷不在书房里,天天锁着,爷可以不用下来罢了。”宝玉笑道:“虽锁着,也
要下来的。”钱升李贵都笑道:“爷说的是。就托懒不下来,倘或遇见赖大爷林二
爷,虽不好说爷,也要劝两句。所有的不是,都派在我们身上,又说我们不教给爷
礼了。”周瑞钱升便一直出角门来。

正说话时,顶头见赖大进来,宝玉忙笼住马,意欲下来。赖大忙上来抱住腿。
宝玉便在镫上站起来,笑着,携手说了几句话。接着又见个小厮带着二三十人,拿
着扫帚簸箕进来,见了宝玉,都顺墙垂手立住,独为首的小厮打了个千儿,说:“请
爷安。”宝玉不知名姓,只微笑点点头儿。马已过去,那人方带人去了。于是出了
角门。外有李贵等六人的小厮并几个马夫,早预备下十来匹马专候,一出角门,李
贵等各上马前引,一阵烟去了,不在话下。

这里晴雯吃了药仍不见病退,急的乱骂大夫,说:“只会哄人的钱,一剂好药
也不给人吃。”麝月笑劝他道:“你太性急了,俗语说:‘病来如山倒,病去如抽
丝。’又不是老君的仙丹,那有这么灵药?你只静养几天,自然就好了。你越急越
着手。”晴雯又骂小丫头子们:“那里攒沙去了!瞅着我病了,都大胆子走了。明
儿我好了,一个个的才揭了你们的皮!”唬的小丫头子定儿忙进来问:“姑娘做什
么?”晴雯道:“别人都死了,就剩了你不成?”说着,只见坠儿也蹭进来了。晴
雯道:“你瞧瞧这小蹄子,不问他还不来呢。这里又放月钱了,又散果子了,你该
跑在头里了。你往前些!我是老虎,吃了你?”坠儿只得往前凑了几步。晴雯便冷
不防欠身,一把将他的手抓住,向枕边拿起一丈青来,向他手上乱戳,又骂道:“要
这爪子做什么?拈不动针,拿不动线,只会偷嘴吃!眼皮子又浅,爪子又轻,打嘴现
世的,不如戳烂了!”坠儿疼的乱喊。麝月忙拉开,按着晴雯躺下,道:“你才出
了汗,又作死!等你好了,要打多少打不得?这会子闹什么。”

晴雯便命人叫宋嬷嬷进来,说道:“宝二爷才告诉了我,叫我告诉你们,坠儿
很懒,宝二爷当面使他,他拨嘴儿不动,连袭人使他,他也背地里骂。今儿务必打
发他出去,明儿宝二爷亲自回太太就是了。”宋嬷嬷听了,心下便知镯子事发,因
笑道:“虽如此说,也等花姑娘回来,知道了,再打发他。”晴雯道:“宝二爷今
儿千叮咛万嘱咐的,什么‘花姑娘’‘草姑娘’的,我们自然有道理!你只依我的
话,快叫他家的人来领他出去。”麝月道:“这也罢了。早也是去,晚也是去,早
带了去,早清净一日。”宋嬷嬷听了,只得出去唤了他母亲来,打点了他的东西。
又见了晴雯等,说道:“姑娘们怎么了?你侄女儿不好,你们教导他,怎么撵出去?
也到底给我们留个脸儿。”晴雯道:“这话只等宝玉来问他,与我们无干。”那媳
妇冷笑道:“我有胆子问他去?他那一件事不是听姑娘们的调停?他纵依了,姑娘们
不依,也未必中用。比如方才说话,虽背地里,姑娘就直叫他的名字,在姑娘们就
使得,在我们就成了野人了!”

晴雯听说,越发急红了脸,说道:“我叫了他的名字了。你在老太太、太太跟
前告我去,说我野,也撵出我去!”麝月道:“嫂子你只管带了人出去,有话再说。
这个地方岂有你叫喊讲理的?你见谁和我们讲过理?别说嫂子你,就是赖大奶奶、林
大娘也得担待我们三分。就是叫名字,从小儿直到如今,都是老太太吩咐过的,你
们也知道的:恐怕难养活,巴巴的写了他的小名儿各处贴着,叫万人叫去,为的是
好养活,连挑水挑粪花子都叫得,何况我们!连昨儿林大娘叫了一声‘爷’,老太
太还说呢。此是一件。二则我们这些人,常回老太太、太太的话去,可不叫着名回
话,难道也称‘爷’?那一日不把‘宝玉’两字叫二百遍,偏嫂子又来挑这个了!过
一天嫂子闲了,在老太太、太太跟前听听我们当着面儿叫他,就知道了。嫂子原也
不得在老太太、太太跟前当些体统差使,成年家只在三门外头混,怪不得不知道我
们里头的规矩。这里不是嫂子久站的,再一会,不用我们说话,就有人来问你了。
有什么分证的话,且带了他去,你回了林大娘,叫他来找二爷说话。家里上千的人,
他也跑来,我也跑来,我们认人问姓还认不清呢!”说着,便叫小丫头子:“拿了
擦地的布来擦地!”那媳妇听了,无言可对,亦不敢久站,赌气带了坠儿就走。宋
嬷嬷忙道:“怪道你这嫂子不知规矩。你女儿在屋里一场,临去时也给姑娘们磕个
头。没有别的谢礼,他们也不希罕,不过磕个头尽心罢咧,怎么说走就走?”坠儿
听了,只得翻身进来,给他两个磕头。又找秋纹等,他们也并不睬他。那媳妇声
叹气,口不敢言,抱恨而去。

晴雯方才又闪了风,着了气,反觉更不好了。翻腾至掌灯,刚安静了些,只见
宝玉回来,进门就声顿脚。麝月忙问原故,宝玉道:“今儿老太太喜喜欢欢的给
了这件褂子,谁知不防,后襟子上烧了一块。幸而天晚了,老太太、太太都不理论。”
一面脱下来。麝月瞧时,果然有指顶大的烧眼,说:“这必定是手炉里的火迸上了。
这不值什么,赶着叫人悄悄拿出去叫个能干织补匠人织上就是了。”说着,就用包
袱包了,叫了一个嬷嬷送出去,说:“赶天亮就有才好,千万别给老太太、太太知
道。”婆子去了半日,仍就拿回来,说:“不但织补匠,能干裁缝、绣匠并做女工
的,问了,都不认的这是什么,都不敢揽。”麝月道:“这怎么好呢?明儿不穿也
罢了。”宝玉道:“明儿是正日子,老太太、太太说了,还叫穿过这个去呢。偏头
一日就烧了,岂不扫兴!”

晴雯听了半日,忍不住,翻身说道:“拿来我瞧瞧罢!没那福气穿就罢了,这
会子又着急。”宝玉笑道:“这话倒说的是。”说着,便递给晴雯,又移过灯来,
细瞧了一瞧。晴雯道:“这是孔雀金线的。如今咱们也拿孔雀金线,就像界线似的
界密了,只怕还可混的过去。”麝月笑道:“孔雀线现成的,但这里除你,还有谁
会界线?”晴雯道:“说不的我挣命罢了。”宝玉忙道:“这如何使得?才好了些,
如何做得活!”晴雯道:“不用你蝎蝎螫螫的,我自知道。”一面说,一面坐起来,
挽了一挽头发,披了衣裳。只觉头重身轻,满眼金星乱迸,实实掌不住。待不做,
又怕宝玉着急,少不得狠命咬牙捱着。便命麝月只帮着拈线。晴雯先拿了一根比一
比,笑道:“这虽不很像,要补上也不很显。”宝玉道:“这就很好,那里又找俄
罗斯国的裁缝去?”晴雯先将里子拆开,用茶杯口大小一个竹弓钉绷在背面,再将
破口四边用金刀刮的散松松的,然后用针缝了两条,分出经纬,亦如界线之法,先
界出地子来,后依本纹来回织补。补两针,又看看;织补不上三五针,便伏在枕上
歇一会。宝玉在旁,一时又问:“吃些滚水不吃?”一时又命:“歇一歇。”一时
又拿一件灰鼠斗篷替他披在背上,一时又拿个枕头给他靠着。急的晴雯央道:“小
祖宗,你只管睡罢!再熬上半夜,明儿眼睛抠搂了,那恰怎么好?”

宝玉见他着急,只得胡乱睡下,仍睡不着。一时只听自鸣钟已敲了四下,刚刚
补完;又用小牙刷慢慢的剔出毛来。麝月道:“这就很好,要不留心,再看不出
的。”宝玉忙要了瞧瞧,笑说:“真真一样了。”晴雯已嗽了几声,好容易补完了,
说了一声:“补虽补了,到底不像。我也再不能了!”“嗳哟”了一声,就身不由
主睡下了。

要知端的,且看下回分解。