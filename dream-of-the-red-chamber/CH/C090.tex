\chapter{失绵衣贫女耐嗷嘈~送果品小郎惊叵测}

却说黛玉自立意自戕之后,渐渐不支,一日竟至绝粒。从前十几天内,贾母等
轮流看望,他有时还说几句话;这两日索性不大言语。心里虽有时昏晕,却也有时
清楚。贾母等见他这病不似无因而起,也将紫鹃雪雁盘问过两次。两个那里敢说?
便是紫鹃欲向侍书打听消息,又怕越闹越真,黛玉更死得快了,所以见了侍书,毫
不提起。那雪雁是他传话弄出这样原故来,此时恨不得长出百十个嘴来说“我没说”,
自然更不敢提起。到了这一天黛玉绝粒之日,紫鹃料无指望了,守着哭了会子,因
出来偷向雪雁道:“你进屋里来,好好儿的守着他,我去回老太太、太太和二奶奶
去。今日这个光景,大非往常可比了。”雪雁答应,紫鹃自去。

这里雪雁正在屋里伴着黛玉,见他昏昏沉沉,小孩子家那里见过这个样儿,只
打量如此便是死的光景了,心中又痛又怕,恨不得紫鹃一时回来才好。正怕着,只
听窗外脚步走响,雪雁知是紫鹃回来,才放下心了,连忙站起来,掀着里间帘子等
他。只见外面帘子响处,进来了一个人,却是侍书。那侍书是探春打发来看黛玉的,
见雪雁在那里掀着帘子,便问道:“姑娘怎么样?”雪雁点点头儿,叫他进来。侍
书跟进来,见紫鹃不在屋里,瞧了瞧黛玉,只剩得残喘微延,唬的惊疑不止。因问:
“紫鹃姐姐呢?”雪雁道:“告诉上屋里去了。”那雪雁此时只打量黛玉心中一无所
知了,又见紫鹃不在面前,因悄悄的拉了侍书的手问道:“你前日告诉我说的什么
王大爷给这里宝二爷说了亲,是真话么?”侍书道:“怎么不真!”雪雁道:“多早
晚放定的?”侍书道:“那里就放定了呢?那一天我告诉你时,是我听见小红说的。
后来我到二奶奶那边去,二奶奶正和平姐姐说呢,道:‘那都是门客们借着这个事
讨老爷的喜欢,往后好拉拢的意思。别说大太太说不好,就是大太太愿意,说那姑
娘好,那大太太眼里看的出什么人来?再者,老太太心里早有了人了,就在咱们园
子里的,大太太那里摸的着底呢。老太太不过因老爷的话,不得不问问罢咧。’又
听见二奶奶说:‘宝玉的事,老太太总是要亲上作亲的,凭谁来说亲,横竖不中用。’”
雪雁听到这里,也忘了神了,因说道:“这是怎么说!白白的送了我们这一位的命
了。”侍书道:“这是从那里说起?”雪雁道:“你还不知道呢!前日都是我和紫鹃姐
姐说来着,这一位听见了,就弄到这步田地了。”侍书道:“你悄悄儿的说罢,看仔
细他听见了。”雪雁道:“人事都不醒了,瞧瞧罢,左不过在这一两天了。”正说着,
只见紫鹃掀帘进来说:“这还了得!你们有什么话还不出去说,还在这里说!索性逼
死他就完了。”侍书道:“我不信有这样奇事。”紫鹃道:“好姐姐,不是我说,你又
该恼了!你懂得什么呢?懂得也不传这些舌了。”

这里三个人正说着,只听黛玉忽然又嗽了一声,紫鹃连忙跑到炕沿前站着,侍
书雪雁也都不言语了。紫鹃弯着腰,在黛玉身后轻轻问道:“姑娘,喝口水罢?”
黛玉微微答应了一声。雪雁连忙倒了半钟滚白水,紫鹃接了托着,侍书也走近前来。
紫鹃和他摇头儿,不叫他说话,侍书只得咽住了。站了一回,黛玉又嗽了一声。紫
鹃趁势问道:“姑娘,喝水呀!”黛玉又微微应了一声,那头似有欲抬之意,那里抬
得起?紫鹃爬上炕去,爬在黛玉傍边,端着水,试了冷热,送到唇边,扶了黛玉的
头,就到碗边喝了一口。紫鹃才要拿时,黛玉意思还要喝一口,紫鹃便托着那碗不
动。黛玉又喝了一口,摇摇头儿,不喝了。喘了一口气,仍旧躺下。半日,微微睁
眼,说道:“刚才说话不是侍书么?”紫鹃答应道:“是。”侍书尚未出去,因连忙
过来问候。黛玉睁眼看了,点点头儿,又歇了一歇,说道:“回去问你姑娘好罢。”
侍书见这番光景,只当黛玉嫌烦,只得悄悄的退出去了。

原来那黛玉虽则病势沉重,心里却还明白。起先侍书雪雁说话时,他也模糊听
见了一半句,却只作不知,也因实无精神答理。及听了雪雁侍书的话,才明白过前
头的事情原是议而未成的。又兼侍书说是凤姐说的,老太太的主意,亲上作亲,又
是园中住着的,非自己而谁?因此一想,阴极阳生,心神顿觉清爽许多,所以才喝
了两口水,又要想问侍书的话。恰好贾母、王夫人、李纨、凤姐听见紫鹃之言都赶
着来看。黛玉心中疑团已破,自然不似先前寻死之意了。虽身骨软弱,精神短少,
却也勉强答应一两句了。凤姐因叫过紫鹃,问道:“姑娘也不至这样。这是怎么说,
你这样唬人?”紫鹃道:“实在头里看着不好,才敢去告诉的。回来见姑娘竟好了
许多,也就怪了。”贾母笑道:“你也别信他。他懂得什么?看见不好就言语,这倒
是他明白的地方。小孩子家不嘴懒脚嫩就好。”说了一回,贾母等料着无妨,也就
去了。正是:
心病终须心药治,解铃还是系铃人。

不言黛玉病渐减退。且说雪雁紫鹃背地里都念佛。雪雁向紫鹃说道:“亏他好
了!只是病的奇怪,好的也奇怪。”紫鹃道:“病的倒不怪,就只好的奇怪。想来宝
玉和姑娘必是姻缘。人家说的:‘好事多磨。’又说道:‘是姻缘棒打不回。’这么看
起来,人心天意,他们两个竟是天配的了。再者,你想那一年,我说了林姑娘要回
南去,把宝玉没急死了,闹得家翻宅乱;如今一句话又把这一个弄的死去活来:可
不说的三生石上百年前结下的么?”说着,两个悄悄的抿着嘴笑了一回。雪雁又道:
“幸亏好了,咱们明儿再别说了。就是宝玉娶了别的人家儿的姑娘,我亲见他在那
里结亲,我也再不露一句话了。”紫鹃笑道:“这就是了。”

不但紫鹃和雪雁在私下里讲究,就是众人也都知道黛玉的病也病的奇怪,好也
好得奇怪,三三两两,唧唧哝哝议论着。不多几时,连凤姐儿也知道了,邢王二夫
人也有些疑惑,倒是贾母略猜着了八九。那时正值邢王二夫人、凤姐等在贾母房中
说闲话,说起黛玉的病来。贾母道:“我正要告诉你们。宝玉和林丫头是从小儿在
一处的,我只说小孩子们怕什么。以后时常听得林丫头忽然病,忽然好,都为有了
些知觉了。所以我想他们若尽着搁在一块儿,毕竟不成体统。你们怎么说?”王夫
人听了,便呆了一呆,只得答应道:“林姑娘是个有心计儿的。至于宝玉,呆头呆
脑,不避嫌疑是有的。看起外面,却还都是个小孩儿形象。此时若忽然或把那一个
分出园外,不是倒露了什么痕迹了么?古来说的:‘男大须婚,女大须嫁。’老太太
想,倒是赶着把他们的事办办也罢了。”贾母皱了一皱眉,说道:“林丫头的乖僻,
虽也是他的好处,我的心里不把林丫头配他,也是为这点子。况且林丫头这样虚弱,
恐不是有寿的。只有宝丫头最妥。”王夫人道:“不但老太太这么想,我们也是这么。
但林姑娘也得给他说了人家儿才好。不然,女孩儿家长大了,那个没有心事?倘或
真与宝玉有些私心,若知道宝玉定下宝丫头,那倒不成事了。”贾母道:“自然先给
宝玉娶了亲,然后给林丫头说人家。再没有先是外人、后是自己的,况且林丫头年
纪到底比宝玉小两岁。依你们这么说,倒是宝玉定亲的话,不许叫他知道倒罢了。”
凤姐便吩咐众丫头们道:“你们听见了?宝二爷定亲的话,不许混吵嚷;若有多嘴的,
提防着他的皮!”贾母又向凤姐道:“凤哥儿,你如今自从身上不大好,也不大管园
里的事了。我告诉你,须得经点儿心。不但这个,就像前年那些人喝酒耍钱,都不
是事。你还精细些,少不得多分点心儿,严紧严紧他们才好。况且我看他们也就还
服你些。”凤姐答应了。娘儿们又说了一回话,方各自散了。

从此,凤姐常到园中照料。一日,刚走进大观园,到了紫菱洲畔,只听见一个
老婆子在那里嚷。凤姐走到跟前,那婆子才瞧见了,早垂手侍立,口里请了安。凤
姐道:“你在这里闹什么?”婆子道:“蒙奶奶们派我在这里看守花果,我也没有差
错,不料邢姑娘的丫头说我们是贼。”凤姐道:“为什么呢?”婆子道:“昨儿我们
家的黑儿跟着我到这里玩了一回,他不知道,又往邢姑娘那边去瞧了一瞧,我就叫
他回去了。今儿早起,听见他们丫头说,丢了东西了。我问他丢了什么,他就问起
我来了。”凤姐道:“问了你一声,也犯不着生气呀。”婆子道:“这里园子,到底是
奶奶家里的,并不是他们家里的。我们都是奶奶派的,贼名儿怎么敢认呢?”凤姐
照脸啐了一口,厉声道:“你少在我跟前唠唠叨叨的!你在这里照看,姑娘丢了东西,
你们就该问哪。怎么说出这些没道理的话来!把老林叫了来,撵他出去。”丫头们答
应了。只见邢岫烟赶忙出来,迎着凤姐陪笑道:“这使不得,没有的事。事情早过
去了。”凤姐道:“姑娘,不是这个话。倒不讲事情,这名分上太岂有此理了。”岫
烟见婆子跪在地下告饶,便忙请凤姐到里边去坐。凤姐道:“他们这种人,我知道
他,除了我,其馀都没上没下的了。”岫烟再三替他讨饶,只说自己的丫头不好。
凤姐道:“我看着邢姑娘的分上,饶你这一次!”婆子才起来磕了头,又给岫烟磕了
头,才出去了。

这里二人让了坐,凤姐笑问道:“你丢了什么东西了?”岫烟笑道:“没有什么
要紧的,是一件红小袄儿,已经旧了的。我原叫他们找,找不着就罢了。这小丫头
不懂事,问了那婆子一声,那婆子自然不依了。这都是小丫头糊涂不懂事,我也骂
了几句。已经过去了,不必再提了。”凤姐把岫烟内外一瞧,看见虽有些皮绵衣裳,
已是半新不旧的,未必能暖和。他的被窝多半是薄的。至于房中桌上摆设的东西,
就是老太太拿来的,却一些不动,收拾的干干净净。凤姐心上便很爱敬他,说道:
“一件衣裳原不要紧,这时候冷,又是贴身的,怎么就不问一声儿呢?这撒野的奴
才,了不得了!”说了一回,凤姐出来,各处去坐了一坐,就回去了。到了自己房
中,叫平儿取了一件大红洋绉的小袄儿,一件松花色绫子一抖珠儿的小皮袄,一条
宝蓝盘锦厢花线裙,一件佛青银鼠褂子,包好叫人送去。

那时岫烟被那老婆子聒噪了一场,虽有凤姐来压住,心上终是不定。想起:“许
多姐妹们在这里,没有一个下人敢得罪他的,独自我这里,他们言三语四,刚刚凤
姐来碰见。”想来想去,终是没意思,又说不出来。正在吞声饮泣,看见凤姐那边
的丰儿送衣裳过来。岫烟一看,决不肯受。丰儿道:“奶奶吩咐我说:‘姑娘要嫌是
旧衣裳,将来送新的来。’”岫烟笑谢道:“承奶奶的好意。只是因我丢了衣裳,他
就拿来,我断不敢受的。拿回去,千万谢你们奶奶!承你奶奶的情,我算领了。”倒
拿个荷包给了丰儿,那丰儿只得拿了去了。不多时又见平儿同着丰儿过来,岫烟忙
迎着问了好,让了坐。平儿笑说道:“我们奶奶说:姑娘特外道的了不得!”岫烟道:
“不是外道,实在不过意。”平儿道:“奶奶说:‘姑娘要不收这衣裳,不是嫌太旧,
就是瞧不起我们奶奶。’刚才说了:我要拿回去,奶奶不依我呢。”岫烟红着脸笑谢
道:“这样说了,叫我不敢不收。”又让了一回茶。

平儿和丰儿回去,将到凤姐那边,碰见薛家差来的一个老婆子,接着问好。平
儿便问道:“你那里去的?”婆子道:“那边太太、姑娘叫我来请各位太太、奶奶、
姑娘们的安。我才刚在奶奶前问起姑娘来,说姑娘到园中去了。可是从邢姑娘那里
来么?”平儿道:“你怎么知道?”婆子道:“方才听见说,真真的二奶奶和姑娘们
的行事叫人感念。”平儿笑了一笑说:“你回来坐着罢。”婆子道:“我还有事,改日
再过来瞧姑娘罢。”说着走了。平儿回来,回覆了凤姐。不在话下。

且说薛姨妈家中被金桂搅得翻江倒海,看见婆子回来,说起岫烟的事,宝钗母
女二人不免滴下泪来。宝钗道:“都为哥哥不在家,所以叫邢姑娘多吃几天苦。如
今还亏凤姐姐不错。咱们底下也得留心,到底是咱们家里人。”说着,只见薛蝌进
来说道:“大哥哥这几年在外头相与的都是些什么人!连一个正经的也没有。来一起
子,都是些狐群狗党。我看他们那里是不放心,不过将来探探消息儿罢咧。这两天
都被我赶出去了。以后吩咐了门上,不许传进这种人来。”薛姨妈道:“又是蒋玉函
那些人哪?”薛蝌道:“蒋玉函却倒没来,倒是别人。”薛姨妈听了薛蝌的话,不觉
又伤起心来,说道:“我虽有儿,如今就像没有的了。就是上司准了,也是个废人。
你虽是我侄儿,我看你还比你哥哥明白些,我这后辈子全靠你了。你自己从今后要
学好。再者,你聘下的媳妇儿,家道不比往时了。人家的女孩儿出门子不是容易,
再没别的想头,只盼着女婿能干,他就有日子过了。若邢丫头也像这个东西——”
说着把手往里头一指,道:“我也不说了。邢丫头实在是个有廉耻有心计儿的,又
守得贫,耐得富。只是等咱们的事过去了,早些儿把你们的正经事完结了,也了我
一宗心事。”薛蝌道:“琴妹妹还没有出门子,这倒是太太烦心的一件事。至于这个,
可算什么呢。”大家又说了一回闲话。

薛蝌回到自己屋里,吃了晚饭,想起邢岫烟住在贾府园中,终是寄人篱下,况
且又穷,日用起居不想可知。况兼当初一路同来,模样儿性格儿都知道的。可知天
意不均:如夏金桂这种人,偏叫他有钱,娇养得这般泼辣;邢岫烟这种人,偏叫他
这样受苦。阎王判命的时候,不知如何判法的?想到闷来,也想吟诗一首,写出来
出出胸中的闷气,又苦自己没有工夫,只得混写道:
蛟龙失水似枯鱼,两地情怀感索居。
同在泥涂多受苦,不知何日向清虚!
写毕,看了一回,意欲拿来粘在壁上,又不好意思,自己沉吟道:“不要被人看见
笑话。”又念了一遍,道:“管他呢,左右粘上自己看着解闷儿罢。”又看了一回,
到底不好,拿来夹在书里。又想:“自己年纪可也不小了,家中又碰见这样飞灾横
祸,不知何日了局。致使幽闺弱质,弄得这般凄凉寂寞!”

正在那里想时,只见宝蟾推进门来,拿着一个盒子,笑嘻嘻放在桌上。薛蝌站
起来让坐。宝蟾笑着向薛蝌道:“这是四碟果子,一小壶儿酒:大奶奶叫给二爷送
来的。”薛蝌陪笑道:“大奶奶费心。但是叫小丫头们送来就完了,怎么又劳动姐姐
呢?”宝蟾道:“好说。自家人,二爷何必说这些套话?再者我们大爷这件事,实在
叫二爷操心,大奶奶久已要亲自弄点什么儿谢二爷,又怕别人多心。二爷是知道的,
咱们家里都是言合意不合,送点子东西没要紧,倒没的惹人七嘴八舌的讲究。所以
今儿些微的弄了一两样果子,一壶酒,叫我亲自悄悄儿的送来。”说着,又笑瞅了
薛蝌一眼,道:“明儿二爷再别说这些话,叫人听着怪不好意思的。我们不过也是
底下的人,伏侍的着大爷,就伏侍的着二爷,这有何妨呢?”薛蝌一则秉性忠厚,
二则到底年轻,只是向来不见金桂和宝蟾如此相待,心中想到刚才宝蟾说为薛蟠之
事,也是情理,因说道:“果子留下罢,这个酒儿,姐姐只管拿回去。我向来的酒
上实在很有限,挤住了偶然喝一钟,平白无事是不能喝的,难道大奶奶和姐姐还不
知道么?”宝蟾道:“别的我作得主,独这一件事,我可不敢应。大奶奶的脾气儿
二爷是知道的,我拿回去,不说二爷不喝,倒要说我不尽心了。”薛蝌没法,只得
留下。宝蟾方才要走,又到门口往外看看,回过头来向着薛蝌一笑,又用手指着里
面说道:“他还只怕要来亲自给你道乏呢。”薛蝌不知何意,反倒讪讪的起来,因说
道:“姐姐替我谢大奶奶罢。天气寒,看凉着。再者自己叔嫂,也不必拘这些个礼。”
宝蟾也不答言,笑着走了。

薛蝌始而以为金桂为薛蟠之事,或者真是不过意,备此酒果给自己道乏,也是
有的。及见了宝蟾这种鬼鬼祟祟、不尴不尬的光景,也觉有几分。却自己回心一想:
“他到底是嫂子的名分,那里就有别的讲究了呢?或者宝蟾不老成,自己不好意思
怎么着,却指着金桂的名儿,也未可知。然而到底是哥哥的屋里人,也不好……”
忽又一转念:“那金桂素性为人毫无闺阁理法,况且有时高兴,打扮的妖调非常,
自以为美,又怎么不是怀着坏心呢?不然,就是他和琴妹妹也有了什么不对的地方
儿,所以设下这个毒法儿,要把我拉在浑水里,弄一个不清不白的名儿,也未可知?”
想到这里,索性倒怕起来了。正在不得主意的时候,忽听窗外“噗哧”的笑了一声,
把薛蝌倒唬了一跳。

未知是谁,下回分解。