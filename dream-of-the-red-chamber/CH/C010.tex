\chapter{金寡妇贪利权受辱~张太医论病细穷源}

话说金荣因人多势众,又兼贾瑞勒令赔了不是,给秦钟磕了头,宝玉方才不吵
闹了。大家散了学,金荣自己回到家中,越想越气,说:“秦钟不过是贾蓉的小舅
子,又不是贾家的子孙,附学读书,也不过和我一样。因他仗着宝玉和他相好,就
目中无人。既是这样,就该干些正经事,也没的说;他素日又和宝玉鬼鬼祟祟的,
只当人家都是瞎子看不见。今日他又去勾搭人,偏偏撞在我眼里,就是闹出事来,
我还怕什么不成?”他母亲胡氏听见他咕咕唧唧的,说:“你又要管什么闲事?好容
易我和你姑妈说了,你姑妈又千方百计的和他们西府里琏二奶奶跟前说了,你才得
了这个念书的地方儿。若不是仗着人家,咱们家里还有力量请的起先生么?况且人
家学里茶饭都是现成的,你这二年在那里念书,家里也省好大的嚼用呢!省出来的,
你又爱穿件体面衣裳。再者你不在那里念书,你就认得什么薛大爷了?那薛大爷一
年也帮了咱们七八十两银子。你如今要闹出了这个学房,再想找这么个地方儿,我
告诉你说罢,比登天的还难呢!你给我老老实实的玩一会子睡你的觉去,好多着呢!”
于是金荣忍气吞声,不多一时,也自睡觉去了。次日仍旧上学去了,不在话下。

且说他姑妈原给了贾家“玉”字辈的嫡派,名唤贾璜,但其族人那里皆能像宁
荣二府的家势?原不用细说。这贾璜夫妻守着些小小的产业,又时常到宁荣二府里
去请安,又会奉承凤姐儿并尤氏,所以凤姐儿尤氏也时常资助资助他,方能如此度
日。今日正遇天气晴明,又值家中无事,遂带了一个婆子,坐上车,来家里走走,
瞧瞧嫂子和侄儿。说起话儿来,金荣的母亲偏提起昨日贾家学房里的事,从头至尾,
一五一十,都和他小姑子说了。这璜大奶奶不听则已,听了怒从心上起,说道:“这
秦钟小杂种是贾门的亲戚,难道荣儿不是贾门的亲戚?也别太势利了!况且都做的是
什么有脸的事!就是宝玉也不犯向着他到这个田地。等我到东府里瞧瞧我们珍大奶
奶,再和秦钟的姐姐说说,叫他评评理!”金荣的母亲听了,急的了不得,忙说道:
“这都是我的嘴快,告诉了姑奶奶,求姑奶奶快别去说罢!别管他们谁是谁非,倘
或闹出来,怎么在那里站的住?要站不住,家里不但不能请先生,还得他身上添出
许多嚼用来呢!”璜大奶奶说道:“那里管的那些个?等我说了,看是怎么样!”也不
容他嫂子劝,一面叫老婆子瞧了车,坐上竟往宁府里来。

到了宁府,进了东角门,下了车,进去见了尤氏,那里还有大气儿?殷殷勤勤
叙过了寒温,说了些闲话儿,方问道:“今日怎么没见蓉大奶奶?”尤氏说:“他这
些日子不知怎么了,经期有两个多月没有来。叫大夫瞧了,又说并不是喜。那两日
到下半日就懒怠动了,话也懒怠说,神也发涅。我叫他:‘你且不必拘礼,早晚不
必照例上来,你竟养养儿罢。就有亲戚来,还有我呢。别的长辈怪你,等我替你告
诉。’连蓉哥儿我都嘱咐了,我说:‘你不许累他,不许招他生气,叫他静静儿的
养几天就好了。他要想什么吃,只管到我屋里来取。倘或他有个好歹,你再要娶这
么一个媳妇儿,这么个模样儿,这么个性格儿,只怕打着灯笼儿也没处找去呢!’
他这为人行事儿,那个亲戚长辈儿不喜欢他?所以我这两日心里很烦。偏偏儿的早
起他兄弟来瞧他,谁知那小孩子家不知好歹,看见他姐姐身上不好,这些事也不当
告诉他,就受了万分委曲也不该向着他说。谁知昨日学房里打架,不知是那里附学
的学生,倒欺负他,里头还有些不干不净的话,都告诉了他姐姐。婶子你是知道的:
那媳妇虽则见了人有说有笑的,他可心细,不拘听见什么话儿都要忖量个三日五夜
才算。这病就是打这‘用心太过’上得的。今儿听见有人欺负了他的兄弟,又是恼,
又是气。恼的是那狐朋狗友,搬弄是非,调三窝四;气的是为他兄弟不学好,不上
心念书,才弄的学房里吵闹。他为这件事,索性连早饭还没吃。我才到他那边解劝
了他一会子,又嘱咐了他的兄弟几句,我叫他兄弟到那边府里又找宝玉儿去;我又
瞧着他吃了半钟儿燕窝汤,我才过来了。婶子,你说我心焦不心焦?况且目今又没
个好大夫,我想到他病上,我心里如同针扎的一般!你们知道有什么好大夫没有?”

金氏听了这一番话,把方才在他嫂子家的那一团要向秦氏理论的盛气,早吓的
丢在爪洼国去了。听见尤氏问他好大夫的话,连忙答道:“我们也没听见人说什么
好大夫。如今听起大奶奶这个病来,定不得还是喜呢。嫂子倒别教人混治,倘若治
错了,可了不得!”尤氏道:“正是呢。”说话之间,贾珍从外进来,见了金氏,便
问尤氏道:“这不是璜大奶奶么?”金氏向前给贾珍请了安,贾珍向尤氏说:“你让
大妹妹吃了饭去。”贾珍说着话便向那屋里去了。金氏此来原要向秦氏说秦钟欺负
他侄儿的事,听见秦氏有病,连提也不敢提了。况且贾珍尤氏又待的甚好,因转怒
为喜的,又说了一会子闲话,方家去了。

金氏去后,贾珍方过来坐下,问尤氏道:“今日他来又有什么说的?”尤氏答
道:“倒没说什么,一进来脸上倒像有些个恼意似的,及至说了半天话儿,又提起
媳妇的病,他倒渐渐的气色平和了。你又叫留他吃饭,他听见媳妇这样的病,也不
好意思只管坐着,又说了几句话就去了,倒没有求什么事。如今且说媳妇这病,你
那里寻一个好大夫给他瞧瞧要紧,可别耽误了!现今咱们家走的这群大夫,那里要
得?一个个都是听着人的口气儿,人怎么说,他也添几句文话儿说一遍;可倒殷勤
的很,三四个人,一日轮流着,倒有四五遍来看脉!大家商量着立个方儿,吃了也
不见效。倒弄的一日三五次换衣裳、坐下起来的见大夫,其实于病人无益。”贾珍
道:“可是这孩子也糊涂,何必又脱脱换换的。倘或又着了凉,更添一层病,还了
得?任凭什么好衣裳,又值什么呢,孩子的身体要紧。就是一天穿一套新的,也不
值什么。我正要告诉你:方才冯紫英来看我,他见我有些心里烦,问我怎么了,我
告诉他媳妇身子不大爽快,因为不得个好大夫,断不透是喜是病,又不知有妨碍没
妨碍,所以我心里实在着急。冯紫英因说他有一个幼时从学的先生,姓张名友士,
学问最渊博,更兼医理极精,且能断人的生死。今年是上京给他儿子捐官,现在他
家住着呢。这样看来,或者媳妇的病该在他手里除灾也未可定。我已叫人拿我的名
帖去请了。今日天晚,或未必来,明日想一定来的。且冯紫英又回家亲替我求他,
务必请他来瞧的。等待张先生来瞧了再说罢。”

尤氏听说,心中甚喜,因说:“后日是太爷的寿日,到底怎么个办法?”贾珍
说道:“我方才到了太爷那里去请安,兼请太爷来家受一受一家子的礼。太爷因说
道:‘我是清净惯了的,我不愿意往你们那是非场中去。你们必定说是我的生日,
要叫我去受些众人的头,你莫如把我从前注的《阴骘文》给我好好的叫人写出来刻
了,比叫我无故受众人的头还强百倍呢!倘或明日后日这两天一家子要来,你就在
家里好好的款待他们就是了。也不必给我送什么东西来。连你后日也不必来。你要
心中不安,你今日就给我磕了头去。倘或后日你又跟许多人来闹我,我必和你不依。’
如此说了,今日我是再不敢去的了。且叫赖升来,吩咐他预备两日的筵席。”

尤氏因叫了贾蓉来:“吩咐赖升照例预备两日的筵席,要丰丰富富的。你再亲
自到西府里请老太太、大太太、二太太和你琏二婶子来逛逛。你父亲今日又听见一
个好大夫,已经打发人请去了,想明日必来。你可将他这些日子的病症细细的告诉
他。”贾蓉一一答应着出去了。正遇着刚才到冯紫英家去请那先生的小子回来了,
因回道:“奴才方才到了冯大爷家,拿了老爷名帖请那先生去,那先生说是:‘方才
这里大爷也和我说了,但只今日拜了一天的客,才回到家,此时精神实在不能支持,
就是去到府上也不能看脉,须得调息一夜,明日务必到府。’他又说:‘医学浅薄,
本不敢当此重荐,因冯大爷和府上既已如此说了,又不得不去,你先替我回明大人
就是了。大人的名帖着实不敢当。’还叫奴才拿回来了。哥儿替奴才回一声儿罢。”
贾蓉复转身进去,回了贾珍尤氏的话,方出来叫了赖升,吩咐预备两日的筵席的话。
赖升答应,自去照例料理,不在话下。

且说次日午间,门上人回道:“请的那张先生来了。”贾珍遂延入大厅坐下。茶
毕,方开言道:“昨日承冯大爷示知老先生人品学问,又兼深通医学,小弟不胜钦
敬。”张先生道:“晚生粗鄙下士,知识浅陋。昨因冯大爷示知,大人家第谦恭下士,
又承呼唤,不敢违命。但毫无实学,倍增汗颜。”贾珍道:“先生不必过谦,就请先
生进去看看儿妇,仰仗高明,以释下怀。”于是贾蓉同了进去,到了内室,见了秦
氏,向贾蓉说道:“这就是尊夫人了?”贾蓉道:“正是。请先生坐下,让我把贱内
的病症说一说再看脉如何?”那先生道:“依小弟意下,竟先看脉,再请教病源为
是。我初造尊府,本也不知道什么,但我们冯大爷务必叫小弟过来看看,小弟所以
不得不来。如今看了脉息,看小弟说得是不是,再将这些日子的病势讲一讲,大家
斟酌一个方儿。可用不可用,那时大爷再定夺就是了。”贾蓉道:“先生实在高明,
如今恨相见之晚。就请先生看一看脉息可治不可治,得以使家父母放心。”于是家
下媳妇们,捧过大迎枕来,一面给秦氏靠着,一面拉着袖口,露出手腕来。这先生
方伸手按在右手脉上,调息了至数,凝神细诊了半刻工夫。换过左手,亦复如是。
诊毕了,说道:“我们外边坐罢。”

贾蓉于是同先生到外边屋里炕上坐了。一个婆子端了茶来,贾蓉道:“先生请
茶。”茶毕,问道:“先生看这脉息还治得治不得?”先生说:“看得尊夫人脉息,
左寸沉数,左关沉伏,右寸细而无力,右关虚而无神。其左寸沉数者,乃心气虚而
生火;左关沉伏者,乃肝家气滞血亏。右寸细而无力者,乃肺经气分太虚;右关虚
而无神者,乃脾土被肝木克制。心气虚而生火者,应现今经期不调,夜间不寐。肝
家血亏气滞者,应胁下痛胀,月信过期,心中发热。肺经气分太虚者,头目不时眩
晕,寅卯间必然自汗,如坐舟中,脾土被肝木克制者,必定不思饮食,精神倦怠,
四肢酸软。据我看这脉,当有这些症候才对。或以这个的为喜脉,则小弟不敢闻命
矣。”旁边一个贴身伏侍的婆子道:“何尝不是这样呢!真正先生说得如神,倒不用
我们说了。如今我们家里现有好几位太医老爷瞧着呢,都不能说得这样真切。有的
说道是喜,有的说道是病;这位说不相干,这位又说怕冬至前后:总没有个真著话
儿。求老爷明白指示指示。”

那先生说:“大奶奶这个症候,可是众位耽搁了!要在初次行经的时候就用药治
起,只怕此时已全愈了。如今既是把病耽误到这地位,也是应有此灾。依我看起来,
病倒尚有三分治得。吃了我这药看,若是夜间睡的着觉,那时又添了二分拿手了。
据我看这脉息,大奶奶是个心性高强、聪明不过的人。但聪明太过,则不如意事常
有;不如意事常有,则思虑太过:此病是忧虑伤脾,肝木忒旺,经血所以不能按时
而至。大奶奶从前行经的日子问一问,断不是常缩,必是常长的。是不是?”这婆
子答道:“可不是!从没有缩过,或是长两日三日,以至十日不等,都长过的。”先
生听道:“是了,这就是病源了。从前若能以养心调气之药服之,何至于此!这如今
明显出一个水亏火旺的症候来。待我用药看。”于是写了方子,递与贾蓉,上写的
是:

益气养荣补脾和肝汤

人参二钱
白术二钱土炒
云苓三钱
熟地四钱
归身二钱
白芍二钱
川芎
一钱五分
黄芪三钱
香附米二钱
醋柴胡八分
淮山药二钱炒
真阿胶二钱蛤粉
炒
延胡索
钱半酒炒
炙甘草八分
引用建莲子七粒去心
大枣二枚

贾蓉看了说:“高明的很。还要请教先生:这病与性命终久有妨无妨?”先生
笑道:“大爷是最高明的人:人病到这个地位,非一朝一夕的症候了;吃了这药,
也要看医缘了。依小弟看来,今年一冬是不相干的;总是过了春分,就可望全愈了。”
贾蓉也是个聪明人,也不往下细问了。

于是贾蓉送了先生去了,方将这药方子并脉案都给贾珍看了,说的话也都回了
贾珍并尤氏了。尤氏向贾珍道:“从来大夫不像他说的痛快,想必用药不错的。”贾
珍笑道:“他原不是那等混饭吃久惯行医的人,因为冯紫英我们相好,他好容易求
了他来的。既有了这个人,媳妇的病或者就能好了。他那方子上有人参,就用前日
买的那一斤好的罢。”贾蓉听毕了话,方出来叫人抓药去煎给秦氏吃。

不知秦氏服了此药,病势如何,且听下回分解。