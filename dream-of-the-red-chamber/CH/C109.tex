\chapter{候芳魂五儿承错爱~还孽债迎女返真元}

话说宝钗叫袭人问出原故,恐宝玉悲伤成疾,便将黛玉临死的话与袭人假作闲
谈,说是:“人在世上,有意有情,到了死后,各自干各自的去了,并不是生前那
样的人死后还是那样。活人虽有痴心,死的竟不知道。况且林姑娘既说仙去,他看
凡人是个不堪的浊物,那里还肯混在世上?只是人自己疑心,所以招出些邪魔外祟
来缠扰。”宝钗虽是与袭人说话,原说给宝玉听的。袭人会意,也说是:“没有的
事。若说林姑娘的魂灵儿还在园里,我们也算相好,怎么没有梦见过一次?”宝玉
在外面听着,细细的想道:“果然也奇。我知道林妹妹死了,那一日不想几遍,怎
么从没梦见?想必他到天上去了,瞧我这凡夫俗子不能交通神明,所以梦都没有一
个儿。我如今就在外间睡,或者我从园里回来,他知道我的心,肯与我梦里一见。
我必要问他实在那里去了,我也时常祭奠。若是果然不理我这浊物,竟无一梦,我
便也不想他了。”主意已定,便说:“我今夜就在外间睡,你们也不用管我。”宝
钗也不强他,只说:“你不用胡思乱想。你没瞧见太太因你园里去了,急的话都说
不出来?你这会子还不保养身子,倘或老太太知道了,又说我们不用心。”宝玉道:
“白这么说罢咧,我坐一会子就进来。你也乏了,先睡罢。”宝钗料他必进来的,
假意说道:“我睡了,叫袭姑娘伺候你罢。”

宝玉听了,正合机宜。等宝钗睡下,他便叫袭人麝月另铺设下一副被褥,常叫
人进来瞧二奶奶睡着了没有。宝钗故意装睡,也是一夜不宁。那宝玉只当宝钗睡着,
便与袭人道:“你们各自睡罢,我又不伤感。你若不信,你就伏侍我睡了再进去,
只要不惊动我就是了。”袭人果然伏侍他睡下,预备下了茶水,关好了门,进里间
去照应了一回,各自假寐,等着宝玉若有动静再出来。宝玉见袭人进去了,便将坐
更的两个婆子支到外头。他轻轻的坐起来,暗暗的祝赞了几句,方才睡下。起初再
睡不着,以后把心一静,谁知竟睡着了,却倒一夜安眠。直到天亮,方才醒来,拭
了拭眼,坐着想了一回,并无有梦。便叹口气道:“正是‘悠悠生死别经年,魂魄
不曾来入梦’!”宝钗反是一夜没有睡着,听见宝玉在外边念这两句,便接口道:
“这话你说莽撞了。若林妹妹在时,又该生气了。”宝玉听了,自觉不好意思,只
得起来,搭讪着进里间来,说:“我原要进来,不知怎么一个盹儿就打着了。”宝
钗道:“你进来不进来,与我什么相干?”

袭人也本没有睡,听见他们两个说话,即忙上来倒茶。只见老太太那边打发小
丫头来问:“宝二爷昨夜睡的安顿么?若安顿,早早的同二奶奶梳洗了就过去。”
袭人道:“你去回老太太,说:‘宝玉昨夜很安顿,回来就过来。’”小丫头去了。
宝钗连忙梳洗了,莺儿袭人等跟着,先到贾母那里行了礼。便到王夫人那边起,至
凤姐,都让过了。仍到贾母处,见他母亲也过来了。大家问起:“宝玉晚上好么?”
宝钗便说:“回去就睡了,没有什么。”众人放心,又说些闲话。

只见小丫头进来,说:“二姑奶奶要回去了。听见说,孙姑爷那边人来,到大
太太那里说了些话,大太太叫人到四姑娘那边说,不必留了,让他去罢。如今二姑
奶奶在大太太那边哭呢,大约就过来辞老太太。”贾母众人听了,心中好不自在,
都说:“二姑娘这么一个人,为什么命里遭着这样的人!一辈子不能出头,这可怎
么好呢。”说着,迎春进来,泪痕满面。因是宝钗的好日子,只得含着泪,辞了众
人要回去。贾母知道他的苦处,也不便强留,只说道:“你回去也罢了,但只不用
伤心。碰着这样人也是没法儿的。过几天我再打发人接你去罢。”迎春道:“老太
太始终疼我,如今也疼不来了。可怜我没有再来的时候儿了。”说着,眼泪直流。
众人都劝道:“这有什么不能回来的呢?比不得你三妹妹隔得远,要见面就难了。”
贾母等想起探春,不觉也大家落泪。为是宝钗的生日,只得转悲作喜说:“这也不
难。只要海疆平静,那边亲家调进京来,就见的着了。”大家说:“可不是这么着
么?”说着,迎春只得含悲而别。大家送了出来,仍回贾母那里。从早至暮,又闹
了一天,众人见贾母劳乏,各自散了。

独有薛姨妈辞了贾母,到宝钗那里,说道:“你哥哥是今年过了,直要等到皇
恩大赦的时候,减了等,才好赎罪。这几年叫我孤苦伶仃,怎么处!我想要给你二
哥哥完婚,你想想好不好?”宝钗道:“妈妈是因为大哥哥娶了亲,唬怕了的,所
以把二哥哥的事也疑惑起来。据我说,很该办。邢姑娘是妈妈知道的,如今在这里
也很苦。娶了去,虽说咱们穷,究竟比他傍人门户好多着呢。”薛姨妈道:“你得
便的时候,就去回明老太太,说我家没人,就要择日子了。”宝钗道:“妈妈只管
和二哥哥商量,挑个好日子,过来和老太太、大太太说了,娶过去,就完了一宗事。
这里大太太也巴不得娶了去才好。”薛姨妈道:“今日听见史姑娘也就回去了,老
太太心里要留你妹妹在这里住几天,所以他住下了。我想他也是不定多早晚就走的
人了,你们姐妹们也多叙几天话儿。”宝钗道:“正是呢。”于是薛姨妈又坐了一
坐,出来辞了众人回去了。

却说宝玉晚间归房,因想:“昨夜黛玉竟不入梦,或者他已经成仙,所以不肯
来见我这种浊人,也是有的;不然,就是我的性儿太急了,也未可知。”便想了个
主意,向宝钗说道:“我昨夜偶然在外头睡着,似乎比在屋里睡的安稳些,今日起
来,心里也觉清净。我的意思,还要在外头睡两夜,只怕你们又来拦我。”宝钗听
了,明知早晨他嘴里念诗自然是为黛玉的事了,想来他那个呆性是不能劝的,倒好
叫他睡两夜,索性自己死了心也罢了,况兼昨夜听他睡的倒也安静。便道:“好没
来由,你只管睡去,我们拦你作什么?但只别胡思乱想的招出些邪魔外祟来。”宝
玉笑道:“谁想什么。”袭人道:“依我劝,二爷竟还是屋里睡罢。外边一时照应
不到,着了凉,倒不好。”宝玉未及答言,宝钗却向袭人使了个眼色儿。袭人会意,
道:“也罢,叫个人跟着你罢,夜里好倒茶倒水的。”宝玉便笑道:“这么说,你
就跟了我来。”袭人听了,倒没意思起来,登时飞红了脸,一声也不言语。宝钗素
知袭人稳重,便说道:“他是跟惯了我的,还叫他跟着我罢。叫麝月五儿照料着也
罢了。况且今日他跟着我闹了一天,也乏了,该叫他歇歇了。”宝玉只得笑着出来。
宝钗因命麝月五儿给宝玉仍在外间铺设了,又嘱咐两个人:“醒睡些。要茶要水,
都留点神儿。”两个答应着。出来看见宝玉端然坐在床上,闭目合掌,居然像个和
尚一般,两个也不敢言语,只管瞅着他笑。宝钗又命袭人出来照应。袭人看见这般,
却也好笑,便轻轻的叫道:“该睡了。怎么又打起坐来了?”宝玉睁开眼看见袭人,
便道:“你们只管睡罢,我坐一坐就睡。”袭人道:“因为你昨日那个光景,闹的
二奶奶一夜没睡,你再这么着成什么事?”宝玉料着自己不睡,都不肯睡,便收拾
睡下。袭人又嘱咐了麝月等几句,才进去关门睡了。这里麝月五儿两个人也收拾了
被褥,伺候宝玉睡着,各自歇下。

那知宝玉要睡越睡不着,见他两个人在那里打铺,忽然想起那年袭人不在家
时,晴雯麝月两个人服事,夜间麝月出去,晴雯要唬他,因为没穿衣服着了凉,后
来还是从这个病上死的。想到这里,一心移在晴雯身上去了。忽又想起凤姐说五儿
给晴雯“脱了个影儿”,因将想晴雯的心又移在五儿身上。自己假装睡着,偷偷儿
的看那五儿,越瞧越像晴雯,不觉呆性复发。听了听里间已无声息,知是睡了;但
不知麝月睡了没有,便故意叫了两声,却不答应。五儿听见了宝玉叫人,便问道:
“二爷要什么?”宝玉道:“我要漱漱口。”五儿见麝月已睡,只得起来,重新剪
了蜡花,倒了一钟茶来,一手托着漱盂。却因赶忙起来的,身上只穿着一件桃红绫
子小袄儿,松松的挽着一个儿。宝玉看时,居然晴雯复生。忽又想起晴雯说的“早
知担了虚名,也就打个正经主意了”,不觉呆呆的呆看,也不接茶。

那五儿自从芳官去后,也无心进来了。后来听说凤姐叫他进来伏侍宝玉,竟比
宝玉盼他进来的心还急。不想进来以后,见宝钗袭人一般尊贵稳重,看着心里实在
敬慕;又见宝玉疯疯傻傻,不似先前的丰致;又听见王夫人为女孩子们和宝玉玩笑
都撵了,所以把那女儿的柔情和素日的痴心,一概搁起。怎奈这位呆爷今晚把他当
作晴雯,只管爱惜起来。那五儿早已羞得两颊红潮,又不敢大声说话,只得轻轻的
说道:“二爷,漱口啊。”宝玉笑着接了茶在手中,也不知道漱了没有,便笑嘻嘻
的问道:“你和晴雯姐姐好不是啊?”五儿听了,摸不着头脑,便道:“都是姐妹,
也没有什么不好的。”宝玉又悄悄的问道:“晴雯病重了,我看他去,不是你也去
了么?”五儿微微笑着点头儿。宝玉道:“你听见他说什么了没有?”五儿摇着头
儿道:“没有。”宝玉已经忘神,便把五儿的手一拉。五儿急的红了脸,心里乱跳,
便悄悄说道:“二爷,有什么话只管说,别拉拉扯扯的。”宝玉才撒了手,说道:
“他和我说来着:‘早知担了个虚名,也就打正经主意了。’你怎么没听见么?”
五儿听了,这话明明是撩拨自己的意思,又不敢怎么样,便说道:“那是他自己没
脸。这也是我们女孩儿家说得的吗?”宝玉着急道:“你怎么也是这么个道学先生!
我看你长的和他一模一样,我才肯和你说这个话,你怎么倒拿这些话遭塌他?”

此时五儿心中也不知宝玉是怎么个意思,便说道:“夜深了,二爷睡罢,别紧
着坐着,看凉着了。刚才奶奶和袭人姐姐怎么嘱咐来!”宝玉道:“我不凉。”说
到这里,忽然想起五儿没穿着大衣裳,就怕他也像晴雯着了凉,便问道:“你为什
么不穿上衣裳就过来?”五儿道:“爷叫的紧,那里有尽着穿衣裳的空儿?要知道
说这半天话儿时,我也穿上了。”宝玉听了,连忙把自己盖的一件月白绫子绵袄儿
揭起来递给五儿,叫他披上。五儿只不肯接,说:“二爷盖着罢,我不凉。我凉,
我有我的衣裳。”说着,回到自己铺边,拉了一件长袄披上。又听了听,麝月睡的
正浓,才慢慢过来说:“二爷今晚不是要养神呢吗?”宝玉笑道:“实告诉你罢:
什么是养神!我倒是要遇仙的意思。”五儿听了,越发动了疑心,便问道:“遇什
么仙?”宝玉道:“你要知道,这话长着呢。你挨着我来坐下,我告诉你。”五儿
红了脸,笑道:“你在那里躺着,我怎么坐呢?”宝玉道:“这个何妨?那一年冷
天,也是你晴雯姐姐和麝月姐姐玩,我怕冻着他,还把他揽在一个被窝儿里呢。这
有什么?大凡一个人,总别酸文假醋的才好。”五儿听了,句句都是宝玉调戏之意,
那知这位呆爷却是实心实意的话。五儿此时走开不好,站着不好,坐下不好,倒没
了主意。因拿眼一溜,抿着嘴儿笑道:“你别混说了。看人家听见,什么意思?怨
不得人家说你专在女孩儿身上用工夫。你自己放着二奶奶和袭人姐姐,都是仙人儿
似的,只爱和别人混搅。明儿再说这些话,我回了二奶奶,看你什么脸见人。”正
说着,只听外面“咕咚”一声,把两个人吓了一跳。里间宝钗咳嗽了一声,宝玉听
见连忙嘴儿,五儿也就忙忙的息了灯,悄悄的躺下了。原来宝钗袭人因昨夜不曾
睡,又兼日间劳乏了一天,所以睡去,都不曾听见他们说话,此时院中一响,猛然
惊醒,听了听,也无动静。宝玉此时躺在床上,心里疑惑:“莫非林妹妹来了,听
见我和五儿说话,故意吓我们的?”翻来覆去,胡思乱想,五更以后,才朦胧睡去。

却说五儿被宝玉鬼混了半夜,又兼宝钗咳嗽,自己怀着鬼胎,生怕宝钗听见了,
也是思前想后,一夜无眠。次日一早起来,见宝玉尚自昏昏睡着,便轻轻儿的收拾
了屋子。那时麝月已醒,便道:“你怎么这么早起来了?你难道一夜没睡吗?”五
儿听这话又似麝月知道了的光景,便只是讪笑,也不答言。一时宝钗袭人也都起来,
开了门。见宝玉尚睡,却也纳闷:“怎么在外头两夜睡的倒这么安稳呢?”及宝玉
醒来,见众人都起来了,自己连忙爬起。揉着眼睛,细想昨夜又不曾梦见,可是“仙
凡路隔”了。慢慢的下了床,又想昨夜五儿说的“宝钗袭人都是天仙一般”,这话
却也不错,便怔怔的瞅着宝钗。

宝钗见他发怔,虽知他为黛玉之事,却也定不得梦不梦,只是瞅的自己倒不好
意思的,便道:“你昨夜可遇见仙了么?”宝玉听了,只道昨晚的话宝钗听见了,
笑着勉强说道:“这是那里的话?”那五儿听了这一句,越发心虚起来,又不好说
的,只得且看宝钗的光景。只见宝钗又笑着问五儿道:“你听见二爷睡梦里和人说
话来着么?”宝玉听了,自己坐不住,搭讪着走开了。五儿把脸飞红,只得含糊道:
“前半夜倒说了几句,我也没听真。什么‘担了虚名’,又什么‘没打正经主意’,
我也不懂,劝着二爷睡了。后来我也睡了,不知二爷还说来着没有。”宝钗低头一
想:“这话明是为黛玉了。但尽着叫他在外头,恐怕心邪了,招出些花妖柳怪来。
况兼他的旧病,原在姐妹上情重,只好设法将他的心意挪移过来,然后能免无事。”
想到这里,不免面红耳热起来,也就讪讪的进房梳洗去了。

且说贾母两日高兴,略吃多了些,这晚有些不受用;第二天,便觉着胸口饱闷。
鸳鸯等要回贾政,贾母不叫言语,说:“我这两日嘴馋些,吃多了点子。我饿一顿
就好了,你们快别吵嚷。”于是鸳鸯等并没有告诉人。这日晚间,宝玉回到自己屋
里,见宝钗自贾母王夫人处才请了晚安回来。宝玉想着早起之事,未免赧颜抱惭,
宝钗看他这样的,也晓得是没意思的光景。因想着他是个痴情人,要治他的这个病,
少不得仍以痴情治之。想了想,便问宝玉道:“你今夜还在外头睡去罢咧?”宝玉
自觉没趣,便道:“里头外头都是一样的。”宝钗意欲再说,反觉碍难出口。袭人
道:“罢呀,这倒是什么道理呢?我不信睡的那么安顿。”五儿听见这话,连忙接
口道:“二爷在外头睡,别的倒没有什么,只爱说梦话,叫人摸不着头脑儿,又不
敢驳他的回。”袭人便道:“我今日挪出床上睡睡,看说梦话不说。你们只管把二
爷的铺盖铺在里间就完了。”宝钗听了,也不作声。宝玉自己惭愧,那里还有强嘴
的分儿,便依着搬进来。一则宝玉抱歉,欲安宝钗之心;二则宝钗恐宝玉思郁成疾,
不如稍示柔情,使得亲近,以为移花接木之计。于是当晚袭人果然挪出去。这宝玉
固然是有意负荆,那宝钗自然也无心拒客,从过门至今日,方才是雨腻云香,氤氲
调畅。从此“二五之精,妙合而凝”。此是后话不提。

且说次日宝玉宝钗同起,宝玉梳洗了,先过贾母这边来。这里贾母因疼宝玉,
又想宝钗孝顺,忽然想起一件东西来。便叫鸳鸯开了箱子,取出祖上所遗的一个汉
玉,虽不及宝玉他那块玉石,挂在身上却也希罕。鸳鸯找出来递与贾母,便说道:
“这件东西,我好像从没见的。老太太这些年还记得这样清楚,说是那一箱什么匣
子里装着,我按着老太太的话一拿就拿出来了。老太太这会子叫拿出来做什么?”
贾母道:“你那里知道?这块玉还是祖爷爷给我们老太爷,老太爷疼我,临出嫁的
时候叫了我去,亲手递给我的。还说:‘这玉是汉朝所佩的东西,很贵重,你拿着
就像见了我的一样。’我那时还小,拿了来也不当什么便撩在箱子里。到了这里,
我见咱们家的东西也多,这算得什么,从没带过,一撩便撩了六十多年。今儿见宝
玉这样孝顺,他又丢了一块玉,故此想着拿出来给他,也像是祖上给我的意思。”
一时宝玉请了安,贾母便喜欢道:“你过来,我给你一件东西瞧瞧。”宝玉走到床
前,贾母便把那块汉玉递给宝玉。宝玉接来一瞧,那玉有三寸方圆,形似甜瓜,色
有红晕,甚是精致。宝玉口口称赞。贾母道:“你爱么?这是我祖爷爷给我的,我
传了你罢。”宝玉笑着,请了个安谢了,又拿了要送给他母亲瞧。贾母道:“你太
太瞧了,告诉你老子,又说疼儿子不如疼孙子了。他们从没见过。”宝玉笑着去了。
宝钗等又说了几句话,也辞了出来。

自此,贾母两日不进饮食,胸口仍是膨闷,觉得头晕目眩,咳嗽。邢王二夫人、
凤姐等请安,见贾母精神尚好,不过叫人告诉贾政,立刻来请了安。贾政出来,即
请大夫看脉。不多一时,大夫来诊了脉,说是有年纪的人,停了些饮食,感冒些风
寒,略消导发散些就好了。开了方子,贾政看了,知是寻常药品,命人煎好进服。
以后贾政早晚进来请安。一连三日,不见稍减。贾政又命贾琏打听好大夫,“快去
请来瞧老太太的病。咱们家常请的几个大夫,我瞧着不怎么好,所以叫你去。”贾
琏想了一想,说道:“记得那年宝兄弟病的时候,倒是请了一个不行医的来瞧好了
的,如今不如找他。”贾政道:“医道却是极难的,越是不兴时的大夫倒有本领。
你就打发人去找来罢。”贾琏即忙答应去了,回来说道:“这刘大夫新近出城教书
去了,过十来天进城一次。这时等不得,又请了一位,也就来了。”贾政听了,只
得等着,不提。

且说贾母病时,合宅女眷无日不来请安。一日,众人都在那里,只见看园内腰
门的老婆子进来回说:“园里的栊翠庵的妙师父知道老太太病了,特来请安。”众
人道:“他不常过来,今儿特来,你们快请进来。”凤姐走到床前回了贾母。岫烟
是妙玉的旧相识,先走出去接他。只见妙玉头带妙常冠,身上穿一件月白素绸袄儿,
外罩一件水田青缎镶边长背心,拴着秋香色的丝绦,腰下系一条淡墨画的白绫裙,
手执麈尾念珠,跟着一个侍儿,飘飘拽拽的走来。岫烟见了问好,说是:“在园内
住的时候儿,可以常来瞧瞧你;近来因为园内人少,一个人轻易难出来。况且咱们
这里的腰门常关着,所以这些日子不得见你。今儿幸会。”妙玉道:“头里你们是
热闹场中,你们虽在外园里住,我也不便常来亲近。如今知道这里的事情也不大好,
又听说是老太太病着,又惦记着你,还要瞧瞧宝姑娘。我那管你们关不关?我要来
就来,我不来,你们要我来也不能啊。”岫烟笑道:“你还是这种脾气。”

一面说着,已到贾母房中。众人见了,都问了好。妙玉走到贾母床前问候,说
了几句套话。贾母便道:“你是个女菩萨,你瞧瞧我的病可好的了好不了?”妙玉
道:“老太太这样慈善的人,寿数正有呢。一时感冒,吃几帖药,想来也就好了。
有年纪的人,只要宽心些。”贾母道:“我倒不为这些。我是极爱寻快乐的。如今
这病也不觉怎么着,只是胸膈饱闷。刚才大夫说是气恼所致。你是知道的,谁敢给
我气受?这不是那大夫脉理平常么?我和琏儿说了,还是头一个大夫说感冒伤食的
是,明儿还请他来。”说着,叫鸳鸯:“吩咐厨房里办一桌净素菜来,请妙师父这
里便饭。”妙玉道:“我吃过午饭了,我是不吃东西的。”王夫人道:“不吃也罢,
咱们多坐一会,说些闲话儿罢。”妙玉道:“我久已不见你们,今日来瞧瞧。”又
说了一回话,便要走。回头见惜春站着,便问道:“四姑娘为什么这样瘦?不要只
管爱画劳了心。”惜春道:“我久不画了。如今住的房屋不比园里的显亮,所以没
兴头画。”妙玉道:“你如今住在那一所?”惜春道:“就是你才来的那个门东边
的屋子,你要来很近。”妙玉道:“我高兴的时候来瞧你。”惜春等说着送了出去。
回身过来,听见丫头们回说大夫在贾母那边呢,众人暂且散去。

那知贾母这病日重一日,延医调治不效,以后又添腹泻。贾政着急,知病难医,
即命人到衙门告诉,日夜同王夫人亲侍汤药。一日,见贾母略进些饮食,心里稍宽,
只见老婆子在门外探头。王夫人叫彩云看去,问问是谁。彩云看了是陪迎春到孙家
去的人,便道:“你来做什么?”婆子道:“我来了半日,这里找不着一个姐姐们,
我又不敢冒撞,我心里又急。”彩云道:“你急什么?又是姑爷作践姑娘不成么?”
婆子道:“姑娘不好了!前儿闹了一场,姑娘哭了一夜,昨日痰堵住了。他们又不
请大夫,今日更利害了。”彩云道:“老太太病着呢,别大惊小怪的。”王夫人在
内已听见了,恐老太太听见不受用,忙叫彩云带他外头说去。岂知贾母病中心静,
偏偏听见,便道:“迎丫头要死了么?”王夫人便道:“没有。婆子们不知轻重,
说是这两日有些病,恐不能就好,到这里问大夫。”贾母道:“瞧我的大夫就好,
快请了去。”王夫人便叫彩云:“叫这婆子去回大太太去。”那婆子去了。这里贾
母便悲伤起来,说是:“我三个孙女儿:一个享尽了福死了;三丫头远嫁,不得见
面;迎丫头虽苦,或者熬出来,不打量他年轻轻儿的就要死了!留着我这么大年纪
的人活着做什么!”王夫人鸳鸯等解劝了好半天。那时宝钗李氏等不在房中,凤姐
近来有病,王夫人恐贾母生悲添病,便叫人叫了他们来陪着,自己回到房中,叫彩
云来埋怨:“这婆子不懂事!以后我在老太太那里,你们有事,不用来回。”丫头
们依命不言。岂知那婆子刚到邢夫人那里,外头的人已传进来,说:“二姑奶奶死
了。”邢夫人听了,也便哭了一场。现今他父亲不在家中,只得叫贾琏快去瞧看。
知贾母病重,众人都不敢回。可怜一位如花似月之女,结缡年馀,不料被孙家揉搓,
以致身亡。又值贾母病笃,众人不便离开,竟容孙家草草完结。

贾母病势日增,只想这些孙女儿。一时想起湘云,便打发人去瞧他。回来的人
悄悄的找鸳鸯。因鸳鸯在老太太身旁,王夫人等都在那里,不便上去,到了后头,
找了琥珀,告诉他道:“老太太想史姑娘,叫我们去打听。那里知道史姑娘哭的了
不得,说是姑爷得了暴病,大夫都瞧了,说这病只怕不能好,若是变了痨病,还可
捱个四五年。所以史姑娘心里着急。又知道老太太病,只是不能过来请安。还叫我
别在老太太跟前提起来,倘或老太太问起来,务必托你们变个法儿回老太太才好。”
琥珀听了,了一声,也就不言语了,半日说道:“你去罢。”琥珀也不便回,心
里打算告诉鸳鸯叫他撒谎去,所以来到贾母床前。见贾母神色大变,地下站着一屋
子的人,嘁嘁喳喳的说:“瞧着是不好。”也不敢言语了。这里贾政悄悄的叫贾琏
到身旁,向耳边说了几句话。贾琏轻轻的答应,出去了,便传齐了现在家里的一干
人,说:“老太太的事,待好出来了,你们快快分头派人办去。头一件,先请出板
来瞧瞧,好挂里子。快到各处将各人的衣服量了尺寸,都开明了,便叫裁缝去做孝
衣。那棚杠执事都讲定了。厨房里还该多派几个人。”赖大等回道:“二爷,这些
事不用爷费心,我们早打算好了,只是这项银子在那里领呢?”贾琏道:“这种银
子不用外头去,老太太自己早留下了。刚才老爷的主意,只要办的好,我想外面也
要好看。”赖大等答应,派人分头办去。

贾琏复回到自己房中,便问平儿:“你奶奶今儿怎么样?”平儿把嘴往里一努,
说:“你瞧去。”贾琏进内,见凤姐正要穿衣,一时动不得,暂且靠在炕桌儿上。
贾琏道:“你只怕养不住了,老太太的事,今儿明儿就要出来了,你还脱得过么?
快叫人将屋里收拾收拾,就该扎挣上去了。若有了事,你我还能回来么?”凤姐道:
“咱们这里还有什么收拾的!不过就是这点子东西,还怕什么?你先去罢,看老爷叫
你。我换件衣裳就来。”贾琏先回到贾母房里,向贾政悄悄的回道:“诸事已交派
明白了。”贾政点头。外面又报:“太医来了。”贾琏接入,诊了一回。大夫出来,
悄悄的告诉贾琏:“老太太的脉气不好,防着些。”贾琏会意,与王夫人等说知。
王夫人即忙使眼色叫鸳鸯过来,叫他把老太太的装裹衣服预备出来。鸳鸯自去料
理。

贾母睁眼要茶喝,邢夫人便进了一杯参汤。贾母刚用嘴接着喝,便道:“不要
这个,倒一钟茶来我喝。”众人不敢违拗,即忙送上来。一口喝了,还要,又喝一
口,便说:“我要坐起来。”贾政等道:“老太太要什么,只管说,可以不必坐起
来才好。”贾母道:“我喝了口水,心里好些儿,略靠着和你们说说话儿。”珍珠
等用手轻轻的扶起,看见贾母这会子精神好了些。

未知生死,下回分解。