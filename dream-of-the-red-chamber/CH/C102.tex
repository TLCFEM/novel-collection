\chapter{宁国府骨肉病灾~大观园符水驱妖孽}

话说王夫人打发人来唤宝钗,宝钗连忙过来请了安。王夫人道:“你三妹妹如
今要出嫁了,你们作嫂子的大家开导开导他,也是你们姊妹之情。况且他也是个明
白孩子,我看你们两个也很合的来。只是我听见说,宝玉听见他三妹妹出门子,哭
的了不的。你也该劝劝他才是。如今我的身子是十病九痛的,你二嫂子也是三日好
两日不好。你还心地明白些,诸事该管的,也别说只管吞着,不肯得罪人。将来这
一番家事都是你的担子。”宝钗答应着。王夫人又说道:“还有一件事,你二嫂子昨
儿带了柳家媳妇的丫头来,说补在你们屋里。”宝钗道:“今日平儿才带过来,说是
太太和二奶奶的主意。”王夫人道:“是呦,你二嫂子和我说,我想也没要紧,不便
驳他的回。只是一件,我见那孩子眉眼儿上头也不是个很安顿的。起先为宝玉房里
的丫头狐狸似的,我撵了几个,那时候你也自然知道,才搬回家去的。如今有你,
固然不比先前了。我告诉你,不过留点神儿就是了。你们屋里,就是袭人那孩子还
可以使得。”宝钗答应了,又说了几句话,便过来了。饭后到了探春那边,自有一
番殷勤劝慰之言,不必细说。

次日,探春将要起身,又来辞宝玉。宝玉自然难割难分。探春倒将纲常大体的
话,说的宝玉始而低头不语,后来转悲作喜,似有醒悟之意。于是探春放心辞别众
人,竟上轿登程,水舟陆车而去。

先前众姊妹们都住在大观园中,后来贾妃薨后,也不修葺。到了宝玉娶亲,林
黛玉一死,史湘云回去,宝琴在家住着,园中人少,况兼天气寒冷,李纨姊妹、探
春、惜春等俱挪回旧所。到了花朝月夕,依旧相约玩耍。如今探春一去,宝玉病后
不出屋门,益发没有高兴的人了。所以园中寂寞,只有几家看园的人住着。

那日,尤氏过来送探春起身,因天晚省得套车,便从前年在园里开通宁府的那
个便门里走过去了。觉得凄凉满目,台榭依然,女墙一带都种作园地一般,心中怅
然如有所失。因到家中,便有些身上发热。扎挣一两天,竟躺倒了。日间的发烧犹
可,夜里身热异常,便谵语绵绵。贾珍连忙请了大夫看视,说感冒起的,如今缠经
入了足阳明胃经,所以谵语不清,如有所见,有了大秽即可身安。尤氏服了两剂,
并不稍减,更加发起狂来。贾珍着急,便叫贾蓉来:“打听外头有好医生,再请几
位来瞧瞧。”贾蓉回道:“前儿这个大夫是最兴时的了,只怕我母亲的病不是药治得
好的。”贾珍道:“胡说,不吃药,难道由他去罢?”贾蓉道:“不是说不治,为的
是前日母亲往西府去,回来是穿着园子里走过来的。一到了家就身上发烧,别是撞
客着了罢。外头有个毛半仙,是南方人,卦起的很灵,不如请他来占算占算。看有
信儿呢,就依着他;要是不中用,再请别的好大夫来。”

贾珍听了,即刻叫人请来;坐在书房内喝了茶,便说:“府上叫我,不知占什
么事?”贾蓉道:“家母有病,请教一卦。”毛半仙道:“既如此,取净水洗手,设
下香案,让我起出一课来看就是了。”一时,下人安排定了,他便怀里掏出卦筒来,
走到上头,恭恭敬敬的作了一个揖,手内摇着卦筒,口里念道:“伏以太极两仪,
交感,图书出而变化不穷,神圣作而诚求必应。兹有信官贾某,为因母病,虔
请伏羲、文王、周公、孔子四大圣人,鉴临在上,诚感则灵,有凶报凶,有吉报吉。
先请内象三爻。”说着,将筒内的钱倒在盘内,说:“有灵的,头一爻就是‘交’。”
拿起来又摇了一摇,倒出来,说是“单”。第三爻又是“交”。检起钱来,嘴里说是:
“内爻已示,更请外象三爻,完成一卦。”起出来,是“单拆单”。那毛半仙收了卦
筒和铜钱,便坐下问道:“请坐,请坐,让我来细细的看看。这个卦乃是‘未济’
之卦。世爻是第三爻,午火兄弟劫财,晦气是一定该有的。如今尊驾为母问病,用
神是初爻,真是父母爻动出官鬼来。五爻上又有一层官鬼,我看令堂太夫人的病是
不轻的。还好,还好,如今子亥之水休囚,寅木动而生火。世爻上动出一个子孙来,
倒是克鬼的。况且日月生身,再隔两日,子水官鬼落空,交到戌日就好了。但是父
母爻上变鬼,恐怕令尊大人也有些关碍。就是本身世爻比劫过重,到了水旺土衰的
日子也不好。”说完了,便撅着胡子坐着。

贾蓉起先听他捣鬼,心里忍不住要笑;听他讲的卦理明白,又说生怕父亲也不
好,便说道:“卦是极高明的,但不知我母亲到底是什么病?”毛半仙道:“据这卦
上,世爻午火变水相克,必是寒火凝结。若要断得清楚,揲蓍也不大明白,除非用
‘大六壬’才断的准。”贾蓉道:“先生都高明的么?”毛半仙道:“知道些。”贾蓉
便要请教,报了一个时辰。毛先生便画了盘子,将神将排定算去,是戌上白虎。“这
课加做‘魄化课’。大凡白虎乃是凶将,乘旺象气受制,便不能为害。如今乘着死
神死煞及时令囚死,则为饿虎,定是伤人。就如魄神受惊消散,故名‘魄化’。这
课象说是人身丧魄,忧患相仍,病多丧死,讼有忧惊。按象有日暮虎临,必定是傍
晚得病的。象内说:‘凡占此课,必定旧宅有伏虎作怪,或有形响。’如今尊驾为大
人而占,正合着虎在阳忧男,在阴忧女,此课十分凶险呢。”贾蓉没有听完,唬得
面上失色道:“先生说的很是,但与那卦又不大相合,到底有妨碍么?”毛半仙道:
“你不用慌,待我慢慢的再看。”低着头又咕哝了一会子,便说:“好了,有救星了。
算出巳上有贵神救解,谓之‘魄化魂归’,先忧后喜,是不妨事的,只要小心些就
是了。”

贾蓉奉上卦金,送了出去,回禀贾珍,说是:“母亲的病,是在旧宅傍晚得的,
为撞着什么‘伏尸白虎’。”贾珍道:“你说你母亲前日从园里走回来的,可不是那
里撞着的!你还记得你二婶娘到园里去,回来就病了?他虽没有见什么,后来那些丫
头老婆们都说是山子上一个毛烘烘的东西,眼睛有灯笼大,还会说话,他把二奶奶
赶回来了,唬出一场病来。”贾蓉道:“怎么不记得!我还听见宝二叔家的焙茗说:
晴雯做了园里芙蓉花的神了;林姑娘死了,半空里有音乐,必定他也是管什么花儿
了。想这许多妖怪在园里,还了得。头里人多阳气重,常来常往不打紧;如今冷落
的时候,母亲打那里走,还不知踹了什么花儿呢,不然就是撞着那一个。那卦也还
算是准的。”贾珍道:“到底说有妨碍没有呢?”贾蓉道:“据他说,到了戌日就好
了。只愿早两天好,或除两天才好。”贾珍道:“这又是什么意思?”贾蓉道:“那
先生若是这样准,生怕老爷也有些不自在。”正说着,里头喊说:“奶奶要坐起到那
边园里去,丫头们都按捺不住。”贾珍等进去安慰,只闻尤氏嘴里乱说:“穿红的来
叫我!穿绿的来赶我!”地下这些人又怕又好笑。贾珍便命人买些纸钱,送到园里烧
化。果然那夜出了汗,便安静些。到了戌日,也就渐渐的好起来。

由是,一人传十,十人传百,都说大观园中有了妖怪,唬得那些看园的人也不
修花补树、灌溉果蔬。起先晚上不敢行走,以致鸟兽逼人;近来甚至日间也是约伴
持械而行。过了些时,果然贾珍也病,竟不请医调治,轻则到园化纸许愿,重则详
星拜斗。贾珍方好,贾蓉等相继而病。如此接连数月,闹的两府俱怕。从此风声鹤
唳,草木皆妖。园中出息一概全蠲,各房月例重新添起,反弄的荣府中更加拮据。
那些看园的没有了想头,个个要离此处,每每造言生事,便将花妖树怪编派起来,
各要搬出,将园门封固,再无人敢到园中。以致崇楼高阁,琼馆瑶台,皆为禽兽所
栖。

却说晴雯的表兄吴贵正住在园门口。他媳妇自从晴雯死后,听见说作了花神,
每日晚间便不敢出门。这一日吴贵出门买东西,回来晚了。那媳妇子本有些感冒着
了,日间吃错了药,晚上吴贵到家,已死在炕上。外面的人因那媳妇子不大妥当,
便说妖怪爬过墙来吸了精去死的。于是老太太着急的了不得,另派了好些人将宝玉
的住房围住,巡逻打更。这些小丫头们还说,有看见红脸的,有看见很俊的女人的,
吵嚷不休,唬的宝玉天天害怕。亏得宝钗有把持,听见丫头们混说,便吓唬着要打,
所以那些谣言略好些。无奈各房的人都是疑人疑鬼的不安静,也添了人坐更,于是
更加了好些食用。

独有贾赦不大很信,说:“好好儿的园子,那里有什么鬼怪。”挑了个风清日暖
的日子,带了好几个家人,手内持着器械,到园踹看动静。众人劝他不依。到了园
中,果然阴气逼人。贾赦还扎挣前走,跟的人都探头缩脑的。内中有个年轻的家人,
心内已经害怕,只听唿的一声,回过头来,只见五色灿烂的一件东西跳过去了,唬
的“嗳哟”一声,腿子发软,就躺倒了。贾赦回身查问,那小子喘嘘嘘的回道:“亲
眼看见一个黄脸红胡子绿衣裳一个妖精!走到树林子后头山窟窿里去了。”贾赦听
了,便也有些胆怯,问道:“你们都看见么?”有几个推顺水船儿的回说:“怎么没
瞧见?因老爷在头里,不敢惊动罢了。奴才们还掌得住。”说得贾赦害怕,也不敢再
走。急急的回来,吩咐小子们:“不用提及,只说看遍了,没有什么东西。”心里实
也相信,要到真人府里请法官驱邪。岂知那些家人无事还要生事,今见贾赦怕了,
不但不瞒着,反添些穿凿,说得人人吐舌。贾赦没法,只得请道士到园作法,驱邪
逐妖。择吉日,先在省亲正殿上铺排起坛场来。供上三清圣像,旁设二十八宿并马、
赵、温、周四大将,下排三十六天将图像。香花灯烛设满一堂,钟鼓法器排列两边,
插着五方旗号。道纪司派定四十九位道众的执事,净了一天坛。三位法官行香取水
毕,然后擂起法鼓。法师们俱戴上七星冠,披上九宫八卦的法衣,踏着登云履,手
执牙笏,便拜表请圣。又念了一天的消灾驱邪接福的《洞玄经》,以后便出榜召将。
榜上大书“太乙、混元、上清三境灵宝符演教大法师,行文敕令本境诸神到坛听
用”。

那日两府上下爷们仗着法师擒妖,都到园中观看,都说:“好大法令,呼神遣
将的闹起来,不管有多少妖怪也唬跑了。”大家都挤到坛前。只见小道士们将旗幡
举起,按定五方站住,伺候法师号令。三位法师,一位手提宝剑,拿着法水,一位
捧着七星皂旗,一位举着桃木打妖鞭,立在坛前。只听法器一停,上头令牌三下,
口中念起咒来,那五方旗便团团散布。法师下坛,叫本家领着到各处楼阁殿亭,房
廊屋舍,山崖水畔,洒了法水,将剑指画了一回。回来,连击令牌,将七星旗祭起,
众道士将旗幡一聚接下,打妖鞭望空打了三下。本家众人都道拿住妖怪,争着要看,
及到跟前,并不见有什么形响。只见法师叫众道士拿取瓶罐,将妖收下,加上封条,
法师朱笔书符收起,令人带回在本观塔下镇住,一面撤坛谢将。贾赦恭敬叩谢了法
师。贾蓉等小弟兄背地都笑个不住,说:“这样的大排场,我打量拿着妖怪,给我
们瞧瞧到底是些什么东西,那里知道是这样搜罗。究竟妖怪拿去了没有?”贾珍听
见,骂道:“糊涂东西!妖怪原是聚则成形,散则成气,如今多少神将在这里,还敢
现形吗?无非把这妖气收了,便不作祟,就是法力了。”众人将信将疑,且等不见响
动再说。

那些下人只知妖怪被擒,疑心去了,便不大惊小怪,往后果然没人提起了。贾
珍等病愈复原,都道法师神力。独有一个小厮笑说道:“头里那些响动,我也不知
道。就是跟着大老爷进园这一日,明明是个大公野鸡飞过去了。拴儿吓离了眼,说
的活像,我们都替他圆了个谎,大老爷就认真起来。倒瞧了个很热闹的坛场。”众
人虽然听见,那里肯信,究无人敢住。

一日,贾赦无事,正想要叫几个家下人搬住园中看守,惟恐夜晚藏匿奸人。方
欲传出话去,只见贾琏进来,请了安,回说:“今日到大舅家去,听见一个荒信,
说是二叔被节度使参进来,为的是失察属员,重征粮米,请旨革职的事。”贾赦听
了,吃惊道:“只怕是谣言罢?前儿你二叔带书子来说,探春于某日到了任所,择了
某日吉时,送了你妹子到了海疆,路上风恬浪静,合家不必挂念。还说节度认亲,
倒设席贺喜。那里有做了亲戚倒提参起来的?且不必言语,快到吏部打听明白,就
来回我。”贾琏即刻出去,不到半日回来,便说:“才到吏部打听,果然二叔被参。
题本上去,亏得皇上的恩典,没有交部,便下旨意,说是:‘失察属员,重征粮米,
苛虐百姓,本应革职,姑念初膺外任,不谙吏治,被属员蒙蔽,着降三级,加恩仍
以工部员外上行走,并令即日回京。’这信是准的。正在吏部说话的时候,来了一
个江西引见的知县,说起我们二叔是很感激的。但说是个好上司,只是用人不当,
那些家人在外招摇撞骗,欺凌属员,已经把好名声都弄坏了。节度大人早已知道,
也说我们二叔是个好人。不知怎么样,这回又参了。想是忒闹得不好,恐将来弄出
大祸,所以借了一件失察的事情参的,倒是避重就轻的意思,也未可知。”贾赦未
听说完,便叫贾琏:“先去告诉你婶子知道,且不必告诉老太太就是了。”贾琏去回
王夫人。

未知有何话说,下回分解。