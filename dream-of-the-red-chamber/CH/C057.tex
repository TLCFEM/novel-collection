\chapter{慧紫鹃情辞试莽玉~慈姨妈爱语慰痴颦}

话说宝玉听王夫人唤他,忙至前边来,原来是王夫人要带他拜甄夫人去。宝玉
自是欢喜,忙去换衣服,跟了王夫人到那里。见甄家的形景,自与荣宁不甚差别,
或有一二稍盛的。细问,果有一宝玉。甄夫人留席,竟日方回。宝玉方信。因晚间
回家来,王夫人又吩咐预备上等的席面,定名班大戏,请过甄夫人母女。后二日,
他母女便不作辞,回任去了,无话。

这日宝玉因见湘云渐愈,然后去看黛玉。正值黛玉才歇午觉,宝玉不敢惊动,
因紫鹃正在回廊上手里做针线,便上来问他:“昨日夜里咳嗽的可好些?”紫鹃道:
“好些了。”宝玉笑道:“阿弥陀佛!宁可好了罢。”紫鹃笑道:“你也念起佛来,
真是新闻。”宝玉笑道:“所谓‘病急乱投医’了。”一面说,一面见他穿着弹墨
绫薄绵袄,外面只穿着青缎夹背心,宝玉便伸手向他身上抹了一抹,说道:“穿这
样单薄,还在风口里坐着,时气又不好,你再病了,越发难了。”紫鹃便说道:“从
此咱们只可说话,别动手动脚的:一年大、二年小的,叫人看着不尊重。打紧的那
起混帐行子们背地里说你,你总不留心,还自管和小时一般行为,如何使得?姑娘
常常吩咐我们,不叫和你说笑。你近来瞧他,远着你还恐远不及呢。”说着,便起
身携了针线进别的房里去了。

宝玉见了这般景况,心中像浇了一盆冷水一般,只瞅着竹子发了一回呆,——
因祝妈正在那里刨土种竹,扫竹叶子。顿觉一时魂魄失守,随便坐在一块山石上出
神,不觉滴下泪来。直呆了一顿饭的工夫,千思万想,总不知如何是可。偶值雪雁
从王夫人屋里取了人参来,从此经过,忽扭头看见桃花树下石上一人,手托着腮颊,
正出神呢:不是别人,却是宝玉。雪雁疑惑道:“怪冷的,他一个人在这里做什么?
春天凡有残疾的人肯犯病,敢是他也犯了呆病了?”一边想,一边就走过来,蹲着
笑道:“你在这里做什么呢?”宝玉忽见了雪雁,便说道:“你又做什么来找我?
你难道不是女儿?他既防嫌,不许你们理我,你又来寻我,倘被人看见,岂不又生
口舌?你快家去罢!”

雪雁听了,只当是他又受了黛玉的委屈,只得回至屋里。黛玉未醒,将人参交
给紫鹃。紫鹃因问他:“太太做什么呢?”雪雁道:“也睡中觉呢,所以等了这半
天。姐姐,你听笑话儿:我因等太太的工夫,和玉钏儿姐姐坐在下屋里说话儿,谁
知赵姨奶奶招手儿叫我。我只当有什么话说,原来他和太太告了假,出去给他兄弟
伴宿坐夜,明儿送殡去。跟他的小丫头子小吉祥儿没衣裳,要借我的月白绫子袄儿。
我想他们一般也有两件子的,往这地方去,恐怕弄坏了,自己的舍不得穿,故此借
别人的穿。借我的,弄坏了也是小事,只是我想他素日有什么好处到咱们跟前?所
以我说:我的衣裳簪环,都是姑娘叫紫鹃姐姐收着呢。如今先得去告诉他,还得回
姑娘,费多少事,别误了你老人家出门,不如再转借罢。”紫鹃笑道:“你这个小
东西儿,倒也巧。你不借给他,你往我和姑娘身上推,叫人怨不着你。他这会子就
去呀,还是等明日一早才去呢?”雪雁道:“这会子就走,只怕此时已去了。”紫
鹃点头。雪雁道:“只怕姑娘还没醒呢。是谁给了宝玉气受?坐在那里哭呢!”紫
鹃听了,忙问:“在那里?”雪雁道:“在沁芳亭后头桃花底下呢。”

紫鹃听了,忙放下针,又嘱咐雪雁:“好生听叫。要问我,答应我就来。”说
着,便出了潇湘馆,一径来寻宝玉。走至宝玉跟前,含笑说道:“我不过说了那么
句话,为的是大家好。你就一气跑了这风地里来哭,弄出病来还了得!”宝玉忙笑
道:“谁赌气了!我因为听你说的有理,我想你们既这样说,自然别人也是这样说,
将来渐渐的都不理我了。我所以想到这里,自己伤起心来了。”紫鹃也便挨他坐着。
宝玉笑道:“方才对面说话,你还走开,这会子怎么又来挨着我坐?”紫鹃道:“你
都忘了?几日前头,你们姐儿两个正说话,赵姨娘一头走进来,——我才听见他不
在家,所以我来问你。正是前日你和他才说了一句‘燕窝’,就不说了,总没提起,
我正想着问你。”宝玉道:“也没什么要紧,不过我想着宝姐姐也是客中,既吃燕
窝,又不可间断,若只管和他要,也太托实。虽不便和太太要,我已经在老太太跟
前略露了个风声,只怕老太太和凤姐姐说了。我告诉他的,竟没告诉完。如今我听
见一日给你们一两燕窝,这也就完了。”紫鹃道:“原来是你说了,这又多谢你费
心。我们正疑惑,老太太怎么忽然想起来叫人每一日送一两燕窝来呢?这就是了。”
宝玉笑道:“这要天天吃惯了,吃上三二年就好了。”紫鹃道:“在这里吃惯了,
明年家去,那里有这闲钱吃这个?”

宝玉听了,吃了一惊,忙问:“谁家去?”紫鹃道:“妹妹回苏州去。”宝玉
笑道:“你又说白话。苏州虽是原籍,因没了姑母,无人照看才接了来的。明年回
去找谁?可见撒谎了。”紫鹃冷笑道:“你太看小了人。你们贾家独是大族,人口
多的,除了你家,别人只得一父一母,房族中真个再无人了不成?我们姑娘来时,
原是老太太心疼他年小,虽有叔伯,不如亲父母,故此接来住几年。大了该出阁时,
自然要送还林家的,终不成林家女儿在你贾家一世不成?林家虽贫到没饭吃,也是
世代书香人家,断不肯将他家的人丢给亲戚,落的耻笑。所以早则明年春,迟则秋
天,这里纵不送去,林家亦必有人来接的了。前日夜里姑娘和我说了,叫我告诉你,
将从前小时玩的东西,有他送你的,叫你都打点出来还他;他也将你送他的打点在
那里呢。”

宝玉听了,便如头顶上响了一个焦雷一般。紫鹃看他怎么回答,等了半天,见
他只不作声。才要再问,只见晴雯找来说:“老太太叫你呢。谁知在这里。”紫鹃
笑道:“他这里问姑娘的病症,我告诉了他半天,他只不信,你倒拉他去罢。”说
着,自己便走回房去了。晴雯见他呆呆的,一头热汗,满脸紫胀,忙拉他的手一直
到怡红院中。袭人见了这般,慌起来了,只说时气所感,热身被风扑了。无奈宝玉
发热事犹小可,更觉两个眼珠儿直直的起来,口角边津液流出,皆不知觉。给他个
枕头,他便睡下;扶他起来,他便坐着;倒了茶来,他便吃茶。众人见了这样,一
时忙乱起来,又不敢造次去回贾母,先要差人去请李嬷嬷来。一时李嬷嬷来了,看
了半天:问他几句话,也无回答;用手向他脉上摸了摸,嘴唇人中上着力掐了两下,
掐得指印如许来深,竟也不觉疼。李嬷嬷只说了一声:“可了不得了!”“呀”的
一声,便搂头放身大哭起来。急得袭人忙拉他说:“你老人家瞧瞧可怕不怕,且告
诉我们,去回老太太、太太去。你老人家怎么先哭起来?”李嬷嬷捶床捣枕说:“这
可不中用了!我白操了一世的心了!”

袭人因他年老多知,所以请他来看,如今见他这般一说,都信以为实,也哭起
来了。晴雯便告诉袭人方才如此这般。袭人听了,便忙到潇湘馆来,见紫鹃正伏侍
黛玉吃药,也顾不得什么,便走上来问紫鹃道:“你才和我们宝玉说了些什么话?
你瞧瞧他去!你回老太太去,我也不管了!”说着,便坐在椅上。黛玉忽见袭人满
面急怒,又有泪痕,举止大变,更不免也着了忙,因问怎么了。袭人定了一回,哭
道:“不知紫鹃姑奶奶说了些什么话,那个呆子眼也直了,手脚也冷了,话也不说
了,李妈妈掐着也不疼了,已死了大半个了!连妈妈都说不中用了,那里放声大哭,
只怕这会子都死了!”黛玉听此言,李妈妈乃久经老妪,说不中用了,可知必不中
用,“哇”的一声,将所服之药,一口呕出,抖肠搜肺、炙胃扇肝的,哑声大嗽了
几阵。一时面红发乱,目肿筋浮,喘的抬不起头来。

紫鹃忙上来捶背。黛玉伏枕喘息了半晌,推紫鹃道:“你不用捶!你竟拿绳子
来勒死我,是正经!”紫鹃说道:“我并没说什么,不过是说了几句玩话,他就认
真了。”袭人道:“你还不知道他那傻子,每每玩话认了真?”黛玉道:“你说了
什么话?趁早儿去解说,他只怕就醒过来了。”紫鹃听说,忙下床,同袭人到了怡
红院。谁知贾母王夫人等已都在那里了。贾母一见了紫鹃,便眼内出火,骂道:“你
这小蹄子,和他说了什么?”紫鹃忙道:“并没敢说什么,不过说几句玩语。”谁
知宝玉见了紫鹃,方“嗳呀”了一声,哭出来了。众人一见,都放下心来。贾母便
拉住紫鹃,只当他得罪了宝玉,所以拉紫鹃命他赔罪。谁知宝玉一把拉住紫鹃,死
也不放,说:“要去连我带了去!”众人不解,细问起来,方知紫鹃说要回苏州去,
一句玩话引出来的。贾母流泪道:“我当有什么要紧大事!原来是这句玩话。”又
向紫鹃道:“你这孩子,素日是个伶俐聪敏的,你又知道他有个呆根子,平白的哄
他做什么?”薛姨妈劝道:“宝玉本来心实,可巧林姑娘又是从小儿来的,他姊妹
两个一处长得这么大,比别的姊妹更不同。这会子热剌剌的说一个去,别说他是个
实心的傻孩子,便是冷心肠的大人,也要伤心。这并不是什么大病,老太太和姨太
太只管万安,吃一两剂药就好了。”

正说着,人回:“林之孝家的,赖大家的,都来瞧哥儿来了。”贾母道:“难
为他们想着,叫他们来瞧瞧。”宝玉听了一个“林”字,便满床闹起来说:“了不
得了,林家的人接他们来了!快打出去罢!”贾母听了,也忙说:“打出去罢!”
又忙安慰说:“那不是林家的人,林家的人都死绝了,再没人来接他,你只管放心
罢!”宝玉道:“凭他是谁,除了林妹妹,都不许姓林了!”贾母道:“没姓林的
来,凡姓林的都打出去了。”一面吩咐众人:“以后别叫林之孝家的进园来,你们
也别说‘林’字儿。孩子们,你们听了我这句话罢!”众人忙答应,又不敢笑。一
时宝玉又一眼看见了十锦子上陈设的一只金西洋自行船,便指着乱说:“那不是
接他们来的船来了?湾在那里呢。”贾母忙命拿下来。袭人忙拿下来,宝玉伸手要。
袭人递过去,宝玉便掖在被中,笑道:“这可去不成了!”一面说,一面死拉着紫
鹃不放。

一时人回:“大夫来了。”贾母忙命快进来。王夫人、薛姨妈、宝钗等暂避入
里间,贾母便端坐在宝玉身旁。王太医进来,见许多的人,忙上去请了贾母的安,
拿了宝玉的手,诊了一回。那紫鹃少不得低了头。王太医也不解何意,起身说道:
“世兄这症,乃是急痛迷心。古人曾云痰迷有别,有气血亏柔饮食不能熔化痰迷者,
有怒恼中痰急而迷者,有急痛壅塞者。此亦痰迷之症,系急痛所致,不过一时壅蔽,
较别的似轻些。”贾母道:“你只说怕不怕,谁和你背药书呢!”王太医忙躬身笑
道:“不妨,不妨。”贾母道:“果真不妨?”王太医道:“实在不妨。都在晚生
身上。”贾母道:“既这么着,请外头坐,开了方儿。吃好了呢,我另外预备谢礼,
叫他亲自捧了,送去磕头;要耽误了,我打发人去拆了太医院的大堂。”王太医只
管躬身陪笑说:“不敢,不敢。”他原听说“另具上等谢礼命宝玉去磕头”,故满
口说“不敢”,竟未听见贾母后来说拆太医院之戏语,犹说不敢,贾母与众人反倒
笑了。

一时按方煎药,药来服下,果觉比先安静。无奈宝玉只不肯放紫鹃,只说:“他
去了,就是要回苏州去了。”贾母王夫人无法,只得命紫鹃守着他,另将琥珀去伏
侍黛玉。黛玉不时遣雪雁来探消息。这晚间宝玉稍安,贾母王夫人等方回去了,一
夜还遣人来问几次信。李奶奶带宋妈等几个年老人用心看守,紫鹃、袭人、晴雯等
日夜相伴。有时宝玉睡去,必从梦中惊醒,不是哭了,说黛玉已去,便是说有人来
接。每一惊时,必得紫鹃安慰一番方罢。彼时贾母又命将祛邪守灵丹及开窍通神散
各样上方秘制诸药,按方饮服,次日又服了王太医药,渐次好了起来。宝玉心下明
白,因恐紫鹃回去,倒故意作出佯狂之态。紫鹃自那日也着实后悔,如今日夜辛苦,
并没有怨意。袭人心安神定,因向紫鹃笑道:“都是你闹的,还得你来治。也没见
我们这位呆爷,‘听见风儿就是雨’,往后怎么好!”暂且按下。

且说此时湘云之症已愈,天天过来瞧看,见宝玉明白了,便将他病中狂态形容
给他瞧,引的宝玉自己伏枕而笑。原来他起先那样,竟是不知的,如今听人说还不
信。无人时,紫鹃在侧,宝玉又拉他的手,问道:“你为什么唬我?”紫鹃道:“不
过是哄你玩罢咧,你就认起真来。”宝玉道:“你说的有情有理,如何是玩话呢?”
紫鹃笑道:“那些话,都是我编的。林家真没了人了。纵有也是极远的族中,也都
不在苏州住,各省流寓不定。纵有人来接,老太太也必不叫他去。”宝玉道:“便
老太太放去,我也不依。”紫鹃笑道:“果真的不依?只怕是嘴里的话。你如今也
大了,连亲也定下了,过二三年再娶了亲,你眼睛里还有谁了!”宝玉听了,又惊
问:“谁定了亲?定了谁?”紫鹃笑道:“年里我就听见老太太说要定了琴姑娘呢,
不然,那么疼他?”宝玉笑道:“人人只说我傻,你比我更傻!不过是句玩话,他
已经许给梅翰林家了。果然定下了他,我还是这个形景了?先是我发誓赌咒,砸这
劳什子,你都没劝过吗?我病的刚刚的这几日才好了,你又来怄我!”一面说,一
面咬牙切齿的,又说道:“我只愿这会子立刻我死了,把心迸出来,你们瞧见了。
然后连皮带骨,一概都化成一股灰,再化成一股烟,一阵大风,吹的四面八方,都
登时散了,这才好!”一面说,一面又滚下泪来。

紫鹃忙上来握他的嘴,替他擦眼泪,又忙笑解释道:“你不用着急。这原是我
心里着急,才来试你。”宝玉听了,更又诧异,问道:“你又着什么急?”紫鹃笑
道:“你知道,我并不是林家的人,我也和袭人鸳鸯是一伙的。偏把我给了林姑娘
使,偏偏他又和我极好,比他苏州带来的还好十倍,一时一刻,我们两个离不开。
我如今心里却愁他倘或要去了,我必要跟了他去的。我是合家在这里,我若不去,
辜负了我们素日的情长;若去,又弃了本家。所以我疑惑,故说出这谎话来问你,
谁知你就傻闹起来!”宝玉笑道:“原来是你愁这个,所以你是傻子!从此后再别
愁了。我告诉你一句打趸儿的话:活着,咱们一处活着;不活着,咱们一处化灰、
化烟。如何?”紫鹃听了,心下暗暗筹画。忽有人回:“环爷兰哥儿问候。”宝玉
道:“就说难为他们,我才睡了,不必进来。”婆子答应去了。紫鹃笑道:“你也
好了,该放我回去瞧瞧我们那一个去了。”宝玉道:“正是这话。我昨夜就要叫你
去,偏又忘了。我已经大好了,你就去罢。”紫鹃听说,方打叠铺盖妆奁之类。宝
玉笑道:“我看见你文具儿里头有两三面镜子,你把那面小菱花的给我留下罢。我
搁在枕头傍边,睡着好照,明日出门带着也轻巧。”紫鹃听说,只得与他留下。先
命人将东西送过去,然后别了众人,自回潇湘馆来。

黛玉近日闻得宝玉如此形景,未免又添些病症,多哭几场。今儿紫鹃来了,问
其原故,已知大愈,仍遣琥珀去伏侍贾母。夜间人静后,紫鹃已宽衣卧下之时,悄
向黛玉笑道:“宝玉的心倒实,听见咱们去,就这么病起来。”黛玉不答。紫鹃停
了半晌,自言自语的说道:“一动不如一静。我们这里就算好人家,别的都容易,
最难得的是从小儿一处长大,脾气情性都彼此知道的了。”黛玉啐道:“你这几天
还不乏,趁这会子不歇一歇,还嚼什么蛆?”紫鹃笑道:“倒不是白嚼蛆,我倒是
一片真心为姑娘。替你愁了这几年了:又没个父母兄弟,谁是知疼着热的?趁早儿
老太太还明白硬朗的时节,作定了大事要紧。俗语说:‘老健春寒秋后热。’倘或
老太太一时有个好歹,那时虽也完事,只怕耽误了时光,还不得趁心如意呢。公子
王孙虽多,那一个不是三房五妾,今儿朝东,明儿朝西?娶一个天仙来,也不过三
夜五夜也就撂在脖子后头了。甚至于怜新弃旧反目成仇的,多着呢。娘家有人有势
的还好,要像姑娘这样的,有老太太一日好些,一日没了老太太,也只是凭人去欺
负罢了。所以说,拿主意要紧。姑娘是个明白人,没听见俗语说的:‘万两黄金容
易得,知心一个也难求!’”

黛玉听了,便说道:“这丫头今日可疯了!怎么去了几日,忽然变了一个人?我
明日必回老太太,退回你去,我不敢要你了。”紫鹃笑道:“我说的是好话,不过
叫你心里留神,并没叫你去为非作歹。何苦回老太太,叫我吃了亏,又有什么好处。”
说着,竟自己睡了。黛玉听了这话,口内虽如此说,心内未尝不伤感。待他睡了,
便直哭了一夜,至天明,方打了一个盹儿。次日,勉强盥漱了,吃了些燕窝粥。便
有贾母等亲来看视了,又嘱咐了许多话。

目今是薛姨妈的生日,自贾母起,诸人皆有祝贺之礼,黛玉也只得备了两色针
线送去。是日也定了一班小戏,请贾母与王夫人等。独有宝玉与黛玉二人不曾去。
至晚散时,贾母等顺路又瞧了他二人一遍,方回房去了。次日,薛姨妈家又命薛蝌
陪诸伙计吃了一天酒。连忙了三四天,方才完结。

因薛姨妈看见邢岫烟生得端雅稳重,且家道贫寒,是个钗荆裙布的女儿,便欲
说给薛蟠为妻。因薛蟠素昔行止浮奢,又恐遭塌了人家女儿。正在踌躇之际,忽想
起薛蝌未娶,看他二人,恰是一对天生地设的夫妻,因谋之于凤姐儿。凤姐儿笑道:
“姑妈素知我们太太有些左性的,这事等我慢谋。”因贾母去瞧凤姐儿时,凤姐儿
便和贾母说:“姑妈有一件事要求老祖宗,只是不好启齿。”贾母忙问何事,凤姐
儿便将求亲一事说了。贾母笑道:“这有什么不好启齿的,这是极好的好事,等我
和你婆婆说,没有不依的。”因回房来,即刻就命人叫了邢夫人过来,硬作保山。
邢夫人想了一想:薛家根基不错,且现今大富,薛蝌生得又好,且贾母又作保山。
将计就计,便应了。贾母十分喜欢,忙命人请了薛姨妈来。二人见了,自然有许多
谦辞。邢夫人即刻命人去告诉邢忠夫妇。他夫妇原是此来投靠邢夫人的,如何不依,
早极口的说:“妙极。”贾母笑道:“我最爱管闲事,今日又管成了一件事,不知
得多少谢媒钱?”薛姨妈笑道:“这是自然的。纵抬了整万银子来,只怕不稀罕。
但只一件,老太太既是作媒,还得一位主亲才好。”贾母笑道:“别的没有,我们
家折腿烂手的人还有两个。”说着,便命人去叫过尤氏婆媳二人来。贾母告诉他原
故,彼此忙都道喜。贾母吩咐道:“咱们家的规矩,你是尽知的,从没有两亲家争
礼争面的。如今你算替我在当中料理,不可太省,也不可太费,把他两家的事周全
了回我。”尤氏忙答应了。薛姨妈喜之不尽,回家命写了请帖,补送过宁府。尤氏
深知邢夫人情性,本不欲管,无奈贾母亲自嘱咐,只得应了,惟忖度邢夫人之意行
事。薛姨妈是个无可无不可的人,倒还易说。这且不在话下。

如今薛姨妈既定了邢岫烟为媳,合宅皆知。邢夫人本欲接出岫烟去住,贾母因
说:“这又何妨?两个孩子又不能见面,就是姨太太和他一个大姑子,一个小姑子,
又何妨?况且都是女孩儿,正好亲近些呢。”邢夫人方罢。那薛蝌岫烟二人,前次
途中曾有一面知遇,大约二人心中皆如意。只是那岫烟未免比先时拘泥了些,不好
和宝钗姐妹共处闲谈;又兼湘云是个爱取笑的,更觉不好意思。幸他是个知书达礼
的,虽是女儿,还不是那种佯羞诈鬼、一味轻薄造作之辈。宝钗自那日见他起,想
他家业贫寒;二则别人的父母皆是年高有德之人,独他的父母偏是酒糟透了的人,
于女儿分上平常;邢夫人也不过是脸面之情,亦非真心疼爱;且岫烟为人雅重,迎
春是个老实人,连他自己尚未照管齐全,如何能管到他身上,凡闺阁中家常一应需
用之物,或有亏乏,无人照管,他又不与人张口。宝钗倒暗中每相体贴接济,也不
敢叫邢夫人知道,也恐怕是多心闲话之故。如今却是众人意料之外,奇缘作成这门
亲事。岫烟心中先取中宝钗,有时仍与宝钗闲话,宝钗仍以姊妹相呼。

这日宝钗因来瞧黛玉,恰值岫烟也来瞧黛玉,二人在半路相遇。宝钗含笑唤他
到跟前,二人同走。至一块石壁后,宝钗笑问他:“这天还冷的很,你怎么倒全换
了夹的了?”岫烟见问,低头不答。宝钗便知道又有了原故,因又笑问道:“必定
是这个月的月钱又没得,凤姐姐如今也这样没心没计了。”岫烟道:“他倒想着不
错日子给的。因姑妈打发人和我说道:一个月用不了二两银子,叫我省一两给爹妈
送出去,要使什么,横竖有二姐姐的东西,能着些搭着就使了。姐姐想:二姐姐是
个老实人,也不大留心。我使他的东西,他虽不说什么,他那些丫头妈妈,那一个
是省事的?那一个是嘴里不尖的?我虽在那屋里,却不敢很使唤他们。过三天五天,
我倒得拿些钱出来,给他们打酒买点心吃才好。因此,一月二两银子还不够使。如
今又去了一两,前日我悄悄的把棉衣服叫人当了几吊钱盘缠。”宝钗听了,愁叹道:
“偏梅家又合家在任上,后年才进来。若是在这里,琴儿过去了,好再商议你的事,
离了这里就完了。如今不完了他妹妹的事,也断不敢先娶亲的。如今倒是一件难事。
再迟两年,我又怕你熬煎出病来。等我和妈妈再商议。”宝钗又指他裙上一个璧玉
佩问道:“这是谁给你的?”岫烟道:“这是三姐姐给的。”宝钗点头道:“他见
人人皆有,独你一个没有,怕人笑话,故此送一个,这是他聪明细致之处。”岫烟
又问:“姐姐此时那里去!”宝钗道:“我到潇湘馆去。你且回去,把那当票子叫
丫头送来我那里,悄悄的取出来,晚上再悄悄的送给你去,早晚好穿。不然,风闪
着还了得!但不知当在那里了?”岫烟道:“叫做什么恒舒,是鼓楼西大街的。”
宝钗笑道:“这闹在一家去了。伙计们倘或知道了,好说‘人没过来,衣裳先来了’。”
岫烟听说,便知是他家的本钱,也不答言,红了脸,一笑走开。

宝钗也就往潇湘馆来。恰正值他母亲也来瞧黛玉,正说闲话呢。宝钗笑道:“妈
妈多早晚来的?我竟不知道。”薛姨妈道:“我这几日忙,总没来瞧瞧宝玉和他,
所以今日瞧他两人。都也好了。”黛玉忙让宝钗坐下,因向宝钗道:“天下的事,
真是人想不到的。拿着姨妈和大舅母说起,怎么又作一门亲家!”薛姨妈道:“我
的儿,你们女孩儿家那里知道?自古道:‘千里姻缘一线牵’。管姻缘的有一位月
下老儿,预先注定,暗里只用一根红丝,把这两个人的脚绊住。凭你两家那怕隔着
海呢,若有姻缘的,终久有机会作成了夫妇。这一件事,都是出人意料之外。凭父
母本人都愿意了,或是年年在一处,已为是定了的亲事,若是月下老人不用红线拴
的,再不能到一处。比如你姐妹两个的婚姻,此刻也不知在眼前,也不知在山南海
北呢!”宝钗道:“惟有妈妈说动话拉上我们!”一面说,一面伏在母亲怀里,笑
说:“咱们走罢。”黛玉笑道:“你瞧瞧!这么大了,离了姨妈,他就是个最老道
的,见了姨妈他就撒娇儿。”薛姨妈将手摩弄着宝钗,向黛玉叹道:“你这姐姐,
就和凤哥儿在老太太跟前一样,着了正经事,就有话和他商量;没有了事,幸亏他
开我的心。我见了他这样,有多少愁不散的?”

黛玉听说,流泪叹道:“他偏在这里这样,分明是气我没娘的人,故意来形容
我。”宝钗笑道:“妈妈,你瞧他这轻狂样儿,倒说我撒娇儿!”薛姨妈道:“也
怨不得他伤心,可怜没父母,到底没个亲人。”又摩挲着黛玉,笑道:“好孩子,
别哭。你见我疼你姐姐,你伤心,不知我心里更疼你呢。你姐姐虽没父亲,到底有
我,有亲哥哥,这就比你强了。我常和你姐姐说,心里很疼你,只是外头不好带出
来。他们这里人多嘴杂,说好话的人少,说歹话的人多:不说你无依靠,为人做人
配人疼;只说我们看着老太太疼你,我们也‘上水’去了。”黛玉笑道:“姨妈
既这么说,我明日就认姨妈做娘。姨妈若是弃嫌,就是假意疼我。”薛姨妈道:“你
不厌我,就认了。”宝钗忙道:“认不得的。”黛玉道:“怎么认不得?”宝钗笑
道:“我且问你:我哥哥还没定亲事,为什么反将邢妹妹先说给我兄弟了?是什么
道理?”黛玉道:“他不在家,或是属相生日不对,所以先说与兄弟了。”宝钗笑
道:“不是这样。我哥哥已经相准了,只等来家才放定,也不必提出人来。我说你
认不得娘的,——细想去!”说着,便和他母亲挤眼儿发笑。黛玉听了,便一头伏
在薛姨妈身上,说道:“姨妈不打他,我不依!”薛姨妈搂着他笑道:“你别信你
姐姐的话,他是和你玩呢。”宝钗笑道:“真个妈妈明日和老太太求了,聘作媳妇,
岂不比外头寻的好?”黛玉便拢上来要抓他,口内笑说:“你越发疯了!”

薛姨妈忙笑劝,用手分开方罢。又向宝钗道:“连邢姑娘我还怕你哥哥遭塌了
他,所以给你兄弟,别说这孩子,我也断不肯给他。前日老太太要把你妹妹说给宝
玉,偏生又有了人家;不然,倒是门子好亲事。前日我说定了邢姑娘,老太太还取
笑说:‘我原要说他的人,谁知他的人没到手,倒被他说了我们一个去了!’虽是
玩话,细想来倒也有些意思。我想宝琴虽有了人家,我虽无人可给,难道一句话也
没说?我想你宝兄弟,老太太那样疼他,你又生得那样,若要外头说去,老太太断
不中意。不如把你林妹妹定给他,岂不四角俱全?”黛玉先还怔怔的听,后来见说
到自己身上,便啐了宝钗一口,红了脸,拉着宝钗笑道:“我只打你!为什么招出
姨妈这些老没正经的话来?”宝钗笑道:“这可奇了。妈妈说你,为什么打我?”
紫鹃忙跑来笑道:“姨太太既有这主意,为什么不和老太太说去?”薛姨妈笑道:
“这孩子急什么!想必催着姑娘出了阁,你也要早些寻一个小女婿子去了。”紫鹃
飞红了脸,笑道:“姨太太真个倚老卖老的。”说着便转身去了。黛玉先骂:“又
与你这蹄子什么相干!”后来见了这样,也笑道:“阿弥陀佛,该该该!也臊了一
鼻子灰去了。”薛姨妈母女及婆子丫鬟都笑起来。

一语未了,忽见湘云走来,手里拿着一张当票,口内笑道:“这是什么帐篇子?”
黛玉瞧了不认得。地下婆子都笑道:“这可是一件好东西!这个乖不是白教的。”
宝钗忙一把接了看时,正是岫烟才说的当票子,忙着折起来。薛姨妈忙说:“那必
是那个妈妈的当票子失落了,回来急的他们找。那里得的?”湘云道:“什么是‘当
票子’?”众婆子笑道:“真真是位呆姑娘,连当票子也不知道。”薛姨妈叹道:
“怨不得他,真真是侯门千金,而且又小,那里知道这个?那里去看这个?就是家下
人有这个,他如何得见。别笑他是呆子,若给你们家的姑娘看了,也都成了呆子呢。”
众婆子笑道:“林姑娘才也不认得。别说姑娘们,就如宝玉,倒是外头常走出去的,
只怕也还没见过呢。”薛姨妈忙将原故讲明,湘云黛玉二人听了,方笑道:“这人
也太会想钱了。姨妈家当铺也有这个么?”众人笑道:“这更奇了,‘天下老鸹一
般黑’,岂有两样的。”薛姨妈因又问:“是那里拾的?”湘云方欲说时,宝钗忙
说:“是一张死了没用的,不知是那年勾了账的。香菱拿着哄他们玩的。”薛姨妈
听了此话是真,也就不问了。

一时人来回:“那府里大奶奶过来请姨太太说话呢。”薛姨妈起身去了。这里
屋内无人时,宝钗方问湘云:“何处拾的?”湘云笑道:“我见你令弟媳的丫头篆
儿悄悄的递给莺儿,莺儿便随手夹在书里,只当我没看见。我等他们出去了,我偷
着看,竟不认得。知道你们都在这里,所以拿来大家认认。”黛玉忙问:“怎么他
也当衣裳不成?既当了,怎么又给你?”宝钗见问,不好隐瞒他两个,便将方才之
事都告诉了他二人。黛玉听了,“兔死狐悲,物伤其类”,不免也要感叹起来了。
湘云听了却动了气,说道:“等我问着二姐姐去!我骂那起老婆子丫头一顿,给你
们出气何如?”说着便要走出去。宝钗忙一把拉住,笑道:“你又发疯了,还不给
我坐下呢。”黛玉笑道:“你要是个男人,出去打一个抱不平儿;你又充什么荆轲、
聂政?真真好笑。”湘云道:“既不叫问他去,明日索性把他接到咱们院里一处住
去,岂不是好?”宝钗笑道:“明日再商量。”说着,人报:“三姑娘、四姑娘来
了。”三人听说,忙掩了口,不提此事。

要知端详,且听下回分解。