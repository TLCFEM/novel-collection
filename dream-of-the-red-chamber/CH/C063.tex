\chapter{寿怡红群芳开夜宴~死金丹独艳理亲丧}

话说宝玉回至房中洗手,因和袭人商议:“晚间吃酒,大家取乐,不可拘泥。
如今吃什么好?早说给他们备办去。”袭人笑道:“你放心,我和晴雯、麝月、秋
纹四个人,每人五钱银子,共是二两;芳官、碧痕、春燕、四儿四个人,每人三钱
银子,他们告假的不算:共是三两二钱银子,早已交给了柳嫂子,预备四十碟果子。
我和平儿说了,已经抬了一罐好绍兴酒藏在那边了。我们八个人单替你做生日。”
宝玉听了,喜的忙说:“他们是那里的钱?不该叫他们出才是。”晴雯道:“他们
没钱,难道我们是有钱的?这原是各人的心。那怕他偷的呢,只管领他的情就是了。”
宝玉听了,笑说:“你说的是。”袭人笑道:“你这个人,一天不捱他两句硬话村
你,你再过不去。”晴雯笑道:“你如今也学坏了,专会调三窝四。”说着,大家
都笑了。宝玉说:“关了院门罢。”袭人笑道:“怪不得人说你是‘无事忙’!这
会子关了门,人倒疑惑起来,索性再等一等。”宝玉点头,因说:“我出去走走。
四儿舀水去,春燕一个跟我来罢。”说着,走至外边,因见无人,便问五儿之事。
春燕道:“我才告诉了柳嫂子,他倒很喜欢。只是五儿那一夜受了委屈烦恼,回去
又气病了,那里来得?只等好了罢。”宝玉听了,未免后悔长叹,因又问:“这事
袭人知道不知道?”春燕道:“我没告诉,不知芳官可说了没有。”宝玉道:“我
却没告诉过他。也罢,等我告诉他就是了。”说毕,复走进来,故意洗手。

已是掌灯时分,听得院门前有一群人进来。大家隔窗悄视,果见林之孝家的和
几个管事的女人走来,前头一人提着大灯笼。晴雯悄笑道:“他们查上夜的人来了。
这一出去,咱们就好关门了。”只见怡红院凡上夜的人,都迎出去了。林之孝家的
看了不少,又吩咐:“别耍钱吃酒,放倒头睡到大天亮。我听见是不依的。”众人
都笑说:“那里有这么大胆子的人。”林之孝家的又问:“宝二爷睡下了没有?”
众人都回:“不知道。”袭人忙推宝玉。宝玉了鞋,便迎出来,笑道:“我还没
睡呢。妈妈进来歇歇。”又叫:“袭人,倒茶来。”林之孝家的忙进来,笑说:“还
没睡呢?如今天长夜短,该早些睡了,明日方起的早。不然,到了明日起迟了,人
家笑话,不是个读书上学的公子了,倒像那起挑脚汉了。”说毕,又笑。宝玉忙笑
道:“妈妈说的是。我每日都睡的早,妈妈每日进来,可都是我不知道的,已经睡
了。今日因吃了面,怕停食,所以多玩一回。”林之孝家的又向袭人等笑说:“该
些普洱茶喝。”袭人晴雯二人忙说:“了一茶缸子女儿茶,已经喝过两碗了。
大娘也尝一碗,都是现成的。”说着,晴雯便倒了来。林家的站起接了,又笑道:
“这些时,我听见二爷嘴里都换了字眼,赶着这几位大姑娘们竟叫起名字来。虽然
在这屋里,到底是老太太、太太的人,还该嘴里尊重些才是。若一时半刻偶然叫一
声使得;若只管顺口叫起来,怕以后兄弟侄儿照样,就惹人笑话这家子的人眼里没
有长辈了。”宝玉笑道:“妈妈说的是。我不过是一时半刻偶然叫一句是有的。”
袭人晴雯都笑说:“这可别委屈了他,直到如今,他可‘姐姐’没离了嘴。不过玩
的时候叫一声半声名字,若当着人,却是和先一样。”林之孝家的笑道:“这才好
呢,这才是读书知礼的。越自己谦逊,越尊重。别说是三五代的陈人、现从老太太、
太太屋里拨过来的,就是老太太、太太屋里的猫儿狗儿,轻易也伤不得他。这才是
受过调教的公子行事。”说毕,吃了茶,便说:“请安歇罢,我们走了。”宝玉还
说:“再歇歇。”那林之孝家的已带了众人又查别处去了。

这里晴雯等忙命关了门,进来笑说:“这位奶奶那里吃了一杯来了?唠三叨四
的,又排场了我们一顿去了。”麝月笑道:“他也不是好意的?少不得也要常提着
些儿,也堤防着,怕走了大褶儿的意思。”说着,一面摆上酒果。袭人道:“不用
高桌,咱们把那张花梨圆炕桌子放在炕上坐,又宽绰,又便宜。”说着,大家果然
抬来。麝月和四儿那边去搬果子,用两个大茶盘,做四五次方搬运了来。两个老婆
子蹲在外面火盆上筛酒。宝玉说:“天热,咱们都脱了大衣裳才好。”众人笑道:
“你要脱,你脱,我们还要轮流安席呢。”宝玉笑道:“这一安席,就要到五更天
了。知道我最怕这些俗套,在外人跟前,不得已的。这会子还怄我,就不好了。”
众人听了,都说:“依你。”

于是先不上坐,且忙着卸妆宽衣。一时将正妆卸去,头上只随便挽着儿,身
上皆是紧身袄儿。宝玉只穿着大红绵纱小袄儿,下面绿绫弹墨夹裤,散着裤脚,系
着一条汗巾,靠着一个各色玫瑰芍药花瓣装的玉色夹纱新枕头,和芳官两个先
拳。当时芳官满口嚷热,只穿着一件玉色红青驼绒三色缎子拼的水田小夹袄,束着
一条柳绿汗巾,底下是水红洒花夹裤,也散着裤腿。头上齐额编着一圈小辫,总归
至顶心,结一根粗辫,拖在脑后,右耳根内只塞着米粒大小的一个小玉塞子,左耳
上单一个白果大小的硬红镶金大坠子,越显得面如满月犹白,眼似秋水还清。引得
众人笑说:“他两个倒像一对双生的弟兄。”袭人等一一斟上酒来,说:“且等一
等再拳。虽不安席,在我们每人手里吃一口罢了。”于是袭人为先,端在唇上吃
了一口,其馀依次下去,一一吃过,大家方团圆坐了。春燕四儿因炕沿坐不下,便
端了两个绒套绣墩近炕沿放下。那四十个碟子,皆是一色白彩定窑的,不过小茶碟
大,里面自是山南海北干鲜水陆的酒馔果菜。

宝玉因说:“咱们也该行个令才好。”袭人道:“斯文些才好,别大呼小叫,
叫人听见。二则我们不识字,可不要那些文的。”麝月笑道:“拿骰子咱们抢红罢。”
宝玉道:“没趣,不好。咱们占花名儿好。”晴雯笑道:“正是,早已想弄这个玩
意儿。”袭人道:“这个玩意虽好,人少了没趣。”春燕笑道:“依我说,咱们竟
悄悄的把宝姑娘、云姑娘、林姑娘请了来,玩一会子,到二更天再睡不迟。”袭人
道:“又开门合户的闹,倘或遇见巡夜的问?”宝玉道:“怕什么!咱们三姑娘也
吃酒,再请他一声才好。还有琴姑娘。”众人都道:“琴姑娘罢了,他在大奶奶屋
里,叨登的大发了。”宝玉道:“怕什么,你们就快请去。”春燕四儿都巴不得一
声,二人忙命开门,各带小丫头分头去请。

晴雯、麝月、袭人三人又说:“他两个去请,只怕不肯来,须得我们去请,死
活拉了来。”于是袭人晴雯忙又命老婆子打个灯笼,二人又去。果然宝钗说夜深了,
黛玉说身上不好。他二人再三央求:“好歹给我们一点体面,略坐坐再来。”众人
听了,却也欢喜。因想不请李纨,倘或被他知道了倒不好,便命翠墨同春燕也再三
的请了李纨和宝琴二人,会齐先后都到了怡红院中。袭人又死活拉了香菱来。炕上
又并了一张桌子,方坐开了。宝玉忙说:“林妹妹怕冷,过这边靠板壁坐。”又拿
了个靠背垫着些。袭人等都端了椅子在炕沿下陪着。黛玉却离桌远远的靠着靠背,
因笑向宝钗、李纨、探春等道:“你们日日说人家夜饮聚赌,今日我们自己也如此。
以后怎么说人?”李纨笑道:“有何妨碍?一年之中不过生日节间如此,并没夜夜
如此,这倒也不怕。”

说着,晴雯拿了一个竹雕的签筒来,里面装着象牙花名签子,摇了一摇,放在
当中。又取过骰子来,盛在盒内,摇了一摇,揭开一看,里面是六点,数至宝钗。
宝钗便笑道:“我先抓,不知抓出个什么来。”说着将筒摇了一摇,伸手掣出一签。
大家一看,只见签上画着一枝牡丹,题着“艳冠群芳”四字。下面又有镌的小字,
一句唐诗,道是:
任是无情也动人。
又注着:“在席共贺一杯。此为群芳之冠,随意命人,不拘诗词雅谑,或新曲一支
为贺。”众人都笑说:“巧得很!你也原配牡丹花。”说着大家共贺了一杯。宝钗
吃过,便笑说:“芳官唱一只我们听罢。”芳官道:“既这样,大家吃了门杯好听。”
于是大家吃酒,芳官便唱:“寿筵开处风光好……”众人都道:“快打回去!这会
子很不用你来上寿。拣你极好的唱来。”芳官只得细细的唱了一只《赏花时》“翠
凤翎毛扎帚,闲踏天门扫落花……”才罢。宝玉却只管拿着那签,口内颠来倒去
念“任是无情也动人”,听了这曲子,眼看着芳官不语。湘云忙一手夺了,撂与宝
钗。

宝钗又掷了一个十六点,数到探春。探春笑道:“还不知得个什么。”伸手掣
了一根出来,自己一瞧,便撂在桌上,红了脸笑道:“很不该行这个令!这原是外
头男人们行的令,许多混帐话在上头。”众人不解,袭人等忙拾起来。众人看时,
上面一枝杏花,那红字写着“瑶池仙品”四字,诗云:
日边红杏倚云栽。
注云:“得此签者,必得贵婿,大家恭贺一杯,再同饮一杯。”众人笑说道:“我
们说是什么呢,这签原是闺阁中取笑的,除了这两三根有这话的,并无杂话。这有
何妨?我们家已有了王妃,难道你也是王妃不成?大喜,大喜!”说着大家来敬探春。
探春那里肯饮,却被湘云、香菱、李纨等三四个人,强死强活,灌了一钟才罢。

探春只叫:“蠲了这个,再行别的。”众人断不肯依。湘云拿着他的手,强掷
了个十九点出来,便该李氏掣。李氏摇了一摇,掣出一根来一看,笑道:“好极!
你们瞧瞧这行子,竟有些意思。”众人瞧那签上,画着一枝老梅,写着“霜晓寒姿”
四字,那一面旧诗是:
竹篱茅舍自甘心。
注云:“自饮一杯,下家掷骰。”李纨笑道:“真有趣,你们掷去罢,我只自吃一
杯,不问你们的废兴。”说着便吃酒,将骰过给黛玉。

黛玉一掷是十八点,便该湘云掣。湘云笑着,揎拳掳袖的,伸手掣了一根出来。
大家看时,一面画着一枝海棠,题着“香梦酣”四字。那面诗道是:
只恐夜深花睡去。
黛玉笑道:“‘夜深’二字改‘石凉’两个字倒好。”众人知他打趣日间湘云醉眠
的事,都笑了。湘云笑指那自行船给黛玉看,又说:“快坐上那船家去罢,别多说
了。”众人都笑了。因看注云:“既云香梦酣,掣此签者,不便饮酒,只令上下
两家各饮一杯。”湘云拍手笑道:“阿弥陀佛,真真好签!”恰好黛玉是上家,宝
玉是下家,二人斟了两杯,只得要饮。宝玉先饮了半杯,瞅人不见,递与芳官。芳
官即便端起来,一仰脖喝了。黛玉只管和人说话,将酒全折在漱盂内了。

湘云便抓起骰子来,一掷个九点,数去该麝月。麝月便掣了一根出来,大家看
时,上面是一枝荼花,题着“韶华胜极”四字,那边写着一句旧诗,道是:
开到荼花事了。
注云:“在席各饮三杯送春。”麝月问:“怎么讲?”宝玉皱皱眉儿,忙将签藏了,
说:“咱们且喝酒罢。”说着,大家吃了三口,以充三杯之数。

麝月一掷个十点,该香菱。香菱便掣了一根并蒂花,题着“联春绕瑞”,那面
写着一句旧诗,道是:
连理枝头花正开。
注云:“共贺掣者三杯,大家陪饮一杯。”

香菱便又掷了个六点,该黛玉。黛玉默默的想道:“不知还有什么好的被我掣
着方好。”一面伸手取了一根。只见上面画着一枝芙蓉花,题着“风露清愁”四字,
那面一句旧诗,道是:
莫怨东风当自嗟。
注云:“自饮一杯,牡丹陪饮一杯。”众人笑说:“这个好极,除了他,别人不配
做芙蓉。”黛玉也自笑了。

于是饮了酒,便掷了个二十点,该着袭人。袭人便伸手取了一枝出来,却是一
枝桃花,题着“武陵别景”四字,那一面写着旧诗,道是:
桃红又见一年春。
注云:“杏花陪一盏,坐中同庚者陪一盏,同姓者陪一盏。”众人笑道:“这一回
热闹有趣。”大家算来:香菱、晴雯、宝钗三人皆与他同庚,黛玉与他同辰,只无
同姓者。芳官忙道:“我也姓花,我也陪他一钟。”于是大家斟了酒。黛玉因向探
春笑道:“命中该招贵婿的!你是杏花,快喝了,我们好喝。”探春笑道:“这是
什么话?大嫂子顺手给他一巴掌!”李纨笑道:“人家不得贵婿,反捱打,我也不
忍得。”众人都笑了。

袭人才要掷,只听有人叫门,老婆子忙出去问时,原来是薛姨妈打发人来了接
黛玉的。众人因问:“几更了?”人回:“二更以后了,钟打过十一下了。”宝玉
犹不信,要过表来瞧了一瞧,已是子初一刻十分了,黛玉便起身说:“我可掌不住
了,回去还要吃药呢。”众人说:“也都该散了。”袭人宝玉等还要留着众人,李
纨探春等都说:“夜太深了不像,这已是破格了。”袭人道:“既如此,每位再吃
一杯再走。”说着,晴雯等已都斟满了酒。每人吃了,都命点灯。袭人等齐送过沁
芳亭河那边,方回来。

关了门,大家复又行起令来。袭人等又用大钟斟了几钟,用盘子攒了各样果菜
与地下的老妈妈们吃。彼此有了三分酒,便拳赢唱小曲儿。那天已四更时分,老
妈妈们一面明吃,一面暗偷,酒缸已罄,众人听了,方收拾盥漱睡觉。芳官吃得两
腮胭脂一般,眉梢眼角,添了许多丰韵,身子图不得,便睡在袭人身上,说:“姐
姐,我心跳的很。”袭人笑道:“谁叫你尽力灌呢。”春燕四儿也图不得,早睡了,
晴雯还只管叫。宝玉道:“不用叫了,咱们且胡乱歇一歇。”自己便枕了那红香枕,
身子一歪,就睡着了。袭人见芳官醉的很,恐闹他吐酒,只得轻轻起来,就将芳官
扶在宝玉之侧,由他睡了。自己却在对面榻上倒下。

大家黑甜一觉,不知所之。及至天明,袭人睁眼一看,只见天色晶明,忙说:
“可迟了!”向对面床上瞧了一瞧,只见芳官头枕着炕沿上,睡犹未醒,连忙起来
叫他。宝玉已翻身醒了。笑道:“可迟了。”因又推芳官起身。那芳官坐起来,犹
发怔揉眼睛。袭人笑道:“不害羞,你喝醉了,怎么也不拣地方儿,乱挺下了?”
芳官听了,瞧了瞧,方知是和宝玉同榻,忙羞的笑着下地说:“我怎么——”却说
不出下半句来。宝玉笑道:“我竟也不知道了。若知道,给你脸上抹些墨。”说着,
丫头进来,伺候梳洗。宝玉笑道:“昨日有扰,今日晚上我还席。”袭人笑道:“罢
罢,今日可别闹了,再闹就有人说话了。”宝玉道:“怕什么,不过才两次罢了。
咱们也算会吃酒了,一坛子酒怎么就吃光了。——正在有趣儿,偏又没了。”袭人
笑道:“原要这么着才有趣儿,必尽了兴,反无味。昨日都好上来了,晴雯连臊也
忘了,我记得他还唱了一个曲儿。”四儿笑道:“姐姐忘了,连姐姐还唱了一个呢!
在席的谁没唱过?”众人听了,俱红了脸,用两手握着,笑个不住。

忽见平儿笑嘻嘻的走来,说:“我亲自来请昨日在席的人,今日我还东,短一
个也使不得。”众人忙让坐吃茶。晴雯笑道:“可惜昨夜没他。”平儿忙问:“你
们夜里做什么来?”袭人便说:“告诉不得你!昨日夜里热闹非常,连往日老太太、
太太带着众人玩,也不及昨儿这一玩:一坛酒我们都鼓捣光了。一个个喝的把臊都
丢了,又都唱起来。四更多天,才横三竖四的打了一个盹儿。”平儿笑道:“好,
白和我要了酒来,也不请我。还说着给我听,气我。”晴雯道:“今儿他还席,必
自来请你,你等着罢。”平儿笑问道:“‘他’是谁?谁是‘他’?”晴雯听了,
把脸飞红了,赶着打,笑说道:“偏你这耳朵尖,听的真!”平儿笑道:“呸!不
害臊的丫头!这会子有事,不和你说。我有事,去了回来再打发人来请。一个不到,
我是打上门来的。”宝玉等忙留他,已经去了。

这里宝玉梳洗了,正喝茶,忽然一眼看见砚台底下压着一张纸,因说道:“你
们这么随便混压东西,也不好。”袭人晴雯等忙问:“又怎么了?谁又有了不是了?”
宝玉指道:“砚台下是什么?一定又是那位的样子,忘记收的。”晴雯忙启砚拿了
出来,却是一张字帖儿。递给宝玉看时,原来是一张粉红笺纸,上面写着:“槛外
人妙玉恭肃遥叩芳辰。”宝玉看毕,直跳了起来,忙问:“是谁接了来的?也不告
诉!”袭人晴雯等见了这般,不知当是那个要紧的人来的帖子,忙一齐问:“昨儿
是谁接下了一个帖子?”四儿忙跑进来,笑说:“昨儿妙玉并没亲来,只打发个妈
妈送来。我就搁在这里,谁知一顿酒喝的就忘了。”众人听了道:“我当是谁,大
惊小怪,这也不值的。”宝玉忙命:“快拿纸来。”当下拿了纸,研了墨,看他下
着“槛外人”三字,自己竟不知回帖上回个什么字样才相敌,只管提笔出神,半天
仍没主意。因又想:“要问宝钗去,他必又批评怪诞,不如问黛玉去。”想罢,袖
了帖儿,径来寻黛玉。

刚过了沁芳亭,忽见岫烟颤颤巍巍的迎面走来。宝玉忙问:“姐姐那里去?”
岫烟笑道:“我找妙玉说话。”宝玉听了,诧异说道:“他为人孤癖,不合时宜,
万人不入他的目。原来他推重姐姐,竟知姐姐不是我们一流俗人。”岫烟笑道:“他
也未必真心重我,但我和他做过十年的邻居,只一墙之隔。他在蟠香寺修炼,我家
原来寒素,赁房居就,赁了他庙里的房子住了十年。无事到他庙里去作伴,我所认
得的字,都是承他所授:我和他又是贫贱之交,又有半师之分。因我们投亲去了,
闻得他因不合时宜,权势不容,竟投到这里来。如今又两缘凑合,我们得遇,旧情
竟未改易,承他青目,更胜当日。”宝玉听了,恍如听了焦雷一般,喜得笑道:“怪
道姐姐举止言谈,超然如野鹤闲云,原本有来历。我正因他的一件事为难,要请教
别人去。如今遇见姐姐,真是天缘凑合,求姐姐指教。”说着便将拜帖取给岫烟看。
岫烟笑道:“他这脾气竟不能改,竟是生成这等放诞诡僻了。从来没见拜帖上下别
号的,这可是俗语说的‘僧不僧,俗不俗,女不女,男不男’,成个什么理数。”
宝玉听说,忙笑道:“姐姐不知道,他原不在这些人中里,他原是世人意外之人。
因取了我是个些微有知识的,方给我这帖子。我因不知回什么字样才好,竟没了主
意,正要去问林妹妹,可巧遇见了姐姐。”

岫烟听了宝玉这话,且只管用眼上下细细打量了半日,方笑道:“怪道俗语说
的,‘闻名不如见面’,又怪不的妙玉竟下这帖子给你,又怪不的上年竟给你那些
梅花。既连他这样,少不得我告诉你原故。他常说古人中自汉、晋、五代、唐、宋
以来,皆无好诗,只有两句好,说道:‘纵有千年铁门槛,终须一个土馒头。’所
以他自称‘槛外之人’。又常赞:‘文是庄子的好。’故又或称为‘畸人’。他若
帖子上是自称‘畸人’的,你就还他个‘世人’。‘畸人’者,他自称是畸零之人,
你谦自己乃世人扰扰之人,他便喜了。如今他自称‘槛外之人’,是自谓蹈于铁槛
之外了,故你如今只下‘槛内人’,便合了他的心了。”宝玉听了,如醍醐灌顶,
“嗳哟”了一声,方笑道:“怪道我们家庙说是铁槛寺呢,原来有这一说。姐姐就
请,让我去写回帖。”岫烟听了,便自往栊翠庵来。宝玉回房,写了帖子,上面只
写“槛内人宝玉熏沐谨拜”几字。亲自拿了到栊翠庵,只隔门缝儿投进去,便回来
了。

因饭后平儿还席,说红香圃太热,便在榆荫堂中摆了几席新酒佳肴。可喜尤氏
又带了佩凤偕鸾二妾过来游玩。这二妾亦是青年娇憨女子,不常过来的,今既入了
这园,再遇见湘云、香菱、芳、蕊一干女子,所谓“方以类聚,物以群分”二语不
错,只见他们说笑不了,也不管尤氏在那里,只凭丫鬟们去服役,且同众人一一的
游玩。

闲言少述,且说当下众人都在榆荫堂中,以酒为名,大家玩笑,命女先儿击鼓。
平儿采了一枝芍药,大家约二十来人,传花为令,热闹了一回。因人回说:“甄家
有两个女人送东西来了。”探春和李纨尤氏三人出去议事厅相见。这里众人且出来
散一散。佩凤偕鸾两个去打秋千玩耍,宝玉便说:“你两个上去,让我送。”慌的
佩凤说:“罢了,别替我们闹乱子!”

忽见东府里几个人,慌慌张张跑来,说:“老爷殡天了!”众人听了,吓了一
大跳,忙都说:“好好的并无疾病,怎么就没了?”家人说:“老爷天天修炼,定
是功成圆满,升仙去了。”尤氏一闻此言,又见贾珍父子并贾琏等皆不在家,一时
竟没个着己的男子来,未免忙了。只得忙卸了妆饰,命人先到玄真观将所有的道士
都锁了起来,等大爷来家审问;一面忙忙坐车,带了赖升一干老人媳妇出城。又请
大夫看视,到底系何病症。大夫们见人已死,何处诊脉来?素知贾敬导气之术,总
属虚诞,更至参星礼斗,守庚申,服灵砂等,妄作虚为,过于劳神费力,反因此伤
了性命的,如今虽死,腹中坚硬似铁,面皮嘴唇,烧的紫绛皱裂。便向媳妇回说:
“系道教中吞金服砂,烧胀而殁。”众道士慌的回道:“原是秘制的丹砂吃坏了事,
小道们也曾劝说:‘功夫未到,且服不得。’不承望老爷于今夜守庚申时,悄悄的
服了下去,便升仙去了。这是虔心得道,已出苦海,脱去皮囊了。”尤氏也不便听,
只命锁着,等贾珍来发放,且命人飞马报信。一面看视里面窄狭,不能停放,横竖
也不能进城的,忙装裹好了,用软轿抬至铁槛寺来停放。掐指算来,至早也得半月
的工夫贾珍方能来到,目今天气炎热,实不能相待,遂自行主持,命天文生择了日
期入殓。寿木早年已经备下,寄在此庙的,甚是便宜。三日后,便破孝开吊,一面
且做起道场来。因那边荣府里凤姐儿出不来,李纨又照顾姐妹,宝玉不识事体,只
得将外头事务,暂托了几个家里二等管事的。贾、贾、贾珩、贾璎、贾菖、贾
菱等各有执事。尤氏不能回家,便将他继母接来,在宁府看家。这继母只得将两个
未出嫁的女儿带来,一并住着,才放心。

且说贾珍闻了此信,急忙告假,——并贾蓉是有职人员。礼部见当今隆敦孝弟,
不敢自专,具本请旨。原来天子极是仁孝过天的,且更隆重功臣之裔,一见此本,
便诏问贾敬何职。礼部代奏:“系进士出身,祖职已荫其子贾珍。贾敬因年迈多疾,
常养静于都城之外玄真观,今因疾殁于观中。其子珍,其孙蓉,现因国丧,随驾在
此,故乞假归殓。”天子听了,忙下额外恩旨曰:“贾敬虽无功于国,念彼祖父之
忠,追赐五品之职。令其子孙扶柩由北下门入都,恩赐私第殡殓,任子孙尽丧,礼
毕扶柩回籍。外着光禄寺按上例赐祭,朝中由王公以下,准其祭吊。钦此。”此旨
一下,不但贾府里人谢恩,连朝中所有大臣,皆嵩呼称颂不绝。

贾珍父子星夜驰回。半路中又见贾贾二人,领家丁飞骑而来,看见贾珍,
一齐滚鞍下马请安。贾珍忙问:“做什么?”贾回说:“嫂子恐哥哥和侄儿来了,
老太太路上无人,叫我们两个来护送老太太的。”贾珍听了,赞声不绝。又问:“家
中如何料理?”贾等便将如何拿了道士,如何挪至家庙,怕家内无人,接了亲家
母和两个姨奶奶在上房住着。贾蓉当下也下了马,听见两个姨娘来了,喜的笑容满
面。贾珍忙说了几声“妥当”,加鞭便走。店也不投,连夜换马飞驰。一日到了都
门,先奔入铁槛寺,那天已是四更天气。坐更的闻知,忙喝起众人来。贾珍下了马,
和贾蓉放声大哭,从大门外便跪爬起来,至棺前稽颡泣血,直哭到天亮,喉咙都哭
哑了方住。尤氏等都一齐见过,贾珍父子忙按礼换了凶服,在棺前俯伏。无奈自要
理事,竟不能目不视物、耳不闻声,少不得减了些悲戚,好指挥众人。因将恩旨备
述给众亲友听了,一面先打发贾蓉回家来,料理停灵之事。

贾蓉巴不得一声儿,便先骑马跑来。到家,忙命前厅收桌椅,下扇,挂孝幔
子,门前起鼓手棚、牌楼等事。又忙着进来看外祖母、两个姨娘。原来尤老安人年
高喜睡,常常歪着;他二姨娘三姨娘都和丫头们做活计,见他来了,都道烦恼。贾
蓉且嘻嘻的望他二姨娘笑说:“二姨娘,你又来了?我父亲正想你呢。”二姨娘红
了脸,骂道:“好蓉小子!我过两日不骂你几句,你就过不得了,越发连个体统都
没了。还亏你是大家公子哥儿,每日念书学礼的,越发连那小家子的也跟不上。”
说着顺手拿起一个熨斗来,兜头就打,吓得贾蓉抱着头,滚到怀里告饶。尤三姐便
转过脸去,说道:“等姐姐来家再告诉他。”贾蓉忙笑着跪在炕上求饶,因又和他
二姨娘抢砂仁吃。那二姐儿嚼了一嘴渣子,吐了他一脸,贾蓉用舌头都舔着吃了。
众丫头看不过,都笑说:“热孝在身上,老娘才睡了觉。他两个虽小,到底是姨娘
家。你太眼里没有奶奶了,回来告诉爷,你吃不了兜着走。”贾蓉撇下他姨娘,便
抱着那丫头亲嘴,说:“我的心肝,你说得是。咱们馋他们两个。”丫头们忙推他,
恨的骂:“短命鬼!你一般有老婆丫头,只和我们闹。知道的说是玩,不知道的人,
再遇见那样脏心烂肺的、爱多管闲事嚼舌头的人,吵嚷到那府里,背地嚼舌,说咱
们这边混账。”贾蓉笑道:“各门另户,谁管谁的事?都够使的了。从古至今,连
汉朝和唐朝,人还说‘脏唐臭汉’,何况咱们这宗人家!谁家没风流事?别叫我说出
来。连那边大老爷这么利害,琏二叔还和那小姨娘不干净呢。凤婶子那样刚强,瑞
大叔还想他的账:那一件瞒了我?”

贾蓉只管信口开河,胡言乱道。三姐儿了脸,早下炕进里间屋里,叫醒尤老
娘。这里贾蓉见他老娘醒了,忙去请安问好。又说:“老祖宗劳心,又难为两位姨
娘受委屈,我们爷儿们感激不尽。惟有等事完了,我们合家大小登门磕头去。”尤
老安人点头道:“我的儿,倒是你会说话。亲戚们原是该的。”又问:“你父亲好?
几时得了信赶到的?”贾蓉笑道:“刚才赶到的,先打发我瞧你老人家来了,好歹
求你老人家事完了再去。”说着,又和他二姨娘挤眼儿。二姐便悄悄咬牙骂道:“很
会嚼舌根的猴儿崽子!留下我们,给你爹做妈不成?”贾蓉又和尤老娘道:“放心
罢,我父亲每日为两位姨娘操心,要寻两个有根基的富贵人家,又年轻又俏皮两位
姨娘父亲,好聘嫁这二位姨娘。这几年总没拣着,可巧前儿路上才相准了一个。”
尤老娘只当是真话,忙问:“是谁家的?”二姐丢了活计,一头笑,一头赶着打,
说:“妈妈,别信这混账孩子的话。”三姐儿道:“蓉儿,你说是说,别只管嘴里
这么不清不浑的!”说着,人来回话,说:“事已完了,请哥儿出去看了,回爷的
话去呢。”那贾蓉方笑嘻嘻的出来。

不知如何,下回分解。