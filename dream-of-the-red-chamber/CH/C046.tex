\chapter{尴尬人难免尴尬事~鸳鸯女誓绝鸳鸯偶}

话说黛玉直到四更将阑,方渐渐的睡去,暂且无话。

如今且说凤姐儿因见邢夫人叫他,不知何事,忙另穿戴了一番,坐车过来。邢
夫人将房内人遣出,悄悄向凤姐儿道:“叫你来不为别的,有一件为难的事,老爷
托我,我不得主意,先和你商议。老爷因看上了老太太屋里的鸳鸯,要他在房里,
叫我和老太太讨去。我想这倒是常有的事,就怕老太太不给。你可有法子办这件事
么?”凤姐儿听了,忙陪笑道:“依我说,竟别碰这个钉子去。老太太离了鸳鸯,
饭也吃不下去,那里就舍得了?况且平日说起闲话来,老太太常说老爷:‘如今上
了年纪,做什么左一个右一个的放在屋里。头宗耽误了人家的女孩儿,二则放着身
子不保养,官儿也不好生做,成日和小老婆喝酒。’太太听听,很喜欢咱们老爷么?
这会子躲还怕躲不及,这不是‘拿草棍儿戳老虎的鼻子眼儿去’吗?太太别恼:我
是不敢去的。明放着不中用,而且反招出没意思来。老爷如今上了年纪,行事不免
有点儿背晦,太太劝劝才是。比不得年轻,做这些事无碍,如今兄弟、侄儿、儿子、
孙子一大群,还这么闹起来,怎么见人呢?”邢夫人冷笑道:“大家子三房四妾的
也多,偏咱们就使不得?我劝了也未必依。就是老太太心爱的丫头,这么胡子苍白
了又做了官的一个大儿子,要了做屋里人,也未必好驳回的。我叫了你来,不过商
议商议,你先派了一篇的不是!也有叫你去的理?自然是我说去。你倒说我不劝!你
还是不知老爷那性子的!劝不成,先和我闹起来。”

凤姐知道邢夫人禀性愚弱,只知奉承贾赦以自保,次则婪取财货为自得,家下
一应大小事务俱由贾赦摆布。凡出入银钱一经他的手,便克扣异常,以贾赦浪费为
名,“须得我就中俭省,方可偿补。”儿女奴仆,一人不靠,一言不听。如今又听
说如此的话,便知他又弄左性子,劝也不中用了,连忙陪笑说道:“太太这话说的
极是。我能活了多大,知道什么轻重?想来父母跟前,别说一个丫头,就是那么大
的一个活宝贝,不给老爷给谁?背地里的话,那里信的?我竟是个傻子!拿着二爷说
起,或有日得了不是,老爷太太恨的那样,恨不得立刻拿来一下子打死,及至见了
面也罢了,依旧拿着老爷太太心爱的东西赏他。如今老太太待老爷自然也是这么
着。依我说,老太太今儿喜欢,要讨,今儿就讨去。我先过去哄着老太太,等太太
过去了,我搭讪着走开,把屋子里的人我也带开,太太好和老太太说,给了更好,
不给也没妨碍,众人也不能知道。”邢夫人见他这般说,便又喜欢起来,又告诉他
道:“我的主意,先不和老太太说。老太太说不给,这事就死了。我心里想着先悄
悄的和鸳鸯说。他虽害臊,我细细的告诉了他,他要是不言语,就妥了,那时再和
老太太说。老太太虽不依,搁不住他愿意,常言‘人去不中留’,自然这就妥了。”
凤姐儿笑道:“到底是太太有智谋,这是千妥万妥。别说是鸳鸯,凭他是谁,那一
个不想巴高望上、不想出头的?放着半个主子不做,倒愿意做丫头,将来配个小子
就完了呢。”邢夫人笑道:“正是这个话了。别说鸳鸯,就是那些执事的大丫头,
谁不愿意这样呢。你先过去,别露一点风声,我吃了晚饭就过来。”

凤姐儿暗想:“鸳鸯素昔是个极有心胸气性的丫头,虽如此说,保不严他愿意
不愿意。我先过去了,太太后过去,他要依了,便没的话说;倘或不依,太太是多
疑的人,只怕疑我走了风声,叫他拿腔作势的。那时太太又见应了我的话,羞恼变
成怒,拿我出起气来倒没意思。不如同着一齐过去了,他依也罢不依也罢,就疑不
到我身上了。”想毕,因笑道:“才我临来,舅母那边送了两笼子鹌鹑,我吩咐他
们炸了,原要赶太太晚饭上送过来。我才进大门时,见小子们抬车,说太太的车拔
了缝,拿去收拾去了。不如这会子坐了我的车一齐过去倒好。”邢夫人听了,便命
人来换衣裳。凤姐忙着伏侍了一回,娘儿两个坐车过来。凤姐儿又说道:“太太过
老太太那里去,我要跟了去,老太太要问起我过来做什么,那倒不好。不如太太先
去,我脱了衣裳再来。”

邢夫人听了有理,便自往贾母处来。和贾母说了一回闲话儿,便出来,假托往
王夫人屋里去,从后屋门出去,打鸳鸯的卧房门前过。只见鸳鸯正坐在那里做针线,
见了邢夫人站起来。邢夫人笑道:“做什么呢?”一面说,一面便过来接他手内的
针线,道:“我看看你扎的花儿。”看了一看,又道:“越发好了。”遂放下针线,
又浑身打量。只见他穿着半新的藕色绫袄,青缎掐牙坎肩儿,下面水绿裙子。蜂腰
削背,鸭蛋脸,乌油头发,高高的鼻子,两边腮上微微的几点雀瘢。鸳鸯见这般看
他,自己倒不好意思起来,心里便觉诧异,因笑问道:“太太,这会子不早不晚的
过来做什么?”邢夫人使个眼色儿,跟的人退出。邢夫人便坐下,拉着鸳鸯的手,
笑道:“我特来给你道喜来的。”鸳鸯听了,心中已猜着三分,不觉红了脸,低了
头,不发一言。听邢夫人道:“你知道,老爷跟前竟没有个可靠的人,心里再要买
一个,又怕那些牙子家出来的不干不净,也不知道毛病儿,买了来三日两日,又弄
鬼掉猴的。因满府里要挑个家生女儿,又没个好的,不是模样儿不好,就是性子不
好;有了这个好处,没了那个好处。因此常冷眼选了半年,这些女孩子里头,就只
你是个尖儿:模样儿,行事做人,温柔可靠,一概是齐全的。意思要和老太太讨了
你去,收在屋里。你比不得外头新买了来的,这一进去了就开了脸,就封你作姨娘,
又体面,又尊贵。你又是个要强的人,俗语说的,‘金子还是金子换’,谁知竟叫
老爷看中了!你如今这一来,可遂了你素日心高智大的愿了,又堵一堵那些嫌你的
人的嘴。——跟了我回老太太去!”说着,拉了他的手就要走。

鸳鸯红了脸,夺手不行。邢夫人知他害臊,便又说道:“这有什么臊的?又不
用你说话,只跟着我就是了。”鸳鸯只低头不动身。邢夫人见他这般,便又说道:
“难道你还不愿意不成?若果然不愿意,可真是个傻丫头了。放着主子奶奶不做,
倒愿意做丫头!三年两年不过配上个小子,还是奴才。你跟我们去,你知道我的性
子又好,又不是那不容人的人,老爷待你们又好。过一年半载生个一男半女,你就
和我并肩了。家里的人,你要使唤谁,谁还不动?现成主子不做去,错过了机会,
后悔就迟了。”鸳鸯只管低头,仍是不语。邢夫人又道:“你这么个爽快人,怎么
又这样积粘起来?有什么不称心的地方儿,只管说,我管保你遂心如意就是了。”
鸳鸯仍不语。邢夫人又笑道:“想必你有老子娘,你自己不肯说话,怕臊,你等他
们问你呢?这也是理。等我问他们去,叫他们来问你,有话只管告诉他们。”说毕,
便往凤姐儿屋里来。

凤姐儿早换了衣裳,因屋内无人,便将此话告诉了平儿。平儿也摇头笑道:“据
我看来,未必妥当。平常我们背着人说起话来,听他那个主意,未必肯。也只说着
瞧罢了。”凤姐儿道:“太太必来这屋里商量。依了还犹可,要是不依,白讨个没
趣儿,当着你们,岂不脸上不好看。你说给他们炸些鹌鹑,再有什么配几样,预备
吃饭。你且别处逛逛去,估量着走了你再来。”平儿听说,照样传给婆子们,便逍
遥自在的园子里来。

这里鸳鸯见邢夫人去了,必到凤姐房里商议去了,还必定有人来问他,不如躲
了这里。因找了琥珀道:“老太太要问我,只说我病了,没吃早饭,往园子里逛逛
就来。”琥珀答应了。鸳鸯便往园子里来各处游玩。不想正遇见平儿。平儿见无人,
便笑道:“新姨娘来了!”鸳鸯听了,便红了脸,说道:“怪道你们串通一气来算
计我!等着我和你主子闹去就是了!”平儿见鸳鸯满脸恼意,自悔失言,便拉到枫
树底下,坐在一块石上,把方才凤姐过去回来所有的形景言词、始末原由,都告诉
了他。鸳鸯红了脸,向平儿冷笑道:“我只想咱们,好比如袭人、琥珀、素云、紫
鹃、彩霞、玉钏、麝月、翠墨,跟了史姑娘去的翠缕,死了的可人和金钏,去了的
茜雪,连上你我,这十来个人,从小儿什么话儿不说,什么事儿不做?这如今因都
大了,各自干各自的去了,我心里却仍是照旧,有话有事,并不瞒你们。这话我先
放在你心里,且别和二奶奶说:别说大老爷要我做小老婆,就是太太这会子死了,
他三媒六证的娶我去做大老婆,我也不能去!”

平儿方欲说话,只听山石背后哈哈的笑道:“好个没脸的丫头,亏你不怕牙碜!”
二人听了,不觉吃了一惊,忙起身向山后找寻,不是别人,却是袭人,笑着走出来。
问:“什么事情?也告诉告诉我。”说着,三人坐在石上。平儿又把方才的话说了,
袭人听了,说道:“这话论理不该我们说:这个大老爷,真真太下作了。略平头正
脸的,他就不能放手了。”平儿道:“你既不愿意,我教你个法儿。”鸳鸯道:“什
么法儿?”平儿笑道:“你只和老太太说,就说已经给了琏二爷了,大老爷就不好
要了。”鸳鸯啐道:“什么东西!你还说呢!前儿你主子不是这么混说?谁知应到今
儿了。”袭人笑道:“他两个都不愿意。依我说,就和老太太说,叫老太太就说把
你已经许了宝二爷了,大老爷也就死了心了。”鸳鸯又是气,又是臊,又是急,骂
道:“两个坏蹄子,再不得好死的!人家有为难的事,拿着你们当做正经人,告诉
你们与我排解排解,饶不管,你们倒替换着取笑儿。你们自以为都有了结果了,将
来都是做姨娘的!据我看来,天底下的事,未必都那么遂心如意的。你们且收着些
儿罢,别忒乐过了头儿!”

二人见他急了,忙陪笑道:“好姐姐别多心。咱们从小儿都是亲姊妹一般,不
过无人处偶然取个笑儿。你的主意告诉我们知道,也好放心。”鸳鸯道:“什么主
意!我只不去就完了。”平儿摇头道:“你不去,未必得干休。大老爷的性子你是
知道的。虽然你是老太太房里的人,此刻不敢把你怎么样,难道你跟老太太一辈子
不成?也要出去的。那时落了他的手,倒不好了。”鸳鸯冷笑道:“老太太在一日,
我一日不离这里;若是老太太归西去了,他横竖还有三年的孝呢,没个娘才死了,
他先弄小老婆的!等过了三年,知道又是怎么个光景儿呢?那时再说。纵到了至急为
难,我剪了头发做姑子去,不然,还有一死!一辈子不嫁男人,又怎么样?乐得干净
呢!”平儿袭人笑道:“真个这蹄子没了脸,越发信口儿都说出来了。”鸳鸯道:
“已经这么着,臊会子怎么样?你们不信,只管看着就是了。太太才说了,找我老
子娘去,我看他南京找去!”平儿道:“你的父母都在南京看房子,没上来,终久
也寻的着;现在还有你哥哥嫂子在这里。可惜你是这里的家生女儿,不如我们两个
只单在这里。”鸳鸯道:“家生女儿怎么样?‘牛不喝水强按头’吗?我不愿意,难
道杀我的老子娘不成!”

正说着,只见他嫂子从那边走来。袭人道:“他们当时找不着你的爹娘,一定
和你嫂子说了。”鸳鸯道:“这个娼妇,专管是个‘六国贩骆驼’的,听了这话,
他有个不奉承去的!”说话之间,已来到跟前。他嫂子笑道:“那里没有找到,姑
娘跑了这里来!你跟了我来,我和你说话。”平儿袭人都忙让坐。他嫂子只说:“姑
娘们请坐,找我们姑娘说句话。”袭人平儿都装不知道,笑说:“什么话,这么忙?
我们这里猜谜儿呢,等猜了再去罢。”鸳鸯道:“什么话?你说罢。”他嫂子笑道:
“你跟我来,到那里告诉你,横竖有好话儿。”鸳鸯道:“可是太太和你说的那话?”
他嫂子笑道:“姑娘既知道,还奈何我!快来,我细细的告诉你,可是天大的喜事!”
鸳鸯听说,立起身来,照他嫂子脸上下死劲啐了一口,指着骂道:“你快夹着你那
嘴离了这里,好多着呢!什么‘好话’?又是什么‘喜事’?怪道成日家羡慕人家
的丫头做了小老婆,一家子都仗着他横行霸道的,一家子都成了小老婆了!看的眼
热了,也把我送在火炕里去。我若得脸呢,你们外头横行霸道,自己封就了自己是
舅爷;我要不得脸败了时,你们把忘八脖子一缩,生死由我去!”一面骂,一面哭。
平儿袭人拦着劝他。

他嫂子脸上下不来,因说道:“愿意不愿意你也好说,犯不着拉三扯四的。俗
语说的好:‘当着矮人,别说矮话。’姑娘骂我,我不敢还言;这二位姑娘并没惹
着你,‘小老婆’长,‘小老婆’短,人家脸上怎么过的去?”袭人平儿忙道:“你
倒别说这话,他也并不是说我们,你倒别拉三扯四的,你听见那位太太、太爷们封
了我们做小老婆?况且我们两个,也没有爹娘哥哥兄弟在这门子里仗着我们横行霸
道的。他骂的人自由他骂去,我们犯不着多心。”鸳鸯道:“他见我骂了他,他臊
了,没的盖脸,又拿话调唆你们两个。幸亏你们两个明白。原是我急了,也没分别
出来,他就挑出这个空儿来!”他嫂子自觉没趣,赌气去了。鸳鸯气的还骂,平儿
袭人劝他一回,方罢了。

平儿因问袭人道:“你在那里藏着做什么?我们竟没有看见你。”袭人道:“我
因为往四姑娘房里看我们宝二爷去了,谁知迟了一步,说是家去了。我疑惑怎么没
遇见呢,想要往林姑娘家找去,又遇见他的人,说也没去。我这里正疑惑是出园子
去了,可巧你从那里来了。我一闪,你也没看见。后来他又来了,我从这树后头走
到山子石后,我却见你两个说话来了,谁知你们四个眼睛没见我。”一语未了,又
听身后笑道:“四个眼睛没见你?你们六个眼睛还没见我呢。”三人吓了一跳,回
身一看,你道是谁,却是宝玉。袭人先笑道:“叫我好找!你在那里来着?”宝玉
笑道:“我打四妹妹那里出来,迎头看见你走了来,我想来必是找我去的,我就藏
起来了哄你。看你扬着头过去了,进了院子,又出来了,逢人就问,我在那里好笑。
等着你到了跟前,吓你一跳。后来见你也藏藏躲躲的,我就知道也是要哄人了。我
探头儿往前看了一看,却是他们两个,我就绕到你身后头。你出去,我也躲在你躲
的那里了。”平儿笑道:“咱们再往后找找去罢,只怕还找出两个人来,也未可知。”
宝玉笑道:“这可再没有了。”

鸳鸯已知这话俱被宝玉听了,只伏在石头上装睡。宝玉推他笑道:“这石头上
冷,咱们回屋里去睡,岂不好?”说着,拉起鸳鸯来。又忙让平儿来家吃茶,和袭
人都劝鸳鸯走,鸳鸯方立起身来。四人竟往怡红院来。宝玉将方才的话俱已听见,
心中着实替鸳鸯不快,只默默的歪在床上,任他三人在外间说笑。

那边邢夫人因问凤姐儿鸳鸯的父亲,凤姐因说:“他爹的名字叫金彩,两口子
都在南京看房子,不大上来。他哥哥文翔现在是老太太的买办。他嫂子也是老太太
那边浆洗上的头儿。”邢夫人便命人叫了他嫂子金文翔的媳妇来,细细说给他。那
媳妇自是喜欢,兴兴头头去找鸳鸯,指望一说必妥,不想被鸳鸯抢白了一顿,又被
袭人平儿说了几句,羞恼回来。便对邢夫人说:“不中用,他骂了我一场。”因凤
姐儿在旁,不敢提平儿,说:“袭人也帮着抢白我,说了我许多不知好歹的话,回
不得主子的。太太和老爷商议再买罢。谅那小蹄子也没有这么大福,我们也没有这
么大造化。”邢夫人听了,说道:“又与袭人什么相干?他们如何知道呢?”又问:
“还有谁在跟前?”金家的道:“还有平姑娘。”凤姐儿忙道:“你不该拿嘴巴子
把他打回来?我一出了门,他就逛去了,回家来连个影儿也摸不着他!——他必定也
帮着说什么来着?”金家的道:“平姑娘倒没在跟前,远远的看着倒像是他,可也
不真切。不过是我白忖度着。”凤姐便命人去:“快找了他来,告诉我家来了,太
太也在这里,叫他快着来。”丰儿忙上来回道:“林姑娘打发了人下请字儿,请了
三四次,他才去了;奶奶一进门,我就叫他去的。林姑娘说:‘告诉奶奶,我烦他
有事呢。’”凤姐儿听了方罢,故意的还说:“天天烦他!有什么事情?”

邢夫人无计,吃了饭回家,晚上告诉了贾赦。贾赦想了一想,即刻叫贾琏来,
说:“南京的房子还有人看着,不止一家,即刻叫上金彩来。”贾琏回道:“上次
南京信来,金彩已经得了痰迷心窍,那边连棺材银子都赏了,不知如今是死是活。
即便活着,人事不知,叫来无用。他老婆子又是个聋子。”贾赦听了,喝了一声,
又骂:“混帐!没天理的囚攮的,偏你这么知道!还不离了我这里!”的贾琏退出。
一时又叫传金文翔。贾琏在外书房伺候着,又不敢家去,又不敢见他父亲,只得听
着。一时金文翔来了,小么儿们直带入二门里去,隔了四五顿饭的工夫,才出来去
了。贾琏暂且不敢打听,隔了一会,又打听贾赦睡了,方才过来。至晚间凤姐儿告
诉他,方才明白。

且说鸳鸯一夜没睡。至次日,他哥哥回贾母,接他家去逛逛,贾母允了,叫他
家去。鸳鸯意欲不去,只怕贾母疑心,只得勉强出来。他哥哥只得将贾赦的话说给
他,又许他怎么体面,又怎么当家做姨娘,鸳鸯只咬定牙不愿意。他哥哥无法,少
不得回去回复贾赦。贾赦恼起来,因说道:“我说给你,叫你女人和他说去。就说
我的话:‘自古嫦娥爱少年’,他必定嫌我老了。大约他恋着少爷们,多半是看上
了宝玉,只怕也有贾琏。若有此心,叫他早早歇了。我要他不来,以后谁敢收他?
这是一件。第二件,想着老太太疼他,将来外边聘个正头夫妻去。叫他细想:凭他
嫁到了谁家,也难出我的手心!除非他死了,或是终身不嫁男人,我就服了他!要不
然时叫他趁早回心转意,有多少好处。”贾赦说一句,金文翔应一声“是”。贾赦
道:“你别哄我,明儿我还打发你太太过去问鸳鸯。你们说了,他不依,便没你们
的不是;若问他,他再依了,仔细你们的脑袋!”金文翔忙应了又应,退出回家,
也等不得告诉他女人转说,竟自己对面说了这话。把个鸳鸯气的无话可回,想了一
想,便说道:“我便愿意去,也须得你们带了我回声老太太去。”他哥嫂只当回想
过来,都喜之不尽,他嫂子即刻带了他上来见贾母。

可巧王夫人、薛姨妈、李纨、凤姐儿、宝钗等姊妹并外头的几个执事有头脸的
媳妇,都在贾母跟前凑趣儿呢。鸳鸯看见,忙拉了他嫂子,到贾母跟前跪下,一面
哭,一面说,把邢夫人怎么来说,园子里他嫂子怎么说,今儿他哥哥又怎么说,“因
为不依,方才大老爷越发说我‘恋着宝玉’,不然,要等着往外聘,凭我到天上,
这一辈子也跳不出他的手心去,终久要报仇。我是横了心的,当着众人在这里,我
这一辈子,别说是宝玉,就是宝金、宝银、宝天王、宝皇帝,横竖不嫁人就完了!
就是老太太逼着我,一刀子抹死了,也不能从命!伏侍老太太归了西,我也不跟着
我老子娘哥哥去,或是寻死,或是剪了头发当姑子去!要说我不是真心,暂且拿话
支吾:这不是天地鬼神、日头月亮照着!嗓子里头长疔!”原来这鸳鸯一进来时,
便袖内带了一把剪子,一面说着,一面回手打开头发就铰。众婆子丫鬟看见,忙来
拉住,已剪下半绺来了。众人看时,幸而他的头发极多,铰的不透,连忙替他挽上。

贾母听了,气的浑身打战,口内只说:“我通共剩了这么一个可靠的人,他们
还要来算计!”因见王夫人在旁,便向王夫人道:“你们原来都是哄我的!外头孝
顺,暗地里盘算我!有好东西也来要,有好人也来要。剩了这个毛丫头,见我待他
好了,你们自然气不过,弄开了他,好摆弄我!”王夫人忙站起来,不敢还一言。
薛姨妈见连王夫人怪上,反不好劝的了。李纨一听见鸳鸯这话,早带了姊妹们出去。
探春有心的人,想王夫人虽有委屈,如何敢辩,薛姨妈现是亲妹妹,自然也不好辩,
宝钗也不便为姨母辩,李纨、凤姐、宝玉一发不敢辩。这正用着女孩儿之时——迎
春老实,惜春小——因此,窗外听了一听,便走进来,陪笑向贾母道:“这事与太
太什么相干?老太太想一想:也有大伯子的事,小婶子如何知道?”

话未说完,贾母笑道:“可是我老糊涂了。姨太太别笑话我!你这个姐姐,他
极孝顺,不像我们那大太太,一味怕老爷,婆婆跟前不过应景儿。可是我委屈了他。”
薛姨妈只答应“是”,又说:“老太太偏心,多疼小儿子媳妇,也是有的。”贾母
道:“不偏心。”因又说:“宝玉,我错怪了你娘,你怎么也不提我,看着你娘受
委屈?”宝玉笑道:“我偏着母亲说大爷大娘不成?通共一个不是,我母亲要不认,
却推谁去?我倒要认是我的不是,老太太又不信。”贾母笑道:“这也有理。你快
给你娘跪下,你说:太太别委屈了,老太太有年纪了,看着宝玉罢。”宝玉听了,
忙走过来,便跪下要说。王夫人忙笑着拉起他来,说:“快起来,断乎使不得,难
道替老太太给我赔不是不成?”宝玉听说,忙站起来。

贾母又笑道:“凤姐儿也不提我!”凤姐笑道:“我倒不派老太太的不是,老
太太倒寻上我了?”贾母听了,和众人都笑道:“这可奇了,倒要听听这个‘不是’?”
凤姐道:“谁叫老太太会调理人?调理的水葱儿似的,怎么怨得人要?我幸亏是孙子
媳妇,我若是孙子,我早要了,还等到这会子呢。”贾母笑道:“这倒是我的不是
了?”凤姐笑道:“自然是老太太的不是了。”贾母笑道:“这么着,我也不要了,
你带了去罢。”凤姐儿道:“等着修了这辈子,来生托生男人,我再要罢。”贾母
笑道:“你带了去,给琏儿放在屋里,看你那没脸的公公还要不要了!”凤姐儿道:
“琏儿不配,就只配我和平儿这一对‘烧糊了的子’,和他混罢咧。”说的众人
都笑起来了。

丫头回说:“大太太来了。”王夫人忙迎出去。

要知端底,下回分解。