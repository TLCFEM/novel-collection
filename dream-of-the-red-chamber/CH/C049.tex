\chapter{琉璃世界白雪红梅~脂粉香娃割腥啖膻}

话说香菱见众人正说笑他,便迎上去笑道:“你们看这首诗:要使得,我就还
学;要还不好,我就死了这做诗的心了。”说着,把诗递与黛玉及众人看时,只见
写道是:
精华欲掩料应难,影自娟娟魄自寒。
一片砧敲千里白,半轮鸡唱五更残。
绿蓑江上秋闻笛,红袖楼头夜倚栏。
博得嫦娥应自问:何缘不使永团?
众人看了,笑道:“这首不但好,而且新巧有意趣。可知俗语说:‘天下无难事,
只怕有心人。’社里一定请你了!”香菱听了,心下不信,料着是他们哄自己的话,
还只管问黛玉宝钗等。

正说之间,只见几个小丫头并老婆子忙忙的走来,都笑道:“来了好些姑娘奶
奶们,我们都不认得;奶奶姑娘们快认亲去。”李纨笑道:“这是那里的话?你到
底说明白了,是谁的亲戚?”那婆子丫头都笑道:“奶奶的两位妹子都来了;还有
一位姑娘,说是薛大姑娘的妹子;还有一位爷,说是薛大爷的兄弟。我这会子请姨
太太去呢,奶奶和姑娘们先上去罢。”说着,一径去了。宝钗笑道:“我们薛蝌和
他妹子来了不成?”李纨笑道:“或者我婶娘又上京来了?怎么他们都凑在一处?这
可是奇事。”

大家来至王夫人上房,只见黑压压的一地。又有邢夫人的嫂子,带了女儿岫烟
进京来投邢夫人的,可巧凤姐之兄王仁也正进京,两亲家一处搭帮来了。走至半路
泊船时,遇见李纨寡婶,带着两个女儿,长名李纹,次名李绮,也上京,大家叙起
来,又是亲戚,因此三家一路同行。后有薛蟠之从弟薛蝌,因当年父亲在京时,已
将胞妹薛宝琴许配都中梅翰林之子为妻,正欲进京聘嫁,闻得王仁进京,他也随后
带了妹子赶来。所以今日会齐了,来访投各人亲戚。于是大家见礼叙过,贾母王夫
人都欢喜非常。贾母因笑道:“怪道昨日晚上灯花爆了又爆,结了又结,原来应到
今日。”一面叙些家常,收了带来的礼物,一面命留酒饭。凤姐儿自不必说,忙上
加忙;李纨宝钗自然和婶母姊妹叙离别之情。黛玉见了,先是欢喜,后想起众人皆
有亲眷,独自己孤单无倚,不免又去垂泪。宝玉深知其情,十分劝慰了一番方罢。

然后宝玉忙忙来至怡红院中,向袭人、麝月、晴雯笑道:“你们还不快着看去!
谁知宝姐姐的亲哥哥是那个样子,他这叔伯兄弟,形容举止另是个样子,倒像是宝
姐姐的同胞兄弟似的。更奇在你们成日家只说宝姐姐是绝色的人物,你们如今瞧见
他这妹子,还有大嫂子的两个妹子,我竟形容不出来了。老天,老天,你有多少精
华灵秀,生出这些人上之人来!可知我‘井底之蛙’,成日家只说现在的这几个人
是有一无二的,谁知不必远寻,就是本地风光,一个赛似一个。如今我又长了一层
学问了。除了这几个,难道还有几个不成?”一面说,一面自笑。袭人见他又有些
魔意,便不肯去瞧。晴雯等早去瞧了一遍回来,带笑向袭人说道:“你快瞧瞧去!
大太太一个侄女儿,宝姑娘一个妹妹,大奶奶两个妹妹,倒像一把子四根水葱儿。”

一语未了,只见探春也笑着进来找宝玉,因说:“咱们诗社可兴旺了。”宝玉
笑道:“正是呢。这是一高兴起诗社,鬼使神差来了这些人。但只一件,不知他们
可学过做诗不曾?”探春道:“我才都问了问,虽是他们自谦,看其光景,没有不
会的。便是不会也没难处,你看香菱就知道了。”晴雯笑道:“他们里头薛大姑娘
的妹妹更好。三姑娘看着怎么样?”探春道:“果然的。据我看来,连他姐姐并这
些人总不及他。”袭人听了,又是诧异,又笑道:“这也奇了,还从那里再寻好的
去呢?我倒要瞧瞧去。”探春道:“老太太一见了,喜欢的无可不可的,已经逼着
咱们太太认了干女孩儿了。老太太要养活,才刚已经定了。”宝玉喜的忙问:“这
话果然么?”探春道:“我几时撒过谎?”又笑道:“老太太有了这个好孙女儿,
就忘了你这孙子了。”宝玉笑道:“这倒不妨,原该多疼女孩儿些是正理。明儿十
六,咱们可该起社了。”探春道:“林丫头刚起来了,二姐姐又病了,终是七上八
下的。”宝玉道:“二姐姐又不大做诗,没有他又何妨。”探春道:“索性等几天,
等他们新来的混熟了,咱们邀上他们岂不好?这会子大嫂子宝姐姐心里自然没有诗
兴的。况且湘云没来,颦儿才好了,人都不合式。不如等着云丫头来了,这几个新
的也熟了,颦儿也大好了,大嫂子和宝姐姐心也闲了,香菱诗也长进了:如此邀一
满社,岂不好?咱们两个如今且往老太太那里去听听,除宝姐姐的妹妹不算外,他
一定是在咱们家住定了的。倘或那三个要不在咱们这里住,咱们央告着老太太,留
下他们也在园子里住了,咱们岂不多添几个人,越发有趣了。”

宝玉听了,喜的眉开眼笑,忙说道:“倒是你明白。我终久是个糊涂心肠,空
喜欢了一会子,却想不到这上头。”说着,兄妹两个一齐往贾母处来。果然王夫人
已认了薛宝琴做干女儿,贾母喜欢非常,不命往园中住,晚上跟着贾母一处安寝。
薛蝌自向薛蟠书房中住下了。贾母和邢夫人说:“你侄女儿也不必家去了,园里住
几天,逛逛再去。”邢夫人兄嫂家中原艰难,这一上京原仗的是邢夫人与他们治房
舍、帮盘缠,听如此说,岂不愿意。邢夫人便将邢岫烟交与凤姐儿。凤姐儿算着园
中姊妹多,性情不一,且又不便另设一处,莫若送到迎春一处去,倘日后邢岫烟有
些不遂意的事,纵然邢夫人知道了,与自己无干。从此后,若邢岫烟家去住的日期
不算,若在大观园住到一个月上,凤姐儿亦照迎春分例,送一分与岫烟。凤姐儿冷
眼岫烟心性行为,竟不像邢夫人及他的父母一样,却是个极温厚可疼的人。因
此凤姐儿反怜他家贫命苦,比别的姊妹多疼他些,邢夫人倒不大理论了。贾母王夫
人等因素喜李纨贤惠,且年轻守节,令人敬服,今见他寡婶来了,便不肯叫他外头
去住。那婶母虽十分不肯,无奈贾母执意不从,只得带着李纹李绮在稻香村住下了。

当下安插既定,谁知忠靖侯史鼎又迁委了外省大员,不日要带家眷去上任,贾
母因舍不得湘云,便留下他了,接到家中。原要命凤姐儿另设一处与他住,史湘云
执意不肯,只要和宝钗一处住,因此也就罢了。

此时大观园中,比先又热闹了多少:李纨为首,馀者迎春、探春、惜春、宝钗、
黛玉、湘云、李纹、李绮、宝琴、邢岫烟,再添上凤姐儿和宝玉,一共十三人。叙
起年庚,除李纨年纪最长,凤姐次之,馀者皆不过十五六七岁,大半同年异月,连
他们自己也不能记清谁长谁幼;并贾母王夫人及家中婆子丫头也不能细细分清,不
过是“姐”“妹”“兄”“弟”四个字,随便乱叫。

如今香菱正满心满意只想做诗,又不敢十分罗唆宝钗,可巧来了个史湘云,那
史湘云极爱说话的,那里禁得香菱又请教他谈诗?越发高了兴,没昼没夜,高谈阔
论起来。宝钗因笑道:“我实在聒噪的受不得了。一个女孩儿家,只管拿着诗做正
经事讲起来,叫有学问的人听了反笑话,说不守本分。一个香菱没闹清,又添上你
这个话口袋子,满口里说的是什么:怎么是‘杜工部之沉郁,韦苏州之淡雅’,又
怎么是‘温八叉之绮靡,李义山之隐僻’。痴痴癫癫,那里还像两个女儿家呢?”
说得香菱湘云二人都笑起来。

正说着,只见宝琴来了,披着一领斗篷,金翠辉煌,不知何物。宝钗忙问:“这
是那里的?”宝琴笑道:“因下雪珠儿,老太太找了这一件给我的。”香菱上来瞧
道:“怪道这么好看,原来是孔雀毛织的。”湘云笑道:“那里是孔雀毛?就是野
鸭子头上的毛做的。可见老太太疼你了:这么着疼宝玉,也没给他穿。”宝钗笑道:
“真是俗语说的,‘各人有各人的缘法’。我也想不到他这会子来,既来了,又有
老太太这么疼他。”湘云道:“你除了在老太太跟前,就在园里,来这两处,只管
玩笑吃喝。到了太太屋里,若太太在屋里,只管和太太说笑,多坐一回无妨;若太
太不在屋里,你别进去。那屋里人多心坏,都是耍咱们的。”说的宝钗、宝琴、香
菱、莺儿等都笑了。宝钗笑道:“说你没心却有心,虽然有心,到底嘴太直了。我
们这琴儿,今儿你竟认他做亲妹妹罢。”湘云又瞅了宝琴笑道:“这一件衣裳也只
配他穿,别人穿了实在不配。”正说着,只见琥珀走来,笑道:“老太太说了:叫
宝姑娘别管紧了琴姑娘,他还小呢,让他爱怎么着就由他怎么着,他要什么东西只
管要,别多心。”宝钗忙起身答应了,又推宝琴笑道:“你也不知是那里来的这点
福气!你倒去罢,恐怕我们委屈了你!我就不信,我那些儿不如你?”

说话之间,宝玉黛玉进来了,宝钗犹自嘲笑。湘云因笑道:“宝姐姐,你这话
虽是玩,却有人真心是这样想呢。”琥珀笑道:“真心恼的再没别人,就只是他。”
口里说,手指着宝玉。宝钗湘云都笑道:“他倒不是这样人。”琥珀又笑道:“不
是他,就是他。”说着,又指黛玉。湘云便不作声。宝钗笑道:“更不是了。我的
妹妹和他的妹妹一样,他喜欢的比我还甚呢,他那里还恼?你信云儿混说,他那嘴
有什么正经。”宝玉素昔深知黛玉有些小性儿,尚不知近日黛玉和宝钗之事,正恐
贾母疼宝琴,他心中不自在。今儿湘云如此说了,宝钗又如此答,再审度黛玉声色
亦不似往日,果然与宝钗之说相符,心中甚是不解。因想:“他两个素日不是这样
的,如今看来,竟更比他人好了十倍。”一时又见林黛玉赶着宝琴叫“妹妹”,并
不提名道姓,真似亲姊妹一般。那宝琴年轻心热,且本性聪敏,自幼读书识字,今
在贾府住了两日,大概人物已知;又见众姊妹都不是那轻薄脂粉,且又和姐姐皆和
气,故也不肯怠慢。其中又见林黛玉是个出类拔萃的,便更与黛玉亲敬异常。宝玉
看着,只是暗暗的纳罕。

一时宝钗姊妹往薛姨妈房内去后,湘云往贾母处来,林黛玉回房歇着。宝玉便
找了黛玉来,笑道:“我虽看了《西厢记》,也曾有明白的几句说了取笑,你还曾
恼过。如今想来,竟有一句不解,我念出来,你讲讲我听。”黛玉听了,便知有文
章,因笑道:“你念出来我听听。”宝玉笑道:“那《闹简》上有一句说的最好:
‘是几时孟光接了梁鸿案?’这五个字不过是现成的典,难为他‘是几时’三个虚
字,问的有趣。是几时接了?你说说我听听。”黛玉听了,禁不住也笑起来,因笑
道:“这原问的好。他也问的好,你也问的好。”宝玉道:“先时你只疑我,如今
你也没的说了。”黛玉笑道:“谁知他竟真是个好人,我素日只当他藏奸。”因把
说错了酒令,宝钗怎样说他,连送燕窝,病中所谈之事,细细的告诉宝玉,宝玉方
知原故。因笑道:“我说呢!正纳闷‘是几时孟光接了梁鸿案’,原来是从‘小孩
儿家口没遮拦’上就接了案了。”

黛玉因又说起宝琴来,想起自己没有姊妹,不免又哭了。宝玉忙劝道:“这又
自寻烦恼了。你瞧瞧,今年比旧年越发瘦了,你还不保养。每天好好的,你必是自
寻烦恼,哭一会子,才算完了这一天的事。”黛玉拭泪道:“近来我只觉心酸,眼
泪却像比旧年少了些的。心里只管酸痛,眼泪却不多。”宝玉道:“这是你哭惯了,
心里疑惑,岂有眼泪会少的!”

正说着,只见他屋里的小丫头子送了猩猩毡斗篷来,又说:“大奶奶才打发人
来说:下了雪,要商议明日请人做诗呢。”一语未了,只见李纨的丫头走来请黛玉。
宝玉便邀着黛玉同往稻香村来。黛玉换上掐金挖云红香羊皮小靴,罩了一件大红羽
绉面白狐狸皮的鹤氅,系一条青金闪绿双环四合如意绦,上罩了雪帽。二人一齐踏
雪行来,只见众姊妹都在那里,都是一色大红猩猩毡与羽毛缎斗篷,独李纨穿一件
哆罗呢对襟褂子,薛宝钗穿一件莲青斗纹锦上添花洋线番丝的鹤氅。邢岫烟仍是
家常旧衣,并没避雨之衣。一时湘云来了,穿着贾母给他的一件貂鼠脑袋面子、大
毛黑灰鼠里子、里外发烧大褂子,头上带着一顶挖云鹅黄片金里子大红猩猩毡昭君
套,又围着大貂鼠风领。黛玉先笑道:“你们瞧瞧,孙行者来了。他一般的拿着雪
褂子,故意妆出个小骚鞑子样儿来。”湘云笑道:“你们瞧我里头打扮的。”一面
说,一面脱了褂子,只见他里头穿着一件半新的靠色三厢领袖秋香色盘金五色绣龙
窄小袖掩衿银鼠短袄,里面短短的一件水红妆缎狐肷褶子,腰里紧紧束着一条蝴
蝶结子长穗五色宫绦,脚下也穿着鹿皮小靴,越显得蜂腰猿背,鹤势螂形。众人笑
道:“偏他只爱打扮成个小子的样儿,原比他打扮女儿更俏丽了些。”

湘云笑道:“快商议做诗。我听听是谁的东家?”李纨道:“我的主意。想来
昨儿的正日已自过了,再等正日还早呢,可巧又下雪,不如咱们大家凑个热闹,又
给他们接风,又可以做诗。你们意思怎么样?”宝玉先道:“这话很是,只是今儿
晚了,若到明儿,晴了又无趣。”众人都道:“这雪未必晴。纵晴了,这一夜下的
也够赏了。”李纨道:“我这里虽然好,又不如芦雪庭好。我已经打发人笼地炕去
了,咱们大家拥炉做诗。老太太想来未必高兴。况且咱们小玩意儿,单给凤丫头个
信儿就是了。你们每人一两银子就够了,送到我这里来。”指着香菱、宝琴、李纹、
李绮、岫烟,“五个不算外,咱们里头二丫头病了不算,四丫头告了假也不算,你
们四分子送了来,我包管五六两银子也尽够了。”宝钗等一齐应诺。因又拟题限韵,
李纨笑道:“我心里早已定了。等到了明日临期,横竖知道。”说毕,大家又说了
一回闲话,方往贾母处来,当日无话。

到了次日清早,宝玉因心里惦记着,这一夜没好生得睡,天亮了就爬起来。掀
起帐子一看,虽然门窗尚掩,只是窗上光辉夺目,心内早踌躇起来,埋怨定是晴了,
日光已出。一面忙起来揭起窗屉,从玻璃窗内往外一看,原来不是日光,竟是一夜
的雪,下的将有一尺厚,天上仍是搓绵扯絮一般。宝玉此时喜欢非常,忙唤起人来,
盥漱已毕,只穿一件茄色哆罗呢狐狸皮袄,罩一件海龙小鹰膀褂子,束了腰,披上
玉针蓑,带了金藤笠,登上沙棠屐,忙忙的往芦雪庭来。出了院门,四顾一望,并
无二色,远远的是青松翠竹,自己却似装在玻璃盆内一般。于是走至山坡之下。顺
着山脚刚转过去,已闻得一股寒香扑鼻,回头一看,却是妙玉那边栊翠庵中有十数
枝红梅如胭脂一般,映着雪色,分外显得精神,好不有趣。宝玉便立住,细细的赏
玩了一回方走。只见蜂腰板桥上一个人打着伞走来,是李纨打发了请凤姐儿去的
人。宝玉来至芦雪庭,只见丫头婆子正在那里扫雪开径。原来这芦雪庭盖在一个傍
山临水河滩之上,一带几间茅檐土壁,横篱竹牖,推窗便可垂钓,四面皆是芦苇掩
覆。一条去径,逶迤穿芦度苇过去,便是藕香榭的竹桥了。众丫头婆子见他披蓑带
笠而来,都笑道:“我们才说正少一个渔翁,如今果然全了。姑娘们吃了饭才来呢,
你也太性急了。”宝玉听了,只得回来。刚至沁芳亭,见探春正从秋爽斋出来,围
着大红猩猩毡的斗篷,带着观音兜,扶着个小丫头,后面一个妇人打着一把青绸油
伞。宝玉知道他往贾母处去,遂站在亭边等他来到,二人一同出园前去。

宝琴正在里间房内梳洗更衣。一时众姐妹来齐,宝玉只嚷饿了,连连催饭。好
容易等摆上饭来,头一样菜是牛乳蒸羊羔,贾母就说:“这是我们有年纪人的药,
没见天日的东西,可惜你们小孩子吃不得。今儿另外有新鲜鹿肉,你们等着吃罢。”
众人答应了。宝玉却等不得,只拿茶泡了一碗饭,就着野鸡爪子忙忙的爬拉完了。
贾母道:“我知道你们今儿又有事情,连饭也不顾吃了。”就叫:“留着鹿肉给他
晚上吃罢。”凤姐儿忙说:“还有呢,吃残了的倒罢了。”湘云就和宝玉计较道:
“有新鹿肉,不如咱们要一块,自己拿了园里弄着,又吃又玩。”宝玉听了,真和
凤姐要了一块,命婆子送进园去。

一时大家散后,进园齐往芦雪庭来,听李纨出题限韵。独不见湘云宝玉二人。
黛玉道:“他两个人再到不得一处,要到了一处,生出多少事来。这会子一定算计
那块鹿肉去了。”正说着,只见李婶娘也走来看热闹,因问李纨道:“怎么那一个
带玉的哥儿和那一个挂金麒麟的姐儿,那样干净清秀,又不少吃的,他两个在那里
商议着要吃生肉呢,说的有来有去的。我只不信,肉也生吃得的?”众人听了,都
笑道:“了不得,快拿了他两个来。”黛玉笑道:“这可是云丫头闹的。我的卦再
不错。”李纨即忙出来,找着他两个,说道:“你们两个要吃生的,我送你们到老
太太那里吃去,那怕一只生鹿,撑病了不与我相干。这么大雪,怪冷的,快替我做
诗去罢。”宝玉忙笑道:“没有的事!我们烧着吃呢。”李纨道:“这还罢了。”
只见老婆子们拿了铁炉、铁叉、铁丝蒙来,李纨道:“留神,割了手不许哭。”说
着,方进去了。

那边凤姐打发平儿回复不来,为发放年例正忙着呢。湘云见了平儿,那里肯放?
平儿也是个好玩的,素日跟着凤姐儿无所不至,见如此有趣,乐得玩笑,因而退去
手上的镯子,三个人围着火,平儿便要先烧三块吃。那边宝钗黛玉平素看惯了,不
以为异,宝琴等及李婶娘深为罕事。探春和李纨等已议定了题韵。探春笑道:“你
们闻闻,香气这里都闻见了,我也吃去。”说着,也找了他们来。李纨也随来,说:
“客已齐了,你们还吃不够吗?”湘云一面吃,一面说道:“我吃这个方爱吃酒,
吃了酒才有诗。若不是这鹿肉,今儿断不能做诗。”说着,只见宝琴披着凫靥裘,
站在那里笑。湘云笑道:“傻子!你来尝尝。”宝琴笑道:“怪腌的。”宝钗笑
道:“你尝尝去,好吃的很呢。你林姐姐弱,吃了不消化,不然,他也爱吃。”宝
琴听了,就过去吃了一块,果然好吃,就也吃起来。一时凤姐儿打发小丫头来叫平
儿,平儿说:“史姑娘拉着我呢,你先去罢。”小丫头去了。一时,只见凤姐儿也
披了斗篷走来,笑道:“吃这样好东西,也不告诉我!”说着,也凑在一处吃起来。
黛玉笑道:“那里找这一群花子去!罢了罢了,今日芦雪庭遭劫,生生被云丫头作
践了。我为芦雪庭一大哭。”湘云冷笑道:“你知道什么!‘是真名士自风流’。
你们都是假清高,最可厌的。我们这会子腥的膻的大吃大嚼,回来却是锦心绣口。”
宝钗笑道:“你回来若做的不好了,把那肉掏出来,就把这雪压的芦苇子上些,
以完此劫。”

说着,吃毕,洗了一回手。平儿带镯子时,却少了一个,左右前后乱找了一番,
踪迹全无。众人都诧异。凤姐儿笑道:“我知道这镯子的去向,你们只管做诗去。
我们也不用找,只管前头去,不出三日包管就有了。”说着又问:“你们今儿做什
么诗?老太太说了,离年又近了,正月里还该做些灯谜儿大家玩笑。”众人听了,
都笑道:“可是呢,倒忘了。如今赶着做几个好的,预备着正月里玩。”说着,一
齐来至地炕屋内,只见杯盘果菜俱已摆齐了,墙上已贴出诗题、韵脚、格式来了。
宝玉湘云二人忙看时,只见题目是《即景联句》,“五言排律一首,限‘二萧’韵。”
后面尚未列次序。李纨道:“我不大会做诗,我只起三句罢,然后谁先得了谁先联。”
宝钗道:“到底分个次序。”

要知端底,且看下回分解。