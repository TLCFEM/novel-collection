\chapter{醉金刚轻财尚义侠~痴女儿遗帕惹相思}

话说黛玉正在情思萦逗、缠绵固结之时,忽有人从背后拍了一下,说道:“你
作什么一个人在这里?”黛玉唬了一跳,回头看时,不是别人,却是香菱。黛玉道:
“你这个傻丫头,冒冒失失的唬我一跳。这会子打那里来?”香菱嘻嘻的笑道:“我
来找我们姑娘,总找不着。你们紫鹃也找你呢,说琏二奶奶送了什么茶叶来了。回
家去坐着罢。”一面说,一面拉着黛玉的手,回潇湘馆来,果然凤姐送了两小瓶上
用新茶叶来。黛玉和香菱坐了,谈讲些这一个绣的好,那一个扎的精,又下一回棋,
看两句书,香菱便走了,不在话下。

且说宝玉因被袭人找回房去,只见鸳鸯歪在床上看袭人的针线呢,见宝玉来
了,便说道:“你往那里去了?老太太等着你呢,叫你过那边请大老爷的安去。还
不快去换了衣裳走呢!”袭人便进房去取衣服。宝玉坐在床沿上褪了鞋,等靴子穿
的工夫,回头见鸳鸯穿着水红绫子袄儿,青缎子坎肩儿,下面露着玉色绸,大红
绣鞋,向那边低着头看针线,脖子上围着紫绸绢子。宝玉便把脸凑在脖项上,闻那
香气,不住用手摩挲,其白腻不在袭人以下。便猴上身去,涎着脸笑道:“好姐姐,
把你嘴上的胭脂赏我吃了罢!”一面说,一面扭股糖似的粘在身上。鸳鸯便叫道:
“袭人你出来瞧瞧!你跟他一辈子,也不劝劝他,还是这么着。”袭人抱了衣裳出
来,向宝玉道:“左劝也不改,右劝也不改,你到底是怎么着?你再这么着,这个
地方儿可也就难住了。”一边说,一边催他穿衣裳,同鸳鸯往前面来。

见过贾母,出至外面,人马俱已齐备。刚欲上马,只见贾琏请安回来正下马。
二人对面,彼此问了两句话,只见旁边转过一个人来,说:“请宝叔安。”宝玉看
时,只见这人生的容长脸儿,长挑身材,年纪只有十八九岁,甚实斯文清秀。虽然
面善,却想不起是那一房的,叫什么名字。贾琏笑道:“你怎么发呆?连他也不认
得?他是廊下住的五嫂子的儿子芸儿。”宝玉笑道:“是了,我怎么就忘了。”因
问他:“你母亲好?这会子什么勾当?”贾芸指贾琏道:“找二叔说句话。”宝玉
笑道:“你倒比先越发出挑了,倒像我的儿子。”贾琏笑道:“好不害臊!人家比
你大五六岁呢,就给你作儿子了?”宝玉笑道:“你今年十几岁?”贾芸道:“十
八了。”原来这贾芸最伶俐乖巧的,听宝玉说像他的儿子,便笑道:“俗话说的好,
‘摇车儿里的爷爷,拄拐棍儿的孙子’。虽然年纪大,‘山高遮不住太阳’。只从
我父亲死了,这几年也没人照管,宝叔要不嫌侄儿蠢,认做儿子,就是侄儿的造化
了。”贾琏笑道:“你听见了?认了儿子,不是好开交的。”说着笑着进去了。宝
玉笑道:“明儿你闲了,只管来找我,别和他们鬼鬼祟祟的。这会子我不得闲儿,
明日你到书房里来,我和你说一天话儿,我带你园里玩去。”说着,扳鞍上马,众
小厮随往贾赦这边来。

见了贾赦,不过是偶感些风寒。先述了贾母问的话,然后自己请了安;贾赦先
站起来回了贾母问的话,便唤人来:“带进哥儿去太太屋里坐着。”宝玉退出来,
至后面,到上房,邢夫人见了,先站了起来请过贾母的安,宝玉方请安。邢夫人拉
他上炕坐了,方问别人,又命人倒茶。茶未吃完,只见贾琮来问宝玉好。邢夫人道:
“那里找活猴儿去!你那奶妈子死绝了,也不收拾收拾。弄的你黑眉乌嘴的,那里
还像个大家子念书的孩子?”正说着,只见贾环贾兰小叔侄两个也来请安。邢夫人
叫他两个在椅子上坐着。贾环见宝玉同邢夫人坐在一个坐褥上,邢夫人又百般摸索
抚弄他,早已心中不自在了,坐不多时,便向贾兰使个眼色儿要走。贾兰只得依他,
一同起身告辞。

宝玉见他们起身,也就要一同回去。邢夫人笑道:“你且坐着,我还和你说话。”
宝玉只得坐了。邢夫人向他两个道:“你们回去,各人替我问各人的母亲好罢。你
姑姑姐姐们都在这里呢,闹的我头晕!今儿不留你们吃饭了。”贾环等答应着便出
去了。宝玉笑道:“可是姐姐们都过来了?怎么不见?”邢夫人道:“他们坐了会
子,都往后头不知那屋里去了。”宝玉说:“大娘说‘有话说’,不知是什么话?”
邢夫人笑道:“那里什么话,不过叫你等着同姐妹们吃了饭去,还有一个好玩的东
西给你带回去玩儿。”娘儿两个说着,不觉又晚饭时候,请过众位姑娘们来,调开
桌椅,罗列杯盘。母女姊妹们吃毕了饭,宝玉辞别贾赦,同众姊妹们回家,见过贾
母王夫人等,各自回房安歇,不在话下。

且说贾芸进去,见了贾琏,因打听:“可有什么事情?”贾琏告诉他说:“前
儿倒有一件事情出来,偏偏你婶娘再三求了我,给了芹儿了。他许我说:‘明儿园
里还有几处要栽花木的地方,等这个工程出来,一定给你就是了。’”那贾芸听了,
半晌说道:“既这么着,我就等着罢。叔叔也不必先在婶娘跟前提我今儿来打听的
话,到跟前再说也不迟。”贾琏道:“提他做什么!我那里有这工夫说闲话呢。明
日还要到兴邑去走一走,必须当日赶回来方好。你先等着去。后日起更以后,你来
讨信,早了我不得闲。”说着,便向后面换衣服去了。

贾芸出了荣国府回家,一路思量,想出一个主意来,便一径往他舅舅卜世仁家
来。原来卜世仁现开香料铺,方才从铺子里回来,一见贾芸,便问:“你做什么来
了?”贾芸道:“有件事求舅舅帮衬:要用冰片、麝香,好歹舅舅每样赊四两给我,
八月节按数送了银子来。”卜世仁冷笑道:“再休提赊欠一事!前日也是我们铺子
里一个伙计,替他的亲戚赊了几两银子的货,至今总没还,因此我们大家赔上,立
了合同,再不许替亲友赊欠,谁要犯了,就罚他二十两银子的东道。况且如今这个
货也短,你就拿现银子到我们这小铺子里来买,也还没有这些,只好倒扁儿去,这
是一件。二则你那里有正经事?不过赊了去又是胡闹。你只说舅舅见你一遭儿就派
你一遭儿不是,你小人儿家很不知好歹,也要立个主意,赚几个钱,弄弄穿的吃的,
我看着也喜欢。”贾芸笑道:“舅舅说的有理。但我父亲没的时候儿,我又小,不
知事体。后来听见母亲说,都还亏了舅舅替我们出主意料理的丧事。难道舅舅是不
知道的:还是有一亩地,两间房子,在我手里花了不成?‘巧媳妇做不出没米的饭
来’,叫我怎么样呢?还亏是我呢,要是别的死皮赖脸的,三日两头儿来缠舅舅,
要三升米二升豆子,舅舅也就没法儿呢!”卜世仁道:“我的儿,舅舅要有,还不
是该当的?我天天和你舅母说,只愁你没个算计儿。你但凡立的起来,到你们大屋
里,就是他们爷儿们见不着,下个气儿和他们的管事的爷们嬉和嬉和,也弄个事儿
管管。前儿我出城去,碰见你们三屋里的老四,坐着好体面车,又带着四五辆车,
有四五十小和尚道士儿,往家庙里去了。他那不亏能干,就有这个事到他身上了?”
贾芸听了唠叨的不堪,便起身告辞。卜世仁道:“怎么这么忙?你吃了饭去罢。”
一句话尚未说完,只见他娘子说道:“你又糊涂了!说着没有米,这里买了半斤面
来下给你吃,这会子还装胖呢。留下外甥挨饿不成?”卜世仁道:“再买半斤来添
上就是了。”他娘子便叫女儿:“银姐,往对门王奶奶家去问:有钱借几十个,明
儿就送了来的。”夫妻两个说话,那贾芸早说了几个“不用费事”,去的无影无踪
了。

不言卜家夫妇,且说贾芸赌气离了舅舅家门,一径回来,心下正自烦恼,一边
想,一边走。低着头,不想一头就碰在一个醉汉身上,把贾芸一把拉住,骂道:“你
瞎了眼?碰起我来了!”贾芸听声音像是熟人,仔细一看,原来是紧邻倪二。这倪
二是个泼皮,专放重利债,在赌博场吃饭,专爱喝酒打架。此时正从欠钱人家索债
归来,已在醉乡,不料贾芸碰了他,就要动手。贾芸叫道:“老二,住手!是我冲
撞了你。”倪二一听他的语音,将醉眼睁开,一看见是贾芸,忙松了手,趔趄着笑
道:“原来是贾二爷。这会子那里去?”贾芸道:“告诉不得你,平白的又讨了个
没趣儿。”倪二道:“不妨。有什么不平的事告诉我,我替你出气。这三街六巷凭
他是谁,若得罪了我醉金刚倪二的街坊,管叫他人离家散!”贾芸道:“老二,你
别生气,听我告诉你这缘故。”便把卜世仁一段事告诉了倪二。倪二听了大怒道:
“要不是二爷的亲戚,我就骂出来。真真把人气死!也罢,你也不必愁,我这里现
有几两银子,你要用只管拿去。我们好街坊,这银子是不要利钱的。”一头说,一
头从搭包内掏出一包银子来。

贾芸心下自思:“倪二素日虽然是泼皮,却也因人而施,颇有义侠之名。若今
日不领他这情,怕他臊了,反为不美。不如用了他的,改日加倍还他就是了。”因
笑道:“老二,你果然是个好汉!既蒙高情,怎敢不领?回家就照例写了文约送过来。”
倪二大笑道:“这不过是十五两三钱银子,你若要写文约,我就不借了。”贾芸听
了,一面接银子,一面笑道:“我遵命就是了。何必着急!”倪二笑道:“这才是
呢。天气黑了,也不让你喝酒了,我还有点事儿,你竟请回罢。我还求你带个信儿
给我们家:叫他们关了门睡罢,我不回家去了。倘或有事,叫我们女孩儿明儿一早
到马贩子王短腿家找我。”一面说,一面趔趄着脚儿去了。不在话下。

且说贾芸偶然碰见了这件事,心下也十分稀罕,想那倪二倒果然有些意思,只
是怕他一时醉中慷慨,到明日加倍来要,便怎么好呢。忽又想道:“不妨,等那件
事成了,可也加倍还的起他。”因走到一个钱铺里,将那银子称了称,分两不错,
心上越发喜欢。到家先将倪二的话捎给他娘子儿,方回家来。他母亲正在炕上拈线,
见他进来,便问:“那里去了一天?”贾芸恐母亲生气,便不提卜世仁的事,只说:
“在西府里等琏二叔来着。”问他母亲:“吃了饭了没有?”他母亲说:“吃了。
还留着饭在那里。”叫小丫头拿来给他吃。

那天已是掌灯时候,贾芸吃了饭,收拾安歇,一宿无话。次日起来,洗了脸,
便出南门大街,在香铺买了冰麝,往荣府来。打听贾琏出了门,贾芸便往后面来。
到贾琏院门前,只见几个小厮,拿着大高的苕帚在那里扫院子呢。忽见周瑞家的从
门里出来叫小厮们:“先别扫,奶奶出来了。”贾芸忙上去笑问道:“二婶娘那里
去?”周瑞家的道:“老太太叫,想必是裁什么尺头。”正说着,只见一群人簇拥
着凤姐出来了。贾芸深知凤姐是喜奉承爱排场的,忙把手逼着,恭恭敬敬抢上来请
安。凤姐连正眼也不看,仍往前走,只问他母亲好:“怎么不来这里逛逛?”贾芸
道:“只是身上不好,倒时常惦记着婶娘,要瞧瞧,总不能来。”凤姐笑道:“可
是你会撒谎!不是我提,他也就不想我了。”贾芸笑道:“侄儿不怕雷劈,就敢在
长辈儿跟前撒谎了?昨儿晚上还提起婶娘来,说:‘婶娘身子单弱,事情又多,亏
了婶娘好精神,竟料理的周周全全的。要是差一点儿的,早累的不知怎么样了。’”

凤姐听了,满脸是笑,由不的止了步,问道:“怎么好好儿的,你们娘儿两个
在背地里嚼说起我来?”贾芸笑着道:“只因我有个好朋友,家里有几个钱,现开
香铺,因他捐了个通判,前儿选着了云南不知那一府,连家眷一齐去。他这香铺也
不开了,就把货物攒了一攒,该给人的给人,该贱发的贱发。像这贵重的,都送给
亲友,所以我得了些冰片、麝香。我就和我母亲商量,贱卖了可惜,要送人也没有
人家儿配使这些香料。因想到婶娘往年间还拿大包的银子买这些东西呢,别说今年
贵妃宫中,就是这个端阳节所用,也一定比往常要加十几倍:所以拿来孝敬婶娘。”
一面将一个锦匣递过去。凤姐正是办节礼用香料,便笑了一笑,命丰儿:“接过芸
哥儿的来,送了家去,交给平儿。”因又说道:“看你这么知好歹,怪不得你叔叔
常提起你来,说你好,说话明白,心里有见识。”贾芸听这话入港,便打进一步来,
故意问道:“原来叔叔也常提我?”凤姐见问,便要告诉给他事情管的话,一想又
恐他看轻了,只说得了这点儿香料,便许他管事了。因且把派他种花木的事一字不
提,随口说了几句淡话,便往贾母屋里去了。

贾芸自然也难提,只得回来。因昨日见了宝玉,叫他到外书房等着,故此吃了
饭,又进来,到贾母那边仪门外绮散斋书房里来。只见茗烟在那里掏小雀儿呢。贾
芸在他身后,把脚一跺,道:“茗烟小猴儿又淘气了!”茗烟回头,见是贾芸,便
笑道:“何苦二爷唬我们这么一跳。”因又笑说:“我不叫茗烟了,我们宝二爷嫌
‘烟’字不好,改了叫‘焙茗’了。二爷明儿只叫我焙茗罢。”贾芸点头笑着同进
书房,便坐下问:“宝二爷下来了没有?”焙茗道:“今日总没下来。二爷说什么,
我替你探探去。”说着,便出去了。

这里贾芸便看字画古玩。有一顿饭的工夫,还不见来。再看看要找别的小子,
都玩去了。正在烦闷,只听门前娇音嫩语的叫了一声“哥哥呀”。贾芸往外瞧时,
是个十五六岁的丫头,生的倒甚齐整,两只眼儿水水灵灵的,见了贾芸,抽身要躲,
恰值焙茗走来,见那丫头在门前,便说道:“好,好,正抓不着个信儿呢!”贾芸
见了焙茗,也就赶出来,问:“怎么样?”焙茗道:“等了半日,也没个人过。这
就是宝二爷屋里的。”因说道:“好姑娘,你带个信儿,就说廊上二爷来了。”那
丫头听见,方知是本家的爷们,便不似从前那等回避,下死眼把贾芸钉了两眼。听
那贾芸说道:“什么‘廊上’‘廊下’的,你只说芸儿就是了。”半晌,那丫头似
笑不笑的说道:“依我说,二爷且请回去,明日再来。今儿晚上得空儿,我替回罢。”
焙茗道:“这是怎么说?”那丫头道:“他今儿也没睡中觉,自然吃的晚饭早,晚
上又不下来,难道只是叫二爷这里等着挨饿不成?不如家去,明儿来是正经。就便
回来有人带信儿,也不过嘴里答应着罢咧。”贾芸听这丫头的话简便俏丽,待要问
他的名字,因是宝玉屋里的,又不便问,只得说道:“这话倒是。我明日再来。”
说着,便往外去了。焙茗道:“我倒茶去。二爷喝了茶再去。”贾芸一面走,一面
回头说:“不用,我还有事呢。”口里说话,眼睛瞧那丫头还站在那里呢。

那贾芸一径回来。至次日,来至大门前,可巧遇见凤姐往那边去请安,才上了
车,见贾芸过来,便命人叫住,隔着窗子笑道:“芸儿,你竟有胆子在我跟前弄鬼!
怪道你送东西给我,原来你有事求我。昨儿你叔叔才告诉我,说你求他。”贾芸笑
道:“求叔叔的事,婶娘别提,我这里正后悔呢。早知这样,我一起头儿就求婶娘,
这会子早完了,谁承望叔叔竟不能的!”凤姐笑道:“哦!你那边没成儿,昨儿又
来找我了?”贾芸道:“婶娘辜负了我的孝心。我并没有这个意思,要有这个意思,
昨儿还不求婶娘吗?如今婶娘既知道了,我倒要把叔叔搁开,少不得求婶娘,好歹
疼我一点儿。”凤姐冷笑道:“你们要拣远道儿走么!早告诉我一声儿,多大点子
事,还值的耽误到这会子。那园子里还要种树种花儿,我正想个人呢,早说不早完
了?”贾芸笑道:“这样明日婶娘就派我罢?”凤姐半晌道:“这个我看着不大好,
等明年正月里的烟火灯烛那个大宗儿下来,再派你不好?”贾芸道:“好婶娘,先
把这个派了我,果然这件办的好,再派我那件罢。”凤姐笑道:“你倒会拉长线儿!
罢了,要不是你叔叔说,我不管你的事。我不过吃了饭就过来。你到午错时候来领
银子,后日就进去种花儿。”说着,命人驾起香车,径去了。

贾芸喜不自禁。来至绮散斋打听宝玉,谁知宝玉一早便往北静王府里去了。贾
芸便呆呆的坐到晌午。打听凤姐回来,去写个领票来领对牌,至院外,命人通报了,
彩明走出来要了领票,进去批了银数、年月,一并连对牌交给贾芸。贾芸接来看那
批上批着二百两银子,心中喜悦,翻身走到银库上领了银子,回家告诉他母亲,自
是母子俱喜。次日五更,贾芸先找了倪二还了银子,又拿了五十两银子出西门找到
花儿匠方椿家里去买树,不在话下。

且说宝玉自这日见了贾芸,曾说过明日着他进来说话,这原是富贵公子的口
角,那里还记在心上,因而便忘怀了。这日晚上,却从北静王府里回来,见过贾母
王夫人等回至园内。换了衣服,正要洗澡,袭人被宝钗烦了去打结子去了,秋纹碧
痕两个去催水。檀云又因他母亲病了,接出去了;麝月现在家中病着;还有几个做
粗活听使唤的丫头,料是叫不着他,都出去寻伙觅伴的去了。不想这一刻的工夫,
只剩了宝玉在屋内。偏偏的宝玉要喝茶,一连叫了两三声,方见两三个老婆子走进
来。宝玉见了,连忙摇手说:“罢罢,不用了。”老婆子们只得退出。宝玉见没丫
头们,只得自己下来,拿了碗,向茶壶去倒茶。只听背后有人说道:“二爷看烫了
手,等我倒罢。”一面说,一面走上来接了碗去。宝玉倒唬了一跳,问:“你在那
里来着?忽然来了,唬了我一跳!”那丫头一面递茶,一面笑着回道:“我在后院
里。才从里间后门进来,难道二爷就没听见脚步响么?”宝玉一面吃茶,一面仔细
打量那丫头:穿着几件半新不旧的衣裳,倒是一头黑鸦鸦的好头发,挽着儿,容
长脸面,细挑身材,却十分俏丽甜净。宝玉便笑问道:“你也是我屋里的人么?”
那丫头笑应道:“是。”宝玉道:“既是这屋里的,我怎么不认得?”那丫头听说,
便冷笑一声道:“爷不认得的也多呢,岂止我一个。从来我又不递茶水拿东西,眼
面前儿的一件也做不着,那里认得呢?”宝玉道:“你为什么不做眼面前儿的呢?”
那丫头道:“这话我也难说。只是有句话回二爷:昨日有个什么芸儿来找二爷,我
想二爷不得空儿,便叫焙茗回他;今日来了,不想二爷又往北府里去了。”刚说到
这句话,只见秋纹碧痕嘻嘻哈哈的笑着进来,两个人共提着一桶水,一手撩衣裳,
趔趔趄趄泼泼撒撒的。那丫头便忙迎出去接。

秋纹碧痕,一个抱怨“你湿了我的衣裳”,一个又说“你踹了我的鞋”。忽见
走出一个人来接水,二人看时,不是别人,原来是小红。二人便都诧异,将水放下,
忙进来看时,并没别人,只有宝玉,便心中俱不自在。只得且预备下洗澡之物。待
宝玉脱了衣裳,二人便带上门出来,走到那边房内,找着小红,问他:“方才在屋
里做什么?”小红道:“我何曾在屋里呢?因为我的绢子找不着,往后头找去,不
想二爷要茶喝,叫姐姐们,一个儿也没有,我赶着进去倒了碗茶,姐姐们就来了。”
秋纹兜脸啐了一口道:“没脸面的下流东西!正经叫你催水去,你说有事,倒叫我
们去,你可抢这个巧宗儿!一里一里的,这不上来了吗?难道我们倒跟不上你么?你
也拿镜子照照,配递茶递水不配?”碧痕道:“明儿我说给他们,凡要茶要水拿东
西的事,咱们都别动,只叫他去就完了。”秋纹道:“这么说,还不如我们散了,
单让他在这屋里呢。”二人你一句我一句正闹着,只见有个老嬷嬷进来传凤姐的话
说:“明日有人带花儿匠来种树,叫你们严紧些,衣裳裙子别混晒混晾的。那土山
上都拦着围幕,可别混跑。”秋纹便问:“明日不知是谁带进匠人来监工?”那老
婆子道:“什么后廊上的芸哥儿。”秋纹碧痕俱不知道,只管混问别的话,那小红
心内明白,知是昨日外书房所见的那人了。

原来这小红本姓林,小名红玉,因“玉”字犯了宝玉黛玉的名,便改唤他做“小
红”,原来是府中世仆,他父亲现在收管各处田房事务。这小红年方十四,进府当
差,把他派在怡红院中,倒也清幽雅静。不想后来命姊妹及宝玉等进大观园居住,
偏生这一所儿,又被宝玉点了。这小红虽然是个不谙事体的丫头。因他原有几分容
貌,心内便想向上攀高,每每要在宝玉面前现弄现弄。只是宝玉身边一干人都是伶
牙俐爪的,那里插的下手去?不想今日才有些消息,又遭秋纹等一场恶话,心内早
灰了一半,正没好气,忽然听见老嬷嬷说起贾芸来,不觉心中一动,便闷闷的回房,
睡在床上,暗暗思量,翻来覆去,自觉没情没趣的。忽听的窗外低低的叫道:“红
儿,你的娟子我拾在这里呢。”小红听了,忙走出来看时,不是别人,正是贾芸。
小红不觉粉面含羞,问道:“二爷在那里拾着的?”只见那贾芸笑道:“你过来,
我告诉你。”一面说一面就上来拉他的衣裳。那小红臊的转身一跑,却被门槛子绊
倒。

要知端底,下回分解。