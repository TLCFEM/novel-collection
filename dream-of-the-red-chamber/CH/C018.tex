\chapter{皇恩重元妃省父母~天伦乐宝玉呈才藻}

话说彼时有人回,工程上等着糊东西的纱绫,请凤姐去开库;又有人来回,请
凤姐收金银器皿。王夫人并上房丫鬟等皆不得空儿。宝钗因说道:“咱们别在这里
碍手碍脚。”说着,和宝玉等便往迎春房中来。

王夫人日日忙乱,直到十月里才全备了:监办的都交清帐目;各处古董文玩,
俱已陈设齐备;采办鸟雀,自仙鹤、鹿、兔以及鸡、鹅等,亦已买全,交于园中各
处饲养;贾蔷那边也演出二三十出杂戏来;一班小尼姑、道姑也都学会念佛诵经。
于是贾政略觉心中安顿。遂请贾母到园中,色色斟酌,点缀妥当,再无些微不合之
处,贾政才敢题本。本上之日,奉旨:“于明年正月十五日上元之日贵妃省亲。”
贾府奉了此旨,一发日夜不闲,连年也不能好生过了。

转眼元宵在迩。自正月初八,就有太监出来先看方向,何处更衣,何处燕坐,
何处受礼,何处开宴,何处退息。又有巡察地方总理关防太监,带了许多小太监来
各处关防,挡围幕,指示贾宅人员何处出入,何处进膳,何处启事种种仪注。外面
又有工部官员并五城兵马司打扫街道,撵逐闲人。贾赦等监督匠人扎花灯烟火之
类,至十四日,俱已停妥。这一夜,上下通不曾睡。

至十五日五鼓,自贾母等有爵者,俱各按品大妆。此时园内帐舞蟠龙,帘飞绣
凤,金银焕彩,珠宝生辉,鼎焚百合之香,瓶插长春之蕊,静悄悄无一人咳嗽。贾
赦等在西街门外,贾母等在荣府大门外。街头巷口,用围幕挡严。正等的不耐烦,
忽见一个太监骑着匹马来了,贾政接着,问其消息。太监道:“早多着呢!未初用
晚膳,未正还到宝灵宫拜佛,酉初进大明宫领宴看灯方请旨。只怕戌初才起身呢。”
凤姐听了道:“既这样,老太太和太太且请回房,等到了时候再来也还不迟。”于
是贾母等自便去了。园中俱赖凤姐照料。执事人等,带领太监们去吃酒饭,一面传
人挑进蜡烛,各处点起灯来。

忽听外面马跑之声不一,有十来个太监,喘吁吁跑来拍手儿。这些太监都会意,
知道是来了,各按方向站立。贾赦领合族子弟在西街门外,贾母领合族女眷在大门
外迎接,半日静悄悄的。忽见两个太监骑马缓缓而来,至西街门下了马,将马赶出
围幕之外,便面西站立;半日又是一对,亦是如此。少时便来了十来对,方闻隐隐
鼓乐之声。一对对凤龙旌,雉羽宫扇,又有销金提炉,焚着御香,然后一把曲柄
七凤金黄伞过来,便是冠袍带履,又有执事太监捧着香巾、绣帕、漱盂、拂尘等物。
一队队过完,后面方是八个太监抬着一顶金顶鹅黄绣凤銮舆,缓缓行来。贾母等连
忙跪下。早有太监过来,扶起贾母等来,将那銮舆抬入大门往东一所院落门前,有
太监跪请下舆更衣。于是入门,太监散去,只有昭容、彩嫔等引着元春下舆。只见
苑内各色花灯灼,皆系纱绫扎成,精致非常。上面有一灯匾,写着“体仁沐德”
四个字。元春入室更衣,复出上舆进园。只见园中香烟缭绕,花影缤纷,处处灯光
相映,时时细乐声喧,说不尽这太平景象,富贵风流。

却说贾妃在轿内看了此园内外光景,因点头叹道:“太奢华过费了。”忽又见
太监跪请登舟。贾妃下舆登舟,只见清流一带,势若游龙,两边石栏上,皆系水晶
玻璃各色风灯,点的如银光雪浪;上面柳杏诸树,虽无花叶,却用各色绸绫纸绢及
通草为花,粘于枝上,每一株悬灯万盏;更兼池中荷荇凫鹭诸灯,亦皆系螺蚌羽毛
做就的,上下争辉,水天焕彩,真是玻璃世界,珠宝乾坤。船上又有各种盆景,珠
帘绣幕,桂楫兰桡,自不必说了。

已而入一石港,港上一面匾灯,明现着“蓼汀花溆”四字。看官听说:这“蓼
汀花溆”及“有凤来仪”等字,皆系上回贾政偶试宝玉之才,何至便认真用了?想
贾府世代诗书,自有一二名手题咏,岂似暴富之家,竟以小儿语搪塞了事呢?只因
当日这贾妃未入宫时,自幼亦系贾母教养。后来添了宝玉,贾妃乃长姊,宝玉为幼
弟,贾妃念母年将迈,始得此弟,是以独爱怜之。且同侍贾母,刻不相离。那宝玉
未入学之先,三四岁时,已得元妃口传教授了几本书,识了数千字在腹中。虽为姊
弟,有如母子。自入宫后,时时带信出来与父兄说:“千万好生扶养:不严不能成
器,过严恐生不虞,且致祖母之忧。”眷念之心,刻刻不忘。前日贾政闻塾师赞他
尽有才情,故于游园时聊一试之,虽非名公大笔,却是本家风味;且使贾妃见之,
知爱弟所为,亦不负其平日切望之意。因此故将宝玉所题用了。那日未题完之处,
后来又补题了许多。

且说贾妃看了四字,笑道:“‘花溆’二字便好,何必‘蓼汀’?”侍坐太监
听了,忙下舟登岸,飞传与贾政,贾政即刻换了。彼时舟临内岸,去舟上舆,便见
琳宫绰约,桂殿巍峨,石牌坊上写着“天仙宝境”四大字,贾妃命换了“省亲别墅”
四字。于是进入行宫,只见庭燎绕空,香屑布地,火树琪花,金窗玉槛;说不尽帘
卷虾须,毯铺鱼獭,鼎飘麝脑之香,屏列雉尾之扇。真是:
金门玉户神仙府,桂殿兰宫妃子家。
贾妃乃问:“此殿何无匾额?”随侍太监跪启道:“此系正殿,外臣未敢擅拟。”
贾妃点头。礼仪太监请升座受礼,两阶乐起。二太监引赦、政等于月台下排班上殿,
昭容传谕曰:“免。”乃退。又引荣国太君及女眷等自东阶升月台上排班,昭容再
谕曰:“免。”于是亦退。

茶三献,贾妃降座,乐止,退入侧室更衣,方备省亲车驾出园。至贾母正室,
欲行家礼,贾母等俱跪止之。贾妃垂泪,彼此上前厮见,一手挽贾母,一手挽王夫
人,三人满心皆有许多话,但说不出,只是呜咽对泣而已。邢夫人、李纨、王熙凤、
迎春、探春、惜春等,俱在旁垂泪无言。半日,贾妃方忍悲强笑,安慰道:“当日
既送我到那不得见人的去处,好容易今日回家,娘儿们这时不说不笑,反倒哭个不
了,一会子我去了,又不知多早晚才能一见!”说到这句,不禁又哽咽起来。邢夫
人忙上来劝解。贾母等让贾妃归坐,又逐次一一见过,又不免哭泣一番。然后东西
两府执事人等在外厅行礼。其媳妇丫鬟行礼毕。贾妃叹道:“许多亲眷,可惜都不
能见面!”王夫人启道:“现有外亲薛王氏及宝钗黛玉在外候旨。外眷无职,不敢
擅入。”贾妃即请来相见。一时薛姨妈等进来,欲行国礼,元妃降旨免过,上前各
叙阔别。又有原带进宫的丫鬟抱琴等叩见,贾母连忙扶起,命入别室款待。执事太
监及彩嫔昭容各侍从人等,宁府及贾赦那宅两处自有人款待,只留三四个小太监答
应。母女姊妹,不免叙些久别的情景及家务私情。

又有贾政至帘外问安行参等事。元妃又向其父说道:“田舍之家,盐布帛,
得遂天伦之乐;今虽富贵,骨肉分离,终无意趣。”贾政亦含泪启道:“臣草芥寒
门,鸠群鸦属之中,岂意得征凤鸾之瑞。今贵人上锡天恩,下昭祖德,此皆山川日
月之精华,祖宗之远德,钟于一人,幸及政夫妇。且今上体天地生生之大德,垂古
今未有之旷恩,虽肝脑涂地,岂能报效万一!惟朝乾夕惕,忠于厥职。伏愿圣君万
岁千秋,乃天下苍生之福也。贵妃切勿以政夫妇残年为念。更祈自加珍爱,惟勤慎
肃恭以侍上,庶不负上眷顾隆恩也。”贾妃亦嘱以“国事宜勤,暇时保养,切勿记
念”。贾政又启:“园中所有亭台轩馆,皆系宝玉所题;如果有一二可寓目者,请
即赐名为幸。”元妃听了宝玉能题,便含笑说道:“果进益了。”贾政退出。元妃
因问:“宝玉因何不见?”贾母乃启道:“无职外男,不敢擅入。”元妃命引进来。
小太监引宝玉进来,先行国礼毕,命他近前,携手揽于怀内,又抚其头颈笑道:“比
先长了好些——”一语未终,泪如雨下。

尤氏、凤姐等上来启道:“筵宴齐备,请贵妃游幸。”元妃起身,命宝玉导引,
遂同诸人步至园门前。早见灯光之中,诸般罗列,进园先从“有凤来仪”、“红香
绿玉”、“杏帘在望”、“蘅芷清芬”等处,登楼步阁,涉水缘山,眺览徘徊。一
处处铺陈华丽,一桩桩点缀新奇。元妃极加奖赞,又劝:“以后不可太奢了,此皆
过分。”既而来至正殿,降谕免礼归坐,大开筵宴,贾母等在下相陪,尤氏、李纨、
凤姐等捧羹把盏。

元妃乃命笔砚伺候,亲拂罗笺,择其喜者赐名。因题其园之总名曰“大观园”,
正殿匾额云“顾恩思义”,对联云:
天地启宏慈,赤子苍生同感戴;
古今垂旷典,九州万国被恩荣。
又改题:“有凤来仪”赐名“潇湘馆”。“红香绿玉”改作“怡红快绿”,赐名“怡
红院”。“蘅芷清芬”赐名“蘅芜院”。“杏帘在望”赐名“浣葛山庄”。正楼曰
“大观楼”。东面飞楼曰“缀锦楼”。西面叙楼曰“含芳阁”。更有“蓼风轩”、
“藕香榭”、“紫菱洲”、“荇叶渚”等名。匾额有“梨花春雨”、“桐剪秋风”、
“荻芦夜雪”等名。又命旧有匾联不可摘去。于是先题一绝句云:
衔山抱水建来精,多少工夫筑始成。
天上人间诸景备,芳园应锡大观名。

题毕,向诸姐妹笑道:“我素乏捷才,且不长于呤咏,姐妹辈素所深知,今夜
聊以塞责,不负斯景而已。异日少暇,必补撰《大观园记》并《省亲颂》等文,以
记今日之事。妹等亦各题一匾一诗,随意发挥,不可为我微才所缚。且知宝玉竟能
题咏,一发可喜。此中潇湘馆蘅芜院二处,我所极爱;次之怡红院浣葛山庄;此四
大处,必得别有章句题咏方妙。前所题之联虽佳,如今再各赋五言律一首,使我当
面试过,方不负我自幼教授之苦心。”宝玉只得答应了,下来自去构思。

迎春、探春、惜春三人中,要算探春又出于姊妹之上,然自忖似难与薛林争衡,
只得随众应命。李纨也勉强作成一绝。贾妃挨次看姊妹们的题咏,写道是:

旷性怡情(匾额)

迎
春
园成景物特精奇,奉命羞题额旷怡。
谁信世间有此境,游来宁不畅神思?

文采风流(匾额)

探
春
秀水明山抱复回,风流文采胜蓬莱。
绿裁歌扇迷芳草,红衬湘裙舞落梅。
珠玉自应传盛世,神仙何幸下瑶台。
名园一自邀游赏,未许凡人到此来。

文章造化(匾额)

惜
春
山水横拖千里外,楼台高起五云中。
园修日月光辉里,景夺文章造化功。

万象争辉(匾额)

李
纨
名园筑就势巍巍,奉命多惭学浅微。
精妙一时言不尽,果然万物有光辉。

凝晖钟瑞(匾额)

薛宝钗
芳园筑向帝城西,华日祥云笼罩奇。
高柳喜迁莺出谷,修篁时待凤来仪。
文风已著宸游夕,孝化应隆归省时。
睿藻仙才瞻仰处,自惭何敢再为辞?

世外仙源(匾额)

林黛玉
宸游增悦豫,仙境别红尘。
借得山川秀,添来气象新。
香融金谷酒,花媚玉堂人。
何幸邀恩宠,宫车过往频。
元妃看毕,称赏不已,又笑道:“终是薛林二妹之作与众不同,非愚姊妹所及。”
原来黛玉安心今夜大展奇才,将众人压倒,不想元妃只命一匾一咏,倒不好违谕多
做,只胡乱做了一首五言律应命便罢了。

时宝玉尚未做完,才做了“潇湘馆”与“蘅芜院”两首,正做“怡红院”一首,
起稿内有“绿玉春犹卷”一句。宝钗转眼瞥见,便趁众人不理论,推他道:“贵人
因不喜‘红香绿玉’四字,才改了‘怡红快绿’。你这会子偏又用‘绿玉’二字,
岂不是有意和他分驰了?况且蕉叶之典故颇多,再想一个改了罢。”宝玉见宝钗如
此说,便拭汗说道:“我这会子总想不起什么典故出处来!”宝钗笑道:“你只把
‘绿玉’的‘玉’字改作‘蜡’字就是了。”宝玉道:“绿蜡’可有出处?”宝钗
悄悄的咂嘴点头笑道:“亏你今夜不过如此,将来金殿对策,你大约连‘赵钱孙李’
都忘了呢!唐朝韩翊咏芭蕉诗头一句:‘冷烛无烟绿蜡干’都忘了么?”宝玉听了,
不觉洞开心意,笑道:“该死,该死!眼前现成的句子竟想不到。姐姐真是‘一字
师’了!从此只叫你师傅,再不叫姐姐了。”宝钗也悄悄的笑道:“还不快做上去,
只姐姐妹妹的!谁是你姐姐?那上头穿黄袍的才是你姐姐呢。”一面说笑,因怕他耽
延工夫,遂抽身走开了。

宝玉续成了此首,共有三首。此时黛玉未得展才,心上不快。因见宝玉构思太
苦,走至案旁,知宝玉只少“杏帘在望”一首,因叫他抄录前三首,却自己吟成一
律,写在纸条上,搓成个团子,掷向宝玉跟前。宝玉打开一看,觉比自己做的三首
高得十倍,遂忙恭楷誊完呈上。元妃看道是:

有凤来仪

宝
玉
秀玉初成实,堪宜待凤凰。
竿竿青欲滴,个个绿生凉。
迸砌防阶水,穿帘碍鼎香。
莫摇分碎影,好梦正初长。

蘅芷清芬
蘅芜满静苑,萝薜助芬芳。
软衬三春草,柔拖一缕香。
轻烟迷曲径,冷翠湿衣裳。
谁咏池塘曲?谢家幽梦长。

怡红快绿
深庭长日静,两两出婵娟。
绿蜡春犹卷,红妆夜未眠。
凭栏垂绛袖,倚石护清烟。
对立东风里,主人应解怜。

杏帘在望
杏帘招客饮,在望有山庄。
菱荇鹅儿水,桑榆燕子梁。
一畦春韭熟,十里稻花香。
盛世无饥馁,何须耕织忙。
元妃看毕,喜之不尽,说:“果然进益了!”又指“杏帘”一首为四首之冠,遂将
“浣葛山庄”改为“稻香村”。又命探春将方才十数首诗另以锦笺誊出,令太监传
与外厢。贾政等看了,都称颂不已。贾政又进《归省颂》。元妃又命以琼酪金脍等
物,赐与宝玉并贾兰。此时贾兰尚幼,未谙诸事,只不过随母依叔行礼而已。

那时贾蔷带领一班女戏子在楼下,正等得不耐烦,只见一个太监飞跑下来,说:
“做完了诗了,快拿戏单来!”贾蔷忙将戏目呈上,并十二个人的花名册子。少时,
点了四出戏:第一出《豪宴》,第二出《乞巧》,第三出《仙缘》,第四出《离魂》。
贾蔷忙张罗扮演起来,一个个歌有裂石之音,舞有天魔之态,虽是妆演的形容,却
做尽悲欢的情状。

刚演完了,一个太监托着一金盘糕点之属进来,问:“谁是龄官?”贾蔷便知
是赐龄官之物,连忙接了,命龄官叩头。太监又道:“贵妃有谕,说:‘龄官极好,
再做两出戏,不拘那两出就是了。’”贾蔷忙答应了,因命龄官做《游园》《惊梦》
二出。龄官自为此二出非本角之戏,执意不从,定要做《相约》《相骂》二出。贾
蔷扭不过他,只得依他做了。元妃甚喜,命:“莫难为了这女孩子,好生教习。”
额外赏了两匹宫绸,两个荷包,并金银锞子之类。然后撤筵,将未到之处复又游玩。
忽见山环佛寺,忙盥手进去焚香拜佛,又题一匾云“苦海慈航”。又额外加恩与一
班幽尼女道。

少时,太监跪启:“赐物俱齐,请验按例行赏。”乃呈上略节。元妃从头看了
无话,即命照此而行。太监下来,一一发放。原来贾母的是金玉如意各一柄,沉香
拐杖一根,枷楠念珠一串,“富贵长春”宫缎四匹,“福寿绵长”宫绸四匹,紫金
“笔锭如意”锞十锭,“吉庆有馀”银锞十锭。邢夫人等二分,只减了如意、拐、
珠四样。贾敬、贾赦、贾政等每分御制新书二部,宝墨二匣,金银盏各二只,表礼
按前。宝钗黛玉诸姊妹等,每人新书一部,宝砚一方,新样格式金银锞二对。宝玉
和贾兰是金银项圈二个,金银锞二对。尤氏、李纨、凤姐等皆金银锞四锭,表礼四
端。另有表礼二十四端,清钱五百串,是赏与贾母王夫人及各姊妹房中奶娘众丫鬟
的。贾珍、贾琏、贾环、贾蓉等皆是表礼一端,金银锞一对。其余彩缎百匹,白银
千两,御酒数瓶,是赐东西两府及园中管理工程、陈设、答应及司戏、掌灯诸人的。
外又有清钱三百串,是赐厨役、优伶、百戏、杂行人等的。

众人谢恩已毕,执事太监启道:“时已丑正三刻,请驾回銮。”元妃不由的满
眼又滴下泪来,却又勉强笑着,拉了贾母王夫人的手不忍放,再四叮咛:“不须记
挂,好生保养!如今天恩浩荡,一月许进内省视一次,见面尽容易的,何必过悲?倘
明岁天恩仍许归省,不可如此奢华糜费了。”贾母等已哭的哽噎难言。元妃虽不忍
别,奈皇家规矩违错不得的,只得忍心上舆去了。这里众人好容易将贾母劝住,及
王夫人搀扶出园去了。

未知如何,下回分解。